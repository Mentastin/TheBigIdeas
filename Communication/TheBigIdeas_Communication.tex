\documentclass{report}
\usepackage[utf8]{inputenc}
\usepackage{amsmath}
\usepackage{amsthm}
\usepackage{amssymb}
\usepackage{mathtools}
\usepackage{tikz}
\usepackage{lipsum}
\usepackage{ulem}
\usepackage{contour}
\usepackage{xcolor}
\usepackage{bbold}
\usepackage{url}
\usepackage{slashed}
\usepackage{cancel}
\usepackage{listings}
\usepackage{color}
\usepackage{physics}
\usepackage{siunitx}
\usepackage[skins, breakable]{tcolorbox}
\usepackage[greek,english]{babel}
\tcbset{breakable}
\usetikzlibrary{patterns, snakes, decorations.markings, trees}

\title{Some of the Big Ideas in communication}
\author{Joseph Cunningham}
\date{}


% --- New commands ---
\newcommand{\R}{\mathbb{R}}
\newcommand{\C}{\mathbb{C}}
% --- Define custom environments ---

\newtcolorbox{example}{enhanced,sharp corners=all,colback=white,colframe=black,toprule=0pt,bottomrule=0pt,leftrule=1pt,rightrule=0pt, overlay unbroken={
\draw[black,line width=1pt] (frame.north west) -- ++(.2,0);
\draw[black,line width=1pt] (frame.south west) -- ++(.2,0);
}, overlay first={
\draw[black,line width=1pt] (frame.north west) -- ++(.2,0);
}, overlay last={
\draw[black,line width=1pt] (frame.south west) -- ++(.2,0);
}, title={\underline{Example}}, attach boxed title to top,
boxed title style={empty,size=minimal,toprule=2pt,top=4pt},
coltitle=black, left=2mm, top=2mm, bottom=2mm}

\newtcolorbox{params}{enhanced,sharp corners=all,colback=white,colframe=black,toprule=0pt,bottomrule=0pt,leftrule=1pt,rightrule=0pt, overlay unbroken={
\draw[black,line width=1pt] (frame.north west) -- ++(.2,0);
\draw[black,line width=1pt] (frame.south west) -- ++(.2,0);
}, overlay first={
\draw[black,line width=1pt] (frame.north west) -- ++(.2,0);
}, overlay last={
\draw[black,line width=1pt] (frame.south west) -- ++(.2,0);
}, left=2mm, top=2mm, bottom=2mm}

\newtcolorbox{definition}{enhanced,sharp corners=all,colback=white,colframe=red,toprule=0pt,bottomrule=0pt,leftrule=1pt,rightrule=0pt, overlay unbroken={
\draw[red,line width=1pt] (frame.north west) -- ++(.2,0);
\draw[red,line width=1pt] (frame.south west) -- ++(.2,0);
}, overlay first={
\draw[red,line width=1pt] (frame.north west) -- ++(.2,0);
}, overlay last={
\draw[red,line width=1pt] (frame.south west) -- ++(.2,0);
}, title={DEF}, attach boxed title to top,
boxed title style={empty,size=minimal,toprule=2pt,top=4pt},
coltitle=red, left=2mm, top=2mm, bottom=2mm}

\newtcolorbox{eigenschap}{enhanced,sharp corners=all,colback=white,colframe=green,toprule=0pt,bottomrule=0pt,leftrule=1pt,rightrule=0pt, overlay unbroken={
\draw[green,line width=1pt] (frame.north west) -- ++(.2,0);
\draw[green,line width=1pt] (frame.south west) -- ++(.2,0);
}, overlay first={
\draw[green,line width=1pt] (frame.north west) -- ++(.2,0);
}, overlay last={
\draw[green,line width=1pt] (frame.south west) -- ++(.2,0);
}, left=2mm, top=2mm, bottom=2mm}
% --- Define underlines ---
\renewcommand{\ULdepth}{1.8pt}
\contourlength{0.9pt}
\renewcommand{\ULthickness}{.7pt}
\newcommand{\udef}[1]{%
\textcolor{red}{\uline{\phantom{#1}}}%
  \llap{\contour{white}{#1}}%
}
\newcommand{\ueig}[1]{%
\textcolor{green}{\uline{\phantom{#1}}}%
  \llap{\contour{white}{#1}}%
}
\newcommand{\undline}[1]{%
\uline{\phantom{#1}}%
  \llap{\contour{white}{#1}}%
}
% --- Code style ---
\lstdefinestyle{program}{numbers=left}
\lstdefinestyle{snippet}{}
\usepackage{courier}
\lstset{basicstyle=\selectfont\ttfamily}
% --- More settings ---
\newlength\tindent
\setlength{\tindent}{\parindent}
%\setlength{\parindent}{0pt}
\renewcommand{\indent}{\hspace*{\tindent}}

% Start at section zero:
\setcounter{section}{-1}
% Include \paragraph in ToC:
\setcounter{tocdepth}{5}
% Number \subsubsection:
\setcounter{secnumdepth}{3}

\graphicspath{ {./images/} }
% --- End setup ---

\begin{document}
\maketitle
\tableofcontents

\chapter{Preface}

\part{Communication basics}
\setcounter{chapter}{0} % Reset chapter counter

\part{Stories}
\setcounter{chapter}{0} % Reset chapter counter
Importance.
\chapter{Narrative structure}
\section{Non-fiction}
\subsection{Opening}
Straightforward, cold
\subsubsection{Cold opening}
story, anecdote
\subsection{Body}
Not too many points (3-5 major points)
\subsection{Close}
\subsubsection{Call to action}
More effective if to be done immediately.

\chapter{Rhetorical principles}
show, don't tell (even in writing)
\section{Giving a speech}
\url{https://youtu.be/6fBpfui5gO8}
Distill to a single bold message. Then make statement true.

Never spoil message. Why, what is solution, \ldots, give solution.

You do not need to be expert. Facilitate by getting others to be the expert.

Economy of storytelling.

Book: truth in comedy

From ``How To Speak by Patrick Winston'' - MIT OCW
Start

1. Do not start a talk with a joke.

2. Promise - Tell them what they gonna learn at the end of your talk.

3. Cycle – make your idea repeated many times in order to be completely clear for everyone.
4. Make a “Fence” around your idea so that it can be distinguished from someone else’s idea.

5. Verbal punctuation – sum up information within your talk some times to make listeners get back on.

6.  Ask a question - intriguing one


Place and Time



7. Best time for having a lecture is 11 am.
 (not too early and not after lunch)
8. The place should be well lit.

9. The place should be seen and checked before the lecture.
10. The place should not be full less than a half, it must be chosen according to the amount of listeners.



Tools

For teaching.

1. Board – it’s got graphics, speed, target. Watch your hands! Don’t hold them behind your back, it’s better to keep them straight and use for pointing at the board.

2. Props – use them in order to make your ideas visual.

Visual perception is the most effective way to interact with listeners.



For Job Talk. Exposing, Slides

3. Don’t put too many words on a slide. Slides should just reflect what you’re saying, not the other way around. Pictures attracts attention and people start to wait for your explanation – use that tip.

4. Make slide as easy as you can – no title, no distracting pictures, frames, points and so on.

5. Do not use laser pointer – due to that you lose eye contact with the audience. Instead you can make the arrows just upon a slide. 



Informing

 

Show to your listeners your stuff is cool and interesting.

You have to be able to:

-show your vision of that problem

-show that you’ve done particular things (by steps)

All of that should be done real quick in no more than 5 min.

Persuade your listeners you’re not a rookie (Prof. Winston contrived to do that from the very first seconds of his talk)



Getting Famous

If you want to your ideas be remembered you’ve got to have
 "5 S"


- Symbols associate with your ideas (visual perception is the best way to attract attention)

- Slogan (describing your idea)

- Surprise (common fallacy that is no longer true, for instance, just after you’ve told about it)

- Salient Idea (not necessarily important but the one that sticks out)

- Story (how you did it, how it works…)



How to End



- Don’t put collaborators at the end, do that at the beginning.
- Question’s the worst way to end a talk.

- It’s good to end with a Contribution slide – to sum up everything you’ve told with your OWN decision.

- At the very end you could tell a joke since people then will leave the event feeling fun and thus keep a good memory of your talk.
- "Thank you (for listening)" isn’t good ending, it’s trite at least. You can end with a quote of a prominent person (my own knowledge), with a salute to people (how much you valued the time being here, the people over here..., “I’d like to get back, it was fun!”




\chapter{Shorthand for complex ideas}
\section{What makes something evocative}
\section{Metaphors}
Tools in a toolbelt.
\section{Slogans}
Try to find the value in what people do / are saying

Libertarianism is the least we should achieve.
\section{Fancy names}
\subsection{The power of names}
\subsection{Potentially predictive psychology theories}
Pyramid of Maslow
Stages of grief (Kübler-Ross)
Pygmalion (Rosenthal)
\subsection{Other theories}
Overton window
Community of strength >< Community of vulnerability (Simplican)
Discovery, Debate, Acceptance, Arrogance (DDAA) (Mo Gawdat)
\subsection{Observations}
Dunning-Kruger
Baader-Meinhof phenomenon
Arousal non-concordance
Hedonic adaptation: the observed tendency of humans to quickly return to a relatively stable level of happiness despite major positive or negative events or life changes.
Heidegger's hammer.
Peter principle
\subsection{Studies}
Stanley Milgram
Ash
Stanford prison (discredited?)

\chapter{Writing}
\section{Tips to start writing}
Just start writing, even if you do not know what the conclusion will be or how it will turn out. Often the text will flow differently anyway.

Outline. The use of small steps.

Best stories come from living your life.

When stuck, ask for help.

Read out loud
\section{Visual}
\subsection{Layout}
\subsection{Font}
\subsection{Table of contents}
\section{Styles (theories of communication)}
https://criticalthinkeracademy.com/courses/a-essays/lectures/315864
\subsection{Classic style}
(Steven Pinker)
The model scene for classic style is one person speaking to another, a conversation between equals.

The writer uses prose as a window to describe a world, and to draw attention to the objects and actions going on within this world.

The assertions, the claims that the writer wants to make, are depicted in this world, and the writer tries to get the audience, the reader, to see what is depicted by positioning the reader so that he or she can see what the writer sees.

The writer wants to reveal some truth about the world, and their goal is to get the reader to see this truth, through a conversational dialogue about the world that the writer has created, but that is imaginatively accessible to both of them.

Classic style aims at the presentation of an objective, disinterested truth about the world — a truth that can be confirmed by anyone with a suitable background and position to see it. 
\subsection{Romantic style}
where prose is viewed as a MIRROR to the SELF, not a WINDOW to a world beyond the self.
\subsection{Reflexive style}
where the author wants to draw the reader’s attention to the act of writing itself, and to the challenges the writer faces.
\subsection{Practical style}
Classic and practical style have a common interest in clarity and directness in writing, but they value this for different reasons.

In practical style, clarity is a virtue because its primary goal is to be easily understood by the reader, so that it can help the reader with whatever practical problem they’re facing. 
\section{Tips}
https://philosophicaldisquisitions.blogspot.com/2014/09/steven-pinkers-guide-to-classic-style.html
\section{Writing scripts}
\subsection{Format}

\part{(Neuro)Psychology}
\setcounter{chapter}{0} % Reset chapter counter
Autism: inability to abstract. Kitchen with chair moved is different kitchen. Church not just pentagon with spire and cross, but a specific building.
\chapter{Personality}
Helps to appreciate how different we can be, even if wrong. 

“Big Five” personality traits: openness to experience, conscientiousness, extraversion, agreeableness and neuroticism
\chapter{Intelligence}
fixed / fluid
\chapter{Teaching and learning}
\section{Domains}
Procedural knowledge >< declarative knowledge
\subsection{Cognitive}
\subsection{Psychomotor}
\subsubsection{Motor coordination}
\subsection{Affective}
\section{Types}
\section{Techniques}
Pomodoro
\section{Transfer}
\section{Factors affecting learning}
\subsection{External factors}
\begin{enumerate}
\item Heredity
\item Status of student
\item Physical environment
\end{enumerate}
\subsection{Internal factors}
\begin{enumerate}
\item Goals or purposes
\item Motivational behaviour
\item Interest
\item Attention
\item Drill or practice
\item Fatigue
\item Aptitude
\item Attitude
\item Emotional conditions
\item Speed, accuracy and retention
\item Learning activities
\item Testing
\item Guidance
\end{enumerate}
\section{Tips}
Testing, spacing, inter-leaving (\url{https://www.youtube.com/watch?v=Y_B6VADhY84})

\chapter{Problem solving}
From ``Bulletproof problem solving'':
\begin{itemize}
\item first define problems correctly;
\item breaking problems down into smaller parts + logic trees
\item prioritisation of solutions should be done by weighing the two factors:
\begin{itemize}
\item scale of impact
\item your ability to influence outcomes
\end{itemize}
\item egalitarian work processes can help a team beat individual biases;
\item there is lots of useful data in the real world if you can take the time to look.
\end{itemize}

\part{Affective communication}
TODO better title??? Empathetic communication??
\setcounter{chapter}{0} % Reset chapter counter

\chapter{Body language and non-literal communication}
\section{Flirting}
\subsection{HOT APE}

\chapter{Negotiation and persuasion}
\section{How people make decisions}
show how you add value

\subsection{Personal (dis)liking}
If people like you, they will go out of their way.

People do not need to like you; they need to think they know what motivates you.


\section{Call to action}
\section{Attention}
\section{Leverage}
People like fairness! Handing out Mars bars (Brian Brushwood).

\chapter{Debating}
Very many arguments are arguments from authority.
- Trusted authorities are a useful heuristic for what to believe
- Epistemological interdependence

\chapter{Conflict resolution}
Getting to Zero - Jason Gaddis

\part{Information mapping}
\setcounter{chapter}{0} % Reset chapter counter
\chapter{Information design}
\section{Olog}
\chapter{Visualisation}
\section{Visual clarity}
\section{Data visualisation}
\section{Visualisation of processes}

\part{Communicating to self: life improvements}
\setcounter{chapter}{0} % Reset chapter counter
\chapter{The value of lists}
\section{Types of lists}
\subsection{Things learned}
\subsection{Things to learn}
\subsection{Media to consume}
\subsection{Media consumed}
\subsection{Diary}
Keep a diary
\subsection{Contacts}
\chapter{Routines}
Flow triggers:
\begin{itemize}
\item Risk
\item Novelty
\item Complexity
\item Unpredictability
\item Pattern recognition
\end{itemize}
Ways to implement:
\begin{itemize}
\item Start by editing previous work (pattern recognition)
\item Go somewhere new to work.
\end{itemize}

Easier and better to take things away than adding things (like going to the gym).
Themes per year. Year of less. Cfr. CGP Grey. No Fail states. Fades into background, gets incorporated into thinking. Openended. Meaning can change. Important to change trend, not individual decision.

\chapter{What to do}
Proactive serendipity

a little at a time: We think willpower is more finite than it is

apparently meditation is quite good

\chapter{Productivity}
batching

\appendix

\chapter{Tools}

\chapter{Bibliography}

\url{https://en.wikipedia.org/wiki/Information_mapping}

Visual Display of Quantitative Information

\end{document}