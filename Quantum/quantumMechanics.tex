\chapter{Measurement theory}

\chapter{Postulates}

\section{States, observables and measurement}
\subsection{States}
\begin{definition}
A \udef{state} on a Hilbert space is a positive trace class operator with trace $1$.
\end{definition}

\begin{proposition}
Let $\rho$ be a state. The following are equivalent:
\begin{enumerate}
\item $\rho$ is extremal;
\item $\rho$ is a projector;
\item $\rho = \rho^2$;
\item $\Tr[\rho^2] = 1$;
\item $\norm{\rho} = 1$;
\item $S(\rho) \defeq -\Tr[\rho\ln\rho] = 0$.
\end{enumerate}
\end{proposition}
The quantity $S(\rho) \defeq -\Tr[\rho\ln\rho]$ is called the \udef{von Neumann entropy}.

\subsection{Effects}
\begin{definition}
An \udef{effect} on a Hilbert space is a bounded operator in $[0,\id]$.
\end{definition}

\chapter{Quantum theory}
\section{Quantum statistics}

\section{Time evolution}
\subsection{The Schrödinger equation}
\[ H = i\hbar\od{}{t} \]

\begin{lemma}
$U = e^{Ht / i\hbar} = e^{-iHt/\hbar}$.
\end{lemma}


\subsection{Schrödinger and Heisenberg pictures}
\begin{lemma}
Let $\mathcal{H}$ be a Hilbert space and $\{e_i\}$ an orthonormal basis. Let $U$ be a unitary operator on $\mathcal{H}$. Then $\{Ue_i\}$ is also an orthonormal basis for $\mathcal{H}$.
\end{lemma}
$\{e_i\}$ gives Schrödinger and $\{Ue_i\}$ gives Heisenberg.



\subsection{Adiabatic theorem}

\begin{theorem}
Let $\mathcal{H}$ be a Hilbert space and consider a path $H: \interval{0,1}\to \SelfAdjoints(\mathcal{H}): s\mapsto H(s)$ of self-adjoint operators on the Hilbert space.

Take $\epsilon >0$. If the Hamiltonian of a system is given by $H(\epsilon t)$ for times $t\in \interval{0,\epsilon^{-1}}$. Then the state $\rho$ of the system satisfies
\[ i\epsilon \od{\rho(s)}{s} = \big[H(s), \rho(s)\big], \]
where $s = \epsilon t \in \interval{0,1}$.

Assume $\lambda(s)\in \spec\big(H(s)\big)$ is an isolated point of the spectrum for all $s\in \interval{0,1}$. It is an eigenvalue and let $P(s)$ be the orthogonal projector onto the eigenspace. Set $\Delta(s) \defeq \inf d\big(\lambda, \spec(H(s))\setminus\{\lambda(s)\}\big)$.

Assume $\rho(0) \leq P(0)$. Consider the fidelity $F(s) \defeq \Tr\big(P(s)\rho(s)\big)$. Then
\[ F(1) \geq 1 - \epsilon O\left(\frac{\norm{H'}^2}{\Delta^3} + \frac{\norm{H^{\prime\prime}}}{\Delta^2}\right). \]
\end{theorem}
\begin{proof}
We first write down a differential equation for the fidelity (where $\prime = \od{}{s}$)
\[ \od{F(s)}{s} = \Tr(P'\rho) + \Tr(P\rho'). \]
The second term is zero because $P$ commutes with $H$:
\begin{align*}
\Tr(P\rho') &= -\epsilon^{-1}i\Tr\Big(P\big[H, \rho\big]\Big) \\
&= -\epsilon^{-1}i\Tr\Big(P(H\rho - \rho H)\Big) \\
&= -\epsilon^{-1}i\Tr\Big(PH\rho\Big) +\epsilon^{-1}i\Tr\Big(\rho HP\Big) \\
&= -\epsilon^{-1}i\Tr\Big(PH\rho\Big) +\epsilon^{-1}i\Tr\Big(\rho PH\Big) \\
&= -\epsilon^{-1}i\Tr\Big(\big[PH, \rho\big]\Big) = 0,
\end{align*}
by \zref{m-traceCommutatorCompactSA}.

Now set $Q(s) = \id - P(s)$ and consider the pseudoinverse $(H-\lambda\id)^+$ (TODO by continuous functional calculus?). Then $(H-\lambda\id)(H-\lambda\id)^+ = Q = (H-\lambda\id)^+(H-\lambda\id)$ by continuous functional calculus (TODO ref). Now we can expand $P' = PP'Q + QP'P$ (by \zref{m-derivativeIdempotentOffDiagonal}), so
\begin{align*}
\od{F(s)}{s} &= \Tr(P'\rho) \\
&= \Tr(PP'Q\rho + QP'P\rho) \\
&= \Tr\Big(PP'(H-\lambda\id)^+(H-\lambda\id)\rho + (H-\lambda\id)(H-\lambda\id)^+P'P\rho\Big) \\
&= \Tr\Big(P'(H-\lambda\id)^+(H-\lambda\id)\rho P + (H-\lambda\id)^+P'P\rho (H-\lambda\id)\Big) \\
&= \Tr\Big(P'(H-\lambda\id)^+\big(H\rho P - \rho \lambda P\big) + (H-\lambda\id)^+P' \big(P\rho H - \lambda P\rho \big)\Big) \\
&= \Tr\Big(P'(H-\lambda\id)^+\big(H\rho - \rho H\big)P + (H-\lambda\id)^+P'P\big(\rho H - H\rho \big)\Big) \\
&= \Tr\Big(P'(H-\lambda\id)^+\big[H,\rho\big]P - (H-\lambda\id)^+P'P\big[H, \rho\big]\Big) \\
&= \Tr\Big(PP'(H-\lambda\id)^+\big[H,\rho\big] - (H-\lambda\id)^+P'P\big[H, \rho\big]\Big) \\
&= \Tr\Big(P'Q(H-\lambda\id)^+\big[H,\rho\big] - (H-\lambda\id)^+QP'\big[H, \rho\big]\Big) \\
&= \Tr\Big(P'(H-\lambda\id)^+\big[H,\rho\big] - (H-\lambda\id)^+P'\big[H, \rho\big]\Big) \\
&= \Tr\Big(\big[P', (H-\lambda\id)^+\big]\cdot\big[H,\rho\big]\Big) \\
&= i\epsilon\Tr\Big(\big[P', (H-\lambda\id)^+\big]\rho'\Big) 
\end{align*}
Integrating this w.r.t. $s$ gives
\[ F(1) - F(0) = i\epsilon\int_0^1 \Tr\Big(\big[P', (H-\lambda\id)^+\big]\rho'\Big)  \diff{s}. \]
We fill in that $F(0) = 1$ and perform integration by parts to obtain
\begin{align*}
F(1) &= 1 + i\epsilon\int_0^1 \Tr\Big(\big[P', (H-\lambda\id)^+\big]\rho'\Big)  \diff{s} \\
&= 1 + i\epsilon\Tr\Big(\big[(H-\lambda\id)^+, P'\big]\rho\Big)_0^1 - i\epsilon\int_0^1 \Tr\Big(\big[(H-\lambda\id)^+, P'\big]'\rho\Big)  \diff{s}
\end{align*}
\end{proof}


\begin{proposition}
Let $\mathcal{H}$ be a Hilbert space, $H: \interval{0,1}\to \SelfAdjoints(\mathcal{H})$ and $P(s)$ the projector on an eigenspace associated to some subset of the spectrum $\sigma'$ that is separated from the rest by a gap $g(s)$. Let $\rho(s)$ be the solution the the Heisenberg equation
\[ i\epsilon \od{\rho(s)}{s} = [H(s), \rho(s)] \qquad \rho(0)= P(0). \]
Then we can write
\[ \rho(s) = \sum_{n=0}^N \epsilon^nB_n(s) - \epsilon^N\int_0^s U_\epsilon(s,r)\dot{B}_N(r)U_\epsilon(r,s) \diff{r}, \]
where the $B_n$ satisfy the recursion relation
\[ \begin{cases}
B_0(s) = P(s) \\
B_n(s) = \frac{1}{2\pi}\int_\Gamma R_H(z)\big[P, \dot{B}_{n-1}\big]R_H(z) \diff{z} + S_n - 2PS_nP
\end{cases}, \]
where $\Gamma$ is some Jordan curve that contains $\sigma'$, but not the rest of the spectrum,
\[ S_n \defeq \sum_{m=1}^{n-1}B_mB_{n-m} \]
and $U_\epsilon(s,t)$ is the propagator that satisfies
\[ \begin{cases}
i\epsilon \pd{}{s}U_\epsilon(s,r) = HU_\epsilon(s,r) \\
U_\epsilon(r,r) = \id.
\end{cases} \]
\end{proposition}

\begin{lemma}
Let $\mathcal{H}$ be a Hilbert space and $H: \interval{0,1}\to \SelfAdjoints(\mathcal{H})$ a function that belongs to the Gevrey class $G^\alpha(R)$. Then
\[ \norm{\od[k]{B_n(s)}{s}} \leq L(n,k) \defeq \frac{1}{(10n + 0.3)^2}g^{-2n-k}\big(2CR(k+3n)\big)^{k+3n}. \]
\end{lemma}
\begin{proof}

\end{proof}

\chapter{Approximations}
\section{Approximating eigenvectors}
\subsection{Power series expansion of a non-degenerate level}

\section{Approximating evolutions}

\chapter{Investigations of systems}
\section{Stepped potentials}
(Use density to solve general potentials?)
\section{Coulomb interaction}
\section{Harmonic oscillator}