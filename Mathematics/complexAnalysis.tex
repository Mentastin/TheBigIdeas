\begin{definition}
A \udef{complex function} is a function in $(U\subseteq \C \to \C)$.

A \udef{region} of $\C$ is an open, connected subset of $\C$.
\end{definition}

\section{Prerequisites}
TODO: move
\subsection{The complex numbers as a real algebra}
The set of complex numbers $\C$ is isomorphic to the algebra $\R^2$ where multiplication is defined by
\[ (a, b)\cdot (x, y) \defeq (ax-by, ay+bx). \]
Thus the regular representation $\lambda_{(a,b)}$ has matrix
\[ \begin{pmatrix}
a & -b \\ b & a
\end{pmatrix}. \]

\subsection{Curves and regions in $\C$}
\subsubsection{Jordan curves}
\begin{definition}
A \udef{Jordan curve} in $\C$ is a continuous injective function $\gamma: \interval{0,1}\to \C$ such that $\gamma(0) = \gamma(1)$.
\end{definition}
Sometimes people use ``Jordan curve'' to mean a continuous injective function $\gamma: \interval{0,1}\to \C$, without the condition $\gamma(0) = \gamma(1)$. Then our Jordan curve is called a ``closed Jordan curve''.

\begin{theorem}
Let $\gamma$ be a Jordan curve. Then $\C\setminus\im(\gamma)$ has a separation $(I,E)$, such that
\begin{enumerate}
\item $I$ and $E$ are connected;
\item $I$ is bounded;
\item $E$ is unbounded;
\item the boundary of both $I$ and $E$ in $\C$ is $\im(\gamma)$.
\end{enumerate}
\end{theorem}
\begin{proof}
\url{https://www.jstor.org/stable/2323369?origin=crossref&seq=3}
\end{proof}

\subsubsection{Rectifiable curves}
\begin{definition}
A curve $\gamma: \interval{0,1}\to \C$ is called \udef{rectifiable} if it is of bounded variation.
\end{definition}
TODO: rectifiable curves are exactly the curves
(1) that we can Stieltjes-integrate over and (2) have defined arc-length.


\section{Holomorphic functions}
\begin{definition}
Let $f:U\subseteq \C \to \C$ be a complex function. We say
\begin{enumerate}
\item $f$ is \udef{holomorphic  at $z\in U$} if the limit $\lim_{h\to 0} \dfrac{f(z+h) - f(z)}{h}$
exists;
\item $f$ is \udef{holomorphic in $S\subset U$} if it is holomorphic at every point in $S$;
\item $f$ is \udef{holomorphic} if it is holomorphic at every point in $U$;
\item $f$ is \udef{entire} if $U=\C$ and it is holomorphic at every point in $\C$.
\end{enumerate}
We denote by $\holomorphic(A)$ the set of functions that are holomorphic on a neighbourhood of $A$.
\end{definition}
Note that the functions in $\holomorphic(A)$ do not in general have the same domains!!

\begin{lemma}
Holomorphic functions are continuous.
\end{lemma}
\begin{lemma}
Let $f,g$ be holomorphic. Then
\begin{enumerate}
\item $f+g$ is holomorphic and $(f+g)' = f'+g'$;
\item $fg$ is holomorphic and $(fg)' = f'g+fg'$;
\item if $g(z_0)\neq 0$, then $f/g$ is holomorphic at $z_0$ and
\[ (f/g)' = \frac{f'g - fg'}{g^2}; \]
\item the chain rule holds.
\end{enumerate}
\end{lemma}

\subsection{Cauchy-Riemann equations}
The space of complex numbers $\C$ is a real $2$-dimensional vector space.

\begin{lemma}
Let $f: U\subseteq \C \to C$ be a complex function. Then $f$ is holomorphic \textup{if and only if} it is Fréchet differentiable as a function $F: V\subseteq \R^2 \to \R^2$ and $\diff F$ is the regular representation of some $z\in \C$.
\end{lemma}
\begin{corollary}[Cauchy-Riemann equations]
Let $f: U\subseteq \C \to C$ be a complex function. Then $f$ is complex differentiable at $z_0 = a+bi$ \textup{if and only if} the real derivatives
\[ \pd{\Re(f(a + ib))}{a},\; \pd{\Re(f(a + ib))}{b},\;
\pd{\Im(f(a + ib))}{a} \;\;\text{and}\;\; \pd{\Im(f(a + ib))}{b}. \]
exist, are continuous and the equations
\[
\pd{\Re(f(a + ib))}{a} = \pd{\Im(f(a + ib))}{b} \quad\text{and}\quad \pd{\Re(f(a + ib))}{b} = -\pd{\Im(f(a + ib))}{a} \]
hold. In particular
\[ |f'(z_0)|^2 = \det J_f(a,b) \]
where $\det J_f(a,b)$ is the Jacobian determinant of $f$ as a function $V\subseteq \R^2 \to \R^2$.
\end{corollary}
\begin{proof}
We need the Jacobian
\[ \begin{pmatrix}
\pd{\Re(f(a + ib))}{a} & \pd{\Re(f(a + ib))}{b} \\
\pd{\Im(f(a + ib))}{a} & \pd{\Im(f(a + ib))}{b}
\end{pmatrix} \]
to be of the form $\begin{pmatrix}
x & -y \\ y & x
\end{pmatrix}$ in order for it to be a regular representation of some $f' =\pd{\Re(f(a + ib))}{a} + i\pd{\Im(f(a + ib))}{a} \in \C$. TODO ref requirement continuously differentiable.

The Jacobian determinant is a simple calculation:
\begin{align*}
\det J_f(a,b) &= \pd{\Re(f(a + ib))}{a}\pd{\Im(f(a + ib))}{b} - \pd{\Im(f(a + ib))}{a}\pd{\Re(f(a + ib))}{b} \\
&= \pd{\Re(f(a + ib))}{a}^2 + \pd{\Im(f(a + ib))}{a}^2 = |f'|^2.
\end{align*}
\end{proof}
\begin{corollary}
Let $f: U\subseteq \C \to C$ be a holomorphic function. Consider the function $f_1: V\subseteq \R^2 \to \R: (a,b) \mapsto \Re(f(a+ib))$ and $f_2: V\subseteq \R^2 \to \R: (a,b) \mapsto \Im(f(a+ib))$. Then $\nabla^2 f_1 = 0$ and $\nabla^2 f_2 = 0$.
\end{corollary}
Thus both the real and imaginary part satisfy Laplace's equation. We already use \ref{holomorphicFunctionIsAnalytic}, even though it is proved later. This corollary is not needed to establish \ref{holomorphicFunctionIsAnalytic}.
\begin{proof}
We calculate, using Schwarz's theorem \ref{SchwarzTheorem} and the fact that $f$ is infinitely differentiable \ref{holomorphicFunctionIsAnalytic},
\[ \nabla^2 f_1 = \pd[2]{f_1}{a} + \pd[2]{f_1}{b} = \pd{}{a}\Big(\pd{f_2}{b}\Big) - \pd{}{b}\Big(\pd{f_2}{a}\Big) = 0. \]
A similar calculation gives $\nabla^2 f_2 = 0$.
\end{proof}

\subsection{Cauchy's theorem, Morera's theorem and the integral formula}

\begin{theorem}[Cauchy's theorem] \label{CauchyTheorem}
Let $f: U\subseteq \C \to \C$ be a holomorphic complex function and $\gamma$ a rectifiable Jordan curve whose interior lies in $U$. Then
\[ \oint_\gamma f(z)\diff{z} = 0. \]
\end{theorem}
TODO: looser requirements for $\gamma$
\begin{proof}
TODO: generalised Stokes + Cauchy-Riemann!
\end{proof}
\begin{corollary}
Let $f: U\subseteq \C \to \C$ be a holomorphic complex function on a region $U$. Then there exists a primitive $F: U\to \C$ such that $\od{F}{z} = f$.
\end{corollary}
\begin{proof}
Take $x_0\in U$ and define 

The integral $\int_\gamma f(z)\diff{z}$ depends only on the endpoints of $\gamma$.
\end{proof}

\begin{theorem}[Morera's theorem]
Let $f: U\subseteq \C \to \C$ be a complex function on an open set $U$ such that
\[ \oint_\gamma f(z) \diff{z} = 0 \]
for every triangle $\gamma$ in $U$, then $f$ is holomorphic.
\end{theorem}
\begin{proof}
$\int f(z)\diff{z}$ is a primitive, so $f$ is complex differentiable. TODO
\end{proof}


\begin{note}
Positive orientation is counterclockwise. I.e. $e^{it}$ rotates with positive orientation.

Local view: positive orientation if interior is on the left.
\end{note}

\begin{theorem}[Cauchy's integral formula] \label{CauchyIntergralFormula}
Let $f: U\subseteq \C \to \C$ be a holomorphic complex function and $\gamma$ a positively oriented Jordan curve whose interior lies in $U$. Then
\[ f(z) = \frac{1}{2\pi i}\oint_\gamma \frac{f(\zeta)}{\zeta - z}\diff{\zeta} \]
for any point $z$ in the interior of $\gamma$.
\end{theorem}
\begin{proof}
Let $C_{z,\epsilon}$ be a small circle inside $\gamma$ around $z$ of radius $\epsilon$ and opposite orientation (TODO explicate!). Then $\frac{f(\zeta)}{\zeta - z}$ is holomorphic in the region between $\gamma$ and $C_{z,\epsilon}$, so
\begin{align*}
\oint_\gamma \frac{f(\zeta)}{\zeta - z}\diff{\zeta} &= -\oint_{C_{z,\epsilon}} \frac{f(\zeta)}{\zeta - z}\diff{\zeta} \\
&= -\oint_{C_{z,\epsilon}} \frac{f(\zeta)-f(z)}{\zeta - z} + \frac{f(z)}{\zeta - z}\diff{\zeta} \\
&= -\oint_{C_{z,\epsilon}} \frac{f(\zeta)-f(z)}{\zeta - z}\diff{\zeta} -\oint_{C_{z,\epsilon}}\frac{f(z)}{\zeta - z}\diff{\zeta}.
\end{align*}
In the limit $\epsilon \to 0$, the first part is bounded by
\[ \left|\oint_{C_{z,\epsilon}} \frac{f(\zeta)-f(z)}{\zeta - z}\diff{\zeta}\right| \leq \sup_\zeta\left|\frac{f(\zeta)-f(z)}{\zeta - z}\right| \cdot |2\pi\epsilon| \to 0 \]
because $\frac{f(\zeta)-f(z)}{\zeta - z}$ remains bounded. For the second part, we have
\begin{align*}
-\oint_{C_{z,\epsilon}} \frac{f(z)}{\zeta - z}\diff{\zeta} &= f(z)\oint_{C_{z,\epsilon}} \frac{\diff{\zeta}}{\zeta - z} \\
&= f(z)\int_0^{2\pi}\frac{\epsilon ie^{-it}}{\epsilon e^{-it}}\diff{t} \\
&= f(z)2\pi i.
\end{align*}
\end{proof}
\begin{corollary}
Let $f: U\subseteq \C \to \C$ be a holomorphic complex function. Then $f$ has infinitely many complex derivatives and
\[ f^{(n)}(z) = \frac{n!}{2\pi i}\oint_\gamma \frac{f(\zeta)}{(\zeta-z)^{n+1}}\diff{\zeta} \]
for all $z$ in the interior of $\gamma$.
\end{corollary}
\begin{proof}
TODO
\end{proof}

\begin{proposition} \label{holomorphicFunctionIsAnalytic}
Let $f$ be holomorphic in an open set $\Omega$ and $D$ a
disc centered at $z_0$ whose closure is contained in $\Omega$. Then $f$ has a power series expansion at $z_0$
\[ f(z) = \sum_{n=0}^\infty a_n(z-z_0)^n \]
for all $z\in D$. The coefficients are given by
\[ a_n = \frac{f^{(n)}(z_0)}{n!} \qquad \text{for all $n\geq 0$}. \]
\end{proposition}
\begin{proof}
TODO
\end{proof}
\begin{corollary}
Let $f: \Omega\subseteq \C \to \C$ be a function and $\Omega$ an open set. Then $f$ is holomorphic \textup{if and only if} $f$ is analytic.
\end{corollary}
\begin{corollary}
If $f$ is holomorphic in an open
set that contains the closure of a disc $D$ centered at $z_0$ and of radius $R$,
then
\[  |f^{(n)}(z_0)| \leq \frac{n!\norm{f}_C}{R^n}, \]
where $\norm{f}_C = \sup_{z\in C}|f(z)|$.
\end{corollary}
\begin{proof}
TODO
\end{proof}
Thus the distance from $z_0$ to the nearest singular point is the radius of convergence of the power series.
\begin{corollary}[Liouville's theorem] \label{liouvilleTheoremAnalysis}
If $f$ is entire and bounded, then $f$ is contant.
\end{corollary}
\begin{proof}
It suffices to show that $f'(z_0) = 0$. This can be seen by taking $R\to\infty$ in the previous inequality.
\end{proof}
\begin{corollary}[Fundamental theorem of algebra]
Let $P(z) = a_nz^n + \ldots + a_0$ be a polynomial of degree $n\geq 1$ with complex coefficients. Then $P(z)$ has precisely $n$ roots. If these roots are denoted $w_1, \ldots, w_n$, then we can write
\[ P(z) = a_n(z-w_1)(z-w_2)\ldots(z-w_n). \]
\end{corollary}

\begin{proposition}
Let $f: \Omega\subseteq \C\to \C$ be a complex function that is holomorphic in a region $U\subseteq \Omega$ that contains a closed annulus $\setbuilder{z\in \C}{r\leq|z-z_0|\leq R}$ for some $z_0\in \C$ and $r,R\in \R$. Then $f$ has a Laurent series expansion that converges in the interior of the annulus:
\[ f(z) = \sum_{i=-\infty}^\infty a_i(z-z_0)^i \]
for all $z\in \setbuilder{z\in \C}{r < |z-z_0| < R}$.
\end{proposition}
\begin{proof}
TODO
\end{proof}
TODO uniqueness ??? \url{https://en.wikipedia.org/wiki/Laurent_series#Uniqueness}

\subsection{Analytic continuation}
\begin{proposition}
Let $f:U\subseteq \C\to \C$ be a holomorphic function defined on a region. Suppose there exists a sequence of distinct points with limit point in $U$ on which $f$ vanishes. Then $f = \underline{0}$.
\end{proposition}
\begin{proof}
TODO
\end{proof}
\begin{corollary} \label{zerosHolomorphicFunctionIsolated}
Let $f:U\subseteq \C\to \C$ be a holomorphic function defined on a region. Then the zeros of $f$ are isolated.
\end{corollary}
\begin{proof}
TODO
\end{proof}
\begin{corollary} \label{holomorphicFunctionsCoincidingOnSetWithLimitPoint}
Let $f,g:U\subseteq \C\to \C$ be holomorphic functions defined on a common region. Suppose there exists a sequence of distinct points $\seq{z_n}$ with limit point in $U$ such that $f(z_n) = g(z_n)$ for all $n$. Then $f = g$.
\end{corollary}
In particular this holds if $f$ and $g$ agree on some non-empty open subset of $U$.
\begin{corollary}
Let $U\subseteq U' \subseteq \C$ be regions and $f:U\subseteq\to \C$ a holomorphic function. Then there exists at most one holomorphic function $f'$ on $U'$ such that $f'|_U = f$.
\end{corollary}

\begin{definition}
In this case the function $f'$ is called an \udef{analytic continuation} of $f$ into $\Omega'$.
\end{definition}

\subsubsection{Constructing analytic continuations}
\begin{lemma}
Let $f: U\subseteq \C\to \C$ be a holomorphic function on an open set and $z_0\in U$. Let $R$ be the radius of convergence of the Taylor series of $f$ at $z_0$. Then $f$ has an analytic continuation to $U\cup \ball_{z_0, R}$, which is defined by the Taylor series on $\ball_{z_0, R}\setminus U$.
\end{lemma}
\begin{proof}
The Taylor series is holomorphic on the interior of its disk of convergence, by \ref{powerSeriesHolomorphic}.

Since $z_0\in U$ and $U$ is open, there exists a disk $\ball(z_0, \epsilon)$ whose closure lies in $U$. Then the Taylor series and $f$ coincide on this disk by \ref{holomorphicFunctionIsAnalytic}.

Since this disk contains a limit point, $f$ and the Taylor series coincide on $U\cap \ball(z_0, R)$ by \ref{holomorphicFunctionsCoincidingOnSetWithLimitPoint}.
\end{proof}

\begin{proposition}[Symmetry principle] \label{symmetryPrinciple}
Let $\Omega$ be an open subset of $\C$ such that $\overline{\Omega} = \Omega$. Denote by $\Omega^+$ the part of $\Omega$ that lies in the upper half plane and by $\Omega^-$ the part lying in the lower half plane. Set $I = \R \cap \Omega$.

Let $f^+: \Omega^+ \to \C$ and $f^-: \Omega^- \to \C$ be holomorphic functions that extend continuously to $I$ and
\[ \forall x\in I: \; f^+(x) = f^-(x). \]
Then the compound function
\[ f: \Omega \to \C: z\mapsto f(z) = \begin{cases}
f^+(z) & z\in \Omega^+ \\
f^+(z) = f^-(z) & z\in I \\
f^-(z) & z\in \Omega^-
\end{cases} \]
is holomorphic on all of $\Omega$.
\end{proposition}

\begin{proposition}[Schwarz reflection principle]
Let $\Omega, \Omega^+, \Omega^-$ and $I$ be as in \ref{symmetryPrinciple} and $f: \Omega^+ \to \C$ a holomorphic function that extends continuously to $I$. Then there exists a holomorphic function $g: \Omega\to \C$ such that $g|_{\Omega^+} = f$.
\end{proposition}
\begin{proof}
Idea: define $g|_{\Omega^-}(z) = \overline{f(\overline{z})}$. TODO details.
\end{proof}

\subsubsection{Analytic continuation along a curve}
\begin{theorem}[Monodromy theorem]
Let $f:D\subseteq \C \to \C$ be a holomorphic function and $\Omega$ the set of all points that admit an analytic continuation.

If two curves are homotopic in $\Omega$, then they give the same analytic continuation.
\end{theorem}

\subsection{Limits and holomorphic functions}
\begin{proposition}
Let $\seq{f_n}$ be a sequence of holomorphic functions on $\Omega$ that converges uniformly to a function $f$ in every compact subset of $\Omega$. Then
\begin{enumerate}
\item $f$ is holomorphic in $\Omega$;
\item $\seq{f_n'}$ converges uniformly to $f'$ on every compact subset of $\Omega$;
\item for all $k\in \N$, the sequence $\seq{f_n^{(k)}}$ converges uniformly to $f^{(k)}$ on every compact subset of $\Omega$.
\end{enumerate}
\end{proposition}
\begin{proof}
(1) TODO (+ \url{https://www.math.wustl.edu/~sk/limits.pdf})

(2) TODO

(3) By induction on $k$.
\end{proof}
TODO: this is part of the motivation for the compact-open topology (use this terminology?).

\begin{proposition}
Let $\Omega$ be a open set in $\C$ and $F: \Omega\times[0,1]\to \C$ a function such that
\begin{itemize}
\item $F(\cdot,s)$ is holomorphic (in the first variable) for all $s\in [0,1]$;
\item $F$ is continuous.
\end{itemize}
Then the function
\[ f: \Omega \to \C: z\mapsto f(z) = \int_0^1F(z,s)\diff{s} \]
is holomorphic.
\end{proposition}

\begin{proposition}[Runge's approximation theorem]
Let $K\subseteq \C$ be compact.
\begin{enumerate}
\item Any function holomorphic in a neighbourhood of $K$ can be approximated uniformly on $K$ by a sequence of rational functions whose singularities are in $K^c$.
\item If $K^c$ is connected, then the approximating functions can be taken to be polynomials.
\item If $K^c$ is not connected, then there exists a function $f$ holomorphic on a neighbourhood of $K$ that cannot be approximated uniformly by polynomials on $K$.
\end{enumerate}
\end{proposition}
If inner and outer part of annulus then Laurent series (i.e.\ particular form of approximating rationals)


\begin{proposition}[Mergelyan's theorem]
The approximating functions in Runge's approximation theorem can be taken to be polynomials \textup{if and only if} $K^c$ is connected.
\end{proposition}

\section{Singularities}
\subsection{Laurent series}
\begin{proposition}
Let $f:\Omega \to \C$ be a holomorphic function and let $\Omega$ contain two concentric circles with center $z_0$ and radii $0<r< R$. Then $f$ has a Laurent expansion
\[ f(z) = \sum_{n=-\infty}^{\infty}a_n(z-z_0)^n \]
that converges on the annulus between the concentric circles.
\end{proposition}
\begin{proof}
Call the outer circle $C_1$ and the inner $C_2$. Then we can use the integral formula to write
\[ f(z) = \frac{1}{2\pi i}\oint_{C_1}\frac{f(\zeta)}{\zeta - z}\diff{\zeta} - \frac{1}{2\pi i}\oint_{C_2}\frac{f(\zeta)}{\zeta - z}\diff{\zeta} \]
TODO
\end{proof}

\subsection{Isolated singularities}

\subsection{Non-isolated singularities}

\section{Meromorphic functions}
\subsection{Zeros and poles}
\begin{lemma} \label{holomorphicZeroLemma}
Let $f:\Omega\to \C$ be a holomorphic function on an open set $\Omega$ that has a zero at $z_0\in \Omega$ and does not vanish identically on any neighbourhood of $z_0$. Then there exists an open neighbourhood $U\subseteq \Omega$ of $z_0$ such that
\[ f|_U(z) = (z-z_0)^ng(z) \]
for all $z\in U$, some holomorphic $g:U\to \C\setminus\{0\}$ and $n\in \N$.

Additionally, the $n$ is uniquely determined by $f$ and $z_0$; it is independent of $U$.
\end{lemma}
\begin{proof}
TODO
\end{proof}

In \ref{holomorphicZeroLemma}, the restriction to $U$ is there to make sure that $g$ is nowhere zero. If all we want is for $g$ to not be zero at $z_0$, we extend it to all of $\Omega$.

\begin{lemma} \label{holomorphicZeroFactorisationLemma}
Let $f:\Omega\to \C$ be a holomorphic function on an open set $\Omega$ that has a zero at $z_0\in \Omega$.
Then
\begin{enumerate}
\item there exists a holomorphic function $g: \Omega \to \C$ such that $f(z) = (z-z_0)g(z)$ for all $z\in \Omega$;
\item if $f$ does not vanish identically on any neighbourhood of $z_0$, then there exists a holomorphic function $g: \Omega \to \C$ and $n\in \N$ such that
\begin{itemize}
\item $f(z) = (z-z_0)^ng(z)$ for all $z\in \Omega$;
\item $g(z_0) \neq 0$.
\end{itemize}
\end{enumerate}
\end{lemma}
\begin{proof}
We first prove (2), then (1).

(2) Pick some open disk around $z_0$ that lies in $\Omega$. Apply \ref{holomorphicZeroLemma} to obtain some $U,n,g'$. Take $\epsilon >0$ such that $\ball(z_0, \epsilon) \subseteq U$. On $\ball(z_0, \epsilon)$, define $g(z) \defeq g'(z)$. On $\Omega\setminus\cball(z_0, \epsilon /2)$, define $g(z) \defeq \frac{f(z)}{(z-z_0)^n}$. Where these two definitions overlap, they yield the same function. Clearly, the resulting $g$ is holomorphic and $g(z_0) = g'(z_0) \neq 0$.

(1) If $f$ does not vanish identically on any neighbourhood of $z_0$, then (1) follows from (2).

Now suppose $f$ vanishes identically on some neighbourhood $U\subseteq \Omega$. Then set $g(z) = 0$ for all $z\in U$ and $g(z) = \frac{f(z)}{z-z_0}$ for $z\neq z_0$. Where these two definitions overlap, they yield the same function. The resulting $g$ is holomorphic.
\end{proof}

\begin{definition}
Let $f:\Omega\to \C$ be a function on an open set $\Omega$ and $z_0\in \Omega$ such that $f$ is not identically zero on any neighbourhood of $z_0$.
\begin{itemize}
\item If $f$ is holomorphic with a zero at $z_0$, the $n$ in \ref{holomorphicZeroLemma} is called the \udef{multiplicity} or \udef{order} of the zero $z_0$.
\item If $\lim_{z\to z_0} 1/f(z) = 0$ and is $1/f$ holomorphic in a neighbourhood of $z_0$, then $z_0$ is called a \udef{pole} of $f$. The $n$ in the expansion \ref{holomorphicZeroLemma} of $1/f$ is called the \udef{multiplicity} or \udef{order} of the pole $z_0$.
\end{itemize}
If $n = 1$, we call the zero or pole \udef{simple}. 
\end{definition}

Poles lie in $\overline{\Omega}\setminus\Omega$. TODO sort out definition meromorphic function.

\begin{lemma}
Let $f:\Omega\to \C$ be a function on an open set $\Omega$ that has a pole at $z_0\in \overline{\Omega}$. Then there exists a neighbourhood $U$ of $z_0$ such that $f$ is holomorphic on $U\setminus\{z_0\}$.
\end{lemma}
This means a pole in necessarily an isolated singularity (TODO ref later)

\begin{definition}
Let $f: \Omega\subseteq \C \to \C$ be a function. We call $f$ \udef{meromorphic} if either $f$ or $1/f$ is holomorphic at each point $z\in \Omega$.
\end{definition}

\begin{lemma}
Let $f: \Omega\subseteq \C \to \C$ be a function. Then $f$ is meromorphic \textup{if and only if} for each point $z\in \Omega$, either $f$ is holomorphic or $z$ is a pole.
\end{lemma}

\subsection{Laurent series and residues}
\begin{proposition}
Let $f:\Omega\to \C$ be a function on an open set $\Omega$ that has a pole of order $n$ at $z_0$, then the Laurent series of $f$ at $z_0$ is of the form
\[  f(z) = \sum_{i=-n}^\infty a_i(z-z_0)^i \]
and converges on a punctured disk $B(z_0, \epsilon)\setminus\{z_0\}$.
\end{proposition}
\begin{proof}
By \ref{holomorphicZeroLemma} we have that 
\[ (1/f)|_{B(z_0,\epsilon)}(z) = (z-z_0)^ng(z) \]
for some holomorphic, non-vanishing $g$. Thus $f|_{B(z_0,\epsilon)}(z) = (z-z_0)^{-n}1/g(z)$. Now $1/g$ is holomorphic, so we can expand it as a power series. This makes the product a Laurent series starting at $-n$.

To conclude, we remark that the inner radius of the annulus on which the Laurent series converges is zero: by \ref{LaurentSeriesConvergence}
\[ r = \limsup_{i\to\infty}\left(|a_{-i}|^{1/i}\right) = 0. \]
\end{proof}
TODO: pole iff Laurent series has finite principal part!

\subsubsection{Partial fraction decomposition}
\begin{proposition}[Partial fraction decomposition] \label{partialFractionDecomposition}
Let $f: \Omega\subseteq \C \to \C$ be a meromorphic function with finitely many poles $z_0, \ldots z_n$. Let $P_{z_k}(z)$ be the principal part of the Laurent expansion around $z_k$. Then we have the decomposition
\[ f(z) = \sum_{k=0}^nP_k(z) + F(z) \]
where $F(z)$ is a holomorphic function.
\end{proposition}
\begin{proof}
The function $F(z) = f(z) - \sum_{k=0}^nP_k(z)$ has no poles and thus is holomorphic. Indeed, for all $z\in\Omega \setminus \{z_0, \ldots z_n\}$, both $f(z)$ and $\sum_{k=0}^nP_k(z)$ are holomorphic, meaning that $F(z)$ is too.

Take some pole $z_k$. Then the Laurent series of $F(z)$ around $z_k$ is of the form
\[ F(z) = P_k(z) + \sum_{i=0}^\infty a_{k,i}(z-z_k)^i - \sum_{j=0}^n P_j(z) = \sum_{i=0}^\infty a_{k,i}(z-z_k)^i - \sum_{\substack{j=0\\ j\neq k}}^n P_j(z), \]
which has no singularity at $z_k$.
\end{proof}

TODO link with integrals of fractional functions.

\subsubsection{Residues}
\begin{definition}
Let $f: \Omega\subseteq \C \to \C$ have a pole at $z_0$. Given the Laurent series expansion
\[  f(z) = \sum_{i=-n}^\infty a_i(z-z_0)^i, \]
the coefficient $a_{-1}$ is called the \udef{residue} at the pole $z_0$. We write $\Res_{z_0}f \defeq a_{-1}$.
\end{definition}

\begin{proposition}
Let $f$ have a pole of order $n$ at $z_0$, then
\[ \Res_{z_0}f = \lim_{z\to z_0}\frac{1}{(n-1)!}\left(\od{}{z}\right)^{n-1}(z-z_0)^nf(z). \]
\end{proposition}
\begin{proof}
This follows straight from the series expansion
\[ (z-z_0)^nf(z) = a_{-n} + a_{-n+1}(z-z_0) + \ldots + a_{-1}(z-z_0)^{n-1} + a_0(z-z_0)^n + \ldots \]
\end{proof}
\begin{corollary}
If $f$ has a simple pole at $z_0$, then $\Res_{z_0} f = \lim_{z\to z_0}(z-z_0)f(z)$.
\end{corollary}

\begin{proposition}
Let $f: \Omega\subseteq \C\to \C$ be a meromorphic function that has one pole at $z_0$. Let $\gamma$ be a simple curve such that $z_0\in \int\gamma \subseteq \Omega$. Then
\[ \oint_\gamma f(z) \diff{z} = 2\pi i \Res_{z_0}f. \]
\end{proposition}
\begin{proof}
By truncated Laurent expansion. TODO
\end{proof}
\begin{corollary}[Residue formula] \label{residueFormula}
Let $f: \Omega\subseteq \C\to \C$ be a meromorphic function and $\gamma$ a simple curve that encompasses $N$ poles $z_1, \ldots, z_N$. Then
\[ \oint_{\gamma} f(z)\diff{z} = 2\pi i \sum_{k=1}^N\Res_{z_k}f. \]
\end{corollary}

\subsection{The argument principle}
\begin{proposition}[Argument principle]
Let $f$ be a meromorphic function in an open set containing a simple curve $\gamma$ and its interior. Assume $f$ has no poles or zeros on $\gamma$. Then
\[ \frac{1}{2\pi i}\oint_\gamma \frac{f'(z)}{f(z)}\diff{z} = \text{number of zeros of $f$ inside $\gamma$} \;-\; \text{number of poles of $f$ inside $\gamma$} \]
where the poles and zeros are counted with their multiplicities.
\end{proposition}
\begin{proof}
TODO
\end{proof}

\begin{theorem}[Rouché's theorem] \label{RoucheTheorem}
Let $f$ and $g$ be holomorphic functions on an open set that contains a closed simple curve $\gamma$ and its interior. If
\[ |f(z)| > |g(z)| \qquad \forall z\in \gamma, \]
then $f$ and $f+g$ have the same number of zeros inside $\gamma$.
\end{theorem}
\begin{proof}
For $t\in [0, 1]$ we define
\[ f_t(z) = f(z) + tg(z). \]
This means $f_0 = f$ and $f_1 = f+g$. The condition $|f(z)| > |g(z)|$ ensures that $f_t$ has no zeros on $\gamma$. Also $f_t$ is holomorphic and thus has no poles. So the number of zeros $n_t$ of $f_t$ inside $\gamma$ is given by
\[ n_t = \frac{1}{2\pi i}\oint_\gamma \frac{f_t'(z)}{f_t(z)}\diff{z}. \]
It is then enough to observe that $n_t$ is continuous as a real function of $t$.
\end{proof}

\begin{theorem}[Open mapping theorem] \label{openMappingTheorem}
Let $f: U\subseteq \C\to \C$ a holomorphic function defined on a region. Then either $f$ is constant or an open map.
\end{theorem}
\begin{proof}
Suppose $f$ is not constant. We then need to prove that $f$ is open. Take some open subset $W\subseteq U$ and $w\in W$. We show that there exists an open neighbourhood of $f(w)$ that is a subset of $f^\imf(W)$.

Consider the function $f - \constant{f(w)}$, which is holomorphic.
By \ref{zerosHolomorphicFunctionIsolated}, we can find an open ball $\ball(w, \delta)$ such that $w$ is the only root of $f - \constant{f(w)}$ in $\cball(w, \delta)$. We can take $\delta$ small enough that $\cball(w, \delta) \subseteq W$.

Now $\sphere(w,\delta)$ is closed and bounded, and thus compact by Heine-Borel (TODO ref), so $|f - \constant{f(w)}|$ attains a minimum $\epsilon \geq 0$ on $\sphere(w,\delta)$ by the extreme value theorem \ref{extremeValueTheorem}. This minimum is not zero by construction, so $\epsilon > 0$.

Finally we claim $\ball\big(f(w), \epsilon\big) \subseteq f^\imf(W)$. Take $w'\in \ball\big(f(w), \epsilon\big)$ and consider the function
\[ f - \constant{w'} = \big(f - \constant{f(w)}\big) + \big(\constant{f(w)} - \constant{w'}\big) \]
defined on $\cball(w, \delta)$.

Since $\big|f(z) - w\big| \geq \epsilon > \big|f(w) - w'\big|$, we can apply Rouché's theorem \ref{RoucheTheorem}: $f - \constant{w'}$ has one zero inside $\cball(w,\delta)$ since $f - \constant{f(w)}$ does, and thus $w'\in f^\imf(W)$.
\end{proof}


\subsubsection{Maximum modulus results}
\begin{proposition}[Maximum modulus principle] \label{maximumModulusPrinciple}
Let $f: U\subseteq \C\to \C$ a holomorphic function defined on a region. If $|f|$ attains a maximum on $U$ \textup{if and only if} $f$ is constant.
\end{proposition}
\begin{proof}
If $f$ is a constant, then $|f|$ clearly attains a maximum.

Now suppose $|f|$ attains a maximum at $z_0$. By the open mapping theorem, \ref{openMappingTheorem}, either $f$ is constant or open. Assume, towards a contradiction, that $f$ is open. Then $\im(f)$ contains a ball around $f(z_0)$ and so $|f(z_0)|$ is not a maximum of $|f|$.
\end{proof}
\begin{corollary}
Let $U$ be a region with compact closure $\overline{U}$. If $f$ is holomorphic on $U$ and continuous on $\overline{U}$, then
\[ \sup_{z\in U}|f(z)| \leq \max_{z\in \overline{U}\setminus U}|f(z)|. \]
\end{corollary}
\begin{proof}
Since $\overline{U}$ is compact, $|f|$ attains a maximum on $\overline{U}$ by the extreme value theorem \ref{extremeValueTheorem}. This maximum cannot lie in $U$ by the maximum modulus principle, so it lies on $\overline{U}\setminus U$.
\end{proof}

\begin{proposition}[Phragmén-Lindelöf principle]
Let $S'_\epsilon\subseteq S\subseteq \C$ be regions for all $\epsilon > 0$ with each $S'_\epsilon$ bounded, $M\geq 0$ and $f$ a function that is holomorphic on $S$ and continuous on $\overline{S'}$. Let $\{h_\epsilon\}_{\epsilon > 0}$ be set of functions such that
\begin{itemize}
\item $h_\epsilon \to \constant{1}_{\overline{S}}$ pointwise as $\epsilon \to 0$;
\item $\sup_{z\in \partial S'_\epsilon}\big|fh_\epsilon(z)\big| \leq M$ for all $\epsilon > 0$;
\item $\sup_{z\in S\setminus \overline{S'_\epsilon}}\big|fh_\epsilon(z)\big| \leq M$  for all $\epsilon > 0$.
\end{itemize}
Then $\sup_{z\in S}\big|f(z)\big| \leq M$.
\end{proposition}
The last two points are implied by $\sup_{z\in \overline{S}\setminus S'_\epsilon}\big|fh_\epsilon(z)\big| \leq M$ for all $\epsilon > 0$.
\begin{proof}
Take $\epsilon >0$. Then $fh_\epsilon$ is bounded by $M$ on $\overline{S'_\epsilon}$ by the maximum modulus principle \ref{maximumModulusPrinciple}. It is bounded by $M$ on $S\setminus \overline{S'_\epsilon}$ by assumption.

Since, for all $z\in S$, $|h_\epsilon f(z)| \leq M$ and $|h_\epsilon f(z)| \to |f(z)|$ as $\epsilon \to 0$, we have $|f(z)|\leq M$.
\end{proof}
\begin{corollary}
Let $0< \alpha < \pi$. Let $f$ be a function that is holomorphic on the open sector $S = \setbuilder{z}{-\alpha < \arg(z) < \alpha}$ and continuous on its boundary. If
\begin{itemize}
\item $|f(z)|\leq 1$ for all $z\in \partial S$;
\item $|f(z)| \leq Ce^{c|z|^\rho}$ for some $c, C > 0$ and $\rho \in \interval[co]{0, \frac{\pi}{2\alpha}}$;
\end{itemize}
then $|f(z)| \leq 1$ for all $z\in S$.
\end{corollary}
\begin{proof}
Take $\rho'\in \interval[o]{\rho, \frac{\pi}{2\alpha}}$ and set $h_\epsilon(z) = e^{-\epsilon z^{\rho'}}$, where we take $\big(re^{i\theta}\big)^{\rho'} = r^{\rho'}e^{i\rho'\theta}$ for $-\pi < \theta <\pi$.

Now $|h_\epsilon(z)| = e^{-\epsilon r^{\rho'}\cos(\theta\rho')}$. If $z = re^{i\theta}\in S$, then $-\alpha < \theta < \alpha$ and $0 < \rho' < \frac{\pi}{2\alpha}$ imply 
\[ \frac{-\pi}{2} = -\alpha\frac{\pi}{2\alpha} < -\alpha\rho' < \theta\rho' < \alpha\rho' < \alpha\frac{\pi}{2\alpha} = \frac{\pi}{2}, \]
so $0< \cos(\alpha\rho') < \cos(\theta\rho')$. Then we calculate (using $\rho'-\rho > 0$)
\begin{align*}
\left(\frac{c+1}{\epsilon\cos(\rho'\alpha)}\right)^{\frac{1}{\rho'-\rho}} \leq r &\implies \frac{c+1}{\epsilon\cos(\rho'\alpha)} \leq r^{\rho' -\rho} \\
&\implies \frac{c+1}{\epsilon\cos(\rho'\theta)} \leq r^{\rho' -\rho} \\
&\implies c+1 \leq \epsilon\cos(\rho'\theta)r^{\rho' -\rho} \\
&\implies c - \epsilon\cos(\rho'\theta)r^{\rho' -\rho} \leq -1.
\end{align*}
Thus for all $r\geq \left(\frac{c+1}{\epsilon\cos(\rho'\alpha)}\right)^{\frac{1}{\rho'-\rho}}$ we have
\[ |fh_\epsilon(re^{i\theta})| \leq Ce^{cr^\rho} \cdot e^{-\epsilon r^{\rho'}\cos(\theta\rho')} = Ce^{cr^\rho - \epsilon \cos(\theta\rho') r^{\rho'}} = Ce^{\big(c - \epsilon\cos(\rho'\theta)r^{\rho' -\rho}\big)r^\rho} \leq Ce^{-r^\rho}. \]
So, if $r \geq \max\left\{\left(\frac{c+1}{\epsilon\cos(\rho'\alpha)}\right)^{\frac{1}{\rho'-\rho}}, \big(\ln C\big)^{1/\rho}\right\}$, then 
\[ |fh_\epsilon(re^{i\theta})| \leq Ce^{-r^\rho} \leq Ce^{-\big((\ln C)^{1/\rho}\big)^\rho} = Ce^{-\ln C} = CC^{-1} = 1. \]
Now set $S'_\epsilon \defeq \setbuilder{z\in \C}{-\alpha < \arg(z) < \alpha, |z| < \max\left\{\left(\frac{c+1}{\epsilon\cos(\rho'\alpha)}\right)^{\frac{1}{\rho'-\rho}}, \big(\ln C\big)^{1/\rho}\right\}}$. This satisfies the requirements for the Phragmèn-Lindelöf principle.
\end{proof}
\begin{corollary}[Hadamard three-lines lemma]
Let $f$ be a function that is holomorphic and bounded on the open strip $S = \setbuilder{z}{0 < \Im(z) < 1}$ and continuous on its boundary. Set $M_0 = \sup_{x\in \R}|f(x)|$ and $M_1 = \sup_{x\in \R}|f(x+i)|$. Then
\[ \sup_{x\in \R}|f(x+iy)| \leq M_0^{1-y}M_1^y \]
for all $0\leq y\leq 1$.
\end{corollary}
\begin{proof}
Let $M \defeq \sup_{z\in S}|f(z)|$. We apply the Phragmèn-Lindelöf principle to the function $F(z) \defeq M_0^{-1-iz}M_1^{iz}f(z)$, with $h_\epsilon(z) \defeq e^{-\epsilon z^2}$, to obtain that $\sup_{z\in S}|F(z)| \leq 1$.

For all $x\in \R$, we have
\[ |M_0^{-1-ix}M_1^{ix}f(x)h_\epsilon(x)| \leq |M_0^{-1-ix}M_1^{ix}|M_0 e^{-\epsilon x^2} \leq M_0M_0^{-1} = 1 \]
and
\[ |M_0^{-1-i(x+i)}M_1^{i(x+i)}f(x+i)h_\epsilon(x+i)| \leq |M_0^{-ix}M_1^{ix -1}|M_1 e^{-\epsilon |x+i|^2} \leq M_1^{-1}M_1 =  1. \]
Now set $S'_\epsilon \defeq \setbuilder{z\in \C}{0<\Im(z)< 1; -\sqrt{\frac{\ln M + \epsilon}{\epsilon}} < x < \sqrt{\frac{\ln M + \epsilon}{\epsilon}}}$. To see that this satisfies the Phragmèn-Lindelöf principle, take $z = x+iy \in S\setminus S_\epsilon'$.
Then we have the implications
\[ \frac{\ln M + \epsilon}{\epsilon} \leq x^2 \quad\implies\quad Me^{\epsilon} \leq e^{\epsilon x^2} \quad\implies\quad Me^{-\epsilon x^2 + \epsilon} \leq 1 \]
and so
\begin{align*}
\big|M_0^{-1-iz}M_1^{iz}f(z)h_\epsilon(z)\big| &= \big|M_0^{-1-ix +y}M_1^{ix - y}f(z)e^{-\epsilon (x^2 + 2ixy - y^2)}\big| \\
&= M_0^{-1 +y}M_1^{- y}\big|f(z)\big|e^{-\epsilon x^2 + \epsilon y^2} \\
&\leq M_0^{-1 +1}M_1^{0}Me^{-\epsilon x^2 + 1} \\
&= Me^{-\epsilon x^2 + 1} \leq 1.
\end{align*}
We conclude that $1 \geq \sup_{z\in S}|F(z)|$, so for all $y\in \interval{0,1}$, we have
\[ 1 \geq \sup_{x \in \R}|F(x+iy)| = \sup_{x \in \R}|M_0^{-1-ix+y}M_1^{ix-y}f(x+iy)| = M_0^{y-1}M_1^{-y}\sup_{x \in \R}|f(x+iy)| \]
and so $|f(x+iy)| \leq M_0^{1-y}M_1^{y}$.
\end{proof}


\subsection{The ring of polynomials over $\meromorphic_\Omega$}

\begin{definition}
Let $\Omega\subseteq \C$ be an open set. Let $\meromorphic_\Omega$ be the set of meromorphic functions in $(\Omega \to \C)$.
\end{definition}

\begin{proposition}
For any $\Omega \subseteq \C$, the set $\meromorphic_\Omega$ is a field. With
\begin{itemize}
\item as zero the constant function $\underline{0}$;
\item as identity the constant function $\underline{1}$.
\end{itemize}
\end{proposition}

\subsubsection{Poles and roots}
\begin{definition}
We say a polynomial in $\meromorphic_\Omega[X]$ has a pole at $z\in \Omega$, if one of its coefficients has a pole at $z$.
\end{definition}

\begin{lemma} \label{polesProductPolynomial}
Let $p,q\in \meromorphic_\Omega[X]$ be monic and $r = pq$. If either $p$ or $q$ has a pole at $z_0\in \Omega$, then $r$ has a pole at $z_0$ as well.
\end{lemma}
\begin{proof}
Let $h$ be the maximal order of $z_0$ as a pole among the coefficients of $p$ and $k$ the maximal order of $z_0$ as a pole among the coefficients of $q$. Then both $\lim_{z\to z_0}(z-z_0)^h p(z)$ and $\lim_{z\to z_0}(z-z_0)^k q(z)$ are non-zero polynomials (due to being monic) with constant coefficients and $h+k \geq 1$. 

Now assume $r(z)$ is holomorphic at $z_0$, then
\[ 0 = \lim_{z\to z_0}(z-z_0)^{h+k}r(z) = \Big(\lim_{z\to z_0}(z-z_0)^h p(z)\Big)\Big(\lim_{z\to z_0}(z-z_0)^k q(z)\Big) \neq 0, \]
which is a contradiction.
\end{proof}
\begin{corollary}
Let $p \in \meromorphic_\Omega[x]$ be monic and have a prime decomposition $p = \prod_{k=1}^nq_k^{m_k}$. Then $z_0\in \Omega$ is a pole of $p(z)$ \textup{if and only if} it is a pole of one of the $q_k(z)$.
\end{corollary}

\begin{proposition}
Let $p,q\in \meromorphic_\Omega[x]$ be monic and relatively prime, then $p(z)$ and $q(z)$ are relatively prime as polynomials in $\C[X]$ for all $z\in \Omega \setminus (S_p \cup S_q \cup H)$, where
\begin{itemize}
    \item $S_p$ is the set of poles of $p$;
    \item $S_q$ is the set of poles of $q$;
    \item $H$ is isolated and closed in $\Omega$.
\end{itemize}
\end{proposition}
\begin{proof}
By Bézout's identity, we can write $\alpha[x](z)p[x](z) + \beta[x](z)q[x](z) = f(z)$ for some monic $\alpha,\beta\in \meromorphic_\Omega[x]$. Let $H$ be the set of poles and zeros of $f(z)$ (which contains $S_p \cup S_q$ by \ref{polesProductPolynomial}). For all $z_0\notin H$ we have
\[ \left(\frac{\alpha[x](z_0)}{f(z_0)}\right)p[x](z_0) + \left(\frac{\beta[x](z_0)}{f(z_0)}\right)q[x](z_0) = 1, \]
meaning $p[x](z_0)$ and $q[x](z_0)$ are relatively prime (TODO ref).
\end{proof}
\begin{corollary}
Let $p\in \meromorphic_\Omega[x]$ be monic and irreducible. Then $p(z)$ has simple roots for all $z\in\Omega\setminus H$ for some isolated and closed set $H\subseteq \Omega$.
\end{corollary}
\begin{proof}
If the degree of $p$ is $1$, then the result is immediate. Assume the degree of $p$ is greater than $1$.

For any $z\in \Omega$, any non simple root of $p[x](z)$ must also be a root of $\pd{p[x](z)}{x}$. As $p[x]$ is irreducible, $p[x]$ and $\pd{p[x]}{x}$ are relatively prime. By the proposition $p[x](z_0)$ and $\pd{p[x](z_0)}{x}$ are relatively prime for all $z_0\in \Omega\setminus H$, which means they do not share roots.
\end{proof}

\subsubsection{Algebroid functions}

\subsubsection{Puiseux series}

\section{Conformal mappings}

\section{Special functions}
\subsection{The gamma function}

\begin{proposition} \label{gammaFunction}
Consider the function defined by
\[ \Gamma: \C^{\rightarrow} \to \C: z \mapsto \int_0^\infty e^{-t}t^{z-1}\diff{t},  \]
where $\C^{\rightarrow}$ is the open right half plane $\setbuilder{z\in \C}{\Re(z) > 0}$. Then this function
\begin{enumerate}
\item is well-defined, i.e.\ the integral converges for all points in its domain;
\item is analytic;
\item satisfies $\Gamma(z+1) = z\Gamma(z)$ for all $z \in \C^{\rightarrow}$;
\item has an analytic continuation to a meromorphic function on 
$\C$ whose only singularities are simple poles at the negative integers $-\N$. The residue of $\Gamma$ at $z = -n$ is $(-1)n /n!$.
\end{enumerate}
\end{proposition}
\begin{proof}
(1) TODO: consider two parts $\int_0^a e^{-t}t^{z-1}\diff{t} + \int_a^\infty e^{-t}t^{z-1}\diff{t}$, where $a$ is such that $t^{z-1} \leq e^{t/2}$ for all $t \geq a$.

(2) TODO

(3) We use integration by parts.
\end{proof}

\begin{definition}
The \udef{gamma function} is the function $\Gamma: \C\setminus (-\N) \to \C$ defined by the analytic continuation in \ref{gammaFunction}.
\end{definition}

\begin{proposition}
Laurent series of the gamma function for $0< |z| < 1$: 
\[ \Gamma(z) = \frac{1}{z} - \gamma + \Big(\frac{\gamma^2}{2} \frac{\pi^2}{12}\Big)z + \ldots, \]
where $\gamma \defeq \int_0^\infty e^{-t}\log\big(\frac{1}{t}\big)\diff{t}$ is \udef{Euler's constant}.
\end{proposition}
\begin{proof}
Plugging in Taylor series
\[ t^z = e^{z\log(t)} = 1 + zlog(t) + \frac{z^2}{z}\log(t)^2 + \ldots. \]
Check dominated convergence!
\end{proof}

\begin{proposition}
As $z\to \infty$, the gamma function has the following asymptotic expansion:
\[ \Gamma(z) \sim e^{-z}z^{z-1/2}\sqrt{2\pi}\Big(1 + \frac{1}{12z} + \frac{1}{288z^2} + \ldots \Big). \]
\end{proposition}
The first term of this expansion is known as \udef{Stirling's formula}.
\begin{proof}
TODO
\end{proof}

\section{Banach $*$-algebras of complex-valued functions}
\begin{lemma}
Let $\sSet{X,\xi}$ be a convergence space. Then
\begin{enumerate}
\item $\sup_{x\in X}|f| <\infty$ for all $f\in \cont_0(X)$;
\item if $X$ is compact, then $\sup_{x\in X}|f| <\infty$ for all $f\in \cont(X)$.
\end{enumerate}
\end{lemma}
\begin{proof}
(1) Take arbitrary $f\in \cont_0(X)$. Take $\cball_\C(1,0)$, which is a neighbourhood of $0$. Then there exists a compact $K\subseteq X$ such that $f^{\imf}(K^c) \subseteq \cball_\C(0,1)$, so $\sup |f^{\imf}(K^c)| \leq 1$.

Now $\sup_{x\in X}|f| = \max\big\{\sup |f^{\imf}(K^c)|, \sup |f^{\imf}(K)|\big\}$ and $\sup |f^{\imf}(K)| = \max |f^\imf(K)| < \infty$ by the extreme value theorem \ref{extremeValueTheorem}.

(2) Follows immediately from the extreme value theorem \ref{extremeValueTheorem}.
\end{proof}

\begin{definition}
Let $\sSet{X,\xi}$ be a convergence space. Then
\begin{itemize}
\item we equip $\cont_0(X)$ with the supremum norm, which makes it a Banach-$*$-algebra;
\item if $X$ is compact, then we equip $\cont(X)$ with the supremum norm, which makes it a Banach-$*$-algebra.
\end{itemize} 
\end{definition}

\begin{proposition} \label{unitisationOnePointCompactificationIsomorphism}
Let $\sSet{X,\xi}$ be a convergence space. Then the function
\[ \Phi: \proj_1(-)^\dagger + \constant{\proj_2(-)}_{X^\dagger}: \cont_0(X)^\dagger \to \cont(X^\dagger): (f, \lambda) \mapsto f^\dagger + \constant{\lambda}_{X^\dagger}. \]
is a unital $*$-algebra isomorphism.
\end{proposition}
\begin{proof}
The function is well-defined, since $f^\dagger$ is continuous by \ref{vanishesAtInfinityIffBasepointExtensionContinuous}. It is a bijection, since its inverse is given by
\[ \cont(X^\dagger) \to \cont_0(X)^\dagger: f\mapsto \big(f|_X - \constant{f(\infty)}_X, f(\infty)\big). \]
The fact that it is a $*$-algebra homomorphism follows from \ref{basepointExtensionToRingLemma}.

Unitality follows from $\big(\constant{0}|_X\big)^\dagger = \constant{0}_{X^\dagger}$ (\ref{basepointExtensionToRingLemma}).
\end{proof}

\begin{lemma} \label{compactCharactersLemma}
Let $X$ be a compact topological space. Set
\begin{itemize}
\item $I_x \defeq \setbuilder{f\in \cont(X)}{f(x) = 0}$ for all $x\in X$;
\item $N(I)\defeq \setbuilder{x\in X}{\forall f\in I: \; f(x) = 0}$ for all ideals $I\subseteq \cont(X)$.
\end{itemize}
Then we have
\begin{enumerate}
\item $N(I) = \emptyset$ \textup{if and only if} $I = \cont(X)$;
\item if $I\subseteq \cont(X)$ is a maximal ideal, then $I = I_x$ for some $x\in X$.
\end{enumerate}
\end{lemma}
\begin{proof}
(1) First assume $N(I) = \emptyset$. Then $\big\{f^{\preimf}(\C\setminus\{0\})\big\}_{f\in I}$ is an open cover of $X$. We can take a finite subcover $C$ by \ref{topologyCompactnessOpenCover}. Now set
\[ g \defeq \sum_{f^{\preimf}(\C\setminus\{0\})\in C} f^2. \]
By construction, $g\in I$ (since $I$ is an ideal) and $g$ is invertible (since it is nowhere $0$), thus $I = \cont(X)$ by \ref{properIdealNoUnit}.

Now suppose $I = \cont(X)$. Then $\constant{1}\in I$ and $\constant{1}(x) \neq 0$ for all $x\in X$, so $N(I)=\emptyset$.

(2) Let $I\subseteq \cont(X)$ be a maximal (proper) ideal. Then $N(I) \neq \emptyset$, so there exists $x\in N(I)$. It is clear that $I\subseteq I_x$. By maximality, we have $I = I_x$.
\end{proof}

See also \url{https://math.stackexchange.com/questions/87277/proof-that-ideals-in-c0-1-are-of-the-form-m-c-that-should-not-involve-zorn}.


\begin{proposition}  \label{charactersFunctionAlgebraCompactSpace}
Let $X$ be a Hausdorff topological space. Then
\begin{enumerate}
\item if $X$ is compact, then $\evalMap_{-}|_{\cont(X)}: X \to \widehat{\cont(X)}$ is a homeomorphism;
\item if $X$ is locally compact, then $\evalMap_{-}|_{\cont_0(X)}: X \to \widehat{\cont_0(X)}$ is a homeomorphism.
\end{enumerate}
\end{proposition}
\begin{proof}
(1) For all $x\in X$, we clearly have $\evalMap_x\in \widehat{\cont(X)}$. 

We first show that $\evalMap_{-}|_{\cont(X)}$ is injective. Suppose $x\neq y \in X$. Since $X$ is Hausdorff, $\{x\}$ and $\{y\}$ are disjoint closed sets by \ref{FrechetCharacterisation}. Since $X$ is compact Hausdorff, it is normal by \ref{compactHausdorffSpacesNormal}, so we can find $f\in \cont(X, \R)$ such that $f(x) = 0$ and $f(y) = 1$ by Urysohn's lemma \ref{UrysohnLemma}. This function is still continuous if we extend the codomain to $\C$ by \ref{continuityRestrictionExpansion}.  Then
\[ \evalMap_x(f) = f(x) = 0 \neq 1 = f(y) = \evalMap_y(f), \]
so $\evalMap_x|_{\cont(X)} \neq \evalMap_y|_{\cont(X)}$.

To show surjectivity, take $\varphi\in\widehat{\cont(X)}$. Then $\ker(\varphi)$ is a maximal ideal, see \ref{characterMaximalIdealsComplex}. Thus it is equal to $I_x = \ker\big(\evalMap_x|_{\cont(X)}\big)$ for some $x\in X$ by \ref{compactCharactersLemma}. This means that $\varphi = \lambda \evalMap_x$ for some $\lambda\in C$. Since, by \ref{charactersUnital}, $1 = \varphi(\constant{1}) = \lambda\evalMap_x(\constant{1}) = \lambda$, we have $\lambda = 1$ and $\varphi = \evalMap_x|_{\cont(X)}$.

Now $\evalMap_{-}|_{\cont(X)}$ is continuous by \ref{characteristicPropertyInitialFinalConvergence}, since, for all $f\in \cont(X)$,
\[ \begin{tikzcd}
\widehat{\cont(X)} \ar[r, "\evalMap_f"] & \C \\ X \ar[u, "{\evalMap_{-}|_{\cont(X)}}"] \ar[ur, swap, "f"]
\end{tikzcd} \qquad\text{commutes,} \]
which is true because, for all $x\in X$,
\[ (\evalMap_f\circ \evalMap_{-}|_{\cont(X)})(x) = \evalMap_f\big(\evalMap_x\big) = \evalMap_x(f) = f(x). \]
Since $\widehat{\cont(X)}$ is Hausdorff, by \ref{weakTopologyLCTVS}, we have that $\evalMap_{-}|_{\cont(X)}$ is a homeomorphism by \ref{compactToHausdorffHomeomorphism}.

(2) We first show that $\evalMap_{-}|_{\cont_0(X)}$ is injective. Suppose $x\neq y \in X$. Now $\topMod(X^\dagger)$ is compact and Hausdorff by \ref{onePointCompactificationHausdorff}, and thus normal by \ref{compactHausdorffSpacesNormal}. The sets $\{x\}$, $\{y\}$ and $\{\infty\}$ are disjoint closed sets by \ref{FrechetCharacterisation}.

By the Tietze extension theorem \ref{TietzeExtension}, we can extend the function
\[ \{x, y, \infty\}\to \R: z\mapsto \begin{cases}
1 & (z = x) \\ 2 & (z = y) \\ 0 & (z = \infty)
\end{cases} \]
to a continuous function. This extension is still continuous if we extend the codomain to $\C$ by \ref{continuityRestrictionExpansion}. Call this function $f$. It is also continuous as a function $f: X^\dagger \to \C$, since $X^\dagger$ is stronger than $\topMod(X^\dagger)$. Now $f|_X$ is continuous by \ref{functionVanishingAtInftyIffRestrictionOfContinuousBasepointPreservingFunction}, so 
\[ \evalMap_x|_{\cont_0(X)}\big(f|_X\big) = f(x) = 1 \neq 2 = f(y) = \evalMap_y|_{\cont_0(X)}\big(f|_X\big), \]
so $\evalMap_x|_{\cont_0(X)} \neq \evalMap_y|_{\cont_0(X)}$.

To show that $\evalMap_{-}|_{\cont_0(X)}$ is surjective, take arbitrary $\varphi\in\widehat{\cont_0(X)}$. Then we have a unital extension $\varphi^\dagger\in \widehat{\cont_0(X)^\dagger}$. Take the isomorphism $\Phi: \cont_0(X)^\dagger \to \cont(X^\dagger)$ of \ref{unitisationOnePointCompactificationIsomorphism} and consider $\varphi^\dagger\circ \Phi^{-1}: \cont(X^\dagger)\to \C$, which is an algebra homomorphism by construction and nonzero because it is unital. So $\varphi^\dagger\circ \Phi^{-1}\in \widehat{\cont(X^\dagger)}$ and $\varphi^\dagger\circ \Phi^{-1} = \evalMap_a|_{\cont_0(X)}$ for some $a\in X^\dagger$ by (1).

Take arbitrary $f\in \cont_0(X)$. Then $f^\dagger \in \cont(X^\dagger)$ by \ref{vanishesAtInfinityIffBasepointExtensionContinuous} and we calculate
\begin{align*}
(\varphi^\dagger\circ \Phi^{-1})(f^\dagger) &= \varphi^\dagger\big(f^\dagger|_X - \constant{f^\dagger(\infty)}, f^\dagger(\infty)\big) \\
&= \varphi^\dagger\big(f^\dagger|_X, 0\big) \\
&= \varphi^\dagger(f, 0) = \varphi(f) + 0 = \varphi(f).
\end{align*}

Now suppose, towards a contradiction, that $a = \infty$. Then, for all $f\in \cont_0(X)$ we have
\[ \varphi(f) = (\varphi^\dagger\circ \Phi^{-1})(f^\dagger) = \evalMap_\infty(f^\dagger) = 0, \]
which is a contradiction since $\varphi \neq \constant{0}$. So $a\in X$ and we have
\[ \varphi(f) = (\varphi^\dagger\circ \Phi^{-1})(f^\dagger) = \evalMap_a(f^\dagger) = \evalMap_a(f) \]
for all $f\in \cont_0(X)$, which shows that $\evalMap_{-}|_{\cont_0(X)}$ is surjective.

Finally we need to show that $\evalMap_{-}|_{\cont_0(X)}$ and its inverse is continuous. By \ref{LCTopologicalHausdorffInitialConvergence}, $X$ has the initial convergence w.r.t. $\cont_0(X)$, so, for all $x\in X$ and $F\in \powerfilters(X)$, we have
\begin{align*}
F \overset{X}{\longrightarrow} x &\iff \forall f\in \cont_0(X): \; \upset f^{\imf\imf}(F)\to f(x) \\
&\iff \forall f\in \cont_0(X): \; \upset \big(\evalMap_{-}|_{\cont_0(X)}\big)^{\imf\imf}(F)(f) \to \big(\evalMap_{-}|_{\cont_0(X)}\big)(f)(x) \\
&\iff \upset \big(\evalMap_{-}|_{\cont_0(X)}\big)^{\imf\imf}(F) \overset{\widehat{\cont_0(X)}}{\longrightarrow} \big(\evalMap_{-}|_{\cont_0(X)}\big)(f),
\end{align*}
which concludes the proof that $\evalMap_{-}|_{\cont_0(X)}$ is a homeomorphism.
\end{proof}


\begin{proposition}[Gelfand-Kolmogorov]
Let $X,Y$ be compact topological spaces and $h: \cont(X)\to \cont(Y)$ a unital algebra homomorphism. Then
\begin{enumerate}
\item there exists a function $g: Y\to X$ such that $h$ equals the pre-composition $g^\star$;
\begin{enumerate}
\item this $g$ is continuous;
\item this $g$ is unique if $X$ is Hausdorff;
\end{enumerate}
\item if $h$ is an isomorphism, then $g$ is a homeomorphism.
\end{enumerate}
\end{proposition}
Note that we do not assume $h$ to be continuous w.r.t. any convergence.

TODO: clean up calculations.
\begin{proof}
(1) Let both $\cont(X)$ and $\cont(Y)$ have the algebraic convergence, then $h: \cont(X)\to \cont(Y)$ is continuous. By  \ref{adjointContinuousFunction}, $h$ has a unique adjoint $h^\star: \dual{\cont(Y)} \to \dual{\cont(X)}$. We claim that $\im\big(h^\star|_{\widehat{\cont(Y)}}\big) \subseteq \widehat{\cont(X)}$, so we can consider the function $h^\star: \widehat{\cont(Y)}\to \widehat{\cont(X)}$. Indeed, for all $\varphi\in \widehat{\cont(Y)}$, we have that $h^\star(\varphi) = \varphi \circ h$ is an algebra homomorphism. It is also non-zero because $h^\star(\varphi)(\constant{1}) = \varphi\big(h(\constant{1})\big) = \varphi(\constant{1}) = 1$, by \ref{charactersUnital}.

For all $f\in \cont(X)$, consider the diagram
\[ \begin{tikzcd}
\widehat{\cont(X)} \ar[r, "\evalMap_f"] & \C \\
\widehat{\cont(Y)} \ar[u, "h^\star"] & {} \\
Y \ar[u, "{\evalMap_{-}|_{\cont(Y)}}"] \ar[uur, swap, "h(f)"]
\end{tikzcd}, \]
which commutes, because, for all $y\in Y$, we have
\[ \big(\evalMap_f \circ h^\star \circ \evalMap_{-}|_{\cont(Y)}\big)(y) = \evalMap_f\big(h^\star (\evalMap_y)\big) = \evalMap_f(\evalMap_y \circ h) = (\evalMap_y\circ h)(f) = h(f)(y). \]
Since $h(f)$ is a character and thus continuous by \ref{charactersUnital}, we have that $h^\star \circ \evalMap_{-}|_{\cont(Y)}$ is continuous by \ref{characteristicPropertyInitialFinalConvergence}.

Then, since $\evalMap_{-}|_{\cont(X)}$ is invertible by \ref{charactersFunctionAlgebraCompactSpace}, we can set $g \defeq \big(\evalMap_{-}|_{\cont(X)}\big)^{-1} \circ h^\star \circ \evalMap_{-}|_{\cont(Y)}: Y\to X$. This is also continuous by \ref{charactersFunctionAlgebraCompactSpace}.

We now show that $h = g^\star$. First observe that for arbitrary $f\in \cont(X)$, we have $f\circ \big(\evalMap_{-}|_{\cont(X)}\big)^{-1} = \evalMap_f$. Indeed, each character in $\widehat{\cont(X)}$ is of the form $\evalMap_x$ for some $x\in X$ by \ref{charactersFunctionAlgebraCompactSpace}, so
\[ \Big(f\circ \big(\evalMap_{-}|_{\cont(X)}\big)^{-1}\Big)(\evalMap_x) = \Big(f\circ \big(\evalMap_{-}|_{\cont(X)}\big)^{-1} \circ \evalMap_{-}|_{\cont(X)}\Big)(x) = f(x) = \evalMap_x(f) = \evalMap_f\big(\evalMap_x\big). \]

By a previous calculation, we have
\[ g^\star(f) = f\circ g = f\circ (\evalMap_{-}|_{\cont(X)})^{-1} \circ h^\star \circ \evalMap_{-}|_{\cont(Y)} = \evalMap_f \circ h^\star \circ \evalMap_{-}|_{\cont(Y)} = h(f). \]

Finally, suppose, towards a contradiction, that $g_1^\star = h = g_2^\star$ but $g_1 \neq g_2$. Then there exists $y\in Y$ such that $g_1(y)\neq g_2(y)$. Since $X$ is Hausdorff, $\{g_1(y)\}$ and $\{g_2(y)\}$ are disjoint closed sets by \ref{FrechetCharacterisation}. Since $X$ is compact Hausdorff, it is normal by \ref{compactHausdorffSpacesNormal}, so we can find $f\in \cont(X, \R)$ such that $f\big(g_1(y)\big) = 0$ and $f\big(g_2(y)\big) = 1$ by Urysohn's lemma \ref{UrysohnLemma}. This function is still continuous if we extend the codomain to $\C$ by \ref{continuityRestrictionExpansion}.
Now
\[ g_1^\star(f)(y) = (f\circ g_1)(y) = f\big(g_1(y)\big) = 0 \neq 1 = f\big(g_2(y)\big) = (f\circ g_2)(y) = g_2^\star(f)(y), \]
so $g_1^\star(f) \neq g_2^\star(f)$ and thus $g_1^\star \neq g_2^\star$, which is a contradiction.

(2) If $h$ is an isomorphism, then $g^{-1} = \big(\evalMap_{-}|_{\cont(Y)}\big)^{-1} \circ (h^{-1})^\star \circ \evalMap_{-}|_{\cont(X)}$, which is continuous.
\end{proof}
