\chapter{Vector spaces}

Gauss-Jordan reduction

TODO projective transformations

orientation
\url{https://en.wikipedia.org/wiki/Orientation_(vector_space)}
also for fixed set of $n$ vectors

\url{http://www.physics.rutgers.edu/~gmoore/618Spring2018/GTLect2-LinearAlgebra-2018.pdf}

\section{Formal definition}
A vector space is a collection of vectors, which are objects that have a natural addition and scalar multiplication.
\begin{definition}
A \udef{vector space} over a field $\mathbb{F}$ is a set $V$ together with an \udef{addition}
\[ +: V\times V \to V \]
and a \udef{scalar multiplication}
\[ \cdot: \mathbb{F}\times V \to V \]
such that $(V,+)$ is a commutative group and the following properties hold:
\begin{itemize}[leftmargin=4cm]
\item[\textbf{Distributivity 1}] $\lambda\cdot(v+w) = \lambda v + \lambda w$ for all $\lambda \in \mathbb{F}$ and all $v,w \in V$.
\item[\textbf{Distributivity 2}] $(\lambda_1+\lambda_2)\cdot v = \lambda_1 v + \lambda_2 v$ for all $\lambda_1, \lambda_2 \in \mathbb{F}$ and all $v \in V$.
\item[\textbf{Mixed associativity}] $\lambda_1\cdot(\lambda_2\cdot v) = (\lambda_1 \lambda_2) \cdot v$ for all $\lambda_1, \lambda_2 \in \mathbb{F}$ and all $v \in V$.
\item[\textbf{Multiplicative identity}] $1\cdot v = v$ for all $v \in V$.
\end{itemize}
This vector space can be denoted $\sSet{\mathbb{F}, V, +}$.
\end{definition}
In the definition we have used the following convention: for all $v,w\in V$ and $\lambda\in \mathbb{F}$, we denote $+(v,w)$ as $v+w$ and $\cdot(\lambda, v)$ as $\lambda \cdot v$ or $\lambda v$.

We call the elements of the field \udef{scalars} and the elements of the set $V$ \udef{vectors}. The zero of the group is known as the \udef{zero vector}.

Almost always we will actually be interested in $\mathbb{F} = \R$ or $\mathbb{F} = \C$.
\subsection{Examples}
\begin{enumerate}
\item The $n$-tuples in $\mathbb{F}^n$ with pointwise addition and multiplication. If the entries of the $n$-tuples are written one above the other in a column, it is called a \udef{column vector}.
\item The polynomials in $\mathbb{F}[X]$.
\item The polynomials in $\mathbb{F}[X]_{\leq n}$ of maximally degree $n$.
\item For any set $S$, the functions $(S\to \mathbb{F})$, denoted $\mathbb{F}^S$, with pointwise addition and multiplication.
\item For any topological space $X$, the continuous functions in $(X\to \C)$, denoted $\cont(X)$.
\item The trivial vector space $\{ 0\}$. A vector space can never be empty, because a commutative group always has a neutral element.
\item The set of all possible \textit{displacements} in (Euclidean) space forms a vector space. Once we have chosen an origin, we can view space as a vector space.
\end{enumerate}
\subsection{Some elementary lemmas}
\begin{lemma}
Given the vector space $(\mathbb{F}, V, +)$  and arbitrary $u,v,w\in V$ and $\lambda \in \mathbb{F}$, we have
\begin{enumerate}
\item $0v = 0 = \lambda \cdot 0$;
\item $(-1)v = -v = 1(-v)$;
\item $(-\lambda)v = -(\lambda v) = \lambda(-v)$;
\item $u+v = w+v \implies u = w$.
\end{enumerate}
By $-v$ we mean the additive inverse of $v$.
\end{lemma}
\begin{proof}
\begin{enumerate}
\item First, use distributivity to get
\[ 0v = (0+0)v = 0v + 0v. \]
The apply the previous lemma to $0+0v = 0v = 0v+0v$ to get $0=0v$. The equality $\lambda\cdot 0 = 0$ is proved analogously.
\item To show that $(-1)\cdot v$ is the additive inverse of $v$, i.e.\ $-v$, we simply add $(-1)\cdot v + v$ and observe the result is $0$.
\[ (-1)\cdot v + v = (-1)\cdot v + 1\cdot v = (1+(-1))\cdot v = 0\cdot v = 0. \]
\item Similar to the previous point.
\item The additive inverse $-v$ exists, so we can just add it left and right.
\end{enumerate}
\end{proof}
\subsection{Subspaces}
\begin{definition}
A \textit{subset} $U$ of a vector space $V$ is called a \udef{subspace} of $V$ if $U$ is also a vector space.
\end{definition}
The subset $U$ automatically inherits a lot of the structure of $V$. We only need to verify a couple of conditions.
\begin{proposition}[Subspace criterion] \label{subspaceCriterion}
A subset $U$ of a vector space $V$ is a subspace of $V$ \textup{if and only if} $U$ satisfies the following conditions:
\begin{enumerate}
\item \textbf{Additive identity}: $0 \in U$. Alternatively it is enough to show that $U$ is not empty.
\item \textbf{Closed under addition}: $v,w \in U$ implies $v+w\in U$;
\item \textbf{Closed under scalar multiplication}: $\lambda \in \mathbb{F}$ and $u\in U$ implies $\lambda u \in U$.
\end{enumerate}
\end{proposition}
Alternatively the last two criteria are equivalent to:
\[ v,w\in U; \lambda \in \mathbb{F} \qquad \text{implies} \qquad v+\lambda w \in U. \]

If the question is whether a set is a subspace, this criterion is almost always the answer. An elementary application:
\begin{proposition}
Any arbitrary intersection of subspaces is a subspace.
\end{proposition}

\section{Basis and dimension}
\subsection{Linear combinations and span}
\begin{definition}
A \udef{linear combination} of vectors $v_1, \ldots, v_n$ is a vector of the form
\[ a_1v_1 + \ldots + a_nv_n \]
where $a_1, \ldots, a_n \in \mathbb{F}$.
\end{definition}
\begin{definition}
The set of all linear combinations of a set of vectors $D$ in $V$ is called called the \udef{span} of $D$.
\[ \Span_{\mathbb{F}}(D) = \left\{ \sum_i a_iv_i \;|\; v_i\in D, a_i \in \mathbb{F} \right\} \]
If $D=\emptyset$, we conventionally say that $\Span(D) = \{0\}$.

$\Span(D)$ is also written $<D>$.
\end{definition}
Note that $D$ may be an infinite set, but the linear combinations are always finite sums. Often the set $D$ is simply a finite set $v_1,\ldots, v_n$

\begin{proposition}
The span of a set of vectors in $V$ is the smallest subspace of $V$ containing all the vectors in the set.
\end{proposition}

\begin{definition}
If $V = \Span(D)$, then the set $D$ \udef{spans} $V$.
\end{definition}

\begin{lemma}
Let $D,E$ be subsets of a vector space $V$.
\begin{enumerate}
\item $\Span(D) = \Span(\Span(D))$;
\item $D\subset E \implies \Span(D) \subset \Span(E)$.
\end{enumerate}
\label{span}
\end{lemma}

\begin{definition}
A vector space is \udef{finite-dimensional} if it is spanned by a finite set of vectors.

A vector space is \udef{infinite-dimensional} if it is not finite-dimensional.
\end{definition}
\subsection{Linear independence}
\begin{definition}
A set of vectors $D$ is \udef{linearly independent} if the only linear combinations in $D$ that equal $0$ are the trivial ones with all scalars zero. i.e.\,
\[ \text{If} \qquad \sum_{i=1}^n a_iv_i = 0 \qquad \text{then} \qquad a_1=\ldots=a_n = 0  \]
assuming the $v_i$ are vectors and the $a_i$ are scalars.

\udef{Linear dependence} is the opposite of linear independence.
\end{definition}
The empty set $D=\emptyset$ is taken as linearly independent. No non-trivial combinations of vectors in $\emptyset$ are equal to zero, because there are no non-trivial combinations of vectors in $\emptyset$.

\begin{lemma}
Let $D$ be a linearly dependent set of vectors. Then there exists a vector $v\in D$ such that
\begin{enumerate}
\item $v$ is a linear combination of other vectors in $D$;
\item $v\in \Span(D\setminus\{v\})$;
\item $\Span(D) = \Span(D\setminus\{v\})$.
\end{enumerate}
\label{linearDependence}
\end{lemma}
\begin{proof}
Take a linear combination of vectors in $D$ equalling zero,
\[ \sum_i a_iv_i = 0. \]
By linear dependence such a combination can be found such that not all $a_i$ are zero. In particular at least two must be non-zero. Take $a_j\neq 0$. Then
\[ v_j = \sum_{i\neq j}\frac{a_iv_i}{a_j}. \]

To prove the last point, take a $u\in \Span(D)$. Then
\[ u = \sum_i b_iv_i = b_j v_j + \sum_{i\neq j} b_iv_i = b_j\sum_{i\neq j}\frac{a_iv_i}{a_j} + \sum_{i\neq j} b_iv_i = \sum_{i\neq j}\left(\frac{b_ja_i}{a_j}+b_i\right)v_i.  \]
So $u\in \Span(D\setminus\{v\})$. The opposite inclusion is obvious. 
\end{proof}

\subsection{Bases}
\begin{definition}
A \udef{basis} of a vector space $V$ is a set of vectors in $V$ that spans $V$ and is linearly independent.
\end{definition}
\begin{example}
The \udef{standard basis} or \udef{natural basis} of $\mathbb{F}^n$ is given by
\begin{align*}
(1,0,0,&\ldots,0), \\
(0,1,0,&\ldots,0), \\
(0,0,1,&\ldots,0), \\
&\ldots \\
(0,0,0,&\ldots,1).
\end{align*}
We will denote it $\mathcal{E}$ or $\mathcal{E}_n$.
\end{example}
\subsubsection{In finite-dimensional spaces}
\begin{proposition}
A finite set $\{v_1, \ldots, v_n\}$ of vectors in $V$ is a basis of $V$ \textup{if and only if} every $v\in V$ can be written uniquely in the form
\[ v = a_1v_1 + \ldots + a_nv_n, \]
where $a_1, \ldots, a_n \in \mathbb{F}$.
\end{proposition}
\begin{proof}
We prove both directions.
\begin{itemize}
\item[$\boxed{\Rightarrow}$] Suppose $\{v_1, \ldots, v_n\}$ is a basis of $V$. Then any vector $v$ can be written as $a_1v_1 + \ldots + a_nv_n$, because the basis spans the space. We just need to show the decomposition is unique. To that end, assume there was another decomposition $v = b_1v_1 + \ldots + b_nv_n$. Subtracting both decompositions gives
\[ 0 = (a_1-b_1)v_1 + \ldots + (a_n-b_n)v_n. \]
Because $\{v_1, \ldots, v_n\}$ is linearly independent, $a_i = b_i$ for all $i$.
\item[$\boxed{\Leftarrow}$] Now suppose every vector has such a decomposition. Clearly $\{v_1, \ldots, v_n\}$ spans $V$. The unique decomposition of $0$ gives linear independence.
\end{itemize}
\end{proof}

\begin{theorem}[Steinitz exchange lemma] \label{SteinitzExchange}
Let $V$ be a vector space.
If $U = \{u_1, \ldots, u_m\}$ is a linearly independent set of $m$ vectors in $V$, and $W = \{ w_1, \ldots, w_n \}$ spans $V$, then:
\begin{enumerate}
\item $m\leq n$;
\item There is a set $\{u_1, \ldots, u_m, w'_{m+1}, \ldots, w'_n\} \supset U$ that spans $V$ where $w'_{m+1},\ldots, w'_n \in W$.
\end{enumerate}
\end{theorem}
\begin{proof}
We obtain the set $\{u_1, \ldots, u_m, w'_{m+1}, \ldots, w'_n\}$ by starting with the list $B_0 = (w_1, \ldots, w_n)$ and applying the following steps for each element $u_i \in U$, in the process defining sets $B_1, \ldots, B_m$. Each of these sets spans $V$.
\begin{enumerate}
\item Add $u_i$ to $B_{i-1}$. The set is now linearly dependent, because $B_{i-1}$ spans $V$.
\item By lemma \ref{linearDependence}, we can find a vector $v$ that is a linear combination of $B_{i-1}\setminus \{v\}$. Because $u_1,\ldots, u_i$ are linearly independent, we can choose this vector to be an element of $W$. Define $B_i = B_{i-1}\setminus\{v\}$. By lemma \ref{linearDependence}, $B_i$ still spans $V$, as required.
\end{enumerate}
This process only stops when we have had all elements of $U$.
\end{proof}
\begin{corollary}
If a vector space $V$ has a basis with $n$ vectors, then any basis of $V$ has $n$ vectors. \label{nBasis}
\end{corollary}

\begin{theorem}
Suppose $V$ is a finite-dimensional vector space spanned by $D = \{v_1, \ldots, v_n\}$.
\begin{enumerate}
\item We can find a subset of $D$ that is a basis of $V$, i.e.\ $D$ can be reduced to a basis;
\item Each linearly independent set of vectors can be expanded to a basis.
\end{enumerate}
\label{basis}
\end{theorem}
\begin{proof}
\begin{enumerate}
\item Remove $0$ from $D$, if it is an element. If $D$ is not linearly independent, find a vector in $D$ that is a linear combination of other vectors in $D$. Repeat until the set is linearly independent. This process stops due to the finite number of vectors. The set spans $V$ at every step.
\item Follows easily from the Steinitz exchange lemma, taking $W$ to be a basis.
\end{enumerate}
\end{proof}
\begin{corollary}
Every finite-dimensional vector space has a basis. \label{existenceBasis}
\end{corollary}

Thanks to corollaries \ref{nBasis} and \ref{existenceBasis}, the following definition makes sense:
\begin{definition}
The \udef{dimension} of a finite-dimensional vector space is the length of any basis of the vector space.
The dimension of $V$ (if $V$ is finite-dimensional) is denoted by $\dim V$ or $\dim_\mathbb{F}V$.\footnote{The latter notation is particularly useful if when distinguishing between real and complex vector spaces, because every complex vector space can be seen as a real vector space. In this case $\dim_\R V = 2\dim_\C V$, because $v$ and $iv$ are linearly independent over $\R$.}

If $V = \{0\}$, we take $\dim V = 0$.
\end{definition}

\begin{corollary}
Every linearly independent set of vectors in $V$ with length $\dim V$ is a basis of $V$. \label{maxLinearlyIndependent}
\end{corollary}
\begin{corollary}
Every spanning set of vectors in $V$ with length $\dim V$ is a basis of $V$.
\end{corollary}

\begin{proposition}
Let $V$ be a finite-dimensional vector space and $U$ a subspace of $V$. Then
\begin{enumerate}
\item $U$ is finite-dimensional and $\dim U \leq \dim V$;
\item $\dim U = \dim V \iff U=V$.
\end{enumerate}
\label{vectorSpaceEquality}
\end{proposition}
\begin{proof}
We construct a basis for $U$ using the following process:
\begin{enumerate}
\item If U=\{0\}, then we can take the basis $\emptyset$ and we are done. If $U\neq \{0\}$, we choose a nonzero vector $v_1 \in U$.
\item If $U$ is the span of all the vectors we have chosen, we are done. If not choose a vector in $U$, not in the span of the other vectors.
\item Repeat step (2).
\end{enumerate}
By construction, the chosen set of vectors is linearly independent. By the Steinitz exchange lemma this process must stop. In particular it must stop before reaching $\dim V$ vectors.

If the process reaches this upper bound, then by corollary \ref{maxLinearlyIndependent}, the set of vectors in $U$ is also a basis for $V$.
\end{proof}
We now have two tools for proving equalities of finite-dimensional vector spaces: either by proving both inclusions, or by leveraging point (2) of the previous proposition.

\subsubsection{In infinite-dimensional spaces}
Our definition of a basis of a vector spaces still makes sense for infinite-dimensional vector spaces, and many results of the previous section still make sense for infinite-dimensional vector spaces.

For infinite-dimensional vector spaces, there are, however, other notions of basis we might be interested in. In particular, our definition of basis requires all vectors to be constructible as \emph{finite} linear combinations of basis elements. In some contexts we might want to relax this to allow infinite combinations as well. For that, of course, we need some notion of infinite sum. Often we construct infinite sums as the limit of a sequence of finite sums, in which case we need a topology on our vector space that allows us to take limits.\footnote{Although other options exist, such as taking sums over hyperintegers.}  

In order to distinguish our purely algebraic definition of basis from these other notions of basis, a basis in the sense defined above is sometimes known as an \udef{algebraic basis} of \udef{Hamel basis}.

We will be discussing Hamel bases in this section.

\begin{theorem} \label{extensionReductionBasis}
Let $V$ be a vector space.
\begin{enumerate}
\item Any spanning set contains a basis.
\item Any linearly independent subset can be expanded to a basis.
\end{enumerate}
\label{infBasis}
\end{theorem}
\begin{proof}
Requires the axiom of choice. We will use Zorn's lemma twice.
\begin{enumerate}
\item Let $S$ be a spanning subset of $V$. Define
\[ \mathcal{A} = \{ D\subset S \;|\; \text{$D$ is linearly independent}\} \]
ordered by inclusion. It is easy to see that any chain on $\mathcal{A}$ has an upper bound on $\mathcal{A}$, by just taking the union which is still linearly independent. It follows from Zorn's lemma that $\mathcal{A}$ has a maximal element $R$. 
We show that $\Span(R) \supset S$ by contradiction. If $\Span(R) \not\supset S$, we can consider $R\cup \{v\}$ for some $v \in S$ that is not in $\Span(R)$ and we obtain an element of $\mathcal{A}$ which is greater than a maximal element. This is a contradiction. Then from $\Span(R) \supset S$ we conclude, using lemma \ref{span}
\[ \Span(R) = \Span(\Span(R)) \supset \Span(S) = V \]
from which it follows that $\Span(R) = V$.
\item Let $S$ be a linearly independent subset of $V$. Define
\[ \mathcal{A} = \{ D\subset V \;|\; S \subset D \; \text{and $D$ is linearly independent}\} \]
ordered by inclusion. 
It is easy to see that any chain on $\mathcal{A}$ has an upper bound on $\mathcal{A}$, by just taking the union. It follows from Zorn's lemma that $\mathcal{A}$ has a maximal element $R$. We show that $\Span(R) = V$ by contradiction. If $\Span(R) \neq V$, we can consider $R\cup \{v\}$ for some $v\notin \Span(R)$ and we obtain an element of $\mathcal{A}$ which is greater than a maximal element. This is a contradiction.
\end{enumerate}
\end{proof}
\begin{corollary} \label{existenceHamelBasis}
Every vector space has a Hamel basis
\end{corollary}

\begin{theorem}[Dimension theorem for vector spaces]
Given a vector space $V$, any two bases have the same cardinality.
\end{theorem}
\begin{proof}
The finite-dimensional case has already been proved. Suppose $A$ is a basis of $V$ with $|A| \geq \aleph_0$. Let $B$ be another basis of $V$. Each element $a\in A$ can be written as a finite combination of elements in $B$. Collect all the elements that go into the finite linear combination in a finite set $B_a \subset B$. We claim
\[ B = \bigcup_{a\in A} B_a. \]
Indeed, assume $b\in B \setminus (\cup_{a\in A} B_a)$. Since $A$ spans $V$, so does $\cup_{a\in A} B_a$. Thus $b$ can be written as a non-trivial combination of vectors in $\cup_{a\in A} B_a\subset B$, contradicting the linear independence of $B$. Then we have
\[ |B| = \left| \bigcup_{a\in A}B_a \right| \leq \aleph_0 \cdot |A| = |A| \]
A similar argument gives
\[ |A| \leq \aleph_0 \cdot |B| = |B|. \]
By the Schröder–Bernstein theorem \ref{SchroederBernstein}, we conclude $|A| = |B|$.
\end{proof}
TODO: does this proof work with only the ultrafilter lemma?

Thus the notion of dimension (also known as \udef{Hamel-dimension}) also makes sense for infinite-dimensional vector spaces, except it is a cardinality, not a number.

TODO: do we need a strong cardinality assignment? (Assumed for now)

Many textbooks state results using dimensions only for the finite-dimensional case. As we will see, these results almost always generalise directly to the infinite-dimensional case as well, if we assume the axiom of choice.

\begin{note}
The inverse of this theorem (i.e.\ the infinite-dimensional analogue of proposition \ref{vectorSpaceEquality}) does not hold: infinite-dimensional vector spaces always have proper subspaces with a basis of the same cardinality. This is obvious because dropping one vector in the Hamel basis of an infinite-dimensional vector space will not change the cardinality, but will make it a proper subspace.
\end{note}

 \begin{corollary}
 Let $V$ and $W$ be vector spaces.
 \begin{enumerate}
 \item If $\dim V > \dim W$, then no linear map from $V$ to $W$ is injective.
 \item If $\dim V < \dim W$, then no linear map from $V$ to $W$ is surjective.
 \end{enumerate}
 \end{corollary}

\begin{lemma}
Let $V$ be an infinite-dimensional vector space over a field $\mathbb{F}$. Assume $|\mathbb{F}|\leq \dim_{\mathbb{F}} V$, then $\dim_{\mathbb{F}} V = |V|$. \label{vsCardinality}
\end{lemma}
\begin{proof}
Let $B$ be a basis of $V$. It is supposed infinite. There is a surjection
\[\bigcup_{n\in\N}(\mathbb{F}\times B)^{n} \to V: (a_i,v_i)^{i<n} \mapsto \sum_{i<n}a_iv_i. \]
So we have
\[ |V| \leq \left|\bigcup_{n\in\N}(F\times B)^{n}\right| = \sum_{n\in \N}|F\times B|^n \leq \aleph_0\cdot |\mathbb{F}| \cdot |B| = \max\{\aleph_0, |\mathbb{F}|, |B|\} = |B|. \]
Thus $|V|\leq \dim_{\mathbb{F}} V$. The other inequality is obvious. By the Schröder–Bernstein theorem \ref{SchroederBernstein}, we conclude $\dim_{\mathbb{F}} V = |V|$.
\end{proof}

\section{Constructing vector spaces}
\subsection{Sums of subspaces}
\begin{definition}
Suppose $\{U_i\}_{i\in I}$ a set of subspaces of a vector space $V$. The \udef{sum} of these subspaces, denoted $\sum_{i\in I}U_i$, is the set of all finite linear combinations of elements in $\bigcup_{i\in I}U_i$:
\[ \sum_{i\in I}U_i = \Span\left(\bigcup_{i\in I} U_i\right) = \setbuilder{\sum_{i\in J} u_i}{\text{$J\subset I$ finite}, u_i\in \bigcup_{i\in I}U_i}. \]
\end{definition}
For finite sums this reduces to
\[ U_1+\ldots + U_m = \setbuilder{\sum_{i=1}^m u_i}{u_1\in U_1, \ldots, u_m\in U_m}. \]

\begin{proposition} \label{basisSum}
Let $\{U_i\}_{i\in I}$ be a set of subspaces of a vector space $V$ and $\beta_i$ a basis of $U_i$ for all $i\in I$. Then
\[ \sum_{i\in I}U_i = \Span\left(\bigcup_{i\in I}\beta_i\right). \]
\end{proposition}
\begin{proof}
From $\bigcup_{i\in I}\beta_i \subseteq \bigcup_{i\in I} U_i$, we get $\Span\left(\bigcup_{i\in I}\beta_i\right) \subseteq \Span\left(\bigcup_{i\in I} U_i\right) = \sum_{i\in I}U_i$.

Conversely, take $u\in \sum_{i\in I}U_i$. Then $u = \sum_{j\in J}u_j$ where $J$ is finite subset of $I$ and $u_i\in U_i$. Now each $u_j$ can be written as $\sum_k a_{j,k}v_{j,k}$, where $a_{j,k}$ are scalars and $v_{j,k}$ are vectors in $\beta_j$. So
\[ u = \sum_{j,k}a_{j,k}v_{j,k}, \]
which is a finite linear combination of vectors in $\bigcup_{i\in I}\beta_i$. So $u\in \Span\left(\bigcup_{i\in I}\beta_i\right)$.
\end{proof}

\begin{proposition}
Let $V$ be a vector space and $A,B,C$ subspaces. Then
\begin{enumerate}
\item $A+(B\cap C) \subseteq (A+B)\cap (A+C)$;
\item $(A+B)\cap C \supseteq (A\cap C) + (B\cap C)$. 
\end{enumerate}
\end{proposition}
\begin{proof}
(1) Take $v = v_1+v_2 \in A+(B\cap C)$ where $v_2 \in B$ and $v_2 \in C$, so $v_1+v_2\in A+B$ and $v_1+v_2\in A+C$.

(2) Take $v = v_1+v_2\in (A\cap C) + (B\cap C)$. Then $v_1,v_2\in C$ and thus $v\in (A+B)\cap C$.
\end{proof}

\begin{theorem}[Dimension of a sum]
Let $U_1$ and $U_2$ be subspaces of a finite-dimensional vector space, then
\[ \dim(U_1 + U_2) = \dim U_1 + \dim U_2 - \dim(U_1\cap U_2). \]
\label{dimOfASum}
\end{theorem}
\begin{proof}
Let $\dim U_1 = r, \dim U_2 = s$ and $\dim(U_1\cap U_2) = t$. Then $t\leq r$ and $t\leq s$.  Take a basis $\{v_1,\ldots, v_t\}$ of $U_1\cap U_2$. This can be expanded to a basis $\beta_{U_1} = \{ v_1, \ldots, v_t, u_{t+1}, \ldots u_{r} \}$ of $U_1$ and also to a basis $\beta_{U_2} = \{ v_1, \ldots, v_t, u'_{t+1}, \ldots u'_{s} \}$ of $U_2$. We will show that $\{ v_1, \ldots, v_t, u_{t+1}, \ldots u_{r}, u'_{t+1}, \ldots, u'_{s} \}$ is a basis of $U_1\cap U_2$. This completes the proof because
\begin{align*}
\dim(U_1 + U_2) &= t + (s-t) + (r-t) = s + r -t\\
&= \dim U_1 + \dim U_2 - \dim(U_1\cap U_2).
\end{align*}
The spanning property is easy. Linear independence is slightly more difficult: Take a linear combination
\[ \sum_{i=1}^t\alpha_i v_i + \sum^r_{j=t+1}\beta_ju_j + \sum^s_{k=t+1}\beta'_ku_k' =0. \]
We must show this combination is trivial. Indeed observe that
\[ \sum_{i=1}^t\alpha_i v_i + \sum^r_{j=t+1}\beta_ju_j  =-\sum^s_{k=t+1}\beta'_ku_k'. \]
The left-hand side is a vector in $U_1$, the right-hand side is a vector in $U_2$, so it must lie in $U_1\cap U_2$, so we rewrite the left-hand side as
\[ \sum_{i=1}^t\lambda_i v_i =  -\sum^s_{k=t+1}\beta'_ku_k'.\]
Due to $\beta_{U_2}$ being a basis, this linear combination must be trivial and all $\beta'_k$ are zero. This leaves us 
\[ \sum_{i=1}^t\alpha_i v_i + \sum^r_{j=t+1}\beta_ju_j =0 \]
from our original linear combination. Due to $\beta_{U_2}$ being a basis this combination must also be trivial. 
\end{proof}
\begin{note}
If $\dim(U_1\cap U_2)<\dim U_1$ and $\dim(U_1\cap U_2)< \dim U_2$, this proof generalises to infinite-dimensional vector spaces.
\end{note}

\subsection{(Internal) direct sum}
\begin{definition}
Suppose $\{U_i\}_{i\in I}$ is a set of subspaces of $V$. The sum $\sum_{i\in I}U_i$ is called a \udef{direct sum} if each element $u$ of the sum can be \emph{uniquely} written as
\[ u = \sum_{i\in I}u_i \qquad (u_i\in U_i) \]
where only finitely many of the $u_i$ are nonzero.

In this case we write $\bigoplus_{i\in I} U_i$, or $U_1 \oplus \ldots \oplus U_m$ if $I = \{1,\ldots, m\}$. 
\end{definition}

\begin{proposition}[Conditions for a direct sum] \label{directSumCriterion}
Let $\{U_i\}_{i\in I}$ be a set of subspaces of a vector space $V$ and $\beta_i$ a basis of $U_i$ for all $i\in I$. Let $U,W\subseteq V$ also be subspaces of $V$.
\begin{enumerate}
\item The sum $\sum_{i\in I}U_i$ is direct \textup{if and only if} $0$ has the unique decomposition as in the definition.
\item The sum $\sum_{i\in I}U_i$ is direct \textup{if and only if} the union $\bigcup_{i\in I}\beta_i$ is disjoint and linearly independent.
\item The sum $U+W$ is direct \textup{if and only if} $U\cap W = \{0\}$.
\end{enumerate}
\end{proposition}
\begin{proof}
TODO
\end{proof}
\begin{corollary}
Let $\{U_i\}_{i\in I}$ be a set of subspaces of a vector space $V$ and $\beta_i$ a basis of $U_i$ for all $i\in I$. Then
\[ \dim\left(\bigoplus_{i \in I}U_i\right) = \sum_{i\in I} \dim U_i \]
\end{corollary}

\begin{definition}
In a vector space $V$, a subspace $W$ is a \udef{complementary subspace} (or a \udef{complement}) of the subspace $U$ if $V = U \oplus W$.
\end{definition}

\begin{proposition}
Let $V$ be a vector space, then each subspace of $V$ has a complement.
\end{proposition}
\begin{proof}
Let $U$ be a subspace of $V$. Then, by \ref{existenceHamelBasis}, we can find a basis $B$ of $U$ and, by \ref{extensionReductionBasis}, we can extend it to a basis $D$ of $V$. Now $V = U \oplus \Span(D\setminus B)$ by \ref{basisSum} and \ref{directSumCriterion}.
\end{proof}
Note this requires the axiom of choice, and is in fact equivalent with it.
\begin{corollary}
Suppose $V$ is finite-dimensional and $U_1,\ldots, U_m$ are subspaces of $V$. Then $U_1+\ldots+ U_m$ is a direct sum \textup{if and only if}
\[ \dim(U_1+\ldots+U_m) = \dim U_1 + \ldots \dim U_m. \]
\label{directSumDimensionCriterion}
\end{corollary}

\subsection{External direct sum}
\begin{definition}
Let $U,W$ be vector spaces over the same field $\mathbb{F}$. We define the vector space  $U\oplus W$, called the \udef{(external) direct sum}, as the set $U\times W$ with the operations
\[ \begin{cases}
(u_1,w_1) + (u_2, w_2) = (u_1 +_U u_2, w_1 +_W w_2) & (u_1,u_2 \in U; w_1, w_2 \in W) \\
r\cdot (u,w) = (ru, rw) & (r\in \mathbb{F}; u\in U; w \in W)
\end{cases} \]
In general we can define a direct sum of an arbitrary collection of vector spaces $\{U_i\}_{i\in I}$, denoted
\[ \bigoplus_{i\in I}U_i \]
as the vector space with as field the subset of the Cartesian product $\prod_{i\in I}U_i$ where all but finitely many of the terms are zero. The operations are defined point-wise.
\end{definition}

\begin{proposition}
Suppose $V_1, \ldots V_m$ are vector spaces over $\mathbb{F}$. Then
\[ \dim(V_1\oplus\ldots \oplus V_m) = \dim V_1 + \ldots + \dim V_m \]
\label{dimDirectSum}
\end{proposition}
\begin{proof}
We construct a basis $\beta$ of $V_1\oplus\ldots \oplus V_m$ from bases $\beta_{V_i}$ of $V_i$:
\[ \beta = (\beta_{V_1} \times \{0 \} \times \ldots \times \{0 \}) \cup (\{0 \} \times \beta_{V_2}\times \{0 \} \times \ldots \times \{0 \}) \cup \ldots \cup (\{0 \}\times \ldots \times \{0 \} \times \beta_{V_m}). \]
All these unions are disjunct, so
\begin{align*}
|\beta| &= |(\beta_{V_1} \times \{0 \} \times \ldots \times \{0 \}) \cup  \ldots \cup (\{0 \}\times \ldots \times \{0 \} \times \beta_{V_m})| \\
&= |(\beta_{V_1} \times \{0 \} \times \ldots \times \{0 \})| + \ldots + |(\{0 \}\times \ldots \times \{0 \} \times \beta_{V_m})| \\
&= | \beta_{V_1}| + \ldots + |\beta_{V_m}| \\
&= \dim V_1 + \ldots + \dim V_m.
\end{align*}
\end{proof}

\begin{proposition}
Let $U,W$ be subspaces of $V$. Then the external direct sum of $U$ and $W$ is isomorphic to the internal direct sum of $U$ and $W$.
\end{proposition}
\begin{proof}
The map $f: U\times W\to V: (u,w) \mapsto u+w$ is an isomorphism.
\end{proof}
For this reason we use the same symbol for both.

\begin{definition}
Let $V,W, X,Y$ be vector spaces over $\mathbb{F}$. Let $S: V\to X$ and $T: W\to Y$ be linear maps. Then the \udef{direct sum} of $S$ and $T$ is a linear map
\[ S\oplus T: V \oplus W \to X\oplus Y: (v,w) \mapsto (S(v), T(v)). \]
\end{definition}

\begin{lemma}
Let $V,W$ be vector spaces over a field $\F$ and $A,C\in\Lin(V)$ and $B,D\in\Lin(W)$. Then
\begin{enumerate}
\item $a(A\oplus B) + b(C\oplus D) = (aA+bC)\oplus (aB + bD)$;
\item $(A\oplus B)(C\oplus D) = AC\oplus BD$;
\item $(A\oplus B)^k = A^k\oplus B^k$.
\end{enumerate}
\end{lemma}

\subsubsection{Matrix representation}
TODO: move
Assume $V$ and $W$ are finite-dimensional vector spaces with resp. bases $\{\vec{e}_i\}_{i=1}^m$ and $\{\vec{f}_j\}_{j=1}^n$.
As in the proof of proposition \ref{dimDirectSum}, we can take the basis $\{\vec{e}_i\}_i\times\{0\} \cup \{0\}\times\{\vec{f}_j\}_j$ of $V\oplus W$.

We can naturally fit the basis into a list of $m+n$ elements:
\[ (\vec{e}_1,0),\ldots (\vec{e}_m, 0), (0, \vec{f}_1), \ldots, (0,\vec{f}_n)  \]
\subsubsection{Linear maps}
TODO: also move
Let $S: V\to X$ and $T:W\to Y$ be linear maps, with matrix representations $A$ and $B$, respectively. The matrix representation of $S\oplus T$ is given by
\[ A\oplus B = \begin{bmatrix}
A & 0 \\
0 & B
\end{bmatrix} \]
with respect to the basis $\{\vec{e}_i\}_i\times\{0\} \cup \{0\}\times\{\vec{f}_j\}_j$.


\section{Linear maps}
\begin{definition}
Let $(\mathbb{R}, V, +)$ and $(\mathbb{R}, W, +)$ be vector spaces over the same field. A \udef{linear map} or \udef{linear transformation} is a function $L:V\to W$ with the following properties:
\begin{itemize}[leftmargin=3cm]
\item[\textbf{Additivity}] $L(u+v) = L(u)+L(v)$ for all $u,v \in V$;
\item[\textbf{Homogeneity}] $L(\lambda v) = \lambda L(v)$ for all $\lambda \in \mathbb{R}$ and all $v\in V$.
\end{itemize}
These conditions are equivalent to the condition that
\[ L(\lambda_1 v_1 + \lambda_2v_2) = \lambda_1L(v_1) + \lambda_2 L(v_2) \qquad \text{for all $\lambda_1,\lambda_2\in \mathbb{F}$ and all $v_1,v_2\in V$.} \]
We denote the set of all linear maps from $V$ to $W$ as $\Lin_\mathbb{F}(V,W)$, or $\Lin(V,W)$. The set of endomorphisms on $V$ is denoted $\Lin(V) \defeq \End(V) = \Lin(V,V)$.
\end{definition}

\begin{lemma} \label{linearMaps}
Let $L\in \Lin(V,W)$.
\begin{enumerate}
\item $L(0) = 0$ and $L(-v) = -L(v)$
\item $L\left(\sum^n_{i=1}\lambda_i v_i\right) = \sum_{i=1}^n\lambda_i L(v_i)$.
\item A linear map is completely determined by the images of a basis of $V$.
\item Let $D$ be a set of vectors. Then $L[D]$ is linearly independent \textup{if and only if} $D$ is linearly independent.
\end{enumerate}
\end{lemma}

\subsection{Examples}
\begin{enumerate}
\item The zero map that maps everything to zero.
\item Identity maps.
\item Differentiation of polynomials.
\item Integration of polynomials.
\item Shifting elements in a list.
\item Projections.
\end{enumerate}

\begin{definition}
A (linear) \udef{operator} between two vector spaces $V$ and $W$ is a linear partial function $T: V \not\to W$ such that the domain $\dom(T)$ is a vector space.

We also say an operator is a function $T: \dom(T)\subseteq V\to W$.
\end{definition}
The requirement that $\dom(T)$ be a subspace of $V$ is necessary for linearity to make sense!

Some authors (e.g\ Axler) use the word ``operator'' to mean a linear endomorphism.

\subsection{Image and kernel}
\begin{definition}
Let $L \in \Lin(V,W)$. The \udef{kernel} or \udef{null space} of $L$ is the set of vectors that $L$ maps to zero:
\[ \ker(L) = \{ v\in V \;|\; L(v) = 0 \}. \]
\end{definition}
\begin{proposition} \label{kernelSubspace}
The kernel of $L\in \Lin(V,W)$ is a subspace of $V$.
\end{proposition}
\begin{definition}
The dimension of the kernel of a linear map is its \udef{nullity}.
\end{definition}
\begin{proposition} \label{injectivityKernelTriviality}
Let $L\in\Lin(V,W)$. Then $L$ is injective if and only if $\ker(L) = 0$.
\end{proposition}
TODO: generalise to groups
\begin{proof}
We show both implications.
\begin{itemize}
\item[\boxed{\Rightarrow}] We know $\{0\}\subset \ker(L)$ by lemma \ref{linearMaps}. Suppose $v\in \ker(L)$, then $L(v) = 0 = L(0)$. So $v=0$ by injectivity and $\{0\}\supset \ker(L)$.
\item[\boxed{\Leftarrow}] Suppose $u,v \in V$ such that $L(u)=L(v)$. Then
\[ 0 = L(u) - L(v) = L(u-v). \]
Thus $u-v\in \ker(L)$, meaning $u-v = 0$ and $u=v$.
\end{itemize}
\end{proof}

\begin{definition}
Let $L \in \Lin(V,W)$. The \udef{image} or \udef{range} of $L$ is the set of vectors that are of the form $L(v)$ for some $v\in V$:
\[ \im(L) = \{ L(v) \;|\; v\in V \}. \]
\end{definition}
\begin{proposition}
The range of $L\in \Lin(V,W)$ is a subspace of $W$.
\end{proposition}
\begin{definition}
The dimension of the image of a linear map is its \udef{rank}.
\end{definition}

\begin{theorem}
Every short exact sequence of vector spaces splits.
\end{theorem}
\begin{proof}
Let
\[ \begin{tikzcd}
0 \rar & U \rar{S} & V \rar{T} & W \rar & 0
\end{tikzcd} \]
be a short exact sequence of vector spaces.
By the splitting lemma TODO ref, it is enough to find a left inverse of $S$. Pick a basis $\beta$ of $U$. Because $S$ is injective, $S[\beta]$ is linearly independent and we can extend it to a basis $\beta'$. We can now define the left inverse by specifying how the basis elements are mapped, by \ref{linearMaps}. To wit: $\beta'\setminus S[\beta]$ is mapped to $0$ and each element $S[\beta]$ has exactly one origin be injectivity and it is to this origin that it is now mapped.
\end{proof}
\begin{corollary} \label{directSumKernelImage}
Let $L \in \Lin(V,W)$. Then
\[ V \cong \ker L \oplus \im L. \]
\end{corollary}
\begin{proof}
Given $L$ we have the short exact sequence
\[ \begin{tikzcd}
0 \rar & \ker L \ar[r, hook] & V \rar{L} & \im L \rar & 0.
\end{tikzcd} \]
The isomorphism then follows from the splitting lemma TODO ref.
\end{proof}
\begin{corollary}[Dimension theorem for linear maps] \label{dimensionLinearMaps}
Let $L \in \Lin(V,W)$. Then
\[ \dim(V) = \dim(\ker L) + \dim(\im L). \]
\end{corollary}
This corollary is also known as the rank-nullity theorem or the fundamental theorem of linear maps.
\begin{proof}
By $\dim(V) = \dim(\ker L \oplus \im L) = \dim(\ker L) + \dim(\im L)$.

Alternatively this can be proven directly as follows:

Take a basis $\beta_0$ of $\ker(L)$. We can expand this to a basis $\beta$ of $V$, by theorem \ref{infBasis}. It is easy to show that $L[\beta\setminus \beta_0]$ is a basis of $\im(L)$. Now $L[\beta\setminus \beta_0] =_c \beta\setminus \beta_0$ and $(\beta\setminus \beta_0) \cap \beta_0 = \emptyset$. Thus $|\beta| = |(\beta\setminus \beta_0) \cup \beta_0| = |\beta\setminus \beta_0| + |\beta_0|$. This proves the assertion.
\end{proof}
\begin{corollary}
Let
\[ \begin{tikzcd}
0 \rar & V_1 \rar & V_2 \rar & \ldots \rar & V_n \rar & 0
\end{tikzcd} \]
be an exact sequence of vector spaces, then
\[ \sum_{i=1}^n (-1)^i\dim(V_i) = 0. \]
\end{corollary}
\begin{proof}
Let $f_i$ be the map $V_i\to V_{i+1}$. By exactness $\im f_i=\ker f_{i+1}$ and $\dim(\im f_i)=\dim(\ker f_{i+1})$. By the previous corollary $\dim(V_i) = \dim(\ker f_i) + \dim(\im f_i)$. Then
\[ \sum_{i=1}^n (-1)^i\dim(V_i) = \sum_{i=1}^n (-1)^i\dim(\ker f_i) + \sum_{i=1}^n (-1)^i\dim(\ker f_{i+1}) = \sum_{i=2}^{n} (-1)^i\dim(\ker f_i) - \sum_{i=2}^{n} (-1)^i\dim(\ker f_{i}) = 0. \]
\end{proof}
\begin{corollary} \label{dimensionImageSmaller}
Let $L \in \Lin(V,W)$. Then
\[ \dim(\im L) \leq \dim(V). \]
\end{corollary}
\begin{proof}
TODO ref cardinal arithmetic.
\end{proof}

\begin{lemma} \label{rankMapComposition}
Let $S,T$ be compatible linear maps. Then
\[ \text{rank of $ST$}\;\leq\;\min\{\text{rank of $S$, rank of $T$}\}. \]
If $T$ is invertible, then the rank of $ST$ equals the rank of $S$. Similarly if $S$ is invertible, then the rank of $ST$ equals the rank of $T$.
\end{lemma}
\begin{proof}
Clearly $\im(ST) \subset \im(S)$, so $\dim\im(ST)\leq \dim\im(S)$.
We also have $ST = S|_{\im T}T$, where $S|_{\im T}$ is $S$ restricted to $\im T$.  Then corollary \ref{dimensionImageSmaller} applied to $S|_{\im T}$ gives $\dim\im(ST)\leq \dim\im T$. Together these inequalities give the result.

To show equality in the invertible case, first assume $T$ invertible:
\[ \dim\im ST \leq \dim\im STT^{-1} = \dim\im S. \]
Together with the first inequality this gives an equality. The case for $S$ invertible is similar.
\end{proof}

\begin{proposition} \label{kernelCompositionLinearMaps}
Let $S,T$ be compatible linear maps. Then
\begin{enumerate}
\item $\ker(ST)\supseteq \ker(T)$;
\item $\dim\ker(ST) = \dim\ker(T) + \dim(\im(T)\cap\ker(S))$.
\end{enumerate}
\end{proposition}
\begin{proof}
(1) $x\in\ker(T) \implies (ST)x = S(Tx) = S(0) = 0 \implies x\in\ker(ST)$.
(2) Consider the restriction $T|_{\ker(ST)}$. Applying the dimension theorem gives
\[ \dim\ker(ST) = \dim\ker(T|_{\ker(ST)}) + \dim\im(T|_{\ker(ST)}) = \dim\ker(ST) = \dim\ker(T) + \dim\im(T|_{\ker(ST)}) , \]
so it is enough to show $\im(T|_{\ker(ST)}) = \im(T)\cap\ker(S)$. First take $v\in\im(T|_{\ker(ST)}$, then there exists some $w\in\ker(ST)$ such that $v=Tw$, meaning $v\in\im(T)$. Also $Sv = STw = 0$, meaning $v\in\ker(S)$.

Then take $v\in\im(T)\cap\ker(S)$, so we can find a $w$ such that $v = Tw$. Also $Sv = STw = 0$, so $w\in\ker(ST)$ and $v\in\im(T|_{\ker{ST}})$.
\end{proof}

\subsubsection{Algebraic operations on linear maps}
\begin{definition}
Suppose $K,L \in \Lin_{\mathbb{F}}(V,W)$ and $\lambda \in \mathbb{F}$.
\begin{itemize}
\item The \udef{sum} $K+L$ is defined by $(K+L)(v) = Kv+Lv$ for all $v\in V$;
\item The \udef{scalar product} is defined by $(\lambda K)(v) = \lambda K(v)$ for all $v\in V$.
\end{itemize}
\end{definition}
\begin{proposition}
\begin{itemize}
\item The sum of linear maps is again a linear maps. Scalar multiples of linear maps are linear maps.
\item With addition and scalar multiplication defined as above, $\Lin_\mathbb{F}(V,W)$ is a vector space.
\end{itemize}
\end{proposition}

\begin{definition}
Let $K\in \Lin_\mathbb{F}(U,V)$ and $L\in \Lin_\mathbb{F}(V,W)$. The \udef{product} $LK$ is defined as the composition
\[ (LK)(u) = L(K(u)) \qquad \text{for all $u\in U$.} \]
If the product of two linear maps $K,L$ makes sense, we call the linear maps \udef{compatible}.
\end{definition}
\begin{proposition}
The product of two (compatible) linear maps is a linear map.
\end{proposition}
\begin{proposition}[Algebraic properties of linear maps]
The product of linear maps has the following properties. 
\begin{itemize}[leftmargin=4.2cm]
\item[\textbf{Associativity}] Let $L_1, L_2, L_3$ be compatible linear maps, then
\[ (L_1L_2)L_3 = L_1(L_2L_3) \]
\item[\textbf{Identity}] Let $L\in \Lin(V,W)$. The identity maps $I_V:V\to V$ and $I_W:W\to W$ are linear and have the property that
\[ LI_V = I_W L = L. \]
\item[\textbf{Distributive properties}]
$ (S_1+S_2)T = S_1T + S_2T \qquad \text{and} \qquad S(T_1 + T_2) = ST_1 + ST_2 $
whenever $T,T_1, T_2 \in \Lin(U,V)$ and $S,S_1, S_2\in \Lin(V,W)$.
\end{itemize}
These properties mean that for any vector space $V$, $\Lin(V)$ forms a unital algebra.
\end{proposition}
Note that multiplication of linear maps is not commutative, not even for maps that are compatible both ways.

\subsection{Invertibility and isomorphisms}
\begin{proposition} \label{inverseLinear}
Let $L$ be a linear map. If $L$ is invertible as a function (i.e.\ bijective), its inverse $L^{-1}$ is linear.
\end{proposition}
\begin{proof}
We calculate for $x,y$ vectors and $a\in\mathbb{F}$
\[ L^{-1}(ax + y) = L^{-1}(aLL^{-1}x + LL^{-1}y) = L^{-1}L(aL^{-1}x + L^{-1}y) = aL^{-1}x + L^{-1}y. \]
\end{proof}

\begin{definition}
\begin{itemize}
\item An invertible linear map is called an \udef{isomorphism}.
\item Two vector spaces $V,W$  are \udef{isomorphic} if there is an isomorphism between them. This is denoted $V\cong W$.
\end{itemize}
\end{definition}

\begin{proposition} \label{isomorphicDimension}
Let $V,W$ be vector spaces over the same field $\mathbb{F}$ and $n\in \N$. Then
\begin{enumerate}
\item $V\cong W \iff \dim V = \dim W$;
\item $V \cong \mathbb{F}^n \iff \dim V = n$;
\item $\F^n \cong \F^m \iff n=m$.
\end{enumerate}
\label{isomorphicCondition}
\end{proposition}
\begin{proof}
We prove the first statement. The second and third follow easily, using $\dim_\mathbb{F} \mathbb{F}^n = n$.
\begin{itemize}
\item[$\boxed{\Rightarrow}$] Let $T:V\to W$ be an isomorphism. Then $\ker T = \{0\}$ and $\im T = W$. Thus
\[ \dim V = \dim \ker T + \dim \im T = 0 + \dim W = \dim W. \]
\item[$\boxed{\Leftarrow}$] Assume $\dim V = \dim W$. Thus there exists an invertible function from a basis of $V$ to a basis of $W$. This can be extended by linearity to a function on $V$, because it is defined on a Hamel basis. It is easy to see this function is linear and bijective.
\end{itemize}
\end{proof}

\begin{proposition} \label{mappingOfBasisByIsomorphism}
Let $L\in\Lin(V,W)$ be an isomorphism. Let $\beta$ be a basis of $V$, then $L[\beta]$ is a basis of $W$.
\end{proposition}

\begin{proposition} \label{invertibleFiniteDim}
Suppose $V$ is a finite-dimensional vector space and $L\in \Lin(V)$ is a linear map on $V$, then
\[ L \;\text{is invertible} \iff L \;\text{is injective} \iff L \;\text{is surjective} \]
\end{proposition}
\begin{proof}
All we need to prove is
\[ L \;\text{is injective} \iff L \;\text{is surjective} \]
\begin{itemize}
\item[$\boxed{\Rightarrow}$] Assume $L$ injective. Then $\ker L = \{0\}$. By the dimension theorem for linear maps, theorem \ref{dimensionLinearMaps}
\[ \dim \im L = \dim V - \dim \ker L = \dim V. \]
Because $\im L \subset V$ and using proposition \ref{vectorSpaceEquality}, we conclude that $\im L = V$ and thus $L$ is surjective.
\item[$\boxed{\Leftarrow}$] Assume $L$ surjective. Then, by the dimension theorem for linear maps,
\[ \dim \ker L = \dim V - \dim \im L = 0, \]
which means $L$ is injective.
\end{itemize}
\end{proof}
Remark that the proof of the first implication uses proposition \ref{vectorSpaceEquality}, and thus cannot be generalised to infinite-dimensional vector spaces. In the proof of the second implication the subtraction of infinite cardinals is only uniquely defined if  $\dim V > \dim \im L$, which is clearly not the case.

\begin{example}
Counterexamples to the previous theorem in the infinite-dimensional case are given by the left shift map on $\mathbb{F}^\N$ (which is injective, but not surjective) and the right shift map on $\mathbb{F}^\N$ (which is surjective, but not injective).
\end{example}


\subsection{Special types of linear maps}
\subsubsection{Finite-rank operators}
\begin{definition}
A linear map $T: V\to V$ is said to be a \udef{finite-rank operator} if it has finite rank.
\end{definition}
\subsubsection{Idempotents}

\begin{lemma}
Let $V$ be a vector space and $U\subseteq V$ a subspace. Then for any complement $W$ of $U$ in $V$,
\[ P: U\oplus W = V \to V: u+w \mapsto u \]
is an idempotent such that $\im P = U$.
\end{lemma}

\begin{proposition} \label{directSumKernelImageIdempotent}
Let $V$ be a vector space and $P$ an idempotent linear map. Then
\[ V = \im P \oplus \ker P. \]
\end{proposition}
\begin{proof}
For any $v\in V$, we can write $v= (v-Pv)+Pv$ where $Pv\in \im P$ and $(v-Pv)\in \ker P$ because
\[ P(v-Pv) = Pv- P^2v = Pv - Pv = 0. \]
So we have $V = \im P + \ker P$. To show that the sum is direct, we take $u\in \im P \cap \ker P$. Then $u = Pw$ for some $w\in V$ and applying $P$ gives $0 = Pu = P^2w = Pw = 0$. So the sum is direct by \ref{directSumCriterion}.
\end{proof}

\begin{lemma} \label{idempotentImageEquivalence}
Let $V$ be a vector space, $P$ an idempotent linear map and $v\in V$. Then $v\in \im P$ \textup{if and only if} $v = Pv$.
\end{lemma}
\begin{proof}
We have that $v\in \im P$ iff $\exists u\in V: Pu = v$. Then we have $v = Pu = P^2u = Pv$.
\end{proof}

\begin{lemma}
Let $V$ be a vector space and $P$ an idempotent linear map. Then $P' \defeq \id_V - P$ is an idempotent linear map such that
\[ \im P' = \ker P \qquad\text{and}\qquad \ker P' = \im P. \]
Consequently, $V = \im P\oplus \im P'$.
\end{lemma}
\begin{proof}
Clearly $\id_V - P$ is idempotent: $(\id_V - P)^2 = \id_V - P - P + P^2 = \id_V - 2P + P = \id_V - P$. It is enough to show that $\im P' = \ker P$, because $P = \id_V - P'$.

Assume $v\in \ker P$. Then $P'v = v - Pv = v-0 = v$, so $v\in \im P'$.

Assume $v\in \im P'$. Then, by \ref{idempotentImageEquivalence}, $v = P'v = v - Pv$, so $Pv = 0$.

The last remark follows from \ref{directSumKernelImageIdempotent}
\end{proof}



TODO trace:
\begin{lemma}
Let $V$ be a vector space and $P$ an idempotent linear map. Then
\[ \Tr(P) = \dim(\im(P)). \]
\end{lemma}


\subsubsection{Invariant, reducing and irreducible subspaces}
\begin{definition}
Let $V$ be a vector space, $T$ a linear operator on $V$ and $U\subseteq V$ a subspace. Then $U$ is called
\begin{itemize}
    \item \udef{invariant} under $T$ is $T[U]\subseteq U$;
    \item \udef{reducing} for $T$ if $V = U\oplus W$ and both $U$ and $W$ are invariant under $T$;
    \item \udef{irreducible} w.r.t. $T$ if for all $W\subseteq U$ such that $W$ is reducing for $T$, we have $W = U$ or $W = \emptyset$.
\end{itemize}
\end{definition}

\begin{lemma}
Let $V$ be a vector space, $T$ a linear operator on $V$ and $P$ an idempotent operator on $V$ with image $U = P[V]$. Then 
\begin{enumerate}
\item $U$ is invariant under $T$ \textup{if and only if} $PTP = TP$;
\item the following are equivalent:
\begin{enumerate}
\item $U$ is reducing for $T$;
\item $P[V]$ and $(\id_V - P)[V]$ are invariant under $T$;
\item $PT = PTP = TP$;
\item $PT = TP$.
\end{enumerate} \textup{if and only if} .
\end{enumerate}
\end{lemma}
\begin{proof}
(1) The invariance of $U$ under $T$ can be stated as $TP[V] \subseteq \im P$. By \ref{idempotentImageEquivalence} this can be restated as $TPv = PTPv$ for all $v\in V$.

(2) Points (a) and (b) are equivalent by \ref{idempotentImageEquivalence}.

Points (b) and (c) are equivalent by point (1) and $(\id_V - P)T(\id_V - P) = T - PT - TP + PTP = T(\id_V - P) + PTP - PT$.

Point (d) follows immediately from (c). The converse follows from $P(TP) = P(PT) = PT$ and $(PT)P = (TP)P = TP$.
\end{proof}

\subsubsection{Irreducible operators}
\begin{definition}
Let $V$ be a vector space and $T$ an operator on $V$. Then $T$ is called \udef{irreducible} if $V$ is irreducible w.r.t. $T$.
\end{definition}

\section{Sets of vectors}
\begin{proposition}
Let $V$ be vector space and consider a function $f: \powerset{V} \to \powerset{V}$ and define
\[ \mathcal{X} = \setbuilder{X\subseteq V}{A \subseteq X \implies f(A)\subseteq X}. \]
Then $\mathcal{X}$ is closed under arbitrary intersections and thus a complete sublattice of $\powerset(V)$.

The closure operator into $\mathcal{X}$ is given by the intersection of all supersets in $\mathcal{X}$.
\end{proposition}

Most of the types of sets of vectors in this section are of this form.

\subsection{Convex sets}
\begin{definition}
A subset $C$ of a real or complex vector space $V$ is called \udef{convex} if for all $x,y\in C$ and $0\leq r \leq 1$, $rx + (1-r)y\in C$.

The closure of a set $X\subseteq V$ into the lattice of convex sets is called the \udef{convex hull} of $X$ and is denoted $\convex(X)$.
\end{definition}
Note that this is a stronger property than metric convexity!
\begin{example}
Let $C$ be the set of all vectors with norm in $\Q\cap [0,1]$. The is metrically convex, but not a convex set of vectors.
\end{example}

\begin{lemma}
Let $V$ be a vector space and $X\subseteq V$ a subset. Then $\convex(X) = \setbuilder{rx + (1-r)y}{0\leq r \leq 1, x,y\in B}$.
\end{lemma}

\begin{lemma}
Let $V$ be a vector space an $X\subseteq V$ a subset. Then the following are equivalent:
\begin{enumerate}
\item $X$ is convex;
\item for all $0\leq r \leq 1$, $rX + (1-r)X \subseteq X$.
\end{enumerate}
\end{lemma}

\begin{lemma} \label{translationScalingConvexSet}
Let $V$ be a vector space over $\F$, $v\in V$, $\lambda\in \F$ and $X\subseteq V$ a convex subset subset. Then $v+\lambda X$ is convex.
\end{lemma}
\begin{proof}
Take $v+\lambda x_1, v+\lambda x_2 \in v+\lambda X$ and $r\in [0,1]$. Then
\[ r(v+\lambda x_1) + (1-r)(v+\lambda x_2) = v + \lambda(rx_1 + (1-r)x_2) \in v+\lambda X. \]
\end{proof}

\subsection{Cones}
\begin{definition}
A subset $C$ of a real or complex vector space $V$ is called a \udef{cone} if for all real $r>0$, $rC \subseteq C$. A cone is called
\begin{itemize}
\item \udef{pointed} if it contains the origin and \udef{blunt} if not;
\item \udef{flat} if $\exists x\neq 0: x\in C \land -x\in C$, and \udef{salient} if not.
\end{itemize}
\end{definition}

The closure of a set $X\subseteq V$ into the lattice of cones is given by $\R\cdot X$.

\begin{lemma} \label{convexityAdditiveClosure}
A cone $C$ is convex if and only if $C + C \subseteq C$. 
\end{lemma}
\begin{proof}
Assume $C$ convex. Take $v,w\in C$, then $v/2 + w/2\in C$ by convexity and so $v+w = 2(v/2+w/2)\in C$.

Assume $C$ closed under addition. Take $v,w\in C$ and $\lambda\in[0,1]$. Then $(1-\lambda)v$ and $\lambda w$ are elements of $C$ and so the convex combination $(1-\lambda)v + \lambda w$ is too.
\end{proof}

\subsection{Balanced set}
\begin{definition}
A subset $B$ of a vector space $V$ over a field $\F$ with valuation $|\cdot|$ is called \udef{balanced} if for all $|r|\leq 1$, $rC \subseteq C$.

\begin{itemize}
\item The closure of a set $X\subseteq V$ into the lattice of balanced sets is called the \udef{balanced hull} of $X$ and is denoted $\balanced(X)$.
\item The dual closure of a set $X\subseteq V$ into the lattice of balanced sets is called the \udef{balanced core} of $X$ and is denoted $\balancedCore(X)$.
\end{itemize}
\end{definition}
Note that $0$ is an element of any balanced set. The lattice of balanced sets is closed under unions and thus a complete sublattice of $\powerset(X)$.

\begin{lemma}
Let $V$ be a vector space and $B\subseteq V$ a subset. Then
\begin{enumerate}
\item $\balanced(B) = \bigcup_{|r|\leq 1}rB = \ball(0,1)\cdot B$;
\item $\balancedCore(B) = \begin{cases}
\bigcap_{|r|\geq 1}rB & 0\in B\\
\emptyset & 0\notin B
\end{cases}$.
\end{enumerate}
\end{lemma}

\begin{lemma}
Let $V$ be a vector space and $B\subseteq V$ a subset. Then the following are equivalent:
\begin{enumerate}
\item $B$ is balanced;
\item $\balanced(B) = \ball(0,1)\cdot B \subseteq B$;
\item $B\subseteq \balancedCore(B)$;
\item for all $|r|\geq 1$, $C\subseteq rC$.
\end{enumerate}
\end{lemma}

\begin{lemma} \label{balancedLemma}
Let $V$ be a vector space and $B\subseteq V$ a balanced subset. Then
\begin{enumerate}
\item for all $\lambda\in \F$: $\lambda B = |\lambda| B$.
\end{enumerate}
\end{lemma}

\begin{lemma} \label{balancedCoreConvexSet}
The balanced core of a convex set is convex.
\end{lemma}
\begin{proof}
Let $B\subseteq V$ be a convex subset of a vector space $V$. Then
$\balancedCore(B) = \begin{cases}
\bigcap_{|r|\geq 1}rB & 0\in B\\
\emptyset & 0\notin B
\end{cases}$. The empty set is convex. For all $r\in \F$, $rB$ is convex by \ref{translationScalingConvexSet} and arbitrary intersections of convex sets are convex.
\end{proof}

\subsection{Absolutely convex sets}
\begin{definition}
A subset $B$ of a vector space $V$ over a field $\F$ with valuation $|\cdot|$ is called \udef{absolutely convex} or \udef{disked} if it is convex and balanced.

The closure of a set $X\subseteq V$ into the lattice of balanced sets is called the \udef{absolute convex hull} or \udef{disked hull} of $X$ and is denoted $\disked(X)$.
\end{definition}

\begin{lemma}
Let $V$ be a vector space and $X\subseteq V$ a subset. Then the following are equivalent:
\begin{enumerate}
\item $X$ is absolutely convex;
\item for all $x,y\in X$ and $|r| \leq 1$:  $rx + (1-r)y\in X$;
\item for all $|a|+|b| \leq 1$, $aX +bX \subseteq X$;
\item for all $|a|+|b| \leq |c|$, $aX +bX \subseteq cX$.
\end{enumerate}
\end{lemma}

\begin{lemma}
Let $V$ be a vector space and $X\subseteq V$ a subset. Then $\disked(X) = \convex(\balanced(X))$.
\end{lemma}
In general $\disked(X) \neq \balanced(\convex(X))$.

\subsection{Absorbing sets}
\begin{definition}
Let $V$ be a vector space and $A,B\subseteq V$. The $A$ \udef{absorbs} $B$ if there exists a real $r>0$ such that for all $|c| \geq r$: $B\subseteq cA$.

The set $A$ is called \udef{absorbing} if it absorbs $\{v\}$ for all $v\in V$.
\end{definition}

\begin{lemma}
Let $V$ be a vector space and $A\subseteq V$ a subset. Then the following are equivalent:
\begin{enumerate}
\item $A$ is absorbing;
\item for all $v\in V$ there exists an $\epsilon\in \F$ such that $\epsilon v\in A$.
\end{enumerate}
\end{lemma}

\subsection{Translation invariance}
TODO Unique factorisation through $(x,y)\mapsto y-x$. (Universal property)

eg kernel, commutator, metric

\subsubsection{Quotient spaces}
TODO: need closed $U$? For quotient map to be continuous? TODO show quotient topology.

\begin{proposition}
Let $V$ be a vector space. Then $\mathfrak{q}\subset V\times V$ is a congruence \textup{if and only if} the set
\[ U_\mathfrak{q} = \setbuilder{w-v}{(v,w)\in\mathfrak{q}} \]
is a vector space.
\end{proposition}
\begin{proof} Then

- $\mathfrak{q}$ is reflexive iff $0\in U_\mathfrak{q}$;

- $\mathfrak{q}$ is symmetric iff $U_\mathfrak{q}$ is closed under multiplication with $-1$;

- $\mathfrak{q}$ is transitive iff $U_\mathfrak{q}$ is closed under addition;

- $\mathfrak{q}$ is a subalgebra of $V\oplus V$ iff $U_\mathfrak{q}$ is closed under addition and scalar multiplication.

As $U_\mathfrak{q}$ is a subset of $V$, we use the subspace criterion.
\end{proof}
Then the equivalences
\[ [v]_\mathfrak{q}=[w]_\mathfrak{q} \iff (v,w)\in\mathfrak{q} \iff w-v\in U_\mathfrak{q} \iff w+U_\mathfrak{q} = v+U_\mathfrak{q} \]
motivate the following definition:
\begin{definition}
Let $V$ be a vector space.
\begin{itemize}
\item An \udef{affine subset} of $V$ is a subset of $V$ of the form $v+U$ for some $v\in V$ and some subspace $U$ of $V$.
\item An affine subset $v+U$ is \udef{parallel} to $U$.
\end{itemize}
Suppose $U$ subspace of $V$. The \udef{quotient vector space} $V/U$ is the vector space of all affine subsets of $V$ parallel to $U$:
\[ V/U = \{ v+U \;|\; v\in V \}, \]
which is a vector space by virtue of being a quotient algebra.

We call the dimension of $V/U$ the \udef{codimension} of $U$ in $V$:
\[ \codim(U) = \dim(V/U). \]
\end{definition}

\begin{proposition}
Let $U$ be a subspace of a vector space $V$. Then
\[ \dim V = \dim U + \dim V/U = \dim U + \codim U.  \]
\end{proposition}
\begin{proof}
Apply the dimension theorem for linear maps to the quotient map.
\end{proof}

\begin{definition}
Let $f:V\to W$ be a linear map of vector spaces. The \udef{cokernel} of $f$ is the quotient space
\[ \coker(f) = W/\im(f). \]
The dimension of the cokernel is called the \udef{corank}.
\end{definition}
\begin{lemma}
Let $U$ be a subspace of a vector space $V$. The codimension of $U$ is the corank of the inclusion $U\hookrightarrow V$:
\[ \codim(U) = \dim\coker(U\hookrightarrow V). \]
\end{lemma}

\begin{proposition} \label{splittingMap}
Let $T\in \Lin(V,W)$. Then $T$ induces a linear map
\[ \tilde{T}: V/\ker(T) \to W: v +\ker(T) \mapsto Tv \]
with the following properties:
\begin{enumerate}
\item $\tilde{T}$ is injective;
\item $\im\tilde{T} = \im T$;
\item $\tilde{T}$ is an isomorphism from $V/\ker(T)$ to $\im T$.
\end{enumerate}
\end{proposition}

 TODO each short exact sequence of vector spaces splits \url{https://en.wikipedia.org/wiki/Rank%E2%80%93nullity_theorem}






\chapter{Modules}
TODO: $*$-modules
\section{Representation theory}
A representation $G\to \GL(V)$ gives $V$ the structure of a $G$-module.

A map $\varphi$ between two representations $V,W$ of $G$ is a vector space map $\varphi: V\to W$ such that
\begin{center}
\begin{tikzcd}
V\arrow[r, "\varphi"]\arrow[d, "g"]& W\arrow[d, "g"] \\V\arrow[r, "\varphi"]& W
\end{tikzcd}
commutes for every $g\in G$.
\end{center}
In other symbols $\forall g\in G: g\varphi = \varphi g$.
We call such a map $G$-linear.

Define isomorphism and isomorphic.

Eg trivial representation: $gv = v$.

A \udef{subrepresentation} of a representation $V$ is a vector subspace $W$ of $V$ that is invariant under $G$: $\forall g\in G: g[W]\subseteq W$.

$\ker \varphi$, $\coker \varphi$, $\Im \varphi$ subrepresentations.

Irrep has no proper, non-trivial subrepresentation.

\begin{lemma} \label{existenceIrreps}
Every representation has an irreduciple subrepresentation.
\end{lemma}
\begin{proof}
If $V$ has no non-trivial subrepresentations, then $V$ is simple and we are done. 
Otherwise take the set of non-trivial subrepresentations. This forms a poset ordered by inclusion and by the maximal chain principle \ref{ZornEquivalents} this poset has a maximal chain $C$ and it is clear that $\bigcap C$ is a simple subrepresentation. In particular $\bigcap C$ is closed because it is an intersection of closed subspaces.
\end{proof}

Representation gives representation on dual.

Direct sums and tensor products of representations. Also symmetric and exterior powers.

$\Hom(V,W)$ has representation via $V^*\otimes W$.

A space is irreducible if and only if it is completely reducible and indecomposable.

\begin{proposition}
Let $\varphi$ be a $G$-linear map. If $\varphi$ is invertible as a function, its inverse is also $G$-linear.
\end{proposition}
\begin{proof}
Assume $\varphi$ invertible and take an arbitrary $g\in G$. Then
\[ g\varphi = \varphi g \implies \varphi^{-1} g = g \varphi^{-1} \]
by multiplying left and right by $\varphi^{-1}$. So
\[ \forall g\in G: g \varphi^{-1} = \varphi^{-1} g \]
meaning $\varphi^{-1}$ is $G$-linear.
\end{proof}

\begin{proposition}
Let $(V_1,\rho_1)$ and $(V_2,\rho_2)$ be isomorphic, then $V_1$ is irreducible \textup{if and only if} $V_2$ is irreducible.
\end{proposition}

\begin{proposition}
Let $(V,\rho)$ be a representation of $G$. Every element of the centre $Z(G)$ of $G$ defines an isomorphism $V\to V$.
\end{proposition}
\begin{proof}
Every $g_0\in Z(G)$ defines a $G$-linear map $\rho(g_0)$:
\[ \forall g\in G: \rho(g_0)\rho(g) = \rho(g_0g) = \rho(gg_0) = \rho(g) \rho(g_0). \]
The map $\rho(g_0)$ has an inverse $\rho(g_0^{-1})$.
\end{proof}

\begin{proposition}
Let $G$ be an Abelian group. The irreps of $G$ are $1$-dimensional and thus homomorphisms
\[ \rho: G\to \GL(\C). \]
\end{proposition}
\begin{proof}
Let $V$ be an irrep. The action of each $g\in G$ is an isomorphism and thus a scalar multiple by Schur's lemma. Thus every subspace of $V$ must be invariant, so also a subrepresentation. This means $V$ may not have any proper, non-trivial subspaces, meaning it is $1$-dimensional.
\end{proof}

\section{Hilbert modules}
Let $B$ be a $C^*$-algebra. A \udef{Hilbert $B$-module} $E$ is essentially a $B$-module with a $B$-valued inner product $\inner{\cdot, \cdot}_B: E\times E \to B$.

To be more precise: a (right) Hilbert $B$-module is a complex Banach space $E$ equipped with a right $B$-module structure and a positive\footnote{I.e. $\inner{\xi, \xi}_B$ is an element of the positive cone $B^+$.} definite $B$-valued inner product which is linear in the second and anti-linear in the first and satisfies, for all $\xi,\eta \in E$ and $b\in B$
\[ (\inner{\xi,\eta}_B)^* = \inner{\eta,\xi}_B, \qquad \inner{\xi, \eta}_B b = \inner{\xi,\eta\cdot b}_B, \qquad \text{and} \qquad \norm{\xi}^2 = \norm{\inner{\xi,\xi}_B}. \]
We can also define left Hilbert $B$-modules analogously. The Hilbert $\C$-modules are precisely the complex Hilbert spaces.

Any $C^*$-algebra $B$ can be seen as a Hilbert $B$-module by equipping it with the following inner product:
\[ \inner{\cdot, \cdot}_B: B\times B\to B: (a,b) \mapsto a^*b. \]

If $E$ and $F$ are Hilbert $B$-modules, then a map $T: E \to F$ is called \udef{adjointable} if there exists a map $T^*: F \to E$ such that for all $\xi\in E, \eta\in F: \inner{T\xi,\eta}_B = \inner{\xi, T^*\eta}_B$. Adjointable operators are bounded and $B$-linear.



\chapter{Algebras}
TODO $\GL(A)$ which forms group under multiplication.

Multiplicative map: preserves multiplication.

Anti-commute

representation = algebra homomorphism with linear operators on a vector space.

Envelope of a representation: module to algebra.

\section{Semisimple algebra}
\begin{proposition}
Let $V$ be a Hilbert space with a subspace $U$. Let $A$ be a bounded linear operator on $V$. If $U$ is stable under $A$, then $U^\perp$ is stable under $A^*$.
\end{proposition}
\begin{proof}
Let $u\in U$ and $v\in U^\perp$. Then from
\[ \inner{u, A^*v} = \inner{Au,v} = 0 \]
we see that $A^*v \in U^\perp$. Thus $U^\perp$ is stable under $A^*$.
\end{proof}
\begin{corollary}
Let $D$ be a ring of bounded linear operators on the Hilbert space $V$. If $D=D^*$, then $V$ is a semisimple $D$-module.
\end{corollary}
\begin{proof}
Let $S$ be the set of direct sums of simple subrepresentations of $V$:
\[ S = \setbuilder{\bigoplus_{i\in I}V_i }{\text{$V_i$ simple subrepresentations of $V$ and all $V_i$ are orthogonal}}. \]
Then $S$ is a poset ordered by inclusion. Now any chain $C$ in $S$ has an upper bound $\bigcup C$ and $\bigcup C$ is in $S$ because every $v\in\bigcup C$ can be written uniquely as a finite linear sum
\[ v = \sum_{\substack{i\in J\\ \text{$J$ finite}}} v_i \qquad v_i\in V_i. \]
By Zorn's lemma $S$ has a maximal element $U$, which is closed by \ref{directSumOrthogonalClosed}. We now claim that $U=V$. Assume, towards a contradiction, that $U\neq V$. Then $U^\perp$ is stable under $D$ by the proposition, closed by \ref{orthogonalComplementClosed} and thus contains a simple subrepresentation $W$ by \ref{existenceIrreps}. Then $U\subset U\oplus W \in S$, meaning $U$ is not a maximal element. This is a contradiction.
\end{proof}
Note that for the corollary it is important that $V$ be a Hilbert space, not only for the condition $D=D^*$ which could also be fulfilled by a set of symmetric operators or a group of unitary operators.


\section{Graded algebras and filtrations}

\section{Tensor algebra}
\[ \mathcal{T}(V) \defeq \R \bigoplus_{n=1}^\infty V^n = \R \bigoplus_{n=1}^\infty \underbrace{V\otimes \ldots \otimes V}_{\text{$n$ times}}. \]

Transpose: $v\otimes w \to w\otimes v$.

\subsection{Tensor product}
+ Graded tensor product

\section{Matrix algebras}
\begin{definition}
TODO
\end{definition}

\subsection{Natural isomorphism}
Remove parentheses block matrix.

Also $A^{n\times n} \cong \C^{n\times }\otimes A$.

\chapter{Lie groups and Lie algebras}
\section{Definitions}
\[ g(x) = \exp(ix^aX_a) \]
Lie algebra has the operation $[X_a,X_b]= X_aX_b - X_bX_a$. 
\section{Matrix groups}
We now consider some extremely important examples of topological groups: the matrix groups.
If we take the set of real, $N\times N$ matrices with a non-zero determinant, it turns out that they form a group with the matrix multiplication:
\begin{enumerate}
\item The matrix multiplication is associative;
\item The identity is the identity matrix $\mathbb{1}$;
\item Because their determinant is not zero, every matrix in this set has an inverse.
\item Because the matrices are square, the multiplication of two matrices gives a matrix of the same dimensions. In other words the matrix multiplication is a closed operation.
\end{enumerate}
We call this group the \udef{real general linear group} $\GL(N, \R)$. It also has a complex counterpart, the complex general linear group $\GL(N, \C)$.

TODO: topological
We can also immediately see that the operations of matrix multiplication and inversion are smooth. (For inversion this is obviously only true after restriction to the open subset of invertible matrices, which luckily all matrix Lie groups are in turn a subset of). This follows quite readily because both operations are in effect comprised of addition and multiplication operations, which are infinitely differentiable. (E.g. $A^{-1} = \frac{1}{\det(A)}\mathrm{adj}(A)$)

\begin{example}
TODO: $A^2 = \mathbb{1}$
\end{example}

These groups, along with all their subgroups, are known as the matrix groups and are very important in physics.

\subsubsection{Continuous parameters}
It is sometimes interesting to know how many degrees of freedom a particular set of transformations has. For example, rotations in the 2D plane are characterized with one parameter: the angle of rotation. In 3D we need three parameters. This notion of continuous parameter is formalised below.

\begin{definition}
A function $A : \R \to \GL(n, \C)$ is called a \udef{one-parameter subgroup} of $\GL(n, \C)$ if
\begin{enumerate}
\item $A$ is continuous,
\item $A(0) = \mathbb{1}_n$,
\item $A(t+s) = A(t)A(s)$ for all $t,s \in \R$.
\end{enumerate}
We also call the image of $A$ a one-parameter subgroup.
\end{definition}

A one-parameter subgroup has one continuous parameter. A subgroup of $\GL(n, \C)$ with $m$ continuous parameters, is a function $A : \R^m \to \GL(n, \C)$ such that each function of the form
\[ x \mapsto A(a_1, a_2, \ldots , a_{i-1}, x, a_{i+1}, \ldots, a_m) \]
gives a one-parameter subgroup for fixed $a_1,\ldots, a_m$.

We can speak of an $m$-parameter subgroup because, while different parametrisations may be found, any subgroup of $\GL(n,\C)$ constructed in this way must always be constructed with the same number of parameters. To see that this must be the case, consider two parametrised subgroups $A : \R^m \to \GL(n, \C)$ and $B : \R^{m'} \to \GL(n, \C)$ with the same image.

TODO !! + dimension of manifold

\subsubsection{Examples}
We now give names to the most important matrix groups, and list the number of continuous parameters.
\begin{enumerate}
\item General linear group
\[ \GL(N,\R) = \{N\times N \;\text{real matrices}, \; \det M \neq 0\} \]
\begin{itemize}
\item We have $N^2$ independent parameters (= the entries of the matrix), so $\dim \GL(N,\R) = N^2$
\item Each complex number can be described with two real ones, so $\dim \GL(N, \C) = 2N^2$
\end{itemize}
\item Special linear group
\[ \SL(N,\R) = \{M\in\GL(N,\R), \;\det M = 1\} \]
\begin{itemize}
\item $\dim \SL(N,\R) = N^2-1$: 1 dimension is used to fix determinant.
\item $\dim \SL(N,\C) = 2(N^2-1)$: 1 dimension is used to fix the real part of the determinant, and 1 to fix the imaginary part.
\end{itemize}
\item Unitary matrices
\[ \U(N) = \{U\in\GL(N,\C), U^\dagger\mathbb{1}_NU = \mathbb{1}_N\} \]
\begin{itemize}
\item $U^\dagger U$ is Hermitian, meaning that the complex transpose of $U$ is $U$.
\item $U^\dagger U = \mathbb{1}_N$ yields only $N^2$ independent equations, not $2N^2$ because of the Hermiticity of the equation.
\item $\dim \U(N) = 2N^2-N^2$ = $N^2$
\end{itemize}
\item Special unitary groups
\[ \SU(N) = \{U\in\U(N),\; \det U = 1\} \]
\begin{itemize}
\item For unitary matrices we have that $|\det U| = 1$. This fixes one continuous parameter and thus one dimension.
\item $\dim \SU(N) = N^2 - 1$
\end{itemize}
\item Orthogonal groups
\begin{itemize}
\item $\Ogroup(N) = \{O\in\GL(N,\R),\; O^\intercal\mathbb{1}_N O = \mathbb{1}_N\}$
\begin{itemize}
\item $O^\intercal O$ is symmetric, so $\frac{N(N+1)}{2}$ independent equations (half the matrix already fixed by the other half)
\item $\dim \Ogroup = N^2 - \frac{N(N+1)}{2} = \frac{N(N-1)}{2}$
\end{itemize}
\item $\SO(N) = \{O\in\Ogroup(N),\; \det O = 1\}$
\begin{itemize}
\item For orthogonal matrices $\det O = \pm 1$. This does not fix any continuous parameters.
\item $\dim \SO = \dim \Ogroup = \frac{N(N-1)}{2}$
\end{itemize}
\end{itemize}
\item Using a non definite metric $\eta = \diag(\mathbb{1}_p, -\mathbb{1}_q)$
\begin{itemize}
\item $\U(p,q) = \{ U\in \GL(N,\C), U^\dagger\eta U = \eta \}$
\item $\Ogroup(p,q) = \{ O\in \GL(N,\R), O^\intercal\eta O = \eta \}$ \\
In particular $\SO(1,3)$ is the \udef{Lorentz group} (with mostly minus convention).
\end{itemize}
\end{enumerate}

Here are some of the most important examples written more explicitly in terms of their continuous parameters:
\begin{itemize}
\item $\U(1) \equiv \{z\in\mathbb{C}|\; |z|=1\}, \boldsymbol{\cdot}$ has one real parameter. Every element $z$ of this group can be written $z=e^{i\alpha}$ for a real $\alpha$.
\item $\SO(2)$ has one real parameter.
\[ R(\theta) = \begin{pmatrix}\cos(\theta) & -\sin(\theta)\\ \sin(\theta) & \cos(\theta)\end{pmatrix} \]
\item $\SO(3)$ has three real parameters.
\[ R(\theta_{12},\theta_{13},\theta_{23}) = R_1(\theta_{12})R_2(\theta_{13})R_3(\theta_{23}) \]
where
\[R_1(\theta_{12}) = \begin{pmatrix}\cos(\theta_{12}) & -\sin(\theta_{12})&0\\ \sin(\theta_{12}) & \cos(\theta_{12})&0\\0&0&1\end{pmatrix}\]
\[R_2(\theta_{13}) = \begin{pmatrix}\cos(\theta_{13}) &0& -\sin(\theta_{13})\\0&1&0\\ \sin(\theta_{13}) &0& \cos(\theta_{13})\end{pmatrix}\]
\[R_3(\theta_{23}) = \begin{pmatrix}1&0&0\\ 0&\cos(\theta_{23}) & -\sin(\theta_{23})\\0& \sin(\theta_{23}) & \cos(\theta_{23})\end{pmatrix}\]
\item $\SU(2)$ has three real parameters and its elements can be seen as complex $2\times 2$ rotations.
\[ \U(\alpha, \beta, \gamma) = \begin{pmatrix}\cos\theta e^{i\alpha} & -\sin\theta e^{i\beta}\\ \sin\theta e^{-i\beta} & \cos\theta e^{-i\alpha}\end{pmatrix} \]
\end{itemize}


\chapter{Representation theory}
\section{Finite groups}
\subsection{Character tables}
\subsubsection{For $\mathbb{Z}_n$}
Denoting $\mathbb{Z}_n = \{\bar{0}, \bar{1}, \bar{2},\ldots, \overline{n-1}\}$
\[ \begin{array}{l|lllll}
g_i & \bar{0} & \bar{1} & \bar{2} & \hdots & \overline{n-1} \\
|\text{Cl}| & 1 & 1 & 1 & \hdots & 1 \\ \hline
\chi_0 & 1 & 1 & 1 & \hdots & 1 \\
\chi_1 & 1 & \omega_n & \omega^2_n & \hdots & \omega_n^{n-1} \\
\chi_2 & 1 & \omega_n^2 & \omega^4_n & \hdots & \omega_n^{2(n-1)} \\
\vdots & \vdots & \vdots & \vdots &  & \vdots \\
\chi_{n-1} & 1 & \omega_n^{n-1} & \omega_n^{2(n-1)} & \hdots & \omega_n^{(n-1)(n-1)}
\end{array} \]

\subsection{Complete reducibility of complex representations}

\subsection{Schur's lemma, isotypic decomposition and duals}
\subsection{Orthogonality in the character tables}
\subsection{The sum of squares formula}
\subsection{The number of irreps is the number of conjugacy classes}
\subsection{Dimensions of irreps divide the order of the group}

\chapter{Normed spaces and inner product spaces}
In this chapter we will always use either $\mathbb{F} = \R$ or $\mathbb{F} = \C$.

TODO: \url{https://math.stackexchange.com/questions/2151779/normed-vector-spaces-over-finite-fields/2568231}
\section{Normed spaces}
\begin{definition}
A \udef{norm} on a vector space $V$ is a function
\[ \norm{\cdot}: V \to \mathbb{R} \]
that has the following properties:
\begin{itemize}[leftmargin=6cm]
\item[\textbf{Triangle inequality}\footnote{Also known as the property of being \udef{subadditive}.}] $\norm{u+v} \leq \norm{u}+\norm{v}$;
\item[\textbf{Absolute homogeneity}] $\norm{\lambda v} = |\lambda|\cdot\norm{v}$;
\item[\textbf{Point-separating}] If $\norm{v} = 0$, then $v = 0$.
\end{itemize}
A \udef{seminorm} is a function $V\to \mathbb{R}$ that is subadditive and absolutely homogeneous.

A \udef{normed space} $(\mathbb{F},V,+,\norm{\cdot})$ is a vector space $(\mathbb{F},V,+)$ equipped with a norm $\norm{\cdot}$.
\end{definition}
\begin{lemma}
A subadditive, absolutely homogenous function $f:V\to \R$ is non-negative:
\[ f: V\to \R_{\geq 0}. \]
Thus norms and seminorms are functions $V\to \R_{\geq 0}$.
\end{lemma}
\begin{proof}
TODO
\end{proof}

\begin{lemma}[Reverse triangle inequality]
Let $(V,\norm{\cdot})$ be a normed space. Then $\forall v,w\in V: |\norm{v}-\norm{w}|\leq \norm{v-w}$.
\end{lemma}
\begin{proof}
$\norm{v} = \norm{v-w+w} \leq \norm{v-w} + \norm{w}$.
\end{proof}

\begin{definition}
A vector with norm 1 is called a \udef{unit vector}. Unit vectors are often written with a hat:
\[ \norm{\vhat{v}} = 1. \]
\end{definition}

\begin{lemma}
A subspace of a normed vector space is a normed space, with the norm given by the restriction of the norm in the larger space.
\end{lemma}

\begin{proposition}
Every normed space can be viewed as a metric space with the metric $d:V\times V \to [0,\infty[$ given by
\[ d(x,y) = \norm{x-y}. \]
This metric has the properties of
\begin{itemize}[leftmargin=6cm]
\item[\textbf{Translation invariance}] $d(x+a, y+a) = d(x,y)$;
\item[\textbf{Scaling}] $d(\lambda x, \lambda y) = |\lambda|d(x,y)$.
\end{itemize}
Conversely, any metric with translation invariance and scaling determines a norm:
\[ \norm{x} = d(x,\vec{0}). \]
Passing from norm to metric back to norm, we recover the original norm.
\end{proposition}
\begin{lemma}
A linear map $L:V\to W$ between normed spaces is an isometry for the metric \textup{if and only if} it preserves the norm, i.e.
\[ \forall v\in V: \quad \norm{v}_V = \norm{L(v)}_W. \]
\end{lemma}
\begin{proof}
Assume $L$ is an isometry, then
\[ \norm{v} = d(v,\vec{0}) = d(L(v),L(\vec{0})) = \norm{L(v) - L(\vec{0})} = \norm{L(v) - \vec{0}} = \norm{L(v)}. \]
Assume $L$ preserves the norm, then
\[ d(L(v_1), d(v_2)) = \norm{L(v_1)-L(v_2)} = \norm{L(v_1-v_2)} = \norm{v_1-v_2} = d(v_1,v_2). \]
\end{proof}

\begin{proposition}
Let $V$ be a normed vector space, then the norm $\norm{\cdot}:V\to \R$ is a continuous map.
\end{proposition}
\begin{proof}
The reverse triangle inequality, $|\norm{v}-\norm{w}| \leq norm{v-w}$, implies that the norm is Lipschitz continuous with Lipschitz constant $1$, so we can use \ref{lemma:LipschitzcontinuousContinuous}.
\end{proof}

\begin{definition}
Let $V$ be a vector space. Two norms $\norm{\cdot}_1$ and $\norm{\cdot}_2$ on $V$ are \udef{equivalent} if there exist $a,b\in \R$ such that
\begin{align*}
\forall v\in V: a\norm{v}_1&\leq \norm{v}_2 \\
\forall v\in V: b\norm{v}_2&\leq \norm{v}_1
\end{align*}
\end{definition}

\begin{proposition}
Equivalent norms induce the same topology.
\end{proposition}
\begin{proof}
TODO
\end{proof}

Because normed product spaces are metric spaces, we have a notion of convergence and can define infinite sums:
\begin{definition}
In a normed space $V$, we can define a \udef{infinite linear combination} as an infinite sum
\[ \sum_{i\in I} c_i v_i  \]
where $\{v_i\}_{i\in I}$ is a set of vectors and $\{c_i\}_{i I}$ a set of scalars, if that sum converges in the norm topology.
\end{definition}
\begin{note}
This finite sum is defined using nets:
Ordered by inclusion, the set $J = \{I'\subset I \;|\; I' \; \text{is finite}\}$ is a directed set. This means
\[ \left(\sum_{i\in A}c_iv_i \right)_{A\in J} \]
is a net. The infinite sum is defined if this net converges.
\end{note}

\begin{lemma}
Every proper subspace $U$ of a normed vector space $V$ has empty interior.
\end{lemma}
\begin{proof}
Suppose $U$ has a non-empty interior. Then it contains some ball $B(u,\epsilon)$. Now every vector in $V$ can be translated and rescaled to fit inside the ball $B(u,\epsilon)$. Indeed let $v\in V$ and set $u' = u+ \frac{\epsilon}{2\norm{v}}v \in B(u,\epsilon)$. Then, since $U$ is a subspace $v = \frac{2\norm{v}}{\epsilon}(u'-u)\in U$. So $U=V$.
\end{proof}

\begin{lemma}[Riesz's lemma] \label{lemma:RieszsLemma}
Let $V$ be a normed vector space. Given a non-dense subspace $X$ and a number $\theta<1$, there exists a unit vector $v\in V$ such that
\[ \theta \leq d(X,v) = \inf_{x\in X}\norm{x-v}. \]
\end{lemma}
\begin{proof}
Take a vector $v_1$ not in the closure of $X$ and put $a = \inf_{x\in X}\norm{x-v_1}$. Then $a>0$ by lemma \ref{lemma:sequencesSupInf}. For $\epsilon > 0$, let $x_1\in X$ be such that $\norm{x_1+v_1}<a+\epsilon$. Then take
\[ v = \frac{v_1 - x_1}{\norm{v_1-x_1}} \qquad \text{so} \qquad \norm{v}=1. \]
And
\[ \inf_{x\in X}\norm{x-v} = \inf_{x\in X}\norm{x-\frac{v_1 - x_1}{\norm{v_1-x_1}}} = \inf_{x\in X}\norm{\frac{x-v_1 + x_1}{\norm{v_1-x_1}}} = \frac{\inf_{x\in X}\norm{x-v_1}}{\norm{v_1-x_1}} \geq \frac{a}{a+\epsilon}. \]
By choosing $\epsilon >0$ small, $a/(a+\epsilon)$ can be made arbitrarily close to $1$.
\end{proof}
For finite-dimensional spaces we can even take $\theta=1$.

\subsection{Linear independence and bases in normed spaces}
\url{https://math.stackexchange.com/questions/1518029/are-uncountable-schauder-like-bases-studied-used}

\subsection{Finite-dimensional normed (sub)spaces}

\begin{lemma} \label{lemma:coordinateContinuity}
Let $V$ be a normed vector space and $\{x_1, \ldots, x_n\}$ a linearly independent set of vectors. There exists a $c>0$ such that $\forall \alpha_1,\ldots, \alpha_n \in \mathbb{F}$:
\[ \norm{\alpha_1x_1 + \ldots + \alpha_nx_n} \geq c(|\alpha_1|+\ldots+|\alpha_n|) . \]
\end{lemma}
\begin{proof}
TODO ref locally convex spaces?
\end{proof}
TODO This is equivalent with continuity of coordinate functions.

\begin{proposition} \label{prop:finiteDimComplete}
Every finite-dimensional subspace of a normed vector space is complete.
\end{proposition}
\begin{proof}
Take a basis $\{e_i\}_{i=1}^n$ and let $c$ be as in lemma \ref{lemma:coordinateContinuity}. Consider an arbitrary Cauchy sequence $(v_k)_{k\in\N}$. We can write
\[ v_k = \alpha_{k,1}e_1 + \ldots + \alpha_{k,n}e_n. \]
We claim that $(\alpha_{k,i})_{k\in\N}$ is Cauchy in $\mathbb{F}$ for all $1\leq i\leq n$. Indeed, take an $\epsilon>0$. By the Cauchy nature of $(v_k)_{k\in\N}$ we can find a $k_0$ such that $\forall k', k''>k_0:$
\[ c\epsilon > \norm{v_{k'} - v_{k''}} \geq \norm{\sum_{i=1}^n (\alpha_{k',i}-\alpha_{k'',i})e_i}\geq c\sum_{i=1}^n |\alpha_{k',i}-\alpha_{k'',i}| \geq c |\alpha_{k',i}-\alpha_{k'',i}|. \]
Dividing left and right by $c$ gives exactly the Cauchy condition for each $1\leq i\leq n$. By the completeness of $\R$ or $\C$, each of these sequences has a limit $\alpha_i$.
Then $v= \sum_{i=1}^n\alpha_ie_i$ is an element of the subspace. The sequence $(v_k)$ converges to $v$ because
\[ \norm{v_k-v} = \norm{\sum_{i=1}^n (\alpha_{k,i}-\alpha_i)e_i} \leq \sum_{i=1}^n |\alpha_{k,i}-\alpha_i|\norm{e_i} \]
and the right-hand side goes to zero as $k\to \infty$.
\end{proof}
\begin{corollary} \label{corollary:finiteDimClosed}
Every finite-dimensional subspace of a normed vector space is closed.
\end{corollary}
TODO ref for proof.

\begin{proposition}
On a finite-dimensional vector space all norms are equivalent.
\end{proposition}
\begin{proof}
Let $\{e_i\}_{i=1}^n$ be a basis and take an arbitrary vector $v = \sum_{i=1}^nv_ie_i$. Let $\norm{\cdot}_1$ and $\norm{\cdot}_2$ be two norms.
We calculate
\[ \norm{v}_1 \leq \sum_{i=1}^n|v_i|\norm{e_i}_1 \leq k\sum_{i=1}^n|v_i| \leq \frac{k}{c_2}\norm{v}_2 \]
where the first inequality is the triangle inequality, the second comes from $k=\max\norm{e_i}_1$ and the third is lemma \ref{lemma:coordinateContinuity}. A similar calculation gives the other necessary inequality.
\end{proof}

\begin{proposition}
In a finite-dimensional normed space $V$, any subset $M \subseteq V$ is compact if and only if $M$ is closed and bounded.
\end{proposition}
\begin{proof}
TODO + ref Heine Borel property
\end{proof}


TODO: move up?
\begin{proposition} \label{prop:compactnessUnitBall}
The closed unit ball of a vector space is compact \textup{if and only if} the vector space is finite-dimensional.
\end{proposition}
\begin{proof}
One direction is given by the previous proposition. For the other direction, we show the contrapositive: let the vector space be infinite-dimensional.
We define a sequence of unit vectors $(e_i)_{i\in\N}$ recursively as follows:
\begin{itemize}
\item $e_1$ is just a unit vector;
\item for $e_{n+1}$ apply Riesz's lemma \ref{lemma:RieszsLemma} to the subspace $\Span\{e_i\}_{i=1}^n$ and $\theta = 1/2$. This subspace cannot be dense, because it is a closed (by corollary \ref{corollary:finiteDimClosed}) finite-dimensional subspace of an infinite-dimensional vector space.
\end{itemize}
This yields a sequence such that for all $m,n$
\[ \norm{e_m - e_n}\geq \frac{1}{2}. \]
This sequence is not Cauchy and thus not convergent.
\end{proof}







\subsection{Norms on constructed vector spaces}
\subsubsection{Direct sum}
\[ \norm{x\oplus y}_{X\oplus Y} = \norm{x}_X + \norm{y}_Y \]
TODO + arbitrary direct sums.
\subsubsection{The graph norm}
Let $L:V\to W$ be a linear map between normed spaces. The graph of $L$
\[ \setbuilder{(v,w)\in V\oplus W}{w = Lv} \]
has a natural norm inherited from the direct sum:
\[ \norm{(v,Lv)} = \norm{v}_V + \norm{Lv}_W. \]
This norm can also be seen as a norm on $V$: the \udef{graph norm} induced by $L$ is defined as
\[ \norm{v}_L := \norm{v}_V + \norm{Lv}_W. \]


\section{Operators on normed spaces}


\subsection{Bounded operators}
\begin{definition}
An operator $L$ between normed vector spaces is called \udef{bounded} if it is Lipschitz continuous.

In other words, there exists an $M>0$ such that $\forall v\in \dom(L)$
\[ \norm{L(v)} \leq M \norm{v}. \]

The set of bounded operators from $V$ to $W$ is denoted $\Bounded(V,W)$. If $V=W$, we write $\Bounded(V)$.
\end{definition}

\begin{theorem} \label{theorem:boundedLinearMaps}
Let $L$ be a linear operator between normed spaces $V,W$. The following are equivalent:
\begin{enumerate}
\item $L$ is continuous everywhere in $\dom(L)$;
\item $L$ is continuous at $x_0 \in \dom(L)$;
\item $L$ is continuous at $0$;
\item $L$ is bounded.
\end{enumerate}
\end{theorem}
\begin{proof}
We proceed round-robin-style:
\begin{itemize}[leftmargin=2cm]
\item[$\boxed{(1) \Rightarrow (2)}$] Trivial.
\item[$\boxed{(2) \Rightarrow (3)}$] Let $\seq{x_n}$ converge to $0$, then
\[ \lim_{n\to\infty}L(x_n) = \lim_{n\to\infty}L(x_n+x_0) - L(x_0) = L(\lim_{n\to\infty}x_n+x_0) - L(x_0) = L(x_0) - L(x_0) = 0. \]
Continuity follows because normed vector spaces are sequential spaces.
\item[$\boxed{(3) \Rightarrow (4)}$] From continuity at zero, there exists a $\delta>0$ such that $\norm{L(h)} = \norm{L(h)-L(0)} \leq 1$ for all $h\in \dom(L)$ with $\norm{h}\leq \delta$. Thus for all nonzero $v\in \dom(L)$
\[ \norm{L(v)} = \norm{\frac{\norm{v}}{\delta}L(\delta \frac{v}{\norm{v}})} = \frac{\norm{v}}{\delta}\norm{L(\delta \frac{v}{\norm{v}})}\leq \frac{\norm{v}}{\delta}. \]
\item[$\boxed{(4) \Rightarrow (1)}$] Lipschitz continuity implies continuity \ref{lemma:LipschitzcontinuousContinuous}.
\end{itemize}
\end{proof}
\begin{corollary} \label{corollary:boundedAntiLinearMaps}
An anti-linear map between complex vector spaces is also continuous \textup{if and only if} it is bounded.
\end{corollary}
\begin{proof}
An anti-linear map $A:V\to W$ is an $\R$-linear map $A:V_\R\to W_\R$. Now $V_\R, W_\R$ have the same norms as $V,W$ and thus the same topology. So $A:V\to W$ is continuous if and only if $A:V_\R\to W_\R$ is continuous.
\end{proof}
\begin{corollary}
All norm-decreasing homomorphisms are continuous.
\end{corollary}

\begin{proposition}
Let $V,W$ be normed spaces. Then $T:V\to W$ is bounded \textup{if and only if}
$T^{-1}[B(\vec{0},1)]$ has nonempty interior.
\end{proposition}
\begin{proof}
TODO!
\end{proof}

\begin{lemma} \label{lemma:kerClosed}
Let $T$ be a bounded linear operator. Then $\ker(T)$ is closed.
\end{lemma}
\begin{proof}
Suppose $T$ bounded and thus continuous. Then $\ker L = L^{-1}[\{0\}]$ and thus closed, by proposition \ref{prop:continuity}.
\end{proof}
\begin{proof}
Let $v\in \overline{\ker(T)}$. Then find a sequence $(v_n)$ in $\ker(T)$ that converges to $v$. Then by continuity $(Tv_n)$ converges to $Tv$, but for all $n\in\N: Tx_n = 0$, so the limit is $Tv=0$. Thus $v\in\ker(T)$, making it closed.
\end{proof}

\begin{proposition}\label{prop:continuousMapCriterion}
Let $L:V\to W$ be a linear map between normed spaces.
\begin{enumerate}
\item If $V$ is finite-dimensional, then $L$ is continuous.
\item If $W$ is finite-dimensional, then $L$ is continuous \textup{if and only if} $\ker L$ is closed.
\end{enumerate}
\end{proposition}
\begin{proof}
\begin{enumerate}
\item This follows from a consideration of the graph norm $\norm{v}_L = \norm{v}+\norm{Lv}$ and the fact that on a finite-dimensional space any two norms are equivalent: for all $v$ we can choose an $M$ such that
\[ \norm{Lv}\leq \norm{v}_L \leq M\norm{v}. \]
\item Assume $W$ finite-dimensional. Consider the map $\bar{L}:V/\ker L\to W: v+\ker{L}\mapsto L(v)$, defined in proposition \ref{prop:splittingMap}. Then $V/\ker L$ is isomorphic to a subspace of $W$ and thus is finite-dimensional. By the first point, $\bar{L}$ must be continuous. Let $\pi: V\to V/\ker L$ denote the quotient map, which is continuous (TODO is this where closure of $\ker L$ is used?). Then $L = \bar{L}\circ \pi$ is a composition of continuous maps and thus continuous.

Conversely, we have the lemma \ref{lemma:kerClosed}.
\end{enumerate}
\end{proof}

\subsubsection{The normed space of bounded operators}
\begin{lemma} \label{lemma:operatorNorm}
Let $(V,\norm{\cdot}_V)$ and $(W,\norm{\cdot}_W)$ be normed spaces and $L\in\Lin(V, W)$. Then $L$ is bounded \textup{if and only if}
\[ \sup\setbuilder{\frac{\norm{Lx}_W}{\norm{x}_V}}{x\in V\setminus\{0\}} \] 
exists.
\end{lemma}
\begin{definition}
Let $(V,\norm{\cdot}_V)$ and $(W,\norm{\cdot}_W)$ be normed spaces and $L\in\Lin(V, W)$ bounded. Then
\[ \norm{L} \defeq \sup\setbuilder{\frac{\norm{Lx}_W}{\norm{x}_V}}{x\in V\setminus\{0\}} \]
is called the \udef{operator norm} of $L$.
\end{definition}

\begin{proposition} \label{prop:BoundedSpace}
Let $(V,\norm{\cdot}_V)$ and $(W,\norm{\cdot}_W)$ be normed spaces. Then the set $\Bounded(V,W)$ of bounded linear maps is a normed subspace of $\Lin(V,W)$ equipped with the operator norm.
\end{proposition}

\begin{proposition} \label{prop:operatorNorm}
Let $L\in\Bounded(V,W)$ be a bounded operator and let $B(\vec{0},\epsilon)$ be an open ball centered at $\vec{0}$. Then
\begin{align*}
\norm{L} &= \frac{\sup L[B(\vec{0},\epsilon)]}{\epsilon} \\
&= \frac{\sup L[\overline{B}(\vec{0},\epsilon)]}{\epsilon} \\
&= \sup\setbuilder{\norm{Lx}}{\norm{x} = 1}.
\end{align*}
\end{proposition}
\begin{proof}
TODO
\end{proof}

\begin{lemma}
Let $S,T$ be compatible bounded operators. Then
\[ \norm{ST} \leq \norm{S}\norm{T}. \]
\end{lemma}
\begin{proof}
$\norm{ST} = \sup\setbuilder{\frac{\norm{STx}}{\norm{x}}}{\norm{x}=1} \leq \sup\setbuilder{\frac{\norm{S}\norm{Tx}}{\norm{x}}}{\norm{x}=1}\leq \norm{S}\;\norm{T}$.
\end{proof}

\subsubsection{Operators bounded below}
\begin{definition}
Let $T$ be a bounded linear operator. We say $T$ is \udef{bounded below} if
\[ \exists b>0:\forall v\in \dom(T): \quad \norm{Tv}\geq b\norm{v} \]
\end{definition}


\begin{proposition} \label{prop:boundedBelow}
Let $T:V\to W$ be a bounded operator that is bounded below. Then
\begin{enumerate}
\item $T$ is injective;
\item if $T$ is surjective, the inverse $T^{-1}: W\to V$ exists and is bounded.
\end{enumerate}
\end{proposition}
\begin{proof}
To show $T$ is injective, take $x_1,x_2\in \dom T$ such that $Tx_1 = Tx_2$. Then
\[ 0 = \norm{Tx_1 - Tx_2} = \norm{T(x_1 - x_2)} \geq b\norm{x_1 - x_2} \geq 0. \]
So $\norm{x_1 - x_2} = 0$ and thus $x_1=x_2$.

The existence of $T^{-1}$ is then clear. For boundedness notice that $T^{-1}y \in \dom(T)$, so because $T$ is bounded below,
\[ \norm{T^{-1}y} \leq \frac{1}{b}\norm{TT^{-1}y} = \frac{1}{b}\norm{y}. \]
\end{proof}

\begin{lemma} \label{lemma:boundedBelowBounded}
Let $T:\dom(V)\to W$ be an injective operator. Then $T$ is bounded \textup{if and only if} $T^{-1}:\im(T)\to \dom(T)$ is bounded below.
\end{lemma}
\begin{proof}
Assume $T$ bounded. Then for all $x\in\im T$: $\norm{x} = \norm{TT^{-1}x} \leq \norm{T}\norm{T^{-1}x}$, so $T^{-1}$ is bounded below by $1/\norm{T}$.

Assume $T^{-1}$ bounded below. Then for all $x\in\dom(T)$: $\norm{x} = \norm{T^{-1}Tx} \geq b\norm{Tx}$, so $T$ is bounded by $1/b$.
\end{proof}

\subsection{Closed operators}
\begin{definition}
Let $T:\dom(T)\subseteq X\to Y$ be an operator. Then $T$ is a \udef{closed operator} if $\graph(T)$ is closed in $X\oplus Y$.
\end{definition}
This is not the same as a closed map in the topological sense!

The most important property of closed operators is given by the following proposition. It is sometimes taken as the definition.
\begin{proposition} \label{prop:closedGraphEquivalence}
Let $X,Y$ be normed spaces and $T: \dom(T)\subset X \to Y$ be a linear operator. Then
the following are equivalent:
\begin{enumerate}
\item $T$ is a closed operator;
\item if $(x_n)_{n\in\N}\subset \dom(T)$ converges to $x\in X$ and $(Tx_n)_{n\in\N}$ converges to $y$, then $x\in\dom(T)$ and $Tx = y$.
\end{enumerate}
\end{proposition}
TODO: remove domain from proposition? (? If domain is closed?)
\begin{corollary}
All bounded operators have closed graph. (? If domain is closed?)
\end{corollary}
The converse is not true in general.

\url{https://en.wikipedia.org/wiki/Unbounded_operator#Closed_linear_operators}
\url{https://en.wikipedia.org/wiki/Closed_graph_theorem_(functional_analysis)}

\begin{proposition} \label{prop:algebraClosedOperators}
Let $T$ be a closed and $S$ a bounded operator, then
\begin{enumerate}
\item $S+T$ is closed;
\item $TS$ is closed;
\item if $T$ is injective, then $T^{-1}: \im(T) \to \dom(T)$ is closed.
\end{enumerate}
\end{proposition}
\begin{proof}
(1) TODO

(2) TODO

(3) We use \ref{prop:closedGraphEquivalence}. Take $\seq{y_n}\subset \dom(T^{-1})$ such that $y_n\to y$ and $T^{-1}y_n\to x$. Set $x_n = T^{-1}y_n$, so then $Tx_n = y_n\to y$. Because $T$ is closed it follows that $Tx = y$, so $T^{-1}y = x$, meaning $T^{-1}$ is closed.
\end{proof}
TODO example $ST$ need not be closed.

\begin{lemma}
Let $T$ be a closed operator, then $\ker(T)$ is closed.
\end{lemma}
\begin{proof}
Let $\seq{x_n}\subset\ker(T)$ be a convergent sequence. Then $\seq{Tx_n}$ is identically zero and thus converges to $0$. By closedness of $T$, $Tx = 0$ and thus $x\in\ker(T)$. 
\end{proof}
We have already proven this for bounded operators, see \ref{lemma:kerClosed}.

\begin{lemma}
Let $T:X\not\to Y$ be an injective closed operator. Then $T^{-1}:\im(T)\to X$ is also closed.
\end{lemma}
\begin{proof}
Take $\seq{y_n}\subset \im(T)$ such that $y_n\to y$ and $T^{-1}y_n \to x$. Then $T(T^{-1}y_n) = y_n \to y$, so by closedness of $T$, we have $Tx = y$, and thus $T^{-1}y = x$.
\end{proof}

\subsubsection{Closable operators}
\begin{definition}
A linear operator is called \udef{closable} if it has closed extension.
\end{definition}

\begin{proposition}
A linear operator $T$ is closable \textup{if and only if} for all sequences $\seq{x_n}\subset\dom(T)$
\[ \left(x_n\to 0 \land T(x_n)\to v\right) \quad\implies\quad v = 0. \]
\end{proposition}
\begin{proof}
TODO
\end{proof}

\begin{lemma}
A closable operator $T$ has a minimal closed extension $\overline{T}$, which is given by the closure of the graph of $T$.
\end{lemma}
\begin{proof}
TODO
\end{proof}

\subsection{Compact operators}
\begin{definition}
A linear map $L:V\to W$ between normed spaces is called \udef{compact} if $L[\overline{B}(\vec{0}, 1)]$ is relatively compact.

I.e. the image of the closed unit ball has compact closure.

The space of compact maps from $V$ to $W$ is denoted $\mathcal{K}(V,W)$.
\end{definition}

These operators were introduced to study equations of the form
\[ (T-\lambda I)x(t) = p(t). \]

\begin{proposition}
Let $L\in\Lin(V,W)$. The following are equivalent:
\begin{enumerate}
\item $L$ is compact;
\item the image of any bounded subset of $V$ is relatively compact in $W$;
\item there exists a neighbourhood $U$ of $0$ in $V$ such that the image of $U$ is a subset of a compact set in $W$;
\item for any bounded sequence $(x_n)_{n\in\N} \subseteq V$, then sequence $(Lx_n)_{n\in\N}$ contains a converging subsequence.
\end{enumerate}
\end{proposition}
\begin{proof}
TODO
\end{proof}
\begin{corollary}
All maps of finite rank are compact.
\end{corollary}
\begin{proof}
Closed balls in $\C^n$ are compact.
\end{proof}

\begin{proposition}
Let $V$ be a normed space. Then $\mathcal{K}(V)$ is a closed two-sided ideal in $\Bounded(V)$.
\end{proposition}

\begin{lemma}
The identity map on $X$ is compact \textup{if and only if} $X$ is finite-dimensional.
\end{lemma}
\begin{proof}
The unit ball is compact iff $X$ is finite-dimensional, by \ref{prop:compactnessUnitBall}.
\end{proof}
\begin{corollary}
Let $T\in\Compact(X,Y)$. If $T$ is injective and $T^{-1}$ bounded, then $X$ is finite-dimensional.
\end{corollary}
\begin{proof}
In this case $\id_X = T^{-1}T$ is compact by TODO ref.
\end{proof}

\section{Inner product spaces}
\begin{definition}
An \udef{inner product} on a vector space $V$ is a function
\[ \inner{\cdot,\cdot}: V\times V \to \mathbb{F}  \]
that has the following properties:
\begin{itemize}[leftmargin=4.5cm]
\item[\textbf{Linearity}] in the \emph{second}\footnote{Some authors take linearity in the first component.} component
\[\inner{v,\lambda_1 w_1 + \lambda_2 w_2} = \lambda_1\inner{v,w_1} + \lambda_2\inner{v,w_2},\]
where $\lambda_1,\lambda_2 \in \mathbb{F}$ and $v,w_1,w_2\in V$.
\item[\textbf{Conjugate symmetry}\footnote{This is for $\mathbb{F} = \C$. For $\mathbb{F} = \R$ this reduces to normal symmetry $\inner{v,w} = \inner{w,v}$.}] $\inner{v,w} = \overline{\inner{w,v}}$ for all $v,w\in V$.
\item[\textbf{Positivity}\footnote{By conjugate symmetry we know that $\inner{v,v}$ is a real number, so this condition makes sense.}] $\inner{v,v} \geq 0$ for all $v\in V$.
\item[\textbf{Definiteness}]$\inner{v,v} = 0$ if and only if $v= 0$.
\end{itemize}
An \udef{inner product space} or \udef{pre-Hilbert space} $(\mathbb{F}, V,+,\inner{\cdot,\cdot})$ is a vector space $(\mathbb{F}, V,+)$ together with an inner product $\inner{\cdot,\cdot}$ on $V$.

A real finite-dimensional inner product space is called a \udef{Euclidean space}.
\end{definition}
\begin{lemma}
An inner product over a complex vector space $V$ is anti-linear in the first component.
\end{lemma}

\begin{lemma} \label{lemma:nonDegeneracyInnerProduct}
Definiteness implies the inner product on $V$ is non-degenerate:
\[ [\forall u\in V:\inner{u,v} = 0] \implies v = 0. \]
\end{lemma}
The converse is not true.

There are some generalised notions of inner product:
\begin{definition}
Let $V$ be a complex vector space.
\begin{enumerate}
\item A \udef{sesquilinear form} is a function $V\times V\to \C$ that is linear in the second component and anti-linear in the first.
\item A \udef{Hermitian form} is a conjugate symmetric sesquilinear form.
\item A \udef{pre-inner product} is a positive Hermitian form, i.e. an inner product without the requirement of definiteness.
\end{enumerate}
\end{definition}

\begin{example}
\begin{enumerate}
\item The \udef{standard inner product} on $\R^n$ is given by
\[ \inner{a,b} = \inner{\begin{bmatrix}
a_1 \\ \vdots \\ a_n
\end{bmatrix},\begin{bmatrix}
b_1 \\ \vdots \\ b_n
\end{bmatrix}} = \begin{bmatrix}
a_1 & \hdots & a_n
\end{bmatrix}\begin{bmatrix}
b_1 \\ \vdots \\ b_n
\end{bmatrix} = a^\transp b \]
This is also known as the \udef{dot product} $a\cdot b$.
\item The \udef{standard inner product} on $\C^n$ is given by
\[ \inner{a,b} = \inner{\begin{bmatrix}
a_1 \\ \vdots \\ a_n
\end{bmatrix},\begin{bmatrix}
b_1 \\ \vdots \\ b_n
\end{bmatrix}} = \begin{bmatrix}
\bar{a}_1 & \hdots & \bar{a}_n
\end{bmatrix}\begin{bmatrix}
b_1 \\ \vdots \\ b_n
\end{bmatrix} = \bar{a}^\transp b \]
\item The \udef{Frobenius inner product} on $\C^{m\times n}$ is given by
\[ \inner{A,B}_F =  \Tr(\overline{A}^\transp B) = \overline{\vectorisation_C(A)}^\transp \vectorisation_C(B)\]
\item On the vector space $\mathcal{C}[a,b]$ of continuous real functions on $[a,b]$, we can take the inner product
\[ \inner{f,g} = \int_a^b f(x)\cdot g(x) \diff{x}. \]
\end{enumerate}
\end{example}

\begin{definition}
Two vectors $u,v \in V$ are \udef{orthogonal} if $\inner{u,v} =0$. This is denoted $u\perp v$.
\end{definition}
\begin{lemma} \label{lemma:elementaryOrthogonality}
Let $V$ be an inner product space.
\begin{enumerate}
\item $0$ is the only vector orthogonal to itself.
\item $0$ is orthogonal to all $v\in V$;
\item Let $x,y\in V$. If, for all $v\in V$, $\inner{v,x} = \inner{v,y}$, then $x=y$.
\end{enumerate}\end{lemma}
\begin{proof}
The first is a consequence of definiteness, the second a consequence of linearity: $\inner{v,0} = \inner{v,0\cdot0} = 0\inner{v,0} = 0$.

The third is also a consequence of linearity: assume $\forall v\in V: \inner{v,x} = \inner{v,y}$, then $\inner{v,x-y}=0$ and $x-y$ is orthogonal to all $v\in V$ and in particular to $0$. Thus $x-y$ must be zero.
\end{proof}

\begin{proposition}
Every inner product gives rise to a norm, defined by
\[ \norm{\cdot} = \sqrt{\inner{\cdot,\cdot}}. \]
\end{proposition}
\begin{proof}
The only non-trivial part is the triangle inequality. This will be proved later using the Cauchy-Schwarz inequality.
\end{proof}


\begin{lemma}
Let $V$ be an inner product space. Then
\[ \norm{v+w}^2 = \norm{v}^2+\norm{w}^2+2\Re\inner{v,w} \]
\end{lemma}
\begin{lemma} \label{lemma:orthogonalDecomposition}
Let $v,w\in V$, with $w\neq 0$. We can decompose $v$ as a multiple of $w$ and a vector $u$ orthogonal to $w$:
\[ v = cw+u = \left(\frac{\inner{v,w}}{\norm{w}^2}\right)w + \left( v- \frac{\inner{v,w}w}{\norm{w}^2} \right). \]
\end{lemma}
\begin{proof}
The only thing to check is $\inner{w, v- \frac{\inner{v,w}w}{\norm{w}^2}} = 0$, which is a simple calculation.
\end{proof}

\subsection{Pythagoras and Cauchy-Schwarz}
\begin{theorem}[Pythagorean theorem]
Suppose $u\perp v$. Then $\norm{u+v}^2 = \norm{u}^2 + \norm{v}^2$.
\end{theorem}
\begin{proof}
\[ \norm{u+v}^2 = \inner{u+v,u+v} = \inner{u,u}+ \inner{u,v} + \inner{v,u} + \inner{v,v} = \norm{u}^2 + \norm{v}^2. \]
\end{proof}

\begin{theorem}[Cauchy-Schwarz-Bunyakovsky inequality.] \label{theorem:CauchySchwarz}
Let $V$ be a vector space with a pre-inner product $\inner{\cdot,\cdot}$. Let $v,w\in V$. Then
\[ |\inner{v,w}|^2\leq \inner{v,v}\cdot\inner{w,w}. \]
Suppose $\inner{\cdot,\cdot}$ is definite (i.e. an inner product), then
this is an equality \textup{if and only if} $v$ and $w$ are scalar multiples.
\end{theorem}
This result is also known as the Cauchy-Schwarz inequality, or the CSB inequality.
\begin{proof}
Consider 
\[ \inner{v-\lambda w, v-\lambda w} = \inner{v,v}-\lambda\inner{v,w}-\overline{\lambda}\inner{w,v} + |\lambda|^2 \inner{w,w} \geq 0. \]
Suppose $\inner{v,w}=re^{i\theta}$ (if $\mathbb{F} = \R$, then $\theta=0$ or $\theta = \pi$). The inequality must still hold for all $\lambda$ of the form $te^{-i\theta}$ for some $t\in \R$. The inequality thus becomes
\[ 0\leq \inner{v,v}-te^{-i\theta}re^{i\theta}-te^{i\theta}re^{-i\theta} + t^2 \inner{w,w} = \inner{v,v}-2rt + t^2 \inner{w,w}. \]
On the right we have a quadratic formula in $t$. This may never be negative and the discriminant may therefore not be positive. Calculating the discriminant gives $(2r)^2 - 4\inner{v,v}\inner{w,w}$. Thus
\[ 0\geq r^2 - \inner{v,v}\inner{w,w} = |\inner{v,w}|^2 - \inner{v,v}\inner{w,w}. \]
\end{proof}
In the case of an inner product, there is a simpler proof:
\begin{proof}
Take the decomposition from lemma \ref{lemma:orthogonalDecomposition} and apply the Pythagorean theorem to obtain
\[ \norm{v}^2 = \frac{|\inner{v,w}|^2}{\norm{w}^2} + \norm{u}^2 \geq \frac{|\inner{v,w}|^2}{\norm{w}^2}. \]
This also shows the claim about scalar multiples.
\end{proof}
\begin{corollary} \label{lemma:innerBoundedFunctionals}
Let $V$ be an inner product space. The functions
\[\inner{v,\cdot}: V\to \mathbb{F}: x\mapsto \inner{v,x} \]
are bounded linear functionals for all $v\in V$.
\end{corollary}
\begin{corollary} \label{corollary:preInnerProductCSBZero}
Let $V$ be a vector space with a pre-inner product $\inner{\cdot,\cdot}$. Then
\[ \inner{x,x}=0\lor\inner{y,y}=0 \quad\implies\quad \inner{x,y} = 0. \]
\end{corollary}
\begin{definition}
The Cauchy-Schwarz inequality allows us to define the \udef{angle} $\theta$ between two vectors $v,w$ by
\[ \cos\theta = \frac{\inner{v,w}}{\norm{v}\norm{w}}.\]
\end{definition}
\begin{lemma}
If $v\perp w$, then the angle between them is $\pi/2 + k\pi$.
\end{lemma}

TODO CS special case of Hölder inequality.

\begin{theorem}[Triangle inequality]
Let $v,w\in V$. Then
\[ \norm{v+w} \leq \norm{v}+\norm{w} \qquad\text{or, equivalently}\qquad \norm{v}-\norm{w}\leq \norm{v-w}. \]
This inequality is an equality if and only if one of $u,v$ is a nonnegative multiple of the other.
\end{theorem}
\begin{proof}
We calculate
\begin{align*}
\norm{v+w}^2 &= \norm{v}^2+\norm{w}^2+2\Re\inner{v,w} \\
&\leq \norm{v}^2+\norm{w}^2+2|\inner{v,w}| \\
&\leq \norm{v}^2+\norm{w}^2+2\norm{v}\norm{w} \\
&= (\norm{v}+\norm{w})^2.
\end{align*}
The substitution $v\to v-w$ gives the second form.
\end{proof}

\subsection{Parallelogram law and polarisation}
\begin{theorem}[Parallelogram law] \label{theorem:parallelogramLaw}
Let $V$ be an inner product space and $v,w\in V$. Then
\[ \norm{v+w}^2 + \norm{v-w}^2 = 2 (\norm{v}^2+\norm{w}^2). \]
\end{theorem}
\begin{proof}
We calculate
\begin{align*}
\norm{v+w}^2 + \norm{v-w}^2 = \inner{v+w, v+w}+\inner{v-w,v-w} = 2(\norm{v}^2 + \norm{w}^2).
\end{align*}
\end{proof}
\begin{corollary}[Appolonius' identity]
Let $V$ be an inner product space and $x,y,z\in V$. Then
\[ \norm{z-x}^2 + \norm{z-y}^2 = \frac{1}{2}\norm{x-y}^2 + 2\norm*{z-\frac{1}{2}(x+y)}^2. \]
\end{corollary}
\begin{proof}
Apply the parallelogram law to $u = \frac{1}{2}(z-x)$ and $v = \frac{1}{2}(z-y)$.
\end{proof}

\begin{theorem}[Polarisation identities] \label{theorem:polarisationIdentities}
Polarisation identities allow us to recover the inner product from the norm.
\begin{enumerate}
\item For real inner product spaces, $\mathbb{F} = \R$:
\begin{align*}
\inner{v,w} &= \frac{1}{2}(\norm{v+w}^2 - \norm{v}^2-\norm{w}^2) \\
&= \frac{1}{2}(\norm{v}^2 + \norm{w}^2-\norm{v-w}^2) \\
&= \frac{1}{4}(\norm{v+w}^2 - \norm{v-w}^2) = \frac{1}{4}\sum_{k=0}^1 (-1)^k\norm{v+(-1)^k w}^2.
\end{align*}
\item For complex inner product spaces, $\mathbb{F} = \C$:
\[ \inner{x,y} = \frac{1}{4}\sum_{k=0}^3 i^k\norm{i^k x+y}^2. \]
\item For general sesquilinear forms:
\[ S(x,y) = \frac{1}{4}\sum_{k=0}^3 i^k S(i^k x+y, i^k x+y). \]
\end{enumerate}
\end{theorem}
\begin{corollary}
A sesquilinear form is Hermitian \textup{if and only if} $\inner{v,v}$ is real for all $v\in V$.
\end{corollary}
\begin{proof}
The direction $\Rightarrow$ is obvious. For the other direction, assume $\inner{v,v}$ is real for all $v\in V$ and in particular $\inner{u+i^kv,u+i^kv}$ is real. We calculate
\begin{align*}
\overline{\inner{u,v}} &= \frac{1}{4}\sum^3_{k=0}(-i)^k\inner{u+i^kv,u+i^kv} \\
&= \frac{1}{4}\sum^3_{k=0}(-i)^k\inner{v+(-i)^ku,v+(-i)^ku} & &\text{Using (conjugate) linearity and $i^k(-i)^k=1$}\\
&= \frac{1}{4}\sum^3_{k=0}i^k\inner{v+i^ku,v+i^ku} & &\text{Substituting $k\to k+2$}\\
&= \inner{v,u}.
\end{align*}
\end{proof}
Not all norms on vector spaces can be obtained from an inner product. If a norm can be obtained from an inner product, we can use polarisation to recover the inner product. If a norm cannot be obtained from an inner product, the putative inner product suggested by polarisation will turn out not to be an inner product.
\begin{proposition}
A norm can be obtained from an inner product \textup{if and only if} it satisfies the parallelogram law.
\end{proposition}
\begin{corollary}
The space $l^p$ is an inner product space \textup{if and only if} $p=2$.
\end{corollary}
\begin{proof}
The inner product on $l^2$ is defined by $\inner{x_n, y_n} = \sum_{n=1}^\infty \overline{x_n}y_n$.

If $p\neq 2$ we can find a counterexample to the parallelogram law: let $x=(1,1,0,0,\ldots)\in l^p$ and $y = (1,-1,0,0,\ldots)\in l^p$. Then
\[ \norm{x}_p = \norm{y}_p = 2^{1/p} \qquad \text{and} \qquad \norm{x+y} = \norm{x-y} = 2 \]
and the parallelogram law is then not valid if $p\neq 2$.
\end{proof}

\section{Orthogonal and orthonormal sets of vectors}
\begin{definition}
\begin{itemize}
\item A set of vectors $D$ is called \udef{orthogonal} if for any two vectors $v,w\in D$, $v\perp w$ \textup{if and only if} $v\neq w$.
\item A set of vectors $D$ is called \udef{orthonormal} if for any two vectors $v,w\in D$,
\[ \inner{v,w} = \begin{cases}
0 & (v\neq w) \\ 1 & (v=w)
\end{cases}. \]
\end{itemize}
In particular an orthonormal set is an orthogonal set of unit vectors.
\end{definition}

\subsection{Orthogonal sets and sequences}
\begin{lemma} \label{lemma:orthogonalLinearlyIndependent}
Every orthogonal set of vectors is linearly independent.
\end{lemma}
\begin{lemma}
Every subset of an orthogonal (resp. orthonormal) set is orthogonal (resp. orthonormal).
\end{lemma}

\begin{theorem}[Gram-Schmidt procedure]
Every finite set of linearly independent vectors $D = \{v_1,\ldots, v_n\}$ can be transformed into an orthonormal set $D' = \{e_1,\ldots,e_n\}$ with the same number of vectors such that the spans are the same: $\Span(D') = \Span(D)$.
\end{theorem}
\begin{proof}
The procedure goes as follows:
\begin{align*}
e_1 &= \frac{v_1}{\norm{v_1}} \\
e_2 &= \frac{v_2 - \inner{e_1,v_2}e_1}{\norm{v_2 - \inner{e_1,v_2}e_1}} \\
&\hdots \\
e_j &= \frac{v_j - \inner{e_1,v_j}e_1- \ldots - \inner{e_{j-1},v_j}e_{j-1}}{\norm{v_2 - \inner{e_1,v_2}e_1- \ldots - \inner{e_{j-1},v_j}e_{j-1}}} \\
&\hdots
\end{align*}
\end{proof}

If we only need an orthogonal set $\{y_1,\ldots,y_n\}$, not an orthonormal one, we can use the procedure
\[ y_{k+1} = v_{k+1} - \sum_{i=1}^k \frac{\inner{v_{k+1}, y_i}}{\inner{y_i,y_i}}y_i. \]

\begin{lemma} \label{lemma:orthogonality}
Let $(\mathbb{F}, V,+,\inner{\cdot,\cdot})$ be an inner product space. Then
\[ \inner{v,w}=0 \qquad \iff \qquad \forall a\in\mathbb{F}:\;\norm{v}\leq\norm{v+aw}.  \]
\end{lemma}
\begin{proof}
The implication $\Rightarrow$ is a consequence of the Pythagorean theorem. For the other implication, assume $\forall a\in\mathbb{F}:\;\norm{v}\leq\norm{v+aw}$. Then
\[ \norm{v}^2 \leq \norm{v-aw}^2 = \norm{v}^2 - 2\Re\inner{v,aw} + \norm{aw}^2 \]
which implies $2\Re\inner{v,aw} \leq a^2\norm{w}^2$. Let $\inner{v,w} = re^{i\theta}$. (If $\mathbb{F} = \R$, then $\theta=0$.) Then in particular the inequality holds for all $a=te^{i\theta}$ with $t\in\R$. This yields
\[ 2\Re(te^{-i\theta}re^{i\theta}) \leq t^2\norm{w}^2 \qquad \text{or}\qquad 2rt\leq t^2\norm{w}^2. \]
Letting $t\geq 0$, we can divide out a $t$: $2r\leq t\norm{w}^2$. Then letting $t\to 0$ gives $r=0$ and thus $\inner{v,w}=0$.
\end{proof}

\begin{proposition}
Let $V$ be an inner product space and $D = \{e_1,\ldots, e_n\}$ a finite orthonormal set of vectors. Then $\forall v\in V$
\[ \inf_{c_i\in\mathbb{F}}\norm{v-\sum_{i=1}^nc_ie_i} = \norm{v-\sum_{i=1}^n\inner{e_i,v}e_i} \]
\end{proposition}
\begin{proof}
We calculate
\begin{align*}
\norm{v-\sum_{i=1}^nc_ie_i}^2 &= \inner{v-\sum_{i=1}^nc_ie_i,v-\sum_{j=1}^nc_je_j} \\
&= \norm{v} - \sum_{j=1}^n c_j\inner{v,e_j} - \sum_{i=1}^n\bar{c}_i\inner{e_i,v} + \sum_{i,j=1}^n\bar{c}_ic_j\inner{e_i,e_j} \\
&= \norm{v} - 2\Re\left(\sum_{i=1}^nc_i\overline{\inner{e_i,v}}\right) + \sum_{i=1}^n|c_i|^2 \\
&= \sum_{i=1}^n\left(|c_i|^2 - 2\Re\left(\sum_{i=1}^nc_i\overline{\inner{e_i,v}}\right) + |\inner{e_i,v}|^2\right) +\norm{v} - \sum_{i=1}^n|\inner{e_i;v}|^2 \\
&= \sum_{i=1}^n|c_i - \inner{e_i,v}|^2 +\norm{v} - \sum_{i=1}^n|\inner{e_i,v}|^2.
\end{align*}
This is clearly minimised when $c_i = \inner{e_i,v}$.
\end{proof}
\begin{corollary}
Let $v\in\Span(D)$, then $v = \sum_{i=1}^n \inner{e_i,v}e_i$.
\end{corollary}
We call the numbers $\inner{e_i,v}$ the \udef{Fourier coefficients} of $v$ w.r.t. $D$.
\begin{proof}
In this case $\inf_{c_i\in\mathbb{F}}\norm{v-\sum_{i=1}^nc_ie_i} = 0$.
\end{proof}
\begin{corollary}[Bessel inequality]
Let $\{e_i\}_{i\in I}$ be an orthonormal family and $v\in V$, then
\[ \sum_{i\in I}|\inner{e_i,v}|^2 = \sup \left\{\sum_{\substack{i\in I' \subset I\\ I' \;\text{finite}}} |\inner{e_i,v}|^2 \right\} \leq \norm{v}^2. \]
\end{corollary}
\begin{proof}
In the previous proof,
\[ 0 \leq \norm{v-\sum_{i=1}^nc_ie_i}^2 = \sum_{i=1}^n|c_i - \inner{e_i,v}|^2 +\norm{v} - \sum_{i=1}^n|\inner{e_i,v}|^2 = \norm{v} - \sum_{i=1}^n|\inner{e_i,v}|^2. \]
Where we have set $c_i = \inner{e_i,v}$. Thus the supremum must also be $\leq \norm{v}$.
\end{proof}
\begin{corollary}
For any $v\in V$, $\inner{e_i,v} = 0$ except for countably many $i\in I$. \label{corollary:countableComponents}
\end{corollary}
\begin{proof}
Ref TODO. \url{https://proofwiki.org/wiki/Uncountable_Sum_as_Series}.
\end{proof}
TODO: link with metric topology being sequential?

\begin{corollary}[Riemann-Lebesgue lemma]
For any sequence $\seq{e_i}_{i\in J \subset I}$, we have
\[ \lim_{i\in J} \inner{e_i,v} = 0. \]
\end{corollary}

\begin{corollary}
We can also obtain the Cauchy-Schwarz inequality from the Bessel inequality.
\end{corollary}
\begin{proof}
Let $x,y\in V$. Then $\{x/\norm{x}\}$ is an orthonormal set. Applying the Bessel inequality for $y$ gives $\norm{y}^2 \geq |\inner{x/\norm{x}, y}|^2 \implies |\inner{x,y}|^2\leq \norm{x}^2\norm{y}^2 \implies |\inner{x,y}| \leq \norm{x}\;\norm{y}$.
\end{proof}

\subsection{Orthonormal bases}
\begin{definition}
Let $D$ be an orthonormal set of vectors in an inner product space $V$, then $D$ is said to be
\begin{enumerate}
\item \udef{maximal}, if it is a maximal element in the set of orthonormal sets ordered by inclusion;
\item \udef{total}, if the smallest closed subspace that includes $D$ is $V$ (i.e. $\Span(D)$ is dense in $V$);
\item an \udef{orthonormal basis} (o.n. basis) or a \udef{Hilbert basis} if any vector in $V$ can be written as a (possibly infinite) linear combination of elements of $D$.
\end{enumerate}
\end{definition}
\begin{note}
Hilbert bases are in general not Hamel bases.  E.g., take $\R^\mathbb{N}$. Then 
\begin{align*}
(1,0,0,&\ldots), \\
(0,1,0,&\ldots), \\
(0,0,1,&\ldots), \\
&\ldots
\end{align*}
is an orthonormal basis, but not a Hamel basis (consider $(1,1,1,\ldots)$).
\end{note}

\begin{proposition}
If $V$ is finite-dimensional, then the notions of maximal orthonormal set, total orthonormal set and orthonormal set coincide. Such an orthonormal set is also a (Hamel) basis of $V$.
\end{proposition}
\begin{proof}
Corollaries of Gram-Schmidt.
\end{proof}

\begin{proposition}
Let $V$ be an inner product space and $D = \{e_i\}_{i\in I}$ an orthonormal set. The $D$ is an o.n. basis \textup{if and only if} $D$ is total.
\end{proposition}
\begin{proof}
$\boxed{\Rightarrow}$ Assume $D$ an o.n. basis. Then there exists a sequence of partial sums converging to any element $v\in V$. Each of these partial sums is a finite linear combination of elements in $D$ and thus this sequence is a sequence in $\Span(D)$. This means $v\in\overline{\Span(D)}$.

$\boxed{\Leftarrow}$ Assume $\overline{\Span(D)} = V$. Because the topology on $V$ is a metric topology, we can find a sequence $(v_n)$ in $\Span(D)$ that converges to any $v\in V$.
\end{proof}


\begin{proposition} \label{prop:totalONBParsevalEquivalence} \label{prop:plancherel}
Let $V$ be an inner product space and $D = \{e_i\}_{i\in I}$ an orthonormal set. The following are equivalent:
\begin{enumerate}
\item $D$ is an orthonormal basis of $V$;
\item $D$ is total in $V$;
\item for all $v,w\in V$,
\[ \inner{v,w} = \sum_{i\in I}\inner{v,e_i}\inner{e_i,w}; \]
\item \textup{(Parseval's identity)} for all $v\in V$,
\[ \norm{v}^2 = \sum_{i\in I}|\inner{e_i,v}|^2; \]
\item for all $v\in V$: if $v\perp D$, then $v=0$;
\item \textup{(Plancherel formula)} for all $v\in V$,
\[ v = \sum_{i\in I}\inner{e_{i},v}e_{i}. \]
\end{enumerate}
\end{proposition}
\begin{proof}
We proceed round-robin-style.
\begin{itemize}[leftmargin=2cm]
\item[$\boxed{(1) \Rightarrow (2)}$] Assume $D$ an o.n. basis. Then there exists a net of partial sums converging to any element $v\in V$. Each of these partial sums is a finite linear combination of elements in $D$ and thus this net is a net in $\Span(D)$. This means $v\in\overline{\Span(D)}$.
\item[$\boxed{(2) \Rightarrow (3)}$] Fix $v,w\in V$. Because $V$ is a metric spaces and thus sequential, we can find sequences $(v_j)_{j\in J}$ and $(w_k)_{k\in K}$ in $\Span(D)$ converging to $v$ and $w$. Now the linear maps $u\mapsto \overline{\inner{u, e_i}}$ and $u\mapsto \inner{e_i, u}$ are bounded by Cauchy-Schwarz and thus continuous by theorem \ref{theorem:boundedLinearMaps} (TODO corollary CSB). Then we can calculate, using the fact that each $v_j$ and $w_k$ is a finite linear combination of $e_i$,
\begin{align*}
\inner{v,w} &= \inner{\lim_{j}v_j, \lim_k w_k} = \lim_{j}\lim_{k}\inner{v_j,w_k} \\
&= \lim_{j}\lim_{k}\inner{\sum_{i=1}^{N_{j}}\inner{e_i,v_j}e_i,\sum_{i'=1}^{N_k}\inner{e_{i'},w_k}e_{i'}} \\
&= \lim_{j}\lim_{k}\sum_{i=1}^{N_{j}}\sum_{i'=1}^{N_k}\inner{v_j,e_i}\inner{e_{i'},w_k}\inner{e_i,e_{i'}} = \lim_{j}\lim_{k}\sum_{i=1}^{N_{j}}\sum_{i'=1}^{N_k}\inner{v_j,e_i}\inner{e_{i'},w_k}\delta_{i,i'} \\
&= \lim_{j}\lim_{k}\sum_{i=1}^{\min\{N_{j},N_{k}\}}\inner{v_j,e_i}\inner{e_i,w_k} \\
&= \lim_{j}\lim_{k}\sum_{i\in I}\inner{v_j,e_i}\inner{e_i,w_k} \\
&= \sum_{i\in I}\lim_{j}\lim_{k}\inner{v_j,e_i}\inner{e_i,w_k} \\
&= \sum_{i\in I}\inner{v,e_i}\inner{e_i,w}.
\end{align*}
For the interchange of the limits and the summation in the penultimate equality we can use Tannery's theorem, \ref{theorem:tannery}. Indeed $|\inner{e_i,w_k}|$ is bounded by $\norm{w_k}$ by the Bessel inequality. By the continuity of the norm we have $\lim_k \norm{w_k} = \norm{w}$, so the sequence $\norm{w_k}$ is bounded.
\item[$\boxed{(3) \Rightarrow (4)}$] Set $v=w$.
\item[$\boxed{(4) \Rightarrow (5)}$] If $v\perp D$, then
\[ \norm{v}^2 = \sum_{i\in I}|\inner{e_i,v}|^2 = 0 \qquad\text{which implies $v=0$.} \]
\item[$\boxed{(5) \Rightarrow (6)}$] The vector $v-\sum_{i\in I}\inner{e_i,v}e_i$ is perpendicular to $D$:
\[ \forall e_j\in D: \quad \inner{e_j, v-\sum_{i\in I}\inner{e_i,v}e_i} = \inner{e_j, v}-\sum_{i\in I}\inner{e_i,v}\inner{e_j,e_i} = \inner{e_j, v} - \inner{e_j, v} = 0. \]
So $v-\sum_{i\in I}\inner{e_i,v}e_i = 0$ and the Plancherel formula holds.
\item[$\boxed{(6) \Rightarrow (1)}$] By definition of o.n. basis.
\end{itemize}
\end{proof}

\begin{lemma}
Every orthonormal basis is a maximal orthonormal family.
\end{lemma}
\begin{proof}
Let $D = \{e_i\}_{i\in I}$ be an o.n. basis. Assume, towards a contradiction, that there exists an o.n. family $D' \supsetneq D$. Let $x\in D'\setminus D$. Then, using the Plancherel formula and the fact that $D'$ is orthogonal,
\[ x = \sum_{i\in I}\inner{e_{i},x}e_{i} = \sum_{i\in I} 0 = 0. \]
As $0$ can never be an element of an o.n. family, this is a contradiction.  
\end{proof}
There are maximal orthonormal families that are not bases.
\begin{example}
Consider the space $l^2(\N)$ and take the subspace $X$ generated by the family of elements
\[ \left( \sum_{n=1}^\infty n^{-1}e_n, e_2,e_3,e_4,\ldots \right) \]
with the inner product induced by the inner product of $l^2$. In this space $F=\{e_2,e_3,\ldots\}$ is orthonormal and maximal, but not a basis.
\end{example}

Maximal orthonormal families feel like kinds bases, especially given the next couple of results. We would really like the concepts of orthonormal basis and maximal orthonormal family to coincide. Spaces in which they do not are missing something; they are not complete. This is one good reason we are most often interested in Hilbert spaces.

\begin{theorem}
\begin{itemize}
\item Every vector space has a maximal orthonormal set.
\item Every orthonormal set can be extended to a maximal orthonormal set.
\end{itemize}
\end{theorem}
\begin{proof}
The first statement follows easily from the second. The second statement is proved using Zorn's lemma. Let $S$ be an orthonormal set. Define
\[ \mathcal{A} = \{ D\subset V \;|\; S\subset D \; \text{and $D$ is orthonormal} \} \]
ordered by inclusion. It is easy to see that any chain on $\mathcal{A}$ has an upper bound on $\mathcal{A}$, by just taking the union which is still orthonormal. It follows from Zorn's lemma that $\mathcal{A}$ has a maximal element $R$. This is by definition an orthonormal basis.

In the finite-dimensional case this can also be proved using Gram-Schmidt.
\end{proof}

\begin{proposition}
Given a vector space $V$, any two maximal orthonormal sets have the same cardinality.
\end{proposition}
\begin{proof}
Take $D = \{e_i\}_{i\in I}$ and $D' = \{f_j\}_{j\in J}$ maximal orthonormal sets.
\end{proof}
\begin{definition}
An inner product space is \udef{separable} if it is separable as a metric space, i.e. it admits a countable dense subset.
\end{definition}
\begin{proposition}
An inner product space is separable \textup{if and only if} it admits an orthonormal basis with at most countably many vectors.
\end{proposition}
\begin{proof}
TODO
\end{proof}

\subsection{Orthogonal complements}
\begin{definition}
Let $U$ be a subset of an inner product space $V$. The \udef{orthogonal complement} $U^\perp$ of $U$ is the set of vectors in $V$ that are orthogonal to every vector in $U$:
\[ U^\perp = \{ v\in V\;|\; \inner{v,u}=0\; \forall u\in U \}. \]
\end{definition}
We can also consider the orthogonal complement of a subspace with respect to another subspace, not the full space.
\begin{definition}
Let $U\subseteq W$ be subsets of an inner product space $V$. The \udef{orthogonal complement} of $U$ with respect to $W$ is the set of vectors in $W$ that are orthogonal to every vector in $U$:
\[ W\ominus U = \{ w\in W\;|\; \inner{w,u}=0\; \forall u\in U \}. \]
\end{definition}

\begin{proposition} \label{prop:OrthogonalComplementProperties}
Let $U,W$ be \emph{subsets} of an inner product space $V$.
\begin{enumerate}
\item $U^\perp$ is a subspace of $V$;
\item $U^\perp = \Span(U)^\perp$;
\item $\{0\}^\perp = V$;
\item $V^\perp = \{0\}$;
\item $U\cap U^\perp \subset \{0\}$;
\item If $U\subset W$, then $W^\perp \subset U^\perp$.
\end{enumerate}
\end{proposition}

\begin{proposition} \label{prop:ominusUnderIsometry}
Let $V$ be an inner product space and $T:V\to V$ an isometry. Let $A\supseteq B$ be subspaces. Then
\[ T[A\ominus B] = T[A]\ominus T[B]. \]
\end{proposition}
\begin{proof}
Take $v\in T[A\ominus B]$. Then there exists an $x\in A$ such that $T(x) = v$ and $\inner{x,b}=0$ for all $b\in B$. Then by isometry $\inner{T(x), T(b)}=0$ for all $b\in B$. So $v\in T[A]\ominus T[B]$. This reasoning can be inverted to give the other inclusion. 
\end{proof}

\begin{proposition} \label{prop:linearDeMorgan}
Let $W_1,W_2$ be subspaces of an inner product space $V$. Then
\[ (W_1+W_2)^\perp = W_1^\perp \cap W_2^\perp. \]
\end{proposition}
\begin{proposition} \label{prop:orthogonalComplementClosed}
Let $U$ be a \emph{subset} of an inner product space $V$. Then $U^\perp$ is closed and $\overline{U}^\perp = U^\perp$. This can be rephrased as
\[ \overline{U}^\perp = \overline{U^\perp} = U^\perp. \]
Also
\[\overline{U} \subset (U^\perp)^\perp. \]
\end{proposition}
\begin{proof}
Let $x\in \overline{U^\perp}$. Then there exists a sequence $(x_i)$ in $U^\perp$ that converges to $x$. For all $u\in U$, the functional $\inner{u,\cdot}:y\mapsto \inner{u,y}$ is bounded (by Cauchy-Schwarz). Thus all these functionals are continuous. Applying any one to the sequence $x_i$ gives a sequence of zeros. Thus $\inner{u,x} = 0$ for all $u\in U$. Thus $x\in U^\perp$ and hence $U^\perp \supset \overline{U^\perp}$ meaning $U^\perp$ is closed.

Now $\overline{U}\supset U$, so $\overline{U}^\perp \subset U^\perp$. For the other inclusion, take an $x\in U^\perp$. Take an arbitrary $y\in \overline{U}$. Then there exists a sequence $(y_i)$ in $U$ that converges to $y$. Apply the bounded functional $\inner{x,\cdot}$ to the sequence $(y_i)$, yielding a sequence of zeros. Thus $\inner{x,y}=0$. Thus $x\in \overline{U}^\perp$.

Finally let $u\in \overline{U}$. Take a sequence $u_i\to u$. Take an arbitrary element $x\in U^\perp$. As before $\inner{x,u} = \lim_i\inner{x,u_i} = 0$. So $u\in (U^\perp)^\perp$.
\end{proof}
\begin{corollary} \label{corollary:orthogonalComplementClosed}
If the subset $U$ is dense in $V$, then $U^\perp = \{0\}$. 
\end{corollary}
\begin{proof}
\[ U^\perp = \overline{U}^\perp = V^\perp = \{0\}. \]
\end{proof}
\begin{proposition}
Let $U$ be a finite-dimensional subspace of an inner product space $V$.
\begin{enumerate}
\item $V=U\oplus U^\perp$;
\item $U = (U^\perp)^\perp$.
\end{enumerate}
\end{proposition}
Notice that $V$ may be infinite dimensional!
\begin{proof}
We start with the first point. The sum $U + U^\perp$ is definitely direct, $U\oplus U^\perp$, by proposition \ref{prop:OrthogonalComplementProperties} and the criterion for a direct sum, proposition \ref{prop:directSumCriterion}. Clearly $U\oplus U^\perp\subseteq V$, so we just need to show that $V \subseteq U\oplus U^\perp$.

To that end, take a vector $v\in V$. Let $\{e_i\}_{i=1}^n$ be an orthonormal basis of $U$. We can write
\[ v = \left(v - \sum_{i=1}^n\inner{v,e_i}e_i\right) + \left(\sum_{i=1}^n\inner{v,e_i}e_i\right). \]
The first part is an element of $U^\perp$, the second of $U$, so $v\in U\oplus U^\perp$.

For the second point: any finite-dimensional subspace $U$ is automatically closed, so $U = \overline{U} \subset (U^\perp)^\perp$, by proposition \ref{prop:orthogonalComplementClosed}. For the other inclusion, take $v\in (U^\perp)^\perp$. By the first point, we can write $v = v_1 + v_2$ where $v_1\in U$ and $v_2\in U^\perp$. Because $v\in (U^\perp)^\perp$ and $v_2\in U^\perp$, we must have
\[ 0 = \inner{v_2, v} = \inner{v_2, v_1+v_2} = \inner{v_2, v_1} + \inner{v_2,v_2} = \norm{v_2}. \]
So $v=v_1\in U$.
\end{proof}

TODO all projection results for projection onto finite dim? See proposition before Bessel inequality. In fact better: projection onto summand of direct sum! Put under decompositions.


A result dual to proposition \ref{prop:linearDeMorgan} also holds in finite-dimensional spaces:
\begin{proposition}
Let $W_1,W_2$ be subspaces a finite-dimensional space $V$. Then
\[ (W_1\cap W_2)^\perp = W_1^\perp + W_2^\perp. \]
\end{proposition}
\begin{proof}
We start by applying proposition \ref{prop:linearDeMorgan} to $W_1^\perp$ and $W_2^\perp$:
\[ (W_1^\perp+W_2^\perp)^\perp = (W_1^\perp)^\perp \cap (W_2^\perp)^\perp = W_1 \cap W_2. \]
Taking the orthogonal complement of both sides gives the result. In infinite dimensions $(W_1^\perp+W_2^\perp)$ is not necessarily closed. 
\end{proof}



\section{Maps on inner product spaces}

\begin{lemma}[Continuity of inner product]
Let $V$ be an inner product space. Then the inner product is a continuous function $V\times V \to \mathbb{F}$.
\end{lemma}
\begin{proof}
We show that if $x_n \to x$ and $y_n \to y$, then $\inner{x_n,y_n}\to \inner{x,y}$. By the triangle and Cauchy-Schwarz inequalities
\begin{align*}
|\inner{x_n,y_n}-\inner{x,y}| &= |\inner{x_n,y_n}-\inner{x_n,y}+\inner{x_n,y} - \inner{x,y}| \\
&\leq |\inner{x_n, y_n-y}| + |\inner{x_n-x, y}| \\
&\leq \norm{x_n}\norm{y_n-y} + \norm{x_n-x}\norm{y}.
\end{align*}
Because the right-hand side converges to $0$, the left-hand side must too.
\end{proof}

\subsection{Bounded operators}
\begin{lemma} \label{lemma:operatorNormInnerProduct}
Let $T\in\Bounded(V,W)$, then
\begin{align*}
\norm{T} &= \sup_{w\in \im(T),v \in \dom(T)} \frac{|\inner{w,Tv}|}{\norm{w}\,\norm{v}} \\
&= \sup\setbuilder{|\inner{w,Tv}|}{w\in \im(T)\;\land\; v\in\dom{T}\;\land\; \norm{w} = 1 = \norm{v}} \\
&= \sup_{w\in W,v \in \dom(T)} \frac{|\inner{w,Tv}|}{\norm{w}\,\norm{v}} \\
&= \sup\setbuilder{|\inner{w,Tv}|}{w\in W\;\land\; v\in\dom{T}\;\land\; \norm{w} = 1 = \norm{v}}.
\end{align*}
\end{lemma}
\begin{proof}
We prove
\[ \norm{T} \leq \sup_{w\in \im(T),v \in \dom(T)} \frac{|\inner{w,Tv}|}{\norm{w}\,\norm{v}} \leq \sup_{w\in W,v \in \dom(T)} \frac{|\inner{w,Tv}|}{\norm{w}\,\norm{v}} \leq \norm{T}. \]
The first two inequalities follow from the characterisation \ref{prop:operatorNorm}
\[ \norm{T} = \sup_{v \in \dom(T)} \frac{\norm{Tv}}{\norm{v}} = \sup_{v \in \dom(T)} \frac{\inner{Tv,Tv}}{\norm{Tv}\,\norm{v}} \]
and the inclusions
\begin{align*}
\setbuilder{\frac{|\inner{w,Tv}|}{\norm{w}\,\norm{v}}}{v\in\dom(T), w = Tv} &\subseteq \setbuilder{\frac{|\inner{w,Tv}|}{\norm{w}\,\norm{v}}}{v\in\dom(T), w\in\im(T)}\\
&\qquad\quad\subseteq \setbuilder{\frac{|\inner{w,Tv}|}{\norm{w}\,\norm{v}}}{v\in\dom(T), w\in V}.
\end{align*}
The last equality follows from the Cauchy-Schwarz inequality \ref{theorem:CauchySchwarz}:
\[ \frac{|\inner{w,Tv}|}{\norm{w}\,\norm{v}} \leq \frac{\norm{w}\,\norm{Tv}}{\norm{w}\,\norm{v}} = \frac{\norm{Tv}}{\norm{v}} \leq \frac{\norm{T}\,\norm{v}}{\norm{v}} = \norm{T} \]
for all $v\in\dom(T), w\in V$. 
\end{proof}

\subsection{Isometries}
\begin{lemma} \label{lemma:equalityOfMapsInnerProductSpaces}
Let $V$ be an inner product space and $S,T\in\Hom(V)$. Then $S=T$ \textup{if and only if}
\[ \forall v,w\in V: \inner{Tv,w} = \inner{Sv,w}. \]
\end{lemma}
\begin{proof}
The direction $\boxed{\Rightarrow}$ is obvious. For the other direction, use
\[ 0 = \inner{Tv,w} - \inner{Sv,w} = \inner{(T-S)v,w} \]
for all $v,w$. In particular $w=(T-S)v$. The result follows from definiteness of the inner product.
\end{proof}

\begin{lemma}
Let $V,W$ be inner product spaces. Let $f:V\to W$ be a function. Then $f$ preserves the metric (i.e. is an isometry) \textup{if and only if} $f$ also preserves the inner product:
\[ \forall x,y \in V: \quad \inner{f(x),f(y)}_W = \inner{x,y}_V. \]
\end{lemma}
The proof is a simple application of the polarisation identities.

\begin{definition}
Let $V,W$ be an inner product spaces. A linear map $U\in\Hom(V,W)$ is called \udef{unitary} if it is an isometry and invertible.

Unitary operators on real vector spaces are also called \udef{orthogonal operators}.
\end{definition}
Because every isometry is injective (see lemma \ref{lemma:isometryInjective}), it is enough for a linear map to be isometric and surjective to be unitary.

\begin{lemma}
Every unitary map is bounded and has norm $1$.
\end{lemma}
\begin{proof}
Let $U: V\to W$ be a unitary map between inner product spaces. Then $\forall v\in V: \norm{U(v)} = \norm{v}$.
\end{proof}

Unitary operators transform orthonormal bases to orthonormal bases:
\begin{proposition}
Let $T\in \Hom(V,W)$ with $V,W$ inner product spaces and let $V$ have an orthonormal basis $\{e_i\}_{i\in I}$. Then $T$ is unitary \textup{if and only if} $\{Te_i\}_{i\in I}$ is an orthonormal basis of $W$.
\end{proposition}
\begin{proof}
Assume $T$ unitary. The family $\{Te_i\}_{i\in I}$ is certainly orthonormal, by preservation of the inner product. Now let $w\in W$ and so $T^{-1}w\in V$. By the Plancherel formula, proposition \ref{prop:plancherel}, we can write
\[ T^{-1}w = \sum_{n=1}^\infty \inner{e_{i_n},T^{-1}w}e_{i_n} = \lim_{N\to\infty}\sum_{n=1}^N \inner{e_{i_n},T^{-1}w}e_{i_n} \]
and so
\[ w = TT^{-1}w = T\lim_{N\to\infty}\sum_{n=1}^N \inner{e_{i_n},T^{-1}w}e_{i_n} = \lim_{N\to\infty}\sum_{n=1}^N \inner{e_{i_n}T^{-1}w}Te_{i_n} \]
because $T$ is bounded and thus continuous, by theorem \ref{theorem:boundedLinearMaps}.
Thus $\{Te_i\}_{i\in I}$ is an orthonormal basis of $W$.

Conversely, assume $\{Te_i\}_{i\in I}$ is an orthonormal basis of $W$. We first prove $T$ is bounded, which is a simple application of Parseval's identity, proposition \ref{prop:totalONBParsevalEquivalence}:
\[ \norm{Tv}^2 = \sum_{i\in I}|\inner{Te_i,Tv}|^2 = \sum_{i\in I}|\inner{e_i,v}|^2 = \norm{v}^2. \]
The rest of the proof is again an application of the Plancherel formula.
\end{proof}

\begin{lemma}
Let $U$ be a unitary map. If $\lambda$ is an eigenvalue of $U$, then $|\lambda| = 1$.
\end{lemma}
\begin{proof}
Let $v$ be an eigenvector associated to the eigenvalue $\lambda$. Then
\[ \inner{v,v} = \inner{L(v),L(v)} = \inner{\lambda v, \lambda v} = \lambda^2\inner{v,v},  \]
so $\lambda^2 = 1$.
\end{proof}

\subsection{Symmetric operators}
\begin{definition}
Let $(\mathbb{F},V,+,\inner{\cdot,\cdot})$ be an inner product space. A linear operator $L$ is called \udef{symmetric} if, $\forall v,w\in \dom(L)$
\[ \inner{L(v),w} = \inner{v,L(w)}. \]
\end{definition}

\begin{proposition}
Let $V$ be an inner product space and $L$ a symmetric operator on $V$. Then eigenvectors of $L$ associated to different eigenvalues are orthogonal.
\end{proposition}
\begin{proof}
Let $v,w$ be eigenvectors of $L$ with eigenvalues $\lambda, \mu$ such that $\lambda \neq \mu$. Then
\[ \lambda\inner{v,w} = \inner{\lambda v,w}=\inner{L(v),w} = \inner{v,L(w)} = \inner{v,\mu w} = \mu \inner{v,w} \]
and consequently $\inner{v,w} =0$.
\end{proof}

\subsection{Impact on subspaces}
\subsubsection{Invariant and reducing subspaces}
\begin{definition}
Let $V$ be an inner product space and $T$ a linear operator on $V$.
\begin{itemize}
\item A subspace $U\subseteq V$ is said to be \udef{invariant} under $T$ if $T[U] \subset U$.
\item A subspace $U\subseteq V$ is said to be \udef{reducing} for $T$ if both $U$ and $U^\perp$ are invariant under $T$.
\end{itemize}
\end{definition}

\subsection{Quadratic form associated with an operator}
\begin{definition}
Let $T$ be a linear operator on an inner product space $V$. The \udef{quadratic form associated with $T$} is
\[ Q_T: \dom(T)\to \F: u\mapsto \inner{u,Tu}. \]
\end{definition}
\begin{lemma} \label{lemma:symmetricRealQuadraticForm}
If $T$ is a symmetric operator, then its associated quadratic form is real-valued.
\end{lemma}
\begin{proof}
Assume $T$ symmetric, then for all $u\in\dom(T)$
\[ Q_T(u) = \inner{u,Tu} = \inner{Tu,u} = \overline{\inner{u,Tu}} = \overline{Q(u)}. \]
\end{proof}

\subsubsection{Rayleigh quotient}
\begin{definition}
Let $T$ be a linear operator on an inner product space $V$. The \udef{Rayleigh quotient} for $T$ is 
\[ J_T: \dom(T)\setminus\{0\}\to \F: u\mapsto \frac{Q(u)}{\norm{u}^2} = \frac{\inner{u,Tu}}{\norm{u}^2}. \]
We may also write just $J$ if the intended operator $T$ is clear.
\end{definition}

\subsubsection{Numerical range}
\url{https://users.math.msu.edu/users/shapiro/pubvit/downloads/numrangenotes/numrange_notes.pdf}

\url{https://pskoufra.info.yorku.ca/files/2016/07/Numerical-Range.pdf}

\url{http://www.math.wm.edu/~ckli/nrnote}

\url{https://link-springer-com.ezproxy.ulb.ac.be/content/pdf/10.1007%2F978-3-319-01448-7.pdf}

\begin{definition}
Let $T$ be a linear operator on an inner product space $V$ and $J_T$ the Rayleigh quotient of $T$. The range $\NumRange(T) \defeq \im(J_T)$ is known as the \udef{numerical range}.
\end{definition}

The numerical range of $T$ can equivalently be defined as the image of the unit sphere under the quadratic form associated to $T$.

\begin{lemma}
Let $V$ be an inner product space and $T$ a bounded symmetric operator on $V$. Then
\begin{enumerate}
\item the directional derivative $\partial_v(J_T(u))$ exists if $u\neq 0$ and is equal to (TODO remove and place in proof?)
\[ \partial_v(J_T)|_u = \frac{\inner{u,u}\Big( \inner{v,Tu} + \inner{u,Tv} \Big) - \inner{u,Tu}\Big(\inner{u,v}+\inner{v,u}\Big)}{\inner{u,u}^2}; \]
\item $u\in V\setminus \{ 0 \}$ is a critical point of $J_T$ \textup{if and only if} $u$ is an eigenvector of $T$ with corresponding eigenvalue $\lambda = J_T(u)$.
\end{enumerate}
\end{lemma}
\begin{proof}
TODO: critical point in $\C$ v $\R$?? (For symmetric operators $J$ is real valued)
\ref{prop:derivativeBilinearForm}
\end{proof}

\subsubsection{Numerical radius}
\begin{definition}
Let $T$ be a linear operator on an inner product space $V$. Then
\[ \nr(T) \defeq \sup_{u\in \dom(T)\setminus\{0\}} |J_T(u)| \]
is the \udef{numerical radius}.
\end{definition}
If $Q_T$ is the quadratic form associated to an operator $T$, we have
\[ |Q_T(u)| \leq \norm{u}^2\nr(T). \]

\begin{proposition}
Let $T$ be a bounded operator on an inner product space $V$, then $\forall u\in \dom(T)\setminus\{0\}$
\[ |J_T(u)| \leq \nr(T) \leq \norm{T}. \]
If $T$ is also symmetric and has $\dom(T)=V$, then $\norm{T} = \nr(T)$.
\end{proposition}
\begin{proof}
The first claim follows simply from the Cauchy-Schwarz inequality \ref{theorem:CauchySchwarz}
\[ |J(u)| \leq \frac{\norm{u}\,\norm{Tu}}{\norm{u}^2} = \frac{\norm{Tu}}{\norm{u}} \leq \frac{\norm{T}\norm{u}}{\norm{u}} = \norm{T}. \]
For the second claim we need to also show the inverse inequality. By \ref{lemma:operatorNormInnerProduct} it is enough to show that $|\inner{w,Tv}| \leq \nr(T)$ for all $w,v\in V$ with $\norm{v} = 1 = \norm{w}$.

Take arbitrary unit vectors $v,w\in V$ and let $\theta$ be such that $|\inner{w,Tv}| = e^{i\theta}\inner{w,Tv}$. Then $\inner{e^{-i\theta}w,Tv}$ is real, so, viewing it as a sesquilinear form, the imaginary parts of the polarisation identity \ref{theorem:polarisationIdentities} cancel:
\begin{align*}
\inner{e^{-i\theta}w,Tv} &= \frac{1}{4}\sum_{k=0}^3i^k \inner{(i^ke^{-i\theta}w + v), T((i^ke^{-i\theta}w + Tv))} \\
&= \frac{1}{4}\Big( \inner{v+e^{-i\theta}w, T(v+e^{-i\theta}w)} - \inner{v-e^{-i\theta}w, T(v-e^{-i\theta}w)} \Big),
\end{align*}
where we have used that the quadratic form is real by \ref{lemma:symmetricRealQuadraticForm}.

Thus
\begin{align*}
|\inner{w,Tv}| &= |\inner{e^{-i\theta}w,Tv}| \\
&\leq \frac{1}{4}\Big( |\inner{v+e^{-i\theta}w, T(v+e^{-i\theta}w)}| + |\inner{v-e^{-i\theta}w, T(v-e^{-i\theta}w)}| \Big) \\
&\leq \frac{1}{4}\nr(T)\Big( \norm{v+w}^2 + \norm{v-w}^2 \Big) \\
&= \frac{1}{4}\nr(T)\Big( 2\norm{v}^2 + 2\norm{w}^2 \Big) = \nr(T),
\end{align*}
where we have used the fact that $v,w$ are unit vectors and the parallelogram law \ref{theorem:parallelogramLaw}.
\end{proof}

\chapter{Bilinear and multilinear maps}
TODO: bilinear maps, bilinear forms = bilinear functionals
orthogonality

\section{Bilinear form}
\begin{definition}
Let $V$ be a vector space over a field $\F$. A function $B:V\times V\to \F$ is called a \udef{bilinear form} on $V$ if for all $v\in V$, both $B(v,-)$ and $B(-,v)$ are linear.
\end{definition}

\subsection{Quadratic forms}
\begin{definition}
Let $V$ be a vector space over a field $\F$ and $B$ a bilinear form on $V$. The function
\[ q_B: V\to \F: v\mapsto B(v,v) \]
is called the associated \udef{quadratic form}.
\end{definition}

\begin{proposition}
Let $V$ be a vector space over a field $\F$ and $B$ a bilinear form on $V$. Then
\[ B(v,w) + B(w,v) = q_B(v+w) - q_B(v) - q_B(w). \]
\end{proposition}
\begin{corollary}
If $B$ is symmetric and $\F$ is not of characteristic $2$, then we can recover the bilinear form from the associated quadratic form:
\[ B(v,w) = \frac{1}{2}\Big(q_B(v+w) - q_B(v) - q_B(w)\Big). \]
\end{corollary}
So there is a bijection between symmetric and bilinear forms over fields not of characteristic $2$.

\begin{proposition}
Let $V$ be a vector space over a field $\F$ and $q: V\to \F$ a function. Then $q$ is the quadratic form associated to some bilinear form \textup{if and only if}
\begin{itemize}
    \item $\forall \lambda \in \F\forall v\in V: \; q(\lambda v) = \lambda^2 q(v)$;
    \item the parallelogram law holds: $\forall v,w\in V$
    \[ q(v+w) + q(v-w) = 2(q(v)+q(w)). \]
\end{itemize}
\end{proposition}
\begin{proof}
First assume $q$ is the quadratic form associated with the bilinear form $B$. Then $q(\lambda v) = B(\lambda v, \lambda v) = \lambda^2 B(v,v) = \lambda^2 q(v)$. The proof of the parallelogram law is the same as in an inner product space.

For the converse, we need to show that $q(v+w) - q(v) - q(w)$ is bilinear. TODO! (only real / complex??)
\end{proof}

\subsubsection{Finite dimensional quadratic forms}
TODO matrix representation of $q$.
\begin{definition}
Let $V$ be a finite dimensional vector space and $q$ a quadratic form on $V$. A basis $\{e_i\}_{i\in I}$ is called
\begin{itemize}
\item \udef{$q$-orthogonal} if $q(e_i + e_j) = q(e_i) + q(e_j)$ for all $i\neq j \in I$;
\item \udef{$q$-orthonormal} if it is orthogonal and $q(e_i)\in \{-1,0,1\}$ for all $i \in I$.
\end{itemize}
Let $\{e_i\}_{i\in I}$ be an orthonormal basis and
\begin{itemize}
    \item $p = |\setbuilder{e_i}{q(e_i) = 1}|$;
    \item $n = |\setbuilder{e_i}{q(e_i) = -1}|$;
    \item $z = |\setbuilder{e_i}{q(e_i) = 0}|$.
\end{itemize}
Then we call the triple $(p,n,z)$ the \udef{signature} of the basis $\{e_i\}_{i\in I}$.
\end{definition}

\begin{theorem}[Sylvester's law of inertia]
Let $V$ be a finite dimensional vector space and $q$ a quadratic form on $V$.
\begin{enumerate}
\item If $V$ is a real vector space, then any orthonormal basis has the same signature.
\item If $V$ is a complex vector space, then for any orthonormal basis, the pair $(p+n,z)$ is the same.
\end{enumerate}
\end{theorem}
\begin{proof}
TODO
\end{proof}

\begin{proposition}
Let $V$ be a finite dimensional vector space and $q$ a quadratic form on $V$.
\begin{enumerate}
\item If $\F$ is not of characteristic $2$, then $V$ has an orthogonal basis.
\item If $\F$ is a spin field, then $V$ has an orthonormal basis.
\end{enumerate}
\end{proposition}
TODO spin field.
\begin{proof}
TODO
\end{proof}


\section{Tensor product}
\url{https://kconrad.math.uconn.edu/blurbs/linmultialg/tensorprod.pdf}
\subsection{Free vector space}
Given any set, we can construct a vector space by viewing each element in the set as a (linearly) independent (basis) vector. The vector space then consists of formal linear combinations of these vectors.

To be more precise:
\begin{definition}
Let $S$ be a set and $K$ a field. Then define
\[ F_K(S) \defeq \setbuilder{f\in(S\to K)}{f^{-1}[K\setminus\{0\}]\;\text{is finite}}. \]
Define the following operations on $F_K(S)$:
\begin{align*}
+ &: F(S)\times F(S) \to F(S): (f,g)\mapsto (f+g: x\mapsto f(x)+g(x)) \\
\cdot &: K\times F(S) \to F(S): (\lambda,f)\mapsto (\lambda f: x\mapsto \lambda f(x)) \\
\end{align*}
The operations $+,\cdot$ are well-defined and make $F(S)$ into a vector space, called the \udef{free vector space} over $S$.
\end{definition}

\begin{proposition}
Let $S$ be a set. Then we can identify $S$ with a subset of $F(S)$ by
\[ \iota: S\hookrightarrow F(S): x\mapsto \chi_{\{x\}}. \]
With this identification $S$ forms a basis for $F(S)$.
\end{proposition}

\begin{lemma}
Let $V$ be a vector space over $K$ and $\beta$ a basis for $V$. Then $V\cong  F_K(V)$.
\end{lemma}

\subsubsection{The free functor}
\begin{proposition}
The operation $F$ of finding the free vector space over a set can be extended to an contravariant functor
\[ F: \cat{Set} \to \cat{Vect}. \]
\end{proposition}
\begin{proof}
Let $f:X\to Y$ be a function between sets.
\end{proof}
TODO: just specific instance up to isomorphism?? covariant?? isomorphism class of all vector spaces with basis $S$?? Is this why tensor product only up to isomorphism??

\subsubsection{Universal property}
\begin{proposition}
Let $\phi$ be an arbitrary function from $S$ to a vector space $W$ over a field $K$, then there exists a unique linear map $\overline{\phi}: F(S)\to W$ such that the diagram
\[ \begin{tikzcd}
S \ar[r,"\iota"] \ar[dr,"\phi"] & F(S) \ar[d,dashed,"\overline{\phi}"] \\
& W
\end{tikzcd} \qquad \text{commutes.} \]
Furthermore, $F(S)$ is the unique $K$-vector space with this property.
\end{proposition}

\subsection{Abstract definition}
The idea behind the tensor product of two vector spaces $V,W$ over a field $K$ is to create the most general set of pairings that is a vector space and such that the pairings are bilinear: $\forall \lambda\in K: \forall v_1,v_2,v\in V:\forall w_1,w_2,w\in W$:
\[ (\lambda v_1+v_2, w) = \lambda (v_1,w)+(v_2,w) \qquad\text{and}\qquad (v,\lambda w_1+w_2) = \lambda (v,w_1) + (v,w_2). \]
This will be realised as a quotient of a free vector space.

To be more precise:
\begin{definition}
Let $V,W$ be vector spaces over a field $K$.
Consider the set $\operatorname{Field}(V)\times \operatorname{Field}(W)$, which we will refer to as $V\times W$. Construct the sets
\begin{align*}
R_1 &= \setbuilder{(\lambda(v_1,w)+(v_2,w),(\lambda v_1+v_2, w))\in F_K(V\times W)}{\lambda\in K; v_1,v_2\in V; w\in W} \\
R_2 &= \setbuilder{(\lambda(v,w_1)+(v,w_2),(v, \lambda w_1 + w_2))\in F_K(V\times W)}{\lambda\in K; v_1,v_2\in V; w\in W} \\
R &= R_1\cup R_2
\end{align*}
The \udef{tensor product} $V\otimes W$ of the vector spaces $V$ and $W$ is the quotient vector space
\[ V\otimes W := F(V\times W)/R^\equiv \]
where $R^\equiv$ is the reflexive symmetric transitive closure of $R$.

The equivalence class $[(v,w)]$ is denoted $v\otimes w$. An element of $V\otimes W$ that can be written as $v\otimes w$ is called a \udef{pure tensor} or \udef{simple tensor}.
\end{definition}
In order for the definition to be well-defined, we need for $R^\equiv$ to be a congruence on $F$.

The \udef{tensor product} $V\otimes W$ of two vector spaces $V$ and $W$ over a common field $K$ is the quotient vector space
\[ V\otimes W := F(V\times W)/\sim \]
where $\sim$ is the equivalence relation over the the free vector space $F(V\times W)$ with the properties of
\begin{itemize}
\item \textit{Distributivity}: $(v+v', w) \sim (v,w) + (v',w)$ and $(v, w+w') \sim (v,w) + (v,w')$.
\item \textit{Scalar multiples}: $c(v,w) \sim (cv,w) \sim (v,cw)$.
\end{itemize}



TODO definition via bases: $V\otimes W = F(\beta_V\times \beta_W)$

\begin{lemma} \label{tensorProductLinearlyIndependentBasis}
Let $V,W$ be vector spaces over a field $K$. Then
\[ V\otimes W = \setbuilder{\sum_{i=1}^n v_i\otimes w_i}{n\in \N, \text{$\{v_i\}_{i=1}^n\subseteq V$ and $\{w_i\}_{i=1}^n\subseteq W$ is a linearly independent}}. \]
\end{lemma}
\begin{proof}
Suppose $v_{j} = \sum_{k\neq j}\lambda_kv_k$. Then 
\[ \sum_{i=1}^n v_i\otimes w_i = \sum_{k\neq j}\lambda_kv_k\otimes w_k + \sum_{i\neq j} v_i\otimes w_i = \sum_{i\neq j}\lambda_iv_i\otimes w_i + \sum_{i\neq j} v_i\otimes w_i = \sum_{i\neq j} (\lambda_i+1)v_i\otimes w_i. \]
The argument for $\{w_i\}$ is similar.
\end{proof}

\subsection{Universal property}
See also proposition \ref{dimHomset}.


\subsection{Tensor product of linear maps}
The tensor product also operates on linear maps between vector spaces.
\begin{definition}
Given two linear maps $S: V\to X$ and $T:W\to Y$, then the \udef{tensor product} of the linear maps $S$ and $T$ is the linear map
\[ S\otimes T: V\otimes W \to X\otimes Y \]
defined by
\[ (S\otimes T)(v\otimes w) = S(v)\otimes T(w). \]
For vectors that are not pure tensors, this definition is extended by linearity.
\end{definition}
\begin{lemma}
The tensor product of linear maps is well-defined.
\end{lemma}

With this definition the tensor product becomes a bifunctor from the category of vector spaces to itself, covariant in both arguments.

TODO: functional calculus on tensor product.
TODO: tensor product of operator algebras


\subsection{Operator-valued matrices}
\begin{proposition}
Let $A$ be an algebra over a field $\mathbb{F}$ and $\beta$ any set. Consider the direct sum $A^\beta = \bigoplus_{i\in\beta}A$. Then $A^\beta \cong F_\F(\beta)\otimes A$.
\end{proposition}
\begin{proof}
TODO
\end{proof}


\subsection{Matrix representation}
\subsubsection{Finding a basis}
Assume $V$ and $W$ are finite-dimensional vector spaces with resp. bases $\{\vec{e}_i\}_i$ and $\{\vec{f}_j\}_j$. Then the set $\{ \vec{e}_i\otimes \vec{f}_j \}_{i,j}$ forms a basis for $V\otimes W$. Indeed,
\begin{itemize}
\item Take two arbitrary vectors $\vec{v} = \sum_i a_i \vec{e}_i \in V$ and $\vec{w} = \sum_j b_j \vec{f}_j \in W$.
Using the distributivity and scalar multiples properties of $\sim$, we can write the tensor product $\vec{v}\otimes \vec{w}$ as
\begin{equation} \vec{v}\otimes \vec{w} = (\sum_i a_i \vec{e}_i)\otimes(\sum_j b_j \vec{f}_j) = \sum_{i,j}a_ib_j (\vec{e_i}\otimes \vec{f}_j). \label{eq:vtensorw} \end{equation}
So any pure tensor can be written as the sum of vectors of the form $\vec{e}_i\otimes \vec{f}_j$. In general a vector in $V\otimes W$ can be written as a finite sum of pure tensors, meaning the set of vectors $\{ \vec{e}_i\otimes \vec{f}_j \}_{i,j}$ spans $V\otimes W$.
\item For linear independence we, observe that for any linearly independent $v_1, v_2, w_1, w_2$, the vector $v_1\otimes w_1 + v_2\otimes w_2$ cannot be written as a pure tensor.
\end{itemize}

Clearly it follows that
\[ \dim(V\otimes W) = \dim(V)\cdot\dim(W) \]

\subsubsection{Coordinates and the outer product}
The coordinates of a vector with respect to the basis $\{ \vec{e}_i\otimes \vec{f}_j \}_{i,j}$ can naturally be put into a matrix. Taking the tensor product of two vectors corresponds to taking the outer product of their coordinate vectors. That is, setting $\co(v) = \vec{v}$ and $\co(w) = \vec{w}$, we get
\[ \co(v\otimes w)_{i,j} = a_ib_j = (\vec{v}\vec{w}^\transp)_{i,j} \]
which follows from \eqref{eq:vtensorw} above.

For this reason $\otimes$ is also used to denote the outer product.

If we want a proper column vector as our coordinate vector, we can apply row-by-row vectorisation to this matrix.
\[ \co(v\otimes w) = \vectorisation_R(\vec{v}\vec{w}^\transp) = \vec{v}\otimes\vec{w} = \co(v)\otimes\co(w). \]
where $\otimes$ is also used to denote the Kronecker product.

Coordinates for vectors that are not pure tensors can easily be found by the linearity of the coordinate map.
\subsubsection{Linear maps and the Kronecker product}
Letting the coordinates be columns, we can hope to find a matrix for the linear map $S\otimes T$. Fix bases for the spaces $V,W,X,Y$. Let $A$ and $B$ be the matrices of $S$ and $T$ with respect to these bases. Use these bases to fix the bases for $V\otimes W$ and $X\otimes Y$.

\begin{eigenschap}
The matrix of the map $S\otimes T$ with respect to these bases is the matrix $A\otimes B$, where $\otimes$ is the Kronecker product.
\end{eigenschap}

This follows from a simple calculation:
\begin{align*}
\co\left(S\otimes T(v\otimes w)\right) &= \co\left(S(v)\otimes T(w)\right) \\
&= \co\left(S(v)\right)\otimes \co\left(T(w)\right) \\
&= A\co(v)\otimes B\co(v) \\
&= (A\otimes B)\co(v)\otimes\co(w) \qquad (\text{using the mixed product}) \\
&= (A\otimes B)\co(v\otimes w).
\end{align*}
Again this calculation can be extended to non-pure tensors by linearity.

\subsection{Properties}
TODO currying.
\url{https://math.stackexchange.com/questions/679584/why-is-texthomv-w-the-same-thing-as-v-otimes-w}
And reference later!
\subsection{Multilinear maps}
\begin{definition}
Let $V^k = V\times \ldots \times V$. A function $f: V^k\to \R$ is \udef{$k$-linear} if it is linear in each of its arguments.
\begin{itemize}
\item A $k$-linear function $f:V^k\to \R$ is \udef{symmetric} if for all permutations $\sigma\in S_k$
\[ f(v_{\sigma(1)},\ldots, v_{\sigma(k)}) = f(v_1,\ldots, v_k). \]
\item A $k$-linear function $f:V^k\to \R$ is \udef{alternating} if for all permutations $\sigma\in S_k$
\[ f(v_{\sigma(1)},\ldots, v_{\sigma(k)}) = (\sgn\sigma)f(v_1,\ldots, v_k). \]
\end{itemize}
We call the space of all alternating $k$-linear maps $A_k(V)$.
\end{definition}
In particular $A_1(V) = V^*$.
\begin{note}
Given a $k$-linear function $f$ and a permutation $\sigma\in S_k$, we define the $k$-linear function $\sigma f$ by
\[ (\sigma f)(v_1,\ldots, v_k) = f(v_{\sigma(1)},\ldots, v_{\sigma(k)}). \]
Then a symmetric map is one such that $\sigma f = f$ for all $\sigma\in S_k$ and an alternating map is one such that $\sigma f = (\sgn \sigma)f$ for all $\sigma\in S_k$.
\end{note} 
\begin{lemma}
Let $\sigma,\tau \in S_k$ and $f$ a $k$-linear map on $V$. Then $\tau(\sigma f) = (\tau \sigma)f$.
\end{lemma}
\subsubsection{The symmetrising and alternating maps}
\begin{definition}
Let $f$ be a $k$-linear map on a vector space $V$.
\begin{itemize}
\item The \udef{symmetrisation} of $f$, $Sf$, is the map
\[ Sf = \frac{1}{k!}\sum_{\sigma\in S_k}\sigma f. \]
\item The \udef{anti-symmetrisation} or \udef{skew-symmetrisation} of $f$, $Af$, is the map
\[ Af = \frac{1}{k!}\sum_{\sigma\in S_k}(\sgn \sigma)\sigma f. \]
\end{itemize}
\end{definition}
\begin{lemma}
\begin{enumerate}
\item The $k$-linear map $Sf$ is symmetric. If $f$ is symmetric, then $Sf = f$.
\item The $k$-linear map $Af$ is alternating. If $f$ is alternating, then $Af = f$.
\end{enumerate}
\end{lemma}
\begin{lemma} \label{idempotenceA}
Let $f$ be a $k$-linear functional and $g$ an $l$-linear functional on $V$. Then
\[ A(A(f)\otimes g) = A(f\otimes g) = A(f\otimes A(g)). \]
\end{lemma}
\subsubsection{The wedge product}
\begin{definition}
Let $f\in A_k(V)$ and $g\in A_l(V)$. The \udef{wedge product} of $f$ and $g$ is given by
\[ f\wedge g = \frac{(k+l)!}{k!l!}A(f\otimes g). \]
\end{definition}
We can also write
\[ (f\wedge g)(v_1,\ldots,v_{k+l}) = \frac{1}{k!l!}\sum_{\sigma\in S_{k+l}}(\sgn \sigma)f(v_{\sigma(1)},\ldots,v_{\sigma(k)})g(v_{\sigma(k+1)},\ldots, v_{\sigma(k+l)}). \]
We can reduce redundancies in this definition in the following way:
We call $\sigma\in S_{k+l}$ a \udef{$(k,l)$-shuffle} if
\[ \sigma(1)<\ldots<\sigma(k) \qquad \text{and}\qquad \sigma(k+1)<\ldots<\sigma(k+l). \]
The we write
\[ (f\wedge g)(v_1,\ldots,v_{k+l}) = \sum_{\text{$(k,l)$-shuffles $\sigma$}}(\sgn \sigma)f(v_{\sigma(1)},\ldots,v_{\sigma(k)})g(v_{\sigma(k+1)},\ldots, v_{\sigma(k+l)}). \]
\begin{proposition}
Let $f\in A_k(V)$ and $g\in A_l(V)$. Then
\[ f\wedge g = (-1)^{kl}g\wedge f. \]
\end{proposition}
\begin{lemma}
The wedge product is associative:
\[ (f\wedge g)\wedge h = f\wedge (g\wedge h). \]
\end{lemma}
Proof using \ref{idempotenceA}.

\begin{lemma}
Let $\alpha^1,\ldots, \alpha^k$ be linear functionals on $V$ and $v_1,\ldots,v_k\in V$, then
\[ (\alpha^1\wedge\alpha^k)(v_1,\ldots, v_k) = \det[\alpha^i(v_j)]. \]
\end{lemma}

\subsection{Tensors}
A $(p,k)$-tensor is a multilinear function $V^k\to V^p$.

\section{Real, complex and quaternionic vector spaces}
\begin{definition}
A function $f$ between complex vector spaces is \udef{anti-linear} (or \udef{conjugate-linear}) in the first component:
\[f(\lambda_1 v_1 + \lambda_2 v_2) = \overline{\lambda_1}f(v_1) + \overline{\lambda_2}f(v_2),\]
where $\lambda_1,\lambda_2 \in \C$ and $v_1,v_2\in \dom(f)$.
\end{definition}
\subsection{Complex structure on a real vector space}
\begin{definition}
Let $V$ be a real vector space. A \udef{complex structure} on $V$ is a linear map $J: V\to V$ such that $J^2 = -I_V$.
\end{definition}

\subsection{The real vector spaces associated to a complex vector space}
Let $V = (\C, V, +)$. Then define $V_\R \defeq (\R,V,+)$.

every anti-linear map $A:V\to W$ is an $\R$-linear map $A:V_\R\to W_\R$. (They are equal as sets).

\section{Quotient algebras of dual systems}
\begin{definition}
Let $\sSet{X,Y,b}$ be a dual system over a field $\F$. The algebra generated by this dual system is
\[ A(X,Y,b) \defeq \bigoplus_{n\in\N}(X\oplus Y)^{\otimes n}/\genIdealBuilder{x\otimes y - b(x,y)\vec{1}}{x\in X, y\in Y}. \]
\end{definition}
TODO universal algebra presented by generators and relation.

\subsection{The $\Z_2$-grading}
TODO parity grading grading respects ideal!

\begin{proposition}
Let $\sSet{X,Y,b}$ be a dual system. There is a faithfull superalgebra representation
\[ A(X,Y,b) \hookrightarrow \End\left(\bigoplus_{n\in \N}Y^{\otimes n}\right) \]
given by the extension (TODO universal property x2) of
\[ X\oplus Y \to \End(Y^{\otimes n}): x+y \mapsto \Big( y_1\otimes\ldots y_n \mapsto b(x,y_1) y\otimes y_2\otimes \ldots \otimes y_n \Big). \]
\end{proposition}

\section{Clifford algebras}
\begin{definition}
Let $V$ be a vector space over a field $\mathbb{F}$ and $q$ a quadratic form defined on $V$.
Let $\mathcal{T}(V)$ be the tensor algebra
\[ \mathcal{T}(V) \defeq \mathbb{F}\oplus \bigoplus_{n=1}^\infty V^n = \mathbb{F}\oplus \bigoplus_{n=1}^\infty \underbrace{V\otimes \ldots \otimes V}_{\text{$n$ times}}. \]
Let $\mathcal{I}(V,q)$ be the (two-sided) ideal in $\mathcal{T}(V)$ generated by
\[ \setbuilder{\vec{v}\otimes \vec{v} - q(v) \vec{1}}{\vec{v}\in V}. \]
Then the \udef{Clifford algebra} $\Cl(V,q)$ associated with $V$ and $q$ is the quotient
\[ \Cl(V,q) \defeq \mathcal{T}(V)/\mathcal{I}(V,q). \]
We call
\begin{itemize}
    \item elements of the Clifford algebra \udef{multivectors};
    \item elements of $\Span(V^k)$ \udef{$k$-vectors};
    \item elements of $V$ \udef{vectors}; we use bold face to denote these elements (e.g.\ $\vec{v}$);
    \item elements of $\F\vec{1}$ \udef{scalars}.
\end{itemize}
Elements of the Clifford algebra are called \udef{multivectors}.
\end{definition}
Let $\pi_q$ be the canonical projection
\[ \pi_q: \mathcal{T}(V) \to \Cl(V,q). \]


TODO: make all vectors bold!

\begin{lemma}
The embedding $V\hookrightarrow \Cl(V,q)$ is faithful, i.e.\ $\pi_q|_V$ is injective.
\end{lemma}
\begin{proof}
TODO
\end{proof}
Clearly $\pi_q|_V(\vec{v})^2 = q(\vec{v}) \vec{1}$ for all $\vec{v}\in V$.

\begin{lemma} \label{vectorInverseCliffordAlgebra}
Let $\Cl(V,q)$ be a Clifford algebra. Then
\[ \Cl^\times(V,q) \cap V = \setbuilder{\vec{v}\in V}{q(v) \neq 0}. \]
The inverse of $\vec{v}\in \Cl^\times(V,q) \cap V$ is $\vec{v}^{-1} = \vec{v}/q(\vec{v})$.
\end{lemma}
\begin{proof}
First take $\vec{v}\in \setbuilder{\vec{v}\in V}{q(\vec{v}) \neq 0}$. This definition of $\vec{v}^{-1}$ is indeed a multiplicative inverse: $\vec{v}\vec{v}^{-1} = \vec{v}^2/q(\vec{v}) = q(\vec{v})/q(\vec{v})\vec{1} = \vec{1}$.

Conversely, take $\vec{v}\in \Cl^\times(V,q) \cap V$. If $q(\vec{v}) = 0$, then $\vec{v}$ would be a zero divisor (as $\vec{v}\vec{v} = q(\vec{v})\vec{1} = 0$). No zero divisor can have an inverse (\ref{inverseZeroDivisor}), 
\end{proof}

\begin{lemma} \label{CliffordRelation}
The algebra $Cl(V,q)$ is generated by the vector space $V$ and $\vec{1}$, subject to the relations
\[ \vec{v}\vec{v} = q(\vec{v})\vec{1} \qquad \forall \vec{v}\in V. \]
\end{lemma}
TODO generators of an algebra!


Clifford algebras can also be defined by their universal property:
\begin{proposition}[Universal property of Clifford algebras] \label{CliffordUniversalProperty}
Let $V$ be a vector space over a field $\mathbb{F}$ and $q$ a quadratic form on $V$. 

Then for any unital associative algebra $A$ over $\mathbb{F}$ and linear map $j: V \to A$ such that
\[ j(\vec{v})^2 = q(\vec{v}) \vec{1} \qquad \forall \vec{v}\in V \]
there exists a unique algebra homomorphism $\widetilde{j}: \Cl(V,q)\to A$ such that the following diagram commutes:
\[ \begin{tikzcd}
V \rar{\pi_q|_V} \ar[dr, swap, "{j}"] & \Cl(V,q) \dar[dashed]{\widetilde{j}} \\
 & A
\end{tikzcd} \]
Furthermore, $\Cl(V,q)$ is the unique associative $\mathbb{F}$-algebra with this property.
\end{proposition}
\begin{corollary}
Let $(V,q)$ and $(V',q')$ be vector spaces with quadratic forms. If a linear map $f:V\to V'$ preserves to quadratic form, $q'\circ f = q$, then $f$ extends to a unique algebra homomorphism
\[ \widetilde{f}: \Cl(V,q) \to \Cl(V',q'). \]
Now let $(V^{\prime\prime},q^{\prime\prime})$ be another vector space equipped with a quadratic form and let $g: V'\to V^{\prime\prime}$ be a linear map preserving the quadratic form. Then
\[ \widetilde{g\circ f} = \widetilde{g}\circ\widetilde{f}. \]
Also isomorphisms of vector spaces extend to isomorphisms of Clifford algebras.
\end{corollary}
\begin{proof}
Let the algebra $A$ of the proposition be $\Cl(V',q')$. Then $\pi_q|_V\circ f$ satisfies the requirement for $j$:
\[ [(\pi_{q'}|_V\circ f)(\vec{v})]^2 = q'(f(\vec{v}))^2 \vec{1} = q(\vec{v})^2\vec{1}. \]
Thus by the proposition, there is a unique extension of $f:V\to V'$ to a map $\Cl(V,q) \to \Cl(V',q')$.

The composition relation follows from uniqueness.
\end{proof}
\begin{corollary} \label{qOrthogonalMaps}
The orthogonal group
\[ \Ogroup(V,q) = \setbuilder{g\in\GL(V)}{q\circ g = q} \]
extends canonically to a group of automorphisms of $\Cl(V,q)$:
\[ \Ogroup(V,q) \subset \Aut(\Cl(V,q)). \]
\end{corollary}

\subsection{Scalar and outer products}
\begin{lemma}
Let $\Cl(V,q)$ be a Clifford algebra and $\vec{v},\vec{w}\in V$. Then $\vec{v}\vec{w} + \vec{w}\vec{v}$ is a scalar multiple of the identity.
\end{lemma}
\begin{proof}
We can calculate
\[ q(\vec{v}+\vec{w})\vec{1} = (\vec{v}+\vec{w})^2 = \vec{v}^2 + \vec{v}\vec{w} + \vec{w}\vec{v} + \vec{w}^2 = \vec{v}\vec{w} + \vec{w}\vec{v} + q(\vec{v})\vec{1} + q(\vec{w})\vec{1}, \]
so $vw+wv = [q(\vec{v}+\vec{w})-q(\vec{v})-q(\vec{w})]\vec{1}$.
\end{proof}

\begin{definition}
Let $\F$ be a field whose characteristic is not $2$ and $\Cl(V,q)$ a Clifford algebra over $\F$. We can then write, for $\vec{v},\vec{w}\in V$
\begin{align*}
\vec{v}\vec{w} &= \frac{\vec{v}\vec{w}+\vec{w}\vec{v}}{2} + \frac{\vec{v}\vec{w}-\vec{w}\vec{v}}{2} \\
&\defeq \vec{v}\cdot \vec{w} + \vec{v}\wedge \vec{w}.
\end{align*}
We call the symmetric part $\vec{v}\cdot \vec{w}$ the \udef{scalar product} (sometimes also called the \udef{inner product}),
and the antisymmetric part $\vec{v}\wedge \vec{w}$ the \udef{outer product}.
\end{definition}
In the sequel, whenever we talk about the scalar and outer product, we will always assume the characteristic of the field is not $2$.

\begin{lemma}
Let $\Cl(V,q)$ be a Clifford algebra and $\vec{v}\in V$. We have $\vec{v}\cdot \vec{v} = \vec{v}\vec{v} = \vec{v}^2 = q(\vec{v})\vec{1}$.
\end{lemma}
So $\vec{v}^2$ can cause no confusion.

\begin{lemma} \label{CliffordAlgebraVectorSwap}
Let $\Cl(V,q)$ be a Clifford algebra and $\vec{v},\vec{w}\in V$.
Then
\begin{enumerate}
\item $\vec{w}\vec{v} = 2(\vec{w}\cdot \vec{v}) - \vec{v}\vec{w}$;
\item $\vec{w}\vec{v} = 2(\vec{w}\wedge \vec{v}) + \vec{v}\vec{w} = -2(\vec{v}\wedge \vec{w}) + \vec{v}\vec{w}$;
\item $\vec{v}\vec{w}^2 = 2 (\vec{v}\cdot \vec{w})\vec{v}\vec{w} - \vec{v}^2 \vec{w}^2$;
\item $\vec{v}\vec{w}\vec{v} = 2(\vec{v}\cdot \vec{w})\vec{v} - q(\vec{v})\vec{w}$.
\end{enumerate}
\end{lemma}

\begin{lemma} \label{CliffordIdentities}
Let $\Cl(V,q)$ be a Clifford algebra and $\vec{v},\vec{w}\in V$.
Then
\begin{enumerate}
\item $\vec{v}(\vec{v}\wedge \vec{w}) = (\vec{w}\wedge \vec{v})\vec{v}$;
\end{enumerate}
\end{lemma}
\begin{proof}
(1) We calculate
\[ 2\vec{v}(\vec{v}\wedge \vec{w}) = q(\vec{v})\vec{w} - \vec{v}\vec{w}\vec{v} = \vec{w}q(\vec{v}) - \vec{v}\vec{w}\vec{v} = (\vec{w}\vec{v} - \vec{v}\vec{w})\vec{v} = 2(\vec{w}\wedge \vec{v})\vec{v}. \]
\end{proof}

\begin{definition}
Let $\Cl(V,q)$ be a Clifford algebra. We call $\vec{v},\vec{w}\in V$
\begin{itemize}
\item \udef{orthogonal} or \udef{perpendicular} if $\vec{v}\cdot \vec{w} = 0$;
\item \udef{parallel} if $\vec{v}\wedge \vec{w} = 0$.
\end{itemize}
\end{definition}
\begin{lemma}
Let $\Cl(V,q)$ be a Clifford algebra and $\vec{v},\vec{w}\in V$. The following are equivalent:
\begin{enumerate}
\item $\vec{v}$ and $\vec{w}$ are orthogonal;
\item $\vec{v}\vec{w} = \vec{v}\wedge \vec{w}$;
\item $\vec{v}\vec{w} = -\vec{w}\vec{v}$;
\item $q(\vec{v}+\vec{w}) = q(\vec{v})+ q(\vec{w})$;
\end{enumerate}
as are
\begin{enumerate} \setcounter{enumi}{4}
\item $\vec{v}$ and $\vec{w}$ are parallel;
\item $\vec{v}\vec{w} = \vec{v}\cdot \vec{w}$.
\end{enumerate}
\end{lemma}


\subsection{Involutions}
\subsubsection{Grade involution}
Given a Clifford algebra $\Cl(V,q)$, consider the map $\alpha: V \to V: \vec{v}\mapsto -\vec{v}$ on the vector space $V$.
Now $\alpha$ is always an element of $\Ogroup(V,q)$, so by \ref{qOrthogonalMaps} it extends to a map on the Clifford algebra.
\[ \widetilde{\alpha}: \Cl(V,q) \to \Cl(V,q). \]
Since $\alpha^2 = I_V$, we have that
\[ \widetilde{\alpha}^2 = \widetilde{\alpha^2} = \widetilde{I_V} = I_{\Cl(V,q)} \]
meaning $\widetilde{\alpha}$ is an involution on the Clifford algebra. From now on we drop the tilde and just write $\alpha: \Cl(V,q) \to \Cl(V,q)$ for the \udef{grade involution}.

\begin{lemma}
The grade involution $\alpha: \Cl(V,q) \to \Cl(V,q)$
\begin{enumerate}
\item is an algebra homomorphism and thus multiplicative:
\[ \alpha(xy) = \alpha(x)\alpha(y) \qquad\forall x,y\in \Cl(V,q); \]
\item is unital, $\alpha(\vec{1}) = \vec{1}$, and thus preserves inverses:
\[ \alpha(x^{-1}) = \alpha(x)^{-1} \qquad\forall x\in \Cl(V,q); \]
\item generates a $\Z_2$-grading 
\[ \Cl(V,q) = \Cl^0(V,q)\oplus \Cl^1(V,q). \]
\end{enumerate}
\end{lemma}

\subsubsection{Transpose}
The transpose map defined on the tensor algebra $\mathcal{T}(V)$, i.e.\ the linear map that reverses to order of homogeneous elements:
\[ v_1\otimes \ldots \otimes v_r \mapsto v_r \otimes \ldots \otimes v_1, \]
preserves the ideal $\mathcal{I}(V,q)$, and so determines a well-defined map on the Clifford algebra $\Cl(V,q)$:
\[ (-)^t: \Cl(V,q)\to \Cl(V,q). \]
\begin{lemma}
The transpose $(-)^t: \Cl(V,q)\to \Cl(V,q)$ is
\begin{enumerate}
\item an involution;
\item an anti-automorphism:
\[ \forall x,y\in \Cl(V,q): \quad (xy)^t = y^tx^t; \]
\item unital.
\end{enumerate}
\end{lemma}

\subsubsection{Clifford conjugation}
\begin{definition}
The composition of the grade involution and the transpose is called \udef{Clifford conjugation}:
\[ x\mapsto \overline{x} \defeq \alpha(x^t). \]
\end{definition}
\begin{lemma}
The grade involution and transpose commute
\[ \alpha \circ(-)^t = (-)^t\circ \alpha \]
and Clifford conjugation is thus equal to both.
\end{lemma}

\begin{lemma}
Clifford conjugation is
\begin{enumerate}
\item an involution;
\item an anti-automorphism:
\[ \forall x,y\in \Cl(V,q): \quad (xy)^t = y^tx^t; \]
\item unital.
\end{enumerate}
\end{lemma}

\subsubsection{Quaternion types of Clifford algebra types}

\subsection{The norm mapping}
\begin{definition}
We define the \udef{norm mapping} $N$ by
\[ N: \Cl(V,q)\to \Cl(V,q): x\mapsto x \overline{x}. \]
\end{definition}
\begin{lemma} \label{normIsQuadraticForm}
\begin{enumerate}
\item If $v\in V$, then $N(v) = q(v)$.
\item $\alpha\circ N = N\circ \alpha$.
\end{enumerate}
\end{lemma}

\begin{proposition}
Let $\Cl(V,q)$ be a Clifford algebra and $\vec{v},\vec{w}\in V$. Then
\begin{enumerate}
\item $N(\vec{v}\vec{w}) = \vec{v}^2 \vec{w}^2$;
\item $N(\vec{v}\wedge \vec{w}) = \vec{v}^2\vec{w}^2 - (\vec{v}\cdot \vec{w})^2$;
\end{enumerate}
\end{proposition}
\begin{proof}
(1) Because $\vec{v}\vec{w}\in\Cl(V,q)^0$, we have $N(\vec{v}\vec{w}) = (\vec{v}\vec{w})(\vec{w}\vec{v}) = \vec{v}^2 \vec{w}^2$.

(2) Because $\vec{v}\wedge \vec{w}\in\Cl(V,q)^0$, we have $N(\vec{v}\wedge \vec{w}) = (\vec{v}\wedge \vec{w})(\vec{w}\wedge \vec{v})$. We use $\vec{v}\wedge \vec{w} = -\vec{w}\wedge \vec{v} = \vec{v}\vec{w} - \vec{v}\cdot \vec{w}$ to calculate
\[ N(\vec{v}\wedge \vec{w}) = -(\vec{v}\vec{w} - \vec{v}\cdot \vec{w})(\vec{v}\vec{w} - \vec{v}\cdot \vec{w}) = -\vec{v}\vec{w}\vec{v}\vec{w} + 2(\vec{v}\cdot \vec{w})\vec{v}\vec{w} - (\vec{v}\cdot\vec{w})^2 \]
Then \ref{CliffordAlgebraVectorSwap} gives $\vec{v}\vec{w}\vec{v}\vec{w} = 2 (\vec{v}\cdot \vec{w})\vec{v}\vec{w} - \vec{v}^2 \vec{w}^2$.
Plugging this back in gives
\[ N(\vec{v}\wedge \vec{w}) = -2 (\vec{v}\cdot \vec{w})\vec{v}\vec{w} + \vec{v}^2 \vec{w}^2 + 2(\vec{v}\cdot \vec{w})\vec{v}\vec{w} - (\vec{v}\cdot\vec{w})^2 = \vec{v}^2 \vec{w}^2 - (\vec{v}\cdot\vec{w})^2. \]
\end{proof}


\subsection{Clifford algebras as filtered algebras}
\begin{proposition}
A Clifford algebra $\Cl(V,q)$ has a filtration $F_k \defeq \Span(V^{k})$.
\end{proposition}

We then have an associated graded algebra and a grade operator.

\begin{proposition}
Let $V$ be a vector space and $q$ a quadratic form on $V$.
\begin{enumerate}
\item As graded algebras, $\Cl(V,q)$ is naturally isomorphic to the exterior algebra ${\textstyle\bigwedge}^* V$.
\item As algebras ${\textstyle\bigwedge}^* V \cong \Cl(V,0)$.
\item As vector spaces, there is an isomorphism
\[ {\textstyle\bigwedge}^* V \to \Cl(V,q): v_1\wedge \ldots \wedge v_n \mapsto \frac{1}{r!}\sum_{\sigma\in S_n}\sgn(\sigma)v_{\sigma(1)}\hdots v_{\sigma(r)} \]
compatible with the fibrations.
\end{enumerate}
\end{proposition}

\begin{lemma}
Let $\Cl(V,q)$ be a Clifford algebra and $\vec{v},\vec{w}\in V$. Then $\grade{\vec{v}\vec{w}}_1 = 0$.
\end{lemma}
\begin{proof}
We have $\alpha(\vec{v}\vec{w}) = (-\vec{v})(-\vec{w}) = \vec{v}\vec{w}$, so $\vec{v}\vec{w}\in\Cl^0(V,q)$, while $V\subseteq \Cl^1(V,q)$.
\end{proof}

\begin{proposition}
Let $\Cl(V,q)$ be a Clifford algebra and $\vec{v}_1, \ldots, \vec{v}_k\in V$. Then $\vec{v}_1\ldots\vec{v}_k \in V^{k-1}$ \textup{if and only if} $\vec{v}_1,\ldots,\vec{v}_k$ are linearly dependent.
\end{proposition}
\begin{proof}
If $\vec{v}_1, \ldots, \vec{v}_k\in V$ are linearly dependent, then we can eliminate one of the $\vec{v}_j$ and then $\vec{v}_1\ldots\vec{v}_k \in V^{k-1}$.

Now assume $\vec{v}_1, \ldots, \vec{v}_k\in V$ are linearly independent.
\end{proof}

\subsection{Orthogonal decomposition}
As $q(u,v)$ is a bilinear form, we can consider orthogonal subspaces with respect to it. Then $V=V_1\oplus V_2$ is a $q$-orthogonal decomposition if and only if $\forall v_1\in V_1, v_2\in V_2$:
\[ q(v_1,v_2) = 0 \qquad \iff \qquad q(v_1+v_2) = q(v_1) + q(v_2). \]
\begin{proposition}
Let $V=V_1\oplus V_2$ be a $q$-orthogonal decomposition. Then there is a natural isomorphism of Clifford algebras
\[ \Cl(V,q) \;\to\; \Cl(V_1,q|_{V_1}) \hat{\otimes} \Cl(V_2, q|_{V_2}):\quad v_1+v_2\mapsto v_1\otimes \vec{1} + \vec{1}\otimes v_2. \]
\end{proposition}

\section{Subgroups of a Clifford algebra}

\begin{proposition}
Let $V$ be a finite-dimensional real or complex vector space of dimension $\dim V = n$. Then the group $\Cl^\times(V,q)$ of multiplicative units in the Clifford algebra is a Lie group of dimension $2^n$ and the corresponding Lie algebra $\mathfrak{cl}^\times(V,q)$ is the full Clifford algebra $\Cl(V,q)$ with the Lie bracket
\[ [x,y] = xy - yx.  \]
\end{proposition}

\subsection{Inner automorphisms of $\Cl(V,q)$}
Characteristic for field not 2!!


\begin{proposition} \label{AdOrthogonalDecomposition}
Let $\vec{v}\in V\cap \Cl^\times(V,q)$. Then for all $\vec{w}\in V$:
\[ \Ad_{\vec{v}}(\vec{w}) = \frac{\vec{v}\vec{w}\vec{v}}{q(\vec{v})} = \frac{2\vec{v}\cdot\vec{w}}{q(\vec{v})}\vec{v} - \vec{w}. \]
In particular $\Ad_{\vec{v}}[V] = V$.
\end{proposition}
\begin{proof}
From \ref{vectorInverseCliffordAlgebra} we have $q(\vec{v})\neq 0$ and $\vec{v}^{-1} = \vec{v}/q(\vec{v})$.
We then calculate
\[ q(\vec{v})\Ad_{\vec{v}}(\vec{w}) = q(\vec{v})\vec{v}\vec{w}\vec{v}^{-1} = \vec{v}\vec{w}\vec{v} = (\vec{v}\vec{w} + \vec{w}\vec{v} - \vec{w}\vec{v})\vec{v} = (\vec{v}\vec{w}+\vec{w}\vec{v})\vec{v} - q(\vec{v})\vec{w}. \]
\end{proof}

\begin{lemma} \label{AdOrthogonalMap}
Let $\vec{v}\in V$ such that $q(\vec{v})\neq 0$. Then $\Ad_{\vec{v}}\in \Ogroup(V,q)$.
\end{lemma}
TODO renew notation
\begin{proof}
Clearly $\Ad_{\vec{v}}$ is invertible. We then calculate using \ref{AdOrthogonalDecomposition}
\begin{align*}
q(\Ad_v(w)) &= q\left(\frac{q(v,w)}{q(v)}v - w\right) = q\left(\frac{q(v,w)}{q(v)}v,-w\right) + q\left(\frac{q(v,w)}{q(v)}v\right) + q(w) \\
&= -\frac{q(v,w)}{q(v)}q(v,w)+\left(\frac{q(v,w)}{q(v)}\right)^2q(v) + q(w) = q(w).
\end{align*}
\end{proof}


\subsection{Pin and Spin groups}
\begin{definition}
Let $P(V,q)$ be the subgroup of $\Cl^\times(V,q)$ generated by elements $v\in V$ with $q(v)\neq 0$.

We also define the group
\[ \Gamma(V,q) \defeq \setbuilder{x\in\Cl^\times(V,q)}{\Ad_x[V] = V}. \]
which is called the \udef{Clifford group}, \udef{Lipschitz group} or \udef{Clifford-Lipschitz group}.
\end{definition}
That $\Gamma(V,q)$ is a group follows from the following observation: If $\Ad_x[V]=V$ and $\Ad_y[V]=V$, then
\[ \Ad_{xy}[V] = \Ad_x[\Ad_y[V]] = \Ad_x[V] = V. \]


\begin{lemma} \label{PsubgroupVpreserving}
There is an inclusion
\[ P(V,q) \subset \Gamma(V,q). \]
\end{lemma}
\begin{proof}
The generators of $P(V,q)$ are in $\Gamma(V,q)$ by \ref{AdOrthogonalDecomposition} and $\Gamma(V,q)$ is a group.
\end{proof}

\begin{lemma}
The $\Ad$ function defines a representation
\[ \Ad: P(V,q) \to \Ogroup(V,q). \]
\end{lemma}
\begin{proof}
This mapping is well-defined by \ref{AdOrthogonalMap} and the identity
\[ \Ad_{xy} = \Ad_x\circ \Ad_y. \]
This identity also shows that the mapping is a group homomorphism, and thus that it is a representation.
\end{proof}

\begin{lemma}
Let $x\in \Gamma(V,q)$. Then
\begin{enumerate}
\item $\alpha(x)\in \Gamma(V,q)$;
\item $x^t\in \Gamma(V,q)$.
\end{enumerate}
\end{lemma}
\begin{proof}
We calculate
\[ V = \alpha[V] = \alpha[\Gamma(V,q)] = \alpha(\alpha(x)) V \alpha(x)^{-1} = \Ad_{\alpha(x)}[V] \]
and
\[ V = (\alpha[V])^t = \alpha(x^t) V (x^t)^{-1} = \Ad_{x^t}[V] \]
and the third follows by multiplicative closure.
\end{proof}
A consequence of this lemma is that $N(x)\in \Gamma(V,q)$ for all $x\in \Gamma(V,q)$. But we will show that something stronger holds, namely $N(x)\in\F^\times$.

\begin{definition}
The \udef{Pin group} of $(V,q)$ is the subgroup $\Pin(V,q)$ of $P(V,q)$ generated by the elements $v\in V$ with $q(v) = \pm 1$.

The \udef{Spin group} of $(V,q)$ is defined by
\[ \Spin(V,q) = \Pin(V,q) \cap \Cl^0(V,q) \]
where $\Cl^0(V,q)$ is the even subalgebra of $\Cl(V,q)$.
\end{definition}

\subsection{The twisted adjoint representation}
Consider the map $\Ad_v$ acting on the $q$-orthogonal decomposition (TODO ref)
\[ V = \Span\{v\}\oplus\Span\{v\}^\perp \qquad \text{for some $v\in P(V,q)$,} \]
where $\Span\{v\}^\perp = \setbuilder{w\in V}{q(v,w)=0}$. Then, by the formula
\[ \Ad_v(w) = \frac{q(v,w)}{q(v)}v - w ,\]
we see that elements of $\Span\{v\}$ are mapped to themselves:
\begin{align*}
\Ad_v(\lambda v) &= \frac{q(v,\lambda v)}{q(v)}v - \lambda v = \lambda\frac{q(2v)-2q(v)}{q(v)}v - \lambda v \\
&= 2\lambda v -\lambda v = \lambda v.
\end{align*}
and that elements $w\in\Span\{v\}^\perp$ are mapped to $-w$.

This means that $\Ad_v$ is orientation-preserving if $\dim(V)$ is odd and orientation-reversing otherwise.

We would prefer the action of $\Ad_v$ to do the opposite: fix the hyperplane $\Span\{v\}^\perp$ and invert $\Span\{v\}$. To that end we introduce the twisted adjoint representation.
\begin{definition}
The \udef{twisted adjoint representation} $\widetilde{\Ad}: \Cl^\times(V,q) \to \GL(\Cl(V,q))$ is defined by
\[ \widetilde{\Ad}_x(y) = \alpha(x)yx^{-1} \qquad \forall x\in \Cl^\times(V,q), \forall y\in \Cl(V,q) \]
where $\alpha$ is the grade involution.
\end{definition}
\begin{lemma}
Let $x,y\in \Cl^\times(V,q)$ and $v,w\in V$. Then
\begin{enumerate}
\item $\widetilde{\Ad}_{xy} = \widetilde{\Ad}_x\circ \widetilde{\Ad}_y$;
\item $\widetilde{\Ad}_x = \Ad_x$ if $x\in \Cl^0(V,q)$;
\item $\widetilde{\Ad}_v(w) = w-\frac{q(v,w)}{q(v)}v$.
\end{enumerate}
\end{lemma}

We have
\[ \Gamma(V,q) = \setbuilder{x\in\Cl^\times(V,q)}{\widetilde{\Ad}_x[V] = V}. \]

\begin{proposition}
Let $V$ be a finite-dimensional vector space over a field $\mathbb{F}$ and $q$ non-degenerate. Then the kernel of the homomorphism
\[ \widetilde{\Ad}: \Gamma(V,q) \to \GL(V) \]
is exactly the group $\mathbb{F}^\times$.
\end{proposition}
\begin{proof}
Choose an orthogonal basis $v_1,\ldots, v_n$ for $V$ w.r.t. the bilinear form $q(-,-)$ (TODO ref; also proof here only finite-dim: can it generalise?). Suppose $x\in \Cl^\times(V,q)$ is in the kernel of $\widetilde{\Ad}$, then
\[ \alpha(x)v = vx \qquad \text{for all $v\in V$.} \]
Now we can write $x = x_0 + x_1$ where $x_0$ is even and $x_1$ is odd and both are polynomial expressions in $v_1,\ldots, v_n$. Making use of $v_iv_j = \pm v_jv_i$, we can write $x_0 = a_0 + v_1a_1$ where $a_0,a_1$ are polynomial expressions in $v_2,\ldots, v_n$. Then $a_0$ is even and $a_1$ is odd, so
\[ v_1a_0 + v_1^2a_1 = v_1(a_0+v_1a_1) = (a_0+v_1a_1)v_1 = a_0v_1 + v_1 a_1 v_1 = v_1a_0-v_1^2 a_1. \]
Thus $v^2_1a_1 = -q(v_1)a_1 = 0$, so $a_1=0$ and $x_0$ does not involve $v_1$. By induction $x_0$ does not involve any of $v_1,\ldots, v_n$ and thus $x_0 = \lambda \vec{1}$ for $\lambda\in\mathbb{F}$.

A similar argument shows that $x_1$ is independent of $v_1,\ldots, v_n$ and thus $x_1=0$. Here it is important that $v_1x_1 = -x_1v_1$, because in $x_1 = a_0 + v_1a_1$, $a_0$ is now odd and $a_1$ even.

Thus $x = x_0+x_1 = \lambda \vec{1}$ and $x\neq 0$, so $x\in \mathbb{F}^\times$.
\end{proof}
This proof only works for the \textit{twisted} adjoint representation, not the adjoint representation.

It is also clearly important that $q$ be non-degenerate.

\begin{corollary} \label{normHomomorphism}
The restriction of the norm $N$ to $\Gamma(V,q)$ gives a homomorphism
\[ N: \Gamma(V,q) \to \mathbb{F}^\times. \]
\end{corollary}
\begin{proof}
If we can show that $N[\Gamma(V,q)]\subset \mathbb{F}^\times$, then the multiplicativity of $N$ follows from
\[ N(xy) = xy\alpha((xy)^t) = xy\alpha(y^t)\alpha(x^t) = xN(y)\alpha(x^t) = x\alpha(x^t)N(y) = N(x)N(y). \]

Thus by the proposition it is enough to show that for all $x\in \Gamma(V,q)$, $N(x)\in \ker(\widetilde{\Ad})$. Because $\alpha(x)vx^{-1}\in V$, the transpose leaves it unchanged:
\[ \alpha(x)vx^{-1} = (\alpha(x)vx^{-1})^t = (x^t)^{-1}v\alpha(x^t). \]
This can be rewritten as
\[ v = x^t\alpha(x)vx^{-1}(\alpha(x^t))^{-1} = \alpha(\alpha(x^t)x)v(\alpha(x^t)x)^{-1} = \widetilde{\Ad}_{N(x)}(v). \]
Thus $N(x)\in \ker(\widetilde{\Ad})$.
\end{proof}
\begin{corollary}
Let $x\in \Gamma(V,q)$, then $\widetilde{\Ad}_x \in \Ogroup(V,q)$. Thus there is a group homomorphism
\[ \widetilde{\Ad}: \Gamma(V,q) \to \Ogroup(V,q). \]
\end{corollary}
\begin{proof}
First assume $v\in V^\times = \setbuilder{v\in V}{q(v)\neq 0}\subset \Gamma(V,q)$. Then by \ref{normIsQuadraticForm} and the previous corollary
\[ q(\widetilde{\Ad}_x(v)) = N(\widetilde{\Ad}_x(v)) = N(\alpha(x)vx^{-1}) = N(\alpha(x))N(v)N(x^{-1}) = N(v)N(x)N(x^{-1}) = N(v) = q(v). \]

Now assume $q(v) = 0$. If $q(\widetilde{\Ad}_x(v))$ were not zero, then $\widetilde{\Ad}_x(v)\in V^\times$ and thus
\[ q(v) = q(\widetilde{\Ad}_{x^{-1}}\circ\widetilde{\Ad}_x(v)) = q(\widetilde{\Ad}_x(v)) \neq 0 \]
which is a contradiction.
\end{proof}

\subsection{Double coverings}
We define $SP(V,q) \defeq P(V,q)\cap \Cl^0(V,q)$.
\begin{theorem}
The homomorphisms
\[ \widetilde{\Ad}: P(V,q)\to \Ogroup(V,q) \qquad \text{and} \qquad \widetilde{\Ad}: SP(V,q)\to \SO(V,q) \]
are surjective. So we have the short exact sequence
\[ \begin{tikzcd}
1 \rar & \F^\times \rar & \Gamma(V,q) \rar{\widetilde{\Ad}} & \Ogroup(V,q) \rar & 1.
\end{tikzcd} \]
\end{theorem}

\begin{proposition}
The images $\widetilde{\Ad}(\Pin(V,q))$ and $\widetilde{\Ad}(\Spin(V,q))$ are both normal subgroups of $\Ogroup(V,q)$.
\end{proposition}
\begin{proof}
By a simple calculation we have $\widetilde{\Ad}_{f(v)} = f\circ\widetilde{\Ad}_v\circ f^{-1}$ for all $v \in V$ and $f\in\Ogroup(V,q)$.
\end{proof}

\begin{definition}
A field $\mathbb{F}$ of characteristic $\neq 2$ is called \udef{spin} if for all $a\in \mathbb{F}^\times$ at least one of the equations $t^2 = a$ and $t^2 = -a$ has a solution $t$ in $\mathbb{F}$.
\end{definition}
Thus $\mathbb{F}$ is spin if
\[ \mathbb{F}^\times = (\mathbb{F}^\times)^2 \cup (-(\mathbb{F}^\times)^2). \]

\begin{lemma}
The following fields are spin:
\begin{enumerate}
\item $\R$;
\item $\C$;
\item $\mathbb{F}_p$ with $p$ prime and $p \equiv 3\mod 4$.
\end{enumerate}
\end{lemma}

\begin{theorem}
Let $V$ be a finite-dimensional vector space over a spin field
$\mathbb{F}$, and suppose $q$ is a non-degenerate quadratic form on $V$. Then there are
short exact sequences
\[ \begin{tikzcd}
1 \rar & F \rar & \Spin(V,q) \rar{\widetilde{\Ad}} & \SO(V,q) \rar & 1
\end{tikzcd} \]
\[ \begin{tikzcd}
1 \rar & F \rar & \Pin(V,q) \rar{\widetilde{\Ad}} & \Ogroup(V,q) \rar & 1
\end{tikzcd} \]
where
\[ F = \begin{cases}
\Z_2 = \{1,-1\} & \sqrt{-1}\notin\mathbb{F} \\
\Z_4 = \{\pm 1, \pm\sqrt{-1}\}	& \text{otherwise.}
\end{cases} \]
This result holds for general fields if $\SO(V,q)$ and $\Ogroup(V,q)$ are replaced by appropriate normal subgroups of $\Ogroup(V,q)$.
\end{theorem}
\begin{proof}
Suppose $x = v_1\ldots v_r\in\Pin(V,q)$ is in the kernel of $\widetilde{\Ad}$. Then $x\in \mathbb{F}^\times$ and so
\[ x^2 = N(x) = N(v_1)\ldots N(v_r) = \pm 1 \]
by \ref{normHomomorphism}.
\end{proof}

\begin{proposition}
Let $\mathbb{F}$ be a spin field. Then either
\[ \Gamma(V,q)=P(v,q) \qquad \text{or} \qquad \Gamma(V,q)/P(V,q) \cong \Z_2. \]
\end{proposition}
\begin{proof}
TODO

We have a group homomorphism $\widetilde{\Ad}: \Gamma(V,q)\to \Ogroup(V,q)$ with $\ker(\widetilde{\Ad}) = \F^\times$. 
\end{proof}

\section{Real and complex Clifford algebras}

\begin{definition}
$q$-orthonormal basis.
\end{definition}

\begin{proposition}
There is an algebra isomorphism
\[ \Cl_{r,s} \cong \Cl^0_{r,s+1} \qquad \forall r,s\in\N. \]
In particular $\Cl_n \cong \Cl^0_{n+1}$.
\end{proposition}
\begin{proof}
Take a $q$-orthonormal basis $\{e_i\}_{i=1}^{r+s+1}$ of $\R^{r+s+1}$ and let $\R^{r+s}$ be spanned by the basis $\{e_i\}_{i=1}^{r+s}$. Then define a linear map $f:\R^{r+s}\to \Cl^0_{r,s+1}$ by
\[ f(e_i) = e_{r+s+1}e_i \qquad (i=1,\ldots,r+s). \]
We hope to apply the universal property \ref{CliffordUniversalProperty} to extend it to a map $\widetilde{f}: \Cl_{r,s}\to \Cl^0_{r,s+1}$. So we check $f(v)^2 = q(v) \vec{1}$. Indeed, let $v = \sum_{i=1}^{r+s}v_ie_i$, then
\[ f(v)^2 = \sum_{i,j=1}^{r+s}v_iv_je_{r+1}e_ie_{r+1}e_j = -\sum_{i,j=1}^{r+s}v_iv_je_{r+1}e_{r+1}e_ie_j = \sum_{i,j=1}^{r+s}v_iv_je_ie_j = q(v) \vec{1}. \]

It is easy to see $\widetilde{f}$ is bijective.
\end{proof}


\subsection{Geometric algebra}
\begin{definition}
A \udef{geometric algebra} is a Clifford algebra $\Cl(\R^n, \norm{\cdot}^2)$, where $\norm{\cdot}$ is the standard norm. We denote the $n$-dimensional geometric algebra by $\G^n$.
\end{definition}

In particular we have $u\cdot v = \inner{u,v}$.

\subsubsection{Calculations}

\begin{proposition}
Let $\vec{v}$ be a vector in $\G^n$ and $\vhat{e}$ a unit vector. Then
\[ \vec{v} = \inner{\vec{v},\vhat{e}}\vhat{e} + (\vec{v}\wedge \vhat{e})\;\vhat{e}. \]
The first term commutes with $\vhat{e}$ and the second term anti-commutes with $\vhat{e}$.
\end{proposition}
\begin{proof}
We calculate
\[ \vec{v} = \vec{v}\vhat{e}^2 = (\vec{v}\vhat{e})\vhat{e} = \Big(\inner{\vec{v},\vhat{e}} + \vec{v}\wedge \vhat{e}\Big)\vhat{e} = \inner{\vec{v},\vhat{e}}\vhat{e} + (\vec{v}\wedge \vhat{e})\vhat{e}. \]
The first term obviously commutes with $\vhat{e}$. The anti-commutivity of the second term follows from \ref{CliffordIdentities}.
\end{proof}

\subsubsection{The pseudoscalar}
TODO orientations and stuff

\subsubsection{Rotors and rotations}
\url{https://marctenbosch.com/quaternions/#h_16}
-> why composition of rotors works.

\begin{proposition}
$e^{\theta} \leftrightarrow \vhat{a}\vhat{b}$
\end{proposition}

\begin{lemma}
Let $\vhat{u}, \vhat{v}$ be unit vectors in a geometric algebra $\G^n$ and let $\{\vec{e}_1, \vec{e}_2\}$ be an orthonormal basis of $\Span\{\vhat{u},\vhat{v}\}$. Then
\[ \vhat{v}\vhat{u} = \cos\theta + \sin\theta \vhat{e}_1\vhat{e}_2 = e^{\theta \vhat{e}_1\vhat{e}_2}, \]
for some $\theta \in [ 0,2\pi ]$.
\end{lemma}
\begin{proof}
TODO
\end{proof}


In the plane $\Span\{\vhat{e}_1,\vhat{e}_2\}$, a rotation that maps $\vhat{u}$ to $\vhat{v}$ should act as $\lambda_{\vhat{v}\vhat{u}}$. On vectors perpendicular to the plane it should act as the identity.


\begin{proposition}
Let $\vhat{u}, \vhat{v}$ be unit vectors in a geometric algebra $\G^n$ and let $\{\vec{e}_1, \vec{e}_2\}$ be an orthonormal basis of $\Span\{\vhat{u},\vhat{v}\}$. Set $\vec{i} \defeq \vec{e}_1\wedge \vec{e}_2 = \vec{e}_1\vec{e}_2$. Then the rotation $R_{\vhat{v}\from\vhat{u}}$ that maps $\vhat{u}$ to $\vhat{v}$ can be written as
\[ R_{\vhat{v}\from\vhat{u}}: V \to V: \vec{w} \mapsto e^{\vec{i}\theta/2}\vec{w}e^{-\vec{i}\theta/2}. \]
Thus $R_{\vhat{v}\from\vhat{u}} = \Ad_{e^{\vec{i}\theta/2}} = \widetilde{\Ad}_{e^{\vec{i}\theta/2}}$.
\end{proposition}
\begin{proof}
We write $\vec{w} = \vec{w}_\parallel + \vec{w}_\perp \in \Span\{\vhat{u},\vhat{v}\} \oplus \Span\{\vhat{u},\vhat{v}\}^\perp$.
Then
\begin{align*}
R_{\vhat{v}\from\vhat{u}}(\vec{w}) &= e^{\vec{i}\theta}\vec{w}_\parallel + \vec{w}_\perp \\
&= e^{\vec{i}\theta/2}e^{\vec{i}\theta/2}\vec{w}_\parallel + e^{\vec{i}\theta/2}e^{-\vec{i}\theta/2}\vec{w}_\perp \\
&= e^{\vec{i}\theta/2}\vec{w}_\parallel e^{-\vec{i}\theta/2} + e^{\vec{i}\theta/2}\vec{w}_\perp e^{-\vec{i}\theta/2} \\
&= e^{\vec{i}\theta/2}(\vec{w}_\parallel + \vec{w}_\perp) e^{-\vec{i}\theta/2} = e^{\vec{i}\theta/2}\vec{w} e^{-\vec{i}\theta/2}.
\end{align*}
We have used that elements of $\Span\{\vhat{u},\vhat{v}\}$ commute with $\vec{i}$ and elements of $\Span\{\vhat{u},\vhat{v}\}^\perp$ anticommute with $\vec{i}$.
\end{proof}
\begin{corollary}
Let $\vhat{n}$ be a unit vector in $\G^3$ and $\vec{I}$ a unit pseudoscalar. Then a rotation of $\theta$ radians around $\vhat{n}$ is given by
\[ R_{\vhat{n},\theta}: V \to V: \vec{w} \mapsto e^{\vec{n}\vec{I}\theta/2}\vec{w}e^{-\vec{n}\vec{I}\theta/2}. \]
Thus $R_{\vhat{n},\theta} = \Ad_{e^{\vec{n}\vec{I}\theta/2}} = \widetilde{\Ad}_{e^{\vec{n}\vec{I}\theta/2}}$. The direction is determined by the orientation of $\vec{I}$.
\end{corollary}
\begin{proof}
TODO
\end{proof}

\begin{definition}
We call an element of $\G^n$ of the form $e^{\vec{i}\theta/2}$ a \udef{rotor}.
\end{definition}

TODO $\Spin^+$!

\section{Representations}

\begin{definition}
Let $K \subseteq k$ be fields, $V$ a vector space over $k$ and $q$ a quadratic form on $V$. Then a \udef{$K$-representation} of the Clifford algebra $\Cl(V,q)$ is a $k$-algebra homomorphism
\[ \rho: \Cl(V,q) \to \Hom_K(W,W) \]
where $W$ is a finite dimensional vector space over $K$. The space $W$ is then a \udef{$\Cl(V,q)$-module} over $K$.
\end{definition}

Usually we will take the field $K$ to be $\R,\C,\mathbb{H}$.

\begin{lemma}
\begin{enumerate}
\item A complex representation of $\Cl_{r,s}$ automatically extends to a representation of
\[ \Cl_{r,s}\otimes_\R \C \cong \cCl_{r+s}. \]
\item A quaternionic representation of $\Cl_{r,s}$ is automatically complex.
\end{enumerate}
\end{lemma}

\section{Lie algebra structures}

\begin{proposition}
The Lie subalgebra of $(\Cl_n, [\cdot,\cdot])$ corresponding to the subgroup $\Spin_n\subset \Cl_n^\times$ is
\[ \mathfrak{spin}_n = {\textstyle \bigwedge^2}\R^n. \]
In particular, $\bigwedge^2\R^n$ is closed under the bracket operation.
\end{proposition}
\begin{proof}
We are looking for tangent vectors to the submanifold $\Spin_n$ at $\vec{1}$. Fix an orthonormal basis $e_1,\ldots, e_n$ of $\R^n$ and consider the curve
\[ \gamma(t) = (e_i\cos t+ e_j\sin t)(-e_i\cos t+ e_j\sin t) = (\cos^2 t - \sin^2 t)+2e_ie_j\sin t\cos t = \cos(2t)\sin(2t)e_ie_j. \]
This curve lies in $\Spin_n$, satisfies $\gamma(0) = \vec{1}$ and its tangent vector at $\gamma(0)$ is $2e_ie_j$. Hence $\mathfrak{spin}_n$ contains $\Span_\R\{e_ie_j\} = \bigwedge^2\R^n$. Since $\dim_\R(\mathfrak{spin}_n)$
\end{proof}






\section{Geometry and geometric algebra}
\url{http://www.faculty.luther.edu/~macdonal/GAConstruct.pdf}
\subsection{Definitions}

\begin{definition}
A \udef{geometric algebra} $\mathfrak{G}$ is a real unital associative algebra of the form
\[ \mathfrak{G} = \bigoplus_{r\in \N}\mathfrak{G}_r \]
such that
\begin{align*}
\mathfrak{G}_0 &= \Span\{\vec{1}\} \\
\mathfrak{G}_r &= \Span\setbuilder{\vec{a}_1\vec{a}_2\ldots\vec{a}_r}{\vec{a}_1,\ldots,\vec{a}_r \in \mathfrak{G}_1, \; \forall i,j\leq r: \vec{a}_i \vec{a}_j = -\vec{a}_j \vec{a}_i} & \text{for all $r>1$}.
\end{align*}
We also assume that the multiplication satisfies
\[ \forall \vec{a} \in \mathfrak{G}_1: \quad \vec{a}^2 = \vec{a}\vec{a} = \lambda \vec{1} \in \mathfrak{G}_0 \qquad \text{for some $\lambda \in \R^{> 0}$} \]
and that for each element $a$ of $\mathfrak{G}_r\setminus\{0\}$ there exists a vector $\vec{a}\in\mathfrak{G}_1$ such that $\vec{a}a \in \mathfrak{G}_{r+1}\setminus\{0\}$.

We then call
\begin{itemize}
\item $\sqrt{\lambda}$ the \udef{magnitude of $\vec{a}$}, denoted $|\vec{a}|$;
\item the projection $\mathfrak{G} \to \mathfrak{G}_r$ the \udef{grade operator}, denoted $\grade{\cdot}_r$;
\item the multiplication of $\mathfrak{G}$ the \udef{geometric product} on $\mathfrak{G}$;
\item elements of $\mathfrak{G}$ \udef{multivectors};
\item elements of $\mathfrak{G}_r$ \udef{$r$-vectors} or \udef{homogenous multivectors}; in particular $0$-vectors are called \udef{scalars}, $1$-vectors \udef{vectors}, $2$-vectors \udef{bivectors} \ldots
\item $r$-vectors of the form $\vec{a}_1\vec{a}_2\ldots\vec{a}_r$ where $\vec{a}_1,\ldots,\vec{a}_r \in \mathfrak{G}_1$ anti-commute are called \udef{simple $r$-vectors} or \udef{$r$-blades}.
\end{itemize}
We use lowercase letters $a,b,c \ldots$ to denote multivectors, Greek letters $\mu, \nu, \lambda \ldots$ for scalars and bold letters $\vec{u}, \vec{v}, \vec{w} \ldots$ for vectors. Often we will use subscripts to denote the grade of a multivector, e.g\ $a_r$ is an $r$-vector. Capital letters with subscript, e.g\ $A_r$, will be used to denote $r$-blades.
\end{definition}

So for any $a\in \mathfrak{G}$, we can write
\[ a = \grade{a}_0 + \grade{a}_1 + \ldots + \grade{a}_n  = \sum_{r=0}^n \grade{a}_i  \] for some $n\in\N$.

We make the convention that negative grades are always zero.

Because of the assumption that no $\mathfrak{G}_r$ is trivial, we need $\mathfrak{G}_1$ to be infinite-dimensional.


\begin{definition}
Let $\mathfrak{G}$ be a geometric algebra. We define the \udef{reverse} operation $\dagger$ on $\mathfrak{G}$ as the unique linear operation such that
\begin{align*}
\vec{1}^\dagger &= \vec{1} \\
\vec{u}^\dagger &= \vec{u} \qquad \vec{u}\in\mathfrak{G}_1 \\
(\vec{v}_1 \vec{v}_2 \ldots \vec{v}_r)^\dagger &= \vec{v}_r \ldots \vec{v}_2 \vec{v}_1 \qquad \vec{v}_1, \ldots, \vec{v}_r \in \mathfrak{G}_1.
\end{align*}
\end{definition}
Note that we have specified $\dagger$ on all basis elements of $\mathfrak{G}$, so it is well-defined and uniquely determined, cfr. \ref{linearMaps}.

\begin{lemma}
Let $a,b \in \mathfrak{G}$. Then
\begin{enumerate}
\item $(a^\dagger)^\dagger = a$;
\item $(ab)^\dagger = b^\dagger a^\dagger$;
\item $\grade{a^\dagger}_r = \grade{a}_r^\dagger = (-1)^{r(r-1)/2}\grade{a}_r$;
\item $\grade{a}_r = (-1)^{r(r-1)/2}\grade{a}_r^\dagger$;
\item $\grade{a_rb_s}_t = (-1)^{\frac{1}{2}(r(r-1) + s(s-1) + t(t-1))}\grade{b_sa_r}_t$.
\end{enumerate}
\end{lemma}
Notice we are only interested in the exponent of $(-1)$ modulo $2$.

\begin{definition}
Let $\mathfrak{G}$ be a geometric algebra. We define the \udef{inner product} $\cdot$ on homogeneous multivectors by
\[ a_r\cdot b_s = \begin{cases}
0 & \text{$r=0$ or $s=0$} \\
\grade{a_rb_s}_{|r-s|} & \text{else}.
\end{cases}  \]
The inner product is bilinear on homogeneous multivectors and can thus be extended linearly to arbitrary multivectors.
\end{definition}
So for arbitrary multivectors we have
\[ a\cdot b = \sum_r\sum_s \grade{a}_r\cdot \grade{b}_s. \]

\begin{definition}
Let $\mathfrak{G}$ be a geometric algebra. We define the \udef{outer product} $\wedge$ on homogeneous multivectors by
\[ a_r\wedge b_s = \grade{a_rb_s}_{r+s} \]
The outer product is bilinear on homogeneous multivectors and can thus be extended linearly to arbitrary multivectors.
\end{definition}
So for arbitrary multivectors we have
\[ a\wedge b = \sum_r\sum_s \grade{a}_r\wedge \grade{b}_s. \]
For scalars $\lambda \in \mathfrak{G}^0$, we have
\[ a \wedge \lambda = \lambda\wedge a = \lambda a \qquad a \in \mathfrak{G}. \]
We have explicitly excluded this in the inner product.

\begin{lemma}
If $a = \vec{v}_1 \vec{v}_2\ldots \vec{v}_r$, then $a\in \bigoplus_{i\leq r}\mathfrak{G}_i$.

If $a$ is an $r$-blade and $\beta$ an orthogonal basis for $\mathfrak{G}_1$, then there exist $\vec{e}_1,\ldots, \vec{e}_r \in \beta$ such that
\[ a = \lambda \vec{e}_1 \ldots \vec{e}_r \]


 be an $r$-blade, then
\[\vec{v}_1 \vec{v}_2\ldots \vec{v}_r = \vec{v}_1 \wedge \vec{v}_2 \wedge\ldots\wedge \vec{v}_r.\]
\end{lemma}

\begin{note}
We introduce an order of operations (from highest priority to lowest):
\begin{enumerate}
\item outer product;
\item inner product;
\item geometric product.
\end{enumerate}
\end{note}

\begin{lemma}
Let $a_r,b_s$ be homogeneous vectors in $\mathfrak{G}$. Then
\begin{enumerate}
\item $a_r\cdot b_s = (-1)^{s(r-1)}b_s\cdot a_r$ for $r\geq s$;
\item $a_r \wedge b_s = (-1)^{rs}b_s \wedge a_r$.
\end{enumerate}
\end{lemma}
\begin{proof}
(1) We calculate
\[ a_r\cdot b_s = \grade{a_rb_s}_{|r-s|} = (-1)^{\frac{1}{2}(r(r-1) + s(s-1) + |r-s|(|r-s|-1))}\grade{a_rb_s}_{|r-s|}. \]
We can simplify the exponent, assuming $r\geq s$, to
\[ r^2 + s^2 -r -sr \equiv r+s+r+sr \equiv s+sr \equiv sr-s \mod 2.\]
(2) We calculate
\[ a_r \wedge b_s = \grade{a_rb_s}_{r+s} = (-1)^{\frac{1}{2}(r(r-1) + s(s-1) + (r+s)((r+s)-1))}\grade{a_rb_s}_{r+s}. \]
We can simplify the exponent to
\[ r^2 -r+s^2 -s + rs \equiv r-r+s-s+rs \equiv rs \mod 2. \]
\end{proof}

By the fact that the definitions of inner and outer product make sense it is obvious that the geometric product does not preserve grade, or even homogeneity. It does, however, preserve a $\Z_2$-grading: we can split
\[ \mathfrak{G} = \mathfrak{G}_\text{even} \oplus \mathfrak{G}_\text{odd} \qquad \text{where}\quad \begin{cases}
\mathfrak{G}_\text{even} \defeq \bigoplus_{r\in \N}\mathfrak{G}_{2r} \\
\mathfrak{G}_\text{odd}\; \defeq \bigoplus_{r\in \N}\mathfrak{G}_{2r+1}.
\end{cases} \]
We have the linear projection operators $\grade{\cdot}_+:\mathfrak{G}\to \mathfrak{G}_\text{even}$ and $\grade{\cdot}_-:\mathfrak{G}\to \mathfrak{G}_\text{odd}$. We also write $\mathfrak{G}_+$ instead of $\mathfrak{G}_\text{even}$ and $\mathfrak{G}_-$ instead of $\mathfrak{G}_\text{odd}$.
\begin{proposition}
Let $\mathfrak{G} = \mathfrak{G}_\text{+} \oplus \mathfrak{G}_\text{-}$ be a geometric algebra and let $p,q\in\{+,-\}\cong \Z_2$. Then
\[ \mathfrak{G}_p\mathfrak{G}_q \subset \mathfrak{G}_{pq}. \]
\end{proposition}
\begin{proof}
The grading operators $\grade{\cdot}_\pm$ are linear maps and thus determined by their action on basis elements. Thus it is enough to show that
\[ \grade{A_rB_s}_+ =  \begin{cases}
A_rB_s & (r+s \equiv 0 \mod 2) \\
0 & (r+s \equiv 1 \mod 2)
\end{cases} \qquad \grade{A_rB_s}_- =  \begin{cases}
0 & (r+s \equiv 0 \mod 2) \\
A_rB_s & (r+s \equiv 1 \mod 2)
\end{cases} \]
for any $r$-blade $A_r$ and $s$-blade $B_s$. In fact by associativity of the geometric product, it is enough to show
\[ \vec{v}A_r \]
\end{proof}

\begin{lemma}
Let $\vec{u}, \vec{v} \in \mathfrak{G}_1$. Then
\[ \vec{u}\cdot \vec{v} = \grade{\vec{u}\vec{v}}_0 = \frac{1}{2}(\vec{u}\vec{v} + \vec{v}\vec{u}). \]
\end{lemma}
\begin{proof}
We start from $(\vec{u}+\vec{v})^2 = \vec{u}^2 + \vec{u}\vec{v} + \vec{v}\vec{u} + \vec{v}^2$ and rearrange to get
\[ \vec{u}\vec{v} + \vec{v}\vec{u} = (\vec{u}+\vec{v})^2 - \vec{u}^2 - \vec{v}^2 = |\vec{u}+\vec{v}|^2 - |\vec{u}|^2 - |\vec{v}|^2  \]
which is scalar. Since $\grade{\vec{u}\vec{v}}_0 = \grade{\vec{v}\vec{u}}_0$, we have
\[ \frac{1}{2}(\vec{u}\vec{v} + \vec{v}\vec{u}) = \frac{1}{2}\grade{\vec{u}\vec{v} + \vec{v}\vec{u}}_0 = \grade{\vec{u}\vec{v}}_0 = \vec{u}\cdot \vec{v}. \]
\end{proof}
\begin{corollary}
The geometric inner product restricted to $\mathfrak{G}_1$ is bilinear, symmetric and positive definite. It is thus an inner product as previously defined.
\end{corollary}
The associated definitions are thus also applicable here. In particular two vectors $\vec{u},\vec{v}$ are called \udef{orthogonal} if $\vec{u}\cdot \vec{v} = 0$. By the lemma this is the case when $\vec{u}\vec{v} = - \vec{v}\vec{u}$. This means that the $r$ vectors making up $r$-blades are linearly independent, \ref{orthogonalLinearlyIndependent}.




\begin{lemma}
For any algebra satisfying the other axioms, the direct sum $\mathfrak{G} = \bigoplus_{r\in\N}\mathfrak{G}_r$ is well-defined.
\end{lemma}
\begin{proof}
We need to show that for all $r>s\in \N$, we have $\mathfrak{G}_r\cap\mathfrak{G}_s = \{0\}$. Assume, towards a contradiction, that here exist $a_r,b_s$ such that $a_r = b_s$. Then both $a_r$ and $b_s$ can be written as sums of blades. Now let $D$ be the set of all vectors featured in a blade in this sum. By Gram-Schmidt, we can find an orthogonal basis for $D$ and rewrite $a_r$ and $b_s$ in this basis. As $r>s$, we can find elements of this orthogonal basis to multiply $a_r$ with such that it becomes zero, but $b_s$ remains non-zero (unless it already was zero). TODO: improve proof.
\end{proof}


\begin{lemma}
Let $\vec{u}, \vec{v}_1,\ldots, \vec{v}_n$ be vectors in $\mathfrak{G}_1$. Then
\[ \vec{u}\cdot (\vec{v}_1 \vec{v}_2 \ldots \vec{v}_n) = \sum_{i=1}^n (-1)^{k+1}(\vec{u}\cdot \vec{v}_i)\vec{v}_1\ldots\breve{\vec{v}}_i\ldots \vec{v}_n, \]
where the breve indicates the vector under it is omitted from the product.
\end{lemma}
\begin{proof}

\end{proof}

\begin{lemma}
\[ \vec{u}a_r = \vec{u}\cdot a_r + \vec{u}\wedge a_r = \grade{\vec{u}a_r}_{r-1} + \grade{\vec{u}a_r}_{r+1}. \]
\end{lemma}

\begin{proposition}
Let $\vec{v}\in\mathfrak{G}_1$ and $a_r\in\mathfrak{G}_r$. Then

\end{proposition}





\begin{lemma}
The outer product is associative:
\[ a\wedge(b\wedge c) = (a\wedge b)\wedge c \]
The inner product is not associative, but homogeneous multivectors obey
\begin{align*}
a_r\cdot(b_s \cdot c_t) &= (a_r\wedge b_s)\cdot c_t & &\text{for $r+s\leq t$ and $r,s>0$} \\
a_r\cdot(b_s \cdot c_t) &= (a_r\cdot b_s)\cdot c_t & &\text{for $r+t\leq s$}
\end{align*}
\end{lemma}

\begin{lemma}
\[ \vec{u}\wedge a\wedge \vec{v}\wedge b = -\vec{v}\wedge a\wedge \vec{u}\wedge b  \]
\end{lemma}
$\vec{v}\wedge a \wedge \vec{v} \wedge b = 0$

\subsection{Affine spaces}
\subsection{Projections on 1D spaces}
\[ \sin(\theta) = \norm{a_\perp}/ \norm{a} \qquad \cos(\theta) = \norm{a_\parallel}/\norm{a}. \]
\subsection{The geometric product}

\subsection{Hodge duality}
\subsection{Cross product and triple product}
Cross product not associative

Triple product nice way to find normal vectors with specific orientation.

\chapter{Coordinates and matrices}

\url{file:///C:/Users/user/Downloads/2013%20Matrix%20Computations%204th(1).pdf}
\url{file:///C:/Users/user/Downloads/(Cambridge%20mathematical%20textbooks)%20Garcia,%20Stephan%20Ramon_%20Horn,%20Roger%20A.%20-%20A%20Second%20Course%20in%20Linear%20Algebra-Cambridge%20University%20Press%20(2017)(1).pdf}

TODO Haynsworth inertia additivity formula

\section{Coordinates}
In this chapter we will purely be interested in finite-dimensional spaces. From proposition \ref{isomorphicDimension} we know that for any $n$-dimensional vector space $V$ over a field $\mathbb{F}$, $V\cong \mathbb{F}^n$. If we choose a basis $\beta$, we can explicitly give an isomorphism.
\begin{definition}
Let $V$ be an $n$-dimensional vector space with basis $\beta = \{e_1,\ldots, e_n\}$. Because $\beta$ is a basis, we can uniquely write every $v\in V$ as $a_1e_1+\ldots +a_ne_n$. Then we define the \udef{coordinate map w.r.t. $\beta$} as
\[ \co_\beta: V \to \mathbb{F}^n: v \mapsto \co_\beta(v) = \co_\beta(a_1e_1+\ldots +a_ne_n) = \begin{pmatrix}
a_1 \\ \vdots \\ a_n
\end{pmatrix}. \]
The vector $\begin{pmatrix}
a_1 \\ \vdots \\ a_n
\end{pmatrix}$ is called a \udef{coordinate vector}.

The vector $\co_\beta(v)$ is also denoted $[v]_\beta$.
\end{definition}
This coordinate map is indeed an isomorphism.

We conventionally write coordinate vectors as column vectors in $\F^n$. We will represent such vectors in bold type: $\vec{v}\in\F^n$.

\begin{lemma}
Let $\mathcal{E}$ be the standard basis of $\F^n$. Then
\[ \co_\mathcal{E} = \id = \co_\mathcal{E}^{-1}. \]
\end{lemma}

\section{Matrices}
\begin{definition}
A matrix is a rectangular grid of numbers. If it has $m$ rows and $n$ columns, we call it a \udef{$(m\times n)$-matrix}.
If the numbers are elements of the field $\mathbb{F}$, we denote the set of $(m\times n)$-matrices as $\mathbb{F}^{m\times n}$.

If $m=n$,we call the matrix a \udef{square matrix}.
\end{definition}
\begin{example}
An example of a ($2\times 4$)-matrix:
\[ \begin{bmatrix}
1 &2 &3\\4 &5&6
\end{bmatrix} = \begin{pmatrix}
1 &2 &3\\4 &5&6
\end{pmatrix} \]
Sometimes square brackets are used, sometimes parentheses. It's just a matter of style.
\end{example}
Matrices are usually denoted using capital letters.

\begin{definition}
Let $n,m\in\N_0$.
\begin{itemize}
\item The \udef{zero matrix} of dimension $n\times m$ is the $(n\times m)-$matrix
\[ \mathbb{0}^{n\times m} = \begin{pmatrix}
0 & \hdots & 0 \\
\vdots & \ddots & \\
0 & \hdots & 0
\end{pmatrix}. \]
\item The \udef{matrix of ones} or \udef{all-ones matrix} of dimension $n\times m$ is the $(n\times m)-$matrix
\[ \mathbb{J}^{n\times m} = \begin{pmatrix}
1 & \hdots & 1 \\
\vdots & \ddots & \\
1 & \hdots & 1
\end{pmatrix}. \]
\item The \udef{identity matrix} of dimension $n$ is the $(n\times n)$-matrix
\[ \mathbb{1}_n = \begin{pmatrix}
1 & 0 & 0 & \hdots & 0\\
0 & 1 & 0 & \hdots & 0\\
0 & 0 & 1 & \hdots & 0\\
\vdots & \vdots & \vdots & \ddots & \vdots \\
0 & 0 & 0 & \hdots & 1
\end{pmatrix} \]
The components of $\mathbb{1}_n$ are given by
\[ [\mathbb{1}_n]_{ij} = \delta_{ij}. \]
\end{itemize}
If $m=n$, we abbreviate these matrices as $\mathbb{0}_n$ and $\mathbb{J}_n$.
\end{definition}

\begin{lemma}
The $(m\times n)$-matrices in $\F^{m\times n}$ naturally form a vector space with point-wise addition and scalar multiplication. 
\end{lemma}

\begin{definition}
We call matrices $A,B$ \udef{conformal} for a certain operation if the operation is defined on these matrices. So $A,B$ are conformal for addition if they have the same dimensions. 
\end{definition}

\subsection{Components of matrices}
The element on the $i^\text{th}$ row and $j^\text{th}$ column of a matrix $A$ is denoted $[A]_{i,j}$ or $a_{i,j}$. These numbers are known as the \udef{components} of the matrix.

Vectors in $\F^n$ can be seen as matrices by writing them as column vectors. In this way we identify $\F^n$ with $\F^{n\times 1}$.

Let $\vec{v}\in \F^n$. The components of $\vec{v}$ are of the form $[\vec{v}]_{i,1}$. We abbreviate this to $[\vec{v}]_i$.

\begin{lemma}
The functions
\[ [-]_{ij}: A\mapsto [A]_{ij} \qquad \text{and} \qquad [-]_i: \vec{v}\mapsto [\vec{v}]_i  \]
are linear.
\end{lemma}

Conversely, consider a set of numbers $a_{i,j}$ where $i\in (1:m), j\in (1:n)$. Then by $[a_{i,j}]$ we mean the matrix consisting of those numbers.

We can also consider components after applying a function. For some linear map $f$, we often write
\[ f_i \defeq [-]_i\circ f. \]

\begin{definition}
Let $A$ be an $(m\times n)$-matrix.
\begin{itemize}
\item If $1\leq j\leq m$, then $[A]_{j,-}$ denotes the $(1\times n)$-matrix consisting of row $j$ of $A$.
\item If $1\leq k\leq n$, then $[A]_{-,k}$ denotes the $(m\times 1)$-matrix consisting of column $k$ of $A$.
\end{itemize}
\end{definition}

\begin{lemma}
Let $A,B\in \F^{m\times n}$ and $\lambda\in \F$, then
\begin{itemize}
\item $[A+B]_{ij} = [A]_{ij} + [B]_{ij}$;
\item $[\lambda A]_{ij} = \lambda[A]_{ij}$.
\end{itemize}
\end{lemma}

\begin{definition}
Let $A\in\F^{m\times n}$ be a matrix. A component $[A]_{i,j}$ is
\begin{itemize}
\item on the \udef{diagonal} if $i=j$;
\item \udef{off-diagonal} $i\neq j$;
\item on the $k^\text{th}$ \udef{superdiagonal} if $j = i+k$;
\item on the $k^\text{th}$ \udef{subdiagonal} if $j = i-k$.
\end{itemize}
\end{definition}

\subsubsection{Submatrices}
\begin{definition}
Let $A\in\F^{m\times n}$ be a matrix and $I\subseteq 0:m$ and $J\subseteq 0:n$ sets, then $[A]_{I,J}$ is the matrix consisting only of those entries whose row number is in $I$ and whose column number is in $J$. A matrix of this form is called a \udef{submatrix}.

In this context, we abbreviate
\begin{itemize}
\item $I = 0:m$ and $J = 0:n$ by $-$;
\item $I=\{i\}$ by $i$.
\end{itemize}
\end{definition}

\begin{example}
Let
\[ A = \begin{pmatrix}
1 & 2 & 3 & 4 \\ 5 & 6 & 7 & 8 \\ 9 & 10 & 11 & 12
\end{pmatrix}, \]
then
\[ [A]_{0:2,0:3} = \begin{pmatrix}
1 & 2 & 3 \\ 5 & 6 & 7
\end{pmatrix}, \qquad [A]_{1,-} = \begin{pmatrix}
5 & 6 & 7 & 8
\end{pmatrix} \qquad\text{and}\qquad [A]_{-, 2} =  \begin{pmatrix}
3 \\ 7 \\ 11
\end{pmatrix}. \]
\end{example}

\begin{lemma} \label{standardBasisFromIdentityMatrix}
Let $\{\vec{e}_i\}_{i=0}^n$ be the standard basis of $\F^n$. Then
\[ \vec{e}_i = [\mathbb{1}_n]_{-, i}. \]
\end{lemma}

\begin{lemma}
Let $A\in \F^{m\times n}$. Then $A = [A]_{-,-}$.
\end{lemma}

\begin{lemma} \label{componentsSubmatrixFromEnumeration}
Let $A\in\F^{m\times n}$ be a matrix and $I\subseteq 0:m$ and $J\subseteq 0:n$ sets.
Then $\big[[A]_{I,J}\big]_{r,s} = [A]_{\pi_I(r), \pi_J(s)}$.
\end{lemma}

TODO $\pi_I$ is the unique monotonic enumeration of $I$.

\subsubsection{Types of matrices}
\begin{definition}
Let $A\in\F^{n\times n}$. We say
\begin{itemize}
\item $A$ is \udef{upper triangular} if $i>j \implies [A]_{i,j} = 0$;
\item $A$ is \udef{strictly upper triangular} if $i\geq j \implies [A]_{i,j} = 0$;
\item $A$ is \udef{lower triangular} if $i<j \implies [A]_{i,j} = 0$;
\item $A$ is \udef{strictly lower triangular} if $i\leq j \implies [A]_{i,j} = 0$;
\item $A$ is \udef{triangular} if it is upper or lower triangular.
\end{itemize}
We say
\begin{itemize}
\item $A$ is \udef{diagonal} if $i\neq j \implies [A]_{i,j} = 0$; in this case we write $A = \diag([A]_{11},\ldots, [A]_{nn})$;
\item $A$ is \udef{tridiagonal} if $|i\neq j| \geq 2 \implies [A]_{i,j} = 0$;
\item $A$ is \udef{bidiagonal} if it is tridiagonal and triangular.
\end{itemize}
We say
\begin{itemize}
\item $A$ is a \udef{permutation matrix} if exactly one entry in each row and in each column is 1; all other entries are 0.
\end{itemize}
\end{definition}

\subsection{Matrix multiplication}
Assume we have $n$ vectors $v_1,\ldots, v_n$ in $\F^m$. We may be interested in linear combinations of these vectors, say $a_1v_1+\ldots + a_nv_n$. We can collect the coefficients $a_i$ in a column vector in $\F^n$. The vectors $v_i$ can be written as columns and placed in a matrix.

Consider the action that pairs such a matrix of column vectors with the element in its column space determined by a column matrix. This action is called \udef{matrix multiplication} and is denoted by juxtaposing the matrix and the vector (sometimes separated by a dot).

\begin{example}
Let $v_1 = (1,3,4)$ and $v_2 = (2,5,6)$ be vectors in $\R^3$. These can be placed as columns in a matrix:
\[ \begin{pmatrix}
1 & 2 \\ 3 & 5 \\ 4 & 6
\end{pmatrix} \]
Consider the linear combination $2v_1 + v_2$, we can write this as the matrix multiplication
\[ 2v_1 + v_2 = \begin{pmatrix}
1 & 2 \\ 3 & 5 \\ 4 & 6
\end{pmatrix}\begin{pmatrix}
2 \\ 1
\end{pmatrix} = \begin{pmatrix}
2\cdot 1 + 1\cdot 2 \\
2\cdot 3 + 1\cdot 5 \\
2\cdot 4 + 1\cdot 6
\end{pmatrix} = \begin{pmatrix}
4 \\ 11 \\ 14
\end{pmatrix} \]
\end{example}
\begin{example}
A very important case (and one we will explore in more detail later) is given by systems of linear equations. We might have the following equations:
\[ \begin{cases}
2x + y -z = 3 \\
-x + y +3z = 2 \\
x+y = -2
\end{cases} \]
This can be rewritten as
\[ x\begin{pmatrix}
2 \\-1 \\ 1
\end{pmatrix} + y \begin{pmatrix}
1 \\1 \\ 1
\end{pmatrix} + z\begin{pmatrix}
-1 \\ 3 \\ 0
\end{pmatrix} = \begin{pmatrix}
3 \\2 \\ -2
\end{pmatrix} \]
where each row is an equation. In this case the coefficients are the unknowns $x,y,z$. So using the notation of matrix multiplication, the equations become
\[ \begin{pmatrix}
2 & 1 & -1 \\
-1 & 1 & 3 \\
1 & 1 & 0
\end{pmatrix}\begin{pmatrix}
x \\ y \\ z
\end{pmatrix} = \begin{pmatrix}
3 \\ 2 \\ -2
\end{pmatrix}. \]
\end{example}

We can view matrix multiplication as a function
\[ \F^{m\times n}\times \F^n \to \F^m: (A,\vec{v}) \mapsto \begin{pmatrix}
\sum_{i=1}^n [A]_{1,i}[\vec{v}]_i \\ \vdots \\ \sum_{i=1}^n [A]_{m,i}[\vec{v}]_i
\end{pmatrix}. \]


Let $B$ be an $(m\times n)$-matrix and $\vec{v}$ a vector in $\F^n$. Then the matrix multiplication $B\cdot \vec{v}$ gives a vector in $\F^m$. This can be used as the input for another matrix multiplication, if multiplied by a $(k\times m)$ matrix $A$. So the expression $A(B\vec{v})$makes sense.

Now we would like to define matrix multiplication between the matrices $B,A$ by the condition that
\[ (A\cdot B)\vec{v} = A(B\vec{v}). \]
In other words we are asserting the associativity of the matrix multiplication.

Consider the component equations: $[A(B\vec{v})]_i = [(B\cdot A)\vec{v}]_i$. We can then calculate:

\begin{align*}
[A(B\vec{v})]_i &= \sum_{j=1}^m [A]_{i,j}[B\vec{v}]_j = \sum_{j=1}^m [A]_{i,j}(\sum_{k=1}^n [B]_{j,k}[\vec{v}]_k) = \sum_{j=1}^m\sum_{k=1}^n [A]_{i,j}[B]_{j,k}[\vec{v}]_k \\ &= \sum_{k=1}^n\left(\sum_{j=1}^m[A]_{i,j}[B]_{j,k}\right)[\vec{v}]_k \eqdef  [(A\cdot B)\vec{v}]_i
\end{align*}
The last equation can only be satisfied for all $[v]_i$ if the matrix multiplication is defined such that
\[ [A\cdot B]_{i,k} \defeq \sum_{j=1}^m[A]_{i,j}[B]_{j,k}. \]

Of course our construction only works if the dimensions of $A,B$ are such that $A(B\vec{v})$ is well-defined.

\begin{definition}
Let $A,B$ be matrices.
\begin{itemize}
\item The matrices $A,B$ are conformal for multiplication if $A\in \F^{k\times m}$ and $B\in\F^{m\times n}$ for some $k,m,n\in\N_0$.
\item If $A$ and $B$ are conformal, we define the product $AB$ by
\[ [AB]_{i,k} = \sum_{j=1}^m[A]_{i,j}[B]_{j,k}. \]
\end{itemize}
Thus matrix multiplication can be thought of as a map $\F^{k\times m}\times \F^{m\times n} \to \F^{k\times n}$
\end{definition}
The compatibility requirement can be abbreviated by
\[ [k\times m] \cdot [m\times n] = [k\times n]. \]

Notice that the matrix multiplication $\F^{m\times n}\times \F^n \to \F^m$ we originally defined is a special case of this more general matrix multiplication if we identify $\F^n$ with $\F^{n\times 1}$ and $\F^m$ with $\F^{m\times 1}$ (that is, we view $\F^n,\F^m$ as column vectors). In this case we have the multiplication
\[ [m\times n] \cdot [n\times 1] = [m\times 1]. \]

\begin{lemma} \label{linearityMatrixMultiplication}
The matrix multiplication map $\F^{k\times m}\times \F^{m\times n} \to \F^{k\times n}$ is linear in both arguments.
\end{lemma}
\begin{proof}
For linearity in the first argument, assume $A_1,A_2\in\F^{k\times m}$ and $\lambda\in \F$. Then
\[ [(\lambda A_1+ A_2)B]_{ij} = \sum_{j=1}^m[\lambda A_1+ A_2]_{i,j}[B]_{j,k} = \lambda \left(\sum_{j=1}^m[A_1]_{i,j}[B]_{j,k}\right) + \left(\sum_{j=1}^m[A_2]_{i,j}[B]_{j,k}\right). \]
The proof of linearity in second argument is similar.
\end{proof}

TODO: associative class! (In fact category!)

\begin{lemma} \label{submatrixMultiplication}
Let $A\in \F^{k\times m}$, $B\in \F^{m\times n}$, $I\subseteq 0:k$ and $J\subseteq 0:n$. Then
\[ [A]_{I,-}\cdot [B]_{-,J} = [AB]_{I,J}. \]
\end{lemma}
\begin{proof}
Using \ref{componentsSubmatrixFromEnumeration}, we have
\begin{align*}
\big[[A]_{I,-}\cdot [B]_{-,J}\big]_{r,s} &= \sum_{p\in 0:m}\big[[A]_{I,-}\big]_{r, p} \big[[B]_{-,J}\big]_{p,s} \\
&= \sum_{p\in 0:m}[A]_{\pi_I(r), p}[B]_{p, \pi_J(s)} \\
&= [A\cdot B]_{\pi_I(r), \pi_J(s)} \\
&= \big[[A\cdot B]_{I,J}\big]_{r,s}.
\end{align*}
\end{proof}

\subsubsection{Identity}
TODO: the identity matrix is an identity for matrix multiplication!


\begin{lemma}
Let $A\in\F^{m\times n}$. Then
\[ A = \mathbb{1}_m\cdot A = A \cdot \mathbb{1}_n. \]
\end{lemma}
\begin{proof}
By a simple calculation in components:
\[ [\mathbb{1}_m\cdot A]_{ij} = \sum_{k=1}^m\delta_{ik}[A]_{kj} = [A]_{ij}. \]
The other equation is similar.
\end{proof}

\begin{lemma} \label{matrixOfOnesMultiplication}
Let $l,m,n\in\N_0$, then
\[ \mathbb{J}^{l\times m}\mathbb{J}^{m\times n} = m\mathbb{J}^{l\times n}. \]
\end{lemma}
\begin{proof}
\[ [\mathbb{J}^{l\times m}\mathbb{J}^{m\times n}]_{i,j} = \sum_{k=1^m}[\mathbb{J}^{l\times m}]_{i,k}[\mathbb{J}^{m\times n}]_{k,j} = \sum_{k=1^m}1 = m. \]
\end{proof}
\begin{corollary}
Let $a,b,c,d\in\R$, then
\[ (a\mathbb{1}_n + b\mathbb{J}_n)(c\mathbb{1}_n + d\mathbb{J}_n) = ac\mathbb{1}_n + (ad+bc +bdn)\mathbb{J}_n \]
\end{corollary}

\subsubsection{Left and right inverses}
\begin{definition}
Let $A\in \mathbb{F}^{m\times n}$. A matrix $B$ is a \udef{left inverse} of $A$ if $BA = \mathbb{1}_n$. A matrix $B$ is a \udef{right inverse} of $A$ if $AB = \mathbb{1}_m$.
\end{definition}
Not all matrices have a left and/or right inverses.

\subsubsection{Square matrices}
\begin{lemma}
For any $n\in\N_0$, the vector space $\F^{n\times n}$ of square matrices is a monoid with as operation matrix multiplication and as neutral element the identity matrix $\mathbb{1}_n$.
\end{lemma}

In particular we can define integer powers of matrices. We set $A^0 = \mathbb{1}_n$ by convention.

We also have that if square matrices have both a left inverse and a right inverse, they are the same. See \ref{leftRightInverseMonoid}.

In fact, we have something stronger:
\begin{lemma}
Let $A\in \mathbb{F}^{n\times n}$ be a square matrix. Then $A$ has a left inverse \textup{if and only if} $A$ has a right inverse. Both inverses are the same.
\end{lemma}
\begin{proof}
Consider the map $f:\F^{n\times n}\to \F^{n\times n}: B\mapsto AB$. (This will later be called the left regular representation of $A$). 
 
We follow a chain of implications.
\begin{itemize}
\item \textit{$A$ has left inverse $\Rightarrow$ $f$ is injective}. Assume there exist matrices $B_1,B_2$ such that $AB_1 = AB_2$, then
\[ 0 = A^{-1}0 = A^{-1}(AB_1-AB_2) = A^{-1}A(B_1-B_2) = B_1-B_2. \]
\item \textit{$f$ is injective $\Rightarrow$ $f$ is surjective}. By \ref{invertibleFiniteDim}.
\item \textit{$f$ is surjective $\Rightarrow$ $A$ has right inverse}. By surjectivity there exists a matrix $B$ such that $f(B) = AB=\mathbb{1}_n$. Then $B$ is a right inverse by definition.
\end{itemize}
We have proven that the existence of a left inverse implies the existence of a right inverse. The opposite implication is obtained by considering the right regular representation $f:\F^{n\times n}\to \F^{n\times n}: B\mapsto BA$.
\end{proof}
\begin{definition}
A square matrix is called \udef{invertible} or \udef{nonsingular} or \udef{nondegenerate} if it has a left and right inverse.

If it is not invertible, it is called \udef{singular} or \udef{degenerate}.

We denote the inverse of $A$ as $A^{-1}$.
\end{definition}
\begin{lemma}
Let $A,B\in \mathbb{F}^{n\times n}$ be invertible matrices. Then $AB$ is invertible with inverse
\[ (AB)^{-1} = B^{-1}A^{-1}. \]
\end{lemma}

\begin{lemma}
Let $a,b\in\R$. Then
\[ (a\mathbb{1}_n +b\mathbb{J}_n)^{-1} = \frac{1}{a(a+nb)}[(a+nb)\mathbb{1}_n -b\mathbb{J}_n] = \frac{1}{a}\mathbb{1}_n - \frac{b}{a(a+nb)}\mathbb{J}_n. \]
\end{lemma}
\begin{proof}
By multiplication, using \ref{matrixOfOnesMultiplication}.
\end{proof}

\begin{definition}
Let $A\in\F^{n\times n}$. We say $A$ is \udef{strictly diagonally dominant} if
\[ \sum_{\substack{j \in 1:n \\ j\neq i}}|[A]_{ij}| < |[A]_{ii}|. \]
\end{definition}
\begin{proposition} \label{invertibleDiagonallyDominant}
If $A\in\F^{n\times n}$ is strictly diagonally dominant, then it is invertible.
\end{proposition}
\begin{proof}
We prove the contraposition. Assume, then, that $A$ is not invertible, so there exists an $x\neq 0$ such that $Ax = 0$ (TODO ref). So then for all $i\in 1:n$ we have
\[ \sum_{j = 1}^n [A]_{ij}x_j. \]
Setting $|x_m| = \max_{i\in 1:n}|x_i|$, we have
\[ [A]_{mm}x_m = -\sum_{\substack{j\in 1:n \\ j\neq m}}[A]_{mj}x_j \]
and so
\[ |[A]_{mm}|\cdot |x_m| = |\sum_{\substack{j\in 1:n \\ j\neq m}}[A]_{mj}x_j| \leq \sum_{\substack{j\in 1:n \\ j\neq m}}|[A]_{mj}|\cdot |x_j| \leq \sum_{\substack{j\in 1:n \\ j\neq m}}|[A]_{mj}|\cdot |x_m|.\]
In particular this means $A$ cannot be strictly diagonally dominant.
\end{proof}

\subsection{Matrices and linear maps}
\subsubsection{Matrices as maps $\F^n\to \F^m$}
The matrix multiplication $\F^{m\times n}\times \F^n \to \F^m$ can be curried to produce a map $\ell$. The input of $\ell$, i.e.\ a matrix in $\F^{m\times n}$ is typically written as a subscript. So, for a matrix $A\in\F^{m\times n}$ we have
\[ \ell_A: \F^n \to \F^m: \vec{v}\mapsto \ell_A(\vec{v}) = A\vec{v}. \]
Because the matrix multiplication is linear in the second argument (see \ref{linearityMatrixMultiplication}), the map $\ell_A: \F^n \to \F^m$ is linear for all matrices $A\in \F^{m\times n}$. So we have
\[ \ell: \F^{m\times n}\to \Hom(\F^n, \F^m). \]

Additionally this map $\ell$ is linear due to the matrix multiplication being linear in the first argument (see again \ref{linearityMatrixMultiplication}).

\begin{proposition} \label{ellIsomorphism}
For all $n,m\in\N_0$, the map
\[ \ell: \F^{m\times n}\to \Hom(\F^n, \F^m) \]
is an isomorphism.
\end{proposition}
\begin{proof}
We will explicitly construct an inverse. Take some $L\in\Hom(\F^n, \F^m)$. Now $L$ is completely determined by the images of the elements in the standard basis $\mathcal{E}= \{\vec{e}_i\}_{i=1}^n$, so we just need to find a matrix $A$ such that $\ell_A$ maps the basis elements to the same elements as $L$.

Now $A\vec{e}_1$ is just the first column of $A$. So we set the first column of $A$ to be $L(\vec{e}_1)$. Similarly $A\vec{e}_i$ is the $i^\text{th}$ column of $A$ and we set it equal to $L(\vec{e}_i)$. This gives the required matrix.
\end{proof}
The matrix $A$ that satisfies $\ell_A = L$ is often denoted $A_L$.

The construction in the previous proof is important for practically finding matrices associated with linear maps. The construction can be recapped as follows:
\[ A_L = \begin{pmatrix}
L(\vec{e}_1) & L(\vec{e}_2) & \hdots & L(\vec{e}_n)
\end{pmatrix} = \begin{pmatrix}
L_1(\vec{e}_1) & L_1(\vec{e}_2) & \hdots & L_1(\vec{e}_n)  \\
L_2(\vec{e}_1) & L_2(\vec{e}_2) & & \\
\vdots & & \ddots & \\
L_m(\vec{e}_1) & & & L_m(\vec{e}_n)
\end{pmatrix} \qquad [A_L]_{ij} = L_i(\vec{e}_j) \]
where $\mathcal{E}= \{\vec{e}_i\}_{i=1}^n$ is the standard basis of $\F^n$ and $L_i(\vec{e}_j) = [L(\vec{e}_j)]_i$ is the $i^\text{th}$ component of $L(\vec{e}_j)$.

If $\F^n$ is equipped with the standard inner product, then this can also be written as $L_i(\vec{e}_j) = \inner{\vec{e}_i, L(\vec{e}_j)}$, so
\[ A_L = \begin{pmatrix}
\inner{\vec{e}_1, L(\vec{e}_1)} & \inner{\vec{e}_1, L(\vec{e}_2)} & \hdots & \inner{\vec{e}_1, L(\vec{e}_n)}  \\
\inner{\vec{e}_2, L(\vec{e}_1)} & \inner{\vec{e}_2, L(\vec{e}_2)} & & \\
\vdots & & \ddots & \\
\inner{\vec{e}_m, L(\vec{e}_1)} & & & \inner{\vec{e}_m, L(\vec{e}_n)}
\end{pmatrix}. \]

\begin{proposition}
The map $\ell$ translates matrix multiplication into function composition:
\[ \ell_{AB} = \ell_A \circ \ell_B \qquad\text{and}\qquad \ell^{-1}(f\circ g) = \ell^{-1}(f)\ell^{-1}(g). \]
\end{proposition}
\begin{proof}
Let $A\in \F^{k\times m}$ and $B\in \F^{m\times n}$. Let $\vec{v}\in\F^n$, then
\[ \ell_{AB}(\vec{v}) = (AB)\vec{v} = A(B\vec{v}) = A(\ell_B(\vec{v})) = \ell_A(\ell_B(\vec{v})) = (\ell_A\circ\ell_B)(\vec{v}). \]
\end{proof}

\begin{lemma}
For all $n\in\N$ we have $\id_{\F^{n\times n}} = \ell_{\mathbb{1}_n}$.
\end{lemma}

\begin{lemma} \label{invertibleMapInvertibleMatrix}
Let $A$ be a matrix over $\F$. Then $\ell_A$ is invertible \textup{if and only if} $A$ is square and invertible.
\end{lemma}
\begin{proof}
Assume $\ell_A: \F^n\to\F^m$ invertible. Then $\F^n\cong\F^m$, so $m=n$ by \ref{isomorphicDimension}. Also $\ell^{-1}((\ell_A)^{-1})$ is an inverse of A because
\[ \ell^{-1}((\ell_A)^{-1})\cdot A = \ell^{-1}((\ell_A)^{-1})\cdot\ell^{-1}(\ell_A) = \ell^{-1}[(\ell_A)^{-1}\ell_A] = \ell^{-1}(\id_{\F^n}) = \mathbb{1}_n. \]

Conversely, assume $A$ invertible with inverse $A^{-1}$. Then $\ell_{A^{-1}}$ is the inverse of $\ell_A$:
\[ \ell_{A^{-1}}\ell_A = \ell_{\mathbb{1}_n} = \id_{\F^n} \qquad \text{and}\qquad \ell_A\ell_{A^{-1}} = \ell_{\mathbb{1}_n} = \id_{\F^n}. \]
\end{proof}

\subsubsection{Linear maps as matrices}
We can associate a matrix to any linear map by passing to coordinates. Let $L: V\to W$ be a linear map from an $n$-dimensional vector space $V$ to an $m$-dimensional vector space $W$. If we fix bases $\mathcal{V}$ of $V$ and $\mathcal{W}$ of $W$, then $\ell^{-1}$ associates a unique matrix with the linear map
\[ \co_{\mathcal{W}}\circ L \circ \co^{-1}_{\mathcal{V}}: \F^n\to\F^m. \]
In other words, $A$ is the unique matrix such that
\[ \begin{tikzcd}
V\rar{L}\dar[swap]{\co_{\mathcal{V}}} & W \dar{\co_{\mathcal{W}}} \\
\F^n \rar{\ell_A} & \F^m
\end{tikzcd} \qquad \text{commutes.} \]
We call this matrix $A \defeq \ell^{-1}(\co_{\mathcal{W}}\circ L \circ \co^{-1}_{\mathcal{V}})$ the \udef{matrix of the linear map $L$} w.r.t. the bases $\mathcal{V}$ and $\mathcal{W}$. This matrix is denoted
\[ (L)_{\mathcal{V}}^{\mathcal{W}} \qquad \text{or}\qquad \prescript{\mathcal{W}}{}{(L)}^{\mathcal{V}} \qquad \text{or}\qquad (L)_{\mathcal{W}\leftarrow \mathcal{V}}. \]

The commutativity of the diagram translates to the following lemma:
\begin{lemma} \label{commutativityMatrixLinearMap}
Let $L:V\to W$ be a linear map and $\mathcal{V},\mathcal{W}$ bases of $V,W$ respectively. Then
\[ (L)^\mathcal{W}_\mathcal{V} \circ \co_\mathcal{V} = \co_{\mathcal{W}}\circ L. \]
\end{lemma}
A practical way to calculate matrices associated with linear maps is given by the following lemma.
\begin{lemma}
Let $L:V\to W$ be a linear map and $\mathcal{V}=\{\vec{v}_i\}_{i=1}^n,\mathcal{W} = \{\vec{w}_i\}_{i=1}^m$ bases of $V,W$ respectively. Then
\[ (L)^\mathcal{W}_\mathcal{V} = \begin{pmatrix}
\co_\mathcal{W}(L(\vec{v}_1)) & \co_\mathcal{W}(L(\vec{v}_2)) & \hdots & \co_\mathcal{W}(L(\vec{v}_n)).
\end{pmatrix} \]
\end{lemma}
\begin{proof}
The $i^\text{th}$ column of $(L)^\mathcal{W}_\mathcal{V}$ is equal to $(L)^\mathcal{W}_\mathcal{V}\vec{e}_i = (L)^\mathcal{W}_\mathcal{V}(\co_\mathcal{V}(\vec{v}_i))$, where $\vec{e}_i$ is the $i^\text{th}$ element of the standard basis $\mathcal{E}$. This is equal to $\co_{\mathcal{W}}(L(\vec{v}))$ by \ref{commutativityMatrixLinearMap}.
\end{proof}

\begin{proposition} \label{algebraMatricesLinearMaps}
Let $U,V,W$ be vector spaces with bases $\mathcal{U},\mathcal{V},\mathcal{W}$, resp., and $S:V\to W, T:U\to V$ linear maps. Then
\[ (S)_\mathcal{V}^\mathcal{W}(T)_\mathcal{U}^\mathcal{V} = (S\circ T)_{\mathcal{U}}^{\mathcal{W}}. \]
\end{proposition}
\begin{proof}
We calculate
\begin{align*}
\ell^{-1}(\co_{\mathcal{W}}\circ S \circ \co^{-1}_{\mathcal{V}})\ell^{-1}(\co_{\mathcal{V}}\circ T \circ \co^{-1}_{\mathcal{U}}) &= \ell^{-1}(\co_{\mathcal{W}}\circ S \circ \co^{-1}_{\mathcal{V}}\circ\co_{\mathcal{V}}\circ T \circ \co^{-1}_{\mathcal{U}}) \\
&= \ell^{-1}(\co_{\mathcal{W}}\circ (S \circ T) \circ \co^{-1}_{\mathcal{U}}) = (S\circ T)_{\mathcal{U}}^{\mathcal{W}}.
\end{align*}
\end{proof}

\begin{proposition}
Let $V,W$ be finite-dimensional vector spaces over a field $\mathbb{F}$ with bases $\mathcal{V}$ and $\mathcal{W}$, respectively. The mapping
\[ (-)^{\mathcal{W}}_{\mathcal{V}}:\Hom_\mathbb{F}(V,W) \to \mathbb{F}^{m\times n}: L\mapsto (L)^{\mathcal{W}}_{\mathcal{V}} \]
is an isomorphism.
\end{proposition}
\begin{proof}
It is equal to
\[ \ell^{-1}\circ(\co_\mathcal{W})_*\circ(\co^{-1}_\mathcal{V})^*. \]
Now $\ell$ is an isomorphism by \ref{ellIsomorphism} and thus $\ell^{-1}$ is one by \ref{inverseLinear}. The maps $(\co_\mathcal{W})_*$ and $(\co^{-1}_\mathcal{V})^*$ are injective maps between finite-dimensional spaces by \ref{monicEpicInPrePostComposition} and thus isomorphisms by \ref{invertibleFiniteDim}.
So we have a composition of isomorphisms, which is an isomorphism.
\end{proof}
\begin{corollary}
Let $V,W$ be finite-dimensional vector spaces over a field $\mathbb{F}$, then
\[ \dim_{\mathbb{F}}\Hom_\mathbb{F}(V,W) = (\dim_{\mathbb{F}} V)\cdot (\dim_{\mathbb{F}} W). \] \label{dimHomset}
\end{corollary}

\begin{lemma}
Let $A\in\F^{m\times n}$ be a matrix and let $\mathcal{E}_m$ and $\mathcal{E}_n$ be the standard bases of $\F^m$ and $\F^n$. Then
\[ (\ell_{A})_{\mathcal{E}_n}^{\mathcal{E}_m} = A. \]
\end{lemma}

\subsubsection{Changing basis with matrices}
In particular we can apply all the theory of the previous section to the identity map $\id: V\to V$. Let $\beta, \beta'$ be two bases of $V$. Then \ref{commutativityMatrixLinearMap} gives
\[ \co_{\beta'}(v) = (\id)_{\beta}^{\beta'}\co_\beta(v) \]
which captures the effect of transforming from one basis to another.
\begin{definition}
Matrices of the form $(\id)_{\beta}^{\beta'}$ are called \udef{transition matrices} or \udef{change-of-basis matrices}.
\end{definition}
\begin{lemma}
Let $\beta, \beta'$ be two bases of a vector space $V$. Then
\[ \left((\id)_{\beta}^{\beta'}\right)^{-1} = (\id)_{\beta'}^{\beta}. \]
\end{lemma}
\begin{proof}
This follows from \ref{algebraMatricesLinearMaps} which gives
\[ (\id)_{\beta'}^{\beta}(\id)_{\beta}^{\beta'} = (\id)_\beta^\beta = \mathbb{1}_n. \]
\end{proof}

Let $L\in \Hom(V)$. If we know $(L)_\beta^\beta$, we can calculate $(L)_{\beta'}^{\beta'}$ using
\begin{align*}
(L)_{\beta'}^{\beta'} &= (\id)_{\beta}^{\beta'}(L)_\beta^\beta(\id)_{\beta'}^{\beta} \\
&= \left((\id)_{\beta'}^{\beta}\right)^{-1}(L)_\beta^\beta(\id)_{\beta'}^{\beta}
\end{align*}
\begin{definition}
Let $A,B\in \mathbb{F}^{n\times n}$, then $A$ and $B$ are called \udef{similar} if there exists an invertible matrix $P\in\mathbb{F}^{n\times n}$ such that
\[ B = P^{-1}A P\] 
\end{definition}

Any similar matrices may be seen as matrices of the same linear transformation w.r.t. different bases.



\section{The transpose and standard inner product}
\subsection{The transpose}
\begin{definition}
Let $A\in \mathbb{F}^{m\times n}$. The \udef{transpose} of $A$, denoted $A^\transp$, is defined by
\[ [A^\transp]_{ij} = [A]_{ji}. \]
\end{definition}
\begin{lemma}
The transpose is a linear operation.
\end{lemma}

\begin{lemma}
Let $A,B$ be matrices such that $AB$ is defined, then
\[ (AB)^\transp = B^\transp A^\transp. \]
\end{lemma}
\begin{proof}
We simply calculate
\[ [(AB)^\transp]_{ij} = [AB]_{ji} = \sum_{k}[A]_{jk}[B]_{ki} = \sum_{k}[B]_{ki}[A]_{jk} = \sum_{k}[B^\transp]_{ik}[A^\transp]_{kj} = [B^\transp A^\transp]_{ij}. \]
\end{proof}

\begin{definition}
\begin{itemize}
\item A square matrix $A$ such that $A=A^\transp$ is called \udef{symmetric}.
\item A square matrix $A$ such that $A=-A^\transp$ is called \udef{skew symmetric}.
\end{itemize}
\end{definition}

\subsection{The standard inner product}
Let $\vec{v},\vec{w}\in\F^n$. Then the standard inner product is given by
\[ \inner{\vec{v},\vec{w}} \defeq \overline{\vec{v}}^\transp \vec{w}. \]
This reduces to $\vec{v}^\transp\vec{w}$ if $\F = \R$.

In the sequel we will always assume $\F^n$ is equipped with the standard inner product, unless otherwise specified.

The inner product also induces a norm. There are multiple norms that may be of interest. To avoid confusion we may denote the norm that arises from the inner product with a subscript 2:
\[ \norm{\vec{v}} = \norm{\vec{v}}_2 = \sqrt{\inner{\vec{v},\vec{v}}}. \]

\begin{lemma} \label{componentsFromStandardInnerProduct}
Let $\{\vec{e}_i\}_{i=1}^n$ be the standard basis of $\F^n$ and $\{\vec{f}_i\}_{i=1}^m$ the standard basis of $\F^m$. Let $A\in\F^{n\times m}$. Then
\[ [A]_{i,j} = \vec{e}^\transp_i A \vec{f}_j = \inner{\vec{e}_i, A\vec{f}_j}. \]
\end{lemma}
\begin{proof}
Using \ref{standardBasisFromIdentityMatrix} and \ref{submatrixMultiplication}, we calculate
\begin{align*}
\vec{e}^\transp_i A \vec{f}_j &= [\mathbb{1}_n]_{-,i}^\transp \cdot A \cdot [\mathbb{1}_m]_{-,j} \\
&= [\mathbb{1}_n]_{i, -} \cdot [A]_{-,-} \cdot [\mathbb{1}_m]_{-,j} \\
&= [\mathbb{1}_n\cdot A]_{i,-}\cdot [\mathbb{1}_m]_{-,j} \\
&= [\mathbb{1}_n\cdot A\cdot\mathbb{1}_m]_{i,j} = [A]_{i,j}.
\end{align*}
We also use that $\overline{\vec{e}_i} = \vec{e}_i$.
\end{proof}

\subsubsection{The norm}
The identification $\F^n \cong \F^{n\times 1}$ gives $\F^n$ an operator norm in addition to the norm derived from the inner product. These norms coincide since, for any unit vector $\lambda\in \F$, we have
\[ \norm{\vec{v}\lambda} = |\lambda| \norm{\vec{v}} = \vec{v}. \]

\begin{lemma}
Let $\vec{v}\in\F^n$. Then $\norm{\vec{v}^\transp} = \norm{\vec{v}}$.
\end{lemma}
\begin{proof}
Restratement of \ref{normVectorSupUnitvectors}.
\end{proof}

\begin{lemma} \label{matrixElementsBoundedByNorm}
Let  $\{\vec{e}_i\}_{i=1}^n$ be the standard basis of $\F^n$ and $\{\vec{f}_i\}_{i=1}^m$ the standard basis of $\F^m$. Let $A\in\F^{n\times m}$. Then $\big|[A]_{i,j}\big| \leq \norm{A}$.
\end{lemma}
\begin{proof}
We have, by \ref{componentsFromStandardInnerProduct} and \ref{operatorNormInnerProduct},
\[ \big|[A]_{i,j}\big| = \big|[A]_{i,j}\big| = |\inner{\vec{e}_i, A\vec{f}_j}| \leq \norm{A}. \]
\end{proof}

\subsubsection{The adjoint}
\begin{definition}
Let $A\in \F^{m\times n}$. We call $A^* \defeq \ell^{-1}\big(\ell(A)^*\big)$ the \udef{adjoint} of $A$, where we have equiped $\F^m$ and $\F^n$ with their standard inner products.
\end{definition}

\begin{lemma}
Let $A\in \F^{m\times n}$. Then $A^* = \overline{A}^\transp$.
\end{lemma}
In other words, the adjoint is the conjugate transpose.
\begin{proof}
Take arbitrary $\vec{v}\in\F^m$ and $\vec{w}\in\F^n$. Then
\[ \inner{\vec{v}, A\vec{w}} = \overline{\vec{v}}^\transp A\vec{w} = \overline{\overline{A}^\transp\vec{v}}^\transp\vec{w} = \inner{\overline{A}^\transp\vec{v}, \vec{w}}. \]
TODO!
\end{proof}

\begin{definition}
Let $A\in\F^{m\times n}$ be a matrix. For properties related to the adjoint, we say $A$ has the property if $\ell_A$ has the property.

In addition, we say
\begin{itemize}
\item $A$ is \udef{Hermitian} if $A$ is self-adjoint;
\item $A$ is \udef{skew Hermitian} if $A$ is swew-adjoint.
\end{itemize}
\end{definition}
Thus, e.g.\
\begin{itemize}
\item if $A^*A = AA^*$, then $A$ is called normal;
\item if $A^2 = A = A^*$, then $A$ is called an orthogonal projection;
\item if $A^*A = \mathbb{1}_n$, then $A$ is called isometric;
\item if $A^*A = \mathbb{1}_n$ and $AA^* = \mathbb{1}_m$, then $A$ is called unitary.
\end{itemize}

\begin{lemma}
If $A\in \F^{m\times n}$ is unitary, then $m=n$.
\end{lemma}
\begin{proof}
TODO
\end{proof}

\begin{lemma}
Let $U\in\F^{n\times n}$ be a matrix. Then the following are equivalent
\begin{enumerate}
\item $U$ is unitary, i.e.\ $U^*U=\mathbb{1}_n$;
\item the columns of $U$ are orthonormal;
\item $UU^*=\mathbb{1}_n$;
\item $U^*$ is unitary;
\item the row of $U$ are orthonormal;
\item $U$ is invertible and $U^{-1} = U^*$.
\end{enumerate}
\end{lemma}
Thus every square isometric matrix is unitary.
\begin{proof}
TODO
\end{proof}

\begin{corollary}
If $\alpha,\beta$ are orthonormal bases of $\F^{n}$, then the transition matrix $(\id)_\alpha^\beta$ is unitary.
\end{corollary}


\subsection{Orthogonal matrices}
\begin{proposition}
Let $A\in\R^{n\times n}$ and $\R^n$ have the standard inner product. The following are equivalent:
\begin{enumerate}
\item The columns of $A$ form an orthonormal basis of $\R^n$;
\item The rows of $A$ form an orthonormal basis of $\R^n$;
\item $A^\transp A = \mathbb{1}_n$;
\item $A^{-1} = A^\transp$;
\end{enumerate}
\end{proposition}
\begin{definition}
A matrix $A\in\R^{n\times n}$ satisfying any of the above is called an \udef{orthogonal matrix}.
\end{definition}
We can formulate the spectral theorem as follows:
\begin{proposition}
Let $A\in\R^{n\times n}$ be a symmetric matrix. The there exists an orthogonal matrix $P\in \R^{n\times n}$ such that $P^{-1}AP = P^\transp A P$ is a diagonal matrix.
\end{proposition}

$\det(A) = \pm 1$. Orthogonal matrix means orthogonal transformation.
For orthogonal matrix $P$: $\norm{A}_F = \norm{PA}_F$
s

\subsection{Inequalities}
\subsubsection{Cauchy-Schwarz corollaries}
\begin{lemma}
Let $A\in \C^{n\times n}$ be positive ($A\geq 0$). Then
\[ |[A]_{i,j}| \leq \sqrt{[A]_{i,i}[A]_{j,j}}. \]
\end{lemma}
\begin{proof}
Consider the energy form $\inner{-, -}_A = \inner{-, A-} = \overline{(-)}^\transp A(-)$. This is a pre-inner product. Let $\{\vec{e}_i\}_{i=0}^n$ be the standard basis of $\F^n$. Then $[A]_{i,j} = \inner{\vec{e}_i, \vec{e}_j}_A$ by \ref{componentsFromStandardInnerProduct}. Thus, in particular, $\sqrt{[A]_{i,i}} = \sqrt{\inner{\vec{e}_i, \vec{e}_i}_A} = \norm{\vec{e}_i}_A$.

The Cauchy-Schwarz inequality \ref{CauchySchwarz} then yields the result.
\end{proof}

\section{Block matrices}
Any given matrix can be \textit{interpreted} as consisting of submatrices. Eg,
\[ \begin{bmatrix}
1&2&3\\4&5&6\\7&8&9
\end{bmatrix} \quad \text{can be viewed as} \quad 
\begin{bmatrix}
\begin{pmatrix}
1 & 2
\end{pmatrix} & 3 \\
\begin{pmatrix}
4 & 5 \\ 7 & 8
\end{pmatrix} & \begin{pmatrix}
6 \\ 9
\end{pmatrix}
\end{bmatrix}
\quad \text{with partitioning} \quad
\left[\begin{array}{cc|c}
1&2&3\\ \hline
4&5&6\\7&8&9
\end{array}\right]
 \]
\begin{definition}
A matrix consisting of submatrices (also known as blocks) is called a \udef{block matrix} or \udef{partitioned matrix}.

If $A\in\F^{n\times n}, \vec{x},\vec{y}\in\F^n$ and $c\in \F$, we call the block matrices
\[ \begin{bmatrix}
c & \vec{x}^\transp \\ \vec{y} & A
\end{bmatrix}, \begin{bmatrix}
\vec{x}^\transp & c \\ A & \vec{y}
\end{bmatrix}, \begin{bmatrix}
A & \vec{x} \\
\vec{y}^\transp & c
\end{bmatrix}, \quad\text{and}\quad \begin{bmatrix}
\vec{x} & A \\ c & \vec{y}^\transp
\end{bmatrix} \]
\udef{bordered matrices}. They have been obtained from $A$ by \udef{bordering}.
\end{definition}

The partition can be specified at the level of the dimensions: let $A$ be an $(m\times n)$-matrix, let $m_1,\ldots, m_k\leq m$ be the number of rows in each horizontal partition and let $n_1,\ldots, n_l\leq n$ be the number of columns in each vertical partition. Clearly
\[ m = \sum_{i=1}^k m_i \qquad \text{and} \qquad n = \sum_{i=1}^l n_i. \]

We then say the block matrix has dimensions $(m_1|\ldots|m_k) \times (n_1|\ldots|n_l)$, or $A$ is a $(m_1|\ldots|m_k) \times (n_1|\ldots|n_l)$-matrix.

\begin{example}
The block matrix considered above is a $(1|2)\times (2|1)$-matrix.
\end{example}

We write $A_{i,j}$ or $(A)_{i,j}$ for the block in the $i^\text{th}$ horizontal partition and the $j^\text{th}$ vertical partition. So if $A$ is a $(m_1|\ldots|m_k) \times (n_1|\ldots|n_l)$-matrix, we can write
\[ A = \begin{bmatrix}
A_{1,1} & \hdots & A_{1,l} \\
\vdots & \ddots & \vdots \\
A_{k,1} & \hdots & A_{k,l}
\end{bmatrix} \qquad \text{where} \qquad A_{i,j}\in\F^{m_i\times n_j}. \]

Any $m\times n$-matrix can be partitioned into columns, which means identifying it with an $m\times (1|1|\ldots|1)$-matrix.

Similarly, any $m\times n$-matrix can be partitioned into rows, which means identifying it with a $(1|1|\ldots|1)\times n$-matrix.

\begin{lemma}
Let $A$ be a partitioned matrix of dimensions $(m_1|\ldots|m_k) \times (n_1|\ldots|n_l)$, then $A^\transp$ is a partitioned matrix of dimensions $(n_1|\ldots|n_l) \times (m_1|\ldots|m_k)$ and
\[ (A)_{i,j} = (A^\transp)_{j,i}. \]
\end{lemma}

\begin{definition}
Let $A,B$ be matrices. We can then form the \udef{direct sum} matrix $A\oplus B$ as follows:
\[ A\oplus B = \begin{pmatrix}
A & \mathbb{0} \\ \mathbb{0} & B
\end{pmatrix}. \]
\end{definition}

\begin{proposition}
Let $A$ be a partitioned matrix of dimensions $(m_1|\ldots|m_k) \times (n_1|\ldots|n_l)$ and $B$ a partitioned matrix of dimensions $(n_1|\ldots|n_l)\times (p_1|\ldots|p_q)$.

The matrix product of $A$ and $B$ is an $(m_1|\ldots|m_k) \times (p_1|\ldots|p_q)$-matrix with blocks
\[ (AB)_{i,j} = \sum_{t}A_{i,t}B_{t,j}. \]
That is, multiplication of two block matrices can be carried out as if their blocks were scalars.
\end{proposition}
We can abbreviate the dimension requirements as
\[ [(m_1|\ldots|m_k) \times (n_1|\ldots|n_l)]\cdot[(n_1|\ldots|n_l)\times (p_1|\ldots|p_q)] = [(m_1|\ldots|m_k) \times (p_1|\ldots|p_q)]. \]
\begin{corollary} \label{multiplicationBlockMatrices}
Let $A,B$ be matrices of dimensions $k\times l$ and $l \times m$. Then
\[ AB = [AB]_{-,-} = \sum_k[A]_{-,k}[B]_{k, -}; \]
and
\begin{itemize}
\item we can partition $A$ into rows and $B$ into columns to get
\[ [AB]_{i,j} = [A]_{i,-}[B]_{-, j}; \]
\item we can partition $B$ into columns to get
This can also be written as
\[ [AB]_{-, j} = [A]_{-,-}[B]_{-,j} = A[B]_{-, j}; \]
\item we can partition $A$ into rows to get
This can also be written as
\[ [AB]_{i, -} = [A]_{i,-}[B]_{-,-} = [A]_{i,-}B. \]
\end{itemize}
\end{corollary}
In particular this means we can write
\[ AB = \begin{pmatrix}
A
\end{pmatrix} \begin{pmatrix}
[B]_{-,1} & \hdots & [B]_{-, m}
\end{pmatrix} = \begin{pmatrix}
A[B]_{-,1} & \hdots & A[B]_{-,m}
\end{pmatrix}. \]

\subsection{Identities and inverses}
\begin{lemma}
Let $X,Y,Z$ be conformal matrices and $Y,Z$ invertible, then
\[ \begin{bmatrix}
Y & X \\ \mathbb{0} & Z
\end{bmatrix}^{-1} = \begin{bmatrix}
Y^{-1} & -Y^{-1}XZ^{-1} \\ \mathbb{0} & Z^{-1}.
\end{bmatrix}. \]
In particular
\[ \begin{bmatrix}
\mathbb{1} & X \\ \mathbb{0} & \mathbb{1}
\end{bmatrix}^{-1} = \begin{bmatrix}
\mathbb{1} & -X \\ \mathbb{0} & \mathbb{1}
\end{bmatrix}. \]
\end{lemma}
\begin{corollary}
If $A$ is an upper triangular matrix with nonzero diagonal entries, the $A$ is invertible and $[A^{-1}]_{i,i} = [A]_{i,i}^{-1}$
\end{corollary}
\begin{proof}
Induction on dimension.
\end{proof}

\begin{lemma}
Let $X,Y,Z$ be conformal matrices. Then
\[ \begin{bmatrix}
Y & X \\ \mathbb{0} & Z
\end{bmatrix} = \begin{bmatrix}
\mathbb{1} & A \\ \mathbb{0} & \mathbb{1}
\end{bmatrix}\begin{bmatrix}
Y & X-AZ + YA \\ \mathbb{0} & Z
\end{bmatrix}\begin{bmatrix}
\mathbb{1} & -A \\ \mathbb{0} & \mathbb{1}
\end{bmatrix} \]
for all conformal $A$.
\end{lemma}

\begin{lemma} \label{schurComplementLemma}
Let $A,B,C,D$ be conformal matrices. Then, if $D$ is invertible 
\[ M = \begin{bmatrix}
A & B \\ C & D
\end{bmatrix} = \begin{bmatrix}
\mathbb{1} & BD^{-1} \\ \mathbb{0} & \mathbb{1}
\end{bmatrix}\begin{bmatrix}
A-BD^{-1}C & \mathbb{0} \\ \mathbb{0} & D
\end{bmatrix}\begin{bmatrix}
\mathbb{1} & \mathbb{0} \\ D^{-1}C & \mathbb{1}
\end{bmatrix}  \]
and if $A$ is invertible,
\[ M = \begin{bmatrix}
A & B \\ C & D
\end{bmatrix} = \begin{bmatrix}
\mathbb{1} & \mathbb{0} \\ CA^{-1} & \mathbb{1}
\end{bmatrix}\begin{bmatrix}
A & \mathbb{0} \\ \mathbb{0} & D - CA^{-1}B
\end{bmatrix}\begin{bmatrix}
\mathbb{1} & A^{-1}B \\ \mathbb{0} & \mathbb{1}
\end{bmatrix}.  \]
\end{lemma}
\begin{definition}
The matrix $M/D \defeq A-BD^{-1}C$ is the \udef{Schur complement} of $D$ in $M$.

Similarly $M/A \defeq D-CA^{-1}B$ is the Schur complement of $A$ in $M$.
\end{definition}
\begin{lemma}
Let $A,B,C,D$ be conformal matrices such that all requisite matrices are invertible. Then
\begin{align*}
M^{-1} &= \begin{bmatrix}
A & B \\ C & D
\end{bmatrix}^{-1} = \begin{bmatrix}
\mathbb{1} & \mathbb{0} \\ -D^{-1}C & \mathbb{1}
\end{bmatrix}\begin{bmatrix}
(M/D)^{-1} & \mathbb{0} \\ \mathbb{0} & D^{-1}
\end{bmatrix}\begin{bmatrix}
\mathbb{1} & -BD^{-1} \\ \mathbb{0} & \mathbb{1}
\end{bmatrix} \\
&= \begin{bmatrix}
(M/D)^{-1} & -(M/D)^{-1}BD^{-1} \\ -D^{-1}C(M/D)^{-1} & D^{-1}+D^{-1}C(M/D)^{-1}BD^{-1}\end{bmatrix} \\
M^{-1} &= \begin{bmatrix}
A & B \\ C & D
\end{bmatrix}^{-1} = \begin{bmatrix}
\mathbb{1} & -A^{-1}B \\ \mathbb{0} & \mathbb{1}
\end{bmatrix}\begin{bmatrix}
A^{-1} & \mathbb{0} \\ \mathbb{0} & (M/A)^{-1}
\end{bmatrix}\begin{bmatrix}
\mathbb{1} & \mathbb{0} \\ -CA^{-1} & \mathbb{1}
\end{bmatrix} \\
&= \begin{bmatrix}
A^{-1}+A^{-1}B(M/A)^{-1}CA^{-1} & -A^{-1}B(M/A)^{-1} \\-(M/A)^{-1}CA^{-1} & (M/A)^{-1}
\end{bmatrix}.
\end{align*}
\end{lemma}

\begin{lemma}
Let $A,B$ be conformal matrices. Then
\[ \begin{bmatrix}
AB & A \\ \mathbb{0} & \mathbb{0}
\end{bmatrix} \qquad\text{and}\qquad \begin{bmatrix}
\mathbb{0} & A \\ \mathbb{0} & BA
\end{bmatrix} \]
are similar
\end{lemma}
\begin{proof}
$\begin{bmatrix}
\mathbb{1} & \mathbb{0} \\ B & \mathbb{1}
\end{bmatrix}\begin{bmatrix}
AB & A \\ \mathbb{0} & \mathbb{0}
\end{bmatrix} = \begin{bmatrix}
\mathbb{0} & A \\ \mathbb{0} & BA
\end{bmatrix}\begin{bmatrix}
\mathbb{1} & \mathbb{0} \\ B & \mathbb{1}
\end{bmatrix}.$
\end{proof}

\begin{proposition}
Let $A,B,C,U,V$ be conformal matrices. Then
\begin{enumerate}
\item \textup{(Push-through identity)} $(\mathbb{1}+UV)^{-1}U = U(\mathbb{1}+VU)^{-1}$;
\item $(\mathbb{1}+A)^{-1} \begin{aligned}[t]
&= \mathbb{1}-(\mathbb{1}+A)^{-1} A\\
&= \mathbb{1}-A(\mathbb{1}+A)^{-1}
\end{aligned}$
\item $(\mathbb{1} + UV)^{-1} = \mathbb{1} - U(\mathbb{1}+VU)^{-1}V$;
\item \textup{(Woodbury identity)} $(B + UCV)^{-1} = B^{-1} - B^{-1}U(C^{-1}+VB^{-1}U)VB^{-1}$.
\end{enumerate}
\end{proposition}
\begin{proof}
(1) From $U(\mathbb{1} + VU) = (\mathbb{1} + UV)U$.

(2) From $\mathbb{1} = (\mathbb{1}+A)(\mathbb{1}+A)^{-1} = (\mathbb{1}+A)^{-1} + A(\mathbb{1}+A)^{-1}$.

(3) $\begin{aligned}[t]
(\mathbb{1} + UV)^{-1} &= \mathbb{1}-(\mathbb{1} + UV)^{-1}UV & &\text{using (2)} \\
&= \mathbb{1}-U(\mathbb{1} + VU)^{-1}V & &\text{using (1)}.
\end{aligned}$

(4) $\begin{aligned}[t]
(B + UCV)^{-1} &= (B(\mathbb{1} + B^{-1}UCV))^{-1} \\
&= (\mathbb{1} + (B^{-1}U)(CV))^{-1}B^{-1} \\
&= (\mathbb{1}-(B^{-1}U)(\mathbb{1} + (CV)(B^{-1}U))^{-1}(CV))B^{-1} & &\text{using (3)}\\
&= B^{-1}-B^{-1}U(\mathbb{1} + CVB^{-1}U)^{-1}CVB^{-1} \\
&= B^{-1}-B^{-1}U(C^{-1}(\mathbb{1} + CVB^{-1}U))^{-1}VB^{-1} \\
&= B^{-1}-B^{-1}U(C^{-1} + VB^{-1}U)^{-1}VB^{-1}.
\end{aligned}$
\end{proof}
\begin{corollary}[Sherman-Morrison formula]
Let $A\in\F^{n\times n}$ and $\vec{u},\vec{v}\in\F^n$. 
Then $A+\vec{u}\vec{v}^\transp$ is invertible iff $1 + \vec{v}^\transp A^{-1}\vec{u} \neq 0$. In this case
\[ (A+\vec{u}\vec{v}^\transp)^{-1} = A^{-1} - \frac{A^{-1}\vec{u}\vec{v}^\transp A^{-1}}{1 + \vec{v}^\transp A^{-1}\vec{u}}. \]
\end{corollary}
\begin{corollary}[Hua's identity]
Let $A,B$ be conformal matrices. Then
\begin{align*}
(A+B)^{-1} &= A^{-1} - A^{-1}(B^{-1}+A^{-1})^{-1}A^{-1} \\
&= A^{-1} - A^{-1}(AB^{-1}+\mathbb{1})^{-1} \\
(A-B)^{-1} &= A^{-1} + A^{-1}B(A-B)^{-1} \\
&= \sum_{k=0}^\infty (A^{-1}B)^kA^{-1}.
\end{align*}
\end{corollary}

\section{Vector spaces associated with a matrix}
\subsection{Row and column space}
\begin{definition}
Let $A$ be an $(n\times m)$-matrix. It can be partitioned into both rows and columns. Let $R_1,\ldots, R_n$ be the rows of $A$ and $C_1,\ldots, C_m$ the columns of $A$:
\[ A = \begin{pmatrix}
R_1 \\ \vdots \\ R_n
\end{pmatrix} = \begin{pmatrix}
C_1 & \hdots & C_m
\end{pmatrix}. \]
Then
\begin{itemize}
\item $\Span\{R_1,\ldots, R_n\}$ is the \udef{row space} $\Row(A)$ of $A$; and
\item $\Span\{C_1,\ldots, C_m\}$ is the \udef{column space} $\Col(A)$ of $A$.
\end{itemize}
We call $\dim\Row(A)$ the \udef{row rank} and $\dim\Col(A)$ the \udef{column rank}.
\end{definition}
Clearly $\Col(A) = \Row(A^\transp)$.

\begin{lemma} \label{columnSpace}
Let $A\in \F^{m\times n}$ and $\vec{b}\in \F^m$. Then
\[ \exists \vec{x}\in\F^n: A\vec{x}=\vec{b} \quad\iff\quad \vec{b}\in\Col(A). \]
Moreover, if the columns of $A$ are linearly independent, then the $\vec{x}\in\F^n$ is unique. 
\end{lemma}
\begin{proof}
By \ref{multiplicationBlockMatrices}
\begin{align*}
A\vec{x} &= [A\vec{x}]_{-,-} = \sum_{j=1}^n[A]_{-,j}[\vec{x}]_{j,-} = \sum_{j=1}^n[A]_{-,j}[\vec{x}]_{j} \\
&= [A]_{-,1}[\vec{x}]_1 + \ldots [A]_{-,n}[\vec{x}]_n  \\
&= C_1[\vec{x}]_1 + \ldots C_n[\vec{x}]_n.
\end{align*}
\end{proof}

\begin{proposition} \label{rowColSubspaces}
Let $A$ and $B$ be matrices. Then
\begin{enumerate}
\item $\Col(B)\subseteq \Col(A) \iff B = AX$ for some matrix $X$;
\item $\Row(B)\subseteq \Row(A) \iff B = YA$ for some matrix $Y$.
\end{enumerate}
\end{proposition}
Note that for point (1) to hold, $A$ and $B$ must have the same number of rows. For point (2) they must have the same number of columns.
\begin{proof}
(1) $\boxed{\Rightarrow}$ Assume $\Col(B)\subseteq \Col(A)$. Then   by \ref{columnSpace}, for each column $\vec{b}_j = [B]_{-,j}$ of $B$ we can find an $\vec{x}_j\in \F^n$ such that $A\vec{x}_j = \vec{b}_j$. Then
\[ B = \begin{pmatrix}
\vec{b}_1 & \hdots & \vec{b}_k
\end{pmatrix} = \begin{pmatrix}
A\vec{x}_1 & \hdots & A\vec{x}_k
\end{pmatrix} = A \begin{pmatrix}
\vec{x}_1 & \hdots & \vec{x}_k
\end{pmatrix} = AX.\]

$\boxed{\Leftarrow}$ By \ref{multiplicationBlockMatrices} we can write
\[ AB = \begin{pmatrix}
A[B]_{-,1} & \hdots & A[B]_{-,m}
\end{pmatrix}, \]
so every column in $AB$ is of the form $A[B]_{-,i}$, which is a linear combination of the columns in $A$.

(2) We simply calculate using point (1):
\[ \Row(B)\subseteq \Row(A) \iff \Col(B^\transp)\subseteq \Col(A^\transp) \iff B^\transp = A^\transp X \iff B = X^\transp A. \]
\end{proof}
\begin{corollary} \label{RowColSpaceProduct}
Let $A,B$ be conformal matrices. Then
\begin{enumerate}
\item $\Col(AB)\subseteq \Col(A)$;
\item $\Row(AB)\subseteq \Row(B)$.
\end{enumerate}
\end{corollary}
\begin{corollary} \label{RowColSpaceInverse}
A matrix $A$ is invertible \textup{if and only if} $\Col(A) = \Row(A) = \F^n$.
\end{corollary}
\begin{proof}
$\boxed{\Rightarrow}$ From $A = \mathbb{1}_nA$ we see that $\Col(A)\subseteq\Col(\mathbb{1}_n) = \F^n$.

From $\mathbb{1}_n = AA^{-1}$ we see that $\Col(\mathbb{1}_n)\subseteq\Col(A)$, so $\Col(\mathbb{1}_n) = \Col(A)$. The calculation of the row space is similar.

$\boxed{\Leftarrow}$ From $\Col(A) = \Col(\mathbb{1}_n)$, there exists an $X$ such that $\mathbb{1}_n = AX$, so $X$ is the inverse of $A$.
\end{proof}
\begin{corollary}
Let $A\in\F^{m\times n}$. Then
\[ \dim\Row(A) = \dim\Col(A) \]
i.e.\ the row rank equals the column rank.
\end{corollary}
\begin{proof}
Take a basis for $\Col(A)$ and let $X$ have these vectors as columns. Then $\Col(X) = \Col(A)$, so $A = XY$ for some $Y\in\F^{k\times n}$.

By point 2. of the proposition, we have $\Row(A)\subseteq \Row(Y)$ and due to the dimensions, we have $\dim\Row(Y)\leq k$. So
\[ \dim\Row(A) \leq \dim\Row(Y) \leq k = \dim\Col(A). \]

Consider $A^\transp$, which can be factorised as before by taking a basis of its column space and putting the vectors in the columns of $X'$. Then $A^\transp = X'Y'$. As before, we have
\[ \dim\Col(A) = \dim\Row(A^\transp) \leq \dim\Col(A^\transp) = \dim\Row(A). \]

Combining the inequalities gives $\dim\Col(A) = \dim\Row(A)$.
\end{proof}
\begin{definition}
We can unambiguously call $\dim\Col(A)=\dim\Row(A)$ the \udef{rank} of the matrix $A$. We write $\Rank(A)$.
\end{definition}

\begin{lemma} \label{extendToInvertible}
Let $A\in\F^{m\times n}$.
\begin{enumerate}
\item If $\Rank(A) = m$, then $n\geq m$ and there exists $X\in\F^{(n-m)\times n}$ such that
\[ \begin{bmatrix}
A \\ X
\end{bmatrix} \in \F^{n\times n} \quad \text{is invertible.} \]
\item If $\Rank(A) = n$, then $m\geq n$ and there exists $Y\in\F^{m\times (m-n)}$ such that
\[ \begin{bmatrix}
A & Y
\end{bmatrix} \in \F^{m\times m} \quad \text{is invertible.} \]
\end{enumerate}
\end{lemma}
\begin{proof}
(1) In this case the rows are linearly independent and elements of $\F^n$, so they can be extended to a basis of $\F^n$. This extension is the matrix $X$.

(2) Transpose and apply (1).
\end{proof}

\begin{lemma}[Full-rank factorisation]
Let $A\in\F^{m\times n}$ and $k=\dim\Col(A)$. Then $A$ can be factorised as $A=XY$ where $X\in\F^{m\times k}$, $Y\in\F^{k\times n}$ and
\[ k = \Rank(A) = \Rank(X) = \Rank(Y). \]
Moreover, for any matrix $X$, the following are equivalent:
\begin{enumerate}
\item the columns of $X$ form a basis of $\Col(A)$;
\item there is a unique $Y\in\F^{k\times n}$ such that $A=XY$.
\end{enumerate}
Clearly considering $A^\transp$ yields dual equivalences.
\end{lemma}
\begin{proof}
By \ref{columnSpace} $Xv = [A]_{-,j}$ has a unique solution $v=y_j$ for all $j$ if and only if the columns of $X$ are linearly independent and $[A]_{-,j}$ is in their span, i.e.\ they from a basis for $\Col(A)$.
\end{proof}

\begin{proposition} \label{imageColumnSpace}
Let $L$ be a linear map. Then
\[ \im(L) = \Col(A_L). \]
\end{proposition}
This implies that the rank of $L$ is the rank of $A$.

\begin{proposition}
Let $A,B\in\F^{m\times n}$. Then
\[ \Col(A)=\Col(B) \iff \text{$\exists$ invertible $X$ such that $A = BX$}. \]
\end{proposition}
\begin{proof}
$\boxed{\Leftarrow}$ follows from \ref{rowColSubspaces}.

$\boxed{\Rightarrow}$ Let $C$ be a matrix whose columns form a basis of $\Col(A)=\Col(B)$. Then we can find matrices $S,T$ such that $CS = A$ and $CT = B$ are full-rank factorisations and these can be extended to invertible matrices by \ref{extendToInvertible}
\[ X_1 = \begin{bmatrix}
S \\ U
\end{bmatrix} \in \F^{n\times n} \qquad X_2 = \begin{bmatrix}
T \\ V
\end{bmatrix} \in \F^{n\times n}. \]
Now we claim $X = X_2^{-1}X_1$ fulfils the requirements:
\[ BX = (CT + 0V)X_2^{-1}X_1 = \begin{bmatrix}
C & \mathbb{0}
\end{bmatrix}X_2(X_2^{-1}X_1) = \begin{bmatrix}
C & \mathbb{0}
\end{bmatrix}X_1 = CS = A. \]
\end{proof}

\subsection{Null space}
\begin{definition}
Let $A\in \F^{m\times n}$ be a matrix. The \udef{null space} $\Null(A)$ of $A$ is the kernel of $\ell_A$. The dimension of $\Null(A)$ is called the \udef{nullity} of $A$.
\end{definition}
In other words:
\[ \Null(A) = \setbuilder{\vec{v}\in \F^n}{A\vec{v} = 0}. \]

\begin{proposition}
Let $A\in\F^{m\times n}$ be a matrix, then
\[ \Null(A) = \Col(A^*)^\perp. \]
\end{proposition}
\begin{proof}
$\vec{v}\in\Null(A) \iff A\vec{v} = 0 \iff \forall \vec{w}\in\F^n:\inner{A\vec{v},\vec{w}}=0 \iff \forall \vec{w}\in\F^n: \inner{\vec{v},A^*\vec{w}}=0 \iff \vec{v}\in\Col(A^*)^\perp$.
\end{proof}

\begin{lemma} \label{dimensionTheoremMatrices}
Let $A\in \F^{m\times n}$ be a matrix. Then
\[ \Rank(A) + \dim\Null(A) = n. \]
\end{lemma}
\begin{proof}
This is the dimension theorem applied to $\ell_A$, using $\im(\ell_A) = \Col(A)$.
\end{proof}

We can also formulate \ref{kernelCompositionLinearMaps} for matrices:
\begin{proposition} \label{nullSpaceProduct}
Let $A,B$ be conformal matrices. Then
\begin{enumerate}
\item $\Null(AB)\supseteq \Null(B)$;
\item $\dim\Null(AB) = \dim\Null(B) + \dim(\Col(B)\cap\Null(A))$.
\end{enumerate}
\end{proposition}
Note that (1) is the opposite inclusion to $\Col(AB)\subseteq \Col(A)$ and $\Row(AB)\subseteq \Row(B)$.
\begin{corollary}[Sylvester's law of nullity]
Let $A,B$ be square matrices. Then
\[ \max\{\dim\Null(A),\dim\Null(B)\} \leq \dim\Null(AB) \leq \dim\Null(A) + \dim\Null(B). \]
\end{corollary}

\subsection{The rank of a matrix}
\subsection{Rank equalities and inequalities}
\begin{proposition} \label{rankMultiplication}
Let $A,B,C,D$ be conformal matrices, then
\begin{enumerate}
\item $\Rank(AB) \leq \min\{\Rank(A),\Rank(B)\}$;
\item $\max\{\Rank(A),\Rank(C)\} \leq \Rank \begin{bmatrix}
A & C
\end{bmatrix}$;
\item $\max\{\Rank(A),\Rank(D)\} \leq \Rank \begin{bmatrix}
A \\ D
\end{bmatrix}$.
\end{enumerate}
\end{proposition}
\begin{proof}
(1) This is the matrix form of \ref{rankMapComposition}. It is also an immediate consequence of \ref{RowColSpaceProduct}.
\end{proof}

\begin{lemma}
Let $A\in\F^{m\times n}$, then
\[ \Rank(A) \leq \min\{m,n\}. \]
\end{lemma}
\begin{definition}
Let $A\in\F^{m\times n}$.
\begin{itemize}
\item If $\Rank(A) = \min\{m,n\}$, we say $A$ has \udef{full rank}.
\item If $\Rank(A) = m$, we say $A$ has \udef{full row rank}.
\item If $\Rank(A) = n$, we say $A$ has \udef{full column rank}.
\end{itemize}
\end{definition}

\begin{lemma}
Let $X,A,Y$ be conformal matrices. Then
\begin{enumerate}
\item if $X$ has full column rank, then $\Rank(A) = \Rank(XA)$;
\item if $Y$ has full row rank, then $\Rank(A) = \Rank(AY)$;
\item $A$ is invertible \textup{if and only if} it has full row rank and full column rank.
\end{enumerate}
\end{lemma}
\begin{proof}
(1) By \ref{dimensionTheoremMatrices}, $\Null(X) = \{0\}$. So $\vec{v}\in\Null(XA) \iff \vec{v}\in\Null(A)$, so $\Null(XA) = \Null(A)$. Then \ref{dimensionTheoremMatrices} implies $\Rank(XA) = \Rank(A)$.

(2) $\Rank(AY) = \Rank(Y^\transp A^\transp) = \Rank(A^\transp) = \Rank(A)$.

(3) Consequence of \ref{RowColSpaceInverse}.
\end{proof}

\begin{proposition}[Sylvester's rank inequality] \label{SylvesterRankInequality}
Let $A\in\F^{m\times k}$ and $B\in\F^{k\times n}$, then
\[ \Rank(AB) \geq \Rank(A)+\Rank(B) - k. \]
In addition, the following are equivalent:
\begin{enumerate}
\item $\Null(A)\subseteq \Col(B)$;
\item $\Rank(AB) = \Rank(A)+\Rank(B) - k$.
\end{enumerate}
In particular if $AB$ is a full-rank factorisation, i.e.\ $\Rank(AB) = k$, then (1) and (2) hold.
\end{proposition}
\begin{proof}
Let $AB = XY$ be a full-rank factorisation of $AB$ and set $r=\Rank(AB)$. Then define
\[ C = \begin{bmatrix}
A & X
\end{bmatrix}\in\F^{m\times(k+r)} \quad\text{and}\quad D=\begin{bmatrix}
B \\ -Y
\end{bmatrix}\in\F^{(k+r)\times n}\]
so $CD = \begin{bmatrix}
A & X
\end{bmatrix}\begin{bmatrix}
B \\ -Y
\end{bmatrix} = AB-XY = 0$.
This means that $\Col(D)\subseteq \Null(C)$, so $\Rank(D)\leq \dim\Null(C)$. Then
\[ \Rank(A) + \Rank(B) \leq \Rank(C) + \Rank(D) \leq \Rank(C) + \dim\Null(C) = k+\Rank(AB). \]
using \ref{dimensionTheoremMatrices} for $\Rank(C) + \dim\Null(C) = k+r$.

Now for the equivalent statements:

$\boxed{(1)\Rightarrow (2)}$ In this case \ref{nullSpaceProduct} becomes
\[ \dim\Null(AB) = \dim\Null(A) + \dim\Null(B). \]
Using \ref{dimensionTheoremMatrices}, this becomes
\[ n - \Rank(AB) = k-\Rank(A) + n-\Rank(B), \]
which can be arranged to give (2).

$\boxed{(2)\Rightarrow (1)}$ Assume, towards contraposition, $\Null(A) \nsubseteq \Col(B)$. Then we can find $\vec{v}\in\Null(A)$ such that $\vec{v}\notin \Col(B)$. Then
\[ \Rank \begin{bmatrix}
B & \vec{v}
\end{bmatrix} = \Rank(B) + 1 \qquad\text{and}\qquad \Rank(A \begin{bmatrix}
B & \vec{v}
\end{bmatrix}) = \Rank\begin{bmatrix}
AB & A\vec{v}
\end{bmatrix} = \Rank(AB). \]
And by the inequality
\[ \Rank(AB) \geq \Rank(A)+\Rank(B)+1-k, \]
so in particular $\Rank(AB) > \Rank(A)+\Rank(B)-k$ and thus $\Rank(AB) \neq \Rank(A)+\Rank(B)-k$.

Finally:

$\boxed{(\Rank(AB) = k)\Rightarrow (1)}$ From $\Rank(AB)\leq\Rank(A)\leq k$ and $\Rank(AB)\leq\Rank(B)\leq k$ we get
\[ \Rank(A) = \Rank(B) = \Rank(AB) = k \]
and so
\[ \Rank(A) + \Rank(B) - k = k =\Rank(AB). \]
\end{proof}
\begin{corollary}
Let $A\in\F^{m\times k}$ and $B\in\F^{k\times n}$ such that $\Null(A)\subseteq \Col(B)$, then
\[ AB = \mathbb{0}_{m\times n} \iff \Col(B)\subseteq \Null(A) \iff \Rank(A) + \Rank(B) = k \]
\end{corollary}

\begin{proposition}[Frobenius's rank inequality]
Let $A,B,C$ be conformal matrices, then
\[ \Rank(ABC) \geq \Rank(AB)+\Rank(BC) - \Rank(B). \]
\end{proposition}
\begin{proof}
Let $B = XY$ be a full-rank factorisation. Then
\begin{align*}
\Rank(ABC) &= \Rank(AXYC) \geq \Rank(AX) + \Rank(YC) - \Rank(B) \\
&\geq \Rank(AXY) + \Rank(XYC) - \Rank(B) = \Rank(AB)+\Rank(BC) - \Rank(B)
\end{align*}
using Sylvester's rank inequality \ref{SylvesterRankInequality}.
\end{proof}

\begin{proposition}
Let $A,B$ be conformal matrices. Then
\[ |\Rank(A)-\Rank(B)|\leq\Rank(A+B)\leq \Rank(A) + \Rank(B). \]
\end{proposition}
\begin{proof}
Let $A=X_1Y_1$ and $B=X_2Y_2$ be full-rank factorisations with $r = \Rank(A)$ and $s=\Rank(B)s$. Define
\[ C = \begin{bmatrix}
X_1 & X_2
\end{bmatrix}\qquad\text{and}\qquad \begin{bmatrix}
Y_1 \\ Y_2
\end{bmatrix}. \]
Then $CD = X_1Y_1 + X_2Y_2 = A+B$.

The second inequality follows from \ref{rankMultiplication}:
\[ \Rank(A+B) = \Rank(CD) \leq \min\{\Rank(C),\Rank(D)\} \leq r+s. \]

The first inequality follows from \ref{rankMultiplication}, which gives $\Rank(C)\geq\max\{r,s\}$ and $\Rank(D)\geq\max\{r,s\}$, Sylvester's rank inequality \ref{SylvesterRankInequality}, which gives
\[ \Rank(A+B) \geq \Rank(C) + \Rank(D) - (r+s) \geq r+r-(r+s) = r-s \]
and
\[ \Rank(A+B) \geq \Rank(C) + \Rank(D) - (r+s) \geq s+s-(r+s) = s-r. \]
These combine to give the first inequality.
\end{proof}

\begin{proposition}[Guttman rank additivity formula]
Let $M=\begin{pmatrix}
A & B \\ C & D
\end{pmatrix}$ and $A$ or $D$ invertible, then
\begin{align*}
\Rank(M) &= \Rank(D) + \Rank(A-BD^{-1}C) \\
&= \Rank(A) + \Rank(D-CA^{-1}B).
\end{align*}
\end{proposition}
\begin{proof}
This uses the Schur complement decomposition \ref{schurComplementLemma} and the fact that the transformation matrices are invertible.
\end{proof}

\subsubsection{The index of matrices}
\begin{proposition}
Let $A\in\F^{n\times n}$ be a square matrix. Then
\begin{enumerate}
\item $\Rank(A^k)\geq \Rank(A^{k+1})$ for all $k\in\N$;
\item if $\Rank(A^k) = \Rank(A^{k+1})$, then for all $p\in\N$
\[ \Rank(A^k) = \Rank(A^{k+p}) \qquad\text{and}\qquad \Col(A^k) = \Col(A^{k+p}); \]
\item there is a least integer $q\in [0,n]$ such that $\Rank(A^q) = \Rank(A^{q+1})$.
\end{enumerate}
\end{proposition}
These assertions remain true for exponent zero so long as we define $A^0 = \mathbb{1}_n$.

\begin{definition}
The \udef{index} of a square matrix is the least positive integer $q$ such that $\Rank(A^q) = \Rank(A^{q+1})$.
\end{definition}
In particular, invertible matrices have index zero.

\section{Matrix operations}
All functions on $\F$ can of course be extended component-wise to functions on $\F^{m\times n}$. In particular, let $\overline{A}$ be the component-wise complex conjugate of $A\in\F^{m\times n}$.
\subsection{Elementary row and column operations}
\begin{definition}
Let $A$ be an $(m\times n)$-matrix. The following are \udef{elementary row operations} (EROs):
\begin{itemize}[leftmargin=3cm]
\item[$\boxed{R_i \to \lambda R_i}$] Replacing a row by a non-zero multiple (i.e.\ $\lambda \neq 0$).
\item[$\boxed{R_i \leftrightarrow R_j}$] Swapping two rows.
\item[$\boxed{R_i \to R_i+ \lambda R_j}$] Adding a multiple of a row to a different row.
\end{itemize}
The \udef{elementary column operations} (ECO) are these operations applied to the columns. The are the operations given by transposing the matrix, applying an ERO and transposing again.

Two matrices $A,B$ are called \udef{row equivalent} if it is possible to obtain $B$ from $A$ by applying EROs. We write $A\sim_R B$ or just $A\sim B$.

Two matrices $A,B$ are called \udef{column equivalent} if it is possible to obtain $B$ from $A$ by applying ECOs. We write $A\sim_C B$.

Two matrices $A,B$ are called \udef{row+column equivalent} if it is possible to obtain $B$ from $A$ by applying EROs and ECOs. We write $A\sim_{R+C} B$.
\end{definition}
In general the theory will be developed for EROs. The relevant results for ECOs are obtained by transposition.

\begin{lemma}
The relations $\sim_R, \sim_C$ and $\sim_{R+C}$ are equivalence relations.
\end{lemma}

\begin{lemma} \label{matricesEROs}
Applying the elementary row operations to an $(n\times m)$-matrix is the same as multiplying from the left by the matrices obtained by applying the ERO to the $(n\times n)$-unit matrix:
\begin{itemize}[leftmargin=3cm]
\item[$\boxed{R_i \to \lambda R_i}$]
\[ \begin{pmatrix}
1 &&&&&& \\
 & \ddots &&&&& \\ 
  & & 1 &&&& \\
 & &  & \lambda &&& \\
 & &  & & 1 && \\
 &&&&& \ddots & \\
 &&&&&& 1
\end{pmatrix} \]
\item[$\boxed{R_i \leftrightarrow R_j}$]
\[ \begin{pmatrix}1&&&&&&\\&\ddots &&&&&\\&&0&&1&&\\&&&\ddots &&&\\&&1&&0&&\\&&&&&\ddots &\\&&&&&&1\end{pmatrix} \]
\item[$\boxed{R_i \to R_i+ \lambda R_j}$]
\[ \begin{pmatrix}1&&&&&&\\&\ddots &&&&&\\&&1&&&&\\&&&\ddots &&&\\&&m&&1&&\\&&&&&\ddots &\\&&&&&&1\end{pmatrix} \]
\end{itemize}
\end{lemma}
\begin{corollary}
ECOs can be performed by multiplying by an invertible matrix from the right.
\end{corollary}
\begin{corollary}
Let $A,B\in\F^{m\times n}$ be matrices.
\begin{enumerate}
\item If $A\sim_R B$, then $\Row(A) = \Row(B)$ and $\Null(A) = \Null(B)$.
\item If $A\sim_C B$, then $\Col(A) = \Col(B)$.
\end{enumerate}
\end{corollary}
\begin{corollary}
Let $A,B\in\F^{m\times n}$ be matrices such that $A\sim_{R+C} B$. Then $\Rank(A) = \Rank(B)$ and $\dim\Null(A) = \dim\Null(B)$.
\end{corollary}
\begin{proof}
The dimension of the null space is preserved because $\dim\Null(A) = n - \Rank(A)$, by \ref{dimensionTheoremMatrices}.
\end{proof}

\subsubsection{Gauss-Jordan elimination}
\begin{definition}
Let $A$ be an $(n\times m)$-matrix.
\begin{itemize}
\item The first non-zero element from the left on each row is called the \udef{leading coefficient} of \udef{pivot} of that row.
\item The matrix $A$ is in \udef{row echelon form} (ref) if
\begin{itemize}
\item rows of all zeros are at the bottom;
\item the leading coefficients of all other rows are one;
\item the leading coefficients of each non-zero row is strictly to the right of the leading coefficient of the row above it.
\end{itemize}
\item The matrix $A$ is in \udef{reduced row echelon form} (rref) if
\begin{itemize}
\item it is in echelon form;
\item all the coefficients in the column above a leading coefficient are zero.
\end{itemize}
\end{itemize}
\end{definition}

\begin{proposition}[Gauss-Jordan elimination]
Any matrix is row equivalent to a matrix in reduced echelon form.
\end{proposition}
The proof gives an algorithm for computing the reduced echelon form. Sometimes the part of the algorithm that brings the matrix to an (unreduced) echelon form is called Gaussian elimination. We first give an algorithm for Gaussian elimination and then use this for full Gauss-Jordan elimination.
\begin{proof}[Proof (Gaussian elimination)]
Let $A\in\F^{m\times n}$ be the input-matrix. The algorithm is recursive, we call it $\operatorname{ref}$.
\begin{enumerate}
\item If $A$ is empty (i.e.\ $m=0$ or $n=0$), then return $A$.
\item Find the entry in the first column of largest non-zero absolute value. Swap the corresponding row with the first row.
\begin{itemize}
\item If all the entries in the first column are zero, return
\[ \begin{pmatrix}
0 & \operatorname{ref}\left([A]_{1:m,2:n}\right)
\end{pmatrix}. \]
\item The algorithm also works if we choose any non-zero entry, not necessarily the largest.
\end{itemize}
\item Divide the first row by $[A]_{1,1}$.
\item Perform the ERO $R_i\to R_i - [A]_{i,1}R_1$ for each row $i\in 2:m$.
\item Return
\[ \begin{pmatrix}
1 & [A]_{1,2:n} \\
0 & \operatorname{ref}\left([A]_{2:m,2:n}\right)
\end{pmatrix}. \]
\end{enumerate}
The algorithm terminates because each subsequent call involves a matrix of strictly smaller dimensions.
\end{proof}
\begin{proof}[Proof (Obtaining the reduced echelon form)]
For each leading coefficient $[A]_{ij}$, perform the EROs $R_k\to R_k-[A]_{k,j}R_i$ for all rows $k\in 1:i-1$.
\end{proof}

\begin{corollary}
Let $A$ be any matrix. Then
\[ A \;\sim_{R+C}\; \begin{pmatrix}
\mathbb{1}_k & 0^{k\times p} \\
0^{q\times k} & 0^{q\times p}
\end{pmatrix}. \]
The row and column ranks are both equal to $k$ and the nullity is equal to $p$.

This also means we can write
\[ A = P\begin{pmatrix}
\mathbb{1}_k & 0^{k\times p} \\
0^{q\times k} & 0^{q\times p}
\end{pmatrix}Q \]
for some invertible matrices $P,Q$.
\end{corollary}
This factorisation of $A$ is called the \udef{rank normal form}.


We can use Gauss-Jordan elimination to find bases for $\Row(A), \Col(A)$ and $\Null(A)$:
\begin{itemize}
\item[$\boxed{\Row(A)}$] The row space is preserved by row equivalence. So we can perform Gauss-Jordan elimination, discard the null rows and the remaining rows will form a basis of $\Row(A)$.
\item[$\boxed{\Col(A)}$] Applying an ERO to a matrix $A$ is the same as multiplying from the left by some invertible matrix $E$. Due to \ref{multiplicationBlockMatrices},
\[ EA = \begin{pmatrix}
E[A]_{-,1} & \hdots & E[A]_{-,m}
\end{pmatrix}, \]
so $\Col(EA) = \ell_E[\Col(A)]$ and because $\ell_E$ is an isomorphism, we have $\Col(A) \cong \Col(EA)$.

In the echelon form it is easy to see that the columns containing leading elements form a basis. By applying the inverse map we see that the corresponding columns in the original matrix form a basis of the original column space $\Col(A)$, see \ref{mappingOfBasisByIsomorphism}.
\item[$\boxed{\Null(A)}$] The row space is preserved by row equivalence, so we first perform Gauss-Jordan elimination $A\sim_R U$. Then solve $Ux=0$ for $x$ (see later).
\end{itemize}

\begin{proposition}
Let $L:\F^n\to\F^m$ be a linear map. Then for any bases $\beta_n,\beta_m$ of $\F^n$ and $\F^m$, we have
\[ (L)_{\beta_n}^{\beta_m} \;\sim_{R+C}\; \begin{pmatrix}
\mathbb{1}_k & 0^{k\times q} \\
0^{p\times k} & 0^{p\times q}
\end{pmatrix} \]
and
\begin{enumerate}
\item $L$ is injective \textup{if and only if} $p=0$;
\item $L$ is surjective \textup{if and only if} $q=0$.
\end{enumerate}
\end{proposition}


\subsubsection{Calculating inverse matrices}
Any invertible square matrix $A$ is row equivalent to $\mathbb{1}_n$. We want to find the matrix associated to this row reduction. One way to keep track of this matrix is to simultaneously apply the EROs to $A$ and $\mathbb{1}_n$:
\begin{lemma}
Let $A\in\F^{n\times n}$ be a square matrix. Then
\[ \left(\begin{array}{c|c} A & \mathbb{1}_n \end{array}\right) \sim_R \left(\begin{array}{c|c} \mathbb{1}_n & A^{-1} \end{array}\right). \]
\end{lemma}
\begin{proof}
$A^{-1}\begin{pmatrix}
A & \mathbb{1_n}
\end{pmatrix} = \begin{pmatrix}
\mathbb{1}_n & A^{-1}
\end{pmatrix}$.
\end{proof}


\subsection{The trace}
\begin{definition}
Let $A\in \mathbb{F}^{n\times n}$ be a square matrix. The \udef{trace} of $A$, denoted $\Tr(A)$, is the sum of the diagonal entries of $A$:
\[ \Tr(A) = \sum_{i=1}^n (A)_{i,i}. \]
\end{definition}
\begin{proposition}
Let $A\in \mathbb{F}^{n\times n}$ be a square matrix, then
\begin{enumerate}
\item $\Tr(cA+B) = c\Tr(A) + \Tr(B)$ for all $c\in\F$;
\item $\Tr(\overline{A}) = \overline{\Tr(A)}$;
\item $\Tr(A^\transp) = \Tr(A)$;
\item $\Tr(A^*) = \overline{\Tr(A)}$.
\end{enumerate}
\end{proposition}
\begin{proposition}
Let $A,B,C \in\F^{n\times n}$ be square matrices. Then
\begin{enumerate}
\item $\Tr(AB) = \sum_{i,j=1}^n [A]_{i,j}[B]_{j,i}$;
\item $\Tr(AB) = \Tr(BA)$;
\item $\Tr(ABC) = \Tr(CAB) = \Tr(BCA)$.
\end{enumerate}
\end{proposition}
\begin{proof}
Point (1) follows by direct computation
\[ \Tr(AB) = \sum_{i=1}^n (AB)_{i,i} = \sum_{i=1}^n\sum_{j=1}^n A_{i,j}B_{j,i}. \]

(2) We use (1) and the fact that $[A]_{i,j}[B]_{j,i} = [B]_{i,j}[A]_{j,i}$.

(3) Follows straight from (2): $\Tr\big((AB)C\big) = \Tr\big(C(AB)\big)$ etc.
\end{proof}
The trace of a product of matrices is invariant under any cyclic permutation of the matrices.

The trace of a product of matrices is not invariant under noncyclic permutation of the matrices:
\[ \Tr(ABC) \neq \Tr(BAC). \]



\begin{lemma} \label{averageTraceOverDiagonal}
Let $A\in\F^{n\times n}$. Then $A$ is similar to a matrix $B$ with
\[ [B]_{i,i} = \frac{1}{n}\Tr(A) \qquad \forall i\leq n. \]
\end{lemma}
\begin{proof}
It is enough to show that this holds for matrices with zero trace: if $A$ is any matrix, then $A- \frac{\Tr(A)}{n}\mathbb{1}$ is a matrix with zero trace. If this is similar to a matrix $S(A- \frac{\Tr(A)}{n}\mathbb{1})S^{-1}$ with zeros on the diagonal, then $SAS^{-1}$ has $\frac{\Tr(A)}{n}$ on the diagonal.

So assume WLOG that $\Tr(A) = 0$ and $A\neq 0$. We can find $\vec{v}\in\F^n$ such that $\{\vec{v}, A\vec{v}\}$ is linearly independent: assume, towards contraposition, that $A\vec{v}$ is a scalar multiple of $\vec{v}$, for all $\vec{v}$. This must be the same multiple $\lambda$ for all $\vec{v}$. If not, i.e.\ there exist $\vec{v},\vec{w}$ such that $A\vec{v} = \lambda_1 \vec{v}$ and $A\vec{w} = \lambda_2 \vec{w}$ with $\lambda_1\neq \lambda_2$, then $\vec{v}+\vec{w}$ is not mapped to a multiple. In this case $A$ is $\lambda \id$, but $\Tr(A) = 0$ fixes $\lambda=0$, which we excluded.

Now we can extend $\{\vec{v}, A\vec{v}\}$ to a basis of $\F^n$ and put these as columns in a matrix $S = \begin{bmatrix}
\vec{v} & A\vec{v} & S_1
\end{bmatrix}$. We can partition $S^{-1} = \begin{bmatrix}
\vec{x}^\transp \\ S_2
\end{bmatrix}$. Then we have
\[ S^{-1}S = \begin{bmatrix}
\vec{x}^\transp \vec{v} & \vec{x}^\transp A\vec{v} & \vec{x}^\transp S_1 \\ S_2\vec{v} & S_2A\vec{v} & S_2S_1
\end{bmatrix} = \mathbb{1}. \]
In particular $\vec{x}^\transp A\vec{v} = 0$. Then
\[ S^{-1}AS = \begin{bmatrix}
\vec{x}^\transp A\vec{v} & \vec{x}^\transp A^2\vec{v} & \vec{x}^\transp A S_1 \\
S_2A\vec{v} & S_2A^2\vec{v} & S_2AS_1
\end{bmatrix}  = \begin{bmatrix}
0 & \star \\ \star & S_2AS_1
\end{bmatrix}. \]
Now $S_2AS_1$ has trace zero and we can repeat the argument. By induction we can make all elements on the diagonal zero.
\end{proof}



\begin{proposition}
The trace of a matrix of a linear map is independent of the choice of basis:
\[ \Tr(L)_{\beta}^{\beta} = \Tr(L)_{\beta'}^{\beta'}. \]
\end{proposition}
\begin{proof}
\[ \Tr(L)_{\beta'}^{\beta'} = \Tr\left[((I)_{\beta'}^{\beta})^{-1}(L)_\beta^\beta(I)_{\beta'}^{\beta}\right] = \Tr\left[(L)_\beta^\beta(I)_{\beta'}^{\beta}((I)_{\beta'}^{\beta})^{-1}\right] = \Tr(L)_\beta^\beta \]
\end{proof}
This allows us to make the following definition:
\begin{definition}
The \udef{trace} of a linear map on a finite-dimensional vector space is the trace of any matrix representation of that map.
\end{definition}
\subsection{The determinant}
\begin{definition}
A map
\[ f: \mathbb{F}^{n\times n}\to \mathbb{F} \]
with the following properties
\begin{itemize}
\item[\textbf{D-1}] $f(\mathbb{1}_n) = 1$;
\item[\textbf{D-2}] $f(A)$ changes sign if two rows in $A$ are swapped;
\item[\textbf{D-3}] $f$ is linear in the first row:
\[ f(\begin{pmatrix}
\lambda A_{1,-} + \mu A'_{1,-} \\ A_{2,-} \\ \vdots \\ A_{n,-} 
\end{pmatrix}) = \lambda f(\begin{pmatrix}
A_{1,-} \\ A_{2,-} \\ \vdots \\ A_{n,-} 
\end{pmatrix}) + \mu f(\begin{pmatrix}
A'_{1,-} \\ A_{2,-} \\ \vdots \\ A_{n,-} 
\end{pmatrix}); \]
\end{itemize}
is called a \udef{determinant map}.
\end{definition}
We will show that there exists one and only one determinant map. We are therefore justified in calling it the determinant.

The determinant of $A$ is often denoted $\det(A)$ or $|A|$.

\begin{lemma} \label{determinantProperties}
Let $f: \mathbb{F}^{n\times n}\to \mathbb{F}$ be a determinant map.
\begin{enumerate}
\item $f$ is linear in each row;
\item if a matrix $A$ has a row of zeros, or two identical rows, then $f(A) = 0$;
\item the ERO $R_i \to R_i+ \lambda R_j$ does not change the determinant;
\item the ERO $R_i \leftrightarrow R_j$ changes the sign of the determinant;
\item the ERO $R_i \to \lambda R_i$ multiplies the determinant by $\lambda$;
\item $f(A)$ is non-zero \textup{if and only if} $A\sim_R \mathbb{1}_n$;
\item $f(A)$ is non-zero \textup{if and only if} $A$ is invertible.
\end{enumerate}
\end{lemma}

\subsubsection{Leibniz formula}
\begin{proposition}[Liebniz formula] \label{LiebnizFormula}
There is one and only one determinant map. It is given by
\[ \det: \F^{n\times n}\to \F: A\mapsto \sum_{\sigma\in S^n}\sgn(\sigma)\prod_{i=1}^n[A]_{i,\sigma(i)}. \]
\end{proposition}
\begin{proof}
Let $f$ be a determinant map and $\mathcal{E} = \{\vec{e}_i\}_{i=1}^n$ the standard basis of $\F^n$, considered as rows and $A\in\F^{n\times n}$. Then
\[ f(A) = f(\begin{pmatrix}
\sum_{j_1=1}^n [A]_{1j_1}\vec{e}_{j_1} \\ \vdots \\ \sum_{j_n=1}^n [A]_{nj_n}\vec{e}_{j_n}
\end{pmatrix}) = \sum_{j_1,\ldots, j_n=1}[A]_{1j_1}\ldots[A]_{nj_n}f(\begin{pmatrix}
\vec{e}_{j_1} \\ \vdots \\ \vec{e}_{j_n}
\end{pmatrix}). \]
Now $f(\begin{pmatrix}
\vec{e}_{j_1} \\ \vdots \\ \vec{e}_{j_n}
\end{pmatrix})$ is only non-zero if all $j_i$ are different, i.e.\ $j_i = \sigma(i)$ for some permutation $\sigma\in S^n$. Also
\[ f(\begin{pmatrix}
\vec{e}_{\sigma(1)} \\ \vdots \\ \vec{e}_{\sigma(n)}
\end{pmatrix}) = \sgn(\sigma)f(\mathbb{1}_n) = \sgn(\sigma). \]
So the only possible candidate for a determinant map is
\[ f(A) = \sum_{\sigma\in S^n}\sgn(\sigma)\prod_{i=1}^n[A]_{i,\sigma(i)}. \]
This satisfies \textbf{D-1}, \textbf{D-2} and \textbf{D-3}.
\end{proof}
In the case of $3\times 3$ the Leibniz formula reduces to
\[ \begin{vmatrix}
a_{11} & a_{12} & a_{13} \\
a_{21} & a_{22} & a_{23} \\
a_{31} & a_{32} & a_{33}
\end{vmatrix} = a_{11}a_{22}a_{33} + a_{12}a_{23}a_{31} + a_{13}a_{21}a_{32} - a_{31}a_{22}a_{13} - a_{32}a_{23}a_{11} - a_{33}a_{23}a_{12}. \]
The rule of Sarrus is the following mnemonic:
\begin{corollary}[Rule of Sarrus]
The determinant of a $3\times 3$ matrix can be seen as the sum of the following diagonals (with the correct sign):
\[ \begin{tikzpicture}
\node (A) {$\begin{vmatrix}
a_{11} & a_{12} & a_{13} \\
a_{21} & a_{22} & a_{23} \\
a_{31} & a_{32} & a_{33}
\end{vmatrix}$};
\node[right of=A, node distance=5.8em] (B) {$\begin{matrix}
a_{11} & a_{12} \\
a_{21} & a_{22} \\
a_{31} & a_{32}
\end{matrix}$};
\draw (A.south west) -- ++(7.5em,4em) ++(.2,.1) node {--};
\draw (A.south west) ++(2em,0) --  ++(7.5em,4em) ++(.2,.1) node {--};
\draw (A.south west) ++(4em,0) --  ++(7.5em,4em) ++(.2,.1) node {--};
\draw (A.north west) -- ++(7.5em,-4em) ++(.2,-.1) node {+};
\draw (A.north west) ++(2em,0) --  ++(7.5em,-4em) ++(.2,-.1) node {+};
\draw (A.north west) ++(4em,0) --  ++(7.5em,-4em) ++(.2,-.1) node {+};
\end{tikzpicture} \]
\end{corollary}
\begin{corollary} \label{determinantBound}
Let $A\in \F^{n\times n}$. Then $|\det(A)| \leq n! \norm{A}^n$.
\end{corollary}
\begin{proof}
We calculate using the Liebniz formula, \ref{matrixElementsBoundedByNorm} and the fact that $\#(S^n) = n!$ (TODO ref),
\begin{align*}
|\det(A)| &= \Big|\sum_{\sigma\in S^n}\sgn(\sigma)\prod_{i=1}^n[A]_{i,\sigma(i)}\Big| \\
&\leq \sum_{\sigma\in S^n}\Big|\prod_{i=1}^n[A]_{i,\sigma(i)}\Big| \\
&= \sum_{\sigma\in S^n}\prod_{i=1}^n\big|[A]_{i,\sigma(i)}\big| \\
&\leq \sum_{\sigma\in S^n}\prod_{i=1}^n\norm{A} \\
&= \sum_{\sigma\in S^n}\norm{A}^n \\
&= n!\norm{A}^n.
\end{align*}
\end{proof}
\begin{corollary}
The determinant function $\det: \F^{n\times n}\to \F$ is continuous.
\end{corollary}

\begin{proposition} \label{expansionDeterminantAroundIdentity}
Let $A\in \F^{n\times n}$ and $\lambda \in \F$. Then
\[ \det(\mathbb{1}_n + \lambda A) = 1 + \lambda \Tr(A) + \lambda^2P(\lambda), \]
where $P(\lambda)$ is a polynomial in $\lambda$ such that $|P(0)|\leq n(n-1)\norm{A}^2$ and $|P(1)| \leq 1 + n\norm{A} + n!\big(1 + \norm{A}\big)^n$.
\end{proposition}
\begin{proof}
We expand $\det(\mathbb{1}_n + \lambda A)$ using the Liebniz formula \ref{LiebnizFormula}:
\begin{align*}
\det(\mathbb{1}_n + \lambda A) &= \sum_{\sigma\in S^n}\sgn(\sigma)\prod_{i=1}^n[\mathbb{1}_n + \lambda A]_{i,\sigma(i)} \\
&= \sum_{\sigma\in S^n}\sgn(\sigma)\prod_{i=1}^n\big(\delta_{i,\sigma(i)} + \lambda [A]_{i,\sigma(i)}\big) \\
&= \prod_{i=1}^n\big(\delta_{i,i} + \lambda [A]_{i,i}\big) + \sum_{\sigma\in S^n\setminus\{\id\}}\sgn(\sigma)\prod_{i=1}^n\big(\delta_{i,\sigma(i)} + \lambda [A]_{i,\sigma(i)}\big).
\end{align*}
Now we expand the first part using \ref{productOfSum}:
\begin{align*}
\prod_{i=1}^n\big(\delta_{i,i} + \lambda [A]_{i,i}\big) &= \prod_{i=1}^n\big(1 + \lambda [A]_{i,i}\big)\\
&= \sum_{\seq{s_i}\in \{0,1\}^n}\prod_{i=0}^{n-1}\bigg(k\mapsto \begin{cases}
1 & (k=0) \\ \lambda[A]_{i,i} & (k=1)
\end{cases}\bigg)(s_i) \\
&= 1 + \lambda \sum_{i=0}^{n-1}[A]_{i,i} + \sum_{\seq{s_i}\in C}\prod_{i=0}^{n-1}\bigg(k\mapsto \begin{cases}
1 & (k=0) \\ \lambda [A]_{i,i} & (k=1)
\end{cases}\bigg)(s_i) \\
&= 1 + \lambda \Tr(A) + Q(\lambda),
\end{align*}
where $C$ is the subset of $\{0,1\}^n$ of strings that contains at least two ones and $Q(\lambda) \defeq \sum_{\seq{s_i}\in C}\prod_{i=0}^{n-1}\Big(k\mapsto \begin{cases}
1 & (k=0) \\ \lambda[A]_{i,i} & (k=1)
\end{cases}\Big)(s_i)$. Thus, by construction, every term of $Q(\lambda)$ is divisible by $\lambda^2$, so $Q(\lambda) = \lambda^2P'(\lambda)$ for some polynomial $P'(\lambda)$.

Now returning to $\sum_{\sigma\in S^n\setminus\{\id\}}\sgn(\sigma)\prod_{i=1}^n\big(\delta_{i,\sigma(i)} + \lambda [A]_{i,\sigma(i)}\big)$, note for every $\sigma\in S^n\setminus\{\id\}$ there must exist at least two distinct $i,j\in \interval[co]{0,n}$ such that $i\neq \sigma(i)$ and $j\neq \sigma(j)$ (TODO ref). The corresponding factors in $\prod_{i=1}^n\big(\delta_{i,\sigma(i)} + \lambda [A]_{i,\sigma(i)}\big)$ are divisible by $\lambda$, so the sum is a polynomial divisible by $\lambda^2$. We write $\lambda^2P^{\prime\prime}(\lambda) = \sum_{\sigma\in S^n\setminus\{\id\}}\sgn(\sigma)\prod_{i=1}^n\big(\delta_{i,\sigma(i)} + \lambda [A]_{i,\sigma(i)}\big)$. Now we define $P(\lambda) \defeq P'(\lambda) + P^{\prime\prime}(\lambda)$.

Finally, we compute the bounds of $P(0)$ and $P(1)$. For $P(0)$, if $\sigma\in S^n$ changes the place of more than two elements, then $\prod_{i=1}^n\big(\delta_{i,\sigma(i)} + \lambda [A]_{i,\sigma(i)}\big)$ contains at least three factors of $\lambda$. Even dividing out two, the contribution to $P^{\prime\prime}(0)$ is zero. If it changes the place of only two elements $i,j$ (i.e.\ it is a transposition), then it is equal to $\big|[A]_{i,j}[A]_{j,i}\big| \leq \norm{A}^2$ by \ref{matrixElementsBoundedByNorm}. Since there are $\frac{n(n-1)}{2}$ transpositions, there are $\frac{n(n-1)}{2}$ such terms. By a similar reasoning, we bound $\prod_{i=0}^{n-1}\Big(k\mapsto \begin{cases}
1 & (k=0) \\ \lambda[A]_{i,i} & (k=1)
\end{cases}\Big)(s_i)$ by $\big|[A]_{i,i}[A]_{j,j}\big|$ if $s$ has ones only at locations $i,j$. Otherwise the corresponding term is zero. Thus for $P'(0)$ we also get a bound of $\frac{n(n-1)}{2}\norm{A}^2$. Adding these together gives $n(n-1)\norm{A}^2$.

For $P(1)$, we have $P(1) = \det(\mathbb{1}_n + A) - 1 - \Tr(A)$, so using \ref{matrixElementsBoundedByNorm} and \ref{determinantBound}, we have
\begin{align*}
\big|P(1)\big| &\leq 1 + \big|\Tr(A)\big| + \big|\det(\mathbb{1}_n + A)\big| \\
&\leq 1 + n\norm{A} + n!\norm{\mathbb{1}_n + A}^n \\
&\leq 1 + n\norm{A} + n!\big(1 + \norm{A}\big)^n.
\end{align*}
\end{proof}

\subsubsection{Laplace expansion}
\begin{definition}
Let $A\in\F^{n\times n}$ be a square matrix. A \udef{minor determinant} or \udef{minor} is the determinant of a submatrix. In particular the \udef{$(i,j)$-minor} $M_{i,j}$ is the minor
\[ M_{i,j} \defeq \det([A]_{(1:n)\setminus\{i\},(1:n)\setminus\{j\}}). \]

The \udef{$(i,j)$-cofactor} $C_{i,j}$ is defined as
\[ C_{i,j} \defeq (-1)^{i+j}M_{i,j}. \]
\end{definition}
\begin{proposition}[Laplace expansion] \label{LaplaceExpansion}
Let $A\in\F^{n\times n}$. Then
\[ \det(A) = \sum_{i=1}^n [A]_{i,j}C_{i,j} = \sum_{i=1}^n [A]_{j,i}C_{j,i} \]
for all $j\in 1:n$.
\end{proposition}
This is also known as ``expansion along the $j^\text{th}$ column'', resp., row.
\begin{proof}
Fix some $j\in 1:n$. Then we can calculate
\begin{align*}
\det(A) &=  \sum_{\sigma\in S^n}\sgn(\sigma)\prod_{k=1}^n[A]_{k,\sigma(k)} = \sum_{i=1}^n\sum_{\substack{\sigma\in S^n \\ \sigma(i)=j}}\sgn(\sigma)\prod_{k=1}^n[A]_{k,\sigma(k)} \\
&= \sum_{i=1}^n [A]_{i,j}\sum_{\substack{\sigma\in S^n \\ \sigma(i)=j}}\sgn(\sigma)\prod_{k\in (1:n)\setminus\{j\}}^n[A]_{k,\sigma(k)} = \sum_{i=1}^n [A]_{i,j}C_{i,j}.
\end{align*}
The expression for column expansion can be obtained by transposition.
\end{proof}

\subsubsection{Volume}

\subsubsection{Properties}
\begin{lemma}
Let $A\in\F^{n\times n}$. Then
\[ \det(A^\transp) = \det(A). \]
\end{lemma}
\begin{proof}
We calculate using the Leibniz formula:
\begin{align*}
\det(A^\transp) &= \sum_{\sigma\in S^n}\sgn(\sigma)\prod_{i=1}^n[A^\transp]_{i,\sigma(i)} = \sum_{\sigma\in S^n}\sgn(\sigma)\prod_{i=1}^n[A]_{\sigma(i),i} \\
&= \sum_{\sigma\in S^n}\sgn(\sigma)\prod_{i=1}^n[A]_{i,\sigma^{-1}(i)} = \sum_{\sigma\in S^n}\sgn(\sigma^{-1})\prod_{i=1}^n[A]_{i,\sigma^{-1}(i)} = \det(A).
\end{align*}
\end{proof}

\begin{lemma}
Let $A\in\F^{n\times n}$. Then
\begin{enumerate}
\item $\det(\lambda A) = \lambda^n\det(A)$;
\item $\det(\overline{A}) = \overline{\det(A)}$;
\item $\det(A^*) = \overline{\det(A)}$;
\item if $A$ is triangular, then
\[ \det(A) = \prod_{i=1}^n [A]_{i,i}. \]
\end{enumerate}
\end{lemma}

\begin{proposition}
Let $A,B\in\F^{n\times n}$. Then
\[ \det(AB) = \det(A)\det(B). \]
\end{proposition}
\begin{proof}
First assume $\det(B) = 0$. Then $AB$ is not invertible, for if $AB$ were invertible, $B$ would have inverse $(AB)^{-1}A$. So $\det(AB) = 0$ and the formula holds.

Now assume $\det(B) \neq 0$, then $A\mapsto \det(AB)/\det(B)$ satisfies the definition of a determinant map and thus is equal to $\det$.
\end{proof}
\begin{corollary}
If $U$ is unitary, then $|\det(U)| = 1$.
\end{corollary}
\begin{proof}
$1=\det{\mathbb{1}} = \det(U^*U) = \det(U^*)\det(U) = \overline{\det(U)}\det(U) = |\det(U)|$.
\end{proof}
\begin{corollary}
Let $A,B\in\F^{n\times n}$. Then
\[ \det(AB) = \det(BA). \]
In fact we can arbitrarily commute any matrix product inside the determinant.
\end{corollary}
\begin{corollary}
The determinant of a matrix of a linear map is independent of the choice of basis:
\[ \det(L)_{\beta}^{\beta} = \det(L)_{\beta'}^{\beta'}. \]
\end{corollary}
This allows us to make the following definition:
\begin{definition}
The \udef{determinant} of a linear map on a finite-dimensional vector space is the determinant of any matrix representation of that map.
\end{definition}

\begin{lemma}
Let $A\in\F^{m\times m}$ and $n\in\N$. Then
\[ \det\begin{bmatrix}
\mathbb{1}_n & \mathbb{0} \\ \mathbb{0} & A
\end{bmatrix} = \det(A) = \det\begin{bmatrix}
A & \mathbb{0} \\ \mathbb{0} & \mathbb{1}_n
\end{bmatrix}. \]
\end{lemma}
\begin{proof}
By induction on $n$.
\end{proof}

\begin{lemma}
Let $A,B,C$ be conformal matrices and $A,D$ square. Then
\[ \det\begin{bmatrix}A & B \\ \mathbb{0} & D \end{bmatrix} = \det(A)\det(D). \]
\end{lemma}
\begin{proof}
This follows from
\[ \begin{bmatrix}A & B \\ \mathbb{0} & D \end{bmatrix} = \begin{bmatrix}\mathbb{1} & \mathbb{0} \\ \mathbb{0} & D \end{bmatrix} \begin{bmatrix}\mathbb{1} & B \\ \mathbb{0} & \mathbb{1} \end{bmatrix} \begin{bmatrix}A & \mathbb{0} \\ \mathbb{0} & \mathbb{1} \end{bmatrix}, \]
the product rule, the previous lemma and the fact that the central matrix is triangular with only ones on the diagonal.
\end{proof}

\begin{lemma} \label{blockDeterminant}
Let $M = \begin{pmatrix}
A & B \\ C & D
\end{pmatrix}$ be a partitioned matrix. Then
\begin{align*}
\det(M) &= \det(A)\det(D-CA^{-1}B) \\
&= \det(D)\det(A-BD^{-1}C)
\end{align*}
if $A$, resp. $D$, is invertible.
\end{lemma}
\begin{proof}
This follows straight from the Schur complement.
\end{proof}
\begin{corollary}[Cauchy expansion of the determinant]
Let $A\in\F^{n\times n}, \vec{x},\vec{y}\in\F^n$ and $c\in \F$. Then
\[ \det \begin{bmatrix}
c & \vec{x}^\transp \\ \vec{y} & A
\end{bmatrix} = (c-\vec{x}^\transp A^{-1}\vec{y})\det(A) = c\det(A) - \vec{x}^\transp \adj(A) \vec{y}. \]
\end{corollary}
\begin{corollary}
Let $A,B,C,D\in\F^{n\times n}$.
\begin{enumerate}
\item If $A$ or $D$ is invertible and commutes with $B$, then
\[ \det\begin{bmatrix}
A & B \\ C & D
\end{bmatrix} = \det(DA-CB). \]
\item If $A$ or $D$ is invertible and commutes with $C$, then
\[ \det\begin{bmatrix}
A & B \\ C & D
\end{bmatrix} = \det(AD-BC). \]
\end{enumerate}
\end{corollary}

\begin{lemma}
Let $a,b\in\R$. Then
\[ \det(a\mathbb{1}_n+b\mathbb{J}_n) = a^{n-1}(a+nb). \]
\end{lemma}
\begin{proof}
We can partition $a\mathbb{1}_n+b\mathbb{J}_n$ and use the ERO $R_i\to R_i-R_1$ for $1<i\leq n$ to obtain:
\[ a\mathbb{1}_n+b\mathbb{J}_n = \begin{pmatrix}
a+b & b\mathbb{J}^{1\times (n-1)} \\
b\mathbb{J}^{(n-1)\times 1} & a\mathbb{1}_{n-1}+b\mathbb{J}_{n-1}
\end{pmatrix} = \begin{pmatrix}
a+b & b\mathbb{J}^{1\times (n-1)} \\
-a\mathbb{J}^{(n-1)\times 1} & a\mathbb{1}_{n-1}
\end{pmatrix}. \]
Then by \ref{blockDeterminant}, we have
\[ \det(a\mathbb{1}_n+b\mathbb{J}_n) = a^{n-1}(a+b +b\mathbb{J}^{1\times (n-1)}\mathbb{J}^{(n-1)\times 1}) = a^{n-1}(a+nb). \]
\end{proof}

\begin{lemma}[Weinstein-Aronszajn identity]
Let $A\in\F^{m\times n}$ and $B\in\F^{n\times m}$. Then
\[ \det(\mathbb{1}_{m} + AB) = \det(\mathbb{1}_{n} + BA). \]
Also, for any $\lambda\in\R_0$,
\[ \det(AB - \lambda\mathbb{1}_{m}) = (-\lambda)^{m-n}\det(BA - \lambda\mathbb{1}_{n}). \]
\end{lemma}
This is also sometimes referred to as the Sylvester determinant identity.
\begin{proof}
Applying the two equalities in \ref{blockDeterminant} to the matrix
\[ M = \begin{bmatrix}
\mathbb{1}_m & -A \\
B & \mathbb{1}_n
\end{bmatrix} \]
give
\begin{align*}
M &= \det(\mathbb{1}_m)\det(\mathbb{1}_n - B\mathbb{1}_m^{-1}(-A)) = \det(\mathbb{1}_n+BA) \\
&= \det(\mathbb{1}_n)\det(\mathbb{1}_m - (-A)\mathbb{1}_n^{-1}B) = \det(\mathbb{1}_m+AB).
\end{align*}
\end{proof}
\begin{corollary}[Matrix determinant lemma]
Let $A\in\F^{n\times n}$ and $\vec{u},\vec{v}\in\F^n$. Then
\begin{align*}
\det(A+\vec{u}\vec{v}^\transp) &= \det(A)\det(\mathbb{1}_n + A^{-1}\vec{u}\vec{v}^\transp) \\
&= \det(A)(\mathbb{1}_n + \vec{v}^\transp A^{-1}\vec{u}) \\
&= \det(A) + \vec{v}^\transp \adj(A)\vec{u},
\end{align*}
which is interesting because $\vec{v}^\transp A^{-1}\vec{u} \in\F$  and $\vec{v}^\transp \adj(A)\vec{u}\in\F$ are scalars.
\end{corollary}

TODO
\[ \log\det M=\Tr\log M \]


\subsection{Adjugate}
\begin{definition}
Let $A\in\F^{n\times n}$ be a square matrix. The \udef{adjugate matrix} or \udef{classical adjoint} $\adj(A)$ is the transposed cofactor matrix:
\[ [\adj(A)]_{ij} = C_{ji} \]
where $C_{ij}$ is the $(i,j)-$cofactor.
\end{definition}

\begin{lemma}
Let $A\in\F^{n\times n}$, $\vec{b}\in \F^n$ and $k\in(1:n)$. Then
\[ \det(\left[\begin{cases}
[A]_{ij} & (j\neq k) \\
[\vec{b}]_i & (j=k)
\end{cases}\right]) = [\adj(A)b]_k \]
\end{lemma}
\begin{proof}
We calculate
\[ [\adj(A)\vec{b}]_k = \sum_l [\adj(A)]_{kl}[\vec{b}]_l = \sum_l C_{lk}[\vec{b}]_l. \]
Now in the definition of $C_{lk}$, the $k^\text{th}$ column is excluded. So the $(l,k)-$cofactor of $A$ is the same as the $(l,k)-$cofactor of 
\[ \left[\begin{cases}
[A]_{ij} & (j\neq k) \\
[\vec{b}]_i & (j=k)
\end{cases}\right] \]
which is the matrix where the $k^\text{th}$ column of $A$ is replaced by $\vec{b}$. Then $\sum_l C_{lk}[\vec{b}]_l$ is the determinant of this matrix by \ref{LaplaceExpansion}.
\end{proof}

\begin{proposition} \label{adjunctDeterminant}
Let $A\in\F^{n\times n}$. Then
\[ A\cdot\adj(A) = \adj(A)\cdot A = \det(A) \mathbb{1}_n. \]
\end{proposition}
\begin{proof}
\[ [\adj(A)\cdot A]_{ij} = \sum_k [A]_{ik}C_{jk} \]
Clearly if $i=j$, we have the Laplace expansion \ref{LaplaceExpansion} and the expression equals $\det(A)$. If $i\neq j$, then the $j^\text{th}$ row does not enter into the expression (it is left out of the $(i,j)-$minor) and thus may just as well be replaced by a copy of the $i^\text{th}$ row. In this case we get the expression for the determinant of a matrix with two identical rows. This must be $0$.
\end{proof}
\begin{corollary} \label{inverseAdjunctDeterminant}
Let $A\in\F^{n\times n}$ be invertible. Then
\[ A^{-1} = \frac{1}{\det(A)}\adj(A). \]
\end{corollary}

\begin{proposition}
Let $A,B\in\F^{n\times n}$ be square matrices. Then
\begin{enumerate}
\item $\adj(\mathbb{0}_n) = \mathbb{0}_n$;
\item $\adj(\mathbb{1}_n) = \mathbb{1}_n$;
\item $\adj(\lambda A) = \lambda^{n-1}\adj(A)$ for any $\lambda\in \F$;
\item $\adj(A^\transp) = \adj(A)^\transp$;
\item $\det(\adj(A)) = \det(A)^{n-1}$;
\item $\adj(AB) = \adj(A)\adj(B)$.
\end{enumerate}
\end{proposition}
\begin{proof}
Points 2. and 5. are direct consequences of \ref{adjunctDeterminant}. TODO rest.
\end{proof}
Cauchy-Binet

\subsection{Generalised inverses or pseudoinverses}
\subsubsection{Moore-Penrose pseudoinverse}

\subsection{Pfaffian}

\subsection{Vectorisation}
For calculations it is often useful to put the matrix of coordinates into a column vector. The process of fitting a matrix into a column vector is known as the \udef{vectorisation} of a matrix. Two obvious ways to do this are by going row-by-row or column by column.
\begin{itemize}
\item Column-by-column we get
\[ \vectorisation_C: \R^{m\times n}\to \R^{mn\times 1}: A \mapsto \vectorisation_C(A) = [a_{1,1},\ldots,a_{m,1}, a_{1,2},\ldots, a_{m,2}, \;\; \ldots \;\; , a_{1,n}, \ldots, a_{m,n}]^\transp \]
\begin{example}
\[ \text{If} \qquad A = \begin{pmatrix}
a & b \\ c & d
\end{pmatrix}, \qquad \text{then} \qquad \vectorisation_C(A) = \begin{pmatrix}
a \\ c \\ b \\ d
\end{pmatrix}. \]
\end{example}
\item Row-by-row we get
\[ \vectorisation_R: \R^{m\times n}\to \R^{mn\times 1}: A \mapsto \vectorisation_R(A) = [a_{1,1},\ldots,a_{1,n}, a_{2,1},\ldots, a_{2,n}, \;\; \ldots \;\; , a_{m,1}, \ldots, a_{m,n}]^\transp \]
\begin{example}
\[ \text{If} \qquad A = \begin{pmatrix}
a & b \\ c & d
\end{pmatrix}, \qquad \text{then} \qquad \vectorisation_R(A) = \begin{pmatrix}
a \\ b \\ c \\ d
\end{pmatrix}. \]
\end{example}
\end{itemize}
Obviously these are related by
\[ \vectorisation_C(A) = \vectorisation_R(A^\transp) \]

Vectorisation is a self-adjunction in the monoidal closed structure of any category of matrices. (TODO)
\subsection{The Hadamard product}
\[ \vectorisation(A\circ B) = \vectorisation(A) \circ \vectorisation(B) \]
This works for both $\vectorisation_C$ and $\vectorisation_R$.
\subsection{The outer product}
Given two column vectors $\vec{u} = [u_1 \hdots u_m]^\transp$ and $\vec{v} = [v_1 \hdots v_n]^\transp$, the \udef{outer product} is defined by
\[ \vec{u}\otimes \vec{v} = \vec{u}\vec{v}^\transp = \begin{bmatrix}
u_1 \\ \vdots \\ u_m
\end{bmatrix} \begin{bmatrix}
v_1 & \hdots & v_n
\end{bmatrix} = \begin{bmatrix}
u_1v_1 & \hdots & u_1v_n \\
\vdots & \ddots & \vdots \\
u_mv_1 & \hdots & u_m v_n
\end{bmatrix} \]
\subsection{The Kronecker product}
Given two matrices $A,B$ of dimensions $m\times n$ and $p\times q$, the \udef{Kronecker product} yields a $(pm\times qn)$-matrix:
\[ A\otimes B = \begin{bmatrix}
a_{1,1}B & \hdots & a_{1,n}B \\
\vdots & \ddots & \vdots \\
a_{m,1}B & \hdots & a_{m,n}B
\end{bmatrix} \]

The same symbol is used as for the outer product, because the Kronecker product can be seen as a generalisation of the outer product. Indeed for column vectors $\vec{u}, \vec{v}$ we have the identities
\begin{align}
\vec{u}\otimes_{\text{Kron}} \vec{v} &= \vectorisation_R(\vec{u}\otimes_\text{outer} \vec{v}) \\
&= \vectorisation_C(\vec{v}\otimes_\text{outer} \vec{u})
\end{align}
and \[ \vec{u}\otimes_\text{outer}\vec{v} = \vec{u}\otimes_{\text{Kron}} \vec{v}^\transp \]

In terms of matrix elements we can write, assuming the indices start at zero
\[ (A\otimes B)_{i,j} = a_{\lfloor{i/p}\rfloor, \lfloor{j/q}\rfloor}b_{i\%p,j\%q}. \]
where $\%$ denotes the remainder. For indices starting from 1, we have
\[ (A\otimes B)_{i,j} = a_{\lceil{i/p}\rceil, \lceil{j/q}\rceil}b_{(i-1)\%p+1,(j-1)\%q+1}. \]

\subsubsection{Properties}
The Kronecker product is bilinear and associative.
\begin{itemize}
\item[\textbf{Transpose}]
\[ (A\otimes B)^\transp = A^\transp \otimes B^\transp \]
\item[\textbf{Determinant}]
\[ \det(A\otimes B) = \det(A)^m\det(B)^n \]
if $A$ is an $n\times n$ matrix and $B$ an $m\times m$ matrix.
\item[\textbf{Trace}]
\[ \Tr(A\otimes B) = \Tr(A)\Tr(B) \]
\item[\textbf{Mixed product}]
Let $A,B,C,D$ be conformal matrices, then
\[ (A\otimes B)(C \otimes D) = (AC)\otimes (BD). \]
The proof is as follows:
\begin{align}
(A\otimes B)(C \otimes D) &= \begin{bmatrix}
a_{1,1}B & \hdots & a_{1,n}B \\
\vdots & \ddots & \vdots \\
a_{m,1}B & \hdots & a_{m,n}B
\end{bmatrix}\begin{bmatrix}
c_{1,1}D & \hdots & c_{1,n}D \\
\vdots & \ddots & \vdots \\
c_{m,1}D & \hdots & c_{m,n}D
\end{bmatrix} \\
&= \begin{bmatrix}
\sum_k a_{1,k}c_{k,1} BD & \hdots & \sum_k a_{1,k}c_{k,p} BD \\
\vdots & \ddots & \vdots \\
\sum_k a_{m,k}c_{k,1} BD & \hdots & \sum_k a_{m,k}c_{k,p} BD
\end{bmatrix}  =  \begin{bmatrix}
(AC)_{1,1} BD & \hdots & (AC)_{1,p} BD \\
\vdots & \ddots & \vdots \\
(AC)_{m,1} BD & \hdots & (AC)_{m,p} BD
\end{bmatrix}\\
&= (AC)\otimes (BD)
\end{align}
Using the fact that multiplication of two block matrices can be carried out as if their blocks were scalars.

As an immediate consequence:
\[ A\otimes B = (I_n\otimes B)(A\otimes I_k) = (A\otimes I_k)(I_n \otimes B). \]

\item[\textbf{Inverse}]
The product $A\otimes B$ is invertible iff $A$ and $B$ are invertible. In that case the inverse is given by
\[ (A\otimes B)^{-1} = A^{-1}\otimes B^{-1}. \]
This follows easily from the mixed product.
\item[\textbf{Moore-Penrose pseudoinverse}]
\[ (A\otimes B)^+ = A^+\otimes B^+. \]

\item[\textbf{A vectorisation trick}]
Let $A,B,C$ be matrices of dimensions $k\times l, l\times m$ and $m\times n$. Then
\[ \vectorisation(ABC) = (C^\transp\otimes A)\vectorisation(B). \]
From this we obtain some other formulations:
\begin{align}
\vectorisation(ABC) &= (I_n\otimes AB)\vectorisation(C) \\
&= (C^\transp B^\transp \otimes I_k)\vectorisation(A) \\
\vectorisation(AB) &= (I_m\otimes A)\vectorisation(B) = (B^\transp \otimes I_k)\vectorisation(A)
\end{align}
\end{itemize}

\subsection{The commutator}
\begin{theorem}[Shoda's theorem]
Let $A\in\F^{n\times n}$. Then there exists $X,Y$ such that $A=[X,Y] = XY-YX$ \textup{if and only if} $\Tr(A) = 0$.
\end{theorem}
\begin{proof}
Assume $A = [X,Y]$. Then $\Tr(A) = \Tr(XY) -\Tr(YX) = \Tr(XY) - \Tr(XY) = 0$.

Conversely, assume $\Tr(A) = 0$. By \ref{averageTraceOverDiagonal} $A$ is similar to a matrix $B$ that has zeros on the diagonal. Let $X = \diag(1,2,\ldots, n)$ and
\[ [Y]_{i,j} = \begin{cases}
(i-j)^{-1}[B]_{i,j} & (i\neq j) \\
1 & (i=j).
\end{cases} \]
Then
\[ [X,Y]_{i,j} = [XY-YX]_{i,j} = i[Y]_{i,j}-j[Y]_{i,j} = (i-j)[Y]_{i,j} = [B]_{i,j}. \]
So $B$ is the commutator $[X,Y]$. Then $A$ is the commutator $[SXS^{-1}, SYS^{-1}]$.
\end{proof}

\subsection{Kruskal rank and spark}
\begin{definition}
Let $A\in \F^{m\times n}$. The \udef{Kruskal rank} is the largest number $\KruskalRank(A)$ such that all $\KruskalRank(A)$-sets of columns are linearly independent.
\end{definition}
\begin{lemma}
Let $A\in \F^{m\times n}$. Then $\KruskalRank(A) \leq \Rank(A)$.
\end{lemma}
\begin{proof}
If $\Rank(A) = k$, then there exists a $k$-set of columns that spans $\Col(A)$. Then every linearly independent set is smaller than $k$ by the Steinitz exchange lemma \ref{SteinitzExchange} and $\KruskalRank(A) \leq k$.
\end{proof}

\begin{definition}
Let $A\in \F^{m\times n}$. Then the \udef{spark} of $A$ is defined as
\[ \operatorname{spark}(A) \defeq \min\setbuilder{\norm{x}_0}{x\in\Null(A)} \]
where $\norm{x}_0$ is the number of non-zero elements of $x$.
\end{definition}
TODO $\norm{\cdot}_0$ norm for finite fields.

\begin{lemma}
Let $A\in \F^{m\times n}$. Then
\[ \operatorname{spark}(A) = \KruskalRank(A) + 1. \]
\end{lemma}
\begin{proof}
TODO
\end{proof}

\begin{lemma}
Let $A\in \F^{m\times n}$. If $\Rank(A) = n$, then $\KruskalRank(A) = n$.
\end{lemma}

\begin{proposition}
Let $A\in \F^{m\times n}$. Then
\[ \KruskalRank(A) \geq \frac{1}{\mu(A)} \]
where $\mu(A) = \max_{i\neq j} \frac{|\inner{[A]_{_,i},[A]_{-,j}}|}{\norm{[A]_{_,i}}\norm{[A]_{_,j}}}$.
\end{proposition}
\begin{proof}
TODO
\end{proof}


\section{Eigenvalues and eigenvectors}
\subsection{The spectrum}
In this section we study vectors that are mapped to multiples of themselves by a given matrix $A$, i.e.\ vectors $\vec{v}$ such that
\[ A\vec{v} = \lambda \vec{v} \qquad\text{for some $\lambda\in\F$.} \]
Clearly for this to be possible, $A$ needs to be square.
\begin{definition}
Suppose $A\in \F^{n\times n}$.
\begin{itemize}
\item  A scalar $\lambda\in \mathbb{F}$ is called an \udef{eigenvalue} of $A$ if there exists a $\vec{v}\in \F^n$ such that $\vec{v}\neq 0$ and $A\vec{v} = \lambda v$.
\item Such a vector $\vec{v}$ is called an \udef{eigenvector}.
\item The set of all eigenvectors associated with an eigenvalue $\lambda$ is called the \udef{eigenspace} $E_\lambda(A)$. Because
\[ E_\lambda(A) = \ker(\ell_{\lambda \mathbb{1}_{n} - A}), \]
it is indeed a vector space.

The dimension of $E_\lambda(A)$ is the \udef{geometric multiplicity} of $\lambda$.
\end{itemize}
The set of all eigenvalues is called the \udef{spectrum} of $A$.
\end{definition}
\begin{proposition}
Let $A\in \F^{n\times n}$ and $\lambda\in \mathbb{F}$, then
\[ \text{$\lambda$ is an eigenvalue of $A$} \qquad \iff \qquad \text{$\lambda \mathbb{1}_{n} - A$ is invertible.} \]
\end{proposition}
\begin{proof}
The equation $A\vec{v} = \lambda \vec{v}$ is equivalent to $(A-\lambda \mathbb{1}_n)\vec{v} = 0$. So there exist eigenvectors associated to $\lambda$ iff the kernel of $\ell_{A-\lambda \mathbb{1}_n}$ is not trivial iff $\ell_{A-\lambda \mathbb{1}_n}$ is injective (\ref{injectivityKernelTriviality}) iff $\ell_{A-\lambda \mathbb{1}_n}$ is invertible (\ref{invertibleFiniteDim}) iff $A-\lambda \mathbb{1}_n$ is invertible (\ref{invertibleMapInvertibleMatrix}).
\end{proof}

\begin{proposition}[Gerschgorin circle theorem]
Let $A\in \F^{n\times n}$. If $\lambda$ is an eigenvalue of $A$, then there is an $i\in 1:n$ such that
\[ |\lambda - [A]_{ii}| \leq \sum_{\substack{j\in 1:n \\ j \neq i}}|[A]_{ij}|. \]
\end{proposition}
\begin{proof}
If $|\lambda - [A]_{ii}| > \sum_{\substack{j\in 1:n \\ j \neq i}}|[A]_{ij}|$ for all $i$, then $(A-\lambda\mathbb{1})$ is strictly diagonally dominant and thus invertible by \ref{invertibleDiagonallyDominant}.
\end{proof}

\begin{proposition}
Let $A\in\F^{n\times n}$ be a matrix. Suppose $\lambda_1, \ldots, \lambda_m$ are distinct eigenvalues of $A$ and $\vec{v}_1,\ldots, \vec{v}_m$ are corresponding eigenvectors. Then $\{\vec{v}_1,\ldots, \vec{v}_m\}$ is linearly independent.
\end{proposition}
\begin{proof}
The proof goes by contradiction. Assume $\{\vec{v}_1,\ldots, \vec{v}_m\}$ is linearly dependent. Let $k$ be the smallest positive integer such that
\[ \vec{v}_k \in \Span\{\vec{v}_1,\ldots, \vec{v}_{k-1}\}. \]
So there exists a nontrivial linear combination
\[ \vec{v}_k = a_1\vec{v}_1+\ldots +a_{k-1}\vec{v}_{k-1}. \]
Multiplying by $A$ gives
\[ \lambda_k\vec{v}_k = a_1\lambda_k\vec{v}_1+\ldots +a_{k-1}\lambda_k\vec{v}_{k-1}. \]
Multiplying the previous combination by $\lambda_k$ and subtracting both equations gives
\[ 0= a_1(\lambda_k-\lambda_1)\vec{v}_1 +\ldots + a_{k-1}(\lambda_k - \lambda_{k-1})\vec{v}_{k-1}. \]
By assumption of linear independence of $\{\vec{v}_1,\ldots, \vec{v}_{k-1}\}$ this combination must be trivial, however none of the $(\lambda_k-\lambda_i)$ can be zero, so all the $a_i$ must be zero. This is a contradiction with the assumption of linear dependence.
\end{proof}
\begin{corollary}
The matrix $A\in\F^{n\times n}$ has at most $n$ linearly independent eigenvalues.
\end{corollary}
\begin{corollary}
Suppose $\lambda_1, \ldots, \lambda_m$ are distinct eigenvalues of $A$. Then
\[ E_{\lambda_1}(A) \oplus \ldots \oplus E_{\lambda_m}(A) \]
is a direct sum. Furthermore, the sum of geometric multiplicities is less than or equal to the dimension of $V$:
\[ \dim E_{\lambda_1}(A) + \ldots + \dim E_{\lambda_m}(A) \leq \dim V. \]
\end{corollary}

\subsubsection{The characteristic equation}
\begin{definition}
Let $A\in\F^{n\times n}$. The \udef{characteristic polynomial} $p_A(x)$ of $A$. Is the polynomial
\[ p_A(x) \defeq \det(x\mathbb{1}_n - A). \]
\end{definition}
The characteristic polynomial is also sometimes defined as $\det(A - x\mathbb{1}_n)$. This differs by a sign $(-1)^{n}$.
\begin{lemma}
The characteristic polynomial of any square matrix is a monic polynomial.
\end{lemma}
\begin{lemma}
The characteristic polynomials of similar matrices are identical.
\end{lemma}

\begin{proposition}
Let $A\in\F^{n\times n}$. Then $p_A(x)$ can be factorised as
\[ p_A(x) = \prod_{i=1}^m (x - \lambda_i)^{m_i}  \]
where $\lambda_i$ are the eigenvalues of $A$ and the multiplicities $m_i$ are positive integers such that $\sum_{i=1}^m m_i = n$.
\end{proposition}
\begin{corollary}
The eigenvalues of $A$ are the solutions of the equation
\[ p_A(x) = 0. \]
\end{corollary}
\begin{corollary}
The determinant of a matrix is the product of its eigenvalues, counting algebraic multiplicity: for $A\in\F^{n\times n}$
\[ \det(A) = \prod_{i=1}^m\lambda_i^{m_i}. \]
\end{corollary}
\begin{proof}
We have
\[ \det(A) = (-1)^n\det(0-A) = (-1)^np_A(0) = (-1)^n\prod_{i=1}^m(0 - \lambda_i)^{m_i} = \prod_{i=1}^m\lambda_i^{m_i}. \]
\end{proof}

\begin{definition}
Let $A\in\F^{n\times n}$. The equation
\[ p_A(x) = \det(x\mathbb{1}_n - A) = 0 \]
is called the \udef{characteristic equation} of $A$.

Let $\lambda$ be a solution of the characteristic equation. The multiplicity of $\lambda$ as a root of $p_A(x)$ is the \udef{algebraic multiplicity} of $\lambda$.
\end{definition}
\begin{lemma}
Let $A\in\F^{n\times n}$ and $\lambda$ be an eigenvalue of $A$.

The geometric multiplicity of $\lambda$ is less than or equal to the algebraic multiplicity of $\lambda$.
\end{lemma}
\begin{proof}
Set $k=\dim E_\lambda$. Take a basis of $E_\lambda(A)$ and extend it to a basis $\beta$ of $\F^{n}$. With respect to this basis the matrix of $\ell_A$ is of the form
\[ (\ell_A)_\beta^\beta =  \begin{pmatrix}
\lambda \mathbb{1}_k & B \\ 0 & C
\end{pmatrix} = P^{-1}(\ell_A)_\mathcal{E}^\mathcal{E}P = P^{-1}AP \]
for some matrices $B,C$ and some invertible matrix $P$ where $\mathcal{E}$ is the standard basis of $\F^n$. Then 
\[ p_A(x)= p_{P^{-1}AP} = p_{\lambda \mathbb{1}_{k}}(x)p_C(x) = (\lambda - x)^kp_C(x), \]
so the algebraic multiplicity of $\lambda$ is at least the geometric multiplicity $k$. It may be greater if $\lambda$ is also an eigenvector of $C$, but in this case the eigenvector is a linear combination of the eigenvectors already chosen for $\beta$.
\end{proof}

\subsubsection{Diagonalisable matrices}
\begin{definition}
A matrix $A\in\F^{n\times n}$ is called \udef{diagonalisable} if $\F^n$ has a basis of eigenvectors of $A$.
\end{definition}
\begin{proposition}
Let $A\in\F^{n\times n}$ and let $\lambda_1,\ldots, \lambda_m$ denote the distinct eigenvalues of $A$. The following are equivalent:
\begin{enumerate}
\item $A$ is diagonalisable;
\item there exist $1$-dimensional subspaces $U_1,\ldots, U_n$ of $A$, each invariant under $\ell_A$, such that
\[ \F^n = U_1\oplus \ldots \oplus U_n; \]
\item $\F^n = E_{\lambda_1}(A) \oplus \ldots \oplus E_{\lambda_m}(A);$
\item $n = \dim E_{\lambda_1}(A) + \ldots + \dim E_{\lambda_m}(A);$
\item for each $\lambda_i$ the geometric multiplicity is equal to the algebraic multiplicity and the sum of algebraic multiplicities is $n$.
\end{enumerate}
\end{proposition}
In the case of complex vector spaces, the sum of algebraic multiplicities is always $n$ by the fundamental theorem of algebra.
\begin{corollary}
If $A\in\F^{n\times n}$ has $n$ distinct eigenvalues, then $A$ is diagonalisable.
\end{corollary}
So a matrix may fail to be diagonalisible for two reasons: not enough geometric multiplicity or not enough geometric and algebraic multiplicity

\subsection{Spectral theorem}
\begin{theorem}
Let $V$ be a complex finite-dimensional inner product space. Let $L$ be an operator on $V$. Then
\begin{enumerate}
\item there exists an orthonormal basis of $V$ consisting of eigenvectors of $L$ \textup{if and only if} $L$ is normal;
\item if $L$ is self-adjoint, then the eigenvalues of $L$ are real.
\end{enumerate}
\end{theorem}

TODO: replace following: + in real case we need self-adjoint!
\begin{theorem}[Spectral theorem for matrices]
Let $V$ be a finite-dimensional inner product space over $\R$ or $\C$. Let $L=L^*$ be self-adjoint. Then
\begin{enumerate}
\item there exists an orthonormal basis of $V$ consisting of eigenvectors of $L$;
\item the eigenvalues of $L$ are real.
\end{enumerate}
\end{theorem}
\begin{proof}
We first prove the theorem for complex vector spaces. The proof is by finite induction:

By the fundamental theorem of algebra the characteristic polynomial has at least 1 root $\lambda_1$. Choose a corresponding eigenvector $\vec{v}_1$. Then by
\[ \lambda_1\inner{\vec{v}_1,\vec{v}_1} = \inner{\vec{v}_1, L\vec{v}_1} = \inner{L\vec{v}_1, \vec{v}_1} = \overline{\lambda_1}\inner{\vec{v}_1, \vec{v}_1}, \]
$\lambda_1$ is real.

Now $\Span\{\vec{v}_1\}^\perp$ is invariant under $L$:
\[ \vec{x}\in \Span\{\vec{v}_1\}^\perp \quad\iff\quad \inner{\vec{x},\vec{v}_1} = 0 \quad\implies\quad \inner{L\vec{x},\vec{v}_1} = \inner{\vec{x},L\vec{v}_1} = \lambda_1\inner{\vec{x},\vec{v}_1} = 0. \]

We can now apply the same argument to $L|_{\Span\{\vec{v}_1\}^\perp}:\Span\{\vec{v}_1\}^\perp\to \Span\{\vec{v}_1\}^\perp$, whose eigenvector are orthogonal to $v_1$. Finite induction then finishes the proof in the complex case. 

In the real case: we can linearly extend $L$ to be an operator $L_\C$ on the complexification $V_\C$. Then $L_\C$ has formally the same characteristic polynomial as $L$, except now interpreted as a function $\C\to\C$, not $\R\to\R$. Now we know the roots of $p_{L_\C}(x)$ are real, so they are also roots of $p_L(x)$. The rest of the proof can be completed in the same way. 
\end{proof}

The spectral decomposition is a special case of both the Schur decomposition and the singular value decomposition.

TODO: Kronecker product: multiply eigenvalues: all there (by multiplicity)

\subsection{Computing eigenvalues and vectors}
\subsubsection{Power method}
+ inverse

\subsubsection{Deflation}
\url{https://quickfem.com/wp-content/uploads/IFEM.AppE_.pdf}

\subsubsection{QR}

\section{Matrix classes and decompositions}
\subsection{Matrix classes}
\subsubsection{Rank-1 projections}
\begin{proposition}
Let $P\in\F^{n\times n}$. Then $P$ is a rank-1 (orthogonal) projection \textup{if and only if} there is a unit vector $\vec{u}\in\F^n$ such that $P= \vec{u}\vec{u}^*$.
\end{proposition}
\begin{proof}
Due to $P$ being rank-1, we can find a unit vector $\vec{u}$ such that $\Col(P) = \Span\{\vec{u}\}$. So the columns of $P$ are all multiples of $\vec{u}$, meaning we can write $P$ as $\vec{u}\vec{v}^*$ for some $\vec{v}\in\F^n$.

Now $P = P^*= (\vec{u}\vec{v}^*)^* = \vec{v}\vec{u}^*$, so $\vec{u}\vec{v}^* = \vec{v}\vec{u}^*$ and thus $\Col(P) = \Span\{\vec{v}\}$, meaning $\vec{v} = \lambda \vec{u}$.

Also $P^2 = \vec{u}\vec{v}^*\vec{u}\vec{v}^* = \vec{u}\inner{v,u}\vec{v}^* = \inner{v,u}\vec{u}\vec{v}^*$, so $1 = \inner{\vec{v},\vec{u}} = \overline{\lambda}\inner{\vec{u},\vec{u}} = \overline{\lambda}$.

So $\vec{v}=\vec{u}$ and $P= \vec{u}\vec{u}^*$.
\end{proof}
We write $P_{\vec{u}}$ to denote $\vec{u}\vec{u}^*$. Then in particular $P_{\vec{u}}\vec{v} = \inner{\vec{u},\vec{v}}\vec{u}$.

\subsubsection{Householder matrices}
\begin{definition}
Let $\vec{w}\in\F^n$ be a non-zero vector. Set $\vec{u} = \vec{w}/\norm{\vec{w}}$. Then the corresponding \udef{Householder matrix} is
\[ U_{\vec{w}} \defeq \mathbb{1}_n - 2P_{\vec{u}} = \mathbb{1}_n - 2 \frac{\vec{w}\vec{w}^*}{\inner{\vec{w},\vec{w}}} = \mathbb{1}_n - 2 \frac{\vec{w}\vec{w}^*}{\vec{w}^*\vec{w}}. \]
The corresponding transformation $\ell_{U_{\vec{w}}}$ is called a \udef{Householder transformation}.
\end{definition}
The Householder transformation reflects vectors across a hyperplane orthogonal to $\vec{w}$.

\begin{lemma}
Householder matrices are unitary, Hermitian and involutive.
\end{lemma}

For any two vectors of the same length, we can construct a unitary matrix that maps one to the other, using Householder matrices.

\begin{proposition}
Let $\vec{x},\vec{y}\in\F^n$ such that $\norm{\vec{x}} = \norm{\vec{y}} \neq 0$. Let
\[ \sigma = \begin{cases}
1& (\inner{\vec{x}\vec{y}} = 0) \\ -\overline{\inner{\vec{x},\vec{y}}}/|\inner{\vec{x},\vec{y}}| & (\inner{\vec{x}\vec{y}} \neq 0),
\end{cases} \]
and let $\vec{w} = \vec{y}-\sigma \vec{x}$. Then $\sigma U_{\vec{w}}$ is unitary and $\sigma U_{\vec{w}}\vec{x} = \vec{y}$.
\end{proposition}
The use of $\sigma$ is purely to improve numerical stability.

\subsubsection{Upper Hessenberg matrices}
\begin{definition}
A matrix $A$ is called an \udef{upper Hessenberg matrix} if $i>j+1\implies [A]_{i,j}=0$.
\end{definition}
This is a matrix of the form
\[ \begin{bmatrix}
\star & \star & \star & \star & \star \\
\star & \star & \star & \star & \star \\
0 & \star & \star & \star & \star \\
0 & 0 & \star & \star & \star \\
0 & 0 & 0 & \star & \star \\
\end{bmatrix} \]
Every square matrix is unitarily similar to an upper Hessenberg matrix, and the unitary
similarity can be constructed from a sequence of Householder matrices and complex rotations.

\subsection{Matrix decompositions}
\subsubsection{LU and LDU factorisation}
\subsubsection{QR factorisation}
The QR factorization of an $m \times n$ matrix $A$ is a factorisation
$A=QR$
where $Q\in\F^{m\times n}$ has orthonormal columns and $R\in\F^{n\times n}$ is square upper triangular.
This requires that $m\geq n$.

\begin{proposition}[QR factorisation]
Let $A\in\F^{m\times n}$ and $m\geq n$. Then
\begin{enumerate}
\item there exists a unitary $U\in\F^{m\times m}$ and an upper triangular matrix $R\in\F^{n\times n}$ such that
\[ A = U \begin{bmatrix}
R \\ \mathbb{0}
\end{bmatrix} \]
we can take $R$ to have real, non-negative values on the diagonal;
\item writing $U = \begin{bmatrix}
Q & Q'
\end{bmatrix}$, we get the decomposition
\[ A = QR \]
where $Q$ has orthonormal columns;
\item if $\Rank(A)= n$, then fixing the values on the diagonal of $R$ to be positive makes the factorisation unique; all values on the diagonal are non-zero.
\end{enumerate}
\end{proposition}
\begin{proof}
We can find a (unitary) Householder transformation $U_1$ that maps the first columns $[A]_{-,1}$ to $\norm{[A]_{-,1}}\vec{e}_1$. So
\[ U_1 A = \begin{bmatrix}
\norm{[A]_{-,1}} & \star \\ \mathbb{0} & A_1
\end{bmatrix} \]
Then we can do the same for $A_1$, meaning $U_2 = \mathbb{1}\oplus U'$ transforms $A$ as
\[ U_2U_1A = \begin{bmatrix}
\norm{[A]_{-,1}} & \star & \star \\
0 & \norm{[A_1]_{-,1}} & \star \\
\mathbb{0} & \mathbb{0} & A_2
\end{bmatrix}. \]
Repreating this gives the required factorisation.
\end{proof}
The factorisation $A = U \begin{bmatrix}
R \\ \mathbb{0}
\end{bmatrix}$ is called the wide QR factorisation and $A=QR$ the (narrow) QR factorisation.

\subsection{Polar decomposition}
\subsubsection{Singular value decomposition}
spectral decomposition is a special case
\subsubsection{Schur decomposition}
spectral decomposition is a special case




\section{Systems of linear equations}
\url{https://encyclopediaofmath.org/wiki/Motzkin_transposition_theorem}
TODO
A homogeneous system of linear equations with more variables than equations has non-zero solutions.

An inhomogeneous system of linear equations with more equations thanvariables has no solution for some choice of the constant terms.

Calculation of inverse via row reduction.

free and bounded variables.

\begin{lemma}
If $Ax=b$ is consistent for all $b\in \F^n$, then $A$ has a right inverse $B$, i.e.\ $AB = \mathbb{1}$.
\end{lemma}
\begin{proof}
For each $e_i$ in the standard basis we can find a $c_i$ such that $Ac_i = e_i$. Then
\[ A \begin{pmatrix}
c_1 & c_2 & \hdots & c_n
\end{pmatrix} = \begin{pmatrix}
Ac_1 & Ac_2 & \hdots & Ac_n
\end{pmatrix} = \begin{pmatrix}
e_1 & e_2 & \hdots & e_n
\end{pmatrix} = \mathbb{1}_n. \]
\end{proof}

\subsection{Cramer's rule}
\begin{proposition}
\[ x_i = \frac{\det(A_i)}{\det(A)} \]
\end{proposition}
\begin{proof}
$x = \begin{pmatrix}
x_1 \hdots x_n
\end{pmatrix}^\transp$
\begin{align*}
x_i &= \det \begin{pmatrix}
e_1 & \hdots & e_{i-1} & x & e_{i+1} & \hdots & e_n
\end{pmatrix} \\
&= \det \begin{pmatrix}
A^{-1}a_1 & \hdots & A^{-1}a_{i-1} & A^{-1}b & A^{-1}a_{i+1} & \hdots & A^{-1}a_n
\end{pmatrix} \\
&= \det (A^{-1}\begin{pmatrix}
a_1 & \hdots & a_{i-1} & b & a_{i+1} & \hdots & a_n
\end{pmatrix}) = \det(A^{-1}A_i) \\
&= \frac{\det(A_i)}{\det(A)}.
\end{align*}
\end{proof}


\section{Polynomials applied to endomorphisms}
\section{The spectra of matrices}
What are eigenvectors of rotation? -> complex eigenvalues.

Real matrix: complex conjugate eigenvalues have complex conjugate eigenvectors

Finite order endomorphisms are diagonalisable over $\C$ (or any algebraically closed field where the characteristic of the field does not divide the order of the endomorphism) with roots of unity on the diagonal. This follows since the minimal polynomial is separable, because the roots of unity are distinct.

See \url{https://en.wikipedia.org/wiki/Minimal_polynomial_(linear_algebra)}


\section{Euclidean geometry}
\begin{definition}
The \udef{$n$-dimensional Euclidean space} is $\R^n$ equipped with the inner product
\[ \inner{\begin{pmatrix}
x_1 \\ x_2 \\ \vdots \\ x_n
\end{pmatrix}, \begin{pmatrix}
y_1 \\ y_2 \\ \vdots \\ y_n
\end{pmatrix}} = x_1y_1 + x_2y_2 \ldots x_ny_n. \]
\end{definition}

TODO: $z$-vector points into page due to orientation (point up out of the page would give it a left-handed orientation).

\subsection{Affine subspaces}

Line as intersection of planes with normals $n_1,n_2$. Then direction of line is $n_1\times n_2$.

\subsection{Distances}

\subsection{Distance point to plane}
\begin{lemma}
\[ d(P,\pi) = \frac{|p_x\alpha + p_y\beta +p_z\gamma -d|}{\sqrt{\alpha^2 + \beta^2 + \gamma^2}} \]
\end{lemma}

Thus in the Cartesian expression for a plane, $\alpha x + \beta y + \gamma z = d$, the $d$ is the distance to the origin. (Also $(\alpha, \beta, \gamma)$ is the normal vector).

\subsection{Angles}

\subsection{Rotations}

\subsection{Spheres}











\chapter{Indices and symbols}
\section{Contravariant and covariant vectors and tensors}
When working in finite-dimensional spaces with specified bases, we often get expressions of the form
\[ v = \sum_{i=1}^n a_i \vec{e}_i. \]
We may replace this expression with
\[ v = a^i \vec{e}_i \]
if we take the convention that if an index is repeated once up and once down, then there is a sum over all values of that index. This is the \udef{Einstein summation convention}.

Note that coordinates have their indices up, and basis vectors have their indices down.

Now in the dual space, we have the dual basis $\{\varphi^j\}_j$. In the dual space we take the opposite convention: coordinates have their indices down, and dual basis vectors have their indices up, so
\[ \varphi = b_j \varphi^j. \]
This allows us to write
\[ \varphi(v) = \varphi(a_i \vec{e}_i) = a_i b^j \varphi^j(\vec{e}_i) = a_i b^i \]
where for the last equality we have used that $\varphi^j(\vec{e}_i)$ only does not vanish if $i=j$ and is $1$ in this case.

We call vectors in $V$ \udef{contravariant} vectors and vectors in $V^*$ \udef{covariant} vectors, or covectors.

Per convention we put the coordinates of contravariant vectors in column vectors. This means, by proposition \ref{transpDual}, we must put the coordinates of covariant vectors in row vectors. Indeed, let $v=a^i \vec{e}_i\in V$ and $\varphi = b_j\varphi^j \in V^*$, then
\[ \varphi(v) = a^ib_i = \begin{bmatrix}
a^1 & \hdots & a^n
\end{bmatrix}\begin{bmatrix}
b_1 \\ \vdots  \\ b_n
\end{bmatrix}. \]

We can view covariant vectors as functions that take a contravariant vector and produce a number, and we can view contravariant vectors as functions that take a covariant vector and produce a number. In general we may have linear functions that accept several co- and contravariant vectors and produce a number. By (TODO), such functions are tensor products of various co- and contravariant vectors. They would have multiple up- and down-indices. e.g\
\[ \vec{T} = \tensor{T}{^i_j_k^l^m}(\vec{e}_i\otimes \vec{e}^j\otimes \vec{e}^k\otimes \vec{e}_l \otimes \vec{e}_m) \]

Where $\tensor{T}{^i_j_k^l^m}$ are the coordinates w.r.t. the basis vectors $\vec{e}_i\otimes \vec{e}^j\otimes \vec{e}^k \otimes\vec{e}_l \otimes \vec{e}_m$.

For example, once the basis has been chosen, matrices map contravariant vectors to contravariant vectors. And contravariant vectors map covariant vectors to numbers, so by reverse currying a matrix is maps a contravariant and a covariant vector to a number.

For the indices of matrices we have taken the convention that the first index is for rows and the second for the columns. For a constant row index, the column index spells out a covariant vector, so the column index is down. Conversely, the row index is up. A matrix $A$ with components $(A)_{i,j}$ becomes
\[ \tensor{A}{^i_j}(\vec{e}_i\otimes \vec{e}^j). \]
This is consistent with the observation that the matrix sends a vector to a function on covectors, in other words is a function which accepts vectors in first place and covectors in second place.

In the expressions so far only repeated indices were present. Such repeated indices are called  \udef{bound indices} or \udef{dummy indices}. They may be replaced in the expression by other letters, so long as there is no clash. If an index is not repeated, it is a \udef{free index} and may not just be changed.

\subsection{``Tensors are objects that transform as tensors''}
Variants:
\begin{itemize}
\item ``$N$ arbitrary numbers are not the components of a vector'' (Peres p.65)
\end{itemize}


\section{Covectors}

\subsection{Multi-index notation}
Let $e_1,\ldots, e_n$ be a basis for a real vector space $V$. Let $\alpha^1,\ldots, \alpha^n$ be the dual basis for $V^*$. A \udef{multi-index}
\[ I = (i_1,\ldots,i_k)\]
is a $k$-tuple of numbers $\in (1,\ldots,n)$. We write
\[ \begin{cases}
e_I \defeq e_{i_1}\otimes\ldots\otimes e_{i_k}\\
\alpha^I \defeq \alpha^{i_1}\wedge\ldots \wedge\alpha^{i_k}
\end{cases}. \]
The covector $\alpha^I$ is completely determined by the values in $I$, the order only changes the sign. A multi-index $I = (i_1,\ldots,i_k)$ is \udef{ascending} if
\[ 1\leq i_1<\ldots<i_k\leq n. \]
\begin{proposition}
Let $I,J$ be ascending multi-indices of length $k$, then
\[ \alpha^I(e_J) = \begin{cases}
1 & I=J \\ 0& I\neq J
\end{cases}. \]
\end{proposition}
\begin{proposition}
The covectors $\alpha^I$, with $I$ an ascending multi-index of length $k$, form a basis of $A_k(V)$.
\end{proposition}
\begin{corollary}
If $\dim V=n$, then
\[ \dim A_k(V) = \begin{pmatrix}
n\\k
\end{pmatrix}. \]
\end{corollary}

\section{Symmetrisation and anti-symmetrisation of indices}

\[ T_{\{a_1\dots a_n\}} = \frac{1}{n!} \sum_{\sigma\in S_n} T_{a_{\sigma(1)} \dots a_{\sigma(n)}} \]

\[ T_{[a_1\dots a_n]} = \frac{1}{n!} \sum_{\sigma\in S_n} (\sgn \sigma)T_{a_{\sigma(1)} \dots a_{\sigma(n)}} \]
\section{Symbols}
\subsection{Kronecker delta}
\begin{definition}
The \udef{Kronecker delta} is defined by
\[ \delta_{ij} = \delta^i_j = \begin{cases}
1 & (i=j) \\
0 & (i \neq j)
\end{cases}.\]
\end{definition}
\subsection{Levi-Civita symbol}
\begin{definition}
The \udef{Levi-Civita symbol} is defined by
\[ \varepsilon_{a_{1}\ldots a_{n}} = \begin{cases}
+1 & \text{$(a_{1},\ldots, a_{n})$ is an even permutation of $(1,\ldots, n)$} \\
-1 & \text{$(a_{1},\ldots, a_{n})$ is an odd permutation of $(1,\ldots, n)$} \\
0 & \text{otherwise}
\end{cases}.\]
The indices may be placed up or down.
\end{definition}

\begin{lemma} \label{LeviCivitaProduct}
The Levi-Civita symbol is given by the explicit expression
\[ \varepsilon_{a_{1}\ldots a_{n}} = \prod_{1\leq i<j\leq n}\sgn(a_j-a_i).\]
\end{lemma}

\begin{proposition}
Working in $n$ dimensions, when all $i_1,\ldots i_n;j_1,\ldots, j_n$ take values in $\{ 1,\ldots, n \}$:
\begin{enumerate}
\item $\displaystyle \varepsilon_{i_1\ldots i_n}\varepsilon^{j_1\ldots j_n} = n!\delta^{j_1}_{[i_1}\ldots \delta^{j_n}_{i_n]} = \sum_{\sigma\in S_n} (-1)^{{\sgn}(\sigma)} \delta^{j_1}_{i_{\sigma(1)}} \dots \delta^{j_n}_{i_{\sigma(n)}}$
\item $\displaystyle \varepsilon _{i_{1}\dots i_{n}}\varepsilon ^{i_{1}\dots i_{n}}=n!$
\item $\displaystyle \varepsilon _{i_{1}\dots i_{k}~i_{k+1}\dots i_{n}}\varepsilon ^{i_{1}\dots i_{k}~j_{k+1}\dots j_{n}}=k!(n-k)!~\delta _{[i_{k+1}}^{j_{k+1}}\dots \delta _{i_{n}]}^{j_{n}}$.
\end{enumerate}
\end{proposition}
\begin{proof}
\begin{enumerate}
\item Both sides of the equation are a sum over the same indices. We consider each term in the sum separately and show that the sums are equal term-by-term. We split the terms into two categories.
\begin{enumerate}
\item First consider the case that $j_1\ldots j_n$ is not a permutation of $(1,\ldots, n)$, i.e.\ a number is repeated. Then $\varepsilon_{i_1\ldots i_n}\varepsilon_{j_1\ldots j_n}$ is automatically zero. The right-hand side is definitely zero if the $i$s do not take the same values as the $j$s. If they do take the same values, there is a number that is repeated at least twice. For every term in the sum over permutations, there is another term with the repeated $i$s swapped, which also adds a minus due to the change of sign of the permutation. Hence the sum over permutations is zero.
\item Now assume that $j_1\ldots j_n$ is a permutation of $(1,\ldots, n)$. Then either the $i$s are also a permutation, or $\delta^{j_1}_{i_{\sigma(1)}} \dots \delta^{j_n}_{i_{\sigma(n)}}$ is always zero. The only possible non-zero term is with a $\sigma\in S_n$ such that $j_k = i_{\sigma(k)}$ for all $k$. If $\sgn(i_1,\ldots, i_n) = \sgn(j_1,\ldots, j_n)$, then $\sgn(\sigma)=1$ and both sides match. If $\sgn(i_1,\ldots, i_n) = -\sgn(j_1,\ldots, j_n)$, then $\sgn(\sigma)=-1$ and both sides again match. 
\end{enumerate}
So, in fact, we have shown something slightly stronger, namely 
\[ \varepsilon_{i_1\ldots i_n}\varepsilon_{j_1\ldots j_n} = n!\delta^{j_1}_{[i_1}\ldots \delta^{j_n}_{i_n]} \]
where there is no sum over indices.
\item The number of permutations of any $n$-element set number is exactly $n!$. Every permutation is either even or odd and $(+1)^2 = (-1)^2 = 1$. Non-permutations do not contribute to the sum.
\item The sum on the left only has terms where the $i$s and $j$s are permutations of $(1,\ldots, n)$. In each such term we can bring the indices with values $1-k$ to the first $k$ spots, each by a transposition. Because both Levi-Civita symbols have the same first $k$ indices, each will need the same number of transpositions and thus the sign does not change. Then by considering lemma \ref{LeviCivitaProduct} we see that we have obtained a product of cases 1. and 2. This yields the answer.

\end{enumerate}
\end{proof}
\begin{corollary}
In two dimensions, where all $i,j,m,n$ each take values in $\{1,2\}$,
\begin{enumerate}
\item $\varepsilon _{ij}\varepsilon ^{mn}={\delta _{i}}^{m}{\delta _{j}}^{n}-{\delta _{i}}^{n}{\delta _{j}}^{m}$
\item $\varepsilon _{ij}\varepsilon ^{in}={\delta _{j}}^{n}$
\item $\varepsilon _{ij}\varepsilon ^{ij}=2.$
\end{enumerate}
\end{corollary}
\begin{corollary}
In three dimensions, where all $i,j,k,m,n$ each take values in $\{1,2,3\}$,
\begin{enumerate}
\item $\varepsilon _{ijk}\varepsilon ^{imn}={\delta _{j}}^{m}{\delta _{k}}^{n}-{\delta _{j}}^{n}{\delta _{k}}^{m}$
\item $\varepsilon _{jmn}\varepsilon ^{imn}={\delta _{j}}^{i}$
\item $\varepsilon _{ijk}\varepsilon ^{ijk}=6.$
\end{enumerate}
\end{corollary}
\begin{proposition}
Working in 3 dimensions,
\begin{align*}
\varepsilon _{ijk}\varepsilon _{lmn}&={\begin{vmatrix}\delta _{il}&\delta _{im}&\delta _{in}\\\delta _{jl}&\delta _{jm}&\delta _{jn}\\\delta _{kl}&\delta _{km}&\delta _{kn}\\\end{vmatrix}}\\[6pt]&=\delta _{il}\left(\delta _{jm}\delta _{kn}-\delta _{jn}\delta _{km}\right)-\delta _{im}\left(\delta _{jl}\delta _{kn}-\delta _{jn}\delta _{kl}\right)+\delta _{in}\left(\delta _{jl}\delta _{km}-\delta _{jm}\delta _{kl}\right).
\end{align*}
This can directly be generalised to $n$ dimensions.
\end{proposition}

\section{Writing matrix operations using using tensor notation}
A matrix $A$ with components $(A)_{i,j}$ becomes
\[ \tensor{A}{^i_j}(\vec{e}_i\otimes \vec{e}^j). \]
\subsection{Trace}
The trace of $\tensor{A}{^i_j}$ is $\tensor{A}{^i_i}$.
\subsection{Matrix multiplication}
\[ \tensor{(AB)}{^{i}_{k}}=\tensor{A}{^{i}_{j}}\tensor{B}{^{j}_{k}} \]
which in particular for matrix-vector multiplication becomes
\[ (Av)^i = \tensor{A}{^i_j} v^j. \]
\subsection{Transpose}
The transpose of $\tensor{A}{^i_j}$ is $\tensor{(A^\transp)}{^j_i}$.

Or: $(A^\transp)_{ab} = A_{ba}$ and $\tensor{(A^\transp)}{_i^j} = \tensor{A}{^j_i}$?
\subsection{Determinant}
\begin{align*}
\det(A) &= \varepsilon^{j_1\ldots j_n}\tensor{A}{^{1}_{j_1}}\ldots \tensor{A}{^{n}_{j_n}} \\
&= \frac{1}{n!}\varepsilon_{i_1\ldots i_n}\varepsilon^{j_1\ldots j_n}\tensor{A}{^{i_1}_{j_1}}\ldots \tensor{A}{^{i_n}_{j_n}}
\end{align*}

\chapter{Ordered vector spaces}
TODO link ordered groups.
\begin{definition}
Let $\sSet{\R, V, +}$ be a real vector space and $\precsim$ a preorder on the set $V$. Then $\precsim$ is a \udef{vector preorder} if it is compatible with the vector space structure as follows: $\forall x,y,z\in V, \lambda\in\R$
\begin{enumerate}
\item $x\precsim y$ implies $x+z \precsim y+z$;
\item if $\lambda\geq 0$, then $x \precsim y$ implies $\lambda x \precsim \lambda y$.
\end{enumerate}
We call $(\R, V, +, \precsim)$ a \udef{preordered vector space}.

\begin{itemize}
\item If $\precsim$ is a partial order, we call $(\R, V, +, \precsim)$ a \udef{partially ordered vector space} or simply a \udef{ordered vector space}.
\item If $\sSet{V, \precsim}$ is a lattice, we call $(\R, V, +, \precsim)$ a \udef{vector lattice} or a \udef{Riesz space}.
\end{itemize}
\end{definition}

\begin{lemma} \label{positiveConeOrderCharacterisation}
Let $\sSet{\R, V, +}$ be a real vector space and $\precsim$ a preorder on the set $V$. The compatibility of the order can equivalently be expressed by:
$\forall x,y,z\in V, \lambda\in\R$
\[ \begin{cases}
\text{$x \precsim y$ implies $x+z \precsim y+z$;} \\
\text{if $\lambda\geq 0$ and $0 \precsim x$, then $0 \precsim \lambda x$.}
\end{cases} \]
\end{lemma}

\begin{lemma} \label{elementaryVectorPreorderManipulations}
Let $V$ be a preordered vector space. For all $v,w \in V$ we have
\[ v \precsim w \;\iff\; 0 \precsim  w - v \;\iff\; -w \precsim -v.  \]
\end{lemma}
\begin{proof}
We get the implications
\[ v \precsim w \implies 0 \precsim  w - v \implies -w \precsim -v \implies v-w \precsim 0 \implies v\precsim w \]
by subsequently adding $-v, -w, v,w$ to both sides by compatibility of the order.
\end{proof}
\begin{corollary}
Let $V$ be a preordered vector space and $\alpha \in \R\setminus\{0\}$. Then for all $v,w\in V$
\[ v \precsim w \quad \iff \quad \begin{cases}
\alpha v \precsim \alpha w & (0 < \alpha) \\
\alpha v \succsim \alpha w & (\alpha > 0)
\end{cases}. \] 
\end{corollary}

\begin{lemma} \label{additionVectorInequalities}
Let $V$ be a preordered vector space. For all $v,w, x, y \in V$ we have
\[ \begin{cases}
v \precsim w \\ x \precsim y
\end{cases} \implies v+ x \precsim w+y. \]
\end{lemma}
\begin{proof}
We calculate $v + x \leq w + x \leq w+y$.
\end{proof}

\begin{example}
\begin{itemize}
\item The finite-dimensional vector spaces $\R^n$ with coordinate-wise addition, scalar multiplication and order are Riesz spaces.
\item The finite-dimensional vector spaces $\R^n$ with coordinate-wise addition, scalar multiplication and lexicographical order are Riesz spaces.
\item Let $X$ be a set. The set $(X\to \R)$ is a real vector space with point-wise addition and scalar multiplication. If the order is also defined point-wise, i.e.\ $f \leq g$ iff $\forall x\in X: f(x) \leq g(x)$, then $(X\to \R)$ is a Riesz space.
\end{itemize}
\end{example}

\begin{lemma}
Let $X$ be a topological space. The spaces
\begin{enumerate}
\item $\cont(X,\R)$;
\item $\cont_0(X,\R)$;
\item $\cont_c(X,\R)$; and
\item $\cont_b(X,\R)$
\end{enumerate}
with point-wise operations are Riesz spaces.
\end{lemma}
\begin{proof}
In all these cases the join and meet of $f,g$ are given by
\begin{align*}
f \vee g &= \frac{1}{2}(f+g)+ \frac{1}{2}|f-g| \\
f \wedge g &= \frac{1}{2}(f+g) - \frac{1}{2}|f-g|.
\end{align*}
So the join and meet are still continuous and have the same properties as $f,g$.
\end{proof}

\section{Upsets and downsets}
\begin{lemma} \label{sumMultipleUpDownsets}
Let $V$ be a preordered vector space, $S\subseteq V$ a subset, $v\in V$ and $\alpha\in \R$. Then
\begin{enumerate}
\item if $\alpha > 0$, then $(\alpha S)^u = \alpha S^u$ and $(\alpha S)^l = \alpha S^l$;
\item if $\alpha < 0$, then $(\alpha S)^u = \alpha S^l$ and $(\alpha S)^l = \alpha S^u$;
\item $(S+v)^u = S^u + v$ and $(S+v)^l = S^l + v$.
\end{enumerate}
\end{lemma}
\begin{corollary}
Let $V$ be a preordered vector space, $S\subseteq V$ a subset, $v\in V$ and $\alpha\in \R$. Then
\begin{enumerate}
\item $\sup(S+v) = \sup(S)+v$ and $\inf(S+v) = \inf(S)+v$;
\item if $\alpha > 0$, then $\sup(\alpha S) = \alpha \sup(S)$ and $\inf(\alpha S) = \alpha \inf(S)$;
\item if $\alpha < 0$, then $\sup(\alpha S) = \alpha \inf(S)$ and $\inf(\alpha S) = \alpha \sup(S)$.
\end{enumerate}
\end{corollary}

\section{The positive cone}
\begin{definition}
Let $V$ be a preordered vector space. The subset
\[ V^+ \defeq \setbuilder{v\in V}{0 \precsim v} \]
is called the \udef{positive cone} of $V$. The elements of the positive cone $V^+$ are called the \udef{positive elements} of $V$.
\end{definition}
That the positive cone is in fact a cone follows from \ref{positiveConeOrderCharacterisation}
\begin{proposition} \label{positiveCone}
Let $V$ be a vector space.
\begin{enumerate}
\item A vector preorder on $V$ is uniquely determined by its positive cone:
\[ x \precsim y \quad\iff\quad y-x \in V^+. \]
\item The positive cone of a vector preorder is pointed and convex.
\item Any pointed convex cone in $V$ determines a (unique) vector preorder.
\item A vector preorder is a partial order \textup{if and only if} the positive cone is salient.
\end{enumerate}
\end{proposition}
Convexity is equivalent to closure under addition (see \ref{convexityAdditiveClosure})
\begin{proof}
(1) This is just \ref{elementaryVectorPreorderManipulations}.

(2) $V^+$  is pointed by reflexivity: $0\precsim 0$. It is closed under addition by \ref{additionVectorInequalities}.

(3) Compatibility with addition is immediate from the definition of the order. Compatibility with scalar multiplication is due to it being a cone (see \ref{positiveConeOrderCharacterisation}). Reflexivity is equivalent with pointedness. Finally transitivity follows from closure under addition:
\begin{align*}
 \begin{cases}
x\precsim y \\ y\precsim x
\end{cases} &\iff \quad \begin{cases}
0 \precsim y -x \\ 0 \precsim z-y
\end{cases} \iff\quad \begin{cases}
y-x \in V^+ \\ z-y \in V^+
\end{cases} \\
&\implies (z-y)+(y-x) = z-x \in V^+ \iff x \precsim z. 
\end{align*}

(4) We have $x\precsim y$ and $y\precsim x$ iff $(y-x) \in V^+$ and $-(y-x) \in V^+$. Thus both salience and anti-symmetry are equivalent to this situation implying $x-y = 0$.
\end{proof}


\begin{lemma} \label{scalarMultiplicationInequalities}
Let $V$ be a preordered vector space, $v\in V^+$ and $\alpha\in \R$.
\begin{enumerate}
\item If $\alpha \geq 1$, then $\alpha v \succsim v$.
\item If $\alpha \leq 1$, then $\alpha v \precsim v$.
\end{enumerate}
\end{lemma}
\begin{proof}
(1) We have $v\succsim 0$ and $(\alpha-1) \geq 0$, so $(\alpha-1)v \succsim 0$ and $\alpha v \succsim v$.

(2) We have $v\succsim 0$ and $(\alpha-1) \leq 0$, so $(\alpha-1)v \precsim 0$ and $\alpha v \precsim v$.
\end{proof}

\section{Riesz spaces}

\begin{lemma} \label{lemmaRieszSpaces}
Let $V$ be a Riesz space, $u,v,w\in V$ and $\alpha\in \R$, then
\begin{enumerate}
\item $-(v \wedge w) = (-v)\vee (-w)$ and $-(v \vee w) = (-v)\wedge (-w)$;
\item if $\alpha \geq 0$, then $\alpha(v \wedge w) = (\alpha v)\wedge (\alpha w)$ and $\alpha(v \vee w) = (\alpha v)\vee (\alpha w)$;
\item if $\alpha \leq 0$, then $\alpha(v \wedge w) = (\alpha v)\vee (\alpha w)$ and $\alpha(v \vee w) = (\alpha v)\wedge (\alpha w)$;
\item $u+(v \wedge w) = (u+v)\wedge (u+w)$ and $u+(v \vee w) = (u+v)\vee (u+w)$.
\end{enumerate}
\end{lemma}
\begin{proof}
We apply \ref{imagePolars} to

(1) the reverse order-embedding $v\mapsto -v$;

(2) the order-embedding $v\mapsto \alpha v$ for $\alpha > 0$; (if $\alpha = 0$ the result is trivial);

(3) the reverse order-embedding $v\mapsto \alpha v$ for $\alpha > 0$; (if $\alpha = 0$ the result is trivial);

(4) the order-embedding $v\mapsto u+v$.
\end{proof}

\begin{proposition}[Riesz decomposition]
Let $V$ be a Riesz space and $v,w_1,w_2\in V^+$ such that $v \leq w_1 + w_2$. Then $\exists v_1, v_2\in V^+$ such that $v = v_1 + v_2$ and $v_1 \leq w_2, v_2 \leq w_2$.
\end{proposition}
\begin{proof}
Set $v_1 = v\wedge w_1$ and $v_2 = v - v_1$. These satisfy all the properties. We verify the inequality $v_2 \leq w_2$: from $w_2 \geq v - w_1$ we get
\[ w_2 = 0\vee w_2 \geq 0\vee (v - w_1) = v + (-v)\vee(-w_1) = v - v\wedge w_1 = v-v_1 = v_2. \]
\end{proof}

\begin{proposition} \label{sumAsMeetJoin}
Let $V$ be a Riesz space and $v,w\in V$, then
\[ (v \vee w) + (v \wedge w) = v+w. \]
\end{proposition}
\begin{proof}
We calculate
\[ (v \vee w) + (v \wedge w) = v + 0 \vee (w-v) + w + (v-w)\wedge 0 = (v + w) + 0 \vee (w-v) - 0 \vee (w-v) = v + w. \]
\end{proof}

\begin{proposition}
Let $V$ be a Riesz space, $u,v,w\in V$ and $x,y,z\in V^+$. Then
\begin{enumerate}
\item $(u+v)\vee (2w) \leq u\vee w + v\vee w$;
\item $(u+v)\vee z \leq u\vee z + v\vee z$;
\item $(x+y)\wedge z \leq x\wedge z + y\wedge z$.
\end{enumerate}
\end{proposition}
\begin{proof}
(1) From $u\leq u\vee w$ and $v\leq v\vee w$, we get $u+v \leq u\vee w + v\vee w$. Similarly from $w\leq u\vee w$ and $w\leq v\vee w$, we get $2w \leq u\vee w + v\vee w$. Together this gives (1).

(2) We have $2z \geq z$ by \ref{scalarMultiplicationInequalities}.

(3) TODO (use Birkhoff inequality??)
\end{proof}

\begin{proposition}[Infinite distributivity in Riesz spaces]
Let $V$ be a Riesz space, $v\in V$ and $S\subseteq V$ a subset. Then
\begin{enumerate}
\item if $\bigvee S$ exists, then $\left(\bigvee S\right) \wedge v = \bigvee (S\wedge v)$;
\item if $\bigwedge S$ exists, then $\left(\bigwedge S\right) \vee v = \bigwedge (S\vee v)$;
\end{enumerate}
\end{proposition}
\begin{proof}
We already have the inequality $\left(\bigvee S\right) \wedge v \geq \bigvee (S\wedge v)$ from \ref{infiniteDistributiveInequalities}. To show the other inequality, it is enough to show that for 
\end{proof}
\begin{corollary}
Riesz spaces are distributive lattices.
\end{corollary}

\subsection{Positive elements}
\subsubsection{Positive and negative parts}
\begin{definition}
Let $V$ be a Riesz space and $v\in V$. Then we define
\begin{align*}
v^+ &\defeq v \vee 0 \\
v^- &\defeq (-v) \vee 0 = - (v \wedge 0).
\end{align*}
We call $v^+$ the \udef{positive part} of $v$ and $v^-$ the \udef{negative part} of $v$.
\end{definition}

\begin{lemma} \label{MeetJoinAsPositiveNegative}
Let $V$ be a Riesz space and $v,w\in V$. Then
\begin{enumerate}
\item $v\vee w = (v-w)^+ + w = (v-w)^- + v$;
\item $v\wedge w = v - (v-w)^+ = w - (v-w)^-$.
\end{enumerate}
\end{lemma}
\begin{proof}
We calculate $v\vee w = (v-w)\vee 0 + w = (v-w)^+ + w$. The other equalities are similar.
\end{proof}

\begin{proposition} \label{PositiveNegativeElements} \label{minimalPositiveDecomposition} 
Let $V$ be a Riesz space and $v,w\in V$. Then
\begin{enumerate}
\item $v^+, v^- \in V^+$;
\item $v= v^+ - v^-$;
\item $v^+\perp v^-$ (i.e.\ $v^+ \wedge v^- = 0$).
\end{enumerate}
Furthermore,
\begin{enumerate} \setcounter{enumi}{3}
\item if $p,q\in V$ satisfy 1 and 2, i.e.\ $p,q\in V^+$ and $v = p-q$, then $p \geq v^+$ and $q \geq v^-$; we may say $v=v^+-v^-$ is the minimal such decomposition; 
\item the elements $v^+, v^-$ are uniquely determined by properties 2 and 3. 
\end{enumerate}
Also
\begin{enumerate} \setcounter{enumi}{5}
\item $(-v)^- = v^+$ and $(-v)^+ = v^-$;
\item if $\alpha \geq 0$, then $(\alpha v)^+ = \alpha v^+$ and $(\alpha v)^- = \alpha v^-$;
\item $-v^- \leq v \leq v^+$;
\item $v\leq w$ \textup{if and only if} $v^+ \leq w^+$ and $v^- \geq w^-$.
\end{enumerate}
\end{proposition}
\begin{proof}
(1) Evident from definitions.

(2) We calculate $v^+ - v = (v \vee 0) - v = (v-v) \vee (0-v) = 0\vee (-v) = v^-$.

(3) We calculate $0 = v^- - v^- = v^-  + (v\wedge 0) = (v^- + v)\wedge (0 + v^-) = v^+ \wedge v^-$.

(4) From $v\leq p$ and $0\leq p$, we get $v^+ = v \vee 0 \leq p$. Then we also have $v^- = v^+ - v \leq p - v = q$.

(5) Assume $p,q\in V$ satisfy (2) and (3), then (1) automatically follows from (3). Using \ref{MeetJoinAsPositiveNegative}, we calculate
\[ 0 = p\wedge q = p - (p-q)^+ = p - v^+. \]
So $p = v^+$ and $q = p - v = v^+ - v = v^-$.

(6) It is evident that $(-v)^- = (--v)\vee 0 = v\vee 0$.

(7) We calculate $\alpha v^+ = \alpha (v \vee 0) = (\alpha v) \vee 0 = (\alpha v)^+$; the calculation for $\alpha v^-$ is similar.

(8) This is clear from $-v^- = v\wedge 0 \;\leq\; v \;\leq\; v \vee 0 = v^+$.

(9) $v\leq w$ implies $v^+ = v\vee 0 \leq w\vee 0 = w^+$ and $-v^- = v\wedge 0 \leq w\wedge 0 = - w^-$.

Conversely, we have $v = v^+ - v^- \leq w^+ - w^- = w$.
\end{proof}


\begin{proposition} \label{triangleInequalityPositiveNegativeElements}
Let $V$ be a Riesz space and $v,w\in V$. Then
\begin{enumerate}
\item $(v+w)^+ \leq v^+ + w^+$;
\item $(v+w)^- \leq v^- + w^-$.
\end{enumerate}
\end{proposition}
\begin{proof}
From \ref{PositiveNegativeElements} we get $v \leq v^+$ and $w\leq w^+$, so $v+w \leq v^+ + w^+$. Also $0 \leq v^+ + w^+$. So
\[ (v+w)^+ = (v+w)\vee 0 \leq v^+ + w^+. \]
Then we also have
\[ (v+w)^- = (-v-w)^+ \leq (-v)^+ + (-w)^+ = v^- + w^-. \]
\end{proof}

\subsubsection{Absolute value}
\begin{definition}
Let $V$ be a Riesz space and $v\in V$. Then the \udef{absolute value} of $v$ is
\[ |v| \defeq v\vee (-v) = -(v\wedge (-v)). \]
\end{definition}

If the Riesz space is a real function space with pointwise order, then $|f| = |\cdot|\circ f$ as usual, where $|\cdot|: \R\to \R$ is the usual absolute value function.

\begin{lemma} \label{absoluteValue}
Let $V$ be a Riesz space, $v,w\in V$ and $\alpha\in \R$. Then
\begin{enumerate}
\item $|v| = v^+ + v^-$;
\item $|v| \in V^+$;
\item if $v\in V^+$, then $|v| = v$;
\item $|v| = |-v|$;
\item $|\alpha v| = |\alpha|\cdot |v|$;
\item $\big||v|\big| = |v|$;
\item $0 \leq v^+ \leq |v|$ and $0 \leq v^- \leq |v|$;
\item $|v| = 0$ \textup{if and only if} $v = 0$.
\end{enumerate}
\end{lemma}
\begin{proof}
We prove (1):
\[ |v| = v\vee (-v) = (2v)\vee 0 - v = 2v^+ - v = 2v^+ - (v^+ - v^-) = v^+ - v^- \]
and (6), using the absorption law:
\[ \big||v|\big| = |v|\vee (-|v|) = v \vee (-v) \vee \big( v\wedge (-v) \big) = v \vee (-v) = |v|. \]
The rest are immediate consequences, using the results of \ref{PositiveNegativeElements}.
\end{proof}

\begin{lemma}
Let $V$ be a Riesz space and $v,w\in V$, then
\begin{enumerate}
\item $(v+w)\vee (v-w) = v + |w|$;
\item $(v+w)\wedge (v-w) = v - |w|$;
\end{enumerate}
or, equivalently,
\begin{enumerate} \setcounter{enumi}{2}
\item $v \vee w = \frac{1}{2}\big(v+w + |v - w|\big)$;
\item $v \wedge w = \frac{1}{2}\big(v+w - |v - w|\big)$.
\end{enumerate}
\end{lemma}
\begin{proof}
We calculate, using \ref{lemmaRieszSpaces}
\[ (v+w)\vee (v-w) = v+ w\vee(-w) = v+ |w| \quad\text{and}\quad (v+w)\wedge (v-w) = v + w\wedge(-w) = v-|w|. \]
The next two equalities follow by the substitutions $v+w \leftrightarrow v$ and $v-w \leftrightarrow w$.
\end{proof}
\begin{corollary}
Let $V$ be a Riesz space and $v,w\in V$, then
\[ |v - w| = (v \vee w) - (v \wedge w). \]
\end{corollary}
\begin{corollary} \label{meetJoinAbsoluteValues}
Let $V$ be a Riesz space and $v,w\in V$, then
\begin{enumerate}
\item $|v| \vee |w| = \frac{1}{2}\Big(|v|+|w| + \big||v| - |w|\big|\Big)$;
\item $|v| \wedge |w| = \frac{1}{2}\Big(|v|+|w| - \big||v| - |w|\big|\Big)$.
\end{enumerate}
\end{corollary}
\begin{proof}
Substitute $v\to |v|$ and $w\to |w|$.
\end{proof}
\begin{corollary}
Let $V$ be a Riesz space and $u,v,w\in V$, then
\begin{enumerate}
\item $|u\vee v - u\vee w| + |u\wedge v - u\wedge w| = |v-w|$;
\item $|v^+-w^+|\leq |v-w|$ and $|v^- - w^-|\leq |v-w|$.
\end{enumerate}
\end{corollary}
\begin{proof}
(1) Using the proposition, we get
\[ |u\vee v - u\vee w| + |u\wedge v - u\wedge w| = (u\vee v)\vee(u\vee w) - (u\vee v)\wedge (u\vee w) + (u\wedge v)\vee(u\wedge w) - (u\wedge v)\wedge(u\wedge w). \]
Using distributivity this simplifies to $(v\vee w) - (v\wedge w) = |v-w|$.

(2) Using (1) we have
\[ |v-w| = |0\vee v - 0\vee w| + |0\wedge v - 0\wedge w| = |v^+ -w^+| + |v^- - w^-| \geq \begin{cases}
|v^+ -w^+| \\
|v^- - w^-|.
\end{cases}  \]
\end{proof}

\begin{proposition}
Let $V$ be a Riesz space and $v,w\in V$, then
\begin{enumerate}
\item $|v|\vee |w| = \frac{1}{2}\Big( |v+w| + |v - w| \Big)$;
\item $|v|\wedge |w| = \frac{1}{2}\Big| |v+w| - |v - w| \Big|$;
\end{enumerate}
or, equivalently,
\begin{enumerate} \setcounter{enumi}{2}
\item $|v+w|\vee |v-w| = |v| + |w|$;
\item $|v+w|\wedge |v-w| = \big| |v| - |w| \big|$.
\end{enumerate}
Also
\begin{enumerate} \setcounter{enumi}{4}
\item $|v|+|w| = |v+w| + |v-w| - \big||v|-|w|\big|$;
\item $|v+w|+|v-w| = 2|v| + 2|w| - \big||v+w|-|v-w|\big|$;
\end{enumerate}
and
\begin{enumerate} \setcounter{enumi}{6}
\item $|v|+|w| = \big||v|-|w|\big| + \big||v+w|-|v-w|\big|$.
\end{enumerate}
\end{proposition}
The only real trick is in the proof of (1). All the other results follow from elementary substitutions.
\begin{proof}
(3,4,6) Are equivalent to (1,2,5) by the replacements $v \leftrightarrow v+w$ and $w \leftrightarrow v-w$.

(1) We calculate
\begin{align*}
|v|\vee |w| &= v\vee (-v)\vee w \vee (-w) = \big(v\vee(-w)\big)\vee \big((-v)\vee w\big) \\
&= \frac{1}{2}\Big((v-w) + |v + w|\Big)\vee \frac{1}{2}\Big( (-v+w) + |- v - w| \Big) \\
&= \frac{1}{2}|v+ w| + \frac{1}{2}\big((v-w)\vee (-v+w)\big) = \frac{1}{2}\Big( |v+w| + |v - w| \Big).
\end{align*}

(5) Using \ref{sumAsMeetJoin}, (1) and \ref{meetJoinAbsoluteValues} we get
\begin{align*}
|v|+|w| &= |v|\vee|w| + |v|\wedge |w| \\
&= \frac{1}{2}\Big( |v+w| + |v - w| \Big) + \frac{1}{2}\Big(|v|+|w| - \big||v| - |w|\big|\Big).
\end{align*}
This simplifies to the required equation.

(7) Follows from substituting (6) into (5).

(2) Follows from (7) and \ref{meetJoinAbsoluteValues}.
\end{proof}
\begin{corollary}
Let $V$ be a Riesz space and $v,w\in V$, then the following are equivalent:
\begin{enumerate}
\item $|v|\wedge |w| = 0$;
\item $|v+w| = |v-w|$;
\item $|v|\vee |w| = |v+w|$.
\end{enumerate}
\end{corollary}

The absolute value also satisfies the triangle inequality.
\begin{proposition}[Triangle and reverse triangle inequality in Riesz spaces]
Let $V$ be a Riesz space and $v,w\in V$, then
\[ |v| + |w| \geq \big|v+w\big| \geq \big||v|-|w|\big|. \]
\end{proposition}
\begin{proof}
The first inequality is the triangle inequality. It follows straight from \ref{triangleInequalityPositiveNegativeElements}.

The second inequality is the reverse triangle inequality and follows from the triangle inequality as in \ref{reverseTriangleInequality}.
\end{proof}

\subsection{Subsets}
\begin{definition}
Let $V$ be a Riesz space. A subset $E$ is called
\begin{itemize}
\item a \udef{Riesz subspace} if it is both a subspace and a sublattice;
\item \udef{solid} if for all $v\in E$ the interval $[-|v|,|v|]$ is a subset of $E$;
\item a \udef{band} if for all subsets $S\subseteq E$ we have $\sup(S) \subset E$. 
\end{itemize}
\end{definition}

\begin{lemma}
Let $V$ be a Riesz space and $v,w\in V$, then
\[ |w|\leq |v| \iff -|v| \leq w \leq |v|. \]
\end{lemma}
\begin{proof}
We have $w \leq |w|$ and $|w| \leq |v|$, so $w\leq |v|$. Also $-w \leq |-w| = |w|$, so $-|v| \leq -|w| \leq w$.

Conversely, $-|v| \leq w$ implies $-w\leq |v|$. So $|w| = w\vee (-w) \leq |v|$.
\end{proof}
\begin{corollary}
Let $V$ be a Riesz space and $E\subseteq V$ a subset. Then $E$ is solid \textup{if and only if}
\[ \forall v\in E: \forall w\in V: \; |w|\leq |v| \implies w\in E. \]
\end{corollary}

\begin{lemma}
Let $V$ be a Riesz space and $E\subseteq V$ a subset. Then $E$ is an (order) ideal \textup{if and only if} it is a solid Riesz subspace.
\end{lemma}
\begin{proof}
TODO
\end{proof}


\subsection{Disjointness}

\subsection{Archimedean}

\chapter{Some results and applications}
\section{Rotations}
Rodrigues' rotation formula

eigenvectors and eigenvalues of rotation.
\section{Pauli matrices}

\[ \sigma_x = \begin{pmatrix}
0 & 1 \\ 1 & 0
\end{pmatrix} \qquad \sigma_y = \begin{pmatrix}
0 & -i \\ i & 0
\end{pmatrix} \qquad \sigma_z = \begin{pmatrix}
1 & 0 \\ 0 & -1
\end{pmatrix} \]
All have eigenvalues $\pm 1$. The eigenspaces are spanned by
\[ v_{x+} = \frac{1}{\sqrt{2}}\begin{pmatrix}
1 \\ 1
\end{pmatrix}, \quad v_{x-} = \frac{1}{\sqrt{2}}\begin{pmatrix}
1 \\ -1
\end{pmatrix}, \quad v_{y+} = \frac{1}{\sqrt{2}}\begin{pmatrix}
1 \\ i
\end{pmatrix}, \quad v_{y-} = \frac{1}{\sqrt{2}}\begin{pmatrix}
1 \\ -i
\end{pmatrix}, \quad v_{z+} = \begin{pmatrix}
1 \\ 0
\end{pmatrix}, \quad v_{z-} = \begin{pmatrix}
0 \\ 1
\end{pmatrix}, \quad  \]

\[ \Tr[\sigma_i \sigma_j] = \delta_{ij} \]

