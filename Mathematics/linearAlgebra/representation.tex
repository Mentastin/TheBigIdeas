\chapter{Modules}
TODO: $*$-modules
\section{Representation theory}
A representation $G\to \GL(V)$ gives $V$ the structure of a $G$-module.

A map $\varphi$ between two representations $V,W$ of $G$ is a vector space map $\varphi: V\to W$ such that
\begin{center}
\begin{tikzcd}
V\arrow[r, "\varphi"]\arrow[d, "g"]& W\arrow[d, "g"] \\V\arrow[r, "\varphi"]& W
\end{tikzcd}
commutes for every $g\in G$.
\end{center}
In other symbols $\forall g\in G: g\varphi = \varphi g$.
We call such a map $G$-linear.

Define isomorphism and isomorphic.

Eg trivial representation: $gv = v$.

A \udef{subrepresentation} of a representation $V$ is a vector subspace $W$ of $V$ that is invariant under $G$: $\forall g\in G: g[W]\subseteq W$.

$\ker \varphi$, $\coker \varphi$, $\Im \varphi$ subrepresentations.

Irrep has no proper, non-trivial subrepresentation.

\begin{lemma} \label{existenceIrreps}
Every representation has an irreduciple subrepresentation.
\end{lemma}
\begin{proof}
If $V$ has no non-trivial subrepresentations, then $V$ is simple and we are done. 
Otherwise take the set of non-trivial subrepresentations. This forms a poset ordered by inclusion and by the maximal chain principle \ref{ZornEquivalents} this poset has a maximal chain $C$ and it is clear that $\bigcap C$ is a simple subrepresentation. In particular $\bigcap C$ is closed because it is an intersection of closed subspaces.
\end{proof}

Representation gives representation on dual.

Direct sums and tensor products of representations. Also symmetric and exterior powers.

$\Hom(V,W)$ has representation via $V^*\otimes W$.

A space is irreducible if and only if it is completely reducible and indecomposable.

\begin{proposition}
Let $\varphi$ be a $G$-linear map. If $\varphi$ is invertible as a function, its inverse is also $G$-linear.
\end{proposition}
\begin{proof}
Assume $\varphi$ invertible and take an arbitrary $g\in G$. Then
\[ g\varphi = \varphi g \implies \varphi^{-1} g = g \varphi^{-1} \]
by multiplying left and right by $\varphi^{-1}$. So
\[ \forall g\in G: g \varphi^{-1} = \varphi^{-1} g \]
meaning $\varphi^{-1}$ is $G$-linear.
\end{proof}

\begin{proposition}
Let $(V_1,\rho_1)$ and $(V_2,\rho_2)$ be isomorphic, then $V_1$ is irreducible \textup{if and only if} $V_2$ is irreducible.
\end{proposition}

\begin{proposition}
Let $(V,\rho)$ be a representation of $G$. Every element of the centre $Z(G)$ of $G$ defines an isomorphism $V\to V$.
\end{proposition}
\begin{proof}
Every $g_0\in Z(G)$ defines a $G$-linear map $\rho(g_0)$:
\[ \forall g\in G: \rho(g_0)\rho(g) = \rho(g_0g) = \rho(gg_0) = \rho(g) \rho(g_0). \]
The map $\rho(g_0)$ has an inverse $\rho(g_0^{-1})$.
\end{proof}

\begin{proposition}
Let $G$ be an Abelian group. The irreps of $G$ are $1$-dimensional and thus homomorphisms
\[ \rho: G\to \GL(\C). \]
\end{proposition}
\begin{proof}
Let $V$ be an irrep. The action of each $g\in G$ is an isomorphism and thus a scalar multiple by Schur's lemma. Thus every subspace of $V$ must be invariant, so also a subrepresentation. This means $V$ may not have any proper, non-trivial subspaces, meaning it is $1$-dimensional.
\end{proof}

\section{Hilbert modules}
Let $B$ be a $C^*$-algebra. A \udef{Hilbert $B$-module} $E$ is essentially a $B$-module with a $B$-valued inner product $\inner{\cdot, \cdot}_B: E\times E \to B$.

To be more precise: a (right) Hilbert $B$-module is a complex Banach space $E$ equipped with a right $B$-module structure and a positive\footnote{i.e.\ $\inner{\xi, \xi}_B$ is an element of the positive cone $B^+$.} definite $B$-valued inner product which is linear in the second and anti-linear in the first and satisfies, for all $\xi,\eta \in E$ and $b\in B$
\[ (\inner{\xi,\eta}_B)^* = \inner{\eta,\xi}_B, \qquad \inner{\xi, \eta}_B b = \inner{\xi,\eta\cdot b}_B, \qquad \text{and} \qquad \norm{\xi}^2 = \norm{\inner{\xi,\xi}_B}. \]
We can also define left Hilbert $B$-modules analogously. The Hilbert $\C$-modules are precisely the complex Hilbert spaces.

Any $C^*$-algebra $B$ can be seen as a Hilbert $B$-module by equipping it with the following inner product:
\[ \inner{\cdot, \cdot}_B: B\times B\to B: (a,b) \mapsto a^*b. \]

If $E$ and $F$ are Hilbert $B$-modules, then a map $T: E \to F$ is called \udef{adjointable} if there exists a map $T^*: F \to E$ such that for all $\xi\in E, \eta\in F: \inner{T\xi,\eta}_B = \inner{\xi, T^*\eta}_B$. Adjointable operators are bounded and $B$-linear.



\chapter{Algebras}
TODO $\GL(A)$ which forms group under multiplication.

Multiplicative map: preserves multiplication.

Anti-commute

representation = algebra homomorphism with linear operators on a vector space.

Envelope of a representation: module to algebra.

\section{Definition}
\begin{definition}
Vector space with associative bilinear function.
\end{definition}


\section{Semisimple algebra}
\begin{proposition}
Let $V$ be a Hilbert space with a subspace $U$. Let $A$ be a bounded linear operator on $V$. If $U$ is stable under $A$, then $U^\perp$ is stable under $A^*$.
\end{proposition}
\begin{proof}
Let $u\in U$ and $v\in U^\perp$. Then from
\[ \inner{u, A^*v} = \inner{Au,v} = 0 \]
we see that $A^*v \in U^\perp$. Thus $U^\perp$ is stable under $A^*$.
\end{proof}
\begin{corollary}
Let $D$ be a ring of bounded linear operators on the Hilbert space $V$. If $D=D^*$, then $V$ is a semisimple $D$-module.
\end{corollary}
\begin{proof}
Let $S$ be the set of direct sums of simple subrepresentations of $V$:
\[ S = \setbuilder{\bigoplus_{i\in I}V_i }{\text{$V_i$ simple subrepresentations of $V$ and all $V_i$ are orthogonal}}. \]
Then $S$ is a poset ordered by inclusion. Now any chain $C$ in $S$ has an upper bound $\bigcup C$ and $\bigcup C$ is in $S$ because every $v\in\bigcup C$ can be written uniquely as a finite linear sum
\[ v = \sum_{\substack{i\in J\\ \text{$J$ finite}}} v_i \qquad v_i\in V_i. \]
By Zorn's lemma $S$ has a maximal element $U$, which is closed by \ref{directSumOrthogonalClosed}. We now claim that $U=V$. Assume, towards a contradiction, that $U\neq V$. Then $U^\perp$ is stable under $D$ by the proposition, closed by \ref{orthogonalComplementClosed} and thus contains a simple subrepresentation $W$ by \ref{existenceIrreps}. Then $U\subset U\oplus W \in S$, meaning $U$ is not a maximal element. This is a contradiction.
\end{proof}
Note that for the corollary it is important that $V$ be a Hilbert space, not only for the condition $D=D^*$ which could also be fulfilled by a set of symmetric operators or a group of unitary operators.


\section{Graded and filtered algebras}
TODO; move to rings. TODO move filtration:
\begin{definition}
Let $X$ be a set. A \udef{filtration} on $X$ is a family of subsets $\seq{X_i}_{i=0}^\infty$ such that $X_i \subseteq X_{i+1}$ and $X = \bigcup_{i=0}^\infty X_i$.
\end{definition}

\subsection{Graded algebras}
\begin{definition}
Let $A$ be an algebra and let $S$ be a semigroup. An \udef{$S$-grading} on $A$ is a set $\{A_s\}_{s\in S}$ of vector subspaces of $A$ indexed by $S$ such that $A_sA_t \subseteq A_{st}$ and $A = \bigoplus_{s\in S}A_s$.
\end{definition}


\begin{proposition}
Let $A$ be an algebra over a field $\F$, $f:A\to A$ a diagonalisable algebra homomorphism. Then
\begin{enumerate}
\item $\spec(f)$ is a multiplicative subsemigroup $S$ of $\F$;
\item $A_s \defeq \setbuilder{a\in A}{f(a) = sa}$ defines an $S$-grading on $\F$.
\end{enumerate}
\end{proposition}
TODO!!!

\begin{corollary}
Let $A$ be an algebra and $f$ and involutive algebra homomorphism. Then this involution defines a $Z_2$-grading.
\end{corollary}

\subsubsection{Grade operator}
\begin{definition}
Let $A = \bigoplus_{i=0}^\infty A_k$ be a graded algebra. Then, for $r\in \N$, we call the projection $A\to A_r: a\mapsto \grade{a}_r$ respecting this decomposition the \udef{grade operator}.
\end{definition}

\subsection{$\Z_2$-graded or superalgebras}

\subsection{Filtered algebra}
\begin{definition}
A \udef{filtered algebra} is an algebra $F$ together with a filteration $\seq{F_i}_{i=0}^\infty$ of subspaces such that $F_i\cdot F_j \subseteq F_{i+j}$.
\end{definition}

\subsubsection{Associated graded algebra}
\begin{definition}
Let $\sSet{F,\seq{F_i}_{i=0}^\infty}$ be a filtered algebra. The \udef{associated graded algebra} is defined as
\[ G = \bigoplus_{i=0}^\infty G_i \qquad\text{where}\qquad G_i = \begin{cases}
F_0 & (i=0) \\
F_{i}/F_{i-1} & (i \geq 1).
\end{cases} \]
\end{definition}
TODO define the multiplication!

TODO: $G$ is isomorphic to $F$ as vector space, but \emph{not} as an algebra!!


\section{Tensor algebra}
\[ \mathcal{T}(V) \defeq \R \bigoplus_{n=1}^\infty V^n = \R \bigoplus_{n=1}^\infty \underbrace{V\otimes \ldots \otimes V}_{\text{$n$ times}}. \]

Transpose: $v\otimes w \to w\otimes v$.

\subsection{Tensor product}
+ Graded tensor product

\section{Matrix algebras}
\begin{definition}
TODO
\end{definition}

\subsection{Natural isomorphism}
Remove parentheses block matrix.

Also $A^{n\times n} \cong \C^{n\times }\otimes A$.

\chapter{Lie groups and Lie algebras}
\section{Definitions}
\[ g(x) = \exp(ix^aX_a) \]
Lie algebra has the operation $[X_a,X_b]= X_aX_b - X_bX_a$. 
\section{Matrix groups}
We now consider some extremely important examples of topological groups: the matrix groups.
If we take the set of real, $N\times N$ matrices with a non-zero determinant, it turns out that they form a group with the matrix multiplication:
\begin{enumerate}
\item The matrix multiplication is associative;
\item The identity is the identity matrix $\mathbb{1}$;
\item Because their determinant is not zero, every matrix in this set has an inverse.
\item Because the matrices are square, the multiplication of two matrices gives a matrix of the same dimensions. In other words the matrix multiplication is a closed operation.
\end{enumerate}
We call this group the \udef{real general linear group} $\GL(N, \R)$. It also has a complex counterpart, the complex general linear group $\GL(N, \C)$.

TODO: topological
We can also immediately see that the operations of matrix multiplication and inversion are smooth. (For inversion this is obviously only true after restriction to the open subset of invertible matrices, which luckily all matrix Lie groups are in turn a subset of). This follows quite readily because both operations are in effect comprised of addition and multiplication operations, which are infinitely differentiable. (e.g\ $A^{-1} = \frac{1}{\det(A)}\mathrm{adj}(A)$)

\begin{example}
TODO: $A^2 = \mathbb{1}$
\end{example}

These groups, along with all their subgroups, are known as the matrix groups and are very important in physics.

\subsubsection{Continuous parameters}
It is sometimes interesting to know how many degrees of freedom a particular set of transformations has. For example, rotations in the 2D plane are characterized with one parameter: the angle of rotation. In 3D we need three parameters. This notion of continuous parameter is formalised below.

\begin{definition}
A function $A : \R \to \GL(n, \C)$ is called a \udef{one-parameter subgroup} of $\GL(n, \C)$ if
\begin{enumerate}
\item $A$ is continuous,
\item $A(0) = \mathbb{1}_n$,
\item $A(t+s) = A(t)A(s)$ for all $t,s \in \R$.
\end{enumerate}
We also call the image of $A$ a one-parameter subgroup.
\end{definition}

A one-parameter subgroup has one continuous parameter. A subgroup of $\GL(n, \C)$ with $m$ continuous parameters, is a function $A : \R^m \to \GL(n, \C)$ such that each function of the form
\[ x \mapsto A(a_1, a_2, \ldots , a_{i-1}, x, a_{i+1}, \ldots, a_m) \]
gives a one-parameter subgroup for fixed $a_1,\ldots, a_m$.

We can speak of an $m$-parameter subgroup because, while different parametrisations may be found, any subgroup of $\GL(n,\C)$ constructed in this way must always be constructed with the same number of parameters. To see that this must be the case, consider two parametrised subgroups $A : \R^m \to \GL(n, \C)$ and $B : \R^{m'} \to \GL(n, \C)$ with the same image.

TODO !! + dimension of manifold

\subsubsection{Examples}
We now give names to the most important matrix groups, and list the number of continuous parameters.
\begin{enumerate}
\item General linear group
\[ \GL(N,\R) = \{N\times N \;\text{real matrices}, \; \det M \neq 0\} \]
\begin{itemize}
\item We have $N^2$ independent parameters (= the entries of the matrix), so $\dim \GL(N,\R) = N^2$
\item Each complex number can be described with two real ones, so $\dim \GL(N, \C) = 2N^2$
\end{itemize}
\item Special linear group
\[ \SL(N,\R) = \{M\in\GL(N,\R), \;\det M = 1\} \]
\begin{itemize}
\item $\dim \SL(N,\R) = N^2-1$: 1 dimension is used to fix determinant.
\item $\dim \SL(N,\C) = 2(N^2-1)$: 1 dimension is used to fix the real part of the determinant, and 1 to fix the imaginary part.
\end{itemize}
\item Unitary matrices
\[ \U(N) = \{U\in\GL(N,\C), U^\dagger\mathbb{1}_NU = \mathbb{1}_N\} \]
\begin{itemize}
\item $U^\dagger U$ is Hermitian, meaning that the complex transpose of $U$ is $U$.
\item $U^\dagger U = \mathbb{1}_N$ yields only $N^2$ independent equations, not $2N^2$ because of the Hermiticity of the equation.
\item $\dim \U(N) = 2N^2-N^2$ = $N^2$
\end{itemize}
\item Special unitary groups
\[ \SU(N) = \{U\in\U(N),\; \det U = 1\} \]
\begin{itemize}
\item For unitary matrices we have that $|\det U| = 1$. This fixes one continuous parameter and thus one dimension.
\item $\dim \SU(N) = N^2 - 1$
\end{itemize}
\item Orthogonal groups
\begin{itemize}
\item $\Ogroup(N) = \{O\in\GL(N,\R),\; O^\intercal\mathbb{1}_N O = \mathbb{1}_N\}$
\begin{itemize}
\item $O^\intercal O$ is symmetric, so $\frac{N(N+1)}{2}$ independent equations (half the matrix already fixed by the other half)
\item $\dim \Ogroup = N^2 - \frac{N(N+1)}{2} = \frac{N(N-1)}{2}$
\end{itemize}
\item $\SO(N) = \{O\in\Ogroup(N),\; \det O = 1\}$
\begin{itemize}
\item For orthogonal matrices $\det O = \pm 1$. This does not fix any continuous parameters.
\item $\dim \SO = \dim \Ogroup = \frac{N(N-1)}{2}$
\end{itemize}
\end{itemize}
\item Using a non definite metric $\eta = \diag(\mathbb{1}_p, -\mathbb{1}_q)$
\begin{itemize}
\item $\U(p,q) = \{ U\in \GL(N,\C), U^\dagger\eta U = \eta \}$
\item $\Ogroup(p,q) = \{ O\in \GL(N,\R), O^\intercal\eta O = \eta \}$ \\
In particular $\SO(1,3)$ is the \udef{Lorentz group} (with mostly minus convention).
\end{itemize}
\end{enumerate}

Here are some of the most important examples written more explicitly in terms of their continuous parameters:
\begin{itemize}
\item $\U(1) \equiv \{z\in\mathbb{C}|\; |z|=1\}, \boldsymbol{\cdot}$ has one real parameter. Every element $z$ of this group can be written $z=e^{i\alpha}$ for a real $\alpha$.
\item $\SO(2)$ has one real parameter.
\[ R(\theta) = \begin{pmatrix}\cos(\theta) & -\sin(\theta)\\ \sin(\theta) & \cos(\theta)\end{pmatrix} \]
\item $\SO(3)$ has three real parameters.
\[ R(\theta_{12},\theta_{13},\theta_{23}) = R_1(\theta_{12})R_2(\theta_{13})R_3(\theta_{23}) \]
where
\[R_1(\theta_{12}) = \begin{pmatrix}\cos(\theta_{12}) & -\sin(\theta_{12})&0\\ \sin(\theta_{12}) & \cos(\theta_{12})&0\\0&0&1\end{pmatrix}\]
\[R_2(\theta_{13}) = \begin{pmatrix}\cos(\theta_{13}) &0& -\sin(\theta_{13})\\0&1&0\\ \sin(\theta_{13}) &0& \cos(\theta_{13})\end{pmatrix}\]
\[R_3(\theta_{23}) = \begin{pmatrix}1&0&0\\ 0&\cos(\theta_{23}) & -\sin(\theta_{23})\\0& \sin(\theta_{23}) & \cos(\theta_{23})\end{pmatrix}\]
\item $\SU(2)$ has three real parameters and its elements can be seen as complex $2\times 2$ rotations.
\[ \U(\alpha, \beta, \gamma) = \begin{pmatrix}\cos\theta e^{i\alpha} & -\sin\theta e^{i\beta}\\ \sin\theta e^{-i\beta} & \cos\theta e^{-i\alpha}\end{pmatrix} \]
\end{itemize}


\chapter{Representation theory}
\section{Finite groups}
\subsection{Character tables}
\subsubsection{For $\mathbb{Z}_n$}
Denoting $\mathbb{Z}_n = \{\bar{0}, \bar{1}, \bar{2},\ldots, \overline{n-1}\}$
\[ \begin{array}{l|lllll}
g_i & \bar{0} & \bar{1} & \bar{2} & \hdots & \overline{n-1} \\
|\text{Cl}| & 1 & 1 & 1 & \hdots & 1 \\ \hline
\chi_0 & 1 & 1 & 1 & \hdots & 1 \\
\chi_1 & 1 & \omega_n & \omega^2_n & \hdots & \omega_n^{n-1} \\
\chi_2 & 1 & \omega_n^2 & \omega^4_n & \hdots & \omega_n^{2(n-1)} \\
\vdots & \vdots & \vdots & \vdots &  & \vdots \\
\chi_{n-1} & 1 & \omega_n^{n-1} & \omega_n^{2(n-1)} & \hdots & \omega_n^{(n-1)(n-1)}
\end{array} \]

\subsection{Complete reducibility of complex representations}

\subsection{Schur's lemma, isotypic decomposition and duals}
\subsection{Orthogonality in the character tables}
\subsection{The sum of squares formula}
\subsection{The number of irreps is the number of conjugacy classes}
\subsection{Dimensions of irreps divide the order of the group}