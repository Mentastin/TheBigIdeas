\chapter{Bilinear and multilinear maps}
TODO: bilinear maps, bilinear forms = bilinear functionals
orthogonality
\section{Quadratic forms}
\begin{definition}
Let $V$ be a finite-dimensional vector space over a field $\mathbb{F}$. A \udef{quadratic form} defined on $V$ is a function
\[ q: V \to \mathbb{F} \]
 such that
 \begin{itemize}
 \item $q(\lambda v) = \lambda^2 q(v)$ for all $\lambda\in\mathbb{F}$ and $v\in V$;
 \item the function
 \[ V\times V \to \mathbb{F}: (u,v)\mapsto q(u,v) = q(u+v)-q(u) -q(v) \]
is bilinear. 
 \end{itemize}
\end{definition}
Sylvester's law of inertia

\section{Tensor product}
\url{https://kconrad.math.uconn.edu/blurbs/linmultialg/tensorprod.pdf}
\subsection{Free vector space}
Given any set, we can construct a vector space by viewing each element in the set as a (linearly) independent (basis) vector. The vector space then consists of formal linear combinations of these vectors.

To be more precise:
\begin{definition}
Let $S$ be a set and $K$ a field. Then define
\[ F_K(S) \defeq \setbuilder{f\in(S\to K)}{f^{-1}[K\setminus\{0\}]\;\text{is finite}}. \]
Define the following operations on $F_K(S)$:
\begin{align*}
+ &: F(S)\times F(S) \to F(S): (f,g)\mapsto (f+g: x\mapsto f(x)+g(x)) \\
\cdot &: K\times F(S) \to F(S): (\lambda,f)\mapsto (\lambda f: x\mapsto \lambda f(x)) \\
\end{align*}
The operations $+,\cdot$ are well-defined and make $F(S)$ into a vector space, called the \udef{free vector space} over $S$.
\end{definition}

\begin{proposition}
Let $S$ be a set. Then we can identify $S$ with a subset of $F(S)$ by
\[ \iota: S\hookrightarrow F(S): x\mapsto \chi_{\{x\}}. \]
With this identification $S$ forms a basis for $F(S)$.
\end{proposition}

\begin{lemma}
Let $V$ be a vector space over $K$ and $\beta$ a basis for $V$. Then $V\cong  F_K(V)$.
\end{lemma}

\subsubsection{The free functor}
\begin{proposition}
The operation $F$ of finding the free vector space over a set can be extended to an contravariant functor
\[ F: \cat{Set} \to \cat{Vect}. \]
\end{proposition}
\begin{proof}
Let $f:X\to Y$ be a function between sets.
\end{proof}
TODO: just specific instance up to isomorphism?? covariant?? isomorphism class of all vector spaces with basis $S$?? Is this why tensor product only up to isomorphism??

\subsubsection{Universal property}
\begin{proposition}
Let $\phi$ be an arbitrary function from $S$ to a vector space $W$ over a field $K$, then there exists a unique linear map $\overline{\phi}: F(S)\to W$ such that the diagram
\[ \begin{tikzcd}
S \ar[r,"\iota"] \ar[dr,"\phi"] & F(S) \ar[d,dashed,"\overline{\phi}"] \\
& W
\end{tikzcd} \qquad \text{commutes.} \]
Furthermore, $F(S)$ is the unique $K$-vector space with this property.
\end{proposition}

\subsection{Abstract definition}
The idea behind the tensor product of two vector spaces $V,W$ over a field $K$ is to create the most general set of pairings that is a vector space and such that the pairings are bilinear: $\forall \lambda\in K: \forall v_1,v_2,v\in V:\forall w_1,w_2,w\in W$:
\[ (\lambda v_1+v_2, w) = \lambda (v_1,w)+(v_2,w) \qquad\text{and}\qquad (v,\lambda w_1+w_2) = \lambda (v,w_1) + (v,w_2). \]
This will be realised as a quotient of a free vector space.

To be more precise:
\begin{definition}
Let $V,W$ be vector spaces over a field $K$.
Consider the set $\operatorname{Field}(V)\times \operatorname{Field}(W)$, which we will refer to as $V\times W$. Construct the sets
\begin{align*}
R_1 &= \setbuilder{(\lambda(v_1,w)+(v_2,w),(\lambda v_1+v_2, w))\in F_K(V\times W)}{\lambda\in K; v_1,v_2\in V; w\in W} \\
R_2 &= \setbuilder{(\lambda(v,w_1)+(v,w_2),(v, \lambda w_1 + w_2))\in F_K(V\times W)}{\lambda\in K; v_1,v_2\in V; w\in W} \\
R &= R_1\cup R_2
\end{align*}
The \udef{tensor product} $V\otimes W$ of the vector spaces $V$ and $W$ is the quotient vector space
\[ V\otimes W := F(V\times W)/R^\equiv \]
where $R^\equiv$ is the reflexive symmetric transitive closure of $R$.

The equivalence class $[(v,w)]$ is denoted $v\otimes w$. An element of $V\otimes W$ that can be written as $v\otimes w$ is called a \udef{pure tensor} or \udef{simple tensor}.
\end{definition}
In order for the definition to be well-defined, we need for $R^\equiv$ to be a congruence on $F$.

The \udef{tensor product} $V\otimes W$ of two vector spaces $V$ and $W$ over a common field $K$ is the quotient vector space
\[ V\otimes W := F(V\times W)/\sim \]
where $\sim$ is the equivalence relation over the the free vector space $F(V\times W)$ with the properties of
\begin{itemize}
\item \textit{Distributivity}: $(v+v', w) \sim (v,w) + (v',w)$ and $(v, w+w') \sim (v,w) + (v,w')$.
\item \textit{Scalar multiples}: $c(v,w) \sim (cv,w) \sim (v,cw)$.
\end{itemize}



TODO definition via bases: $V\otimes W = F(\beta_V\times \beta_W)$

\subsection{Universal property}
See also proposition \ref{dimHomset}.


\subsection{Tensor product of linear maps}
The tensor product also operates on linear maps between vector spaces.
\begin{definition}
Given two linear maps $S: V\to X$ and $T:W\to Y$, then the \udef{tensor product} of the linear maps $S$ and $T$ is the linear map
\[ S\otimes T: V\otimes W \to X\otimes Y \]
defined by
\[ (S\otimes T)(v\otimes w) = S(v)\otimes T(w). \]
For vectors that are not pure tensors, this definition is extended by linearity.
\end{definition}
\begin{lemma}
The tensor product of linear maps is well-defined.
\end{lemma}

With this definition the tensor product becomes a bifunctor from the category of vector spaces to itself, covariant in both arguments.

TODO: functional calculus on tensor product.
TODO: tensor product of operator algebras


\subsection{Operator-valued matrices}
\begin{proposition}
Let $A$ be an algebra over a field $\mathbb{F}$ and $\beta$ any set. Consider the direct sum $A^\beta = \bigoplus_{i\in\beta}A$. Then $A^\beta \cong F_\F(\beta)\otimes A$.
\end{proposition}
\begin{proof}
TODO
\end{proof}


\subsection{Matrix representation}
\subsubsection{Finding a basis}
Assume $V$ and $W$ are finite-dimensional vector spaces with resp. bases $\{\vec{e}_i\}_i$ and $\{\vec{f}_j\}_j$. Then the set $\{ \vec{e}_i\otimes \vec{f}_j \}_{i,j}$ forms a basis for $V\otimes W$. Indeed,
\begin{itemize}
\item Take two arbitrary vectors $\vec{v} = \sum_i a_i \vec{e}_i \in V$ and $\vec{w} = \sum_j b_j \vec{f}_j \in W$.
Using the distributivity and scalar multiples properties of $\sim$, we can write the tensor product $\vec{v}\otimes \vec{w}$ as
\begin{equation} \vec{v}\otimes \vec{w} = (\sum_i a_i \vec{e}_i)\otimes(\sum_j b_j \vec{f}_j) = \sum_{i,j}a_ib_j (\vec{e_i}\otimes \vec{f}_j). \label{eq:vtensorw} \end{equation}
So any pure tensor can be written as the sum of vectors of the form $\vec{e}_i\otimes \vec{f}_j$. In general a vector in $V\otimes W$ can be written as a finite sum of pure tensors, meaning the set of vectors $\{ \vec{e}_i\otimes \vec{f}_j \}_{i,j}$ spans $V\otimes W$.
\item For linear independence we, observe that for any linearly independent $v_1, v_2, w_1, w_2$, the vector $v_1\otimes w_1 + v_2\otimes w_2$ cannot be written as a pure tensor.
\end{itemize}

Clearly it follows that
\[ \dim(V\otimes W) = \dim(V)\cdot\dim(W) \]

\subsubsection{Coordinates and the outer product}
The coordinates of a vector with respect to the basis $\{ \vec{e}_i\otimes \vec{f}_j \}_{i,j}$ can naturally be put into a matrix. Taking the tensor product of two vectors corresponds to taking the outer product of their coordinate vectors. That is, setting $\co(v) = \vec{v}$ and $\co(w) = \vec{w}$, we get
\[ \co(v\otimes w)_{i,j} = a_ib_j = (\vec{v}\vec{w}^\transp)_{i,j} \]
which follows from \eqref{eq:vtensorw} above.

For this reason $\otimes$ is also used to denote the outer product.

If we want a proper column vector as our coordinate vector, we can apply row-by-row vectorisation to this matrix.
\[ \co(v\otimes w) = \vectorisation_R(\vec{v}\vec{w}^\transp) = \vec{v}\otimes\vec{w} = \co(v)\otimes\co(w). \]
where $\otimes$ is also used to denote the Kronecker product.

Coordinates for vectors that are not pure tensors can easily be found by the linearity of the coordinate map.
\subsubsection{Linear maps and the Kronecker product}
Letting the coordinates be columns, we can hope to find a matrix for the linear map $S\otimes T$. Fix bases for the spaces $V,W,X,Y$. Let $A$ and $B$ be the matrices of $S$ and $T$ with respect to these bases. Use these bases to fix the bases for $V\otimes W$ and $X\otimes Y$.

\begin{eigenschap}
The matrix of the map $S\otimes T$ with respect to these bases is the matrix $A\otimes B$, where $\otimes$ is the Kronecker product.
\end{eigenschap}

This follows from a simple calculation:
\begin{align*}
\co\left(S\otimes T(v\otimes w)\right) &= \co\left(S(v)\otimes T(w)\right) \\
&= \co\left(S(v)\right)\otimes \co\left(T(w)\right) \\
&= A\co(v)\otimes B\co(v) \\
&= (A\otimes B)\co(v)\otimes\co(w) \qquad (\text{using the mixed product}) \\
&= (A\otimes B)\co(v\otimes w).
\end{align*}
Again this calculation can be extended to non-pure tensors by linearity.

\subsection{Properties}
TODO currying.
\url{https://math.stackexchange.com/questions/679584/why-is-texthomv-w-the-same-thing-as-v-otimes-w}
And reference later!
\subsection{Multilinear maps}
\begin{definition}
Let $V^k = V\times \ldots \times V$. A function $f: V^k\to \R$ is \udef{$k$-linear} if it is linear in each of its arguments.
\begin{itemize}
\item A $k$-linear function $f:V^k\to \R$ is \udef{symmetric} if for all permutations $\sigma\in S_k$
\[ f(v_{\sigma(1)},\ldots, v_{\sigma(k)}) = f(v_1,\ldots, v_k). \]
\item A $k$-linear function $f:V^k\to \R$ is \udef{alternating} if for all permutations $\sigma\in S_k$
\[ f(v_{\sigma(1)},\ldots, v_{\sigma(k)}) = (\sgn\sigma)f(v_1,\ldots, v_k). \]
\end{itemize}
We call the space of all alternating $k$-linear maps $A_k(V)$.
\end{definition}
In particular $A_1(V) = V^*$.
\begin{note}
Given a $k$-linear function $f$ and a permutation $\sigma\in S_k$, we define the $k$-linear function $\sigma f$ by
\[ (\sigma f)(v_1,\ldots, v_k) = f(v_{\sigma(1)},\ldots, v_{\sigma(k)}). \]
Then a symmetric map is one such that $\sigma f = f$ for all $\sigma\in S_k$ and an alternating map is one such that $\sigma f = (\sgn \sigma)f$ for all $\sigma\in S_k$.
\end{note} 
\begin{lemma}
Let $\sigma,\tau \in S_k$ and $f$ a $k$-linear map on $V$. Then $\tau(\sigma f) = (\tau \sigma)f$.
\end{lemma}
\subsubsection{The symmetrising and alternating maps}
\begin{definition}
Let $f$ be a $k$-linear map on a vector space $V$.
\begin{itemize}
\item The \udef{symmetrisation} of $f$, $Sf$, is the map
\[ Sf = \frac{1}{k!}\sum_{\sigma\in S_k}\sigma f. \]
\item The \udef{anti-symmetrisation} or \udef{skew-symmetrisation} of $f$, $Af$, is the map
\[ Af = \frac{1}{k!}\sum_{\sigma\in S_k}(\sgn \sigma)\sigma f. \]
\end{itemize}
\end{definition}
\begin{lemma}
\begin{enumerate}
\item The $k$-linear map $Sf$ is symmetric. If $f$ is symmetric, then $Sf = f$.
\item The $k$-linear map $Af$ is alternating. If $f$ is alternating, then $Af = f$.
\end{enumerate}
\end{lemma}
\begin{lemma} \label{idempotenceA}
Let $f$ be a $k$-linear functional and $g$ an $l$-linear functional on $V$. Then
\[ A(A(f)\otimes g) = A(f\otimes g) = A(f\otimes A(g)). \]
\end{lemma}
\subsubsection{The wedge product}
\begin{definition}
Let $f\in A_k(V)$ and $g\in A_l(V)$. The \udef{wedge product} of $f$ and $g$ is given by
\[ f\wedge g = \frac{(k+l)!}{k!l!}A(f\otimes g). \]
\end{definition}
We can also write
\[ (f\wedge g)(v_1,\ldots,v_{k+l}) = \frac{1}{k!l!}\sum_{\sigma\in S_{k+l}}(\sgn \sigma)f(v_{\sigma(1)},\ldots,v_{\sigma(k)})g(v_{\sigma(k+1)},\ldots, v_{\sigma(k+l)}). \]
We can reduce redundancies in this definition in the following way:
We call $\sigma\in S_{k+l}$ a \udef{$(k,l)$-shuffle} if
\[ \sigma(1)<\ldots<\sigma(k) \qquad \text{and}\qquad \sigma(k+1)<\ldots<\sigma(k+l). \]
The we write
\[ (f\wedge g)(v_1,\ldots,v_{k+l}) = \sum_{\text{$(k,l)$-shuffles $\sigma$}}(\sgn \sigma)f(v_{\sigma(1)},\ldots,v_{\sigma(k)})g(v_{\sigma(k+1)},\ldots, v_{\sigma(k+l)}). \]
\begin{proposition}
Let $f\in A_k(V)$ and $g\in A_l(V)$. Then
\[ f\wedge g = (-1)^{kl}g\wedge f. \]
\end{proposition}
\begin{lemma}
The wedge product is associative:
\[ (f\wedge g)\wedge h = f\wedge (g\wedge h). \]
\end{lemma}
Proof using \ref{idempotenceA}.

\begin{lemma}
Let $\alpha^1,\ldots, \alpha^k$ be linear functionals on $V$ and $v_1,\ldots,v_k\in V$, then
\[ (\alpha^1\wedge\alpha^k)(v_1,\ldots, v_k) = \det[\alpha^i(v_j)]. \]
\end{lemma}

\subsection{Tensors}
A $(p,k)$-tensor is a multilinear function $V^k\to V^p$.

\section{Real, complex and quaternionic vector spaces}
\begin{definition}
A function $f$ between complex vector spaces is \udef{anti-linear} (or \udef{conjugate-linear}) in the first component:
\[f(\lambda_1 v_1 + \lambda_2 v_2) = \overline{\lambda_1}f(v_1) + \overline{\lambda_2}f(v_2),\]
where $\lambda_1,\lambda_2 \in \C$ and $v_1,v_2\in \dom(f)$.
\end{definition}
\subsection{Complex structure on a real vector space}
\begin{definition}
Let $V$ be a real vector space. A \udef{complex structure} on $V$ is a linear map $J: V\to V$ such that $J^2 = -I_V$.
\end{definition}

\subsection{The real vector spaces associated to a complex vector space}
Let $V = (\C, V, +)$. Then define $V_\R \defeq (\R,V,+)$.

every anti-linear map $A:V\to W$ is an $\R$-linear map $A:V_\R\to W_\R$. (They are equal as sets).

\section{Clifford algebras}
\begin{definition}
Let $V$ be a vector space over a field $\mathbb{F}$ and $q$ a quadratic form defined on $V$.
Let $\mathcal{T}(V)$ be the tensor algebra
\[ \mathcal{T}(V) \defeq \mathbb{F}\oplus \bigoplus_{n=1}^\infty V^n = \mathbb{F}\oplus \bigoplus_{n=1}^\infty \underbrace{V\otimes \ldots \otimes V}_{\text{$n$ times}}. \]
Let $\mathcal{I}(V,q)$ be the (two-sided) ideal in $\mathcal{T}(V)$ generated by
\[ \setbuilder{v\otimes v - q(v)\cdot \vec{1}}{v\in V}. \]
Then the \udef{Clifford algebra} $\Cl(V,q)$ associated with $V$ and $q$ is the quotient
\[ \Cl(V,q) \defeq \mathcal{T}(V)/\mathcal{I}(V,q). \]
\end{definition}
Let $\pi_q$ be the canonical projection
\[ \pi_q: \mathcal{T}(V) \to \Cl(V,q). \]

\begin{lemma}
The embedding $V\hookrightarrow \Cl(V,q)$ is faithful, i.e. $\pi_q|_V$ is injective.
\end{lemma}
\begin{proof}
TODO
\end{proof}
Clearly $\pi_q|_V(v)^2 = q(v)\cdot \vec{1}$ for all $v\in V$.


\begin{lemma} \label{CliffordRelations}
The algebra $Cl(V,q)$ is generated by the vector space $V$ and $\vec{1}$, subject to the relations
\[ v\cdot v = q(v)\cdot \vec{1} \qquad \forall v\in V. \]
If the characteristic of the field $\mathbb{F}$ is not $2$, then 
\[ v\cdot w + w\cdot v = [q(v+w)-q(v)-q(w)]\cdot \vec{1} = q(v,w)\cdot \vec{1}. \]
\end{lemma}

Clifford algebras can also be defined by their universal property:
\begin{proposition} \label{CliffordUniversalProperty}
Let $V$ be a vector space over a field $\mathbb{F}$ and $q$ a quadratic form on $V$. 

Then for any unital associative algebra $A$ over $\mathbb{F}$ and linear map $j: V \to A$ such that
\[ j(v)^2 = q (v)\cdot \vec{1} \qquad \forall v\in V \]
there exists a unique algebra homomorphism $\widetilde{j}: \Cl(V,q)\to A$ such that the following diagram commutes:
\[ \begin{tikzcd}
V \rar{\pi_q|_V} \ar[dr, swap, "{j}"] & \Cl(V,q) \dar[dashed]{\widetilde{j}} \\
 & A
\end{tikzcd} \]
Furthermore, $\Cl(V,q)$ is the unique associative $\mathbb{F}$-algebra with this property.
\end{proposition}
\begin{corollary}
Let $(V,q)$ and $(V',q')$ be vector spaces with quadratic forms. If a linear map $f:V\to V'$ preserves to quadratic form, $q'\circ f = q$, then $f$ extends to a unique algebra homomorphism
\[ \widetilde{f}: \Cl(V,q) \to \Cl(V',q'). \]
Now let $(V^{\prime\prime},q^{\prime\prime})$ be another vector space equipped with a quadratic form and let $g: V'\to V^{\prime\prime}$ be a linear map preserving the quadratic form. Then
\[ \widetilde{g\circ f} = \widetilde{g}\circ\widetilde{f}. \]
Also isomorphisms of vector spaces extend to isomorphisms of Clifford algebras.
\end{corollary}
\begin{proof}
Let the algebra $A$ of the proposition be $\Cl(V',q')$. Then $\pi_q|_V\circ f$ satisfies the requirement for $j$:
\[ [(\pi_{q'}|_V\circ f)(v)]^2 = q'(f(v))^2\cdot \vec{1} = q(v)^2\cdot \vec{1}. \]
Thus by the proposition, there is a unique extension of $f:V\to V'$ to a map $\Cl(V,q) \to \Cl(V',q')$.

The composition relation follows from uniqueness.
\end{proof}
\begin{corollary} \label{qOrthogonalMaps}
The orthogonal group
\[ \Ogroup(V,q) = \setbuilder{g\in\GL(V)}{q\circ g = q} \]
extends canonically to a group of automorphisms of $\Cl(V,q)$:
\[ \Ogroup(V,q) \subset \Aut(\Cl(V,q)). \]
\end{corollary}

Graded algebra.

\begin{proposition}
Let $V$ be a vector space and $q$ a quadratic form on $V$.
\begin{enumerate}
\item As graded algebras, $\Cl(V,q)$ is naturally isomorphic to the exterior algebra ${\textstyle\bigwedge}^* V$.
\item As algebras ${\textstyle\bigwedge}^* V \cong \Cl(V,0)$.
\item As vector spaces, there is an isomorphism
\[ {\textstyle\bigwedge}^* V \to \Cl(V,q): v_1\wedge \ldots \wedge v_n \mapsto \frac{1}{r!}\sum_{\sigma\in S_n}\sgn(\sigma)v_{\sigma(1)}\hdots v_{\sigma(r)} \]
compatible with the fibrations.
\end{enumerate}
\end{proposition}

\subsection{Involutions}
\subsubsection{Grade involution}
Given a Clifford algebra $\Cl(V,q)$, consider the map $\alpha: V \to V: v\mapsto -v$ on the vector space $V$.
Now $\alpha$ is always an element of $\Ogroup(V,q)$, so by \ref{qOrthogonalMaps} it extends to a map on the Clifford algebra.
\[ \widetilde{\alpha}: \Cl(V,q) \to \Cl(V,q). \]
Since $\alpha^2 = I_V$, we have that
\[ \widetilde{\alpha}^2 = \widetilde{\alpha^2} = \widetilde{I_V} = I_{\Cl(V,q)} \]
meaning $\widetilde{\alpha}$ is an involution on the Clifford algebra. From now on we drop the tilde and just write $\alpha: \Cl(V,q) \to \Cl(V,q)$ for the \udef{grade involution}.

\begin{lemma}
The grade involution $\alpha: \Cl(V,q) \to \Cl(V,q)$
\begin{enumerate}
\item is an algebra homomorphism and thus multiplicative:
\[ \alpha(xy) = \alpha(x)\alpha(y); \]
\item is unital, $\alpha(\vec{1}) = \vec{1}$, and thus preserves inverses:
\[ \alpha(x^{-1}) = \alpha(x)^{-1}; \]
\item generates a $\Z_2$-grading 
\[ \Cl(V,q) = \Cl^0(V,q)\oplus \Cl^1(V,q). \]
\end{enumerate}
\end{lemma}

\subsubsection{Transpose}
The transpose map defined on the tensor algebra $\mathcal{T}(V)$, i.e. the linear map that reverses to order of homogeneous elements:
\[ v_1\otimes \ldots \otimes v_r \mapsto v_r \otimes \ldots \otimes v_1, \]
preserves the ideal $\mathcal{I}(V,q)$, and so determines a well-defined map on the Clifford algebra $\Cl(V,q)$:
\[ (-)^t: \Cl(V,q)\to \Cl(V,q). \]
\begin{lemma}
The transpose $(-)^t: \Cl(V,q)\to \Cl(V,q)$ is
\begin{enumerate}
\item an involution;
\item an anti-automorphism:
\[ \forall x,y\in \Cl(V,q): \quad (xy)^t = y^tx^t; \]
\item unital.
\end{enumerate}
\end{lemma}

\subsubsection{Clifford conjugation}
\begin{definition}
The composition of the grade involution and the transpose is called \udef{Clifford conjugation}:
\[ x\mapsto \overline{x} \defeq \alpha(x^t). \]
\end{definition}
\begin{lemma}
The grade involution and transpose commute
\[ \alpha \circ(-)^t = (-)^t\circ \alpha \]
and Clifford conjugation is thus equal to both.
\end{lemma}

\begin{lemma}
Clifford conjugation is
\begin{enumerate}
\item an involution;
\item an anti-automorphism:
\[ \forall x,y\in \Cl(V,q): \quad (xy)^t = y^tx^t; \]
\item unital.
\end{enumerate}
\end{lemma}

\subsubsection{Quaternion types of Clifford algebra types}

\subsection{The norm mapping}
\begin{definition}
We define the \udef{norm mapping} $N$ by
\[ N: \Cl(V,q)\to \Cl(V,q): x\mapsto x \overline{x}. \]
\end{definition}
\begin{lemma} \label{normIsQuadraticForm}
\begin{enumerate}
\item If $v\in V$, then $N(v) = q(v)$.
\item $\alpha\circ N = N\circ \alpha$.
\end{enumerate}
\end{lemma}

\subsection{Orthogonal decomposition}
As $q(u,v)$ is a bilinear form, we can consider orthogonal subspaces with respect to it. Then $V=V_1\oplus V_2$ is a $q$-orthogonal decomposition if and only if $\forall v_1\in V_1, v_2\in V_2$:
\[ q(v_1,v_2) = 0 \qquad \iff \qquad q(v_1+v_2) = q(v_1) + q(v_2). \]
\begin{proposition}
Let $V=V_1\oplus V_2$ be a $q$-orthogonal decomposition. Then there is a natural isomorphism of Clifford algebras
\[ \Cl(V,q) \;\to\; \Cl(V_1,q|_{V_1}) \hat{\otimes} \Cl(V_2, q|_{V_2}):\quad v_1+v_2\mapsto v_1\otimes \vec{1} + \vec{1}\otimes v_2. \]
\end{proposition}

\section{The Pin and Spin groups}
\begin{lemma}
Let $\Cl(V,q)$ be a Clifford algebra. Then the group of multiplicative units in the Clifford algebra
\[ \Cl^\times(V,q) = \setbuilder{x\in \Cl(V,q)}{\exists x^{-1}\in \Cl(V,q): x^{-1}x = xx^{-1} = \vec{1}} \]
contains all elements $v\in V$ with $q(v)\neq 0$.
\end{lemma}
\begin{proof}
From \ref{CliffordRelations} we have $v^2 = q(v)\vec{1}$, so $v^{-1} = v/q(v)$.
\end{proof}

\begin{proposition}
Let $V$ be a finite-dimensional real or complex vector space of dimension $\dim V = n$. Then the group $\Cl^\times(V,q)$ of multiplicative units in the Clifford algebra is a Lie group of dimension $2^n$ and the corresponding Lie algebra $\mathfrak{cl}^\times(V,q)$ is the full Clifford algebra $\Cl(V,q)$ with the Lie bracket
\[ [x,y] = xy - yx.  \]
\end{proposition}

\subsection{Inner automorphisms of $\Cl(V,q)$}
Characteristic for field not 2!!


\begin{proposition} \label{AdOrthogonalDecomposition}
Let $v\in V\subset \Cl(V,q)$ be an element with $q(v)\neq 0$. Then for all $w\in V$:
\[ \Ad_v(w) = \frac{q(v,w)}{q(v)}v - w. \]
In particular $\Ad_v[V] = V$.
\end{proposition}
\begin{proof}
From \ref{CliffordRelations} we have $v^2 = q(v)\vec{1}$ and $vw+wv = q(v,w)\vec{1}$. Then $v = q(v)v^{-1}$ and thus
\[ q(v)\Ad_v(w) = q(v)vwv^{-1} = vwv = v(q(v,w)\vec{1} - vw) = q(v,w)v - q(v)w. \]
\end{proof}

\begin{lemma} \label{AdOrthogonalMap}
Let $v\in V$ such that $q(v)\neq 0$. Then $\Ad_v\in \Ogroup(V,q)$.
\end{lemma}
\begin{proof}
Clearly $\Ad_v$ is invertible. We then calculate using \ref{AdOrthogonalDecomposition}
\begin{align*}
q(\Ad_v(w)) &= q\left(\frac{q(v,w)}{q(v)}v - w\right) = q\left(\frac{q(v,w)}{q(v)}v,-w\right) + q\left(\frac{q(v,w)}{q(v)}v\right) + q(w) \\
&= -\frac{q(v,w)}{q(v)}q(v,w)+\left(\frac{q(v,w)}{q(v)}\right)^2q(v) + q(w) = q(w).
\end{align*}
\end{proof}


\subsection{Pin and Spin groups}
\begin{definition}
Let $P(V,q)$ be the subgroup of $\Cl^\times(V,q)$ generated by elements $v\in V$ with $q(v)\neq 0$.

We also define the group
\[ \Gamma(V,q) \defeq \setbuilder{x\in\Cl^\times(V,q)}{\Ad_x[V] = V}. \]
which is called the \udef{Clifford group}, \udef{Lipschitz group} or \udef{Clifford-Lipschitz group}.
\end{definition}
That $\Gamma(V,q)$ is a group follows from the following observation: If $\Ad_x[V]=V$ and $\Ad_y[V]=V$, then
\[ \Ad_{xy}[V] = \Ad_x[\Ad_y[V]] = \Ad_x[V] = V. \]


\begin{lemma} \label{PsubgroupVpreserving}
There is an inclusion
\[ P(V,q) \subset \Gamma(V,q). \]
\end{lemma}
\begin{proof}
The generators of $P(V,q)$ are in $\Gamma(V,q)$ by \ref{AdOrthogonalDecomposition} and $\Gamma(V,q)$ is a group.
\end{proof}

\begin{lemma}
The $\Ad$ function defines a representation
\[ \Ad: P(V,q) \to \Ogroup(V,q). \]
\end{lemma}
\begin{proof}
This mapping is well-defined by \ref{AdOrthogonalMap} and the identity
\[ \Ad_{xy} = \Ad_x\circ \Ad_y. \]
This identity also shows that the mapping is a group homomorphism, and thus that it is a representation.
\end{proof}

\begin{lemma}
Let $x\in \Gamma(V,q)$. Then
\begin{enumerate}
\item $\alpha(x)\in \Gamma(V,q)$;
\item $x^t\in \Gamma(V,q)$.
\end{enumerate}
\end{lemma}
\begin{proof}
We calculate
\[ V = \alpha[V] = \alpha[\Gamma(V,q)] = \alpha(\alpha(x))\cdot V\cdot \alpha(x)^{-1} = \Ad_{\alpha(x)}[V] \]
and
\[ V = (\alpha[V])^t = \alpha(x^t)\cdot V \cdot (x^t)^{-1} = \Ad_{x^t}[V] \]
and the third follows by multiplicative closure.
\end{proof}
A consequence of this lemma is that $N(x)\in \Gamma(V,q)$ for all $x\in \Gamma(V,q)$. But we will show that something stronger holds, namely $N(x)\in\F^\times$.

\begin{definition}
The \udef{Pin group} of $(V,q)$ is the subgroup $\Pin(V,q)$ of $P(V,q)$ generated by the elements $v\in V$ with $q(v) = \pm 1$.

The \udef{Spin group} of $(V,q)$ is defined by
\[ \Spin(V,q) = \Pin(V,q) \cap \Cl^0(V,q) \]
where $\Cl^0(V,q)$ is the even subalgebra of $\Cl(V,q)$.
\end{definition}

\subsection{The twisted adjoint representation}
Consider the map $\Ad_v$ acting on the $q$-orthogonal decomposition (TODO ref)
\[ V = \Span\{v\}\oplus\Span\{v\}^\perp \qquad \text{for some $v\in P(V,q)$,} \]
where $\Span\{v\}^\perp = \setbuilder{w\in V}{q(v,w)=0}$. Then, by the formula
\[ \Ad_v(w) = \frac{q(v,w)}{q(v)}v - w ,\]
we see that elements of $\Span\{v\}$ are mapped to themselves:
\begin{align*}
\Ad_v(\lambda v) &= \frac{q(v,\lambda v)}{q(v)}v - \lambda v = \lambda\frac{q(2v)-2q(v)}{q(v)}v - \lambda v \\
&= 2\lambda v -\lambda v = \lambda v.
\end{align*}
and that elements $w\in\Span\{v\}^\perp$ are mapped to $-w$.

This means that $\Ad_v$ is orientation-preserving if $\dim(V)$ is odd and orientation-reversing otherwise.

We would prefer the action of $\Ad_v$ to do the opposite: fix the hyperplane $\Span\{v\}^\perp$ and invert $\Span\{v\}$. To that end we introduce the twisted adjoint representation.
\begin{definition}
The \udef{twisted adjoint representation} $\widetilde{\Ad}: \Cl^\times(V,q) \to \GL(\Cl(V,q))$ is defined by
\[ \widetilde{\Ad}_x(y) = \alpha(x)yx^{-1} \qquad \forall x\in \Cl^\times(V,q), \forall y\in \Cl(V,q) \]
where $\alpha$ is the grade involution.
\end{definition}
\begin{lemma}
Let $x,y\in \Cl^\times(V,q)$ and $v,w\in V$. Then
\begin{enumerate}
\item $\widetilde{\Ad}_{xy} = \widetilde{\Ad}_x\circ \widetilde{\Ad}_y$;
\item $\widetilde{\Ad}_x = \Ad_x$ if $x\in \Cl^0(V,q)$;
\item $\widetilde{\Ad}_v(w) = w-\frac{q(v,w)}{q(v)}v$.
\end{enumerate}
\end{lemma}

We have
\[ \Gamma(V,q) = \setbuilder{x\in\Cl^\times(V,q)}{\widetilde{\Ad}_x[V] = V}. \]

\begin{proposition}
Let $V$ be a finite-dimensional vector space over a field $\mathbb{F}$ and $q$ non-degenerate. Then the kernel of the homomorphism
\[ \widetilde{\Ad}: \Gamma(V,q) \to \GL(V) \]
is exactly the group $\mathbb{F}^\times$.
\end{proposition}
\begin{proof}
Choose an orthogonal basis $v_1,\ldots, v_n$ for $V$ w.r.t. the bilinear form $q(-,-)$ (TODO ref; also proof here only finite-dim: can it generalise?). Suppose $x\in \Cl^\times(V,q)$ is in the kernel of $\widetilde{\Ad}$, then
\[ \alpha(x)v = vx \qquad \text{for all $v\in V$.} \]
Now we can write $x = x_0 + x_1$ where $x_0$ is even and $x_1$ is odd and both are polynomial expressions in $v_1,\ldots, v_n$. Making use of $v_iv_j = \pm v_jv_i$, we can write $x_0 = a_0 + v_1a_1$ where $a_0,a_1$ are polynomial expressions in $v_2,\ldots, v_n$. Then $a_0$ is even and $a_1$ is odd, so
\[ v_1a_0 + v_1^2a_1 = v_1(a_0+v_1a_1) = (a_0+v_1a_1)v_1 = a_0v_1 + v_1 a_1 v_1 = v_1a_0-v_1^2 a_1. \]
Thus $v^2_1a_1 = -q(v_1)a_1 = 0$, so $a_1=0$ and $x_0$ does not involve $v_1$. By induction $x_0$ does not involve any of $v_1,\ldots, v_n$ and thus $x_0 = \lambda\cdot \vec{1}$ for $\lambda\in\mathbb{F}$.

A similar argument shows that $x_1$ is independent of $v_1,\ldots, v_n$ and thus $x_1=0$. Here it is important that $v_1x_1 = -x_1v_1$, because in $x_1 = a_0 + v_1a_1$, $a_0$ is now odd and $a_1$ even.

Thus $x = x_0+x_1 = \lambda\cdot \vec{1}$ and $x\neq 0$, so $x\in \mathbb{F}^\times$.
\end{proof}
This proof only works for the \textit{twisted} adjoint representation, not the adjoint representation.

It is also clearly important that $q$ be non-degenerate.

\begin{corollary} \label{normHomomorphism}
The restriction of the norm $N$ to $\Gamma(V,q)$ gives a homomorphism
\[ N: \Gamma(V,q) \to \mathbb{F}^\times. \]
\end{corollary}
\begin{proof}
If we can show that $N[\Gamma(V,q)]\subset \mathbb{F}^\times$, then the multiplicativity of $N$ follows from
\[ N(xy) = xy\alpha((xy)^t) = xy\alpha(y^t)\alpha(x^t) = xN(y)\alpha(x^t) = x\alpha(x^t)N(y) = N(x)N(y). \]

Thus by the proposition it is enough to show that for all $x\in \Gamma(V,q)$, $N(x)\in \ker(\widetilde{\Ad})$. Because $\alpha(x)vx^{-1}\in V$, the transpose leaves it unchanged:
\[ \alpha(x)vx^{-1} = (\alpha(x)vx^{-1})^t = (x^t)^{-1}v\alpha(x^t). \]
This can be rewritten as
\[ v = x^t\alpha(x)vx^{-1}(\alpha(x^t))^{-1} = \alpha(\alpha(x^t)x)v(\alpha(x^t)x)^{-1} = \widetilde{\Ad}_{N(x)}(v). \]
Thus $N(x)\in \ker(\widetilde{\Ad})$.
\end{proof}
\begin{corollary}
Let $x\in \Gamma(V,q)$, then $\widetilde{\Ad}_x \in \Ogroup(V,q)$. Thus there is a group homomorphism
\[ \widetilde{\Ad}: \Gamma(V,q) \to \Ogroup(V,q). \]
\end{corollary}
\begin{proof}
First assume $v\in V^\times = \setbuilder{v\in V}{q(v)\neq 0}\subset \Gamma(V,q)$. Then by \ref{normIsQuadraticForm} and the previous corollary
\[ q(\widetilde{\Ad}_x(v)) = N(\widetilde{\Ad}_x(v)) = N(\alpha(x)vx^{-1}) = N(\alpha(x))N(v)N(x^{-1}) = N(v)N(x)N(x^{-1}) = N(v) = q(v). \]

Now assume $q(v) = 0$. If $q(\widetilde{\Ad}_x(v))$ were not zero, then $\widetilde{\Ad}_x(v)\in V^\times$ and thus
\[ q(v) = q(\widetilde{\Ad}_{x^{-1}}\circ\widetilde{\Ad}_x(v)) = q(\widetilde{\Ad}_x(v)) \neq 0 \]
which is a contradiction.
\end{proof}

\subsection{Double coverings}
We define $SP(V,q) \defeq P(V,q)\cap \Cl^0(V,q)$.
\begin{theorem}
The homomorphisms
\[ \widetilde{\Ad}: P(V,q)\to \Ogroup(V,q) \qquad \text{and} \qquad \widetilde{\Ad}: SP(V,q)\to \SO(V,q) \]
are surjective. So we have the short exact sequence
\[ \begin{tikzcd}
1 \rar & \F^\times \rar & \Gamma(V,q) \rar{\widetilde{\Ad}} & \Ogroup(V,q) \rar & 1.
\end{tikzcd} \]
\end{theorem}

\begin{proposition}
The images $\widetilde{\Ad}(\Pin(V,q))$ and $\widetilde{\Ad}(\Spin(V,q))$ are both normal subgroups of $\Ogroup(V,q)$.
\end{proposition}
\begin{proof}
By a simple calculation we have $\widetilde{\Ad}_{f(v)} = f\circ\widetilde{\Ad}_v\circ f^{-1}$ for all $v \in V$ and $f\in\Ogroup(V,q)$.
\end{proof}

\begin{definition}
A field $\mathbb{F}$ of characteristic $\neq 2$ is called \udef{spin} if for all $a\in \mathbb{F}^\times$ at least one of the equations $t^2 = a$ and $t^2 = -a$ has a solution $t$ in $\mathbb{F}$.
\end{definition}
Thus $\mathbb{F}$ is spin if
\[ \mathbb{F}^\times = (\mathbb{F}^\times)^2 \cup (-(\mathbb{F}^\times)^2). \]

\begin{lemma}
The following fields are spin:
\begin{enumerate}
\item $\R$;
\item $\C$;
\item $\mathbb{F}_p$ with $p$ prime and $p \equiv 3\mod 4$.
\end{enumerate}
\end{lemma}

\begin{theorem}
Let $V$ be a finite-dimensional vector space over a spin field
$\mathbb{F}$, and suppose $q$ is a non-degenerate quadratic form on $V$. Then there are
short exact sequences
\[ \begin{tikzcd}
1 \rar & F \rar & \Spin(V,q) \rar{\widetilde{\Ad}} & \SO(V,q) \rar & 1
\end{tikzcd} \]
\[ \begin{tikzcd}
1 \rar & F \rar & \Pin(V,q) \rar{\widetilde{\Ad}} & \Ogroup(V,q) \rar & 1
\end{tikzcd} \]
where
\[ F = \begin{cases}
\Z_2 = \{1,-1\} & \sqrt{-1}\notin\mathbb{F} \\
\Z_4 = \{\pm 1, \pm\sqrt{-1}\}	& \text{otherwise.}
\end{cases} \]
This result holds for general fields if $\SO(V,q)$ and $\Ogroup(V,q)$ are replaced by appropriate normal subgroups of $\Ogroup(V,q)$.
\end{theorem}
\begin{proof}
Suppose $x = v_1\ldots v_r\in\Pin(V,q)$ is in the kernel of $\widetilde{\Ad}$. Then $x\in \mathbb{F}^\times$ and so
\[ x^2 = N(x) = N(v_1)\ldots N(v_r) = \pm 1 \]
by \ref{normHomomorphism}.
\end{proof}

\begin{proposition}
Let $\mathbb{F}$ be a spin field. Then either
\[ \Gamma(V,q)=P(v,q) \qquad \text{or} \qquad \Gamma(V,q)/P(V,q) \cong \Z_2. \]
\end{proposition}
\begin{proof}
TODO

We have a group homomorphism $\widetilde{\Ad}: \Gamma(V,q)\to \Ogroup(V,q)$ with $\ker(\widetilde{\Ad}) = \F^\times$. 
\end{proof}

\section{Real and complex Clifford algebras}

\begin{definition}
$q$-orthonormal basis.
\end{definition}

\begin{proposition}
There is an algebra isomorphism
\[ \Cl_{r,s} \cong \Cl^0_{r,s+1} \qquad \forall r,s\in\N. \]
In particular $\Cl_n \cong \Cl^0_{n+1}$.
\end{proposition}
\begin{proof}
Take a $q$-orthonormal basis $\{e_i\}_{i=1}^{r+s+1}$ of $\R^{r+s+1}$ and let $\R^{r+s}$ be spanned by the basis $\{e_i\}_{i=1}^{r+s}$. Then define a linear map $f:\R^{r+s}\to \Cl^0_{r,s+1}$ by
\[ f(e_i) = e_{r+s+1}e_i \qquad (i=1,\ldots,r+s). \]
We hope to apply the universal property \ref{CliffordUniversalProperty} to extend it to a map $\widetilde{f}: \Cl_{r,s}\to \Cl^0_{r,s+1}$. So we check $f(v)^2 = q(v)\cdot \vec{1}$. Indeed, let $v = \sum_{i=1}^{r+s}v_ie_i$, then
\[ f(v)^2 = \sum_{i,j=1}^{r+s}v_iv_je_{r+1}e_ie_{r+1}e_j = -\sum_{i,j=1}^{r+s}v_iv_je_{r+1}e_{r+1}e_ie_j = \sum_{i,j=1}^{r+s}v_iv_je_ie_j = q(v)\cdot \vec{1}. \]

It is easy to see $\widetilde{f}$ is bijective.
\end{proof}

\section{Representations}

\begin{definition}
Let $K \subseteq k$ be fields, $V$ a vector space over $k$ and $q$ a quadratic form on $V$. Then a \udef{$K$-representation} of the Clifford algebra $\Cl(V,q)$ is a $k$-algebra homomorphism
\[ \rho: \Cl(V,q) \to \Hom_K(W,W) \]
where $W$ is a finite dimensional vector space over $K$. The space $W$ is then a \udef{$\Cl(V,q)$-module} over $K$.
\end{definition}

Usually we will take the field $K$ to be $\R,\C,\mathbb{H}$.

\begin{lemma}
\begin{enumerate}
\item A complex representation of $\Cl_{r,s}$ automatically extends to a representation of
\[ \Cl_{r,s}\otimes_\R \C \cong \cCl_{r+s}. \]
\item A quaternionic representation of $\Cl_{r,s}$ is automatically complex.
\end{enumerate}
\end{lemma}

\section{Lie algebra structures}

\begin{proposition}
The Lie subalgebra of $(\Cl_n, [\cdot,\cdot])$ corresponding to the subgroup $\Spin_n\subset \Cl_n^\times$ is
\[ \mathfrak{spin}_n = {\textstyle \bigwedge^2}\R^n. \]
In particular, $\bigwedge^2\R^n$ is closed under the bracket operation.
\end{proposition}
\begin{proof}
We are looking for tangent vectors to the submanifold $\Spin_n$ at $\vec{1}$. Fix an orthonormal basis $e_1,\ldots, e_n$ of $\R^n$ and consider the curve
\[ \gamma(t) = (e_i\cos t+ e_j\sin t)(-e_i\cos t+ e_j\sin t) = (\cos^2 t - \sin^2 t)+2e_ie_j\sin t\cos t = \cos(2t)\sin(2t)e_ie_j. \]
This curve lies in $\Spin_n$, satisfies $\gamma(0) = \vec{1}$ and its tangent vector at $\gamma(0)$ is $2e_ie_j$. Hence $\mathfrak{spin}_n$ contains $\Span_\R\{e_ie_j\} = \bigwedge^2\R^n$. Since $\dim_\R(\mathfrak{spin}_n)$
\end{proof}






\section{Geometry and geometric algebra}
\subsection{Definitions}

\begin{definition}
A \udef{geometric algebra} $\mathfrak{G}$ is a real unital associative algebra of the form
\[ \mathfrak{G} = \bigoplus_{r\in \N}\mathfrak{G}_r \]
such that
\begin{align*}
\mathfrak{G}_0 &= \Span\{\vec{1}\} \\
\mathfrak{G}_r &= \Span\setbuilder{\vec{a}_1\vec{a}_2\ldots\vec{a}_r}{\vec{a}_1,\ldots,\vec{a}_r \in \mathfrak{G}_1, \; \forall i,j\leq r: \vec{a}_i \vec{a}_j = -\vec{a}_j \vec{a}_i} & \text{for all $r>1$}.
\end{align*}
We also assume that the multiplication satisfies
\[ \forall \vec{a} \in \mathfrak{G}_1: \quad \vec{a}^2 = \vec{a}\vec{a} = \lambda \vec{1} \in \mathfrak{G}_0 \qquad \text{for some $\lambda \in \R^{> 0}$} \]
and that for each element $a$ of $\mathfrak{G}_r\setminus\{0\}$ there exists a vector $\vec{a}\in\mathfrak{G}_1$ such that $\vec{a}a \in \mathfrak{G}_{r+1}\setminus\{0\}$.

We then call
\begin{itemize}
\item $\sqrt{\lambda}$ the \udef{magnitude of $\vec{a}$}, denoted $|\vec{a}|$;
\item the projection $\mathfrak{G} \to \mathfrak{G}_r$ the \udef{grade operator}, denoted $\grade{\cdot}_r$;
\item the multiplication of $\mathfrak{G}$ the \udef{geometric product} on $\mathfrak{G}$;
\item elements of $\mathfrak{G}$ \udef{multivectors};
\item elements of $\mathfrak{G}_r$ \udef{$r$-vectors} or \udef{homogenous multivectors}; in particular $0$-vectors are called \udef{scalars}, $1$-vectors \udef{vectors}, $2$-vectors \udef{bivectors} \ldots
\item $r$-vectors of the form $\vec{a}_1\vec{a}_2\ldots\vec{a}_r$ where $\vec{a}_1,\ldots,\vec{a}_r \in \mathfrak{G}_1$ anti-commute are called \udef{simple $r$-vectors} or \udef{$r$-blades}.
\end{itemize}
We use lowercase letters $a,b,c \ldots$ to denote multivectors, Greek letters $\mu, \nu, \lambda \ldots$ for scalars and bold letters $\vec{u}, \vec{v}, \vec{w} \ldots$ for vectors. Often we will use subscripts to denote the grade of a multivector, e.g. $a_r$ is an $r$-vector. Capital letters with subscript, e.g. $A_r$, will be used to denote $r$-blades.
\end{definition}

So for any $a\in \mathfrak{G}$, we can write
\[ a = \grade{a}_0 + \grade{a}_1 + \ldots + \grade{a}_n  = \sum_{r=0}^n \grade{a}_i  \] for some $n\in\N$.

We make the convention that negative grades are always zero.

Because of the assumption that no $\mathfrak{G}_r$ is trivial, we need $\mathfrak{G}_1$ to be infinite-dimensional.


\begin{definition}
Let $\mathfrak{G}$ be a geometric algebra. We define the \udef{reverse} operation $\dagger$ on $\mathfrak{G}$ as the unique linear operation such that
\begin{align*}
\vec{1}^\dagger &= \vec{1} \\
\vec{u}^\dagger &= \vec{u} \qquad \vec{u}\in\mathfrak{G}_1 \\
(\vec{v}_1 \vec{v}_2 \ldots \vec{v}_r)^\dagger &= \vec{v}_r \ldots \vec{v}_2 \vec{v}_1 \qquad \vec{v}_1, \ldots, \vec{v}_r \in \mathfrak{G}_1.
\end{align*}
\end{definition}
Note that we have specified $\dagger$ on all basis elements of $\mathfrak{G}$, so it is well-defined and uniquely determined, cfr. \ref{linearMaps}.

\begin{lemma}
Let $a,b \in \mathfrak{G}$. Then
\begin{enumerate}
\item $(a^\dagger)^\dagger = a$;
\item $(ab)^\dagger = b^\dagger a^\dagger$;
\item $\grade{a^\dagger}_r = \grade{a}_r^\dagger = (-1)^{r(r-1)/2}\grade{a}_r$;
\item $\grade{a}_r = (-1)^{r(r-1)/2}\grade{a}_r^\dagger$;
\item $\grade{a_rb_s}_t = (-1)^{\frac{1}{2}(r(r-1) + s(s-1) + t(t-1))}\grade{b_sa_r}_t$.
\end{enumerate}
\end{lemma}
Notice we are only interested in the exponent of $(-1)$ modulo $2$.

\begin{definition}
Let $\mathfrak{G}$ be a geometric algebra. We define the \udef{inner product} $\cdot$ on homogeneous multivectors by
\[ a_r\cdot b_s = \begin{cases}
0 & \text{$r=0$ or $s=0$} \\
\grade{a_rb_s}_{|r-s|} & \text{else}.
\end{cases}  \]
The inner product is bilinear on homogeneous multivectors and can thus be extended linearly to arbitrary multivectors.
\end{definition}
So for arbitrary multivectors we have
\[ a\cdot b = \sum_r\sum_s \grade{a}_r\cdot \grade{b}_s. \]

\begin{definition}
Let $\mathfrak{G}$ be a geometric algebra. We define the \udef{outer product} $\wedge$ on homogeneous multivectors by
\[ a_r\wedge b_s = \grade{a_rb_s}_{r+s} \]
The outer product is bilinear on homogeneous multivectors and can thus be extended linearly to arbitrary multivectors.
\end{definition}
So for arbitrary multivectors we have
\[ a\wedge b = \sum_r\sum_s \grade{a}_r\wedge \grade{b}_s. \]
For scalars $\lambda \in \mathfrak{G}^0$, we have
\[ a \wedge \lambda = \lambda\wedge a = \lambda a \qquad a \in \mathfrak{G}. \]
We have explicitly excluded this in the inner product.

\begin{lemma}
If $a = \vec{v}_1 \vec{v}_2\ldots \vec{v}_r$, then $a\in \bigoplus_{i\leq r}\mathfrak{G}_i$.

If $a$ is an $r$-blade and $\beta$ an orthogonal basis for $\mathfrak{G}_1$, then there exist $\vec{e}_1,\ldots, \vec{e}_r \in \beta$ such that
\[ a = \lambda \vec{e}_1 \ldots \vec{e}_r \]


 be an $r$-blade, then
\[\vec{v}_1 \vec{v}_2\ldots \vec{v}_r = \vec{v}_1 \wedge \vec{v}_2 \wedge\ldots\wedge \vec{v}_r.\]
\end{lemma}

\begin{note}
We introduce an order of operations (from highest priority to lowest):
\begin{enumerate}
\item outer product;
\item inner product;
\item geometric product.
\end{enumerate}
\end{note}

\begin{lemma}
Let $a_r,b_s$ be homogeneous vectors in $\mathfrak{G}$. Then
\begin{enumerate}
\item $a_r\cdot b_s = (-1)^{s(r-1)}b_s\cdot a_r$ for $r\geq s$;
\item $a_r \wedge b_s = (-1)^{rs}b_s \wedge a_r$.
\end{enumerate}
\end{lemma}
\begin{proof}
(1) We calculate
\[ a_r\cdot b_s = \grade{a_rb_s}_{|r-s|} = (-1)^{\frac{1}{2}(r(r-1) + s(s-1) + |r-s|(|r-s|-1))}\grade{a_rb_s}_{|r-s|}. \]
We can simplify the exponent, assuming $r\geq s$, to
\[ r^2 + s^2 -r -sr \equiv r+s+r+sr \equiv s+sr \equiv sr-s \mod 2.\]
(2) We calculate
\[ a_r \wedge b_s = \grade{a_rb_s}_{r+s} = (-1)^{\frac{1}{2}(r(r-1) + s(s-1) + (r+s)((r+s)-1))}\grade{a_rb_s}_{r+s}. \]
We can simplify the exponent to
\[ r^2 -r+s^2 -s + rs \equiv r-r+s-s+rs \equiv rs \mod 2. \]
\end{proof}

By the fact that the definitions of inner and outer product make sense it is obvious that the geometric product does not preserve grade, or even homogeneity. It does, however, preserve a $\Z_2$-grading: we can split
\[ \mathfrak{G} = \mathfrak{G}_\text{even} \oplus \mathfrak{G}_\text{odd} \qquad \text{where}\quad \begin{cases}
\mathfrak{G}_\text{even} \defeq \bigoplus_{r\in \N}\mathfrak{G}_{2r} \\
\mathfrak{G}_\text{odd}\; \defeq \bigoplus_{r\in \N}\mathfrak{G}_{2r+1}.
\end{cases} \]
We have the linear projection operators $\grade{\cdot}_+:\mathfrak{G}\to \mathfrak{G}_\text{even}$ and $\grade{\cdot}_-:\mathfrak{G}\to \mathfrak{G}_\text{odd}$. We also write $\mathfrak{G}_+$ instead of $\mathfrak{G}_\text{even}$ and $\mathfrak{G}_-$ instead of $\mathfrak{G}_\text{odd}$.
\begin{proposition}
Let $\mathfrak{G} = \mathfrak{G}_\text{+} \oplus \mathfrak{G}_\text{-}$ be a geometric algebra and let $p,q\in\{+,-\}\cong \Z_2$. Then
\[ \mathfrak{G}_p\mathfrak{G}_q \subset \mathfrak{G}_{pq}. \]
\end{proposition}
\begin{proof}
The grading operators $\grade{\cdot}_\pm$ are linear maps and thus determined by their action on basis elements. Thus it is enough to show that
\[ \grade{A_rB_s}_+ =  \begin{cases}
A_rB_s & (r+s \equiv 0 \mod 2) \\
0 & (r+s \equiv 1 \mod 2)
\end{cases} \qquad \grade{A_rB_s}_- =  \begin{cases}
0 & (r+s \equiv 0 \mod 2) \\
A_rB_s & (r+s \equiv 1 \mod 2)
\end{cases} \]
for any $r$-blade $A_r$ and $s$-blade $B_s$. In fact by associativity of the geometric product, it is enough to show
\[ \vec{v}A_r \]
\end{proof}

\begin{lemma}
Let $\vec{u}, \vec{v} \in \mathfrak{G}_1$. Then
\[ \vec{u}\cdot \vec{v} = \grade{\vec{u}\vec{v}}_0 = \frac{1}{2}(\vec{u}\vec{v} + \vec{v}\vec{u}). \]
\end{lemma}
\begin{proof}
We start from $(\vec{u}+\vec{v})^2 = \vec{u}^2 + \vec{u}\vec{v} + \vec{v}\vec{u} + \vec{v}^2$ and rearrange to get
\[ \vec{u}\vec{v} + \vec{v}\vec{u} = (\vec{u}+\vec{v})^2 - \vec{u}^2 - \vec{v}^2 = |\vec{u}+\vec{v}|^2 - |\vec{u}|^2 - |\vec{v}|^2  \]
which is scalar. Since $\grade{\vec{u}\vec{v}}_0 = \grade{\vec{v}\vec{u}}_0$, we have
\[ \frac{1}{2}(\vec{u}\vec{v} + \vec{v}\vec{u}) = \frac{1}{2}\grade{\vec{u}\vec{v} + \vec{v}\vec{u}}_0 = \grade{\vec{u}\vec{v}}_0 = \vec{u}\cdot \vec{v}. \]
\end{proof}
\begin{corollary}
The geometric inner product restricted to $\mathfrak{G}_1$ is bilinear, symmetric and positive definite. It is thus an inner product as previously defined.
\end{corollary}
The associated definitions are thus also applicable here. In particular two vectors $\vec{u},\vec{v}$ are called \udef{orthogonal} if $\vec{u}\cdot \vec{v} = 0$. By the lemma this is the case when $\vec{u}\vec{v} = - \vec{v}\vec{u}$. This means that the $r$ vectors making up $r$-blades are linearly independent, \ref{orthogonalLinearlyIndependent}.




\begin{lemma}
For any algebra satisfying the other axioms, the direct sum $\mathfrak{G} = \bigoplus_{r\in\N}\mathfrak{G}_r$ is well-defined.
\end{lemma}
\begin{proof}
We need to show that for all $r>s\in \N$, we have $\mathfrak{G}_r\cap\mathfrak{G}_s = \{0\}$. Assume, towards a contradiction, that here exist $a_r,b_s$ such that $a_r = b_s$. Then both $a_r$ and $b_s$ can be written as sums of blades. Now let $D$ be the set of all vectors featured in a blade in this sum. By Gram-Schmidt, we can find an orthogonal basis for $D$ and rewrite $a_r$ and $b_s$ in this basis. As $r>s$, we can find elements of this orthogonal basis to multiply $a_r$ with such that it becomes zero, but $b_s$ remains non-zero (unless it already was zero). TODO: improve proof.
\end{proof}


\begin{lemma}
Let $\vec{u}, \vec{v}_1,\ldots, \vec{v}_n$ be vectors in $\mathfrak{G}_1$. Then
\[ \vec{u}\cdot (\vec{v}_1 \vec{v}_2 \ldots \vec{v}_n) = \sum_{i=1}^n (-1)^{k+1}(\vec{u}\cdot \vec{v}_i)\vec{v}_1\ldots\breve{\vec{v}}_i\ldots \vec{v}_n, \]
where the breve indicates the vector under it is omitted from the product.
\end{lemma}
\begin{proof}

\end{proof}

\begin{lemma}
\[ \vec{u}a_r = \vec{u}\cdot a_r + \vec{u}\wedge a_r = \grade{\vec{u}a_r}_{r-1} + \grade{\vec{u}a_r}_{r+1}. \]
\end{lemma}

\begin{proposition}
Let $\vec{v}\in\mathfrak{G}_1$ and $a_r\in\mathfrak{G}_r$. Then

\end{proposition}





\begin{lemma}
The outer product is associative:
\[ a\wedge(b\wedge c) = (a\wedge b)\wedge c \]
The inner product is not associative, but homogeneous multivectors obey
\begin{align*}
a_r\cdot(b_s \cdot c_t) &= (a_r\wedge b_s)\cdot c_t & &\text{for $r+s\leq t$ and $r,s>0$} \\
a_r\cdot(b_s \cdot c_t) &= (a_r\cdot b_s)\cdot c_t & &\text{for $r+t\leq s$}
\end{align*}
\end{lemma}

\begin{lemma}
\[ \vec{u}\wedge a\wedge \vec{v}\wedge b = -\vec{v}\wedge a\wedge \vec{u}\wedge b  \]
\end{lemma}
$\vec{v}\wedge a \wedge \vec{v} \wedge b = 0$

\subsection{Affine spaces}
\subsection{Projections on 1D spaces}
\[ \sin(\theta) = \norm{a_\perp}/ \norm{a} \qquad \cos(\theta) = \norm{a_\parallel}/\norm{a}. \]
\subsection{The geometric product}

\subsection{Hodge duality}
\subsection{Cross product and triple product}
Cross product not associative

Triple product nice way to find normal vectors with specific orientation.