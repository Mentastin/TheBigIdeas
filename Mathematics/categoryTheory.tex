\url{https://arxiv.org/pdf/1912.10642.pdf}
\url{https://arxiv.org/pdf/0810.1279.pdf}
\url{http://katmat.math.uni-bremen.de/acc/acc.pdf}

\chapter{Basic concepts}
\section{Categories}
\subsection{Definitions and examples}
\begin{definition}
A \udef{category} $\cat{C}$ consists of
\begin{enumerate}
\item a class, denoted $\ob(\cat{C})$, of \udef{objects} $X,Y,Z,\ldots$
\item a class, denoted $\mor(\cat{C})$, of \udef{morphisms} or \udef{arrows} $f,g,h,\ldots$
\item a partial binary function $\circ: \mor(\cat{C}) \times \mor(\cat{C}) \not\to \mor(\cat{C})$ of \udef{composition}
\item functions $\dom: \mor(\cat{C}) \to \ob(\cat{C})$ and $\codom: \mor(\cat{C}) \to \ob(\cat{C})$; we call $\dom(f)$ the \udef{domain} and $\codom(f)$ the \udef{codomain} of the morphism $f$; we write
\[ f: X \to Y \qquad\text{or}\qquad \begin{tikzcd}
X \rar{f} & Y
\end{tikzcd}  \]
to mean $f$ is a morphism with $\dom(f) = X$ and $\codom(f) = Y$;
\item a function $\id: \ob(\cat{C}) \to \mor(\cat{C}): X\mapsto (\id_X: X\to X)$, whose images are called \udef{identity morphisms};
\end{enumerate}
such that
\begin{itemize}
\item $g\circ f$ is defined \emph{if and only if} $\codom(f) = \dom(g)$; in this case
\[ g\circ f: \dom(f) \to \codom(g); \]
\item for any $f: X\to Y$, we have $\id_Y\circ f = f\circ \id_X = f$;
\item $\circ$ is associative.
\end{itemize}
We also write $gf$ or $g\cdot f$ instead of $gf$.
\end{definition}

Two arrows are called \udef{parallel} if they have the same domain and codomain.

TODO: we may characterise categories as typed associative classes.

\begin{example}
The terminology suggests that we can think of the objects as sets and the morphisms as functions. We will indeed often consider categories where the objects are sets (potentially with extra structure) and the morphisms are (a certain class of) functions (e.g.\ structure preserving functions).

Some examples of such categories are
\begin{itemize}
\item All sets are objects in the category $\cat{Set}$ and all functions are morphisms.
\item The category $\cat{Poset}$ has partially ordered sets as objects and order-preserving functions as morphisms.
\item A set may be identified with the set of its singleton subsets. One way to make this a category is by considering only the identity functions of the form
\[ \id_{\{a\}}: \{a\} \to \{a\}: a\mapsto a. \]
A category is \udef{discrete} if every morphism is an identity.
\end{itemize}

There are also categories that are not of this type:
\begin{itemize}
\item A preorder $(P, \precsim)$ can be regarded as a category. The elements of $P$ are the objects of the category and for $x,y\in P$ there exists a unique morphism $x\to y$ if and only if $x\precsim y$. Transitivity implies the existence of composites and reflexivity the existence of identity morphisms.
\item In particular (TODO: Von Neumann??) ordinals are preorders and thus define categories. We use the notation $\mathbb{0}, \mathbb{1}, \mathbb{2}, \ldots$ for the categories defined by $0,1,2,\ldots$. Also $\bbomega$ is the category defined by $\omega$.
\end{itemize}
\end{example}

If the category is small enough, we can depict it using a diagram. Points are objects and arrows are morphisms.

\begin{example}
\begin{align*}
\mathbb{1}&: \begin{tikzcd}
0 \arrow[loop left]
\end{tikzcd} \\
\mathbb{2}&: \begin{tikzcd}[ampersand replacement=\&]
0 \arrow[loop left] \ar[r] \& 1 \arrow[loop right]
\end{tikzcd} & &\text{\udef{walking arrow} or \udef{free arrow}} \\
\mathbb{I}&: \begin{tikzcd}[ampersand replacement=\&]
0 \arrow[loop left] \ar[r, shift left] \& 1 \ar[l, shift left] \arrow[loop right]
\end{tikzcd} & &\text{\udef{walking isomorphism} or \udef{free isomorphism}} \\
\mathbb{3}&: \begin{tikzcd}[ampersand replacement=\&]
0 \arrow[loop left] \ar[r] \ar[rr, bend left] \& 1 \ar[r] \arrow[loop below] \& 2 \arrow[loop right]
\end{tikzcd}
\end{align*}
\end{example}

\subsection{Hom sets and sizes of categories}
\begin{definition}
A category $\cat{C}$ is \udef{small} if its collection of morphisms is a set.

A category $\cat{C}$ is \udef{locally small} for any objects $X,Y$ if the collection of morphisms from $X$ to $Y$ is a set. This set is called a \udef{hom-set} and is denoted $\Hom(X,Y)$ or $\cat{C}(X,Y)$.
\end{definition}

\begin{lemma}
If $\cat{C}$ is a small category, then $\ob(\cat{C})$ is a set.
\end{lemma}
\begin{proof}
TODO: replacement with $X\mapsto \id_X$.
\end{proof}

\subsubsection{Thin categories}
\begin{definition}
A category with at most one morphism in each category is called \udef{thin}.
\end{definition}

\begin{lemma}
A small category
\begin{itemize}
\item with at most one element in every hom-set defines a preorder;
\item such that for all objects $X,Y$, the set $\Hom(X,Y)\cup \Hom(Y,X)$ is at most a singleton defines a partial order.
\end{itemize}
\end{lemma}
We call such categories \udef{preorder categories} and \udef{poset categories}.

\subsubsection{Skeletal categories}
\begin{definition}
The \udef{isomorphism class} of an object in a category is the collection of objects isomorphic to the object.

A category $\cat{C}$ is \udef{skeletal} if it contains one object in each isomorphism class.
\end{definition}
\begin{lemma}
Let $\cat{C}$ be a category.
There is a unique - up to isomorphism - skeletal category equivalent to $\cat{C}$.
\end{lemma}
This category is called the \udef{skeleton} of $\cat{C}$, denoted $\cat{skC}$.
\begin{proof}
TODO
\end{proof}

\subsubsection{Initial, terminal and zero objects; zero morphisms}
\begin{definition}
Let $\cat{C}$ be a category.
\begin{itemize}
\item An object $X\in\ob(\cat{C})$ is called \udef{initial} if $\cat{C}(X,Y)$ contains one element for each $Y\in\ob(\cat{C})$.
\item An object $Y\in\ob(\cat{C})$ is called \udef{terminal} if $\cat{C}(X,Y)$ contains one element for each $X\in\ob(\cat{C})$.
\item An object $Z\in\ob(\cat{C})$ is called a \udef{zero object} if $Z$ is both initial and terminal.
\end{itemize}
\end{definition}
Initial and terminal objects are dual notions.

\begin{proposition}
Let $\cat{C}$ be a category. Any two initial objects are isomorphic. Any two terminal objects are isomorphic.
\end{proposition}
\begin{proof}
Let $X_1,X_2$ be initial objects let $f:X_1\to X_2$ and $g:X_2\to X_1$ be the unique morphisms with these domains / codomains. Then $g\circ f: X_1 \to X_1 \in \cat{C}(X_1, X_1) = \{\id_{X_1}\}$. Similarly $f$ is also a left inverse of $g$. Thus $X_1$ and $X_2$ are isomorphic.

The case for terminal objects is dual.
\end{proof}

\begin{definition}
Let $\cat{C}$ be a category with zero object $0$. A \udef{zero morphism} $f: X\to Y$ is a morphism that can be factored as $f = h\circ g$, where $g: X\to 0$ and $h: 0\to Y$.
\end{definition}

\begin{proposition}
Let $\cat{C}$ be a category with zero object $0$.
\begin{enumerate}
\item Every hom-set contains exactly one zero morphism.
\item The composition of a zero morphism and any other morphism is a zero morphism.
\end{enumerate}
\end{proposition}

\subsection{(Commuting) diagrams}
Informally a diagram in a category is a drawing with points and arrows.

We always assume that all identity arrows and composite arrows are present in the diagram, even if they are rarely drawn.

We say that such a diagram commutes if, for any two points $x,y$ in the diagram, every path from $x$ to $y$ defines the same morphism.

\begin{example}
Let $(N_1, 0_1, S_1)$ and $(N_2, 0_2, S_2)$ be Peano systems and $\pi$ the unique isomorphism defined in \ref{existenceUniquenessPeano}. The condition
\[ \forall n\in N_1: \pi(S_1n) = S_2\pi(n) \]
is equivalent to saying
\[ \begin{tikzcd}
N_1 \ar[r, "\pi"] \ar[d, "S_1"] & N_2 \ar[d, "S_2"] \\
N_1 \ar[r, "\pi"] & N_2
\end{tikzcd} \qquad \text{commutes.} \]
The two paths in the diagram correspond to the two sides of the equation.
\end{example}

More formally we have the following definition:
\begin{definition}
Let $J$ be a category.
A \udef{diagram} of type (or shape) $\cat{J}$ in a category $\cat{C}$ is a (covariant) functor
\[ D: \cat{J} \to \cat{C}. \]
The category $\cat{J}$ is called the \udef{index category} or \udef{scheme} of the diagram $D$.

Usually we are interested in diagrams where $\cat{J}$ is small or even finite. In these cases we call the diagram \udef{small} or \udef{finite}.

If the index category $\cat{J}$ is a preordered set, then we call the diagram a \udef{commutative diagram}.

Let $\cat{I}$ be a subcategory of $\cat{J}$. Then the functor $D$ restricted to $\cat{I}$ is a \udef{subdiagram} of $D$.
\end{definition}
To actually draw such a diagram, you draw each object and morphism in the index category and name it using its image under $D$.

There is no requirement of injectivity, so the same object or morphism in $\cat{C}$ may appear several times in the diagram.

Every path in the index category with the same start $X$ and end $Y$ defines a morphism in the hom-set $\Hom(X,Y)$. In a preorder category there is at most one morphism in each hom-set, so each such path must correspond to the same morphism. Thus the informal and formal definitions of commutative diagram agree.

\begin{example}
The commutative square shown in the last example has the shape
\[ \mathbb{S}: \begin{tikzcd}
\boldsymbol{1} \ar[loop left] \ar[r] \ar[dr] \ar[d] & \boldsymbol{2} \ar[d] \ar[loop right] \\
\boldsymbol{3} \ar[loop left] \ar[r] & \boldsymbol{4} \ar[loop right]
\end{tikzcd}. \]
All morphisms in the category have been drawn. We also call this preorder category the \udef{commutative square}.
\end{example}

Any two paths in a commutative diagram with the same start and end yield an equation of morphism. Constructing proofs with these equations is known as \udef{diagram chasing} or \udef{abstract nonsense}.

\begin{example}
Warning! When drawing the diagram you draw the index category and label with the object category. You do not draw the image in the object category! This is not necessarily a preorder category when the index category is a preorder category. In fact it is not necessarily even a category. So, for example, the functor $F$ that maps
\[ \begin{tikzcd}
X_1 \dar{f} & X_2 \dar{g} \\ Y_1 & Y_2
\end{tikzcd} \qquad\text{to}\qquad \begin{tikzcd}
X = F(X_1) = F(X_2) \ar[d, shift right, swap, "f' = F(f)"] \ar[d, shift left, "g' = F(g)"] \\ Y = F(Y_1) = F(Y_2)
\end{tikzcd}  \]
should be drawn as
\[ \begin{tikzcd}
X \dar{f'} & X \dar{g'} \\ Y & Y
\end{tikzcd}, \qquad\text{not}\qquad \begin{tikzcd}
X \ar[d, swap, shift right, "f'"] \ar[d, shift left, "g'"] \\ Y
\end{tikzcd} \]
\end{example}


\subsubsection{Diagram chasing results}
\begin{lemma} \label{commutingRectangle}
Consider the diagram
\[ \begin{tikzcd}
X \rar{f}\dar{k} & Y \rar{g}\dar{h} & Z\dar{l} \\
A \rar{p} & B \rar{q} & C
\end{tikzcd}. \]
Suppose the left and right squares commute (as subdiagrams), then the whole diagram commutes.
\end{lemma}
\begin{proof}
There are two pairs of nodes that may have more than one morphism between them: $(X,C)$ and $(A,Z)$. We show that there is only one morphism in $\Hom(X,C)$. The other case is similar.

The question boils down to whether $qhf = qpk$. Now we know that $hf = pk$ because the left square commutes, so the identity is automatic.
\end{proof}

\subsection{Categories and associative classes}
\begin{lemma}
Let $\cat{C}$ be a category. We may view $\cat{C}$ as an associative class $\sSet{\mor(\cat{C})\uplus \{u\}, \widehat{\circ}}$, where
\[ \widehat{\circ}: (\mor(\cat{C})\uplus \{u\}) \times (\mor(\cat{C})\uplus \{u\}) \to (\mor(\cat{C})\uplus \{u\}): (f,g) \mapsto \begin{cases}
u & (\text{$f=u$ or $g=u$ or $fg$ is undefined}) \\
fg & (\text{otherwise}).
\end{cases} \]
\end{lemma}
This means we can use the results about associative classes in this context.

\subsubsection{Duality}
Intuitively category-theoretic duality is obtained by ``flipping all the arrows''.
\begin{definition}
Let $\cat{C}$ be a category. The \udef{opposite category} $\cat{C^{op}}$ is obtained by
\begin{itemize}
\item replacing the composition $\circ$ with its dual $\circ^d$;
\item swapping the domain and codomain functions
\[ \dom_{\cat{C^{op}}} \defeq \codom_{\cat{C}} \qquad \text{and} \qquad \codom_{\cat{C^{op}}} = \dom_{\cat{C}}.  \]
\end{itemize}
In particular $\ob(\cat{C}) = \ob(\cat{C^{op}})$ and $\mor(\cat{C}) = \mor(\cat{C^{op}})$.
\end{definition}
As we often suppress the composition $\circ$, and both the original and dual category have the same objects, it may sometimes be unclear in which category we are working. We then may write $f^o$ the emphasis that we are working in the dual category.

\begin{lemma}
The category $(\cat{C}^{op})^{op}$ is again the category $\cat{C}$.
\end{lemma}


A statement about the category $\cat{C^{op}}$ can often be interpreted as a statement about $\cat{C}$. This is referred to as the ``dual statement''. In other words: a statement in $\cat{C}$ is true if and only if the dual statement is true in $\cat{C^{op}}$.

If we prove that something is true in a collection of categories including $\cat{C}$ and $\cat{C^{op}}$, we have in effect proven both the statement and the dual statement. We then say the second follows from the first by duality.


\subsubsection{Mono- and epimorphisms}
\begin{definition}
Let $\cat{C}$ be a category. Then a morphism $f \in \mor(\cat{C})$ is called
\begin{itemize}
\item a \udef{monomorphism} or \udef{monic} if it is left-cancellative.
\item an \udef{epimorphism} or \udef{epic} if it is right-cancellative.
\end{itemize}
\end{definition}




\subsection{Left and right inverses}
\begin{definition}
Let $\cat{C}$ be a category and $f:X\to Y$ a morphism.
\begin{itemize}
\item A \udef{left inverse} (or \udef{retraction}) of $f$ is a morphism $g: Y\to X$ such that
\[ g\circ f = \id_X. \]
In this case we call $f$ a \udef{split monomorphism} or \udef{split monic}.
\item A \udef{right inverse} (or \udef{section}) of $f$ is a morphism $h: Y\to X$ such that
\[ f\circ h = \id_Y. \]
In this case we call $f$ a \udef{split epimorphism} or \udef{split epic}.
\item A \udef{(two-sided) inverse} of $f$ is a morphism that is both a left and a right inverse. In this case we call $f$ an \udef{isomorphism}.
\end{itemize}
If an isomorphism exists from $X$ to $Y$ (or equivalently from $Y$ to $X$), we say $X$ and $Y$ are \udef{isomorphic}, and write $X\cong Y$.
\end{definition}

An \udef{endomorphism} is a morphism $f$ such that $\dom(f) = \codom(f)$. An \udef{automorphism} is an endomorphism that is an isomorphism.

Note that a left or right inverse in the sense of category theory is \emph{not the same} as in the sense of an associative class.

\begin{lemma}
Let $f:X\to Y$ be a morphism. If $g: Y\to X$ is a left inverse and $h: Y\to X$ a right inverse of $f$, then $g=h$ and $f$ is an isomorphism. In particular an isomorphism can have at most one inverse.
\end{lemma}
\begin{proof}
The proof is identical to \ref{leftRightInverse}:
\[ g = g\id_Y = g(fh) = (gf)h = \id_Xh = h. \]
\end{proof}

\begin{lemma}\mbox{}
\begin{enumerate}
\item Every split monomorphism is a monomorphism.
\item Every split epimorphism is a epimorphism.
\end{enumerate}
\end{lemma}
The converses are \emph{not true}.

\begin{lemma}
In the category $\cat{Set}$
\begin{enumerate}
\item every monomorphism is split;
\item the assertion that every epimorphism is split is equivalent to the axiom of choice.
\end{enumerate}
\end{lemma}
\begin{proof}
This is a restatement of \ref{injectiveInverse} and \ref{surjectiveInverse}.
\end{proof}

\begin{lemma} \label{singleSplitImpliesDoubleSplit}
If a monomorphism is split epic, it is also split monic, and thus an isomorphism.

If an epimorphism is split monic, it is also split epic, and thus an isomorphism.
\end{lemma}
\begin{proof}
Assume a morphism $f:X\to Y$ is monic and has a right inverse $r$. We claim $r$ is also a left inverse. Indeed
\[ fr = \id_Y \implies frf = \id_Yf = f\id_X \implies rf = \id_x  \]
where the second implication is because $f$ is monic. The second claim of the lemma follows by duality.
\end{proof}


\begin{lemma}
In a preorder category, every morphism that is monic or epic, is an isomorphism.

In a poset category, every morphism that is monic or epic, is an identity.
\end{lemma}

\begin{lemma}
Let $f:X\to Y$ and $g:Y\to X$ be morphisms in a category $\cat{C}$. Then
\begin{enumerate}
\item $g$ is a left inverse of $f$ if and only if the diagram
\[ \begin{tikzcd}
X \rar{f} \ar[rr, bend right, "\id_X"] & Y \rar{g} & X
\end{tikzcd} \qquad \text{commutes;} \]
\item $g$ is a right inverse of $f$ if and only if the diagram
\[ \begin{tikzcd}
Y \rar{g} \ar[rr, bend right, "\id_Y"] & X \rar{f} & Y
\end{tikzcd} \qquad \text{commutes.} \]
\end{enumerate}
\end{lemma}


\subsubsection{Groups and groupoids}
\begin{definition}
A \udef{groupoid} is a category in which every morphism is an isomorphism.

A \udef{group} is a groupoid with one object.
\end{definition}

\begin{lemma}
For any category $\cat{C}$ there exists a subcategory containing all the objects of $\cat{C}$ and only the morphisms of $\cat{C}$ that are isomorphisms. This subcategory is called the \udef{maximal groupoid} inside $\cat{C}$.
\end{lemma}

\subsubsection{The core of a category}
\begin{definition}
Let $\cat{C}$ be a category.
\begin{itemize}
\item The \udef{core} of $\cat{C}$ is the groupoid whose objects are the objects of $\cat{C}$ and whose morphisms are the isomorphisms of $\cat{C}$.
\end{itemize}
\end{definition}

Thus a category is a groupoid if and only if it equals its core.

\subsection{Subcategories}
\begin{definition}
Let $\cat{C}$ be a category. We call $\cat{D}$ a \udef{subcategory} of $\cat{C}$ if
\begin{itemize}
\item $\ob(\cat{D}) \subseteq \ob(\cat{C})$;
\item $\mor(\cat{D}) \subseteq \mor(\cat{C})$;
\item $\forall f\in \mor(\cat{D}): \; \dom(f)\in\ob(\cat{D}) \land \codom(f)\in\ob(\cat{D})$;
\item $\forall X\in \ob(\cat{D}): \; \id_X\in \mor(\cat{D})$;
\item $\forall f,g \in \mor(\cat{D}): \; fg \in \mor(\cat{D})$.
\end{itemize}
We call the subcategory \udef{full} if for all $X,Y\in \ob(\cat{D}): \; \cat{C}(X,Y) = \cat{D}(X,Y)$.
\end{definition}

\section{Functors}
\begin{definition}
A \udef{(covariant) functor} $F:\cat{C}\to\cat{D}$ consists of
\begin{enumerate}
\item a function $F: \ob(\cat{C}) \to \ob(\cat{D})$;
\item a function $F: \mor(\cat{C}) \to \mor(\cat{D})$;
\end{enumerate}
satisfying the functorial properties:
\begin{itemize}
\item $F(f: a\to b) = F(f): F(a) \to F(b)$;
\item for any composable pair $g,f$ in $\cat{C}$, $F(g\circ f) = F(g)\circ F(f)$;
\item for each object $c$ in $\cat{C}$, $F(\id_c) = \id_{F(c)}$.
\end{itemize}
A \udef{contravariant functor} $F:\cat{C}\to\cat{D}$ is similar, except it satisfies
\begin{itemize}
\item $F(f: a\to b) = F(f): F(b) \to F(a)$;
\item for any composable pair $g,f$ in $\cat{C}$, $F(g\circ f) = F(f)\circ F(g)$;
\item for each object $c$ in $\cat{C}$, $F(\id_c) = \id_{F(c)}$.
\end{itemize}
An \udef{endofunctor} is a functor between a category and itself.
\end{definition}

\begin{lemma}
The composition of two contravariant functors is covariant.
\end{lemma}

\begin{lemma}
Let $\cat{C}$ be a category. The functions
\[ \begin{cases}
\id_{\ob(\cat{C})} = \id_{\ob(\cat{C^{op}})} \\
\id_{\mor(\cat{C})} = \id_{\mor(\cat{C^{op}})}
\end{cases} \qquad \text{form a contravariant functor $\operatorname{OP}: \cat{C} \to \cat{C^{op}}$.
} \]
We can compose any contravariant functor $F: \cat{C} \to \cat{D}$ with the functor $\operatorname{OP}$ to obtain a covariant functor $F: \cat{C} \to \cat{D^{op}}$ or $F: \cat{C^{op}} \to \cat{D}$.
\end{lemma}

\begin{example}
Examples of covariant functors:
\begin{itemize}
\item For any category $\cat{C}$, there is an \udef{identity functor} $\id_\cat{C}$ that maps objects and morphisms to themselves.
\item There is an endofunctor $\mathcal{P}:\cat{Set} \to \cat{Set}$ that maps sets to their powerset and functions $f:A\to B$ to their image function $f[\cdot]:\mathcal{P}(A)\to \mathcal{P}(B)$.
\item \udef{Forgetful functors} are functors from the category of some type of structured set to a category of structured set with less (or) no structure. For example, we have a forgetful functor $U: \cat{Poset} \to \cat{Set}$ that maps the cateory $\cat{Poset}$ into the category $\cat{Set}$ by forgetting the order. 
\end{itemize}
Examples of contravariant functors:
\begin{itemize}
\item The contravariant endofunctor $\mathcal{P}:\cat{Set} \to \cat{Set}$ that maps sets to their powerset and functions $f:A\to B$ to their inverse image function $f^{-1}[\cdot]:\mathcal{P}(B)\to \mathcal{P}(A)$.
\end{itemize}
\end{example}

\begin{lemma} \label{functorMorphismPreservation}
Functors preserve split monomorphisms, split epimorphisms and isomorphisms.
\end{lemma}
Functors do not necessarily preserve monomorphisms and epimorphisms.
\begin{corollary}
Let $F:\cat{C}\to \cat{D}$ be a functor. Let $f:X \to Y$ be monic (resp. epic). If $F(f)$ is not monic (resp. epic), then $f$ is not split monic (resp. split epic).
\end{corollary}

\begin{lemma}
Functors preserve commutative diagrams: Let $D:\cat{J}\to \cat{C}$ be a commutative diagram and $F:\cat{C}\to \cat{B}$ a functor. Then $F\circ D$ is a commutative diagram.
\end{lemma}

\subsection{Functors as morphisms}
\begin{definition}
We define $\cat{Cat}$ as the category whose objects are small categories and whose morphisms are functors between them.

We define $\cat{CAT}$ as the category whose objects are locally small categories and whose morphisms are functors between them.
\end{definition}
HOW DOES $\cat{CAT}$ MAKE SENSE?

\begin{lemma}
$\cat{Cat}$ is not small, but it is locally small.

$\cat{CAT}$ is not locally small.
\end{lemma}
There is an inclusion functor $\cat{Cat} \hookrightarrow \cat{CAT}$.

We can obviously consider isomorphisms in the category $\cat{Cat}$, but this is not a very natural notion to work with. For example, a category is not typically isomorphic to its opposite category.

The more natural concept is equivalence of categories.

\begin{lemma} \label{varianceOfComposite}
Let $F,G$ be composable morphisms in $\cat{CAT}$, i.e.\ functors. Then $FG$ is contravariant if exactly one of $F,G$ is contravariant and covariant otherwise.
\end{lemma}


\subsection{Properties of functors}
\begin{definition}
A functor $F:\cat{C}\to\cat{D}$ is called
\begin{itemize}
\item \udef{full} if for all $x,y\in \cat{C}$, the map $\cat{C}(x,y) \to \cat{D}(Fx,Fy)$ is surjective;
\item \udef{faithful} if for all $x,y\in \cat{C}$, the map $\cat{C}(x,y) \to \cat{D}(Fx,Fy)$ is injective;
\item and \udef{essentially surjective on objects} if for every object $d\in\cat{D}$ there is some $c\in\cat{C}$ such that $d$ is isomorphic to $Fc$.
\end{itemize}
These are local definitions. We also call $F$
\begin{itemize}
\item an \udef{embedding} if it is injective on objects;
\item a \udef{full embedding} if it is full, faithful and an embedding.
\end{itemize}
\end{definition}

\begin{lemma} \label{isomorphismCreationReflection}
If $F:\cat{C}\to\cat{D}$ is a full and faithful functor, the $F$ reflects isomorphisms:
If $f$ is a morphism in $\cat{C}$ so that $Ff$ is an isomorphism in $\cat{D}$, then $f$ is an isomorphism.
\end{lemma}
Thus, if $x$ and $y$ are objects in $\cat{C}$ such that $Fx$ and $Fy$ are isomorphic in $\cat{D}$, then $x$ and $y$ are isomorphic in $\cat{C}$.

By \ref{functorMorphismPreservation} any functor preserves isomorphisms.

\begin{proposition}
Let $F:\cat{C}\to\cat{D}$ be a functor.
\begin{enumerate}
\item If $F$ is an embedding, then the image of $F$ is a subcategory of $\cat{D}$.
\item If $F$ is a full embedding, then the image of $F$ is a full subcategory of $\cat{D}$.
\end{enumerate}
\end{proposition}

\subsubsection{Abstract and concrete categories}
\begin{definition}
A category $\cat{C}$ is called \udef{concrete} if there exists a faithful functor $F: \cat{C} \to \cat{Set}$.
If a category is not concrete, it is called \udef{abstract}.
\end{definition}

\begin{proposition} \label{injectiveMonoSurjectiveEpi}
In any concrete category
\begin{enumerate}
\item $f$ is an injective function implies $f$ is monomorphic;
\item $f$ is a surjective function implies $f$ is epimorphic.
\end{enumerate}
In the category $\cat{Set}$, the opposite implications also hold.
\end{proposition}
\begin{proof}
Set $f:X\to Y$.
\begin{enumerate}
\item Assume $f$ injective. Let $g,h: Z\to X$ such that $g\neq h$. Because we are in a concrete category, this means there exists a $z\in Z$ such that $g(z) \neq h(z)$. By injectivity of $f$ this means $f(g(z)) \neq f(h(z))$ and so $fg \neq fh$. We conclude $f$ is monic by contraposition.
\item Assume $f$ surjective. Let $g,h: Y\to Z$ such that $g\neq h$. Because we are in a concrete category, this means there exists a $y\in Y$ such that $g(y) \neq h(y)$. By surjectivity of $f$ we can find an $x\in X$ such that $f(x) = y$. So $g(f(x)) \neq h(f(x))$ and we conclude $gf \neq hf$, meaning $f$ is epic by contraposition.
\end{enumerate}
Suppose we are now in the category $\cat{Set}$.
\begin{enumerate}
\item Assume $f$ monic. To prove injectivity, take arbitrary $x_1,x_2\in X$ and assume $f(x_1) = f(x_2)$. Now define the constant functions $x_1: Z\to X: z\mapsto x_1$ and $x_2: Z\to X: z\mapsto x_2$. Then
\[ fx_1 = f(x_1) = f(x_2) = f(x_2) \qquad \implies \qquad x_1 = x_2 \]
because $f$ monic.
\item Assume $f$ epic. Left compose $f$ with both the characteristic function $\chi_{f[X]}$ and the constant function $1: Y\to \{1\}: y\mapsto 1$. It is clear $\chi_{f[X]}f = 1f$, so $\chi_{f[X]} = 1$ meaning $f$ is surjective.
\end{enumerate}
\end{proof}

\subsection{Comma categories}
\begin{definition}
Let $\cat{A}, \cat{B}$ and $\cat{C}$ be categories and $S, T$ functors of type $\begin{tikzcd}
\cat{A} \ar[r, "S"] & \cat{C} & \lar[swap]{T} \cat{B}
\end{tikzcd}$. The \udef{comma category} $S\downarrow T$ is the category with
\begin{itemize}
\item as objects tuples of the form $((A,B), h)$, where $A\in\ob(\cat{A}), B\in\ob(\cat{B})$ and $h: S(A) \to T(B)$ a morphism in $\cat{C}$;
\item as morphisms pairs $(f,g): ((A,B), h) \to ((A',B'), h')$ where $f:A\to A'$, $g: B\to B'$ such that
\[ \begin{tikzcd}
S(A) \rar{S(f)} \dar{h} & S(A') \dar{h'} \\
T(B) \rar{T(g)} & T(B')
\end{tikzcd} \qquad\text{commutes.} \]
\end{itemize}
Composition is defined by $(f',g') \circ (f,g) = (f'\circ f, g'\circ g)$ and the identity on $((A,B),h)$ is $(\id_A, \id_B)$.
\end{definition}
The composition $(f',g') \circ (f,g) = (f'\circ f, g'\circ g)$ is a morphism in the comma category by \ref{commutingRectangle}.

\subsubsection{Arrow category}
\begin{definition}
An \udef{arrow category} is a comma category of type $\begin{tikzcd}
\cat{C} \ar[r, "\id_\cat{C}"] & \cat{C} & \lar[swap]{\id_\cat{C}} \cat{C}
\end{tikzcd}$. In is denoted $\cat{C}^\to$.
\end{definition}

\subsubsection{Slice and coslice categories}
\begin{definition}
Let $\cat{A}$ be a category and $a$ an object in $\cat{A}$.
\begin{itemize}
\item The \udef{slice category} over $a$, denoted $\cat{A}/a$, is the comma category of $\begin{tikzcd}
\cat{A} \rar{\id_\cat{A}} & \cat{A} & \lar[swap]{(0\mapsto a)} \cat{\mathbb{1}}
\end{tikzcd}$.
\item The \udef{coslice category} under $a$, denoted $a/\cat{A}$, is the comma category of $\begin{tikzcd}
\cat{\mathbb{1}} \rar{(0\mapsto a)} & \cat{A} & \cat{A} \lar[swap]{\id_\cat{A}} 
\end{tikzcd}$.
\end{itemize}
\end{definition}

Objects in a slice category $\cat{A}/a$ are always of the form $((X, a),h: X\to a)$ and morphisms are always of the form $(f, \id_a)$. So we may drop the dependence on $a$.
\begin{lemma}
Let $\cat{A}$ be a category and $a$ an object in $\cat{A}$.
\begin{enumerate}
\item The objects of the slice category $\cat{A}/a$ may be written as $(X,h)$. A morphism $f: X\to Y$ in $\cat{A}$ may be seen as a morphism $f: (X, h) \to (Y, h')$ in $\cat{A}/a$ if
\[ \begin{tikzcd}[column sep=small]
X \ar[rr, "f"]\arrow[dr, swap, "h"]& & Y \arrow[dl, "h'"]  \\
& a &
\end{tikzcd} \qquad\text{commutes.} \]
\item The objects of the coslice category $a/\cat{A}$ may be written as $(X,h)$. A morphism $f: X\to Y$ in $\cat{A}$ may be seen as a morphism $f: (X, h) \to (Y, h')$ in $\cat{A}/a$ if
\[ \begin{tikzcd}[column sep=small]
& a & \\
X \ar[rr, swap, "f"]\ar[ur, "h"]& & Y \ar[ul, swap, "h'"] 
\end{tikzcd} \qquad\text{commutes.} \]
\end{enumerate}
\end{lemma}


\subsection{Constructions of categories}
\subsubsection{Subcategories}
\begin{definition}
Let $\cat{C}$ be a category. A category $\cat{D}$ is a \udef{subcategory} of $\cat{C}$ if all its objects are objects of $\cat{C}$ and all its morphisms are morphisms of $\cat{C}$.
\end{definition}


\subsubsection{Product categories}
\begin{definition}
The \udef{product} $\cat{C}\times \cat{D}$ of categories $\cat{C}$ and $\cat{D}$  is a category consisting of
\begin{enumerate}
\item ordered pairs $(c,d)$ where $c\in \cat{C}$ and $d\in \cat{D}$ as objects;
\item ordered pairs $(f,g): (c,d) \to (c',d')$ where $f:c\to c'$ and $g:d\to d'$ as morphisms.
\end{enumerate}
Composition and identities are defined component-wise.
\end{definition}



\section{Naturality}
\subsection{Natural transformations}
\begin{definition}
Let $\cat{C}, \cat{D}$ be categories and $F,G: \cat{C}\to \cat{D}$ functors. A \udef{natural transformation} $\alpha: F \Rightarrow G$ consists of
\begin{enumerate}
\item an arrow $\alpha_c: Fc\to Gc$ in $\cat{D}$, called a \udef{component} of the natural transformation, for each object $c\in \cat{C}$
\end{enumerate}
such that, for any morphism $f:c\to c'$ in $\cat{C}$, 
\[ \begin{tikzcd}
Fc \ar[r, "\alpha_c"] \ar[d, swap, "Ff"] & Gc \ar[d, "Gf"] \\
Fc' \ar[r, "\alpha_{c'}"] & Gc'
\end{tikzcd} \qquad \text{commutes}.\]
A \udef{natural isomorphism} is a natural transformation in which every component is an isomorphism. We write $F\cong G$ in this case.
\end{definition}

We represent the natural transformation $\alpha: F \Rightarrow G$ between $F,G: \cat{C}\to \cat{D}$ diagrammatically as
\[ \begin{tikzcd}[column sep=huge]
\cat{C}
  \arrow[bend left=50]{r}[name=U,label=above:$F$]{}
  \arrow[bend right=50]{r}[name=D,label=below:$G$]{} &
\cat{D}
  \arrow[shorten <=10pt,shorten >=10pt,Rightarrow,to path={(U) -- node[label=right:$\alpha$] {} (D)}]{}
\end{tikzcd} \]

The origin of this commutative square can be found in the following picture:

\begin{figure}[h!]
\centering
\begin{tikzpicture}[commutative diagrams/every diagram]
\matrix[matrix of math nodes, commutative diagrams/every cell] (mC) at (0,1.5) {c \\ \\ c' \\};
\path[commutative diagrams/.cd,every arrow,every label](mC-1-1)edge[blue] node {$f$}(mC-3-1);
\matrix[matrix of math nodes, commutative diagrams/every cell] (mD) at (6,0) {Fc &&  \\ && Gc \\ Fc'  && \\ && Gc' \\};
\path[commutative diagrams/.cd,every arrow,every label](mD-1-1)edge[red] node {$\alpha_c$}(mD-2-3);
\path[commutative diagrams/.cd,every arrow,every label] (mD-2-3) edge[blue] node {$Gf$} (mD-4-3);
\path[commutative diagrams/.cd,every arrow,every label](mD-3-1)edge[red] node {$\alpha_{c'}$}(mD-4-3);
\path[commutative diagrams/.cd,every arrow,every label](mD-1-1)edge[blue] node {$Ff$}(mD-3-1);

\path[commutative diagrams/.cd,every arrow,every label](mC-1-1)edge node {$F$}(mD-1-1);
\path[commutative diagrams/.cd,every arrow,every label](mC-1-1)edge[pos=0.34, swap] node {$G$}(mD-2-3);

\path[commutative diagrams/.cd,every arrow,every label](mC-3-1)edge node {$F$}(mD-3-1);
\path[commutative diagrams/.cd,every arrow,every label](mC-3-1)edge node[pos=0.36, swap] {$G$}(mD-4-3);

\node[draw=black,fit=(mC),circle, label={above left:$\cat{C}$}] (catC) {};
\node[draw=black,fit=(mD),circle, label={above right:$\cat{D}$}] (catD) {};
\end{tikzpicture}
\end{figure}
Thus a natural transformation is one that shifts the image of a functor along the morphisms in the target category.

This leads us to the following proposition, which is sometimes used to define natural transformations:
\begin{proposition}
Let $F,G:\cat{C}\to\cat{D}$ be functors. The natural transformations $\alpha: F\Rightarrow G$ correspond bijectively to functors $H: \cat{C}\times\mathbb{2}\to \cat{D}$ such that
\[ \begin{tikzcd}
\cat{C} \ar[r,"i_0"] \ar[dr, swap, "F"] & \cat{C}\times\mathbb{2} \ar[d,"H"] & \cat{C} \ar[l, "i_1"] \ar[dl, "G"] \\
& \cat{D} &
\end{tikzcd} \qquad \text{commutes.} \]
Here $i_0, i_1$ are the functors that map objects $c$ to $(c,0)$ and $(c,1)$ respectively.

The same holds for natural isomorphisms, if $\mathbb{2}$ is replaced by $\mathbb{I}$.
\end{proposition}

\begin{example}
Let $M$ be a monoid. A \udef{dynamical system} is a functor $F: \cat{B}M\to\cat{C}$ from the delooping space to some category. Often this category is $\cat{Top}$ or $\cat{Vect}$. Then the single object of $\cat{B}M$ is mapped to an object $c$ in $\cat{C}$ and the morphisms of $\cat{B}M$ become actions on the object: each element $m\in M$ specifies a morphism in $\cat{B}M$, which is mapped to a morphism in $\cat{C}$
\[ m: c\to c: x\mapsto m\cdot x. \]

A morphism of dynamical systems, or \udef{intertwiner}, is a map between natural transformation $\alpha:F\to G$. This can be seen as a map between objects in $\cat{C}$ that is covariant, in the sense that
\[ \alpha(m\cdot x) = m\cdot \alpha(x). \]
\end{example}

\subsubsection{Natural transformations as canonical maps}
A canonical map $\alpha$ is a map between objects that arises naturally from the definition or the construction of the objects.

TODO: make rigourous! explicate for universal algebra.

In some sense the canonical map works at the level of structure and thus must commute with morphisms.

A \udef{canonical map} is a functor such that there exists a natural transformation from the identit functor to it.

TODO: what about between different categories?






\subsection{Equivalence of categories}
\begin{definition}
An \udef{equivalence of categories} consists of functors $F: \cat{C} \to \cat{D}$ and $G: \cat{D} \to \cat{C}$ such that we have natural isomorphisms $\id_\cat{C} \cong GF$ and $\id_\cat{D} \cong FG$.

The categories $\cat{C}, \cat{D}$ are \udef{equivalent} if there exists an equivalence between them. We write $\cat{C} \simeq \cat{D}$.
\end{definition}

Except for considerations of size, this equivalence is an equivalence relation.
\begin{lemma}
The notion of equivalence is reflexive, symmetric and transitive.
\end{lemma}



\begin{proposition}
A functor defining an equivalence of categories is full, faithful, and essentially surjective on objects.

Assuming the axiom of choice, any functor with these properties defines an equivalence of categories.
\end{proposition}


\begin{lemma}
An equivalence between skeletal categories is an isomorphism of categories
\end{lemma}
\begin{corollary}
Two categories are equivalent \textup{if and only if} their skeletons are isomorphic.
\end{corollary}

\begin{definition}
A category $\cat{C}$ is called
\begin{itemize}
\item \udef{essentially small} if it is equivalent to a small category (i.e.\ its skeleton is small);
\item \udef{essentially discrete} if it is equivalent to a discrete category (i.e.\ its skeleton is discrete).
\end{itemize}
\end{definition}

\begin{lemma}
Every preorder category is equivalent to a poset category.
\end{lemma}



\section{Monoidal categories}
\begin{definition}
A \udef{monoidal category} is a category $\cat{C}$ equipped with
\begin{enumerate}
\item a functor $\otimes: \cat{C}\times\cat{C} \to \cat{C}$ called the \udef{tensor product};
\item an object $1\in \cat{C}$ called the \udef{unit object} or the \udef{tensor unit}
\end{enumerate}
We write $(\cat{C}, \otimes, 1, \alpha, \lambda, \rho)$.
\end{definition}

\subsection{Enrichment}
\begin{definition}
Let $(\cat{V}, \otimes, 1, \alpha, \lambda, \rho)$ be a monoidal category. A \udef{$\cat{V}$-category} $\cat{C}$ (or \udef{$\cat{V}$-enriched category} or \udef{category enriched over $\cat{V}$}) contains
\begin{enumerate}
\item 
\end{enumerate}
such that
\end{definition}


\chapter{Higher category theory}
\section{2-categories}
\subsection{Functor categories}
\begin{proposition} \label{functorCategory}
Let $\cat{C}, \cat{D}$ be categories. There is a category of functors $\cat{C}\to \cat{D}$ with as morphisms natural transformations called the \udef{functor category} $[\cat{C},\cat{D}]$.
\end{proposition}
\begin{proof}
There are clearly identity natural transformations $I:F\Rightarrow F$ for each functor $F: \cat{C}\to \cat{D}$ with components given by $I_{Fc}$ for all $c\in\cat{C}$.

To verify composition, let $\alpha: F \Rightarrow G$ and $\beta: G\Rightarrow H$ be natural transformations between parallel functors $F,G,H: \cat{C}\to \cat{D}$. Then the composition	$\beta\alpha: F\Rightarrow H$ has components
\[ (\beta\alpha)_c = \beta_c\alpha_c. \]
We just need to show $\beta\alpha$ is a natural transformation. Consider the diagram
\[ \begin{tikzcd}
Fc \ar[r, "\alpha_c"] \ar[d, "Ff"] & Gc \ar[d, "Gf"] \ar[r, "\beta_c"] & Hc \ar[d,"Hf"] \\
Fc' \ar[r, "{\alpha_{c'}}"] & Gc' \ar[r, "{\beta_{c'}}"] & Hc'
\end{tikzcd} \]
Both squares commute by naturality of $\alpha,\beta$. The rectangle then commutes: the only non-trivial part is the equality of paths $Fc\to Hc'$, but any such path can be deformed into another by flipping over squares (TODO better reference).
\end{proof}

\subsection{The 2-category of categories}


\chapter{Representability and universal properties}
\section{The Yoneda lemma}
\url{https://math.stackexchange.com/questions/149376/excessive-use-of-the-yoneda-lemma}
\url{https://qchu.wordpress.com/2012/04/02/the-yoneda-lemma-i/}
\begin{lemma}
Let $\cat{C},\cat{D}$ be categories. For every $c\in\cat{C}$, there is a functor defined by
\[ \evalMap_c: [\cat{C},\cat{D}] \to \cat{D}: F\mapsto Fc. \]
\end{lemma}
\begin{proof}
A natural transformation $\alpha: F \Rightarrow G$ is mapped to its component $\alpha_c: Fc \to Gc$. By \ref{functorCategory}, for all $F$, $I_F\mapsto I_{Fc}$ and the composition of composable natural transformations $\beta,\alpha$ has the component $\beta_c\alpha_c$ at $c$.
\end{proof}
So this gives us a functor
\[ \evalMap: (\cat{Set^C}, \cat{C}) \to \cat{Set}: \quad (F,c)\mapsto Fc.\]

Let $\cat{C}$ be a small category. Then for a given $c\in\cat{C}$, $\cat{C}(c,-)$ is the covariant functor represented by $c$ and thus an object in $[\cat{C},\cat{\Set}]$, which is locally small. Then we can consider the covariant functor represented by $\cat{C}(c,-)$,
\[ [\cat{C},\cat{\Set}]\to \cat{Set}:\quad F \mapsto \Hom(\cat{C}(c,-), F). \]

By currying, we can consider the two-sided represented functor $\cat{C}(-,-): \cat{C^{op}}\times\cat{C}\to \cat{Set}$ as a contravariant functor $\cat{C} \to \cat{Set^C}: c\mapsto \cat{C}(c,-)$. Composing this with the contravariant functor represented by $F$ gives the covariant functor
\[ \cat{C} \to \cat{Set}:\quad c \mapsto \Hom(\cat{C}(c,-), F). \]

Combining these two functors we have a bifunctor
\[ (\cat{Set^C}, \cat{C}) \to \cat{Set}: \quad(F,c) \mapsto \Hom(\cat{C}(c,-), F). \]

If we now let $\cat{C}$ be only locally small, not necessarily small, $\cat{Set^C}$ is no longer necessarily locally small. It turns out however that $\Hom(\cat{C}(c,-), F)$ still always yields a set. This assertion is part of the Yoneda lemma. The other part is the assertion that there is a natural isomorphism $\Phi$:
\[ \begin{tikzcd}
\cat{C}\times\cat{Set^C} \ar[rr, bend left=50, "{\Hom(\cat{C}(c,-), F)}"] \ar[rr, bend right=50, "\evalMap_c(F)"] & \Phi\Downarrow\cong & \cat{Set}
\end{tikzcd}. \]

\begin{theorem}[Yoneda lemma] \label{YonedaLemma}
Let $\cat{C}$ be a locally small category.

Given a functor $F: \cat{C} \to \cat{Set}$ and an object $c$ in $\cat{C}$, the function
\[ \Phi_{c,F}: \Hom(\cat{C}(c,-), F) \to Fc: \alpha \mapsto \alpha_c(I_c) \]
is a bijection.

Also $\Phi$ is a natural transformation with components $\Phi_{c,F}$.
\end{theorem}
\begin{corollary}[Yoneda embedding] \label{YonedaEmbedding}
Let $\cat{C}$ be a locally small category. The functors
\[ \begin{tikzcd}
\cat{C} \ar[r, hook] & \cat{Set^{C^{op}}} \\
c \ar[d, swap, "f"]\ar[r, mapsto] \ar[r,mapsto, shift right=1.7em] & \cat{C}(o(-),c) \ar[d, "f_*"] \\
d \ar[r,mapsto] & \cat{C}(o(-),d)
\end{tikzcd} \qquad\text{and}\qquad \begin{tikzcd}
\cat{C^{op}} \ar[r, hook] & \cat{Set^C} \\
c \ar[r, mapsto] \ar[r,mapsto, shift right=1.7em] & \cat{C}(c,-) \\
d \ar[u, "f^o"] \ar[r,mapsto] & \cat{C}(d,-) \ar[u, swap, "f^*"]
\end{tikzcd} \]
define full and faithful embeddings.
\end{corollary}
\begin{proof}
The functors are clearly injective on objects as morphisms with different domain or codomain are different.

Consider first the left embedding. We want the transformation $f_*$ to be natural (this is the transformation with all components equal to $f_*$). Indeed the functors $\cat{C}(o(-),c)$ and $\cat{C}(o(-),c)$ map morphisms $g^o$ in $\cat{C^{op}}$ to $g^*$. Now $f_*$ and $g^*$ clearly commute, proving the naturality of $f_*$.

Then the function $\cat{C}(c,d) \to \Hom(\cat{C}(-,c),\cat{C}(-,d)): f\mapsto f_*$ is the inverse of the Yoneda map $\Phi_{c,\cat{C}(-,d)}$:
\[ \Phi_{c,\cat{C}(-,d)}(f_*) = f_*(I_c) = f\circ I_c = f. \]
Thus by the Yoneda lemma it is bijection, meaning
\[ \cat{C}(c,d) \cong \Hom(\cat{C}(-,c),\cat{C}(-,d)) \]
and the left functor is full and faithful by definition.

The statement for the right functor can be proven analogously. Alternatively, we can see that it is a dual statement in the following way: taking the category to be $\cat{C^{op}}$, we get
\[ \cat{C^{op}}(c,d) \cong \Hom(\cat{C^{op}}(-,c),\cat{C^{op}}(-,d)) \]
or
\[ o[\cat{C}(d,c)] \cong \Hom(o\cdot\cat{C}(c,-)\cdot o,o\cdot\cat{C}(d,-)\cdot o) \]
now for every $\alpha \in \Hom(\cat{C}(c,-),\cat{C}(d,-))$, there is an $\alpha^o$ in $\Hom(o\cdot\cat{C}(c,-)\cdot o,o\cdot\cat{C}(d,-)\cdot o)$, so we have an isomorphism
\[ \cat{C}(d,c) \cong \Hom(\cat{C}(c,-),\cat{C}(d,-)). \]
\end{proof}

It is common to refer both to the Yoneda lemma \ref{YonedaLemma} and its corollary on the Yoneda embedding, \ref{YonedaEmbedding}, as the Yoneda lemma.

\begin{proposition} \label{representablyIsomorphic}
Let $\cat{C}$ be a locally small category. The following are equivalent:
\begin{enumerate}
\item $f:x\to y$ is an isomorphism in $\cat{C}$;
\item $f_*: \cat{C}(-,x) \Rightarrow \cat{C}(-,y)$ is a natural isomorphism;
\item $f^*: \cat{C}(y,-) \Rightarrow \cat{C}(x,-)$ is a natural isomorphism.
\end{enumerate}
\end{proposition}
\begin{proof}
We have already shown $f_*$ and $f^*$ are natural transformations in \ref{YonedaEmbedding}.
The proof follows from the Yoneda embedding \ref{YonedaEmbedding} and \ref{isomorphismCreationReflection}.

Alternatively, a partial proof is as follows: If $f$ has an inverse $f^{-1}$, then $(f^{-1})^*$ is an inverse of $f^*$ and $(f^{-1})_*$ an inverse of $f_*$.
\end{proof}

\section{Representable functors}
\subsection{Functors represented by objects}
\begin{definition}
Let $\cat{C}$ be a locally small category and $c$ an object in $\cat{C}$.

Given a morphism $f: X\to Y$ in $\cat{C}$, we can define the functions
\begin{enumerate}
\item $f_*: \cat{C}(c,X) \to \cat{C}(c,Y): g\mapsto fg$ defined by post-composition;
\item $f^*: \cat{C}(Y,c) \to \cat{C}(X,c): g\mapsto gf$ defined by pre-composition.
\end{enumerate}
These are morphisms in the category $\cat{Set}$.

Based on this we can define the \udef{covariant functor represented by $c$} $\cat{C}(c,-)$ and the \udef{contravariant functor represented by $c$} $\cat{C}(-,c)$:
\[ \begin{tikzcd}
\cat{C} \ar[r, "{\cat{C}(c,-)}"] & \cat{Set} \\
x \ar[d, swap, "f"]\ar[r, mapsto] \ar[r,mapsto, shift right=1.7em] & \cat{C}(c,x) \ar[d, "f_*"] \\
y \ar[r,mapsto] & \cat{C}(c,y)
\end{tikzcd} \qquad\qquad \begin{tikzcd}
\cat{C} \ar[r, "{\cat{C}(-,c)}"] & \cat{Set} \\
x \ar[r, mapsto] \ar[r,mapsto, shift right=1.7em] & \cat{C}(x,c) \ar[d, "f^*"] \\
y \ar[u, "f"] \ar[r,mapsto] & \cat{C}(y,c) 
\end{tikzcd} \]
We also have the \udef{two-sided represented functor} $\cat{C}(-,-):\cat{C}\times \cat{C} \to \cat{Set}$, which is contravariant in the first argument and covariant in the second:
\[ \begin{tikzcd}
\cat{C}\times\cat{C} \ar[r, "{\cat{C}(-,-)}"] & \cat{Set} \\
(x,y) \ar[d, shift left]\ar[r, mapsto] \ar[r,mapsto, shift right=1.7em] & \cat{C}(x,y) \ar[d, "{f^*h_*: g\mapsto hgf}"] \\
(w,z) \ar[u, shift left, "{(f,h)}"] \ar[r,mapsto] & \cat{C}(w,z)
\end{tikzcd} \]
\end{definition}
The placement of the asterisk indicates the placement of the function being composed with $f$: below means to the right and above to the left. This is consistent with the notation for indicating bases in the part on matrix representations of linear functions (TODO ref).

\begin{proposition}
Let $\cat{C}$ be a locally small category and $f: X\to Y$ a morphism. Then $f_*$ and $f^*$ are dual as follows:
\begin{align*}
(f^\text{o})^* &= of_*o = (f^*)^o; \\
(f^\text{o})_* &= of^*o = (f^*)^o.
\end{align*}
\end{proposition}
\begin{proof}
We start with the locally small category $\cat{C^{op}}$, the morphism $f^\text{o}: Y\to X$ and an object $c$. Then 
\[ (f^\text{o})^*: \cat{C^{op}}(X,c) \to \cat{C^{op}}(Y,c): g^\text{o} \mapsto g^\text{o}f^\text{o}. \]
We can rewrite this as
\[ (f^\text{o})^*: o[\cat{C}(c,X)] \to o[\cat{C}(c,Y)]: g^o \mapsto (fg)^o \]
or
\[ o|_{o[\cat{C}(Y,c)]}(f^\text{o})^*o|_{\cat{C}(c, X)}: \cat{C}(c,X) \to \cat{C}(c,Y): g \mapsto fg. \]
This last function is exactly $f_*$. The other equality follows by duality: replacing $\cat{C}$ with $\cat{C^{op}}$ and $f: X\to Y$ with $f^o: Y\to X$ gives
\[ (f^o)_* = o|_{o[\cat{C^{op}}(X,c)]}(f^\text{oo})^*o|_{\cat{C^{op}}(c, Y)} = o|_{\cat{C}(c,X)}f^*o|_{o[\cat{C}(Y,c)]} \]
so
\[ f^* =  o|_{o[\cat{C}(c,X)]}(f^o)_* o|_{\cat{C}(Y,c)}. \]
\end{proof}
\begin{corollary} \label{dualityRepresentedFunctors}
The co- and contravariant represented functors are dual in the following sense:
\begin{align*}
o\cdot \cat{C^{op}}(c,-) \cdot o &= \cat{C}(-,c) \\
o\cdot \cat{C^{op}}(-,c) \cdot o &= \cat{C}(c,-) \\
\end{align*}
\end{corollary}
Note that $o$ still works as a function on sets of morphisms in $\cat{C}$, even though they are now said to be in the category $\cat{Set}$. We also use that $o$ acts on transformations as $o(\alpha) = o\alpha o$ (TODO!).
\begin{proof}
We verify the equality for all objects and morphisms:
\[ (o\cdot \cat{C^{op}}(c,-) \cdot o)(x) = (o\cdot \cat{C^{op}}(c,-))(x) = o(\cat{C^{op}}(c,x)) = o(o(\cat{C}(x,c))) = \cat{C}(x,c) \]
and
\[ (o\cdot \cat{C^{op}}(c,-) \cdot o)(f) = (o\cdot \cat{C^{op}}(c,-))(f^o) = o((f^o)_*) =o(of^*o) = oof^*oo = f^*. \]
\end{proof}

\begin{proposition} \label{monicEpicInPrePostComposition}
Let $\cat{C}$ be a locally small category and $f: X\to Y$ a morphism. Then
\begin{enumerate}
\item $f: X\to Y$ is monic \textup{if and only if} for all $c$ in $\cat{C}$, $f_*: \cat{C}(c,X) \to \cat{C}(c,Y)$ is injective;
\end{enumerate}
and, dually,
\begin{enumerate}
\setcounter{enumi}{1}
\item $f:X\to Y$ is epic \textup{if and only if} for all $c$ in $\cat{C}$, $f^*: \cat{C}(Y,c) \to \cat{C}(X,c)$ is injective.
\end{enumerate}
\end{proposition}
\begin{proof}
Assume $f$ monic. To show injectivity, assume $f_*(g) = f_*(h)$, which means $fg = fh$. By monicity $g = h$, showing $f^*$ is injective.

The converse is equally direct, as is the second statement.

The second statement also follows by duality, recognising that because $o$ is bijective when restricted to the relevant sets, $of^*o$ is injective if and only if $f^*$ is, by \ref{injectiveMonoSurjectiveEpi} and \ref{monicEpicCompositions}.
\end{proof}

\subsection{Representable functors}
\begin{definition}
Let $\cat{C}$ be a locally small category.

A covariant / contravariant functor $F: \cat{C} \to \cat{Set}$ is \udef{representable} if there is an object $c\in\cat{C}$ such that $F$ is naturally isomorphic to the covariant / contravariant functor represented by $c$, i.e.\
\[ F \cong \begin{cases}
\cat{C}(c,-) & \text{$F$ covariant} \\
\cat{C}(-,c) & \text{$F$ contravariant.}
\end{cases} \]
We say $F$ is \udef{represented} by the object $c\in\cat{C}$.

A \udef{representation} of a functor $F: \cat{C} \to \cat{Set}$ is given by an object $c\in\cat{C}$ and a natural isomorphism $\alpha$ from $F$ to the functor represented by $c$.
\end{definition}
A functor is represented by at most one object, up to isomorphism, by \ref{representablyIsomorphic}.

By the Yoneda lemma, every natural isomorphism $\alpha$ corresponds to en element $\alpha_c(I_c) \in Fc$.
\begin{definition}
A \udef{univeral property} of an object $c\in \cat{C}$ is expressed by a representable functor $F$ together with a \udef{universal element} $x\in Fc$ that corresponds to a natural isomorphism between $F$ and the functor represented by $c$.
\end{definition}
\section{The category of elements}
\chapter{Limits and colimits}
\chapter{Adjunctions}
\chapter{Monads}

\chapter{Abelian categories}
Five lemma. Short exact sequences. Split exact. Exact functors.


\url{https://mathoverflow.net/questions/363720/short-exact-sequences-every-mathematician-should-know}