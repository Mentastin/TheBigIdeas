\section{Functions on locally compact groups}

\begin{proposition}
Let $G$ be a locally compact group. Then any function $f\in \cont_c(G)$ is uniformly continuous.
\end{proposition}

\begin{proposition} \label{integralCompactSupportContinuous}
Let $G$ be a locally compact group, $\mu$ a Borel measure on $G$ and $f\in \cont_c(G)$.
Then the function $g\mapsto \int_G f(xg)\diff{\mu(x)}$ is continuous on $G$.
\end{proposition}

\section{Haar integration}

\begin{lemma}
Let $G$ be a topological group and $\mathcal{B}$ the Borel $\sigma$-algebra of $G$. Then $x\cdot A$ and $A\cdot x$ are measurable for all $A\in \mathcal{B}$ and $x\in G$.
\end{lemma}

\begin{definition}
Let $G$ be a group. A measure $\mu$ on $G$ such that $\mu(x\cdot A) = \mu(A)$ for all measureable sets $A$ and $x\in G$,
is called \udef{left-invariant}.
\end{definition}

\begin{theorem}[Haar measure]
Let $G$ be a locally compact group. There exists a non-zero left-invariant Radon measure $\mu$ $G$.
It is uniquely determined up to scalar multiples.
\end{theorem}
\begin{proof}
TODO
\end{proof}

\begin{definition}
This measure is called the \udef{Haar measure} of $G$. Let $\mu$ be the Haar measure of $G$. We abbreviate $L^p(G,\mu)$ by $L^p(G)$.
\end{definition}

\begin{lemma}
Let $\mu$ be the Haar measure on a locally compact group $G$. For any non-empty open $U\subseteq G$ and compact $K\subseteq G$ there exists an $n\in \N$ such that $\mu(K)\leq n\mu(U)$.
\end{lemma}
\begin{proof}
Because $\bigcup_{x\in G} x\cdot U = G$, the family $\setbuilder{x\cdot U}{x\in G}$ is an open cover of $G$. For any compact $K\subseteq G$, $\setbuilder{x\cdot U \cap K}{x\in G}$ is an open cover of $K$, which has a finite subcover $\setbuilder{x_i\cdot U \cap K}{x_1,\ldots, x_n \in G}$ by \ref{topologyCompactnessOpenCover}. Then
\[ \mu(K) = \mu\left(\bigcup_{i=1}^n x_i\cdot U \cap K\right) \leq \sum_{i=1}^n \mu(x_i\cdot U \cap K) \leq \sum_{i=1}^n \mu(x_i\cdot U) = \sum_{i=1}^n \mu(U) = n\mu(U). \]
\end{proof}
\begin{corollary} \label{HaarConsequences}
Let $\mu$ be the Haar measure on a locally compact group $G$.
\begin{enumerate}
\item Let $U\subseteq G$ be an open set, then $\mu(U) =0 \iff U = \emptyset$.
\item Let $f: G\to \R^{\geq 0}$ be a positive continuous function, then $\int_G f \diff{\mu} = 0 \iff f= 0$.
\item Every compact set has finite measure.
\end{enumerate}
\end{corollary}
\begin{proof}
(1) If $U = \emptyset$, then $\mu(U) = 0$ by definition. Assume, towards a contradiction, $\mu(U) = 0$ but $U \neq \emptyset$. Then for any compact $K\subseteq G$ we have $\mu(K) \leq n\mu(U) = 0$ by the lemma.

By weak inner regularity all open sets are null sets. By outer regularity all sets are null sets and thus $\mu$ is the zero measure, which is not the Haar measure.

(2) By \ref{functionPropertiesFromIntegral} the open set $f^\preimf\big(\,]0, \infty[\,\big)$ has measure zero and so must be empty by (1).

(3) By local finiteness, there exists an open set $U$ of finite measure. Take $K$ a compact set. Then $\mu(K) \leq n\mu(U)$, which is finite.
\end{proof}

\begin{proposition}
Let G be a locally compact group. The following are equivalent:
\begin{enumerate}
\item there exist $x\in G$ such that $\{x\}$ has non-zero Haar measure;
\item the counting measure is a Haar measure;
\item $G$ is a discrete group.
\end{enumerate}
\end{proposition}
\begin{proof}
$(1) \Rightarrow (2)$ Let $\mu$ be a Haar measure and let $\{x\}$ have non-zero Haar measure. Then $\mu(\{x\})^{-1}\mu$ is also a Haar measure. For all $y\in G$ we have
\[ mu(\{x\})^{-1}\mu(\{y\}) = mu(\{x\})^{-1}\mu(xy^{-1}\cdot\{y\}) = mu(\{x\})^{-1}\mu(\{x\}) = 1, \]
so $\mu(\{x\})^{-1}\mu$ is the counting measure by \ref{countingMeasureCriterion}.

$(2) \Rightarrow (3)$ Every point (in particular $1$) has a compact neighbourhood. By \ref{HaarConsequences} this compact neighbourhood has finite measure and thus is a finite set. The neighbourhood contains an open set of finite measure. TODO use Hausdorff!

$(3) \Rightarrow (1)$ If $G$ is discrete, then $\{x\}$ is open and has non-vanishing Haar measure by \ref{HaarConsequences}.
\end{proof}

\begin{proposition}
Let $G$ be a locally compact group. Then $G$ has finite Haar measure \textup{if and only if} $G$ is compact.
\end{proposition}
\begin{proof}
If $G$ is compact, then it has finite Haar measure by \ref{HaarConsequences}.

Now assume $G$ has finite Haar measure. Let $U$ be a compact neighbourhood of $1$. By \ref{HaarConsequences}, $\mu(U) \neq 0$. Consider the set
\[ \setbuilder{x_1\cdot U\cup \ldots \cup x_n\cdot U}{n\in \N, \forall i < j \leq n: x_i\cdot U \perp x_j\cdot U} \]
as a subset of the poset $\sSet{\powerset(G), \subseteq}$. Every chain in this set is finite as $n = \mu(U)^{-1}\mu(x_1\cdot U\cup \ldots \cup x_n\cdot U) \leq \mu(G)$. Take a maximal chain with maximum $K = y_1\cdot U\cup \ldots \cup y_n\cdot U$. Now $K$ is compact (TODO ref) and for all $z\in G$ we have $K\mesh z\cdot K$, because otherwise the chain would not have been maximal. Thus $G = K\cdot K^{-1}$, which is a compact set by (TODO ref).
\end{proof}

\subsection{The modular function}
\begin{definition}
Let $G$ be a locally compact group and $\mu$ a Haar measure on $G$. Then $\mu_x$ defined by
\[ \mu_x(A) \defeq \mu(A \cdot x) \]
is also a Haar measure. There exists a unique function $\Delta: G \to \R^{> 0}$ such that
\[ \mu_x = \Delta(x)\mu. \]
This function is called the \udef{modular function} of $G$.

If $\Delta = \underline{1}$, then $G$ is called \udef{unimodular}.
\end{definition}
The modular function is independent of the choice of Haar measure. Clearly every commutative group is unimodular.

\begin{lemma} \label{rightShiftHaarIntegral}
Let $G$ be a locally compact group, $f\in L^1(G)$ and $y\in G$. Then
\begin{enumerate}
\item $\int_G f(yx)\diff{x} = \int_G f(x)\diff{x}$;
\item $\int_G f(xy)\diff{x} = \Delta(y^{-1})\int_G f(x)\diff{x}$.
\end{enumerate}
\end{lemma}
\begin{proof}
TODO 
\end{proof}

\begin{proposition}
Let $G$ be a locally compact group. The modular function $\Delta: G \to \sSet{\R^{> 0},\cdot}$ is a continuous group homomorphism.
\end{proposition}
\begin{proof}
Let $\mu$ be a Haar measure.

To show $\Delta$ is a group homomorphism, take $x,y\in G$. Let $A$ be a measurable set with finite and non-zero measure. Then
\[ \Delta(xy)\mu(A) = \mu(A \cdot xy) = \mu\big((A\cdot x)\cdot y\big) = \Delta(y)\mu(A\cdot x) = \Delta(y)\Delta(x)\mu(A). \]

For continuity, take some $f\in\cont_c(G)$ with non-zero integral (TODO show existence). Set $c = \int_G f\diff{\mu} \neq 0$. Then by \ref{rightShiftHaarIntegral} we have
\[ \Delta(y) = c^{-1}\int_G f(xy^{-1})\diff{x} = \int_G f\circ \rho_{y^{-1}}(x)\diff{x}, \]
which is continuous by \ref{integralCompactSupportContinuous} because $f\circ \rho_{y^{-1}} \in \cont_c(G)$ (TODO ref).
\end{proof}
\begin{corollary}
Every compact group is unimodular.
\end{corollary}
\begin{proof}
The image $f^{\imf}[G]$ is a compact subgroup of $\sSet{\R^{> 0},\cdot}$ by \ref{compactConstructions}. The only compact subgroup of $\sSet{\R^{> 0},\cdot}$ is $\{1\}$ (TODO ref).
\end{proof}

\subsection{The character group}
\begin{definition}
Let $A$ be a locally compact abelian group. A \udef{character} of $A$ is a continuous group homomorphism $\chi: A\to \T$.

The set of all characters forms a group under pointwise multiplication, called the \udef{character group}. It is denoted $\widehat{A}$.
\end{definition}

\subsection{Convolution}
\begin{definition}
Let $G$ be a locally compact group that is a countable union of compact subsets and $f,g\in L^1(G)$. The \udef{convolution} $f*g$ is defined by
\begin{align*}
(f*g)(x) &= \int_Gf(y)g(y^{-1}x)\diff{\mu(y)} \\
&= \int_Gf(xz)g(z^{-1})\diff{\mu(z)}.
\end{align*}
\end{definition}
TODO: show that the above equality is an equality (using translation invariance of Haar measure).

\begin{lemma}
Let $G$ be a locally compact group that is a countable union of compact subsets and $f,g\in L^1(G)$.
The convolution $f*g$ is defined a.e., is a function in $L^1(G)$ and satisfies
\[ \norm{f*g}_1 \leq \norm{f}_1\norm{g}_1. \]
\end{lemma}
\begin{proof}
We calculate,
\begin{align*}
\int_G\int_G |f(y)g(y^{-1}x)| \diff{\mu(y)}\diff{\mu(x)} &= \int_G\int_G |f(y)g(y^{-1}x)| \diff{\mu(x)}\diff{\mu(y)} \\
&= \int_G |f(y)| \int_G |g(y^{-1}x)| \diff{\mu(x)}\diff{\mu(y)} \\
&= \int_G |f(y)| \int_G |g(x)| \diff{\mu(x)}\diff{\mu(y)} \\
&= \int_G |f(y)| \;\norm{g}_1\diff{\mu(y)} \\
&= \norm{f}_1\norm{g}_1 <\infty,
\end{align*}
where we have used Tonelli's theorem \ref{tonellisTheorem} and left-translation-invariance. By \ref{functionPropertiesFromIntegral}, we have that $\int_G |f(y)g(y^{-1}x)| \diff{\mu(y)}<\infty$ a.e.\ and thus $(f*g)(x) = \int_G f(y)g(y^{-1}x) \diff{\mu(y)}$ is defined a.e.

Using (TODO ref inequality int abs value), we have
\begin{align*}
\norm{f*g}_1 = \int_G|f*g|\diff{\mu} &= \int_G\left|\int_G f(y)g(y^{-1}x) \diff{\mu(y)}\right|\diff{\mu(x)} \\
&\leq \int_G\int_G |f(y)g(y^{-1}x)| \diff{\mu(y)}\diff{\mu(x)} \\
&= \norm{f}_1\norm{g}_1 <\infty.
\end{align*}
This shows $f*g\in L^1(G)$ and the inequality.
\end{proof}

\begin{proposition}
Let $G$ be a locally compact group that is a countable union of compact subsets. Then $L^1(G)$ is a Banach algebra under the convolution product.
\end{proposition}
\begin{proof}
We need to show associativity and distributivity. TODO.
\end{proof}


\subsection{The Fourier transform}
\begin{definition}
Let $A$ be a locally compact abelian group and $f\in L^1(A)$. The \udef{Fourier transform} of $f$ is the function
\[ \hat{f}: \widehat{A} \to \C: \chi \mapsto \hat{f}(\chi) = \int_A f(x)\overline{\chi(x)}\diff{x}. \]
\end{definition}
