\chapter{Functions on measure spaces}

\subsubsection{Operators on sequence spaces}
TODO Gribanov's theorems

3.7.1, 3.7.2 of Hanson / Yakovlev.

\subsubsection{Series in Banach spaces}
TODO
\url{https://link.springer.com/content/pdf/10.1007%2F978-0-8176-4687-5_3.pdf}
\begin{definition}
Let $\seq{x_n}$ be a sequence in a Banach space $X$. As for series of scalars, we say a series $\sum_{n=1}^\infty x_n$ is
\begin{itemize}
\item \udef{unconditionally convergent} if $\sum_{n=1}^\infty x_{\sigma(n)}$ converges for every permutation $\sigma$ of $\N$;
\item \udef{absolutely convergent} if $\sum_{n=1}^\infty \norm{x_n} < \infty$.
\end{itemize}
\end{definition}

\begin{proposition} \label{absoluteUnconditionalConvergenceBanach}
Let $\seq{x_n}$ be a sequence in a Banach space $X$. If $\sum_{n=1}^\infty$ converges absolutely, then it converges unconditionally.
\end{proposition}
\begin{proof}
Assume absolute convergence, so $\sum\norm{x_i}<\infty$. Then (for $m< n$)
\[ \norm{\sum_{i=1}^n x_i - \sum_{i=1}^m x_i} = \norm{\sum_{i=m+1}^n x_i} \leq \sum_{i=m+1}^n\norm{x_i} = \sum_{i=1}^n \norm{x_i} - \sum_{i=1}^m \norm{x_i}, \]
and because $\sum\norm{x_i}$ converges, it is a Cauchy sequence and by the inequality so is $\sum x_i$. By completeness this sequence is convergent.

By (TODO ref) $\sum\norm{x_{\sigma(i)}}$ converges for any permutation $\sigma$ of $\N$. We can then repeat the argument to show $\sum x_{\sigma(i)}$ is also convergent and thus unconditionally convergent.
\end{proof}





\section{Bochner integration}

TODO: the Bochner integral is the unique extension of the integral of simple functions to the set of Bochner measurable functions???? (i.e.\ simple functions dense in Bochner space, with $L^1$ metric)
\begin{definition}
Let $(\Omega, \mathcal{A},\mu)$ be a measure space and $Y$ a normed vector space. Then a Bochner measurable function $f:\Omega\to Y$ is called \udef{Bochner integrable} if there exists a sequence of integrable simple functions $\seq{s_n}\subset\SF(\Omega,Y)$ such that
\[ \lim_{n\to\infty}\int_\Omega \norm{f-s_n}\diff{\mu} = 0. \]
Take such a sequence $\seq{s_n}$. The \udef{Bochner integral} of $f$ on $\Omega$ w.r.t. $\mu$ is defined as
\[ \int_\Omega f\diff{\mu} \defeq \lim_{n\to\infty}\int_\Omega s_n\diff{\mu}. \]
\end{definition}

\begin{lemma}
The Bochner integral is well-defined: let $\seq{s_n},\seq{t_n}\in \prescript{\N}{}{\SF(\Omega,Y)}$ be sequences such that
\[ \lim_{n\to\infty}\int_\Omega \norm{f-s_n}\diff{\mu} = 0 = \lim_{n\to\infty}\int_\Omega \norm{f-t_n}\diff{\mu}.  \]
Then
\begin{enumerate}
\item the limits $\lim_{n\to\infty}\int_\Omega s_n\diff{\mu}$ and $\lim_{n\to\infty}\int_\Omega t_n\diff{\mu}$ exist;
\item $\lim_{n\to\infty}\int_\Omega s_n\diff{\mu} = \lim_{n\to\infty}\int_\Omega t_n\diff{\mu}$.
\end{enumerate}
\end{lemma}
\begin{proof}
TODO
\end{proof}

\begin{proposition}[Bochner integrability criterion] \label{BochnerIntegrabilityCondition}
Let $(\Omega, \mathcal{A},\mu)$ be a measure space and $Y$ a normed vector space.

A Bochner measurable function $f$ is Bochner integrable \textup{if and only if}
\[ \int_\Omega \norm{f} \diff{\mu} < \infty. \]
\end{proposition}

\begin{proposition}
Linearity and monotonicity.
\end{proposition}

\begin{theorem}[Hille's theorem] \label{HilleTheorem}
Let $(\Omega, \mathcal{A},\mu)$ be a measure space, $X,Y$ normed vector spaces and $T: X\not\to Y$ a closed operator. If $T\circ f$ is integrable, then
\[ \int_\Omega (T\circ f)\diff{\mu} = T\left(\int_\Omega f\diff{\mu}\right). \]
\end{theorem}
\begin{proof}
TODO
\end{proof}
\begin{corollary} \label{boundedOperatorUnderIntegral}
If $T$ is bounded, then $T\circ f$ is integrable and
\[ \int_\Omega (T\circ f)\diff{\mu} = T\left(\int_\Omega f\diff{\mu}\right). \]
\end{corollary}
\begin{proof}
TODO: show that $T\circ f$ is integrable!
\end{proof}

TODO Dominated convergence.

\subsection{Integration of bounded operators}
\begin{lemma} \label{integralBoundedOperator}
Let $X$ be a normed space and $(\Omega, \mathcal{A},\mu)$ a measure space. Let $T: \Omega \to \Bounded(X)$ be a function. If $T$ is integrable, then for all $x\in X$, $Tx$ is integrable and
\[ \left(\int_\Omega T\diff{\mu}\right)x = \int_\Omega Tx\diff{\mu}. \]
\end{lemma}
\begin{proof}
The evaluation map $\evalMap_x$ is linear and bounded by $\norm{x}$ for all $x\in X$, so we can use \ref{boundedOperatorUnderIntegral}.
\end{proof}

\chapter{Orthogonal polynomials and random matrices}
TODO: not harmonic analysis, but just parked

\section{Orthogonal polynomials}
\begin{definition}

\end{definition}