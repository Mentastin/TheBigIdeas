TODO analysis in vector notation. See Stat inf AQFT

\section{Fields}
\subsection{What is a field?}
People mean different things when they say field. We have already encountered the algebraic structure (e.g\ the fields $\R$ and $\mathbb{Q}$). In physics the term field usually means we associate a value or an object to each point in space. The vector field along a curve that we have already seen can be seen as a field in one dimensional space. In this section we will restrict our attention to Euclidean space. Details for other geometries will follow. In modern high energy physics the term field often refers specifically to fields of operators.

\begin{definition}
A field $F$ associates an element of a set $X$ to each point in space.
\[ F: \mathbb{E}^n \to X \]
Where $n$ is typically 2 or 3. Depending on $X$ we call the field differently:
\begin{itemize}
\item For $X \subset \R$ we call $F$ a \udef{scalar field}.
\item For $X$ a three-dimensional vector field, we call $F$ a \udef{vector field}.
\end{itemize}
\end{definition}

Examples of scalar fields include temperature in space and pressure distribution in a fluid. The flow of a fluid can be modeled using a vector field.

\section{Differential calculus}
There are two important cases: a field may be two or three dimensional. Because two dimensional fields can be seen as a special case of three dimensional ones, we will assume $n=3$ for the rest of this section. This means that the field is essentially a function with three variables:
\[ F(x,y,z) \]
This means there are three differential operators $\pd{}{x}, \pd{}{y}, \pd{}{z}$. We also fix and orthonormal basis for $\mathbb{E}^3$ with basisvectors
\[\vhat{x},\; \vhat{y},\; \vhat{z}\]
in that order.

\subsection{Nabla, the vector differential operator}
The main operations we can do in scalar and vector fields (taking the gradient, divergence and curl) can be expressed in terms of the \udef{nabla} operator $\vnabla$ (also known as the \udef{del} operator), which can be seen as a \textit{vector} operator with three components:
\[ \vnabla = \pd{}{x}\vhat{x} + \pd{}{y}\vhat{y} + \pd{}{z}\vhat{z} \]
Formally this means it is an operator that when acting on a number produces a vector. 

More intuitively it means that is also makes sense to use it in conjunction with the dot and cross products. It must however be remembered that $\vnabla$ is not really a vector and we cannot just use vector identities with it (even if they do sometimes turn out to be correct). When in doubt, write out the components.

\subsubsection{Gradient}
Say $F$ is a scalar field. A first, obvious, question for calculus to solve is: how fast does $F$ vary?

TODO after Taylor expansion. TODO directional derivative + intuition

\[ \diff{F} = (\grad F)\cdot (\diff{\vec{l}}) \]
where
\[ \grad F = \left(\pd{}{x}\vhat{x} + \pd{}{y}\vhat{y} + \pd{}{z}\vhat{z}\right)F = \pd{F}{x}\vhat{x} + \pd{F}{y}\vhat{y} + \pd{F}{z}\vhat{z} \]

Now
\[ \diff{F} = \grad F \cdot \diff{\vec{l}} = |\grad F||\diff{\vec{l}}|\cos\theta. \]
From this it is clear that $\diff{F}$ is largest if $\theta = 0$ and smallest (i.e.\ zero) if $\theta = \pi$. Fixing $\theta = 0$ and viewing $F$ as a one dimensional function (with variable $l = |\vec{l}|$) along this line, we see
\[ \od{F}{l} = |\grad F|. \]
This leads us to a geometrical interpretation of the gradient:
\begin{itemize}
\item The gradient $\grad F$ points in the direction of maximum increase of the function $F$.
\item Locally the field does not vary perpendicular to $\grad F$.
\item The magnitude $|\grad F|$ gives the slope (rate of increase) along this maximal direction.
\end{itemize}

We call a point $(x,y,z)$ a \udef{stationary point} if $\grad F = 0$ at $(x,y,z)$. As for single variable calculus, local maxima and minima are stationary points.

\subsubsection{Divergence}
Assume we have a vector field
\[ \vec{v}: \mathbb{E}^3 \to \mathbb{R}^3: (x,y,z) \mapsto \vec{v}(x,y,z) = v_x \vhat{x} + v_y \vhat{y} + v_z \vhat{z} \]
we can then define the \udef{divergence} as
\begin{align*}
\nabla\cdot \vec{v} &= \left(\pd{}{x}\vhat{x} + \pd{}{y}\vhat{y} + \pd{}{z}\vhat{z}\right)\cdot (v_x \vhat{x} + v_y \vhat{y} + v_z \vhat{z}) \\
&= \pd{v_x}{x} + \pd{v_y}{y} + \pd{v_z}{z}
\end{align*}
The divergence of a vector function is a \textit{scalar}.

Intuitively the divergence can be thought of as the amount the vector field spreads out (diverges) from the point in question. If we think of the vector field as modeling the flow of a fluid, then a point of positive divergence is a source and a point of negative divergence is a drain.

TODO figure (like 18 in electro)

\subsubsection{Curl}
For a vector field $\vec{v}$, the \udef{curl} can be defined as follows:
\begin{align*}
\curl \vec{v} &= \begin{vmatrix}
\vhat{x} & \vhat{y} & \vhat{z} \\
\pd{}{x} & \pd{}{y} & \pd{}{z} \\
v_x & v_y & v_z
\end{vmatrix} \\
&= \left(\pd{v_z}{y} - \pd{v_y}{z}\right)\vhat{x} + \left(\pd{v_x}{z} - \pd{v_z}{x}\right)\vhat{y} + \left(\pd{v_y}{x} - \pd{v_x}{y}\right)\vhat{z}.
\end{align*}
The curl of a vector function is a \textit{vector}.

Intuitively the curl is a measure of how much the vector swirls around the point in question. Again viewing the vector field as the flow of some liquid, the curl indicates how much a paddle wheel fixed at that point would rotate (TODO fig, like 19 in electro + paddle wheel).

\subsection{Properties of vector derivatives}
\begin{note}
We now assume
\begin{itemize}
\item $k\in \R$ is a constant.
\item $f$ and $g$ are scalar fields.
\item $\vec{A}$ and $\vec{B}$ are vector fields.
\end{itemize}
\end{note}
All the vector derivatives are linear:
\[ \begin{cases}
\grad(kf+g) = k\grad f + \grad g \\
\nabla\cdot(k\vec{A}+ \vec{B}) = k\nabla\cdot\vec{A} + \nabla\cdot \vec{B} \\
\curl(k \vec{A}+ \vec{B}) = k\curl \vec{A} + \curl \vec{B}
\end{cases} \]
\subsubsection{Product rules}
There are several relevant products to consider: scalar times scalar ($fg$), scalar times vector ($f \vec{A}$), dot product ($\vec{A}\cdot \vec{B}$) and cross product ($\vec{A}\times \vec{B}$). Accordingly, there are six product rules. Each can easily be verified by writing out the components and using the standard product rule from single variable calculus.
\begin{itemize}
\item For gradients
\begin{itemize}
\item[(a)] $\grad(fg) = f\grad g + g\grad f$
\item[(b)] $\grad(\vec{A}\cdot \vec{B}) = \vec{A}\times(\curl \vec{B}) + \vec{B}\times(\curl\vec{A}) + (\vec{A}\cdot\vnabla)\vec{B} +(\vec{B}\cdot\vnabla)\vec{A}$
\end{itemize}
\item For divergences
\begin{itemize}
\item[(c)] $\nabla\cdot(f\vec{A}) = f(\nabla\cdot\vec{A}) + \vec{A}\cdot(\grad f)$
\item[(d)] $\nabla\cdot(\vec{A}\times \vec{B}) = \vec{B}\cdot(\curl \vec{A}) - \vec{A}\cdot(\curl\vec{B})$
\end{itemize}
\item For curls
\begin{itemize}
\item[(e)] $\curl(f\vec{A}) = f(\curl \vec{A}) - \vec{A}\times (\grad f)$
\item[(f)] $\curl(\vec{A}\times \vec{B}) = (\vec{B}\cdot\vnabla)\vec{A} - (\vec{A}\cdot\vnabla)\vec{B} + \vec{A}(\nabla\cdot \vec{B}) - \vec{B}(\nabla\cdot \vec{A})$
\end{itemize}
\end{itemize}
One strange feature of these product rules is the occurrence of terms of the form $(\vec{A}\cdot\vnabla)\vec{B}$. This is clearer (at least to my mind) when written out in components (in three dimensions)
\[ (\vec{A}\cdot\vnabla)\vec{B} = \left(A_x  \pd{B_x}{x}\right)\vhat{x} + \left(A_y  \pd{B_y}{y}\right)\vhat{y} + \left(A_z  \pd{B_z}{z}\right)\vhat{z}. \]

\subsubsection{Quotient rules}
These can be easily obtained from the product rules.
\begin{align*}
\grad \left(\frac{f}{g}\right) &= \frac{g\grad f - f\grad g}{g^2} \\
\nabla\cdot \left(\frac{\vec{A}}{g}\right) &= \frac{g(\nabla\cdot \vec{A}) - \vec{A}\cdot (\grad g)}{g^2} \\
\curl \left(\frac{\vec{A}}{g}\right) &= \frac{g(\curl\vec{A}) + \vec{A}\times(\grad g)}{g^2}
\end{align*}

\subsection{Second derivatives}
\subsubsection{The Laplacian}
We introduce a second order operator, the \udef{Laplacian} $\nabla^2$:
\begin{align*}
\nabla^2 &= \vnabla \cdot \vnabla \\
&= \left(\pd{}{x}\vhat{x} + \pd{}{y}\vhat{y} + \pd{}{z}\vhat{z}\right)\cdot \left(\pd{}{x}\vhat{x} + \pd{}{y}\vhat{y} + \pd{}{z}\vhat{z}\right) \\
&= \pd[2]{}{x}\vhat{x} + \pd[2]{}{y}\vhat{y} + \pd[2]{}{z}\vhat{z}
\end{align*}
This can be applied to a scalar field:
\[ \nabla^2 F \equiv \pd[2]{F}{x}\vhat{x} + \pd[2]{F}{y}\vhat{y} + \pd[2]{F}{z}\vhat{z} \]
Or to a vector field by applying the Laplacian to each component individually:
\[ \nabla^2 \vec{v} \equiv (\nabla^2 v_x)\vhat{x} + (\nabla^2 v_y)\vhat{y} + (\nabla^2 v_z)\vhat{z} \]
\subsubsection{Constructing second derivatives from first order derivatives}
There are five ways we can make second derivative operators by mixing gradient, divergence and curl:
\begin{enumerate}
\item Divergence of gradient:
\begin{align*}
\nabla\cdot (\grad F) &= \left(\pd{}{x}\vhat{x} + \pd{}{y}\vhat{y} + \pd{}{z}\vhat{z}\right)\cdot \left(\pd{F}{x}\vhat{x} + \pd{F}{y}\vhat{y} + \pd{F}{z}\vhat{z}\right) \\
&= \pd[2]{F}{x}\vhat{x} + \pd[2]{F}{y}\vhat{y} + \pd[2]{F}{z}\vhat{z} = \nabla^2 F
\end{align*}
So this is just the Laplacian.
\item The curl of a gradient:
\begin{align*}
\curl(\grad F) &= \left(\pd{}{y}\pd{F}{z} - \pd{}{z}\pd{F}{y}\right)\vhat{x} + \left(\pd{}{z}\pd{F}{x} - \pd{}{x}\pd{F}{z}\right)\vhat{y} + \left(\pd{}{x}\pd{F}{y} - \pd{}{y}\pd{F}{x}\right)\vhat{z} \\
&= 0
\end{align*}
This is an important fact that hinges on the fact that cross derivatives commute.
\item The gradient of the divergence $\grad(\nabla\cdot \vec{v})$ is \textit{not} the same as the Laplacian of a vector. It does not have a special name.
\item The divergence of a curl:
\begin{align*}
\nabla\cdot(\curl \vec{v}) &= \pd{}{x}\left(\pd{v_z}{y} - \pd{v_y}{z}\right) + \pd{}{y}\left(\pd{v_x}{z} - \pd{v_z}{x}\right) + \pd{}{z}\left(\pd{v_y}{x} - \pd{v_x}{y}\right) \\
&= \pd{}{y}\pd{v_x}{z} - \pd{}{z}\pd{v_x}{y} + \pd{}{x}\pd{v_z}{y} - \pd{}{z}\pd{v_y}{x} + \pd{}{x}\pd{v_z}{y} - \pd{}{z}\pd{v_z}{x}
&= 0
\end{align*}
Again this hinges on the fact the cross derivatives commute.
\item The curl of a curl gives nothing new:
\[ \curl(\curl \vec{v}) = \grad(\nabla\cdot \vec{v}) - \nabla^2 \vec{v} \]
\end{enumerate}
We repeat two important facts for future reference:
\begin{eigenschap}
\begin{itemize}
\item The curl of a gradient is always \textbf{zero}:
\[ \curl(\grad F) = 0 \]
\item The divergence of a curl is always \textbf{zero}:
\[ \nabla\cdot(\curl \vec{v}) = 0 \]
\end{itemize}
\end{eigenschap}

\subsection{With respect to which coordinates?}
Sometimes we will deal with maps that look like fields (they depend on $x,y$ and $z$ coordinate), but also depend on other variables, like time. We can still use all the results from this section. All derivatives are partial, so we just calculate as if the other variables were constant.

Sometimes we will deal with maps that depend on two or more sets of spatial coordinates. For example, the electric field in a point may depend on the locations of various charged particles. In this case we can still use the notation and result from this section, we just need to specify with respect to which set of coordinates we are applying the derivative.

For example, using the compact notation $\vec{r_1} = (x_1,y_1,z_1)$ and $\vec{r_2} = (x_2,y_2,z_2)$, we may have a quantity $T(\vec{r_1}, \vec{r_2}) = T(x_1,y_1,z_1,x_2,y_2,z_2)$.
We can now write $\vnabla_{\vec{r_1}}T$ to mean
\[ \vnabla_{\vec{r_1}}T =  \pd{T}{x_1}\vhat{x} + \pd{T}{y_1}\vhat{y} + \pd{T}{z_1}\vhat{z} \]
and $\vnabla_{\vec{r_2}}T$ to mean
\[ \vnabla_{\vec{r_2}}T =  \pd{T}{x_2}\vhat{x} + \pd{T}{y_2}\vhat{y} + \pd{T}{z_2}\vhat{z} \]

\subsection{Miscellaneous identities}
\[ \vec{a}\times (\curl \vec{a}) = \grad \left(\frac{a^2}{2}\right) - (\vec{a}\cdot \nabla)\vec{a} \]

\subsection{Tensor derivatives}

\section{Integral calculus}
\subsection{Line integrals}
Given a curve $\gamma$ we define the \udef{line integral} between $\gamma(a)$ and $\gamma(b)$ as

\begin{definition}
\begin{itemize}
\item The line integral of a scalar field $F$ is given by
\[ \int_a^b F[\gamma(t)]\;|\vec{\gamma'(t)}|\; \diff{t}  \]
\item The line integral of a vector field $\vec{v}$ is given by
\[ \int_a^b \vec{v}[\gamma(x)]\cdot \vec{\gamma'}(x) \diff{x}. \]
This is usually written in the following way:
\[ \int_{\vec{a}}^{\vec(b)} \vec{v}\cdot \diff{\vec{l}} \qquad \text{or} \qquad \int_\gamma \vec{v}\cdot \diff{\vec{l}} \]
where $\vec{a} = \gamma(a)$ and $\vec{b} = \gamma(b)$
\end{itemize}
\end{definition}

If $\gamma$ is a closed loop (so $\vec{a} = \vec{b}$) we write
\[ \oint \vec{v}\cdot \diff{\vec{l}} \]

In general the value of the line integral depends on the curve $\gamma$. There exist some vector fields such that line integrals only depend on the endpoints, not on the path. Such vector fields are called \udef{conservative}. For any conservative field $\vec{u}$:
\[ \oint \vec{u}\cdot \diff{\vec{l}} = 0. \]

\subsection{Surface integrals}
For a given surface $\mathcal{S}$ and vector field $\vec{v}$, we define the \udef{surface integral}
TODO
\[ \iint_{\mathcal{S}} \vec{v}\cdot \diff{\vec{a}} \]
\[ \oiint \vec{v}\cdot \diff{\vec{a}} \]

\subsection{Volume integrals}
TODO
\[ \iiint_{\mathcal{V}} F \diff{\tau} \]

\section{Fundamental theorems of vector calculus}
In this section we will state three very important results that are analoguous to the fundamental theorem of calculus:
\[ \int_a^b \left(\od{f}{x}\right)\diff{x} = f(b) - f(a) \]

That is we will be inversing the operations of gradient divergence and curl using integrals.

\subsection{Fundamental theorem for gradient}
The fundamental theorem for gradients states that for any scalar field $F$
\[ \boxed{ \int_{\vec{a}}^{\vec{b}}(\grad F)\cdot \diff{\vec{l}} = F(\vec{b}) - F(\vec{a})  } \]

An intuitive explanation can be given as follows: TODO figure.

This fundamental theorem is valid for any curve. Thus for any scalar field, the vector field $\grad F$ is \ueig{conservative}.

\subsection{Fundamental theorem for divergence}
The fundamental theorem for divergences is also known as \textbf{Gauss's theorem}, \textbf{Green's theorem} or simply the \textbf{divergence theorem}.
\[ \boxed{ \iiint_{\mathcal{V}}(\nabla\cdot \vec{v}\diff{\tau}) = \oiint_{\mathcal{S}} \vec{v}\cdot \diff{\vec{a}} } \]
Where $\mathcal{S}$ is the surface of the volume $\mathcal{V}$.

If we view the vector field as the flow of a fluid, the divergence theorem can be interpreted as
\[ \iiint(\text{sources within the volume}) = \oiint (\text{flow out through the surface}). \]
\subsection{Fundamental theorem for curl}
The fundamental theorem for curls is known as \textbf{Stokes' theorem}.
\[ \boxed{ \iint_{\mathcal{S}}(\curl \vec{v})\cdot \diff{\vec{a}} = \oint_{\mathcal{P} \vec{v}\cdot \diff{\vec{l}}} } \]
Where $\mathcal{P}$ is the perimeter of the path $\mathcal{S}$.

Two comments
\begin{itemize}
\item The expressions on both sides of the equals sign have a sign ambiguity: The surface integral changes sign if you orient the surface differently and the line integral changes sign if you change direction of traversal.
\item The surface integral does not depend on the exact surface chosen, only on its perimeter. This also means that
\[ \oiint(\curl \vec{v})\cdot \diff{\vec{a}} = 0. \]
\end{itemize}

\section{Integrating by parts}
Integration by parts in vector calculus is analogous to the one dimensional case:
Use the product rule, integrate both sides and invoke the fundamental theorem. For vector calculus we have more product rules to exploit. Assume $f$ is a scalar field and $\vec{A}$ and $\vec{B}$ are vector fields.
\[ \iiint_{\mathcal{V}}f(\nabla\cdot \vec{A}) \diff{\tau} = - \iiint_{\mathcal{V}}\vec{A}\cdot(\grad f) \diff{\tau} + \oiint_{\mathcal{S}}f\cdot (\vec{A}\cdot \diff{\vec{a}}) \]
\[ \iint_\mathcal{S} f(\curl \vec{A})\cdot \diff{\vec{a}} = \iint_\mathcal{S}[\vec{A}\times(\grad f)]\cdot \diff{\vec{a}} + \oint_\mathcal{P} f\cdot (\vec{A}\cdot \diff{\vec{l}}) \]
\[ \iiint_\mathcal{V} \vec{B}\cdot (\curl \vec{A})\diff{\tau} = \iiint_\mathcal{V} \vec{A}\cdot(\curl \vec{B})\diff{\tau} + \oiint_{\mathcal{S}}(\vec{A}\times \vec{B})\cdot \diff{\vec{a}} \]

\section{Other coordinate systems}
\subsection{General coordinate transformations}
TODO


The formula for $\nabla\cdot X$ is incorrect. The notation with the 'usual' dot product is misleading. Properly it is for a diagonal metric:
\[\nabla\cdot F = \frac 1\rho\frac{\partial(\rho F^i)}{\partial x^i}\]
where $\rho=\sqrt{\det g}$ is the coefficient of the differential volume element $dV=\rho\, dx^1\wedge\ldots \wedge dx^n$, meaning $\rho$ is also the Jacobian determinant, and where $F^i$ are the components of $F$ with respect to an unnormalized basis.

In polar coordinates we have $\rho=\sqrt{\det g}=r$, and:
\[\nabla\cdot X = \frac 1r \frac{\partial(r X^r)}{\partial r} 
+ \frac 1r\frac{\partial(r X^\theta)}{\partial \theta}\]

In the usual normalized coordinates $X=\vhat X^{r}\frac{\partial}{\partial r} + \vhat X^{\theta}\frac 1r\frac{\partial}{\partial\theta}$ this becomes:
\[\nabla\cdot X = \frac 1r \frac{\partial(r \vhat X^{r})}{\partial r} 
+ \frac 1r\frac{\partial \vhat X^{\theta}}{\partial \theta}\]
which agrees with the usual formula given in calculus books.

\subsection{Spherical coordinates}
\begin{itemize}
\item \textit{Gradient}
\[ \grad F = \pd{F}{r}\vhat{r} + \frac{1}{r} \pd{F}{\theta}\vhat{\theta} + \frac{1}{r\sin\theta}\pd{F}{\phi}\vhat{\phi} \]
\item \textit{Divergence}
\[ \nabla\cdot \vec{v} = \frac{1}{r^2}\pd{r^2v_r}{r} + \frac{1}{r\sin\theta}\pd{\sin\theta v_\theta}{\theta} + \frac{1}{r\sin\theta}\pd{v_\phi}{\phi} \]
\item \textit{Curl}
\[ \curl \vec{v} = \frac{1}{r\sin\theta}\left(\pd{\sin\theta v_\theta}{\theta} - \pd{v_\theta}{\phi}\right)\vhat{r} + \frac{1}{r}\left(\frac{1}{\sin\theta}\pd{v_r}{\phi} - \pd{rv_\phi}{r}\right)\vhat{\theta} + \frac{1}{r}\left(\pd{rv_\theta}{r} - \pd{v_r}{\theta}\right)\vhat{\phi}  \]
\item \textit{Laplacian}
\[ \nabla^2 F = \frac{1}{r^2} \pd{}{r}\left(r^2 \pd{F}{r}\right) + \frac{1}{r^2\sin\theta} \pd{}{\theta}\left(\sin\theta \pd{F}{\theta}\right) + \frac{1}{r^2\sin^2}\theta\pd[2]{F}{z} \]
\end{itemize}
\subsection{Cylindrical coordinates}
\begin{itemize}
\item \textit{Gradient}
\[ \grad F = \pd{F}{s}\vhat{s} + \frac{1}{s} \pd{F}{\phi}\vhat{\phi} + \pd{F}{z}\vhat{z} \]
\item \textit{Divergence}
\[ \nabla\cdot \vec{v} = \frac{1}{s}\pd{sv_s}{s} + \frac{1}{s}\pd{v_\phi}{\phi} + \pd{v_z}{z} \]
\item \textit{Curl}
\[ \curl \vec{v} = \left(\frac{1}{s}\pd{v_z}{\phi} - \pd{v_\phi}{z}\right)\vhat{s} + \left(\pd{v_s}{z} - \pd{v_z}{s}\right)\vhat{\phi} + \frac{1}{s}\left(\pd{sv_\phi}{s} - \pd{v_s}{\phi}\right)\vhat{z}  \]
\item \textit{Laplacian}
\[ \nabla^2 F = \frac{1}{s} \pd{}{s}\left(s \pd{F}{s}\right) + \frac{1}{s^2} \pd[2]{F}{\phi} + \pd[2]{F}{z} \]
\end{itemize}
\subsection{Polar coordinates}
This is just like for cylindrical coordinates, except nothing depends on the $z$ coordinate and vectors of a vector field do not have a component in the $z$ direction. Thus we may set $v_z$, and anything that is operated on by $\pd{}{z}$, to zero.
\begin{itemize}
\item \textit{Gradient}
\[ \grad F = \pd{F}{s}\vhat{s} + \frac{1}{s} \pd{F}{\phi}\vhat{\phi} \]
\item \textit{Divergence}
\[ \nabla\cdot \vec{v} = \frac{1}{s}\pd{sv_s}{s} + \frac{1}{s}\pd{v_\phi}{\phi} \]
\item \textit{Curl}
\[ \curl \vec{v} = \frac{1}{s}\left(\pd{sv_\phi}{s} - \pd{v_s}{\phi}\right)\vhat{z}  \]
\item \textit{Laplacian}
\[ \nabla^2 F = \frac{1}{s} \pd{}{s}\left(s \pd{F}{s}\right) + \frac{1}{s^2} \pd[2]{F}{\phi} \]
\end{itemize}

\section{Potentials}
\subsection{Irrotational fields}
\udef{Irrotational fields} are fields where the curl vanishes everywhere.
\begin{eigenschap}
For a vector field $\vec{F}$ the following conditions are equivalent:
\begin{enumerate}[(a)]
\item $\curl \vec{F} = 0 \;$ everywhere.
\item $\int_{\vec{a}}^{\vec{b}}\vec{F}\cdot \diff{\vec{l}}$ is independent of the path, for any given end points.
\item $\oint \vec{F}\cdot \diff{\vec{l}} = 0$ for any closed loop.
\item $\vec{F}$ is the gradient of some scalar function, called the \udef{potential}
\[ \vec{F} = - \grad V \]
\end{enumerate}
\end{eigenschap}
The potential is not unique, any constant can be added to $V$ without changing $\grad V$. The minus sign in the definition is conventional.
\subsection{Solenoidal fields}
\udef{Solenoidal fields} are fields where the divergence vanishes everywhere.
\begin{eigenschap}
For a vector field $\vec{F}$ the following conditions are equivalent:
\begin{enumerate}[(a)]
\item $\nabla\cdot \vec{F} = 0 \;$ everywhere.
\item $\iint \vec{F}\cdot \diff{\vec{a}}$ is independent of the exact path, for a given perimeter.
\item $\oiint \vec{F}\cdot \diff{\vec{a}} = 0$ for any closed surface.
\item $\vec{F}$ is the curl of some vector function, called the \udef{vector potential}
\[ \vec{F} = \curl \vec{A} \]
\end{enumerate}
\end{eigenschap}
\subsection{Helmholtz theorem}
\subsubsection{Decomposition in irrotational and solenoidal field}
An arbitrary vector field $\vec{F}$ can always be written as the sum of an irrotational and a solenoidal field, and thus also as the sum of the gradient of a scalar and the curl of a vector.
\[ \vec{F} = - \grad V + \curl \vec{A} \]
This is sometimes known as the \textbf{Helmholtz decomposition}.
\subsubsection{Fields with prescribed divergence and curl}
Say we have a scalar field $D$ and a solenoidal vector fields $\vec{C}$. Can we find a vector field $\vec{F}$ such that the divergence and curl are given by $D$ and $\vec{C}$ respectively?
\[ \begin{cases}
\nabla\cdot \vec{F} = D \\ \curl \vec{F} = \vec{C}
\end{cases} \]
The answer is \textit{yes} if $\vec{C}(\vec{r})$ and $D(\vec{r})$ go to zero at infinity faster than $\frac{1}{r^2}$. They even define $\vec{F}$ uniquely if $\vec{F}$ goes to zero at infinity.

The vector field field can be constructed as follows:
\[ \vec{F} = -\nabla\cdot U + \curl \vec{W} \]
where
\[ \begin{cases}
U(\vec{r}) = \frac{1}{2\pi}\int \frac{D(\vec{r'})}{|\vec{r} - \vec{r'}|}\diff{\tau'} \\
\vec{W}(\vec{r}) = \frac{1}{2\pi}\int \frac{\vec{C}(\vec{r'})}{|\vec{r} - \vec{r'}|}\diff{\tau'}
\end{cases} \]

\section{Laplace's equation}
TODO intro + see electro
\subsection{Uniqueness theorems}
\subsection{Method of images}
\subsection{Seperation of variables}
