\url{file:///C:/Users/user/Downloads/0-8176-4442-3.pdf}
\url{https://link.springer.com/content/pdf/10.1007%2F978-0-387-84895-2.pdf}
\url{https://zr9558.files.wordpress.com/2014/08/a-guide-to-distribution-theory-and-fourier-transforms.pdf}
\url{file:///C:/Users/user/Downloads/978-0-8176-4675-2.pdf}

TODO: Hölder, Minkowski, Lyapounov

TODO: Moore-Osgood

\chapter{Measure theory}
TODO (bounded) finitely additive signed measures form Riesz spaces.

\section{Measurable spaces and measurable functions}
\begin{definition}
Let $\Omega$ be a non-empty set and $\mathcal{A}$ a $\sigma$-algebra on $\Omega$.

The pair $(\Omega, \mathcal{A})$ is called a \udef{measurable space}.

The elements of $\mathcal{A}$ are called \udef{events}.
\end{definition}


\subsection{Properties of measure spaces}
\subsubsection{Separated measure spaces}
\begin{definition}
Let $\sSet{\Omega,\mathcal{A}}$ be a measurable space and $x,y\in \Omega$. Then $x,y$ are called \udef{separated} or \udef{distinguishable} if there exists $A\in \mathcal{A}$ such that $x\in A$ and $y\notin A$.

The measurable space $\sSet{\Omega,\mathcal{A}}$ is called \udef{separated} if all distinct $x,y\in \Omega$ are separated.
\end{definition}

\begin{lemma}
The relation that relates indistinguishable points is an equivalence relation.

This relation is the identity relation \textup{if and only if} the measurable space is separated.
\end{lemma}

\subsection{Measurable functions}
\begin{definition}
Let $(\Omega_1, \mathcal{A}_1)$ and $(\Omega_2, \mathcal{A}_2)$ be measurable spaces.

A function $f:\Omega_1 \to \Omega_2$ is called \udef{measurable} if
\[ \forall E\in\mathcal{A}_2: f^{\preimf}(E) \in\mathcal{A}_1. \]
We may also say $f$ is \udef{$\mathcal{A}_1/\mathcal{A}_2$-measurable} to emphasise which $\sigma$-algebras are being used.

We denote the set of $\mathcal{A}_1/\mathcal{A}_2$-measurable functions by $\meas(\mathcal{A}_1, \mathcal{A}_2)$, or by $\meas(\Omega_1, \Omega_2)$ if the $\sigma$-algebras are clear from context.
\end{definition}
Suppose we are moving around in $\Omega_1$ and tracking the output of the function in $\Omega_2$. We would like to be able to explore the contents of an event in $\Omega_2$ from within an event in $\Omega_1$.

\begin{lemma} \label{measurableFromGeneratingSet}
Let $(\Omega_1, \mathcal{A}_1)$ and $(\Omega_2, \mathcal{A}_2)$ be measurable spaces and $\mathcal{A}_2$ is generated by $\mathcal{S}$.

Then $f: \Omega_1\to \Omega_2$ is measurable \textup{if and only if} $\forall S\in\mathcal{S}: f^{-1}[S] \in \mathcal{A}_1$.
\end{lemma}

\begin{lemma} \label{constantFunctionsMeasurable}
Any constant function $\underline{a}: \sSet{\Omega_1,\mathcal{A}_1} \to \sSet{\Omega_2,\mathcal{A}_2}: x\mapsto a$ is measurable.
\end{lemma}
\begin{proof}
For any $A\in\mathcal{A}_2$ we have
\[ \underline{a}^{\preimf}(A) = \begin{cases}
\emptyset & (a\notin A) \\
\Omega_1 & (a\in A).
\end{cases} \]
In both cases $\underline{a}^{\preimf}(A)\in\mathcal{A}_1$.
\end{proof}

\begin{lemma} \label{measurabilityCodomainPartition}
Let $(\Omega_1, \mathcal{A}_1)$, $(\Omega_2, \mathcal{A}_2)$ be measurable spaces, $f: \Omega_1 \to \Omega_2$ a function and $B\in \mathcal{A}_2$.

Then $f$ is measurable \textup{if and only if} both $f|^B$ and $f|^{B^c}$ are measurable.
\end{lemma}
\begin{proof}
For the direction $\Rightarrow$, let $\iota_1: f^\preimf(B)\to \Omega_2$ and $\iota_2: f^\preimf(B)^c\to \Omega_2$ be the inclusion functions, which are measurable by definition of sub-measurable space. Then $f|^B = f\circ \iota_1$ and $f|^{B^c} = f\circ \iota_2$ are measurable.

For the other direction, take $A\in \mathcal{A}_2$. Then
\[ f^{\preimf}(A) = f^\preimf(A\cap B) \cup f^\preimf(A\cap B^c) = (f|^B)^\preimf(A\cap B) \cup (f|^{B^c})^\preimf(A\cap B^c) \in \mathcal{A}_1, \]
so $f$ is measurable.
\end{proof}

\subsubsection{$\sigma$-algebras generated by functions}
\begin{definition}
Let $\Omega_1$ be a set and $\sSet{\Omega_2, \mathcal{A}}$ a measurable space. Let $f: \Omega_1\to \Omega_2$ be a function. The \udef{pull-back $\sigma$-algebra} or the $\sigma$-algebra \udef{generated} by $f$ is
\[ \sigma(f) \defeq \setbuilder{f^{-1}[A]}{A\in \mathcal{A}}. \]
\end{definition}

The pull-back $\sigma$-algebra is a $\sigma$-algebra.

\begin{lemma}
Let $(\Omega_1, \mathcal{A}_1)$ and $(\Omega_2, \mathcal{A}_2)$ be measurable spaces and $f: \Omega_1\to \Omega_2$ a function. Then $f$ is $\mathcal{A}_1/\mathcal{A}_2$-measurable \textup{if and only if} $\sigma(f) \subseteq \mathcal{A}_1$.
\end{lemma}

\begin{lemma}
Let $\Omega_1$ be a set, $\sSet{\Omega_2, \mathcal{A}}$ a measurable space and $f: \Omega_1\to \Omega_2$ a function. Then $x,y\in \sSet{\Omega_1,\sigma(f)}$ are indistinguishable \textup{if and only if} $f(x), f(y) \in \sSet{\Omega_2,\mathcal{A}}$ are indistinguishable.
\end{lemma}
\begin{proof}
TODO For all $A\in\mathcal{A}$, we have $f(x)\in A \iff x\in f^{-1}[A]$ and $y\in f^{-1}[A]\iff f(y)\in A$. The indistinguishability of $x,y$ means the left-hand sides are equivalent. The indistinguishability of $f(x),f(y)$ means the right-hand sides are equivalent.
\end{proof}
\begin{corollary}
Let $\Omega_1$ be a set, $\sSet{\Omega_2, \mathcal{A}}$ a measurable space and $f: \Omega_1\to \Omega_2$ a function. Then
\begin{enumerate}
\item $\ker(f)$ is a subset of the indistinguishability relation of $\sigma(f)$;
\item if $\sSet{\Omega_2, \mathcal{A}}$ is separated, then $ker(f)$ equals the indistinguishability relation of $\sigma(f)$.
\end{enumerate}
\end{corollary}


\subsubsection{The Doob-Dynkin property}
\begin{definition}
A measurable set $\sSet{\Omega,\mathcal{A}}$ is said to have the \udef{Doob-Dynkin property} if for any set $\Omega_1$, measurable space $\sSet{\Omega_2,\mathcal{A}_2}$ and function $f: \Omega_1 \to \Omega_2$, a function $g: \Omega_1\to \Omega$ is $\sigma\{f\}/\mathcal{A}$-measurable \textup{if and only if} it factors through $f$, i.e.
\[\exists h\in \meas(\mathcal{A}_2\to \mathcal{A}):\quad g = h\circ f. \]
\end{definition}

\begin{example}
The measurable space $\sSet{\Omega, \mathcal{A}} = \sSet{\{0,1\}, \{\emptyset,\Omega\}}$ does not have the Doob-Dynkin property. 

Take, for example, any sets $\Omega_1, \Omega_2$ and equip $\Omega_2$ with the $\sigma$-algebra $\{\emptyset, \Omega_2\}$. Then for any function $f:\Omega_1 \to \Omega_2$, we have $\sigma(f) = \{\emptyset, \Omega_1\}$ and thus any function $g: \Omega_1 \to \Omega$ is measurable. Clearly it is not true that any function between arbitrary sets factors through any other function.
\end{example}


\begin{proposition}
A measurable set $\sSet{\Omega,\mathcal{A}}$ has the Doob-Dynkin property \textup{if and only if} $\mathcal{A}$ is
\begin{itemize}
\item separated; and
\item for any measurable space $\sSet{\Omega_1,\mathcal{A}_1}$, subset $D\subseteq \Omega_1$ and measurable function $f: \sSet{D,\mathcal{A}_1|_D} \to \sSet{\Omega,\mathcal{A}}$, $f$ can be extended to a measurable function on $\sSet{\Omega_1,\mathcal{A}_1}$.
\end{itemize}
\end{proposition}
\begin{proof}
TODO {functionsLeftRightRelations}

\url{http://www.numdam.org/article/SPS_1990__24__46_0.pdf}
\url{https://mathoverflow.net/questions/263863/does-the-doob-dynkin-lemma-hold-for-any-measurable-space-that-separates-points}

\url{https://math.stackexchange.com/questions/2193181/proof-of-doob-dynkin-lemma-when-x-isnt-assumed-surjective}
\end{proof}


\subsection{Borel $\sigma$-algebras}
\begin{definition}
Let $(X,\mathcal{T})$ be a topological space. The $\sigma$-algebra on $X$ generated by $\mathcal{T}$ is called the \udef{Borel $\sigma$-algebra} of $(X,\mathcal{T})$. The measurable space consisting of $X$ equipped with the Borel $\sigma$-algebra is called the \udef{Borel-measurable space}.
\end{definition}
Clearly the Borel $\sigma$-algebra is also generated by the closed sets.

\begin{lemma}
Let $(X,\mathcal{T})$ be a topological space. If $\mathcal{T}$ has a countable basis $\mathcal{B}$, then the Borel $\sigma$-algebra is generated by the basis $\mathcal{B}$.
\end{lemma}

TODO:
\begin{align*}
\mu(B) &= \sup\setbuilder{\mu(C)}{C\subseteq B, \; \text{$C$ compact}} \\
&= \inf\setbuilder{\mu(O)}{B\subseteq O, \; \text{$O$ open}}.
\end{align*}
cfr compact open topology??

\begin{lemma}
Every continuous function between topological spaces is a measurable function between Borel-measurable spaces.
\end{lemma}

\begin{proposition} \label{pointWiseConvergenceMeasurable}
Let $(\Omega,\mathcal{A})$ be a measurable space and $\sSet{X,\xi}$ a $T_1$ and $C_2$ topological space. We equip $X$ with its Borel $\sigma$-algebra $\mathcal{B}$.

Suppose that a sequence of measurable functions $f_n : \Omega \to Y$ converges pointwise to a function $f:\Omega\to X$. Then $f$ is measurable.
\end{proposition}
\begin{proof}
Pick arbitrary $B\in \mathcal{B}$. By \ref{AnySetCountableIntersectionOfOpenSets} we can write $B = \bigcap_{k\in\N}O_k$ for some sequence $\seq{O_k}$ of open sets. Then we have, using the fact that $\sSet{X,\xi}$ is a sequntial space (TODO ref),
\begin{align*}
x\in f^{\preimf}[B] &\iff f(x)\in B \\
&\iff \forall k\in\N: f(x)\in O_k\\
&\iff \forall k\in\N: \exists n_0\in\N: \forall n\geq n_0: f_n(x)\in O_k \\
&\iff \forall k\in\N: \exists n_0\in\N: \forall n\geq n_0: x\in f^{\preimf}_n[O_k] \\
&\iff x\in \bigcap_{k\in\N}\bigcup_{n_0\in\N}\bigcap_{n\geq n_0}f^{\preimf}_n[O_k].
\end{align*}
So $f^{\preimf}[B] = \bigcap_{k\in\N}\bigcup_{n_0\in\N}\bigcap_{n\geq n_0}f^{\preimf}_n[O_k]$, which is an element of $\mathcal{A}$ because all $f^{\preimf}_n[O_k]$ are in $\mathcal{A}$.
\end{proof}

\url{https://math.stackexchange.com/questions/1343860/limit-of-measurable-functions-is-measurable}


\subsubsection{Measurable real-valued functions}
Unless stated otherwise we always consider the real numbers equipped with the Borel $\sigma$-algebra.

\begin{proposition}[Doob-Dynkin lemma]
TODO: $[0,1]$ and $[-\infty,\infty]$ have the Doob-Dynkin property.
\end{proposition}

\begin{proposition}
Let $\mathcal{B}$ be the Borel $\sigma$-algebra on $\overline{\R}$. Then the following equalities hold for any $b\in\overline{\R}$:
\begin{align*}
\mathcal{B} &= \sigma\setbuilder{\interval[c]{a,b}}{a\in \overline{\R}}\\
&= \sigma\setbuilder{\interval[o]{a,b}}{a\in \overline{\R}} \\
&= \sigma\setbuilder{\interval[oc]{a,b}}{a\in \overline{\R}} \\
&= \sigma\setbuilder{\interval[co]{a,b}}{a\in \overline{\R}}.
\end{align*}
\end{proposition}
\begin{proof}
TODO
\end{proof}
\begin{corollary} \label{measurablesSetsRealMeasurableFunction}
Let $\sSet{\Omega, \mathcal{A}}$ be a measurable space and $f: \Omega \to \overline{\R}$ a function. Then the following are equivalent:
\begin{enumerate}
\item $f$ is measurable;
\item $\setbuilder{\{f\geq a\}\big.}{a\in\R} \subseteq \mathcal{A}$;
\item $\setbuilder{\{f > a\}\big.}{a\in\R} \subseteq \mathcal{A}$;
\item $\setbuilder{\{f \leq a\}\big.}{a\in\R} \subseteq \mathcal{A}$;
\item $\setbuilder{\{f < a\}\big.}{a\in\R} \subseteq \mathcal{A}$.
\end{enumerate}
\end{corollary}

\begin{proposition} \label{measurablesSetsTwoRealMeasurableFunctions}
Let $\sSet{\Omega, \mathcal{A}}$ be a measurable space and $f,g: \Omega \to \overline{\R}$ measurable functions. Then
\begin{enumerate}
\item $\{f < g\} \in \mathcal{A}$;
\item $\{f \leq g\} \in \mathcal{A}$;
\item $\{f = g\} \in \mathcal{A}$;
\item $\{f \neq g\} \in \mathcal{A}$.
\end{enumerate}
\end{proposition}
\begin{proof}
(1) This follows from \ref{measurablesSetsRealMeasurableFunction} and
\[ \{f < g \} = \bigcup_{r\in\Q}\{f<r\}\cap \{r < g\}. \]

(2) Complement of (1).

(3) From $\{f = g\} = \{f \leq g\}\cap \{g \leq f\}$.

(4) Complement of (3).
\end{proof}

\begin{proposition} \label{operationsOnRealMeasurableFunctions}
Let $\sSet{\Omega, \mathcal{A}}$ be a measurable space and $f,g: \Omega \to \overline{\R}$ measurable functions and $a,b\in \overline{\R}$. Then
\begin{enumerate}
\item $f+g$ is measurable;
\item $f\cdot g$ is measurable.
\end{enumerate}
\end{proposition}
\begin{proof}
(1) For all $a\in\R$, we have $\{f + g \geq a\} = \{f \geq a - g\}$. The right-hand side is in $\mathcal{A}$ for all $a\in R$ by \ref{measurablesSetsTwoRealMeasurableFunctions}. The inclusion of the left-hand side in $\mathcal{A}$ proves the measurability of $f+g$ by \ref{measurablesSetsRealMeasurableFunction}.

(2) We use \ref{measurabilityCodomainPartition} to show measurability by partitioning the codomain of $f\cdot g$ into $\{-\infty, 0, +\infty\}$ and $\R\setminus\{0\}$.
For the former restriction, we have
\begin{align*}
(f\cdot g)^\preimf(\{+\infty\}) = \big(\{f = +\infty\}\cap \{g > 0\}\big) &\cup \big(\{f > 0\} \cap \{g = +\infty\}\big) \\ &\cup \big(\{f = -\infty\}\cap \{g < 0\}\big) \\ &\cup \big(\{f < 0\} \cap \{g = -\infty\}\big),
\end{align*}
so $(f\cdot g)^\preimf(\{+\infty\})\in\mathcal{A}$. Similarly $(f\cdot g)^\preimf(\{-\infty\})\in\mathcal{A}$ and $(f\cdot g)^\preimf(\{0\})\in\mathcal{A}$.

Now consider the latter restriction. We can write $(f\cdot g)|^{\R\setminus\{0\}}$ as $f'\cdot g'$, where $f'$ and $g'$ take only finite and non-zero values.

From point (1) and the equality
\[ f'\cdot g' = \frac{1}{4}(f'+g')^2 - \frac{1}{4}(f'-g')^2, \]
the measurability of $f'\times g'$ follows if we can show the measurability of $h^2$ and $-h$, given a measurable $h:\Omega \to \R$.

To show measurability of $h^2$, take arbitrary $a\in\R$. Then
\[ \{h^2 \geq a\} = \begin{cases}
\Omega & (a < 0) \\
\{h \geq \sqrt{a}\} \cup \{h \leq -\sqrt{a}\} & (a \geq 0)
\end{cases}, \]
so $\{h^2 \geq a\} \in \mathcal{A}$ and $h^2$ is measurable by \ref{measurablesSetsRealMeasurableFunction}.

To show measurability of $-h$, take arbitrary $a\in\R$. Then
\[ \{ -g \geq a \} = \{g \leq -a\} \in \mathcal{A}, \]
so $-g$ is measurable by \ref{measurablesSetsRealMeasurableFunction}.
\end{proof}
\begin{corollary}
The function $af+bg$ is measurable for all $a,b\in\overline{\R}$.
\end{corollary}
\begin{proof}
Constant functions are always measurable, by \ref{constantFunctionsMeasurable}, and $af + bg = \underline{a}f + \underline{b}g$, which is measurable by the proposition.
\end{proof}

\begin{proposition} \label{limitOperationsOnRealMeasurableFunctions}
Let $\sSet{\Omega, \mathcal{A}}$ be a measurable space and $\seq{f_n: \Omega \to \overline{\R}}_{n\in\N}$ a sequence measurable functions. Then
\begin{enumerate}
\item $\inf_{n\in\N}f_n$ is measurable;
\item $\sup_{n\in\N}f_n$ is measurable;
\item $\liminf_{n\in\N}f_n$ is measurable;
\item $\limsup_{n\in\N}f_n$ is measurable.
\end{enumerate}
\end{proposition}
\begin{proof}
(1) Take arbitrary $a\in\R$. Then
\[ \{\inf_{n\in\N}f_n \geq a\} = \bigcap_{n\in\N}\{f_n \geq a\}, \]
so $\{\inf_{n\in\N}f_n \geq a\} \in \mathcal{A}$ and thus $\inf_{n\in\N}f_n$ is measurable by \ref{measurablesSetsRealMeasurableFunction}.

(2) We have $\sup_{n\in\N}f_n = -\inf_{n\in\N}(-f_n)$, so the result follows from \ref{operationsOnRealMeasurableFunctions} and point (1).

(3) We have $\liminf_{n\in\N}f_n = \sup_{n\in\N} \inf_{m\geq n}f_m$.

(4) We have $\limsup_{n\in\N}f_n = \inf_{n\in\N} \sup_{m\geq n}f_m$.
\end{proof}
\begin{corollary}
If $\seq{f_n}\to f$ pointwise, then $f$ is measurable.
\end{corollary}
\begin{proof}
We have $\lim_{n\in\N}f_n = \limsup_{n\in\N}f_n$.
\end{proof}
This corollary is a special case of \ref{pointWiseConvergenceMeasurable}.
\begin{corollary} \label{infiniteSumMeasurable}
Let $\sSet{\Omega, \mathcal{A}}$ be a measurable space and $\seq{f_n: \Omega \to \overline{\R}}_{n\in\N}$ a sequence measurable functions such that $\sum_{n\in\N}f_n$ converges everywhere. Then $\sum_{n\in\N}f_n$ is measurable.
\end{corollary}

\subsubsection{Measurable complex-valued functions}

\begin{lemma} \label{modulusMeasurable}
We have $|\cdot| \in \meas(\C,\R)$.
\end{lemma}

\begin{proposition}
Let $\sSet{\Omega,\mathcal{A}}$ be a measurable space and $f: \Omega \to \C$. Then $f$ is measurable \textup{if and only if} both $\Re\circ f$ and $\Im\circ f$ are measurable.
\end{proposition}
\begin{proof}
TODO
\end{proof}


\begin{proposition}
Let $\sSet{\Omega,\mathcal{A}}$ be a measurable space. Then $\meas(\Omega, \C)$ is a complex vector space.
\end{proposition}
\begin{proof}
TODO. Cfr. \ref{operationsOnRealMeasurableFunctions}.
\end{proof}


\section{Measures and generalisations}
\subsection{Contents}
\begin{definition}
Let $\mathcal{S}$ be a semi-ring on a set $\Omega$ and $\sSet{M,+,0}$ a commutative monoid. An $M$-valued \udef{content} on $\mathcal{S}$ is a map $\mu: \mathcal{S} \to M$ satisfying
\begin{itemize}
\item $\mu(\emptyset) = 0$;
\item $\mu$ is \udef{(finitely) additive}: let $\seq{A_n}_{n=0}^k\in \mathcal{S}^*$ be a string of pairwise disjoint sets such that $A_0 \uplus \ldots \uplus A_{k} \in \mathcal{S}$, then
\[ \mu(A_0 \uplus \ldots \uplus A_{k}) = \mu(A_0) + \ldots + \mu(A_{k}). \]
\end{itemize}
If $M = \overline{\R^+}$, then we call the content \udef{positive}. A positive measure is called \udef{finite} if $\im(\mu) \subseteq \R^+$.
\end{definition}
Note that $\emptyset\in \mathcal{S}$ by \ref{emptysetElementSemiring}.

\begin{lemma}
If $\mathcal{S}$ is a ring and $\mu$ a content on $\mathcal{S}$, then finite additivity is implied by
\[ \forall A\perp B\in \mathcal{S}:\qquad \mu(A\uplus B) = \mu(A) + \mu(B). \]
\end{lemma}

\begin{example}
For semirings finite additivity is not implied by binary additivity. Take the semiring
\[ \mathcal{S} = \big\{\emptyset,\; \{1\},\; \{2\},\; \{3\},\; \{1,2,3\}\big\} \]
and consider the content
\[ \mu(\emptyset) = 0,\quad \mu\{1\} = 1, \quad \mu\{2\} = 1, \quad \mu\{3\} = 1, \quad \mu\{1,2,3\} = 0. \]
This satisfies binary additivity, but not finite additivity.
\end{example}

\begin{lemma} \label{contentsOnSemiRingToRing}
Let $\mathcal{S}$ be a semi-ring on $\Omega$. Every content $\mu_\mathcal{S}$ on $\mathcal{S}$ extends uniquely to a content $\mu$ on $\mathfrak{R}\{\mathcal{S}\}$, the ring generated by $\mathcal{S}$.
\end{lemma}
\begin{proof}
The extension of $\mathcal{S}$ to $\mathfrak{R}\{\mathcal{S}\}$ is exactly the closure under disjoint unions, see \ref{ringFromSemiRing}. We extend $\mu$ to this closure by defining
\[ \mu(A\uplus B) = \mu(A) + \mu(B) \]
for all disjoint $A,B\in \mathfrak{R}\{\mathcal{S}\}$. 

Now both $A$ and $B$ are finite disjoint unions of sets in $\mathcal{S}$ and thus so is $A\uplus B$. If $A\uplus B \in \mathcal{S}$, then the extension corresponds to the original definition by finite additivity.
\end{proof}

\begin{lemma} \label{emptysetNullset}
If there exists an $A\in \mathcal{S}$ such that $\mu(A)$ is cancellative, then the requirement $\mu(\emptyset) = 0$ is redundant.
\end{lemma}
\begin{proof}
We calculate $\mu(A) = \mu(A \uplus \emptyset) = \mu(A) + \mu(\emptyset)$, so $\mu(\emptyset) = 0$.
\end{proof}
For positive contents this is requirement is $\mu(A) < \infty$.

\begin{proposition} \label{semiringPositiveContent}
Let $\mathcal{S}$ be a semi-ring on a set $\Omega$, $\mu$ a positive content and $A,B\in \mathcal{S}$. Then
\begin{enumerate}
\item $A\subseteq B$ implies $\mu(A) \leq \mu(B)$;
\item $A\cup B\in\mathcal{S}$ implies $\mu(A \cup B) \leq \mu(A) + \mu(B)$.
\end{enumerate}
\end{proposition}
\begin{proof}
(1) Because $\mathcal{S}$ is a semi-ring, we can write $B = A \uplus (B\setminus A) = A \uplus (C_0 \uplus \ldots \uplus C_k)$ for some $C_0,\ldots, C_k\in\mathcal{S}$. By finite additivity, we have
\[ \mu(B) = \mu(A) + \mu(C_0) + \ldots + \mu(C_k). \]
As $\mu(C_0) + \ldots + \mu(C_k)$ is positive, we have $\mu(A) \leq \mu(B)$.

(2) We can write $A\cup B = (A\setminus B) \uplus (A\cap B) \uplus (B\setminus A)$. Because $\mathcal{S}$ is a semi-ring, we can write $A\setminus B = C_0 \uplus \ldots \uplus C_k$ and $B\setminus A = D_0 \uplus \ldots \uplus D_l$ for some $C_0,\ldots, C_k, D_0,\ldots, D_l\in\mathcal{S}$. By finite additivity, we have
\begin{align*}
\mu(A\cup B) &= & \hspace{-5em} \mu(C_0) + \ldots + \mu(C_k) + &\mu(A\cap B) + \mu(D_0) + \ldots + \mu(D_l) \\
\mu(A) &= & \hspace{-5em} \mu(C_0) + \ldots + \mu(C_k) + &\mu(A\cap B) \\
\mu(B) &= & &\mu(A\cap B) + \mu(D_0) + \ldots + \mu(D_l).
\end{align*}
Thus
\[ \mu(A)+\mu(B) = \mu(C_0) + \ldots + \mu(C_k) + 2\mu(A\cap B) + \mu(D_0) + \ldots + \mu(D_l). \]
As $\mu(A\cap B)$ is positive, we have $\mu(A\cup B) \leq \mu(A)+\mu(B)$.
\end{proof}

\begin{proposition} \label{ringPositiveContent}
Let $\mathcal{S}$ be a \emph{ring} on a set $\Omega$, $\mu$ a positive content and $A,B\in \mathcal{S}$. Then
\begin{enumerate}
\item $A\subseteq B$ and $\mu(A)<\infty$ implies $\mu(B\setminus A) = \mu(B) - \mu(A)$;
\item $\mu(A \cup B) + \mu(A\cap B) = \mu(A) + \mu(B)$;
\item let $\{A_n\}_{i\in I}$ be a set of pairwise disjoint sets in $\mathcal{S}$ such that $\biguplus_{i\in I}A_i \in \mathcal{S}$, then
\[ \mu\Big(\biguplus_{i\in I}A_n\Big) \geq \sum_{i\in I}\mu(A_n). \]
\end{enumerate}
\end{proposition}
\begin{proof}
(1) As $\mathcal{S}$ is a ring, we have $B\setminus A\in\mathcal{S}$. The result follows from rearranging $\mu(B) = \mu\big((B\setminus A) \uplus A\big) = \mu(B\setminus A) + \mu(A)$.

(2) We calculate
\begin{align*}
\mu(A \cup B) + \mu(A\cap B) &= \mu\big((A\setminus B)\uplus (A\cap B) \uplus (B\setminus A)\big) + \mu(A\cap B) \\
&= \mu(A\setminus B) + \mu(A\cap B) + \mu(B\setminus A) + \mu(A\cap B) \\
&= \mu\big((A\setminus B) \uplus (A\cap B)\big) + \mu\big((B\setminus A) \uplus \mu(A\cap B)\big) \\
&= \mu(A) + \mu(B).
\end{align*}

(3) We have
\[ \biguplus_{i\in I} A_i = \bigcup_{\substack{F\subseteq I \\ \text{finite}}}\biguplus_{i\in F}A_i. \]
Because the content is monotone (by \ref{semiringPositiveContent}), we have
\[ \sum_{i\in I}\mu(A_i) = \sup_{\substack{F\subseteq I \\ \text{finite}}}\sum_{i\in F}\mu(A_i) = \sup_{\substack{F\subseteq I \\ \text{finite}}}\mu\Big(\biguplus_{i\in F}A_i\Big) \leq \mu\Big(\bigcup_{\substack{F\subseteq I \\ \text{finite}}}\biguplus_{i\in F}A_i\Big) = \mu\Big(\biguplus_{i\in I} A_i\Big), \]
by \ref{orderPreservingFunctionLatticeOperations}.
\end{proof}

\subsubsection{Null sets}
\begin{definition}
Let $\mathcal{S}$ be a semi-ring on a set $\Omega$ and $\mu$ a positive content. A set $A\subseteq \Omega$ is called a \udef{null set} if there exists $B\in \mathcal{S}$ such that $A\subseteq B$ and $\mu(B) = 0$.

The content is called \udef{complete} if all null sets are elements of $\mathcal{S}$.
\end{definition}
If $A\in\mathcal{S}$ is a null set, then $\mu(A) \leq \mu(B) = 0$, so $\mu(A) = 0$.

\begin{lemma} \label{measureNullSet}
Let $\mu$ be a positive content on a semi-ring $\mathcal{S}$. If $A\in \mathcal{S}$ is a null set, then $\mu(A) = 0$.
\end{lemma}
\begin{proof}
By definition there exists $B\in\mathcal{S}$ such that $A\subseteq B$ and $\mu(B) = 0$. By monotonicity $0 = \mu(B)\geq \mu(A) \geq 0$, so $\mu(A) = 0$.
\end{proof}

\begin{definition}
Let $\mu$ be a positive content on a semi-ring with universe $\Omega$.

A proposition $P(x)$ referencing some $x\in\Omega$ is said to be true \udef{almost everywhere} (or a.e.) if $\setbuilder{x\in\Omega}{\text{$P(x)$ is false}}$ is a null set.
\end{definition}

\subsection{Pre-measures}
\begin{definition}
A \udef{pre-measure} is a content $\mu:\mathcal{S} \to M$ that is \udef{countably additive} or \udef{$\sigma$-additive}: let $\seq{A_n}$ be a sequence of pair-wise disjoint sets in $\mathcal{S}$ such that $\biguplus_{n\in \N}A_n \in \mathcal{S}$, then
\[ \mu\left(\biguplus_{n\in\N}A_n\right) = \sum_{n\in\N}\mu(A_n). \]
\end{definition}

\begin{example}
\begin{itemize}
\item For any non-empty set $\Omega$ define
\[ \mu: \powerset(\Omega)\to [0,\infty]: E\mapsto \begin{cases}
\#(E) & \text{($E$ is finite)} \\
\infty & \text{($E$ is infinite)}
\end{cases}. \]
Then $(\Omega, \powerset(\Omega),\mu)$ is a measure space and $\mu$ is called the \udef{counting measure}.
\item Let $(\Omega,\mathcal{A})$ be a measurable space and $x\in\Omega$ and define
\[ \delta_x: \mathcal{A}\to [0,\infty]: E\mapsto \begin{cases}
1 & (x\in E) \\ 0 & (x\notin E)
\end{cases}. \]
Then $(\Omega,\mathcal{A},\delta_x)$ is a measure space and $\delta_x$ is called a \udef{Dirac measure}.
\end{itemize}
\end{example}

\begin{lemma} \label{countingMeasureCriterion}
Let $\Omega$ be a set. A measure on $\powerset(\Omega)$ is the counting measure \textup{if and only if} $\mu(\{\omega\}) = 1$ for all $\omega\in\Omega$.
\end{lemma}
\begin{proof}
For any finite $A\subseteq \Omega$ we have $A = \biguplus_{a\in A}\{a\}$ and thus $\mu(A) = \#(A)$ be $\sigma$-additivity.

If $A\subseteq \Omega$ is infinite, take $k\in \N$. We can find a subset $B\subseteq A$ of cardinality $k$ and thus we can write $\mu(A) = \mu(B) + \mu(A\setminus B) \geq \mu(B) = k$. Thus $\mu(A) \geq k$ for all $k\in \N$, meaning $\mu(A)$ is infinite.
\end{proof}

\begin{proposition} \label{premeasureSubadditive}
Let $\mathcal{S}$ be a semi-ring and $\mu: \mathcal{S} \to \overline{\R^+}$ a pre-measure. Then for any sequence $\seq{A_n}_{n\in \N}$ of sets in $\mathcal{S}$ such that $\bigcup_{n\in\N}A_n \in \mathcal{S}$, we have
\[ \mu\Big(\bigcup_{n\in\N}A_n\Big) \leq \sum_{n\in\N}\mu(A_n). \]
\end{proposition}
\begin{proof}
Take a sequence $\seq{A_n}$ in $\mathcal{S}$ such that $\bigcup_{n\in\N}A_n\in\mathcal{S}$ and define
\[ A_n' \defeq A_n \setminus\bigg(\bigcup_{k=0}^{n-1}E_k\bigg), \]
which is a finite disjoint union of sets in $\mathcal{S}$. Consider the sequence $\seq{A^{\prime\prime}_n}_{n\in\N}$ which contains the disjoint components of each $A'_n$ in order. This is a disjoint sequence in $\mathcal{S}$ and $\bigcup_{n\in \N}A_n = \biguplus_{n\in \N}A_n^{\prime\prime}$.
Then
\[ \mu\Big(\bigcup_{n\in\N}A_n\Big) \;=\; \mu\Big(\biguplus_{n\in\N}A_n^{\prime\prime}\Big) \;=\; \sum_{n\in\N}\mu(A_n^{\prime\prime}) \;\leq\; \sum_{n\in\N}\mu(A_n). \]
The last inequality follows from \ref{semiringPositiveContent}.
\end{proof}

\begin{lemma} \label{premeasureSubadditivityEquivalences}
Let $\mathcal{S}$ be a ring and $\mu: \mathcal{S} \to \overline{\R^+}$ a content. Then the following are equivalent:
\begin{enumerate}
\item $\mu$ is a pre-measure;
\item for any sequence $\seq{A_n}_{n\in \N}$ of disjoint sets in $\mathcal{S}$ such that $\biguplus_{n\in\N}A_n \in \mathcal{S}$, we have
\[ \mu\Big(\biguplus_{n\in\N}A_n\Big) \leq \sum_{n\in\N}\mu(A_n); \]
\item for any sequence $\seq{A_n}_{n\in \N}$ of sets in $\mathcal{S}$ such that $\bigcup_{n\in\N}A_n \in \mathcal{S}$, we have
\[ \mu\Big(\bigcup_{n\in\N}A_n\Big) \leq \sum_{n\in\N}\mu(A_n). \]
\end{enumerate}
\end{lemma}
\begin{proof}
$(1) \Leftrightarrow (2)$ Immediate from \ref{ringPositiveContent}.

$(2) \Leftrightarrow (3)$ The direction $\Leftarrow$ is immediate. The other direction follows from \ref{premeasureSubadditive}.
\end{proof}

\begin{proposition} \label{premeasureChainContinuous}
Let $\mathcal{S}$ be a ring and $\mu$ a positive content. The following are equivalent:
\begin{enumerate}
\item $\mu$ is a pre-measure;
\item $\mu$ is $\sigma$-chain-continuous;
\end{enumerate}
and imply
\begin{enumerate} \setcounter{enumi}{2}
\item $\lim_{n\to\infty}\mu(A_n) = \mu(A)$ for all monotonically decreasing sequences $\seq{A_n}_{n\in\N}$ of sets in $\mathcal{S}$ such that $A = \bigcap_{n\in\N} A_n \in\mathcal{A}$ and $\mu(A_0) < \infty$.
\end{enumerate}
If $\mu$ is finite, then all statements are equivalent.
\end{proposition}
\begin{proof}
$(1) \Rightarrow (2)$ Let $\{A_n\}_{n\in\N}$ be a countable chain. Then $\{B_n\}_{n\in \N}$, where $B_n \defeq A_{n}\setminus A_{n-1}$ (and $B_0 = A_0$) is a pairwise disjoint sequence. Also $\bigcup_{n\in \N}A_n = \biguplus_{n\in \N} B_n$ and $A_n = \biguplus_{k=0}^n B_k$. Then
\begin{multline*}
\mu\Big(\bigcup_{n\in \N}A_n\Big) = \mu\Big(\biguplus_{n\in \N} B_n\Big) = \sum_{n\in\N}\mu(B_n) = \lim_{n\to\infty}\sum_{k=0}^n\mu(B_k) \\ = \lim_{n\to\infty}\mu\Big(\biguplus_{k=0}^n B_k\Big) = \lim_{n\to\infty}\mu(A_n) = \sup_{n\in\N}\mu(A_n).
\end{multline*}

$(1) \Leftarrow (2)$ Let $\{B_n\}_{n\in\N}$ be a countable pairwise disjoint set of subsets. Set $A_n = \biguplus_{k=0}^nB_k$. Then $\{A_n\}_{n\in\N}$ is a countable chain, $\biguplus_{n\in \N} B_n = \bigcup_{n\in \N}A_n$, so
\begin{multline*}
\mu\Big(\biguplus_{n\in\N}B_n\Big) = \mu\Big(\bigcup_{n\in\N}A_n\Big) = \sup_{n\in\N}\mu(A_n) = \lim_{n\in\N}\mu(A_n) \\ = \lim_{n\in\N}\mu(\biguplus_{k=0}^nB_n) = \lim_{n\to\infty}\sum_{k=0}^n\mu(B_n) = \sum_{n\in\N}\mu(B_n).
\end{multline*}

$(2) \Rightarrow (3)$ Let $\{C_n\}_{n\in\N}$ be a monotonically decreasing sequence with $\mu(C_0) < \infty$. Then $D_n \defeq C_0\setminus C_n$ is an increasing sequence. The result follows by \ref{ringPositiveContent} and continuity of the subtraction.

The equivalence in the case that $\mu$ is finite is also clear.
\end{proof}

\begin{example}
Let $\mu$ be a content on the finite-cofinite algebra of $\N$ such that
\[ \mu(A) = \begin{cases}
0 & (\text{$A$ is finite}) \\ \infty & (\text{$A$ is cofinite})
\end{cases}. \]
This is clearly not a premeasure, but does satisfy point (3) of the proposition.
\end{example}

\subsection{Measures}
\begin{definition}
A \udef{measure} is a pre-measure that is defined on a $\sigma$-algebra.

If $\mu$ is a measure on a $\sigma$-algebra $\mathcal{A}$with universe $\Omega$, then we call $\sSet{\Omega, \mathcal{A}, \mu}$ a \udef{measure space}. The measure space is called
\begin{itemize}
\item \udef{finite} if $\mu(\Omega) < \infty$;
\item \udef{$\sigma$-finite} if there exists a sequence $\seq{A_n}\in \mathcal{A}^\N$ such that $\Omega = \bigcup_{n\in\N}A_n$ and $\mu(A_n)<\infty$ for all $n\in\N$;
\item \udef{semi-finite} if for all $A\in \mathcal{A}$ such that $\mu(A) \neq 0$, there exists $B\in \mathcal{A}$ such that $B\subseteq A$ and $0< \mu(B) \subseteq \infty$.
\end{itemize}
\end{definition}

\begin{lemma} \label{sigmaFiniteSequences}
Let $\sSet{\Omega, \mathcal{A}, \mu}$ be a measure space. Then the following are equivalent:
\begin{enumerate}
\item $\mu$ is $\sigma$-finite;
\item there exists a monotone sequence $\seq{B_n}_{n\in \N}\in\mathcal{A}^\N$ such that $\bigcup_{n\in\N}B_n = \Omega$ and $\mu(B_n) < \infty$ for all $n\in\N$;
\item there exists a disjoint sequence $\seq{C_n}_{n\in \N}\in\mathcal{A}^\N$ such that $\biguplus_{n\in\N}C_n = \Omega$ and $\mu(C_n) < \infty$ for all $n\in\N$.
\end{enumerate}
\end{lemma}
\begin{proof}
$(1) \Rightarrow (2)$ By $\sigma$-finiteness, there exists a sequence $\seq{A_n}\in \mathcal{A}^\N$ such that $\Omega = \bigcup_{n\in\N}A_n$ and $\mu(A_n)<\infty$ for all $n\in\N$.

We set $B_n \defeq \bigcup_{k\leq n} A_k$, which is a sequence in $\mathcal{A}$ that satisfies the requirements.

$(2) \Rightarrow (3)$ We set $C_n \defeq B_n\setminus B_{n-1}$, (with $C_0 = B_0$). This is a sequence in $\mathcal{A}$ that satisfies the requirements.

$(3) \Rightarrow (1)$ Immediate.
\end{proof}

\begin{lemma}
Every $\sigma$-finite measure is semi-finite.
\end{lemma}
\begin{proof}
Let $\mu$ be a $\sigma$-finite measure on a measurable space $\sSet{\Omega,\mathcal{A}}$. Let $A\in\mathcal{A}$ be such that $\mu(A) \neq 0$. We need to find a $B\in \mathcal{A}$ such that $B\subseteq A$ and $0<\mu(B)<\infty$.

By \ref{sigmaFiniteSequences}, there exists a disjoint sequence $\seq{C_n}_{n\in \N}\in\mathcal{A}^\N$ such that $\biguplus_{n\in\N}C_n = \Omega$ and $\mu(C_n) < \infty$ for all $n\in\N$. Then $A = A\cap \biguplus_{n\in\N}C_n = \biguplus_{n\in \N}A\cap C_n$. Since
\[ 0\neq \mu(A) = \mu\Big(\biguplus_{n\in \N}A\cap C_n\Big) = \sum_{n\in\N}\mu(A\cap C_n), \]
and $\mu(A\cap C_n) \geq 0$ for all $n\in \N$, there must exist some $k\in \N$ such that $0< \mu(A\cap C_k)$. We also have $\mu(A\cap C_k) \subseteq \mu(C_k) < \infty$. Thus we can take $B = A\cap C_k$. 
\end{proof}

\begin{lemma}
Let $\sSet{\Omega, \mathcal{A}, \mu}$ be a semi-finite measure space and $A\in\mathcal{A}$ such that $\mu(A) = \infty$. Then
\begin{enumerate}
\item $\sup\setbuilder{\mu(B)}{B\in \mathcal{A}, B\subseteq A, \mu(B)<\infty} = \infty$;
\item for all $C\geq 0$ there exists $B\in \mathcal{A}$ such that $B\subseteq A$ and $\mu(B) \geq C$.
\end{enumerate} 
\end{lemma}
\begin{proof}
(1) Set $\mathcal{B} \defeq \setbuilder{B\in \mathcal{A}}{B\subseteq A, \mu(B)<\infty}$ and $s \defeq \sup\setbuilder{\mu(B)}{B\in \mathcal{B}}$. Then there exists a sequence $\seq{B_n}$ in $\mathcal{B}$ such that $\mu(B_n) \to s$. Now $\seq{\bigcup_{k=0}^n B_k}$ is a monotonically increasing sequence in $\mathcal{B}$ and
\[ s = \lim_n \mu(B_n) \leq \lim_n \mu\Big(\bigcup_{k=0}^n B_k\Big) \leq s. \]
Now suppose, towards a contradiction, that $s < \infty$. Then
\[ \mu\Big(\bigcup_{k=0}^\infty B_k\Big) = \lim_n \mu\Big(\bigcup_{k=0}^n B_k\Big) = s <\infty, \]
so $\bigcup_{k=0}^\infty B_k\in \mathcal{B}$. Since
\[ \infty = \mu(A) = \mu\bigg(\Big(\bigcup_{k=0}^\infty B_k\Big)\uplus \Big(A\setminus \bigcup_{k=0}^\infty B_k\Big)\bigg) = s + \mu\Big(A\setminus\bigcup_{k=0}^\infty B_k\Big), \]
we need to have that $\mu\Big(A\setminus\bigcup_{k=0}^\infty B_k\Big) = \infty$. Then we can find $C \subseteq A\setminus\bigcup_{k=0}^\infty B_k$ such that $C\in\mathcal{A}$ and $0<\mu(C)\subseteq \infty$. Thus $C\uplus \bigcup_{k=0}^\infty B_k \in \mathcal{B}$ and
\[ \mu\Big(C\uplus \bigcup_{k=0}^\infty B_k\Big) = \mu(C) + \mu\Big(\bigcup_{k=0}^\infty B_k\Big) = \mu(C) + s > s, \]
which is a contradiction.

(2) Immediate from (1).
\end{proof}


\subsubsection{Restrictions of measures}
\begin{lemma} \label{submeasurespace}
Let $\sSet{\Omega,\mathcal{A},\mu}$ be a measure space and let $E\in\mathcal{A}$ be an event. Then $\mathcal{A}' = \powerset(E)\cap\mathcal{A}$ is a $\sigma$-algebra on $E$ and $\sSet{E,\mathcal{A}',\mu|_{\mathcal{A}'}}$ is a measure space.
\end{lemma}
We will often write $\mu|_E$ to mean $\mu|_{\mathcal{A}'}$.
\begin{proof}
First we show that $\mathcal{A}'$ is a $\sigma$-algebra on $E$, using \ref{setAlgebraCriteria}.
\begin{itemize}
\item Clearly $E\in \mathcal{A}$ and $E\in\powerset(E)$, so $E\in\mathcal{A}\cap \powerset(E) = \mathcal{A}'$.
\item Complements are taken w.r.t. $E$. Take $A\in \mathcal{A}'$. Then $E\setminus A$ is a subset of $E$ and thus in $\powerset(E)$. Also $E\setminus A\in \mathcal{A}$, because it is a measure-theoretic ring (see \ref{setAlgebraCoincidence}). Thus $A^c = E\setminus A \in \mathcal{A}\cap \powerset(E) = \mathcal{A}'$.
\item Let $\seq{A_n}_{n\in \N}$ be a sequence of sets in $\mathcal{A}'$. Then $\bigcup_{n\in \N}A_n$ is a subset of $E$ and thus in $\powerset(E)$. Also $\bigcup_{n\in \N}A_n\in\mathcal{A}$ because $\mathcal{A}$ is a $\sigma$-algebra. Thus $\bigcup_{n\in \N}A_n\in\mathcal{A}'$.
\end{itemize}
\end{proof}

\subsubsection{Extensions to measures}
\begin{proposition}
Let $\mathcal{F}\subseteq \powerset(\Omega)$ be a $\pi$-system and $\mu_1, \mu_2: \sigma(\mathcal{F}) \to \overline{\R^+}$ two measures with $\mu_1(\Omega) = \mu_2(\Omega) < \infty$, then $\mu_1 = \mu_2$.
\end{proposition}
\begin{proof}
Define
\[ \mathcal{D} \defeq \setbuilder{A\in \sigma(\mathcal{F})}{\mu_1(A) = \mu_2}. \]
Clearly $\mathcal{F} \subseteq \mathcal{D}$, so if $\mathcal{D}$ is a Dynkin system, then $\sigma(\mathcal{F}) \subseteq \mathcal{D}$ by the $\pi-\lambda$ theorem \ref{piLambdaTheorem}. Then $\sigma(\mathcal{F}) = \mathcal{D}$ by construction and thus $\mu_1 = \mu_2$.

We just need to verify the definition of Dynkin system:
\begin{itemize}
\item $\Omega\in\mathcal{D}$ by assumption;
\item if $A\in\mathcal{D}$, then, by \ref{ringPositiveContent}, $\mu_1(A^c) = \mu_1(\Omega) - \mu_1(A) = \mu_2(\Omega) - \mu_2(A) = \mu_2(A^c)$, so $A^c\in \mathcal{D}$;
\item if $\seq{A_n}_{n\in\N}$ is a disjoint sequence of sets in $\mathcal{D}$, then
\[ \mu_1\Big(\biguplus_{n\in\N}A_n\Big) = \sum_{n\in\N}\mu_1(A_n) = \sum_{n\in\N}\mu_2(A_n) = \mu_2\Big(\biguplus_{n\in\N}A_n\Big), \]
so $\biguplus_{n\in\N}A_n\in\mathcal{D}$.
\end{itemize}
\end{proof}
\begin{corollary} \label{sigmaFiniteUniqueExtension}
Let $\mathcal{F}\subseteq \powerset(\Omega)$ be a $\pi$-system and $\mu_1, \mu_2: \sigma(\mathcal{F}) \to \overline{\R^+}$ two measures. Suppose
\begin{itemize}
\item $\mu_1$ and $\mu_2$ agree on $\mathcal{F}$, i.e.\ $\mu_1(E) = \mu_2(E)$ for all $E\in\mathcal{F}$; and
\item there exists a sequence $\seq{E_n}\subseteq \mathcal{F}$ such that
\begin{itemize}
\item $\mu_1(E_n) = \mu_2(E_n) < \infty$ for all $n\in\N$; and
\item $\bigcup_{n\in\N} E_n = \Omega$.
\end{itemize}
\end{itemize}
Then $\mu_1 = \mu_2$.
\end{corollary}
\begin{proof}
For all $E\in \mathcal{F}$, we have $\mu_1|_{E_n} = \mu_2|_{E_n}$ by the proposition. Pick some $A\in \sigma(\mathcal{F})$. Then we can construct a disjoint sequence
\[ A_n \defeq (A\cap E_n)\setminus \bigg(\bigcup_{k=0}^{n-1}E_k\bigg) \]
and calculate
\[ \mu_1(A) = \sum_{n\in \N}\mu_1(A_n) = \sum_{n\in \N}\mu_1|_{E_n}(A_n) = \sum_{n\in \N}\mu_2|_{E_n}(A_n) = \sum_{n\in \N}\mu_1(A_n) = \mu_2(A). \]
\end{proof}

\subsubsection{New measures from old ones}
\begin{lemma} \label{measuresPositiveLinear}
Let $(\Omega,\mathcal{A})$ be a measurable space. Positive linear combinations of measures are measures: let $\mu_1,\mu_2$ be measures on $(\Omega,\mathcal{A})$ and $0 \leq c \in \R$. Then $c\mu_1 + \mu_2$ is a measure on $(\Omega,\mathcal{A})$.
\end{lemma}

\begin{proposition}[Pushforward measure] \label{pushforwardMeasure}
Let $\sSet{\Omega_1, \mathcal{A}, \mu}$ be a measure space and $\sSet{\Omega_2, \mathcal{B}}$ a measurable space. Let $f: \Omega_1 \to \Omega_2$ be a measurable function. Then
\[ \nu = \mu\circ f^{-1}|_{\mathcal{B}}: \mathcal{B}\to [0,\infty]: B \mapsto \mu(f^{-1}[B]) \]
is a measure on $\sSet{\Omega_2, \mathcal{B}}$.
\end{proposition}
\begin{proof}
Clearly this is well-defined due to $f$ being measurable. We have
\[ \nu(\emptyset) = \mu(f^{-1}[\emptyset])  = \mu(\emptyset) = 0 \]
for every sequence $(E_n)$ in $\mathcal{B}$ of pairwise disjoint sets, we have
\[ \nu\left(\biguplus_{n\in\N}E_n\right) = \mu\left(f^{-1}\left[\biguplus_{n\in\N}E_n\right]\right) = \mu\biguplus_{n\in\N} f^{-1}[E_n]  = \sum_{n\in\N}\mu(f^{-1}[E_n]) = \sum_{n\in\N}\nu(E_n). \]
\end{proof}

\subsection{Outer measures}
\begin{definition}
Let $\Omega$ be a non-empty set. An \udef{outer measure} on $\Omega$ is a map $\nu: \powerset(\Omega)\to \overline{\R^+}$ satisfying
\begin{itemize}
\item $\nu(\emptyset) = 0$;
\item if $E\subseteq F\subseteq \Omega$, then $\nu(E)\leq \nu(F)$;
\item $\nu$ is \udef{$\sigma$-subadditive}: for every sequence $\seq{E_n}$ in $\powerset(\Omega)$, we have
\[ \nu\left(\bigcup_{n\in\N}E_n\right) \leq \sum_{n\in\N}\nu(E_n). \]
\end{itemize}
\end{definition}
A countably additive outer measure is a premeasure, but in general an outer measure is neither a premeasure nor a measure.

\begin{lemma}
In the definition we can replace $\sigma$-subadditivity by subadditivity of countable, pairwise disjoint sequences.
\end{lemma}
Cfr. \ref{premeasureSubadditive}.
\begin{proof}
Clearly $\sigma$-subadditivity implies the subadditivity of countable, pairwise disjoint sequences.

Now assume the subadditivity of countable, pairwise disjoint sequences and take a sequence $\seq{E_n}$ in $\powerset(\Omega)$. Then define
\[ E_n' \defeq E_n \setminus\bigg(\bigcup_{k=0}^{n-1}E_k\bigg), \]
which forms a disjoint sequence such that $\bigcup_{n\in \N}E_n = \biguplus_{n\in \N}E_n'$. We have
\[ \nu\Big(\bigcup_{n\in\N}E_n\Big) \;=\; \nu\Big(\biguplus_{n\in\N}E_n'\Big) \;\leq\; \sum_{n\in\N}\nu(E_n') \;\leq\; \sum_{n\in\N}\nu(E_n). \]
The last inequality follows from the monotonicity of the outer measure.
\end{proof}


\subsubsection{Carathéodory measurability}
\begin{definition}
Let $\Omega$ be a set and $\rho: \powerset(\Omega)\to \overline{\R^+}$ a function. Then $A\subseteq \Omega$ is called \udef{(Carathéodory) measurable} w.r.t.\ $\rho$ if
\[ \forall B\subseteq \Omega: \; \rho(B) = \rho(B\cap A) + \rho(B\cap A^c). \]
\end{definition}
Note that $A\subseteq \Omega$ is measurable if and only if $A^c$ is measurable.

\begin{lemma}
Let $\Omega$ be a set and $\rho: \powerset(\Omega)\to \overline{\R^+}$ a function. If $\rho(\emptyset) \neq 0$, then there exist no Carathéodory measurable sets.
\end{lemma}
\begin{proof}
We prove by contraposition, so assume that there exists a Carathéodory meaurable $A\subseteq \Omega$.
Then
\begin{align*}
\rho(A) &= \rho(A\cap A) + \rho(A\cap A^c) \\
&= \rho(A) + \rho(\emptyset),
\end{align*}
which implies that $\rho(\emptyset) = 0$.
\end{proof}

\begin{lemma} \label{CaratheodoryMeasurableInequality}
Let $\Omega$ be a set and $\nu: \powerset(\Omega)\to \overline{\R^+}$ an outer measure on $\Omega$. A subset $A\subseteq \Omega$ is Carathéodory measurable \textup{if and only if}
\[ \forall B\subseteq \Omega: \; \nu(B) \geq \nu(B\cap A) + \nu(B\cap A^c). \]
\end{lemma}
\begin{proof}
The other inequality follows from subadditivity.
\end{proof}

\begin{lemma} \label{nullSetsCaratheodoryMeasurable}
Let $\Omega$ be a set, $\nu: \powerset(\Omega)\to \overline{\R^+}$ an outer measure on $\Omega$ and $A\subseteq \Omega$ such that $\nu(A) = 0$ or $\nu(A^c) = 0$, then $A$ is Carathéodory measurable.
\end{lemma}
\begin{proof}
First assume $\nu(A) = 0$. Take arbitrary $B\subseteq \Omega$. Then $\nu(B\cap A) = 0$ by monotonicity. Also by monotonicity, we have
\[ \nu(B) \geq \nu(B\cap A^c) = \nu(B\cap A) + \nu(B\cap A^c) \]
and so $A$ is Carathéodory measurable by \ref{CaratheodoryMeasurableInequality}.

The case of $\nu(A^c) = 0$ is similar.
\end{proof}

\begin{lemma} \label{caratheodoryExtensionLemma}
Let $\Omega$ be a set, $\rho: \powerset(\Omega) \to \overline{\R^+}$ a function and $\seq{A_i}_{i\in\N}$ a pairwise disjoint sequence of subsets of $\Omega$ that are Carathéodory measurable w.r.t.\ $\rho$. Then
\[ \rho(B) = \sum_{i=0}^n \rho(B\cap A_i) + \rho\big(B\cap (\bigcap_{i=0}^nA_i^c)\big) \]
for all $B\in \powerset(\Omega)$.
\end{lemma}
\begin{proof}
The proof is by induction on $n$. The base case $n=0$ is just the Carathéodory measurability of $A_0$.

For the induction step, we calculate
\begin{align*}
\rho(B) &= \sum_{i=0}^n \rho(B\cap A_i) + \rho\big(B\cap (\bigcap_{i=0}^nA_i^c)\big) \\
&= \sum_{i=0}^n \rho(B\cap A_i) + \rho\big(B\cap (\bigcap_{i=0}^nA_i^c)\cap A_{n+1}\big) + \rho\big(B\cap (\bigcap_{i=0}^nA_i^c)\cap A_{n+1}^c\big) \\
&= \sum_{i=0}^n \rho(B\cap A_i) + \rho\big(B\cap A_{n+1}\big) + \rho\big(B\cap (\bigcap_{i=0}^{n+1}A_i^c)\big) \\
&= \sum_{i=0}^{n+1}\rho(B\cap A_i) + \rho\big(B\cap (\bigcap_{i=0}^{n+1}A_i^c)\big).
\end{align*}
We conclude by induction.
\end{proof}

\begin{proposition} \label{caratheodoryMeasurableAlgebra}
Let $\Omega$ be a set and $\rho: \powerset(\Omega) \to \overline{\R^+}$ a function such that $\rho(\emptyset) = 0$. Let $\mathcal{A}$ be the set of Carathéodory measurable subsets of $\Omega$. Then
\begin{enumerate}
\item $\mathcal{A}$ is an algebra;
\item if $\rho$ is an outer measure, then $\mathcal{A}$ is a $\sigma$-algebra.
\end{enumerate}
\end{proposition}
\begin{proof}
(1) We use \ref{setAlgebraCriteria} the check that $\mathcal{A}$ is an algebra.
\begin{itemize}
\item For all $B\in \powerset(\Omega)$, we have
\[ \rho(B) = \rho(B\cap \Omega) + 0 = \rho(B\cap \Omega) + \rho(\emptyset) = \rho(B\cap \Omega) + \rho(B\cap \Omega^c), \]
so $\Omega\in \mathcal{A}$.
\item Closure under complementation is clear as the definition of Carathéodory measurability is symmetric w.r.t.\ complementation.
\item Assume $A_1,A_2 \in \mathcal{A}$. Take arbitrary $B\in\powerset(\Omega)$. Then
\begin{align*}
\rho(B) &= \rho(B\cap A_1) + \rho(B\cap A_1^c) \\
&= \rho(B\cap A_1) + \rho(B\cap A_1^c\cap A_2) + \rho(B\cap A_1^c\cap A_2^c) \\
&= \rho(B\cap A_1) + \rho(B\cap A_1^c\cap A_2) + m\big(B\cap (A_1\cup A_2)^c\big) \\
&= \rho\big(B\cap (A_1\cup A_2)\cap A_1\big) + \rho\big(B\cap (A_1\cup A_2)\cap A_1^c\big) + \rho\big(B\cap (A_1\cup A_2)^c\big) \\
&= \rho\big(B\cap (A_1\cup A_2)\big) + \rho\big(B\cap (A_1\cup A_2)^c\big),
\end{align*}
so $A_1\cup A_2\in \mathcal{A}$.
\end{itemize}

(2) To show $\mathcal{A}$ is a $\sigma$-algebra, it is enough to show closure under countable disjoint unions by \ref{conditionAlgebraIsSigmaAlgebra}. Let $\seq{A_n}$ be a sequence of pairwise disjoint sets in $\mathcal{A}$. 

By \ref{caratheodoryExtensionLemma} and monotonicity, we have
\[ \rho(B) \geq \sum_{i=0}^n \rho(B\cap A_i) + \rho\big(B\cap (\bigcap_{i=0}^\infty A_i^c)\big) \]
for all $n\in \N$. Thus, taking the limit $n\to\infty$,
\[ \rho(B) \geq \sum_{i=0}^\infty \rho(B\cap A_i) + \rho\Big(B\cap \big(\biguplus_{i=0}^\infty A_i\big)^c\Big) \geq \rho\Big(B\cap \big(\biguplus_{i=0}^\infty A_i\big)\Big) + \rho\Big(B\cap \big(\biguplus_{i=0}^\infty A_i\big)^c\Big), \]
which is enough to prove the  Carathéodory measurability of $\biguplus_{i=0}^\infty A_i$ by \ref{CaratheodoryMeasurableInequality}.
\end{proof}


\subsubsection{Construction of outer measures}
\begin{proposition} \label{outerMeasureConstruction}
Let $\mathcal{X}$ be a family of subsets of $\Omega$ such that $\emptyset\in \mathcal{X}$ and let $\rho: \mathcal{X}\to \overline{\R^+}$ be a function such that $\rho(\emptyset) = 0$. Then
\[ \rho^*: \powerset(\Omega) \to \overline{R^+}:\quad X\mapsto  \inf\setbuilder{\sum_{n\in\N}\rho(A_n)}{\big(\seq{A_n}_{n\in\N}\subseteq \mathcal{X}\big) \land \big(X\subseteq \bigcup_{n\in\N}A_n\big)} \]
is an outer measure on $\Omega$.
\begin{enumerate}
\item If $\rho$ is a pre-measure, then $\rho(A) = \rho^*(A)$ for all $A\in\mathcal{X}$.
\item If $\rho$ is a measure, then $\rho^*(X) = \inf\setbuilder{\rho(A)}{(A\in \mathcal{X}) \land (X\subseteq A)}$.
\end{enumerate}
\end{proposition}
\begin{proof}
We first verify that $\rho^*$ is an outer measure.
\begin{itemize}
\item We have $\rho(\emptyset) = 0$, so $\rho^*(\emptyset)\leq 0$, but there is no element of $\overline{\R^+}$ smaller than $0$.
\item Take $E\subseteq F\subseteq \Omega$. Then every cover of $F$ is a cover of $E$. So the least cover of $E$ is smaller than the least cover of $F$.
\item Fix a sequence $\seq{E_n}$ in $\powerset(\Omega)$. Take some arbitrary $\epsilon > 0$. For each $E_n$, we can find some sequence $\seq{A_{n,k}}\subseteq \mathcal{X}$ such that $E_n\subseteq \bigcup_{k\in\N}A_{n,k}$ and $\sum_{k\in\N}\rho(A_{n,k}) \leq \rho^*(E_n) + \frac{\epsilon}{2^n}$.
Now the countable set $\seq{A_{n,k}}_{n,k\in \N}$ is a cover of $\bigcup_{n\in\N}E_n$, so
\[ \rho^*\Big(\bigcup_{n\in\N}E_n\Big) \leq \sum_{n,k\in\N}\rho(A_{n,k}) = \sum_{n\in\N}\sum_{k\in\N}\rho(A_{n,k}) \leq \sum_{n\in\N}\big(\rho^*(E_n)+ \frac{\epsilon}{2^n}\big) = 2\epsilon + \sum_{n\in\N}\rho^*(E_n), \]
where we have used \ref{joinMeetUnion} and the fact that the infinite sums are suprema over finite sums.

As the $\epsilon> 0$ was arbitrary, we have that
\[ \rho^*\Big(\bigcup_{n\in\N}E_n\Big) \leq \sum_{n\in\N}\rho^*(E_n). \]
\end{itemize}


(1) Assume $\rho$ is a pre-measure and take $A\in\mathcal{X}$. By construction, $\rho^*(A) \leq \rho(A)$. 

For the other inequality, take an arbitrary sequence $\seq{A_n}_{n\in\N}$ in $\mathcal{X}$ such that $A \subseteq \bigcup_{n\in\N}A_n$. Then
\[ \sum_{n\in\N}\rho(A_n) \geq \sum_{n\in\N}\rho(A_n\cap A) \geq \rho\Big(\bigcup_{n\in\N}(A_n\cap A)\Big) = \rho(A), \]
using monotonicity \ref{semiringPositiveContent} and subadditivity \ref{premeasureSubadditive}.

(2) We calculate
\begin{align*}
\inf\setbuilder{\rho(A)}{(A\in \mathcal{X}) \land (X\subseteq A)} &\geq \inf\setbuilder{\sum_{n\in\N}\rho(A_n)}{\big(\seq{A_n}_{n\in\N}\subseteq \mathcal{X}\big) \land \big(X\subseteq \bigcup_{n\in\N}A_n\big)} \\
&\geq \inf\setbuilder{\rho\Big(\bigcup_{n\in\N}A_n\Big)}{\big(\seq{A_n}_{n\in\N}\subseteq \mathcal{X}\big) \land \big(X\subseteq \bigcup_{n\in\N}A_n\big)} \\
&\geq \inf\setbuilder{\rho(A)}{(A\in \mathcal{X}) \land (X\subseteq A)},
\end{align*}
where we have used first that $\rho(A)$ is a special case of the sum with just one term, next subadditivity of $\rho$ and finally the fact that $\mathcal{X}$ is a $\sigma$-algebra in this case, so $\bigcup_{n\in \N}A_n\in \mathcal{X}$ for all sequences $\seq{A_n}\in \mathcal{X}^\N$.
\end{proof}

\subsubsection{The extension theorem}
\begin{proposition} \label{caratheodoryLemma}
Let $\Omega$ be a set and $\rho: \powerset(\Omega) \to \overline{\R^+}$ a function such that $\rho(\emptyset) = 0$. Let $\mathcal{A}$ be the set of Carathéodory measurable subsets of $\Omega$. Then
\begin{enumerate}
\item $\rho|_{\mathcal{A}}$ is a complete content on the algebra $\mathcal{A}$;
\item if $\rho$ is an outer measure, then $\rho|_{\mathcal{A}}$ is a complete measure.
\end{enumerate}
\end{proposition}
\begin{proof}
(1) We have that $\mathcal{A}$ is an algebra by \ref{caratheodoryMeasurableAlgebra}. We just need to prove additivity. Take disjoint $A,B\in\mathcal{A}$. Then
\begin{align*}
\rho(A\uplus B) &= \rho\big((A\uplus B)\cap A\big) + \rho\big((A\uplus B)\cap A^c\big) \\
&= \rho(A) + \rho(B).
\end{align*}
Completeness follows from \ref{nullSetsCaratheodoryMeasurable}.

(2) In this case we have that $\mathcal{A}$ is a $\sigma$-algebra by \ref{caratheodoryMeasurableAlgebra}. The result follows straight from (1) and \ref{premeasureSubadditivityEquivalences}, because outer measures are $\sigma$-subadditive. 
\end{proof}

\begin{theorem}[Carathéodory-Fréchet-Hahn-Kolmogorov-Hopf extension theorem] \label{premeasureExtensionTheorem}
Let $\mathcal{S}$ be a semi-ring on a set $\Omega$ and $\mu: \mathcal{S}\to \overline{\R^+}$ a pre-measure, then $\mu$ can be extended to a complete measure on $\sigma(\mathcal{S})$.
\end{theorem}
\begin{proof}
We first construct the outer measure $\mu^*$ as in \ref{outerMeasureConstruction} and then restrict this to a complete measure as in \ref{caratheodoryLemma}. If we can prove that each $A\in \mathcal{S}$ is Carathéodory measurable, then the $\sigma$-algebra on which the measure is defined must contain $\sigma(\mathcal{S})$. In this case \ref{outerMeasureConstruction} gives $\mu^*(A) = \mu(A)$. Thus restricting the measure to $\sigma(\mathcal{S})$ gives the first result.

We prove that each set in $\mathcal{S}$ is Carathéodory measurable. Take $A\in \mathcal{S}$. Pick an arbitrary $B\subseteq \Omega$ and fix some $\epsilon > 0$. Then we can find a sequence $\seq{B_n}_{n\in\N}$ in $\mathcal{S}$, that covers $B$, such that
\[ \sum_{n\in\N}\mu(B_n) \leq \mu^*(B) + \epsilon. \]
The sets $B_n\cap A$ are all in $\mathcal{S}$ and cover $B\cap A$. The sets $B_n\setminus A$ can be written as $B_n\setminus A = \biguplus_{k\in\N}D_{n,k}$, where $D_{n,k}\subseteq\mathcal{S}$ and only finitely many are non-empty.
We have
\begin{align*}
\mu^*(B\cap A) &\leq \sum_{n\in\N}\mu(B_n\cap A) \\
\mu^*(B\setminus A) &\leq \sum_{n,k\in\N} \mu(D_{n,k}).
\end{align*}
Then
\[ \mu(B_n) = \mu\Big((B_n\cap A) \uplus \biguplus_{k\in\N}D_{n,k}\Big) = \mu(B_n\cap A) + \sum_{k\in \N}\mu(D_{n,k}),  \]
by finite additivity (as there are only finitely many non-zero terms of the sum), and we get
\begin{align*}
\mu^*(B\cap A) + \mu^*(B\setminus A) &\leq \sum_{n\in\N}\mu(B_n\cap A) + \sum_{n,k\in\N} \mu(D_{n,k}) \\
&= \sum_{n\in\N}\Big(\mu(B_n\cap A) + \sum_{k\in\N}\mu(D_{n,k})\Big) \\
&= \sum_{n\in\N}\mu(B_n) \leq \mu^*(B) + \epsilon.
\end{align*}
As $\epsilon >0$ is arbitrary, we have $\mu^*(B) \geq \mu^*(B\cap A) + \mu^*(B\setminus A)$. By \ref{CaratheodoryMeasurableInequality}, this is enough to show the Carathéodory measurability of $A$.
\end{proof}

\subsubsection{Regular outer measures}
\begin{definition}
Let $\Omega$ be a non-empty set. An outer measure $\nu: \powerset(\Omega)\to \overline{\R^+}$ is called \udef{regular} if for every set $A\subseteq \Omega$ there exists a Carathéodory measurable set $B\supseteq A$ such that $\nu(A) = \nu(B)$.
\end{definition}

\begin{proposition}
Let $\nu$ be a regular outer measure on $\Omega$ with $\nu(\Omega) < \infty$. Then a set $A\subseteq \Omega$ is Carathéodory measurable \textup{if and only if}
\[ \nu(A) + \nu(A^c) = \nu(\Omega). \]
\end{proposition}
\begin{proof}
The direction $\Rightarrow$ is immediate.

For the direction, pick some arbitrary $B\subseteq\Omega$. Then, by regularity, $B$ is contained in some Carathéodory measurable $C\subseteq \Omega$ such that $\nu(B) = \nu(C)$.

Consider the following:
\begin{alignat*}{5}
\nu(&A) & &= \quad & \nu(A&\cap C) & &+ & \nu(A&\cap C^c) \\
&+ &&& & &&& &+ \\
\nu(&A^c) & \quad &= & \nu(A^c&\cap C) & \quad&+\quad & \nu(A^c&\cap C^c) \\
&\rotatebox[origin=c]{90}{$=$} &&& & &&& &\rotatebox[origin=c]{90}{$\leq$} \\
\nu(&\Omega) & &= & \nu(&C) & &+ & \nu(&C^c)
\end{alignat*}
The three equations in the rows are instances of the measurability of $C$. The left column is the assumption on $A$ and the right column follows by subadditivity from $C^c = (A\cap C^c) \cup (A^c\cap C^c)$. By consistencey of the square, we can extract the following inequality from the center column:
\[ \nu(C) \geq \nu(A\cap C) + \nu(A^c\cap C). \]
For $B$ we then have
\begin{align*}
\nu(B) = \nu(C) &\geq \nu(A\cap C) + \nu(A^c\cap C) \\
&\geq \nu(A\cap B) + \nu(A^c\cap B),
\end{align*}
by monotonicity. By \ref{CaratheodoryMeasurableInequality}, this is enough to prove measurability of $A$. 
\end{proof}

\begin{proposition}
Let $\Omega$ be a set, $\mathcal{S}$ a semi-ring on $\Omega$ and $\rho: \mathcal{S}\to\overline{\R^+}$ a premeasure. Let $\rho^*:\powerset(\Omega)\to \overline{\R^+}$ be the outer measure constructed in \ref{outerMeasureConstruction}. Then
\begin{enumerate}
\item $\rho^*(X) = \inf\setbuilder{\rho^*(B)}{\text{$B\supseteq X$ is $\rho^*$-measurable}}$;
\item $\rho^*$ is regular.
\end{enumerate}
\end{proposition}
\begin{proof}
(1) We have $\rho^*(X) \leq \rho^*(B)$ for all $\rho^*$-measurable superset $B$ of $X$. Taking the infimum over the $B$s preserves the inequality, so $\rho^*(X) \leq \inf\setbuilder{\rho^*(B)}{\text{$B\supseteq X$ is $\rho^*$-measurable}}$. For the other inequality we calculate
\begin{align*}
\rho^*(X) &= \inf\setbuilder{\sum_{n\in\N}\rho(A_n)}{\big(\seq{A_n}_{n\in\N}\subseteq \mathcal{S}\big) \land \big(X\subseteq \bigcup_{n\in\N}A_n\big)} \\
&\geq \inf\setbuilder{\rho\Big(\bigcup_{n\in \N}A_n\Big)}{\big(\seq{A_n}_{n\in\N}\subseteq \mathcal{S}\big) \land \big(X\subseteq \bigcup_{n\in\N}A_n\big)} \\
&\geq \inf\setbuilder{\rho\Big(B\Big)}{\text{$B\supseteq X$ is $\rho^*$-measurable}},
\end{align*}
where we have used $\sigma$-subadditivity of the outer measure and the fact that all $A_n\in \mathcal{S}$ are $\rho^*$-measurable by \ref{premeasureExtensionTheorem}, so $\bigcup_{n\in \N}A_n$ is $\rho^*$-measurable by \ref{caratheodoryMeasurableAlgebra}.

(2) Take an arbitrary $A\subseteq \Omega$. For all $n\in \N$, we can find $x_n\in \setbuilder{\rho^*(B)}{\text{$B\supseteq A$ is $\rho^*$-measurable}}$ such that $x_n \leq \rho^*(A) + \frac{1}{n}$. 

Then we can find some sequence $\seq{B_n}$ of $\rho^*$-measurable supersets of $A$ such that $x_n = \rho^*(B_n)$.

We have that $\bigcap_{n\in\N}B_n$ is measurable by \ref{caratheodoryMeasurableAlgebra}. We claim that $\rho^*\Big(\bigcap_{n\in\N}B_n\Big) = \rho^*(A)$. Indeed $A \subseteq \bigcap_{n\in\N}B_n$ and \ref{orderPreservingFunctionLatticeOperations} imply
\[ \rho^*\Big(\bigcap_{n\in\N}B_n\Big) \leq \inf_{n\in\N}\rho^*(B_n) = \inf_{n\in\N}\Big(\rho^*(A)+ \frac{1}{n}\Big) = \rho^*(A) \leq \rho^*\Big(\bigcap_{n\in\N}B_n\Big). \] 
\end{proof}
\begin{corollary}
Let $\nu$ be an outer measure on $\Omega$. Then
\[ \nu': \powerset(\Omega) \to \overline{R^+}:\quad X\mapsto  \inf\setbuilder{\nu(B)}{\text{$B\supseteq A$ is $\nu$-measurable}} \]
is a regular outer measure.
\end{corollary}
This outer measure is called the \udef{regular outer measure associated to $\nu$}.
\begin{proof}
Let $\mathcal{A}$ be the $\sigma$-algebra (by \ref{caratheodoryMeasurableAlgebra}) of $\nu$-measurable sets. Then, because $\mathcal{A}$ is a $\sigma$-algebra, the function $\nu'$ is a the outer measure of \ref{outerMeasureConstruction} generated by the measure $\nu|_\mathcal{A}$.
\end{proof}

\begin{proposition}
Regular outer measures preserve joins of countable chains.
\end{proposition}
\begin{proof}
Let $\nu$ be a regular outer measure and $\seq{A_n}_{n\in\N}$ a monotonically growing sequence of sets. Then we can find a sequence $\seq{B_n}$ of measurable sets such that $\nu(A_n) = \nu(B_n)$ for all $n\in\N$. Now set $C_n = \bigcap_{k \geq n}B_k$. It is clear that $\seq{C_n}$ is monotonic and each $C_n$ is measurable by \ref{caratheodoryMeasurableAlgebra}. As $A_n\subseteq B_k$ for all $k\geq n$, we have $A_n\subseteq C_n$ and thus $\nu(A_n) \leq \nu(C_n) \leq \nu(B_n) = \nu(A_n)$, so $\nu(A_n) = \nu(C_n)$. 

Now, by \ref{orderPreservingFunctionLatticeOperations} and \ref{premeasureChainContinuous} (which we may apply due to \ref{caratheodoryLemma}), we have
\[ \nu\Big(\bigcup_{n\in\N}A_n\Big) \geq \sup_{n\in \N}\nu(A_n) = \sup_{n\in \N}\nu(C_n) = \nu\Big(\bigcup_{n\in\N}C_n\Big) \geq \nu\Big(\bigcup_{n\in\N}A_n\Big), \]
so $\nu\Big(\bigcup_{n\in\N}A_n\Big) = \sup_{n\in \N}\nu(A_n)$.
\end{proof}

\subsection{Constructions}
\subsubsection{Product measures}

\begin{proposition} \label{productMeasure}
Let $\sSet{\Omega_1, \mathcal{A}_1,\mu_1}$, $\sSet{\Omega_2, \mathcal{A}_2, }$ be measure spaces. Then there exists a $\mu$ on $\mathcal{A}_1\otimes\mathcal{A}_2$ that satisfies
\[ \mu(A_1\times A_2) = \mu_1(A_1)\cdot\mu_2(A_2) \qquad \forall A_1\in\mathcal{A}_1, A_2\in\mathcal{A}_2. \]
If $\mu_1$ and $\mu_2$ are $\sigma$-finite, then this measure is unique and $\sigma$-finite.
\end{proposition}
\begin{proof}
The set $\mathcal{B} \defeq \setbuilder{A_1\times A_2}{A_1\in \mathcal{A}_1, A_2\in \mathcal{A}_2}$ is a semi-ring by \ref{productSemiring}. 

If $\mu$ is a pre-measure, then, by \ref{premeasureExtensionTheorem}, it extends to a measure on $\sigma(\mathcal{B}) = \mathcal{A}_1\otimes\mathcal{A}_2$.

We verify that $\mu$ is a pre-measure:
\begin{itemize}
\item $\mu(\emptyset) = \mu(\emptyset \times\emptyset) = \mu_1(\emptyset)\cdot \mu_2(\emptyset) = 0\cdot 0 = 0$.
\item Take $\seq{B_k\times C_k} \in \mathcal{B}^\N$ pairwise disjoint such that $\biguplus_{k\in\N}B_k\times C_k = B\times C\in \mathcal{B}$. Now consider the equivalence relation on $B$ which relates $x$ and $y$ if $\forall k\in\N: x\in B_k \iff y\in B_k$. Now partition $B = \biguplus_{i\in\N} B'_{i}$ according to this equivalence relation. 

Every $B_k$ is a disjoint union of $B_i'$s and we write $B_k = \biguplus_{i\in I_k}B_i'$ (indeed, if $x$ and $y$ are in the same equivalence class $B_i'$ and $x\in B_k$, then $y\in B_k$). 

We partition $C = \biguplus_{j\in\N}C'_j$ and define $J_k$ similarly. We calculate
\begin{align*}
\mu\Big(\biguplus_{k\in\N} B_k\times C_k \Big) &= \mu(B\times C) \\
&= \mu_1(B)\cdot\mu_2(C) \\
&= \mu_1\Big(\biguplus_{i\in\N}B'_{i}\Big)\cdot\mu_2\Big(\biguplus_{j\in\N}C'_{j}\Big) \\
&= \sum_{i,j\in\N}\mu_1(B'_{i})\cdot\mu_2(C'_{j}) \\
&= \sum_{k\in\N} \Big(\sum_{i\in I_k}\mu_1(B'_{i})\Big)\cdot\Big(\sum_{j\in J_k}\mu_2(C'_{j})\Big) \\
&= \sum_{k\in\N} \mu_1\Big(\biguplus_{i\in I_k}B'_{i}\Big)\cdot\mu_2\Big(\biguplus_{j\in J_k}C'_{j}\Big) \\
&= \sum_{k\in\N} \mu_1(B_k)\cdot\mu_2(C_{k}) \\
&= \sum_{k\in\N} \mu(B_k\times C_{k}).
\end{align*}
\end{itemize}
Now assume $\mu_1$ and $\mu_2$ are $\sigma$-finite, so there exist sequences $\seq{D_n}_{n\in\N}\in \mathcal{A}_1^\N$ and $\seq{E_n}_{n\in\N}\in \mathcal{A}_2^\N$ of finite measure whose unions are, resp., $\Omega_1$ and $\Omega_2$. Then $\seq{D_n\times E_n}_{n\in \N}$ is a sequence of finite measure whose union is $\Omega_1\times\Omega_2$. This shows $\sigma$-finiteness.

Uniqueness then follows from \ref{sigmaFiniteUniqueExtension}.
\end{proof}

\begin{definition}
The unique measure from \ref{productMeasure} is called the \udef{product measure} and is denoted $\mu_1\otimes \mu_2$.
\end{definition}

\begin{proposition} \label{setSectionMeasurable}
Let $\sSet{\Omega_1, \mathcal{A}_1,\mu_1}$, $\sSet{\Omega_2, \mathcal{A}_2, }$ be measure spaces and $Q\in \mathcal{A}_1\otimes\mathcal{A}_2$. Then
\begin{enumerate}
\item $\setbuilder{\omega\in\Omega_2}{(x,\omega)\in Q} = xQ \in\mathcal{A}_2$ for all $x\in\Omega_1$;
\item $\setbuilder{\omega\in\Omega_1}{(\omega, y)\in Q} = Qy \in\mathcal{A}_1$ for all $y\in\Omega_2$.
\end{enumerate}
\end{proposition}
\begin{proof}
(1) Fix arbitrary $x\in\Omega_1$.

If $Q = A_1\times A_2$ for some $A_1\in\mathcal{A}_1$ and $A_2\in\mathcal{A}_2$, then
\[ xQ = \setbuilder{\omega\in\Omega_2}{(x,\omega) \in A_1\times A_2} = \begin{cases}
\emptyset & (x\notin A_1) \\
A_2 & (x\in A_1)
\end{cases}, \]
which is an element of $\mathcal{A}_2$, so the result follows.

In order to extend the result to the rest of $\mathcal{A}_1\otimes \mathcal{A}_2$, it is enough to prove that the set $\mathcal{B}$ of all $P\subseteq \Omega_1\times \Omega_2$ such that $xP\in\mathcal{A}_2$ is a $\sigma$-algebra

We verify that $\mathcal{B}$ is a $\sigma$-algebra using \ref{setAlgebraCriteria}:
\begin{itemize}
\item We have $\setbuilder{\omega\in\Omega_2}{(x,\omega)\in \Omega_1\times \Omega_2} = \Omega_2 \in\mathcal{A}_2$, so $\Omega_1\times\Omega_2\in\mathcal{B}$.
\item Take $P\in\mathcal{B}$. Then $xP^c = (xP)^c \in \mathcal{A}_2$, so $P^c\in \mathcal{B}$.
\item Take $\seq{P_n}_{n\in\N}\in \mathcal{B}^\N$, then $x\Big(\bigcup_{n\in\N}P_n\Big) = \bigcup_{n\in\N}xP_n \in \mathcal{A}_2$. So $\bigcup_{n\in\N}P_n\in \mathcal{B}$.
\end{itemize}
\end{proof}
\begin{corollary} \label{partialApplicationMeasurable}
Let $\sSet{\Omega_1, \mathcal{A}_1}$, $\sSet{\Omega_2, \mathcal{A}_2}$, $\sSet{\Omega_3, \mathcal{A}_3}$ be measurable sets and $f: \Omega_1\times\Omega_2 \to \Omega_3$ a measurable function. For all $x\in\Omega_1, y\in\Omega_2$,
\begin{enumerate}
\item $f(x,-)$ is $\mathcal{A}_2/\mathcal{A}_3$-measurable;
\item $f(-,y)$ is $\mathcal{A}_1/\mathcal{A}_3$-measurable.
\end{enumerate}
\end{corollary}
\begin{proof}
(1) Take $B\in \mathcal{A}_3$. Then $\big(f(x,-)\big)^{\preimf}(B) = x\big(f^\preimf(B)\big)$. This is an element of $\mathcal{A}_2$ by the proposition and the measurability of $f$.

(2) Similar.
\end{proof}

\section{Measures on topological spaces}
\begin{definition}
Let $\sSet{X, \xi}$ be a topological space. A measure or outer measure such that all sets in the Borel-$\sigma$-algebra are measurable is called a \udef{Borel} measure.
\end{definition}


\begin{definition}
Let $\sSet{X, \xi}$ be a topological space with Borel $\sigma$-algebra $\mathcal{B}$. A measure $\mu: \mathcal{B} \to [0,\infty]$ is called \udef{locally finite} if every $x\in X$ has a neighbourhood $U_x$ such that $\mu(U_x)$ is finite.
\end{definition}

\subsection{Borel regular measures}
\begin{definition}
Let $\sSet{X, \xi}$ be a topological space. A measure or outer measure is \udef{Borel regular} or \udef{outer regular} if
\end{definition}

\subsubsection{Radon measures}
\begin{definition}
Let $\sSet{X, \topology_X}$ be a topological space with associated Borel $\sigma$-algebra $\mathcal{B}$. A \udef{(outer) Radon measure} is a measure $\mu: \mathcal{B}\to [0,\infty]$ such that
\begin{itemize}
\item $\mu$ is locally finite;
\item $\mu(A) = \inf\setbuilder{\mu(U)}{\text{$U\supseteq A$ is open}}$ for all $A\in \mathcal{B}$;
\item $\mu(A) = \sup\setbuilder{\mu(K)}{\text{$K\subseteq A$ is compact}}$ for all \emph{open} $A\in \mathcal{B}$.
\end{itemize}
We say $\mu$ is \udef{outer regular} and \udef{weakly inner regular}.
\end{definition}

TODO: need each compact set has finite measure? (Equivalent to locally finite for locally compact spaces)

TODO: Borel regular measure that gives finite mass to each compact set.


\subsection{Contents and measures on the real line}
\begin{definition}
A subset $I\subseteq \R$ is called \udef{elementary} if it is of one of the following forms:
\begin{itemize}
\item $\interval[c]{a,b}$ for some $a,b\in \overline{\R}$;
\item $\interval[o]{a,b}$ for some $a,b\in \overline{\R}$;
\item $\interval[co]{a,b}$ for some $a,b\in \overline{\R}$;
\item $\interval[oc]{a,b}$ for some $a,b\in \overline{\R}$.
\end{itemize}
We define the \udef{elementary pre-measure} $\lambda$ on the elementary sets by $\lambda(I) = |b-a|$ and $\lambda(\R) = \infty$.

We will call a family of subsets of the real line \udef{admissible} if it contains all elementary sets. We will call a content or measure $\mu$ on the real line \udef{admissible} if it is defined on an admissible family of sets and satisfies $\mu(I) = \lambda(I)$.
\end{definition}

\begin{lemma}
The set of elementary subsets of the real line is a semi-algebra and the elementary pre-measure $\lambda$ is a positive pre-measure on the set of elementary sets.
\end{lemma}
\begin{proof}
Let $\mathcal{E}$ be the family of elementary subsets of the real line. We verify \ref{setAlgebraCriteria}. Take $A,B\in \mathcal{E}$. Let $a_A,b_A$ be the bounds of $A$ and $a_B, b_B$ the bounds of $B$.

Then $A\cap B$ is an interval between $\max\{a_A, a_B\}$ and $\min\{b_A, b_B\}$.

And $A^c$ consists of the disjoint union of the interval from $-\infty$ to $a_A$ and the interval from $b_A$ to $\infty$.

Now we verify that $\lambda$ is a content. Firstly,
\[ \lambda(\emptyset) = \lambda\big(\interval[o]{1,1}\big) = |1-1| = 0. \]
Next suppose $I = I_0\uplus \ldots \uplus I_{k-1}$, where $I,I_0,\ldots, I_{k-1}$ are elementary sets. If $I$ is an interval from $a_0$ to $a_k$, then we can order the other intervals such that $I_m$ is an interval from $a_m$ to $a_{m+1}$ and $a_0 \leq \ldots \leq a_k$.

Then we have
\[ \mu(I) = a_k - a_0 = (a_k - a_{k-1}) + \ldots (a_1 - a_0) = \mu(I_{k-1}) + \ldots + \mu(I_0). \]


TODO extend to $\sigma$-additivity.
\end{proof}

We will be extending this content to much larger algebra's and $\sigma$-algebras in different ways. We can, however, always do the minimal extension to an algebra as detailed in \ref{contentsOnSemiRingToRing}.

From now on we will use ``elementary set'' to be an element of this algebra.

\subsubsection{Translation invariance and uniqueness}
TODO

\subsubsection{The Peano-Jordan content}
\begin{definition}
A subset $A\subseteq \R$ is called \udef{Peano-Jordan measurable} if for all $\epsilon > 0$, there exists a pair of sets elementary $B_\epsilon, C_\epsilon \subseteq \R$ such that
\begin{itemize}
\item $B_\epsilon \subseteq A\subseteq C_\epsilon$;
\item $\lambda(C_\epsilon \setminus B_\epsilon) \leq \epsilon$.
\end{itemize}
The Peano-Jordan content of a Peano-Jordan measurable set $A$ is equal to
\[ \lambda_J(A) \defeq \lim_{\epsilon\to 0}\lambda(B_\epsilon) = \lim_{\epsilon\to 0}\lambda(C_\epsilon). \]
\end{definition}
Notice that $\lambda(C_\epsilon \setminus B_\epsilon)$ is well-defined because $\lambda$ is defined on the whole \emph{algebra} of elementary sets.

\begin{lemma}
Let $A\subseteq \R$ be a Peano-Jordan measurable set. Then the Peano-Jordan content is well-defined, i.e.\ $\lim_{\epsilon\to 0}\lambda(B_\epsilon) = \lim_{\epsilon\to 0}\lambda(C_\epsilon)$.
\end{lemma}
\begin{proof}
If there exists some $\epsilon' >0$ such that $B_{\epsilon'}$ is infinite, then $C_\epsilon$ must be infinite for all $\epsilon >0$. We then need to show that $\lim_{\epsilon\to 0}\lambda(B_\epsilon) = \infty$.

Suppose, towards a contradiction, that $\lambda(B_\epsilon)$ does not tend to infinity. Then there exists a constant $K\geq 0$ such that $\lambda(B_\epsilon) \leq K$ for all $\epsilon >0$. Then we calculate, using \ref{ringPositiveContent},
\[ \lambda(C_\epsilon \setminus B_\epsilon) = \lambda(C_\epsilon) - \lambda(B_\epsilon) \geq \infty - K = \infty > \epsilon. \]
This is a contradiction.

Now assume $\lambda(B_\epsilon)$ is finite for all $\epsilon >0$. Then we can use \ref{ringPositiveContent} to obtain
\[ \lambda(C_\epsilon \setminus B_\epsilon) = \lambda(C_\epsilon) - \lambda(B_\epsilon) \leq \epsilon. \]
We conclude $\lim_{\epsilon\to 0}\lambda(B_\epsilon) = \lim_{\epsilon\to 0}\lambda(C_\epsilon)$ by continuity of subtraction.
\end{proof}

\begin{proposition} \label{supInfPeanoJordanContent}
Take $A\subseteq \R$.
\begin{enumerate}
\item If $A$ is Peano-Jordan measurable, then
\[ \sup_{\substack{B\subseteq A \\ \text{$B$ is elementary}}} \hspace{-1.5em}\lambda(B) \quad = \inf_{\substack{A\subseteq C \\ \text{$C$ is elementary}}} \hspace{-1.5em}\lambda(C) = \lambda_J(A). \]
\item If
\[ \sup_{\substack{B\subseteq A \\ \text{$B$ is elementary}}} \hspace{-1.5em}\lambda(B) \quad = \inf_{\substack{A\subseteq C \\ \text{$C$ is elementary}}} \hspace{-1.5em}\lambda(C) \]
and this quantity is finite, then $A$ is Peano-Jordan measurable.
\end{enumerate}
\end{proposition}
The condition
\[ \sup_{\substack{B\subseteq A \\ \text{$B$ is elementary}}} \hspace{-1.5em}\lambda(B) \quad = \inf_{\substack{A\subseteq C \\ \text{$C$ is elementary}}} \hspace{-1.5em}\lambda(C) \]
is sometimes taken as the (inequivalent) definition of Peano-Jordan measurable. One problem with this is that the Peano-Jordan measurable sets would not form an algebra (see example).
\begin{proof}
(1) By construction it is clear that
\[ \sup_{\substack{B\subseteq A \\ \text{$B$ is elementary}}} \hspace{-1.5em}\lambda(B) \quad \leq \inf_{\substack{A\subseteq C \\ \text{$C$ is elementary}}} \hspace{-1.5em}\lambda(C). \]
Also
\[ \{B_\epsilon\}_{\epsilon > 0} \subseteq \setbuilder{B}{\text{$B\subseteq A$ and $B$ is elementary}} \quad\text{and}\quad \{C_\epsilon\}_{\epsilon > 0} \subseteq \setbuilder{C}{\text{$A\subseteq C$ and $C$ is elementary}}. \]
By compatibility of the order with the convergence, we have
\[ \inf_{\substack{A\subseteq C \\ \text{$C$ is elementary}}} \hspace{-1.5em}\lambda(C) \quad \leq \quad \lim_{\epsilon\to 0}\lambda(C_\epsilon) = \lambda_J(A) = \lim_{\epsilon\to 0}\lambda(B_\epsilon) \quad \leq \sup_{\substack{B\subseteq A \\ \text{$B$ is elementary}}} \hspace{-1.5em}\lambda(B), \]
which proves the assertions.

(2) Set
\[ x = \sup_{\substack{B\subseteq A \\ \text{$B$ is elementary}}} \hspace{-1.5em}\lambda(B) \quad = \inf_{\substack{A\subseteq C \\ \text{$C$ is elementary}}} \hspace{-1.5em}\lambda(C) \]
and take $\epsilon >0$ arbitrary.

We can find an elementary $B_\epsilon \subseteq A$ such that $\lambda(B_\epsilon) \geq x-\epsilon/2$ and $C_\epsilon \supseteq A$ such that $\lambda(C_\epsilon) \leq x+\epsilon/2$. These sets satisfy the definition.
\end{proof}

\begin{example}
\begin{itemize}
\item There are Jordan measurable sets that are not elementary. For example, $\setbuilder{n^{-1}}{n\in (1:)} \cup \{0\}$ is Jordan measurable with Jordan content $0$.

Indeed, take $B_\epsilon = \{0\}$ for all $\epsilon$ and $C_\epsilon = \interval{0,\epsilon} \cup\bigcup_{n=1}^{\floor{\epsilon^{-1}}} \{n^{-1}\}$.
\item The rationals $\Q$ is not Jordan measurable. Indeed all elementary subsets of $\Q$ have zero content and all elementary supersets of $\Q$ have content $\infty$.
\item Point two of \ref{supInfPeanoJordanContent} does not hold if the quantity is infinite. For example, take $A = \Q \cup \interval{0,+\infty}$. This also shows that if we took the condition in \ref{supInfPeanoJordanContent} to be the definition of Peano-Jordan measurability, then the Peano-Jordan measurable sets would not form an algebra.
\end{itemize}
\end{example}

\begin{lemma} \label{PeanoJordanAlgebra}
The set of Peano-Jordan measurable sets is an algebra, which is admissible. The Peano-Jordan content is a content. It is also admissible.
\end{lemma}
\begin{proof}
Admissibility is clear. We need to show that the set of Jordan measurable sets forms an algebra.

We prove this from the fact that elementary sets form an algebra, using \ref{setAlgebraCriteria}.

\begin{itemize}
\item Clearly $\R\in\mathcal{A}$.
\item Let $A$ be Jordan measurable. Then there exist families $B_\epsilon^A$ and $C_\epsilon^{A}$ of elementary sets such that $B_\epsilon^A \subseteq A \subseteq C_\epsilon^{A}$. Then $(B_\epsilon^A)^c$ and $(C_\epsilon^{A})^c$ are also families of elementary sets and $(B_\epsilon^A)^c \subseteq A^c \subseteq (C_\epsilon^{A})^c$. By \ref{setDifferenceComplement}, we have $(B_\epsilon^A)^c \setminus (C_\epsilon^{A})^c = B_\epsilon^A \setminus C_\epsilon^{A}$. This shows that $A^c$ is Jordan measurable.
\item Let $B$ be Jordan measurable. Then there exist families $B_\epsilon^B$ and $C_\epsilon^{B}$ of elementary sets such that $B_\epsilon^B \subseteq B \subseteq C_\epsilon^{B}$. We have
\[ B_{\epsilon/2}^A \cup B_{\epsilon/2}^B \subseteq A\cup B \subseteq C_{\epsilon/2}^A \cup C_{\epsilon/2}^B. \]
Then $\big(C_{\epsilon/2}^A \cup C_{\epsilon/2}^B\big) \setminus \big(B_{\epsilon/2}^A \cup B_{\epsilon/2}^B\big) \subseteq \big(C_{\epsilon/2}^A \setminus B_{\epsilon/2}^A\big) \cup \big(C_{\epsilon/2}^B \setminus B_{\epsilon/2}^B\big)$, so
\begin{align*}
\lambda\Big(\big(C_{\epsilon/2}^A \cup C_{\epsilon/2}^B\big) \setminus \big(B_{\epsilon/2}^A \cup B_{\epsilon/2}^B\big)\Big) &\leq \lambda\big(C_{\epsilon/2}^A \setminus B_{\epsilon/2}^A\big) + \lambda\big(C_{\epsilon/2}^B \setminus B_{\epsilon/2}^B\big) \\
&= \frac{\epsilon}{2} + \frac{\epsilon}{2} = \epsilon,
\end{align*}
by \ref{semiringPositiveContent}.
\end{itemize}

Finally we show that the Peano-Jordan content is a content.
\begin{itemize}
\item The Peano-Jordan content of $\emptyset$ is $0$ by admissibility and the fact that the elementary pre-measure is a pre-measure.
\item Now take disjoint Peano-Jordan measurable sets $A,B$ and families $B_\epsilon^A, C_\epsilon^A, B_\epsilon^B, C_\epsilon^B$ of elementary sets that witness the Peano-Jordan measurability. Clearly $B_\epsilon^A$ and $B_\epsilon^B$ are disjoint.

As before, we may take $B_\epsilon^A\uplus B_\epsilon^B$ and $C_\epsilon^A \cup C_\epsilon^B$ to witness the Peano-Jordan measurability of $A\uplus B$. We can then calculate the Peano-Jordan content as
\begin{align*}
\lambda_J(A\uplus B) &= \lim_{\epsilon \to 0}\lambda(B_\epsilon^A\uplus B_\epsilon^B) \\
&= \lim_{\epsilon \to 0}\lambda(B_\epsilon^A) + \lambda(B_\epsilon^B) \\
&= \lim_{\epsilon \to 0}\lambda(B_\epsilon^A) + \lim_{\epsilon \to 0}\lambda(B_\epsilon^B) \\
&= \lambda_J(A) + \lambda_J(B).
\end{align*}
\end{itemize}
\end{proof}

\begin{proposition}
Let $\seq{A_n}_{n\in \N}$ be a sequence of disjoint Peano-Jordan measurable sets such that $\biguplus_{n\in\N}A_n$ is Peano-Jordan measurable. Then
\[ \mu\Big(\biguplus_{n\in\N}A_n\Big) = \sum_{n\in\N}\mu(A_n). \]
\end{proposition}
\begin{proof}
Given \ref{ringPositiveContent}, we only need to show
\[ \mu\Big(\biguplus_{n\in\N}A_n\Big) \leq \sum_{n\in\N}\mu(A_n). \]

TODO Tannery
\end{proof}

\subsubsection{The Lebesgue measure}
\begin{definition}
Define $\mathfrak{I}$ as the subset of sequences $\seq{(a_n,b_n)}$ in $(\overline{\R}\times\overline{\R})^\N$ such that
\[ \ldots < b_{n-1}< a_n < b_n < a_{n+1} < \ldots \]

The \udef{Lebesgue outer measure} on $\R$, denoted by $\lambda^*$ is defined by
\[ \lambda^*(A) = \inf\setbuilder{\sum_{n\in \N}(b_n-a_n)}{\text{$\seq{(a_n,b_n)}\in \mathfrak{I}$ such that $A \subseteq \bigcup_{n\in\N} \interval[o]{a_n, b_n}$}}. \]

The set of $\lambda^*$-measurable sets is called the \udef{Lebesgue $\sigma$-algebra} and the associated measure $\lambda$ is called the \udef{Lebegue measure}.
\end{definition}
We say $\seq{(a_n,b_n)}\in \mathfrak{I}$ covers $A$ if $A \subseteq \bigcup_{n\in\N} \interval[o]{a_n, b_n}$.

\begin{lemma}
The Lebesgue outer measure on $\R$ is an outer measure on $\R$ and for all $a
\leq b$
\[ \lambda^*([a,b]) = \lambda^*(]a,b]) = \lambda^*([a,b[) = \lambda^*(]a,b[) = b-a. \]
The Lebesgue outer measure of any unbounded interval is $+\infty$.
\end{lemma}
\begin{proof}
We verify the three points in the definition of an outer measure:
\begin{itemize}
\item We have that $\emptyset \subseteq ]0,\epsilon[$ for all $\epsilon > 0$, so  $0 = \inf\setbuilder{\epsilon}{\epsilon > 0} \geq \lambda^*(\emptyset) \geq 0$.
\item If $A\subseteq B$, then any cover of $B$ is a cover of $A$.
\item Let $\seq{A_n}$ be an arbitrary sequence in $\powerset(\Omega)$. If $\sum_{n\in\N}\lambda^*(A_n) = +\infty$, then subadditivity is necessarily satisfied. Now assume $\sum_{n\in\N}\lambda^*(A_n)$ is finite, then each $\lambda^*(A_n)$ is finite. Fix an arbitrary constant $\epsilon > 0$. We can find a $\seq{(a_{n,k},b_{n,k})}_{k\in\N}\in \mathfrak{I}$ that covers $A_n$ and is such that
\[ \sum_{k=0}^\infty(b_{n,k} - a_{n,k}) < \lambda^*(A_n) + \epsilon/2^n. \]
Now $\bigcup_{n\in \N}A_n$ is covered by $\seq{(a_{n,k},b_{n,k})}_{n,k\in\N}$ and
\[ \sum_{n,k \in\N} (b_{n,k} - a_{n,k}) < \sum_{n\in \N}(\lambda^*(A_n) + \epsilon/2^n) = \sum_{n\in \N}\lambda^*(A_n) + \epsilon. \]
Thus $\lambda^*\left(\bigcup_{n\in \N}A_n\right) \leq \sum_{n,k \in\N} (b_{n,k} - a_{n,k}) < \sum_{n\in \N}\lambda^*(A_n) + \epsilon$ for all $\epsilon > 0$. Thus $\lambda^*\left(\bigcup_{n\in \N}A_n\right) \leq \sum_{n\in \N}\lambda^*(A_n)$.
\end{itemize}
It is clear that $\lambda^*(]a,b[) \leq b-a$. For the converse, it is enough to notice that any sequence covering $]a,b[$ must contain a term $(c,d)$ with $c\leq a$ and $d\geq b$. 

We now show $\lambda^*([a,b]) = b-a$. Considering covers of the form $(a-\epsilon, b+\epsilon)$ for arbitrary $\epsilon > 0$, gives $\lambda^*([a,b]) \leq b-a$. For the reverse inequality, we have $b-a = \lambda^*(]a,b[) \leq \lambda^*([a,b])$.

The results for the halfopen intervals follow because 
\[ b-a = \lambda^*([a,b]) \geq \lambda^*(]a,b]) \leq \lambda^*(]a,b[) = b-a \]
and similarly for $\lambda^*([a,b[)$.

The reasoning for unbounded intervals is similar.
\end{proof}
\begin{corollary}
The Lebesgue measure is a measure.
\end{corollary}

\begin{proposition}
Let $\mathcal{B}$ be the Borel $\sigma$-algebra on $\R$ and $\mathcal{A}_{\lambda^*}$ the Lebesgue $\sigma$-algebra.

Then $\sSet{\R, \mathcal{A}_{\lambda^*}, \lambda}$ is the completion of $\sSet{\R, \mathcal{B}, \lambda}$.
\end{proposition}
\begin{proof}
TODO
\end{proof}

\begin{example}
Cantor set: $\sSet{\R, \mathcal{B}, \lambda}$ is not complete.
\end{example}

\begin{proposition}
The Lebesgue measure (restricted to the Borel $\sigma$-algebra) is the only measure $\mu$ on the Borel $\sigma$-algebra such that $\mu([a,b]) = b-a$.
\end{proposition}
\begin{proof}
TODO
\end{proof}

\begin{proposition}
The Lebesgue measure is regular.
\end{proposition}
\begin{proof}
TODO
\end{proof}

\subsection{Pointwise algebras}
\begin{lemma} \label{equalAECongruence}
Let $\sSet{X, \mathcal{X}, \mu}$ be a measure space and $Y$ an $\Omega$-structure. Then the relation $\overset{\text{a.e.}}{=}$ on $(X\to Y)$ is a congruence on the pointwise algebra $(X\to Y)$.
\end{lemma}
\begin{proof}
Take $\omega\in\Omega$ and $x = \seq{(f_i,g_i)}_{i\in(0:\alpha(\omega))}$ a sequence in $(X\to Y)^2$ such that $f_i\mathrel{\overset{\text{a.e.}}{=}} g_i$ for all $i\in \big(0:\alpha(\omega)\big)$. Set $N_i \defeq \setbuilder{x\in X}{f_i(x)\neq g_i(x)}$. This is a null set for all $i\in\big(0:\alpha(\omega)\big)$.

We need to show that $\omega_{X\to Y}(\seq{f_i}_{i\in(0:\alpha(\omega))})\mathrel{\overset{\text{a.e.}}{=}}\omega_{X\to Y}(\seq{f_i}_{i\in(0:\alpha(\omega))})$. Let $N$ be the set $\setbuilder{x\in X}{\omega_{X\to Y}(\seq{f_i(x)}_{i\in(0:\alpha(\omega))}) \neq \omega_{X\to Y}(\seq{g_i(x)}_{i\in(0:\alpha(\omega))})}$.

Clearly $N\subseteq \bigcup_{i\in(0:\alpha(\omega))}N_i$. So $N$ is a null set by subadditivity.
\end{proof}

\section{Lebesgue integration}
\url{https://math.stackexchange.com/questions/2218114/theoretical-advantages-of-lebesgue-integration}
\url{https://math.stackexchange.com/questions/3202630/what-are-the-advantages-of-the-riemann-vs-lebesgue-integral}

We want to define an integral functional $I(f)$.

Riemann: $I(f) = \lim_n I_n(f)$

Lebesgue $I(f) = \lim_n I(f_n)$.


\subsection{Simple functions}
TODO order on function spaces

\begin{definition}
Let $\Omega$ be a set. A \udef{simple function} (or \udef{step function}) on $\Omega$ is a function with a range of finite cardinality.

Let $B$ be a set. We denote the subset of simple functions in $(\Omega\to B)$ by $\SF(\Omega,B)$
\end{definition}

TODO: should $\SF(\Omega, B)$ mean measurable simple function??

\begin{lemma}
Let $s:\Omega\to Y$ be a simple function into a vector space. Then $s$ can be written as
\[ s(x) = \sum_{\lambda \in s[\Omega]}\lambda\cdot \chi_{s^{-1}[\lambda]}(x). \]
Additionally for some $k\in\N$ we can write $s[\Omega] = \bigcup_{i=1}^k\{\lambda_i\}$ and $A_i = s^{-1}[\lambda_i]$. Then the $A_i$ form a partition of $\Omega$ and
\[ s(x) = \sum_{i=1}^k\lambda_i\cdot \chi_{A_i}(x). \]
This is known as the \udef{canonical form} of $s$. Conversely every function of this form is a simple function.
\end{lemma}

\begin{lemma}
Let $(\Omega, \mathcal{A})$ and $(Y, \mathcal{B})$ be measurable spaces such that $\mathcal{B}$ contains all singleton sets. Let $s:\Omega\to Y$ be a simple function.

Then $s$ is measurable \textup{if and only if} $s^{-1}[\lambda]\in\mathcal{A}$ for all $\lambda\in s[\Omega]$.

This is equivalent to saying the partition $\{A_i\}_{i=1}^k$ is a subset of $\mathcal{A}$.
\end{lemma}
\begin{proof}
The direction $\boxed{\Rightarrow}$ is clear, since $\mathcal{B}$ is assumed to contain all singleton sets.

For the $\boxed{\Leftarrow}$ direction, let $B\in \mathcal{B}$. Then
\[ s^{-1}[B] = \bigcup_{\lambda\in s[\Omega]}s^{-1}[B\cap \{\lambda\}], \]
which is a finite union of measurable sets.
\end{proof}

\begin{lemma}
Let $\Omega$ be a set, $G$ an abelian group and $s,t\in\SF(\Omega,G)$. If $s$ and $t$ have canonical forms
\[ s = \sum_{i=1}^k a_i\cdot\chi_{A_i} \qquad\text{and}\qquad t = \sum_{j=1}^l b_j\cdot\chi_{B_j}, \]
then $s+t$ has canonical form
\[ s+t = \sum_{i=1}^k\sum_{j=1}^l (a_i+b_j)\chi_{A_i}\chi_{B_j} = \sum_{i,j \in (1:k)\times(1:l)}(a_i+b_j)\chi_{A_i\cap B_j}. \]
\end{lemma}


\begin{proposition} \label{measurableFunctionPointwiseLimitSimpleFunctions}
Let $(\Omega, \mathcal{A}, \mu)$ be a measure space and let $f:\Omega\to[0,+\infty]$ be a positive function. Then
\begin{enumerate}
\item there exists an increasing sequence of positive step functions $\seq{s_n}$ that converges point-wise to $f$;
\item if $f$ is measurable, then $\seq{s_n}$ can be taken to be a sequence of measurable simple functions.
\end{enumerate}
\end{proposition}
The converse is also true and is given by \ref{pointWiseConvergenceMeasurable}.
\begin{proof}
(1) If we can find an increasing sequence of positive step functions $\seq{t_n}$ that converges point-wise to $\id:[0,+\infty]\to[0,+\infty]$, then
\[ f = \id\circ f = \lim_{n\to\infty} t_n\circ f = \sup_{n\in\N}(t_n\circ f) \]
and so $s_n = t_n\circ f$ gives the sequence we are looking for. And we can find such a sequence $(t_n)$. For example
\[ t_n = n\charFunc{\interval[co]{n,+\infty}}+\sum_{k=1}^{n2^n}\frac{k-1}{2^n}\charFunc{\interval[co]{\frac{k-1}{2^n},\frac{k}{2^n}}}. \]

(2) We note that each $t_n$ is measureable, so $s_n = t_n \circ f$ is measurable if $f$ is measurable.
\end{proof}


\subsubsection{Integration of simple functions}
\begin{definition}
Let $(\Omega, \mathcal{A}, \mu)$ be a measure space and $Y$ a vector space. Let
\[ s:\Omega \to Y: x\mapsto \sum_{\lambda \in \im(s)}\lambda\cdot \chi_{s^{-1}[\lambda]}(x) = \sum_{i=1}^k\lambda_i\cdot\chi_{A_i}(x)  \]
be a measurable simple function. We define the \udef{integral} of $s$ over $\Omega$ w.r.t. $\mu$ as
\[ \int_\Omega s\diff{\mu} \defeq \sum_{\lambda \in \im(s)}\lambda\cdot \mu(s^{-1}[\lambda]) = \sum_{i=1}^k\mu(A_i)\cdot\lambda_j. \]
Using the convention that $0\times \infty = 0$. (TODO: clarify)

We call $s$ \udef{integrable} if the integral is finite (TODO clarify + below).

For any $E\in\mathcal{A}$ we also define the integral some measurable simple function $s:\Omega\to Y$ over $E$ by
\[ \int_E s \diff{\mu} \defeq \int_E s|_E \diff{\mu|_E}, \]
where the last integral is taken over the measure space $(E,\mathcal{A}',\mu|_{\mathcal{A}'})$ as defined in \ref{submeasurespace}.
\end{definition}

The integral can be seen as a map $\SF(\Omega,Y)\to \overline{Y}$, where $\overline{Y}$ is the Dedekind-MacNeille completion (TODO!).

\begin{lemma} \label{measureFromIntegralCharacteristicFunctions}
Let $(\Omega, \mathcal{A}, \mu)$ be a measure space and $E\in\mathcal{A}$. Then
\[ \mu(E) = \int_\Omega \chi_E \diff{\mu}. \]
\end{lemma}

\begin{lemma} \label{simpleIntegralOverSubset}
Let $(\Omega, \mathcal{A}, \mu)$ be a measure space, $Y$ a vector space over $\F$, $E\in\mathcal{A}$ and $s:\Omega\to Y$ a measurable simple function. Then
\[ \int_E s \diff{\mu} = \int_\Omega s\cdot\chi_{E} \diff{\mu}. \]
\end{lemma}
\begin{proof}
We have
\begin{align*}
\int_E s \diff{\mu} &= \int_E s|_E \diff{\mu|_E}\\
&= \sum_{\lambda \in \im(s|_E)}\lambda\cdot \mu|_E\big(s|_E^{\preimf}(\lambda)\big) \\
&= \sum_{\lambda \in \im(s|_E)}\lambda\cdot \mu\big(s^{\preimf}(\lambda)\cap E\big) \\
&= 0 + \sum_{\lambda \in \im(s|_E)\setminus\{0\}}\lambda\cdot \mu\big(s^{\preimf}(\lambda)\cap E\big) \\
&= 0 + \sum_{\lambda \in \im(s\cdot \chi_E)\setminus\{0\}}\lambda\cdot \mu\big((s\cdot \chi_E)^{\preimf}(\lambda)\big) \\
&= 0\cdot \mu\big((s\cdot \chi_E)^{\preimf}(0)\big) + \sum_{\lambda \in \im(s\cdot \chi_E)\setminus\{0\}}\lambda\cdot \mu\big((s\cdot \chi_E)^{\preimf}(\lambda)\big) \\
&= \sum_{\lambda \in \im(s\cdot \chi_E)}\lambda\cdot \mu\big((s\cdot \chi_E)^{\preimf}(\lambda)\big) \\
&= \int_\Omega s\cdot\chi_E \diff{\mu}.
\end{align*}
\end{proof}


\begin{proposition} \label{integrationLinear} \label{integrationOrderPreserving}
Let $(\Omega, \mathcal{A}, \mu)$ be a measure space and $Y$ a vector space over $\F$. Then
\begin{enumerate}
\item the integral is linear: $\forall c\in \F$ and $\forall s,t\in\SF(\Omega, Y)$:
\[ \int_\Omega (c\cdot s + t)\diff{\mu} = c\cdot \int_\Omega s\diff{\mu} + \int_\Omega t\diff{\mu}. \]
\end{enumerate}
If $Y$ is a normed space, then
\begin{enumerate} \setcounter{enumi}{1}
\item for all $s\in\SF(\Omega, Y)$:
\[ \int_\Omega \norm{s}\diff{\mu} \leq \norm{\int_\Omega s \diff{\mu}}; \]
\item if $E_1\subseteq E_2$ are events in $\mathcal{A}$, then
\[ \int_{E_1}\norm{s}\diff{\mu} \leq \int_{E_2}\norm{s}\diff{\mu}. \]
\end{enumerate} \setcounter{enumi}{3}
If $Y$ is an ordered space, then
\begin{enumerate}
\item if $s(x)\leq t(x)$ for all $x\in\Omega$, then
\[ \int_\Omega s\diff{\mu} \leq \int_\Omega t\diff{\mu}. \]
\end{enumerate}
\end{proposition}
\begin{proof}
(1) Let $s,t$ have canonical forms
\[ s = \sum_{i=1}^k a_i\cdot\chi_{A_i} \qquad\text{and}\qquad t = \sum_{i=1}^k b_i\cdot\chi_{B_i}. \]
Then we calculate
\begin{align*}
\int_\Omega (c\cdot s + t)\diff{\mu} &= \sum_{i=1}^k\sum_{j=1}^l\mu(A_i\cap B_j)\cdot(c a_i + b_j) \\
&= c\sum_{i=1}^k\sum_{j=1}^l\mu(A_i\cap B_j)\cdot a_i +  \sum_{i=1}^k\sum_{j=1}^l\mu(A_i\cap B_j) \cdot b_j \\
&= c\sum_{i=1}^k\mu(A_i)\cdot a_i +  \sum_{j=1}^l\mu(B_j) \cdot b_j \\
&= c\int_\Omega s\diff{\mu} + \int_\Omega t\diff{\mu}.
\end{align*}

The property (2) is just the triangle inequality. The other properties can be proven in a similar fashion to (1).
\end{proof}

\begin{proposition}
Let $(\Omega, \mathcal{A}, \mu)$ be a measure space and $s:\Omega\to [0,+\infty[$ a measurable simple function. The map
\[ \nu:\mathcal{A}\to [0,+\infty]: E\mapsto \int_E s\diff{\mu} = \int_\Omega s\cdot \chi_E\diff{\mu} \]
defines a measure.
\end{proposition}
\begin{proof}
From the positive linearity of both measures and integration (\ref{measuresPositiveLinear}, \ref{integrationLinear}) it is enough to consider $s = \chi_A$ for some $A\in\mathcal{A}$. In this case
\[ \nu(E) = \int_\Omega\chi_A\chi_B \diff{\mu} = \int_\Omega\chi_{A\cap B} \diff{\mu} = \mu(A\cap E). \]
It is clear that $\nu(\emptyset) = 0$. For $\sigma$-additivity, let $(E_n)$ be a sequence of disjoint sets in $\mathcal{A}$ and calculate
\[ \nu\left(\biguplus_{n\in\N}E_n\right) = \mu\left(A\cap \biguplus_{n\in\N}E_n\right) = \mu\left(\biguplus_{n\in\N}(A\cap E_n)\right) = \sum_{n\in\N}\mu(A\cap E_n) = \sum_{n\in\N}\nu(E_n). \]
\end{proof}
\begin{corollary} \label{integralContinuousInDomain}
Let $(\Omega, \mathcal{A}, \mu)$ be a measure space, $s:\Omega\to[0,+\infty]$a measurable simple function and $(E_n)$ a converging sequence in $\mathcal{A}$. Then
\[ \lim_{n\to\infty}\int_{E_n}s\diff{\mu} = \int_{\lim_{n\to\infty} E_n}s\diff{\mu}. \]
\end{corollary}
\begin{proof}
Define the measure $\nu: E\mapsto \int_E s\diff{\mu}$. Then by TODO ref
\[ \lim_{n\to\infty}\int_{E_n}s\diff{\mu} = \lim_{n\to\infty}\nu(E_n) = \nu(\lim_{n\to\infty}E_n) =\int_{\lim_{n\to\infty}E_n}s\diff{\mu}. \]
\end{proof}

\subsection{Positive real functions}
\begin{definition}
Let $(\Omega, \mathcal{A}, \mu)$ be a measure space and let $f:\Omega\to[0,+\infty]$ be a positive measurable function. Define

We define the \udef{Lebesgue integral} of $f$ on $\Omega$ w.r.t. $\mu$ as
\[ \int_\Omega f \diff{\mu} \defeq \sup\setbuilder{\int_\Omega s \diff{\mu}}{s\in\SF(\Omega, \interval[co]{0,+\infty})\;\land\; s\leq f}. \]
We call $f$ \udef{integrable} when $\int_\Omega f \diff{\mu} < \infty$.
\end{definition}
For simple functions this definition corresponds to the previous one by \ref{integrationOrderPreserving}.

This definition a priori makes sense even when $f$ is not assumed to be measurable. However the integral has undesirable properties in this case, such as not being additive.

\begin{example}
If $\delta_x$ is the Dirac measure associated to a point $x\in\Omega$ in a measurable space, then
\[ \int_\Omega f \diff{\delta_x} = f(x). \]
\end{example}

\begin{lemma} \label{integralOverSubset}
Let $(\Omega, \mathcal{A}, \mu)$ be a measure space, $E\in\mathcal{A}$ and $f:\Omega\to \R^+$ a measurable function. Then
\[ \int_E f \diff{\mu} = \int_\Omega f\cdot\chi_{E} \diff{\mu}. \]
\end{lemma}
\begin{proof}
TODO follows from \ref{simpleIntegralOverSubset}
\end{proof}
\begin{corollary}
Let $(\Omega, \mathcal{A}, \mu)$ be a measure space, $E\in\mathcal{A}$ and $f:\Omega\to \R^+$ a measurable function. Then
\begin{enumerate}
\item $\int_\Omega f\diff{\mu} = \int_{\Omega\setminus E} f\diff{\mu}+\int_E f\diff{\mu}$.
\end{enumerate}
\end{corollary}

\begin{proposition} \label{propertiesIntegralPositiveFunctions}
Let $(\Omega, \mathcal{A}, \mu)$ be a measure space and let $f,g:\Omega\to[0,+\infty]$ be positive measurable functions. Then
\begin{enumerate}
\item if $f\leq g$, then $\int_\Omega f\diff{\mu} \leq \int_\Omega g\diff{\mu}$;
\item if $E_1\subseteq E_2$ are events in $\mathcal{A}$, then $\int_{E_1}f\diff{\mu} \leq \int_{E_2}f\diff{\mu}$;
\item \textup{(Beppo Levi's lemma)} if $(f_n)$  is an increasing sequence of positive functions that converges to $f$ point-wise, then $(\int_\Omega f_n\diff{\mu})_n$ is an increasing sequence and
\[ \lim_{n\to\infty}\int_\Omega f_n\diff{\mu} = \int_\Omega \lim_{n\to\infty}f_n\diff{\mu} = \int_\Omega f\diff{\mu}; \]
\item the integral is positive linear: $\forall c\geq 0$:
\[ \int_\Omega(cf+g)\diff{\mu} = c\int_\Omega f\diff{\mu} + \int_\Omega g\diff{\mu}; \]
\item if $\seq{E_n}_{n\in \N}$ is a sequence of pairwise disjoint events in $\mathcal{A}$, then
\[ \int_{\biguplus_{n\in \N}E_n}f\diff{\mu} = \sum_{n\in \N}\int_{E_n}f\diff{\mu}. \]
\end{enumerate}
\end{proposition}
\begin{proof}
(1) For all $s\in \SF(\Omega, \interval[co]{0,+\infty})$ we have that $s\leq f$ implies $s\leq g$.

(2) This follows from $f\cdot\chi_{E_1}\leq f\cdot\chi_{E_2}$ and \ref{simpleIntegralOverSubset}.

(3) That the sequence $(\int_\Omega f_n\diff{\mu})_n$ is increasing follows from point 1. For increasing sequences the limits are suprema, by monotone convergence \ref{sequenceMonotoneConvergence}. Also, for all $m\in\N$, we have $f_m\leq\sup_{n\in\N}f_n$, which implies, by point 1., that $\int_\Omega f_m\diff{\mu}\leq \int_\Omega \sup_{n\in\N}f_n\diff{\mu}$. So
\[ \lim_{n\to\infty}\int_\Omega f_n\diff{\mu} = \sup_{n\in\N}\int_\Omega f_n\diff{\mu} \leq \int_\Omega \sup_{n\in\N}f_n\diff{\mu} = \int_\Omega \lim_{n\to\infty}f_n\diff{\mu}. \]
For the other inequality, it is enough to prove that $c\int_\Omega s\diff{\mu} \leq \lim_{n\to\infty}\int_\Omega f_n\diff{\mu}$ for all $0<c<1$ and $s\in\SF(\Omega,[0,+\infty[)$ such that $s\leq f$. Fix such a $c$ and $s = \sum_{i=1}^k\lambda_i\chi_{A_i}$. Consider the sets
\[ E_n = \setbuilder{x\in\Omega}{cs(x)\leq f_n(x)} = \bigcup_{i=1}^k \left(f^{-1}_n[\,[c\lambda_i, +\infty]\,]\cap A_i\right). \]
Then $(E_n)$ is an increasing sequence in $\mathcal{A}$ with $\Omega= \bigcup_{n\in\N}E_n$ and
\[ c\int_\Omega s\diff{\mu} = \int_{\bigcup_n E_n} cs\diff{\mu} = \lim_{n\to\infty}\int_{E_n} cs\diff{\mu} \]
by \ref{integralContinuousInDomain}. Also 
\[ \int_{E_n} cs\diff{\mu} \leq \int_{E_n} f_n\diff{\mu} \leq \int_{\Omega} f_n\diff{\mu} \]
by the previous points and so the result follows from the fact that limits preserve inequalities, \ref{limitPreservesInequality}.

(4) Take sequences $(s_n)$ and $(t_n)$ of positive measurable step functions
that increase pointwise to $f$ and $g$, respectively. Then $cs_n+t_n$ converges pointwise to $cf+g$ by the linearity of the limit and
\begin{align*}
\int_\Omega(cf+g)\diff{\mu} &= \lim_{n\to\infty}\int_\Omega(cs_n+t_n)\diff{\mu} \\
&= \lim_{n\to\infty}\left(c\int_\Omega s_n\diff{\mu}+\int_\Omega t_n\diff{\mu}\right) \\
&= c\lim_{n\to\infty}\int_\Omega s_n\diff{\mu}+\lim_{n\to\infty}\int_\Omega t_n\diff{\mu} \\
&= c\int_\Omega f\diff{\mu} + \int_\Omega g\diff{\mu}.
\end{align*}

(5) We calculate
\begin{align*}
\sum_{n\in \N}\int_{E_n}f\diff{\mu} &= \lim_{n\to \infty} \sum_{i=0}^n \int_{E_i}f \diff{\mu} \\
&= \lim_{n\to \infty} \sum_{i=0}^n \int_{\Omega}f\cdot\charFunc{E_i} \diff{\mu} \\
&= \lim_{n\to \infty} \int_{\Omega}f\cdot \sum_{i=0}^n\charFunc{E_i} \diff{\mu} \\
&= \lim_{n\to \infty} \int_{\Omega}f\cdot \charFunc{\biguplus_{i=0}^nE_i} \diff{\mu} \\
&= \int_{\Omega}f\cdot \lim_{n\to \infty}\charFunc{\biguplus_{i=0}^nE_i} \diff{\mu} \\
&= \int_{\Omega}f\cdot \charFunc{\biguplus_{n\in \N}E_n} \diff{\mu} \\
&= \int_{\biguplus_{n\in \N}E_n}f \diff{\mu},
\end{align*}
using \ref{integralOverSubset}, the previous parts of this proposition and \ref{indicatorFunctionHomeomorphism}.
\end{proof}

\begin{proposition}[Fatou's lemma] \label{FatouLemma}
Let $\sSet{\Omega, \mathcal{A}, \mu}$ be a measure space and $\seq{h_n}$ any sequence of positive measurable functions in $(\Omega\to[0,+\infty])$. Then
\[ f: \Omega\to[0,+\infty]: x\mapsto \liminf_{n\to\infty}h_n(x) \]
is measurable and
\[ \int_\Omega \liminf_{n\to\infty}h_n\diff{\mu} \leq \liminf_{n\to\infty}\int_\Omega h_n\diff{\mu}. \]
\end{proposition}
\begin{proof}
Consider the sequence $\seq{f_n}$ defined by $f_n(x) = \inf_{k\geq n}h_k(x)$. This is an increasing sequence of positive functions and $f_n \leq h_n$ for all $n\in\N$. Each $f_n$ is measurable because
\[ f_n^{-1}[\,[t,+\infty]\,] = \bigcap_{k=n}^\infty h_k^{-1}[\,[t,+\infty]\,]. \]
This shows that $f$ is measurable by \ref{pointWiseConvergenceMeasurable}. Then we can use Beppo Levi's lemma \ref{propertiesIntegralPositiveFunctions} to obtain
\[ \int_\Omega\liminf_{n\to\infty}h_n\diff{\mu} = \lim_{n\to\infty}\int_\Omega f_n\diff{\mu} = \liminf_{n\to\infty}\int_\Omega f_n\diff{\mu} \leq \liminf_{n\to\infty}\int_\Omega h_n\diff{\mu} \]
using monotonicity of the integral \ref{propertiesIntegralPositiveFunctions} and liminf \ref{propertiesIntegralPositiveFunctions} for the last inequality.
\end{proof}

\begin{proposition} \label{functionPropertiesFromIntegral}
Let $\seq{\Omega, \mathcal{A}, \mu}$ be a measure space and $f:\Omega\to[0,+\infty]$ a positive measurable function. Then
\begin{enumerate}
\item $\int_\Omega f\diff{\mu} = 0$ \textup{if and only if} $f(x) = 0$ a.e.;
\item if $f$ is integrable, then $f(x)< +\infty$ a.e.\
\end{enumerate}
\end{proposition}
\begin{proof}(1) Set $E = \setbuilder{x\in\R}{f(x) \neq 0}$. Then $f(x) = 0$ a.e.\ is equivalent to $\mu(E) = 0$.

Assume $\mu(E) = 0$, then 
\[ \int_\Omega f\diff{\mu} = \int_{\Omega\setminus E} f\diff{\mu} + \int_E f\diff{\mu} = 0+\int_E f\diff{\mu} \leq \sup_{x\in E}(f(x)). \]
Now for all $s\in\SF(\Omega\to \interval[co]{0,+\infty})\cap \downset f$ we have
\[ 0\leq \int_E s \diff{\mu}\leq \max_x(s(x))\mu(E) = 0  \]
and so the supremum $\int_E f\diff{\mu}$ is zero as well.

Now assume $\int_\Omega f\diff{\mu} = 0$. Consider the sets $E_n = f^{-1}[\,]\frac{1}{n},+\infty]\,]$. Then $\frac{1}{n}\chi_{E_n}\in \downset f$ is simple. Hence $\frac{1}{n}\mu(E_n)\leq \int_\Omega f\diff{\mu} = 0$, meaning $\mu(E_n) = 0$ for all $n$. So $\mu(E) = \sup_{n\in\N}\mu(E_n) = 0$.

(2) Towards contraposition, assume the set $E = \setbuilder{x\in\R}{f(x) = +\infty}$ has non-zero measure. Then $a/\mu(E)\chi_E\in\SF(\Omega, \interval[co]{0,+\infty})\cap \downset f$ for all real $a>0$ and
\[  \int_\Omega f\diff{\mu} \geq \int_\Omega \frac{a\chi_E}{\mu(E)}\diff{\mu} = a  \]
so $f$ is not integrable.
\end{proof}

\subsubsection{Tonelli's theorem}

\begin{lemma} \label{tonelliLemma}
Let $\sSet{\Omega_1, \mathcal{A}_1,\mu_1}$, $\sSet{\Omega_2, \mathcal{A}_2, }$ be $\sigma$-finite measure spaces and $Q\in \mathcal{A}_1\otimes\mathcal{A}_2$. Then the section functions
\[ s_1^Q: \Omega_1 \to \overline{\R^+} :x \mapsto \mu_2(xQ) \qquad\text{and}\qquad s_2^Q: \Omega_2 \to \overline{\R^+} :y \mapsto \mu_1(Qy) \]
are measurable. Additionally, we have
\[ (\mu_1\otimes \mu_2)(Q) = \int_{\Omega_1}s_1^Q\diff{\mu_1} = \int_{\Omega_2}s_2^Q\diff{\mu_2}. \]
\end{lemma}
\begin{proof}
The functions are well-defined by \ref{setSectionMeasurable}.

We restrict our attention to $s_1^Q$. The case for $s_2^Q$ is similar. Fix arbitrary $x\in\Omega_1$.

If $Q = A_1\times A_2$ for some $A_1\in\mathcal{A}_1$ and $A_2\in\mathcal{A}_2$, then
\[ s_1^Q(x) = \mu_2\setbuilder{\omega\in\Omega_2}{(x,\omega)\in A_1\times A_2} = \left(\begin{cases}
\mu_2(\emptyset) & (x\notin A_1) \\
\mu_2(A_2) & (x\in A_1)
\end{cases}\right) = \mu_2(A_2)\cdot \chi_{A_1}(x), \]
so $s_1^Q$ is measurable and
\[ \int_{\Omega_1}s_1^Q\diff{\mu_1} = \int_{\Omega_1}\mu_2(A_2)\cdot \chi_{A_1}\diff{\mu_1} = \mu_1(A_1)\cdot\mu_2(A_2), \]
by \ref{measureFromIntegralCharacteristicFunctions}.

Set $\mathcal{C} = \setbuilder{A_1\times A_2}{A_1\in\mathcal{A}_1, A_2\in\mathcal{A}_2}$ let $\mathcal{D}$ be the set of $P\in\Omega_1\times\Omega_2$ such that $s_1^P$ is measurable and $(\mu_1\otimes \mu_2)(Q) = \int_{\Omega_1}s_1^Q\diff{\mu_1}$. We want to prove $\sigma(\mathcal{C}) \subseteq \mathcal{D}$. As $\mathcal{C}$ is a $\pi$-system, it is enough to show that $\mathcal{D}$ is a Dynkin system by the $\pi$-$\lambda$ theorem \ref{piLambdaTheorem}.
We verify:
\begin{itemize}
\item $\Omega_1\times\Omega_2\in\mathcal{D}$, because $\Omega_1\times\Omega_2 \in \mathcal{C}\subseteq \mathcal{D}$;
\item Take $P\in\mathcal{D}$. Fix some monotone sequence $\seq{B_n}_{n\in\N}$ in $\mathcal{A}_2$ of finite measure whose union is $\Omega_2$, as in \ref{sigmaFiniteSequences}. 

We calculate (TODO: fix calculation)
\begin{align*}
s_1^{P^c}(x) &= \mu_2(xP^c) = \mu_2\big((xP)^c\big) = \mu_2\bigg(\Big(\bigcup_{n\in\N} B_n\Big)\setminus xP\bigg) \\
&= \sup_{n\in\N}\mu_2\big(B_n\setminus xP\big) = \sup_{n\in\N}\Big(\mu_2(B_n) - \mu_2(B_n\cap xP)\Big) = \sup_{n\in\N}\Big(\mu_2(B_n) - s_1^P(x)\Big).
\end{align*}
As $s_1^P$ is measurable, $s_1^{P^c}$ is measurable by \ref{operationsOnRealMeasurableFunctions} and \ref{limitOperationsOnRealMeasurableFunctions}.

Now similarly fix some monotone sequence $\seq{C_n}_{n\in\N}$ in $\mathcal{A}_1$ of finite measure whose union is $\Omega_1$. TODO finish proof!!

\item If $\seq{P}_{n\in\N}\in\mathcal{D}^\N$ is a sequence of pairwise disjoint sets, then, by a calculation similar to the one in the proof of \ref{setSectionMeasurable}, we have
\[ s_1^{\biguplus_{n\in\N}P_n}(x) = \mu_2\Big(\biguplus_{n\in\N}xP_n\Big) = \sum_{n\in\N}s_1^{P_n}(x). \]
Thus $s_1^{\biguplus_{n\in\N}P_n}$ is measurable by \ref{infiniteSumMeasurable} and $\biguplus_{n\in\N}P_n\in\mathcal{D}$.
\end{itemize}
\end{proof}

\begin{theorem}[Tonelli's theorem] \label{tonellisTheorem}
Let $\sSet{\Omega_1, \mathcal{A}_1, \mu_1}$, $\sSet{\Omega_2, \mathcal{A}_2, \mu_2}$ be $\sigma$-finite measure spaces and $f: \Omega_1\times\Omega_2 \to \overline{\R^+}$ a measurable function. Then the functions
\[ x\mapsto \int_{\Omega_2}f(x,y)\diff{\mu_2(y)} \qquad\text{and}\qquad y\mapsto \int_{\Omega_1}f(x,y)\diff{\mu_1(x)} \]
are measurable. Also
\[ \int_{\Omega_1\times\Omega_2}f\diff{(\mu_1\otimes\mu_2)} = \int_{\Omega_2}\Big(\int_{\Omega_1}f(x,y)\diff{\mu_1(x)}\Big)\diff{\mu_2(y)} = \int_{\Omega_1}\Big(\int_{\Omega_2}f(x,y)\diff{\mu_2(y)}\Big)\diff{\mu_1(x)}. \]
\end{theorem}
\begin{proof}
Let $s: \Omega_1\times\Omega_2 \to \overline{\R^+}$ be a measurable simple function. Then $s(x,-)$ is a measurable (by \ref{partialApplicationMeasurable}) simple function and we have
\[ \int_{\Omega_2}s(x,y)\diff{\mu_2(y)}  = \sum_{\lambda\in\im\big(s(x,-)\big)}\lambda\cdot\mu_2(xs^{-1}(\lambda)). \]
Thus $x\mapsto \int_{\Omega_2}s(x,y)\diff{\mu_2(y)}$ is, by \ref{tonelliLemma}, a linear combination of measurable functions and thus measurable by \ref{operationsOnRealMeasurableFunctions}.

By \ref{measurableFunctionPointwiseLimitSimpleFunctions} we can approximate $f$ by an increasing sequence $\seq{s_n}$ of simple functions that converges to $f$ pointwise.

By Beppo Levi's lemma \ref{propertiesIntegralPositiveFunctions}, we have
\[ \int_{\Omega_2}f(x,y)\diff{\mu_2(y)} = \lim_{n\to\infty}\int_{\Omega_2}s_n(x,y)\diff{\mu_2(y)}, \]
which is a limit of measurable positive functions and thus measurable by \ref{limitOperationsOnRealMeasurableFunctions}.

Now we calculate, using \ref{tonelliLemma},
\begin{align*}
\int_{\Omega_2}\Big(\int_{\Omega_2}s(x,y)\diff{\mu_2(y)}\Big)\diff{\mu_1(x)} &= \sum_{\lambda\in\im\big(s(x,-)\big)}\lambda\cdot \int_{\Omega_2} \mu_2(xs^{-1}(\lambda)) \diff{\mu_1(x)} \\
&= \sum_{\lambda\in\im(s)}\lambda\cdot (\mu_1\otimes\mu_2)\big(s^{-1}(\lambda)\big) \\
&= \int_{\Omega_1\times\Omega_2}s \diff{(\mu_1\otimes
\mu_2)}.
\end{align*}
For the general case, we again argue using Beppo Levi's lemma \ref{propertiesIntegralPositiveFunctions}.
\end{proof}

\subsection{Complex functions}
Let $\Omega$ be a set and $f:\Omega \to \C$ a function.  Then we can uniquely decompose $f$ into $f= u+iv$ where $u,v: \Omega\to \R$. We can further decompose
\[ \begin{cases}
u = u^+ - u^- & \text{such that $u^+u^- = 0$} \\
v = v^+ - v^- & \text{such that $v^+v^- = 0$.}
\end{cases} \]

\begin{lemma}
If $\Omega $ carries a $\sigma$-algebra such that $f$ is measurable, then
\begin{enumerate}
\item $u^+,u^-,v^+,v^-$ are measurable;
\item $|f| = \sqrt{(u^+)^2+ (u^-)^2 + (v^+)^2 + (v^-)^2}$.
\end{enumerate}
\end{lemma}

\begin{definition}
Let $\sSet{\Omega, \mathcal{A},\mu}$ be a measure space. We say a measurable function $f: \Omega \to \C$ is \udef{integrable}, if $|f|:\Omega\to [0,\infty[$ is integrable. In this case we define the \udef{integral} of $f$ as
\[ \int_\Omega f\diff{\mu} = \int_\Omega u^+\diff{\mu} - \int_\Omega u^-\diff{\mu} + i\int_\Omega v^+\diff{\mu} - i\int_\Omega v^-\diff{\mu}. \]
The set of all integrable functions in $(\Omega \to \C)$ is denoted $\mathcal{L} ^1(\Omega,\mathcal{A},\mu)$ or $\mathcal{L} ^1(\mu)$.
\end{definition}

\begin{lemma}
Let $\sSet{\Omega, \mathcal{A},\mu}$ be a measure space and $f: \Omega \to \C$ a measurable function with decomposition $f = u^+ - u^- + i(v^+ -v^-)$. If $f$ is integrable, then $u^+, u^-, v^+$ and $v^-$ are integrable.
\end{lemma}
\begin{proof}
We have by \ref{propertiesIntegralPositiveFunctions}
\[ \int_\Omega u^+ \diff{\mu} = \int_\Omega \sqrt{(u^+)^2} \diff{\mu} \leq \int_\Omega \sqrt{(u^+)^2+ (u^-)^2 + (v^+)^2 + (v^-)^2} \diff{\mu} = \int_\Omega|f|\diff{\mu} < \infty. \]
The argument for $u^-,v^+$ and $v^-$ is similar.
\end{proof}

\begin{lemma} \label{finiteMeasureBoundedFunctionMeasurable}
Let $\sSet{\Omega, \mathcal{A},\mu}$ be a finite measure space and $f: \Omega \to \C$ a function such that $\sup_{x\in \Omega}|f(x)|<\infty$, then $f$ is integrable.
\end{lemma}
\begin{proof}
Set $\sup_{x\in \Omega}|f(x)| = k$. Then
\[ \int_\Omega |f|\diff{\mu} \leq \int_\Omega \underline{k}\diff{\mu} = k\mu(\Omega) < \infty, \]
by \ref{propertiesIntegralPositiveFunctions}.
\end{proof}


\begin{proposition}[Reverse Fatou lemma]
Let $(\Omega, \mathcal{A}, \mu)$ be a measure space and $(h_n)$ any sequence of measurable functions that is dominated by a positive integrable function $g$ (i.e.\ $h_n\leq g$ for all $n\in\N$). Then
\[ f: \Omega\to[0,+\infty]: x\mapsto \limsup_{n\to\infty}h_n(x) \]
is measurable and
\[ \int_\Omega f\diff{\mu} \geq \limsup_{n\to\infty}\int_\Omega h_n\diff{\mu}. \]
\end{proposition}
\begin{proof}
By the corollary $g(x)<+\infty$ a.e.\ and so also $g-h_n<+\infty$ a.e. Applying Fatou's lemma \ref{FatouLemma} gives
\[ \int\liminf g-h_n \leq \liminf\int g-h_n\]
\end{proof}

\begin{theorem}[Dominated convergence] \label{dominatedConvergence}
Let $\sSet{\Omega, \mathcal{A}, \mu}$ be a measure space and $\seq{f_n}$ any sequence of real measurable functions in $(\Omega\to\R)$ that converges pointwise to a function $f$.

If there exists a positive integrable function $g$ such that $|f_n| \leq g$ for all $n\in \N$, then $f$ is integrable and
\[ \lim_{n\to \infty}\int_\Omega f_n \diff{\mu} = \int_\Omega f\diff{\mu}. \]
\end{theorem}
TODO: sufficiency of uniform continuity.
\begin{proof}
TODO
\end{proof}
\begin{corollary}[Bounded convergence theorem] \label{boundedConvergenceTheorem}
Let $\sSet{\Omega, \mathcal{A}, \mu}$ be a finite measure space and $\seq{f_n}$ any sequence of real measurable functions that converges pointwise to a function $f$.
\end{corollary}
TODO: \url{http://mathonline.wikidot.com/summary-of-convergence-theorems-for-lebesgue-integration}

\subsubsection{Fubini's theorem}
\begin{theorem}[Fubini's theorem]
Let $\sSet{\Omega_1, \mathcal{A}_1, \mu_1}$, $\sSet{\Omega_2, \mathcal{A}_2, \mu_2}$ be $\sigma$-finite measure spaces and $f: \Omega_1\times\Omega_2 \to \overline{\R}$ a measurable and integrable function. Then the functions
\[ x\mapsto \int_{\Omega_2}f(x,y)\diff{\mu_2(y)} \qquad\text{and}\qquad y\mapsto \int_{\Omega_1}f(x,y)\diff{\mu_1(x)} \]
are measurable and a.e. integrable. Also
\[ \int_{\Omega_1\times\Omega_2}f\diff{(\mu_1\otimes\mu_2)} = \int_{\Omega_2}\Big(\int_{\Omega_1}f(x,y)\diff{\mu_1(x)}\Big)\diff{\mu_2(y)} = \int_{\Omega_1}\Big(\int_{\Omega_2}f(x,y)\diff{\mu_2(y)}\Big)\diff{\mu_1(x)}. \]
\end{theorem}
Note that we require that $f: \Omega_1\times\Omega_2 \to \overline{\R}$ is measurable. For Tonelli's theorem, \ref{tonellisTheorem}, this is not necessary.
\begin{proof}
TODO!
\end{proof}

TODO Fubini for complex- (and vector-)valued functions.

\subsection{Bochner integration}
TODO: define integration by
\[ \inner{u, \int T(t)\diff{t} v} \defeq \int\inner{u,T(t)v}\diff{t} \]
??
This case Riemann?? Or only finite dim??



TODO: the Bochner integral is the unique extension of the integral of simple functions to the set of Bochner measurable functions???? (i.e.\ simple functions dense in Bochner space, with $L^1$ metric)
\begin{definition}
Let $(\Omega, \mathcal{A},\mu)$ be a measure space and $Y$ a normed vector space. Then a Bochner measurable function $f:\Omega\to Y$ is called \udef{Bochner integrable} if there exists a sequence of integrable simple functions $\seq{s_n}\subset\SF(\Omega,Y)$ such that
\[ \lim_{n\to\infty}\int_\Omega \norm{f-s_n}\diff{\mu} = 0. \]
Take such a sequence $\seq{s_n}$. The \udef{Bochner integral} of $f$ on $\Omega$ w.r.t. $\mu$ is defined as
\[ \int_\Omega f\diff{\mu} \defeq \lim_{n\to\infty}\int_\Omega s_n\diff{\mu}. \]
\end{definition}

\begin{lemma}
The Bochner integral is well-defined: let $\seq{s_n},\seq{t_n}\in \prescript{\N}{}{\SF(\Omega,Y)}$ be sequences such that
\[ \lim_{n\to\infty}\int_\Omega \norm{f-s_n}\diff{\mu} = 0 = \lim_{n\to\infty}\int_\Omega \norm{f-t_n}\diff{\mu}.  \]
Then
\begin{enumerate}
\item the limits $\lim_{n\to\infty}\int_\Omega s_n\diff{\mu}$ and $\lim_{n\to\infty}\int_\Omega t_n\diff{\mu}$ exist;
\item $\lim_{n\to\infty}\int_\Omega s_n\diff{\mu} = \lim_{n\to\infty}\int_\Omega t_n\diff{\mu}$.
\end{enumerate}
\end{lemma}
\begin{proof}
TODO
\end{proof}

\begin{proposition}[Bochner integrability criterion] \label{BochnerIntegrabilityCondition}
Let $(\Omega, \mathcal{A},\mu)$ be a measure space and $Y$ a normed vector space.

A Bochner measurable function $f$ is Bochner integrable \textup{if and only if}
\[ \int_\Omega \norm{f} \diff{\mu} < \infty. \]
\end{proposition}

\begin{proposition}
Linearity and monotonicity.
\end{proposition}

\begin{theorem}[Hille's theorem] \label{HilleTheorem}
Let $(\Omega, \mathcal{A},\mu)$ be a measure space, $X,Y$ normed vector spaces and $T: X\not\to Y$ a closed operator. If $T\circ f$ is integrable, then
\[ \int_\Omega (T\circ f)\diff{\mu} = T\left(\int_\Omega f\diff{\mu}\right). \]
\end{theorem}
\begin{proof}
TODO
\end{proof}
\begin{corollary} \label{boundedOperatorUnderIntegral}
If $T$ is bounded, then $T\circ f$ is integrable and
\[ \int_\Omega (T\circ f)\diff{\mu} = T\left(\int_\Omega f\diff{\mu}\right). \]
\end{corollary}
\begin{proof}
TODO: show that $T\circ f$ is integrable!
\end{proof}

TODO Dominated convergence.

\subsubsection{Weak and strong measurability}
TODO is Bochner measurable measurable with $Y$ given Borel $\sigma$-algebra?

A function $f:X\to B$ is called Bochner-measurable if it is equal $\mu$-almost everywhere to a function $g$ taking values in a separable subspace $B_{0}$ of $B$, and such that the inverse image $g^{-1}[U]$ of every open set $U$ in $B$ belongs to $\Sigma$. Equivalently, $f$ is limit $\mu$-almost everywhere of a sequence of simple functions. 

\begin{theorem}[Pettis measurability theorem]
Let $(\Omega, \mathcal{A},\mu)$ be a measure space and $Y$ a normed vector space. Then $f$ is strongly measurable \textup{if and only if} $f$ is weakly measurable and almost surely separably valued.
\end{theorem}

\subsubsection{Integration of bounded operators}
\begin{lemma} \label{integralBoundedOperator}
Let $X$ be a normed space and $(\Omega, \mathcal{A},\mu)$ a measure space. Let $T: \Omega \to \Bounded(X)$ be a function. If $T$ is integrable, then for all $x\in X$, $Tx$ is integrable and
\[ \left(\int_\Omega T\diff{\mu}\right)x = \int_\Omega Tx\diff{\mu}. \]
\end{lemma}
\begin{proof}
The evaluation map $\evalMap_x$ is linear and bounded by $\norm{x}$ for all $x\in X$, so we can use \ref{boundedOperatorUnderIntegral}.
\end{proof}



\section{Further topics}
TODO rename!

\subsection{Absolute continuity and mutual singularity}
\begin{definition}
Let $\mu,\nu$ be measures on the measurable space $(\Omega,\mathcal{A})$. We say
\begin{itemize}
\item $\nu$ is \udef{absolutely continuous} w.r.t. $\mu$ if $\mu(A)=0\implies \nu(A) = 0$ for all $A\in\mathcal{A}$;
\item $\mu$ and $\nu$ are \udef{mutually singular} if there exists a set $A\in\mathcal{A}$ with $\mu(A) = 0$ and $\nu(A^c) = 0$.
\end{itemize}
\end{definition}

\begin{theorem}[Radon-Nikodym]
Let $\mu,\nu$ be measures on the measurable space $(\Omega,\mathcal{A})$. Then $\nu$ is absolutely continuous w.r.t. $\mu$ \textup{if and only if} there exists a measurable function $f:\Omega\to\R$ (or $\C$?) such that
\[ \nu(A) = \int_\Omega f\cdot \chi_A \diff{\mu} \qquad \forall A\in\mathcal{A}. \]
The function $f$ is uniquely determined a.e.\ (w.r.t. $\mu$).
\end{theorem}

\subsection{Lebesgue decomposition}
\begin{theorem}[Lebesgue decomposition theorem]
Let $\mu, \nu$ be two measures on a measurable space $(\Omega, \mathcal{A})$. Then $\nu$ can be written uniquely as
\[ \nu = \nu_\text{ac} + \nu_\text{sing} \]
where $\mu$ and $\nu_\text{sing}$ are mutually singular and $\nu_\text{ac}$ is absolutely continuous w.r.t. $\mu$. 

\end{theorem}

\subsection{Convolution}
TODO Young's convolution inequality

\section{Duality in integration}
Distributions with kernels will be example.

\chapter{Real Analysis}
Dual numbers

Dini theorem

\section{Limits and convergence}

\subsection{Bachmann-Landau notation}
\subsubsection{Asymptotic bounds: $O, \Theta, \Omega$}
\begin{definition}
Let $(X,\mathcal{T})$ be a topological space. Let $x_0 \in X$ and $f,g: X\setminus\{x_0\}\to \R^+$ be functions. The statement
\begin{itemize}
\item ``$f(x) = O(g(x))$ as $x\to x_0$'' means there exists a neighbourhood $S$ of $x_0$ and a constant $M\in\R$ such that
\[ \forall x\in S:\; f(x) \leq Mg(x); \]
\item ``$f(x) = \Omega(g(x))$ as $x\to x_0$'' means there exists a neighbourhood $S$ of $x_0$ and a constant $M\in\R$ such that
\[ \forall x\in S:\; f(x) \geq Mg(x); \]
\item ``$f(x) = \Theta(g(x))$ as $x\to x_0$'' means there exists a neighbourhood $S$ of $x_0$ and constants $M_1,M_2\in\R$ such that
\[ \forall x\in S:\; M_1g(x) \leq f(x) \leq M_2g(x). \]
\end{itemize}
We may add the word ``uniformly'' to these statements to mean we can take $S=X$.

We may suppress the $x$ dependence for legibility and write e.g\ $f = O(g)$ instead.
\end{definition}

\begin{lemma}
Let $(X,\mathcal{T})$ be a topological space. Let $x_0 \in X$ and $f,g: X\setminus\{x_0\}\to \R^+$ be functions. Then
\begin{enumerate}
\item $f = O(g)$ \textup{if and only if} $g = \Omega(f)$ as $x\to x_0$;
\item $f = \Theta(g)$ \textup{if and only if} $f = O(g)$ and $f = \Omega(g)$ as $x\to x_0$.
\end{enumerate}
\end{lemma}

\begin{lemma}
Let $(X,\mathcal{T})$ be a topological space. Let $x_0 \in X$ and $f: X\setminus\{x_0\}\to \R^+$ be functions. Then ``being $\Theta(f)$ as $x\to x_0$'' is an equivalence relation.
\end{lemma}

\begin{lemma}
Let $(X,\mathcal{T})$ be a topological space. Let $x_0 \in X$ and $f,g: X\setminus\{x_0\}\to \R^+$ be functions. Then
\begin{enumerate}
\item $f(x) = O(g(x))$ as $x\to x_0$ \textup{if and only if} there exists a neighbourhood $S$ of $x_0$ such that $f(x)/g(x)$ is bounded on $S$;
\item $f(x) = \Omega(g(x))$ as $x\to x_0$ \textup{if and only if} there exists a neighbourhood $S$ of $x_0$ such that $f(x)/g(x)$ is bounded below on $S$ by a strictly positive constant.
\end{enumerate}
\end{lemma}

\subsubsection{Asymptotic domination and equality: $o,\sim,\omega$}
\begin{definition}
Let $(X,\mathcal{T})$ be a topological space. Let $x_0 \in X$ and $f,g: X\setminus\{x_0\}\to \R^+$ be functions. The statement
\begin{itemize}
\item ``$f(x) = o(g(x))$ as $x\to x_0$'' means $\lim_{x\to x_0} \frac{f(x)}{g(x)} = 0$;
\item ``$f(x) \sim_{x_0} g(x)$'' means $\lim_{x\to x_0} \frac{f(x)}{g(x)} = 1$;
\item ``$f(x) = \omega(g(x))$ as $x\to x_0$'' means $\lim_{x\to x_0} \frac{f(x)}{g(x)} = \infty$.
\end{itemize}
\end{definition}

\begin{lemma}
Let $(X,\mathcal{T})$ be a topological space. Let $x_0 \in X$ and $f,g: X\setminus\{x_0\}\to \R^+$ be functions.

Then $f = o(g)$ \textup{if and only if} $g = \omega(f)$ as $x\to x_0$.
\end{lemma}

\begin{lemma}
Let $(X,\mathcal{T})$ be a topological space. Let $x_0 \in X$ and $f,g: X\setminus\{x_0\}\to \R^+$ be functions. Then
\begin{enumerate}
\item $f\sim_{x_0} g \iff (f-g)\in o(g)$ as $x\to x_0$;
\item $\sim_{x_0}$ is an equivalence relation;
\item $f \sim_{x_0} g \implies f = \Theta(g)$ as $x\to x_0$.
\end{enumerate}
\end{lemma}

\begin{lemma}
Let $(X,\mathcal{T})$ be a topological space. Let $x_0 \in X$ and $f,g, h,k: X\setminus\{x_0\}\to \R^+$ be functions. Then
\begin{enumerate}
\item if $f = o(h)$ and $g = O(k)$, then $fg = o(hk)$ as $x\to x_0$;
\item if $f = O(h)$ and $g = O(k)$, then $fg = O(hk)$ as $x\to x_0$;
\item if $f = o(h)$ and $h = O(k)$, then $f = o(k)$ as $x\to x_0$.
\end{enumerate}
\end{lemma}

\subsection{Sequences and limits}
\begin{example}
Let $p\in\R$. Then
\[ \lim_{n\to\infty} n^p = \begin{cases}
+\infty & (p>0) \\
1 & (p=0) \\
0 & (p<0).
\end{cases} \]
Let $r\in\R$. Then
\[ \lim_{n\to\infty} r^n = \begin{cases}
+\infty & (r>1) \\
1 & (r=1) \\
0 & (-1<r<1) \\
\text{does not exist} & (r\leq -1).
\end{cases} \]
\end{example}

\begin{lemma} \label{sequenceWellFoundedIffNoDecreasingSubsequence}
Let $\seq{x_n}$ be a sequence in $\R$. Then $\im\seq{x_n}$ is well-founded \textup{if and only if} $\seq{x_n}$ does not have a strictly decreasing subsequence.
\end{lemma}
\begin{proof}
First assume $\seq{x_n}$ has a strictly decreasing subsequence $\seq{y_n}$. Clearly $\im\seq{y_n}$ cannot have a minimum and thus $\im\seq{x_n}$ is not well-founded.

Now assume $\im\seq{x_n}$ is not well-founded. Then there exists a subset $A\subseteq \im\seq{x_n}$ that does not have a minimum. Thus we can find a strictly decreasing sequence $\seq{x_{n_k}}$ in $A$.

Since $\seq{x_{n_k}}$ is strictly decreasing, each $n_k$ is distinct and we can find a monotone subsequence $\seq{n'_k}$ by \ref{injectiveSequenceNaturalNumbersMonotoneSubsequence}. Now $\seq{x_{n'_k}}$ is a strictly decreasing subsequence of $\seq{x_n}$.
\end{proof}

\begin{proposition} \label{monotoneSubsequenceRealSequence}
Every sequence in $\R$ has a monotone subsequence.
\end{proposition}
\begin{proof}
Let $\seq{x_n}$ be a sequence in $\R$. We need to show that $\seq{x_n}$ has either a monotonically inceasing or monotonically decreasing subsequence.

Assume $\seq{x_n}$ does not have a monotonically decreasing subsequence. Then it also does not have a strictly decreasing subsequence and by \ref{sequenceWellFoundedIffNoDecreasingSubsequence}, $\im\seq{x_n}$ is well-founded. We can now recursively construct a sequence of natural numbers $\seq{n_k}$:
\[ \begin{cases}
n_0 = \argmin_{k'\in \N}x_{k'} \\
n_{k+1} = \argmin_{k'> k}x_{k'}.
\end{cases} \]
Now the subsequence $\seq{x_{n_k}}_{k\in \N}$ is increasing.
\end{proof}
\begin{corollary}[Bolzano-Weierstrass]
Every bounded sequence in $\R$ has a convergent subsequence. 
\end{corollary}

\begin{proposition} \label{sequencesToExtrema}
Let $A\subseteq \R$. Then
\begin{enumerate}
\item there exists a monotonically increasing sequence $\seq{x_n}$ in $A$ such that $\lim_{n\to\infty}x_n = \sup A$;
\item there exists a monotonically decreasing sequence $\seq{y_n}$ in $A$ such that $\lim_{n\to\infty}y_n = \inf A$.
\end{enumerate}
\end{proposition}
\begin{proof}
(1) Let $x_0$ be any element of $A$. Clearly $x_0\leq \sup A$. If $x_0 = \sup A$, we may take the constant sequence $\underline{x_0}$. Otherwise there exists an element of $A$ that is strictly larger than $x_0$. Let this be $x_1$. Repeat this process recursively using the axiom of dependent choice to obtain the required sequence (TODO relax to countable choice?).

(2) Dual to (1).
\end{proof}

\subsection{Upper and lower limits}
\begin{definition}
Consider the function $D: \R \to\powerset(\R): x\mapsto \interval[co]{x,\infty}$ and the directional convergences $\xi_D$ and $\xi_{D^\transp}$ w.r.t.\ $D$ and $D^\transp$. Let $\xi$ be the standard convergence on $\R$.
Then
\begin{itemize}
\item the initial convergence w.r.t.\ $\{\id_\R: \R\to \sSet{\R,\xi}, \id_\R: \R\to \sSet{\R,\xi_D}\}$ is called the \udef{lower limit convergence} on $\R$ and $\R$ equipped with this convergence is denoted $\R_{ll}$;
\item the initial convergence w.r.t.\ $\{\id_\R: \R\to \sSet{\R,\xi}, \id_\R: \R\to \sSet{\R,\xi_{D^\transp}}\}$ is called the \udef{upper limit convergence} on $\R$ and $\R$ equipped with this convergence is denoted $\R_{ul}$.
\end{itemize}
The space $\R_{ll}$ is also called the \udef{Sorgenfrey line}.
\end{definition}

\begin{proposition}
The spaces $\R_{ll}$ and $\R_{ul}$ are topological and we have
\begin{enumerate}
\item $\neighbourhood_{ll}(x) = \upset \setbuilder[\big]{\interval[co]{x,x+\epsilon}}{\epsilon > 0}$ and $\neighbourhood_{ul}(x) = \upset \setbuilder[\big]{\interval[oc]{x-\epsilon, x}}{\epsilon > 0}$;
\item $\topology_{ll}$ has a basis $\setbuilder[\big]{\interval[co]{a,b}}{a,b\in \R}$;
\item $\topology_{ul}$ has a basis $\setbuilder[\big]{\interval[oc]{a,b}}{a,b\in \R}$.
\end{enumerate}
\end{proposition}
\begin{proof}
Application of \ref{pretopologicalInitialConvergence}.
\end{proof}
\begin{corollary} \label{SorgenfreyLineC1}
Both $\R_{ll}$ and $\R_{ul}$ are $C1$, Fréchet-Urysohn and sequential.
\end{corollary}

\begin{lemma} \label{lowerLimitConvergenceDecreasingSubsequence}
Let $\seq{x_n}\subseteq \R_{ll}$ be a convergent sequence. Then $\seq{x_n}$ has a monotonically decreasing convergent subsequence.
\end{lemma}
\begin{proof}
Suppose $\seq{x_n} \overset{ll}{\longrightarrow} x$. By \ref{monotoneSubsequenceRealSequence} we know that $\seq{x_n}$ has either a monotonically increasing or a monotonically decreasing subsequence. We claim that any monotonically increasing subsequence of $\seq{x_n}$ has the constant sequence $\underline{x}$ as a subsequence.

Let $\seq{x'_n}$ be a monotonically increasing subsequence. We must have that $\seq{x'_n} \overset{ll}{\longrightarrow} x$. Then $\seq{x'_n}$ has a tail $\setbuilder{x'_k}{k \geq k_0}\subseteq \interval[co]{x,\infty}$. We claim that $\setbuilder{x'_k}{k \geq k_0} = \{x\}$. Indeed, suppose there exists $x'_m \in \setbuilder{x'_k}{k \geq k_0}$ such that $x'_m \neq x$, then $\setbuilder{x'_k}{k \geq m} \perp \interval[co]{x, (x'_m-x)/2}$ and $\seq{x'_n}$ does not converge.

We have shown that $\seq{x'_k}_{k\geq k_0}$ is the constant sequence $\underline{x}$.
\end{proof}

\section{Series}

\subsection{Convergence tests}
\url{https://en.wikipedia.org/wiki/Convergence_tests}

\subsubsection{Ratio test hierarchy}
\url{https://en.wikipedia.org/wiki/Ratio_test}
\subsubsection{Root test hierarchy}
\url{https://en.wikipedia.org/wiki/Root_test#Root_tests_hierarchy}

\begin{proposition}[Cauchy's criterion] \label{rootTest}
Let $\sum_{n=0}^\infty a_n$ be a positive series. Define the number
\[ C = \limsup_{n\to \infty}\sqrt[n]{a_n}. \]
Then
\begin{itemize}
\item if $C<1$, the series converges;
\item if $C>1$, the series diverges;
\item if $C=1$, the test is inconclusive.
\end{itemize}
\end{proposition}

\subsection{Types of series}
\subsubsection{Geometric series}
\begin{proposition}
Take $a,r\in \R$. Then
\[ \sum_{k=0}^{n-1}ar^k = \begin{cases}
a\left(\frac{1-r^n}{1-r}\right) & (r\neq 1) \\
an & (\text{otherwise}).
\end{cases} \]
\end{proposition}
\begin{corollary}
Take $a\in \R$ and $|r|<1$. Then
\[ \sum_{k=0}^\infty a r^k = \frac{a}{1-r}. \]
\end{corollary}

\subsubsection{Harmonic series and $p$-series}
\begin{proposition} \label{pseriesConvergence}
Take $p\in \R$. Then $\sum_{k=1}^\infty \frac{1}{k^p}$
\begin{enumerate}
\item converges if $p > 1$;
\item does not converge if $p\leq 1$.
\end{enumerate}
\end{proposition}



\section{Continuity}

\subsection{Upper and lower limits}
\begin{definition}
Consider the function $D: \R \to\powerset(\R): x\mapsto \interval[co]{x,\infty}$ and the directional convergences $\xi_D$ and $\xi_{D^\transp}$ w.r.t.\ $D$ and $D^\transp$. Let $\xi$ be the standard convergence on $\R$.
Then
\begin{itemize}
\item the initial convergence w.r.t.\ $\{\id_\R: \R\to \sSet{\R,\xi}, \id_\R: \R\to \sSet{\R,\xi_D}\}$ is called the \udef{lower limit convergence} on $\R$ and $\R$ equipped with this convergence is denoted $\R_{ll}$;
\item the initial convergence w.r.t.\ $\{\id_\R: \R\to \sSet{\R,\xi}, \id_\R: \R\to \sSet{\R,\xi_{D^\transp}}\}$ is called the \udef{upper limit convergence} on $\R$ and $\R$ equipped with this convergence is denoted $\R_{ul}$.
\end{itemize}
The space $\R_{ll}$ is also called the \udef{Sorgenfrey line}.
\end{definition}

\begin{proposition}
The spaces $\R_{ll}$ and $\R_{ul}$ are topological and we have
\begin{enumerate}
\item $\neighbourhood_{ll}(x) = \upset \setbuilder{\interval[co]{x,x+\epsilon}}{\epsilon > 0}$ and $\neighbourhood_{ul}(x) = \upset \setbuilder{\interval[oc]{x-\epsilon, x}}{\epsilon > 0}$;
\item $\topology_{ll}$ has a basis $\setbuilder{\interval[co]{a,b}}{a,b\in \R}$;
\item $\topology_{ul}$ has a basis $\setbuilder{\interval[oc]{a,b}}{a,b\in \R}$.
\end{enumerate}
\end{proposition}
\begin{proof}
Application of \ref{pretopologicalInitialConvergence}.
\end{proof}
\begin{corollary} \label{SorgenfreyLineC1}
Both $\R_{ll}$ and $\R_{ul}$ are $C1$, Fréchet-Urysohn and sequential.
\end{corollary}

\begin{lemma}
Let $\seq{x_n}\subseteq \R_{ll}$ be a convergent sequence. Then $\seq{x_n}$ has a monotonically decreasing convergent subsequence.
\end{lemma}
\begin{proof}
Suppose $\seq{x_n} \overset{ll}{\longrightarrow} x$.
\end{proof}

\begin{definition}
Let $f:A\subseteq\R\to\R$ be a function and $p$ a limit point of $A$. Then
\begin{itemize}
\item the \udef{left limit} of $f(x)$ as $x$ approaches $p$ is the limit of $f|_{A\cap]-\infty,p]}$ as $x\to p$, denoted
\[ \lim_{x\overset{<}{\to} p}f(x) \qquad \text{or} \qquad \lim_{x\to p^-}f(x); \]
\item the \udef{right limit} of $f(x)$ as $x$ approaches $p$ is the limit of $f|_{A\cap[p,+\infty[}$ as $x\to p$, denoted
\[ \lim_{x\overset{>}{\to} p}f(x) \qquad \text{or} \qquad \lim_{x\to p^+}f(x). \]
\end{itemize}
\end{definition}
\begin{lemma}
Let $f:A\subseteq\R\to\R$ be a function and $p$ a limit point of both $A\cap]-\infty,p]$ and $A\cap[p,+\infty[$. Then $\lim_{x\to p} f(x)$ exists if the left and right limit exist and are equal to each other. In this case
\[ \lim_{x\to p} f(x) = \lim_{x\to p^-} f(x) = \lim_{x\to p^+} f(x). \]
\end{lemma}

\subsection{Discontinuities}
If a function $f:A\subseteq\R\to\R$ is not continuous at $p$, then $p$ is a limit point of $A$ by \ref{continuityAtIsolatedPoint} and \ref{notLimitPointSingletonOpen}.
\begin{definition}
Let $f:A\subseteq\R\to\R$ be a function and $p\in A$ such that $f$ is not continuous at $p$. Then
\begin{itemize}
\item if $\lim_{x\to p}f(x)$ exists and is finite, we call $p$ a \udef{removable discontinuity};
\item if both $\lim_{x\to p^-}f(x)$ and $\lim_{x\to p^+}f(x)$ exist and are finite, but are different, we call $p$ a \udef{jump discontinuity};
\item if either $\lim_{x\to p^-}f(x)$ or $\lim_{x\to p^+}f(x)$ do not exist, we call $p$ an \udef{essential discontinuity}.
\end{itemize}
Removable and jump discontinuities are also called \udef{discontinuities of the first kind}. Essential discontinuities are also called \udef{discontinuities of the second kind}.
\end{definition}
TODO: should discontinuities of type 1/x be considered essential?

\subsubsection{Jump discontinuities}
\begin{proposition} \label{monotoneDiscontinuities}
Let $f$ be a monotone real-valued function on an interval $I$. Then all discontinuities are jump discontinuities.
\end{proposition}
\begin{theorem}[Darboux-Froda] \label{DarbouxFroda}
Let $f$ be a monotone real-valued function on an interval $I$. Then the set of discontinuities is at most countable.
\end{theorem}

TODO: intervals must be closed / open?? In \ref{monotoneDiscontinuities} and \ref{DarbouxFroda}.


\begin{definition}
Let $f$ be a real function. We define the \udef{difference operator}
\[ \Delta_{x_0}f \defeq \lim_{x\to x_0+} f(x) - \lim_{x\to x_0-} f(x) \]
\end{definition}

\subsection{Extreme and intermediate value theorems}
\begin{theorem}[Fermat's theorem] \label{fermatsTheorem}
Let $a<b\in \R$ and $f: \interval[o]{a,b}\to \R$ be a real function. If $c\in \interval[o]{a,b}$ is such that $f(c)$ is a maximum or a minimum in $\im(f)$ and $f$ is differentiable at $c$, then $f'(c) = 0$.
\end{theorem}
\begin{proof}
First assume $f(c)$ is a minimum. Then $f(c+h)-f(c) \geq 0$ for all $h\in \interval[o]{a-c, b-c}$, so
\[ \frac{f(c+n^{-1})-f(c)}{n^{-1}} \geq 0 \qquad\text{and}\qquad \frac{f(c-n^{-1})-f(c)}{-n^{-1}} \leq 0 \]
for all $n\in \N$ such that $n^{-1}\in \interval[o]{a-c, b-c}$ and $-n^{-1}\in \interval[o]{a-c, b-c}$. Taking both limits, we see that $f'(c) \geq 0$ and $f'(c) \leq 0$, so $f'(c) = 0$.

The case of $f(c)$ being a maximum is analoguous.
\end{proof}

\begin{theorem}[Extreme value theorem] \label{extremeValueTheorem}
Let $\sSet{X,\xi}$ be a convergence space, $f: X\to \R$ a continuous function and $A\subseteq X$ a non-empty compact subset. Then $f^\imf(A)$ has both a minimum and a maximum.
\end{theorem}
\begin{proof}
We have that $f^\imf(A)$ is compact by \ref{compactConstructions} and thus bounded by \ref{compactImpliesBounded}, which means that $f^\imf(A) \subseteq \interval{a,b}$ for some $a,b\in R$ by \ref{boundedSubsetsRealNumbers}.

Thus we have that $\sup f^{\imf}(A) = c \leq b$. We need to show that $c\in A$. We construct a sequence $\seq{x_n}$ as follows: for all $n\in \N$ we let $x_n$ be some element of $\ball(c, n^{-1})\cap f^{\imf}(A)$.

Each of these sets is non-empty. If one of them was empty for some $n$, then $c-n^{-1}$ would be an upper bound, which is a contradiction.

Clearly $\seq{x_n}$ converges to $c$ in $\R$. As $f^\imf(A)$ is closed by \ref{compactClosedSets} and \ref{metricConvergenceHausdorff}, we have $c\in f^\imf(A)$ and $f^{\imf}(A)$ reaches a maximum.

Similarly $f^{\imf}(A)$ reaches a minimum.
\end{proof}
\begin{corollary}[Rolle's theorem] \label{RollesTheorem}
Let $a<b\in \R$ and $f: \interval{a,b}\to \R$ be a continuous real function that is differentiable on $\interval[o]{a,b}$ such that $f(a) = f(b)$. Then there exists $c\in \interval[o]{a,b}$ such that $f'(c) = 0$.
\end{corollary}
\begin{proof}
We know that $f^{\imf}\big(\interval{a,b}\big)$ has both a maximum and a minimum ($\interval{a,b}$ is compact by \ref{closedRealIntervalCompact}). If $f(a) = f(b)$ is both the minimum and maximum, then for all $c\in \interval[o]{a,b}$, $f(c) = f(a) = f(b)$, so $f(c)$ is both the minimum and maximum, which means that $f'(c) = 0$ by \ref{fermatsTheorem}.

If $f(a) = f(b)$ is not both the minimum and maximum, then there exists another $c\in \interval{a,b}\setminus\{a,b\} = \interval[o]{a,b}$ such that $f(c)$ is either a minimum or a maximum. Then $f'(c) = 0$ by \ref{fermatsTheorem}.
\end{proof}


\begin{theorem}[Mean value theorem] \label{meanValueTheorem}
Let $a<b\in \R$ and $f: \interval{a,b}\to \R$ be a continuous real function that is differentiable on $\interval[o]{a,b}$. Then there exists $c\in \interval[o]{a,b}$ such that
\[f'(c) = \frac{f(b)-f(a)}{b-a}. \]
\end{theorem}
\begin{proof}
Define $g(x) = f(x) - x\frac{f(b)-f(a)}{b-a}$. Then
\begin{multline*}
g(a) = f(a) - a\frac{f(b)-f(a)}{b-a} = \frac{bf(a) - af(a)}{b-a} - a\frac{f(b)-f(a)}{b-a} = \\ \frac{bf(a)-af(b)}{b-a} = \frac{bf(a)-bf(b) + bf(b)-af(b)}{b-a} = \\ \frac{bf(b) - af(b)}{b-a} - \frac{bf(b) - bf(a)}{b-a} = f(b) - b \frac{f(b)-f(a)}{b-a} = g(b),
\end{multline*}
so we can apply Rolle's theorem \ref{RollesTheorem} to $g$. Thus there exists $c\in \interval[o]{a,b}$ such that $0 = g'(c) = f'(c) - \frac{f(b)-f(a)}{b-a}$. Thus $f'(c) = \frac{f(b)-f(a)}{b-a}$.
\end{proof}

\begin{proposition}[Intermediate value theorem]
Let $a,b\in \R$ and $f:\interval{a,b}\to \R$ a real function. For all $y\in \interval{f(a), f(b)}$ there exists $c\in \interval{a,b}$ such that $f(c) = y$.
\end{proposition}
\begin{proof}
Follows from \ref{generalisedIntermediateValueTheorem} and \ref{connectedSubsetReals}.
\end{proof}
\begin{corollary} \label{extremaIntermediateValueTheorem}
Let $a,b\in \R$ and $f:\interval{a,b}\to \R$ a real function. For all $y\in \interval{\inf\im(f), \sup\im(f)}$ there exists $c\in \interval{a,b}$ such that $f(c) = y$.
\end{corollary}
\begin{proof}
Both $\inf\im(f)$ and $\sup\im(f)$ are attained on the interval by \ref{extremeValueTheorem}, say at points $b,c\in \interval{a,b}$. We can then apply the intermediate value theorem 
\end{proof}

\section{Integration}
\begin{lemma} \label{continuousRealFunctionIntegrable}
Let $a,b\in \R$ and $f:\interval{a,b}\to \R$ be a continuous real function. Then $f$ is integrable.
\end{lemma}
\begin{proof}
By \ref{extremeValueTheorem}, $\im(f)$ has a minimum $m$ and maximum $M$.

Then $\max\big(\im(|f|)\big) = \max\{|m|, |M|\}<\infty$. Also $\interval{a,b}$ has measure $|b-a|<\infty$, so $f$ is measurable by \ref{finiteMeasureBoundedFunctionMeasurable}.
\end{proof}

\begin{proposition}[Mean value theorem for definite integrals] \label{meanValueTheoremIntegrals}
Let $a<b\in \R$ and $f:\interval{a,b}\to \R$ be a continuous real function. Then there exists $c\in \interval{a,b}$ such that
\[ \int_a^bf(x)\diff{x} = f(c)(b-a). \]
\end{proposition}
TODO in fact we can take $c\in \interval[o]{a,b}$
\begin{proof}
By the extreme value theorem, $f$ reaches both a minimum $m$ and a maximum $M$ on the interval. Then we have
\[ m(b-a) = \int_a^bm\diff{x} \leq \int_a^bf(x)\diff{x} \leq \int_a^bM\diff{x} = M(b-a) \]
and thus $(b-a)^{-1}\int_a^bf(x)\diff{x}$ lies between the extrema. By \ref{extremaIntermediateValueTheorem} there exists $c\in \interval{a,b}$ such that $f(c) = (b-a)^{-1}\int_a^bf(x)\diff{x}$.
\end{proof}

\begin{theorem}[First fundamental theorem of calculus] \label{firstFundamentalTheoremCalculus}
Let $a,b\in \R$ and $f:\interval{a,b}\to \R$ be a continuous real function. Set
\[ F: \interval{a,b} \to \R: x\mapsto \int_a^x f(t) \diff{t}. \]
Then
\begin{enumerate}
\item $F$ is continuous on $\interval{a,b}$;
\item $F$ is differentiable on $\interval[o]{a,b}$;
\item $F'(x) = f(x)$ for all $x\in \interval[o]{a,b}$.
\end{enumerate}
\end{theorem}
\begin{proof}
For any $x,h\in \R$ such that $x, x+h\in \interval{a,b}$, we have
\[ F(x+h) - F(x) = \int_a^{x+h}f(t)\diff{t} - \int_a^{x}f(t)\diff{t} = \int_x^{x+h}f(t)\diff{t}. \]
By the mean value theorem for integrals, \ref{meanValueTheoremIntegrals}, we have $F(x+h) - F(x) = f(c_h)\cdot h$ for some $c_h\in \interval{x,x+h}$. In the limit $h\to 0$, we have $c_h\to x$ be the squeeze theorem and thus $f(c_h)\to f(x)$ by continuity of $f$. Thus
\[ \lim_{h\to 0}F(x+h) - F(x) = \lim_{h\to 0}f(c_h)\cdot h = f(x)\lim_{h\to 0}h = 0, \]
which implies that $F$ is continuous on $\interval{a,b}$.
Also
\[ f(x) = \lim_{h\to 0}f(c_h) = \lim_{h\to 0}\frac{F(x+h) - F(x)}{h} = F'(x). \]
\end{proof}
\begin{corollary} [Weak second fundamental theorem of calculus] \label{weakSecondTheoremCalculus}
Let $a,b\in \R$ and $f:\interval{a,b}\to \R$ be a differentiable real function such that $f': \interval{a,b}\to \R$ is continuous. Then, for all $x\in \interval{a,b}$,
\[ f(x) = f(a) + \int_a^xf'(t)\diff{t}. \]
In particular $\int_a^bf'(t)\diff{t} = f(b)-f(a)$.
\end{corollary}
TODO: We can drop the hypothesis that $f'$ is continuous, so long as it is (Riemann) integrable. This strengthening is the second fundamental theorem of calculus.
\begin{proof}
The function $f'$ is integrable by \ref{continuousRealFunctionIntegrable}.
Set $g: \interval{a,b}\to \R: x\mapsto \int_a^xf'(t)\diff{t}$, which is differentiable by the first fundamental theorem and $g'(x) = f'(x)$ for all $x\in \interval[o]{a,b}$. Thus we can apply the means value theorem \ref{meanValueTheorem} to $f-g$, restricted to $\interval{a,x}$: there exists $c\in \interval[o]{a,x}$ such that
\[ \frac{(f-g)(x) - (f-g)(a)}{x-a} = (f-g)'(c) = 0. \]
This implies that $f-g$ is a constant function. In order to fix this constant, note that $g(a) = 0$, so $f(x) = f(a) + g(x) = f(a) + \int_a^xf'(t)\diff{t}$ and, in particular, $f(b) = f(a) + \int_a^bf'(t)\diff{t}$.
\end{proof}

\subsection{Henstock-Kurzweil integral}

\subsection{Riemann integration}
\begin{lemma}
Let $a,b\in \R$ and $f:\interval{a,b}\to \R$ be a continuous real function. Then $f$ is Riemann integrable.
\end{lemma}
\begin{proof}

\end{proof}
\subsubsection{Riemann-Stieltjes}

\begin{lemma} \label{rectanglePartitionedByMonotonicFunction}
Let $f: \R^+ \to \R^+$ be an invertible monotonic function and $a\in\R^+$. Then
\[ a\cdot f(a) = \int_0^a f(x)\diff{x} + \int_0^{f(a)} f^{-1}(y)\diff{y}. \]
\end{lemma}
\begin{proof}
We will be evaluating the integral $a\cdot f(a) = \iint_{(x,y)\in S}\diff{x}\diff{y}$, where $S = \setbuilder{(x,y)\in \R^2}{x\leq a, y\leq f(a)}$. To that end, note the partition
\begin{align*}
S &= \setbuilder{(x,y)\in S}{y\leq f(x)} \uplus \setbuilder{(x,y)\in S}{y > f(x)} \\
&= \setbuilder{(x,y)\in S}{y\leq f(x)} \uplus \setbuilder{(x,y)\in S}{x < f^{-1}(y)} \\
&\eqdef S_1 \uplus S_2.
\end{align*}
Then we can calculate
\begin{align*}
ab = \iint_{(x,y)\in S_1}\diff{x}\diff{y} + \iint_{(x,y)\in S_2}\diff{x}\diff{y} \\
\int_0^a\int_0^{f(x)}\diff{y}\diff{x} + \int_0^{f(a)}\int_0^{f^{-1}(y)}\diff{x}\diff{y} \\
\int_0^a\int_0^{f(x)}\diff{y}\diff{x} + \int_0^{f(a)}\int_0^{f^{-1}(y)}\diff{x}\diff{y} \\
\int_0^a f(x)\diff{x} + \int_0^{f(a)} f^{-1}(y)\diff{y}.
\end{align*}
\end{proof}



\section{Inequalities}
\subsection{Based on mean value theorems}
\begin{proposition}
Let $a,b\in \R$ and $f:\interval{a,b}\to \R$ be a differentiable real function such that $f': \interval{a,b}\to \R$ is continuous. Then
\[ 0 \leq \int_a^b|f(x)|\diff{x} - \left|\int_a^bf(x)\diff{x}\right| \leq \frac{(b-a)^2}{2}\max_{x'\in\interval{a,b}}|f'(x')|. \]
\end{proposition}
TODO: according to \url{https://citeseerx.ist.psu.edu/viewdoc/download?doi=10.1.1.625.7106&rep=rep1&type=pdf}, we can improve the constant to $1/3$.
\begin{proof}
For all $x,c\in \interval{a,b}$, there exists $d\in\interval{a,b}$ such that
\[ |f(x)-f(c)| = |x-c|\,|f'(d)| \leq |x-c|\max_{x'\in \interval{a,b}}|f'(x')| \]
by the mean value theorem \ref{meanValueTheorem}, and the fact that $f'$ achieves its extrema by the extreme value theorem \ref{extremeValueTheorem}. Thus we have
\[ \int_a^b|f(x)|\diff{x} - |f(c)|(b-a) = \int_a^b|f(x)| - |f(c)|\diff{x} \leq \int_a^b|f(x) - f(x)|\diff{x}\leq \int_a^b|x-c|\diff{x}\max_{x'\in \interval{a,b}}|f'(x')|. \]
We can then calculate
\[ \int_a^b|x-c|\diff{x} = \frac{(c-a)^2}{2} + \frac{(c-b)^2}{2} = c^2 -ac-bc + \frac{a^2}{2} + \frac{b^2}{2}, \]
which, in function of $c$, is a parabola that has a maximum at either $a$ or $b$, so $\int_a^b|x-c|\diff{x}\leq \frac{(b-a)^2}{2}$. Finally we note that we can take $c$ such that $f(c) = \frac{1}{b-a}\int_a^bf(x)\diff{x}$ by \ref{meanValueTheoremIntegrals}. Putting everything together yields the result.
\end{proof}

\begin{lemma}
Let $a,b\in \R$ and $f:\interval{a,b}\to \R$ be a differentiable real function such that $f': \interval{a,b}\to \R$ is continuous. Then
\begin{enumerate}
\item $\max_{t\in \interval{a,b}}|f(t)| \leq \int_a^b \frac{|f(x)|}{b-a} + |f'(x)| \diff{x}$;
\item if $f(a) = f(b) = 0$, then $\max_{t\in \interval{a,b}}|f(t)| \leq \frac{1}{2}\int_a^b|f'(x)|\diff{x}$.
\end{enumerate}
\end{lemma}
\begin{proof}
(1) By \ref{meanValueTheoremIntegrals}, there exists $c\in \interval{a,b}$ such that $|f(c)| = \left|\int_a^b\frac{f(x)}{b-a}\diff{x}\right| \leq \int_a^b\frac{|f(x)|}{b-a}\diff{x}$.
For all $t\in \interval{a,c}$, we have, by \ref{weakSecondTheoremCalculus}, $f(c) = f(t) + \int_t^c f'(x)\diff{x}$, so
\[ |f(t)| \leq |f(c)| + \left|-\int_t^c f'(x) \diff{x}\right| \leq |f(c)| + \int_t^c |f'(x)|\diff{x} \leq |f(c)| + \int_a^b |f'(x)|\diff{x}. \]
For all $t\in \interval{c,b}$, we have $f(t) = f(c) + \int_c^t f'(x)\diff{x}$ and thus
\[ |f(t)| \leq |f(c)| + \left|\int_c^t f'(x) \diff{x}\right| \leq |f(c)| + \int_c^t |f'(x)|\diff{x} \leq |f(c)| + \int_a^b |f'(x)|\diff{x}. \]
Then we have
\[ \max_{t\in \interval{a,b}}|f(t)| \leq |f(c)| + \int_a^b |f'(x)|\diff{x} \leq \int_a^b\frac{|f(x)|}{b-a}\diff{x} + \int_a^b |f'(x)|\diff{x}. \]

(2) By \ref{weakSecondTheoremCalculus}, we have, for all $t\in \interval{a,b}$, $f(t) = \int_a^t f'(x) \diff{x}$ and $f(t) = -\int_t^b f'(x)\ diff{x}$, so
\[ 2|f(t)| = \left|\int_a^t f'(x) \diff{x}\right| + \left|\int_t^b -f'(x) \diff{x}\right| \leq \int_a^t|f'(x)|\diff{x} + \int_t^b |f'(x)|\diff{x} = \int_a^b |f'(x)|\diff{x}. \]
\end{proof}

\subsection{Jensen's inequality}
\begin{theorem}
Let $\sSet{\Omega, \mathcal{A}, \mu}$ be a probability measure space, $f:\Omega\to \R$ a real measurable function and $\varphi:\R\to \R$ a convex function. Then
\[ \varphi\Big(\int_\Omega f \diff{\mu}\Big) \leq \int_\Omega \varphi\circ f \diff{\mu}. \]
\end{theorem}
It is important that $\mu(\Omega) = 1$.
\begin{proof}
Let $x_0 = \int_\Omega f \diff{\mu}$. Since $\varphi$ has a subdifferential $m$ at $x_0$ by \ref{existenceSubdifferential}, we have
\[ \varphi\big(f(x)\big) \geq \varphi(x_0) + m\big(f(x) - x_0\big). \]
Integrating over $x$ gives
\begin{align*}
\int_\Omega \varphi\circ f \diff{\mu} &\geq \int_\Omega\varphi(x_0)\diff{\mu} + \int_\Omega m\big(f(x) - x_0\big) \diff{\mu} \\
&= \varphi(x_0)\int_\Omega\diff{\mu} + m\Big(\int_\Omega f(x)\diff{mu} -\int_\Omega x_0 \diff{\mu}\Big) \\
&= \varphi(x_0) + m\Big(\int_\Omega f(x)\diff{mu} -x_0\Big) \\
&= \varphi(x_0) = \varphi\Big(\int_\Omega f \diff{\mu}\Big).
\end{align*}
\end{proof}

\subsection{Young's inequality}
TODO: compare (+use) Jensen's inequality.

\subsubsection{Hölder conjugates}
\begin{definition}
Two numbers $p,q\in \R^+$ are called \udef{Hölder conjugate} iff $p^{-1} + q^{-1} = 1$.
\end{definition}

\begin{lemma}
Let $p\in \R^+$. Then $p$ is Hölder conjugate to some $q\in \R^+$ \textup{if and only if} $p > 1$.
\end{lemma}

\begin{lemma} \label{HoelderConjugateEquivalents}
Let $p,q\in \R^+$. The following are equivalent:
\begin{enumerate}
\item $p$ and $q$ are Hölder conjugate (i.e.\ $p^{-1}+q^{-1} = 1$);
\item $p+q = pq$;
\item $(p-1)(q-1) = 1$;
\item $(p-1) = (q-1)^{-1}$;
\item $p = \frac{q}{q-1}$;
\item $q = \frac{p}{p-1}$;
\item $pq^{-1} = p-1$;
\item $qp^{-1} = q-1$.
\end{enumerate}
\end{lemma}

\subsubsection{Young's inequality}
\begin{theorem}
Let $a\in \R$ and $f: \interval{0,a}\to \interval{0,f(a)}$ be a strictly increasing continuous function such that $f(0) = 0$. Then
\[ xy \leq \int_0^x f(t)\diff{t} + \int_0^y f^{-1}(t)\diff{t} \]
for all $x\in \interval{0,a}$ and $y\in \interval{0,f(a)}$.
\end{theorem}
\begin{proof}/
Define the real function $g$ by
\[ g(x) \defeq \int_0^x f(t)\diff{t} + \int_0^{f(x)}f^{-1}(t)\diff{t} - xf(x). \]
Then $g$ is differentiable by \ref{firstFundamentalTheoremCalculus} and (TODO ref chain rule and product rule). The derivative is given by $g'(x) = 0$ for all $x\in \interval{0,a}$. Since $g(0) = 0$, we have $g(x) = 0$ for all $x\in\interval{0,a}$ by \ref{weakSecondTheoremCalculus}. Thus
\[ xf(x) = \int_0^x f(t)\diff{t} + \int_0^{f(x)}f^{-1}(t)\diff{t}. \]
Now we have
\begin{align*}
xy &= xf(x) - x\big(f(x)-y\big) \\
&= \int_0^x f(t)\diff{t} + \int_0^{y}f^{-1}(t)\diff{t} + \int_y^{f(x)}f^{-1}(t)\diff{t} - x\big(f(x)-y\big) \\
&= \int_0^x f(t)\diff{t} + \int_0^{y}f^{-1}(t)\diff{t} + \int_y^{f(x)}\big(f^{-1}(t) - x\big)\diff{t} \\
&\leq \int_0^x f(t)\diff{t} + \int_0^{y}f^{-1}(t)\diff{t},
\end{align*}
where the inequality follows from the fact that $\int_y^{f(x)}\big(f^{-1}(t) - x\big)\diff{t}$ is always positive: if $t \leq f(x)$, then $f^{-1}(t)\leq x$ and if $f(x)\leq t$, then $x\leq f^{-1}(t)$. Since either $y\leq f(x)$ or $f(x)\leq y$, the integral is positive in both cases.
\end{proof}
\begin{corollary}[Young's inequality] \label{YoungsInequality}
Let $p,q\in \R$ be such that $p>0, q>0$ and $p^{-1} + q^{-1} = 1$. Then for all $a,b \geq 0$, we have
\[ ab \leq \frac{a^p}{p} + \frac{b^q}{q}. \]
Equality holds \textup{if and only if} $b = a^{p-1}$.
\end{corollary}
\begin{proof}
TODO
\end{proof}
\begin{proof}[Proof using convexity of exponentiation]
If either $a=0$ or $b =0$, then the result is immediate. Now assume $a>0$ and $b>0$. Then
\begin{align*}
ab &= e^{\ln(ab)} \\
&= e^{\ln(a)+\ln(b)} \\
&= e^{p^{-1}p\ln(a)+q^{-1}q\ln(b)} \\
&= e^{p^{-1}\ln(a^p)+(1-p^{-1})\ln(b^q)} \\
&\leq p^{-1}e^{\ln(a^p)}+(1-p^{-1})e^{\ln(b^q)} \\
&= \frac{a^p}{p} + \frac{b^q}{q}.
\end{align*}
The inequality is due to the convexity of the exponentiation. It is an equality (TODO ref strict convexity) iff
\begin{align*}
\ln(b^q) = \ln(a^p) &\iff b^q = a^p \\
&\iff b = a^{pq^{-1}} = a^{p-1}.
\end{align*}
The last equality follows from \ref{HoelderConjugateEquivalents}.
\end{proof}
\begin{proof}[Geometric proof]
From \ref{HoelderConjugateEquivalents}, we have $(x\mapsto x^{p-1})^{-1} = y\mapsto y^{q-1}$. These functions are monotone (TODO ref).

We have either $a^{p-1}\leq b$ or $b\leq a^{p-1} \iff b^{q-1} \leq a$. If the second option holds, we swap $a$ with $b$ and $p$ with $q$.

Now we can calculate, using \ref{rectanglePartitionedByMonotonicFunction},
\begin{align*}
ab &= aa^{p-1} + a(b-a^{p-1}) \\
&= \int_0^a x^{p-1}\diff{x} + \int_0^{a^{p-1}} y^{q-1}\diff{y} + a(b-a^{p-1}) \\
&= \int_0^a x^{p-1}\diff{x} + \int_0^{a^{p-1}} y^{q-1}\diff{y} + \int_{a^{p-1}}^b a\diff{y} \\
&\leq \int_0^a x^{p-1}\diff{x} + \int_0^{a^{p-1}} y^{q-1}\diff{y} + \int_{a^{p-1}}^b y^{q-1}\diff{y} \\
&= \int_0^a x^{p-1}\diff{x} + \int_0^{b} y^{q-1}\diff{y} \\
&= \frac{a^p}{p} + \frac{b^q}{q}.
\end{align*}
TODO picture.

We clearly have equality iff $a^{p-1} = b$.
\end{proof}

\section{Non-standard analysis}
\url{http://www.lightandmatter.com/calc/}

\begin{proposition}
Let $f:\R\to\R$ be a real function. Then $f$ is continuous at $x\in\R$ \textup{if and only if} for all infinitesimal $\delta$ there exists an infinitesimal $\epsilon$ such that
\[ f(x+\delta) = f(x) + \epsilon. \]
Alternatively we can state this as
\[ f(x+\delta) \approx f(x) \]
for all infinitesimal $\delta$.
\end{proposition}

Clearly continuity is a requirement for differentiability: if $f(x+\delta) - f(x)$ is not infinitesimal, then $\frac{f(x+\delta) - f(x)}{\delta}$ will not be finite.

\begin{lemma} \label{chainLemma}
Let $f:\R\to\R$ be a real function and $y:\R\to\R$ a continuous function. Assume $f$ differentiable at $y_0\in\im(y)$. Consider $f$ as depending on $y$ and $y$ as depending on $x$. Then
\[ \left.\dod{f}{y}\right|_{y_0} = \st\left(\frac{\Delta_y f}{\Delta y}\right) = \st\left(\frac{\Delta_x f\circ y}{\Delta_x y}\right). \]
\end{lemma}
\begin{proof}
We calculate, setting $y_0 = y(x_0)$ and using continuity of $y$,
\begin{align*}
\st\left(\frac{\Delta_x f\circ y}{\Delta_x y}\right) &= \st\left(\frac{f(y(x_0+\Delta x)) -f(y(x_0))}{y(x_0+\Delta x) - y(x_0)}\right) = \st\left(\frac{f(y(x_0)+\delta) -f(y(x_0))}{y(x_0)+\delta - y(x_0)}\right) \\
&= \st\left(\frac{f(y(x_0)+\delta) -f(y(x_0))}{\delta}\right) = \left.\dod{f}{y}\right|_{y_0}.
\end{align*}
\end{proof}

\begin{proposition}[Chain rule]
Let $y,f$ be real functions, differentiable at points $x_0$ and $y_0=y(x_0)$, respectively. Then
\[ \left.\dod{f}{y}\right|_{y_0} \left.\dod{y}{x}\right|_{x_0} = \left.\dod{f\circ y}{x}\right|_{x_0}. \]
\end{proposition}
\begin{proof}
We calculate, using \ref{chainLemma},
\[ \left.\dod{f}{y}\right|_{y_0} \left.\dod{y}{x}\right|_{x_0} = \st\left(\frac{\Delta_y f}{\Delta y}\frac{\Delta_x y}{\Delta x}\right) = \st\left(\frac{\Delta_x f\circ y}{\Delta_x y}\frac{\Delta_x y}{\Delta x}\right) = \st\left(\frac{\Delta_x f\circ y}{\Delta x}\right) = \left.\dod{f\circ y}{x}\right|_{x_0}. \]
\end{proof}

\chapter{TODO Formal and function series}
TODO for general convergence fields??
\section{Power series}
\begin{definition}
A \udef{power series} is a partial function of the form
\[ f: \C\not\to \C: z\mapsto \sum_{i=0}^\infty a_i(z-z_0)^i, \]
where $z_0\in \C$ and $\seq{a_n}$ is a sequence of complex numbers. 
\end{definition}
TODO more general algebras.

\begin{proposition}[Cauchy-Hadamard] \label{CauchyHadamard}
Let $f: \C\not\to \C: z\mapsto \sum_{i=0}^\infty a_i(z-z_0)^i$ be a power series. Define the real number $R$ by
\[ \frac{1}{R} \defeq \limsup_{n\to\infty}\left(|a_n|^{1/n}\right). \]
For all $z\in \C$ such that $|z-z_0| < R$, the value $f(z)$ is well-defined.
\end{proposition}
If $\limsup_{n\to\infty}\left(|a_n|^{1/n}\right) \to \infty$, we consider $R$ to be zero.
\begin{proof}
\ref{rootTest} TODO.
\end{proof} 

\begin{definition}
The $R$ defined in \ref{CauchyHadamard} is called the \udef{radius of convergence} of the power series.
\end{definition}

\subsection{Taylor and MacLaurin}
\section{Laurent series}
\begin{definition}
A \udef{Laurent series} is a partial function of the form
\[ f: \C\not\to \C: z\mapsto \sum_{i=-\infty}^\infty a_i(z-z_0)^i \defeq \sum_{i=0}^\infty a_i(z-z_0)^i + a_{-i}(z-z_0)^{-i}, \]
where $z_0\in \C$ and $\seq{a_n}, \seq{a_{-n}}$ are sequences of complex numbers.

The series $\sum_{i=1}^\infty a_{-i}(z-z_0)^{-i}$ is called the \udef{principal part} of the Laurent series.
\end{definition}

\begin{proposition} \label{LaurentSeriesConvergence}
Let $f: \C\not\to \C: z\mapsto \sum_{i=-\infty}^\infty a_i(z-z_0)^i$ be a Laurent series. Define the real numbers $r$ and $R$ by
\begin{align*}
r &\defeq \limsup_{n\to\infty}\left(|a_{-n}|^{1/n}\right); \\
\frac{1}{R} &\defeq \limsup_{n\to\infty}\left(|a_n|^{1/n}\right).
\end{align*}
For all $z\in \C$ such that $r < |z-z_0| < R$, the value $f(z)$ is well-defined.
\end{proposition}
So the domain of convergence of a Laurent series is an annulus around $z_0$.

\section{Puiseux series}
\url{https://en.wikipedia.org/wiki/Puiseux_series}

\chapter{Complex analysis}
\begin{definition}
A \udef{complex function} is a function in $(U\subseteq \C \to \C)$.

A \udef{region} of $\C$ is an open, connected subset of $\C$.
\end{definition}

\section{Prerequisites}
TODO: move
\subsection{The complex numbers as a real algebra}
The set of complex numbers $\C$ is isomorphic to the algebra $\R^2$ where multiplication is defined by
\[ (a, b)\cdot (x, y) \defeq (ax-by, ay+bx). \]
Thus the regular representation $\lambda_{(a,b)}$ has matrix
\[ \begin{pmatrix}
a & -b \\ b & a
\end{pmatrix}. \]

\subsection{Curves and regions in $\C$}
\subsubsection{Jordan curves}
\begin{definition}
A \udef{Jordan curve} in $\C$ is a continuous injective function $\gamma: \interval{0,1}\to \C$ such that $\gamma(0) = \gamma(1)$.
\end{definition}
Sometimes people use ``Jordan curve'' to mean a continuous injective function $\gamma: \interval{0,1}\to \C$, without the condition $\gamma(0) = \gamma(1)$. Then our Jordan curve is called a ``closed Jordan curve''.

\begin{theorem}
Let $\gamma$ be a Jordan curve. Then $\C\setminus\im(\gamma)$ has a separation $(I,E)$, such that
\begin{enumerate}
\item $I$ and $E$ are connected;
\item $I$ is bounded;
\item $E$ is unbounded;
\item the boundary of both $I$ and $E$ in $\C$ is $\im(\gamma)$.
\end{enumerate}
\end{theorem}
\begin{proof}
\url{https://www.jstor.org/stable/2323369?origin=crossref&seq=3}
\end{proof}

\subsubsection{Rectifiable curves}
\begin{definition}
A curve $\gamma: \interval{0,1}\to \C$ is called \udef{rectifiable} if it is of bounded variation.
\end{definition}
TODO: rectifiable curves are exactly the curves
(1) that we can Stieltjes-integrate over and (2) have defined arc-length.


\section{Holomorphic functions}
\begin{definition}
Let $f:U\subseteq \C \to \C$ be a complex function. We say
\begin{enumerate}
\item $f$ is \udef{holomorphic  at $z\in U$} if the limit $\lim_{h\to 0} \dfrac{f(z+h) - f(z)}{h}$
exists;
\item $f$ is \udef{holomorphic in $S\subset U$} if it is holomorphic at every point in $S$;
\item $f$ is \udef{holomorphic} if it is holomorphic at every point in $U$;
\item $f$ is \udef{entire} if $U=\C$ and it is holomorphic at every point in $\C$.
\end{enumerate}
We denote by $\holomorphic(A)$ the set of functions that are holomorphic on a neighbourhood of $A$.
\end{definition}
Note that the functions in $\holomorphic(A)$ do not in general have the same domains!!

\begin{lemma}
Holomorphic functions are continuous.
\end{lemma}
\begin{lemma}
Let $f,g$ be holomorphic. Then
\begin{enumerate}
\item $f+g$ is holomorphic and $(f+g)' = f'+g'$;
\item $fg$ is holomorphic and $(fg)' = f'g+fg'$;
\item if $g(z_0)\neq 0$, then $f/g$ is holomorphic at $z_0$ and
\[ (f/g)' = \frac{f'g - fg'}{g^2}; \]
\item the chain rule holds.
\end{enumerate}
\end{lemma}

\subsection{Cauchy-Riemann equations}
The space of complex numbers $\C$ is a real $2$-dimensional vector space.

\begin{lemma}
Let $f: U\subseteq \C \to C$ be a complex function. Then $f$ is holomorphic \textup{if and only if} it is Fréchet differentiable as a function $F: V\subseteq \R^2 \to \R^2$ and $\diff F$ is the regular representation of some $z\in \C$.
\end{lemma}
\begin{corollary}[Cauchy-Riemann equations]
Let $f: U\subseteq \C \to C$ be a complex function. Then $f$ is complex differentiable at $z_0 = a+bi$ \textup{if and only if} the real derivatives
\[ \pd{\Re(f(a + ib))}{a},\; \pd{\Re(f(a + ib))}{b},\;
\pd{\Im(f(a + ib))}{a} \;\;\text{and}\;\; \pd{\Im(f(a + ib))}{b}. \]
exist, are continuous and the equations
\[
\pd{\Re(f(a + ib))}{a} = \pd{\Im(f(a + ib))}{b} \quad\text{and}\quad \pd{\Re(f(a + ib))}{b} = -\pd{\Im(f(a + ib))}{a} \]
hold. In particular
\[ |f'(z_0)|^2 = \det J_f(a,b) \]
where $\det J_f(a,b)$ is the Jacobian determinant of $f$ as a function $V\subseteq \R^2 \to \R^2$.
\end{corollary}
\begin{proof}
We need the Jacobian
\[ \begin{pmatrix}
\pd{\Re(f(a + ib))}{a} & \pd{\Re(f(a + ib))}{b} \\
\pd{\Im(f(a + ib))}{a} & \pd{\Im(f(a + ib))}{b}
\end{pmatrix} \]
to be of the form $\begin{pmatrix}
x & -y \\ y & x
\end{pmatrix}$ in order for it to be a regular representation of some $f' =\pd{\Re(f(a + ib))}{a} + i\pd{\Im(f(a + ib))}{a} \in \C$. TODO ref requirement continuously differentiable.

The Jacobian determinant is a simple calculation:
\begin{align*}
\det J_f(a,b) &= \pd{\Re(f(a + ib))}{a}\pd{\Im(f(a + ib))}{b} - \pd{\Im(f(a + ib))}{a}\pd{\Re(f(a + ib))}{b} \\
&= \pd{\Re(f(a + ib))}{a}^2 + \pd{\Im(f(a + ib))}{a}^2 = |f'|^2.
\end{align*}
\end{proof}
\begin{corollary}
Let $f: U\subseteq \C \to C$ be a holomorphic function. Consider the function $f_1: V\subseteq \R^2 \to \R: (a,b) \mapsto \Re(f(a+ib))$ and $f_2: V\subseteq \R^2 \to \R: (a,b) \mapsto \Im(f(a+ib))$. Then $\nabla^2 f_1 = 0$ and $\nabla^2 f_2 = 0$.
\end{corollary}
Thus both the real and imaginary part satisfy Laplace's equation. We already use \ref{holomorphicFunctionIsAnalytic}, even though it is proved later. This corollary is not needed to establish \ref{holomorphicFunctionIsAnalytic}.
\begin{proof}
We calculate, using Schwarz's theorem \ref{SchwarzTheorem} and the fact that $f$ is infinitely differentiable \ref{holomorphicFunctionIsAnalytic},
\[ \nabla^2 f_1 = \pd[2]{f_1}{a} + \pd[2]{f_1}{b} = \pd{}{a}\Big(\pd{f_2}{b}\Big) - \pd{}{b}\Big(\pd{f_2}{a}\Big) = 0. \]
A similar calculation gives $\nabla^2 f_2 = 0$.
\end{proof}

\subsection{Cauchy's theorem, Morera's theorem and the integral formula}

\begin{theorem}[Cauchy's theorem] \label{CauchyTheorem}
Let $f: U\subseteq \C \to \C$ be a holomorphic complex function and $\gamma$ a rectifiable Jordan curve whose interior lies in $U$. Then
\[ \oint_\gamma f(z)\diff{z} = 0. \]
\end{theorem}
TODO: looser requirements for $\gamma$
\begin{proof}
TODO: generalised Stokes + Cauchy-Riemann!
\end{proof}
\begin{corollary}
Let $f: U\subseteq \C \to \C$ be a holomorphic complex function on a region $U$. Then there exists a primitive $F: U\to \C$ such that $\od{F}{z} = f$.
\end{corollary}
\begin{proof}
Take $x_0\in U$ and define 

The integral $\int_\gamma f(z)\diff{z}$ depends only on the endpoints of $\gamma$.
\end{proof}

\begin{theorem}[Morera's theorem]
Let $f: U\subseteq \C \to \C$ be a complex function on an open set $U$ such that
\[ \oint_\gamma f(z) \diff{z} = 0 \]
for every triangle $\gamma$ in $U$, then $f$ is holomorphic.
\end{theorem}
\begin{proof}
$\int f(z)\diff{z}$ is a primitive, so $f$ is complex differentiable. TODO
\end{proof}


\begin{note}
Positive orientation is counterclockwise. I.e. $e^{it}$ rotates with positive orientation.

Local view: positive orientation if interior is on the left.
\end{note}

\begin{theorem}[Cauchy's integral formula] \label{CauchyIntergralFormula}
Let $f: U\subseteq \C \to \C$ be a holomorphic complex function and $\gamma$ a positively oriented Jordan curve whose interior lies in $U$. Then
\[ f(z) = \frac{1}{2\pi i}\oint_\gamma \frac{f(\zeta)}{\zeta - z}\diff{\zeta} \]
for any point $z$ in the interior of $\gamma$.
\end{theorem}
\begin{proof}
Let $C_{z,\epsilon}$ be a small circle inside $\gamma$ around $z$ of radius $\epsilon$ and opposite orientation (TODO explicate!). Then $\frac{f(\zeta)}{\zeta - z}$ is holomorphic in the region between $\gamma$ and $C_{z,\epsilon}$, so
\begin{align*}
\oint_\gamma \frac{f(\zeta)}{\zeta - z}\diff{\zeta} &= -\oint_{C_{z,\epsilon}} \frac{f(\zeta)}{\zeta - z}\diff{\zeta} \\
&= -\oint_{C_{z,\epsilon}} \frac{f(\zeta)-f(z)}{\zeta - z} + \frac{f(z)}{\zeta - z}\diff{\zeta} \\
&= -\oint_{C_{z,\epsilon}} \frac{f(\zeta)-f(z)}{\zeta - z}\diff{\zeta} -\oint_{C_{z,\epsilon}}\frac{f(z)}{\zeta - z}\diff{\zeta}.
\end{align*}
In the limit $\epsilon \to 0$, the first part is bounded by
\[ \left|\oint_{C_{z,\epsilon}} \frac{f(\zeta)-f(z)}{\zeta - z}\diff{\zeta}\right| \leq \sup_\zeta\left|\frac{f(\zeta)-f(z)}{\zeta - z}\right| \cdot |2\pi\epsilon| \to 0 \]
because $\frac{f(\zeta)-f(z)}{\zeta - z}$ remains bounded. For the second part, we have
\begin{align*}
-\oint_{C_{z,\epsilon}} \frac{f(z)}{\zeta - z}\diff{\zeta} &= f(z)\oint_{C_{z,\epsilon}} \frac{\diff{\zeta}}{\zeta - z} \\
&= f(z)\int_0^{2\pi}\frac{\epsilon ie^{-it}}{\epsilon e^{-it}}\diff{t} \\
&= f(z)2\pi i.
\end{align*}
\end{proof}
\begin{corollary}
Let $f: U\subseteq \C \to \C$ be a holomorphic complex function. Then $f$ has infinitely many complex derivatives and
\[ f^{(n)}(z) = \frac{n!}{2\pi i}\oint_\gamma \frac{f(\zeta)}{(\zeta-z)^{n+1}}\diff{\zeta} \]
for all $z$ in the interior of $\gamma$.
\end{corollary}
\begin{proof}
TODO
\end{proof}

\begin{proposition} \label{holomorphicFunctionIsAnalytic}
Let $f$ be holomorphic in an open set $\Omega$ and $D$ a
disc centered at $z_0$ whose closure is contained in $\Omega$. Then $f$ has a power series expansion at $z_0$
\[ f(z) = \sum_{n=0}^\infty a_n(z-z_0)^n \]
for all $z\in D$. The coefficients are given by
\[ a_n = \frac{f^{(n)}(z_0)}{n!} \qquad \text{for all $n\geq 0$}. \]
\end{proposition}
\begin{proof}
TODO
\end{proof}
\begin{corollary}
Let $f: \Omega\subseteq \C \to \C$ be a function and $\Omega$ an open set. Then $f$ is holomorphic \textup{if and only if} $f$ is analytic.
\end{corollary}
\begin{corollary}
If $f$ is holomorphic in an open
set that contains the closure of a disc $D$ centered at $z_0$ and of radius $R$,
then
\[  |f^{(n)}(z_0)| \leq \frac{n!\norm{f}_C}{R^n}, \]
where $\norm{f}_C = \sup_{z\in C}|f(z)|$.
\end{corollary}
\begin{proof}
TODO
\end{proof}
Thus the distance from $z_0$ to the nearest singular point is the radius of convergence of the power series.
\begin{corollary}[Liouville's theorem] \label{liouvilleTheoremAnalysis}
If $f$ is entire and bounded, then $f$ is contant.
\end{corollary}
\begin{proof}
It suffices to show that $f'(z_0) = 0$. This can be seen by taking $R\to\infty$ in the previous inequality.
\end{proof}
\begin{corollary}[Fundamental theorem of algebra]
Let $P(z) = a_nz^n + \ldots + a_0$ be a polynomial of degree $n\geq 1$ with complex coefficients. Then $P(z)$ has precisely $n$ roots. If these roots are denoted $w_1, \ldots, w_n$, then we can write
\[ P(z) = a_n(z-w_1)(z-w_2)\ldots(z-w_n). \]
\end{corollary}

\begin{proposition}
Let $f: \Omega\subseteq \C\to \C$ be a complex function that is holomorphic in a region $U\subseteq \Omega$ that contains a closed annulus $\setbuilder{z\in \C}{r\leq|z-z_0|\leq R}$ for some $z_0\in \C$ and $r,R\in \R$. Then $f$ has a Laurent series expansion that converges in the interior of the annulus:
\[ f(z) = \sum_{i=-\infty}^\infty a_i(z-z_0)^i \]
for all $z\in \setbuilder{z\in \C}{r < |z-z_0| < R}$.
\end{proposition}
\begin{proof}
TODO
\end{proof}
TODO uniqueness ??? \url{https://en.wikipedia.org/wiki/Laurent_series#Uniqueness}

\subsection{Analytic continuation}
\begin{proposition}
Let $f:U\subseteq \C\to \C$ be a holomorphic function defined on a region. Suppose there exists a sequence of distinct points with limit point in $U$ on which $f$ vanishes. Then $f = \underline{0}$.
\end{proposition}
\begin{proof}
TODO
\end{proof}
\begin{corollary} \label{zerosHolomorphicFunctionIsolated}
Let $f:U\subseteq \C\to \C$ be a holomorphic function defined on a region. Then the zeros of $f$ are isolated.
\end{corollary}
\begin{proof}
TODO
\end{proof}
\begin{corollary} \label{holomorphicFunctionsCoincidingOnSetWithLimitPoint}
Let $f,g:U\subseteq \C\to \C$ be holomorphic functions defined on a common region. Suppose there exists a sequence of distinct points $\seq{z_n}$ with limit point in $U$ such that $f(z_n) = g(z_n)$ for all $n$. Then $f = g$.
\end{corollary}
In particular this holds if $f$ and $g$ agree on some non-empty open subset of $U$.
\begin{corollary}
Let $U\subseteq U' \subseteq \C$ be regions and $f:U\subseteq\to \C$ a holomorphic function. Then there exists at most one holomorphic function $f'$ on $U'$ such that $f'|_U = f$.
\end{corollary}

\begin{definition}
In this case the function $f'$ is called an \udef{analytic continuation} of $f$ into $\Omega'$.
\end{definition}

\subsubsection{Constructing analytic continuations}
\begin{lemma}
Let $f: U\subseteq \C\to \C$ be a holomorphic function on an open set and $z_0\in U$. Let $R$ be the radius of convergence of the Taylor series of $f$ at $z_0$. Then $f$ has an analytic continuation to $U\cup \ball_{z_0, R}$, which is defined by the Taylor series on $\ball_{z_0, R}\setminus U$.
\end{lemma}
\begin{proof}
The Taylor series is holomorphic on the interior of its disk of convergence, by \ref{powerSeriesHolomorphic}.

Since $z_0\in U$ and $U$ is open, there exists a disk $\ball(z_0, \epsilon)$ whose closure lies in $U$. Then the Taylor series and $f$ coincide on this disk by \ref{holomorphicFunctionIsAnalytic}.

Since this disk contains a limit point, $f$ and the Taylor series coincide on $U\cap \ball(z_0, R)$ by \ref{holomorphicFunctionsCoincidingOnSetWithLimitPoint}.
\end{proof}

\begin{proposition}[Symmetry principle] \label{symmetryPrinciple}
Let $\Omega$ be an open subset of $\C$ such that $\overline{\Omega} = \Omega$. Denote by $\Omega^+$ the part of $\Omega$ that lies in the upper half plane and by $\Omega^-$ the part lying in the lower half plane. Set $I = \R \cap \Omega$.

Let $f^+: \Omega^+ \to \C$ and $f^-: \Omega^- \to \C$ be holomorphic functions that extend continuously to $I$ and
\[ \forall x\in I: \; f^+(x) = f^-(x). \]
Then the compound function
\[ f: \Omega \to \C: z\mapsto f(z) = \begin{cases}
f^+(z) & z\in \Omega^+ \\
f^+(z) = f^-(z) & z\in I \\
f^-(z) & z\in \Omega^-
\end{cases} \]
is holomorphic on all of $\Omega$.
\end{proposition}

\begin{proposition}[Schwarz reflection principle]
Let $\Omega, \Omega^+, \Omega^-$ and $I$ be as in \ref{symmetryPrinciple} and $f: \Omega^+ \to \C$ a holomorphic function that extends continuously to $I$. Then there exists a holomorphic function $g: \Omega\to \C$ such that $g|_{\Omega^+} = f$.
\end{proposition}
\begin{proof}
Idea: define $g|_{\Omega^-}(z) = \overline{f(\overline{z})}$. TODO details.
\end{proof}

\subsubsection{Analytic continuation along a curve}
\begin{theorem}[Monodromy theorem]
Let $f:D\subseteq \C \to \C$ be a holomorphic function and $\Omega$ the set of all points that admit an analytic continuation.

If two curves are homotopic in $\Omega$, then they give the same analytic continuation.
\end{theorem}

\subsection{Limits and holomorphic functions}
\begin{proposition}
Let $\seq{f_n}$ be a sequence of holomorphic functions on $\Omega$ that converges uniformly to a function $f$ in every compact subset of $\Omega$. Then
\begin{enumerate}
\item $f$ is holomorphic in $\Omega$;
\item $\seq{f_n'}$ converges uniformly to $f'$ on every compact subset of $\Omega$;
\item for all $k\in \N$, the sequence $\seq{f_n^{(k)}}$ converges uniformly to $f^{(k)}$ on every compact subset of $\Omega$.
\end{enumerate}
\end{proposition}
\begin{proof}
(1) TODO (+ \url{https://www.math.wustl.edu/~sk/limits.pdf})

(2) TODO

(3) By induction on $k$.
\end{proof}
TODO: this is part of the motivation for the compact-open topology (use this terminology?).

\begin{proposition}
Let $\Omega$ be a open set in $\C$ and $F: \Omega\times[0,1]\to \C$ a function such that
\begin{itemize}
\item $F(\cdot,s)$ is holomorphic (in the first variable) for all $s\in [0,1]$;
\item $F$ is continuous.
\end{itemize}
Then the function
\[ f: \Omega \to \C: z\mapsto f(z) = \int_0^1F(z,s)\diff{s} \]
is holomorphic.
\end{proposition}

\begin{proposition}[Runge's approximation theorem]
Let $K\subseteq \C$ be compact.
\begin{enumerate}
\item Any function holomorphic in a neighbourhood of $K$ can be approximated uniformly on $K$ by a sequence of rational functions whose singularities are in $K^c$.
\item If $K^c$ is connected, then the approximating functions can be taken to be polynomials.
\item If $K^c$ is not connected, then there exists a function $f$ holomorphic on a neighbourhood of $K$ that cannot be approximated uniformly by polynomials on $K$.
\end{enumerate}
\end{proposition}
If inner and outer part of annulus then Laurent series (i.e.\ particular form of approximating rationals)


\begin{proposition}[Mergelyan's theorem]
The approximating functions in Runge's approximation theorem can be taken to be polynomials \textup{if and only if} $K^c$ is connected.
\end{proposition}

\section{Singularities}
\subsection{Laurent series}
\begin{proposition}
Let $f:\Omega \to \C$ be a holomorphic function and let $\Omega$ contain two concentric circles with center $z_0$ and radii $0<r< R$. Then $f$ has a Laurent expansion
\[ f(z) = \sum_{n=-\infty}^{\infty}a_n(z-z_0)^n \]
that converges on the annulus between the concentric circles.
\end{proposition}
\begin{proof}
Call the outer circle $C_1$ and the inner $C_2$. Then we can use the integral formula to write
\[ f(z) = \frac{1}{2\pi i}\oint_{C_1}\frac{f(\zeta)}{\zeta - z}\diff{\zeta} - \frac{1}{2\pi i}\oint_{C_2}\frac{f(\zeta)}{\zeta - z}\diff{\zeta} \]
TODO
\end{proof}

\subsection{Isolated singularities}

\subsection{Non-isolated singularities}

\section{Meromorphic functions}
\subsection{Zeros and poles}
\begin{lemma} \label{holomorphicZeroLemma}
Let $f:\Omega\to \C$ be a holomorphic function on an open set $\Omega$ that has a zero at $z_0\in \Omega$ and does not vanish identically on any neighbourhood of $z_0$. Then there exists an open neighbourhood $U\subseteq \Omega$ of $z_0$ such that
\[ f|_U(z) = (z-z_0)^ng(z) \]
for all $z\in U$, some holomorphic $g:U\to \C\setminus\{0\}$ and $n\in \N$.

Additionally, the $n$ is uniquely determined by $f$ and $z_0$; it is independent of $U$.
\end{lemma}
\begin{proof}
TODO
\end{proof}

In \ref{holomorphicZeroLemma}, the restriction to $U$ is there to make sure that $g$ is nowhere zero. If all we want is for $g$ to not be zero at $z_0$, we extend it to all of $\Omega$.

\begin{lemma} \label{holomorphicZeroFactorisationLemma}
Let $f:\Omega\to \C$ be a holomorphic function on an open set $\Omega$ that has a zero at $z_0\in \Omega$.
Then
\begin{enumerate}
\item there exists a holomorphic function $g: \Omega \to \C$ such that $f(z) = (z-z_0)g(z)$ for all $z\in \Omega$;
\item if $f$ does not vanish identically on any neighbourhood of $z_0$, then there exists a holomorphic function $g: \Omega \to \C$ and $n\in \N$ such that
\begin{itemize}
\item $f(z) = (z-z_0)^ng(z)$ for all $z\in \Omega$;
\item $g(z_0) \neq 0$.
\end{itemize}
\end{enumerate}
\end{lemma}
\begin{proof}
We first prove (2), then (1).

(2) Pick some open disk around $z_0$ that lies in $\Omega$. Apply \ref{holomorphicZeroLemma} to obtain some $U,n,g'$. Take $\epsilon >0$ such that $\ball(z_0, \epsilon) \subseteq U$. On $\ball(z_0, \epsilon)$, define $g(z) \defeq g'(z)$. On $\Omega\setminus\cball(z_0, \epsilon /2)$, define $g(z) \defeq \frac{f(z)}{(z-z_0)^n}$. Where these two definitions overlap, they yield the same function. Clearly, the resulting $g$ is holomorphic and $g(z_0) = g'(z_0) \neq 0$.

(1) If $f$ does not vanish identically on any neighbourhood of $z_0$, then (1) follows from (2).

Now suppose $f$ vanishes identically on some neighbourhood $U\subseteq \Omega$. Then set $g(z) = 0$ for all $z\in U$ and $g(z) = \frac{f(z)}{z-z_0}$ for $z\neq z_0$. Where these two definitions overlap, they yield the same function. The resulting $g$ is holomorphic.
\end{proof}

\begin{definition}
Let $f:\Omega\to \C$ be a function on an open set $\Omega$ and $z_0\in \Omega$ such that $f$ is not identically zero on any neighbourhood of $z_0$.
\begin{itemize}
\item If $f$ is holomorphic with a zero at $z_0$, the $n$ in \ref{holomorphicZeroLemma} is called the \udef{multiplicity} or \udef{order} of the zero $z_0$.
\item If $\lim_{z\to z_0} 1/f(z) = 0$ and is $1/f$ holomorphic in a neighbourhood of $z_0$, then $z_0$ is called a \udef{pole} of $f$. The $n$ in the expansion \ref{holomorphicZeroLemma} of $1/f$ is called the \udef{multiplicity} or \udef{order} of the pole $z_0$.
\end{itemize}
If $n = 1$, we call the zero or pole \udef{simple}. 
\end{definition}

Poles lie in $\overline{\Omega}\setminus\Omega$. TODO sort out definition meromorphic function.

\begin{lemma}
Let $f:\Omega\to \C$ be a function on an open set $\Omega$ that has a pole at $z_0\in \overline{\Omega}$. Then there exists a neighbourhood $U$ of $z_0$ such that $f$ is holomorphic on $U\setminus\{z_0\}$.
\end{lemma}
This means a pole in necessarily an isolated singularity (TODO ref later)

\begin{definition}
Let $f: \Omega\subseteq \C \to \C$ be a function. We call $f$ \udef{meromorphic} if either $f$ or $1/f$ is holomorphic at each point $z\in \Omega$.
\end{definition}

\begin{lemma}
Let $f: \Omega\subseteq \C \to \C$ be a function. Then $f$ is meromorphic \textup{if and only if} for each point $z\in \Omega$, either $f$ is holomorphic or $z$ is a pole.
\end{lemma}

\subsection{Laurent series and residues}
\begin{proposition}
Let $f:\Omega\to \C$ be a function on an open set $\Omega$ that has a pole of order $n$ at $z_0$, then the Laurent series of $f$ at $z_0$ is of the form
\[  f(z) = \sum_{i=-n}^\infty a_i(z-z_0)^i \]
and converges on a punctured disk $B(z_0, \epsilon)\setminus\{z_0\}$.
\end{proposition}
\begin{proof}
By \ref{holomorphicZeroLemma} we have that 
\[ (1/f)|_{B(z_0,\epsilon)}(z) = (z-z_0)^ng(z) \]
for some holomorphic, non-vanishing $g$. Thus $f|_{B(z_0,\epsilon)}(z) = (z-z_0)^{-n}1/g(z)$. Now $1/g$ is holomorphic, so we can expand it as a power series. This makes the product a Laurent series starting at $-n$.

To conclude, we remark that the inner radius of the annulus on which the Laurent series converges is zero: by \ref{LaurentSeriesConvergence}
\[ r = \limsup_{i\to\infty}\left(|a_{-i}|^{1/i}\right) = 0. \]
\end{proof}
TODO: pole iff Laurent series has finite principal part!

\subsubsection{Partial fraction decomposition}
\begin{proposition}[Partial fraction decomposition] \label{partialFractionDecomposition}
Let $f: \Omega\subseteq \C \to \C$ be a meromorphic function with finitely many poles $z_0, \ldots z_n$. Let $P_{z_k}(z)$ be the principal part of the Laurent expansion around $z_k$. Then we have the decomposition
\[ f(z) = \sum_{k=0}^nP_k(z) + F(z) \]
where $F(z)$ is a holomorphic function.
\end{proposition}
\begin{proof}
The function $F(z) = f(z) - \sum_{k=0}^nP_k(z)$ has no poles and thus is holomorphic. Indeed, for all $z\in\Omega \setminus \{z_0, \ldots z_n\}$, both $f(z)$ and $\sum_{k=0}^nP_k(z)$ are holomorphic, meaning that $F(z)$ is too.

Take some pole $z_k$. Then the Laurent series of $F(z)$ around $z_k$ is of the form
\[ F(z) = P_k(z) + \sum_{i=0}^\infty a_{k,i}(z-z_k)^i - \sum_{j=0}^n P_j(z) = \sum_{i=0}^\infty a_{k,i}(z-z_k)^i - \sum_{\substack{j=0\\ j\neq k}}^n P_j(z), \]
which has no singularity at $z_k$.
\end{proof}

TODO link with integrals of fractional functions.

\subsubsection{Residues}
\begin{definition}
Let $f: \Omega\subseteq \C \to \C$ have a pole at $z_0$. Given the Laurent series expansion
\[  f(z) = \sum_{i=-n}^\infty a_i(z-z_0)^i, \]
the coefficient $a_{-1}$ is called the \udef{residue} at the pole $z_0$. We write $\Res_{z_0}f \defeq a_{-1}$.
\end{definition}

\begin{proposition}
Let $f$ have a pole of order $n$ at $z_0$, then
\[ \Res_{z_0}f = \lim_{z\to z_0}\frac{1}{(n-1)!}\left(\od{}{z}\right)^{n-1}(z-z_0)^nf(z). \]
\end{proposition}
\begin{proof}
This follows straight from the series expansion
\[ (z-z_0)^nf(z) = a_{-n} + a_{-n+1}(z-z_0) + \ldots + a_{-1}(z-z_0)^{n-1} + a_0(z-z_0)^n + \ldots \]
\end{proof}
\begin{corollary}
If $f$ has a simple pole at $z_0$, then $\Res_{z_0} f = \lim_{z\to z_0}(z-z_0)f(z)$.
\end{corollary}

\begin{proposition}
Let $f: \Omega\subseteq \C\to \C$ be a meromorphic function that has one pole at $z_0$. Let $\gamma$ be a simple curve such that $z_0\in \int\gamma \subseteq \Omega$. Then
\[ \oint_\gamma f(z) \diff{z} = 2\pi i \Res_{z_0}f. \]
\end{proposition}
\begin{proof}
By truncated Laurent expansion. TODO
\end{proof}
\begin{corollary}[Residue formula] \label{residueFormula}
Let $f: \Omega\subseteq \C\to \C$ be a meromorphic function and $\gamma$ a simple curve that encompasses $N$ poles $z_1, \ldots, z_N$. Then
\[ \oint_{\gamma} f(z)\diff{z} = 2\pi i \sum_{k=1}^N\Res_{z_k}f. \]
\end{corollary}

\subsection{The argument principle}
\begin{proposition}[Argument principle]
Let $f$ be a meromorphic function in an open set containing a simple curve $\gamma$ and its interior. Assume $f$ has no poles or zeros on $\gamma$. Then
\[ \frac{1}{2\pi i}\oint_\gamma \frac{f'(z)}{f(z)}\diff{z} = \text{number of zeros of $f$ inside $\gamma$} \;-\; \text{number of poles of $f$ inside $\gamma$} \]
where the poles and zeros are counted with their multiplicities.
\end{proposition}
\begin{proof}
TODO
\end{proof}

\begin{theorem}[Rouché's theorem] \label{RoucheTheorem}
Let $f$ and $g$ be holomorphic functions on an open set that contains a closed simple curve $\gamma$ and its interior. If
\[ |f(z)| > |g(z)| \qquad \forall z\in \gamma, \]
then $f$ and $f+g$ have the same number of zeros inside $\gamma$.
\end{theorem}
\begin{proof}
For $t\in [0, 1]$ we define
\[ f_t(z) = f(z) + tg(z). \]
This means $f_0 = f$ and $f_1 = f+g$. The condition $|f(z)| > |g(z)|$ ensures that $f_t$ has no zeros on $\gamma$. Also $f_t$ is holomorphic and thus has no poles. So the number of zeros $n_t$ of $f_t$ inside $\gamma$ is given by
\[ n_t = \frac{1}{2\pi i}\oint_\gamma \frac{f_t'(z)}{f_t(z)}\diff{z}. \]
It is then enough to observe that $n_t$ is continuous as a real function of $t$.
\end{proof}

\begin{theorem}[Open mapping theorem] \label{openMappingTheorem}
Let $f: U\subseteq \C\to \C$ a holomorphic function defined on a region. Then either $f$ is constant or an open map.
\end{theorem}
\begin{proof}
Suppose $f$ is not constant. We then need to prove that $f$ is open. Take some open subset $W\subseteq U$ and $w\in W$. We show that there exists an open neighbourhood of $f(w)$ that is a subset of $f^\imf(W)$.

Consider the function $f - \constant{f(w)}$, which is holomorphic.
By \ref{zerosHolomorphicFunctionIsolated}, we can find an open ball $\ball(w, \delta)$ such that $w$ is the only root of $f - \constant{f(w)}$ in $\cball(w, \delta)$. We can take $\delta$ small enough that $\cball(w, \delta) \subseteq W$.

Now $\sphere(w,\delta)$ is closed and bounded, and thus compact by Heine-Borel (TODO ref), so $|f - \constant{f(w)}|$ attains a minimum $\epsilon \geq 0$ on $\sphere(w,\delta)$ by the extreme value theorem \ref{extremeValueTheorem}. This minimum is not zero by construction, so $\epsilon > 0$.

Finally we claim $\ball\big(f(w), \epsilon\big) \subseteq f^\imf(W)$. Take $w'\in \ball\big(f(w), \epsilon\big)$ and consider the function
\[ f - \constant{w'} = \big(f - \constant{f(w)}\big) + \big(\constant{f(w)} - \constant{w'}\big) \]
defined on $\cball(w, \delta)$.

Since $\big|f(z) - w\big| \geq \epsilon > \big|f(w) - w'\big|$, we can apply Rouché's theorem \ref{RoucheTheorem}: $f - \constant{w'}$ has one zero inside $\cball(w,\delta)$ since $f - \constant{f(w)}$ does, and thus $w'\in f^\imf(W)$.
\end{proof}


\subsubsection{Maximum modulus results}
\begin{proposition}[Maximum modulus principle] \label{maximumModulusPrinciple}
Let $f: U\subseteq \C\to \C$ a holomorphic function defined on a region. If $|f|$ attains a maximum on $U$ \textup{if and only if} $f$ is constant.
\end{proposition}
\begin{proof}
If $f$ is a constant, then $|f|$ clearly attains a maximum.

Now suppose $|f|$ attains a maximum at $z_0$. By the open mapping theorem, \ref{openMappingTheorem}, either $f$ is constant or open. Assume, towards a contradiction, that $f$ is open. Then $\im(f)$ contains a ball around $f(z_0)$ and so $|f(z_0)|$ is not a maximum of $|f|$.
\end{proof}
\begin{corollary}
Let $U$ be a region with compact closure $\overline{U}$. If $f$ is holomorphic on $U$ and continuous on $\overline{U}$, then
\[ \sup_{z\in U}|f(z)| \leq \max_{z\in \overline{U}\setminus U}|f(z)|. \]
\end{corollary}
\begin{proof}
Since $\overline{U}$ is compact, $|f|$ attains a maximum on $\overline{U}$ by the extreme value theorem \ref{extremeValueTheorem}. This maximum cannot lie in $U$ by the maximum modulus principle, so it lies on $\overline{U}\setminus U$.
\end{proof}

\begin{proposition}[Phragmén-Lindelöf principle]
Let $S'_\epsilon\subseteq S\subseteq \C$ be regions for all $\epsilon > 0$ with each $S'_\epsilon$ bounded, $M\geq 0$ and $f$ a function that is holomorphic on $S$ and continuous on $\overline{S'}$. Let $\{h_\epsilon\}_{\epsilon > 0}$ be set of functions such that
\begin{itemize}
\item $h_\epsilon \to \constant{1}_{\overline{S}}$ pointwise as $\epsilon \to 0$;
\item $\sup_{z\in \partial S'_\epsilon}\big|fh_\epsilon(z)\big| \leq M$ for all $\epsilon > 0$;
\item $\sup_{z\in S\setminus \overline{S'_\epsilon}}\big|fh_\epsilon(z)\big| \leq M$  for all $\epsilon > 0$.
\end{itemize}
Then $\sup_{z\in S}\big|f(z)\big| \leq M$.
\end{proposition}
The last two points are implied by $\sup_{z\in \overline{S}\setminus S'_\epsilon}\big|fh_\epsilon(z)\big| \leq M$ for all $\epsilon > 0$.
\begin{proof}
Take $\epsilon >0$. Then $fh_\epsilon$ is bounded by $M$ on $\overline{S'_\epsilon}$ by the maximum modulus principle \ref{maximumModulusPrinciple}. It is bounded by $M$ on $S\setminus \overline{S'_\epsilon}$ by assumption.

Since, for all $z\in S$, $|h_\epsilon f(z)| \leq M$ and $|h_\epsilon f(z)| \to |f(z)|$ as $\epsilon \to 0$, we have $|f(z)|\leq M$.
\end{proof}
\begin{corollary}
Let $0< \alpha < \pi$. Let $f$ be a function that is holomorphic on the open sector $S = \setbuilder{z}{-\alpha < \arg(z) < \alpha}$ and continuous on its boundary. If
\begin{itemize}
\item $|f(z)|\leq 1$ for all $z\in \partial S$;
\item $|f(z)| \leq Ce^{c|z|^\rho}$ for some $c, C > 0$ and $\rho \in \interval[co]{0, \frac{\pi}{2\alpha}}$;
\end{itemize}
then $|f(z)| \leq 1$ for all $z\in S$.
\end{corollary}
\begin{proof}
Take $\rho'\in \interval[o]{\rho, \frac{\pi}{2\alpha}}$ and set $h_\epsilon(z) = e^{-\epsilon z^{\rho'}}$, where we take $\big(re^{i\theta}\big)^{\rho'} = r^{\rho'}e^{i\rho'\theta}$ for $-\pi < \theta <\pi$.

Now $|h_\epsilon(z)| = e^{-\epsilon r^{\rho'}\cos(\theta\rho')}$. If $z = re^{i\theta}\in S$, then $-\alpha < \theta < \alpha$ and $0 < \rho' < \frac{\pi}{2\alpha}$ imply 
\[ \frac{-\pi}{2} = -\alpha\frac{\pi}{2\alpha} < -\alpha\rho' < \theta\rho' < \alpha\rho' < \alpha\frac{\pi}{2\alpha} = \frac{\pi}{2}, \]
so $0< \cos(\alpha\rho') < \cos(\theta\rho')$. Then we calculate (using $\rho'-\rho > 0$)
\begin{align*}
\left(\frac{c+1}{\epsilon\cos(\rho'\alpha)}\right)^{\frac{1}{\rho'-\rho}} \leq r &\implies \frac{c+1}{\epsilon\cos(\rho'\alpha)} \leq r^{\rho' -\rho} \\
&\implies \frac{c+1}{\epsilon\cos(\rho'\theta)} \leq r^{\rho' -\rho} \\
&\implies c+1 \leq \epsilon\cos(\rho'\theta)r^{\rho' -\rho} \\
&\implies c - \epsilon\cos(\rho'\theta)r^{\rho' -\rho} \leq -1.
\end{align*}
Thus for all $r\geq \left(\frac{c+1}{\epsilon\cos(\rho'\alpha)}\right)^{\frac{1}{\rho'-\rho}}$ we have
\[ |fh_\epsilon(re^{i\theta})| \leq Ce^{cr^\rho} \cdot e^{-\epsilon r^{\rho'}\cos(\theta\rho')} = Ce^{cr^\rho - \epsilon \cos(\theta\rho') r^{\rho'}} = Ce^{\big(c - \epsilon\cos(\rho'\theta)r^{\rho' -\rho}\big)r^\rho} \leq Ce^{-r^\rho}. \]
So, if $r \geq \max\left\{\left(\frac{c+1}{\epsilon\cos(\rho'\alpha)}\right)^{\frac{1}{\rho'-\rho}}, \big(\ln C\big)^{1/\rho}\right\}$, then 
\[ |fh_\epsilon(re^{i\theta})| \leq Ce^{-r^\rho} \leq Ce^{-\big((\ln C)^{1/\rho}\big)^\rho} = Ce^{-\ln C} = CC^{-1} = 1. \]
Now set $S'_\epsilon \defeq \setbuilder{z\in \C}{-\alpha < \arg(z) < \alpha, |z| < \max\left\{\left(\frac{c+1}{\epsilon\cos(\rho'\alpha)}\right)^{\frac{1}{\rho'-\rho}}, \big(\ln C\big)^{1/\rho}\right\}}$. This satisfies the requirements for the Phragmèn-Lindelöf principle.
\end{proof}
\begin{corollary}[Hadamard three-lines lemma]
Let $f$ be a function that is holomorphic and bounded on the open strip $S = \setbuilder{z}{0 < \Im(z) < 1}$ and continuous on its boundary. Set $M_0 = \sup_{x\in \R}|f(x)|$ and $M_1 = \sup_{x\in \R}|f(x+i)|$. Then
\[ \sup_{x\in \R}|f(x+iy)| \leq M_0^{1-y}M_1^y \]
for all $0\leq y\leq 1$.
\end{corollary}
\begin{proof}
Let $M \defeq \sup_{z\in S}|f(z)|$. We apply the Phragmèn-Lindelöf principle to the function $F(z) \defeq M_0^{-1-iz}M_1^{iz}f(z)$, with $h_\epsilon(z) \defeq e^{-\epsilon z^2}$, to obtain that $\sup_{z\in S}|F(z)| \leq 1$.

For all $x\in \R$, we have
\[ |M_0^{-1-ix}M_1^{ix}f(x)h_\epsilon(x)| \leq |M_0^{-1-ix}M_1^{ix}|M_0 e^{-\epsilon x^2} \leq M_0M_0^{-1} = 1 \]
and
\[ |M_0^{-1-i(x+i)}M_1^{i(x+i)}f(x+i)h_\epsilon(x+i)| \leq |M_0^{-ix}M_1^{ix -1}|M_1 e^{-\epsilon |x+i|^2} \leq M_1^{-1}M_1 =  1. \]
Now set $S'_\epsilon \defeq \setbuilder{z\in \C}{0<\Im(z)< 1; -\sqrt{\frac{\ln M + \epsilon}{\epsilon}} < x < \sqrt{\frac{\ln M + \epsilon}{\epsilon}}}$. To see that this satisfies the Phragmèn-Lindelöf principle, take $z = x+iy \in S\setminus S_\epsilon'$.
Then we have the implications
\[ \frac{\ln M + \epsilon}{\epsilon} \leq x^2 \quad\implies\quad Me^{\epsilon} \leq e^{\epsilon x^2} \quad\implies\quad Me^{-\epsilon x^2 + \epsilon} \leq 1 \]
and so
\begin{align*}
\big|M_0^{-1-iz}M_1^{iz}f(z)h_\epsilon(z)\big| &= \big|M_0^{-1-ix +y}M_1^{ix - y}f(z)e^{-\epsilon (x^2 + 2ixy - y^2)}\big| \\
&= M_0^{-1 +y}M_1^{- y}\big|f(z)\big|e^{-\epsilon x^2 + \epsilon y^2} \\
&\leq M_0^{-1 +1}M_1^{0}Me^{-\epsilon x^2 + 1} \\
&= Me^{-\epsilon x^2 + 1} \leq 1.
\end{align*}
We conclude that $1 \geq \sup_{z\in S}|F(z)|$, so for all $y\in \interval{0,1}$, we have
\[ 1 \geq \sup_{x \in \R}|F(x+iy)| = \sup_{x \in \R}|M_0^{-1-ix+y}M_1^{ix-y}f(x+iy)| = M_0^{y-1}M_1^{-y}\sup_{x \in \R}|f(x+iy)| \]
and so $|f(x+iy)| \leq M_0^{1-y}M_1^{y}$.
\end{proof}


\subsection{The ring of polynomials over $\meromorphic_\Omega$}

\begin{definition}
Let $\Omega\subseteq \C$ be an open set. Let $\meromorphic_\Omega$ be the set of meromorphic functions in $(\Omega \to \C)$.
\end{definition}

\begin{proposition}
For any $\Omega \subseteq \C$, the set $\meromorphic_\Omega$ is a field. With
\begin{itemize}
\item as zero the constant function $\underline{0}$;
\item as identity the constant function $\underline{1}$.
\end{itemize}
\end{proposition}

\subsubsection{Poles and roots}
\begin{definition}
We say a polynomial in $\meromorphic_\Omega[X]$ has a pole at $z\in \Omega$, if one of its coefficients has a pole at $z$.
\end{definition}

\begin{lemma} \label{polesProductPolynomial}
Let $p,q\in \meromorphic_\Omega[X]$ be monic and $r = pq$. If either $p$ or $q$ has a pole at $z_0\in \Omega$, then $r$ has a pole at $z_0$ as well.
\end{lemma}
\begin{proof}
Let $h$ be the maximal order of $z_0$ as a pole among the coefficients of $p$ and $k$ the maximal order of $z_0$ as a pole among the coefficients of $q$. Then both $\lim_{z\to z_0}(z-z_0)^h p(z)$ and $\lim_{z\to z_0}(z-z_0)^k q(z)$ are non-zero polynomials (due to being monic) with constant coefficients and $h+k \geq 1$. 

Now assume $r(z)$ is holomorphic at $z_0$, then
\[ 0 = \lim_{z\to z_0}(z-z_0)^{h+k}r(z) = \Big(\lim_{z\to z_0}(z-z_0)^h p(z)\Big)\Big(\lim_{z\to z_0}(z-z_0)^k q(z)\Big) \neq 0, \]
which is a contradiction.
\end{proof}
\begin{corollary}
Let $p \in \meromorphic_\Omega[x]$ be monic and have a prime decomposition $p = \prod_{k=1}^nq_k^{m_k}$. Then $z_0\in \Omega$ is a pole of $p(z)$ \textup{if and only if} it is a pole of one of the $q_k(z)$.
\end{corollary}

\begin{proposition}
Let $p,q\in \meromorphic_\Omega[x]$ be monic and relatively prime, then $p(z)$ and $q(z)$ are relatively prime as polynomials in $\C[X]$ for all $z\in \Omega \setminus (S_p \cup S_q \cup H)$, where
\begin{itemize}
    \item $S_p$ is the set of poles of $p$;
    \item $S_q$ is the set of poles of $q$;
    \item $H$ is isolated and closed in $\Omega$.
\end{itemize}
\end{proposition}
\begin{proof}
By Bézout's identity, we can write $\alpha[x](z)p[x](z) + \beta[x](z)q[x](z) = f(z)$ for some monic $\alpha,\beta\in \meromorphic_\Omega[x]$. Let $H$ be the set of poles and zeros of $f(z)$ (which contains $S_p \cup S_q$ by \ref{polesProductPolynomial}). For all $z_0\notin H$ we have
\[ \left(\frac{\alpha[x](z_0)}{f(z_0)}\right)p[x](z_0) + \left(\frac{\beta[x](z_0)}{f(z_0)}\right)q[x](z_0) = 1, \]
meaning $p[x](z_0)$ and $q[x](z_0)$ are relatively prime (TODO ref).
\end{proof}
\begin{corollary}
Let $p\in \meromorphic_\Omega[x]$ be monic and irreducible. Then $p(z)$ has simple roots for all $z\in\Omega\setminus H$ for some isolated and closed set $H\subseteq \Omega$.
\end{corollary}
\begin{proof}
If the degree of $p$ is $1$, then the result is immediate. Assume the degree of $p$ is greater than $1$.

For any $z\in \Omega$, any non simple root of $p[x](z)$ must also be a root of $\pd{p[x](z)}{x}$. As $p[x]$ is irreducible, $p[x]$ and $\pd{p[x]}{x}$ are relatively prime. By the proposition $p[x](z_0)$ and $\pd{p[x](z_0)}{x}$ are relatively prime for all $z_0\in \Omega\setminus H$, which means they do not share roots.
\end{proof}

\subsubsection{Algebroid functions}

\subsubsection{Puiseux series}

\section{Conformal mappings}

\section{Banach $*$-algebras of complex-valued functions}
\begin{lemma}
Let $\sSet{X,\xi}$ be a convergence space. Then
\begin{enumerate}
\item $\sup_{x\in X}|f| <\infty$ for all $f\in \cont_0(X)$;
\item if $X$ is compact, then $\sup_{x\in X}|f| <\infty$ for all $f\in \cont(X)$.
\end{enumerate}
\end{lemma}
\begin{proof}
(1) Take arbitrary $f\in \cont_0(X)$. Take $\cball_\C(1,0)$, which is a neighbourhood of $0$. Then there exists a compact $K\subseteq X$ such that $f^{\imf}(K^c) \subseteq \cball_\C(0,1)$, so $\sup |f^{\imf}(K^c)| \leq 1$.

Now $\sup_{x\in X}|f| = \max\big\{\sup |f^{\imf}(K^c)|, \sup |f^{\imf}(K)|\big\}$ and $\sup |f^{\imf}(K)| = \max |f^\imf(K)| < \infty$ by the extreme value theorem \ref{extremeValueTheorem}.

(2) Follows immediately from the extreme value theorem \ref{extremeValueTheorem}.
\end{proof}

\begin{definition}
Let $\sSet{X,\xi}$ be a convergence space. Then
\begin{itemize}
\item we equip $\cont_0(X)$ with the supremum norm, which makes it a Banach-$*$-algebra;
\item if $X$ is compact, then we equip $\cont(X)$ with the supremum norm, which makes it a Banach-$*$-algebra.
\end{itemize} 
\end{definition}

\begin{proposition} \label{unitisationOnePointCompactificationIsomorphism}
Let $\sSet{X,\xi}$ be a convergence space. Then the function
\[ \Phi: \proj_1(-)^\dagger + \constant{\proj_2(-)}_{X^\dagger}: \cont_0(X)^\dagger \to \cont(X^\dagger): (f, \lambda) \mapsto f^\dagger + \constant{\lambda}_{X^\dagger}. \]
is a unital $*$-algebra isomorphism.
\end{proposition}
\begin{proof}
The function is well-defined, since $f^\dagger$ is continuous by \ref{vanishesAtInfinityIffBasepointExtensionContinuous}. It is a bijection, since its inverse is given by
\[ \cont(X^\dagger) \to \cont_0(X)^\dagger: f\mapsto \big(f|_X - \constant{f(\infty)}_X, f(\infty)\big). \]
The fact that it is a $*$-algebra homomorphism follows from \ref{basepointExtensionToRingLemma}.

Unitality follows from $\big(\constant{0}|_X\big)^\dagger = \constant{0}_{X^\dagger}$ (\ref{basepointExtensionToRingLemma}).
\end{proof}

\begin{lemma} \label{compactCharactersLemma}
Let $X$ be a compact topological space. Set
\begin{itemize}
\item $I_x \defeq \setbuilder{f\in \cont(X)}{f(x) = 0}$ for all $x\in X$;
\item $N(I)\defeq \setbuilder{x\in X}{\forall f\in I: \; f(x) = 0}$ for all ideals $I\subseteq \cont(X)$.
\end{itemize}
Then we have
\begin{enumerate}
\item $N(I) = \emptyset$ \textup{if and only if} $I = \cont(X)$;
\item if $I\subseteq \cont(X)$ is a maximal ideal, then $I = I_x$ for some $x\in X$.
\end{enumerate}
\end{lemma}
\begin{proof}
(1) First assume $N(I) = \emptyset$. Then $\big\{f^{\preimf}(\C\setminus\{0\})\big\}_{f\in I}$ is an open cover of $X$. We can take a finite subcover $C$ by \ref{topologyCompactnessOpenCover}. Now set
\[ g \defeq \sum_{f^{\preimf}(\C\setminus\{0\})\in C} f^2. \]
By construction, $g\in I$ (since $I$ is an ideal) and $g$ is invertible (since it is nowhere $0$), thus $I = \cont(X)$ by \ref{properIdealNoUnit}.

Now suppose $I = \cont(X)$. Then $\constant{1}\in I$ and $\constant{1}(x) \neq 0$ for all $x\in X$, so $N(I)=\emptyset$.

(2) Let $I\subseteq \cont(X)$ be a maximal (proper) ideal. Then $N(I) \neq \emptyset$, so there exists $x\in N(I)$. It is clear that $I\subseteq I_x$. By maximality, we have $I = I_x$.
\end{proof}

See also \url{https://math.stackexchange.com/questions/87277/proof-that-ideals-in-c0-1-are-of-the-form-m-c-that-should-not-involve-zorn}.


\begin{proposition}  \label{charactersFunctionAlgebraCompactSpace}
Let $X$ be a Hausdorff topological space. Then
\begin{enumerate}
\item if $X$ is compact, then $\evalMap_{-}|_{\cont(X)}: X \to \widehat{\cont(X)}$ is a homeomorphism;
\item if $X$ is locally compact, then $\evalMap_{-}|_{\cont_0(X)}: X \to \widehat{\cont_0(X)}$ is a homeomorphism.
\end{enumerate}
\end{proposition}
\begin{proof}
(1) For all $x\in X$, we clearly have $\evalMap_x\in \widehat{\cont(X)}$. 

We first show that $\evalMap_{-}|_{\cont(X)}$ is injective. Suppose $x\neq y \in X$. Since $X$ is Hausdorff, $\{x\}$ and $\{y\}$ are disjoint closed sets by \ref{FrechetCharacterisation}. Since $X$ is compact Hausdorff, it is normal by \ref{compactHausdorffSpacesNormal}, so we can find $f\in \cont(X, \R)$ such that $f(x) = 0$ and $f(y) = 1$ by Urysohn's lemma \ref{UrysohnLemma}. This function is still continuous if we extend the codomain to $\C$ by \ref{continuityRestrictionExpansion}.  Then
\[ \evalMap_x(f) = f(x) = 0 \neq 1 = f(y) = \evalMap_y(f), \]
so $\evalMap_x|_{\cont(X)} \neq \evalMap_y|_{\cont(X)}$.

To show surjectivity, take $\varphi\in\widehat{\cont(X)}$. Then $\ker(\varphi)$ is a maximal ideal, see \ref{characterMaximalIdealsComplex}. Thus it is equal to $I_x = \ker\big(\evalMap_x|_{\cont(X)}\big)$ for some $x\in X$ by \ref{compactCharactersLemma}. This means that $\varphi = \lambda \evalMap_x$ for some $\lambda\in C$. Since, by \ref{charactersUnital}, $1 = \varphi(\constant{1}) = \lambda\evalMap_x(\constant{1}) = \lambda$, we have $\lambda = 1$ and $\varphi = \evalMap_x|_{\cont(X)}$.

Now $\evalMap_{-}|_{\cont(X)}$ is continuous by \ref{characteristicPropertyInitialFinalConvergence}, since, for all $f\in \cont(X)$,
\[ \begin{tikzcd}
\widehat{\cont(X)} \ar[r, "\evalMap_f"] & \C \\ X \ar[u, "{\evalMap_{-}|_{\cont(X)}}"] \ar[ur, swap, "f"]
\end{tikzcd} \qquad\text{commutes,} \]
which is true because, for all $x\in X$,
\[ (\evalMap_f\circ \evalMap_{-}|_{\cont(X)})(x) = \evalMap_f\big(\evalMap_x\big) = \evalMap_x(f) = f(x). \]
Since $\widehat{\cont(X)}$ is Hausdorff, by \ref{weakTopologyLCTVS}, we have that $\evalMap_{-}|_{\cont(X)}$ is a homeomorphism by \ref{compactToHausdorffHomeomorphism}.

(2) We first show that $\evalMap_{-}|_{\cont_0(X)}$ is injective. Suppose $x\neq y \in X$. Now $\topMod(X^\dagger)$ is compact and Hausdorff by \ref{onePointCompactificationHausdorff}, and thus normal by \ref{compactHausdorffSpacesNormal}. The sets $\{x\}$, $\{y\}$ and $\{\infty\}$ are disjoint closed sets by \ref{FrechetCharacterisation}.

By the Tietze extension theorem \ref{TietzeExtension}, we can extend the function
\[ \{x, y, \infty\}\to \R: z\mapsto \begin{cases}
1 & (z = x) \\ 2 & (z = y) \\ 0 & (z = \infty)
\end{cases} \]
to a continuous function. This extension is still continuous if we extend the codomain to $\C$ by \ref{continuityRestrictionExpansion}. Call this function $f$. It is also continuous as a function $f: X^\dagger \to \C$, since $X^\dagger$ is stronger than $\topMod(X^\dagger)$. Now $f|_X$ is continuous by \ref{functionVanishingAtInftyIffRestrictionOfContinuousBasepointPreservingFunction}, so 
\[ \evalMap_x|_{\cont_0(X)}\big(f|_X\big) = f(x) = 1 \neq 2 = f(y) = \evalMap_y|_{\cont_0(X)}\big(f|_X\big), \]
so $\evalMap_x|_{\cont_0(X)} \neq \evalMap_y|_{\cont_0(X)}$.

To show that $\evalMap_{-}|_{\cont_0(X)}$ is surjective, take arbitrary $\varphi\in\widehat{\cont_0(X)}$. Then we have a unital extension $\varphi^\dagger\in \widehat{\cont_0(X)^\dagger}$. Take the isomorphism $\Phi: \cont_0(X)^\dagger \to \cont(X^\dagger)$ of \ref{unitisationOnePointCompactificationIsomorphism} and consider $\varphi^\dagger\circ \Phi^{-1}: \cont(X^\dagger)\to \C$, which is an algebra homomorphism by construction and nonzero because it is unital. So $\varphi^\dagger\circ \Phi^{-1}\in \widehat{\cont(X^\dagger)}$ and $\varphi^\dagger\circ \Phi^{-1} = \evalMap_a|_{\cont_0(X)}$ for some $a\in X^\dagger$ by (1).

Take arbitrary $f\in \cont_0(X)$. Then $f^\dagger \in \cont(X^\dagger)$ by \ref{vanishesAtInfinityIffBasepointExtensionContinuous} and we calculate
\begin{align*}
(\varphi^\dagger\circ \Phi^{-1})(f^\dagger) &= \varphi^\dagger\big(f^\dagger|_X - \constant{f^\dagger(\infty)}, f^\dagger(\infty)\big) \\
&= \varphi^\dagger\big(f^\dagger|_X, 0\big) \\
&= \varphi^\dagger(f, 0) = \varphi(f) + 0 = \varphi(f).
\end{align*}

Now suppose, towards a contradiction, that $a = \infty$. Then, for all $f\in \cont_0(X)$ we have
\[ \varphi(f) = (\varphi^\dagger\circ \Phi^{-1})(f^\dagger) = \evalMap_\infty(f^\dagger) = 0, \]
which is a contradiction since $\varphi \neq \constant{0}$. So $a\in X$ and we have
\[ \varphi(f) = (\varphi^\dagger\circ \Phi^{-1})(f^\dagger) = \evalMap_a(f^\dagger) = \evalMap_a(f) \]
for all $f\in \cont_0(X)$, which shows that $\evalMap_{-}|_{\cont_0(X)}$ is surjective.

Finally we need to show that $\evalMap_{-}|_{\cont_0(X)}$ and its inverse is continuous. By \ref{LCTopologicalHausdorffInitialConvergence}, $X$ has the initial convergence w.r.t. $\cont_0(X)$, so, for all $x\in X$ and $F\in \powerfilters(X)$, we have
\begin{align*}
F \overset{X}{\longrightarrow} x &\iff \forall f\in \cont_0(X): \; \upset f^{\imf\imf}(F)\to f(x) \\
&\iff \forall f\in \cont_0(X): \; \upset \big(\evalMap_{-}|_{\cont_0(X)}\big)^{\imf\imf}(F)(f) \to \big(\evalMap_{-}|_{\cont_0(X)}\big)(f)(x) \\
&\iff \upset \big(\evalMap_{-}|_{\cont_0(X)}\big)^{\imf\imf}(F) \overset{\widehat{\cont_0(X)}}{\longrightarrow} \big(\evalMap_{-}|_{\cont_0(X)}\big)(f),
\end{align*}
which concludes the proof that $\evalMap_{-}|_{\cont_0(X)}$ is a homeomorphism.
\end{proof}


\begin{proposition}[Gelfand-Kolmogorov]
Let $X,Y$ be compact topological spaces and $h: \cont(X)\to \cont(Y)$ a unital algebra homomorphism. Then
\begin{enumerate}
\item there exists a function $g: Y\to X$ such that $h$ equals the pre-composition $g^\star$;
\begin{enumerate}
\item this $g$ is continuous;
\item this $g$ is unique if $X$ is Hausdorff;
\end{enumerate}
\item if $h$ is an isomorphism, then $g$ is a homeomorphism.
\end{enumerate}
\end{proposition}
Note that we do not assume $h$ to be continuous w.r.t. any convergence.

TODO: clean up calculations.
\begin{proof}
(1) Let both $\cont(X)$ and $\cont(Y)$ have the algebraic convergence, then $h: \cont(X)\to \cont(Y)$ is continuous. By  \ref{adjointContinuousFunction}, $h$ has a unique adjoint $h^\star: \dual{\cont(Y)} \to \dual{\cont(X)}$. We claim that $\im\big(h^\star|_{\widehat{\cont(Y)}}\big) \subseteq \widehat{\cont(X)}$, so we can consider the function $h^\star: \widehat{\cont(Y)}\to \widehat{\cont(X)}$. Indeed, for all $\varphi\in \widehat{\cont(Y)}$, we have that $h^\star(\varphi) = \varphi \circ h$ is an algebra homomorphism. It is also non-zero because $h^\star(\varphi)(\constant{1}) = \varphi\big(h(\constant{1})\big) = \varphi(\constant{1}) = 1$, by \ref{charactersUnital}.

For all $f\in \cont(X)$, consider the diagram
\[ \begin{tikzcd}
\widehat{\cont(X)} \ar[r, "\evalMap_f"] & \C \\
\widehat{\cont(Y)} \ar[u, "h^\star"] & {} \\
Y \ar[u, "{\evalMap_{-}|_{\cont(Y)}}"] \ar[uur, swap, "h(f)"]
\end{tikzcd}, \]
which commutes, because, for all $y\in Y$, we have
\[ \big(\evalMap_f \circ h^\star \circ \evalMap_{-}|_{\cont(Y)}\big)(y) = \evalMap_f\big(h^\star (\evalMap_y)\big) = \evalMap_f(\evalMap_y \circ h) = (\evalMap_y\circ h)(f) = h(f)(y). \]
Since $h(f)$ is a character and thus continuous by \ref{charactersUnital}, we have that $h^\star \circ \evalMap_{-}|_{\cont(Y)}$ is continuous by \ref{characteristicPropertyInitialFinalConvergence}.

Then, since $\evalMap_{-}|_{\cont(X)}$ is invertible by \ref{charactersFunctionAlgebraCompactSpace}, we can set $g \defeq \big(\evalMap_{-}|_{\cont(X)}\big)^{-1} \circ h^\star \circ \evalMap_{-}|_{\cont(Y)}: Y\to X$. This is also continuous by \ref{charactersFunctionAlgebraCompactSpace}.

We now show that $h = g^\star$. First observe that for arbitrary $f\in \cont(X)$, we have $f\circ \big(\evalMap_{-}|_{\cont(X)}\big)^{-1} = \evalMap_f$. Indeed, each character in $\widehat{\cont(X)}$ is of the form $\evalMap_x$ for some $x\in X$ by \ref{charactersFunctionAlgebraCompactSpace}, so
\[ \Big(f\circ \big(\evalMap_{-}|_{\cont(X)}\big)^{-1}\Big)(\evalMap_x) = \Big(f\circ \big(\evalMap_{-}|_{\cont(X)}\big)^{-1} \circ \evalMap_{-}|_{\cont(X)}\Big)(x) = f(x) = \evalMap_x(f) = \evalMap_f\big(\evalMap_x\big). \]

By a previous calculation, we have
\[ g^\star(f) = f\circ g = f\circ (\evalMap_{-}|_{\cont(X)})^{-1} \circ h^\star \circ \evalMap_{-}|_{\cont(Y)} = \evalMap_f \circ h^\star \circ \evalMap_{-}|_{\cont(Y)} = h(f). \]

Finally, suppose, towards a contradiction, that $g_1^\star = h = g_2^\star$ but $g_1 \neq g_2$. Then there exists $y\in Y$ such that $g_1(y)\neq g_2(y)$. Since $X$ is Hausdorff, $\{g_1(y)\}$ and $\{g_2(y)\}$ are disjoint closed sets by \ref{FrechetCharacterisation}. Since $X$ is compact Hausdorff, it is normal by \ref{compactHausdorffSpacesNormal}, so we can find $f\in \cont(X, \R)$ such that $f\big(g_1(y)\big) = 0$ and $f\big(g_2(y)\big) = 1$ by Urysohn's lemma \ref{UrysohnLemma}. This function is still continuous if we extend the codomain to $\C$ by \ref{continuityRestrictionExpansion}.
Now
\[ g_1^\star(f)(y) = (f\circ g_1)(y) = f\big(g_1(y)\big) = 0 \neq 1 = f\big(g_2(y)\big) = (f\circ g_2)(y) = g_2^\star(f)(y), \]
so $g_1^\star(f) \neq g_2^\star(f)$ and thus $g_1^\star \neq g_2^\star$, which is a contradiction.

(2) If $h$ is an isomorphism, then $g^{-1} = \big(\evalMap_{-}|_{\cont(Y)}\big)^{-1} \circ (h^{-1})^\star \circ \evalMap_{-}|_{\cont(X)}$, which is continuous.
\end{proof}


\chapter{Function spaces}
\section{The $\mathcal{L}^p(X,\diff{\mu})$ spaces}
\begin{definition}
Let $\sSet{X,\mathcal{A},\mu}$ be a measure space and $0<p<\infty \in \R$. The we define
\begin{itemize}
\item $\mathcal{L}^p(X,\mu)$ to be the set of measureable functions $f$ in $(X\to \C)$ such that $|f|^p$ is integrable w.r.t.\ $\mu$.
\item $\mathcal{L}^\infty(X,\mu)$ to be the set of measureable functions $f$ in $(X\to \C)$ such that $\sup_x|f(x)|<\infty$.
\end{itemize}
\end{definition}

\begin{proposition}
Let $0< p<q \leq \infty$ and $\sSet{X,\mathcal{A},\mu}$ be a measure space. Then
\begin{enumerate}
\item $\mathcal{L}^p(X,\mu) \supseteq \mathcal{L}^q(X,\mu)$ \textup{if and only if} $X$ does not contain sets of finite but arbitrary large measure;
\item $\mathcal{L}^p(X,\mu) \subseteq \mathcal{L}^q(X,\mu)$ \textup{if and only if} $X$ does not contain sets of non-zero but arbitrary small measure.
\end{enumerate}
\end{proposition}
\begin{proof}
TODO \url{https://en.wikipedia.org/wiki/Lp_space#Embeddings}
\end{proof}
\begin{corollary}
For all $0< p<q \leq \infty$ and $a,b\in\R$, we have $\mathcal{L}^p([a,b],\lambda) \supseteq \mathcal{L}^q([a,b],\lambda)$.
\end{corollary}

\begin{definition}
Let $0< p<\infty$ and $\sSet{X,\mathcal{A},\mu}$ be a measure space. Then we define the real functional
\[ \norm{\cdot}_{p}: \mathcal{L}^p(X,\mu)\to \R: f\mapsto \left(\int_X |f|^p\diff{\mu}\right)^{1/p}. \]
\end{definition}
We will prove that this functional is a seminorm.

\begin{lemma} \label{pNormAbsolutelyHomogeneous}
Let $0< p<\infty$ and $\sSet{X,\mathcal{A},\mu}$ be a measure space. Then $\norm{\cdot}_p$ is absolutely homogeneous.
\end{lemma}
\begin{proof}
For all $\lambda\in \R$, we have
\[ \norm{\lambda f}_{p} = \left(\int_X |\lambda f|^p\diff{\mu}\right)^{1/p} = \left(|\lambda|^p\int_X |f|^p\diff{\mu}\right)^{1/p} = |\lambda|\left(\int_X |f|^p\diff{\mu}\right)^{1/p}. \]
\end{proof}

\begin{lemma} \label{pnormLemma}
Let $\sSet{X,\mathcal{A},\mu}$ be a measure space, $k,l\in\interval[o]{1,+\infty}$ and $f\in \mathcal{L}^{kl}(X,\mu)$. Then
\[ \norm{|f|^k}_l = \left(\norm{f}_{kl}\right)^k. \]
\end{lemma}
\begin{proof}
We calculate
\begin{align*}
\left(\norm{f}_{kl}\right)^k &= \left(\left(\int_X|f|^{kl}\diff{\mu}\right)^{1/(kl)}\right)^{k} \\
&= \left(\int_X\big||f|^{k}\big|^l\diff{\mu}\right)^{1/l} \\
&= \norm{|f|^k}_l.
\end{align*}
\end{proof}

\begin{theorem}[Hölder's inequality] \label{HoeldersInequality}
Let $p,q\in \R^+$ be Hölder conjugate and $\sSet{X,\mathcal{A},\mu}$ a measure space. Then for all $f\in \mathcal{L}^p(X,\mu), g\in \mathcal{L}^q(X,\mu)$:
\begin{enumerate}
\item $fg\in \mathcal{L}^1(X,\mu)$;
\item $\norm{fg}_1 \leq \norm{f}_p\norm{g}_q$.
\end{enumerate}
We have equality \textup{if and only if} $\exists K\in \R$ such that $|f|^p = K|g|^q$ a.e.
\end{theorem}
\begin{proof}
If either $\norm{f}_p = 0$ or $\norm{g}_q = 0$, then either $|f|^p$ or $|g|^q$ is $0$ a.e. by \ref{functionPropertiesFromIntegral}. Then $fg = 0$ a.e., so $\norm{fg}_1 = 0$.

Now, assuming $\norm{f}_p \neq 0$ and $\norm{g}_q \neq 0$, we can write $f' = \norm{f}_p^{-1}f$ and $g' = \norm{g}_q^{-1}g$.
These functions satisfy $\norm{f'}_p = 1$ and $\norm{g'}_q = 1$, because of \ref{pNormAbsolutelyHomogeneous}.

Now we can apply Young's inequality \ref{YoungsInequality} pointwise to obtain
\[ |f'(x)g'(x)| \leq \frac{|f'|^p}{p} + \frac{|g'|^q}{q} \qquad \forall x\in X. \]
Integrating both sides (and using \ref{pNormAbsolutelyHomogeneous}), gives
\begin{align*}
\frac{\norm{fg}_1}{\norm{f}_p\norm{g}_q} = \norm{f'g'}_1 = \int_X|f'g'|\diff{\mu} &\leq \int_X\frac{|f'|^p}{p}\diff{\mu} + \int_X\frac{|g'|^q}{q}\diff{\mu} \\
&= \frac{\norm{f'}_p^p}{p} + \frac{\norm{g'}_q^q}{q} = \frac{1}{p} + \frac{1}{q} = 1.
\end{align*}
The Hölder inequality follows.

Now assume $\norm{fg}_1 = \norm{f}_p\norm{g}_q$. Then we must have the inequality is in fact an equality, i.e.\ 
\[ \int_X|f'g'|\diff{\mu} = \int_X\left(\frac{|f'|^p}{p} + \frac{|g'|^q}{q}\right)\diff{\mu}. \]
Then $|f'g'| = \frac{|f'|^p}{p} + \frac{|g'|^q}{q}$ a.e.\ by \ref{functionPropertiesFromIntegral}. By \ref{YoungsInequality} the equality $|f'(x)g'(x)| = \frac{|f'(x)|^p}{p} + \frac{|g'(x)|^q}{q}$ only holds if $|f'(x)|^p = |g'(x)|^q$. We conclude that
\[ 1 = \frac{|g'|^q}{|f'|^p} = \frac{\norm{f}_p^p |g|^q}{\norm{g}_q^q |f|^p} = K \frac{|g|^q}{|f|^p} \]
almost everywhere, for some $K\in \R^+$.
\end{proof}
\begin{proof}[Alternative proof]
Using Jensen's inequality \url{https://en.wikipedia.org/wiki/H%C3%B6lder%27s_inequality}
\end{proof}


\begin{theorem}[Minkowski's inequality] \label{MinkowskisInequality}
Let $1< p<\infty$ and $\sSet{X,\mathcal{A},\mu}$ be a measure space. Let $f,g\in\mathcal{L}^p(X,\mu)$ be functions. Then
\[ \norm{f+g}_p \leq \norm{f}_p + \norm{g}_p. \]
\end{theorem}
TODO invert inequality if $p<1$! TODO also inverse Hölder inequality.
\begin{proof}
First we show that $\norm{f+g}_p$ is finite. We show this using the convexity of $\R^+\to \R^+: x\mapsto |x|^p$:
\begin{align*}
|f+g|^p &= 2^p\left|\frac{1}{2}f+ \frac{1}{2}g\right|^p \\
&\leq 2^p\left|\frac{1}{2}|f|+ \frac{1}{2}|g|\right|^p \\
&\leq 2^p\left(\frac{1}{2}|f|^p + \frac{1}{2}|g|^p\right) = 2^{p-1}\left(|f|^p + |g|^p\right) < \infty.
\end{align*}
Then $\norm{f+g}_p^p < \infty$ by \ref{propertiesIntegralPositiveFunctions}, so $\norm{f+g}_p < \infty$.

Now we can calculate (setting $q = \frac{p}{p-1}$, i.e.\ the Hölder conjugate of $p$)
\begin{align*}
\norm{f+g}_p^p &= \int_X|f+g|^p\diff{\mu} \\
&= \int_X|f+g|\cdot |f+g|^{p-1}\diff{\mu} \\
&\leq \int_X\big(|f|+|g|\big)\cdot |f+g|^{p-1}\diff{\mu} \\
&= \norm{|f|\cdot |f+g|^{p-1}}_1 + \norm{|g|\cdot |f+g|^{p-1}}_1 \\
&\leq \norm{f}_p\cdot \norm{|f+g|^{p-1}}_q + \norm{g}_p\cdot \norm{|f+g|^{p-1}}_q \\
&= \big(\norm{f}_p + \norm{g}_p\big)\cdot \norm{|f+g|^{p-1}}_q
\end{align*}
using first the triangle inequality and then Hölder's inequality, \ref{HoeldersInequality}.

Next we calculate
\begin{align*}
\norm{|f+g|^{p-1}}_q &= \left(\norm{f+g}_{(p-1)q}\right)^{p-1} = \left(\norm{f+g}_{(p-1)\frac{p}{p-1}}\right)^{p-1} = \norm{f+g}_{p}^{p-1},
\end{align*}
using \ref{pnormLemma} and \ref{HoelderConjugateEquivalents}.

Putting these two calculations together gives
\[ \norm{f+g}_p^p \leq \big(\norm{f}_p + \norm{g}_p\big)\cdot \norm{f+g}_{p}^{p-1}, \]
or $\norm{f+g}_p \leq \norm{f}_p + \norm{g}_p$.
\end{proof}
\begin{corollary}
Let $0< p<\infty$ and $\sSet{X,\mathcal{A},\mu}$ be a measure space. Then $\norm{\cdot}_{p}$ is a seminorm.
\end{corollary}
\begin{proof}
We need to prove that $\norm{f}_{p}$ is subadditive and absolutely homogeneous. Subadditivity is given by the Minkowski inequality and absolute homogeneity by \ref{pNormAbsolutelyHomogeneous}.
\end{proof}


\section{Classification by support}
\begin{definition}
Let $\sSet{\Omega, \xi}$ be a convergence space, $\sSet{V,\zeta}$ a convergence vector space and $f\in (\Omega\to V)$.
\begin{itemize}
\item The \udef{set-theoretic support} of $f$ is
\[ \supp_\text{st}(f) \defeq \setbuilder{x\in \Omega}{f(x) \neq 0}. \]
\item The \udef{(closed) support} of $f$ is $\supp(f) \defeq \closure_\xi\big(\supp_\text{st}(f)\big)$.
\end{itemize}
\end{definition}

\begin{lemma}
Let $\sSet{\Omega, \xi}$ be a convergence space, $\sSet{V,\zeta}$ a Hausdorff convergence vector space and $A\subseteq \Omega$ a subset. Then
\begin{enumerate}
\item $\setbuilder{f\in (\Omega\to V)_p}{\supp_\text{st}(f)\subseteq A}$ is a closed subset of $(\Omega\to V)_p$;
\item if $A$ is closed, then $\setbuilder{f\in (\Omega\to V)_p}{\supp(f)\subseteq A}$ is a closed subset of $(\Omega\to V)_p$.
\end{enumerate}
\end{lemma}
\begin{proof}
(1) Set $X \defeq \setbuilder{f\in (\Omega\to V)_p}{\supp_\text{st}(f)\subseteq A}$.

We need to show that $\adh_p(X) \subseteq X$. To that end, take $f\in \adh_p(X)$, then, by \ref{principalAdherenceInherence}, there exists $H\in \powerfilters(\Omega\to V)$ such that $X\in H$. For all $b\notin A$, we have $\evalMap_b^{\imf\imf}(H) = \pfilter{0}$. As $\evalMap_b^{\imf\imf}(H) \to f(b)$, we have that $f(b) = 0$ by Hausdorffness, which means that $f\in X$.

(2) If $A$ is closed, then $\overline{\supp_\text{st}(f)}\subseteq A \iff \supp_\text{st}(f) \subseteq A$.
\end{proof}

\subsection{Functions of compact support}
\begin{definition}
Let $\sSet{\Omega, \xi}$ be a convergence space and $\sSet{V,\zeta}$ a convergence vector space. We define
\begin{itemize}
\item $(\Omega\to V)_C \defeq \setbuilder{f\in (\Omega\to V)}{\text{$\supp(f)$ is compact}}$;
\item $\cont_C(\Omega,V) \defeq (\Omega\to V)_C \cap \cont(\Omega, V)$;
\item $W_C \defeq (\Omega\to V)_C \cap W$ for all subspaces $W\subseteq (\Omega\to V)$.
\end{itemize}
\end{definition}
Note that $(\Omega\to V)_C \neq (\Omega\to V)_c$ and $\cont_C(\Omega) \neq \cont_c(\Omega)$!

Observation: if $\Omega$ is Hausdorff, then the set of compact subsets of $\Omega$ forms a $\pi$-system of closed subsets of $\Omega$. This means that

\section{Distributions}
\begin{definition}
Let $\sSet{\Omega, \mathcal{A}, \mu}$ be a measure space and $f: \Omega\to \overline{\R^+}$ a measurable function. The \udef{distribution function} of $f$ is the function
\[ D_f: \overline{\R^+}\to \overline{\R^+}: \alpha \mapsto \mu\setbuilder[\big]{\omega}{f(\omega)>\alpha}.  \]
\end{definition}

\begin{lemma} \label{distributionContinuity}
Let $\sSet{\Omega, \mathcal{A}, \mu}$ be a measure space and $f: \Omega\to \overline{\R^+}$ a measurable function. Then
\begin{enumerate}
\item $D_f$ is monotonically decreasing;
\item $D_f$ is right-continuous.
\end{enumerate}
\end{lemma}
\begin{proof}
(1) Let $\alpha \leq \beta$, then $\setbuilder{\omega}{f(\omega) > \alpha} \supseteq \setbuilder{\omega}{f(\omega) > \beta}$, so
\[ D_f(\alpha) = \mu\setbuilder{\omega}{f(\omega) > \alpha} \geq \mu\setbuilder{\omega}{f(\omega) > \beta} = D_f(\beta). \]


(2) Since the lower limit topology is sequential, by \ref{SorgenfreyLineC1}. It is enough to check that $D_f$ is sequentially continuous.


Take an arbitrary sequence that converges to an arbitrary $x\in\overline{\R^+}_{ll}$. By \ref{lowerLimitConvergenceDecreasingSubsequence}, we can find a decreasing subsequence $\seq{x_n}$. Then $\seq{D_f(x_n)}$ is monotonically increasing.

Now we can calculate
\begin{align*}
\lim_{n\to\infty} D_f(x_n) &= \sup_{n\in\N}D_f(x_n) \\
&= \sup_{n\in\N} \mu\setbuilder{\omega}{f(\omega) > x_n} \\
&= \mu\Big(\bigcup_{n\in\N}\setbuilder{\omega}{f(\omega) > x_n}\Big) \\
&= \mu\setbuilder{\omega}{f(\omega) > x} \\
&= D_f(x),
\end{align*}
where we have used \ref{premeasureChainContinuous}. We have also used the fact that $\bigcup_{n\in\N}\setbuilder{\omega}{f(\omega) > x_n} = \setbuilder{\omega}{f(\omega) > x}$, which we now prove:
The inclusion $\bigcup_{n\in\N}\setbuilder{\omega}{f(\omega) > x_n} \subseteq \setbuilder{\omega}{f(\omega) > x}$ follows from the fact that $\setbuilder{\omega}{f(\omega) > x_n} \subseteq \setbuilder{\omega}{f(\omega) > x}$ for all $n\in \N$. For the other inequality, take $\omega \in \setbuilder{\omega}{f(\omega) > x}$. Then $\omega \in \setbuilder{\omega}{f(\omega) > x+(f(\omega)-x)/2}$ and there exists an $x_m$ such that
\[ \setbuilder{\omega}{f(\omega) > x+(f(\omega)-x)/2} \subseteq \setbuilder{\omega}{f(\omega) > x_m}. \] 
\end{proof}

\begin{lemma} \label{distributionOfFunctionLemma}
Let $\sSet{\Omega, \mathcal{A}, \mu}$ be a measure space and $f,g: \Omega\to \overline{\R^+}$ measurable functions and $c\in \interval[o]{0,\infty}$. Then
\begin{enumerate}
\item if $f \leq g$ a.e., then $D_f \leq D_g$;
\item $D_{cf}(\alpha) = D_f\Big(\frac{\alpha}{c}\Big)$;
\item $D_{f+g}(\alpha+\beta) \leq D_f(\alpha) + D_g(\beta)$;
\item $D_{fg}(\alpha\beta) \leq D_f(\alpha) + D_g(\beta)$.
\end{enumerate}
\end{lemma}
In particular if $f = g$ a.e., then $D_f = D_g$. If $h,h': \Omega\to \C$ are measurable functions, then $D_{|h+h'|}\leq D_{|h|+|h'|} \leq D_{|h|} \leq D_{|h'|}$.
\begin{proof}
(1) Fix $\alpha\in \overline{\R^+}$ and set
\[ N \defeq \{f > g\}, \quad A \defeq \{f \geq \alpha\}, \quad\text{and}\quad B \defeq \{g \geq \alpha\}. \]
For all $\omega\in\Omega$, we have that $f(\omega)\leq g(\omega)$ and $\alpha < f(\omega)$ imply $\alpha < g(\omega)$, so
\[ A\setminus (N\cap A) = A\setminus N = A\cap N^c \subseteq B. \]
Also $N$ is measurable (by \ref{measurablesSetsRealMeasurableFunction}) and a null set (by assumption). 
As $(N\cap A) \subseteq A$ and $N\cap A$ is measurable, we have \[ \mu\big(A\setminus (N\cap A)\big) = \mu(A) - \mu(N\cap A) = \mu(A) - 0 = \mu(A), \]
using \ref{ringPositiveContent} and \ref{measureNullSet}. Thus we have
\[ D_f(\alpha) = \mu(A) = \mu\big(A\setminus (N\cap A)\big) = \mu(A\cap N^c) \leq \mu(B) = D_g(\alpha), \]
where the inequality is given by \ref{semiringPositiveContent}.

(2) We have $\{cf>\alpha\} = \{f>\frac{\alpha}{c}\}$.

(3) Take $\alpha,\beta\in \R^+$ and $\omega\in \Omega$. 
Because the order on $\overline{\R^+}$ is linear, \ref{additionVectorInequalities} gives
\[ \alpha + \beta < f(\omega) + g(\omega) \;\implies\; \big(\alpha < f(\omega)\big) \lor \big(\beta < g(\omega)\big). \]
This implications shows
\[ \{\alpha + \beta < f(\omega)+g(\omega)\} \;\subseteq\; \{\alpha < f(\omega)\} \cup \{\beta < g(\omega)\}. \]
Thus we calculate
\begin{align*}
D_{f+g}(\alpha+\beta) &= \mu\{\alpha + \beta < f(\omega)+g(\omega)\} \\
&\leq \mu\Big(\{\alpha < f(\omega)\} \cup \{\beta < g(\omega)\}\Big) \\
&\leq \mu\{\alpha < f(\omega)\} + \mu \{\beta < g(\omega)\} \\
&= D_f(\alpha) + D_g(\beta)
\end{align*}
by \ref{semiringPositiveContent}.

(4) Take $\alpha,\beta\in \overline{\R^+}$ and $\omega\in \Omega$. Then we have the implication
\[ \{\alpha\beta < f(\omega)g(\omega)\} \;\subseteq\; \{\alpha < f(\omega)\} \cup \{\beta < g(\omega)\}. \]
Thus we calculate
\begin{align*}
D_{fg}(\alpha+\beta) &= \mu\{\alpha\beta < f(\omega)g(\omega)\} \\
&\leq \mu\Big(\{\alpha < f(\omega)\} \cup \{\beta < g(\omega)\}\Big) \\
&\leq \mu\{\alpha < f(\omega)\} + \mu \{\beta < g(\omega)\} \\
&= D_f(\alpha) + D_g(\beta)
\end{align*}
by \ref{semiringPositiveContent}.
\end{proof}



\begin{proposition}
Let $\sSet{\Omega, \mathcal{A}, \mu}$ be a $\sigma$-finite measure space, $\varphi: \R^+\to \R^+$ a monotone differentiable function with $\phi(0) = 0$ and $f: X\to \R^+$ a measurable function such that $\varphi\circ f$ is integrable. Then
\[ \int_\Omega\varphi\circ f \diff{\mu} = \int_0^\infty \varphi'(\alpha)D_f(\alpha)\diff{\alpha}. \]
\end{proposition}
\begin{proof}
We have
\begin{align*}
\int_\Omega\varphi\big(f(\omega)\big) \diff{\mu(\omega)} &= \int_\Omega\int_{0}^{f(\omega)}\varphi'(\alpha)\diff{\alpha} \diff{\mu(\omega)} \\
&= \int_\Omega\int_{0}^{\infty}[\alpha < f(\omega)]\varphi'(\alpha)\diff{\alpha} \diff{\mu(\omega)} \\
&= \int_{0}^{\infty}\varphi'(\alpha)\int_\Omega[\alpha < f(\omega)]\diff{\mu(\omega)}\diff{\alpha} \\
&= \int_{0}^{\infty}\varphi'(\alpha)D_f(\alpha)\diff{\alpha}.
\end{align*}
by Tonelli's theorem \ref{tonellisTheorem} (as $\varphi'$ is positive) and \ref{weakSecondTheoremCalculus}.
\end{proof}
\begin{corollary} \label{LpNormFromDistribution}
Let $\sSet{\Omega, \mathcal{A}, \mu}$ be a $\sigma$-finite measure space, $0<p<\infty$ and $f\in \mathcal{L}^p(X,\mu)$. Then
\[ \norm{f}_p = \left(p\int_0^\infty \alpha^{p-1}D_f(\alpha)\diff{\alpha}\right)^{1/p}. \]
\end{corollary}
\begin{proof}
Take $\varphi: \alpha\mapsto \alpha^p$. Then $\varphi'(\alpha) = p\alpha^{p-1}$ and so
\[ \norm{f}_p^p = \int_\Omega |f|^p \diff{\mu} = \int_\Omega \varphi\circ|f| \diff{\mu} = \int_0^\infty \varphi'(\alpha)D_f(\alpha)\diff{\alpha} = p\int_0^\infty \alpha^{p-1}D_f(\alpha)\diff{\alpha}. \]
\end{proof}

\subsection{Distribution inequalities}
\begin{proposition}[Markov's inequality]
Let $(\Omega, \mathcal{A}, \mu)$ be a measure space, $f:\Omega\to\C$ a measurable function and $\alpha>0$. Then
\[ D_f(\alpha) + \mu\{f = \alpha\} = \mu\{f\geq \alpha\} \leq \frac{1}{\alpha}\int_{\{f\geq \alpha\}}f\diff{\mu} \leq \frac{1}{\alpha}\int_\Omega f\diff{\mu}. \]
\end{proposition}
Also called Chebyshev's inequality or Chebyshev's first inequality. The next inequality would then be Chebyshev's second inequality. 
\begin{proof}
We have $\alpha\cdot \big[f(\omega)\geq \alpha\big]\leq f(\omega)$ for all $\omega\in\Omega$, so
\[ \alpha\cdot \mu\{f\geq \alpha\} = \alpha \int_{\{f\geq \alpha\}}\constant{1}\diff{\mu} = \int_{\{f\geq \alpha\}}\constant{\alpha}\diff{\mu} \leq \int_{\{f\geq \alpha\}}f\diff{\mu} \leq \int_\Omega f\diff{\mu}, \]
using \ref{measureFromIntegralCharacteristicFunctions}, \ref{simpleIntegralOverSubset} and \ref{propertiesIntegralPositiveFunctions}. This implies Markov's inequality.
\end{proof}
\begin{corollary}[Chebyshev's inequality]
Let $(\Omega, \mathcal{A}, \mu)$ be a measure space, $f:\Omega\to\R$ a measurable function, $\varphi:\im(f) \to \R^+$ a monotone measurable function and $\alpha>0$. Then
\[ \mu\{f\geq \alpha\} \leq \frac{1}{\varphi(\alpha)}\int_{\{\varphi\circ f \geq \varphi(\alpha)\}} \varphi\circ f\diff{\mu} \leq \frac{1}{\varphi(\alpha)}\int_\Omega \varphi\circ f\diff{\mu}. \]
\end{corollary}
\begin{proof}
Since $\varphi$ is monotone, we have $\{f\geq \alpha\} \subseteq \{\varphi\circ f\geq \varphi(\alpha)\}$, so
\[ \mu\{f\geq \alpha\} \leq \mu\{\varphi\circ f\geq \varphi(\alpha)\} \leq \frac{1}{\varphi(\alpha)}\int_{\{\varphi\circ f \geq \varphi(\alpha)\}} \varphi\circ f\diff{\mu} \leq \frac{1}{\varphi(\alpha)}\int_\Omega \varphi\circ f\diff{\mu}, \]
using \ref{semiringPositiveContent}.
\end{proof}
By choosing specific functions $\varphi$, we obtain the following forms of Chebyshev's inequality:
\begin{corollary} \label{ChebyshevInequalityCorollary}
Let $(\Omega, \mathcal{A}, \mu)$ be a measure space, $f:\Omega\to\R^+$ a positive measurable function, $g: \Omega\to \C$ a measurable function, $\alpha>0$ and $0<p$. Then
\begin{enumerate}
\item $D_f(\alpha) \leq \mu\{f\geq \alpha\} \leq \frac{1}{\alpha^p}\int_{\{f \geq \alpha\}} f^p\diff{\mu} \leq \frac{1}{\alpha^p}\int_\Omega f^p\diff{\mu}$;
\item $\alpha D_{|g|}(\alpha)^{1/p}\leq \norm{g}_p$.
\end{enumerate}
\end{corollary}
\begin{proof}
(1) The function $x\mapsto x^p$ is monotone. It is also an order similarity, so $\{f \geq \alpha\} = \{f^p \geq \alpha^p\}$.

(2) We use (1) to obtain $D_{|g|}(\alpha) \leq D_{|g|}(\alpha) + \mu\{|g| = \alpha\} \leq \frac{1}{\alpha^p}\int_\Omega |f|^p\diff{\mu} = \left(\frac{\norm{f}_p}{\alpha}\right)^p$.
\end{proof}
\begin{corollary}
Let $(\Omega, \mathcal{A}, \mu)$ be a measure space, $\alpha>0$, $0<p$ and $f:\Omega\to\R^+$ a positive measurable function such that $\norm{f}_p < \infty$. Then
\begin{enumerate}
\item $\lim_{\alpha \to +\infty}\alpha^p D_f(\alpha) = 0$;
\item $\lim_{\alpha \to 0}\alpha^p D_f(\alpha) = 0$.
\end{enumerate}
\end{corollary}
TODO: Hao requires $p\geq 1$ for (2). Why?
\begin{proof}
(1) We have
\begin{align*}
0 \leq \lim_{\alpha \to +\infty}\alpha^p D_f(\alpha) &\leq \lim_{\alpha \to +\infty}\int_{\{f \geq \alpha\}} f^p\diff{\mu} \\
&= \lim_{\alpha \to +\infty}\int_\Omega [f \geq \alpha] f^p\diff{\mu} \\
&= \int_\Omega \lim^p_{\alpha \to +\infty}[f \geq \alpha] f^p\diff{\mu} \\
&= \int_\Omega \constant{0}\diff{\mu} = 0,
\end{align*}
where we have used \ref{functionPropertiesFromIntegral} and the dominated convergence theorem \ref{dominatedConvergence}, since $[f \geq \alpha]\cdot f^p$ is dominated by $f^p$, which is integrable by assumption.

(2) Fix $\alpha > 0$. For all $\beta > \alpha$ we have
\begin{align*}
\alpha^p\big(D_f(\alpha) - D_f(\beta)\big) &= \alpha^p\big(\mu\{f > \alpha\} - \mu\{f >\beta\}\big) \\
&= \alpha^p\mu\big(\{f > \alpha\} \setminus \{f >\beta\}\big) \\
&= \alpha^p\mu\{\alpha < f \leq \beta\} \\
&= \alpha^p\int_\Omega[\alpha < f \leq \beta]\diff{\mu} \\
&\leq \int_\Omega[\alpha < f \leq \beta]f^p\diff{\mu} \\
&\leq \int_\Omega[f \leq \beta]f^p\diff{\mu},
\end{align*}
using \ref{ringPositiveContent}, \ref{measureFromIntegralCharacteristicFunctions} and \ref{propertiesIntegralPositiveFunctions}. Now
\begin{align*}
\lim_{\alpha\to 0}\alpha^p D_f(\alpha) &= \lim_{\alpha\to 0}\alpha^p D_f(\alpha) - \alpha^p D_f(\beta) \\
&= \lim_{\alpha\to 0}\alpha^p \big(D_f(\alpha) - D_f(\beta)\big) \\
&\leq \lim_{\alpha\to 0} \int_\Omega[f \leq \beta]f^p\diff{\mu} \\
&= \int_\Omega[f \leq \beta]f^p\diff{\mu}.
\end{align*}
We have
\[ \lim_{\beta\to 0}\int_\Omega[f \leq \beta]f^p\diff{\mu} = \int_\Omega\lim_{\beta\to 0}[f \leq \beta]f^p\diff{\mu} = \int_\Omega \constant{0}\diff{\mu} = 0 \]
by \ref{functionPropertiesFromIntegral} and the dominated convergence theorem \ref{dominatedConvergence}. Since we are considering the limit $\alpha \to 0$, we can take $\beta$ arbitrarily small, and thus $\int_\Omega[f \leq \beta]f^p\diff{\mu}$ becomes arbitrarily small and we have $\lim_{\alpha \to 0}\alpha^p D_f(\alpha) = 0$.
\end{proof}

\subsection{Decreasing rearrangements}
\begin{definition}
Let $\sSet{\Omega, \mathcal{A}, \mu}$ be a measure space and $f: \Omega\to \R$ a measurable function. The \udef{decreasing rearrangement} of $f$ is the function
\[ \decRearrange{f}: \overline{\R^+}\to \overline{\R^+}: t \mapsto \inf\setbuilder{\alpha > 0}{D_f(\alpha)\leq t} = \inf D_f^{\preimf}\big(\interval{0,t}\big).  \]
\end{definition}

The function $\decRearrange{f}$ is decreasing and supported in $\interval[c]{0,\mu(X)}$.



\begin{proposition}
Let $\sSet{\Omega, \mathcal{A}, \mu}$ be a measure space, $f: \Omega\to \overline{\R^+}$ a measurable function and $s,t\in \overline{\R^+}$. Then
\begin{enumerate}
\item $\decRearrange{f}\big(D_f(t)\big) \leq t$;
\item $D_f\big(\decRearrange{f}(s)\big) \leq s$;
\item $s < \decRearrange{f}(s) \iff t < D_f(s)$.
\end{enumerate}
\end{proposition}
\begin{proof}
(1) We have $t\in D_f^\preimf\big(\interval{0,D_f(t)}\big)$, so
\[ \decRearrange{f}\big(D_f(t)\big) = \inf D_f^\preimf\big(\interval{0,D_f(t)}\big)\leq \inf\{t\} = t. \]

(2)

(3) We calculate
\[ s < \decRearrange{f}(s) \iff s< \inf D_f^{\preimf}\big(\interval{0,t}\big) \iff s\notin D_f^{\preimf}\big(\interval{0,t}\big) \iff D_f(s) \notin \interval{0,t} \iff t < D_f(s), \]
where we have used the fact that $D_f^{\preimf}\big(\interval{0,t}\big)$ is upwards closed and topologically closed.
\end{proof}


\section{The $L^p(X,\diff{\mu})$ spaces}

\subsection{Indentifying functions that are equal a.e.}

TODO: identification of a.e.\ equal functions is compatible with algebraic operations by \ref{equalAECongruence}.

TODO: allow functions that are only defined a.e.


\begin{proposition}
TODO every equivalence class contains exactly one continuous functions (for Borel measures?)
\end{proposition}


\subsection{$L^0(\Omega, \mu)$}
TODO: $\norm{f}_0 = \mu\{f\neq 0\}$ (?)

\subsubsection{Essential supremum}
\begin{definition}
Let $(\Omega, \mathcal{A}, \mu)$ be a measure space and $f\in \meas(\Omega, \C)$. The \udef{essential supremum} of $f$ is
\[ \essup(f) \defeq \inf\setbuilder{\sup_{\omega\in E}|f(\omega)|}{E\in \mathcal{A}, \mu(E^c) = 0}. \]
We call $f$ \udef{essentially bounded} if $\essup(f)<\infty$.
\end{definition}

\begin{lemma}
Let $(\Omega, \mathcal{A}, \mu)$ be a measure space and $f\in \meas(\Omega, \C)$. Then
\[ \essup(f) = \sup D_{|f|}^{\preimf}\big(\interval[o]{0,\infty}\big) = \inf D_{|f|}^{\preimf}\big(\{0\}\big). \]
\end{lemma}

\subsubsection{Convergence in measure}
\begin{definition}
Let $(\Omega, \mathcal{A}, \mu)$ be a measure space, $F\in \powerfilters\big(\meas(\Omega, \C)\big)$ and $f\in \meas(\Omega, \C)$. Then $F$ is said to \udef{converge in measure} to $f$ if
\[ \upset D_{|-|}(\epsilon)^{\imf\imf}(F-f) \overset{\overline{\R^+}}{\longrightarrow} 0 \]
for all $\epsilon > 0$.
\end{definition}

\begin{lemma}
Let $(\Omega, \mathcal{A}, \mu)$ be a measure space. The following are group norms on $\meas(\Omega, \C)$ whose norm topology coincides with convergence in measure:
\begin{enumerate}
\item $n_1: f \mapsto \min\setbuilder[\big]{\epsilon > 0}{D_{|f|}(\epsilon) \leq \epsilon}$;
\item $n_2: f \mapsto \inf_{\epsilon>0} D_{|f|}(\epsilon) + \epsilon$;
\item if $\mu$ is finite, $n_4: f \mapsto \int_\Omega \frac{|f|}{1+|f|}\diff{\mu}$;
\item if $\mu$ is finite, $n_5: f \mapsto \int_\Omega\min\{|f|, \underline{1}\}\diff{\mu}$.
\end{enumerate}
\end{lemma}
\begin{proof}
(1) We first need to show that this is well-defined, i.e.\ that $\setbuilder[\big]{\epsilon > 0}{D_{|f|}(\epsilon) \leq \epsilon}$ has a minimum.

Set $\epsilon' = \inf\setbuilder[\big]{\epsilon > 0}{D_{|f|}(\epsilon) \leq \epsilon}$. We need to show that $D_{|f|}(\epsilon') \leq \epsilon'$, or $0 \leq \epsilon'- D_{|f|}(\epsilon')$. By \ref{sequencesToExtrema}, we can find a decreasing sequence $\seq{x_n}$ in $\setbuilder[\big]{\epsilon > 0}{D_{|f|}(\epsilon) \leq \epsilon}$ that converges to $\epsilon'$.

Since $\epsilon'- D_{|f|}(\epsilon') = \epsilon' + \big(- D_{|f|}(\epsilon')\big)$ is the sum of two increasing functions, it is increasing and $x_n- D_{|f|}(x_n) \geq 0$ for all $n\in \N$. Thus $\seq{x_n- D_{|f|}(x_n)}$ is a sequence that converges to $\epsilon'- D_{|f|}(\epsilon')$ by right-continuity of $D_{|f|}$, \ref{distributionContinuity}. Since limits preserve inequalities, $\epsilon'- D_{|f|}(\epsilon')$ is positive.

Now we prove that $n_1$ is a group seminorm. It is clear that $n_1(f) = n_1(-f)$ for all $f\in\meas(\Omega, \C)$. Next we prove the triangle inequality. Take arbitrary $f,g\in \meas(\Omega, \C)$. We prove that $D_{|f+g|}\big(n_1(f) + n_1(g)\big) \leq n_1(f) + n_1(g)$, since this implies $n_1(f) + n_1(g) \in \setbuilder[\big]{\epsilon > 0}{D_{|f+g|}(\epsilon) \leq \epsilon}$ and thus
\[ n_1(f+g) = \min\setbuilder[\big]{\epsilon > 0}{D_{|f+g|}(\epsilon) \leq \epsilon} \leq n_1(f) + n_1(g). \]
The proof is as follows:
\begin{align*}
D_{|f+g|}\big(n_1(f) + n_1(g)\big) &= \mu\{|f+g| \leq n_1(f) + n_1(g)\}^c \\
&\leq \mu\{|f|+|g| \leq n_1(f) + n_1(g)\}^c \\
&\leq \mu\big(\{|f| \leq n_1(f)\} \cap \{|g| \leq n_1(g)\}\big)^c \\
&= \mu\big(\{|f| \leq n_1(f)\}^c \cup \{|g| \leq n_1(g)\}^c\big) \\
&\leq \mu\{|f| \leq n_1(f)\}^c + \mu\{|g| \leq n_1(g)\}^c \\
&= D_{|f|}\big(n_1(f)\big) + D_{|g|}\big(n_1(g)\big) \\
&\leq n_1(f) + n_1(g),
\end{align*}
where we have used \ref{semiringPositiveContent} and the fact that $D_{|f|}\big(n_1(f)\big)\leq n_1(f)$ and $D_{|g|}\big(n_1(g)\big)\leq n_1(g)$ (i.e.\ that the minimum is reached).
\end{proof}

\begin{definition}
The metric $d_1$ is known as the \udef{Ky Fan metric}. We can define a whole family of Ky Fan metrics
\[ K_\lambda: (f,g) \mapsto \inf\setbuilder[\big]{\epsilon > 0}{D_{|f-g|}(\lambda\epsilon) \leq \epsilon} \]
\end{definition}

\begin{proposition}
$\lim_{\lambda\to 0}K_\lambda(\cdot) = \norm{\cdot}_0$ and $\lim_{\lambda\to \infty}K_\lambda(\cdot) = \norm{\cdot}_\infty$.
\end{proposition}
\begin{proof}
TODO \url{https://statistik.econ.kit.edu/download/doc_secure1/3_StochModels_Appendix.pdf}
\end{proof}

\begin{lemma}
Let $(\Omega, \mathcal{A}, \mu)$ be a measure space. Then convergence in measure is a topological vector space convergence on $\meas(\Omega,\C)$ with neighbourhood filter
\[ \neighbourhood(0) = \bigvee_{\epsilon > 0}\upset D_{|-|}(\epsilon)^{\preimf\imf}(\neighbourhood_\R(0)). \]
\end{lemma}
\begin{proof}
By construction, the function $f\mapsto f+g$ is a homeomorphism for all $g\in \meas(\Omega, \C)$. Thus it is sufficient to consider the vicinity at the origin and verify the conditions in \ref{TVSconstruction}.

Take $F\in\powerfilters\big(\meas(\Omega,\C)\big)$. Then
\begin{align*}
\upset D_{|-|}(\epsilon)^{\imf\imf}(F) \overset{\overline{\R^+}}{\longrightarrow} 0 &\iff \neighbourhood_\R(0) \subseteq \upset D_{|-|}(\epsilon)^{\imf\imf}(F) \\
&\iff \upset D_{|-|}(\epsilon)^{\preimf\imf}(\neighbourhood_\R(0)) \subseteq F.
\end{align*}
This holds for all $\epsilon > 0$ if and only if $\bigvee_{\epsilon>0}\upset d_{|-|}(\epsilon)^{\preimf\imf}(\neighbourhood_\R(0)) \subseteq F$. Thus the convergence is pretopological with the prescribed neighbourhood filter.

We are now ready to verify the conditions in \ref{TVSconstruction}. Take arbitrary $A\in \bigvee_{\epsilon>0}\upset D_{|-|}(\epsilon)^{\preimf\imf}(\neighbourhood_\R(0))$. So for all $\epsilon > 0$ there exists $\delta > 0$ such that $D_{|-|}(\epsilon)^{\preimf}\big(\interval{0,\delta}\big) \subseteq A$, which is equivalent to
\[ \interval{0,\delta} \subseteq D_{|-|}(\epsilon)^{\imf\imf}\big(\interval{0,\delta}\big) \]
Take arbitrary $\lambda\in \C$. \ref{distributionOfFunctionLemma}
\end{proof}

\subsection{Completeness and density}
\begin{theorem}[Riesz-Fisher]
The space $L^p(X,\diff{\mu})$ is complete.
\end{theorem}

For $L^\infty$: essential supremum.

\begin{proposition}
Let $X$ be a set. For all $1\leq p < \infty$, the set of $p$-integrable simple functions is dense in $L^p(\diff{\mu})$.
\end{proposition}


\begin{proposition}
Let $X$ be a locally compact Hausdorff space and $\mu$ a Radon measure on $X$. Then $\cont_c(X)$ is dense in $L^p(\diff{\mu})$ for all $1\leq p < \infty$.
\end{proposition}
\begin{proof}

\end{proof}

\subsection{Operators on $L^p$ spaces}
\subsubsection{Multiplication operators}
\begin{definition}
Let $(\Omega, \mathcal{A}, \mu)$ be a measure space, $f\in \meas(\Omega, \C)$ and $1\leq p \leq \infty$. Then
\[ M_f: L^p(X, \diff{\mu}) \to L^p(X, \diff{\mu}): g\mapsto f\cdot g \]
with domain
\[ \dom(M_f) = \setbuilder{g\in L^p(X, \diff{\mu})}{f\cdot g \in L^p(X, \diff{\mu})} \]
is called a \udef{multiplication operator}.
\end{definition}

\begin{proposition} \label{boundedMultiplicationOperator}
Let $(\Omega, \mathcal{A}, \mu)$ be a semi-finite measure space, $f\in \meas(\Omega, \C)$ and $1\leq p < \infty$. Then $M_f$ is bounded \textup{if and only if} $f$ is essentially bounded.

In this case $\norm{M_f} = \essup(f)$.
\end{proposition}
TODO also holds for $p = \infty$.
\begin{proof}
$\boxed{\Rightarrow}$ We prove the contraposition. Assume $f$ is not essentially bounded. For all $n$, consider $A_n = \{|f| \geq n\}$, which is in $\mathcal{A}$ by \ref{measurablesSetsRealMeasurableFunction}. Then $\mu(A_n) > 0$. By semi-finiteness, we can find $B_n\subseteq A_n$ such that $B_n\in \mathcal{A}$ and $0 < \mu(B_n)< \infty$. Then $\norm{\charFunc{B_n}}_p = \mu(B_n)^{1/p}$, which means $\charFunc{B_n}\in L^p(X, \diff{\mu})$. We have
\begin{align*}
\norm{M_f\charFunc{B_n}}_p^p &= \int_\Omega |f\cdot \charFunc{B_n}|^{p}\diff{\mu} \\
&= \int_\Omega \charFunc{B_n}\cdot|f|^p\diff{\mu} \\
&= \int_{B_n}|f|^p\diff{\mu} \\
&\geq n^p\int_{B_n}\diff{\mu} \\
&= n^p\mu(B_n) = n^p\norm{\charFunc{B_n}}^p,
\end{align*}
which implies $\norm{M_f\charFunc{B_n}}_p \geq n\norm{\charFunc{B_n}}_p$ for all $n\in \N$. Thus $M_f$ is not bounded.

$\boxed{\Leftarrow}$ Set $f' = |f| \wedge \underline{\essup(f)}$, which is measurable by \ref{modulusMeasurable} and \ref{limitOperationsOnRealMeasurableFunctions}. Also $|f| = f'$ almost everywhere. Then, for $g\in L^p(X, \diff{\mu})$, we have
\begin{align*}
\norm{M_fg}_p^p &= \int_\Omega |f\cdot g|^p \diff{\mu} \\
&= \int_\Omega f'^p\cdot |g|^p \diff{\mu} \\
&\leq \essup(f)^p\int_\Omega |g|^p \diff{\mu} = \essup(f)^p\norm{g}_p^p.
\end{align*}
Thus $M_f$ is bounded and $\norm{M_f} \leq \essup(f)$.

Finally, we need to prove $\norm{M_f} \geq \essup(f)$. Take arbitrary $\epsilon > 0$. Consider $A_\epsilon = \{|f| \geq \essup(f)-\epsilon\}$, which is in $\mathcal{A}$ by \ref{measurablesSetsRealMeasurableFunction}. By construciton, $\mu(A_\epsilon) > 0$. By semi-finiteness, we can find $B_\epsilon\subseteq A_\epsilon$ such that $B_\epsilon\in \mathcal{A}$ and $0 < \mu(B_\epsilon)< \infty$. Then $\norm{\charFunc{B_\epsilon}}_p = \mu(B_\epsilon)^{1/p}$, which means $\charFunc{B_\epsilon}\in L^p(X, \diff{\mu})$. We have
\begin{align*}
\norm{M_f\charFunc{B_\epsilon}}_p^p &= \int_\Omega |f\cdot \charFunc{B_\epsilon}|^{p}\diff{\mu} \\
&= \int_\Omega \charFunc{B_\epsilon}\cdot|f|^p\diff{\mu} \\
&= \int_{B_\epsilon}|f|^p\diff{\mu} \\
&\geq (\essup(f)-\epsilon)^p\int_{B_\epsilon}\diff{\mu} \\
&= \big(\essup(f)-\epsilon\big)^p\mu(B_\epsilon) = \big(\essup(f)-\epsilon\big)^p\norm{B_\epsilon}^p,
\end{align*}
which implies $\norm{M_f\charFunc{B_\epsilon}}_p \geq \big(\essup(f)-\epsilon\big)\norm{B_\epsilon}$, and so $\norm{M_f} \geq \essup(f)-\epsilon$ for all $\epsilon >0$. Thus $\norm{M_f} \geq \essup(f)$.
\end{proof}

\subsection{Weak $L^p(X,\diff{\mu})$ spaces}
\begin{definition}
Let $0<p<\infty$ and let $\sSet{X,\mathcal{A}, \mu}$ be a measure space. Then we define the extended real functional
\[ \norm{\cdot}_{p,\infty}: \meas(X, \C)\to \overline{\R}^+: f\mapsto \sup\setbuilder{\alpha D_f(\alpha)^{1/p}}{\alpha > 0}. \]
The \udef{weak $L^p(X,\diff{\mu})$ space}, denoted $L^{p,\infty}(X,\diff{\mu})$, is the subset of functions $f$ in $\meas(X, \C)$ for which $\norm{f}_{p,\infty}$ is finite.
\end{definition}

\begin{lemma}
Let $0<p<\infty$, let $\sSet{X,\mathcal{A}, \mu}$ be a measure space and $f\in \meas(X, \C)$. Then
\[ \norm{f}_{p,\infty} = \inf\setbuilder{C\in \R^+}{\forall \alpha>0: \; D_f(\alpha) \leq \frac{C^p}{\alpha^p}}. \]
\end{lemma}


\begin{proposition}
Let $0<p<\infty$ and let $\sSet{X,\mathcal{A}, \mu}$ be a measure space. Then
\begin{enumerate}
\item the functional $\norm{\cdot}_{p,\infty}$ is constant on the equivalence classes of a.e.\ equality;
\item $\norm{\cdot}_{p,\infty}: \Big(\meas(X, \C)/\overset{a.e.}{=}\Big) \to \overline{\R}^+$ is a quasi-norm;
\item $\norm{\cdot}_{p,\infty} \leq \norm{\cdot}_{p}$.
\end{enumerate}
\end{proposition}
\begin{proof}
(1) If $f = g$ a.e.\ then $D_f = D_g$ by \ref{distributionOfFunctionLemma}.

(2) TODO

(3) By \ref{ChebyshevInequalityCorollary}, we have $\alpha D_f(\alpha)^{1/p} \leq \norm{f}_p$ and thus $\norm{f}_{p,\infty} = \sup_{\alpha > 0}\alpha D_f(\alpha)^{1/p} \leq \norm{f}_p$.
\end{proof}
\begin{corollary}
Let $0<p<\infty$ and let $\sSet{X,\mathcal{A}, \mu}$ be a measure space. Then $L^{p,\infty}(X,\diff{\mu})$ is a subspace of $L^{p}(X,\diff{\mu})$.
\end{corollary}
\begin{proof}
TODO \ref{realPartExtendedRealFunctional}
\end{proof}

\begin{example}
The inclusion $L^{p,\infty}(X,\diff{\mu}) \subseteq L^{p}(X,\diff{\mu})$ is strict in general.

Take $h: \R\to \R: x\mapsto |x|^{-1/p}$. Then
\[ \norm{h}_p^p = \int_\R |x|^{-1} \diff{x} = 2\int_0^\infty x^{-1}\diff{x} = \infty. \]
We also have
\[ D_h(\alpha) = \mu\big\{|x|^{-1/p} > \alpha\big\} = \mu\big\{|x| < \alpha^{-p}\big\} = 2\alpha^{-p}, \]
so
\[ \norm{h}_{p,\infty} = \sup_{\alpha > 0}\alpha\cdot (2\alpha^{-p})^{1/p} = \sup_{\alpha > 0}2^{1/p}\alpha\cdot \alpha^{-1} = 2^{1/p} < \infty. \]
\end{example}

\subsection{Interpolation}

\begin{proposition}
Let $\sSet{X,\Omega, \mu}$ be a $\sigma$-finite measure space.
Let $0<p<q<\infty$ and $f\in L^{p,\infty(X,\diff{\mu})} \cap L^{q,\infty(X,\diff{\mu})}$. Then for all $p<r<q$, we have
\[ \norm{f}_r \leq \left(\frac{r}{r-p} + \frac{r}{q-r}\right)^{1/r}\norm{f}_{p,\infty}^{\frac{r^{-1}- q^{-1}}{p^{-1} - q^{-1}}}\norm{f}_{q,\infty}^{\frac{p^{-1}- r^{-1}}{p^{-1} - q^{-1}}}, \]
with the interpretation $\infty^{-1} = 0$.
\end{proposition}
\begin{proof}
We know that $\alpha D_f(\alpha)^{1/p} \leq \norm{f}_{p,\infty}$, so $\alpha^p D_f(\alpha) \leq \norm{f}_{p,\infty}^p$. Similarly $\alpha^q D_f(\alpha) \leq \norm{f}_{q,\infty}^q$. Thus
\[ D_f(\alpha) \leq \min\left(\frac{\norm{f}^p_p}{\alpha^p}, \frac{\norm{f}^q_q}{\alpha^q}\right). \]
Now we have
\[ \frac{\norm{f}^p_{p,\infty}}{\alpha^p} \leq \frac{\norm{f}^q_{q,\infty}}{\alpha^q} \iff \alpha \leq \left(\frac{\norm{f}_{q,\infty}^q}{\norm{f}_{p,\infty}^p}\right)^{\frac{1}{q-p}} \eqdef B. \]
We calculate, using \ref{LpNormFromDistribution},
\begin{align*}
\norm{f}_r^r &= r \int_0^\infty\alpha^{r-1}D_f(\alpha)\diff{\alpha} \\
&\leq r \int_0^\infty\alpha^{r-1}\min\left(\frac{\norm{f}^p_{p,\infty}}{\alpha^p}, \frac{\norm{f}^q_{q,\infty}}{\alpha^q}\right)\diff{\alpha} \\
&= r \int_0^B\alpha^{r-1}\frac{\norm{f}^{p,\infty}}{\alpha^p}\diff{\alpha} + r \int_B^\infty\alpha^{r-1}\frac{\norm{f}^q_{q,\infty}}{\alpha^q}\diff{\alpha} \\
&= r \norm{f}^p_{p,\infty} \int_0^B\alpha^{r-p-1}\diff{\alpha} + r \norm{f}^q_{q,\infty} \int_B^\infty\alpha^{r-q-1}\diff{\alpha} \\
&= \frac{r}{r-p} \norm{f}^p_{p,\infty} B^{r-p} + \frac{r}{q-r} \norm{f}^q_{q,\infty} B^{r-q} \\
&= \left(\frac{r}{r-p} + \frac{r}{q-r}\right)^{1/r}\left(\norm{f}_{p,\infty}^p\right)^{\frac{r^{-1}- q^{-1}}{p^{-1} - q^{-1}}}\norm{f}_{q,\infty}^{\frac{p^{-1}- r^{-1}}{p^{-1} - q^{-1}}}.
\end{align*}
Observe that the integrals converge because $p<r<q$, so $r-p>0$ and $r-q<0$.
\end{proof}

\subsubsection{The Marcinkiewicz interpolation theorem}
TODO
\subsubsection{The Riesz-Thorin Interpolation Theorem}
The scalars are complex.

\begin{theorem}[Riesz-Thorin interpolation theorem]
Let $\sSet{X,\mu}$ and $\sSet{Y,\nu}$ be $\sigma$-finite measure spaces. Let $p_0,p_1,q_0,q_1\in \interval{1,\infty}$ and let $T: L^{p_0}(X) + L^{p_1}(X) \to L^{q_0}(Y) + L^{q_1}(Y)$ be a linear operator such that there exist $A_0,A_1 \geq 0$ with 
\begin{align*}
\forall f\in L^{p_0}(X): \quad &\norm{Tf}_{q_0} \leq A_0 \norm{f}_{p_0} \\
\forall f\in L^{p_1}(X): \quad &\norm{Tf}_{q_1} \leq A_1 \norm{f}_{p_1}.
\end{align*}
For all $0<\theta <1$, set
\[ p \defeq \frac{p_0p_1}{\theta p_0 + (1-\theta)p_1} \qquad\text{and}\qquad q \defeq \frac{q_0q_1}{\theta q_0 + (1-\theta)q_1}. \]
Then, for all $f\in L^p(X)$, we have
\[ \norm{Tf}_{q} \leq A_\theta \norm{f}_p, \]
with $A_\theta \leq A_0^{1-\theta}A_1^\theta$.
\end{theorem}
\begin{proof}
TODO
\end{proof}

\begin{proposition}[Young's convolution inequality]

\end{proposition}

\subsection{Locally integrable spaces}
\begin{definition}
Let $(\Omega, \mathcal{A}, \mu)$ be a measure space. The \udef{locally $L^p$} space is the space
\[ L^p_\text{loc}(\Omega) \defeq \setbuilder{f \in (\Omega\to\C)}{\text{$f\in L^p(K)$ for all compact $K\subset \Omega$}}. \]
The functions in $L^1_\text{loc}(\Omega)$ are called \udef{locally integrable} on $\Omega$.
\end{definition}
TODO: deal with equivalence classes??

\begin{lemma}

\end{lemma}

\subsection{Sequence spaces}
TODO:  $L^p(A,\mu)$ with $\mu$ counting measure.

Let $J$ be a countable index set and $x:J\to \mathbb{F}$ a sequence indexed by $J$. We define
\[ \norm{x}_p := \left(\sum_{j\in J}|x(j)|^p\right)^{1/p} \qquad\text{and}\qquad \norm{x}_\infty = \sup_{j\in J}|x(j)|. \]
So $\norm{\cdot}_1$ is the standard norm on $\mathbb{F}^n$. For general sequences there is no guarantee that these norms do not diverge.
\begin{definition}
Let $J$ be an index set, $D$ a directed set and $p\geq 1$,
\begin{align*}
\ell^p(J) &= \setbuilder{x:J\to \F}{\norm{x}_p < +\infty},\\
\ell^\infty(J) &= \setbuilder{x:J\to \F}{\norm{x}_\infty < +\infty},\\
c_0(D) &= \setbuilder{x:D\to \F}{\lim_{n\to\infty}|x(n)| = 0}, \\
c_{00}(D) &= \setbuilder{x:D\to \F}{\setbuilder{n\in D}{x(n)\neq 0}\;\text{has finite cardinality}}.
\end{align*}
unless specified we equip $c_0$ and $c_{00}$ with the norm $\norm{\cdot}_\infty$.
\end{definition}

\begin{lemma}
$c_{00}$ is dense in $\ell^p$ if it is equipped with the norm $\norm{\cdot}_p$ and dense in $c_0$ if it is equipped with the norm $\norm{\cdot}_\infty$.
\end{lemma}

Let $1<p,q<\infty$ satisfy $\frac{1}{p}+\frac{1}{q}$. We have the inequalities
\begin{align*}
\norm{xy}_1 &\leq \norm{x}_p\norm{y}_q\qquad\text{(Hölder inequality)} \\
\norm{x+y}_p &\leq \norm{x}_p+\norm{y}_p\qquad\text{(Minkowski inequality)}
\end{align*}
which follow from the general cases (TODO ref) by applying the counting measure.

\begin{proposition}
The continuous dual of $l^p(J)$ is $l^q(J)$ where $1<p,q<\infty$ satisfy $\frac{1}{p}+\frac{1}{q}$.
Also, the continuous dual of $l^1$ is $l^\infty$.
\end{proposition}


\section{Spaces of test functions}
\begin{definition}
Let $\Omega\subseteq X$ be an open subset of a normed vector space. The space of \udef{test functions} on $\Omega$ is $\cont^\infty_c(\Omega)$ equipped with a particular vector space convergence. This space is also commonly denoted $\testFuncs(\Omega)$.

For any compact $K\subseteq \Omega$, we define
\[ \testFuncs_K(\Omega) \defeq \setbuilder{f\in \testFuncs(\Omega)}{\supp(f) \subseteq K}. \]
\end{definition}

\begin{example}
The \udef{bump function}
\[ \phi: \R\to\R: x\mapsto \begin{cases}
e^{1/(x^2-1)} & |x|<1 \\ 0 & |x| \geq 1
\end{cases} \]
is a test function in $\testFuncs(\R)$.
\end{example}

\begin{lemma}
Every test function is bounded.
\end{lemma}
\begin{proof}
TODO
\end{proof}

\subsection{The convergence of test functions}

\begin{definition}
We denote by $\testFuncs(\Omega)$ the convergence space $\cont^\infty_C(\Omega)$ with the inductive limit convergence.

The topological modification of $\testFuncs_c(\Omega)$ is denoted $\testFuncs(\Omega)$.
\end{definition}

\begin{lemma}
The convergence space
\end{lemma}



\section{The module of distributions}
\begin{definition}
    Let $X$ be an open subset of a normed vector space.
    A linear map $T: \testFuncs(X) \to \C$ is a \udef{distribution} on $X$ is an element of the topological dual of $\testFuncs(X)$ equipped with the strong dual topology. It is denoted $\dists(X)$.
\end{definition}

\begin{proposition}
    The space of distributions $\dists(X)$ is a module over the ring $\cont^\infty(X)$.
    
    If $T\in \dists(X)$ and $a\in \cont^\infty(X)$, then the multiplication is defined by
    \[ (aT)(\phi) = T(a\phi) \qquad \forall \phi \in \testFuncs(X). \]
\end{proposition}

\subsection{Types of distributions}
\subsubsection{Regular distributions}
\begin{lemma}
Let $X$ be an open subset of a normed vector space and $f: X\to\C$ a locally integrable function in $L^1_\text{loc}(X)$. Then
\[ T_f: \testFuncs(X) \to \C: \phi \mapsto \int_X f(x)\phi(x)\diff{x} \]
is a distribution.
\end{lemma}
\begin{proof}
TODO We just need to show continuity (from boundedness)
\end{proof}

\begin{definition}
Distributions of the form $T_f$ for some $f\in L^1_\text{loc}(X)$ are called \udef{regular distributions}.
\end{definition}


\begin{lemma} \label{uniquenessIntegratedFunction}
Let $f,g: X\to \C$ be locally, absolutely integrable functions. If $T_f = T_g$, then $f(x) = g(x)$ a.e.
\end{lemma}

\subsubsection{Dirac delta distribution}
\begin{definition}
    Let $X$ be an open subset of a normed vector space and $x_0\in X$. The \udef{Dirac delta distribution} at $x_0$ is the distribution
    \[ \delta_{x_0}: \testFuncs(X) \to \C: \phi \mapsto \phi(x_0). \]
    We write $\delta \defeq \delta_0$.
\end{definition}

\begin{proposition}
Let $X$ be an open subset of a normed vector space containing $0$. Suppose $\seq{f_n: X\to \C}$ is a sequence of functions such that
\begin{enumerate}
\item $\int_X f_n(x)\diff{x} = 1$ for all $n$;
\item there exists a constant $C$ such that $\int_X |f_n(x)|\diff{x} \leq C$ for all $n$;
\item $\lim_{n\to \infty} \int_{|x|>r}|f_n(x)|\diff{x} = 0 $ for all $r > 0$.
\end{enumerate}
If $\phi$ is bounded on $X$ and continuous at $0$, then
\[ \lim_{n\to \infty} \int_X f_n(x)\phi(x)\diff{x} = \phi(0) \]
and in particular $f_n\to \delta$ in $\dists(X)$.
\end{proposition}
Notice that the condition of $\phi$ being bounded on $X$ and continuous at $0$ is much less strong than $\phi\in\testFuncs$.
\begin{proof}
TODO + iff? Then we could define delta sequences as sequences that converge to $\delta$.

TODO link with (and formulation of integral mean value theorem)
\end{proof}
Sequences $\seq{f_n}$ satisfying the assumptions of the proposition are called \udef{delta sequences}.

\begin{lemma}
If $f_n$ is such that $\int f_n(x) \diff{x} = 1$ and the support shrinks to zero, then $f_n$ is a delta sequence.
\end{lemma}

\begin{lemma}
Let $f: \R^N\to\C$ be an integrable function with $\int_{\R^N}f(x)\diff{x} = 1$. Then $\seq{n^Nf(nx)}_{n\in \N}$ is a delta sequence.
\end{lemma}

\subsection{The derivative of a distribution}
\begin{definition}
Let $X$ be an open subset of a normed vector space $V$ and let $T\in \dists(X)$. For $u\in V$ we define the directional derivative of $T$ as
\[ \partial_u(T) \defeq - T\circ \partial_u. \]

In particular, if $V = \R$, we define
\[ T' \defeq \od{}{x}(T) \defeq - T\circ \od{}{x}. \]
\end{definition}

\begin{lemma}
Let $T_f$ be an integral distribution such that $\partial_u(f)$ is well-defined on $X$, then
\[ \partial_u(T_f) = T_{\partial_{u}(f)}. \]
\end{lemma}
\begin{proof}
TODO by partial integration.
\end{proof}

\begin{lemma}
Let $X$ be an open subset of $\R^N$ and let $T\in \dists(X)$. Then
\[ D^\alpha T = (-1)^{|\alpha|}T\circ D^\alpha. \]
\end{lemma}

\begin{proposition}
    Let $X$ be an open subset of a normed vector space $V$, $T\in \dists(X)$ and $u\in V$. Then
    \begin{enumerate}
    \item $\partial_{u}T \in \dists(X)$;
    \item $\partial_{u}$ is linear on the module $\dists(X)$.
    \end{enumerate}
\end{proposition}

\begin{proposition}
Let $\theta$ be the Heaviside function. Then
\[ (T_\theta)' = \delta. \]
\end{proposition}

\subsubsection{Jump discontinuities in integral distributions}
\begin{proposition}
Let $f$ be a function that is $\cont^1$ on $]-\infty, x_0[$ and $]x_0, +\infty[$ for some $x_0\in\R$. Then
the derivative of $f$ as distribution is given by
\[ f' = (f|_{\R\setminus\{x_0\}})' + (\Delta_{x_0}f)\delta_{x_0}. \]
\end{proposition}
Note that if $f$ is continuous at $x_0$, this gives $f' = (f|_{\R\setminus\{x_0\}})'$ as distributions.
\begin{proof}
Let $\phi\in\testFuncs(\R)$. Then we calculate
\begin{align*}
f'(\phi) &= \int_{-\infty}^\infty f(x)\phi'(x)\diff{x} = \int_{-\infty}^{x_0} f(x)\phi'(x)\diff{x} + \int_{x_0}^\infty f(x)\phi'(x)\diff{x} \\
&= \Big[f(x)\phi(x)\Big]_{-\infty}^{x_0} - \int_{-\infty}^{x_0} f'(x)\phi(x)\diff{x} + \Big[f(x)\phi(x)\Big]_{x_0}^{+\infty} - \int_{x_0}^{+\infty} f'(x)\phi(x)\diff{x} \\
&= -\int_{\R\setminus\{x_0\}} f'(x)\phi(x)\diff{x} + \lim_{x\to x_0+}f(x)\phi(x)-\lim_{x\to x_0-}f(x)\phi(x) \\
&= \int_{\R}f|_{\R\setminus\{x_0\}}\phi'(x)\diff{x} + \phi(x_0)\lim_{x\to x_0+}f(x)- \phi(x_0)\lim_{x\to x_0-}f(x) = \int_{\R}(f|_{\R\setminus\{x_0\}})'\phi(x)\diff{x} + (\Delta_{x_0}f)\delta_{x_0}.
\end{align*}
\end{proof}
\begin{corollary}
Let $f$ be a function that is $\cont^1$ on $]-\infty, x_0[$ and $]x_0, +\infty[$ for some $x_0\in\R$. Then
the $k^\text{th}$ derivative of $f$ as distribution is given by
\[ f^{(k)} = (f|_{\R\setminus\{x_0\}})^{(k)} + \sum_{j=0}^{k-1}(\Delta_{x_0}f^{(j)})\delta^{(k-1-j)}_{x_0}. \]
\end{corollary}

\subsection{Convolution}


\section{Sobolev spaces}
TODO: after $L^p$ spaces.
\subsection{Weak and strong derivatives}
\begin{definition}
    Let $X$ be an open subset of a normed vector space, $f\in L^p(X)$ and $D^\alpha$ a derivative on $X$. If there exists $g\in L^q(X)$ such that $D^\alpha T_f = T_g$, then $g$ is the \udef{weak $\alpha$ derivative} of $f$ in $L^q(X)$.
\end{definition}
The weak $\alpha$ derivative is unique if it exists, due to \ref{uniquenessIntegratedFunction}.

The definition of weak derivative translates to the requirement
\[ \int_X f D^\alpha \phi \diff{x} = (-1)^{|\alpha|}\int_X g\phi\diff{x} \qquad \forall \phi\in\testFuncs(X). \]

\begin{definition}
    Let $X$ be an open subset of a normed vector space, $f\in L^p(X)$ and $D^\alpha$ a derivative on $X$. We call $g\in L^q(X)$ a \udef{strong $\alpha$ derivative} if there exists a sequence $\seq{f_n} \subset \cont^\infty(X)$  such that
    \begin{itemize}
    \item $f_n \to f$ in $L^p(X)$; and
    \item $D^{\alpha}f_n \to g$ in $L^q(X)$.
    \end{itemize}
\end{definition}

\begin{theorem}
Let $f\in L^p(X)$. Then $g\in L^q(X)$ is a weak $\alpha$ derivative \textup{if and only if} it is a strong $\alpha$ derivative.
\end{theorem}

\subsection{Sobolev spaces}
\begin{definition}
Let $X$ be an open subset of a normed vector space, $1\leq p \leq \infty$ and $k\in\N$. Then the \udef{Sobolev space} $W^{k,p}(X)$ is defined as
\[ W^{k,p}(X) \defeq \setbuilder{T\in \dists(X)}{\text{$T$ has a weak $\alpha$ derivative in $L^p(X)$ for all $|\alpha|\leq k$}}.
 \]
\end{definition}
In particular each distribution in $W^{k,p}(X)$ is an integral distribution $T_f$ for some $f$ in $L^p(X)$.

\begin{lemma}
    Let $X$ be an open subset of a normed vector space, $1\leq p \leq \infty$ and $k\in\N$. Then
    \[ \testFuncs(X) \subset W^{k,p}(X) \subset L^p(X), \]
    so that $W^{p,k}(X)$ is a dense subspace of $L^p(X)$.
\end{lemma}

\begin{proposition}
A Sobolev space $W^{k,p}(X,\diff{\mu})$ is a Banach space with norm
\[ \norm{f}_{W^{k,p}(X,\diff{\mu})} = \begin{cases}
\left(\sum_{|\alpha|\leq k}\norm{D^\alpha f}^p_{L^{p}(X, \diff{\mu})}\right)^{1/p} & 1\leq p < \infty \\
\max_{|\alpha|\leq k} \norm{D^\alpha f}_{L^{\infty}(X, \diff{\mu})} & p = \infty.
\end{cases} \]
If the measure is clear, we may also write $W^{k,p}(X)$.

In particular $W^{k,2}(X, \diff{\mu})$ is a Hilbert space with inner product
\[ \inner{f,g} \sum_{|\alpha| \leq k} \int_X D^\alpha f(x) \overline{D^\alpha g(x)}\diff{\mu(x)}. \]
The Hilbert space $W^{k,2}(X, \diff{\mu})$ is more commonly denoted $H^k(X, \diff{\mu})$.
\end{proposition}
TODO: alternative definition: use ordinary derivative and take completion???

\begin{proposition}
Let $X$ be an open subset of a normed vector space, $1\leq p < \infty$ and $k\in\N$. Then $W^{k,p}(X)$ coincides with the closure of $\cont^\infty(X) \cap W^{k,p}(X)$ in the $W^{k,p}(X)$-norm.
\end{proposition}

\begin{definition}
Let $X$ be an open subset of a normed vector space, $1\leq p < \infty$ and $k\in\N$. We define $W_0^{k,p}(X)$ to be the closure of $\cont^\infty_c(X)$ in the $W^{k,p}(X)$-norm.
\end{definition}


\chapter{Harmonic analysis}
\chapter{Functions on measure spaces}

\subsubsection{Operators on sequence spaces}
TODO Gribanov's theorems

3.7.1, 3.7.2 of Hanson / Yakovlev.

\subsubsection{Series in Banach spaces}
TODO
\url{https://link.springer.com/content/pdf/10.1007%2F978-0-8176-4687-5_3.pdf}
\begin{definition}
Let $\seq{x_n}$ be a sequence in a Banach space $X$. As for series of scalars, we say a series $\sum_{n=1}^\infty x_n$ is
\begin{itemize}
\item \udef{unconditionally convergent} if $\sum_{n=1}^\infty x_{\sigma(n)}$ converges for every permutation $\sigma$ of $\N$;
\item \udef{absolutely convergent} if $\sum_{n=1}^\infty \norm{x_n} < \infty$.
\end{itemize}
\end{definition}

\begin{proposition} \label{absoluteUnconditionalConvergenceBanach}
Let $\seq{x_n}$ be a sequence in a Banach space $X$. If $\sum_{n=1}^\infty$ converges absolutely, then it converges unconditionally.
\end{proposition}
\begin{proof}
Assume absolute convergence, so $\sum\norm{x_i}<\infty$. Then (for $m< n$)
\[ \norm{\sum_{i=1}^n x_i - \sum_{i=1}^m x_i} = \norm{\sum_{i=m+1}^n x_i} \leq \sum_{i=m+1}^n\norm{x_i} = \sum_{i=1}^n \norm{x_i} - \sum_{i=1}^m \norm{x_i}, \]
and because $\sum\norm{x_i}$ converges, it is a Cauchy sequence and by the inequality so is $\sum x_i$. By completeness this sequence is convergent.

By (TODO ref) $\sum\norm{x_{\sigma(i)}}$ converges for any permutation $\sigma$ of $\N$. We can then repeat the argument to show $\sum x_{\sigma(i)}$ is also convergent and thus unconditionally convergent.
\end{proof}





\section{Bochner integration}

TODO: the Bochner integral is the unique extension of the integral of simple functions to the set of Bochner measurable functions???? (i.e.\ simple functions dense in Bochner space, with $L^1$ metric)
\begin{definition}
Let $(\Omega, \mathcal{A},\mu)$ be a measure space and $Y$ a normed vector space. Then a Bochner measurable function $f:\Omega\to Y$ is called \udef{Bochner integrable} if there exists a sequence of integrable simple functions $\seq{s_n}\subset\SF(\Omega,Y)$ such that
\[ \lim_{n\to\infty}\int_\Omega \norm{f-s_n}\diff{\mu} = 0. \]
Take such a sequence $\seq{s_n}$. The \udef{Bochner integral} of $f$ on $\Omega$ w.r.t. $\mu$ is defined as
\[ \int_\Omega f\diff{\mu} \defeq \lim_{n\to\infty}\int_\Omega s_n\diff{\mu}. \]
\end{definition}

\begin{lemma}
The Bochner integral is well-defined: let $\seq{s_n},\seq{t_n}\in \prescript{\N}{}{\SF(\Omega,Y)}$ be sequences such that
\[ \lim_{n\to\infty}\int_\Omega \norm{f-s_n}\diff{\mu} = 0 = \lim_{n\to\infty}\int_\Omega \norm{f-t_n}\diff{\mu}.  \]
Then
\begin{enumerate}
\item the limits $\lim_{n\to\infty}\int_\Omega s_n\diff{\mu}$ and $\lim_{n\to\infty}\int_\Omega t_n\diff{\mu}$ exist;
\item $\lim_{n\to\infty}\int_\Omega s_n\diff{\mu} = \lim_{n\to\infty}\int_\Omega t_n\diff{\mu}$.
\end{enumerate}
\end{lemma}
\begin{proof}
TODO
\end{proof}

\begin{proposition}[Bochner integrability criterion] \label{BochnerIntegrabilityCondition}
Let $(\Omega, \mathcal{A},\mu)$ be a measure space and $Y$ a normed vector space.

A Bochner measurable function $f$ is Bochner integrable \textup{if and only if}
\[ \int_\Omega \norm{f} \diff{\mu} < \infty. \]
\end{proposition}

\begin{proposition}
Linearity and monotonicity.
\end{proposition}

\begin{theorem}[Hille's theorem] \label{HilleTheorem}
Let $(\Omega, \mathcal{A},\mu)$ be a measure space, $X,Y$ normed vector spaces and $T: X\not\to Y$ a closed operator. If $T\circ f$ is integrable, then
\[ \int_\Omega (T\circ f)\diff{\mu} = T\left(\int_\Omega f\diff{\mu}\right). \]
\end{theorem}
\begin{proof}
TODO
\end{proof}
\begin{corollary} \label{boundedOperatorUnderIntegral}
If $T$ is bounded, then $T\circ f$ is integrable and
\[ \int_\Omega (T\circ f)\diff{\mu} = T\left(\int_\Omega f\diff{\mu}\right). \]
\end{corollary}
\begin{proof}
TODO: show that $T\circ f$ is integrable!
\end{proof}

TODO Dominated convergence.

\subsection{Integration of bounded operators}
\begin{lemma} \label{integralBoundedOperator}
Let $X$ be a normed space and $(\Omega, \mathcal{A},\mu)$ a measure space. Let $T: \Omega \to \Bounded(X)$ be a function. If $T$ is integrable, then for all $x\in X$, $Tx$ is integrable and
\[ \left(\int_\Omega T\diff{\mu}\right)x = \int_\Omega Tx\diff{\mu}. \]
\end{lemma}
\begin{proof}
The evaluation map $\evalMap_x$ is linear and bounded by $\norm{x}$ for all $x\in X$, so we can use \ref{boundedOperatorUnderIntegral}.
\end{proof}

\chapter{Orthogonal polynomials and random matrices}
TODO: not harmonic analysis, but just parked

\section{Orthogonal polynomials}
\begin{definition}

\end{definition}



\chapter{Analysis on metric spaces}
\begin{definition}
A \udef{metric measure space} is a structured set $\sSet{X, d, \mu}$ such that $d$ is a metric on $X$ and $\mu$ is a Borel measure on $\sSet{X,d}$.
\end{definition}
\section{Doubling measures and differentiation}
\subsection{Doubling measures}
\begin{definition}
A metric measure space $\sSet{X,d,\mu}$ is called a \udef{doubling space} and the measure $\mu$ a \udef{doubling measure} if
\begin{itemize}
\item each ball of non-zero radius has finite and non-zero measure;
\item there exists $C\geq 1$ such that
\[ \forall x\in X: \delta >0:\quad \mu\big(\ball_d(x, 2\delta)\big) \leq C \mu\big(\ball_d(x, \delta)\big). \]
\end{itemize}
We call $C$ the \udef{doubling constant}.
\end{definition}
TODO: $C\geq 2$ \url{https://www.acadsci.fi/mathematica/Vol44/vol44pp1015-1030.pdf}

\begin{lemma} \label{doublingMeasureFiniteMeasureOfBallsImpliesCountableCollection}
Let $\sSet{X,d,\mu}$ be a doubling metric measure space and $\big\{\ball(x_i, \epsilon_i)\big\}_{i\in I}$ a set of pairwise disjoint balls such that $\mu\Big(\biguplus_{i\in I}\ball(x_i, \epsilon_i)\Big) < \infty$. Then at most countably many $\epsilon_i$ are non-zero.
\end{lemma}
\begin{proof}
We have
\[ \sum_{i\in I}\mu\big(\ball(x_i, \epsilon_i)\big) \leq \mu\Big(\biguplus_{i\in I}\ball(x_i, \epsilon_i)\Big) < \infty \]
by \ref{ringPositiveContent}, so only countably many balls have non-zero measure by \ref{finiteSumsAreCountable} and each ball with non-zero radius has non-zero measure, by definition.
\end{proof}

\begin{lemma}
Let $\sSet{X,d,\mu}$ be a doubling metric measure space with doublng constant $C$, $x\in X$ and $r,k>0$. Then
\[ \mu\big(\ball(x,kr)\big) \leq C^{\ceil{\log_2(k)}}\ball(x,r). \]
\end{lemma}
\begin{proof}
We have
\[ \mu\big(\ball(x,kr)\big) = \mu\big(\ball(x,2^{\log_2(k)}r)\big) \leq \mu\big(\ball(x,2^{\ceil{\log_2(k)}}r)\big) \leq C^{\ceil{\log_2(k)}}\mu\big(\ball(x,r)\big). \]
\end{proof}

\subsection{Covering theorems}
\subsubsection{Vitali spaces}
\begin{definition}
Let $\sSet{X,d,\mu}$ be a metric measure space.
\begin{itemize}
\item A \udef{Vitali covering} of $E\subseteq X$ is a set $\mathcal{C}$ of closed balls such that 
\[ \forall x\in E: \inf \setbuilder{\epsilon > 0}{\cball(x,\epsilon) \in \mathcal{C}} = 0. \]
\item We call $\sSet{X,d,\mu}$ a \udef{Vitali space} if balls in $X$ have finite measure every Vitali covering of $E$ has a countable disjoint subset that covers almost all of $E$.
\end{itemize}
\end{definition}
TODO: allow Vitali coverings that are not balls.

\begin{theorem}[Vitali's covering theorem]
Every doubling metric measure space $\sSet{X,d,\mu}$ is a Vitali space.
\end{theorem}
\begin{proof}
Take $E\subseteq X$ and let $\mathcal{C}$ be a Vitali covering of $E$. We can restrict $\mathcal{C}$ to the balls contained in it with diameter strictly larger than $0$ and less than $1$. By \ref{5foldCoveringLemma} we can find a pairwise disjoint subset $\mathcal{C}'\subseteq \mathcal{C}$ such that $A\subseteq \bigcup_{\cball(x,r)\in \mathcal{C}'}\cball(x,5r)$.

If $\bigcup \mathcal{C}'$ has finite measure, then it is countable by \ref{doublingMeasureFiniteMeasureOfBallsImpliesCountableCollection}.

TODO
\end{proof}

\begin{proposition}
Let $d$ be the standard metric on $\R^n$ and $\mu$ a Radon measure. Then $\sSet{\R^n, d,\mu}$ is a Vitali space.
\end{proposition}
\begin{proof}
TODO
\end{proof}

\begin{theorem}[Lebesgue's differentiation theorem]
Let $\sSet{X,d,\mu}$ be a Vitali space and $f: X\to \R^+$ a positive locally integrable function. Then
\[ \lim_{r\to 0} \frac{1}{\mu\big(\ball(x, r)\big)}\int_{\ball(x, r)}f\diff{\mu} = f(x) \]
for almost all $x\in X$.
\end{theorem}
\begin{proof}
Let $E$ be the set of points in $X$ where the statement does not hold. 
\end{proof}

\subsubsection{Directionally limited spaces}

TODO Besicovitch-Federer covering theorem.

\section{Maximal functions and differentiation}
\subsection{The Hardy-Littlewood maximal operator}
\begin{definition}
Let $\sSet{X,d}$ be a metric space equipped with a Borel measure $\mu$. The \udef{Hardy-Littlewood maximal operator} $\HLmax$ maps a function $f: X\to \R^+$ to the function $\HLmax(f): X\to \overline{\R^+}$ defined by
\[ \HLmax(f)(x) = \sup_{\delta >0}\frac{1}{\mu\big(\ball(x,\delta)\big)}\int_{\ball(x,\delta)} f \diff{\mu}. \]
\end{definition}

\begin{proposition}
Let $\sSet{X,d}$ be a metric space equipped with a Borel measure $\mu$ and $f: X\to \R^+$ some function. Then
\begin{enumerate}
\item if $\HLmax(f)$ is integrable, then $f = 0$ a.e.;
\item if $\HLmax(f)(x) = 0$ for some $x\in X$, then $f=0$ a.e. 
\end{enumerate}
\end{proposition}
\begin{proof}
TODO!
\end{proof}

\begin{proposition}
Let $\sSet{X,d, \mu}$ be a doubling space with doubling contstant $C$, $f: X\to \R^+$ some function and $\alpha > 0$. Then
\[ \mu\{\HLmax(f) > \alpha\} \leq \frac{C^2}{\alpha}\int_{X}f\diff{\mu}. \]
\end{proposition}
\begin{proof}
For any $R>0$ we define the bounded Hardy-Littlewood maximal operator $\HLmax_R(f)$ by
\[ \HLmax_R(f)(x) \defeq \sup_{R\geq\delta >0}\frac{1}{\mu\big(\ball(x,\delta)\big)}\int_{\ball(x,\delta)} f \diff{\mu}. \]
For all $x\in \{\HLmax_R(f) > \alpha\}$, we can find a ball $\ball(x,r)$ with $0<r\leq R$ such that
\[ \int_{\ball(x,r)}f \diff{\mu} \geq \alpha\mu\big(\ball(x,r)\big). \]
Clearly $\HLmax(f) = \sup_{R>0}\HLmax_R(f)$.

We can now apply \ref{5foldCoveringLemma} to this collection of balls (with an improved constant of $4 = 2^2$ by \ref{improvedConstant5foldCoveringLemma}) to obtain a countable subset $\mathcal{C}$ such that
\begin{align*}
\mu\{\HLmax_R(f) > \alpha\} &\leq \mu\Big(\bigcup_{\ball(x,r)\in \mathcal{C}}\ball(x,4r)\Big) \\
&\leq \sum_{\ball(x,r)\in \mathcal{C}}\mu\Big(\ball(x,4r)\Big) \\
&\leq C^2\sum_{\ball(x,r)\in \mathcal{C}}\mu\Big(\ball(x,r)\Big) \\
&\leq C^2\sum_{\ball(x,r)\in \mathcal{C}}\frac{1}{\alpha}\int_{\ball(x,r)}f \diff{\mu} = \frac{C^2}{\alpha}\int_{\biguplus \mathcal{C}} f \diff{\mu} \leq \frac{C^2}{\alpha}\int_{X} f \diff{\mu}.
\end{align*} 



\end{proof}
\begin{corollary}

\end{corollary}

\section{Approximation}
\subsection{Korovkin approximation}

\begin{theorem}[Korovkin]
Let $\sSet{X,d}$ be a metric space. Let $\varphi: \interval[co]{0,\infty} \to \R^+$ be a continuous function such that
\begin{itemize}
\item $\varphi(t) > 0$ for all $t>0$;
\item $\exists T>0, \epsilon_0>0: \varphi(t) \geq \epsilon_0$ for all $t\geq T$.
\end{itemize}
Let $E$ be a Riesz subspace of $(X\to \R)$ such that $\varphi\circ d(x,-)\in E$ for all $x\in X$ and $\underline{1} \in E$.

Let $\seq{L_n: E\to (X\to \R)}$ be a sequence of positive linear operators such that
\begin{itemize}
\item $\lim_{n\to \infty} L_n(\underline{1}) = \underline{1}$ uniformly; and
\item $\lim_{n\to \infty} x\mapsto L_n(\varphi\circ d(x,-))(x) = \underline{0}$ uniformly for all $x\in X$.
\end{itemize}
Then $\lim_{n\to \infty}L_n(f) = f$ uniformly for all uniformly continuous $f\in E$.
\end{theorem}
\begin{proof}
Let $f\in E$ be uniformly continuous. By uniform continuity (\ref{uniformContinuityMetricSpaces}), there exists a $\delta > 0$ such that $d(x,y)\leq \delta \implies |f(x)-f(y)| \leq \epsilon$.

Now we claim that $\inf\varphi^\imf\big(\interval[co]{\delta, +\infty}\big)$ is strictly positive. Indeed,
\[ \inf\varphi^\imf\big(\interval[co]{\delta, +\infty}\big) = \min \Big(\inf\varphi^\imf\big(\interval{\delta, T}\big), \inf\varphi^\imf\big(\interval[co]{T, +\infty}\big) \Big) \geq \min\Big(\inf\varphi^\imf\big(\interval{\delta, T}\big), \epsilon_0\Big), \]
and $\varphi$ attains its infimum on $\interval{\delta, T}$ by \ref{extremeValueTheorem}, so $\inf\varphi^\imf\big(\interval{\delta, T}\big) = \min\varphi^\imf\big(\interval{\delta, T}\big) > 0$. We can now set $C = \inf\varphi^\imf\big(\interval[co]{\delta, +\infty}\big)^{-1}$.

For all $x,y\in X$ such that $d(x,y) \geq \delta$, we have
\[ |f(x) - f(y)| \leq 2\norm{f}_u \leq \frac{2\norm{f}_\infty}{C}(\varphi\circ d)(x,y). \]
Putting the cases of $d(x,y) \leq \delta$ and $d(x,y) \geq \delta$ together, gives
\[ |f(x) - f(y)| \leq \epsilon + \frac{2\norm{f}_u}{C}(\varphi\circ d)(x,y) \]
for all $x,y\in X$. Or, for fixed $x\in X$, $|f - f(x)\underline{1}| \leq \epsilon\underline{1} + \frac{2\norm{f}_u}{C}(\varphi\circ d)(x,-)$.

Thus, for any $x\in X$ and $n\in \N$, we have
\begin{align*}
|L_n(f)(x) - f(x)L_n(\underline{1})(x)| &\leq |L_n(f - f(x)\underline{1})(x)| \\
&\leq L_n(|f - f(x) \underline{1}|)(x) \\
&\leq \epsilon L_n(\underline{1})(x) + \frac{2\norm{f}_u}{C}L_n\big((\varphi\circ d)(x,-)\big)(x),
\end{align*}
where we have used that each $L_n$ preserves to order. Taking the supremum over $x$ and then the limsup $n\to \infty$ gives
\[ \limsup_{n\to\infty}\norm{L_n(f) - f\cdot L_n(\underline{1})}_u \leq \epsilon \]
for all $\epsilon >0$. Thus $\limsup_{n\to\infty}\norm{L_n(f) - f\cdot L_n(\underline{1})} = 0$, which means it must equal the liminf and $L_n(f) - f\cdot L_n(\underline{1})$ converges uniformly to $0$ (by \ref{realConvergenceOrderConvergence} and \ref{vectorSpaceUniformNorm}). Since $f\cdot L_n(\underline{1})$ converges uniformly to $f$ by assumption and the uniform convergence is a vector space convergence (\ref{vectorSpaceUniformNorm}), $L_n(f)$ converges uniformly to $f$.
\end{proof}
\begin{corollary}[Korovkin's first theorem or Bohman-Korovkin theorem] \label{BohmanKorovkin}
Let $\seq{L_n: \cont(\interval{0,1})\to (\interval{0,1}\to \R)}$ be a sequence of positive linear operators and consider $(-)^i: \interval[0,1] \to \R: x\mapsto x^i$. If $L_n\big((-)^i\big) \to (-)^i$ uniformly for $i\in\{0,1,2\}$, then $L_n(f) \to f$ uniformly for all $f\in\cont(\interval{0,1})$.
\end{corollary}
\begin{proof}
We set $\varphi = (-)^2$. Then
\begin{align*}
(\varphi\circ d)(x,y) &= |x-y|^2 = (x-y)^2 = x^2 -2xy +y^2 \\
&= \big(x^2\underline{1} - 2x\id_X + (-)^2\big)(y) \\
&= \big(x^2(-)^0 - 2x(-)^1 + (-)^2\big)(y),
\end{align*}
and, since the uniform convergence is a vector space convergence (by \ref{vectorSpaceUniformNorm}), we have
\begin{align*}
L_n\big((\varphi\circ d)(x,-)\big)(x) &= \Big(x^2L_n\big((-)^0\big) - 2xL_n\big((-)^1\big) + L_n\big((-)^2\big)\Big)(x) \\
&\to \Big(x^2(-)^0 - 2x(-)^1 + (-)^2\Big)(x) \\
&= x^2 - 2x\, x + x^2 = 2x^2 - 2x^2 = 0.
\end{align*}
Because every continuous function on $\cont(\interval{0,1})$ is uniformly continuous (by the Heine-Cantor theorem \ref{HeineCantorTheorem}), we may apply the theorem.
\end{proof}
\begin{corollary}[Korovkin's second theorem]
TODO. See Ciarlet and Altomare.
\end{corollary}

\subsubsection{Bernstein polynomials and Weierstrass approximation}
\begin{definition}
The polynomials
\[ b_{k,n}(x) = \begin{pmatrix}
n \\ k
\end{pmatrix}x^k(1-x)^{n-k} \]
for $k,n\in \N$ and $k\leq n$ are called \udef{Bernstein polynomials}.
\end{definition}

\begin{lemma} \label{BernsteinPartitionUnity}
The Bernstein polynomials of degree $n$ form a partition of unity:
\[ \sum_{k=0}^n b_{k,n}(x) = 1. \]
\end{lemma}
\begin{proof}
The binomial identity \ref{binomialIdentity} gives
\[ \sum_{k=0}^n b_{k,n}(x) = \sum_{k=0}^n \begin{pmatrix}
n \\ k
\end{pmatrix}x^k(1-x)^{n-k} = (x+1-x)^n = 1^n = 1. \]
\end{proof}

\begin{proposition}
Let $n,k\in \N$. Then
\begin{enumerate}
\item $\displaystyle b_{k,n}(x) = \sum_{m=k}^n \begin{pmatrix}
n \\ m
\end{pmatrix}\begin{pmatrix}
m \\ k
\end{pmatrix}(-1)^{m-k}x^m$;
\item $\displaystyle x^m = \begin{pmatrix}
n \\ m
\end{pmatrix}^{-1}\sum_{k=m}^n \begin{pmatrix}
k \\ m
\end{pmatrix}b_{k,n}(x)$.
\end{enumerate}
\end{proposition}
\begin{proof}
(1) Filling in the binomial identity \ref{binomialIdentity} gives
\begin{align*}
b_{k,n}(x) &= \begin{pmatrix}
n \\ k
\end{pmatrix}x^k(1-x)^{n-k} \\
&= \begin{pmatrix}
n \\ k
\end{pmatrix}\sum_{m=0}^{n-k}\begin{pmatrix}
n-k \\ m
\end{pmatrix}(-1)^{m}x^mx^k \\
&= \sum_{m=k}^{n}\begin{pmatrix}
n \\ k
\end{pmatrix}\begin{pmatrix}
n-k \\ m-k
\end{pmatrix}(-1)^{m-k}x^m \\
&= \sum_{m=k}^{n}\begin{pmatrix}
n \\ m
\end{pmatrix}\begin{pmatrix}
m \\ m-k
\end{pmatrix}(-1)^{m-k}x^m \\
&= \sum_{m=k}^{n}\begin{pmatrix}
n \\ m
\end{pmatrix}\begin{pmatrix}
m \\ k
\end{pmatrix}(-1)^{m-k}x^m,
\end{align*}
where we have used \ref{binomialCoefficientMultiplicationLemma} and \ref{binomialCoefficientLemma}.

(2) We calculate
\begin{align*}
\sum_{k=m}^n \begin{pmatrix}
k \\ m
\end{pmatrix}b_{k,n}(x) &= \sum_{k=m}^n \begin{pmatrix}
k \\ m
\end{pmatrix}\begin{pmatrix}
n \\ k
\end{pmatrix}x^k(1-x)^{n-k} \\
&= \begin{pmatrix}
n \\ m
\end{pmatrix}\sum_{k=m}^n \begin{pmatrix}
n-m \\ k-m
\end{pmatrix}x^k(1-x)^{n-k} \\
&= \begin{pmatrix}
n \\ m
\end{pmatrix}\sum_{k=0}^{n-m} \begin{pmatrix}
n-m \\ k
\end{pmatrix}x^mx^k(1-x)^{n-m-k} \\
&= \begin{pmatrix}
n \\ m
\end{pmatrix}x^m(x+1-x)^{n-m} = \begin{pmatrix}
n \\ m
\end{pmatrix}x^m,
\end{align*}
where we have used \ref{binomialCoefficientMultiplicationLemma} and \ref{binomialIdentity}.
\end{proof}
\begin{corollary}
Let $n\in\N$. Then $\{b_{k,n}\}_{k\leq n}(x)$ is a basis of $\R[x]/(x^{n+1})$.
\end{corollary}

\begin{proposition}
Let $f\in\cont(\interval{0,1})$ and consider $B_n: \cont(\interval{0,1}) \to \cont(\interval{0,1})$ defined by
\[ B_n(f): \interval{0,1} \to \R: x\mapsto \sum_{k=0}^nf\Big(\frac{k}{n}\Big)b_{k,n}(x). \]
Then $B_n(f) \to f$ uniformly.
\end{proposition}
\begin{proof}
We aim to apply the Bohman-Korovkin theorem \ref{BohmanKorovkin}. To this end we calculate
\[ B_n\big(\underline{1}\big)(x) = \sum_{k=0}^nb_{k,n}(x) = 1, \]
by \ref{BernsteinPartitionUnity}. Next
\begin{align*}
B_n\big(\id_{\interval{0,1}}\big)(x) &= \sum_{k=0}^n \frac{k}{n}b_{k,n}(x) \\
&= \sum_{k=0}^n \frac{k}{n} \begin{pmatrix}
n \\ k
\end{pmatrix}x^k(1-x)^{n-k} \\
&= 0 + \sum_{k=1}^{n} \frac{k}{n}\begin{pmatrix}
n \\ k
\end{pmatrix}x^k(1-x)^{n-k} \\
&= \sum_{k=1}^{n} \begin{pmatrix}
n-1 \\ k-1
\end{pmatrix}x^k(1-x)^{n-k} \\
&= \sum_{k=0}^{n-1} \begin{pmatrix}
n-1 \\ k
\end{pmatrix}xx^{k}(1-x)^{n-1-k} \\
&= x (x + 1 - x)^{n-1} = x = \id_{\interval{0,1}}(x),
\end{align*}
by \ref{binomialCoefficientLemma} and \ref{binomialIdentity} for all $n\in \N$. Thus $B_n\big(\id_{\interval{0,1}}\big)$ converges uniformly to $\id_{\interval{0,1}}$.

Finally we need to calculate $B_n\big((-)^2\big)(x) = \sum_{k=0}^n \frac{k^2}{n^2}b_{k,n}(x)$. Now applying multiple times gives
\[ k^2 \begin{pmatrix}
n \\ k
\end{pmatrix} = \big(k(k-1) + k\big)\begin{pmatrix}
n \\ k
\end{pmatrix} = n(n-1) \begin{pmatrix}
n - 2 \\ k - 2
\end{pmatrix} + n \begin{pmatrix}
n-1 \\ k-1
\end{pmatrix}. \]
The second term, like before, gives a contribution of $\frac{x}{n}$. The first term gives a contribution of $\frac{n(n-1)}{n^2}x^2$. Putting everything together gives $B_n\big((-)^2\big)(x) = x^2 - n^{-1}(x^2-x) = \big((-)^2 - n^{-1}(x\mapsto x^2 - x)\big)(x)$. Since uniform convergence is a vector convergence, this converges uniformly to $(-)^2$. Thus we can apply the Bohman-Korovkin theorem.
\end{proof}
\begin{corollary}[Weierstrass approximation theorem]
The polynomials are dense in $\cont(\interval{0,1})$ w.r.t.\ the uniform norm.
\end{corollary}

\subsubsection{Kantorovich polynomials}
\begin{definition}
Take $n\in \N$ and $f\in L^p(\interval{0,1})$. Then the \udef{Kantorovich polynomial} $K_n(f)(x)$ is defined by
\[ K_n(f)(x) \defeq \sum_{k=0}^n\left((n+1)\int_{\frac{k}{n+1}}^{\frac{k+1}{n+1}}f(t)\diff{t}\right)b_{k,n}(x). \]
\end{definition}

\begin{proposition}
Let $f\in L^p(\interval{0,1})$ and $1\leq p <\infty$. Then $K_n(f) \to f$ in $L^p(\interval{0,1})$.
\end{proposition}
\begin{proof}
TODO, see Korovkin-type Theorems and Approximation by Positive Linear Operators - Altomare.
\end{proof}



\chapter{Calculus}
\section{Exploring the concept of change}
TODO: diffeomorphism

In physics how things change is quite important. Much of physics is concerned with the question of, given a particular system at a particular time, how that system will evolve.

We have not yet really introduced a mathematical construct that expresses an idea of change. We will do so here.

In particular we will consider ways to express how the output of a function changes if we (slightly) change its input.

To motivate the discussion below, consider the function represented by the graph in figure TODO. 

Locally at any one point the rate of change of the function can be described using the slope at that point. That makes intuitive sense; when walking up a mountain the slope is a measure for how quickly the altitude changes.

The slope between two points can be calculated by dividing the vertical distance by the horizontal distance. This definition of slope obviously depends on two points. We would quite like to be able to talk about the slope at a single point (the way we would intuitively when walking up a hill). To do that we can just bring both points very close together.

As can be seen on the picture this procedure gives the slope of the tangent line at that point (straight lines have a constant slope).

\subsection{Speed}

At this point we can give an important physical motivating example, namely the speed of an object. Say we throw an apple straight up into the air. Its vertical movement is plotted in figure TODO.

We may want to know its speed at different times. We can calculate speed by taking the displacement and dividing it by the time it takes traverse that distance. We can now make an important distinction between average speed (the slope between two distinct points) and instantaneous speed (the limit when we bring both points together).

\section{The derivative}
As motivated above, the rate of change of a (real) function is the difference in output divided by the difference in input of two points:
\[ \frac{f(y) - f(x)}{y-x}. \]
We conventionally call $h = y-x$. We can then write the above quantity (which is called the \udef{Newton quotient}) as
\[ \frac{f(x+h) - f(x)}{x+h-x} = \frac{f(x+h) - f(x)}{h}. \]
The \udef{derivative} of $f$ at $x$ is then just the limit of the Newton quotient with $h$ going to zero.

This limit does not always exist. If the limit exists for all $x$, the function is called \udef{differentiable}. A function may also be differentiable in some points and not in others.

We can now use the definition and properties of limits to calculate derivatives, such as in the following example. This process is slow and laborious even for relatively simple functions. Luckily the derivative has some important properties that lets us calculate the derivative of many functions with relative ease.

We can define a (real) function that, for any input, calculates the derivative of a particular fixed function $f$ at that point and gives that as its output. This new function is often called the derivative of the function $f$. 

There are many ways to write the derivative of $f$:
\[ \lim_{h\to 0} \frac{f(x+h)-f(h)}{h} \equiv f'(x) \equiv \od{f}{x} \equiv \od{f(x)}{x} \]
A mathematician would want me to emphasize that the expression $\od{f}{x}$ should be read as a whole and is technically \emph{not} a division, but a physicist would say that (in some situations) it can be viewed as such, where $\div{f}$ and $\div{x}$ are (the in this context relevant) infinitesimal variations of $f$ and $x$. Do not tell any mathematicians I said this.

\begin{example}
TODO derivative of polynomial function using limits.
\end{example}

In certain situations a dot is used to indicate a derivative with respect to time (i.e.\ the derivative of a quantity in function of time). So we might for example use $x(t)$ to denote the position in function of time (here $x$ is \emph{not} used to refer to a variable but to a function, the notation is standard and usually it clear from the context what $x$ refers to). We can then use the notation
\[ x'(t) \equiv \dot{x}(t). \]
In fact $\dot{x}(t)$, is just the speed.

When using the notation $\od{f}{x}$, this usually refers to the function that is the derivative of $f$. If we want to evaluate this function in a particular point (say $x_0$), we can write something like this
\[ \left.\od{f}{x}\right|_{x=x_0}. \]

\subsection{Slope of a curve}
\[ \text{slope of the normal} = \frac{-1}{\text{slope of the tangent}} \]

\subsection{Properties of the derivative}
Here we give some properties of the derivative:
\begin{itemize}
\item The derivative is a linear operation:
\[ (f+g)'(x) = f'(x) + g'(x) \]
and
\[ (c\cdot f)'(x) = c\cdot f'(x) \qquad \forall c \in \R \]
\item \ueig{Product rule}
\[ (f\cdot g)'(x) = f(x)\cdot g'(x) + f'(x)g(x) \]
\item The derivative of $\frac{1}{f(x)}$, assuming $f(x) \neq 0$:
\[ \left(\frac{1}{f(x)}\right)' = \frac{f'(x)}{f(x)^2}. \]
\item Combining the previous two properties, we get the quotient rule (assuming $g(x)\neq 0$)
\[ \left(\frac{f(x)}{g(x)}\right)' = \frac{g(x)f'(x) - f(x)g'(x)}{g(x)^2}. \]
\item Finally we have the very important \ueig{chain rule}. This tells us how to take the derivative of composite functions:
\[ (f \circ g)'(x) = f'(g(x))g'(x). \]
We can also write this as
\[ \od{f(g(x))}{x} = \od{f}{g}\od{g}{x}. \]
TODO example
\item Derivative of an inverse
\[ \od{f^{-1}(x)}{x} = \frac{1}{f'(f^{-1}(x))} \]
\end{itemize}

TODO Faà di Bruno

\subsection{Derivatives of some common functions}
Using the results below together with the properties above we can calculate the derivative of a large number of functions.

\begin{itemize}
\item Let $n$ be an integer larger than or equal to $1$ and let $f(x) = x^n$. Then
\[ f'(x) = n x^{n-1}. \]
Using this result together with the property of linearity, we can easily calculate the derivative of any polynomial function. TODO: general exponent
\item The derivatives of the trigonometric functions can be derived from
\[ \sin'(x) = \cos \qquad \text{and} \qquad \cos'(x) = -\sin(x) \]
TODO: list
\item TODO cyclometric
\item TODO hyperbolic
\end{itemize}

\subsubsection{The exponential and logarithm}
Define natural logarithm $\ln$ and \textit{the} exponential function $\exp$.
\[ \od{\ln x}{x} = \frac{1}{x} \]
\[ a^x = e^{x\ln a} \qquad (a>0, x\in \R) \]

\[ e^x = \lim_{n\to \infty}\left(1+\frac{x}{n}\right)^n \]
and growth.

\subsection{Applications of differentiation}
\subsubsection{Extreme values}
Link increasing, decreasing and derivatives. + derivative zero everywhere = constant.
critical points. singulat points. concavity and inflections
\subsubsection{Rolle's lemma}
\subsubsection{Mean-value theorem}
\subsubsection{L'Hôpital's rules}


\subsection{Higher order derivatives}
When we take the derivative of a function, we we get a new function. We can now take the derivative of this new function. This is called taking the second order derivative. This process can be repeated for as long as the derivatives exist. We write the $n$-th order derivative as
\[ f^{(n)}(x) = \od[n]{f}{n}. \]
So for example $f''(x) = f^{(2)}(x)$.

\subsection{Implicit differentiation}

\subsection{Partial derivatives}
\subsubsection{Definition}
\subsubsection{Geometric interpretation}

\subsection{Meaning of the differential $\div{}$}
TODO conventional use + examples with nabla

TODO: put series here!

\subsection{Generalisations and types of derivatives}
TODO: Liebnitz rule!!! + linear.

\section{Integration}
TODO intuition, solving strategies, solving intelligently 
SEE: The electric field (first write all quantities, then )

\subsection{Areas as limits of sums}
\subsubsection{Sums and sigma notation}
\subsubsection{Trapezoid rule}
\subsubsection{Midpoint rule}
\subsubsection{Simpson's rule}

\subsection{The definite integral}
\subsection{Computing different areas and volumes}
\subsubsection{Rotation bodies}
\subsubsection{Surface bounded by function of polar coordinate $\theta$}
\[ \frac{1}{2}\int_{\theta_1}^{\theta_2}[f(\theta)]^2\div{\theta} \]

\subsection{The fundamental theorem of calculus}
\subsubsection{Indefinite integrals}
anti-derivative $+C$
\subsubsection{Some elementary integrals}

\subsection{Properties of integrals}
\subsubsection{Linearity}
\subsubsection{Mean-value theorem}
\subsubsection{Integrals of piece-wise continuous functions}

\subsection{Techniques of integration}
\subsubsection{Integrals of rational functions}
\subsubsection{Substitutions}
+ inverse substitutions
\subsubsection{Integration by parts}


\subsection{Improper integrals}


\subsection{Different types of integrals}
\subsubsection{Riemann}
\subsubsection{Lebesgue}
\subsubsection{Stieltjes}
\subsubsection{Cauchy}

\subsection{From infinite sum to integral}
Using measure

\section{Complex analysis}
holomorphic functions, residue theorem

\subsection{Complex integration and analyticity}
\subsection{Laurent series and isolated singularities}
\subsection{Residue calculus}
\subsection{Conformal mapping}


TODO
Solving intelligently (later using physics): Green functions, method of mirrors (+ cfr. general section on equations)
charge distributions

separation of variables (Legendre polynomials)

going from discrete sum to integral (also opposite with dirac delta). volume int using $\mathcal{V}$ and surface $\mathcal{S}$

surface int goes to zero at infinity.

\section{Dirac delta}
\subsection{In one dimension}
\subsection{In three dimensions}
\subsection{Properties}

\begin{eigenschap}
Composition of the Dirac $\delta$ with a smooth, continuously differentiable function $g$ follows from the following relation
\[ \int_\R \delta(g(x))f(g(x))|g'(x)|\div{x} = \int_{g(\R)} \delta(u)f(u)\div{u} \]
Thus we say that
\[ \delta(g(x)) = \sum_i \frac{\delta(x-x_i)}{|g'(x_i)|}\]
Where $x_i$ are the simple roots of $g$.
\end{eigenschap}

\section{Silly integrals}
\[ \int x^{\diff x}-1 = x\ln(x) - x +c \]

