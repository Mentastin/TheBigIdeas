\section{Exponentiation}
Exponentiation is usually introduced as repeated multiplication, e.g. $3^5 = 3\times 3\times 3 \times 3 \times 3 = 243$.  We call the number being repeatedly multiplied the \udef{base} ($3$ in this case) and the number of times it is multiplied ($5$ in this case) we call the \udef{exponent} or \udef{power}. Taking a real number as the base is not too difficult to do: $3.2564^3 = 3.2564\times 3.2564\times 3.2564 = 34.531324622144\ldots$

Now what happens if we put a real number in the exponent? What is $3^{2.5}$? Because $2.5$ is a rational number, we can still solve this using standard exponentiation and a square root ($3^{2.5} = \sqrt{3^5}$). Something like $3^\pi$ gets even more tricky.

We resolve this situation if the following way: TODO

\subsection{The square of a number}
TODO
!! square of real nonnegative

\subsection{$n^\text{th}$ roots}
TODO in particular of one.


\subsection{Exponential functions}
Corresponds to power for rational numbers.

Assume $a>0$ and $b>0$, and $x$ and $y$ are any real numbers, then the exponential function has the following properties:
\begin{enumerate}
\item $a^0 = 1$
\item $a^{x+y} = a^x a^y$
\item $a^{-x} = \frac{1}{a^x}$
\item $(a^x)^y = a^{xy}$
\item $(ab)^x = a^x b^x$
\end{enumerate}
Limits:
\begin{enumerate}
\item If $a>1$, then $\lim_{x\to -\infty} a^x = 0$ and $ \lim_{x\to \infty} a^x = \infty$
\item If $0 < a < 1$, then $\lim_{x\to -\infty} a^x = \infty$ and $ \lim_{x\to \infty} a^x = 0$
\end{enumerate}

\section{Logarithms}
The logarithm, denoted
\[ \log_a: [0,\infty[ \to [-\infty, \infty] \]
is defined as the inverse of the exponential function with base $a$. Thus
\[ \log_a(a^x) = x \quad \forall x\in\R \qquad \text{and} \qquad a^{\log_a(x)}=x \forall x>0 \]
If $x>0, y>0, a>0, b>0$ and $a \neq 1, b\neq 1$, then
\begin{enumerate}
\item $\log_a 1 = 0$
\item $\log_a(xy) = \log_a x + \log_a y$
\item $\log_a(\frac{1}{x}) = -\log_a x$
\item $\log_a(x^y) = y\log_a x$
\item $\log_a x = \frac{\log_b x}{\log_b a}$
\end{enumerate}


\section{Polynomial and rational functions}
The \udef{polynomial functions} are a very important class of functions. They can be specified by equations of the form
\[ f(x) = \sum_{i=0}^n a_i x^i \]
where the numbers $a_i$ specified by the index $i$ are called the \udef{coefficients} corresponding to the $i^\text{th}$ power of $x$. We can assume that $a_n$ is not zero (if it is we reduce $n$ until it isn't). That way $n$ gives the \udef{degree} of the polynomial expression.

Polynomial functions with a degree of one are called \udef{linear}.

TODO def rational functions

\subsection{Linear functions}
\subsection{Quadratic functions}
focus, directrix, vertex, axis

root formula

\subsection{Fundamental theorem of algebra}

\section{Absolute value}
TODO + standard definition of distance