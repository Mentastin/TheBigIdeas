TODO: Tusi couple

In this section we will give a practical rundown of some of the classic results of Euclidean geometry, especially those results that have elementary proofs not needing more involved machinery. We will focus on practical things like calculating angles, surface areas and volumes. A more general discussion will follow in the section about spaces.

\section{Flat shapes}
Pick theorem

\subsection{Circles} are shapes consisting of all the points at a certain fixed distance from a central point. We call this distance the radius, denoted $r$. 
We can then calculate the circumference
\[ C_\circ = 2\pi r \]
and the surface area
\[ A_\circ = \pi r^2. \]
\begin{definition}
Chord. Diameter.
\end{definition}

\subsection{Rectangles}
Rectangles are made up of four lines which intersect at right angles. Rectangles have a length $l$ and a width $w$. The surface area is
\[A_\square = lw.\]
Squares are rectangles with the same length and width.

\subsection{Triangles}
Triangles are shapes with three sides. To calculate the surface area we choose one side to be our base $b$. The shortest possible distance from the line that extends this side to the point not on this line is called the height $h$. The surface area is then
\[ A_\triangle = \frac{1}{2}bh. \]
TODO picture of proof: box around triangle. Draw line from point to base that splits box in two. In each half the triangle takes up half the area.

\begin{proposition}
A triangle, inscribed in a circle, consisting of two chords and a diameter always has a right angle.
\end{proposition}
\begin{proof}
Rotate the triangle 180\textdegree.
\end{proof}

\section{Solids}
\subsection{Spheres} are shapes consisting of all the points at a certain fixed distance from a central point. We call this distance the radius, denoted $r$. The surface area is
\[ S_\text{sphere} = 4\pi r^2 \]
and the volume is
\[V_\text{sphere} = \frac{4}{3}\pi r^3.\]

\subsection{Prisms} are extrusions of a polygonal base, not necessarily in a direction orthogonal to the plane of the base. If $A$ is the surface area of the base, the volume of the prism is
\[V_\text{prism} = Ah. \]

\subsection{Cylinders} are like prisms, but with plane curves instead of polygons as their base. Again, if $A$ is the surface area of the base, the volume is
\[ V_\text{cylinder} = Ah. \]

\subsection{Triangular pyramids} have a volume
\[ V_\text{pyramid} = \frac{1}{3}Ah \]

\section{Angles}
We often use greek letters like $\alpha, \beta$ or $\gamma$ to denote angles. For quantifying angles we can use degrees. Or we can identify each direction with a point on a circle of radius $1$ and use the distance along the edge of the edge of the circle to represent the angle. We call that distance the angle in \udef{radians}. In that case $360$ degrees (or $360^\circ$) is the full circumference, or $2\pi$ radians (also written as $2\pi$ rad or just $2\pi$); $180^\circ$ is half that, or $\pi$.

So we can convert any angle in degrees to radians by dividing by $360^\circ$ and multiplying by $2\pi$. To convert the other way we divide by $2\pi$ and multiply by $^\circ$.

Radians are often useful to work with because the arc length $L$ of a circular arc with radius $r$ and subtending an angle $\theta$ (measured in radians), is
\[ L = \theta r.\] 

We will usually use radians, and it should be assumed that all angles are expressed in radians, unless degrees or other units are explicitly stated.

\section{Some classic results and theorems}
TODO similar triangles, inner angles

Viviani's theorem

\section{Trigonometry}
TODO
\subsection{Defining sine and cosine}
With triangle, but not all numbers, so circle.

Domain and image

The squaring notation $\sin^2\theta$

Table of angles

\subsubsection{Some useful identities}
\begin{itemize}
\item \textbf{Pythagorean identity}
\[ \cos^2 \theta + \sin^2\theta = 1 \]
Corollary: $\sin^2\alpha - \sin^2\beta = \cos^2\beta - \cos^2\alpha$.
\item \textbf{Periodicity}
\[ \begin{cases}
\cos(\theta + 2\pi) = \cos\theta \\
\sin(\theta + 2\pi) = \sin\theta
\end{cases} \]
\item Cosine is \textbf{even}, sine is \textbf{odd}
\[ \begin{cases}
\cos(-\theta) = \cos\theta \\
\sin(-\theta) = -\sin\theta
\end{cases} \]
\item \textbf{Complementary angles}. Two angles are \udef{complementary} if their sum is $\pi/2$.
\[ \begin{cases}
\cos \left(\frac{\pi}{2} - \theta\right) = \sin\theta \\
\sin \left(\frac{\pi}{2} - \theta\right) = \cos\theta
\end{cases} \]
\item \textbf{Supplementary angles}. Two angles are \udef{supplementary} if their sum is $\pi$.
\[ \begin{cases}
\cos \left(\pi - \theta\right) = -\cos\theta \\
\sin \left(\pi - \theta\right) = \sin\theta
\end{cases} \]
\end{itemize}
TODO fig.
\subsubsection{Addition formulae}
\begin{align*}
\cos(\theta+\phi) &= \cos\theta\cos\phi - \sin\theta\sin\phi \\
\sin(\theta+\phi) &= \sin\theta\cos\phi + \cos\theta\sin\phi \\
\cos(\theta-\phi) &= \cos\theta\cos\phi + \sin\theta\sin\phi \\
\sin(\theta-\phi) &= \sin\theta\cos\phi - \cos\theta\sin\phi
\end{align*}
\subsubsection{Double- and half-angle formulae}
Double-angle
\begin{align*}
\sin 2\theta &= 2\sin\theta\cos\theta \\
\cos 2\theta &= \cos^2\theta - \sin^2\theta \\
&= 2\cos^2\theta - 1 \\
&= 1- 2\sin^2\theta
\end{align*}
Half-angle
\[ \cos^2\theta = \frac{1+\cos 2\theta}{2} \qquad \text{and}\qquad \sin^2 \theta = \frac{1 - \cos 2\theta}{2}. \]
\subsection{Other trigonometric functions}
tangent, cotangent, secant, cosecant (primary / secondary)

\subsection{Angles and sides in triangles.}
TODO figure vertices $A,B,C$ and sides $a,b,c$ opposite.
\subsubsection{Law of sines}
\[ \frac{\sin A}{a} = \frac{\sin B}{b} = \frac{\sin C}{c} \]
\subsubsection{Law of cosines}
\begin{align*}
a^2 &= b^2 + c^2 - 2bc\cos A \\
b^2 &= a^2 + c^2 - 2ac\cos B \\
c^2 &= a^2 + b^2 - 2ab\cos C
\end{align*}

\subsection{Waves}
\subsection{Plane waves}
frequency , wavelength, angular frequency, period

\subsection{The wave equation}

\subsection{Group and phase velocity}

\section{Cyclometric functions}
Name places me geographically.

Restrict domain to make bijective.
Both notations $\sin^{-1}$ and $\arcsin$

TODO + continuity of $\cos^{-1}$

\section{Hyperbolic functions}
Definition using $e$.
\[ \cosh x = \frac{e^x + e^{-x}}{2}, \qquad \sinh x = \frac{e^x - e^{-x}}{2} \]
Origin of name:
\[ \cosh^2 t - \sinh^2 t = 1 \]
(Proof:)
\begin{align*}
\cosh^2 t - \sinh^2 t &= \left(\frac{e^x + e^{-x}}{2}\right)^2 - \left(\frac{e^x - e^{-x}}{2}\right)^2 \\
&= \frac{1}{4}\left(e^{2t} + 2 + e^{-2t} - (e^{2t} - 2 + e^{-2t})\right) \\
&= \frac{2+2}{4} = 1
\end{align*}

Cosh is catenary curve.

Many properties similar to regular sine and cosine:
\begin{itemize}
\item $\cosh 0 =1$ and $\sinh 0= 0$;
\item The hyperbolic cosine is \ueig{even} ($\cosh(-x) = \cosh x$) and the hyperbolic sine is \ueig{odd} ($\sinh(-x) = \sinh x$).
\item Addition formulae (notice sign difference with $\cosh$):
\begin{align*}
\cosh(x+y) &= \cosh x\cosh y + \sinh x\sinh y \\
\sinh(x+y) &= \sinh x\cosh y + \cosh x\sinh y
\end{align*}
\item Double angle formulae (again sign difference for $\cosh$):
\begin{align*}
\sinh 2x &= 2\sinh x\cosh x \\
\cosh 2x &= \cosh^2 x + \sinh^2 x \\
&= 2\cosh^2 x - 1 \\
&= 1 + 2\sinh^2 x
\end{align*}
\end{itemize}
Other hyperbolic functions:
\begin{align*}
\tanh x &= \frac{\sinh x}{\cosh x} = \frac{e^x - e^{-x}}{e^x + e^{-x}} & \sech x &= \frac{1}{\cosh x} = \frac{2}{e^x + e^{-x}} \\
\coth x &= \frac{\cosh x}{\sinh x} = \frac{e^x + e^{-x}}{e^x - e^{-x}} & \csch x &= \frac{1}{\sinh x} = \frac{2}{e^x - e^{-x}}
\end{align*}
Inverse hyperbolic functions:
\begin{align*}
\sinh^{-1} x &= \log_e \left(x + \sqrt{x^2 + 1}\right) \\
\tanh^{-1} x &= \frac{1}{2}\log_{e} \left(\frac{1+x}{1-x}\right) \qquad (-1< x <1) \\
\cosh^{-1} x &= \log_e \left(x + \sqrt{x^2 - 1}\right) \qquad (x \geq 1)
\end{align*}
with restriction because $\cosh$ is not automatically bijective. ($\tanh$?)
