\chapter{$K$-theory for additive categories}
Let $\cat{C}$ be an additive category. Consider the isomorphism classes $[E]$ of objects $E$ in $\cat{C}$ with an addition operation given by
\[ [E]+[F] = [E\oplus F] \qquad E,F\in \cat{C}. \]
\begin{lemma}
The isomorphism classes of $\cat{C}$ form an abelian monoid $M(\cat{C})$ under this addition operation:
\[ E\oplus (F\oplus G) \cong (E\oplus F)\oplus G, \qquad  E\oplus F \cong F\oplus E, \qquad \text{and}\qquad E\oplus 0 \cong E. \]
\end{lemma}
Note that it is necessary to use isomorphism classes, because in general
\[ E\oplus (F\oplus G) \neq (E\oplus F)\oplus G \qquad \text{even though} \qquad E\oplus (F\oplus G) \cong (E\oplus F)\oplus G. \]
\begin{definition}
The \udef{$K$ functor} is in this case just the Grothendieck functor $G$ applied after making the category a monoid:
\[ K(-) \defeq G(M(-)). \]
\end{definition}


\chapter{Topological $K$-theory}
\section{The group $K(X)$}
\begin{definition}
Let $X$ be a compact topological space. The category $\cat{Vect}(X)$ of vector bundles over $X$ with the direct sum is an additive category. We define the $K$ group of $X$ as
\[ K(X) \defeq K(\cat{Vect}(X)). \] 
\end{definition}
Let $\mathcal{E}_n$ be the trivial bundle of rank $n$ over a compact space $X$.
\begin{proposition}
Every element $x$ of $K(X)$ can be written as $[E]-[\mathcal{E}_n]$ for some $n$ and some vector bundle $E$ over $X$.

Moreover, $[E]-[\mathcal{E}_p]=[F]-[\mathcal{E}_q]$ \textup{if and only if} there exists an integer $n$ such that $E\oplus\mathcal{E}_{q+n}\cong F\oplus\mathcal{E}_{p+n}$.
\end{proposition}
\begin{proof}
Immediate from Grothendieck construction and the fact that for all vector bundles $E$ there exists a vector bundle $F$ such that $E\oplus F \cong \mathcal{E}_n$.
\end{proof}
\begin{corollary}
Let $E,F$ be vector bundles over $X$. Then $[E]=[F]$ in $K(X)$ \textup{if and only if} $E\oplus \mathcal{E}_n \cong F\oplus \mathcal{E}_n$ for some $n$.
\end{corollary}

\begin{proposition}
For topological spaces $K$ is a contravariant functor on the category of compact spaces.
\end{proposition}

\section{The group $\widetilde{K}(X)$ for pointed spaces}
\begin{definition}
Let $(X, x_0)$ be a pointed compact space. The projection $\pi: X\to \{x_0\}$ induces a homomorphism $K(\{x_0\})\cong \Z\to K(X)$. The \udef{reduced $K$-theory} $\widetilde{K}(X)$ of $X$ is the cokernel of this homomorphism:
\[ \begin{tikzcd}
0 \rar & \Z \rar & K(X) \rar & \widetilde{K}(X) \rar & 0.
\end{tikzcd} \]
\end{definition}

\begin{lemma}
There is a canonical splitting generated by the inclusion $\{x_0\}\hookrightarrow X$ so that
\[ K(X) \cong \Z\oplus \widetilde{K}(X) \qquad \text{and} \qquad \widetilde{K}(X) \cong\ker(K(\{x_0\}\hookrightarrow X)). \]
\end{lemma}

\begin{proposition}
The composition $\gamma: M(\cat{Vect}(X))\to K(X) \to \widetilde{K}(X)$ is a surjective homomorphism.
Moreover, $\gamma([E])=\gamma([F])$ \textup{if and only if} $E\oplus \mathcal{E}_p \cong F\oplus \mathcal{E}_q$ for some $p,q$.
\end{proposition}
This gives a more direct definition of $\widetilde{K}(X)$ as the quotient of $M(\cat{Vect}(X))$ by the equivalence relation
\[ [E] \sim [F] \quad\iff\quad \exists p,q\in \N:\; E\oplus\mathcal{E}_p \cong F\oplus \mathcal{E}_q.  \]


\section{The relative $K$-group $K(X,Y)$}
\begin{definition}
Let $X$ be a compact topological space and $Y$ a closed subspace. Consider the triples $(E,F,\alpha)$ where $E,F$ are vector bundles over $X$ and $\alpha$ is an isomorphism $E_Y \to F_Y$ where $E_Y$ and $F_Y$ are the vector bundles $E,F$ restricted to $Y$.

We define the sum of two triples to be
\[ (E,F,\alpha) + (E',F',\alpha') = (E\oplus E',F\oplus F',\alpha\oplus\alpha'). \]

We call two triples $(E,F,\alpha), (E',F',\alpha')$ isomorphic if there exist isomorphisms $f:E\to E'$ and $g:F\to F'$ such that the diagram
\[ \begin{tikzcd}
E|_Y \rar{\alpha} \dar[swap]{f|_Y} & F|_Y \dar{g|_Y} \\
E'|_Y \rar{\alpha'} & F'|_Y
\end{tikzcd} \qquad \text{commutes.}\]

We consider the equivalence relation of ``stable isomorphism'' on these triples, that is two triples $(E,F,\alpha), (E',F',\alpha')$ are equivalent if and only if there exist triples $(G,G,I_{G_Y})$ and $(G',G',I_{G'_Y})$ such that
\[ \begin{cases}
(E,F,\alpha)+(G,G,I_{G_Y}) = (E\oplus G,F\oplus G, \alpha\oplus I_{G_Y}) & \text{and} \\ (E',F',\alpha')+(G',G',I_{G'_Y}) = (E'\oplus G',F'\oplus G', \alpha'\oplus I_{G'_Y})
\end{cases} \]
are isomorphic.

Then $K(X,Y)$ is the set of equivalence classes of such triples. We denote the equivalence class of a triple $(E,F,\alpha)$ by $d(E,F,\alpha)$.
\end{definition}

\begin{proposition}
Let $X$ be a compact space and $Y$ a closed subspace. Then
\begin{enumerate}
\item $K(X,Y)$ is an abelian group with as neutral element
\[ 0 = d(G,G,I_{G_Y}) \]
and
\[ d(E,F,\alpha) + d(F,E,\alpha^{-1}) = 0; \]
\item $K(X) \cong K(X,\emptyset)$;
\item $d(E,F,\alpha)+d(F,G,\beta) = d(E,G,\beta\circ \alpha)$.
\end{enumerate}
\end{proposition}
\begin{proof}

\end{proof}

\begin{proposition}
Let $i$ be the homomorphism
\[ i:K(X,Y)\to K(X): d(E,F\alpha) \mapsto [E]-[F] \]
and $j$ the homomorphism
\[ i:K(X)\to K(Y): [E]-[F] \mapsto [E|_Y]-[F|_Y]. \]
Then we have the exact sequence
\[ \begin{tikzcd}
K(X,Y) \rar{i} & K(X) \rar{j} & K(Y).
\end{tikzcd} \]
Moreover, if $Y$ is a retract of $X$ (i.e.\ the inclusion $Y\hookrightarrow X$ admits a left-inverse), then we have the split exact sequence
\[ \begin{tikzcd}
0 \rar & K(X,Y) \rar & K(X) \rar & K(Y) \rar & 0.
\end{tikzcd} \]
\end{proposition}
\begin{corollary}
Let $(X,x_0)$ be a pointed space. Then $\{x_0\}$ is a retract of $X$ and thus
\[ K(X, \{x_0\}) \cong \ker(K(X)\to K(\{x_0\})) \cong \widetilde{K}(X) \]
\end{corollary}

\begin{proposition}
The projection $\pi: X\to X/Y$ induces an isomorphism $K(X/Y,\{y\}) \to K(X,Y)$.
\end{proposition}

\begin{proposition}
Let $Y$ be a closed subspace of a compact space $X$. Then we have the 
exact sequence 
\[ \begin{tikzcd}
\widetilde{K}(X/Y) \rar & \widetilde{K}(X) \rar & \widetilde{K}(Y).
\end{tikzcd} \]
\end{proposition}

\begin{theorem}[Atiyah-Jänich]
Let $X$ be a compact Hausdorff space and $H$ a Hilbert space. Then
\[ [X,\Fred(H)] \cong K(X). \]
\end{theorem}

\section{Clifford modules and the functor $K^{p,q}$}
\begin{proposition}
Let $A,B$ be $\R$-algebras. 
\[ (\cat{C}^A)^B \cong \cat{C}^{A\otimes_\R B} \]
\[ \cat{C}^{A\oplus B}\simeq \cat{C}^A\times \cat{C}^B. \]
\end{proposition}

\begin{definition}
Let $\cat{Vect}(X)^{p,q}$ be the category of $\Cl^{p,q}$-modules $W$ such that $W$ is a vector bundle over $X$.

We define $K^{p,q}(X)$ as the Grothendieck group of the functor
\[ \cat{Vect}(X)^{p,q+1}\to \cat{Vect}(X)^{p,q}  \]
\end{definition}

\begin{theorem}
Let $X$ be a compact space. Then $K^{0,0}$ and $K(0,1)$ are canonically isomorphic to $K(X)$ and $K^{-1}(X)$.
\end{theorem}

\subsection{Description via gradings}
\begin{definition}
Let $E\in \cat{Vect}(X)^{p,q}$. A \udef{grading} of $E$ is an endomorphism $\eta$ of $E$ regarded as an object of $\cat{Vect}(X)$ such that
\begin{enumerate}
\item $\eta^2 = I$;
\item $\eta\rho(e_i) = -\rho(e_i)\eta$.
\end{enumerate}
Equivalently, a grading on $E$ is a $\Cl^{p,q+1}$-structure on $E$ extending the $\Cl^{p,q}$-structure where $\eta = \rho(e_{p+q+1})$.
\end{definition}

\chapter{$K$-theory for $C^*$-algebras}
\section{Homotopy equivalence of unitaries}
TODO ref on homotopy + $\sim_h$.

\begin{lemma} \label{productHomotopy}
If $u_1 \sim_h v_1$ and $u_2 \sim_h v_2$, then $ u_1u_2\sim_h v_1v_2$.
\end{lemma}

\begin{definition}
Let $A$ be a unital $C^*$-algebra. We let $\mathcal{U}_0(A)\subseteq \mathcal{U}(A)$ denote the set of all unitaries homotopic with $\vec{1}$ in $\mathcal{U}(A)$.
\end{definition}

\begin{lemma} \label{homotopyOfUnitaries}
Let $A$ be a unital $C^*$-algebra.
\begin{enumerate}
\item For each self-adjoint element $h\in A$, $\exp(ih)\in\mathcal{U}_0(A)$.
\item If $u\in \mathcal{U}(A)$ with $\sigma(u) \neq \mathbb{T}$, then $u\in\mathcal{U}_0(A)$.
\item If $u,v\in\mathcal{U}(A)$ with $\norm{u-v}<2$, then $u\sim_h v$.
\end{enumerate}
\end{lemma}
\begin{proof} \hspace{1em}
\begin{enumerate}
\item By spectral mapping, \ref{spectralMappingCFC}, and \ref{propertiesFromSpectrum} we see that $\exp(ih)$ is unitary. The homotopy is given by $t\mapsto \exp(ith)$.
\item If $\sigma(u) \neq \mathbb{T}$, then for some real $\theta$, $\exp(i\theta)\notin \sigma(u)$. This means the exponential has a well defined inverse $f: \sigma(u) \to \interval[o]{\theta, \theta+2\pi}: \exp(it)\mapsto t$. By spectral mapping, \ref{spectralMappingCFC}, and \ref{propertiesFromSpectrum} we see that $f(u)$ is self-adjoint. It follows that $u = \exp(if(u))$, so we can conclude using (1).
\item Assume $\norm{u-v}<2$. Then
\[ 2 > \norm{u-v} = \norm{v^*}\norm{u-v} \geq \norm{v^*u - 1} \]
so $-2\notin \sigma(v^*u-1)$ and $-1 \notin \sigma(v^*u)$ by spectral mapping. By (2) $v^*u \sim_h \vec{1}$ and hence $u\sim_h v$ be \ref{productHomotopy}.
\end{enumerate}
\end{proof}

\begin{corollary}
The unitary group $\U(\C^{n\times n})$ is connected for all $n\in\N$.
\end{corollary}
\begin{proof}
Every element in $\C^{n\times n}$ has finite spectrum (the eigenvalues). So we conclude by (2) of \ref{homotopyOfUnitaries}.
\end{proof}
Because $\norm{u-v}\leq \norm{u}+\norm{v} = 2$ for all $u,v\in \Unitaries(A)$, two unitaries are only not homotopic if they lie at a distance of exactly 2.

TODO generalise to $\GL$:
\begin{lemma} \label{sectionConnectedToIdentity}
Let $A$ be a unital $C^*$-algebra. Then
\begin{enumerate}
\item $\mathcal{U}_0(A)$ is a normal subgroup of $\mathcal{U}(A)$;
\item $\mathcal{U}_0(A)$ is open and closed relative to $\mathcal{U}(A)$;
\item an element $u\in \mathcal{U}(A)$ belongs to $\mathcal{U}_0(A)$ \textup{if and only if} for some self-adjoint elements $h_1,\ldots, h_n \in A$
\[ u = \exp(ih_1)\cdot \ldots \cdot \exp(ih_n). \]
\end{enumerate}
\end{lemma}
\begin{proof}
TODO ref.
\begin{enumerate}
\item First note $\mathcal{U}_0(A)$ is closed under multiplication by \ref{productHomotopy}. Let $u_t$ be a continuous path from $\vec{1}$ to $u$. Then $u_t^{-1}$ and (for all $v\in \mathcal{U}(A)$) $v^*u_t v$ are continuous paths from $\vec{1}$ to $u^{-1}$ and $v^*uv$, respectively.
\item TODO.
\end{enumerate}
\end{proof}

\begin{lemma}[Whitehead]
Let $A$ be a unital $C^*$-algebra and $u,v\in\mathcal{U}(A)$. Then
\[ \begin{pmatrix}
u & 0 \\ 0 & v
\end{pmatrix} \sim_h \begin{pmatrix}
uv & 0 \\ 0 & \vec{1}
\end{pmatrix} \sim_h \begin{pmatrix}
vu & 0 \\ 0 & \vec{1}
\end{pmatrix} \sim_h \begin{pmatrix}
v & 0 \\ 0 & u
\end{pmatrix} \qquad \text{in}\;\mathcal{U}(A^{2\times 2}). \]
It follows in particular that
\[ \begin{pmatrix}
u & 0 \\ 0 & u^*
\end{pmatrix} \sim_h \begin{pmatrix}
\vec{1} & 0 \\ 0 & \vec{1}
\end{pmatrix}. \]
\end{lemma}
\begin{proof}
First
\[ \sigma_A \begin{pmatrix}
0 & \vec{1} \\ \vec{1} & 0
\end{pmatrix} = \sigma_\C \begin{pmatrix}
0 & 1 \\ 1 & 0
\end{pmatrix} = \{1\}, \qquad \text{so} \qquad \begin{pmatrix}
0 & \vec{1} \\ \vec{1} & 0
\end{pmatrix} \sim_h \begin{pmatrix}
\vec{1} & 0 \\ 0 & \vec{1}
\end{pmatrix} \]
by (2) of \ref{homotopyOfUnitaries}.
Hence
\[ \begin{pmatrix}
u & 0 \\ 0 & v
\end{pmatrix} = \begin{pmatrix}
u & 0 \\ 0 & \vec{1}
\end{pmatrix}\begin{pmatrix}
0 & \vec{1} \\ \vec{1} & 0
\end{pmatrix}\begin{pmatrix}
v & 0 \\ 0 & \vec{1}
\end{pmatrix}\begin{pmatrix}
0 & \vec{1} \\ \vec{1} & 0
\end{pmatrix} \sim_h \begin{pmatrix}
u & 0 \\ 0 & \vec{1}
\end{pmatrix}\begin{pmatrix}
v & 0 \\ 0 & \vec{1}
\end{pmatrix} = \begin{pmatrix}
uv & 0 \\ 0 & \vec{1}
\end{pmatrix}.
 \]
 The other claims follow in a similar way.
\end{proof}

\begin{lemma} \label{unitaryLifting}
Let $A,B$ be unital $C^*$-algebras and let $\Psi: A \to B$ be a surjective $*$-homomorphism. Then
\begin{enumerate}
\item $\Psi\left(\mathcal{U}_0(A)\right) = \mathcal{U}_0(B)$;
\item if $u\in\Unitaries(B)$, and if $u \sim_h \Psi(v)$ for some $v\in\Unitaries(A)$, then $u$ lifts to a unitary element in $A$.
\end{enumerate}
\end{lemma}
\begin{proof}
TODO
\end{proof}

\begin{proposition} \label{unitariesRetractionOfGL}
Let $A$ be a unital $C^*$-algebra and for all $a\in\GL(A)$, let $u(a)|a|$ be the polar decomposition of $a$. Then
\begin{enumerate}
\item $a\sim_h u(a)$ in $\GL(A)$;
\item for all $v_1,v_2\in\Unitaries(A)$
\[ v_1\sim_h v_2 \quad \text{in}\quad \GL(A) \qquad \iff \qquad v_1\sim_h v_2 \quad \text{in}\quad \Unitaries(A). \]
\end{enumerate}
\end{proposition}
\begin{proof}
TODO
\end{proof}
Thus the polar decomposition gives a deformation retract of $\GL(A)$ onto $\Unitaries(A)$.
TODO also for matrices!!


\section{Projections}
\begin{definition}
Let $A$ be a $C^*$-algebra. Two projections $p,q\in\Projections(A)$ are \udef{orthogonal} if $pq = 0$. We write $p \perp q$.
\end{definition}
TODO: $pq = 0$ iff $qp = 0$.

\begin{lemma}
Let $A$ be a $C^*$-algebra and $p,q\in\Projections(A)$. Then the following are equivalent:
\begin{enumerate}
\item $p+q \in \Projections(A)$;
\item $p$ and $q$ are orthogonal;
\item $p+q \leq 1$.
\end{enumerate}
\end{lemma}
\begin{proof}
TODO
\end{proof}
Every ``almost-idempotent'' can be approximated by a projection:
TODO



\subsection{Equivalence of projections}
\begin{definition}
Let $A$ be a $C^*$-algebra and $p,q\in A$. We write
\begin{enumerate}
\item $p\sim q$ if there exists $v\in A$ such that $p = v^*v$ and $q=vv^*$. This is \udef{(Murray-von Neumann) equivalence}.
\item $p \sim_u q$ if there exists $u\in \mathcal{U}(\tilde{A})$ such that $q = upu^*$. This is \udef{unitary equivalence}.
\end{enumerate}
\end{definition}
TODO: We can also take $A^\dagger$ ipv $\tilde{A}$.

\begin{lemma}
Both Murray-von Neumann equivalence and unitary equivalence are equivalence relations.
\end{lemma}
\begin{proof}
TODO transitivity.
\end{proof}

\begin{lemma}
Let $A$ be a unital $C^*$-algebra and $p,q\in\Projections(A)$. Then
\[ p \sim_u q \qquad \iff p \sim q \quad \text{and}\quad \vec{1} - p \sim \vec{1} - q. \]
\end{lemma}
\begin{proof}
Assume $p \sim_u q$, so we can write $q = upu^*$ for some $u\in\mathcal{U}(A)$. Put $v = up$ and $w = u(\vec{1}-p)$. Then
\begin{align*}
v^*v &= p^*u^*up = p, & vv^* &= upp^*u^* = upu^* = q \\
w^*w &= (\vec{1}-p)u^*u(\vec{1}-p) = \vec{1} - p & ww^* &= u(\vec{1}-p)(\vec{1}-p)u^* = u(\vec{1}-p)u^* = \vec{1} - q.
\end{align*}

Assume the converse. TODO
\end{proof}

\begin{lemma}
Let $p,q\in \Projections(A)$. If $\exists z\in \GL(\tilde{A})$ such that $q = zpz^{-1}$, then $p\sim_u q$.
\end{lemma}
\begin{proof}
We have $zp = qz$ and $pz^* = z^*q$, so $p$ commutes with $z^*z$:
\[ pz^*z = z^*qz = z^*zp. \]
Now put $u = z|z|^{-1}$ and calculate
\[ upu^* = z|z|^{-1}p|z|^{-1}z^* = zp|z|^{-2}z^* = qz(z^*z)^{-1}z^* = q. \]
TODO: clarify rules of calculation.
\end{proof}

\begin{proposition}
Let  $p,q\in \Projections(A)$. If $\norm{p-q}< 1$, then $p \sim_h q$.
\end{proposition}

\begin{proposition} \label{implicationsBeweenEquivalences}
Let $A$ be a $C^*$-algebra and $p,q\in\Projections(A)$. Then
\begin{enumerate}
\item if $p\sim_h q$, then $p \sim_u q$;
\item if $p\sim_u q$, then $p \sim q$;
\end{enumerate}and
\begin{enumerate}
\setcounter{enumi}{2}
\item if $p\sim q$, then $\begin{pmatrix}
p & 0 \\ 0 & 0
\end{pmatrix} \sim_u \begin{pmatrix}
q & 0 \\ 0 & 0
\end{pmatrix}$ in $A^{2\times 2}$;
\item if $p\sim_u q$, then $\begin{pmatrix}
p & 0 \\ 0 & 0
\end{pmatrix} \sim_h \begin{pmatrix}
q & 0 \\ 0 & 0
\end{pmatrix}$ in $A^{2\times 2}$.
\end{enumerate}
\end{proposition}
\begin{proof}
TODO
\end{proof}
\subsubsection{Decomposition into matrix algebras}
\begin{proposition}
Let $A$ be a unital $C^*$-algebra. Let $p_1,\ldots ,p_n$ be pairwise orthogonal and Murray-von Neumann equivalent projections for which $p_1 + \ldots + p_n = \vec{1}$. Then $A \cong (p_1Ap_1)^{n\times n}$.
\end{proposition}
\begin{proof}
TODO
\end{proof}

\subsection{Semigroups of projections}
\begin{definition}
Let $A$ be a $C^*$-algebra. We define $\Projections_\infty(A)$ as
\[ \Projections_n(A) = \Projections(A^{n\times n}) \qquad \Projections_\infty(A) = \bigcup_{n=1}^\infty \Projections_n(A) \]
and equip it with the binary operation $\oplus$:
\[ \forall p,q\in \Projections_\infty(A): \quad p\oplus q = \diag(p,q) = \begin{pmatrix}
p & 0 \\ 0 & q
\end{pmatrix}. \]
The involution on $\Projections_\infty(A)$ is the transposed pointwise application of $*$.
\end{definition}

\begin{lemma}
Let $A$ be a $C^*$-algebra.
\begin{enumerate}
\item If $p\in\Projections_n(A)$ and $q\in\Projections_m(A)$, then $p\oplus q\in\Projections_{n+m}(A)$.
\item The operation $\oplus$ is associative, making $\Projections_\infty(A)$ a semigroup.
\item If $p,q,r, p+q\in\Projections_\infty(A)$, then
\[ (p+q)\oplus r = p\oplus r + q\oplus r \qquad \text{and} \qquad r\oplus(p+q) = r\oplus p +r\oplus q. \]
\end{enumerate}
\end{lemma}
\begin{proof}
The first point follows from
\[ (p\oplus q)^* = \begin{pmatrix}
p^* & 0 \\ 0 & q^*
\end{pmatrix} = \begin{pmatrix}
p & 0 \\ 0 & q
\end{pmatrix} = p\oplus q \]
and
\[ (p\oplus q)^2 = \begin{pmatrix}
p & 0 \\ 0 & q
\end{pmatrix}\begin{pmatrix}
p & 0 \\ 0 & q
\end{pmatrix} = \begin{pmatrix}
p^2 & 0 \\ 0 & q^2
\end{pmatrix} = p\oplus q. \]
The others are easy.
\end{proof}

\begin{definition}
We define a relation $\sim_0$ on $\Projections_\infty(A)$ as follows: for $p\in \Projections_n(A)$ and $p\in \Projections_m(A)$,
\[ p \sim_0 q \defequiv \exists v\in A^{m\times n}: \quad p = v^*v\;\land \; q = vv^*. \]
\end{definition}
If $m=n$, then $\sim_0$ equivalence is Murray-von Neumann equivalence.
\begin{lemma}
The relation $\sim_0$ is an equivalence relation on $\Projections_\infty(A)$.
\end{lemma}

\begin{lemma} \label{sim0properties}
Let $A$ be a $C^*$-algebra and $p,q,r,p',q'\in \Projections_\infty(A)$. Then
\begin{enumerate}
\item $p\sim_0 p\oplus 0^{n\times n}$; in particular, $0 \sim_0 0^{n\times n}$;
\item if $p\sim_0 p'$ and $q\sim_0 q'$, then $p\oplus q \sim_0 p'\oplus q'$;
\item $p\oplus q \sim_0 q\oplus p$;
\item if $p,q\in\Projections_n(A)$ such that $pq = 0$, then $p+q\in\Projections_n(A)$ and $p+q \sim_0 p\oplus q$.
\end{enumerate}
\end{lemma}
\begin{proof}
TODO
\end{proof}

\begin{definition}
We set
\[ \mathcal{V}(A) = \Projections_\infty(A) / \sim_0 \]
and let $[p]_\mathcal{V}$ denote the equivalence class containing $p$. We define addition on $\mathcal{V}(A)$ by
\[ [p]_\mathcal{V} + [q]_\mathcal{V} = [p\oplus q]_\mathcal{V} \qquad \forall p,q\in\Projections_\infty(A). \]
\end{definition}
Clearly $\mathcal{V}(A)$ is a commutative monoid with identity $[0]_0$. The $\mathcal{V}$ comes from ``vector bundle''.

The addition $[p]_\mathcal{V} + [q]_\mathcal{V}$ is well-defined for all projections $p,q$. If $p\perp q$, then 
\[ [p]_\mathcal{V} + [q]_\mathcal{V} = [p+q]_\mathcal{V} \]
by \ref{sim0properties}. In general this does not work, which is essentially the reason we work with matrices in $\Projections_\infty(A)$, not just with projections.

\begin{lemma}
Then $\mathcal{V}(-): \cat{C^*alg} \to \cat{CMon}$ is a functor that sends morphisms $f: A \to B$ in $\cat{C^*alg}$,i.e.\ $*$-homomorphisms, to
\[  \mathcal{V}(f):\mathcal{V}(A) \to \mathcal{V}(B): [p]_\mathcal{V} \mapsto [f(p)]_\mathcal{V}. \]
\end{lemma}
\begin{proof}
We need to check the mapping of morphisms is well-defined. Then the functorial properties are immediate.

First we note that $*$-homomorphisms map projections to projections.

Let $[p]_\mathcal{V} = [q]_\mathcal{V}$. Then $p\sim_0 q$ and thus $\exists v\in A^{m\times n}$ such that $p=v^*v$ and $q = vv^*$. Thus
\[ f(p) = f(v^*v) = f(v)^*f(v) \qquad \text{and}\qquad f(q) = f(vv^*) = f(v)f(v)^*. \]
So $f(p) \sim_0 f(q)$ and thus $[f(p)]_\mathcal{V} = [f(q)]_\mathcal{V}$, meaning the mapping of morphisms is well-defined.
\end{proof}

\begin{definition}
Define a relation $\sim_s$ on $\Projections_\infty(A)$ as follows: for $p,q\in\Projections_\infty(A)$,
\[ p\sim_s q \defequiv \exists r\in\Projections_\infty:\quad p\oplus r \sim_0 q\oplus r. \]
The relation $\sim_s$ is called \udef{stable equivalence}.
\end{definition}

\begin{lemma} \label{stableEquivalence}
Let $A$ be unital. Then for all $p,q\in\Projections_\infty(A)$
\[ p\sim_s q \iff p\oplus \vec{1}_n \sim_0 q\oplus \vec{1}_n \]
for some integer $n$.
\end{lemma}
\begin{proof}
Assume $ p\sim_s q$, so $p\oplus r \sim_0 q\oplus r$ for some $r\in\Projections_n(A)$. By \ref{orthogonalProjection}, we know $(\vec{1}-r)$ is a projection. By the second point of \ref{sim0properties}, we have $p\oplus r \oplus (\vec{1}_n - r) \sim_0 q\oplus r \oplus (\vec{1}_n - r)$.

Now $r(\vec{1}_n - r) = r-r = 0$, so $r\oplus(\vec{1}_ - r) \sim_0 r\oplus(\vec{1}_n - r)$ by the fourth point of \ref{sim0properties}. By the second point
\[ p\oplus \vec{1}_n \sim_0 p\oplus r \oplus (\vec{1}_n - r) \sim_0 q\oplus r \oplus (\vec{1}_n - r) \sim_0 q\oplus \vec{1}_n. \]

The converse is immediate.
\end{proof}

\section{The $K_{00}$ functor}
\begin{definition}
Let $A$ be a $C^*$-algebra. Then we define the functor $K_{00}$ as the composition of two functors
\[ K_{00} = G \circ \mathcal{V}: \cat{C^*alg} \to \cat{Ab} \]
and let $[\cdot]_{0}$ be the mapping
\[ \begin{tikzcd}
\Projections_\infty(A) \rar["{[\cdot]_\mathcal{V}}"] & \mathcal{V}(A) \rar["g_0"] & K_{00}(A)
\end{tikzcd} \]
where $g_0$ is the Grothendieck map $x\mapsto (x,0)$.
\end{definition}
Note that $[\cdot]_{0}$ is not surjective and $g_0$ is not necessarily injective.

\begin{proposition}[The standard picture of $K_{00}$] \label{StandardPictureK00}
Let $A$ be a $C^*$-algebra, then
\begin{align*}
K_{00}(A) &= \setbuilder{[p]_0 - [q]_0}{p,q \in \Projections_\infty(A)} \\
&= \setbuilder{[p]_0 - [q]_0}{p,q \in \Projections_n(A), n\in \N}.
\end{align*}
Moreover,
\begin{enumerate}
\item $[p\oplus q]_0 = [p]_0+[q]_0$ for all projections $p,q\in\Projections_\infty(A)$;
\item $[0_A]_0 = 0$;
\item if $p,q\in\Projections_n(A)$ and $p\sim_h q \in \Projections_n(A)$, then $[p]_0 = [q]_0$.
\end{enumerate}
and
\begin{enumerate}
\setcounter{enumi}{3}
\item if $p,q\in\Projections_n(A)$ and $pq =0$, then $[p+q]_0 = [p]_0+[q]_0$;
\item for all $p,q\in\Projections_\infty(A)$, $[p]_0 = [q]_0 \iff p\sim_s q$.
\end{enumerate}
For $*$-homomorphisms $f,g: K_{00}(A)\to K_{00}(B)$:
\begin{enumerate}
\item $K_{00}(f)([p]_0) = [f(p)]_0$;
\item $\forall p\in\Projections_\infty(A): f([p]_0) = g([p]_0) \implies  f = g$;
\item If $f,g$ are orthogonal, i.e.\ for all $a\in A$: $f(a)\cdot g(a) = 0$, then
\[ K_{00}(f+g) = K_{00}(f)+K_{00}(g). \]
\end{enumerate}
\end{proposition}
\begin{proof}
The first equality is a property of the Grothendieck map. For the second equality, take $p\in\Projections_m(A)$ and $p\in\Projections_n(A)$, then $[p]_0 - [q]_0 = [p\oplus 0^{n\times n}]_0 - [q \oplus^{m\times m}]_0$.
Then
\begin{enumerate}
\item This follows because $[p\oplus q]_\mathcal{V} = [p]_\mathcal{V} + [q]_\mathcal{V}$ and the Grothendieck map is a homomorphism. TODO refs.
\item $[0_A]_0+[0_A]_0 = [0_A\oplus 0_A]_0 = [0_A]_0$, since $0_A\oplus 0_A \sim_0 0_A$.
\item By \ref{implicationsBeweenEquivalences},
\[ p\sim_h q \implies p\sim q \implies p\sim_0 q \implies [p]_\mathcal{V} = [q]_\mathcal{V} \implies [p]_0 = [q]_0. \]
\end{enumerate}
and
\begin{enumerate}
\setcounter{enumi}{3}
\item This is just point (4) of \ref{sim0properties} combined $[p\oplus q]_0 = [p]_0+[q]_0$.
\item If $p\sim_s q$, then $p\oplus r \sim_0 q\oplus r$, so $[p]_0 +[r]_0 = [q]_0 + [r]_0$ and $[p]_0 = [q]_0$ because $K_{00}(A)$ is a group.

Conversely, if $[p]_0 = [q]_0$, then there is an $[r]_\mathcal{V}$ such that $[p]_\mathcal{V} + [r]_\mathcal{V} = [q]_\mathcal{V} + [r]_\mathcal{V}$. Hence
\[ [p\oplus r]_\mathcal{V} = [q\oplus r]_\mathcal{V} \implies p\oplus r \sim_0 q\oplus r \implies p\sim_s q. \]
Note that we cannot conclude $p \sim_0 q$ from $[p]_0 = [q]_0$, because the Grothedieck map is not necessarily injective.
\end{enumerate}
For the morphisms,
\begin{enumerate}
\item $K_{00}(f)([p]_0) = G(\mathcal{V}(f))([p]_0) = \mathcal{V}(f)([p]_0) = [f(p)]_0$.
\item Assume $ \forall p\in\Projections_\infty(A): f([p]_0) = g([p]_0)$. Take an arbitrary $[p]_0-[q]_0\in K_{00}(A)$, then
\[ f([p]_0-[q]_0) = f([p]_0)-f([q]_0) = g([p]_0)-g([q]_0) = g([p]_0-[q]_0). \]
\item For all $p\in\Projections_\infty(A)$:
\begin{align*}
K_{00}(f + g)([p]_0) &= K_{00}([f(p) + g(p)]_0) = K_{00}([f(p)]_0 + [g(p)]_0) =  (K_{00}(f) + K_{00}(g))([p]_0)
\end{align*}
where we have used point $(4)$ of \ref{sim0properties}.
\end{enumerate}
\end{proof}


\begin{proposition}[Universal property of $K_{00}$]
Let $A$ be a $C^*$-algebra and $G$ an Abelian group. Suppose $\nu:\Projections_\infty(A)\to G$ is a function that satisfies
\begin{enumerate}
\item $\nu(p\oplus q) = \nu(p)+\nu(q)$ for all projections $p,q\in\Projections_\infty(A)$;
\item $\nu(0_A) = 0$;
\item if $p,q\in\Projections_n(A)$ and $p\sim_h q \in \Projections_n(A)$, then $\nu(p) = \nu(q)$.
\end{enumerate}
Then there is a unique group homomorphism $\alpha: K_{00}(A) \to G$ which makes the diagram
\[ \begin{tikzcd}
\Projections_\infty(A) \ar[d,"{[\cdot]_0}"] \ar[dr, "\nu"] & \\
K_{00}(A) \ar[r, dashed, swap, "\exists! \alpha"] & G
\end{tikzcd} \qquad \text{commute.}\]
\end{proposition}
In the third point we can also use $\sim_u, \sim_0$ or $\sim_s$:
\begin{proposition}
Let A be a $C^*$-algebra, $G$ an Abelian group, and
$\nu: \Projections_\infty(A) \to G$ a function that satisfies $\nu(0_A) = 0$ and $\nu(p\oplus q) = \nu(p) + \nu(q)$ for all projections $p,q\in\Projections_\infty(A)$. Then the following are equivalent:
\begin{itemize}
\item[$3.$] for all $n$ and all $p,q\in \Projections_n(A)$: if $p\sim_h q$, then $\nu(p) = \nu(q)$;
\item[$3'.$]for all $n$ and all $p,q\in \Projections_n(A)$: if $p\sim_u q$, then $\nu(p) = \nu(q)$;
\item[$3^{\prime\prime}.$] for all  $p,q\in \Projections_\infty(A)$: if $p\sim_0 q$, then $\nu(p) = \nu(q)$;
\item[$3^{\prime\prime\prime}.$] for all  $p,q\in \Projections_\infty(A)$: if $p\sim_s q$, then $\nu(p) = \nu(q)$;
\end{itemize}
\end{proposition}
\begin{proof}
We cyclically prove $(3^{\prime\prime\prime}) \Rightarrow (3^{\prime\prime}) \Rightarrow (3') \Rightarrow (3) \Rightarrow (3^{\prime\prime\prime})$:
\begin{enumerate}
\item[$\boxed{(3^{\prime\prime\prime}) \Rightarrow (3^{\prime\prime})}$] Assume $(3^{\prime\prime\prime})$ and take arbitrary $p,q\in\Projections_\infty(A)$ such that $p\sim_0 q$. Then $p\oplus 0 \sim_0 q\oplus 0$, so $p\sim_s q$ and $\nu(p)=\nu(q)$ by $(3^{\prime\prime\prime})$.
\item[$\boxed{(3^{\prime\prime}) \Rightarrow (3')}$] Assume $(3^{\prime\prime})$ and take arbitrary $p,q\in\Projections_n(A)$ for some $n$ such that $p\sim_u q$. Then $p\sim q$ by \ref{implicationsBeweenEquivalences} and thus $p\sim_0 q$, so $\nu(p)=\nu(q)$ by $(3^{\prime\prime})$.
\item[$\boxed{(3') \Rightarrow (3)}$] Assume $(3')$ and take arbitrary $p,q\in\Projections_n(A)$ for some $n$ such that $p\sim_h q$. Then $p\sim_u q$ by \ref{implicationsBeweenEquivalences}, so $\nu(p)=\nu(q)$ by $(3')$.
\item[$\boxed{(3) \Rightarrow (3^{\prime\prime\prime})}$] Assume $(3)$ and take arbitrary $p,q\in\Projections_\infty(A)$ such that $p\sim_s q$. Then there exists an $r\in\Projections_\infty(A)$ such that $p\oplus r \sim_0 q\oplus r$. If $p\oplus r\in\Projections_m$ and $q\oplus r\in\Projections_n$, then
\[ p \oplus r\oplus 0^{n\times n} \sim_0 p\oplus r \sim_0 q\oplus r \sim_0 q\oplus r\oplus 0^{m\times m}. \]
And since both sides are in $\Projections_{n+m}(A)$, we have $p \oplus r\oplus 0^{n\times n} \sim q\oplus r\oplus 0^{m\times m}$. By \ref{implicationsBeweenEquivalences} this implies
\[ p \oplus r\oplus 0^{n\times n}\oplus 0^{(m+n)\times(m+n)} \sim_h q\oplus r\oplus 0^{m\times m}\oplus 0^{(m+n)\times(m+n)}. \]
Using $(3)$ this gives
\[ \nu(p) + \nu(r) = \nu(p \oplus r\oplus 0^{n\times n}\oplus 0^{(m+n)\times(m+n)}) = \nu(q\oplus r\oplus 0^{m\times m}\oplus 0^{(m+n)\times(m+n)}) = \nu(q) +\nu(r) \]
which implies $\nu(p) = \nu(q)$.
\end{enumerate}
\end{proof}


\begin{proposition}[Homotopy invariance of $K_{00}$] \label{homotopyInvarianceK00}
Let $A,B$ be $C^*$-algebras. If $\varphi,\psi:A\to B$ are homotopic $*$-homomorphisms, then 
\[ K_{00}(\varphi) = K_{00}(\psi). \]
\end{proposition}
\begin{proof}
For every $p\in\Projections_\infty(A)$, $\varphi(p)\sim_h \psi(p)$. So
\[ K_{00}(\varphi)([p]_0) = [\varphi(p)]_0 = [\psi(p)]_0 = K_{00}(\psi)([p]_0), \]
meaning $K_{00}(\varphi) = K_{00}(\psi)$
\end{proof}
\begin{corollary}
If $A$ and $B$ are homotopy equivalent, then $K_{00}(A)\cong K_{00}(B)$.
\end{corollary}
\begin{proof}
If $\psi\circ\varphi \sim_h I_A$, then
\[ K_{00}(\psi)\circ K_{00}(\varphi) = K_{00}(I_A) = I_{K_{00}(A)}. \]
Similarly $\varphi\circ\psi \sim_h I_B$ implies
\[ K_{00}(\varphi)\circ K_{00}(\psi) = K_{00}(I_B) = I_{K_{00}(B)}. \]
So $K_{00}(\varphi): A\to B$ is invertible with inverse $K_{00}(\psi)$.
\end{proof}

\begin{proposition}
The functor $K_{00}$ is not half exact for non-unital $C^*$-algebras.
\end{proposition}
This is essentially the motivation to work with $K_0$, not $K_{00}$.

\section{The $K_{0}$ functor}
\begin{definition}
Let $A$ be a $C^*$-algebra and $\pi: A^\dagger \to \C$ the projection of the second component of $A^\dagger$onto $\C$. Then we define $K_0(A)$ as the kernel of $K_{00}(\pi): K_{00}(A^\dagger)\to K_{00}(\C)$.
\[ \begin{tikzcd}
0 \rar & A \rar[hook, "\iota"] & A^\dagger \rar[shift left, "\pi"] & \lar[hook, shift left, "\lambda"] \C \rar & 0
\end{tikzcd} \]
\end{definition}
\begin{proposition} \label{K00embedsIntoK0}
Let $A$ be a $C^*$-algebra, then $K_{00}(\iota)$ in an embedding $K_{00}(\iota): K_{00}(A) \hookrightarrow K_{0}(A)$.
\end{proposition}
\begin{proof}
To show injectivity, it is enough to note that $\iota$ is split monic. Then $K_{00}(\iota)$ is also split monic in the category $\cat{Ab}$ and thus injective.

To show the image is a subset of $K_0(A)$, take some arbitrary $[p]_0-[q]_0\in K_{00}(A)$ with  $p,q\in\Projections_\infty(A)$. Then we claim 
\[K_{00}(\iota)([p]_0-[q]_0) = K_{00}(\iota)([p]_0)-K_{00}(\iota)([q]_0) =  [\iota(p)]_0-[\iota(q)]_0\]
maps to zero under $K_{00}(\pi)$. Indeed:
\[ K_{00}(\pi)([\iota(p)]_0-[\iota(q)]_0) = K_{00}(\pi)([\iota(p)]_0) - K_{00}(\pi)([\iota(q)]_0) = [\pi(\iota(p))]_0 - [\pi(\iota(q))]_0 = 0 \]
using that fact that $\pi\circ\iota = 0$. So $K_{00}(\iota)([p]_0-[q]_0)\in\ker(K_{00}(\pi)) = K_0(A)$.
\end{proof}
So we can identify $K_{00}(A) \cong \im K_{00}(\iota) \subseteq K_0(A)$ and we can naturally extend $[\cdot]_{0}$ to a function
\[ \Projections_\infty(A)\to K_{00}(A) \hookrightarrow K_0(A). \]

\begin{proposition}
Let $A$ be a unital $C^*$-algebra. Then $K_{00}(A) \cong K_0(A)$.
\end{proposition}
\begin{proof}
By \ref{K00embedsIntoK0} we have $K_{00}(\iota): K_{00}(A) \hookrightarrow K_{0}(A)$. We just need to show $K_{00}(\iota)$ is surjective. To do this we are going to decompose $I_{A^\dagger}$ into the form
\[ I_{A^\dagger} = \iota\circ \mu + \mu'\circ \pi \]
with the property that $\iota\circ \mu \cdot \mu'\circ \pi$ is the zero map, i.e.\ $\iota\circ \mu$ and $\mu'\circ \pi$ are orthogonal. Such a decomposition is given by
\[ \mu: A^\dagger \to A: (a,\alpha) \mapsto a+\alpha \qquad \text{and}\qquad \mu': \C \to A^\dagger: \alpha \mapsto (-\alpha, \alpha),\]
which works because $A$ is unital.

We verify
\begin{align*}
&(\iota\circ \mu)(a,\alpha) + (\mu'\circ \pi)(a,\alpha) = (a+\alpha, 0) + (-\alpha, a\alpha) = (a,\alpha) \\
&(\iota\circ \mu)(a,\alpha) \cdot (\mu'\circ \pi)(a,\alpha) = (a+\alpha,0) \cdot (-\alpha, \alpha) = (-\alpha(a+\alpha) + \alpha(a+\alpha) + 0,0) = (0,0).
\end{align*}

Then take some $s\in K_0(A) = \ker(K_{00}(\pi))$. We calculate
\begin{align*}
s &= I_{K_{00}(A^\dagger)}(s) = K_{00}(I_{A^\dagger})(s) = K_{00}(\iota\circ \mu + \mu'\circ \pi)(s) \\
&= K_{00}(\iota\circ \mu)(s) + K_{00}(\mu'\circ \pi)(s) = (K_{00}(\iota)\circ K_{00}(\mu))(s) + (K_{00}(\mu')\circ K_{00}(\pi))(s) \\
&= K_{00}(\iota)\circ K_{00}(\mu))(s) + 0 \in \im K_{00}(\iota).
\end{align*}
\end{proof}
For this reason $K_0(A)$ is often defined as $K_{00}(A)$ for unital algebras.

\begin{lemma}
Let $A,B$ be $C^*$-algebras. For every morphism $f: A\to B$, we can view $K_{00}(f)$ as a morphism on a subgroup of $K_0(A)$ by identifying it with
\[ K_{00}(\iota)K_{00}(f)K_{00}(\iota)^{-1}. \]
Then $K_{00}(f)$ can uniquely be extended to a morphism $K_0(f): K_0(A)\to K_0(B)$. 

This makes $K_0$ a functor.
\end{lemma}
\begin{proof}
Consider the diagram
\[ \begin{tikzcd}
K_{00}(A) \rar{K_{00}(\iota_A)} \dar{K_{00}(f)} & K_0(A) \rar{\subseteq}\dar[dashed]{K_0(f)} & K_{00}(A^\dagger) \rar{K_{00}(\pi_A)}\dar{K_{00}(f^\dagger)} & K_{00}(\C) \ar[d, equals] \\
K_{00}(B) \rar[swap]{K_{00}(\iota_B)} & K_0(B) \rar[swap]{\subseteq} & K_{00}(B^\dagger) \rar[swap]{K_{00}(\pi_B)} & K_{00}(\C)
\end{tikzcd} \]
which is commutative because functors preserve commutative diagrams. Uniqueness is immediate from the commutativity of the middle square. To show existence, we must show that
\[ K_{00}(f^\dagger)[K_0(A)] \subseteq K_0(B). \]
Take $r \in K_0(A) = \ker K_{00}(\pi_A)$. Then using the fact that $f^\dagger$ commutes with $\pi$, i.e.\ 
\[ \pi_B \circ f^\dagger = f^\dagger\circ \pi_A,\]
we get 
\[ (K_{00}(\pi_B)\circ K_{00}(f^\dagger))(r) = (K_{00}(f^\dagger)\circ K_{00}(\pi_A))(r) = K_{00}(f^\dagger)(0) = 0. \]
So $K_{00}(f^\dagger)(r) \in \ker K_{00}(\pi_B) = K_0(B)$.
\end{proof}

\begin{proposition}[The standard picture of $K_{0}$] \label{StandardPictureK0}
Let $A$ be a $C^*$-algebra, then
\begin{align*}
K_{0}(A) &= \setbuilder{[p]_0 - [s(p)]_0}{p \in \Projections_\infty(A^\dagger)}.
\end{align*}
Moreover, for all $p,q\in\Projections_\infty(A^\dagger)$, the following are equivalent:
\begin{enumerate}
\item $[p]_0-[s(p)]_0 = [q]_0 - [s(q)]_0$,
\item $p\oplus \vec{1}_k \sim_0 q \oplus \vec{1}_l$ in $\Projections_\infty(A^\dagger)$ for some $k,l\in \N$,
\item there exist scalar projections $r_1, r_2$ such that $p\oplus r_1 \sim_0 q \oplus r_2$ in $\Projections_\infty(A^\dagger)$.
\end{enumerate}
If $\varphi:A\to B$ is a $*$-homomorphism, then
\[ K_0(\varphi)([p]_0 - [s(p)]_0) = [\varphi^\dagger(p)]_0 - [s(\varphi^\dagger(p))]_0 \]
for all $p\in \Projections_\infty(A^\dagger)$.
\end{proposition}
\begin{proof}
For all $p\in\Projections_\infty(A^\dagger)$, $[p]_0 - [s(p)]_0$ is in $K_0(A) = \ker K_{00}(\pi)$:
\[ K_{00}(\pi)([p]_0 - [s(p)]_0) = [\pi(p)]_0 - [(\pi\circ s)(p)]_0 = [\pi(p)]_0 - [\pi(p)]_0 = 0. \]
Conversely, let $g\in K_0(A)$, then by the standard picture of $K_{00}(A^\dagger)$ there are $e,f\in \Projections((A^\dagger)^{n\times n})$ such that $g = [e]_0 - [f]_0$. Put
\[ p = \begin{pmatrix}
e & 0 \\ 0 & \vec{1}_n - f
\end{pmatrix}, \qquad q = \begin{pmatrix}
0 & 0 \\ 0 & \vec{1}_n
\end{pmatrix}. \]
Then $(\vec{1}_n - f)$ and $f$ are orthogonal,
\[ (\vec{1}_n - f)f = f - f^2 = f-f = 0. \]
So, by \ref{StandardPictureK00}, $[\vec{1}_n]_0 = [\vec{1}_n -f + f]_0 = [\vec{1}_n -f]_0 + [f]_0$ and we get
\[ [p]_0 - [q]_0 = [e]_0 + [\vec{1}_n - f]_0 - [\vec{1}_n]_0 = [e]_0 - [f]_0 = g. \]
Using $q = s(q)$ and $K_{00}(\pi)(g) = 0$, we get
\[ [s(p)]_0 - [q]_0 = [s(p)]_0 - [s(q)]_0 = K_{00}(s)(g) = (K_{00}(\lambda)\circ K_{00}(\pi))(g) = 0. \]
So $[s(p)]_0 = [q]_0$ and we get $g = [p]_0 - [s(p)]_0$.

We prove the equivalent statements cyclically:
\begin{itemize}
\item[$\boxed{(1) \Rightarrow (3)}$] Suppose $[p]_0-[s(p)]_0 = [q]_0 - [s(q)]_0$ for some $p,q\in\Projections_\infty(A^\dagger)$ this implies
\[ [p\oplus s(q)]_0 = [q\oplus s(p)]_0 \implies p\oplus s(q) \sim_s q\oplus s(p) \qquad \text{in $\Projections_\infty(A^\dagger)$} \]
by \ref{StandardPictureK00}. By \ref{stableEquivalence} this implies $p\oplus s(q)\oplus \vec{1}_n \sim_0 q \oplus s(p) \oplus \vec{1}_n$. Putting $r_1 = s(q)\oplus \vec{1}_n$ and $r_2 = s(p) \oplus \vec{1}_n$, which are scalar projections, this is exactly $(3)$.
\item[$\boxed{(3) \Rightarrow (2)}$] If $r_1$ is a scalar projection in $\Projections_k(A)$ and $r_2$ in $\Projections_l(A)$, then $r_1\sim_0 \vec{1}_k$ and $r_2 \sim_0 \vec{1}_l$, by TODO ref. Hence $p\oplus \vec{1}_k \sim_0 q \oplus \vec{1}_l$.
\item[$\boxed{(2) \Rightarrow (1)}$] 
\end{itemize}
\end{proof}

\begin{proposition}[Half exactness of $K_0$]
Every short exact sequence of $C^*$-algebras
\[ \begin{tikzcd}
0 \rar & I \rar{\varphi} & A \rar{\psi} & B \rar & 0
\end{tikzcd} \]
induces an exact sequence of Abelian groups
\[ \begin{tikzcd}
K_0(I) \rar{K_0(\varphi)} & K_0(A) \rar{K_0(\psi)} & K_0(B).
\end{tikzcd} \]
\end{proposition}

\begin{proposition}
Every split exact sequence of $C^*$-algebras
\[ \begin{tikzcd}
0 \rar & I \rar{\varphi} & A \rar[shift left]{\psi} & B \rar \lar[shift left]{\lambda} & 0
\end{tikzcd} \]
induces a split exact sequence of Abelian groups
\[ \begin{tikzcd}
0 \rar & K_0(I) \rar{K_0(\varphi)} & K_0(A) \rar[shift left]{K_0(\psi)} & K_0(B) \rar \lar[shift left]{K_0(\lambda)} & 0
\end{tikzcd} \]
\end{proposition}

\begin{proposition}
For every pair $A,B$ of $C^*$-algebras,
\[ K_0(A\oplus B) \cong K_0(A)\oplus K_0(B).  \]
\end{proposition}

\subsection{Homotopy, suspensions and cones}
\begin{proposition}[Homotopy invariance of $K_0$]
Let $A,B$ be $C^*$-algebras. If $\varphi,\psi:A\to B$ are homotopic $*$-homomorphisms, then 
\[ K_{0}(\varphi) = K_{0}(\psi). \]
If $A$ and $B$ are homotopy equivalent, then $K_{0}(A)\cong K_{0}(B)$.
\end{proposition}
\begin{proof}
If $\varphi$ is homotopic to $\psi$, then $\varphi^\dagger$ is homotopic to $\psi^\dagger$ and thus $K_{00}(\varphi^\dagger) = K_{00}(\psi^\dagger)$, by \ref{homotopyInvarianceK00}. Restricting to $K_0(A)$ yields the result.
\end{proof}
In particular, $K_0(A) = 0$ for every contractible $C^*$-algebra $A$.

\begin{definition}
Let $A$ be a $C^*$-algebra. The \udef{cone} over $A$ is
\[ CA \defeq \setbuilder{f\in C([0,1], A)}{f(0) = 0}. \]
The \udef{suspension} of $A$ is
\[ SA \defeq \setbuilder{f\in C([0,1], A)}{f(0) = f(1) = 0}. \]
\end{definition}

\begin{lemma}
If the operations are pointwise and the norm the supremum norm, then $CA$ and $SA$ are $C^*$-algebras.
\end{lemma}

\begin{lemma}
Let $A$ be a $C^*$-algebra. The cone $CA$ is contractible. The suspension $SA$ is contractible if $A$ is contractible.
\end{lemma}
\begin{proof}
Let $\gamma_t: CA\to CA$ be defined by $\gamma_t(f)(s) = f(st)$ for all $t\in [0,1]$. Then $\gamma$ defines a contraction of $CA$.

For any contraction $\beta_t: A\to A$ of $A$, $\gamma_t: SA \to SA: f\mapsto \beta_t\circ f$ is a contraction of $SA$.
\end{proof}

\begin{lemma} \label{exactSequenceSuspensionCone}
Let $A$ be a $C^*$-algebra. Then $SA$ is a closed ideal of $CA$ and $A \cong CA/SA$.

We have the short exact sequence
\[ \begin{tikzcd}
 0 \rar & SA \rar[hook, "\iota"] & CA \rar & A \rar & 0.
\end{tikzcd} \]
\end{lemma}
\begin{proof}
Consider the map
\[ CA \to A: f\mapsto f(1). \]
This is a surjective morphism with kernel $SA$. So $SA$ is a closed ideal by ref TODO.
\end{proof}


\begin{definition}
Let $A,B$ be $C^*$-algebras and $\alpha: A\to B$ a morphism. The \udef{mapping cone} for $\alpha$ is
\[ \begin{tikzcd}
C_\alpha \defeq \setbuilder{(a, f) \in A\oplus CB}{f(1) = \alpha(a)}.
\end{tikzcd} \]
\end{definition}
The mapping cone is a $C^*$-algebra.

\begin{lemma}
The mapping cone $\alpha$ is related to $A$ and $B$ in the short exact sequence
\[ \begin{tikzcd}
0 \rar & SB \rar{\iota: f\mapsto (0,f)} & C_\alpha \rar{\pi: (a,f) \mapsto a} & A \rar & 0.
\end{tikzcd} \]
Moreover, the sequence
\[ \begin{tikzcd}
K_0(C_\alpha) \rar{\pi_*} & K_0(A) \rar{\alpha_*} & K_0(B)
\end{tikzcd} \]
is exact.
\end{lemma}
\begin{proof}
TODO
\end{proof}


\section{The $K_1$ functor}
\begin{definition}
As with the projections, we define, for some unital $C^*$-algebra,
\[ \Unitaries_n(A) = \Unitaries(A^{n\times n}), \qquad \Unitaries_\infty(A) = \bigcup^\infty_{n=1}\Unitaries_n(A) \]
as well as the binary operations $\oplus$ on $\Unitaries_\infty(A)$
\[ \forall u,v \in \Unitaries_\infty(A): \quad u\oplus v = \diag(u,v) = \begin{pmatrix}
u & 0 \\ 0 & v
\end{pmatrix}. \]
\end{definition}
\begin{lemma}
Let $A$ be a unital $C^*$-algebra. If $u\in\Unitaries_n(A)$ and $v\in\Unitaries_m(A)$, then $u\oplus v\in\Unitaries_{n+m}(A)$.
\end{lemma}
\subsection{Normalised matrices}
\begin{definition}
The group of \udef{normalised invertible matrices} is
\[ \GL_n^\dagger(A) \defeq \setbuilder{a\in \GL_n(A^\dagger)}{\pi(a) = \vec{1}_n} \]
and the group of \udef{normalised unitary matrices} is
\[ \Unitaries_n^\dagger(A) \defeq \setbuilder{u\in \Unitaries_n(A^\dagger)}{\pi(u) = \vec{1}_n}. \]
The group operation for both is multiplication. We also define
\[ \GL_\infty^\dagger(A) \defeq \bigcup_{n=1}^\infty\GL_n^\dagger(A) \qquad \text{and}\qquad \Unitaries_\infty^\dagger(A) \defeq \bigcup_{n=1}^\infty\Unitaries_n^\dagger(A). \]
\end{definition}

\begin{proposition} \label{normalisedQuotientsIsomorphisms}
Let $A$ be a $C^*$-algebra and $n\in \N\cup\{\infty\}$. Then the groups
\begin{align*}
\GL_n^\dagger(A) / \GL_n^\dagger(A)_0,& & &\Unitaries_n^\dagger(A) / \Unitaries_n^\dagger(A)_0 \\
\GL_n(A^\dagger) / \GL_n(A^\dagger)_0,& & &\Unitaries_n(A^\dagger) / \Unitaries_n(A^\dagger)_0
\end{align*}
are pairwise isomorphic. If $A$ is unital, then 
\[ \GL_n^\dagger(A) \cong \GL_n(A) \qquad \text{and} \qquad \Unitaries_n^\dagger(A) \cong \Unitaries_n(A). \]
\end{proposition}
\begin{proof}
First note that by \ref{sectionConnectedToIdentity}, all quotients are normal subgroups.

Consider the continuous map
\[ \phi: \GL_n^\dagger(A) \to \Unitaries_n^\dagger(A): z\mapsto z|z|^{-1}. \]
The induced map
\[ \psi: \GL_n^\dagger(A) / \GL_n^\dagger(A)_0 \to \Unitaries_n^\dagger(A) / \Unitaries_n^\dagger(A)_0: [z] \mapsto [\phi(z)] \]
is a group homomorphism by (TODO ref) and is bijective: it is clearly surjective. For injectivity, we prove the kernel is trivial.
Indeed let $[\phi(z)] = \vec{1}$, then $z|z|^{-1} \sim_h \vec{1}_n$. By \ref{unitariesRetractionOfGL}, $z|z|^{-1}\sim_h z$, so $z\sim_h \vec{1}_n$ by transitivity.

To prove $\GL_n(A^\dagger) / \GL_n(A^\dagger)_0 \cong \GL_n^\dagger(A) / \GL_n^\dagger(A)_0$, consider the isomorphism $[z] \mapsto [z\pi(z^{-1})]$.

Restricting to unitaries gives the last isomorphism.
\end{proof}

\subsection{Equivalence of unitaries}
\begin{definition}
We define a relation $\sim_1$ on $\Unitaries_\infty(A)$ as follows: for $u\in \Unitaries_n(A)$ and $v\in \Unitaries_m(A)$,
\[ u \sim_1 v \defequiv \exists k\geq \max\{m,n\}: \quad u\oplus \vec{1}_{k-n} \sim_h v\oplus \vec{1}_{k-m} \quad\text{in $\Unitaries_{k}(A)$}. \]
With the convention that $w\oplus 1_0 = w$ for all $w\in\Unitaries_\infty(A)$.
\end{definition}
\begin{lemma}
Let $A$ be a unital $C^*$-algebra. Then for all $u,v,u',v'\in \Unitaries(A)$
\begin{enumerate}
\item $\sim_1$ is an equivalence relation on $\Unitaries_\infty(A)$;
\item $u \sim_1 u\oplus \vec{1}_k$
\item $u\oplus v \sim_1 v\oplus u$;
\item if $u\sim_1 u'$ and $v\sim_1 v'$, then $u\oplus v \sim_1 u'\oplus v'$;
\item if $u,v\in \Unitaries_n(A)$, then $uv\sim_1 vu \sim_1 u\oplus v$.
\end{enumerate}
\end{lemma}

\subsection{The $K_1$ functor}
\begin{definition}
For each $C^*$-algebra $A$, we define
\[ K_1(A) = \Unitaries_\infty(A^\dagger) / \sim_1. \]
Define the binary operation $+$ on $K_1(A)$ by $[u]_1 + [v]_1 = [u\oplus v]_1$.
\end{definition}
Because
\[ [u]_1 + [v]_1 = [u\oplus v]_1 = [uv]_1, \]
the $K_1(A)$ group is naturally a multiplicative group that we are writing in additive notation for uniformity with other $K$ groups.
Notice in particular that
\[ [\vec{1}]_1 = 0. \]

In fact we could have defined
\[ [u]_1 + [v]_1 = [uv]_1, \]
so that there is less dependence on matrices. This is different than for projections where the definition
\[ [p]_0 + [q]_0 = [p+q]_0 \]
did not work because $p+q$ was not necessarily a projection.

\begin{proposition}
Let $A$ be a $C^*$-algebra. Then $K_1(A)$ is isomorphic to any of the following:
\begin{align*}
\GL_\infty^\dagger(A) / \GL_\infty^\dagger(A)_0,& & &\Unitaries_\infty^\dagger(A) / \Unitaries_\infty^\dagger(A)_0 \\
\GL_\infty(A^\dagger) / \GL_\infty(A^\dagger)_0,& & &\Unitaries_\infty(A^\dagger) / \Unitaries_\infty(A^\dagger)_0.
\end{align*}
\end{proposition}
\begin{proof}
The four groups are isomorphic by \ref{normalisedQuotientsIsomorphisms}. Consider $\Unitaries_\infty(A^\dagger) / \Unitaries_\infty(A^\dagger)_0$. We just need to see that $\forall u\in \Unitaries_\infty(A^\dagger)$
\[ u\sim_1 \vec{1} \iff u \in \Unitaries_\infty(A^\dagger)_0\]
by TODO ref. This is clear.
\end{proof}

\begin{proposition}[Universal property of $K_{1}$] \label{univeralPropertyK1}
Let $A$ be a $C^*$-algebra and $G$ an Abelian group. Suppose $\nu:\Unitaries_\infty(A^\dagger)\to G$ is a function that satisfies
\begin{enumerate}
\item $\nu(u\oplus v) = \nu(u)+\nu(v)$ for all unitaries $u,v\in\Unitaries_\infty(A^\dagger)$;
\item $\nu(\vec{1}_A) = 0$;
\item if $u,v\in\Unitaries_n(A^\dagger)$ and $u\sim_h v \in \Unitaries_n(A^\dagger)$, then $\nu(p) = \nu(q)$.
\end{enumerate}
Then there is a unique group homomorphism $\alpha: K_{1}(A) \to G$ which makes the diagram
\[ \begin{tikzcd}
\Unitaries_\infty(A^\dagger) \ar[d,"{[\cdot]_1}"] \ar[dr, "\nu"] & \\
K_{1}(A) \ar[r, dashed, swap, "\exists! \alpha"] & G
\end{tikzcd} \qquad \text{commute.}\]
\end{proposition}

The functor $K_1$ is homotopy invariant, half exact, split exact and respects direct sums.

\section{Exact sequences of $K$-groups}
\subsection{Suspensions}
\begin{lemma}
Let $A$ be a $C^*$-algebra. We have the isomorphisms
\begin{align*}
SA &\cong A\otimes C_0(\R) \\ &\cong C_0(\R, A) \\ &\cong C_0(]0,1[, A) \\ &\cong \setbuilder{f\in C(\mathbb{T}, A)}{f(1)=0}.
\end{align*}
\end{lemma}

\begin{lemma}
Suspension is a functor $S: \cat{C^*alg} \to \cat{C^*alg}$.
\end{lemma}
\begin{proof}
We know the suspension maps $C^*$-algebras to $C^*$-algebras. Let $f: A\to B$ be a $*$-homomorphism. Then $f_*: SA \to SB$ is well-defined and a $*$-homomorphism. The functorial properties are clearly satisfied.
\end{proof}

\begin{theorem}
The functors $K_1$ and $K_0\circ S$ are naturally isomorphic.
\end{theorem}
\begin{proof}
For every $C^*$-algebra $A$ we define
\end{proof}


\subsection{The index map}
Suppose a short exact sequence of $C^*$-algebras
\[  \begin{tikzcd}
0 \rar & I \rar{\varphi} & A \rar{\psi} & B\rar & 0
\end{tikzcd}.\]
Then we want to define a map $\delta_1:  K_1(B)\to K_0(I)$, called the \udef{index map}, such that the sequence
\[ \begin{tikzcd}
K_1(I) \rar{K_1(\varphi)} & K_1(A) \rar{K_1(\psi)} & K_1(B) \dar{\delta_1} \\
K_0(B) & K_0(A) \lar{K_0(\psi)} & K_0(I) \lar{K_0(\varphi)}
\end{tikzcd} \]
is exact.

\subsubsection{Constructing the index map}
Take an element $[u]\in K_1(B) = \Unitaries_\infty(B^\dagger) / \Unitaries_\infty(B^\dagger)_0$. Now the elements of $u\cdot \Unitaries_\infty(B^\dagger)_0$ do not in general lift to unitaries in $A^\dagger$.

We wish to measure to what degree this lifting is not possible. We expect such a map to be well-defined, because the elements of $\Unitaries_\infty(B^\dagger)_0$ should not impede the lifting, by \ref{unitaryLifting}.



To do this, we define a function $\nu': \Unitaries_\infty(B^\dagger)\to \Projections(I^\dagger)$ such that $\nu \defeq [\cdot]_0\circ \nu': \Unitaries_\infty(B^\dagger)\to K_0(I)$ satisfies the universal property of the $K_1$ functor, \ref{univeralPropertyK1}, meaning it uniquely factors through $K_1(B)$, giving a group homomorphism $\delta_1: K_1(B)\to K_0(I)$ satisfying $\delta_1([u]_1) = \nu(u)$ for each $u\in\Unitaries_\infty(B^\dagger)$.


We define the map $\nu': \Unitaries_\infty(B^\dagger)\to \Projections(I^\dagger)$ as follows:

Take $u\in \Unitaries_\infty(B^\dagger)$. First we would like to lift this to a unitary 

\begin{lemma}
Suppose a short exact sequence of $C^*$-algebras
\[  \begin{tikzcd}
0 \rar & I \rar{\varphi} & A \rar{\psi} & B\rar & 0
\end{tikzcd}\]
and let $u\in \Unitaries_n(B^\dagger)$.
\begin{enumerate}
\item There exists a unitary $v\in\Unitaries_{2n}(A^\dagger)$ and a projection $p\in\Projections_{2n}(I^\dagger)$ such that
\[ \psi^\dagger(v) = \begin{pmatrix}
u & 0 \\ 0 & u^*
\end{pmatrix}, \qquad \varphi^\dagger(p) = v \begin{pmatrix}
\vec{1}_n & 0 \\ 0 & 0
\end{pmatrix}v^*, \qquad s(p) = \begin{pmatrix}
\vec{1}_n & 0 \\ 0 & 0
\end{pmatrix}. \]
\item Given these $v,p$, if $w\in\Unitaries_{2n}(A^\dagger)$ and $q\in\Projections_{2n}(I^\dagger)$ satisfy
\[ \psi^\dagger(w) = \begin{pmatrix}
u & 0 \\ 0 & u^*
\end{pmatrix}, \qquad \varphi^\dagger(q) = w \begin{pmatrix}
\vec{1}_n & 0 \\ 0 & 0
\end{pmatrix}w^*, \]
then $s(q) = \diag(\vec{1}_n, 0_n)$ and $p\sim_u q$ in $\Projections_{2n}(I^\dagger)$.
\end{enumerate}
\end{lemma}
\begin{proof}
(1). Because $\diag(u,u^*)\sim_h \diag(1,1)$, we can use the first point of \ref{unitaryLifting} to see that $\diag(u,u^*)$ lifts to a unitary $v\in (A^\dagger)^{2n\times 2n}$, giving the first equation. Also
\[ \psi^\dagger( v \begin{pmatrix}
\vec{1}_n & 0 \\ 0 & 0
\end{pmatrix}v^* ) = \psi^\dagger(v) \begin{pmatrix}
\vec{1}_n & 0 \\ 0 & 0
\end{pmatrix} \psi^\dagger(v^*) = \begin{pmatrix}
u & 0 \\ 0 & u^*
\end{pmatrix}\begin{pmatrix}
\vec{1}_n & 0 \\ 0 & 0
\end{pmatrix}\begin{pmatrix}
u^* & 0 \\ 0 & u
\end{pmatrix} = \begin{pmatrix}
\vec{1}_n & 0 \\ 0 & 0
\end{pmatrix}. \]
Letting $\pi_1: A^\dagger \to A$ be the projection as in \ref{projectionOnACommutes}, this means
\[ \pi_1(\psi^\dagger( v \begin{pmatrix}
\vec{1}_n & 0 \\ 0 & 0
\end{pmatrix}v^* )) = \psi(\pi_1( v \begin{pmatrix}
\vec{1}_n & 0 \\ 0 & 0
\end{pmatrix}v^* )) = 0. \]
By the exactness of the sequence, 
\[ \pi_1( v \begin{pmatrix}
\vec{1}_n & 0 \\ 0 & 0
\end{pmatrix}v^* ) \in \im (\varphi), \]
meaning there is a $p\in (I^\dagger)^{2n\times 2n}$ such that $\varphi^\dagger(p) = v\diag(\vec{1}_n,0)v^*$ and $p$ is a projection by \ref{injectiveLifts}.
For the third equality, $s(p) = \psi^\dagger(\varphi^\dagger(p)) = \diag(\vec{1}_n, 0)$.

(2). That $s(q) = \diag(\vec{1}_n, 0)$ follows from $\psi^\dagger(\varphi^\dagger(p)) = \diag(\vec{1}_n, 0)$ as before.

Then $\psi^\dagger(wv^*) = \vec{1}_{2n}$, so $\psi(\pi_1(wv^*)) = 0$ and $\pi_1(wv^*)\in \im\varphi$ by exactness. So we can find a $z\in (I^\dagger)^{2n\times 2n}$ such that $\varphi^\dagger(z) = wv^*$ and $z$ is unitary by \ref{injectiveLifts}. From $\varphi^\dagger(zpz^*) = \varphi^\dagger(q)$ and the injectivity of $\varphi^\dagger$, we get $q = zpz^*$, meaning $p \sim_u q$ in $\Projections_{2n}(I^\dagger)$.
\end{proof}
We use this lemma to define a function
\[ \nu: \Unitaries_\infty(B^\dagger)\to K_0(I) \]
which maps $u\in\Unitaries_\infty(B^\dagger)$ to $\nu(u) = [p]_0 - [s(p)]_0$ where $p\in\Projections_{2n}(I^\dagger)$ is as in the lemma. This map is well-defined by the lemma.

\begin{lemma}
The map $\nu: \Unitaries_\infty(B^\dagger)\to K_0(I)$ satisfies the universal property of $K_1(B)$:
\begin{enumerate}
\item $\nu(u_1\oplus u_2) = \nu(u_1)+\nu(u_2)$ for all unitaries $u_1,u_2\in\Unitaries_\infty(B^\dagger)$;
\item $\nu(\vec{1}) = 0$;
\item if $u_1,u_2\in\Unitaries_n(B^\dagger)$ and $u_1\sim_h u_2 \in \Unitaries_n(B^\dagger)$, then $\nu(u_1) = \nu(u_2)$.
\end{enumerate}
\end{lemma}
\begin{proof}
(1). For $j=1,2$, let $u_j$ be given. Choose $v_j \in \Unitaries_{2n_j}(A^\dagger)$ and $p_j \in \Projections_{2n_j}(I^\dagger)$ as in the definition of the index map, i.e.\ $\nu(u_j) = [p_j]_0 - [s(p_j)]_0$. Then introduce
\[ y = \begin{pmatrix}
\vec{1}_{n_1} & 0 & 0 & 0 \\
0 & 0 & \vec{1}_{n_2} & 0 \\
0 & \vec{1}_{n_1} & 0 & 0 \\
0 & 0 & 0 & \vec{1}_{n_2}
\end{pmatrix}\in \Unitaries_{2(n_1+n_2)}(\C) \]
\end{proof}

Because the map $\nu$ satisfies the universal property of the $K_1$ functor, \ref{univeralPropertyK1}, it uniquely factors throught $K_1(B)$, giving a group homomorphism $\delta_1: K_1(B)\to K_0(I)$ satisfying $\delta_1([u]_1) = \nu(u)$ for each $u\in\Unitaries_\infty(B^\dagger)$.

The map $\delta_1$ is called the \udef{index map} associated with the short exact sequence.

\subsubsection{Properties of the index map}
With the index map we have the exact sequence:
\[ \begin{tikzcd}
K_1(I) \rar{K_1(\varphi)} & K_1(A) \rar{K_1(\psi)} & K_1(B) \dar{\delta_1} \\
K_0(B) & K_0(A) \lar{K_0(\psi)} & K_0(I) \lar{K_0(\varphi)}
\end{tikzcd} \]

\begin{proposition}[Naturality of the index map] \label{naturalityIndexMap}
Let
\[ \begin{tikzcd}
0 \rar & I \dar{\gamma} \rar{\varphi} & A \dar{\alpha} \rar{\psi} & B \dar{\beta} \rar & 0 \\
0 \rar & I' \rar{\varphi'} & A' \rar{\psi'} & B' \rar & 0
\end{tikzcd} \]
be a commutative diagram of short exact rows of $C^*$-algebras. Let
\[ \delta_1: K_1(B) \to K_0(I) \qquad \text{and} \qquad \delta'_1: K_1(B') \to K_0(I') \]
be the index maps associated with both rows. Then the diagram
\[ \begin{tikzcd}
K_1(B) \rar{\delta_1} \dar{K_1(\beta)} & K_0(I) \dar{K_0(\gamma)} \\
K_1(B') \rar{\delta_1'} & K_0(I')
\end{tikzcd} \qquad \text{commutes.} \]
\end{proposition}

\subsection{Higher $K$-groups}
\begin{proposition}
There is a natural isomorphism between $K_1(A)$ and $K_0(SA)$.
\end{proposition}
\begin{proof}
From the short exact sequence, \ref{exactSequenceSuspensionCone}
\[ \begin{tikzcd}
 0 \rar & SA \rar & CA \rar & A \rar & 0.
\end{tikzcd} \]
we get the long exact sequence, ref TODO.
\[ \begin{tikzcd}
K_1(SA) \rar & K_1(CA) \rar & K_1(A) \rar & K_0(SA) \rar & K_0(CA) \rar & K_0(A)
\end{tikzcd}. \]
Because $CA$ is contractible, we have
\[ \begin{tikzcd}
K_1(SA) \rar & 0 \rar & K_1(A) \rar & K_0(SA) \rar & 0 \rar & K_0(A)
\end{tikzcd}. \]
By exactness this gives $K_1(A) \cong K_0(SA)$. The naturality is given by the naturality of the index map, \ref{naturalityIndexMap}.
\end{proof}

\subsection{Bott periodicity}
\begin{theorem}[Bott periodicity]
The functors $K_0$ and $K_1\circ S$ are naturally isomorphic.
\end{theorem}
\subsection{The six-term exact sequence}

\chapter{$K$-theory for graded $C^*$-algebras}
\section{Van Daele's picture}

\section{Karoubi's picture}
\begin{definition}
Let $A$ be a graded $C^*$-algebra. 
\end{definition}


\chapter{$K$-theory for group $C^*$-algebras}