% blackboard typefaces by Anthony Phan.
% file: mbtest.tex (testfile)
% last modification: 10.08.2001.

\magnification=\magstep1
\overfullrule=0pt
\input testmac.tex

\begintitle
\title{Mbbxxx series}
\subtitle{Informations}
\author{Anthony Phan}
\endtitle

\nonfrenchspacing

The {\it Mbboard}\/ series are quite good quality fonts primarily intended
to be used as blackboard bold fonts in mathematical texts typesetted
in \TeX\ with Computer Modern as main fonts' set. Their design comes
from printed mathematical books and articles, and, thus, from
some commercial fonts. One of the main contribution of the
mbboard-series lies in its completeness as one could se further on,
completeness due to their author's work only.

These fonts have been designed in such a way that they can be used
as text fonts, even if such texts would be hardly readable.
This also means that even if they are mostly intended to be used
in mathematical formulae, they are not ``mathematical fonts''
as understood in the MetaFont experts' community. I believe that
it doesn't matter to much  since these are upright fonts (in some
cases, one would have to insert some negative thin space in subscrits).

The current distribution of these fonts and \TeX's packages
is the {\it mbboard0.4}\/ which appears to be quite good and
stable. The two supported subseries are {\it mbbXX}\/ and
{\it mbbxXX}\/ where {\it XX}\/ is the point size. Other
subseries should be seen as oddities and, then, unsupported.
The author also claims no responsability in translations
of these fonts into other formats (type 1, etc.).

\section{Progress}

\subsection{Mbboard0.0, january 2000}
Space and time initial distribution.
If most glyphs look correct, kerning
properties are to be checked. The
dagesh marking is to be changed.

\subsection{Mbboard0.1, april 2000}
There have been changes in ligtables and designs:

\item{$\bullet$}
There is no more automatic c--t nor s--t
ligature in the basic encoding.
The corresponding glyphs have to
be accessed directly with \cs{\ctlig}
or \cs{\stlig} in text mode.

\item{$\bullet$}
There is now a dagesh sign such that
this sign plus a Hebrew letter
give the Hebrew letter with dagesh
by ligaturing system. Otherwise,
it does nothing. Its location
on mbb's encoding is the former
shereq sign's location.
This last glyph has been removed
from the basic encoding.

\item{$\bullet$}
Some vertical overshots have been corrected.

\item{$\bullet$}
The Roman shaped `a' and `3' have been slightly modified.

\item{$\bullet$}
Dorian pi has been rewritten.

\item{$\bullet$}
Kerning properties are still to
be checked.

\subsection{Mbboard0.2, december 2000}
Just code-check before the first ``expected
to be stable''-distribution.

\subsection{Mbboard0.3, june 2001}
Input files' names have been changed in order
to avoid conflicts with other source files---such
as the bbold's ones (Alan Jeffrey's fonts).

Roundish bowls are experimented. It will lead
to a simpler code: there will be less explicit
intersection control sequences.

\subsection{Mbboard0.4, october 2001}
Some control sequences' names or variables' names
have been changed. Roundish bowls have been (not fully)
implemented. Some important changes have been done:

\item{$\bullet$}
The width of `lambda' is now equal to the width of `A'.
(so that metrics have changed!)

\item{$\bullet$}
Percent sign and related ones have been corrected.
(someone could have told me of that error!)

\item{$\bullet$}
The letter `e' and related ones have been sharpen;
`xi', `zeta' have been slightly changed\dots\ and so on.

\item{$\bullet$} 17pt size has been introduced.

\item{$\bullet$} A whole bold extended font has been created.

\item{$\bullet$}
Kerning properties are still to be checked.

\subsection{Mbboard0.5, october 2001}
The main change is my trial to get rid of every {\tt square\_end}
control sequence. But it alters design quite deeply. So design
itself is at least slightly changed. Some metrics have changed
too just because I could not remember why I've made some odd
choices. All  those changes can not be reported without making
a comparison between this version and the former one.

\item{$\bullet$}
The width of `cedilla' has changed, it's now very thin just
as `ogonek'. (So that metrics have changed!)

\item{$\bullet$}
Kerning properties are still to be checked.

\fuzzytext
\let\proofmode=!
\input mbboard.tex
\normaltext

\section{Different sizes}

\usuals{mbb17 }\usuals{mbbx17 }
\usuals{mbb12 }\usuals{mbbx12 }
\usuals{mbb10 }\usuals{mbbx10 }
\usuals{mbb9 }\usuals{mbbx9 }
\usuals{mbb8 }\usuals{mbbx8 }
\usuals{mbb7 }\usuals{mbbx7 }
\usuals{mbb6 }\usuals{mbbx6 }
\usuals{mbb5 }\usuals{mbbx5 }
%
% tests comparatifs
%
\compare{mbb10 }{cmr10 }
\compare{mbb10 }{cmbx10 }
\compare{mbb10 }{mbbx10 }
\compare{mbb10 }{cmss10 }
%\compare{mbb10 }{ptmr }
\docomparison{mbb10 }{msbm10 }from 65 to 90.
\docomparison{msbm10 }{ptmr }from 65 to 90.

\section{Greek letters}

$\bbalpha\alpha$
$\bbbeta\beta$
$\bbgamma\gamma$ $\bbdelta\delta$ $\bbepsilon\epsilon$
$\bbzeta\zeta$ $\bbeta\eta$ $\bbtheta\theta$
$\bbiota\iota$ $\bbkappa\kappa$ $\bblambda\lambda$
$\bbmu\mu$ $\bbnu\nu$ $\bbxi\xi$ $\bbomicron o$
$\bbpi\pi$ $\bbrho\rho$ $\bbsigma\sigma$ $\bbtau\tau$
$\bbupsilon\upsilon$ $\bbphi\phi$ $\bbchi\chi$ $\bbpsi\psi$
$\bbomega\omega$ $\bbvarepsilon\varepsilon$ $\bbvartheta\vartheta$
$\bbvarsigma\varsigma$ $\bbvarkappa$%\varkappa$
$\bbvarpi\varpi$
$\bbvarrho\varrho$ $\bbvarsigma\varsigma$ $\bbvarphi\varphi$
\par\let\machin=\rm
$\bbAlpha\machin A$ $\bbBeta\machin B$
$\bbGamma\machin\Gamma$
$\bbDelta\machin\Delta$ $\bbEpsilon\machin\rm E$ $\bbZeta\machin Z$
$\bbEta\machin H$ $\bbTheta\machin\Theta$ $\bbIota\machin I$
$\bbKappa\machin K$ $\bbLambda\machin\Lambda$ $\bbMu\machin M$
$\bbNu\machin N$ $\bbXi\machin\Xi$ $\bbOmicron\machin O$
$\bbPi\machin\Pi$ $\bbRho\machin P$ $\bbSigma\machin\Sigma$
$\bbTau\machin T$ $\bbUpsilon\machin\Upsilon$ $\bbPhi\machin\Phi$
$\bbChi\machin X$ $\bbPsi\machin\Psi$ $\bbOmega\machin\Omega$

\section{Accents}

{\mbb
\AA\aa
\c R\u R\=R\d R\.R\"R\`R\'R\^R\v R\~R\H R
\c a\u a\=a\d a\.a\"a\`a\'a\^a\v a\~a\H a
\c o\u o\=o\d o\.o\"o\`o\'o\^o\v o\~o\H o
\c e\u e\=e\d e\.e\"e\`e\'e\^e\v e\~e\H e
\c c}
\medbreak
{\rm\AA\aa
\c R\u R\=R\d R\.R\"R\`R\'R\^R\v R\~R\H R
\c a\u a\=a\d a\.a\"a\`a\'a\^a\v a\~a\H a
\c o\u o\=o\d o\.o\"o\`o\'o\^o\v o\~o\H o
\c e\u e\=e\d e\.e\"e\`e\'e\^e\v e\~e\H e
\c c}

%\section{Random test of gray}

%{\mbb \mixfrom 65 to 90.

%\medbreak

%\mixfrom 97 to 122.

%\medbreak

%\mixfrom  192 to 250.
%}

%\section{Kerning tables}

%{\mbb
%\kerningtable["41,"5A]["41,"5A]
%\kerningtable["41,"5A]["61,"7A]
%\kerningtable["61,"7A]["61,"7A]
%\kerningtable["61,"7A]["41,"5A]
%}

\section{Other basic control sequences}
The most important control sequence is \cs{\mathbb}
which has one argument and is similar to the
classical one. There is also \cs{\mbb}
which changes current font in text or math mode
to blackboard. Those two control sequences are available
in plain\TeX\ and La\TeX.
About text, usual \TeX's control sequences
or ligaturing mechanisms apply.
The c--t and s--t ligatures can be used via
\cs{\ctlig} and \cs{\stlig}
in text mode only, but switching to another
font may produce oddities.
French double quotes are obtained by `{\tt<<}'
and `{\tt>>}'. But french single quotes cannot
be obtained by `{\tt<}' nor `{\tt>}', they have
to be reached by their {\tt charcode}.

{\mbb << This was the noble\stlig\ Roman of them all\dots>>}

\section{The \cs{\bbdagesh} construction}
One of the most current diacritic marks in Hebrew is dagesh.
If Hebrew is quite rarely used in mathematical composition,
to our knowledge it appeared one time in a logic book with
Hebrew letters from aleph to gimmel with, or not, dagesh mark.
The fact that this mark cannot be applied by simply putting aside
lead pieces lead us to design particuliar characters for those
cases. The big question was how to switch from normal characters
to marked ones. The last solution is to use ligaturing properties
to make a dagesh mark meeting a Hebrew character becoming a dagesh
marked Hebrew character. This means that a dagesh sign had to be
designed, and that it had to have a nice behavior with other
characters. By now, we don't know if the solution that we retain
could be considered as a canonical one. It's just the one that
is built-in the current version of our fonts. Look at samples below: 
$$
\bbdagesh\bbaleph\bbdagesh\bbbeth\bbdagesh\bbgimmel
\bbdagesh\bbdalet\bbdagesh\mathbb R
\bbdagesh\mathbb U\bbdagesh\mathbb O
$$

\section{Mbb, the regular blackboard serie}

The following font is actually the only one that is fully supported.
This essentially means that evolution of metrical properties will be
based on them. Other series could get some improvement as this one
may do (I hope so), but that's not certain\dots

\UsualTest{mbb10}

\section{Mbbx, the regular blackboard bold extended serie}

\UsualTest{mbbx10}

\section{Mbbr, a variant that may look better for text (unsupported yet)}

\UsualTest{mbbr10}

\section{Mbbi, an italic version (unsupported yet)}

\UsualTest{mbbi10}

\section{Mbbsl, a slanted version (unsupported yet)}

\UsualTest{mbbsl10}

\section{Mbbcr, a T1-encoded version (unsupported yet)}

As one can see in the following table, the Euro sign has
been introduced in this T1-encoding. I just cannot
remember which symbol I've then removed.

\UsualTest{mbbcr10}

\section{Mbbgr, why not some Greek version? (unsupported yet)}

The following font consists only of a
fairly simple rewriting of some driver
file. Encoding has been inspired by
Silvio Levy's work and some foregoing
contributors. It is not
(at that time) supposed to be a complete
Greek font.

\UsualTest{mbbgr10}

\section{Mbbheb, what about a Hebrew version? (unsupported yet)}

The following font should not be taken too seriously.
There are at least three reasons for that.
The first one is that it seems to contain a lot of
diacritic marks, but the fact is that we don't know
how to deal with them. If they are to be handled by
a \cs{\accent} mecanism, their metrical design
should fit that. If they are not, what should be done?
Secondly, there is the great encoding matter. I can't remember
why I choosed the following encoding. It was inspired,
I think, by some comparisons with existing ones, expecially
with unicode specifications and with Yannis Haralambous' work
on tiqwah fonts. The last reason that comes to my mind
is that I know nothing about Hebrew and, furthermore,
about Hebrew typographical composition.
So, most of this serie is a first draft for an unexpectable
and unusefull Hebrew font.

\UsualTest{mbbheb10}
\bye


Il y a quelques ann\'ees, je d\'ecouvrais l'univers
de \TeX\ par contrainte. Jusqu'alors je m'\'etais
efforc\'e d'y \'echapper pensant que celui-ci me
serait imp\'en\'etrable. M\^u par je ne sais quel
enthousiasme, je finis par m'y mettre en abordant
La\TeX\ sur un Mac. Pas typographe pour deux sous,
ni informaticien pour une roupie, je dus faire face
\`a de nombreuses difficult\'es. Mon plus grand
succ\`es \`a l'\'epoque fut de d\'ecouvrir et de
pouvoit utiliser la fonte {\tt msym} qui r\'esidait
encore sur l'ordinateur dont je me servais. Celle-ci
comblait mes attentes quant \`a disposer de caract\`eres
ad\'equats pour repr\'esenter ---~entre autres~---
les ensembles fondamentaux. H\'elas, d'une plateforme
on doit passer \`a une autre simplement parce que
l'on change de bureau. J'ai alors red\'ecouvert (avec
r\'etiscence) les syst\`emes de type Unix et surtout
les distributions \TeX\ pour ceux-ci. Il n'y avait
plus de {\tt msym} mais seulement un {\tt msbm} qui ne
me plaisait pas. De nombreuses recherches sur le web
se seront sold\'ees par la conviction qu'il fallait
recoder ce qui me faisait d\'efaut. J'ai alors
emprunt\'e une premi\`ere fois le {\tt MetaFontbook}
et j'ai commenc\'e par les premiers exercices\dots
