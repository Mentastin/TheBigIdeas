\section{Curves and parametrisations in Euclidean space}
\subsection{Definition of curves in $\mathbb{E}^n$}
A curve can be describes with a mapping of the form
\[ \gamma: I\subseteq \R \to \mathbb{E}^n: t \mapsto \gamma(t) = (\gamma_1(t), \ldots, \gamma_n(t)) \]
where $I$ is an interval and each of the components $\gamma_i(t)$ are real functions.

We call the curve described by $\gamma$ \udef{differentiable} if each component is infinitely differentiable.

We call $t$ the \udef{parameter} of the curve and $\gamma$ is called the \udef{parametrisation} of the curve. The parametrisation contains information not only about the shape of the curve, but also about how it is traversed. We will often simply use the word curve when we mean parametrisation.

\begin{example}
\begin{itemize}
\item A line is a type of curve and can written as
\[ \gamma: \R \to \mathbb{E}^n: t \mapsto p + tv \]
where $p\in \mathbb{E}^n$ is a point and $v\in \R^n$ is a vector.
\item A circle is a curve and can be written as
\[ \gamma: [0,2\pi]\to \mathbb{E}^2: t\mapsto (m_1 + R\cos t, m_2 + R\sin t) \]
where $m = (m_1,m_2) \in \mathbb{E}^2$ is a point and $R$ is a positive number called the radius.
\item A helix is a curve and can be written as
\[ \gamma: \R\to \mathbb{E}^3: t\mapsto (a\cos t, a\sin t, bt) \]
where $a$ and $b$ are real numbers.
\end{itemize}
\end{example}

\begin{definition}
Let $\gamma: I \to \mathbb{E}^n$ be a curve. A \udef{vector field along $\gamma$} is a map of the form
\[ Y: I \to T\mathbb{E}^n: t\mapsto Y(t) \in T_{\gamma(t)}\mathbb{E}^n \]
\end{definition}

TODO specify components after geometry!!!!

\subsection{Velocity and arc length}
We define the \udef{velocity vector field along the curve} as the map
\[ \vec{\gamma'}: I \to T\mathbb{E}^n: t \mapsto \vec{\gamma'(t)}\equiv (\gamma'_1(t),\ldots, \gamma'_n(t))_{\gamma(t)} \in T_{\gamma(t)}\mathbb{E}^n \]
where $\vec{\gamma_i'}(t)$ means the derivative of $\gamma_i(t)$.

The \udef{speed} of $\gamma$ can then be defined as the function
\[ v: I \to \R: t \mapsto v(t) \equiv \lVert\vec{\gamma'}(t)\lVert \]
and for $a,b \in I$ with $a\leq b$, we call
\[ \int^b_a v(t) \diff{t} = \int^b_a \lVert \vec{\gamma'}(t) \lVert \diff{t} \]
the \udef{length} of the stretch of $\gamma$ between $\gamma(a)$ and $\gamma(b)$. This corresponds to our intuitive notion of length of a curve.

\begin{note}
Another way to define the length of a curve, is by dividing the interval $[a,b]$ into $k$ sections, each of length $\Delta$. Thus we can write
\[ a = t_0 < t_1 < \ldots < t_{k-1} < t_k = b \qquad \text{with} \; t_i - t_{i-1} = \Delta. \]
This defines a broken line with length
\[ \sum^k_{i=1} \lVert \gamma(t_i) - \gamma(t_{i-1})\lVert  = \sum^{k-1}_{i=0} \lVert (\gamma(t_i+ \Delta) - \gamma(t_{i})) / \Delta \lVert \Delta. \]

We could define the length of the curve as the length of such a broken line in the limit of $k \to \infty$ (which also means that $\Delta$ goes to $0$). So
\begin{align*}
\text{length} &= \lim_{k\to\infty, \Delta\to 0} \sum^{k-1}_{i=0} \lVert (\gamma(t_i+ \Delta) - \gamma(t_{i})) / \Delta \lVert \Delta \\
&= \lim_{k\to\infty, \Delta\to 0}\sum^{k-1}_{i=0} \lVert (\gamma'(t_i)) \lVert \Delta \\
&= \int^b_a \lVert \gamma'(u) \lVert \diff{u}
\end{align*}
which is the definition we gave before.
\end{note}

We can also define the \udef{arc length} as the function
\[ s: I \to \R: t\mapsto s(t) \equiv \int_a^t v(u) \diff{u} = \int_a^t \lVert \vec{\gamma'}(u)\lVert \diff{u} \]
for a given $a\in I$. This is quite simply the length of the curve between $\gamma(a)$ and $\gamma(t)$, if $t\geq a$ and minus the length otherwise.

\begin{lemma} \label{convexityArcLength}
Let $I\to \mathbb{E}^n$ be a curve and $t,t'\in I$. Then
\[ \norm{\gamma(t')-\gamma(t)} \leq s(t') - s(t). \]
\end{lemma}
\begin{proof}
Triangle inequality. TODO
\end{proof}

\subsection{Tangent vectors}
A \udef{tangent line} to the curve $\gamma$ in $t_0$ is a line through the point $\gamma(t_0)$ in the direction of $\vec{\gamma'}(t_0)$. This is obviously only defined if $\vec{\gamma'}(t_0) \neq 0$.

Any multiple of $\vec{\gamma'}(t_0)$ in $T_{\gamma(t_0)}\mathbb{E}^n$ is called a \udef{tangent vector} in $\gamma(t_0)$. In particular a tangent vector with norm one is called a \udef{unit tangent vector}.

\begin{note}
With the right assumptions of differentiability etc. it is possible to write down a Taylor expansion of the curve $\gamma$.
\[ \gamma(t) = \gamma(t_0) + (t-t_0)\vec{\gamma'}(t_0) + \ldots + \frac{1}{k!}(t-t_0)^k\vec{\gamma^{(k)}}(t_0) + (t-t_0)^{k+1}R_{k+1}(t) \]
We recognise the expression for the tangent line as the first order approximation
\[ \gamma(t) \approx \gamma(t_0) + (t-t_0)\vec{\gamma'}(t_0). \]
\end{note}

\subsection{Reparametrisations and arc length parametrisations}
Two different parametrisations may look the same when drawn in space.

\begin{definition}
Let $I, \tilde{I} \subseteq \R$ be intervals and $\gamma: I \to \mathbb{E}^n$ a curve.
If $h: \tilde{I} \to I$ is a diffeomorphism, then 
\[ \beta \equiv \gamma \circ h: \tilde{I} \to \mathbb{E}^n \]
is a curve \textit{with the same image as $\gamma$} (i.e. it looks the same in space). We call $\beta$ a \udef{reparametrisation} of $\gamma$.
\end{definition}
Because
\[ \vec{\beta'}(t) = (\gamma\circ h)'(t) = \vec{\gamma}'(h(t))h'(t), \]
$\vec{\beta'}(t)$ and $\vec{\gamma'}(h(t))$ are proportional to each other and thus the tangent lines are the same. Also
\[ v_\beta = |h'|(v_\gamma\circ h). \]

Because the arc length is also a geometric quantity, we would expect it to be the same for both parametrisations. Actually it turns out to be the same up to the sign, because the new parametrisation may traverse the curve in the opposite direction.
\[ s_\beta(t) = \pm s_\gamma(h(t)) \]

\subsubsection{Arc length parametrisation}
We would now like to find a reparametrisation such that the speed is always unity (i.e. one). This turns out to always be possible is the curve is regular.
\begin{definition}
A curve is called \udef{regular} if $v(t) > 0$ for all $t$.
\end{definition}

If the curve is regular, the the arc length is a diffeomorfism, as is it's inverse. Using the inverse in the place of the diffeomorphism $h$, we get exactly the reparametrisation we were looking for.

It turns out that this reparametrisation is relatively unique: If $\beta_1$ and $\beta_2$ are reparametrisations of the same curve, both with speed 1, then $\beta_1(t) = \beta_2(\pm t + c)$, for a constant $c\in\R$. In other words, if we want a reparametrisation with speed 1 everywhere, then that reparametrisation is unique once we have chosen a direction and origin.

\begin{definition}
We call a curve with speed 1 everywhere an \udef{arc length parametrisation}.
\end{definition}

Let $\beta$ be an arc length parametrisation, then the arc length is
\[ s_\beta(t) = \int_a^t \diff{u} = t - a \]

\section{Curves on a sphere}
\begin{proposition}
Let $\gamma: I \to S^n$ be a curve on the unit sphere. Then for all $t, \delta$ such that $t,t+\delta\in I$:
\begin{align*}
|\inner{\gamma(t)| \gamma(t+\delta)}|^2 &= 1 - \norm{P_{\gamma(t)^\perp}\gamma(t+\delta)}\gamma(t)}^2 \\
&\leq 1 - \norm{\gamma(t+\delta) - \gamma(t)}^2 \\
&\leq 1 - (s(t+\delta)-s(t))^2,
\end{align*}
where $P_{\gamma(t)^\perp}$ is the orthogonal projection on the subspace perpendicular to $\Span(\gamma(t))$ and $s(t)$ is the path length.
\end{proposition}
\begin{proof}
Once we have proven the equality, the first inequality follows because $P_{\gamma(t)^\perp}\gamma(t+\delta) = P_{\gamma(t)^\perp}\gamma(t+\delta) -  P_{\gamma(t)^\perp$ and $\norm{P_{\gamma(t)^\perp}} \leq 1$. The second follows by \ref{convexityArcLength}.

Now we prove the equality. Set $\Delta \gamma = \gamma(t+\delta) - \gamma(t)$. Using the fact that $\norm{\gamma(t)} = 1 = \norm{\gamma(t+\delta)}$, we have
\[ 1 = \norm{\gamma(t+\delta)} = \inner{\gamma(t) + \Delta\gamma| \gamma(t) + \Delta\gamma} = 1 + \inner{\gamma(t)| \Delta\gamma} + \inner{\Delta \gamma| \gamma(t)} + \inner{\Delta \gamma| \Delta \gamma}, \]
which implies that $\inner{\gamma(t)| \Delta\gamma} + \inner{\Delta \gamma| \gamma(t)} = - \inner{\Delta \gamma| \Delta \gamma}$.
Using this, we can expand
\begin{align*}
|\inner{\gamma(t)| \gamma(t+\delta)}|^2 &= \inner{\gamma(t)| \gamma(t)+\Delta \gamma}\inner{\gamma(t)+\Delta \gamma| \gamma(t)} \\
&= 1 + \inner{\gamma(t)| \Delta\gamma} + \inner{\Delta \gamma| \gamma(t)} + \inner{\Delta \gamma| \gamma(t)}\inner{\gamma(t)| \Delta \gamma} \\
&= 1 - \inner{\Delta \gamma| \Delta \gamma} + \inner{\Delta \gamma| \gamma(t)}\inner{\gamma(t)| \Delta \gamma} \\
&= 1 - \bra{\Delta\gamma}(\id - \ketbra{\gamma(t)}{\gamma(t)})\ket{\Delta \gamma} = 1 - \norm{P_{\gamma(t)^\perp}\Delta\gamma}^2.
\end{align*}
TODO ref diad projector lemma.
\end{proof}

\section{Curves in flat Euclidean space}
\subsection{Frenet frame for regular curves}
Given a curve $\gamma: I \to \mathbb{E}^2$, we wish to introduce a useful basis for $T_{\gamma(t)}\mathbb{E}^2$. By useful, we mean that it can serve as a natural reference frame for a particle traveling along the curve. The Frenet frame also leads to a natural definition of the curvature of a curve (as well as torsion in 3 dimensional space).

A first obvious vector to introduce in our basis is the unit tangent vector, which we will call $\vec{T}$. Then there is only one way to extend this to a positively oriented orthonormal basis, which we call the normal unit vector $\vec{N}$. If the unit tangent vector is given by $\vec{T} = \left(T_1, T_2\right)$, then\footnote{Effectively we are applying the two-dimensional complex structure to $\vec{T}$. Where a \udef{linear complex structure} is a linear transformation that squares to minus identity.}
\[ \vec{N} = \left(-T_2, T_1\right) \]
\begin{definition}
In two dimensions, the \udef{Frenet frame}, for a curve $\gamma(t)$ at a point $t_0$, is given by $(\vec{T}, \vec{N})$ where
\begin{itemize}
\item $\vec{T}$ is the unit tangent vector to $\gamma(t)$ at $t_0$;
\[ \vec{T} = \frac{\vec{\gamma'}}{\lVert \vec{\gamma'} \lVert} \]
\item $\vec{N}$ is the unique vector such that $(\vec{T}, \vec{N})$ is a positively oriented orthonormal basis, called the \udef{normal unit vector}.
\end{itemize}
\end{definition}

From $(\vec{T}\cdot \vec{T}) = 1$, we see that $(\vec{T}\cdot \vec{T})' = 0$. Using the product rule, we see that it is also equal to $(\vec{T}\cdot \vec{T})' = \vec{T}'\cdot\vec{T} + \vec{T}\cdot\vec{T}' = 2(\vec{T}\cdot \vec{T}')$. Thus we see that $\vec{T}\cdot \vec{T}' = 0$, meaning that the derivative of $\vec{T}$ must be perpendicular to $\vec{T}$. This is true for any unit vector. The unit normal vector is also perpendicular to $\vec{T}$, so we can find a $k \in \R$ such that $\vec{T}' = k \vec{N}$. Because $\vec{N}$ is a unit vector, $k$ is given by $k = \vec{T}'\cdot \vec{N}$.

Following a similar line of reasoning, we see that $\vec{N}'$ must be proportional to $\vec{T}$, with a factor of proportionality equal to $\vec{T}\cdot \vec{N}'$. From
\[ (\vec{T}\cdot \vec{N})' = \vec{T}'\cdot \vec{N} + \vec{T}\cdot \vec{N}' = 0 \]
we see that the factor must equals $-k$.

The quantity $k$ correspond to our intuitive notion of curvature (as we will see), but \textbf{only if the curve is arc length parametrised}. We also want the curvature to be a purely geometric quantity that does not depend on which parametrisation we choose (as $k$ does). So, for regular curves, it makes sense to define the curvature $\kappa$ as the factor $\vec{T}'\cdot \vec{N}$ for an arc length parametrisation of the curve. This uniquely determines the curvature $\kappa$ of the curve up to the sign, which depends on the direction of traversal.

Now for a general regular curve with arc length $s(t)$, we denote the quantities associated with an arc length parametrisation with a tilde:
\[ \begin{cases}
\vec{T}(t) = \vec{\tilde{T}}(s(t)) \\
\vec{N}(t) = \vec{\tilde{N}}(s(t)) \\
\kappa(t) = \tilde{k}(s(t))
\end{cases} \]
We can calculate
\[ \vec{T}'(t) = (\vec{\tilde{T}}(s(t)))' = \vec{\tilde{T}}'(s(t))s'(t) = v(t)\tilde{k}(s(t))\vec{\tilde{N}}(s(t)) = v(t)\kappa(t)N(t) \]

We now have a general expression for $\kappa$:
\[ \kappa = \frac{\vec{T}'\cdot \vec{N}}{v} \]

Summarising
\begin{eigenschap}
The Frenet-Serret formulae are
\[ \begin{cases}
\vec{T}' = \kappa v \vec{N} \\
\vec{N}' = -\kappa v \vec{T}
\end{cases} \]
where $\kappa = \frac{\vec{T}'\cdot \vec{N}}{v}$ is called the \udef{(oriented) curvature} and $v$ is the speed at which the curve is traversed in that point; $v=1$ for arc parametrised curves.
\end{eigenschap}

The curvature $\kappa$ of any regular curve $\gamma$ can be calculated directly from the first and second derivatives of the curve:
\[ \kappa = \frac{\lVert \vec{\gamma'} \times \vec{\gamma''}\lVert}{\lVert\vec{\gamma'}\lVert^3} \]
Or
\[ \kappa = \frac{| \vec{\gamma'} \quad \vec{\gamma''}|}{\lVert\vec{\gamma'}\lVert^3} \]
where $| \vec{\gamma'} \quad \vec{\gamma''}|$ is the determinant with $\vec{\gamma'},\vec{\gamma''}$ seen as column vectors.

The associated collection $\vec{T}, \vec{N}, \kappa$ is called the \udef{Frenet-Serret apparatus}. In three dimensions this also includes the binormal unit vector $\vec{B}$ and torsion $\tau$.

\subsection{Curvature}
First some examples to support the idea that our definition of curvature makes sense.
\begin{example}
\begin{itemize}
\item Let $\gamma$ be a straight line with arc length parametrisation
\[ \gamma: \R \to \mathbb{E}^2: s\mapsto p + s \vec{s} \]
where $\lVert \vec{v}\lVert = 1$. Then $\vec{T}(s) = \vec{v}$ for all $s$ and consequently $\vec{T}' = 0$. Thus a straight line has zero curvature.
\item It can be proven that any curve with zero curvature is a straight line.
\item Let $\gamma$ be a circle centered at $m= (m_1,m_2)$ with radius $R$ and arc length parametrisation
\[ \gamma: \R \to \mathbb{E}^2: s\mapsto \left(m_1 + R\cos \left(\frac{s}{R}\right), m_2 + R\sin \left(\frac{s}{R}\right)\right). \]
Then the Frenet frame, for all $s\in\R$, is given by
\[ \begin{cases}
\vec{T}(s) = \left(-\sin \left(\frac{s}{R}\right), \cos \left(\frac{s}{R}\right)\right) \\
\vec{N}(s) = \left(-\cos \left(\frac{s}{R}\right), -\sin \left(\frac{s}{R}\right)\right)
\end{cases} \]
Thus
\[ \vec{T}'(s) = \left(- \frac{1}{R}\cos \left(\frac{s}{R}\right), - \frac{1}{R}\sin \left(\frac{s}{R}\right)\right) = \frac{1}{R} \vec{N}(s), \]
meaning that $\kappa(s) = \frac{1}{R}$ for all $s\in \R$. A circle with radius $R$ has a constant curvature $1/R$.
\item Every curve with constant, non-zero, curvature is a (part of a) circle with radius $1/|\kappa|$.
\end{itemize}
\end{example}
\subsubsection{Osculating parabola}
Let $\beta$ be an arc length parametrised curve. Then we can write the Taylor expansion:
\begin{align*}
\beta(s) &= \beta(s_0) + (s-s_0)\vec{\beta'}(s_0) + \frac{1}{2}(s-s_0)^2\vec{\beta''}(s_0) + (s-s_0)^3R_3(s) \\
&= \beta(s_0) + (s-s_0)\vec{T}(s_0) + \frac{1}{2}(s-s_0)^2\kappa(s_0)\vec{N}(s_0) + \ldots
\end{align*}
The second order approximation is a parabola, called the \udef{osculating parabola} (which comes from the latin word osculans meaning kissing). Intuitively it may be thought of as the parabola with it's top in $\beta(s_0)$ that most closely matches the curve.

TODO Figure. 

If $\kappa$ is positive, $\beta$ curves towards $\vec{N}$ and $\beta$ locally lies on the same side of $\vec{T}$ as $\vec{N}$. If $\kappa$ is negative, $\beta$ curves away from $\vec{N}$ and $\beta$ locally lies on the opposite side of $\vec{T}$ from $\vec{N}$.

\subsubsection{Osculating circle}
Again let $\beta$ be an arc length parametrised curve.
\begin{definition}
\begin{itemize}
\item We call $1/|\kappa(s_0)|$ the \udef{radius of curvature} of the curve at $s_0$.
\item We call the point
\[ m \equiv \beta(s_0) + (1/\kappa(s_0))\vec{N}(s_0) \]
the \udef{centre of curvature} of the curve at $s_0$.
\item The circle which has as its centre in the centre of curvature and a radius that is the same as the radius of curvature, is called the \udef{osculating circle} of the curve at $s_0$.
\end{itemize}
\end{definition}
The osculating circle may be parametrised as
\[ c(s) = m + R\cos \left(\frac{s-s_0}{R}\right)(- \vec{N}(s_0)) + R\sin \left(\frac{s-s_0}{R}\right)\vec{T}(s_0) \]
where $m$ is the centre of curvature and $R = \frac{1}{|\kappa(s_0)|}$ is the radius of curvature.

We can easily calculate that
\[ \begin{cases}
c(s_0) = \beta(s_0) \\
\vec{c'}(s_0) = \vec{\beta'}(s_0) \\
\vec{c''}(s_0) = \vec{\beta''}(s_0).
\end{cases} \]
We say that the osculating circle approximates the curve to second order. It is the only circle that does that.

The osculating circle gives us quite a useful intuitive interpretation of the curvature:
it is the inverse of the radius of the ``best fitting'' circle.

\subsection{Intrinsic equations}
An \udef{intrinsic equation} of a curve is an equation that defines the curve using a relation between geometrical properties that are intrinsic to the curve and do not depend on the exact parametrisation.

Examples of such intrinsic quantities are: arc length $s$, tangential angle $\theta$ and curvature $\kappa$.
\subsubsection{Tangential angle}
The \udef{tangential angle} $\theta$ is the angle of the unit tangent vector $\vec{T}$ with the line through points $(0,0)$ and $(0,1)$ in $\mathbb{E}^2$ (i.e. the ``$x$-axis''). It is a function $\theta: I \to \R$ such that
\[ \vec{T}(s) = (\cos(\theta(s)), \sin(\theta(s))). \]
The normal unit vector is then given by
\[ \vec{N}(s) = (-\sin(\theta(s)), \cos(\theta(s))). \]
From $\vec{T}' = (-\sin(\theta)\theta', \cos(\theta)\theta') = \theta'\vec{N}$, we see that
\[ \kappa(s) = \theta'(s). \]

\subsubsection{Whewell equations}
Suppose we have an equation for the tangential angle of a curve in function of the arc length ($\theta(s) = \ldots$). We would now like to find a parametrisation for that curve.

Consider the following curve:
\[ \beta(s) = \left(\int_{s_0}^s\cos\theta(u) \diff{u},\, \int^s_{s_0}\sin\theta(u) \diff{u}\right) \]
Then $\vec{\beta'}(s) = (\cos(\theta(s)),\sin(\theta(s)))$ for all $s\in I$, so $\beta$ is arc length parametrised and tangential angle $\theta$ at all points. So $\beta$ is exactly the parametrisation we were looking for.

\begin{example}
\begin{itemize}
\item Straight lines are determined by $\theta = c$ for some constant $c\in\R$.
\item Circles are determined by $\theta(s) = \frac{s}{R}$ where $R\in \R$ is the radius.
\item Catenary curves are determined by $\theta = \arctan \left(\frac{s}{R}\right)$.
\end{itemize}
\end{example}
\subsubsection{Cesàro equations}
Now suppose we have an equation for the curvature of a curve in function of the arc length ($\kappa(s) = \ldots$).

Because $\theta' = \kappa$, the Cesàro equation of a curve can be obtained from the Whewell equation by differentiating it.

The parametrisation is then given by
\[ \beta(s) = \left(\int_{s_0}^s\cos\left(\int_{s_0}^u\kappa(t)\diff{t}\right) \diff{u},\, \int^s_{s_0}\sin\left(\int_{s_0}^u\kappa(t)\diff{t}\right) \diff{u}\right) \]

\begin{example}
\begin{itemize}
\item Line: $\kappa = 0$
\item Circle: $\kappa = 1/R$
\item Logarithmic spiral: $\kappa = C / s$
\item Circle involute: $\kappa = C / \sqrt{s}$
\item Cornu spiral (or clothoid): $\kappa = Cs$
\item Catenary: $\kappa = \frac{a}{s^2 + a^2}$
\end{itemize}
\end{example}

\subsection{Global properties of flat curves}
So far we have mainly looked at \textit{local} properties of curves, like curvature. Now we take a look at some global properties.

\begin{definition}
We call a curve $\gamma:\R\to\mathbb{E}^n$ \udef{closed} if there is a strictly positive number $\omega \in R_0^+$ such that $\gamma(t+\omega) = \gamma(t)$ for all $t\in \R$. We call $\omega$ the \udef{period} of $\gamma$.

If $\omega$ is a period of a curve, then any multiple of $\omega$ is also a period. We call the smallest period the \udef{real period} $\omega$.
\end{definition}

\begin{definition}
A \udef{simple} curve is a curve that does not cross itself. 
\end{definition}
A closed curve $\gamma$ with real period $\omega$ is said to be simple if the restriction $\gamma|_{[0,\omega[}: [0,\omega[ \to \mathbb{E}^n $ is injective.

\begin{definition}
Let $\beta:\R \to \mathbb{E}^2$ be an arc parametrised, closed curve with period $L$.
\begin{itemize}
\item We call
\[ \int_0^L \kappa(s) \diff{s} \]
the \udef{total curvature} of $\beta$.
\item We define the \udef{rotation index} $i_\beta$ of $\beta$ as
\[ i_\beta \equiv \frac{1}{2\pi}\left(\theta(L) - \theta(0)\right) \]
\end{itemize}
\end{definition}
The rotation index must always be an integer, because $\vec{T}(0)$ must be the same as $\vec{T}(L)$. So the angle tangential angles must be the same, modulus $2\pi$.

\begin{eigenschap}
The total curvature is related to the rotation index:
\[ \int_0^L\kappa(s)\diff{s} = 2\pi i_\beta \]
\end{eigenschap}

Finally we formulate three theorems for simple, closed, flat curves (also called \udef{Jordan curves}).
\begin{eigenschap}
\begin{itemize}
\item \textit{Umlaufsatz}. The rotation index of a simple closed curve is $1$ or $-1$.
\item \textit{Jordan's theorem}. A simple closed curve divides the plain onto two parts: a bounded interior and an unbounded exterior.
\item \textit{Isoperimetric inequality}. The surface area of the bounded interior $A$ satisfies the inequality
\[ L^2 \geq 4\pi A \]
where $L$ is a period. This is an equality only if the curve is a circle (and $L$ the real period).
\end{itemize}
\end{eigenschap}

\section{Curves in three dimensional Euclidean space}
\subsection{The Frenet frame}
Let $\gamma$ be a regular curve. Again we define $\vec{T}$ as the unit tangent vector. For the same reason as before, $\vec{T'}$ is perpendicular to $\vec{T}$, so we can use that to define $\vec{N}$. In three dimensions we need a third basis vector. There is only one vector that can be added to the orthonormal vectors $\vec{T}$ and $\vec{N}$ to make \ueig{positively oriented orthonormal basis} of $T_{\gamma(t)}\mathbb{E}^3$, namely $\vec{B} = \vec{T}\times \vec{N}$.
\begin{definition}
In three dimensions, the \udef{Frenet frame}, for a curve $\gamma(t)$ at a point $t_0$, is given by $(\vec{T}, \vec{N}, \vec{B})$ where
\begin{itemize}
\item $\vec{T}$ is the \udef{tangent unit vector} to $\gamma(t)$ at $t_0$
\[ \vec{T} \equiv \frac{\vec{\gamma'}}{\lVert \vec{\gamma'} \lVert} \]
\item $\vec{N}$ is the \udef{normal unit vector}
\[ \vec{N} \equiv \frac{\vec{T'}}{\lVert \vec{T'} \lVert} \]
\item $\vec{B}$ is the \udef{binormal unit vector}
\[ \vec{B} \equiv \vec{T}\times \vec{N} \]
\end{itemize}
\end{definition}

From this definition it is obvious that $\vec{T}'$ is a multiple of $\vec{N}$. As in the one dimensional case, the factor connecting ($k = \lVert \vec{T'} \lVert$) them has geometric significance, so long as the speed is fixed. The big difference is that now the factor $k$ is always positive.

Again we define the curvature $\kappa$ of a curve as the factor $k$ for an arc length parametrisation of the curve. As in the two dimensional case (and following the same reasoning), we have for general regular curves
\[ \vec{T}' = k \vec{N} = v \kappa \vec{N} \]

We now prove that $\vec{B}'= l \vec{N}$ for some factor $l(t)\in \R$. First we write an orthonormal expansion of $\vec{B}'$
\[ \vec{B}' = (\vec{B}'\cdot \vec{T})\vec{T} + (\vec{B}'\cdot \vec{N})\vec{N} + (\vec{B}'\cdot \vec{B})\vec{B}. \]
Now $\vec{B}'\cdot \vec{B}=0$, which we have already shown to be true in the previous section because $\vec{B}$ is a unit vector (like $\vec{T}$). If we derive $\vec{B}\cdot \vec{T} = 0$, we get $\vec{B}'\cdot \vec{T} + \vec{B}\cdot \vec{T}' = \vec{B}'\cdot \vec{T} + k \vec{B}\cdot \vec{N} = \vec{B}'\cdot \vec{T} = 0$. Thus 
\[ \vec{B}' = (\vec{B}'\cdot \vec{N})\vec{N} = l \vec{N} \]

Again, to give $l = \vec{B}'\cdot \vec{N}$ geometric significance, we define the torsion $\tau$ as $-l$ for arc length parametrisations. The minus sign is a classical convention. The torsion \textit{can} be negative. Again, following the same reasoning as we have twice before, we get for general regular curves
\[ \vec{B}' = l \vec{N} = - v \tau \vec{N} \]

\begin{eigenschap}
The Frenet-Serret formulae are
\begin{alignat*}{4}
\vec{T}' &= & & & v\kappa&\vec{N} & & \\
\vec{N}' &= & -v\kappa&\vec{T} & & & +v\tau&\vec{B} \\
\vec{B}' &= & & & -v\tau&\vec{N} & &
\end{alignat*}
where $v = \lVert \vec{\gamma'} \lVert$ is the speed at which the curve is traversed in that point ($v=1$ for arc parametrised curves) and
\begin{itemize}
\item $\kappa = \frac{\vec{T}'\cdot \vec{N}}{v}$ is called the \udef{curvature}
\item $\tau = \frac{- \vec{B}'\cdot \vec{N}}{v}$ is called the \udef{torsion}
\end{itemize}
\end{eigenschap}
The only formula we have not yet proven is the second one. It can easily be seen to be correct if we take the orthonormal expansion $\vec{N}' = (\vec{N}'\cdot \vec{T})\vec{T} + (\vec{N}'\cdot \vec{N})\vec{N} + (\vec{N}'\cdot \vec{B})\vec{B}$ and calculate
\begin{align*}
\vec{N}'\cdot \vec{T} &= - \vec{N}\cdot \vec{T}' = -v\kappa \vec{N}\cdot \vec{N} = -v\kappa \\
\vec{N}'\cdot \vec{N} &= 0 \\
\vec{N}'\cdot \vec{B} &= - \vec{N}\cdot \vec{B}' = v\tau \vec{N}\cdot \vec{N} = v\tau
\end{align*}

The curvature $\kappa$ and torsion $\tau$ of any regular curve $\gamma$ can be calculated directly from the first, second and third derivatives of the curve. As in two dimensions, we have
\[ \kappa = \frac{\lVert \vec{\gamma'} \times \vec{\gamma''}\lVert}{\lVert\vec{\gamma'}\lVert^3} \]
For the torsion we have
\[ \tau = \frac{\vec{\gamma'}\times\vec{\gamma''}\cdot\vec{\gamma'''}}{\lVert\vec{\gamma'}\times\vec{\gamma''}\lVert^2} \qquad \text{or}\qquad \frac{|\vec{\gamma'}\quad\vec{\gamma''}\quad\vec{\gamma'''}|}{\lVert\vec{\gamma'}\times\vec{\gamma''}\lVert^2} \]
where $| \vec{\gamma'} \quad \vec{\gamma''}\quad \vec{\gamma'''}|$ is the determinant with $\vec{\gamma'},\vec{\gamma''}, \vec{\gamma'''}$ seen as column vectors.

The associated collection $\vec{T}, \vec{N}, \vec{B}, \kappa, \tau$ is called the \udef{Frenet-Serret apparatus}.

\subsubsection{Osculating plane}
We define the \udef{osculating plane} in each point as the plane that contains $\vec{T}$ and $\vec{N}$.

Because the Taylor expansion of an arc length parametrised curve $\beta$ is still given by
\begin{align*}
\beta(s) &= \beta(s_0) + (s-s_0)\vec{\beta'}(s_0) + \frac{1}{2}(s-s_0)^2\vec{\beta''}(s_0) + (s-s_0)^3R_3(s) \\
&= \beta(s_0) + (s-s_0)\vec{T}(s_0) + \frac{1}{2}(s-s_0)^2\kappa(s_0)\vec{N}(s_0) + \ldots
\end{align*}
the osculating plane can be seen as the tangent plane and it contains the osculating parabola.

If $\beta$ lies in a plane, then all osculating planes are equal to this plane. That means that $\vec{B}$ does not change, so $\vec{B}'= 0$ and $\tau = 0$. In fact we can state that for any regular curve $\gamma$ with curvature $\kappa > 0$, $\gamma$ lies in a plane if and only if $\tau = 0$.

If a space curve $\gamma$ lies in a plane, then the curvature $\kappa$ of the curve is the absolute value of the curvature of the curve seen as a planar curve in two dimensions.

\section{Surfaces in Euclidean space}
In this section we introduce some of the key concepts