\chapter{Combinatorics}

\url{https://www.cut-the-knot.org/}
\url{https://math.stackexchange.com/questions/785624/show-that-sum-limits-sigma-in-s-n-mboxnumber-of-fixed-points-of-si/785652#785652}

todo: inclusion-exclusion principle + move after analysis
\section{Permutations}
Permutation group $S_n$.

Factorials

\section{Combinations}

\[ \begin{pmatrix}
n \\ k
\end{pmatrix} = \frac{n!}{k!(n-k!)}\]


$(k,m)$ shuffle


\begin{lemma} \label{binomialCoefficientLemma}
Let $n,k\in \N$. Then
\begin{enumerate}
\item $\begin{pmatrix}
n \\ k
\end{pmatrix} = \begin{pmatrix}
n \\ n-k
\end{pmatrix}$;
\item $\begin{pmatrix}
n \\ k
\end{pmatrix} = \frac{n}{k}\begin{pmatrix}
n - 1 \\ k - 1
\end{pmatrix}$.
\end{enumerate}
\end{lemma}
\begin{proof}
(1) We calculate
\[ \begin{pmatrix}
n \\ k
\end{pmatrix} = \frac{n!}{k!(n-k)!} = \frac{n!}{(n-k)!k!} = \frac{n!}{(n-k)!\big(n-(n-k)\big)!} = \begin{pmatrix}
n \\ n-k
\end{pmatrix}. \]

(2) We calculate
\[ \begin{pmatrix}
n \\ k
\end{pmatrix} = \frac{n!}{k!(n-k)!} = \frac{n(n-1)!}{k(k-1)!(n-k)!} = \frac{n}{k}\frac{(n-1)!}{(k-1)!\big((n-1) - (k-1)\big)!} = \frac{n}{k}\begin{pmatrix}
n-1 \\ k-1
\end{pmatrix}. \]
\end{proof}

\begin{proposition}[Pascal's formula] \label{PascalsFormula}
Let $n,k\in \N$. Then
\[ \begin{pmatrix}
n-1 \\ k
\end{pmatrix} + \begin{pmatrix}
n-1 \\ k-1
\end{pmatrix} = \begin{pmatrix}
n \\ k
\end{pmatrix}. \]
\end{proposition}
There is an intuitive explanation: if we want to choose $k$ from $n$, then we can first isolate one of the $n$. Now in choosing $k$, we can either choose the isolated one or not. In the first case we need to choose $k-1$ from the remaining $n-1$, in the second case we need to choosed $k$ from the remaining $n-1$.
\begin{proof}
We can calculate
\begin{align*}
\begin{pmatrix}
n-1 \\ k
\end{pmatrix} + \begin{pmatrix}
n-1 \\ k-1
\end{pmatrix} &= \frac{(n-1)!}{k!(n-1-k)!} + \frac{(n-1)!}{(k-1)!(n-k)!} \\
&= (n-1)!\left(\frac{n-k}{k!(n-k)!} + \frac{k}{k!(n-k)!}\right) \\
&= (n-1)! \frac{n}{k!(n-k)!} \\
&= \frac{n!}{k!(n-k)!} = \begin{pmatrix}
n \\ k
\end{pmatrix}.
\end{align*}
\end{proof}

\begin{theorem}[Binomial identity] \label{binomialIdentity}
Let $n\in \N$. Then
\[ (x+y)^n = \sum_{k=0}^n \begin{pmatrix}
n \\ k
\end{pmatrix}x^k y^{n-k}. \]
\end{theorem}
\begin{proof}
We prove this by induction. For the base case $n=0$, we have
\[ (x+y)^0 = 1 = \begin{pmatrix}
0 \\ 0
\end{pmatrix} = \sum_{k=0}^0 \begin{pmatrix}
0 \\ k
\end{pmatrix}x^k y^{0-k}. \]
Now assume the induction hypothesis $(x+y)^{n-1} = \sum_{k=0}^{n-1} \begin{pmatrix}
n-1 \\ k
\end{pmatrix}x^k y^{n-1-k}$ holds. Then
\begin{align*}
(x+y)^n &= (x+y)^{n-1}(x+y) = (x+y)^{n-1}x + (x+y)^{n-1}y \\
&= \sum_{k=0}^{n-1} \begin{pmatrix}
n-1 \\ k
\end{pmatrix}x^{k+1} y^{n-1-k} + \sum_{k=0}^{n-1} \begin{pmatrix}
n-1 \\ k
\end{pmatrix}x^{k} y^{n-k} &\hspace{-22em}\text{(introduce induction hypothesis)}& \\
&= \sum_{k=1}^{n} \begin{pmatrix}
n-1 \\ k-1
\end{pmatrix}x^{k} y^{n-k} + \sum_{k=0}^{n-1} \begin{pmatrix}
n-1 \\ k
\end{pmatrix}x^{k} y^{n-k} &\hspace{-22em}\text{(relabel $k\to k+1$ in first sum)}& \\
&= \begin{pmatrix}
n-1 \\ n-1
\end{pmatrix}x^ny^{0} + \begin{pmatrix}
n-1 \\ 0
\end{pmatrix}x^0y^n + \sum_{k=1}^{n-1} \begin{pmatrix}
n-1 \\ k-1
\end{pmatrix}x^{k} y^{n-k} + \sum_{k=1}^{n-1} \begin{pmatrix}
n-1 \\ k
\end{pmatrix}x^{k} y^{n-k} \\
&&\hspace{-22em}\text{(isolate $k=n$ in first sum and $k=0$ in the second sum)}& \\
&= x^ny^{0} + x^0y^n + \sum_{k=1}^{n-1} \left(\begin{pmatrix}
n-1 \\ k-1
\end{pmatrix}x^{k} y^{n-k} + \begin{pmatrix}
n-1 \\ k
\end{pmatrix}x^{k} y^{n-k}\right) \\
&= x^ny^{0} + x^0y^n + \sum_{k=1}^{n-1} \left(\begin{pmatrix}
n-1 \\ k-1
\end{pmatrix} + \begin{pmatrix}
n-1 \\ k
\end{pmatrix}\right)x^{k} y^{n-k} &\hspace{-22em}\text{(Pascal's formula \ref{PascalsFormula})}& \\
&= x^ny^{0} + x^0y^n + \sum_{k=1}^{n-1} \begin{pmatrix}
n \\ k
\end{pmatrix}x^{k} y^{n-k} \\
&= \sum_{k=0}^{n} \begin{pmatrix}
n \\ k
\end{pmatrix}x^{k} y^{n-k}.
\end{align*}
\end{proof}
\begin{corollary} \label{consequencesBinomialIdentity}
Let $n\in \N$. Then
\begin{enumerate}
\item $\sum_{k=0}^n \begin{pmatrix}
n \\ k
\end{pmatrix} = 2^n$;
\item $\sum_{k=0}^n (-1)^k\begin{pmatrix}
n \\ k
\end{pmatrix} = 0$;
\item $\sum_{k=0}^n \begin{pmatrix}
n \\ k
\end{pmatrix}x^k = (1+x)^n$.
\end{enumerate}
\end{corollary}
\begin{proof}
(1) Set $x = 1 = y$.

(2) Set $x = -1$ and $y = 1$.

(3) Set $y=1$.
\end{proof}
\begin{corollary}[Multinomial identity] \label{multinomialIdentity}
Let $m,n\in \N$. Then
\[ \Big(\sum_{k=0}^m x_k\Big)^n = \sum_{k_0+\ldots + k_m = n}\begin{pmatrix}
n \\ k_0, \ldots, k_m
\end{pmatrix}x_0^{k_0}\cdot\ldots\cdot x_m^{k_m}, \]
where $\begin{pmatrix}
n \\ k_0, \ldots, k_m
\end{pmatrix} = \frac{n!}{k_0!\ldots k_m!}$.
\end{corollary}

\begin{lemma} \label{binomialCoefficientMultiplicationLemma}
Let $k,m,n\in \N$. Then
\[ \frac{n!}{m!(n-m-k)!} = \begin{pmatrix}
n \\ m
\end{pmatrix}\begin{pmatrix}
n-m \\ k
\end{pmatrix} = \begin{pmatrix}
n \\ k
\end{pmatrix}\begin{pmatrix}
n-k \\ m
\end{pmatrix} = \begin{pmatrix}
n \\ m+k
\end{pmatrix}\begin{pmatrix}
m+k \\ m
\end{pmatrix}. \]
\end{lemma}
\begin{proof}
We calculate
\begin{align*}
\frac{n!}{m!k!(n-m-k)!} &= \frac{n!}{m!(n-m)!}\frac{(n-m)!}{k!(n-m-k)!} \\
&= \begin{pmatrix}
n \\ m
\end{pmatrix}\begin{pmatrix}
n-m \\ k
\end{pmatrix} \\
\frac{n!}{m!k!(n-m-k)!} &= \frac{n!}{m!(n-k)!}\frac{(n-k)!}{k!(n-m-k)!} \\
&= \begin{pmatrix}
n \\ k
\end{pmatrix}\begin{pmatrix}
n-k \\ m
\end{pmatrix} \\
\frac{n!}{m!k!(n-m-k)!} &= \frac{n!}{(n-m-k)!(m+k)!}\frac{(m+k)!}{m!k!} \\
&= \begin{pmatrix}
n \\ m+k
\end{pmatrix}\begin{pmatrix}
m+k \\ m
\end{pmatrix}.
\end{align*}
\end{proof}

\begin{proposition}[Vandermonde's convolution]
Let $m,n,r\in \N$. Then
\[ \begin{pmatrix}
m + n \\ r
\end{pmatrix} = \sum_{k=0}^r \begin{pmatrix}
m \\ k
\end{pmatrix}\begin{pmatrix}
n \\ r-k
\end{pmatrix}. \]
\end{proposition}
This is also known as Vandermonde's identity, but was already known in 1303 by the Chinese mathematician Zhu Shijie.
\begin{proof}
We use \ref{consequencesBinomialIdentity} and \ref{productOfPolynomials} to calculate
\begin{align*}
\sum_{r=0}^{m+n}\begin{pmatrix}
m+n \\ r
\end{pmatrix}x^r &= (1+x)^{m+n} \\
&= (1+x)^{m}(1+x)^{n} \\
&= \bigg(\sum_{i=0}^{m}\begin{pmatrix}
m \\ i
\end{pmatrix}x^i\bigg)\bigg(\sum_{j=0}^{n}\begin{pmatrix}
n \\ j
\end{pmatrix}x^j\bigg) \\
&= \sum_{r=0}^{m+n}\bigg(\sum_{k=0}^r \begin{pmatrix}
m \\ k
\end{pmatrix}\begin{pmatrix}
n \\ r-k
\end{pmatrix}\bigg)x^r.
\end{align*}
This implies equality of all coefficients, which yields the identity.
\end{proof}
TODO: the double counting proof.

\section{Finite sets}
\begin{definition}
We call a set $X$ an $n$-set if its cardinality is $n$.
\end{definition}

\subsection{Covering finite sets}
\begin{definition}
Let $X$ be a finite set.
\begin{itemize}
\item A \udef{cover} is a collection $\mathcal{F}$ of sets such that $\bigcup \mathcal{F} = X$;
\item A \udef{$k$-cover} is a cover containing $k$ sets.
\end{itemize}
\end{definition}

\url{https://core.ac.uk/download/pdf/82680601.pdf}