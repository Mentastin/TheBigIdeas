\chapter{Graph theory}

\begin{definition}
A \udef{graph} $\Gamma$ is a structured set $(V,E)$ where $V$ is a set of \udef{points} or \udef{vertices} and $E$ is a set of \udef{edges} which are sets containing two (not necessarily distinct) \udef{endpoints}.

Given a graph $\Gamma = (V,E)$, a \udef{subgraph} is a graph $\Gamma' = (V', E')$ such that $V'\subseteq V$ and $E'\subseteq E$. This implies the endpoints of all edges in $E'$ must lie in $V'$.

For any set $S$ we can construct the \udef{complete graph} $C(S)$ which contains all possible edges:
\[C(S) = (S, S\cup\setbuilder{e\in\mathcal{P}(S)}{|e|=2}).\]
Given a graph $\Gamma = (V,E)$, the \udef{complementary graph} $\Gamma'$ contains all the vertices of $\Gamma$ and all the edges not in $\Gamma$:
\[ \Gamma' = (V, S\cup\setbuilder{e\in\mathcal{P}(S)}{|e|=2} \setminus E). \]

We call
\begin{itemize}
\item an edge whose endpoints are the same a \udef{loop};
\item a collection of distinct edges that have the same endpoints a \udef{multiple edge};
\item graphs without loops or multiple edges \udef{simple};
\item vertices connected by an edge \udef{adjacent};
\item the number of edges ending in a vertex, the \udef{valency} of the vertex, where loops are counted twice;
\item a vertex that is not the endpoint of any edge, i.e.\ has valency zero, \udef{isolated}.
\end{itemize}

A \udef{path} of length $n$ from  $p$ to $q$ is a set of edges $\{e_1, \ldots, e_n\}$ such that $e_i$ has endpoints $p_{i-1}, p_i$ and $p_0 =p, p_n = q$.

We call
\begin{itemize}
\item a path from $p$ to $p$ \udef{closed} or a \udef{cycle};
\item a graph that contains no cycles \udef{acyclic};
\item a graph in which any two vertices are the endpoints of a path \udef{connected};
\item a graph in which any two vertices are the endpoints of a \emph{unique} path a \udef{tree}.
\end{itemize}
\end{definition}
\begin{example}
Hasse diagrams are acyclic graphs.
\end{example}

\begin{definition}
A \udef{directed graph} or \udef{digraph} is a graph where each edge is an ordered pair instead of just a set, i.e.\ the \udef{initial vertex} is distinguished from the \udef{final vertex}.

The edges in a digraph are also called \udef{arrows} and a finite digraph is called a \udef{quiver}.
\end{definition}
\begin{example}
Hasse diagrams are directed graphs.
\end{example}