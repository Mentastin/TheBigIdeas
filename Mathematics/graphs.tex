\section{Simple graphs and hypergraphs}
\begin{definition}
Let $V$ be a set.
\begin{itemize}
\item A \udef{hypergraph} on $V$ is a structured set $\sSet{V,E}$, where $E$ is an element of $\powerset^2(V)$.
\item A \udef{simple graph} on $V$ is a hypergraph such that each element of $E$ is a doubleton.
\item A \udef{graph} on $V$ is a hypergraph such that each element of $E$ is a singleton or doubleton.
\end{itemize}
We call $V$ the \udef{vertex set} and $E$ the \udef{edge set}.
\end{definition}

TODO simple graphs as symmetric digraphs

\section{Digraphs}
\begin{definition}
Let $G = \sSet{V,E}$ be a relational structure. We call $G$
\begin{itemize}
\item a \udef{directed graph} or \udef{digraph} if it is irreflexive;
\item an \udef{undirected graph} or just \udef{graph} if it is irreflexive and asymmetric.
\end{itemize}
In either of these cases the elements of $V$ are called \udef{vertices}, \udef{nodes} or \udef{points}. We identify $E$ with its graph and elements of this are called \udef{edges}. If $G$ is a digraph, we also call elements of $E$ \udef{arrows}.

Take $x,y \in V$.
\begin{itemize}
\item If $\seq{x,y}$ is an edge in $E$, then we write $xy \defeq \seq{x,y}$.
\item Take $e\in E$. We say $x$ is
\begin{itemize}
\item the \udef{initial vertex} of $e$ if there exists some $a\in V$ such that $e = xa$;
\item the \udef{terminal vertex} of $e$ if there exists some $a\in V$ such that $e = ax$;
\item an \udef{end} or \udef{endvertex} of $e$ if it is either an initial of terminal vertex.
\end{itemize}
\item If $x \nparallel y$, we say $x$ and $y$ are \udef{adjacent} or \udef{neighbours}. This is equivalent to saying $xy$ is an edge.
\item Take $e_1,e_2 \in E$. We say $e_1$ and $e_2$ are \udef{adjacent} if they have a common endvertex.
\item We call $G$ a \udef{complete graph} if $G = U_V$ for some set $V$.
\end{itemize}

We call $|G| \defeq |V|$ the \udef{order} of the graph. We also define $\norm{G} \defeq |E|$.
\end{definition}

\begin{lemma}
A graph $\sSet{V,R}$ can equivalently be encoded as $\sSet{V, E}$ where each element of $E$ is a doubleton subset of $V$. TODO functor
\[ \seq{x,y} \mapsto \{x,y\} \]
\end{lemma}

TODO graphs are the same if we can identify them by flipping edges.

TODO subgraph

\section{Properties of graphs and elements}
\subsection{Edge sets and paths}
\begin{definition}
Let $\sSet{V,E}$ be a digraph and $X,Y\subseteq V$. We define the \udef{edge set}
\[ E(X,Y) \defeq \setbuilder{e\in E}{\exists x\in X, \exists y\in Y: \; e = xy}. \]
In particular, for $x\in V$, we define $E(x, Y) \defeq E(\{x\}, Y)$ and $E(Y, x) \defeq E(Y,\{x\})$.
\end{definition}

\subsubsection{Degrees of vertices}
\begin{definition}
Let $\sSet{V,E}$ be a digraph and $x\in V$. We define
\begin{itemize}
\item the \udef{indegree} of $x$ as $\deg^-(x) \defeq |E(V,x)|$;
\item the \udef{outdegree} of $x$ as $\deg^+(x) \defeq |E(x, V)|$;
\item the \udef{degree} or \udef{valency} of $x$ as $\deg(x) \defeq \deg^-(x) + \deg^+(x)$.
\end{itemize}
If $\deg^-(x) = \deg^+(x)$ for all $x\in V$, then $G$ is called a \udef{balanced} digraph.

If $\deg(x) = 0$, then $x$ is called \udef{isolated}.
\end{definition}

\begin{lemma}[Degree sum formula] \label{degreeSumFormula}
Let $G = \sSet{V,E}$ be a digraph. Then
\[ \norm{G} = \sum_{x\in V}\deg^-(x) = \sum_{x\in V}\deg^+(x) \]
and thus
\[ \norm{G} = \frac{1}{2}\sum_{x\in V}\deg(x). \]
\end{lemma}
\begin{corollary}
Let $\sSet{V,E}$ be a finite digraph. Then the number of vertices of odd degree is even.
\end{corollary}
\begin{proof}
As $\sum_{x\in V}$ is an integer, $\sum_{x\in V}\deg(x)$ must be even.
\end{proof}

\begin{definition}
Let $G = \sSet{V,E}$ be a finite digraph. We define
\begin{itemize}
\item the \udef{average degree} of $G$ $deg(G) \defeq |G|^{-1}\sum_{x\in V}\deg(x)$;
\item the \udef{minimum degree} of $G$ $\delta(G) \defeq \min\setbuilder{\deg(x)}{x\in V}$;
\item the \udef{maximum degree} of $G$ $\Delta(G) \defeq \max\setbuilder{\deg(x)}{x\in V}$.
\end{itemize}
We also define $\deg^\pm(G), \delta^\pm(G)$ and $\Delta^\pm(G)$ for in-/outdegree.
\end{definition}

\begin{lemma}
Let $G = \sSet{V,E}$ be a finite digraph. Then
\[ \deg(G) = 2\frac{\norm{G}}{|G|} \]
\end{lemma}
\begin{proof}
By the degree sum formula \ref{degreeSumFormula}.
\end{proof}

\begin{proposition}
Let $G = \sSet{V,E}$ be a finite digraph with at least one edge. Then there exists a subdigraph $H \subseteq G$ such that
\[ \delta(H) > \frac{\deg(H)}{2} \geq \frac{\deg(G)}{2}. \]
\end{proposition}
\begin{proof}
We constuct a sequence of subgaphs
\[ G = G_0 \supseteq G_1 \subseteq \ldots \]
by deleting a vertex at each step. At each step we choose a vertex $x\in G_i$ such that $\deg(x) \leq \deg(G_i)/2$ and delete it. We stop when no such vertices remain.

Notice that $\deg(G_{i+1}) \geq \deg(G_i)$. Let $v$ be the deleted vertex. Then deleting $v$ removes $2\deg(v)$ from the total degree $\sum_{x\in V_i}\deg(x)$ (because we remove $\deg(v)$ and remove $\deg(v)$ edges that connect $v$). Then
\[ |G_i|\deg(G_{i}) = |G_{i+1}|\deg(G_{i+1}) + 2\deg(v) = (|G_{i}|-1)\deg(G_{i+1}) + 2\deg(v) \leq |G_{i+1}|\deg(G_{i+1}) + \deg(G_i). \]
Rearranging gives $\deg(G_i) \leq \deg(G_{i+1})$.

Now this algorithm is well-defined, no $G_i$ is empty, because $\deg(\sSet{V, \emptyset}) = 0$ (and we can always rerun the algorithm with an extra unconnected vertex).

As the algorithms must terminate, we must have that $\deg(x) > \deg(H)/2$ for all $x\in H$ at the end. In particular $\delta(H) > \frac{\deg(H)}{2}$.
\end{proof}

\subsubsection{Paths and cycles}
\begin{definition}
Let $G = \sSet{V,E}$ be a digraph. A \udef{path} is a non-empty sequence $\seq{x_i}_{i=1}^k$ of distinct vertices such that $x_{i}x_{i+1} \in E$ for all $i\in 1..(k-1)$.

If $G$ is a graph, we allow for either $x_{i}x_{i+1} \in E$ or $x_{i+1}x_{i} \in E$.

We call
\begin{itemize}
    \item $x_1$ the \udef{initial vertex} of the path;
    \item $x_k$ the \udef{terminal vertex} of the path;
    \item $x_1$ and $x_k$ the \udef{ends} or \udef{end vertices} of the path;
    \item $k - 1$ the \udef{length} of the path, which is the number of edges in the path;
    \item the path an \udef{$x_1-x_k$ path}.
\end{itemize}

We say $\seq{x_i}_{i=0}^k$ is a path from $x_1$ to $x_k$.

A set of paths is \udef{independent} if it is disjoint.
\end{definition}

\begin{definition}
Let $G = \sSet{V,E}$ be a digraph. A \udef{cycle} is a path $\seq{x_i}_{i=1}^k$ such that $x_kx_1 \in E$ (or, if $G$ is a graph, $x_1x_k\in E$).

Let $P = \seq{x_i}_{i=1}^k$ be a cycle. We call
\begin{itemize}
\item $k$ the \udef{length} of the cycle;
\item $\min\setbuilder{\len(P)}{\text{$P$ is a cycle}}$ the \udef{girth} of $G$;
\item $\max\setbuilder{\len(P)}{\text{$P$ is a cycle}}$ the \udef{circumference} of $G$;
\item any edge $xy$ such that $x,y\in P$, but $x$ and $y$ are not adjacent in $P$ a \udef{chord}.
\end{itemize}
\end{definition}

\begin{proposition}
Let $G = \sSet{V,E}$ be a finite digraph. Then
\begin{enumerate}
\item $G$ contains a path of at least length $\delta^+(G)$;
\item $G$ contains a cycle of at least length $\delta^+(G)+ 1$, if $\delta^+(G)\geq 2$;
\item if $G$ is a graph, we may replace $\delta^+(G)$ by $\delta(G)$.
\end{enumerate}
\end{proposition}
\begin{proof}
Let $\seq{x_i}_{i=1}^k$ be a path of maximal length. Then all $v$ such that $x_kx \in E$ must lie on the path, otherwise we could extend the path. Thus $k - 1 \geq \deg^+(v) \geq \delta^+(G)$. Connecting to any of these $v$ yields a cycle.
\end{proof}

\subsubsection{Distance}
\begin{definition}
Let $G = \sSet{V,E}$ be a digraph and $x,y\in V$. The \udef{distance} between $x$ and $y$ is
\[ d(x,y) \defeq \min\setbuilder{\len(P)}{\text{$P$ is an $x-y$ path}}. \]
If there do not exist any $x-y$ paths, then we set $d(x,y) \defeq\infty$.

We also define
\begin{itemize}
    \item the \udef{diameter} $\diam(G) \defeq \max_{x,y\in V} d(x,y)$;
    \item the \udef{radius} $\rad(G) \defeq \min_{x\in V}\max{y\in V}d(x,y)$.
\end{itemize}
\end{definition}

\begin{lemma}
Let $G = \sSet{V,E}$ be a digraph. Then $G$ is symmetric \textup{if and only if} $d(x, y) = d(y,x)$ for all $x,y\in V$.
\end{lemma}
\begin{proof}
We have $xy\in E$ iff $d(x,y) = 1$.
\end{proof}

\begin{lemma}
Let $G = \sSet{V,E}$ be a digraph. Then
\[ \rad(G) \leq \diam(G) \leq 2\rad(G). \]
\end{lemma}
\begin{proof}
The first inequality is clear.

For the second inequality: for all $x\in V$ we have 
\end{proof}

\begin{lemma}
Let $G = \sSet{V,E}$ be a digraph that contains a cycle. Then
\[ \text{girth of $G$} \leq 2\diam(G) + 1. \]
\end{lemma}

\subsection{Trees and forests}

\section{$r$-partite graphs}

\section{Matching, covering and packing}

\section{Connectivity}

\section{Planar graphs}

\section{Colourings}

\section{Flows}