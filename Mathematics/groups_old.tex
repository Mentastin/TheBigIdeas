\section{Motivation}
We are interested in studying symmetries. A we say a physical system has a certain \udef{symmetry} if it remains unchanged under some transformation.

Symmetries play an important role in modern physics. According to Noether's theorem every symmetry gives rise to a conserved quantity, which are eminently satisfying. These conserved quantities often form the bedrock of our physical understanding. Concepts such as energy and momentum have withstood the revolutions of relativity and quantum mechanics and are dependable even in the weirdest outreaches of modern high energy physics.

We give some examples of symmetries, to help conceptualise the rest of this otherwise quite theoretical part.
\begin{itemize}
\item If you take the capital letter ``A'' and try writing it backwards, you will still get ``A''. So if you flip the ``A'', i.e.\ mirror it along an axis that runs straight down through the top of the ``A'', it remains unchanged. Here the transformation is the flipping, or reflection, of the ``A''. Hence this kind of symmetry is called reflectional symmetry.
\item If you take a blank, square piece of paper and rotate it \ang{90}, you will still have a blank, square piece of paper that looks exactly like it did before. The transform we have now applied is a rotation, so we call this rotational symmetry.

In fact this rotation induces a family of symmetries: we can apply this rotation twice, three or four times and the paper will still look the same. Of course if we rotate it five times \ang{90}, then this operation is exactly the same as rotating it \ang{90} and has the same outcome, regardless of the presence of rotational symmetry. We conclude then that our family of symmetries has 4 members:
\begin{enumerate}
\item Doing nothing (rotating \ang{0} or \ang{360});
\item Rotating \ang{90};
\item Rotating \ang{180};
\item Rotating \ang{270}.
\end{enumerate}
Mathematicians call this the cyclic group $C_4$. If we allow flipping as well, we get what in known as the dihedral group $D_4$ which has 8 members. Any combination and amount of flipping and rotating by \ang{90} will be identical to one of those eight. In general the cyclic and dihedral groups $C_n$ and $D_n$ are obtained by rotating $360/n\si{\degree}$.
\item If we consider the rotational symmetry of a circle, it is clear that we can rotate over any angle we choose. Obviously the transformations of $C_4$ are still symmetries of the system, but there are many more. Infinitely more in fact. We say $C_4$ is a subgroup of all the symmetries of the circle, which we typically denote $\U(1)$.

It is convenient to label all symmetries by the angle they rotate over. In other words we are specifying a map from the real numbers to a set of transformations. Intuitively it makes sense to say that this group has one continuous parameter. We want to call this kind of group continuous, as opposed to the previous example which was discrete. More precise definitions of these notions will be given later.
\item The classical laws of nature are the same in all inertial frames: they are invariant under Galilean transformations.
A Galilean transformation $S_0 \rightarrow S_0'$ can in general be written in the following way:
\[ \begin{cases}
\vec{x}\rightarrow \vec{x}' = R(\alpha, \beta, \gamma)\vec{x} + \vec{v}_0t + \vec{x}_0 \\
t \rightarrow t' = t+t_0
\end{cases} \]
Where $R(\alpha, \beta, \gamma)$ is a rotation in 3D space. This transformation is defined by $10$ independent continuous parameters.
\item In special relativity the laws of nature are invariant under Poincaré transformations
\[ x^\mu \rightarrow x'^\mu = \Lambda^\mu_{\;\nu} x^\nu + \epsilon^\mu \]
Again this transformation is defined by 10 continuous parameters, 4 from the translation $\epsilon^\mu$ and 6 from the Lorentz transformation.

In this formulation we can view the Lorentz transformation as a generalisation of rotation $R(\alpha, \beta, \gamma)$. The term allowing for a constant speed difference, $\vec{v}_0t$, is now replaced by the ability to ``rotate'' in \textit{four} dimensions. We can now also rotate in time; this is exactly the Lorentz boost we have discussed before. 
\end{itemize}

Based on these examples, we can observe that symmetries have the following properties:
\begin{itemize}
\item If you have two transformations that are symmetries of a theory, the composition of those two symmetries will also be a symmetry: If you apply one transformation your theory stays the same. Now if you apply the second transformation, your theory is still the same. So applying both symmetries is also a symmetry of the theory.
\item The composition of symmetries is associative.
\item The inverse of a symmetry will also leave the theory invariant.
\item The identity transformation is obviously a symmetry.
\end{itemize}
These facts mean that the set of transformations that leave our theory invariant, form what is known as a \textit{group}.

This is very interesting because if we know something is a group, we can immediately make use of all the theorems mathematicians have proved about groups.

\section{Basic definitions}

\begin{definition}
A \udef{group $G$} is a set of elements $\{g_i\}$ endowed with a binary operation
\[ \boldsymbol{\cdot} : G\times G \rightarrow G: (g_1, g_2) \mapsto g_3 = g_1\cdot g_2\]
with the following properties:
\begin{enumerate}
\item $\cdot$ is associative: $g_1\cdot (g_2\cdot g_3) = (g_1\cdot g_2)\cdot g_3$
\item There exists an identity $e$: $g\cdot e = e\cdot g = g$
\item Every element has an  inverse: $\forall g\in G: \exists h \in G: gh = hg = e$ (We call $h = g^{-1}$) 
\end{enumerate}
\end{definition}
As claimed before the set of all the symmetries of a system form a group if we take the binary operation to be the composition operation, $\circ$. So it makes sense that we would be interested in this mathematical construct, but we might wonder whether there was not another construct that was even better at capturing the notion and properties of symmetry. We arrived at the construct of group by listing a number of properties of symmetries, maybe symmetries have more properties that we have not considered and maybe there is another construct that takes those properties into account as well.

It turns out that this is not the case. Cayley's theorem states that every group $G$ is isomorphic to a subgroup of the symmetric group acting on $G$, so in some sense every group is a symmetry group. As a very loose, intuitive application, observe for example that $\mathbb{N},+$ is not a group while $\mathbb{Z},+$ is, because of translational symmetry. In the same way $\mathbb{Z},\cdot$ is not a group, but $\mathbb{R},\cdot$ is, thanks to scaling symmetry.

Based on the definition of a group the following elementary theorems can be proved:
\begin{eigenschap}
\begin{itemize}
\item The identity is unique
\item The inverse of an element is unique
\end{itemize}
\end{eigenschap}

We also introduce the following definitions:

\begin{definition}
\begin{itemize}
\item The \udef{order of $G$} is the number of elements of $G$
\begin{itemize}
\item The order can be finite or infinite.
\end{itemize}
\item A group is called \udef{Abelian} or \udef{commutative} if the group operation $\boldsymbol{\cdot}$ is commutative.
\end{itemize}
\end{definition}

\begin{example}
Examples of Groups:
\begin{enumerate}
\item $Z_n$, the group of all $n^{\text{th}}$ roots of $1$ with the ordinary product, is of order $n$.
\begin{itemize}
\item $Z_2 = \{1,-1\}$
\item $Z_3 = \{1, e^{i2/3\pi}, e^{i1/3\pi}\}$
\end{itemize}
One may remark that this group is the same as (i.e.\ isomorphic to) the cyclic group introduced above.
\item $S_n$, the group of all permutations of $n$ elements, is of order $n!$.
\item Integers with addition.
\item $\mathbb{R}\setminus\{0\}$ with multiplication.
\item The square (i.e.\ $n\times n$) invertible matrices with matrix multiplication form a group.
\end{enumerate}
\end{example}

\subsection{Group isomorphisms}

\begin{definition}
Two groups $G_1, \cdot$ and $G_2,*$ are \udef{isomorphic} if theres is a bijection $\varphi: G_1 \to G_2$ such that
\[ \varphi(x\cdot y) = \varphi(x)*\varphi(y) \qquad \forall x,y\in G_1.\]
We call $\varphi$ an \udef{isomorphism} and write $G_1 \simeq G_2$.
\end{definition}


\begin{example}
The following groups are isomorphic:
\begin{enumerate}
\item $Z_2 = \{1,-1\}, \cdot$
\item $S_2 = \{(ab), (ba)\}, \circ$
\item $\{ \mathbb{1}_N, P_N\}, \cdot$
\end{enumerate}
This can be verified using a multiplication (Cayley) table.
\begin{center}
  \begin{tabular}{ l | c | r }
     & e & g \\ \hline
    e & e & g \\ \hline
    g & g & e \\
  \end{tabular}
\end{center}
\end{example}

\subsection{Subgroups}
\begin{definition}
A \udef{subgroup $H$} of $G$ ($H\subset G$) is a set of elements $h_i\in G$ that satisfy the group axioms
\end{definition}
\begin{eigenschap}
A nonempty \undline{subset $H$} of a group $G$ is a \ueig{subgroup} of $G$ if, and only if, for each pair of elements $h_1,h_2 \in H$, possibly equal, one has $h_1h_2^{-1} \in H$
\end{eigenschap}
\begin{example}
\begin{itemize}
\item $\{e\}, \;G\;$ are trivial subgroups of $G$
\item $Z_2 \subset Z_4$
\item $S_m \subset S_n \quad (m<n)$
\item $\SO(2) \subset \SO(3)$
\end{itemize}
\end{example}
\begin{definition}
The \udef{direct product} $G \equiv H\otimes F$ of two groups $H$ and $F$ is defined with the following operation:
\[ (H\otimes F) \times (H\otimes F) \rightarrow (H\otimes F): ((h_1,f_1),(h_2,f_2)) \mapsto (h_1\cdot h_2, f_1\cdot f_2)\]
\end{definition}
The direct product is a group with
\[ \begin{cases}
e_G = (e_H,e_F) \\
g^{-1} = (h^{-1}, f^{-1})\qquad \forall g = (h,f) \in G.
\end{cases} \]
The groups $F$ and $H$ are subgroups of $G$ and can be recovered by considering, respectively the elements of $G$ of the form $(e_H, g)$ and $(g ,e_F)$.
\begin{example}
The symmetry group of the standard model is the unitary product group
\[\SU(3)_C\times \SU(2)_L \times \U(1)_Y.\]
This will be discussed in more detail later.
\end{example}


\section{Topological groups}
A group is a set with an extra structure layered on top: the group operation that satisfies the group axioms. A topological space is also a set with an extra structure layered on top: the topology, as discussed in a previous part. Now here's a novel idea: let's layer both of these structures on a set at once. This gives no new mathematics because the two structures do not interact in any way; in order for interesting things to occur, we must pose some additional requirements.

\begin{definition}
A \udef{topological group} $G$ is a topological space that is also a a group such that the group operations of
\begin{enumerate}
\item product
\[ G\times G \to G: \, (x,y)\mapsto xy \]
\item and taking inverses
\[ G\to G: \, x\mapsto x^{-1} \]
\end{enumerate}
are \textbf{continuous}.
\end{definition}
TODO also need that points are closed?

\begin{lemma}
The continuity of the product and inverse is equivalent to the continuity of $G\times G \to G: (s,r)\mapsto sr^{-1}$.
\end{lemma}
TODO; reframe as criterion?

\begin{lemma}
Let $G$ be a topological group. The following are homeomorphisms:
\begin{enumerate}
\item $G\to G: s\mapsto s^{-1}$;
\item $G\to G: s\mapsto rs$ for any $r\in G$.
\end{enumerate}
\end{lemma}
An important consequence of this is that the topology of $G$ is determined by the topology near the identity $e$.

Topological groups are also sometimes called continuous groups.

\subsection{Matrix groups}
We now consider some extremely important examples of topological groups: the matrix groups.
If we take the set of real, $N\times N$ matrices with a non-zero determinant, it turns out that they form a group with the matrix multiplication:
\begin{enumerate}
\item The matrix multiplication is associative;
\item The identity is the identity matrix $\mathbb{1}$;
\item Because their determinant is not zero, every matrix in this set has an inverse.
\item Because the matrices are square, the multiplication of two matrices gives a matrix of the same dimensions. In other words the matrix multiplication is a closed operation.
\end{enumerate}
We call this group the \udef{real general linear group} $\GL(N, \R)$. It also has a complex counterpart, the complex general linear group $\GL(N, \C)$.

TODO: topological
We can also immediately see that the operations of matrix multiplication and inversion are smooth. (For inversion this is obviously only true after restriction to the open subset of invertible matrices, which luckily all matrix Lie groups are in turn a subset of). This follows quite readily because both operations are in effect comprised of addition and multiplication operations, which are infinitely differentiable. (e.g\ $A^{-1} = \frac{1}{\det(A)}\mathrm{adj}(A)$)

\begin{example}
TODO: $A^2 = \mathbb{1}$
\end{example}

These groups, along with all their subgroups, are known as the matrix groups and are very important in physics.

\subsubsection{Continuous parameters}
It is sometimes interesting to know how many degrees of freedom a particular set of transformations has. For example, rotations in the 2D plane are characterized with one parameter: the angle of rotation. In 3D we need three parameters. This notion of continuous parameter is formalised below.

\begin{definition}
A function $A : \R \to \GL(n, \C)$ is called a \udef{one-parameter subgroup} of $\GL(n, \C)$ if
\begin{enumerate}
\item $A$ is continuous,
\item $A(0) = \mathbb{1}_n$,
\item $A(t+s) = A(t)A(s)$ for all $t,s \in \R$.
\end{enumerate}
We also call the image of $A$ a one-parameter subgroup.
\end{definition}

A one-parameter subgroup has one continuous parameter. A subgroup of $\GL(n, \C)$ with $m$ continuous parameters, is a function $A : \R^m \to \GL(n, \C)$ such that each function of the form
\[ x \mapsto A(a_1, a_2, \ldots , a_{i-1}, x, a_{i+1}, \ldots, a_m) \]
gives a one-parameter subgroup for fixed $a_1,\ldots, a_m$.

We can speak of an $m$-parameter subgroup because, while different parametrisations may be found, any subgroup of $\GL(n,\C)$ constructed in this way must always be constructed with the same number of parameters. To see that this must be the case, consider two parametrised subgroups $A : \R^m \to \GL(n, \C)$ and $B : \R^{m'} \to \GL(n, \C)$ with the same image.

TODO !! + dimension of manifold

\subsubsection{Examples}
We now give names to the most important matrix groups, and list the number of continuous parameters.
\begin{enumerate}
\item General linear group
\[ \GL(N,\R) = \{N\times N \;\text{real matrices}, \; \det M \neq 0\} \]
\begin{itemize}
\item We have $N^2$ independent parameters (= the entries of the matrix), so $\dim \GL(N,\R) = N^2$
\item Each complex number can be described with two real ones, so $\dim \GL(N, \C) = 2N^2$
\end{itemize}
\item Special linear group
\[ \SL(N,\R) = \{M\in\GL(N,\R), \;\det M = 1\} \]
\begin{itemize}
\item $\dim \SL(N,\R) = N^2-1$: 1 dimension is used to fix determinant.
\item $\dim \SL(N,\C) = 2(N^2-1)$: 1 dimension is used to fix the real part of the determinant, and 1 to fix the imaginary part.
\end{itemize}
\item Unitary matrices
\[ \U(N) = \{U\in\GL(N,\C), U^\dagger\mathbb{1}_NU = \mathbb{1}_N\} \]
\begin{itemize}
\item $U^\dagger U$ is Hermitian, meaning that the complex transpose of $U$ is $U$.
\item $U^\dagger U = \mathbb{1}_N$ yields only $N^2$ independent equations, not $2N^2$ because of the Hermiticity of the equation.
\item $\dim \U(N) = 2N^2-N^2$ = $N^2$
\end{itemize}
\item Special unitary groups
\[ \SU(N) = \{U\in\U(N),\; \det U = 1\} \]
\begin{itemize}
\item For unitary matrices we have that $|\det U| = 1$. This fixes one continuous parameter and thus one dimension.
\item $\dim \SU(N) = N^2 - 1$
\end{itemize}
\item Orthogonal groups
\begin{itemize}
\item $\Ogroup(N) = \{O\in\GL(N,\R),\; O^\intercal\mathbb{1}_N O = \mathbb{1}_N\}$
\begin{itemize}
\item $O^\intercal O$ is symmetric, so $\frac{N(N+1)}{2}$ independent equations (half the matrix already fixed by the other half)
\item $\dim \Ogroup = N^2 - \frac{N(N+1)}{2} = \frac{N(N-1)}{2}$
\end{itemize}
\item $\SO(N) = \{O\in\Ogroup(N),\; \det O = 1\}$
\begin{itemize}
\item For orthogonal matrices $\det O = \pm 1$. This does not fix any continuous parameters.
\item $\dim \SO = \dim \Ogroup = \frac{N(N-1)}{2}$
\end{itemize}
\end{itemize}
\item Using a non definite metric $\eta = \diag(\mathbb{1}_p, -\mathbb{1}_q)$
\begin{itemize}
\item $\U(p,q) = \{ U\in \GL(N,\C), U^\dagger\eta U = \eta \}$
\item $\Ogroup(p,q) = \{ O\in \GL(N,\R), O^\intercal\eta O = \eta \}$ \\
In particular $\SO(1,3)$ is the \udef{Lorentz group} (with mostly minus convention).
\end{itemize}
\end{enumerate}

Here are some of the most important examples written more explicitly in terms of their continuous parameters:
\begin{itemize}
\item $\U(1) \equiv \{z\in\mathbb{C}|\; |z|=1\}, \boldsymbol{\cdot}$ has one real parameter. Every element $z$ of this group can be written $z=e^{i\alpha}$ for a real $\alpha$.
\item $\SO(2)$ has one real parameter.
\[ R(\theta) = \begin{pmatrix}\cos(\theta) & -\sin(\theta)\\ \sin(\theta) & \cos(\theta)\end{pmatrix} \]
\item $\SO(3)$ has three real parameters.
\[ R(\theta_{12},\theta_{13},\theta_{23}) = R_1(\theta_{12})R_2(\theta_{13})R_3(\theta_{23}) \]
where
\[R_1(\theta_{12}) = \begin{pmatrix}\cos(\theta_{12}) & -\sin(\theta_{12})&0\\ \sin(\theta_{12}) & \cos(\theta_{12})&0\\0&0&1\end{pmatrix}\]
\[R_2(\theta_{13}) = \begin{pmatrix}\cos(\theta_{13}) &0& -\sin(\theta_{13})\\0&1&0\\ \sin(\theta_{13}) &0& \cos(\theta_{13})\end{pmatrix}\]
\[R_3(\theta_{23}) = \begin{pmatrix}1&0&0\\ 0&\cos(\theta_{23}) & -\sin(\theta_{23})\\0& \sin(\theta_{23}) & \cos(\theta_{23})\end{pmatrix}\]
\item $\SU(2)$ has three real parameters and its elements can be seen as complex $2\times 2$ rotations.
\[ \U(\alpha, \beta, \gamma) = \begin{pmatrix}\cos\theta e^{i\alpha} & -\sin\theta e^{i\beta}\\ \sin\theta e^{-i\beta} & \cos\theta e^{-i\alpha}\end{pmatrix} \]
\end{itemize}



\section{Group action}
We have seen that symmetry transformations naturally form a group. Based on the concrete set of transformations that are symmetries we saw they form this abstract structure which we called a group. The advantage of working with this abstract entity is that it contains exactly the relevant details about the symmetry. We need not worry ourselves about the peculiarities of the particular system and we can easily make use of results others have obtained solving other problems.

Once we have thoroughly studied the symmetries of our system, we will want a way to move back from studying abstract groups to studying transformations of the system we are actually interested in.

Sometimes there is a natural correspondence between the set of group elements and the set of transformations. If this is the case the group can be interpreted as acting on the system in a \udef{canonical} (or natural) way.

\begin{example}
\begin{itemize}
\item Dihedral group $D_4$ acts quite naturally on a blanc, square piece of paper.
\item The symmetric group $\mathcal{S}_n$ of all permutations of a set of $n$ elements acts naturally on a set of $n$ elements.
\item The group of $n\times n$ matrices acts naturally on $n$-dimensional vectors through matrix multiplication.
\end{itemize}
\end{example}

In general the transition back may not be so clear, simple or natural. For instance there may be a subset of the $n$-dimensional vectors with a symmetry group isomorphic to $D_4$. To what transformations do these group elements correspond? We cannot just rotate and flip these vectors. It is for understanding these cases that the concept of a \udef{group action} is useful.

\subsection{Definition}
We start with a group $G$ and a set $X$. The set $X$ is frequently the set of configurations of the system and thus transformations of the system are functions of the type $f:\,X\to X$; to keep things general, we only assume we have set and we are agnostic as to its origins.
A group action quite simply associates a transformation of the set to every element of the group.

We do however require that this association has some fairly natural features, so that the nature and essence of the group is not lost in transition: the group action must respect the identity element and and group operation. This leads us to the following definition:
\begin{definition}
Let $G$ be a group and $X$ a set, then a \udef{(left) group action $\varphi$} of $G$ on $X$ is a function
\[ \varphi: \, G\times X \to X: \, (g,x)\mapsto \varphi(g,x) = g\cdot x \]
with the properties:
\begin{enumerate}
\item For the identity element $e$ and all $x\in X$: 
\[e\cdot x = x \]
\item For all $g,h \in G$ and $x\in X$:
\[ (gh)\cdot x = g\cdot (h\cdot x) \]
\end{enumerate}
Notice that we have introduced the notation $g\cdot x$ meaning apply the transformation attributed to $g$ through the group action to the element $x$.
\end{definition}

The above definition is for a \textit{left} group action. We can analogously define a right group action. The only difference between the two is that in the right group action in the transformation
\[ x \cdot (gh) = (x\cdot g)\cdot h \]
the transformation associated with $g$ gets applied first. Using the formula $(gh)^{-1} = h^{-1}g^{-1}$ we can always construct a left group action from a right one and vice versa, so typically we only consider left group actions.

An important property is immediately apparent from the definition:
\begin{eigenschap}
The transformation associated with $g$ (i.e.\ $x\mapsto g\cdot x$) is always a bijection because the inverse is given by $x \mapsto g^{-1}\cdot x$.
\end{eigenschap}

\subsection{Types of action}
What follows is simply an enumeration of some properties group actions may have. The action of $G$ on $X$ is called
\begin{enumerate}
\item \udef{transitive} if $X$ is non-empty and for each $x,y$ in $X$ there exists a $g \in G$ such that $g\cdot x = y$.
\item \udef{faithful} if for every distinct $g,h$ in $G$ there exists an $x \in X$ such that $g\cdot x \neq h\cdot x$. In other words the mapping of elements of $G$ to transformations of $X$ is 1-to-1 or injective.
\end{enumerate}

\subsection{Orbits and stabilizers}
\begin{definition}
Consider a group $G$ acting on a set $X$. The \udef{orbit} of an element $x$ of $X$ is denoted $G\cdot x$.
\[ G\cdot x = \{ g\cdot x | g\in G \}. \]
\end{definition}

The \udef{stabilizer subgroup} of $G$ with respect to an element $x$ of $X$ is the set of all elements in $G$ that fix $x$ and is denoted $G_x$.
\[G_x = \{ g\in G | g\cdot x = x \} \]

\subsection{Continuous group action}
A continuous group action on a topological space $X$ is a group action of a topological group $G$ that is continuous: i.e.\,
\[G \times X \to X : \;(g, x) \mapsto g \cdot x \]
is a continuous map.

This is the proper type of group action to use with topological groups, if their topologicalness is relevant and to be preserved.

\subsection{Representations}
If the group action is the action of a group on a vector space such that the transformations the group elements are mapped to are linear transformations, we call this group action a \udef{representation}.

\begin{definition}
A \udef{representation} of a group $G$ on an $n$-dim vector space $V$ is a mapping of the elements of $G$ to the set of invertible linear operations acting on $V$:
\[D: G \rightarrow GL(V): g \mapsto D(g)\]
Such that
\begin{itemize}
\item $D(e) = \mathbb{1}_V$
\item $D(g_1\cdot g_2) = D(g_1)D(g_2) = D(g_3)$
\end{itemize}
\end{definition}

\begin{example}
\begin{itemize}
\item Representations of $Z_3 = \{e,\omega, \omega^2\} \qquad (\omega = e^{i2/3\pi})$
\begin{itemize}
\item Trivial representations
\[D(e) = D(\omega) = D(\omega^2) = \mathbb{1}_V\]
\item Representation $\GL(1, \mathbb{C})$
\[ D(e) = 1, \quad D(\omega) = e^{i\frac{2}{3}\pi} , \qquad D(\omega^2) = e^{i\frac{1}{3}\pi} \]
\item Regular representation:
\[D(e) = 
\begin{pmatrix}
1&0&0\\0&1&0\\0&0&1
\end{pmatrix}, \qquad D(\omega) = \begin{pmatrix}
0&0&1\\1&0&0\\0&1&0
\end{pmatrix}, \qquad D(\omega^2) = \begin{pmatrix}
0&1&0\\0&0&1\\1&0&0
\end{pmatrix}
\]
In general a we can define a regular representation for any finite group $G$ as follows: Let $V$ be a vector space with basis $e_t$ indexed by the elements of $G$, $t \in G$. The mapping $D: e_t \mapsto e_{ts}$ defines the \udef{(left) regular representation} of $G$. This notion can be extended to groups of infinite order.
\end{itemize}
\item The standard representation of a subgroups $H$ of $\GL(n,\C)$ on the vector space $\C^n$ is given by the inclusion:
\[ D: H \to \GL(\C^n) = \GL(n,\C): h \mapsto h \]
\end{itemize}
\end{example}


\begin{definition}
Two representations are \udef{equivalent} if there exists a linear operator $S$ such that
\[D(g) \mapsto D'(g) = S^{-1}D(g)S\]
In other words there exists a similarity transformation $S$
\end{definition}

\begin{definition}
A representation is \udef{unitary} if $\forall g \in G$
\[D(g)D^\dagger(g) = D^\dagger(g)D(g) = \mathbb{1}_V\]
\end{definition}

\begin{definition}
Consider a \undline{representation $D$} of a \undline{group $G$} on a \undline{vector space $V$}
\begin{enumerate}
\item A subspace $W$ of $V$ is called \udef{invariant} if $D(g)w$ is in $W$ for all $w \in W$ and all $g \in G$. An invariant subspace $W$ is called nontrivial if $W\neq\{0\}$ and $W \neq V$.
\item We call $D$ \udef{reducible} if there exists a nontrivial subspace $U$ of $V$ that is invariant under $D$.
\item $D$ is \udef{irreducible} if the only subspaces invariant under all elements of the image of $D$ are $\emptyset$ and $V$
\item $D$ is \udef{completely reducible} if we can decompose $V$ into invariant subspaces:
\[V = U_1\oplus U_2 \oplus \ldots \oplus U_n\]
There then exists a similarity transformation such that
\[\forall g: D(g) = \begin{pmatrix}
D_1(g) & 0 & \dots & 0\\
0 & D_2(g)  & \dots & 0\\
\vdots & & \ddots & \vdots\\
0&0&\dots&D_n(g)
\end{pmatrix}\qquad \text{with}\quad D \equiv D_1\oplus D_2 \oplus \ldots \oplus D_n\]
\end{enumerate}
\end{definition}

\begin{example}
The regular representation of $Z_3$ is completely reducible. The linear operators $D(e), D(\omega)$ and $D(\omega^2)$ have eigenvalues $1,\omega, \omega^2$ with eigenvectors 
\[ \begin{pmatrix}
1\\1\\1
\end{pmatrix}\, \qquad \begin{pmatrix}
1\\ \omega^2 \\ \omega
\end{pmatrix} \qquad \text{and} \qquad \begin{pmatrix}
1 \\ \omega \\ \omega^2
\end{pmatrix}. \]
Each eigenvector generates an invariant subspace. We can then apply the following coordinate transformation
\[ S = \frac{1}{\sqrt{3}}\begin{pmatrix}
1&1&1\\
1&\omega^2 & \omega \\
1&\omega& \omega^2
\end{pmatrix} \]
in order to get the following matrices
\[D'(e) = \begin{pmatrix}
1&0&0\\
0&1&0\\
0&0&1
\end{pmatrix}, \qquad D'(\omega) = \begin{pmatrix}
1 & 0 & 0\\
0& \omega & 0 \\
0&0&\omega^2
\end{pmatrix}, \qquad D'(\omega^2) = \begin{pmatrix}
1 & 0 & 0\\
0& \omega^2 & 0 \\
0&0&\omega
\end{pmatrix}\]
\[ D' = D_1\oplus D_2 \oplus D_3 = \diag\{1,1,2\}\oplus \diag\{1,\omega, \omega^2\} \oplus \diag\{1, \omega^2, \omega\} \]
\end{example}

\subsubsection{Projective representations}
Bargmann theorem



\section{Galilean algebra}
Low velocity limit of Poincaré: Inönü-Wigner contraction.

Mass superselection.

\section{Group extensions}
\begin{definition}
Let $N,Q$ be groups. An \udef{extension} of $Q$ by $N$ is a group $G$ such that
\[
\begin{tikzcd}
1 \ar[r] & N \ar[r, "\iota"] & G \ar[r, "\pi"] & Q \ar[r] & 1
\end{tikzcd}.
\]
is a short exact sequence.
\end{definition}
\begin{lemma}
If $G$ is an extension of $Q$ by $N$, then $G$ is a group (TODO: closure), $\iota(N)$ is a normal subgroup of $G$ and $Q$ is isomorphic to $Q$.
\end{lemma}

\begin{definition}
Two extensions $G,G'$ of $Q$ by $N$ are \udef{equivalent} if there is a homomorphism $T:G\to G'$ making the following diagram commutative:
\[
\begin{tikzcd}
1 \ar[r] & N \ar[r, "\iota"] \ar[equal]{d} & G \ar[r, "\pi"] \ar[d,"T"] & Q \ar[r] \ar[d, equal] & 1 \\
1 \ar[r] & N \ar[r, "\iota"] & G' \ar[r, "\pi"] & Q \ar[r] & 1.
\end{tikzcd}
\]
\end{definition}
\begin{lemma}
If $G,G'$ are equivalent extensions, then they are isomorphic. So equivalence of extension is an equivalence relation.
\end{lemma}
\begin{proof}
The short five lemma (TODO).
\end{proof}
The converse is \emph{not} true! TODO: For instance, there are $8$ inequivalent extensions of the Klein four-group by $\mathbb{Z}/2\mathbb{Z}$, but there are, up to group isomorphism, only four groups of order $8$ containing a normal subgroup of order $2$ with quotient group isomorphic to the Klein four-group.

\section{Grothendieck group}
Given a commutative monoid $M$, the Grothendieck group $G(M)$ is the ``most general'' Abelian group that arises from $M$. Intuitively it is formed by adding additive inverses for all elements of $M$.



 
TODO Grothendieck construction for Abelian monoids: $G(M)$.
Universality, functoriality

Cancellation property: simplified construction.

Grothendieck map $M\to G(M)$ is injective \textup{if and only if} $M$ has cancellation.