\chapter{Riemannian manifolds}
\section{Riemannian metrics}
\begin{definition}
Let $M$ be a smooth manifold. A \udef{Riemannian metric} on $M$ is a smooth covariant $2$-tensor field $g\in T^*M\otimes T^*M$ whose value $g_p$ at each point $p\in M$ is an inner product on $T_pM$. We often use the notation
\[ \inner{v,w}_g \defeq g_p(v,w) \]
where $p\in M$ and $v,w\in T_pM$. Similarly we write $\norm{v}_g \defeq \sqrt{\inner{v,v}_g}$.

A \udef{Riemannian manifold} is a pair $(M,g)$ where $M$ is a smooth manifold and $g$ is a Riemannian metric on $M$.
\end{definition}

Most things work with boundary as well.

\begin{lemma}
Every smooth manifold admits a Riemannian metric.
\end{lemma}
\begin{proof}
TODO with partition of unity.
\end{proof}

\begin{example}
The \udef{Euclidean metric} is the Riemannian metric $g_E$ on the manifold $\R^n$ whose value at each $x\in\R^n$ is the standard inner product on $T_x\R^n$.
\end{example}

\subsection{Isometries}
\begin{definition}
Let $(M_1, g_1)$ and $(M_2, g_2)$ be Riemannian manifolds. An \udef{isometry} from $(M_1, g_1)$ to $(M_2, g_2)$ is a diffeomorphism $\varphi: M_1\to M_2$ such that $\varphi^* g_2 = g_1$.

We say $\varphi: M_1\to M_2$ is a \udef{local isometry} if for each point $p\in M_1$ their is a neighbourhood $U(p)$ such that $\varphi|_U$ is an isometry onto an open subset of $M_2$.

A Riemannian $n$-manifold is called \udef{flat} if it is locally isometric to a Euclidean space.
\end{definition}
An isometry from $(M,g)$ to itself is called an isometry of $(M,g)$. The set of isometries of $(M,g)$ is a group under composition, the \udef{isometry group} of $(M,g)$, denoted $\Iso(M,g)$.

\begin{lemma}
All Riemannian $1$-manifolds are flat.
\end{lemma}

\begin{lemma}
A mapping $\varphi:M\to M'$ between smooth manifolds is an isometry \textup{if and only if} $\varphi$ is a smooth bijection and each differential $\diff\varphi_p:T_pM\to T_{\varphi(p)}M'$ is a linear isometry.
\end{lemma}
\begin{proof}
The only part to prove is that $\varphi$ is automatically a diffeomorphism if it is a smooth bijection. This follows from the global rank theorem \ref{globalRank} because $F$ has contant rank (equal to the dimension of $M$ and $M'$).
\end{proof}

\subsection{Local representations for metrics}
Let $(x^1, \ldots, x^n)$ be smooth local coordinates on the neighbourhood $U\subseteq M$. Then $g|_U$ can be written as
\[ g|_U = g_{ij}\diff{x^i}\otimes\diff{x^j}  \]
The definition of inner product translates to the requirement that $[g(p)]_{ij}$ be a symmetric, non-singular matrix. Using symmetry we get the symmetric product
\[ g|_U = g_{ij}\diff{x^i}\diff{x^j} \]

\begin{example}
The Euclidean metric can be expressed as
\[ g_E = \sum_i\diff{x^i}\diff{x^i} = \delta_{ij}\diff{x^i}\diff{x^i} \]
so $g_{ij} = \delta_{ij}$.
\end{example}

\begin{proposition}
Given a smooth local frame for $TM$ we can construct a smooth orthonormal frame with the same span.
\end{proposition}
\begin{proof}
Gram-Schmidt.
\end{proof}

\subsection{Constructing Riemannian metrics}
\subsection{Riemannian immersions}
\begin{proposition}
Let $(M,g)$ be a Riemannian manifold, $M'$ a smooth manifold and $F:M'\to M$ a smooth map. Then $g' = F^*g$ is a Riemannian metric on $M$ \textup{if and only if} $F$ is an immersion.
\end{proposition}
\begin{proof}
The only reason $g' = F^*g$ may fail to be a metric is if it is not definite. First assume $F$ is not an immersion. Then there exist $p\in M'$ and $v,w\in T_pM'$ such that $\diff{F}_p(v) = \diff{F}_p(w)$ and $v\neq w$. Then $v-w \neq 0$, but
\[ \inner{v-w,v-w}_{g'}= \inner{\diff{F}(v-w),\diff{F}(v-w)}_{g} = \inner{0,0}_g = 0. \]
Conversely, assume $g'$ not definite. Then there exists a $v\neq 0$ such that $0 = \norm{v}_{g'}= \norm{\diff{F}(v)}_{g}$, implying $\diff{F}(v) = 0$. Thus the kernel of $\diff{F}$ is not $\{0\}$, meaning it is not injective by \ref{injectivityKernelTriviality} and thus $F$ is not an immersion by definition. 
\end{proof}
The metric $g' = F^*g$ of the proposition is called the \udef{metric induced by $F$}.

An immersion (resp. embedding) $F: (M,g)\to (M',g')$ is called an \udef{isometric immersion} (resp. \udef{isometric embedding}) if $g' = F^*g$.

\begin{lemma}
Existence of adapted orthonormal frames.
\end{lemma}

\begin{definition}
Let $(M,g)$ be a Riemannian manifold and $M'\subseteq M$ a smooth submanifold. A vector $v\in T_pM$, for some $p\in M'$, is called \udef{normal} to $M'$ if $\inner{v,w}_g = 0$ for every $w\in T_pM'$.

The space of all vectors normal to $M'$ at $p\in M'$ is called the \udef{normal space} $N_pM'$ at $p$.
\end{definition}
Clearly $N_pM' = (T_pM')^\perp$ and
\[ T_pM = T_pM' \oplus N_pM'. \]

\begin{proposition}[Normal bundle]
Let $(M,g)$ be a Riemannian $m$-manifold without boundary and $M'\subseteq M$ a an immersed $n$-submanifold. The set
\[ NM' = \bigsqcup_{p\in M'}N_pM' \]
is a smooth subbundle of $TM|_{M'}$ of rank $(m-n)$.
\end{proposition}
The vector bundle $NM'$ is called the \udef{normal bundle} of $M'$.

A section of the normal bundle $NM'$ is called a \udef{normal vector field} along $M'$.

The \udef{tangential projection} $\pi^\top: TM|_{M'}\to TM'$ and the \udef{normal projection} $\pi^\perp: TM|_{M'}\to NM'$ are the maps that for each $p\in M'$ restrict to the orthogonal projections $T_pM\to T_pM'$ and $T_pM\to N_pM'$.

\begin{lemma}
The tangential and normal projections are smooth bundle homomorphisms
\end{lemma}

\subsection{Riemannian products}
\begin{definition}
Let $(M_1,g_1)$ and $(M_2,g_2)$ be Riemannian manifolds. The product manifold $M_1\times M_2$ has a natural Riemannian metric $g=g_1\oplus g_2$ called the \udef{product metric} defined by
\[ g_{p_1,p_2}: (T_{p_1}M_1\oplus T_{p_2}M_2)^2 \to \R: (v_1+v_2, w_1+w_2) \mapsto g_1|_{p_1}(v_1,w_1) + g_2|_{p_2}(v_2,w_2) \]
where we have identified $T_{(p_1,p_2)}(M_1\times M_2)$ with $T_{p_1}M_1\oplus T_{p_2}M_2$.
\end{definition}

\subsection{Riemannian submersions}
\subsubsection{Horizontal and vertical tangent spaces}
Suppose $M,M'$ are smooth manifolds, $\pi:M\to M'$a smooth submersion and $g$ a Riemannian metric on $M$.

TODO: we can view $M$ as a fibre bundle with as fibres the properly embedded smooth manifolds $M_y = \pi^{-1}(y)$.

At each point $x\in M$ we can split $T_xM$ into two subspaces $V_x \oplus H_x$, the \udef{horizontal} and \udef{vertical tangent spaces} at $x$, defined by
\[ V_x \defeq \ker\diff{\pi}_x = T_x(M_{\pi(x)}) \qquad \text{and} \qquad H_x = (V_x)^\perp. \]
Where the equality $\ker\diff{\pi}_x = T_x(M_{\pi(x)})$ is due to (TODO tangent space to a submanifold). Notice that the definition of $V_x$ does not depend on the metric, but the definition of $H_x$ does.

A \udef{horizontal vector field} on $M$ consists of vectors in the horizontal tangent space and a \udef{vectical vector field} on $M$ consists of vectors in the vertical tangent tangent space on $M$.

A vector field $X$ on $M$ is a \udef{horizontal lift} of a vector field $X'$ on $M'$ if $X$ is horizontal and $\pi$-related to $X$, which means that
\[ \forall x\in M: \; \diff{\pi}_x(X_x) = X'_{\pi(x)} .\]

\begin{proposition}
Let $M,M'$ be smooth manifolds, $\pi:M\to M'$ a smooth submersion and $g$ a Riemannian metric on $M$.
\begin{enumerate}
\item Every smooth vector field $W$ on $M$ can uniquely be expressed as the sum of a smooth horizontal and a smooth vertical vector field:
\[ W = W^H + W^V. \]
\item Every smooth vector field on $M'$ has a unique smooth horizontal lift to $M$.
\item For every $x\in M$ and $v\in H_x$, there is a vector field $X'\in\mathfrak{X}(M')$ whose horizontal lift $X$ satisfies $X_x = v$.
\end{enumerate}
\end{proposition}
The last part of the previous proposition says that any horizontal vector can be extended to a horizontal lift on all of $M$.

Importantly, it is \emph{not true} that every horizontal vector field on $M$ is a horizontal lift.
\begin{example}
Take $\pi: \R^2\to\R: (x,y)\mapsto x$. Let $W$ be the smooth vector field $y\partial_x$ on $\R^2$. At any point $V_p= \Span\{\partial_y\}$ and $H_p= \Span\{\partial_x\}$, so $W$ is horizontal. But there is no vector field on $\R$ whose horizontal lift is $W$. Indeed $\diff{\pi}_p(W) = y\partial_x$ is not constant on $\pi^{-1}(p)$ because it depends on $y$.
\end{example}

\subsubsection{Riemannian submersions}
\begin{definition}
Let $\pi: (M,g)\to (M',g')$ be a smooth submersion between Riemannian manifolds. Then $\pi$ is a \udef{Riemannian submersion} if $\diff{\pi}_x|_{H_x}: H_x\to T_{\pi(x)}M'$ is a (bijective) linear isometry for all $x\in M$.
\end{definition}
Equivalently, the submersion $\pi$ is a Riemannian submersion if the metrics satisfy
\[ \forall x\in M:\; \forall v,w\in H_x:\; g_x(v,w) = g'_{\pi(x)}(\diff{\pi}_x(v), \diff{\pi}_x(w)). \]

\subsubsection{Riemannian coverings}

\subsection{Basic constructions derived from the metric}
\subsubsection{Raising and lowering indices}
Let $M$ be a smooth manifold. Given a Riemannian metric $g$ in $M$, we define a bundle homomorphism
\[ \hat{g}: TM \to T^*M: v\mapsto g_p(v,\cdot). \]
In other words we have $\hat{g}(v)(w) = g_p(v,w)$ for all $p\in M$ and $v,w\in T_pM$.

Musical isomorphisms

\subsubsection{Inner products of tensors}
We define $\inner{\omega, \eta}_g \defeq \inner{\omega^\sharp, \eta^\sharp}$.
Then
\[ \inner{\omega,\eta}= g_{kl}(g^{ki}\omega_i)(g^{lj}\eta_j) = \delta^i_lg^{lj}\omega_i\eta_j = g^{ij}\omega_i\eta_j. \]



\section{Connections}
\subsection{Affine connection}
\begin{definition}
Let $\pi: E\to M$ be a smooth vector bundle over a smooth manifold $M$ and let $\Gamma(E)$ denote the space of sections of $E$. A \udef{connection} in $E$ is a map
\[ \nabla: \mathfrak{X}(M)\times \Gamma(E) \to \Gamma(E): (X,Y)\mapsto \nabla_X Y \]
satisfying the following properties:
\begin{enumerate}
\item $\nabla_X Y$
\end{enumerate}
\end{definition}

\section{Geodesics}


\section{Curvature}