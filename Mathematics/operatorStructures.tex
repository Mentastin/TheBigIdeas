\chapter{Operator structures}
\begin{definition}
Let $A$ be a class. An \udef{operator structure} on $A$ is a structured class $\sSet{A, (O, e)}$ where $O$ is a class and $e: O\times A \not\to A$ a partial function such that $\setbuilder{e(f,-)}{f\in O}$ is closed under function composition.

We will often write $e_f$ for the partial application $e(f, -)$.

A \udef{homogeneous operator structure} is an operator structure of the form $\sSet{A, (A, e)}$.
\end{definition}
We have that for all $f,g\in O$, there exists $h\in O$ such that $e_f\circ e_g = e_h$.

\section{Operator structures}
\subsection{Composition functions}
TODO

\subsection{Operator identity}
\begin{definition}
Let $\sSet{A, (O, e)}$ be an operator structure. We call $f\in O$ an \udef{operator identity} if $e(f,x) =x$ for all $x\in A$ such that $e(f,x)$ is defined.
\end{definition}

\subsection{Operator substructures}
\begin{definition}
Let $\sSet{A, (O, e)}$ be an operator structure $B\subseteq A$ and $F\subseteq O$. Then $\sSet{B, (F, e)}$ is an \udef{operator substructure} of $\sSet{A, (O, e)}$ if it is an operator structure, i.e.\ if $e\imf(F\times B) \subseteq B$.
\end{definition}

\subsubsection{Principal ideals}
\begin{definition}
Let $\sSet{A, (O, e)}$ be an operator structure and $a\in A$. The \udef{principal ideal} $I_a$ of $a$ is the smallest operator structure that has $\sSet{\{a\}, (O,e)}$ as an operator substructure.

We say $a,b\in A$ are \udef{equivalent}, denoted $a\sim b$, if they have the same principal ideals.
\end{definition}
By the definition it is clear that $\sim$ is an equivalence relation.

\begin{lemma} \label{operatorStructurePrincipalIdeal}
Let $\sSet{A, (O, e)}$ be an operator structure and $a\in A$. Then the principal ideal of $a$ is $\{a\}\cup e^\imf(O\times\{a\})$.
\end{lemma}
\begin{proof}
Let $I_a$ be the principal ideal of $a$.
Clearly we have $e^\imf(O\times\{a\}) \subseteq B$, so it is enough to prove that the principal ideal is an operator substructure. Take $e_f(a) \in e^\imf(O\times\{a\})$ and $g\in O$. Then there exists $h\in O$ such that $e_h = e_g\circ e_f$, so $(e_g\circ e_f)(a) = e_h(a) \in e^\imf(O\times\{a\})$.
\end{proof}

\begin{lemma} \label{operatorStructureEquivalence}
Let $\sSet{A, (O, e)}$ be an operator structure and $a,b\in A$. Let $I_a, I_b$ be the principal ideals of $a$ and $b$. Then
\begin{enumerate}
\item $I_a \subseteq I_b$ \textup{if and only if} either $a=b$ or $\exists f\in O: e_f(b) = a$;
\item $a\sim b$ \textup{if and only if} either $a=b$ or $\exists f,g\in O: \; \big(e_f(a) = b\big) \land \big(a = e_g(b)\big)$.
\end{enumerate}
\end{lemma}
\begin{proof}
(1) First assume $I_a \subseteq I_b$. Then $a\in I_b$, so either $a=b$ or $a\in e^\imf(O\times\{b\})$ by \ref{operatorStructurePrincipalIdeal}.

Now assume the right-hand side. If $a=b$, then $I_a =I_b$ and the left-hand side holds. Now assume $e_f(b) = a$, then $I_b$ contains $a$ and is an operator substructure, so $I_a \subseteq I_b$.

(2) Immediate from (1).
\end{proof}


\begin{proposition} \label{operatorStructureEquivalenceCongruence}
Let $\sSet{A, (O, e)}$ be an operator structure. Then $\sim$ is a congruence.
\end{proposition}
\begin{proof}
Take $(a, b) \in {\sim}$. We need to prove that $e_h(a) \sim e_h(b)$ for all $h\in O$.

If $a=b$, then clearly $e_h(a) \sim e_h(b)$ for all $h\in O$. Now assume $a\neq b$. Then by \ref{operatorStructureEquivalence} there exist $f,g\in O$ such that $e_f(a) = b$ and $a = e_g(b)$. By closure of the operator set, there exists $f'\in O$ such that $e_{f'} = e_h \circ e_f$. Similarly there exists $g'\in O$ such that $e_{g'} = e_h\circ e_g$. Then
\[ e_{f'}(a) = (e_h \circ e_f)(a) = e_h(b) \qquad\text{and}\qquad e_{g'}(b) = (e_h \circ e_g)(b) = e_h(a), \]
so $e_h(a) \sim e_h(b)$ by \ref{operatorStructureEquivalence}.
\end{proof}


\section{Homogeneous operator structures}
\subsection{Point identity}
\begin{definition}
Let $\sSet{A, (A, e)}$ be a homogeneous operator structure. We call $a\in A$ a \udef{point identity} if $e(x,a) = x$ for all $x\in A$ such that $e(x,a)$ is defined.
\end{definition}

\subsection{Inverses}
\begin{definition}
Let $\sSet{A, (A, e)}$ be a homogeneous operator structure with (TODO point / operator) identity $f$ and $x\in A$. We call $y\in A$ an \udef{inverse} of $x$ if $e(y,x) = f$. (TODO or $e(x,y)$)
\end{definition}

\subsection{Idempotents}
\begin{definition}
Let $\sSet{A, (A, e)}$ be a homogeneous operator. An element $x\in A$ is called \udef{idempotent} if $e(x,x) = x$.
\end{definition}

\chapter{Operator bistructures and operator $n$-structures}
\begin{definition}
Let $A$ be a class. An \udef{operator bistructure} on $A$ is a structured class $\sSet{A, (L, \lambda), (R,\rho)}$ such that
\begin{itemize}
\item $\sSet{A, (L, \lambda)}$ and $\sSet{A, (R,\rho)}$ are operator structures;
\item $\lambda_f \circ \rho_g = \rho_g\circ \lambda_f$ for all $f\in L$ and $g\in R$.
\end{itemize}
An \udef{operator $n$-structure} on $A$ is a structured class $\sSet{A, \seq{(O_i, e_i)}_{i\in(0:n)}}$ such that
\begin{itemize}
\item $\sSet{A, (O_i, e_i)}$ is an operator structure for all $i\in (0:n)$;
\item $e_i(f,e_j(g,-)) = e_j(g,e_i(f,-))$ for all $i\neq j\in (0:n)$, $f\in e_i$ and $g\in e_j$.
\end{itemize}
\end{definition}

TODO notation: $x>y<z$.

Also $a>b>x<c<d$ is parsed as $a>(b>x<c)<d$

\section{Green's relations}
\begin{definition}
Let $\sSet{A, (L, \lambda), (R,\rho)}$ be an operator bistructure. Then
\begin{itemize}
\item we denote the equivalence of $\sSet{A, (L,\lambda)}$ as $\greensL$;
\item we denote the equivalence of $\sSet{A, (R,\rho)}$ as $\greensR$;
\item $\greensH \defeq \greensL \cap \greensR$;
\item $\greensD \defeq \greensL;\greensR$.
\end{itemize}
These equivalence relations are called \udef{Green's relations}.
\end{definition}

\begin{lemma}
Let $\sSet{A, (L, \lambda), (R,\rho)}$ be an operator bistructure. Then $\sim_L;\sim_R = \sim_R;\sim_L$.
\end{lemma}
\begin{proof}
Assume $x(\greensR;\greensL)z$, meaning $\exists y: x\greensR y$ and $y\greensL z$. Then there exist $a,b\in R$ and $c,d\in L$ such that
\[ \begin{tikzcd}
x \ar[r, maps to, shift left, "\rho_a"] & y \ar[l, maps to, shift left, "\rho_b"] \ar[r, maps to, shift left, "\lambda_c"] & z \ar[l, maps to, shift left, "\lambda_d"]
\end{tikzcd}. \]
Using the fact that the $\lambda$s and $\rho$s commute, we can rearrange such that we also get the mappings along the left and bottom sides of
\[ \begin{tikzcd}[sep=large]
x \ar[r, maps to, shift left, "\rho_a"] \ar[d, maps to, shift left, "\lambda_c"] & y \ar[l, maps to, shift left, "\rho_b"] \ar[d, maps to, shift left, "\lambda_c"] \\
y' \ar[u, maps to, shift left, "\lambda_d"] \ar[r, maps to, shift left, "\rho_a"] & z \ar[u, maps to, shift left, "\lambda_d"] \ar[l, maps to, shift left, "\rho_b"]
\end{tikzcd} \]
for some $y' \in A$. Thus $x(\greensL;\greensR)y$. The other inclusion is similar.
\end{proof}

\begin{corollary}
Let $\sSet{A, (L, \lambda), (R,\rho)}$ be an operator bistructure. Then $\greensD$ is indeed an equivalence relation.
\end{corollary}
\begin{proof}
By \ref{commutingEquivalenceRelations}.
\end{proof}

\begin{proposition}[Green's lemma] \label{GreensLemma}
Let $\sSet{A, (L, \lambda), (R,\rho)}$ be an operator bistructure and $x,y \in A$.
\begin{enumerate}
\item If $x\greensL y$ with $\begin{tikzcd}
x \ar[r, maps to, shift left, "\lambda_a"] & y \ar[l, maps to, shift left, "\lambda_b"]
\end{tikzcd}$, then
\begin{enumerate}
\item $\lambda_a|_{[x]_\greensR}: [x]_\greensR \to [y]_\greensR \qquad \text{is a bijection with inverse} \qquad \lambda_b|_{[y]_\greensR}: [y]_\greensR \to [x]_\greensR$;
\item $\lambda_a|_{[x]_\greensH}: [x]_\greensH \to [y]_\greensH \qquad \text{is a bijection with inverse} \qquad \lambda_b|_{[y]_\greensH}: [y]_\greensH \to [x]_\greensH$.
\end{enumerate}
\item If $x\greensR y$ with $\begin{tikzcd}
x \ar[r, maps to, shift left, "\rho_a"] & y \ar[l, maps to, shift left, "\rho_b"]
\end{tikzcd}$, then
\begin{enumerate}
\item $\rho_a|_{[x]_\greensL}: [x]_\greensL \to [y]_\greensL \qquad \text{is a bijection with inverse} \qquad \rho_b|_{[y]_\greensL}: [y]_\greensL \to [x]_\greensL$;
\item $\rho_a|_{[x]_\greensH}: [x]_\greensH \to [y]_\greensH \qquad \text{is a bijection with inverse} \qquad \rho_b|_{[y]_\greensH}: [y]_\greensH \to [x]_\greensH$.
\end{enumerate}
\end{enumerate}
\end{proposition}
\begin{proof}
Take some arbitrary $x'\in [x]_\greensR$. Then there exist $c,d\in R$ such that $\begin{tikzcd}
x \ar[r, maps to, shift left, "\rho_c"] & x' \ar[l, maps to, shift left, "\rho_d"]
\end{tikzcd}$. Then, because $\lambda$ and $\rho$ commute,
\[ \begin{tikzcd}
x' \ar[r, maps to, shift left, "\rho_d"] & x \ar[l, maps to, shift left, "\rho_c"] \ar[r, maps to, shift left, "\lambda_a"] & y \ar[l, maps to, shift left, "\lambda_b"]
\end{tikzcd} \qquad\text{implies that} \qquad \begin{tikzcd}
x' \ar[r, maps to, shift left, "\lambda_a"] & y' \ar[l, maps to, shift left, "\lambda_b"] \ar[r, maps to, shift left, "\rho_d"] & y \ar[l, maps to, shift left, "\rho_c"]
\end{tikzcd} \]
Thus, for all $x'\in [x]_\greensR$, we have
\begin{itemize}
\item $\lambda_a(x') \in [y]_\greensR$, meaning that $\lambda_a|_{[x]_\greensR}: [x]_\greensR \to [y]_\greensR$ is well-defined;
\item by similar reasoning, we can see that $\lambda_b|_{[y]_\greensR}: [y]_\greensR \to [x]_\greensR$ is also well-defined;
\item $\lambda_b(\lambda_a(x')) = x'$, so the functions are inverse of each other.
\end{itemize}
If $x'\in [x]_\greensH$, then $\lambda_a(x')\greensL \lambda_a(x) = y$ by \ref{operatorStructureEquivalenceCongruence}, so $\lambda_a(x')\in [y]_\greensH$. This means that $\lambda_a|_{[x]_\greensH}: [x]_\greensH \to [y]_\greensH$ is well-defined.

Point (2) is dual.
\end{proof}
\begin{corollary} \label{greensDisomorphism}
Let $\sSet{A, (L, \lambda), (R,\rho)}$ be an operator bistructure, $x, y\in A$ such that $x\greensD y$. Then there exist $a,c\in L$ and $b,d\in R$ such that
\[ \lambda_a\circ\rho_b|_{[x]_\greensH}: [x]_\greensH \to [y]_\greensH \qquad \text{is a bijection with inverse} \qquad \lambda_c\circ \rho_d|_{[y]_\greensH}: [y]_\greensH \to [x]_\greensH. \]
\end{corollary}
\begin{proof}
There exists a $z\in A$ such that $x\greensL z$ and $z\greensR y$. We then just compose the bijections in Green's lemma, keeping in mind that $\lambda$ and $\rho$ commute.
\end{proof}

\section{The intersection of operations and elements}
\subsection{Regular elements and generalised inverses}
\begin{definition}
Let $\sSet{A, (L, \lambda), (R,\rho)}$ be an operator bistructure and $x,y\in A\cap L\cap R$. We say
\begin{itemize}
\item $x$ is \udef{regular} if $\exists a\in A: \; x = (\lambda_x\circ\rho_x) (a) = x>a<x$;
\item $x$ and $y$ \udef{generalised inverses} if $x = x>y<x$ and $y = y>x<y$.
\end{itemize}
\end{definition}

\begin{lemma} \label{idempotentsAreRegular}
Let $\sSet{A, (L, \lambda), (R,\rho)}$ be an operator bistructure. Every idempotent is regular.
\end{lemma}
\begin{proof}
Let $x\in A\cap L\cap R$ be an idempotent. Because $x$ is a left-idempotent, we have $x>x = x$. Because $x$ is a right-idempotent, we have $x<x = x$. Combining the two gives $x = x>x<x$.
\end{proof}

\begin{proposition}
Let $\sSet{A, (L, \lambda), (R,\rho)}$ be an operator bistructure. If $x\in A$ is regular, then every element in $[x]_{\greensD}$ is regular.
\end{proposition}
So it makes sense to call a $\greensD$-class \udef{regular} if it consists of regular elements and \udef{irregular} otherwise.
\begin{proof}
Let $x$ be regular with $x = x>x'<x$ and $x\greensD y$. Then we have $a,c \in L$ and $b,d \in R$ such that
$\lambda_a\circ\rho_b|_{[x]_\greensH}: [x]_\greensH \to [y]_\greensH$ is a bijection with inverse $\lambda_c\circ \rho_d|_{[y]_\greensH}: [y]_\greensH \to [x]_\greensH$, as in \ref{greensDisomorphism}. Then we have
\begin{align*}
y &= a>x<b \\
&= a>(x>x'<x)<b \\
&= a . x>x'<x . b \\
&= a>(c>y<d)>x'<(c>y<d)<b \\
&= (a > c >) y > d > x' > c > y (> d > b) \\
&= y > d > x' > c > y.
\end{align*}
So $y$ is regular.
\end{proof}
\begin{corollary}
If there is an idempotent $x\in [a]_{\greensD}$, then $[a]_{\greensD}$ is regular.
\end{corollary}
\begin{proof}
Every idempotent is regular: \ref{idempotentsAreRegular}.
\end{proof}

\begin{proposition}
Let $\sSet{A, f}$ be an associative class. Then $x$ is regular \textup{if and only if} it has a generalised inverse.
\end{proposition}
\begin{proof}
Clearly every element with a generalised inverse is regular. Conversely, assume $x$ regular with $x = xax$. Then $y = axa$ is a generalised inverse of $x$: $x(axa)x = xax = x$ and $(axa)x(axa) = a(xax)axa = axaxa = axa$.
\end{proof}
Note that we do not have that $x = xyx$ implies $y = yxy$.

\begin{proposition} \label{greensRelationsRegularElements}
Let $\sSet{A, f}$ be an associative class and $x\in A$ a regular element with $x = f(f(x, y), x)$. Then
\begin{enumerate}
\item $f(x,y)$ and $f(y,x)$ are idempotent;
\item $f(y,x) \greensL x$ and $x \greensR f(x, y)$.
\end{enumerate}
\end{proposition}
\begin{proof}
(1) We calculate
\[ f(f(x,y), f(x,y)) = f(f(f(x,y), x), y) = f(x, y)\] and \[f(f(y,x), f(y,x)) = f(y, f(x, f(y,x))) = f(y,x). \]

(2) Using \ref{idealAbsorption}, we have
\[ f(A, x) = f(A, f(x,f(y,x))) \subseteq f(A, f(y,x)) \subseteq f(A,x). \]
Thus $f(A, x) = f(A, f(y,x))$. The second part is dual.
\end{proof}
\begin{corollary}
In a regular $\mathcal{D}$-class each $\mathcal{L}$-class and each $\mathcal{R}$-class contains at least one idempotent.
\end{corollary}
\begin{proof}
Let $[x]_\mathcal{L}$ be an $\mathcal{L}$-class in a regular $\mathcal{D}$-class. By regularity there exists a $y\in A$ such that $xyx = x$. From the proposition, we have that $[x]_\mathcal{L}$ contains the idempotent $yx$ and $[x]_\mathcal{R}$ the idempotent $xy$.
\end{proof}
\begin{corollary} \label{idempotentsHclass}
If $x,x'\in A$ are generalised inverses, then $[xx']_\mathcal{H} = [x]_\mathcal{R}\cap [x']_\mathcal{L}$ and $[x'x]_\mathcal{H} = [x']_\mathcal{R}\cap [x]_\mathcal{L}$.
\end{corollary}
\begin{proof}
From $x\greensR (xx')$ and $(xx')\greensL x'$, we get the first equality. The second is dual.
\end{proof}
We can depict the situation in the corollary as follows:
\[ \hspace{-8.4em} \exists a,b,c,d \in \widetilde{A}: \qquad \begin{tikzcd}[sep=large]
x \ar[r, maps to, shift left, "\rho_a"] \ar[d, maps to, shift left, "\lambda_c"] & xx' \ar[l, maps to, shift left, "\rho_b"] \ar[d, maps to, shift left, "\lambda_c"] \\
x'x \ar[u, maps to, shift left, "\lambda_d"] \ar[r, maps to, shift left, "\rho_a"] & x' \ar[u, maps to, shift left, "\lambda_d"] \ar[l, maps to, shift left, "\rho_b"]
\end{tikzcd} \]

So generalised inverses along one diagonal imply idempotents along the other. In fact, the other direction also holds:
\begin{proposition}
Let $\sSet{A, f}$ be an associative class and $e,f$ idempotents in $A$ such that $e\greensD f$. Then there exist $x\in [e]_\greensR\cap [f]_\greensL$ and $x'\in [e]_\greensL\cap [f]_\greensR$ such that
\begin{itemize}
\item $x,x'$ are generalised inverses;
\item $e = xx'$ and $f = x'x$.
\end{itemize}
\end{proposition}
\begin{proof}
Because $e\greensD f$, we can find $x,x'\in A$ such that
\[ \hspace{-8.4em} \exists a,b,c,d \in \widetilde{A}: \qquad \begin{tikzcd}[sep=large]
e \ar[r, maps to, shift left, "\rho_a"] \ar[d, maps to, shift left, "\lambda_c"] & x \ar[l, maps to, shift left, "\rho_b"] \ar[d, maps to, shift left, "\lambda_c"] \\
x' \ar[u, maps to, shift left, "\lambda_d"] \ar[r, maps to, shift left, "\rho_a"] & f \ar[u, maps to, shift left, "\lambda_d"] \ar[l, maps to, shift left, "\rho_b"]
\end{tikzcd} \]
Then
\begin{align*}
xx' &= (df)(fb) = dfb = e \\
x'x &= (ce)(ea) = cea = f \\
xx'x &= ex = eea = ea = x \\
x'xx' &= fx' = ffb = fb = x',
\end{align*}
which completes the proof.
\end{proof}
\begin{corollary}
Let $\sSet{A, f}$ be an associative class, $y\in A$ and $e,f$ idempotents in $A$. Then $e\greensD f$ \textup{if and only if} there exist generalised inverses $x,x'$ such that $e = xx'$ and $f = x'x$.
\end{corollary}
\begin{proof}
The direction $\Rightarrow$ follows from the proposition. The converse from \ref{greensRelationsRegularElements}.
\end{proof}

\begin{lemma}
Let $\sSet{A, f}$ be an associative class, $x\in A$. Then no $\greensH$-class contains more than one generalised inverse of $x$.
\end{lemma}
\begin{proof}
Assume $x$ has two generalised inverses, $x_1'$ and $x_2'$. From \ref{idempotentsHclass} and \ref{GreensTheoremCorollary} we get that $xx_1' = xx_2'$ and $x_1'x = x_2'x$. Thus
\[ x_1' = x_1'(xx_1') = x_1'xx_2' = (x_1'x)x_2' =  x_2'xx_2' = x_2'. \]
\end{proof}

\begin{proposition}
Let $\sSet{A, f}$ be an associative class and $x,y\in A$. Then $xy\in [x]_\greensR \cap [y]_\greensL$ \textup{if and only if} $[x]_\greensL \cap [y]_\greensR$ contains an idempotent.
\end{proposition}
\begin{proof}
First assume $[x]_\greensL \cap [y]_\greensR$ contains an idempotent $e$. We can depict the situation as
\[ \hspace{-8.4em} \exists a,b,c,d \in \widetilde{A}: \qquad \begin{tikzcd}[sep=large]
x \ar[r, maps to, shift left, "\rho_a"] \ar[d, maps to, shift left, "\lambda_c"] &  \ar[l, maps to, shift left, "\rho_b"] \ar[d, maps to, shift left, "\lambda_c"] \\
e \ar[u, maps to, shift left, "\lambda_d"] \ar[r, maps to, shift left, "\rho_a"] & y \ar[u, maps to, shift left, "\lambda_d"] \ar[l, maps to, shift left, "\rho_b"]
\end{tikzcd} \]
Then we can calculate
\[ xy = deea = dea = xa \in [x]_\greensR \cap [y]_\greensL. \]
Now assume $xy\in [x]_\greensR \cap [y]_\greensL$.
We can depict the situation as
\[ \hspace{-8.4em} \exists a,b,c,d \in \widetilde{A}: \qquad \begin{tikzcd}[sep=large]
x \ar[r, maps to, shift left, "\rho_a"] \ar[d, maps to, shift left, "\lambda_c"] & xy \ar[l, maps to, shift left, "\rho_b"] \ar[d, maps to, shift left, "\lambda_c"] \\
e \ar[u, maps to, shift left, "\lambda_d"] \ar[r, maps to, shift left, "\rho_a"] & y \ar[u, maps to, shift left, "\lambda_d"] \ar[l, maps to, shift left, "\rho_b"]
\end{tikzcd} \]
Now we need to show that $e$ is idempotent. Indeed, starting from the three other corners, we see that $e = cx$ and $e = yb$ and $e = c(xy)b = (cx)(yb) = ee$.
\end{proof}


\chapter{Associative classes}
\begin{definition}
Let $A$ be a class and $f: A\times A\to A$ a binary function. Then $f$ is called \udef{associative} if $\sSet{A, (A, f), (A, f^d)}$ is an operator bistructure.

We call $\sSet{A, f}$ an \udef{associative class}.
\end{definition}

Any property related to the operator structure $\sSet{A, (A,f)}$ is prefixed by ``left'' and any property related to the operator structure $\sSet{A, (A, f^d)}$ is rpefixed by ``right''.

\section{Undefined operations}
We can simulate a partial function by adding an absorbing element $u$ and mapping undefined operations to $u$. The problem with this is that elements can then only be cancellative if they can be composed with anything. Indeed, if $a,b\in A$ such that $f(a,b) = u$. Then $f(a,b) = u = f(a,u)$ and the only way $a$ can be left-cancellative is by having $b=u$.

The problem is that we have made any two undefined operations the same, while we really want any two undefined operations to be different.

TODO: modify set theory to allow $u\neq u$??

\subsection{Connections}
\begin{definition}
Let $A$ be an associative class and $x,y\in A$. We call
\begin{itemize}
\item $\leftconnections{x} \defeq \setbuilder{y\in A}{\text{$yx$ is defined}}$ the class of \udef{left connections};
\item $\rightconnections{x} \defeq \setbuilder{y\in A}{\text{$xy$ is defined}}$ the class of \udef{right connections}.
\end{itemize}
We write
\begin{itemize}
\item $x \preceq_L y$ if $\leftconnections{x} \subseteq \leftconnections{y}$;
\item $x \preceq_R y$ if $\rightconnections{x} \subseteq \rightconnections{y}$.
\end{itemize}
\end{definition}

\begin{lemma}
Let $A$ be an associative class and $x,y,z\in A$. Then
\begin{enumerate}
\item $\leftconnections{xy} = \leftconnections{x}$;
\item $\rightconnections{xy} = \rightconnections{y}$;
\item $x \preceq_L y \iff xz \preceq_L y \iff x \preceq_L yz$ if the relevant terms are defined;
\item $x \preceq_R y \iff zx \preceq_R y \iff x \preceq_R zy$ if the relevant terms are defined.
\end{enumerate}
\end{lemma}


\section{Types and properties of elements}
\subsection{Distinguishability}
\begin{definition}
Let $A$ be an associative class and $x,y\in A$. 
We say
\begin{itemize}
\item $x$ and $y$ are \udef{left-distinguishable} if
\[ x\neq y \quad\implies\quad \exists a\in A: \;\text{$ax$ or $ay$ is defined and $ax \neq ay$}; \]
\item $x$ and $y$ are \udef{right-distinguishable} if
\[ x\neq y \quad\implies\quad \exists a\in A: \; \text{$xa$ or $ya$ is defined and $xa \neq ya$}; \]
\item $x$ and $y$ are \udef{distinguishable} if they are left- \emph{or} right-distinguishable;
\item $x$ and $y$ are \udef{bidistinguishable} if they are left- \emph{and} right-distinguishable.
\end{itemize}
We say
\begin{itemize}
\item $A$ is left-distinguishable if every two elements in $A$ are left-distinguishable;
\item $A$ is right-distinguishable if every two elements in $A$ are right-distinguishable;
\item $A$ is distinguishable if every two elements in $A$ are distinguishable;
\item $A$ is bidistinguishable if every two elements in $A$ are bidistinguishable.
\end{itemize}
\end{definition}

\begin{lemma}
Let $A$ be an associative class and $x,y\in A$. 
Then
\begin{enumerate}
\item $x$ and $y$ are left-distinguishable \textup{if and only if}
\[ x = y \quad\iff\quad \forall a\in A: \;\Big(\text{$ax$ or $ay$ is defined} \implies ax = ay\Big); \]
\item $x$ and $y$ are right-distinguishable \textup{if and only if}
\[ x = y \quad\iff\quad \forall a\in A: \; \Big(\text{$xa$ or $ya$ is defined} \implies xa = ya\Big). \]
\end{enumerate}
In both cases the direction $\Rightarrow$ is automatic.
\end{lemma}

\subsubsection{Cancellation}
\begin{definition}
Let $A$ be an associative class and $x\in A$. 
We call $x$
\begin{itemize}
\item \udef{left-cancellative} or \udef{monic} if $x\cdot -: y\mapsto xy$ is injective;
\item \udef{right-cancellative} or \udef{epic} if $-\cdot x: y\mapsto yx$ is injective.
\end{itemize}
\end{definition}

Left and right cancellative are dual properties.

\begin{lemma}
Let $A$ be an associative class and $x,y$ in $A$.
\begin{enumerate}
\item If $x$ and $y$ are left-(resp. right-)cancellative, then $xy$ is left-(resp. right-)cancellative.
\item If $xy$ is left-cancellative, then $y$ is left-cancellative.
\item If $xy$ is right-cancellative, then $x$ is right-cancellative.
\end{enumerate}
\end{lemma}
\begin{proof}
Let $z_1, z_2\in A$.

(1) Assume that $x$ and $y$ are left-cancellative and $(xy)z_1 = (xy)z_2$. By associativity, we have $x(yz_1) = x(yz_2)$. Thus by injectivity we get first $yz_1 = yz_2$ and then $z_1 = z_2$.

Right-cancellation is similar.

(2) Assume $yz_1 = yz_2$. Then $xyz_1 = xyz_2$, so $z_1 = z_2$ because $xy$ is left-cancellative.

(3) Assume $z_1x = z_2x$. Then $z_1xy = z_2xy$ so $z_1 = z_2$ because $xy$ is right-cancellative.
\end{proof}

\begin{lemma}
Let $A$ be an associative class.
\begin{enumerate}
\item If $\forall x\in A$ there exists a left-cancellative element $a_x$ such that $a_xx$ exists, then $A$ is left-distinguishable.
\item If $\forall x\in A$ there exists a right-cancellative element $a$ such that $ax$ exists, then $A$ is right-distinguishable.
\end{enumerate}
\end{lemma}
\begin{proof}
(1) Take $x,y\in A$ and assume $\forall b\in A: \Big(\text{$bx$ or $by$ is defined} \implies bx = by\Big)$. In particular, this means that $a_xx = a_xy$. By left-cancellation, we have $x=y$.

(2) Dual to (1).
\end{proof}

\subsubsection{Identity and objects}
\begin{definition}
Let $A$ be a class, $f: A\times A \not\to A$ a binary partial function and $e\in A$ an idempotent (i.e.\ $e^2 = e$).

We call $e$
\begin{itemize}
\item a \udef{centre identity} if $xy = xey$ for all $x,y\in A$ such that both sides are defined;
\item a \udef{left identity} if $x = ex$ for all $x\in A$ such that $ex$ is defined;
\item a \udef{right identity} if $x = xe$ for all $x\in A$ such that $xe$ is defined;
\item an \udef{identity} if $e$ is both a left and a right identity.
\end{itemize}

We call $e$
\begin{itemize}
\item a \udef{weak left identity} if $x = ex$ for all $x\in A$ such that $ex$ is defined and $\leftconnections{e} = \leftconnections{x}$;
\item a \udef{weak right identity} if $x = xe$ for all $x\in A$ such that $xe$ is defined and $\rightconnections{e} = \rightconnections{x}$;
\item a \udef{weak identity} if $e$ is both a weak left and a weak right identity.
\end{itemize}
\end{definition}

\begin{lemma}
Let $A$ be an associative class and $e\in A$ an idempotent.
Then
\begin{enumerate}
\item $e$ is a left identity \textup{if and only if} $e$ is left-cancellative;
\item $e$ is a right identity \textup{if and only if} $e$ is right-cancellative.
\end{enumerate}
\end{lemma}
\begin{proof}
(1) First assume $e$ is a left identity. Let $x,y\in A$ be such that $ex = ey$. Then $x = ex = ey = y$.

Now assume that $e$ is left-cancellative and that $ex$ is defined. Then $ex = eex$ by idempotency and thus $x = ex$ by left-cancellation.

(2) Dual.
\end{proof}

\begin{lemma} \label{uniquenessIdentity}
Let $A$ be an associative class and $e,e'\in A$. Then
\begin{enumerate}
\item if $e,e'$ are weak left identities such that $ee' = e'e$, then $e=e'$;
\item if $e,e'$ are weak right identities such that $ee' = e'e$, then $e=e'$.
\end{enumerate}
Also
\begin{enumerate} \setcounter{enumi}{2}
\item if $e$ is a left identity and $e'$ a right identity such that $ee'$ is defined, then $e=e'$;
\end{enumerate}
and
\begin{enumerate} \setcounter{enumi}{3}
\item if $e,e'$ are identities and there exists $x\in A$ such that $ex$ and $e'x$ are both defined, then $e = e'$;
\item if $e,e'$ are identities and there exists $x\in A$ such that $xe$ and $xe'$ are both defined, then $e = e'$.
\end{enumerate}
\end{lemma}
\begin{proof}
(1) We have $\leftconnections{e} = \leftconnections{ee'} = \leftconnections{e'e} = \leftconnections{e'}$, so $e = e'e = ee' = e'$.

(2) Dual.

(3) We have $e = ee' = e'$.

(4) We have $x = ex = ee'x$, so $ee'$ is defined. We conclude with (3).

(5) Dual to (4).
\end{proof}

\begin{lemma} \label{distinguishableIdentity}
Let $A$ be an associative class and $e\in A$.
\begin{enumerate}
\item If $e$ is a left identity and $A$ is right-distinguishable, then $e$ is a weak right identity.
\item If $e$ is a right identity and $A$ is left-distinguishable, then $e$ is a weak left identity.
\end{enumerate}
\end{lemma}
\begin{proof}
(1) Take $x\in A$ such that $xe$ is defined and $\rightconnections{e} = \rightconnections{x}$. Take arbitrary $a\in A$. If $xa$ is defined, then $xea$ is also defined and $xea = xa$. By right-distinguishability, we have $xe = x$. Thus $e$ is a right identity.

(2) Dual.
\end{proof}

\begin{lemma} \label{identityConnection}
Let $A$ be an associative class and $e,x\in A$. Then
\begin{enumerate}
\item if $e$ is a left identity such that $ex$ is defined, then $\leftconnections{e} = \leftconnections{x}$;
\item if $e$ is a right identity such that $xe$ is defined, then $\rightconnections{e} = \rightconnections{x}$;
\end{enumerate}
and
\begin{enumerate} \setcounter{enumi}{2}
\item if $e$ is a left identity such that $xe$ is defined, then $e \preceq_R x$;
\item if $e$ is a right identity such that $ex$ is defined, then $e \preceq_L x$.
\end{enumerate}
\end{lemma}
\begin{proof}
(1) We have $\leftconnections{e} = \leftconnections{ex} = \leftconnections{x}$.

(2) Dual.

(3) Take $a\in \rightconnections{e} = \rightconnections{xe}$. Then $xea$ is defined and $xea = xa$, so $a\in \rightconnections{x}$.

(4) Dual.
\end{proof}


\subsubsection{Inverses}
\begin{definition}
Let $A$ be an associative class and $x\in A$. 
We say an element $y\in A$ is
\begin{itemize}
\item a \udef{left-inverse} of $x$ if $yx$ is a left-identity;
\item a \udef{right-inverse} of $x$ if $xy$ is a right-identity;
\item a \udef{(two-sided) inverse} of $x$ it is both a left- and a right-inverse.
\end{itemize}
We say $(x,y)\in A^2$ is a pair of
\begin{itemize}
\item \udef{mutual left-inverses} if $x$ is a left-inverse of $y$ and $y$ a left-inverse of $x$;
\item \udef{mutual right-inverses} if $x$ is a right-inverse of $y$ and $y$ a right-inverse of $x$.
\end{itemize}
We call $x$ \udef{invertible} if it has a (left/right) mutual inverse.
\end{definition}

\begin{lemma}
Let $A$ be an associative class and $x\in A$.
\begin{enumerate}
\item If $x$ has a left-inverse, then it is left-cancellative.
\item If $x$ has a right-inverse, then it is right-cancellative.
\end{enumerate}
\end{lemma}
\begin{proof}
(1) Let $y\in A$ be a left-inverse of $x$. Take arbitrary $a,b\in A$. Assume $xa = xb$. Then $yxa = yxb$ and so $a = b$.

(2) Dual.
\end{proof}

\begin{proposition} \label{inverseUniqueness}
Let $A$ be an associative class and $x,y_1,y_2,l,r \in A$.
\begin{enumerate}
\item If $A$ is right-distinguishable and both $(x, y_1)$ and $(x,y_2)$ are pairs of mutual left-inverses, then $y_1 = y_2$.
\item If $A$ is left-distinguishable and both $(x, y_1)$ and $(x,y_2)$ are pairs of mutual right-inverses, then $y_1 = y_2$.
\end{enumerate}
Also
\begin{enumerate} \setcounter{enumi}{2}
\item If $l$ is a left-inverse of $x$ and $r$ a right-inverse of $x$, then $l = r$.
\item If $x$ has an inverse, then this inverse is unique.
\end{enumerate}
\end{proposition}
\begin{proof}
(1) We have $y_1 = (y_2x)y_1 = y_2(xy_1)$, so the latter is defined, which implies $xy_1 \preceq_R y_2$ by \ref{identityConnection}. Thus $y_1 \preceq y_2$. Similarly $y_1xy_2$ is defined and this implies $y_2 \preceq y_1$. Thus $\rightconnections{xy_1} = \rightconnections{y_1} = \rightconnections{y_2}$. By \ref{distinguishableIdentity} $xy_1$ is a weak right identity, so $\rightconnections{xy_1} = \rightconnections{y_2}$ implies $y_1 = y_2xy_1 = y_2$.

(2) Dual.

(3) We have $l = l(xr) = (lx)r = r$.

(4) Any inverse of $x$ is both a left- and a right-inverse.
\end{proof}

If $x$ is invertible, we denote the unique inverse by $x^{-1}$.




\begin{lemma}
Let $A$ be an associative class and $x,y \in A$ such that $xy$ is defined.
\begin{enumerate}
\item If $x$ has left-inverse $x^{-L}$ and $y$ has a left-inverse $y^{-L}$, then $xy$ has a left-inverse $y^{-L}x^{-L}$.
\item If $x$ has right-inverse $x^{-R}$ and $y$ has a right-inverse $y^{-R}$, then $xy$ has a right-inverse $y^{-R}x^{-R}$.
\item If $x$ has inverse $x^{-1}$ and $y$ has inverse $y^{-1}$, then $xy$ has inverse $y^{-1}x^{-1}$.
\item If $xy$ has a left-inverse $z$, then $y$ has a left-inverse $zx$.
\item If $xy$ has a right-inverse $z$, then $x$ has a right-inverse $yz$.
\end{enumerate}
\end{lemma}
\begin{proof}
(1) Set $e = x^{-L}x$. We calculate
\[ y^{-L}x^{-L}xy = y^{-L}ey = y^{-L}y, \]
which is a left identity. Then we just need to show that $xyy^{-L}x^{-L}xy = xy$. Indeed
\[ xyy^{-L}x^{-L}xy = xyy^{-L}ey = x(yy^{-L}y) = xy. \]

(2) Dual to (1).

(3) Follows from (1) and (2).

(4) Set $e = z(xy)$. We calculate
\[ e = z(xy) = (zx)y, \]
so $zx$ is a left-inverse of $y$.
\end{proof}
\begin{corollary} \label{productInvertibility}
Let $A$ be an associative class and $x,y \in A$ such that $xy$ is defined. Then $x$ and $y$ are invertible \textup{if and only if} $xy$ and $yx$ are invertible.
\end{corollary}


\begin{lemma}
Let $\sSet{A, f}$ be an associative class and $x,y$ in $A$ with identity $e$. Then
\begin{enumerate}
\item if $x$ has a left inverse, it is left-cancellative;
\item if $x$ has a right inverse, it is right-cancellative;
\item if $x$ has a left inverse and is right-cancellative, it is invertible;
\item if $x$ has a right inverse and is left-cancellative, it is invertible.
\end{enumerate}
\end{lemma}
\begin{proof}
(1) Let $l$ be a left inverse of $x$ and assume $f(x, z_1) = f(x,z_2)$, then $f(l, f(x,z_1)) = f(l, f(x,z_2))$ and thus
\[ z_1 = f(e,z_1) = f(f(l,x), z_1) = f(l, f(x,z_1)) = f(l, f(x,z_2)) = f(f(l,x), z_2) = f(e,z_2) = z_2. \]

(2) Similar.

(3) Let $l$ be a left inverse of $x$. It is enough to show that $l$ is also a right inverse of $x$. We calculate
\[ f(f(x,l), x) = f(x, f(l,x)) = f(x, e) = x = f(e,x). \]
Beacuse $x$ is right-cancellative, this means $f(x,l) = e$ and thus that $l$ is a right inverse.

(4) Similar.
\end{proof}


\section{Left and right relation}
\begin{definition}
Let $\sSet{A, f}$ be an associative class and $x, y\in A$. Then
\begin{itemize}
\item let $L$ be the relation defined by $xLy \defequiv \exists a: f(a, x) = y$;
\item let $R$ be the relation defined by $xRy \defequiv \exists a: f(x, a) = y$;
\item let $L'$ be the relation defined by $xL^!y \defequiv  \exists! a: f(a, x) = y$;
\item let $R'$ be the relation defined by $xR^!y \defequiv \exists! a: f(x, a) = y$.
\end{itemize}
\end{definition}

The relations $L$ and $R$ are dual. The relations $L^!$ and $R^!$ are dual.

\begin{lemma}
The relations $L$ and $R$ are transitive.
\end{lemma}
\begin{proof}
Let $\sSet{A, f}$ be an associative class and $x, y, z\in A$ such that $xLy$ and $yLz$. Then there exist $a,b\in A$ such that $f(a,x) = y$ and $f(b,y) = z$. Then
\[ z = f(b,y) = f\big(b,f(a,x)\big) = f\big(f(b,a), x\big), \]
so $xLz$. The statement for $R$ is dual.
\end{proof}

\begin{lemma}
Let $\sSet{A, f}$ be an associative class and $x\in A$. Then the following are equivalent:
\begin{enumerate}
\item $x$ is right-cancellative
\item $xL \subseteq xL^!$;
\item $xL = xL^!$.
\end{enumerate}
As are the following:
\begin{enumerate}
\item $x$ is left-cancellative
\item $xR \subseteq xR^!$;
\item $xR = xR^!$.
\end{enumerate}
\end{lemma}
\begin{proof}
$(1) \Rightarrow (2)$ Take $y\in xL$. Then there exists $a\in A$ such that $f(a,x) = y$. Assume there exists another $b\in A$ such that $f(b,x) = y$. Then $f(a,x) = f(b,x)$, so $a=b$. Thus the $a\in A$ is unique and $y\in xL^!$.

$(2) \Rightarrow (3)$ The inclusion $xL \supseteq xL^!$ is immediate.

$(3) \Rightarrow (1)$ Take $a,b\in A$ and assume $f(a,x) = f(b,x) = y$. Then $xLy$, so $xL^!y$, so $a=b$.

The proof of the second part is dual.
\end{proof}

\begin{lemma}
Let $\sSet{A, f}$ be an associative class. Then
\begin{enumerate}
\item $L = \bigcup_{a\in A}f(a, -)$;
\item $R = \bigcup_{a\in A}f(-, a)$.
\end{enumerate}
\end{lemma}

For given and fixed $f$, we will often write
\begin{itemize}
\item $\lambda_a: A\to A$ for $f(a, -)$;
\item $\rho_a: A\to A$ for $f(-, a)$.
\end{itemize}

\begin{lemma} \label{lambdaRhoCommute}
For all $a,b\in A$, we have $\lambda_a\circ \rho_b = \rho_b \circ \lambda_a$.
\end{lemma}
\begin{proof}
We calculate, for arbitrary $x\in A$,
\[ \lambda_a\big(\rho_b(x)\big) = f\big(a, f(x,b)\big) = f\big(f(a,x), b\big)  = \rho_b\big(\lambda_a(x)\big). \]
\end{proof}
\begin{corollary} \label{LRcommute}
Let $A$ be a class and $f: A\times A \to A$ an associative binary function. Then $L;R = R;L$.
\end{corollary}
\begin{proof}
Let $x,y\in A$. Then $x(L;R)y$ iff there exist $a,b\in A$ such that the top path in
\[ \begin{tikzcd}
x \ar[r, maps to, "\lambda_a"] \ar[d, maps to, swap, "\rho_b"] & f(a, x) \ar[d, maps to, "\rho_b"] \\
f(x, b) \ar[r, maps to, swap, "\lambda_a"] & y
\end{tikzcd} \]
holds. By \ref{lambdaRhoCommute} this is equivalent to the bottom path holding. The bottom path implies $x(R;L)y$.
\end{proof}

\begin{proposition} \label{functionsLeftRightRelations}
Let $g,h$ be functions in the assocative class of functions with composition $\circ$. Then
\begin{enumerate}
\item $gLh$ \textup{if and only if} $\ker g \subseteq \ker h$;
\item $gRh$ \textup{if and only if} $\im g \supseteq \im h$. 
\end{enumerate}
\end{proposition}

\subsection{Principal ideals}
\begin{definition}
Let $\sSet{A, f}$ be an associative class and $x, y\in A$. Then
\begin{itemize}
\item the \udef{left principal ideal} generated by $x$ is $f(\widetilde{A}, x) = f(A, x)\cup \{x\}$;
\item the \udef{right principal ideal} generated by $x$ is $f(x, \widetilde{A}) = f(x,A)\cup \{x\}$.
\end{itemize}
\end{definition}

\begin{lemma} \label{idealAbsorption}
Let $\sSet{A, f}$ be an associative class and $x, y\in A$. Then
\begin{enumerate}
\item $f(A, f(x,y)) \subseteq f(A, y)$;
\item $f(f(x,y), A) \subseteq f(x, A)$.
\end{enumerate}
\end{lemma}
\begin{proof}
(1) Take $z\in f(A, f(x,y))$. Then there exists $z'\in A$ such that
\[ z = f(z', f(x,y)) = f(f(z', x), y) \in f(A, y). \]

(2) By duality.
\end{proof}

\begin{lemma}
Let $\sSet{A, f}$ be an associative class and $x, y\in A$. Then
\begin{enumerate}
\item $f(A, x) \subseteq f(A,y)$ \textup{if and only if} $\forall a\in A: \exists b\in A: \; f(a, x) = f(b, y)$;
\item $f(x, A) \subseteq f(y, A)$ \textup{if and only if} $\forall a\in A: \exists b\in A: \; f(x, a) = f(y, b)$.
\end{enumerate}
If $A$ contains an identity $e$, then
\begin{enumerate} \setcounter{enumi}{2}
\item $f(A, x) \subseteq f(A,y)$ \textup{if and only if} $\exists b\in A: \; x = f(b, y)$;
\item $f(x, A) \subseteq f(y, A)$ \textup{if and only if} $\exists b\in A: \; x = f(y, b)$.
\end{enumerate}
\end{lemma}

\begin{proposition} \label{invertibilityFromPrincipalIdeals}
Let $\sSet{A, f}$ be an associative class. Then $f$ has an identity and every $x\in A$ is invertible \textup{if and only if}
\[ \forall x\in A: \quad f(x, A) = A = f(A,x). \]
\end{proposition}
\begin{proof}
$\Rightarrow$ We clearly have $f(x, A) \subseteq A$. The other inclusion follows from \ref{idealAbsorption}: $A = f(A, e) = f(A, f(x^{-1}, x)) \subseteq f(A, x)$.

$\Leftarrow$ Pick some $x\in A$, so $x\in f(x,a)$, meaning there exists an $a\in A$ such that $x = xa$. We claim $a$ is a right-identity for $f$. Indeed, take arbitrary $y\in A$. Then $y = f(b,x)$ for some $b\in A$ and so
\[ f(y, a) = f(f(b,x), a) = f(b,f(x,a)) = f(b,x) = y. \]
In the same way we can also find a left-identity. So $A$ contains an identity $e \defeq a$ by \ref{leftRightIdentity}.

Now for all $x\in A$ we have $e\in A = f(x,A)$, so we can find a right-inverse of $x$. Similarly, we can find a left-inverse of $x$. This means $x$ is invertible by \ref{leftRightInverse}.
\end{proof}


\subsection{Green's relations}
\begin{lemma}
Let $\sSet{A, f}$ be an associative class. Then $\greensL$ is the reflexive closure of the symmetric part of $L$ and $\greensR$ is the reflexive closure of the symmetric part of $R$.
\end{lemma}

\begin{lemma}
Let $\sSet{A, f}$ be an associative class and $x, y\in A$. Then the following are equivalent:
\begin{enumerate}
\item $x \greensL y$;
\item $x\big((L\cap L^\transp) \cup \id_A\big)y$
\item $\exists a,b\in \widetilde{A}: (f(a, x) = y) \land (f(b, y) = x)$;
\item $\begin{tikzcd}[sep=large]
x \ar[r, maps to, shift left, "\exists a: \lambda_a"] & y \ar[l, maps to, shift left, "\exists b: \lambda_b"]
\end{tikzcd}$
\item $f(\widetilde{A},x) = f(\widetilde{A}, y)$;
\item $f(A,x) = f(A, y)$;
\end{enumerate}
as are
\begin{enumerate}
\item $x \greensR y$;
\item $x\big((R\cap R^\transp) \cup \id_A\big)y$
\item $\exists a,b\in \widetilde{A}: (f(x, a) = y) \land (f(y, b) = x)$;
\item $\begin{tikzcd}[sep=large]
x \ar[r, maps to, shift left, "\exists a: \rho_a"] & y \ar[l, maps to, shift left, "\exists b: \rho_b"]
\end{tikzcd}$
\item $f(x, \widetilde{A}) = f(y, \widetilde{A})$;
\item $f(x, A) = f(y, A)$.
\end{enumerate}
\end{lemma}


\subsubsection{Egg-box diagrams}


From \ref{commutingEquivalenceRelations} we also know that the $\greensL,\greensR$-egg-box diagram decomposes into blocks, which are $\greensD$-equivalence classes. The columns are $\greensL$-equivalence classes, the rows are $\greensR$-equivalence classes, and the cells are $\greensH$-equivalence classes.

\begin{example}
Let $A = (\{1,2,3\} \to \{1,2,3\})$ with the binary function $A\times A \to A: (f,g)\mapsto f;g$. We can represent an element $f$ of $A$ as $(f(1) f(2) f(3))$. We have
\begin{itemize}
\item $f\greensL g$ if $f$ and $g$ have the same image;
\item $f\greensR g$ if $f$ and $g$ have the same kernel.
\end{itemize}
An egg-box diagram can be drawn as follows:
\[ \begin{array}{|c|c|c|c|c|c|c|}
\hline
\mathbf{(1 1 1)} & \mathbf{(2 2 2)} & \mathbf{(3 3 3)} &&& &  \\ \hline
&& & \mathbf{(1 2 2)}, & \mathbf{(1 3 3)}, & (2 3 3), &  \\
&& & (2 1 1)  & (3 1 1)  & (3 2 2)  &  \\ \hline
&& & (2 1 2), & (3 1 3), & \mathbf{(3 2 3)},  &  \\
&& & \mathbf{(1 2 1)}  & (1 3 1)  & (2 3 2)  &  \\ \hline
&& & (2 2 1), & (3 3 1), & (3 3 2),  &  \\
&& & (1 1 2)  & \mathbf{(1 1 3)}  & \mathbf{(2 2 3)}  &  \\ \hline
&& &&& & \mathbf{(1 2 3)}, (2 3 1), (3 1 2)  \\
&& &&& & (1 3 2), (2 1 3), (3 2 1)  \\ \hline
\end{array} \]
The bold elements are idempotents.
\end{example}

\subsubsection{Greens theorem}

\begin{lemma}
Let $\sSet{A, f}$ be an associative class and $x\in A$. If $x$ is an idempotent, then
\[ \lambda_x|_{[x]_\greensR} = \id_{[x]_\greensR} \qquad \text{and} \qquad \rho_x|_{[x]_\greensL} = \id_{[x]_\greensL}. \]
Thus $x$ is a left identity for $[x]_\greensR$ and a right identity for $[x]_\greensL$.
\end{lemma}
\begin{proof}
Take $y\in [x]_\greensR$. Then there exist $a,b\in \widetilde{A}$ such that $\begin{tikzcd}
x \ar[r, maps to, shift left, "\rho_a"] & y \ar[l, maps to, shift left, "\rho_b"]
\end{tikzcd}$. Then we have
\[ xy = xxb = xb = y. \]
The other claim is dual.
\end{proof}

\begin{theorem}[Green's theorem]
Let $\sSet{A, f}$ be an associative class and $H$ an $\mathcal{H}$-class in $A$. Then either
\begin{enumerate}
\item $H^2\perp H$; or
\item $H^2 = H$, $f|_{H\times H}$ has an identity and each $x\in H$ is invertible.
\end{enumerate}
\end{theorem}
\begin{proof}
Suppose $H^2\cap H \neq \emptyset$, then there exist $a,b\in H$ such that $ab = c\in H$. By the Green's lemma \ref{GreensLemma} we have that $\rho_b:H\to H$ and $\lambda_a: H\to H$ are bijections.

Then for all $h\in H$, $\rho_b(h) = hb \in H$. Again by the Green's lemma, this means that $\lambda_h: H\to H$ is a bijection. Similarly $\rho_h: H\to H$ is a bijection for all $h$. So for all $h\in H$ we have $hH = H = Hh$. The result follows by \ref{invertibilityFromPrincipalIdeals}.
\end{proof}
\begin{corollary} \label{GreensTheoremCorollary}
Let $\sSet{A, f}$ be an associative class and $H$ an $\mathcal{H}$-class in $A$. Then
\begin{enumerate}
\item if $x$ is an idempotent in $H$, then we have the second case;
\item no $\mathcal{H}$-class can contain more than one idempotent. 
\end{enumerate}
\end{corollary}

\subsection{Regular elements and generalised inverses}
\begin{definition}
Let $\sSet{A, f}$ be an associative class and $x, y\in A$. We call
\begin{itemize}
\item $x$ \udef{regular} if $\exists a\in A: \; x = xax$
\item $x$ and $y$ \udef{generalised inverses} if $x = xyx$ and $y = yxy$.
\end{itemize}
\end{definition}

\begin{proposition} 
Let $\sSet{A, f}$ be an associative class. If $x\in A$ is regular, then every element in $[x]_{\greensD}$ is regular.
\end{proposition}
So it makes sense to call a $\greensD$-class \udef{regular} if it consists of regular elements and \udef{irregular} otherwise.
\begin{proof}
Let $x$ be regular with $x = xx'x$ and $x\greensD y$. Then we have $a,b,c,d \in \widetilde{A}$ such that
$\lambda_a\circ\rho_b|_{[x]_\greensH}: [x]_\greensH \to [y]_\greensH$ is a bijection with inverse $\lambda_c\circ \rho_d|_{[y]_\greensH}: [y]_\greensH \to [x]_\greensH$, as in \ref{greensDisomorphism}. Then we have
\[ y = (\lambda_a\circ\rho_b)(x) = (\lambda_a\circ\rho_b)(xx'x) = axx'xb = a(\lambda_c\circ\rho_d)(y)x'(\lambda_c\circ\rho_d)(y)b = ydx'cy. \]
So $y$ is regular.
\end{proof}
\begin{corollary}
If there is an idempotent $x\in [a]_{\greensD}$, then $[a]_{\greensD}$ is regular.
\end{corollary}
\begin{proof}
An idempotent is regular: $x = f(x,x) = f(f(x,x), x)$.
\end{proof}

\begin{proposition}
Let $\sSet{A, f}$ be an associative class. Then $x$ is regular \textup{if and only if} it has a generalised inverse.
\end{proposition}
\begin{proof}
Clearly every element with a generalised inverse is regular. Conversely, assume $x$ regular with $x = xax$. Then $y = axa$ is a generalised inverse of $x$: $x(axa)x = xax = x$ and $(axa)x(axa) = a(xax)axa = axaxa = axa$.
\end{proof}
Note that we do not have that $x = xyx$ implies $y = yxy$.

\begin{proposition} \label{greensRelationsRegularElements}
Let $\sSet{A, f}$ be an associative class and $x\in A$ a regular element with $x = f(f(x, y), x)$. Then
\begin{enumerate}
\item $f(x,y)$ and $f(y,x)$ are idempotent;
\item $f(y,x) \greensL x$ and $x \greensR f(x, y)$.
\end{enumerate}
\end{proposition}
\begin{proof}
(1) We calculate
\[ f(f(x,y), f(x,y)) = f(f(f(x,y), x), y) = f(x, y)\] and \[f(f(y,x), f(y,x)) = f(y, f(x, f(y,x))) = f(y,x). \]

(2) Using \ref{idealAbsorption}, we have
\[ f(A, x) = f(A, f(x,f(y,x))) \subseteq f(A, f(y,x)) \subseteq f(A,x). \]
Thus $f(A, x) = f(A, f(y,x))$. The second part is dual.
\end{proof}
\begin{corollary}
In a regular $\mathcal{D}$-class each $\mathcal{L}$-class and each $\mathcal{R}$-class contains at least one idempotent.
\end{corollary}
\begin{proof}
Let $[x]_\mathcal{L}$ be an $\mathcal{L}$-class in a regular $\mathcal{D}$-class. By regularity there exists a $y\in A$ such that $xyx = x$. From the proposition, we have that $[x]_\mathcal{L}$ contains the idempotent $yx$ and $[x]_\mathcal{R}$ the idempotent $xy$.
\end{proof}
\begin{corollary} \label{idempotentsHclass}
If $x,x'\in A$ are generalised inverses, then $[xx']_\mathcal{H} = [x]_\mathcal{R}\cap [x']_\mathcal{L}$ and $[x'x]_\mathcal{H} = [x']_\mathcal{R}\cap [x]_\mathcal{L}$.
\end{corollary}
\begin{proof}
From $x\greensR (xx')$ and $(xx')\greensL x'$, we get the first equality. The second is dual.
\end{proof}
We can depict the situation in the corollary as follows:
\[ \hspace{-8.4em} \exists a,b,c,d \in \widetilde{A}: \qquad \begin{tikzcd}[sep=large]
x \ar[r, maps to, shift left, "\rho_a"] \ar[d, maps to, shift left, "\lambda_c"] & xx' \ar[l, maps to, shift left, "\rho_b"] \ar[d, maps to, shift left, "\lambda_c"] \\
x'x \ar[u, maps to, shift left, "\lambda_d"] \ar[r, maps to, shift left, "\rho_a"] & x' \ar[u, maps to, shift left, "\lambda_d"] \ar[l, maps to, shift left, "\rho_b"]
\end{tikzcd} \]

So generalised inverses along one diagonal imply idempotents along the other. In fact, the other direction also holds:
\begin{proposition}
Let $\sSet{A, f}$ be an associative class and $e,f$ idempotents in $A$ such that $e\greensD f$. Then there exist $x\in [e]_\greensR\cap [f]_\greensL$ and $x'\in [e]_\greensL\cap [f]_\greensR$ such that
\begin{itemize}
\item $x,x'$ are generalised inverses;
\item $e = xx'$ and $f = x'x$.
\end{itemize}
\end{proposition}
\begin{proof}
Because $e\greensD f$, we can find $x,x'\in A$ such that
\[ \hspace{-8.4em} \exists a,b,c,d \in \widetilde{A}: \qquad \begin{tikzcd}[sep=large]
e \ar[r, maps to, shift left, "\rho_a"] \ar[d, maps to, shift left, "\lambda_c"] & x \ar[l, maps to, shift left, "\rho_b"] \ar[d, maps to, shift left, "\lambda_c"] \\
x' \ar[u, maps to, shift left, "\lambda_d"] \ar[r, maps to, shift left, "\rho_a"] & f \ar[u, maps to, shift left, "\lambda_d"] \ar[l, maps to, shift left, "\rho_b"]
\end{tikzcd} \]
Then
\begin{align*}
xx' &= (df)(fb) = dfb = e \\
x'x &= (ce)(ea) = cea = f \\
xx'x &= ex = eea = ea = x \\
x'xx' &= fx' = ffb = fb = x',
\end{align*}
which completes the proof.
\end{proof}
\begin{corollary}
Let $\sSet{A, f}$ be an associative class, $y\in A$ and $e,f$ idempotents in $A$. Then $e\greensD f$ \textup{if and only if} there exist generalised inverses $x,x'$ such that $e = xx'$ and $f = x'x$.
\end{corollary}
\begin{proof}
The direction $\Rightarrow$ follows from the proposition. The converse from \ref{greensRelationsRegularElements}.
\end{proof}

\begin{lemma}
Let $\sSet{A, f}$ be an associative class, $x\in A$. Then no $\greensH$-class contains more than one generalised inverse of $x$.
\end{lemma}
\begin{proof}
Assume $x$ has two generalised inverses, $x_1'$ and $x_2'$. From \ref{idempotentsHclass} and \ref{GreensTheoremCorollary} we get that $xx_1' = xx_2'$ and $x_1'x = x_2'x$. Thus
\[ x_1' = x_1'(xx_1') = x_1'xx_2' = (x_1'x)x_2' =  x_2'xx_2' = x_2'. \]
\end{proof}

\begin{proposition}
Let $\sSet{A, f}$ be an associative class and $x,y\in A$. Then $xy\in [x]_\greensR \cap [y]_\greensL$ \textup{if and only if} $[x]_\greensL \cap [y]_\greensR$ contains an idempotent.
\end{proposition}
\begin{proof}
First assume $[x]_\greensL \cap [y]_\greensR$ contains an idempotent $e$. We can depict the situation as
\[ \hspace{-8.4em} \exists a,b,c,d \in \widetilde{A}: \qquad \begin{tikzcd}[sep=large]
x \ar[r, maps to, shift left, "\rho_a"] \ar[d, maps to, shift left, "\lambda_c"] &  \ar[l, maps to, shift left, "\rho_b"] \ar[d, maps to, shift left, "\lambda_c"] \\
e \ar[u, maps to, shift left, "\lambda_d"] \ar[r, maps to, shift left, "\rho_a"] & y \ar[u, maps to, shift left, "\lambda_d"] \ar[l, maps to, shift left, "\rho_b"]
\end{tikzcd} \]
Then we can calculate
\[ xy = deea = dea = xa \in [x]_\greensR \cap [y]_\greensL. \]
Now assume $xy\in [x]_\greensR \cap [y]_\greensL$.
We can depict the situation as
\[ \hspace{-8.4em} \exists a,b,c,d \in \widetilde{A}: \qquad \begin{tikzcd}[sep=large]
x \ar[r, maps to, shift left, "\rho_a"] \ar[d, maps to, shift left, "\lambda_c"] & xy \ar[l, maps to, shift left, "\rho_b"] \ar[d, maps to, shift left, "\lambda_c"] \\
e \ar[u, maps to, shift left, "\lambda_d"] \ar[r, maps to, shift left, "\rho_a"] & y \ar[u, maps to, shift left, "\lambda_d"] \ar[l, maps to, shift left, "\rho_b"]
\end{tikzcd} \]
Now we need to show that $e$ is idempotent. Indeed, starting from the three other corners, we see that $e = cx$ and $e = yb$ and $e = c(xy)b = (cx)(yb) = ee$.
\end{proof}

\subsection{Commutation}
\begin{definition}
Let $\sSet{A,f}$ be an associative class and $x,y \in A$. We say $x$ and $y$ \udef{commute} if $f(x,y) = f(y,x)$. We write $x\comm y$.
\end{definition}


\subsubsection{Centraliser or commutant}
\begin{definition}
Let $\sSet{A,f}$ be an associative class and $B\subseteq A$ a subclass. The \udef{centraliser} or \udef{commutant} of $B$ is defined as $Z_A(B) \defeq B^{\comm}$.

In particular we define the \udef{centre} of $A$ as the centraliser of all of $A$: $Z_A \defeq Z_A(A)$.
\end{definition}
Thus
\[ Z_A(B) = \setbuilder{x\in A}{\forall b\in B:\;f(x,b) = f(b,x)}. \].

Note that taking the commuting forms a Galois connection. In particular $B \subseteq B^{\comm\comm}$.

\begin{lemma}
Let $\sSet{A,f}$ be an associative class and $B\subseteq A$. If $f$ is commutative, then $Z_A(B) = A$.
\end{lemma}

\begin{proposition}
Let $\sSet{A,f}$ be an associative class and $B\subseteq A$. Then $Z_A(B)$ is closed under $f$.
\end{proposition}
\begin{proof}
Take arbitrary $x,y\in Z_A(B)$. Take arbitrary $b\in B$. Then
\[ f(f(x,y),b) = f(x,f(y,b)) = f(x,f(b,y)) = f(f(x,b),y) = f(f(b,x),y) = f(b,f(x,y)), \]
which means that $f(x,y)\in Z_A(B)$.
\end{proof}

\subsection{Normaliser}
\begin{definition}
Let $\sSet{A,f}$ be an associative class and $B\subseteq A$ a subclass. An element $x\in A$ is said to \udef{normalise} $B$ if $f(x,B) = f(B,x)$.

The \udef{normaliser} of $B$ in $A$ is the set of all elements in $A$ that normalise $B$:
\[ N_A(B) \defeq \setbuilder{x\in A}{f(x,B) = f(B,x)}. \]
\end{definition}

\begin{proposition}
Let $\sSet{A,f}$ be an associative class and $B\subseteq A$. Then
\begin{enumerate}
\item $N_A(B)$ is closed under $f$;
\item $Z_A(B) \subseteq N_A(B)$;
\item $Z_A(\{a\}) = N_A(\{a\})$ for all $a\in A$.
\end{enumerate}
\end{proposition}
\begin{proof}
(1) Take arbitrary $x,y\in N_A(B)$. Then
\[ f(f(x,y),B) = f(x,f(y,B)) = f(x,f(B,y)) = f(f(x,B),y) = f(f(B,x),y) = f(B,f(x,y)), \]
which means that $f(x,y)\in N_A(B)$.

(2) If $f(x,b) = f(b,x)$ for all $b\in B$, then $f(x,B) = f(B,x)$.

(3) $f(x,\{a\}) = f(x,a)$ and $f(\{a\}, x) = f(a,x)$.
\end{proof}

\subsection{Composition of relations}
\begin{definition}
Let $\mathfrak{R}$ be the class of all relations, with absorbing element $\mathrm{NULL}$ adjoined.
\end{definition}

\begin{proposition}
Let $R,S \in \mathfrak{R}$ be relations. Then
\begin{enumerate}
\item $R \greensL S$ \textup{if and only if} $R$ and $S$ have the same image and cokernel;
\item $R \greensR S$ \textup{if and only if} $R$ and $S$ have the same preimage and kernel.
\end{enumerate}
\end{proposition}
\begin{proof}
TODO
\end{proof}