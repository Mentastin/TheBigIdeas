\chapter{Vector space convergence}

TODO: \url{https://math.stackexchange.com/questions/2001771/existence-of-at-least-one-continuous-coordinate-functional}

\url{https://math.stackexchange.com/questions/60057/does-there-exist-a-linearly-independent-and-dense-subset}

\section{Vector space convergence}
\begin{definition}
Let $\sSet{\F,V,+}$ be a vector space and $\xi$ a convergence on $V$. Then $\sSet{\F,V,+, \xi}$ is a \udef{convergence vector space} (or CVS) if
\begin{itemize}
\item vector addition $+: V\times V \to V$ is continuous;
\item scalar multiplication $\cdot: \F\times V \to V$ is continuous.
\end{itemize}
\end{definition}

\begin{lemma}
If $\sSet{\F,V,+, \xi}$ is a convergence vector space, then $\sSet{V,+, 0, \xi}$ is a convergence group.
\end{lemma}
\begin{proof}
We just need to show that $v\mapsto -v$ is continuous, but this scalar multiplication and thus continuous by assumption.
\end{proof}

\begin{lemma} \label{continuityLemmaVectorConvergence}
If $\sSet{\F,V,+, \xi}$ is a convergence vector space, then
\begin{enumerate}
\item the function $V \to V: v \mapsto \lambda\cdot v$ is a homeomorphism for all $\lambda\in \F\setminus\{0\}$;
\item the function $\F \to \Span\{v\}: \lambda \mapsto \lambda\cdot v$ is continuous and invertible for all $v\in V\setminus\{0\}$. It is a homeomorphism \textup{if and only if} $\Span\{v\}$ is Hausdorff.
\end{enumerate}
\end{lemma}
TODO picture!
\begin{proof}
The functions $\lambda \mapsto (\lambda, v)$ and $v \mapsto (\lambda, v)$ are continuous by \ref{continuousEmbeddingProduct}. Composition with the continuous scalar product gives the result by continuity of composition (\ref{continuityComposition}).

They are both clearly invertible (for the second, note that the kernel is $\{0\}$). The inverse of the first is of the same form and thus immediately continuous.

If $u: \F \to \Span\{v\}: \lambda \mapsto \lambda\cdot v$ is a homeomorphism, then $\Span\{v\}$ is Hausdorff, because $\F$ is.

Now assume $\Span\{v\}$ Hausdorff. We need to show that $u^{-1}$ is continuous. It is enough to show continuity at $0$. We use \ref{pretopologicalContinuityVicinities}, so take $\Gamma \in \vicinity_\F(0)$ and we need to show that $\Gamma \in (u^{-1})^{\imf\imf}\big[\vicinity_{\Span\{v\}}(0)\big]$. WLOG we may take $\Gamma = \ball(0, \epsilon)$.

Now consider $\sphere(0,\epsilon)$, which is compact. Then $u^{\imf}\big(\sphere(0,\epsilon)\big) = \sphere(0,\epsilon)\cdot v$ is compact by \ref{compactConstructions} and thus closed by \ref{compactClosedSets} (as $\Span\{v\}$ we assumed Hausdorff). Now $0 \in \big(\sphere(0,\epsilon)\cdot v\big)^c$, so there exists a vicinity $U$ of $0$ disjoint from $\sphere(0,\epsilon)\cdot v$.

For each $F$ that converges to $0$ in $\Span\{v\}$, $\neighbourhood_\F(0)\cdot F$ also converges to $0$, so there exists $\delta_F>0$ and $C_F\in F$ such that $\ball(0, \delta_F)\cdot C_F \subseteq U$. If $\delta_F > 1$, then $C_F \subseteq \ball(0, \delta_F)\cdot C_F \subseteq \ball(0,\epsilon)\cdot v$, so $\ball(0,\epsilon)\cdot v\in F$.

Now assume $\delta_F \leq 1$. Then $\cball(0, \epsilon\cdot\delta_F^{-1})\setminus \ball(0,\epsilon)$ is compact, which, as before, means $\big(\cball(0, \epsilon\cdot\delta_F^{-1})\setminus \ball(0,\epsilon)\big)\cdot v$ is closed and we take a vicinity $U_F$ of $0$ disjoint from it. Now $C_F\cap U_F\in F$. We also claim that $C_F\cap U_F\subseteq \ball(0,\epsilon)\cdot v$: take $x\in C_F\cap U_F$. Then $x = \lambda v$ and $\delta_F |\lambda| \leq \epsilon$ (as $\ball(0, \delta_F)\cdot C_F \subseteq U$), which is equivalent to $|\lambda| \leq \epsilon\cdot\delta_F^{-1}$. Now because $x\in U_F$, this means $|\lambda| < \epsilon$ and thus $x\in \ball(0,\epsilon)\cdot v$.

As $\ball(0,\epsilon)\cdot v \in F$ for all $F$ that converge to $0$, we have $\ball(0,\epsilon)\cdot v\in \vicinity_{\Span\{v\}}$, so $\ball(0,\epsilon) = \Gamma \in (u^{-1})^{\imf\imf}\big[\vicinity_{\Span\{v\}}(0)\big]$, which is what we had to prove.
\end{proof}

Alternate proof in Beattie / Butzmann.

\begin{proposition} \label{vectorSpaceConvergenceConstruction}
Let $V$ be a vector space over a field $\F$. And $\mathcal{F} \subseteq \powerfilters(V)$ a family of filters. There exists a vector space convergence $\xi$ on $V$ such that $\mathcal{F} = \lim^{-1}_\xi(0)$ \textup{if and only if}
\begin{enumerate}
\item if $F \in \mathcal{F}$ and $G\supseteq F$, then $G\in \mathcal{F}$;
\item if $F,G \in \mathcal{F}$, then $F + G\in \mathcal{F}$;
\item if $F\in \mathcal{F}$, then $\neighbourhood_\F(0)\cdot F \in \mathcal{F}$;
\item if $v\in V$, then $\neighbourhood_\F(0)\cdot v \in \mathcal{F}$;
\item if $F\in \mathcal{F}$ and $\lambda\in \F$, then $\lambda\cdot F \in \mathcal{F}$.
\end{enumerate}
\end{proposition}
Note the similarity with \ref{groupConvergenceConstruction} for convergence groups. A group convergence is completely determined by $\lim^{-1}_\xi(0)$ due to the translation homeomorphisms \ref{shiftHomeomorphism}.
\begin{proof}
Assume first that $\mathcal{F} = \lim^{-1}_\xi(0)$ for some vector space convergence $\xi$.
\begin{enumerate}
\item This is just the monotonicity of the convergence.
\item If $F,G\to 0$, then $F\otimes G \to (0,0)$ by \ref{convergenceFiniteProductFilter}. By continuity of addition we have $F+G\to 0$.
\item The convergence on the scalar field is pretopological, so $\neighbourhood_\F(0)\to 0$. By \ref{convergenceFiniteProductFilter} $\neighbourhood_\F(0)\otimes F \to (0,0)$ and by continuity of the scalar multiplication $\neighbourhood_\F(0)\cdot F \to 0$.
\item By \ref{continuityLemmaVectorConvergence}.
\item By \ref{continuityLemmaVectorConvergence}.
\end{enumerate}

Now assume the five points hold. Define the convergence $\xi$ by $F\to v$ iff $F-v \in \mathcal{F}$. We need to show that this is a convergence and that it makes both the vector addition and scalar multiplication continuous.

Monotonicity is guaranteed by (1). To show the convergence is centered, note that $\mathcal{F} \neq \emptyset$ by (4), so long as $V\neq \emptyset$. Then for any $F\in \mathcal{F}$, $\big\{\{0\}\big\} = 0\cdot F \in \mathcal{F}$ by (5).

To show that the vector addition is continuous, take $F\to (v_1, v_2)$. Then $p_1^{\imf\imf}(F) = F_1\to v_1$ and $p_2^{\imf\imf}(F) = F_2 \to v_2$, i.e.\ $F_1-v_1 \in \mathcal{F}$ and $F_2-v_2 \in \mathcal{F}$. By (1), $(F_1-v_1) + (F_2-v_2) = (F_1+F_2) - (v_1 + v_2) \in \mathcal{F}$, so $F_1+F_2 \to v_1 + v_2$. Thus by \ref{filterFactorisationInequality}, $F_1+F_2 = +^{\imf\imf}[F_1\otimes F_2] \subseteq +^{\imf\imf}[F] \to v_1+v_2$ and the addition is continuous.

Let $G \to (\lambda, v)$. Then $G_1 = p_1^{\imf\imf}(G) \to \lambda$ and $G_2 = p_2^{\imf\imf}(G) \to v$, so $G_1 \supseteq \neighbourhood_\F(\lambda)$. We have
\begin{align*}
\cdot^{\imf\imf}[G] - \lambda\cdot v &\supseteq \cdot^{\imf\imf}[G_1\otimes G_2] - \lambda\cdot v = G_1\cdot G_2 - \lambda\cdot v \\
&\supseteq \neighbourhood_\F(\lambda) \cdot G_2 - \lambda\cdot v \\
&= (\neighbourhood_\F(0) + \lambda)\cdot((G_2 - v) + v) - \lambda\cdot v \\
&\supseteq \lambda\cdot (G_2 - v) + \neighbourhood_\F(0)\cdot(G_2-v) + \lambda\cdot v + \neighbourhood_\F(0)\cdot v - \lambda\cdot v \\
&= \lambda\cdot (G_2 - v) + \neighbourhood_\F(0)\cdot(G_2-v) + \neighbourhood_\F(0)\cdot v \in \mathcal{F}.
\end{align*}
So $\cdot^{\imf\imf}[G] \to \lambda\cdot v$, making the scalar multiplication continuous. Note the last inclusion is not an equality because we go from one instance of $G_2$ and $\neighbourhood_\F(0)$ to two!
\end{proof}
\begin{corollary} \label{vicinityFilterAtOrigin}
Let $\sSet{V, \xi}$ be a convergence vector space. Then
\begin{enumerate}
\item for all $A\in \vicinity_\xi(0)$ and $\lambda\in \F\setminus\{0\}$: $\lambda A\in \vicinity_\xi(0)$;
\item each $A \in \vicinity_\xi(0)$ is absorbent;
\item if $\xi$ is topological, then $\vicinity_\xi(0)$ has a balanced base;
\item if $\xi$ is topological, then $\vicinity_\xi(0)$ has a closed, balanced base.
\end{enumerate}
\end{corollary}
\begin{proof}
Now assume $\xi$ is a topological vector space convergence and $N = \neighbourhood_\xi(0)$.
\begin{enumerate}
\item Take arbitrary $\lambda\in\F\setminus\{0\}$. Then $v\mapsto \lambda v$ is a homeomorphism by \ref{continuityLemmaVectorConvergence}, so $\lambda A \in \vicinity_\xi(\lambda 0) = \vicinity_\xi(0)$, by \ref{homeomorphismPreservation}.
\item For absorbence, take $A\in \vicinity_\xi(0)$ and $v\in V$. As $\neighbourhood_\F(0)\cdot v \to 0$, we must have $A\in \upset\neighbourhood_\F(0)\cdot v$, so there exists $\Gamma \in \neighbourhood_\F(0)$ such that $\Gamma\cdot v \subseteq A$. Now we can find a $r>0$ such that $\ball(0,r)\subseteq \Gamma$, so for all $|c|\geq r^{-1}$ we have $v\in cA$.
\item By point (3) of of the proposition, $\neighbourhood_\F(0)\cdot \vicinity_\xi(0)$ converges to $0$ and thus $\vicinity_\xi(0) \subseteq \neighbourhood_\F(0)\cdot\vicinity_\xi(0)$. Take $A\in \vicinity_\xi(0)$. Then there exists a $\Gamma\in\neighbourhood_\F(0)$ and $B\in \vicinity_\xi(0)$ such that $\Gamma\cdot B \subseteq \vicinity_\xi(0)$. We can find some ball $\ball(0,\epsilon) \subseteq \Gamma$, so $\ball(0,1)\cdot \epsilon B\subseteq \epsilon B \subseteq A$. Thus $\epsilon B$ is balanced and a neighbourhood by point(1). So every $A\in \vicinity_\xi(0)$ contains a balanced set in $\vicinity_\xi(0)$.
\item Take arbitrary $A\in \vicinity_\xi(0)$. By regularity, \ref{topologicalGroupsRegular}, $A$ contains a closed vicinity $B$. By (3), $B$ contains a balanced set $C$. Now consider $\closure_\xi(C)$, which is closed and a a subset of $B$ (as $\closure(C)\subseteq \closure(B) = B$). It is now enough to note that the closure of a balanced set is balanced: if $x$ is the limit of a filter $F$ in $B$, then $r\cdot x$ is the limit of $r\cdot F$ for all $|r|\leq 1$. Now $r\cdot F$ is a filter in $B$ by balance.
\end{enumerate}
\end{proof}

\begin{proposition} \label{vectorSumInherenceAdherence}
Let $\sSet{V,\xi}$ be a vector space convergence and $A,B\subseteq V$. Then
\begin{enumerate}
\item $\adh(A)+\adh(B) \subseteq \adh(A+B)$;
\item $\inh(A)+\inh(B) \subseteq A+\inh(B) \subseteq \inh(A+B)$;
\item $\interior(A)+\interior(B) \subseteq A+\interior(B) \subseteq \interior(A+B)$.
\end{enumerate}
\end{proposition}
TODO: same for closure?
\begin{proof}
(1) We use \ref{productAdherence} and \ref{adherenceInherenceContinuity} to compute
\[ \adh(A)+\adh(B) = +^\imf[\adh(A)\times\adh(B)] = +^\imf[\adh(A\times B)] \subseteq \adh(+^\imf[A\times B]) = \adh(A+B). \]

(2) The inclusion $\inh(A)+\inh(B) \subseteq A+\inh(B)$ is immediate. Now for all $v\in V$ we have $v+\inh(B) = \inh(v+B)$, so
\[ A+\inh(B) = \bigcup_{v\in A}v+\inh(B) = \bigcup_{v\in A}\inh(v+B) \subseteq \inh\left(\bigcup_{v\in A} v+B\right) = \inh(A+B), \]
where we have used the monotonicity of $\inh$ and \ref{orderPreservingFunctionLatticeOperations}.

(3) Similar to (2).
\end{proof}
Notice that the argument used for (2) and (3) does not work for the adherence because $\adh(A)+\adh(B) \nsubseteq A+\adh(B)$ in general.

\begin{lemma}
Let $V$ be a convergence vector space over a field $K$ and $F\subseteq K$ a subfield. Then the $F$-vector space $V_F$ with the same convergence structure is also a convergence vector space.
\end{lemma}
\begin{proof}
It is enough that the restriction of the scalar multiplication to $F$ remains continuous. Alternatively, we can use \ref{vectorSpaceConvergenceConstruction} and note that passing to the field $F$ simply represents a weakening of condition $(5)$.
\end{proof}


\subsection{Initial and final vector space convergences}
\subsubsection{Initial vector space convergence}
\begin{proposition} \label{initialVectorSpaceConvergence}
Let $V$ be a vector space, $\{V_i\}_{i\in I}$ a set of convergence vector spaces and $\{L_i: V \to V_i\}_{i\in I}$ a set of linear maps. Then the initial convergence on $V$ w.r.t. $\{L_i: V \to V_i\}_{i\in I}$ makes $V$ a convergence vector space.
\end{proposition}
\begin{proof}
Continuity of vector addition follows from \ref{initialConvergenceGroup}.

We verify continuity of scalar multiplication $m: \F\times V \to V: (\lambda, v) \mapsto \lambda v$. Using \ref{characteristicPropertyInitialFinalConvergence}, we need to verify that $L_i\circ m$ is continuous for all $i\in I$. Because the $L_i$ are linear, we have
\[ L_i(\lambda v) = \lambda L_i(v) \]
for all $\lambda \in \F, v \in V$. This means that $L_i\circ m = m_i \circ (\id_{V_i}, L_i)$, where $m_i$ is scalar multiplication in $V_i$. Now $(\id_{\F}, L_i)$ is continuous by \ref{continuityFunctionTuple}, so $L_i \circ m$ is continuous.
\end{proof}
\begin{corollary}
Let $\{V_i\}_{i\in I}$ be a set of convergence vector spaces. Then the direct product $\prod_{i\in I} V_i$ with the product convergence is a convergence vector space.
\end{corollary}


\subsubsection{Quotient spaces}
\begin{proposition}
Let $\sSet{V, \xi}$ be a convergence vector space, $W$ a vector space and $q: \sSet{V, \xi} \to W$ a surjective linear function. Then the quotient convergence on $W$ w.r.t. $q$ is a vector space convergence.
\end{proposition}
\begin{proof}
Continuity of vector addition follows from
\ref{quotientConvergenceGroup}.

We verify continuity of scalar multiplication $m: \F\times W \to W: (\lambda, w) \mapsto \lambda w$. Take $F\overset{\F}{\longrightarrow}\lambda$ and $G \overset{W}{\longrightarrow} w$. Then, by \ref{initialFinalConvergence}, there exist $w'\in q^{\preimf}\{w\}$ and $G' \overset{\xi}{\longrightarrow} w'$ such that $q^{\imf\imf}[G'] \subseteq G$. Then
\[ F\cdot G \supseteq F\cdot q^{\imf\imf}[G'] = q^{\imf\imf}[F\cdot G'] \to q(\lambda \cdot w') = \lambda q(w') = \lambda w, \]
which shows that $m$ is continuous.
\end{proof}
\begin{corollary}
Let $\sSet{V,\xi}$ be a convergence vector space and $U\subseteq V$ a subspace. Then $G/U$ is a convergence vector space.
\end{corollary}
\begin{proof}
The function $[\cdot]_U: V\to V/U$ is linear by (TODO ref universal algebra aspects of vector spaces) and \ref{quotientAlgebra}. It is clearly surjective.
\end{proof}
\begin{corollary}
Let $\sSet{V, \xi}$ and $\sSet{W, \zeta}$ be CVSs. Let $f: V\to W$ be a continuous linear function and $U\subseteq G$ a subspace such that $N\subseteq \ker f$. Then there exists a unique continuous linear function $f': A/N \to B$ such that
\[ \begin{tikzcd}
A \arrow[r, "{[\cdot]_N}"] \arrow[dr, swap, "f"] & A/N \arrow[d, dashed, "{f'}"] \\
& B
\end{tikzcd} \qquad\text{commutes.} \]
Further, $f'$ is injective \textup{if and only if} $N = \ker f$.
\end{corollary}
\begin{proof}
The linear function $f'$ is the one from \ref{factorTheorem}. It is continuous by \ref{characteristicPropertyInitialFinalConvergence}.
\end{proof}

\subsubsection{Direct sum}
\begin{definition}
Let $\{\sSet{V_i, \xi_i}\}_{i\in I}$ be a set of convergence vector spaces. Then the \udef{direct sum convergence} is the final \emph{vector space} convergence on $\bigoplus_{i\in I}V_i$ w.r.t. the set $\{e_j: V_j \to \bigoplus_{i\in I}V_i\}$ of natural injections.
\end{definition}
The direct sum convergence is the final vector space convergence. It is equal to the final convergence if and only if the direct sum is trivial (see \ref{algebraicConvergenceStrength}).

\begin{proposition}
Let $\{\sSet{V_i, \xi_i}\}_{i\in I}$ be a set of convergence vector spaces. Then the direct sum convergence is the final convergence on $\bigoplus_{i\in I}V_i$ w.r.t. the set of natural injections $\setbuilder{e_J: \prod_{j\in J} V_j \to \bigoplus_{i\in I}V_i}{\text{$J\subseteq I$ finite}}$.
\end{proposition}
\begin{proof}

\end{proof}
\begin{corollary}
If $I$ is finite, then $\bigoplus_{i\in I}V_i = \prod_{i\in I}V_i$.
\end{corollary}

\begin{proposition}
Let $\{\sSet{V_i, \xi_i}\}_{i\in I}$ be a set of convergence vector spaces. The inclusion map
\[ \bigoplus_{i\in I}V_i \hookrightarrow \prod_{i\in I}V_i \]
is a continuous map into a dense subspace.
\end{proposition}

\subsubsection{Inductive systems}
TODO


\subsection{Continuity}
\subsubsection{Finite dimensional spaces}
\begin{proposition}
Let $\sSet{V,\xi}$ and $\sSet{W,\zeta}$ be convergence vector spaces and $f: V\to W$ a linear map. If $V$ is finite-dimensional and Hausdroff, then $f$ is continuous.
\end{proposition}
\begin{proof}
We prove this by induction on the dimension $n$ of $V$. For the base step $n=1$, we take some non-zero vector $v\in V$ such that $V = \Span\{v\}$. The for all $x\in V$, we have $x= \lambda v$ and thus
\[ f(x) = f(\lambda v) = \lambda f(v). \]
So we have that $f$ is given by
\[ \begin{tikzcd}
V = \Span\{v\} \arrow[rr, "\lambda\cdot v\mapsto \lambda"] && \F \arrow[rr, "\lambda\mapsto \lambda \cdot f(v)"] && \Span\{f(v)\} \arrow[r, hook] & W
\end{tikzcd} \]
which are all continuous functions (in particular by \ref{continuityLemmaVectorConvergence}, using Hausdroffness of $V$).

For the induction step, assume all linear functions $f: V\to W$, where $V$ is Hausdorff and has dimension $\dim(V)< n$ are continuous.

Pick some basis $\{v_1, \ldots, v_n\}$ of $V$.

TODO!
\end{proof}
\begin{corollary}
Every bijective linear map between finite-dimensional Hausdorff convergence vector spaces is a homeomorphism.
\end{corollary}
\begin{corollary}
Every finite-dimensional vector space has a unique convergence that makes it a Hausdorff CVS.
\end{corollary}
\begin{proof}
Unicity is immediate. For existence, every $n$-dimensional vector space is linearly isomorphic to $\F^n$ by \ref{isomorphicDimension} and the Euclidean norm on $\F^n$ gives it a Hausdroff vector convergence. 
\end{proof}

\subsubsection{Hamel coordinate functions}
\begin{lemma} \label{finiteNonZeroHamelCoordinateFunctions}
Let $V$ be a vector space and $\seq{e_i}_{i\in I}$ a Hamel basis of $V$. Let $\seq{\varphi_i}_{i\in I}$ be the associated coordinate functions. Then for all $v\in V$, at most finitely many $\varphi_i(v)$ are non-zero.
\end{lemma}
\begin{proof}
By definition of a Hamel basis, we have $v = \sum_{i\in I}\lambda_i e_i = \sum_{i\in I}\varphi_i(v) e_i$ and the sum must be finite.
\end{proof}
TODO: compare existence dual basis.

\begin{proposition}
Let $\sSet{V,\norm{\cdot}}$ be a Banach space and $\seq{e_i}_{i\in I}$ a Hamel basis of $V$. Let $\seq{\varphi_i}_{i\in I}$ be the associated coordinate functions. At most finitely many $\varphi_i$ are continuous.
\end{proposition}
\begin{proof}
Suppose there existed a countable sequence $\seq{e_i}_{i\in I}$ of basis elements, each with a continuous coordinate function $\varphi_n$. Consider the vector $v = \sum_{n=0}^\infty 2^{-n}e_i/\norm{e_i}$. Then, by continuity, each $\varphi_n(v)$ is non-zero. This is impossible by \ref{finiteNonZeroHamelCoordinateFunctions}.
\end{proof}



\subsection{Topological vector spaces}
\begin{definition}
A \udef{topological vector space} (or TVS) is a convergence vector space that is topological.
\end{definition}
As with convergence groups, any pretopological vector space convergence is topological, see \ref{pretopologicalGroupConvergence}.

\subsubsection{Neighbourhoods and base}
\begin{proposition} \label{TVSconstruction}
Let $V$ be a vector space and $N\in\powerfilters(V)$. Then $N = \neighbourhood_\xi(0)$ for some topological convergence on $V$ \textup{if and only if}
\begin{enumerate}
\item for all $A\in N$ and $\lambda\in \F\setminus\{0\}$: $\lambda A\in N$;
\item each $A \in N$ is absorbent;
\item $N$ has a balanced base;
\item for all $A\in N$, there exists some $B\in N$ such that $B+B\subseteq A$.
\end{enumerate}
\end{proposition}
\begin{proof}
We adapt \ref{vectorSpaceConvergenceConstruction} to the present situation.

First assume $N$ has a balanced and absorbent base. We check the five conditions for $\mathcal{F} = \pfilter{N}$.

\begin{enumerate}
\item Immediate because $\mathcal{F} = \pfilter{N}$.
\item Take $F,G\in \pfilter{N}$. We need to show that $\upset (F + G) \supseteq N$, which means that for all $A \in N$ there exist $B\in F$ and $C\in G$ such that $B+C\subseteq A$. We can take $B = C$ equal to the $B$ of point (2).
\item Take $F\in \pfilter{N}$. We need to show that $\upset(\neighbourhood_\F(0)\cdot F) \supseteq N$, which means that for all $A\in N$ there exists a $\Gamma\in \neighbourhood_\F(0)$ and $B\in F$ such that $\Gamma \cdot B\subseteq A$. We can take $B = \balancedCore(A) \in N \subseteq F$ and $\Gamma = B(0,1)$.
\item Take $v\in V$. We need to show that all $A\in N$ contain $\Gamma\cdot v$ for some $\Gamma \in \neighbourhood_\F(0)$. Because $A$ is absorbent, there exists an $r>0$ such that $v\in cA$ for all $|c|\geq r$. Conversely $c^{-1}v \in A$ for all $|c^{-1}| \leq r^{-1}$. So $\ball(0,r^{-1})\cdot v \subseteq A$ and $B(0,r^{-1}) \in \neighbourhood_\F(0)$.
\item Take $F\in \pfilter{N}$ and $\lambda\in \F$. We need to show that for all $A\in N$ there exists a $B\in F$ such that $\lambda\cdot B\subseteq A$. We can take $B = \lambda^{-1}A \in N\subseteq F$.
\end{enumerate}

Now assume $\xi$ is a topological vector space convergence and $N = \neighbourhood_\xi(0)$. The first 3 points follow from \ref{vicinityFilterAtOrigin}. The fourth from \ref{vicinityFactorisation}.
\end{proof}
\begin{corollary} \label{TVSbase}
Let $V$ be a vector space and $\mathcal{B}\subseteq\powerset(V)$. If $\mathcal{B}$ is such that
\begin{enumerate}
\item all $A\in \mathcal{B}$ are balanced and absorbent;
\item for each $A\in\mathcal{B}$, there exists some $B\in \mathcal{B}$ such that $B+B\subseteq A$;
\end{enumerate}
then $\mathfrak{F}(\mathcal{B}) = \neighbourhood_\xi(0)$ for some topological convergence on $V$.
\end{corollary}
\begin{proof}
We verify the 4 points of the proposition:
\begin{enumerate}
\item From point (2) of the hypothesis, we can prove by induction that for all $A\in\mathcal{B}$ and $n\in \N$, there exists $B\in \mathcal{B}$ such that $2^nB \subseteq A$.

Now take arbitrary $A\in\mathfrak{F}(\mathcal{B})$ and $\lambda\in\F\setminus\{0\}$. Then there exists a finite set $\{B_i\}_{i=0}^k\subseteq \mathcal{B}$ such that $B_0\cap \ldots \cap B_k \subseteq A$. Pick $n\in\N$ such that $2^{-n}\leq |\lambda|$ (and thus $|\lambda^{-1}2^{-n}| \leq 1$). Then we can find a finite set $\{C_i\}_{i=0}^k\subseteq \mathcal{B}$ such that $C_i \subseteq 2^{-n}B_i$. As each $B_i$ is absorbent, we have
\[ C_i \subseteq 2^{-n}B_i = \lambda (\lambda^{-1}2^{-n}) \subseteq \lambda B_i. \]
Thus $(C_0\cap \ldots \cap C_k) \subseteq \lambda (C_0\cap \ldots \cap C_k) \subseteq \lambda A$.
\item By \ref{absorbingSetProperties}, finite intersections of absorbent sets are absorbent. Thus each element of $\mathfrak{F}(\mathcal{B})$ contains an absorbent set and, by \ref{absorbingSetProperties}, is also absorbent.
\item Each element of $\mathfrak{F}(\mathcal{B})$ contains a finite intersection of balanced sets, which is also balanced by \ref{balancedLemma}.
\item Take arbitrary $A\in\mathfrak{F}(\mathcal{B})$. Then there exists a finite set $\{B_i\}_{i=0}^k\subseteq \mathcal{B}$ such that $B_0\cap \ldots \cap B_k \subseteq A$. By assumption, we can find a finite set $\{C_i\}_{i=0}^k\subseteq \mathcal{B}$ such that $C_i + C_i \subseteq B_i$. Then, using \ref{orderPreservingFunctionLatticeOperations}, we have
\[ \Big(\cap_{i\leq k}C_i\Big) + \Big(\cap_{j\leq k}C_j\Big) \subseteq \bigcap_{i,j\leq k} C_i + C_j \subseteq \bigcap_{i\leq k}C_i + C_i \subseteq \bigcap_{i\leq k} B_i \subseteq A. \]
\end{enumerate}
\end{proof}


\subsubsection{Locally convex topological convergence}

\begin{lemma} \label{locallyConvexNeighbourhoodsLemma}
Let $\sSet{V,\xi}$ be a TVS. Then the following are equivalent:
\begin{enumerate}
\item $\xi$ is locally convex;
\item $\neighbourhood_\xi(0)$ is based in the convex sets;
\item $\neighbourhood_\xi(0)$ is based in the absolutely convex sets.
\end{enumerate}
\end{lemma}
\begin{proof}
$(1) \Leftrightarrow (2)$ One direction is immediate, for the other it is enough to note that if $U$ is convex, then so is the translated set $x+U$ for all $x\in V$.

$(2) \Leftrightarrow (3)$ One direction is immediate, the other follows because the balanced core of a convex set is convex by \ref{balancedCoreConvexSet}.
\end{proof}

\begin{proposition}
Let $V$ be a vector space and $N\in\powerfilters(V)$. Then $N = \neighbourhood_\xi(0)$ for some locally convex topological convergence on $V$ \textup{if and only if}
\begin{enumerate}
\item for all $A\in N$ and $\lambda\in \F$: $\lambda A\in N$;
\item each $A \in N$ is absorbent;
\item $N$ has an absolutely convex base.
\end{enumerate}
\end{proposition}
\begin{proof}
This almost completely follows from \ref{TVSconstruction} and \ref{locallyConvexNeighbourhoodsLemma}. We just need to show that for all $A\in N$, there exists some $B\in N$ such that $B+B\subseteq A$. We may take $B = \frac{1}{2}A'$, where $A'$ is a convex subset of $A$, because for all $v,w\in A'$ we have $\frac{1}{2}v + \frac{1}{2}w \in A'$ by convexity.
\end{proof}


\subsection{Properties of subsets}
\begin{proposition} \label{inherenceAdherenceBalanced}
Let $\sSet{V,\xi}$ be a vector space convergence and $A\subseteq V$. Then
\begin{enumerate}
\item if $A$ is balanced, then $\adh(A)$ is balanced;
\item if $A$ is balanced and $0\in\inh(A)$, then $\inh(A)$ is balanced;
\item if $A$ is open and contains the origin, then $\balanced(A)$ is open.
\end{enumerate}
\end{proposition}
\begin{proof}
(1) We use \ref{productAdherence} and \ref{adherenceInherenceContinuity} to compute
\begin{align*}
\cball(0,1)\cdot \adh_\xi(A) &= \cdot^\imf[\adh_\F(\cball(0,1))\times \adh_\xi(A)] = \cdot^\imf[\adh_{\F\otimes \xi}(\cball(0,1)\times A)] \\
&\subseteq \adh_{\xi}\big(\cdot^\imf[\cball(0,1)\times A]\big) = \adh_{\xi}(\cball(0,1)\cdot A) = \adh_\xi(A).
\end{align*}

(2) Take $0<r<1$. Then multiplying by $r$ is a homeomorphism and thus $r\cdot\inh(A) = \inh(r\cdot A) \subseteq \inh(A)$. Now take $r=0$. If $0\in\inh(A)$, then
\[ r\cdot\inh(A) = \{0\} \subseteq \inh(A). \]

(3) For all $r\neq 0$, $r\cdot A$ is open. Thus
\[ \balanced(A) = \bigcup_{|r|\leq 1}r\cdot A = \{0\}\cup \bigcup_{0< r\leq 1}r\cdot A = \bigcup_{0< r\leq 1}r\cdot A \]
is a union of open sets and thus open by \ref{completeClosureTopology}.
\end{proof}
\begin{proof}[Alternative proof of (1)]
Take $|\lambda|\leq 1$ and $v\in \adh_\xi(A)$, then we need to show that $\lambda v \in \adh_\xi(A)$. We have $A\in \vicinity_\xi(v)^{\mesh}$ and $A\subseteq \lambda^{-1}A$. So for all $B\in \vicinity_\xi(v)$:
\[ A\mesh B \quad\implies\quad \lambda^{-1}A\mesh B \quad\implies\quad A\mesh \lambda B. \]
Thus $A\in \vicinity_\xi(\lambda v)^{\mesh}$, which is what we needed to show by \ref{principalAdherenceInherence}.
\end{proof}

\begin{proposition} \label{inherenceAdherenceConvex}
Let $\sSet{V,\xi}$ be a vector space convergence and $A\subseteq V$. Then
\begin{enumerate}
\item if $A$ is convex, then $\adh(A)$, $\inh(A)$ and $\interior(A)$ is convex;
\item if $A$ is open, then $\convex(A)$ is open.
\end{enumerate}
\end{proposition}
TODO same for closure?
\begin{proof}
(1) For all $0<r<1$, we have
\[ r\adh(A) + (1-r)\adh(A) = \adh(rA) + \adh\big((1-r)A\big) \subseteq \adh\big(rA + (1-rA)\big) \subseteq \adh(A) \]
by \ref{vectorSumInherenceAdherence}. 

The argument for $\inh(A)$ and $\interior(A)$ is similar.

(2) We have
\[ \interior(\convex(A)) = \interior\left(\bigcup_{|r|\leq 1} rA + (1-r)A\right) \supseteq \bigcup_{|r|\leq 1} r\interior(A) + (1-r)\interior(A) = \bigcup_{|r|\leq 1} rA + (1-r)A = \convex(A). \]
TODO ref.
\end{proof}

\begin{proposition}
Let $\sSet{V,\xi}$ be a vector space convergence and $A\subseteq V$ a subspace. Then $\adh(A)$ is a subspace.
\end{proposition}
\begin{proof}
Clearly $\adh_\xi(A)$ is not empty. It is then enough to note that $\adh_\xi(A)+\adh_\xi(A)\subseteq \adh_\xi(A)$, by \ref{vectorSumInherenceAdherence}, and $\F\cdot \adh_\xi(A) \subseteq \adh_\xi(A)$, for which we use \ref{productAdherence} and \ref{adherenceInherenceContinuity} to compute
\begin{align*}
\F\cdot \adh_\xi(A) &= \cdot^\imf[\adh_\F(\F)\times \adh_\xi(A)] = \cdot^\imf[\adh_{\F\otimes \xi}(\F\times A)] \\
&\subseteq \adh_{\xi}\big(\cdot^\imf[\F\times A]\big) = \adh_{\xi}(\F\cdot A) = \adh_\xi(A).
\end{align*}
\end{proof}
\begin{corollary} \label{hyperplaneClosedDense}
A hyperplane in a convergence vector space is either closed or dense.
\end{corollary}
\begin{proof}
Let $H$ be a hyperplane in a vector space $V$. Then $H \subseteq \adh(H)$ and $\adh(H)$ is a subspace. Because $H$ is a coatom, we have either $\adh(H) = H$ or $\adh(H) = V$. In the first case $H$ is closed, in the second dense.
\end{proof}
\begin{corollary}
Let $\sSet{V,\xi}$ be a complete Hausdorff CVS. Then any finite dimensional subspace $U$ of $V$ is closed.
\end{corollary}
\begin{proof}
Consider $\adh(U)$. This is a vector space by the proposition. If $\adh(U)$ has the same dimension as $U$, then $\adh(U) = U$ by \ref{vectorSpaceEquality} (as $U \subseteq \adh(U)$). In this case $U$ is closed.

Now assume $\adh(U)$ has a strictly larger dimension than $U$. Then pick a basis $\beta$ of $U$ and enlarge it to a basis $\beta'$ of $\adh(U)$, which is possible by \ref{extensionReductionBasisFiniteDimensions}. Take $w\in \beta'\setminus \beta$.

As $w\in\adh(U)$, there exists $F\to w$ such that $U\in F$
\end{proof}



\subsubsection{Bounded subsets}
\begin{definition}
Let $\sSet{V,\xi}$ be a convergence vector space on a field $\F$. A subset $A\subseteq V$ is called \udef{(von Neumann) bounded} if $\neighbourhood_\F(0) \cdot^{\imf\imf} \upset\{A\} \overset{\xi}{\longrightarrow} 0$.
\end{definition}
Clearly subsets of bounded sets are bounded.

\begin{lemma}
Let $\sSet{V,\xi}$ be a convergence vector space on a field $\F$ and $A\subseteq V$ a subset. 
\begin{enumerate}
\item If $A$ is bounded, then $A$ is absorbed by all vicinities of the origin (i.e.\ by all elements of $\vicinity_{\xi}(0)$).
\item If $\xi$ is topological, the converse also holds.
\end{enumerate}
\end{lemma}
\begin{proof}
We just need to show that ``$A$ is absorbed by all vicinities of the origin'' is equivalent to $\vicinity_\xi(0) \subseteq \neighbourhood_\F(0) \cdot A$. Then (1) follows from the fact that $\vicinity_\xi(0)$ is smaller than all convergent filters and (2) follows from the fact that all filters larger than $\vicinity_\xi(0)$ converge.

The equivalence is immediate by noting that $A$ is absorbed by $V\in \vicinity_\xi(0)$ iff there exists $\epsilon >0$ such that $\ball(0,\epsilon)\cdot A \subseteq V$.
\end{proof}

\begin{lemma} \label{finiteVectorSubsetBounded}
Let $\sSet{V,\xi}$ be a convergence vector space of finite depth. Any finite subset $S\subseteq V$ is bounded.
\end{lemma}
\begin{proof}
Let $S\subseteq V$ be a finite subset. By continuity of the scalar multiplication, $\neighbourhood_\F(0)\cdot^{\imf\imf}\pfilter{x} \to 0$ for all $x\in V$ and thus in particular for all $x\in S$.

Now $\neighbourhood_\F(0) \cdot^{\imf\imf} \upset\{S\} = \bigcup_{x\in S}\neighbourhood_\F(0) \cdot^{\imf\imf} \pfilter{x}$, which converges to $0$ by finite depth.
\end{proof}


\begin{lemma} \label{boundedBaseCriterion}
Let $\sSet{V,\xi}$ be a convergence vector space and $A\subseteq V$ a subset. If $A$ is absorbed by all sets in a base of $\vicinity_{\xi}(0)$, then it is bounded.
\end{lemma}

\begin{proposition}
Let $\sSet{V,\xi}$ be a topological convergence vector space and $B\subseteq V$. Then $B$ is von Neumann bounded \textup{if and only if} $B$ is bounded as a subset of a uniform space.
\end{proposition}
\begin{proof}
First assume $B$ is von Neumann bounded. Then we show boundedness using \ref{topologicalBoundednessLemma}. Take $A\in \entourage_V$, then by \ref{entourageConvergenceGroup}, we can find $U\in \neighbourhood_\xi(0)$ such that $\Delta^{\preimf}(U)\subseteq A$. By von Neumann boundedness, we can find $\epsilon >0$ such that $\ball(0,\epsilon)\cdot B \subseteq U$. Take $m\in\N$ such that $m^{-1}\leq \epsilon$. Then $B\subseteq mU$.

Now assume $B$ is bounded as a subset of a uniform space. Take arbitrary $U\in \neighbourhood_\xi(0)$, then by \ref{entourageConvergenceGroup}, we can find $A\in \entourage_V$ such that $A\subseteq \Delta^{\preimf}(U)$, i.e.\ $\Delta^{\imf}(A) \subseteq U$. There exists a finite subset $S\subseteq V$ such that $B \subseteq \bigcup_{x\in S}A^mx$. By \ref{vectorDeltaLemma}, we have
\[ B \subseteq \bigcup_{x\in S}A^mx = \bigcup_{x\in S}x+\Delta^\imf(A^m) = S+\Delta^\imf(A^m) \subseteq S+m\Delta^\imf(A), \]
where we have used convexity (TODO!)
Now $S$ is bounded by \ref{finiteVectorSubsetBounded}, so $S \subseteq cU$ and we have $B\subseteq (c+m)\Delta^\imf(A)$.
\end{proof}



\begin{lemma} \label{topologicalBoundedness}
Let $\sSet{V,\xi}$ be a topological vector space and $A\subseteq V$ a subset. Then $A$ is bounded \textup{if and only if}
\[ \forall U\in \neighbourhood_{\xi}(0): \exists r > 0: \forall c \geq r: \; A \subseteq cU. \]
\end{lemma}
This lemma says that in the topological case we do not need to check all $|c| \geq r$, only the positive ones.
\begin{proof}
The direction $\Rightarrow$ is immediate. For the converse, pick a balanced base of $\neighbourhood_{\xi}(0)$. Then the result follows immediately from the observation that $cV = |c|V$ (\ref{balancedLemma}) for all balanced sets.
\end{proof}

\begin{lemma} \label{boundedSetLemma}
Let $\sSet{V,\xi}$ be a convergence vector space and $A,B\subseteq V$ bounded subsets. Then
\begin{enumerate}
\item $\lambda A$ is bounded for all $\lambda\in\F$;
\item $A+B$ is bounded;
\item $A\cup B$ is bounded.
\end{enumerate}
\end{lemma}

\begin{lemma} \label{boundedSetsTVS}
Let $\sSet{V,\xi}$ be a topological vector space and $A\subseteq V$ a bounded subset. Then
\begin{enumerate}
\item $\closure_\xi(A)$ is bounded;
\item $\balanced(A)$ is bounded;
\item $\disked(A)$ is bounded if $\xi$ is locally convex.
\end{enumerate}
\end{lemma}
\begin{proof}
(1,2) By \ref{vicinityFilterAtOrigin}, we may take a closed and balanced base of $\vicinity_{\xi}(0)$. 
By \ref{boundedBaseCriterion} it is enough to show absorption by these basis elements. This is immediate, because for all closed and bounded $U$, we have $A\subseteq cU$ iff $\closure_\xi(A) \subseteq cU$ iff $\disked(A) \subseteq cU$.

(3) Similar, now taking convex balanced base, \ref{locallyConvexNeighbourhoodsLemma}.
\end{proof}

\begin{proposition} \label{continuousMappingBoundedSets}
Let $\sSet{V,\xi}, \sSet{W,\zeta}$ be vector convergence spaces, $f:V\to W$ a function and $A\subseteq V$ a bounded subset. Then
\begin{enumerate}
\item if $f$ is linear and continuous, then $f^\imf(A)$ is bounded;
\item if $\zeta$ is topological and $f$ is positively homogeneous and continuous at $0$, then $f^\imf(A)$ is bounded.
\end{enumerate} 
\end{proposition}
\begin{proof}
(1) We have $\vicinity_\zeta(0) \subseteq f^{\imf\imf}\big(\vicinity_\xi(0)\big)$ by \ref{continuityVicinityFilter}.
Take arbitrary $U\in \vicinity_\zeta(0)$. Then there exists $V\in \vicinity_\xi(0)$ such that $f^\imf(V) \subseteq U$. By boundedness of $A$, there exists $r>0$ such that $A \subseteq cV$ for all $|c|\geq r$. Then $f^\imf(A)\subseteq f^\imf(cV) = cf^\imf(V) \subseteq cU$ for all $|c|\geq r$. Thus $f^\imf(A)$ is bounded.

(2) Again we have b$\vicinity_\zeta(0) \subseteq f^{\imf\imf}\big(\vicinity_\xi(0)\big)$ by \ref{continuityVicinityFilter} (using $f(0) = 0$ from \ref{homogeneousFunctionLemma}). Take arbitrary $U\in \vicinity_\zeta(0)$. Then there exists $V\in \vicinity_\xi(0)$ such that $f^\imf(V) \subseteq U$. By boundedness of $A$, there exists $r>0$ such that $A \subseteq cV$ for all $c\geq r$. Then $f^\imf(A)\subseteq f^\imf(cV) = cf^\imf(V) \subseteq cU$ for all $c\geq r$. Thus $f^\imf(A)$ is bounded by \ref{topologicalBoundedness}.
\end{proof}

\begin{proposition} \label{boundedSetsInitialTopology}
Let $V$ be a vector space, $\{f_i:V\to \sSet{Y_i,\zeta_i}\}_{i\in I}$ a set of linear functions to topological vector spaces and $A\subseteq V$ a subset. Then $A$ is bounded \textup{if and only if} $f_{i}^{\imf}(A)$ is bounded for all $i\in I$.
\end{proposition}
\begin{proof}
The direction $\Rightarrow$ is given by \ref{continuousMappingBoundedSets}.

For the converse, note that $\neighbourhood_V(0)$ has a base consisting of sets of the form $f^{\preimf}_i(U)$, with $U\in \neighbourhood_{\zeta_i}(0)$, by \ref{pretopologicalInitialConvergence}.
As $f_{i}^{\imf}(A)$ is bounded, there exists $r>0$ such that $f_{i}^{\imf}(A) \subseteq cU$ for all $|c|\geq r$. Then $A \subseteq f_i^{\preimf}\big(f_i^\imf(A)\big) \subseteq f_i^\preimf(cU) = cf_i^\preimf(U)$ for all $|c|\geq r$.
\end{proof}

\begin{proposition} \label{boundedOnVicinityImpliesContinuous}
Let $\sSet{V,\xi}$ vector convergence space and $\sSet{W,\zeta}$ a topological vector convergence space and $f: V\to W$ a positively homogeneous function.
\begin{enumerate}
\item if there exists $U\in \vicinity_\xi(0)$ such that $f^\imf(U)$ is bounded, then $f$ is continuous at the origin;
\item if $f$ is linear, then this implies the continuity of $f$ everywhere.
\end{enumerate}
\end{proposition}
\begin{proof}
(1) By \ref{pretopologicalContinuityVicinities}, it is enough to verify $\neighbourhood_\zeta(0) \subseteq f^{\imf\imf}\big(\vicinity_\xi(0)\big)$. Take $A\in \neighbourhood_\zeta(0)$. Then $f^\imf(U)\subseteq rA$ for some $r>0$ by boundedness. So $f^\imf(r^{-1}U) \subseteq A$. As $r^{-1}U\in \vicinity_\xi(0)$ by \ref{vicinityFilterAtOrigin}, we have $A \in f^{\imf\imf}\big(\vicinity_\xi(0)\big)$.

(2) The continuity of $f$ is equivalent to the continuity of $f$ at $0$, by \ref{shiftHomeomorphism}.
\end{proof}
\begin{corollary} \label{continuityToNormedSpace}
Let $\sSet{V, \xi}$ be a convergence vector space, $\sSet{W, \norm{\cdot}}$ a normed space and $f: V\to W$ a positively homogeneous function. Then
\begin{enumerate}
\item $f$ is continuous at $0$ \textup{if and only if} $f$ is bounded on some $U\in \vicinity_\xi(0)$;
\item if $f$ is linear, then this is equivalent to the continuity of $f$ everywhere.
\end{enumerate}
\end{corollary}
\begin{proof}
(1) The direction $\Leftarrow$ is given by the proposition.

For the converse, assume $f$ continuous at $0$, then $\ball\big(0,1\big)\in \neighbourhood_W(0) = \neighbourhood_W\big(f(0)\big) \subseteq f^{\imf\imf}[\vicinity_\xi(0)]$ by \ref{continuityVicinityFilter}. So there exists $U\in \vicinity_\xi(0)$ such that $f^{\imf}[U] \subseteq \ball(0,1)$, which means that $f$ is bounded by $1$ on $U$.

(2) The continuity of $f$ is equivalent to the continuity of $f$ at $0$, by \ref{shiftHomeomorphism}.
\end{proof}


\begin{lemma} \label{boundedSetVicinityBase}
Let $\sSet{V,\xi}$ be a convergence vector space. If $U\in \vicinity_\xi(0)$ is bounded, then $\{\epsilon U\}_{\epsilon>0}$ forms a base of $\vicinity_\xi(0)$.
\end{lemma}
\begin{proof}
Note that $\{\epsilon U\}_{\epsilon>0} \subseteq \vicinity_\xi(0)$ by \ref{vicinityFilterAtOrigin}.

Conversely, take $U'\in \vicinity_\xi(0)$, then we can find $r>0$ such that $U\subseteq rU'$ and thus $r^{-1}U\subseteq U'$.
\end{proof}

\begin{proposition}
Let $\sSet{V,\xi}$ be a topological vector space and $U\in\neighbourhood_\xi(0)$.
\begin{enumerate}
\item if $U$ is bounded, then $\xi$ is the initial topology w.r.t. some absolutely homogeneous function $f: V\to \R$;
\item if $U$ is bounded and convex, then $\xi$ is the initial topology w.r.t. some seminorm $f$.
\item if $U$ is bounded and convex and $\xi$ is Hausdorff, then $\xi$ is normable.
\end{enumerate}
\end{proposition}
TODO: (1) implies pseudometrisable? Narici/Beckenstein.
\begin{proof}
(1) Suppose $U$ bounded. Then $\balanced(U)$ is bounded by \ref{boundedSetsTVS}, so we may take $U$ balanced WLOG. Then $\{\epsilon U\}_{\epsilon >0}$ forms a base of $\neighbourhood_\xi(0)$ by \ref{boundedSetVicinityBase}. Consider the gauge $p_U$, which is absolutely continuous by \ref{gaugeProperties}. By \ref{gaugeClassificationLemma}, each $\epsilon U$ is contained in a preimage of $p_U$, so the initial topology w.r.t. $p_U$ is stronger than $\xi$. We just need to show that $p_U$ is continuous, which immediately follows from \ref{boundedOnVicinityImpliesContinuous} and \ref{gaugeClassificationLemma}.If $p_U(v) = 0$ for some 

(2) In this case $p_U$ is a seminorm by \ref{gaugeProperties}.

(3) TODO.
\end{proof}


\section{Algebraic convergence}
\begin{definition}
Let $V$ be a vector space over a field $\F$. The \udef{algebraic convergence} on $V$ is the final convergence on $V$ w.r.t $\setbuilder{\F \to V: \lambda\mapsto \lambda \cdot v}{v\in V}$.

We denote this convergence $\mathfrak{a}$ and thus write $F \overset{\mathfrak{a}}{\longrightarrow} x$ and $x\in \lim_\mathfrak{a} F$.
\end{definition}

\begin{lemma} \label{algebraicConvergence}
Let $V$ be a vector space over a field $\F$ and $F\in \powerfilters(V)$. Then the algebraic convergence is defined by
\[ F \overset{\mathfrak{a}}{\longrightarrow} x \iff \begin{cases}
\exists v\in V: \;F \supseteq \neighbourhood_\F(0)\cdot v & (x = 0) \\
F \supseteq \neighbourhood_\F(1)\cdot x & (x \neq 0).
\end{cases}  \]
\end{lemma}
\begin{proof}
This is an application of \ref{initialFinalConvergence}, which states that $F \overset{\mathfrak{a}}{\longrightarrow} x$ iff there exists some $v\in V$ and $\lambda\in \F$ such that $x = \lambda v$ and $\neighbourhood_\F(\lambda)\cdot v \subseteq F$.

If $x\neq 0$, then $\lambda \neq 0$, so
\[ \neighbourhood_\F(\lambda)\cdot v = \neighbourhood_\F(\lambda)\cdot \lambda^{-1}x = \neighbourhood_\F(\lambda\lambda^{-1})\cdot x = \neighbourhood_\F(1)\cdot x. \]
If $x = 0$, then either $\lambda = 0$ or $v = 0$. In the latter case, we have
\[ \neighbourhood_\F(\lambda)\cdot v = \neighbourhood_\F(\lambda)\cdot 0 = \pfilter{0} = \neighbourhood_\F(0)\cdot 0 = \neighbourhood_\F(0)\cdot v. \]
\end{proof}

\begin{lemma} \label{algebraicConvergenceStrength}
Let $\sSet{V,\xi}$ be a convergence vector space. Then
\begin{enumerate}
\item $\mathfrak{a} \subseteq \xi$;
\item $\mathfrak{a}$ is a vector convergence \textup{if and only if} $V$ is 1D.
\end{enumerate}
\end{lemma}
\begin{proof}
(1) Immediate as $\xi$ must make all functions in $\setbuilder{\F \to V: \lambda\mapsto \lambda \cdot v}{v\in V}$ continuous.

(2) First assume $V$ is 1D, so $V = \Span\{v\}$. Then any filter in $\powerfilters(V)$ is of the form $F\cdot v$ for some $F\in \powerfilters(\F)$. Now take two convergent filters, $F\cdot v\to \lambda v$ and $G\cdot \mu v$. Then by \ref{algebraicConvergence}, we have that $F \overset{\F}{\longrightarrow} \lambda$ and $G \overset{\F}{\longrightarrow} \mu$, so
\[ F\cdot v + G\cdot v = (F+G)\cdot v \to (\lambda +\mu)v = \lambda v + \mu v. \]
by continuity of $(\lambda\mapsto \lambda \cdot v)$. This means that the vector addition is continuous and $\mathfrak{a}$ is a vector space convergence.

Now assume $V$ is not 1D. So we can find linearly independent $v,w$. Consider the sequence $\seq{v + \frac{1}{n}w}_{n\in\N}$, which converges to $v$ in any convergence vector space. But $\ball_\F(1,1)\cdot v$ is not an element of the tail filter (as no element of the sequence is of the form $\lambda v$), so $\neighbourhood_\F(1)\cdot v \nsubseteq \TailsFilter\seq{v + \frac{1}{n}w}$ and the sequence does not converge in the algebraic convergence by \ref{algebraicConvergence}.
\end{proof}

\begin{lemma} \label{constructionsInAlgebraicConvergence}
Let $V$ be a vector space over a field $\F$, $v\in V$ and $A\subseteq V$ a subset. Then
\begin{enumerate}
\item $\begin{aligned}[t]
\vicinity_\mathfrak{a}(0) &= \bigcap_{v\in V} \upset \neighbourhood_\F(0)\cdot v \\
&= \setbuilder{B\in \powerset(V)}{\forall v\in V: \exists \Gamma_v\in \neighbourhood_\F(0):\; \Gamma_v\cdot v\subseteq B} \\
&= \setbuilder{\bigcup_{v\in V} \Gamma_v\cdot v}{\forall v\in V:\; \Gamma_v \in \neighbourhood_\F(0)};
\end{aligned}$
\item $\inh_\mathfrak{a}(A) = \setbuilder{x\in V}{\forall v\in V:\exists \Gamma_v \in \neighbourhood_\F(0):\; x + \Gamma_v\cdot v \subseteq A}$;
\item $\adh_\mathfrak{a}(A) = \setbuilder{x\in V}{\exists v\in V: \forall \Gamma\in\neighbourhood_\F(0):\; (x+\Gamma\cdot v)\mesh A}$.
\end{enumerate}
\end{lemma}
\begin{proof}
(1) The first equality follows straight from \ref{algebraicConvergence}.

(2) We have $\inh_\mathfrak{a}(A) = \setbuilder{x}{A\in \vicinity_\mathfrak{a}(x)} = \setbuilder{x}{A-x\in \vicinity_\mathfrak{a}(0)}$. From (1) we get 
\begin{align*}
\inh_\mathfrak{a}(A) &= \setbuilder{x}{\forall v\in V: \exists \Gamma_v\in \neighbourhood_\F(0): \Gamma_v\cdot v \subseteq A-x} \\
&= \setbuilder{x}{\forall v\in V: \exists \Gamma_v\in \neighbourhood_\F(0): x + \Gamma_v\cdot v \subseteq A}.
\end{align*}

(3) We calculate
\begin{align*}
\adh_\mathfrak{a}(A) &= \big(\inh_\mathfrak{a}(A^c)\big)^c \\
&= \setbuilder{x\in V}{\exists v\in V:\forall \Gamma \in \neighbourhood_\F(0):\; \neg(x + \Gamma\cdot v \subseteq A^c)} \\
&= \setbuilder{x\in V}{\exists v\in V:\forall \Gamma \in \neighbourhood_\F(0):\; (x + \Gamma\cdot v) \mesh A}.
\end{align*}
\end{proof}

\begin{lemma}
Let $V$ be a vector space. Then every subspace $U\subseteq V$ is algebraically closed.
\end{lemma}
\begin{proof}
We need to show that $\adh_\mathfrak{a}(U)\subseteq U$. Take $x\in \adh_\mathfrak{a}(U)$. Then take a $v\in V$ such that $\forall \Gamma\in\neighbourhood_\F(0):\; (x+\Gamma\cdot v)\mesh U$.

Pick some $\Gamma\in\neighbourhood_\F(0)$. Then $x+\lambda v\in U$ for some $\lambda\in \Gamma$. Then take $\ball(0,|\lambda|/2)\in \neighbourhood_\F(0)$, so $x+\mu v\in U$ for some $\mu\in \ball(0,|\lambda|/2)$. In particular $\lambda \neq \mu$. If either $\lambda =0$ or $\mu = 0$, then $x\in U$ and we are done. Suppose $\lambda\neq 0 \neq \mu$. Then
\[ \lambda^{-1}(x+\lambda v) - \mu^{-1}(x+\mu v) = (\lambda^{-1} - \mu^{-1})x \in U. \]
So $x\in U$.
\end{proof}

\subsection{The algebraic interior or core}
\begin{definition}
Let $V$ be a vector space and $A\subseteq V$ a subset. Then algebraic inherence $\inh_\mathfrak{a}(A)$ is also called the \udef{algebraic interior} or \udef{core} of $A$.
\end{definition}

\begin{proposition} \label{coreProperties}
Let $V$ be a vector space and $A \subseteq V$ a subset. Then
\begin{enumerate}
\item $A$ is absorbing \textup{if and only if} $0\in \inh_\mathfrak{a}(A)$;
\item if $A$ is convex, then $\inh_\mathfrak{a}(A)$ is convex and open.
\end{enumerate}
\end{proposition}
\begin{proof}
(1) We have that
\begin{align*}
\text{$A$ is absorbing} &\iff \forall v\in V: \exists \epsilon >0: \; \ball(0,\epsilon)\cdot v\subseteq A \\
&\iff \forall v\in V: \exists \Gamma \in \neighbourhood_\F(0): \; \Gamma\cdot v\subseteq A \\
&\iff 0\in \inh_\mathfrak{a}(A).
\end{align*}

(2) We first show convexity: take $x,y \in \inh_\mathfrak{a}(A)$. Then there exist relevant $v,w,\Gamma_v,\Gamma_w$ such that $x+ \Gamma_v\cdot v \subseteq A$ and $y+ \Gamma_w\cdot w \subseteq A$. By \ref{convexCriteria} we have $\lambda \big(x+ \Gamma_v\cdot v\big) + (1-\lambda)\big(y+ \Gamma_w\cdot w\big)\subseteq A$ for all $0\leq \lambda \leq 1$, so
\[ \lambda x+(1-\lambda)y + (\Gamma_v\cap\Gamma_w)\big(\lambda v+(1-\lambda)w\big) \subseteq \lambda x+(1-\lambda)y + \Gamma_v\cdot \lambda v+\Gamma_w\cdot (1-\lambda)w \subseteq A. \]

To show $\inh_\mathfrak{a}(A)$ is open, we use \ref{openClosedCriteria}. Take $x\in \inh_\mathfrak{a}(A)$. Then for all $v\in V$ we can find a $\Gamma_v\in\neighbourhood_\F(0)$ such that $x+\Gamma_v\cdot v \subseteq A$. This means $x+\bigcup_{v\in V}\Gamma_v\cdot v \subseteq A$. Because the convergence on $\F$ is topological, we may take the $\Gamma_v$ open. To conclude with \ref{openClosedCriteria} it is enough to show that $x+\bigcup_{v\in V}\Gamma_v\cdot v \subseteq \inh_\mathfrak{a}(A)$.

Pick some $y = x+ c_w w \in x+ \bigcup_{v\in V}\Gamma_v\cdot v \subseteq A$, meaning $c_w\in\Gamma_w$. We can find an $0<\epsilon_w<|c_w|$ such that $c_w + \ball(0,\epsilon_w) \subseteq \Gamma_w$ by \ref{openClosedCriteria}. Then for all $1-\frac{\epsilon_w}{|c_w|}<\delta<1$ we have $|c_w - \delta c_w| = (1-\delta)|c_w| < \frac{\epsilon_w}{|c_w|}|c_w| = \epsilon_w$ and so $x+ \delta c_w w \in A$.

Now pick an arbitrary $u\in V$. We have $x+\Gamma_u\cdot u \subseteq A$. By convexity we have
\[ \delta^{-1}(x+ \delta c_w w) + (1-\delta^{-1})\big(x+\Gamma_u\cdot u\big) = x + c_w w + (1-\delta^{-1})\Gamma_u\cdot u \subseteq A. \]
This means that for all $v\in V$ we have $y + (1-\delta^{-1})\Gamma_v\cdot v \subseteq A$ and thus $y\in \inh_\mathfrak{a}(A)$.
\end{proof}

\begin{proposition} \label{algebraicallyOpen}
Let $V$ be a vector space and $A \subseteq V$ an algebraically open subset. Then
\begin{enumerate}
\item $A+U$ is algebraically open for any subspace $U\subseteq V$;
\end{enumerate}
\end{proposition}
\begin{proof}
TODO
\end{proof}




\section{Functionals}
\begin{definition}
Let $V$ be a vector space over a field $\mathbb{F}$.
\begin{enumerate}
\item A \udef{functional} on $V$ is a map $V\to \F$;
\item A \udef{linear functional} on $V$ is a linear map from $V$ to $\mathbb{F}$;
\item A \udef{real functional} on $V$ is a map $V\to \R$.
\end{enumerate}
\end{definition}

\begin{lemma} \label{continuityDominatedFunctional}
Let $V$ be a TVS and $f:V\to \F$ a continuous functional. If $g:V\to \F$ is a functional such that $|g(v)|\leq |f(v)|$ for all $v\in V$, then $g$ is continuous.
\end{lemma}
\begin{proof}
We use \ref{pretopologicalContinuityVicinities} to show continuity. To that end take $K\in \neighbourhood_\F(0)$. Then there exists $\epsilon >0$ such that $\ball(0,\epsilon)\subseteq K$ and so
\[ g^{\preimf}(K) \supseteq g^\preimf[\ball(0,\epsilon)] \supseteq f^\preimf[\ball(0,\epsilon)] \in \neighbourhood_V(0). \]
\end{proof}

\subsection{Linear functionals}
\begin{lemma} \label{kernelHyperplane}
Let $V$ be a vector space and $U\subseteq V$ a subspace. Then $U$ is a hyperplane \textup{if and only if} it is the kernel of a linear functional.
\end{lemma}

\begin{lemma} \label{functionalBoundedNeighbourhood}
Let $f: V\to \F$ be a linear functional and $x\notin \ker(f)$. Let $A\subseteq V$ be a balanced set. Then $(x+A)\perp \ker(f)$ \textup{if and only if} $A \subseteq f^{\preimf}(\ball(0,|f(x)|))$.
\end{lemma}
\begin{proof}
Suppose $A \subseteq f^{\preimf}(\ball(0,|f(x)|))$. Then for all $a\in A$: $f(x+a) = f(x) + f(a) \neq 0$.

Conversely, suppose $A \not\subseteq f^{\preimf}(\ball(0,|f(x)|))$, i.e.\ there exists $a\in A$ such that $|f(a)| \geq |f(x)|$. Then $v= -\frac{f(x)}{f(a)}a\in A$, because $A$ is balanced and so $f(x+ v) = f(x)-\frac{f(x)}{f(a)}f(a) = 0$ and so $(x+A) \mesh \ker(f)$.
\end{proof}

\begin{proposition} \label{linearFunctionalOpen}
Let $V$ be a convergence vector space and $f:V\to \F$ a non-zero linear functional. Then $f$ is an open map.
\end{proposition}
\begin{proof}
It is enough to show $f$ is open when $V$ is equipped with the algebraic convergence.

Let $A$ be an algebraically open map. We use \ref{openClosedCriteria} to show $f^\imf[A]$ is also open. Because $f$ is non-zero, there exists a $v\in V$ such that $f(v) \neq 0$. Take some $y\in f^\imf[A]$. Then there exists an $x\in A$ such that $f(x) = y$. Because $A$ is open, $x\in \inh_\mathfrak{x}(A)$ and there exists $x+ \Gamma_v\in \neighbourhood_\F(0)$ such that $x+\Gamma_v\cdot v \subseteq A$ by \ref{constructionsInAlgebraicConvergence}.

Now $f^\imf[x+ \Gamma_v\in \neighbourhood_\F(0)] = y+\Gamma_v \cdot f(v) \subseteq f^\imf[A]$ and $y+\Gamma_v \cdot f(v)$ is a vicinity of $y$, so we are done.
\end{proof}

\begin{lemma} \label{complexRangeExtensionRealFunctional}
Let $V$ be a complex vector space and $g: V_\R\to \R$ a linear functional. Then there exists a unique linear functional $f: V\to \C$ such that $g = \Re(f)$.
\end{lemma}
\begin{proof}
We can write $f = g + ih$ for some function $h: V\to \R$. Then for all $x\in V$
\[ g(ix)+ih(ix) = f(ix) = if(x) = ig(x) - h(x). \]
Comparing real parts gives $h(x) = - g(ix)$. So $f$ must be given by $f(x) = g(x) - ig(ix)$. Clearly $f$ is real-linear. We just need to verify that this makes $f$ complex-linear. Indeed, take $\lambda = a +ib \in \C = \R+i\R$ arbitrarily. Then for all $v\in V$
\begin{align*}
f(\lambda v) &= f\big((a+ib)v\big) \\
&= af(v) + bf(iv) \\
&= af(v) + b\big(g(iv) - ig(i^2v)\big) \\
&= af(v) + b\big(g(iv) + ig(v)\big) \\
&= af(v) + ib\big(-ig(iv) + g(v)\big) \\
&= af(v) + ibf(v) = (a+ib)f(v) = \lambda f(v).
\end{align*}
\end{proof}

\begin{lemma} \label{linearDependenceLinearFunctionals}
Let $V$ be a vector space and $f_0,\ldots, f_n, f$ linear functionals in $(V\to \F)$. Then the following are equivalent:
\begin{enumerate}
\item $f$ is a linear combination of $f_0,\ldots, f_n$;
\item there exists a $C>0$ such that $f(v) \leq C \max_{0\leq i\leq n}|f_i(v)|$;
\item $\ker(f) \supseteq \bigcap_{0\leq i \leq n}\ker(f_i)$;
\end{enumerate}
\end{lemma}
\begin{proof}
The implications $(1) \Rightarrow (2) \Rightarrow (3)$ are clear.

Now assume $(3)$ holds. Consider the function
\[ \begin{pmatrix}
f_0 \\ \vdots \\ f_n
\end{pmatrix}: V\to \F^{n+1}: v\mapsto \begin{pmatrix}
f_0(v) \\ \vdots \\ f_n(v)
\end{pmatrix}. \]
Due to the assumption, we can find a linear function $F: \F^{n+1}\to \F$ such that $f = F\circ \begin{pmatrix}
f_0 \\ \vdots \\ f_n
\end{pmatrix}$.

This function $F$ can be represented as a matrix by \ref{ellIsomorphism}. Thus $f$ is a linear combination of $f_0,\ldots, f_n$.
\end{proof}
\begin{corollary}
Let $V$ be a vector space and $f_0,\ldots, f_n$ linearly independent linear functionals in $(V\to \F)$. Then there exist $v_0, \ldots, v_n$ such that $f_i(v_j) = \delta_{i,j}$.
\end{corollary}
\begin{proof}
The proof is by induction. The case $n=1$ is clear: if there was no such $a_1$, then $f_1$ would be zero and thus not linearly independent.

Suppose the statement holds for $n-1$ and take $f_0,\ldots, f_n$ linearly independent linear functionals with corresponding $v_0,\ldots, v_{n-1}$. By point (3) of the proposition we can find $v_n \in \bigcap_{0\leq i \leq n}\ker(f_i)\setminus \ker(f_n)$, which after rescaling can be taken to be such that $f_n(v_n) = 1$. By construction $f_i(v_n) = 0$ for $i<n$.

Now replace $v_i$ with $v_i-f_n(v_i)v_n$ and rescale.
\end{proof}

\subsection{The dual space}
\begin{definition}
Let $\sSet{V,\xi}$ be a convergence vector space over a field $\mathbb{F}$.

The \udef{dual} of $V$ is the vector space of all continuous linear functionals on $V$.

The dual is denoted $\sSet{V,\xi}^{*}$ (or just $V^*$ is the convergence is clear from the context).
\end{definition}

\begin{proposition} \label{continuityLinearFunctionals}
Let $\sSet{V, \xi}$ be a CVS and $f:V\to \F$ a linear functional on $V$. Then the following are equivalent:
\begin{enumerate}
\item $f\in V^{*}$, i.e.\ $f$ is continuous;
\item there exists a vicinity $U\in \vicinity_\xi(0)$ such that $f$ is bounded on $U$;
\item $\ker(f)$ is closed;
\item $\ker(f)$ is not dense.
\end{enumerate}
\end{proposition}
\begin{proof}
$(1) \Leftrightarrow (2)$ By \ref{continuityToNormedSpace}.

$(1) \Rightarrow (3)$ Because $\ker(f) = f^{\preimf}(\{0\})$ and $\{0\}$ is closed in $\F$, $\ker(f)$ is closed by \ref{preimageOpenClosed}.

$(3) \Rightarrow (1)$ TODO

$(3) \Leftrightarrow (4)$ By \ref{kernelHyperplane} $\ker(f)$ is a hyperplane and by \ref{hyperplaneClosedDense} this hyperplane is either closed or dense.
\end{proof}

Now assume $\ker(f)$ closed. If $\ker(f) = V$, then $f$ is constant and thus continuous by \ref{continuityConstructions}. If $\ker(f) \neq V$, we can find some some $x\in \ker(f)^c$, which is open. Thus $\ker(f)^c - x$ is a neighbourhood of the origin, meaning we can take a balanced subset $A$ by \ref{vicinityFilterAtOrigin}. Now $(x+A)\perp \ker(f)$ by construction, so $f$ is bounded on $A$ by \ref{functionalBoundedNeighbourhood}.


\begin{lemma} Let $X$ be a normed space and
let $x\in X$ $\omega\in \tdual{X}$ be a bounded linear functional. Then
\begin{align*}
\norm{\omega} &= \sup\setbuilder{|\omega(v)|}{\norm{v}=1 } \\
&= \sup\setbuilder{\frac{|\omega(v)|}{\norm{v}}}{v\neq 0} \\
&= \inf\setbuilder{c>0} {|\omega(v)|\leq c\norm{v}\forall v\in X}
\end{align*}
and
\begin{align*}
\norm{x} &= \sup\setbuilder{|\varphi(x)|}{ \norm{\varphi}=1} \qquad\qquad\quad\\ %TODO: fragile spacing!
&= \sup\setbuilder{\frac{|\varphi(x)|}{\norm{\varphi}}}{\varphi\neq 0}.
\end{align*}
\end{lemma}
TODO move??
\begin{proof}
We prove the third equality. Let $\alpha$ be the infimum. Let $\epsilon>0$, then by the definition $|\omega[(\norm{x}+\epsilon)^{-1}x]|\leq \norm{\omega}$. Hence $|\omega(x)|\leq \norm{\omega}(\norm{x}+\epsilon)$. Letting $\epsilon\to 0$ gives $|\omega(x)|\leq \norm{\omega}\norm{x}$ for all $x$. So $\alpha\leq \norm{\omega}$. On the other hand, $|\omega(x)|\leq c$ for all $x$ with $\norm{x}=1$. Hence $\norm{\omega}\leq \alpha$.
\end{proof}

\subsubsection{The algebraic dual}
\begin{definition}
Let $V$ be a vector space. The \udef{algebraic dual} of $V$ is the dual of $\sSet{V,\mathfrak{a}}$, where $\mathfrak{a}$ is the algebraic convergence.

If no convergence on $V$ has been mentioned, then $V^*$ means the algebraic dual.
\end{definition}

\begin{proposition} \label{algebraicDual}
Let $V$ be a vector space. Then the algebraic dual of $V$ is the set of all linear functionals: $V^* = \Lin(V,\F)$.

Thus $V^* \supseteq \sSet{V,\xi}^*$ for all vector space convergences $\xi$ on $V$.
\end{proposition}
\begin{proof}
We need to show that all linear functionals are continuous when $V$ is equipped with the algebraic convergence. Assume $F\overset{\mathfrak{a}}{\longrightarrow} x$. Then there exists a $v\in V$ such that $\neighbourhood_\F(0)\cdot v+x \subseteq F$ and so $\neighbourhood_\F(0)\cdot f(v)+f(x) \subseteq f^\imf[F]$, meaning $f^\imf[F] \overset{\F}{\longrightarrow} f(x)$. Thus $f$ is continuous.
\end{proof}


\begin{proposition} \label{dualBasisDimension}
Let $V$ be a vector space. Then $\dim V^* \geq \dim V$ and
\[ \dim V^* = \dim V \iff \text{$V$ is finite-dimensional}. \]
If $V$ is finite-dimensional with a basis $v_1, \ldots, v_n$, then the \udef{dual basis} $\varphi_1, \ldots, \varphi_n$ is the set of linear functionals on $V$ such that
\[ \varphi_j(v_k) = \begin{cases}
1 & (k=j), \\ 0 & (k\neq j)
\end{cases}. \]
This dual basis is indeed a basis of $V^*$.
\end{proposition}
\begin{proof}
We first assume $V$ is finite-dimensional and prove the dual basis is a basis, which proves $\dim V^* = \dim V$. We then assume $V$ is infinite-dimensional and prove $\dim V^* \neq \dim V$.\footnote{Reference: \url{https://mathoverflow.net/questions/13322/slick-proof-a-vector-space-has-the-same-dimension-as-its-dual-if-and-only-if-i}}
\begin{enumerate}
\item Assume $V$ is finite-dimensional. To show the dual basis spans $V^*$, take a linear functional $\varphi$. Now define $a_i = \varphi(v_i)$. It is clear that $\varphi = \sum_{i=1}^n a_i\varphi_i$. To show linear independence, take a combination
\[ b_1\varphi_1 + \ldots +b_n\varphi_n =0. \]
Filling in all basis vectors $v_i$ in turn, gives $b_i=0$ for all $i$.
\item Assume $V$ is infinite-dimensional. At first let us assume $\dim_{\mathbb{F}}V \geq |\mathbb{F}|$. Then we can apply lemma \ref{vsCardinality} to obtain $\dim_{\mathbb{F}}V = |V|$. Let $\beta$ be a basis for $V$. The elements of $V^*$ correspond bijectively to functions from $\beta$ to $\mathbb{F}$. Thus
\[ |V^*| = |\mathbb{F}^\beta| = |\mathbb{F}|^{|\beta|} > |\beta| = |V|. \]
Now we relax the condition $\dim_{\mathbb{F}}V \geq |\mathbb{F}|$. We first note that every field contains a subfield that is at most denumerable. Take such a field $K\subset \mathbb{F}$. We introduce the new vector space $W = \Span_K(\beta)$. Every functional from $W$ to $K$ extends to a functional from $V$ to $\mathbb{F}$. Hence
\[ \dim_\mathbb{F} V = \dim_K W < \dim_K W^* \leq \dim_{\mathbb{F}} V^* \]
using $\dim_{K}W \geq |K| \geq \aleph_0$.
\end{enumerate}
\end{proof}
\begin{corollary}
Let $V$ be a finite-dimensional vector space. Then the algebraic convergence is the unique Hausdorff vector space convergence on $V$.
\end{corollary}
\begin{proof}
Consider a basis $v_1, \ldots, v_n$ of $V$ with dual basis $\varphi_1, \ldots, \varphi_n$. Let $\xi$ be some vector space convergence. By definition we have $\mathfrak{a} \subseteq \xi$. Now take $F \overset{\xi}{\longrightarrow} v$. We have $F = v_1\cdot \varphi_1^{\imf\imf}[F] + \ldots + v_n\cdot \varphi_n^{\imf\imf}[F]$. Now each $\varphi_1^{\imf\imf}[F]$ converges in both $\mathfrak{a}$ and $\xi$ by \ref{algebraicDual} and the proposition, so by continuity of addition and scalar multiplication, $F$ also converges in $\mathfrak{a}$. 
\end{proof}
\begin{corollary}
Let $\sSet{V,\xi}$ be a convergence vector space. If $V$ is finite-dimensional, then $\sSet{V,\xi}^* = \sSet{V,\mathfrak{a}}^*$.
\end{corollary}
\begin{proof}
We have $\sSet{V,\xi}^* \subseteq \sSet{V,\mathfrak{a}}^*$ by \ref{algebraicDual}. Because $V$ is finite-dimensional, we obtain equality equality of space from equality of dimension by \ref{vectorSpaceEquality}.
\end{proof}

\subsubsection{The bidual space}
TODO!
\begin{definition}
Let $V$ be a convergence vector space. The \udef{bidual space} is the dual of the dual $\abidual{V} = \adual{(\adual{V})}$.
\end{definition}
TODO continuous convergence!!

\begin{definition}
Let $V$ be a vector space over $\mathbb{F}$ and $v\in V$. The \udef{evaluation map} $\evalMap: V\to \abidual{V}: v\mapsto \evalMap_v$ is given by
\[ \evalMap_v: \adual{V} \to \mathbb{F}: l\mapsto l(v). \]
\end{definition}

\begin{lemma}
Let $V$ be a vector space. The evaluation map $\evalMap: V\to \abidual{V}: v\mapsto \evalMap_v$ is linear:
\[ \forall v,w\in V, a\in\mathbb{F}: \quad \evalMap_{av + w} = a\evalMap_v + \evalMap_w. \]
\end{lemma}
\begin{lemma}
Let $V$ be vector space over $\mathbb{F}$. The evaluation map is injective.
\end{lemma}
\begin{proof}
Assume $\evalMap_v = \evalMap_w$ for some $v,w\in V$. Then
\[ 0 = \evalMap_v - \evalMap_w  = \evalMap_{v-w}. \]
So $\forall l\in \adual{V}: \evalMap_{v-w}(l) = l(v-w) = 0$. Now define the sublinear functional by
\[ p(x) = \begin{cases}
\alpha & x = \alpha(v-w) \\
0 & \text{else}.
\end{cases} \]
Then the functional $f$ defined on $\Span\{v-w\}$ by $f(\alpha(v-w)) = \alpha$ is bounded by $p$ and can be extended to a functional on all $V$ by the Hahn-Banach theorem \ref{sublinearHahnBanach} if $v-w\neq 0$. Then $f(v-w) \neq 0$, which contradicts our assumptions. Thus $v=w$.
\end{proof}

\begin{proposition}
The mapping $\evalMap: V\to \abidual{V}: v\mapsto \evalMap_v$ is an isomorphism \textup{if and only if} $V$ is finite-dimensional.
\end{proposition}
\begin{proof}
Assume $V$ finite dimensional. As the evaluation map is injective, it is an isomorphism by \ref{invertibleFiniteDim}.
The other direction is a dimensional argument by proposition \ref{dualBasisDimension}.
\end{proof}


Just like for algebraic duality, we can define a topological bidual space (or second dual space) $\tbidual{V}$.

\begin{proposition}
Let $V$ be a normed space. 
For each $v\in V$
\[ \evalMap_v: \tdual{V} \to \mathbb{F}: \omega \mapsto \omega(v) \]
is bounded and thus an element of $\tbidual{V}$.

The evaluation map $\evalMap: V \to \tbidual{V}$ is
\begin{enumerate}
\item isometric (and thus injective): $\norm{\evalMap_v} = \norm{v}$;
\item bounded with norm $\norm{\evalMap} = 1$.
\end{enumerate}
\end{proposition}
\begin{proof}
Let $v\in V$. Then
\[ \norm{\evalMap_v} = \sup\setbuilder{\norm{\evalMap_v(\omega)}}{\norm{\omega}=1} = \sup\setbuilder{\norm{\omega(v)}}{\norm{\omega}=1} \leq \sup\setbuilder{\norm{v}\;\norm{\omega}}{\norm{\omega}=1} = \norm{v}. \]

(1) Setting $\omega = \inner{v/\norm{v}, \cdot}$, we get
\[ \norm{\evalMap_v} \leq |\evalMap_v(\omega)| = |\inner{v/\norm{v}, v}| = \norm{v}. \]
Together with the calculation above, this gives $\norm{\evalMap_v} = \norm{v}$.

(2) $\norm{\evalMap} = \sup\setbuilder{\norm{\evalMap_v}}{\norm{v}=1} = \sup\setbuilder{\norm{v}}{\norm{v}=1} = 1$.
\end{proof}

\begin{lemma}
Let $V$ be normed space over $\mathbb{F}$ and $v\in V$. For each $v\in V$
\[ \evalMap_v: \tdual{V} \to \mathbb{F}: \omega \mapsto \omega(v) \]
is bounded with norm $\norm{v}$ and thus $\evalMap\in \tbidual{V}$ with $\norm{\evalMap} = 1$.
\end{lemma}


\subsubsection{Reflexive spaces}
\begin{definition}
A normed space $V$ is \udef{reflexive} if the evaluation map $\evalMap:V\to \tbidual{V}$ is surjective:
\[ \im\evalMap = \tbidual{V}. \]
\end{definition}
If $V$ is reflexive, then $\tbidual{V}$ is isometrically isomorphic to $V$. The converse is not necessarily true.

\begin{lemma}
Every finite-dimensional space is reflexive.
\end{lemma}

\begin{proposition}
A separable normed space $X$ with a non-separable dual space $\tdual{X}$ cannot be reflexive. 
\end{proposition}
\begin{proof}
TODO
\end{proof}
Thus $l^1$ is not reflexive.

\begin{proposition}
If the dual space $\tdual{X}$ of a  normed space $X$ is separable, then $X$ itself is separable. 
\end{proposition}
\begin{proof}
TODO
\end{proof}

\subsubsection{Transposition}
\begin{definition}
Let $f:V\to W \in \Hom_{\mathbb{F}}(V,W)$. The \udef{dual map}\footnote{The dual map $f^t$ is often denoted $f^*$ or $f'$. We avoid this because it clashes with the notation of the Hilbert adjoint.} or \udef{transpose} $f^t$ is the linear map
\[ f^t:W^* \to V^*: l\mapsto f^t(l) = l\circ f. \]
\end{definition}
\begin{lemma}
Let $f\in \Hom(U,V)$ and $g\in \Hom(V,W)$.
\begin{itemize}
\item $(g\circ f)^t = f^t\circ g^t$;
\item $\id^t_V = \id_{\adual{V}}$;
\item $f$ is an isomorphism \textup{if and only if} $f^t$ is an isomorphism;
\item $(f^t)^{-1} = (f^{-1})^t$ 
\end{itemize}
\end{lemma}
TODO: merge
\begin{lemma}
Let $S,T\in\Hom(V,W)$ and $\alpha\in\mathbb{F}$. Then
\begin{enumerate}
\item $(S+T)^t = S^t+T^t$;
\item $(\alpha T)^t = \alpha T^t$
\item if $T$ is invertible, then $T^t$ is invertible and
\[ (T^t)^{-1} = (T^{-1})^t. \]
\end{enumerate}
\end{lemma}

\begin{proposition}
Let $U\subset V$ be a subspace and $T\in \Hom(V,W)$, where $V,W$ are \emph{finite-dimensional}.
\begin{enumerate}
\item $\dim\ker T^t = \dim \ker T + \dim W - \dim V$;
\item $\dim\im T^t = \dim \im T$;
\item $\im T^t = (\ker T)^0$
\item $T$ is injective \textup{if and only if} $T^t$ is surjective.
\end{enumerate}
\end{proposition}
\begin{proof}
\mbox{}
\begin{enumerate}
\item Using $\dim \adual{V} = \dim V$, we have
\begin{align*}
\dim \ker T^t &= \dim(\im T)^0 = \dim W-\dim \im T \\
&= \dim W - (\dim V - \dim \ker T) = \dim \ker T + \dim W - \dim V
\end{align*}
where the equalities come from proposition \ref{annihilatorSpace} and the dimension theorem for linear maps, theorem \ref{dimensionLinearMaps}.
\item Still using these results, we can calculate
\begin{align*}
\dim \im T^t &= \dim \adual{W} - \dim \ker T^t = \dim \adual{W} - \dim (\im T)^0 \\
&= \dim \adual{(\im T)} = \dim \im T.
\end{align*}
\item Take $\varphi = T^t(\psi) \in \im T^t$ where $\psi \in \adual{W}$. If $v\in \ker T$, then
\[ \varphi(v) = \left(T^t(\psi)\right)v = (\psi\circ T)(v) = \psi(Tv) = \psi(0) = 0. \]
Hence $\varphi \in (\ker T)^0$ and $\im T^t\subset (\ker T)^0$. We prove the equality by showing the dimensions are the same. Indeed:
\[ \dim \im T^t = \dim \im T = \dim V - \dim \ker T = \dim(\ker T)^0. \]
\item $T\in\Hom(V,W)$ is injective iff $\ker T = \{0\}$ iff $(\ker T)^0 = \adual{V}$ iff $\im T^t = \adual{V}$ iff $T^t$ is surjective.
\end{enumerate}
\end{proof}

\begin{proposition}
Let $\sSet{V,\xi}$ and $\sSet{W, \zeta}$ be CVSs and $T: V\to W$ a linear map. Then $T^t$ has a restriction to $T^t: W^* \to V^*$ \textup{if and only if} $T$ is continuous.
\end{proposition}
\begin{proof}
TODO! Only for locally convex??
\end{proof}

\begin{proposition} \label{transpDual}
Let $f\in\Hom(V,W)$ and $\mathcal{V}, \mathcal{W}$ bases of $V,W$. The
\[ (f^*)^{\mathcal{V}^*}_{\mathcal{W}^*} = ((f)^{\mathcal{W}}_{\mathcal{V}})^\transp. \] 
\end{proposition}

\begin{definition}
Let $T\in\Bounded(V,W)$. The dual map $T^t: \tdual{W}\to \tdual{V}$ is called the \udef{adjoint} or the \udef{transpose} of $T$.
\end{definition}
The notation $T^t$ is consistent for maps on both the algebraic and topological duals: if $T$ is bounded, $T^t:\adual{W}\to \adual{V}$ restricts to $T^t|_{\tdual{W}} = T^t:\tdual{W}\to \tdual{V}$.

\begin{proposition}
Let $T\in\Bounded(V,W)$. Then the transpose $T^t$ is a bounded operator in $\Bounded(W,V)$ with $\norm{T^t} = \norm{T}$.
\end{proposition}
\begin{proof}
The operator $T^t$ is linear since $\forall f_1,f_2\in \tdual{W}, \forall a\in\mathbb{F}, \forall x\in V:$
\[ (T^t(af_1 + f_2))(x) = (af_1 + f_2)(Tx) = af_1(Tx) + f_2(Tx) = a(T^tf_1)(x) + (T^tf_2)(x). \]
For the equality of norms, we prove two inequalities. First $\forall x\in V, f\in \tdual{W}$
\[ |f(Tx)|\leq \norm{f}\norm{Tx}\leq \norm{f}\norm{x}\norm{T} \implies \frac{|f(Tx)|}{\norm{x}} \leq \norm{f}\norm{T}. \]
taking the supremum over $x\in V$, we get $\norm{T^tf} = \norm{f\circ T}\leq \norm{f}\norm{T}$ and taking the supremum over $f\in \tdual{W}$ gives $\norm{T^t}\leq \norm{T}$. This shows that $T^t$ is bounded.

For the other inequality, we use corollary \ref{existenceBoundedFunctionalOfSameNorm} to the Hahn-Banach theorem: for every $x\in V$, there exists a bounded functional $\omega_x$ such that $\norm{\omega_x}=1$ and $\omega_x(x) = \norm{x}$. Then we can calculate:
\begin{align*}
\norm{Tx} = \omega_{Tx}(Tx) = (T^t\omega_{Tx})(x) \leq \norm{T^t\omega_{Tx}}\norm{x} \leq \norm{T^t}\norm{\omega_{Tx}}\norm{x} = \norm{T^t}\norm{x}
\end{align*}
So $\norm{T}\leq\norm{T^t}$. Combining gives $\norm{T^t}=\norm{T}$.
\end{proof}
\begin{corollary}
The map $T\mapsto T^t$ is an isometric isomorphism in $(\Bounded(X,Y)\to \Bounded(\tdual{Y}, \tdual{X}))$.
\end{corollary}

\begin{lemma}
Let $S,T\in\Bounded(V,W)$ and $\alpha\in\mathbb{F}$. Then
\begin{enumerate}
\item $(S+T)^t = S^t+T^t$;
\item $(\alpha T)^t = \alpha T^t$
\item if $T$ is invertible, then $T^t$ is invertible and
\[ (T^t)^{-1} = (T^{-1})^t. \]
\end{enumerate}
Let $T\in\Bounded(U,V)$ and $S\in\Bounded(V,W)$. Then
\begin{enumerate}
\setcounter{enumi}{3}
\item $(ST)^t = T^tS^t$
\end{enumerate}
\end{lemma}


\subsection{Annihilator subspace}
\begin{definition}
Let $U\subset V$ be a subspace. The \udef{annihilator} of $U$, denoted $U^0$, is the set of functionals that are identically zero on $U$:
\[ U^0 = \left\{ \varphi\in V^*\;|\; \forall u\in U:\varphi(u) = 0 \right\}. \]
\end{definition}
\begin{proposition} \label{annihilatorSpace}
Let $U\subset V$ be a subspace and $T\in \Hom(V,W)$.
\begin{enumerate}
\item $U^0$ is a subspace of $\adual{V}$;
\item $\dim \adual{U} + \dim U^0 = \dim \adual{V}$;
\item $\ker T^t = (\im T)^0$
\item $T$ is surjective \textup{if and only if} $T^t$ is injective.
\end{enumerate}
\end{proposition}
\begin{proof}
\mbox{}
\begin{enumerate}
\item Elementary application of subspace criterion, proposition \ref{subspaceCriterion}.
\item Consider the inclusion $\iota: U\hookrightarrow V$. Then the dimension theorem \ref{dimensionLinearMaps} applied to $\iota'$ gives
\[ \dim \im \iota' + \dim \ker\iota' = \dim V^*. \]
Now $\dim \ker\iota'$ are $\varphi\in V^*$ such that $\varphi \circ \iota = 0$. These are exactly the elements of the annihilator. Any functional on $U$ can be extended to a functional on $V$, so $\iota'$ is surjective and $\dim \im \iota' = \dim U^*$.
\item There are two inclusions. First assume $\varphi \in \ker T'$, so $\forall v\in V$
\[ 0 = (\varphi\circ T)(v) = \varphi(Tv). \]
Thus $\varphi\in(\im T)^0$. The other inclusion uses the same equality.
\item $T\in\Hom(V,W)$ is surjective iff $\im T = W$ iff $(\im T)^0 = \{0\}$ iff $\ker T' = \{0\}$ iff $T'$ is injective.
\end{enumerate}
\end{proof}



\subsection{Real functionals}
\begin{definition}
Let $V$ be a real or complex vector space. Let $f: V\to \R$ be a real functional. We say
\begin{itemize}
\item $f$ is \udef{subadditive} or satisfies the \udef{triangle inequality} if $\forall x,y\in V: f(x+y) \leq f(x) + f(y)$;
\item $f$ is \udef{quasi-subadditive} if $\exists K>0: \forall x,y\in V: f(x+y) \leq K\big(f(x) + f(y)\big)$;
\item $f$ is \udef{point-separating} if $\forall x\in V: f(x) = 0 \implies x = 0$;
\item $f$ is \udef{convex} if $\forall x,y\in V, \lambda\in[0,1]: f(\lambda x + (1-\lambda)y) \leq \lambda f(x) + (1-\lambda)f(y)$.
\end{itemize}
We call $f$
\begin{itemize}
\item \udef{sublinear} if it is subadditive and positively homogeneous;
\item a \udef{seminorm} if it is subadditive and absolutely homogeneous;
\item a \udef{quasi-seminorm} if it is quasi-subadditive and absolutely homogeneous;
\item a \udef{norm} is a point-separating seminorm;
\item a \udef{quasi-norm} is a point-separating quasi-seminorm.
\end{itemize}
\end{definition}

TODO general valued fields.

\begin{lemma}
Let $V$ be a real or complex vector space and $f: V\to \R$ be a real functional. Then
\begin{enumerate}
\item absolute homogeneity $\implies$ positive homogeneity;
\item subadditivity $\implies$ quasi-subadditivity;
\item subadditivity+positive homogeneity $\implies$ convexity $\implies$ subadditivity.
\end{enumerate}
\end{lemma}
Thus norms and seminorms are sublinear.

\begin{lemma} \label{seminormPositivity}
Let $f:V\to \R$ be a quasi-seminorm. Then $\im(f)\subseteq \R^+$.
\end{lemma}
Thus (quasi)-seminorms are often considered as function in $V\to \R^+$.
\begin{proof}
For all $v\in V$ we have $0 = f(v-v) \leq K\big(f(v)+f(-v)\big) = 2Kf(v)$, so $f(v) \geq 0$.
\end{proof}

\begin{proposition}[Reverse triangle inequality] \label{reverseTriangleInequality}
Let $V$ be a vector space and $\norm{\cdot}: V\to \R$ a function that satisfies the triangle inequality and has $\norm{-v} = \norm{v}$ for all $v\in V$. Then $\forall v,w\in V$:
\begin{enumerate}
\item $|\norm{v}-\norm{w}|\leq \norm{v-w}$;
\item $|\norm{v}-\norm{w}|\leq \norm{v+w}$.
\end{enumerate}
In particular this holds if $\norm{\cdot}$ is a norm or seminorm.
\end{proposition}
\begin{proof}
We calculate $\norm{v} = \norm{v-w+w} \leq \norm{v-w} + \norm{w}$, so $\norm{v}-\norm{w}\leq \norm{v-w}$. By swapping $v\leftrightarrow w$ we also get $-\norm{v}+\norm{w}\leq \norm{w-v} = \norm{v-w}$ and thus the first inequality is established.

For the second inequality, set $w\to -w$ and use $\norm{-w} = \norm{w}$.
\end{proof}

\subsubsection{Extended real functionals}
\begin{definition}
Let $V$ be a real or complex vector space. An \udef{extended real functional} is a function $V \to \overline{\R}$.
\end{definition}

\begin{lemma} \label{realPartExtendedRealFunctional}
Let $V$ be a real or complex vector space and $f: V\to \overline{\R}$ an extended real functional. If $f$ is a quasi-seminorm and there exists $v\in V$ such that $f(v)\in\R$, then $f^{\preimf}(\R)$ is a subspace of $V$. 
\end{lemma}
\begin{proof}
We verify the subspace criteria (\ref{subspaceCriterion}). 

By assumption, $f^{\preimf}(\R)$ is not empty.

Take $v,w\in f^{\preimf}(\R)$. As $\im(f) \subseteq \overline{\R}^+$ by \ref{seminormPositivity}, we have $0\leq f(v+w)$. Also $f(v+w)\leq K\big(f(v)+f(w)\big)\in \R$. Thus $f(v+w)\in\R$.

Take $\lambda\in\F$. Then $f(\lambda v) = |\lambda|f(v)\in \R$.
\end{proof}

\subsubsection{Epigraphs}
\begin{definition}
Let $V$ be a vector space and $f: V\to \R$ a real functional on $V$. Then \udef{epigraph} of $f$ is defined as
\[ \epigraph(f) \defeq \setbuilder{(v,r)\in V\times \R}{f(v)\leq r}. \]
\end{definition}

\begin{lemma} \label{epigraphLemma}
Let $V$ be a vector space and $f: V\to \R$ a real functional on $V$. Then for all $v\in V$:
\[ f(v) = \inf\setbuilder{r}{(v,r)\in \epigraph(f)}. \]
\end{lemma}

\begin{proposition}
Let $V$ be a real vector space and $f: V\to \R$ a functional. Then
\begin{enumerate}
\item $f$ is convex \textup{if and only if} $\epigraph(f)$ is a convex subset of $V\oplus \R$;
\item $f$ is positively homogeneous \textup{if and only if} $\epigraph(f)$ is a cone in $V\oplus \R$.
\end{enumerate}
\end{proposition}
\begin{proof}
(1) First assume $f$ convex and pick $(v, s), (w,t)\in \epigraph(f)$ and $\lambda\in [0,1]$. Then we need to show that $(\lambda v + (1-\lambda)w, \lambda s + (1-\lambda)t) \in \epigraph(f)$. This is equivalent to saying $f(\lambda v + (1-\lambda)w) \leq \lambda s + (1-\lambda)t$. Indeed we have $f(\lambda v + (1-\lambda)w) \leq \lambda f(v) + (1-\lambda)f(w) \leq \lambda s + (1-\lambda)t$ by the convexity of $f$.

Conversely, assume $\epigraph(f)$ convex. Then $(v, f(v)), (w,f(w))\in \epigraph(f)$, $(\lambda v + (1-\lambda)w, \lambda f(v) + (1-\lambda)f(w)) \in \epigraph(f)$ for all $\lambda\in [0,1]$. This implies $f(\lambda v + (1-\lambda)w) \leq \lambda f(v) + (1-\lambda)f(w)$.

(2) First assume $f$ is positively homogeneous, take $(v,s)\in \epigraph(f)$ and $r>0$. Then we need to show that $r(v,s) = (rv,rs)\in \epigraph(f)$. This follows because of the implications $f(v)\leq s \implies rf(v) \leq rs \implies f(rv) \leq rs$.

Conversely, assume that $\epigraph(f)$ is a cone. Then $\lambda\cdot \epigraph(f) = \epigraph(f)$ for all $\lambda>0$ by \ref{coneEqualityLemma}. We then calculate using \ref{epigraphLemma}:
\begin{align*}
f(\lambda v) &= \inf\setbuilder{r}{(\lambda v,r)\in \epigraph(f)} \\
&= \inf\setbuilder{r}{(\lambda v,r)\in \lambda\cdot\epigraph(f)} \\
&= \inf\setbuilder{r}{\lambda(v,\lambda^{-1}r)\in \lambda\cdot\epigraph(f)} \\
&= \inf\setbuilder{r}{(v,\lambda^{-1}r)\in \epigraph(f)} \\
&= \inf\setbuilder{\lambda r}{(v,r)\in \epigraph(f)} = \lambda f(v).
\end{align*} 
\end{proof}
\begin{corollary}
A functional on a real vector space is sublinear \textup{if and only if} its epigraph is a convex cone.
\end{corollary}

\subsubsection{Convex functionals}

\begin{proposition}
Let $p: V\to\R$ be convex functional. Then
\[ P: V\to\R: x\mapsto \inf_{t>0} t^{-1}p(tx) \]
is sublinear and $P(x)\leq p(x)$.

Also, if $f:V\to \R$ is a linear functional, then $f\leq p \iff f\leq P$.
\end{proposition}
\begin{proof}
For sublinearity: let $x,y\in V$, then for all $s,t>0$
\[ P(x+y) \leq \frac{s+t}{st}p\left(\frac{st}{s+t}(x+y)\right) = \frac{s+t}{st}p\left(\frac{s}{s+t}(tx)+\frac{t}{s+t}(sy)\right) \leq t^{-1}p(tx) + s^{-1}p(sy). \]
This implies that $P(x+y)\leq P(x)+P(y)$.

For positive homogeneity: let $x\in V,\lambda\geq 0$
\[ P(\lambda x) = \inf_{t>0} t^{-1}p(t\lambda x) = \inf_{t\lambda>0} \lambda (t\lambda)^{-1}p(t\lambda x) = \inf_{t>0} \lambda (t)^{-1}p(tx) = \lambda P(x). \]

Finally we prove that $f\leq p \implies f\leq P$ for linear functionals $f$. For all $t>0$ we have $f(tx) \leq p(tx)$, which implies $f(x) = t^{-1}f(tx) \leq t^{-1}p(tx) \leq P(x)$. So $f\leq P$.
\end{proof}

\subsubsection{Sublinear functionals}

\begin{lemma}
Let $V$ be a \emph{real} vector space and $f: V \to \R$ a sublinear functional. Then $f': V\to \R: x\mapsto \max\{f(x), f(-x)\}$ is a seminorm.
\end{lemma}
We call the seminorm $f'$ the \udef{associated seminorm} of the sublinear functional.
\begin{proof}
Take arbitrary $x,y\in V$ and $a\in \R$

For subadditivity, we have
\begin{align*}
f'(x+y) &= \max\{f(x+y), f(-x-y)\} \\
&\leq \max\{f(x)+f(y), f(-x)+ f(-y)\} \\
&\leq \max\{f(x)+f(y), f(-x)+ f(-y), f(x) + f(-y), f(-x) + f(y)\} \\
&= \max\{f(x), f(-x)\} + \max\{f(y), f(-y)\} \\
&= f'(x) + f'(y).
\end{align*}

For absolute homogeneity, we have,
\begin{align*}
f'(ax) &= \begin{cases}
f'(|a|x) & (a \geq 0) \\ f'(-|a|x) & (a < 0)
\end{cases} \\
&= \begin{cases}
\max\{f(|a|x), f(-|a|x)\} & (a \geq 0) \\ \max\{f(-|a|x), f(|a|x)\} & (a < 0)
\end{cases} \\
&= \max\{|a|f(x), |a|f(-x)\} = |a|f'(x).
\end{align*}
\end{proof}

\begin{lemma} \label{superSubtractiveContinuityEverywhere}
Let $\sSet{V, \xi}$ be a convergence vector space and $f: V\to \R$ a function such $f(0) = 0$ and $|f(v) - f(w)| \leq \max\{|f(v-w)|, |f(w-v)|\}$ for all $v,w\in V$.

Then continuity of $f$ at $0$ implies continuity everywhere.
\end{lemma}
\begin{proof}
Take arbitrary $F\overset{\xi}{\longrightarrow} u$. Then $F-u \to 0$, so $|f^{\imf\imf}(F-u)|\to 0$ by continuity at $0$. Similarly $|f^{\imf\imf}(u-F)|\to 0$. Then, by assymption, $|f^{\imf\imf}(F) - f(u)| \leq \max\{|f^{\imf\imf}(F-u)|, |f^{\imf\imf}(u-F)|\} \to 0$, so $|f^{\imf\imf}(F) - f(u)| \to 0$, by TODO ref squeeze theorem. We conclude that $f^{\imf\imf}(F)\to f(u)$.
\end{proof}

\begin{proposition} \label{sublinearContinuity}
Let $\sSet{V,\xi}$ be a convergence vector space and $f: V\to \R$ a sublinear functional. Then the following are equivalent:
\begin{enumerate}
\item $f$ is continuous;
\item $f$ is continuous at $0$;
\item $f$ is bounded on some $U\in\vicinity_\xi(0)$.
\end{enumerate}
\end{proposition}
\begin{proof}
The implication $(1) \Rightarrow (2)$ is immediate. 

The implication $(2) \Rightarrow (1)$ follows from \ref{superSubtractiveContinuityEverywhere}: 
for arbitrary $v,w\in V$, we have $f(v) = f(v-w+w) \leq f(v-w) + f(w)$, so $f(v) - f(w) \leq f(v-w)$. Similarly $f(w) - f(v) \leq f(w-v)$, so
\begin{align*}
|f(v) - f(w)| &= \max\{f(v) - f(w), f(w) - f(v)\} \\
&\leq \max\{f(v-w), f(w-v)\} \\
&\leq \max\{|f(v-w)|, |f(w-v)|\}.
\end{align*}

The equivalence $(2) \Leftrightarrow (3)$ is given by \ref{continuityToNormedSpace}.
\end{proof}
\begin{corollary} \label{continuityAbsFunctional}
Let $\sSet{V,\xi}$ be a convergence vector space and $f: V\to \R$ a \emph{linear} functional. Then $|f|$ is continuous \textup{if and only if} $f$ is continuous.
\end{corollary}
\begin{proof}
The functional $|f|$ is sublinear and bounded on the same sets as $f$. We can then compare the proposition to \ref{continuityToNormedSpace}.
\end{proof}

\subsubsection{Gauges}
\begin{definition}
Let $V$ be a vector space and $A\subseteq V$ an absorbent subset. The function
\[ p_A: V\to \overline{\R^+}: v\mapsto \inf\setbuilder{\lambda\in \R^{\geq 0}}{v\in \lambda A} \]
is called the \udef{gauge} or \udef{Minkowski functional} of $A$.
\end{definition}

\begin{lemma}
Let $V$ be a vector space and $A,B\subseteq V$. Then
\begin{enumerate}
\item if $A$ absorbs $B$, then $p_A^\imf(B)$ is bounded;
\item if $A$ is balanced and $p_A^\imf(B)$ is bounded, then $A$ absorbs $B$.
\end{enumerate}
\end{lemma}
\begin{proof}
(1) Assume $A$ absorbs $B$. Then there exists $r >0$ such that $B\subseteq cA$ for all $|c|\geq r$. In particular $v\in rA$, so $p_A(v) \leq r$, for all $v\in B$. Thus $p_A^\imf(B)$ is bounded by $r$.

(2) Assume $p_A^\imf(B)$ is bounded. Then we can find an upper bound $s>0$. Take $|c|\geq s$ arbitrarily. We need to show that $B\subseteq cA$.

For all $v\in B$, we can find some $\lambda \leq s$ such that $v\in\lambda A = c\left(\frac{\lambda}{c}\right)A \subseteq cA$. The last inclusion follows because $A$ is balanced and $\left|\frac{\lambda}{c}\right|\leq 1$ (as $\lambda \leq s \leq |c|$).
\end{proof}
\begin{corollary} \label{gaugeWellDefined}
Let $V$ be a vector space and $A\subseteq V$. Then
\begin{enumerate}
\item if $A$ is absorbent, then $p_A(v)$ is finite for all $v\in V$;
\item if $A$ is balanced and $p_A(v)$ is finite for all $v\in V$, then $A$ is absorbent.
\end{enumerate}
\end{corollary}

\begin{lemma} \label{gaugeScaling}
Let $V$ be a vector space and $A\subseteq V$ an absorbent subset. For all $v\in V$ and $t\geq 0$:
\[ p_A(tv) = p_{t^{-1}A}(v) = t p_A(v). \]
Thus $p_A$ is positively homogeneous.
\end{lemma}
\begin{proof}
We calculate
\begin{align*}
p_A(tv) &= \inf\setbuilder{\lambda\in \R^{\geq 0}}{tv\in \lambda A} \\
&= \inf\setbuilder{\lambda\in \R^{\geq 0}}{v\in t^{-1}\lambda A} = p_{t^{-1}A}(v) \\
&= \inf\setbuilder{t\lambda\in \R^{\geq 0}}{v\in \lambda A} \\
&= t\inf\setbuilder{\lambda\in \R^{\geq 0}}{v\in \lambda A} = tf(v).
\end{align*}
\end{proof}

\begin{lemma} \label{semibalancedClosureGauge}
Let $V$ be a vector space and $A\subseteq V$ an absorbent subset. Then $p_A = p_{\semibalanced(A)}$.
\end{lemma}
\begin{proof}
We calculate
\begin{align*}
\setbuilder{\lambda\in\R^{\geq 0}}{v\in \lambda \semibalanced(A)} &= \setbuilder{\lambda\in\R^{\geq 0}}{v\in \lambda\cdot\interval{0,1}\cdot A} \\
&= \setbuilder{\lambda\in\R^{\geq 0}}{\exists k\in \interval{0,1}:\; v\in \lambda k\cdot A} \\
&= \setbuilder{\lambda\in\R^{\geq 0}}{\exists \lambda'\leq \lambda:\; v\in \lambda' \cdot A} \\
&= \upset \setbuilder{\lambda'\in\R^{\geq 0}}{v\in \lambda' \cdot A}.
\end{align*}
Thus the infima of both $\setbuilder{\lambda\in\R^{\geq 0}}{v\in \lambda \semibalanced(A)}$ and $\setbuilder{\lambda'\in\R^{\geq 0}}{v\in \lambda' \cdot A}$ are the same.
\end{proof}

\begin{lemma} \label{gaugeLemma}
Let $V$ be a vector space, $A\subseteq V$ an absorbent subset and $\lambda\in \R^{> 0}$. Then
\begin{enumerate}
\item if $A$ is semibalanced, then $p_A(v) < \lambda \implies \lambda^{-1}v\in A$;
\item $\lambda^{-1}v\in A \implies p_A(v) \leq \lambda$;
\item $p_A^{\preimf}[\ball(0,1)] \;\subseteq\; \semibalanced(A) \;\subseteq\; p_A^{\preimf}[\cball(0,1)]$;
\item $\frac{v}{p_A(v)+\epsilon} \in \semibalanced(A)$ for all $v\in V$ and $\epsilon >0$.
\end{enumerate}
\end{lemma}
\begin{proof}
(1) If $p_A(v) < \lambda$, then there exists $\lambda' \leq \lambda$ such that $v\in \lambda' A$. As we can write $\lambda' = k\lambda$ for some $k\in\interval{0,1}$, we have
\[ v\in \lambda k A \subseteq \lambda \cdot \interval{0,1}\cdot A = \lambda A, \]
where we have used that $A = \semibalanced(A) = \interval{0,1}\cdot A$. Thus $\lambda^{-1}v \in A$.

(2) Assume $\lambda^{-1}v \in A$. Then $v\in \lambda A$, so we immediately have $p_A \leq \lambda$.

(3) This follows from (1), the fact that $p_A(v) < p_A(v) + \epsilon$ and the fact that $p_A = p_{\semibalanced(A)}$, \ref{semibalancedClosureGauge}.
\end{proof}

\begin{lemma} \label{gaugeClassificationLemma}
Let $V$ be a vector space, $f: V\to \R^{\geq 0}$ a positively homogeneous function and $A \subseteq V$ a semibalanced subset.
Then the following are equivalent:
\begin{enumerate}
\item $f = p_A$;
\item $f^{\preimf}[\ball(0,1)] \subseteq A \subseteq f^{\preimf}[\cball(0,1)]$.
\end{enumerate}
\end{lemma}
\begin{proof}
$(1) \Rightarrow (2)$ We calculate, using \ref{gaugeLemma},
\begin{align*}
x\in p_{A}^\preimf[\ball(0,1)] \iff& p_{A}(x) < 1 \\
\implies& x\in A \\
\implies& p_{A}(x) \leq 1 \\
\iff& x\in p_{A}^\preimf[\cball(0,1)].
\end{align*}
Reformulating in terms of sets gives the result.

$(2) \Rightarrow (1)$ We calculate, for arbitrary $v\in V$,
\[ \begin{aligned}
p_A(v) &= \inf\setbuilder{\lambda\in \R^{\geq 0}}{v\in \lambda A} \\
&\leq \inf\setbuilder{\lambda\in \R^{\geq 0}}{v\in \lambda f^{\preimf}[\ball(0,1)]} \\
&= \inf\setbuilder{\lambda\in \R^{\geq 0}}{v\in f^{\preimf}[\ball(0,\lambda)]} \\
&= \inf\setbuilder{\lambda\in \R^{\geq 0}}{f(v) < \lambda} = f(v)
\end{aligned} \quad\text{and}\quad \begin{aligned}
p_A(v) &= \inf\setbuilder{\lambda\in \R^{\geq 0}}{v\in \lambda A} \\
&\geq \inf\setbuilder{\lambda\in \R^{\geq 0}}{v\in \lambda f^{\preimf}[\cball(0,1)]} \\
&= \inf\setbuilder{\lambda\in \R^{\geq 0}}{v\in f^{\preimf}[\cball(0,\lambda)]} \\
&= \inf\setbuilder{\lambda\in \R^{\geq 0}}{f(v) \leq \lambda} = f(v).
\end{aligned} \]
We conclude that $f(v) = p_A(v)$.
\end{proof}

\begin{proposition} \label{gaugeClassification}
Let $V$ be a vector space and $f: V\to \R^{\geq 0}$ a function.
Then the following are equivalent:
\begin{enumerate}
\item $f$ is positively homogenous;
\item $f = p_A$ for some semibalanced, absorbent subset $A$.
\end{enumerate}
\end{proposition}
\begin{proof}
Assume $f$ positively homogeneous. Then we can set $A = f^{\preimf}[\ball(0,1)]$ in \ref{gaugeClassificationLemma} because $f^{\preimf}[\ball(0,1)]$ is semibalanced. Indeed
\[ \interval{0,1}\cdot f^{\preimf}[\ball(0,1)] = f^{\preimf}[\interval{0,1}\cdot\ball(0,1)] = f^{\preimf}[\ball(0,1)]. \]

The converse is given by \ref{gaugeScaling}.
\end{proof}

\begin{lemma} \label{gaugeZeroLemma}
Let $V$ be a vector space, $A\subseteq V$ an absorbent subset and $a\in A$. If there exists a subspace $U\subseteq A$ such that $a\in U$, then $p_A(a) = 0$.
\end{lemma}
\begin{proof}
For all $\epsilon > 0$, $\epsilon^{-1}a\in A$, so $a\in \epsilon A$.
\end{proof}

\begin{proposition} \label{gaugeProperties}
Let $V$ be a vector space and $A\subseteq V$ an absorbent subset. Then
\begin{enumerate}
\item $p_A$ is absolutely homogenous if $A$ is balanced;
\item $p_A$ is sublinear if $A$ is convex;
\item $p_A$ is point-separating if $A$ is balanced and contains only the trivial subspace.
\end{enumerate}
\end{proposition}
\begin{proof}
(1) By \ref{balancedLemma} we have $\mu A = |\mu| A$ and thus
\begin{align*}
p_A(\mu\cdot v) &= \inf\setbuilder{\lambda\in \R^{\geq 0}}{\mu\cdot v\in \lambda A} = \inf\setbuilder{\lambda\in \R^{\geq 0}}{v\in \frac{\lambda}{\mu} A} \\
&= \inf\setbuilder{\lambda\in \R^{\geq 0}}{v\in \frac{\lambda}{|\mu|} A} = \inf\setbuilder{|\mu|\lambda\in \R^{\geq 0}}{v\in \lambda A} = |\mu|\cdot p_A(v).
\end{align*}

(2) We just need to show subadditivity. Positive homogeneity is automatic by \ref{gaugeScaling}. Take $v,w\in V$. Now take arbitrary $\epsilon > 0$, so $(p_A(v)+\epsilon)^{-1}v \in A$ and $(p_A(w)+\epsilon)^{-1}w \in A$ by \ref{gaugeLemma} (as $A$ is semibalanced by \ref{convexAbsorbentImpliesSemibalanced}). By convexity of $A$, we have
\[ \frac{v+w}{p_A(v)+p_A(w)+2\epsilon} = \frac{p_A(v)+\epsilon}{p_A(v)+p_A(w)+2\epsilon}(p_A(v)+\epsilon)^{-1}v + \frac{p_A(w)+\epsilon}{p_A(v)+p_A(w)+2\epsilon}(p_A(w)+\epsilon)^{-1}w \in A. \]
By \ref{gaugeLemma} this means $p_A(v)+p_A(w)+2\epsilon \geq p_A(v+w)$ and because $\epsilon$ was arbitrary, we conclude that $p_A(v+w) \leq p_A(v)+p_A(w)$.

(3) Assume $A$ contains only the trivial subspace. Then for all $v\in V$ there exists some $\lambda\in \F$ such that $\lambda\cdot v\notin A$. Now for all $|c|\geq |\lambda|$, $c\cdot v\notin A$ because $A$ is balanced. Then $p_A(2\lambda\cdot v) \neq 0$ and because $p_A$ is absolutely homogeneous we have $p_A(v) = (2\lambda)^{-1}p_A(2\lambda\cdot v) \neq 0$.
\end{proof}
\begin{corollary}
The gauge of an absolutely convex and absorbent subset is a seminorm. If the subset contains only the trivial subspace, then the gauge is a norm.
\end{corollary}

\begin{proposition} \label{continuityConvexGauge}
Let $\sSet{V, \xi}$ be a convergence vector space and $A\subseteq V$ a convex, absorbent subset. Then $p_A: \sSet{V,\xi} \to \F$ is continuous \textup{if and only if} $A\in\vicinity_\xi(0)$.
\end{proposition}
\begin{proof}
First assume $A\in\vicinity_\xi(0)$. By \ref{gaugeProperties}, we have that $p_A$ is sublinear. By \ref{gaugeClassificationLemma}, we have that $A \subseteq p_A^\preimf[\cball(0,1)]$ and thus that $p_A$ is bounded on $A$. Then $p_A$ is continuous by \ref{sublinearContinuity}.

Now assume $p_A$ continuous. Then $\ball(0,1) \in \neighbourhood_\F(0) = \neighbourhood_\F(p_A(0))$, so $p_A^\preimf[\ball(0,1)] \in \vicinity_\xi(0)$ by \ref{continuityVicinityFilter}. By \ref{gaugeClassificationLemma} (using the fact that $A$ is semibalanced, \ref{convexSemibalanced}), $p_A^\preimf[\ball(0,1)] \subseteq A$, so $A\in\vicinity_\xi(0)$.
\end{proof}
\begin{corollary} \label{gaugeContinuousAlgebraicConvergence}
Let $V$ be a vector space. Then $p_A: \sSet{V,\mathfrak{a}} \to \F$ is continuous for any convex, absorbent subset $A \subseteq V$.
\end{corollary}
\begin{proof}
By \ref{coreProperties}, we have $0\in\inh_\mathfrak{a}(A)$. Then $A\in \vicinity_\mathfrak{a}(0)$ by \ref{principalAdherenceInherence}.
\end{proof}

\begin{lemma} \label{absoluteFunctionalGauge}
Let $\sSet{V, \xi}$ be a convergence vector space and $f: V\to \F$ a linear functional. Then $f$ is continuous \textup{if and only if} $|f| = p_K$ for some $K\in\vicinity_\xi(0)$.
\end{lemma}
\begin{proof}
We have that $|f|$ is positively homogeneous. Set $K \defeq |f|^{\preimf}[\ball(0,1)] = |f|^{\preimf}(\interval[o]{0,1})$. Then $|f| = p_{K}$, by \ref{gaugeClassificationLemma}. 

First assume $f$ is continuous. Then $p_K^\preimf\big[\ball(0,1)\big] = K$ is in $\vicinity_\xi(0)$ by \ref{continuityVicinityFilter}.

Now assume $K\in\vicinity_\xi(0)$. As $K \subseteq p_K^\preimf\big[\cball(0,1)\big]$ (\ref{gaugeClassificationLemma}), we have that $|f|$ is bounded on $K$. It is also sublinear and thus continuous by \ref{sublinearContinuity}.
\end{proof}


\begin{proposition} \label{gaugeInherenceAdherence}
Let $V$ be a vector space and $A\subseteq V$ an absorbent semibalanced subset. Then
\begin{enumerate}
\item $\inh_\mathfrak{a}(A) \subseteq p_A^{\preimf}[\ball(0,1)]$;
\item $p_A^{\preimf}[\cball(0,1)] \subseteq \adh_\mathfrak{a}(A)$.
\end{enumerate}
The inclusions are equalities if $A$ is convex.
\end{proposition}
\begin{proof}
(1) Take $x\in \inh_\mathfrak{a}(A)$. Then by \ref{constructionsInAlgebraicConvergence}, there exists $\Gamma\in\neighbourhood_\F(0)$ such that $v+\Gamma\cdot v = (1 + \Gamma)\cdot v \subseteq A$. There exists $\lambda \in 1 + \Gamma$ that is real and strictly less than $1$. Thus $p_A(x) <1$, meaning $x\in p^\preimf_A\big(\ball(0,1)\big)$.

(Convex) For the other inclusion we use that $p_A$ is continuous by \ref{gaugeContinuousAlgebraicConvergence}. By \ref{adherenceInherenceContinuity} and \ref{gaugeClassificationLemma} we have
\[ p_A^{\preimf}[\ball(0,1)] = p_A^{\preimf}\Big[\interior_\F\big(\ball(0,1)\big)] \subseteq \inh_\mathfrak{a}\big(p_A^{\preimf}[\ball(0,1)]\big) \subseteq \inh_\mathfrak{a}(A). \]

(2) Take $x\in p^\preimf_A\big(\cball(0,1)\big)$, which means that $p_A(x) \leq 1$, so $p_A(x) < 1+\epsilon$ for all $\epsilon > 0$. By \ref{gaugeLemma}, $(1+\epsilon)^{-1}x = x - \epsilon(1+\epsilon)^{-1}x \in A$. Because $\frac{\epsilon}{1+\epsilon} < \epsilon$, we have $x - \epsilon(1+\epsilon)^{-1}x \in x + \ball(0,\epsilon)\cdot x$ and so $x + \ball(0,\epsilon)\cdot x \mesh A$.

Now for all $\Gamma\in\neighbourhood_\F(0)$, we have $\ball(0,\epsilon) \subseteq \Gamma$ for some $\epsilon >0$. Thus $x+ \Gamma\cdot x \mesh A$ and $x\in\adh_\mathfrak{a}(A)$ by \ref{constructionsInAlgebraicConvergence}.

(Convex) For the other inclusion we again use that $p_A$ is continuous by \ref{gaugeContinuousAlgebraicConvergence}. From \ref{gaugeClassificationLemma} we have $A \subseteq p^\preimf_A\big(\cball(0,1)\big)$. Thus, by \ref{adherenceInherenceContinuity},
\[ \adh_\mathfrak{a}(A) \subseteq \adh_\mathfrak{a}\Big(p^\preimf_A\big(\cball(0,1)\big)\Big) \subseteq p_A^\preimf\big(\overline{\cball(0,1)}\big) = p^\preimf_A\big(\cball(0,1)\big). \]
\end{proof}

\begin{example}
TODO square with corner missing.
\end{example}



\begin{proposition} \label{gaugeConstructions}
Let $V$ be a vector space and $A,B\subseteq V$ semibalanced subsets. Then
\begin{enumerate}
\item the gauge of $A\cap B$ is $\max\{p_A, p_B\}$;
\item if $A\subseteq B$, then $p_B \leq p_A$;
\item if $p_B \leq p_A$ and $A$ is convex, then $\adh_\mathfrak{a}(A) \subseteq \adh_\mathfrak{a}(B)$.
\end{enumerate}
\end{proposition}
\begin{proof}
(1) Clearly $p_A \leq p_{A\cap B}$:
\begin{align*}
p_A(v) &= \inf\setbuilder{\lambda\in\R^{\geq 0}}{v\in \lambda A} \\
&\leq \setbuilder{\lambda\in\R^{\geq 0}}{v\in \lambda (A\cap B)} \\
&= p_{A\cap B}(v).
\end{align*}
Similarly $p_B \leq p_{A\cap B}$. Thus $\max\{p_A, p_B\} \leq p_{A\cap B}$.

For the other inequality we use \ref{gaugeLemma}: take $v\in V$ and arbitrary $\epsilon > 0$. Then $\max\{p_A(v)+\epsilon, p_B(v)+\epsilon\} > p_A(v)$, so $\max\{p_A(v)+\epsilon, p_B(v)+\epsilon\}^{-1}v\in A$. Similarly $\max\{p_A(v)+\epsilon, p_B(v)+\epsilon\}^{-1}v\in B$. Thus $\max\{p_A(v)+\epsilon, p_B(v)+\epsilon\}^{-1}v \in A\cap B$ and finally $p_{A\cap B}(v) \leq \max\{p_A(v)+\epsilon, p_B(v)+\epsilon\}$. Taking the limit $\epsilon \to 0$, we have $p_{A\cap B} \leq \max\{p_A, p_B\}$.

(2) First assume $A\subseteq B$. Then $\frac{v}{p_A(v)+\epsilon}\in A$ by \ref{gaugeLemma}, for all $\epsilon >0$. Then $\frac{v}{p_A(v)+\epsilon}\in B$, so $p_B(v) \leq p_A(v)+\epsilon$ (again by \ref{gaugeLemma}). Taking the limit $\epsilon\to 0$ yields the result.

(3) Assume $p_B \leq p_A$ and take $v\in \adh_\mathfrak{a}(A)$. Then $p_B(v) \leq p_A(v) \leq 1$ by \ref{gaugeInherenceAdherence}, which also gives $v\in\adh_\mathfrak{a}(B)$.
\end{proof}

\subsubsection{Seminorms}
\begin{lemma}
The kernel of a seminorm is a vector space.
\end{lemma}
Note this does not follow from \ref{kernelSubspace} because seminorms are not linear.
\begin{proof}
Let $p:V\to \R$ be a seminorm. We verify the subspace criterion \ref{subspaceCriterion}. First $0\in\ker(p)$ because $p(0) = p(0\cdot 0) = |0|p(0) = 0$.

Now take $v,w\in \ker(p)$ and $\lambda\in \F$. Then $0\leq p(v+\lambda w) \leq p(v)+|\lambda|p(w) = 0$, so $v+\lambda w\in\ker(p)$.
\end{proof}

\begin{proposition} \label{initialSeminormConvergence}
Let $V$ be a vector space and $S$ a set of seminorms on $V$. Let $\xi$ be the initial convergence on $V$ w.r.t. $S$. Then
\begin{enumerate}
\item $\xi$ is a topological vector space convergence;
\item $\begin{aligned}[t]
\neighbourhood_\xi(0) &= \mathfrak{F}\setbuilder{p^\preimf[\ball(0,\epsilon)]}{p\in S, \epsilon > 0} \\
&= \mathfrak{F}\setbuilder{p^\preimf[\cball(0,\epsilon)]}{p\in S, \epsilon > 0}
\end{aligned}$
\item $f\in \sSet{V, \xi}^*$ \textup{if and only if}
\begin{itemize}
\item $f\in \sSet{V,\mathfrak{a}}^*$;
\item there exists a finite subset $A\subseteq S$ and $C>0$ such that $|f(v)| \leq C\max_{g\in A} g(v)$ for all $v\in V$.
\end{itemize}
\end{enumerate}
\end{proposition}
\begin{proof}
(1, 2) That $\xi$ is topological follows from \ref{pretopologicalInitialConvergence}, which also gives point (2).

To show $\xi$ is a vector space convergence, we verify the conditions in \ref{TVSconstruction}:
\begin{enumerate}
\item Take $\lambda\in \F$ and $U\in \neighbourhood_\xi(0)$. Then there exist $p\in S$ and $\epsilon > 0$ such that $p^\preimf[\ball(0,\epsilon)] \subseteq U$. Then
\[ \lambda U \subseteq \lambda p^\preimf[\ball(0,\epsilon)] = p^\preimf[|\lambda|\ball(0,\epsilon)] = p^\preimf[\ball(0,|\lambda|\cdot \epsilon)], \]
so $\lambda U\in \neighbourhood_\xi(0)$.
\item Take $U\in \neighbourhood_\xi(0)$, so there exist $p\in S$ and $\epsilon > 0$ such that $p^\preimf[\ball(0,\epsilon)] \subseteq U$. Then
\[ p^\preimf[\ball(0,\epsilon/2)] + p^\preimf[\ball(0,\epsilon/2)] \subseteq p^\preimf[\ball(0,\epsilon)] \subseteq U. \]
\item By \ref{coreProperties}.
\item We claim $p^\preimf[\ball(0,\epsilon)]$ is balanced for all $p\in S$ and $\epsilon > 0$. Indeed for all $|r|\leq 1$
\[ rp^\preimf[\ball(0,\epsilon)] = p^\preimf[|r|\cdot \ball(0,\epsilon)] = p^\preimf[\ball(0,|r|\cdot \epsilon)] \subseteq p^\preimf[\ball(0,\epsilon)]. \]
\end{enumerate}

(3) We have by \ref{continuityLinearFunctionals}
\[ f\in \sSet{V, \xi}^* \iff \exists D>0: \exists U\in \neighbourhood_\xi(0):\; f^\imf[U] \subseteq \cball(0,D). \]
Now $U\in \neighbourhood_\xi(0)$ iff there exists a finite $A = \{p_n^\preimf[\ball(0,\epsilon_n)]\}_{n=0}^N \subseteq \setbuilder{p^\preimf[\ball(0,\epsilon)]}{p\in S, \epsilon > 0}$ such that $\bigcap A\subseteq U$. So WLOG we may take $U$ of this form.
Now 
\begin{align*}
f^\imf\Big[\bigcap A\Big] \subseteq \cball(0,D) &\iff \forall v\in V: \; \Big(\forall n\leq N: p_n(v)\leq \epsilon_n \Big) \implies |f(v)|\leq D \\
&\iff \forall v\in V: \; \max_{n\leq N}\epsilon_n^{-1}p_n(v)\leq 1 \implies |f(v)|\leq D \\
&\iff \forall v\in V: \; \max_{n\leq N}\epsilon_n^{-1}p_n\left(\frac{\max_{n\leq N}\epsilon_n^{-1}p_n(v)}{\max_{n\leq N}\epsilon_n^{-1}p_n(v)} v\right)\leq 1 \implies |f(v)|\leq D \\
&\iff \forall v\in V: \; \left(\max_{n\leq N}\epsilon_n^{-1}p_n(v)\right)^{-1}\max_{n\leq N}\epsilon_n^{-1}p_n(v)\leq 1 \implies \left(\max_{n\leq N}\epsilon_n^{-1}p_n(v)\right)^{-1}|f(v)|\leq D \\
&\iff \forall v\in V: \; 1\leq 1 \implies |f(v)|\leq D\max_{n\leq N}\epsilon_n^{-1}p_n(v) \\
&\iff \forall v\in V: \; |f(v)|\leq D\max_{n\leq N}\epsilon_n^{-1}p_n(v).
\end{align*}
WLOG we may take all $\epsilon_n = \epsilon = \min_{n\leq N}\epsilon_n$. We may then take $C = D/\epsilon$.
\end{proof}
Note that for (1) we cannot use \ref{initialVectorSpaceConvergence}, as the elements of $S$ are not linear.

\begin{proposition} \label{locallyConvexSeminormTopology}
Let $V$ be a vector space. A topological convergence on $V$ is locally convex \textup{if and only if} it is the initial convergence w.r.t. some set $S$ of seminorms on $V$.
\end{proposition}
\begin{proof}
If $V$ has the initial convergence w.r.t. $S$, then $V$ is locally convex by \ref{initialSeminormConvergence} because $p^\preimf[\ball(0,\epsilon)]$ is convex for all $p\in S$.

Now let $V$ be a locally convex TVS, then there exists an absolutely convex base $\mathcal{B}$ of $\neighbourhood(0)$ by \ref{vicinityFilterAtOrigin}. Then $S = \setbuilder{p_B}{B\in \mathcal{B}}$ is a set of continuous seminorms, by \ref{gaugeProperties} and \ref{continuityConvexGauge}.

In order to show that the convergence on $V$ is initial w.r.t. $S$, we verify the form of $\neighbourhood(0)$ given in \ref{initialSeminormConvergence}.

We may take $\mathcal{B}$ to consist of algebraically open convex sets by replacing it with $\inh_\mathfrak{a}^\imf[\mathcal{B}]$, which contains open convex sets by \ref{coreProperties}. Then
\begin{align*}
\neighbourhood(0) &= \mathfrak{F}(\mathcal{B}) \\
&= \mathfrak{F}\setbuilder{p_B^\preimf[\ball(0,1)]}{B\in \mathcal{B}} \\
&= \mathfrak{F}\setbuilder{\epsilon^{-1} p_B^\preimf[\ball(0,1)]}{B\in \mathcal{B}} \\
&= \mathfrak{F}\setbuilder{p_B^\preimf[\ball(0,\epsilon)]}{B\in \mathcal{B}, \epsilon > 0} \\
&= \mathfrak{F}\setbuilder{p^\preimf(\ball(0,\epsilon))}{p\in S, \epsilon > 0},
\end{align*}
where we have used \ref{gaugeInherenceAdherence} and the fact that the convergence is a vector space convergence, so $\epsilon B \in \upset\mathcal{B}$.
\end{proof}

\begin{proposition}
Let $V$ be a vector space. The functions
\begin{align*}
\powerset\setbuilder{A\subseteq V}{\text{$A$ is convex}} &\to \setbuilder{p: V\to \R}{\text{$p$ is a seminorm}}: &&\mathcal{B}\mapsto \setbuilder{p_K}{K\in \mathcal{B}} \\
\setbuilder{p: V\to \R}{\text{$p$ is a seminorm}} &\to \powerset\setbuilder{A\subseteq V}{\text{$A$ is convex}}: &&S\mapsto \setbuilder{p^\preimf[U]}{p\in S, U\in \neighbourhood_\R(0)}
\end{align*}
form an antitone Galois connection, where we order the seminorms pointwise.
\end{proposition}
\begin{proof}
TODO + previous as corollary
\end{proof}

Note: metrisable is not equivalent to normable!






\subsection{Hahn-Banach extension theorems}
\begin{theorem}[Hahn-Banach majorised by convex functionals] \label{convexHahnBanach}
Let $V$ be a real vector space, $U\subset V$ a subspace and $p$ a convex functional on $V$. Let $f:U\to\R$ be a linear functional that is bounded by $p$:
\[ \forall u\in U: \quad f(u) \leq p(u). \]
Then $f$ has an extension $\tilde{f}: V\to \R$ such that $\tilde{f}$ is a linear functional on $V$ bounded by $p$:
\[ \forall v\in V: \tilde{f}(v) \leq p(v) \qquad \text{and} \qquad \forall u\in U: \tilde{f}(u) = f(u). \]
\end{theorem}
\begin{proof}
As a first step, we want to extend $f$ to a functional $g$ on a space that is one dimension larger than $U$. This means $g$ is of the form
\[ g: U\oplus\Span\{v_1\}\to\R: v + \alpha v_1 \mapsto f(v) + \alpha c \]
for some $v_1\in V\setminus U$.

If we want $g$ to be majorised by $p$, then we need to find a $c$ such that
\[ \forall v\in U: \forall \alpha\in\R: \; g(\alpha v_1 + v) = \alpha c + f(v) \leq p(\alpha v_1 + v) \]
this means that we need
\[ \forall v\in U: \forall \alpha\in\R:\; \frac{-p(v - |\alpha|v_1) + f(v)}{|\alpha|} \leq c \leq \frac{p(v + |\alpha|v_1) - f(v)}{|\alpha|} \]
and we can find such a $c$ if and only if
\[ \forall v\in U: \forall \alpha\in\R:\; -p(v - |\alpha|v_1) + f(v) \leq p(v + |\alpha|v_1) - f(v), \]
which is equivalent to $2f(v) \leq p(v+|\alpha|v_1)+p(v-|\alpha|v_1)$. This follows from
\begin{align*}
f(v) \leq p(v) &= p(\tfrac{1}{2}(v+|\alpha|v_1) + \tfrac{1}{2}(v-|\alpha|v_1)) \\
&\leq \tfrac{1}{2}p(v+|\alpha|v_1) + \tfrac{1}{2}p(v-|\alpha|v_1).
\end{align*}
So we can extend the domain of $f$ by one dimension such that it is still majorised by $p$.

We can iterate the construction to extend $f$ by multiple dimensions. Each extension can be viewed as a subset of $V\times \R$, by identifying it with its graph.
Consider the family of all such subsets that determine a majorised extension of $f$ (not just those obtained by iteration of the previous construction!). This is a family of finite character. We apply the Teichmüller-Tukey lemma, \ref{ZornEquivalents}, to obtain a maximal element.

This maximal element has domain $V$, because if it did not, it could be extended and was not a maximal element.
\end{proof}
Clearly if $V$ has a well-ordered Hamel basis, we do not need choice as we can just take successive $v$s in the basis and find $c$s constructively.
\begin{corollary}[Hahn-Banach majorised by sublinear functionals] \label{sublinearHahnBanach}
Any majorant $p$ that is sublinear is also convex and can be used in the Hahn-Banach theorem.
\end{corollary}
\begin{corollary}[Hahn-Banach majorised by seminorms] \label{seminormHahnBanach}
Let $(\mathbb{F},V,+)$ be a real or complex vector space, $U\subset V$ a subspace and $p$ a seminorm on $V$. Let $f:U\to\mathbb{F}$ be a linear functional that is bounded by $p$:
\[ \forall u\in U: \quad |f(u)| \leq p(u). \]
Then $f$ has an extension $\tilde{f}: V\to \R$ such that $\tilde{f}$ is a linear functional on $V$ bounded by $p$:
\[ \forall v\in V: |\tilde{f}(v)| \leq p(v) \qquad \text{and} \qquad \forall u\in U: \tilde{f}(u) = f(u). \]
\end{corollary}
\begin{proof}
First assume $V$ is a \emph{real} vector space. Because every seminorm is a sublinear function, we can use \ref{sublinearHahnBanach} to find an extension $\tilde{f}$. We then just need to check it satisfies $\forall v\in V: |\tilde{f}(v)| \leq p(v)$.
From \ref{sublinearHahnBanach} we know $\forall v\in V: \tilde{f}(v) \leq p(v)$.
To prove $-\tilde{f}(v) \leq p(v)$, we calculate
\[ -\tilde{f}(v) = \tilde{f}(-v) \leq p(-v) = |-1|p(v) = p(v). \]

If $V$ is a \emph{complex} vector space, consider the realification $V_\R$ and apply the preceding proof to obtain a linear functional $g: V_\R \to \R$ that extends $f$ and is majorised by $p$. Then by \ref{complexRangeExtensionRealFunctional} we can find a complex-linear functional $\tilde{f}:V \to \C$ such that $\Re(\tilde{f}) = g$.

We just need to show that $f$ is bounded by $p$. Take arbitrary $v\in V$ and write $\tilde{f}(v) = |\tilde{f}(v)|e^{i\theta}$ then
\[ |\tilde{f}(v)| = \Re|\tilde{f}(v)| = \Re\Big(e^{-i\theta}\tilde{f}(v)\Big) = \Re\Big(\tilde{f}(e^{-i\theta}v)\Big) = g(e^{-i\theta}v) \leq p(e^{-i\theta}v) = |e^{-i\theta}|p(v) = p(v). \]
\end{proof}
\begin{corollary} \label{}
Let $V$ be a locally convex convergence vector space, $U\subseteq V$ a subspace and $f:U\to \F$ a continuous functional. Then $f$ has a continuous extension to $V$.
\end{corollary}
\begin{proof}
We have that $|f| = p_K$ for some $K\in \vicinity_U(0)$ by \ref{absoluteFunctionalGauge}. 
By continuity of the inclusion map, we can find an $M \in \vicinity_V(0)$ such that $M\cap U = K$. Then $|f|\leq p_M$ and $f$ can be extended by the Hahn-Banach extension theorem to a functional $f'$ defined in the whole of $V$. Then $f'$ is bounded on $M$ and thus continuous by \ref{boundedOnVicinityImpliesContinuous}.
\end{proof}
\begin{corollary}
Let $X$ be a normed space and $Z\subset X$ a subspace. Any bounded linear functional in $\tdual{Z}$ can be extended to a bounded linear functional in $\tdual{X}$ with the same norm.
\end{corollary}
\begin{proof}
Let $f:Z\to \mathbb{F}$ be such a functional. Extend $f$ by the previous theorem, \ref{seminormHahnBanach}, using $p(x) = \norm{f}_Z\norm{x}$.
\end{proof}
\begin{corollary} \label{existenceBoundedFunctionalOfSameNorm}
Let $X$ be a normed space and $x_0\neq 0$ an element of $X$. Then there exists a bounded linear functional $\omega_{x_0}$ such that
\[ \norm{\omega_{x_0}} = 1 \qquad \text{and} \qquad \omega_{x_0}(x_0)=\norm{x_0}. \]
\end{corollary}
\begin{proof}
Extend the functional $f: \Span\{x_0\}\to \mathbb{F}$ defined by
\[ f(x) = f(ax_0) = a\norm{x_0}. \]
\end{proof}
\begin{corollary}
Let $X$ be a normed space. Then $\forall x\in X:$
\[ \norm{x} = \sup_{\substack{f\in X' \\ f\neq 0}}\frac{|f(x)|}{\norm{f}}. \]
\end{corollary}
\begin{proof}
We calculate
\[ \norm{x} \geq \sup_{\substack{f\in X' \\ f\neq 0}}\frac{|f(x)|}{\norm{f}} \geq \frac{|\omega_{x}(x)|}{\norm{\omega_{x}}} = \frac{\norm{x}}{1} = \norm{x} \]
where the first inequality follows from $|f(x)|\leq \norm{f}\norm{x}$ for all $f\in X', x\in X$.
\end{proof}

\subsubsection{Hahn-Banach separation}

\begin{lemma} \label{gaugeSeparationLemma}
Let $V$ be a real vector space, $A$ an absorbent, semibalanced set and $x_0 \notin A$. Consider the functional $f_{x_0}: \Span\{x_0\}\to \F: tx_0 \mapsto t$. Then $f_{x_0}(x)\leq p_A(x)$ for all $x\in \Span\{x_0\}$.
\end{lemma}
\begin{proof}
Let $x = tx_0$. If $t\leq 0$, then the inequality is immediate. Suppose $t>0$. Because $p_A(x_0) \geq 1$ (by the converse of \ref{gaugeLemma}), we have
\[ f_{x_0}(x) = f_{x_0}(tx_0) = t \leq tp_A(x_0) = p_A(tx_0) = p_A(x)  \]
using positive homogeneity (\ref{gaugeScaling}).
\end{proof}

\begin{theorem}[Mazur] \label{MazurTheorem}
Let $V$ be a real or complex convergence vector space and $A$ an open and convex set. If $U$ is a subspace such that $A\perp U$, then there exists a closed hyperplane $H \supseteq U$ such that $A\perp H$.
\end{theorem}
\begin{proof}
First suppose $V$ is a \emph{real} vector space. Because $A$ is open, it is algebraically open. Take $a\in A$. Then $0\in a-A = \inh_{\mathfrak{a}}(a-A)$, so $a-A$ is absorbing by \ref{coreProperties}. It is also semibalanced (by \ref{convexAbsorbentImpliesSemibalanced} as it is convex by \ref{translationScalingConvexSet}).

Then we have
\[ U\perp A \iff 0\notin U-A \iff a \notin a-A+U. \]
Consider the functional $f_{a}$ of \ref{gaugeSeparationLemma}, which is majorised by the gauge $p_{a-A+U}$, which is sublinear by \ref{gaugeProperties}. Then $f_a$ can be extended as an $\R$-linear function to all $V$ by the Hahn-Banach extension theorem \ref{sublinearHahnBanach}.

We note that $U\subseteq \ker(f_a)$, because $p_{a-A+U}(u) = 0$ by \ref{gaugeZeroLemma}.

In order to conclude with \ref{functionalBoundedNeighbourhood}, we need to show that $A-a \subseteq f_a^{\preimf}(\ball(0,|f_a(a)|)) = f_a^{\preimf}(\ball(0,1))$.
Indeed $A-a \subseteq U+A-a = \inh_\mathfrak{a}(U+A-a) \subseteq p_{U+A-a}^\preimf[\ball(0,1)] \subseteq f_{a}^\preimf[\ball(0,1)]$ by \ref{algebraicallyOpen} and \ref{gaugeInherenceAdherence} ($U+A-a$ is semibalanced because $A-a$ is).

Note that $\ker(f_a)^c$ contains the open set $A$ and thus $\ker(f_a)$ is not dense by \ref{openDensityLemma}. By \ref{hyperplaneClosedDense} this means that $\ker(f_a)$ is closed.

Now suppose $V$ is a \emph{complex} vector space. We can consider the realification $V_\R$ with the same convergence, which is a real convergence vector space by TODO \ref{}. So we can use the preceding proof to find a real hyperplane $K$ in $V$. Then \ref{realComplexHyperplane} gives that $H = K\cap iK$ is a complex hyperplane in $V$. Now $H$ and $A$ must be disjoint because $K$ and $A$ are disjoint and $H \subseteq K$.

Also $H$ is closed because $K$ and $iK$ are closed (the first by the preceding proof, the second because multiplication by $i$ is a homeomorphism \ref{continuityLemmaVectorConvergence}) and the intersection of two closed sets is closed.
\end{proof}
\begin{corollary} \label{functionalZeroOnClosedSubSpace}
Let $V$ be a locally convex vector space and $M$ a closed subspace. There exists a non-zero bounded linear functional $f$ on $V$ such that $M\subseteq \ker(f)$.
\end{corollary}
\begin{proof}
The set $M^c$ is open and by local convexity it contains a convex set $C$. We may take $C$ open by replacing it with its interior, which is convex by \ref{inherenceAdherenceConvex}. Now $C\perp M$ and we can apply the theorem. Now $\ker(f)$ is closed, so $f$ is bounded by \ref{continuityLinearFunctionals}.
\end{proof}

\begin{theorem}[Hahn-Banach separation theorem] \label{HahnBanachSeparation}
Let $V$ be a convergence vector space. Suppose $A,B$ are disjoint, non-empty, convex sets and that $A$ is open. Then there exists a continuous linear functional $f:V\to \F$ such that $f^\imf[A]$ and $f^\imf[B]$ are disjoint.
\end{theorem}
\begin{proof}
The set $A-B = \bigcup_{b\in B}A-b$ is convex and a union of open sets and thus open by \ref{completeClosureTopology}.
The set $A-B$ and the vector space $\{0\}$ are disjoint, so by \ref{MazurTheorem} we can find a closed hyperplane that is disjoint with $A-B$.

By \ref{kernelHyperplane} and \ref{continuityLinearFunctionals} this is the kernel of a continuous linear functional $f$.
\end{proof}
\begin{corollary}
Let $V$ be a real or complex convergence vector space and $A,B$ as in the proposition. Then there exists a continuous linear functional $f:V\to \F$ and $t\in \R$ such that
\[ \Re f(a) < t \leq \Re f(b) \]
for all $a\in A$ and $b\in B$.
\end{corollary}
This means $A$ and $B$ are separated by a closed affine hyperplane $\ker(f)+v$, where $v \in f^\preimf[\{t\}]$.

We can reverse the inequalities by replacing $f$ by $-f$.
\begin{proof}
Apply the proposition to the realification $V_\R$. This gives us an $\R$-linear functional $g: V\to \R$ such that $g^\imf[A]$ and $g^\imf[B]$ are disjoint convex sets. Additionally $g^{\imf}[A]$ is open in $\R$ by \ref{linearFunctionalOpen}.

Because $g^\imf[A]$ and $g^\imf[B]$ are convex, we either have $g^\imf[A]\leq g^\imf[B]$ or $g^\imf[A]\geq g^\imf[B]$. In the second case we simply replace $g$ by $-g$ to obtain the first case. We may take $t= \sup g^\imf[A]$. This is not in $g^\imf[A]$ because it is open.

If $V$ is a real vector space we take $f=g$ are done. If $V$ is complex, we can find a suitable $f$ by \ref{complexRangeExtensionRealFunctional}.
\end{proof}
\begin{corollary} \label{locallyConvexHahnBanachSeparationClosedSet}
Let $\sSet{V, \xi}$ be a locally convex vector convergence space. Let $B$ be a closed convex set and $v\notin B$, then there exists a continuous linear functional $f:V\to \F$ such that $f(v) \notin \overline{f^\imf[B]}$.
\end{corollary}
\begin{proof}
As $B^c$ is open, it is a neighbourhood of $v$ and we can find an convex neighbourhood $U$ of $v$ in $B^c$. Now replace $U$ by its interior. This is open and still convex by \ref{inherenceAdherenceConvex}, so we can apply the theorem.

Then $f^\imf[U]$ is open by \ref{linearFunctionalOpen} and thus in $\neighbourhood_\F\big(f(v)\big)$. It is however disjoint from $f^\imf[B]$, which means that $f(v) \notin \overline{f^\imf[B]}$.
\end{proof}
\begin{corollary}
Let $V$ be a locally convex TVS. Suppose $A,B$ are disjoint, non-empty, convex sets and that $A$ is compact, $B$ is closed. Then there exists a continuous linear functional $f:V\to \F$ and $s,t\in \R$ such that
\[ \Re f(a) < t < s < \Re f(b) \]
for all $a\in A$ and $b\in B$.
\end{corollary}
\begin{proof}
TODO
\end{proof}
\begin{corollary} \label{locallyConvexDualPair}
Let $V$ be a Hausdorff locally convex convergence vector space and $v\in V$. If $f(v) = 0$ for all $f\in V^*$, then $v = 0$.
\end{corollary}
\begin{proof}
We prove the contrapositive. Assume $v\neq 0$. By \ref{FrechetCharacterisation}, we have that $\{0\}$ is closed, so $\{0\}^c$ is a neighbourhood of $v$. By local convexity, $\{0\}^c$ contains a convex vicinity of $x$. We may take this vicinity to be open by taking the interior (which is still convex by \ref{inherenceAdherenceConvex}). Let this open vicinity be $A$ and set $B = \{0\}$. By Hahn-Banach separation, there exists $f\in \dual{V}$ such that $f^\imf[A]$ and $f^\imf[B] = \{0\}$ are disjoint. Thus $f(v) \neq 0$.
\end{proof}

\subsubsection{Banach limits}
\begin{proposition}
There exists a linear map $L:l^\infty(\N) \to \C$ satisfying
\begin{enumerate}
\item $\displaystyle L(x) = \lim_{n\to \infty}x_n$ if the limit exists;
\item $L((x_{n+1})_{n\in\N}) = L((x_n)_{n\in\N})$;
\item if $\forall n\in\N:x_n\geq 0$, then $L(x) \geq 0$;
\item $\norm{L} = 1$.
\end{enumerate}
Such a linear map is called a \udef{Banach limit}.
\end{proposition}
\begin{proof}
TODO, after Cesàro means.
\end{proof}

\subsection{Continuous functionals}

TODO???
\begin{proposition}
Let $V$ and $W$ be TVSs and $f: V\to W$ a linear function.
\begin{enumerate}
\item If $f$ is continuous and $W$ is Hausdorff, then $\ker(f)$ is closed.
\item If $f$ has closed kernel and finite-dimensional image, then $f$ is continuous.
\end{enumerate}
\end{proposition}
\begin{proof}
(1) Because $W$ is Hausdorff, it is also $T_1$ and thus $\{0\}$ is closed by \ref{FrechetCharacterisation}. Then $\ker(f) = f^{\preimf}(\{0\})$ is closed by \ref{continuity}.

(2) 
\end{proof}
??



\section{General duality theory}
\subsection{Paired spaces}
\begin{definition}
A \udef{pairing} is a triple $\sSet{V,W, \pair{\cdot,\cdot}}$ where $V,W$ are vector spaces over $\mathbb{F}$ and $\pair{\cdot,\cdot}: V\times W\to \mathbb{F}$ is a bilinear form. Often we will write the pairing as just $\sSet{V,W}$.

We say $W$
\begin{itemize}
\item \udef{distinguishes} points of $V$; or
\item is \udef{separating} on $V$; or
\item \udef{separates} $V$
\end{itemize}
if 
\[ \forall v\in V\setminus\{0\}: \exists w\in W: \pair{v,w} \neq 0. \]

A \udef{dual system}, \udef{dual pair} or \udef{duality} over a field $\mathbb{F}$ is a pairing $\sSet{V,W,\pair{\cdot,\cdot}}$ such that $V$ distinguishes points of $W$ and $W$ distinguishes points of $V$.
\end{definition}
We have that $W$ is separating on $V$ \textup{if and only if}

\begin{lemma} \label{dualSystemInjective}
Let $\sSet{V,W, \pair{\cdot,\cdot}}$ be a pairing. Then $W$ separates $V$ \textup{if and only if} $v\mapsto \pair{v, \cdot}$ is injective.
\end{lemma}
\begin{proof}
Suppose $W$ separates $V$ and take arbitrary $v,v'\in V$ such that $\pair{v, \cdot} = \pair{v', \cdot}$. Then $\pair{v-v',\cdot} = \underline{0}$, which by assumption means $v-v' = 0$, or $v= v'$.

Now suppose $v\mapsto \pair{v, \cdot}$ is injective and take arbitrary $v\in V\setminus\{0\}$. If $\pair{v,w} = 0$ for all $w\in W$, then $\pair{v,\cdot} = \pair{0,\cdot}$ and thus $v=0$ by injectivity. This was disallowed by assumption.
\end{proof}

\begin{example}
\begin{itemize}
\item Let $\sSet{V, \xi}$ be a convergence vector space. Then $\sSet{\dual{V}, V, \evalMap(\cdot,\cdot)}$ is a pairing.
\item Let $\sSet{V,\xi}$ be a Hausdorff locally convex convergence vector space. Then $\sSet{\dual{V}, V, \evalMap(\cdot,\cdot)}$ is a dual pair by \ref{locallyConvexDualPair}.
\end{itemize}
\end{example}

\begin{lemma}
Let $\sSet{V,W,\pair{\cdot,\cdot}}$ be a dual pair. Then $\sSet{W,V,\pair{\cdot,\cdot}^d}$ is also a dual pair.
\end{lemma}

\subsection{The weak topology}
\begin{definition}
Let $(X,Y,\pair{\cdot,\cdot})$ be paired vector spaces. 
The initial convergence on $Y$ w.r.t. the set of linear functionals $\setbuilder{\pair{x,\cdot}}{x\in X}$ is called the \udef{weak topology} $\sigma(X,Y)$ on $Y$ for the pair $\sSet{X,Y}$\footnote{What we denote $\sigma(X,Y)$ is usually denoted $\sigma(Y,X)$.}.
\end{definition}

\begin{lemma} \label{weakTopologyLCTVS}
Let $\sSet{X,Y,\pair{\cdot,\cdot}}$ be a pairing. The weak topology $\sigma(X,Y)$ on $Y$ 
\begin{enumerate}
\item is the same as the initial convergence w.r.t. the set of seminorms $\setbuilder{\abspair{x,\cdot}}{x\in X}$;
\item is a locally convex vector space topology;
\item has the neighbourhood filter
\begin{align*}
\neighbourhood_{\sigma(X,Y)}(0) &= \mathfrak{F}\setbuilder{y\in Y}{\exists x\in X: \abspair{x,y} \leq 1} \\
&= \mathfrak{F}(_\pol{X}).
\end{align*}
\end{enumerate}
\end{lemma}
\begin{proof}
(1, 2) Let $\sigma_1$ be the initial convergence w.r.t. $\setbuilder{\pair{x,\cdot}}{x\in X}$ and $\sigma_2$ the initial convergence w.r.t. $\setbuilder{\abspair{x,\cdot}}{x\in X}$.

By \ref{continuityAbsFunctional}, all functionals of the form $\abspair{x,\cdot}: \sSet{V,\sigma_1} \to \R$, for some $x\in X$, are continuous. Thus $\sigma_1 \subseteq \sigma_2$.

By \ref{locallyConvexSeminormTopology}, $\sigma_2$ is a locally convex vector space topology. Thus all functionals of the form $\pair{x,\cdot}: \sSet{V,\sigma_2} \to \R$, for some $x\in X$, are continuous. This means that $\sigma_2 \subseteq \sigma_1$.

(3) The form of the neighbourhood filter also follows from \ref{locallyConvexSeminormTopology}. It is enough to show that $\setbuilder{\abspair{x,\cdot}^{\preimf}(\,[0,\epsilon]\,)}{x\in X, \epsilon > 0} = \setbuilder{y\in Y}{\exists x\in X: \abspair{x,y} \leq 1}$. Then
\begin{align*}
y \in \setbuilder{\abspair{x,\cdot}^{\preimf}(\,[0,\epsilon]\,)}{x\in X, \epsilon > 0} &\iff \exists x\in X: \exists \epsilon > 0: \; \abspair{x,y} \leq \epsilon \\
&\iff \exists x\in X: \exists \epsilon > 0: \; \epsilon^{-1}\abspair{x,y} \leq 1 \\
&\iff \exists x\in X: \exists \epsilon > 0: \; \abspair{\epsilon^{-1}x,y} \leq 1 \\
&\iff \exists x\in X: \; \abspair{x,y} \leq 1 \\
&\iff y\in \setbuilder{y\in Y}{\exists x\in X: \abspair{x,y} \leq 1}.
\end{align*}
\end{proof}

\begin{lemma}
Let $(X,Y,\pair{\cdot,\cdot})$ be a pairing and $x\in X$. Then $\abspair{x,\cdot} = p_{^\pol\{x\}}$.
\end{lemma}

\begin{proposition} \label{functionalContinuityWeakTopology}
Let $\sSet{X,Y,\pair{\cdot,\cdot}}$ be a pairing and $f: Y\to \F$ a linear functional. Then
\[ \dual{(Y,\sigma(X,Y))} = \setbuilder{\pair{x,\cdot}}{x\in X}. \]
\end{proposition}
\begin{proof}
This inclusion $\supseteq$ is immediate by the definition of initial topology.

Now take $f\in \dual{(Y,\sigma(X,Y))}$. By point (3) of \ref{initialSeminormConvergence} there exists a finite set $\{x_1, \ldots, x_n\}$ such that $|f(v)| \leq C\max_{i}\abspair{x_i, \cdot}$ for all $v\in V$. By \ref{linearDependenceLinearFunctionals}, this means that $f$ is some linear combination of the $\pair{x_i,\cdot}$, say $\sum_{i=0}^n \alpha_i \pair{x_i, \cdot}$. So
\[ f = \sum_{i=0}^n \alpha_i \pair{x_i, \cdot} = \pair{\sum_{i=0}^n \alpha_i x_i, \cdot}. \]
\end{proof}
\begin{corollary} \label{dualSystemBijection}
Let $\sSet{X,Y,\pair{\cdot,\cdot}}$ be a pairing such that $Y$ separates $X$. Then $X\to Y^*: x\mapsto \pair{x,\cdot}$ is a bijection.
\end{corollary}
In particular this holds if $\sSet{X,Y,\pair{\cdot,\cdot}}$ is a dual system.
\begin{proof}
Injectivity is given by \ref{dualSystemInjective}. Surjectivity by the proposition.
\end{proof}
\begin{corollary}
Let $V$ be a convergence vector space. Then $V^* = \sSet{V, \sigma(V^*, V)}^*$.

The weak topology is the weakest convergence $\tau$ such that $V^* = \sSet{V, \tau}^*$.
\end{corollary}
Thus for any convergence vector space $\sSet{V,\xi}$, the space $\sSet{V, \sigma(V^*, V)}$ is a locally convex TVS with the same continuous functionals.

\subsubsection{Weak continuity}
\begin{definition}
Let $T: \sSet{V,\xi} \to \sSet{W,\zeta}$ be a linear operator between convergence vector spaces. Then $T$ is called \udef{weakly continuous} if
\[ T: \sSet{V, \sigma(V^*,V)} \to \sSet{W, \sigma(W^*,W)} \]
is continuous.
\end{definition}

\begin{lemma}
If $T: V\to W$ is continuous, then it is weakly continuous.
\end{lemma}
\begin{proof}
By the characteristic property of the initial convergence \ref{characteristicPropertyInitialFinalConvergence}, the weak continuity of $T$ is equivalent to the continuity of $f\circ T$ for all $f\in W^*$. This holds because the composition of continuous functions is continuous, \ref{continuityComposition}.
\end{proof}


\subsubsection{The pair $(V^*, V)$}
\begin{definition}
Let $V$ be a vector space and $V^*$ the continuous dual under some convergence. Any convergence on $V$ such that the continuous dual is still $V^*$ is called a \udef{convergence of the dual pair}.
\end{definition}
Any property that only depends on continuous linear functionals is the same for any convergence of the dual pair.

The weak topology is the weakest convergence of the dual pair.

\begin{proposition}
Let $V$ be a vector space and $A\subseteq V$ a convex subset. Let $\xi$ be a locally convex vector space convergence on $V$. Then $\closure_\xi(A) = \closure_{\sigma(V^*, V)}(A)$.
\end{proposition}
In other words the closure of convex sets is the same for all locally convex convergences of the dual pair.
\begin{proof}
From \ref{principalInherenceAdherenceProperties}, we have $\closure_\xi(A) \subseteq \closure_{\sigma(V^*, V)}(A)$.

Now assume $x\notin \closure_\xi(A)$. Then, by \ref{locallyConvexHahnBanachSeparationClosedSet}, we can find a continuous functional $f\in V^*$ such that $f(x) \notin \overline{f^{\imf}\big(\closure_\xi(A)\big)}$. So not every neighbourhood of $f(x)$ meshes with $f^{\imf}\big(\closure_\xi(A)\big)$; take some neighbourhood $U$ that is disjoint from it. By continuity and \ref{continuityVicinityFilter}, we have that $f^{\preimf}(U)$ is a $\sigma(V^*, V)$-neighbourhood of $x$ that must be disjoint from $\closure_\xi(A)$ (we have used that $\sigma(V^*, V)$ is topological, \ref{weakTopologyLCTVS}). This implies that $A$ is also disjoint from $f^{\preimf}(U)$.

Thus $x\notin \closure_{\sigma(V^*, V)}(A)$ by \ref{interiorClosureMembership}.
\end{proof}


\subsubsection{Weak-$*$ topology}
\begin{definition}
Let $(X,Y,\pair{\cdot,\cdot})$ be paired vector spaces. The \udef{weak-$*$ topology} $\sigma^*(X,Y)$ on $X$ for this pairing is the weak topology $\sigma(Y,X)$ for the pairing $(Y,X,\pair{\cdot,\cdot}^d)$.
\end{definition}

\begin{proposition} \label{weak*continuousFunctional}
Let $X$ be a Banach space and let $X'$ have the weak-$*$ topology. Then a linear functional $\theta: X'\to \C$ is continuous \textup{if and only if}
\[ \exists x\in X: \forall \omega\in X': \quad \theta(\omega) = \omega(x). \]
\end{proposition}
\begin{proof}
TODO 9.2 in lecture notes.
\end{proof}

\subsection{Polars}
\subsubsection{Polar sets}
\begin{definition}
Let $\sSet{X,Y,\pair{\cdot,\cdot}}$ be a pairing and $B\subseteq Y$ a subset. The \udef{polar} of $B$ is the polar w.r.t. the relation $\pol$ on $(Y,X)$ defined by
\[ y\pol x \qquad\iff\qquad \abspair{x, y} \leq 1. \]
Conventionally, we also use $\pol$ to denote $\pol^\transp$. Thus the \udef{bipolar} $B^{\pol\pol}$ of $B$ is $B^{\pol\pol^\transp}$.
\end{definition}

\begin{lemma} \label{polarLemma}
Let $\sSet{X,Y,\pair{\cdot,\cdot}}$ be a pairing and $B\subseteq Y$ a subset. Then
\begin{align*}
B^{\pol} &= \setbuilder{x\in X}{\sup_{y\in B}\abspair{x,y} \leq 1} \\
&= \bigcap_{y\in B}\setbuilder{x\in X}{\abspair{x,y}\leq 1}.
\end{align*}
and for the gauge of $B^\pol$, we have $p_{B^\pol}(x) = \sup_{y\in B}\abspair{x,y}$.
\end{lemma}
\begin{proof}
The expression for the polar is straightforward. For the gauge we calculate
\begin{align*}
p_{B^\pol}(x) &= \inf \setbuilder{\lambda\in\overline{\R^+}}{x\in \lambda B^\pol} \\
&= \inf \setbuilder{\lambda\in\overline{\R^+}}{\lambda^{-1}x\in B^\pol} \\
&= \inf \setbuilder{\lambda\in\overline{\R^+}}{\sup_{y\in B}\abspair{\lambda^{-1}x,y}\leq 1} \\
&= \inf \setbuilder{\lambda\in\overline{\R^+}}{\sup_{y\in B}\abspair{x,y}\leq \lambda} \\
&= \sup_{y\in B}\abspair{x,y}.
\end{align*}
\end{proof}

\begin{lemma} \label{polarPropertiesLemma}
Let $\sSet{X,Y,\pair{\cdot,\cdot}}$ be a pairing and $B\subseteq Y$ a subset. Then
\begin{enumerate}
\item for all $\lambda \neq 0$: $(\lambda B)^{\pol} = \lambda^{-1}B^{\pol}$;
\item $B^{\pol}$ is absolutely convex;
\item $B^{\pol}$ is $\sigma^*(X,Y)$-closed.
\end{enumerate}
\end{lemma}
\begin{proof}
(1) We calculate
\begin{align*}
x\in (\lambda B)^{\pol} &\iff \forall y\in B:\; \abspair{x,\lambda y} \leq 1 \\
&\iff \forall y\in B:\; \abspair{\lambda x,y} \leq 1 \\
&\iff \lambda x\in B^\pol \\
&\iff x\in \lambda^{-1}B^\pol.
\end{align*}

(2) We use \ref{absolutelyConvexCriteria}, so we take $x,y\in B^\pol$ and $|r|\leq 1$. We need to show that $rx + (1-r)y\in B^\pol$. Indeed, for arbitrary $z\in B$ we have
\begin{align*}
\abspair{rx + (1-r)y, z} &= |r\pair{x,z} + (1-r)\pair{y,z}| \\
&\leq |r|\;\abspair{x,z} + |1-r|\;\abspair{y,z} \\
&\leq |r| + |1-r| \\
&\leq 1,
\end{align*}
where the last inequality follows from $1 = |1 - r + r| \leq |1-r| + |r|$. Since $z\in B$ was chosen arbitrarily, we have $rx + (1-r)y\in B^\pol$.

(3) We have, using \ref{polarOfUnion},
\begin{align*}
B^\pol &= \left(\bigcup_{y\in B}\{y\}\right)^\pol \\
&= \bigcap_{y\in B} \{y\}^\pol \\
&= \bigcap_{y\in B}\setbuilder{x\in X}{\abspair{x,y}\leq 1} \\
&= \bigcap{y\in B}\abspair{\cdot, y}^{\preimf}[\cball(0,1)],
\end{align*}
which is an intersection of closed sets (as each $\abspair{\cdot, y}$ is continuous in the $\sigma(X,Y)$ topology) and thus closed.
\end{proof}

\begin{proposition}[Bipolar theorem] \label{bipolarTheorem}
Let $\sSet{X,Y,\pair{\cdot,\cdot}}$ be a pairing and $B\subseteq Y$ a subset. Then
\[ B^{\pol\pol} = \overline{\disked(B)}^{\sigma(X,Y)}. \]
\end{proposition}
\begin{proof}
We have $B\subseteq B^{\pol\pol}$ by \ref{reflexiveGaloisCorollary}. Then $\disked(B) \subseteq B^{\pol\pol}$ because $B^{\pol\pol}$ is absolutely convex by \ref{polarPropertiesLemma}. Similarly $\overline{\disked(B)}^{\sigma(X,Y)} \subseteq B^{\pol\pol}$, because $B^{\pol\pol}$ is $\sigma(X,Y)$-closed.

The other inclusion is proved by contradiction. Assume, to this end, that $y_0\in B^{\pol\pol} \setminus \overline{\disked(B)}^{\sigma(X,Y)}$. Then we can apply Hahn-Banach separation \ref{locallyConvexHahnBanachSeparationClosedSet} to obtain a continuous functional $f$ such that $f^\imf\left[\overline{\disked(B)}\right]$ is disjoint from some open neighbourhood $U$ of $f(y_0)$.

Now $\overline{\disked(B)}$ is absolutely convex by \ref{inherenceAdherenceBalanced} and \ref{inherenceAdherenceConvex}, so $f^\imf\left[\overline{\disked(B)}\right]$ is also absolutely convex by linearity. Then, because $|U|$ is open, there exists $t\in |U|$ such that $|f(b)| < t < |f(y_0)|$ for all $b\in \overline{\disked(B)}$. Thus $\sup_{b\in \overline{\disked(B)}}|f(b)| < |f(y_0)|$. By rescaling $f$ we can take $\sup_{b\in \overline{\disked(B)}}|f(b)| \leq 1 < |f(y_0)|$.

By \ref{functionalContinuityWeakTopology}, we can find some $x\in X$ such that $f = \pair{x,\cdot}$. Then $\sup_{b\in \overline{\disked(B)}}\abspair{x, b} \leq 1$ implies $x\in \overline{\disked(B)}^\pol \subseteq B^\pol$. Finally $1 < \abspair{x, y_0}$ implies $y_0\notin B^{\pol\pol}$. This is a contradiction.
\end{proof}

\subsubsection{Weak-$*$-compactness}
\begin{theorem}[Banach/Bourbaki-Alaoglu]
Let $(X,Y,\pair{\cdot,\cdot})$ be paired vector spaces and $A\subseteq Y$ a neighbourhood of the origin in the weak topology. Then $A^\pol$ is $\sigma^*(X,Y)$-compact.
\end{theorem}
\begin{proof}

\end{proof}

\subsubsection{Orthogonal complements}
\begin{definition}
Let $\sSet{X,Y,\pair{\cdot,\cdot}}$ be a pairing and $B\subseteq Y$ a subset. The \udef{orthogonal complement} of $B$ is the polar w.r.t. the relation $\perp$ on $(Y,X)$ defined by
\[ y\perp x \qquad\iff\qquad \pair{x, y} = 0. \]
\end{definition}

\begin{proposition} \label{perpAsPolar}
Let $\sSet{X,Y,\pair{\cdot,\cdot}}$ be a pairing and $B\subseteq Y$ a subset. Then $B^\perp = \Span(B)^\pol$.
\end{proposition}
\begin{proof}
We show both inclusions. First assume $x\in B^\perp$. We need to show that $\abspair{x,y} \leq 1$ for all $y\in \Span(B)$. Indeed we can write $y = \sum_{j=1}^n c_jy_j$ for some $c_j\in\F$ and $y_j\in B$. Then
\[ \abspair{x,y} = \abspair{x, \sum_{j=1}^n c_jy_j} \leq \sum_{j=1}^n |c_j|\;\abspair{x, y_j} = 0 \leq 1, \]
as $y_j\perp x$.

Now assume $x\in \Span(B)^\pol$. If $\abspair{x,y} = 0$ for all $y\in B$, then $x\in B^\perp$ and we are done. Now assume, towards a contradiction, that this is not the case. Then there exists $y\in B$ such that $\abspair{x,y} = \epsilon \neq 1$. Then $\abspair{x,2\epsilon^{-1}y} = 2$, but $2\epsilon^{-1}y\in\Span(B)$, so $\abspair{x,2\epsilon^{-1}y} \leq 1$. This is a contradiction.
\end{proof}
\begin{corollary} \label{corollaryPerpAsPolar}
Let $\sSet{X,Y,\pair{\cdot,\cdot}}$ be a pairing and $B\subseteq Y$ a subset. Then
\begin{enumerate}
\item $B^\perp$ is a subspace;
\item $B^\perp = \Span(B)^\perp = \Span(B^\perp)$;
\item $\{0\}^\perp = X$;
\item $Y$ separates $X$ \textup{if and only if} $Y^\perp = \{0\}$;
\item $B^\perp$ is $\sigma^*(X,Y)$-closed.
\item $B^{\perp\perp} = \overline{\Span(B)}^{\sigma(X,Y)}$;
\item if $\Span(B)$ is $\sigma(X,Y)$-dense in $Y$ and $Y$ separates $X$, then $B^\perp = \{0\}$.
\end{enumerate}
\end{corollary}
\begin{proof}
(1) As $\Span(B)^\pol$ is convex by \ref{polarPropertiesLemma}, we just need to show it is closed under multiplication by $2$, \ref{convexSubspace}. Now $2\cdot \Span(B)^\pol = (2^{-1}\cdot\Span(B))^\pol = \Span(B)^\pol$ by \ref{polarPropertiesLemma}.

(2) We have $B^\perp = \Span(B)^\pol = \Span\big(\Span(B)\big)^\pol = \Span(B)^\perp$. The second equality follows straight from (1).

(3) For all $y\in Y$ we have $y\perp 0$.

(4) We have $0\in Y$, as for all $y\in Y$ we have $y\perp 0$. For all $x\in X\setminus\{0\}$, the following are equivalent: $x\notin Y^\perp$ and $\exists y\in Y: \pair{x,y}\neq 0$. Thus $Y$ being separating on $X$ is equivalent to $\forall x\in X\setminus \{0\}: x\notin Y^\perp$, which is equivalent to $Y^\perp \subseteq \{0\}$.

(5) By \ref{polarPropertiesLemma}.

(6) We have 
\[ B^{\perp\perp} = \Span(B^\perp)^\pol = \big(B^\perp\big)^\pol = \Span(B)^{\pol\pol} = \overline{\disked(\Span(B))}^{\sigma(X,Y)} = \overline{\Span(B)}^{\sigma(X,Y)}, \]
using the bipolar theorem \ref{bipolarTheorem}.

(7) In this case we have $B^\perp = B^{\perp\perp\perp} = \left(\overline{\Span(B)}^{\sigma(X,Y)}\right)^\perp = Y^\perp = \{0\}$.
\end{proof}


\begin{proposition}
Let $\sSet{X,Y,\pair{\cdot,\cdot}}$ be a pairing and $W_1,W_2$ subspaces of $Y$. Then
\[ (W_1+W_2)^\perp = W_1^\perp \cap W_2^\perp. \]
\end{proposition}
\begin{proof}
For a vector $v\in X$,
\begin{align*}
v\in (W_1+W_2)^\perp = (W_1 \cup W_2)^\perp &\implies \forall x\in W_1\cup W_2: \inner{v,x} = 0 \\
&\implies v\in W_1^\perp \cap W_2^\perp
\end{align*}
and
\begin{align*}
v\in W_1^\perp \cap W_2^\perp &\implies \forall x\in W_1, y\in W_2: \inner{v,x} = 0 = \inner{v,y} \\
&\implies \forall x\in W_1, y\in W_2:\inner{v, x+y} = 0 \implies v\in (W_1+W_2)^\perp.
\end{align*}
\end{proof}
TODO: dual result for closed subspaces.



\subsubsection{Annihilator subspaces}
\begin{lemma}
Let $\sSet{X,Y,\pair{\cdot,\cdot}}$ be a pairing and $V\subseteq Y$ a subspace. Then
\[ V^{\pol} = \setbuilder{x\in X}{\forall y\in V:\;\pair{x,y} = 0}. \]
\end{lemma}

\subsubsection{Polar topologies}

\begin{lemma} \label{weaklyBoundedLemma}
Let $\sSet{X,Y,\pair{\cdot,\cdot}}$ be a pairing and $A\subseteq Y$ a subset. The following are equivalent:
\begin{enumerate}
\item $A$ is weakly bounded;
\item $\sup_{y\in A}\abspair{x, y}$ is finite for all $x\in X$;
\item $A^\pol$ is an absorbent subset of $X$.
\end{enumerate}
\end{lemma}
\begin{proof}
$(1) \Leftrightarrow (2)$ Immediate from \ref{boundedSetsInitialTopology} and \ref{weakTopologyLCTVS}, noting that
\[ \sup_{y\in A}\abspair{x, y} = \sup\Big(\abspair{x,-}^{\imf}(A)\Big). \]

$(2) \Leftrightarrow (3)$ By \ref{polarLemma}, point (2) is equivalent to the finiteness of $p_{A^\pol}(x)$ for all $x\in X$. By \ref{gaugeWellDefined} (and the fact that $A^\pol$ is balanced, \ref{polarPropertiesLemma}) we have that this is equivalent to the absorbence of $A^\pol$.
\end{proof}

\begin{proposition}
Let $\sSet{X,Y,\pair{\cdot,\cdot}}$ be a pairing and $\mathcal{A}\subseteq\powerset(Y)$ a set of weakly bounded sets. Then the filter
\[ N = \mathfrak{F}\setbuilder{\lambda A^\pol}{\lambda\in \F, A\in \mathcal{A}} \]
is the neighbourhood filter of $0$ in a locally convex topology on $X$.
\end{proposition}
\begin{proof}
We use \ref{TVSbase} to verify that this filter is the neighbourhood filter of a topological vector space. Indeed, the sets $\lambda A^\pol$ are absorbent by \ref{boundedSetLemma} and \ref{weaklyBoundedLemma}. They are balanced by \ref{polarPropertiesLemma}. Finaly by convexity (\ref{polarPropertiesLemma}), we have $\frac{1}{2}\lambda A^\pol + \frac{1}{2} \lambda A^\pol \subseteq \lambda A^\pol$.

As noted, the basis sets are convex, so the topology is locally convex.
\end{proof}

\begin{definition}
Let $\sSet{X,Y,\pair{\cdot,\cdot}}$ be a pairing and $\mathcal{A}\subseteq\powerset(Y)$ a set of weakly bounded sets. The topology $\xi$ on $X$ with neighbourhood filter
\[ \neighbourhood_\xi(0) \defeq \mathfrak{F}\setbuilder{\lambda A^\pol}{\lambda\in \F, A\in \mathcal{A}} \]
is called the \udef{polar topology} determined by $\mathcal{A}$.
\end{definition}

\begin{proposition}
Let $\sSet{X,Y,\pair{\cdot,\cdot}}$ be a dual system and  $\mathcal{A}\subseteq\powerset(Y)$ a set of weakly bounded sets. Then the polar topology determined by $\mathcal{A}$ is Hausdorff \textup{if and only if} $\Span\Big(\bigcup \mathcal{A}\Big)$ is $\sigma(X,Y)$-dense in $Y$.
\end{proposition}
\begin{proof}

\end{proof}

\subsection{Adjoints}
\begin{definition}
Let $\sSet{X,Y,\pair{\cdot,\cdot}}$ and $\sSet{X',Y',\pair{\cdot,\cdot}'}$ be two pairings and $T: Y\to Y'$ a linear function. An \udef{adjoint} or \udef{transpose} is a linear operator $S: X'\to X$ such that
\[ \pair{x, Ty}' = \pair{Sx, y} \qquad \forall x\in X', \; y\in Y. \]
\end{definition}

\begin{proposition}
Let $\sSet{X,Y,\pair{\cdot,\cdot}}$ and $\sSet{X',Y',\pair{\cdot,\cdot}'}$ be two pairings, $T: Y\to Y'$ a linear function. Then $T$ has an adjoint $S: X'\to X$ \textup{if and only if} it is weakly continuous, i.e.\ continuous as a function
\[ T: \sSet{Y, \sigma(X,Y)} \to \sSet{Y', \sigma(X',Y')}. \]
The adjoint is unique if $Y$ separates $X$.
\end{proposition}
\begin{proof}
The weak continuity of $T$ is, by the characteristic property of the initial convergence \ref{characteristicPropertyInitialFinalConvergence}, equivalent to the continuity of
\[ \pair{x', \cdot}' \circ T = \pair{x', T(\cdot)}' : \sSet{Y, \sigma(X,Y)} \to \F \]
for all $x'\in X'$. In other words, $\pair{x', T(\cdot)}' \in \dual{\sSet{Y, \sigma(X,Y)}}$ for all $x'\in X'$. By \ref{functionalContinuityWeakTopology}, this is equivalent to the existence of some $x\in X$ such that $\pair{x', T(\cdot)}' = \pair{x,\cdot}$. We set $S(x') = x$.

Thus the weak continuity of $T$ is equivalent to the existence of some adjoint function $S$, without the requirement that it be linear. We now just need to show that $S$ can always be taken to be linear.

Indeed, pick some Hamel basis $\mathcal{X}$ of $X'$ and define $S'$ by setting $S'|_{\mathcal{X}} = S|_{\mathcal{X}}$ and extending linearly to the whole of $X'$. It is clear that $S'$ is an adjoint: set $x' = \sum_{x\in\mathcal{X}}\lambda_x x$ (with only finitely many $\lambda_x$ non-zero) and calculate
\begin{multline*}
\pair{x', T(\cdot)}' = \pair{\sum_{x\in\mathcal{X}}\lambda_x x, T(\cdot)}' = \sum_{x\in\mathcal{X}}\lambda_x \pair{x, T(\cdot)} = \sum_{x\in\mathcal{X}}\lambda_x \pair{S(x), \cdot} = \\ \sum_{x\in\mathcal{X}}\lambda_x \pair{S'(x), \cdot} = \pair{S'\left(\sum_{x\in\mathcal{X}}\lambda_x x\right), \cdot} = \pair{S'(x'), \cdot}.
\end{multline*}

Finally we note that the choice of $S(x')$ is unique if $Y$ separates $X$ by \ref{dualSystemBijection}.
\end{proof}
\begin{corollary}
The adjoint of a weakly continuous operator is weak-$*$ continuous.
\end{corollary}
\begin{proof}
Suppose $S$ is an adjoint of $T$. By symmetry of the definition, $T$ is an adjoint of $S$ when considering the dual pairings. Thus $S$ is weak-$*$ continuous by the proposition.
\end{proof}

\begin{lemma}
Let $\sSet{X,Y,\pair{\cdot,\cdot}}$, $\sSet{X',Y',\pair{\cdot,\cdot}'}$ be pairings and $T: Y\to Y'$ a weakly continuous linear function with adjoint $S: X'\to X$. Then for all subsets $A\subseteq Y$, we have
\[ \big(T^{\imf}(A)\big)^\pol = S^{\preimf}(A^\pol). \]
\end{lemma}
\begin{proof}
We calculate, using \ref{polarLemma},
\begin{align*}
S^{\preimf}(A^\pol) &= S^{\preimf}\Big(\bigcap_{y\in A}\setbuilder{x\in X}{\abspair{x,y}\leq 1}\Big) \\
&= \bigcap_{y\in A}S^{\preimf}\Big(\setbuilder{x\in X}{\abspair{x,y}\leq 1}\Big) \\
&= \bigcap_{y\in A}\Big(\setbuilder{x'\in X'}{\abspair{S(x'),y}\leq 1}\Big) \\
&= \bigcap_{y\in A}\Big(\setbuilder{x'\in X'}{\abspair{x',T(y)}\leq 1}\Big) \\
&= \bigcap_{y'\in T^{\imf}(A)}\Big(\setbuilder{x'\in X'}{\abspair{x',y'}\leq 1}\Big) \\
&= \big(T^{\imf}(A)\big)^\pol.
\end{align*}
\end{proof}

\subsection{Adjoints TODO}
\begin{definition}
Let $\sSet{X,Y,\pair{\cdot,\cdot}}$ and $\sSet{X',Y',\pair{\cdot,\cdot}'}$ be two pairings and $T: Y\not\to Y'$ a linear operator. An \udef{adjoint} or \udef{transpose} is a linear operator $S: X'\not\to X$ such that
\[ \pair{x, Ty}' = \pair{Sx, y} \qquad \forall x\dom(S), \; y\in\dom(T). \]
\end{definition}

\begin{lemma}
Let $\sSet{X,Y,\pair{\cdot,\cdot}}$ and $\sSet{X',Y',\pair{\cdot,\cdot}'}$ be two pairings and $T: Y\not\to Y'$ a linear operator.

Let $S_1, S_2: X'\not\to X$ be adjoints of $T$ then for all $x\in \dom(S_1)\cap\dom(S_2)$ we have $S_1(x) - S_2(x) \in \dom(T)^\perp$.

Conversely, let $S$ be an adjoint of $T$ and $x\in\dom(S)$. Then for all $v\in \dom(T)^\perp$ there exists an adjoint $S'$ such that $S'(x) = S(x) + v$.
\end{lemma}
\begin{proof}
For all $u\in \dom(T)$ we have
\[ \pair{S_1(x) - S_2(x), u} = \pair{S_1(x), u} - \pair{S_2(x), u} = \pair{x, Tu}' - \pair{x, Tu}' = 0. \]
So $(S_1(x) - S_2(x)) \in \dom(T)^\perp$.

For the converse, pick some $x\in X$ and set $S' = S + \pair{x,\cdot}v$. This is an adjoint: for all $a\in \dom(T), b\in \dom(S') = \dom(S)$ we have
\[  \inner{S'b,a} = \inner{Sb, a} + \pair{x,b}\pair{v,a} = \inner{Sb, a} = \inner{b,Ta}'. \]
\end{proof}
\begin{corollary}
Let $\sSet{X,Y,\pair{\cdot,\cdot}}$ and $\sSet{X',Y',\pair{\cdot,\cdot}'}$ be two pairings, $T: Y\not\to Y'$ a linear operator and $S_1, S_2$ two adjoints of $T$.

If $Y$ separates $X$ and $T$ is $\sigma(X,Y)$-densely defined, then for all $x\in \dom(S_1)\cap\dom(S_2)$ we have $S_1(x) = S_2(x)$.
\end{corollary}
\begin{proof}
We have, by \ref{corollaryPerpAsPolar}, $\dom(T)^\perp = \{0\}$. So $S_1(x) - S_2(x) = 0$.
\end{proof}
\begin{corollary}
Let $\sSet{X,Y,\pair{\cdot,\cdot}}$ and $\sSet{X',Y',\pair{\cdot,\cdot}'}$ be two pairings, such that $Y$ separates $X$.

Let $T: Y\not\to Y'$ be a linear operator. Then
\[ \bigcup\setbuilder{\graph(S)}{\text{$S\in (X'\not\to X)$ is an adjoint of $T$}} \]
is the graph of an operator \textup{if and only if} $T$ is $\sigma(X,Y)$-densely defined.
\end{corollary}

\begin{definition}
Let $\sSet{X,Y,\pair{\cdot,\cdot}}$ and $\sSet{X',Y',\pair{\cdot,\cdot}'}$ be two pairings. Let $T: Y\not\to Y'$ be a linear operator.

We define the adjoint $T^*$ as the \emph{relation} on $(X',X)$ with graph
\[ \graph(T^*) \defeq \bigcup\setbuilder{\graph(S)}{\text{$S\in (X'\not\to X)$ is an adjoint of $T$}}. \]
\end{definition}

\begin{lemma}
Let $\sSet{X,Y,\pair{\cdot,\cdot}}$ and $\sSet{X',Y',\pair{\cdot,\cdot}'}$ be two pairings, such that $Y$ separates $X$. Let $T: Y\not\to Y'$ be a linear operator with $\sigma(X,Y)$-dense domain.

If $S$ is an adjoint of $T$ that is defined everywhere, then $T^* = S$.
\end{lemma}

\begin{lemma} \label{pairAdjointRelationLemma}
Let $\sSet{X,Y,\pair{\cdot,\cdot}}$ and $\sSet{X',Y',\pair{\cdot,\cdot}'}$ be two pairings. Let $T: Y\not\to Y'$ be a linear operator and $(x,y)\in X'\times X$.

Then $(x, y)\in T^*$ \textup{if and only if}
\[ \forall z\in\dom(T): \; \pair{x, T(z)} = \pair{y, z}. \]
\end{lemma}
\begin{proof}
$\boxed{\Rightarrow}$ If $(x, y)\in T^*$, then there exists an adjoint $S: X'\not\to X$ such that $S(x) = y$. For all $z\in \dom(T)$ we have $\pair{x, T(z)}' = \pair{S(x), z} = \pair{y, z}$.

$\boxed{\Leftarrow}$ The function defined by $S(x) = y$ and extended to $\Span\{x\}$ by linearity is an adjoint.
\end{proof}

\begin{proposition} \label{pairAdjointDomain}
Let $\sSet{X,Y,\pair{\cdot,\cdot}}$ and $\sSet{X',Y',\pair{\cdot,\cdot}'}$ be two pairings. Let $T: Y\not\to Y'$ be a linear operator. Then
\[ \dom(T^*) = \setbuilder{x\in X'}{\text{$\dom(T)\to \F: u\mapsto \pair{x, Tu}$ is a $\sigma(X,Y)$-continuous functional}}. \]
\end{proposition}
\begin{proof}
$\boxed{\subseteq}$ If $\omega_x: u\mapsto \inner{x, Tu}$ is $\sigma(X,Y)$-continuous, then it can be extended to a continuous functional on all of $Y$ by 


------ TODO: rework from here!! ------

its domain can be extended by continuity to $\overline{\dom(T)}$, which is a Hilbert space. This extended functional has a Riesz vector $x^*$ such that $\omega_x = u\mapsto \inner{x^*, u}$. The linear operator with domain $\Span\{x\}$ that maps $x$ to $x^*$ is then an adjoint.

$\boxed{\supseteq}$ If $x\in\dom(T^*)$, then, using the Cauchy-Schwarz inequality,
\[ |\inner{x,Tu}| = |\inner{T^*x,u}| \leq \norm{T^*x}\;\norm{u}, \]
so the functional $u\mapsto \inner{x, Tu}$ is bounded.
\end{proof}
\begin{corollary}
The domain $\dom(T^*)$ is a vector space and in particular contains $0$.
\end{corollary}

\begin{proposition} \label{HilbertAdjointGaloisConnection}
Let $H, K$ be Hilbert spaces. Take $T\in (H\not\to K)$ and $S\in (K\not\to H)$. Then
\[ S \subseteq T^* \iff T\subseteq S^*. \]
Thus $(*,*)$ is an antitone Galois connection between $\sSet{(H\not\to K), \subseteq}$ and $\sSet{(K\not\to H), \subseteq}$.
\end{proposition}
\begin{proof}
We have $S \subseteq T^*$ iff $S$ is an adjoint of $T$ iff $T$ is an adjoint of $S$ (by \ref{adjointRequirementSymmetric}) iff $T\subseteq S^*$.
\end{proof}
\begin{corollary} \label{HilbertAdjointAntitone}
Let $S,T: H\not\to K$ be operators between Hilbert spaces such that $S\subseteq T$. Then $T^* \subseteq S^*$.
\end{corollary}
\subsubsection{Properties of the adjoint relation}

\begin{proposition}
Let $T$ be an operator between Hilbert spaces and $\lambda\in\C$. If $\lambda \neq 0$, then
\[ \begin{pmatrix}
\id & 0 \\ 0 & \overline{\lambda}\id
\end{pmatrix} \graph(T^*) = (\lambda T)^*. \]
\end{proposition}
Note that if $T^*$ is a function (i.e.\ if $T$ is densely defined), then $\begin{pmatrix}
\id & 0 \\ 0 & \overline{\lambda}\id
\end{pmatrix} \graph(T^*) = \overline{\lambda}T^*$. We write the former in the proposition, because we have not made this assumption.

If $\lambda = 0$ and $T: H\not\to K$, then
\[ \begin{pmatrix}
\id & 0 \\ 0 & 0
\end{pmatrix} \graph(T^*) = \big(0: \dom(T^*)\to H\big) \subseteq \big(0: K\to H\big) = (0 T)^*, \]
where the last equality is given by \ref{adjointBoundedEverywhereDefined}.
\begin{proof}
For the inclusion $\subseteq$, take $f$ to be an adjoint of $T$. It is enough to show that $\overline{\lambda}f$ is an adjoint of $\lambda T$. This follows from
\[ \inner{\overline{\lambda}f(w), v} = \lambda\inner{f(w), v} = \lambda\inner{w,Tv} = \inner{w,\lambda Tv} \qquad \forall w\in \dom(f), v\in \dom(T). \]

For the other inclusion, let $f$ be an adjoint of $\lambda T$. It is enough to show that $\overline{\lambda^{-1}}f$ is an adjoint of $T$, because then $f = \overline{\lambda}\cdot\overline{\lambda^{-1}}f \subseteq \begin{pmatrix}
\id & 0 \\ 0 & \overline{\lambda}\id
\end{pmatrix} \graph(T^*)$. Indeed
\[ \inner{\overline{\lambda^{-1}}f(w), v} = \lambda^{-1}\inner{f(w),v} = \inner{w,\lambda^{-1}\lambda Tv} = \inner{w,Tv} \quad \forall w\in \dom(f), v\in \dom(T). \]
\end{proof}

\begin{proposition} \label{adjointGraph}
Let $T: H\not\to K$ be an operator between Hilbert spaces. Then
\begin{align*}
\graph(T^*) &= \left( \begin{pmatrix}
0 & -\id \\ \id & 0
\end{pmatrix}\graph(T) \right)^\perp 
=  \begin{pmatrix}
0 & -\id \\ \id & 0
\end{pmatrix}\graph(T)^\perp.
\end{align*}
If $T$ is densely defined, then $T^*$ is a closed operator.
\end{proposition}
\begin{proof}
We have
\[ \graph(T^*) = \bigcup\setbuilder{\graph(S)}{\text{$S\in (K\not\to H)$ is an adjoint of $T$}}. \]
Take an adjoint $S$ and $(w, Sw)$ in $\graph(S)$. Then for all $v\in\dom(T)$:
\[ 0 = \inner{w, Tv}_K - \inner{Sw, v}_H = \inner{w, Tv}_K + \inner{Sw, -v}_H = \inner{(w, Sw), (Tv,-v)}_{K\oplus H}. \]
So $(Tv,-v) = \begin{pmatrix}
0 & -\id \\ \id & 0
\end{pmatrix} (v,Tv) \in \graph(S)^\perp $.

The final equality follows from \ref{perpUnderIsometry}, using the fact that $\begin{pmatrix}
0 & -\id \\ \id & 0
\end{pmatrix}$ is a surjective isometry.

If $T$ is densely defined, then $T^*$ is a function by \ref{maximalAdjointIsOperator}. It is closed by \ref{orthogonalComplementClosed}.
\end{proof}
\begin{corollary} \label{adjointDenselyDefinedClosable}
Let $T: H\not\to K$ be a densely defined operator between Hilbert spaces.
Then
\begin{enumerate}
\item $\graph(T^{**}) = \overline{\graph(T)}$;
\item $T^*$ is densely defined \textup{if and only if} $T$ is closable;
\item If $T$ is closable, then $\overline{T} = T^{**}$.
\end{enumerate}
\end{corollary}
\begin{proof}
From the proposition we have
\begin{align*}
\graph(T^{**}) &=  \begin{pmatrix}
0 & -\id \\ \id & 0
\end{pmatrix}\graph(T^*)^\perp 
=  \begin{pmatrix}
0 & -\id \\ \id & 0
\end{pmatrix}\left(\begin{pmatrix}
0 & -\id \\ \id & 0
\end{pmatrix}\graph(T)^\perp\right)^\perp \\
&= \begin{pmatrix}
0 & -\id \\ \id & 0
\end{pmatrix}^2\graph(T)^{\perp\perp} = -\graph(T)^{\perp\perp}
= \overline{\graph(T)}.
\end{align*}
The right-hand side is the graph of an operator iff $T$ is closable and the left-hand side is the graph of an operator iff $T^*$ is densely defined, by \ref{maximalAdjointIsOperator}.

For a closable operator, the closure is defined by $\overline{\graph(T)} = \graph(\overline{T})$.
\end{proof}

\begin{proposition} \label{adjointBoundedEverywhereDefined}
Let $T: H\to K$ be a densely defined operator between Hilbert spaces. Then $\dom(T^*) = K$ \textup{if and only if} $T$ is bounded.
\end{proposition}
\begin{proof}
The direction $\Leftarrow$ is given by \ref{adjointDomain}.

For the other direction, note that $T^*$ is closed by \ref{adjointGraph}. Then $T^*$ is bounded by the closed graph theorem \ref{closedGraphTheorem}. We use the direction $\Leftarrow$ to see that $\dom(T^{**}) = H$. Similarly, $T^{**}$ is closed by \ref{adjointGraph} and bounded by the closed graph theorem \ref{closedGraphTheorem}. Thus $T\subseteq \overline{T} = T^{**}$ is bounded.
\end{proof}

An important application of this proposition is the Hellinger-Toeplitz theorem \ref{HellingerToeplitz}.

\begin{proposition} \label{adjointAlgebraicProperties}
Let $T,S$ be compatible operators between Hilbert spaces. Then
\begin{enumerate}
\item $S^* + T^* \subseteq (S+T)^*$;
\item $S^*T^* \subseteq (TS)^*$.
\end{enumerate}
\end{proposition}
\begin{proof}
(1) Let $f$ be an adjoint of $S$ and $g$ an adjoint of $T$. It is enough to see that $f+g$ is an adjoint of $S+T$. Indeed $\forall w\in \dom(f + g), v\in \dom(S+T)$
\[ \inner{(f + g)(w), v} = \inner{f(w),v} + \inner{g(w), Tv} = \inner{w,Sv} + \inner{w,Tv} = \inner{w,(S+T)v}. \]

(2) Let $f$ be an adjoint of $T$ and $g$ an adjoint of $S$. It is enough to see that $gf$ is an adjoint of $TS$. Indeed
\[ \inner{g\circ f(w), v} = \inner{f(w), Sv} = \inner{w,TSv} \qquad \forall w\in \dom(g\circ f), v\in \dom(TS). \]
\end{proof}


There exist various conditions that make the inclusions in \ref{adjointAlgebraicProperties} equalities.
\begin{proposition} \label{equalityAlgebraicPropertiesAdjoint}
Let $T,S$ be compatible operators between Hilbert spaces.
\begin{enumerate}
\item if $T$ is densely defined, $\dom(S) \subseteq \dom(T)$ and $\dom\big((S+T)^*\big) \subseteq \dom(T^*)$, then $S^* + T^* = (S+T)^*$;
\item if $T$ is densely defined, $\im(S)\subseteq \dom(T)$ and $\dom\big((TS)^*)\subseteq \dom(T^*)$, then $S^*T^* = (TS)^*$;
\item if $S$ is densely defined and $\im(S)$ has finite codimension, then $S^*T^* = (TS)^*$.
\end{enumerate}
\end{proposition}
\begin{proof}
(1) By \ref{adjointAlgebraicProperties}, we have
\[ (S+T)^* - T^* \subseteq (S+T-T)^* = S^*, \]
where the last equality is due to $\dom(S) \subseteq \dom(T)$. Now take $x,y$ such that $x\in \dom\big((S+T)^*\big)$. Then $T^*(x)$ exists and we have the implications
\begin{align*}
x(S+T)^*y \iff& x\big((S+T)^* - T^* + T^*\big)y \\
\iff& \exists z: \; x\big((S+T)^* - T^*\big)z \land (z+T^*(x) = y) \\
\implies& \exists z: \; x(S^*)z \land (z+T^*(x) = y) \\
\iff& x(S^* + T^*)y.
\end{align*}
Thus $(S+T)^* \subseteq S^* + T^*$.

(2) We need to prove $(TS)^* \subseteq S^*T^*$. Assume $(x,y)\in (TS)^*$. By \ref{adjointRelationLemma}, we have
\[ \forall z\in \dom(TS):\; \inner{x, TS(z)} = \inner{y, z}. \]
Because $\im(S)\subseteq \dom(T)$, we have $\dom(TS) = \dom(S)$. Also, by assumption, $x\in \dom(T^*)$. So we have
\[ \forall z\in \dom(S):\; \inner{x, TS(z)} = \inner{T^*(x), S(z)} = \inner{y, z}, \]
which means that $\big(T^*(x), y\big)\in S^*$, so $(x,y)\in S^*T^*$.

(3)
\end{proof}
\begin{corollary}
If $T$ is bounded and everywhere defined, then
\[ S^* + T^* = (S+T)^* \qquad\text{and}\qquad S^*T^* = (TS)^*. \]
\end{corollary}


\begin{lemma} \label{HilbertAdjointLemma}
Let $S,T\in\Bounded(H,K)$ and $\lambda \in \mathbb{F}$.
\begin{enumerate}
\item $(T^*)^* = T$;
\item $(S+T)^* = S^* + T^*$;
\item $(\lambda T)^* = \bar{\lambda}T^*$;
\item $\id_V^* = \id_V$.
\end{enumerate}
Let $T\in\Bounded(H_1,H_2), S\in\Bounded(H_2,H_3)$
\begin{enumerate}
\setcounter{enumi}{4}
\item $(ST)^* = T^*S^*$.
\end{enumerate}
\end{lemma}

\begin{note}
Useful exercise: The identities of \ref{HilbertAdjointLemma} can also be proven by elementary manipulations. For example, to prove (1), we take arbitrary $v\in H$ and $w\in K$, Then
\[ \inner{w,Tv} = \inner{T^*w,v} = \overline{\inner{v,T^*w}} = \overline{\inner{(T^*)^*v,w}} = \inner{w, (T^*)^*v}. \]
By lemma \ref{elementaryOrthogonality} we have $Tv = (T^*)^*v$ for all $v\in V$. 
\end{note}

\subsubsection{Adjoints of densely defined operators}
The adjoint of an operator is a function if and only the operator is densely defined.

\begin{proposition} \label{adjointRangeCriterion}
Let $S: K\not\to H$ and $T: H\not\to K$ be linear operators between Hilbert spaces. If
\[ \im(S\cap T^*) = H \qquad\text{and}\qquad \im(T\cap S^*) = K, \]
then $S$ and $T$ are densely defined with $S^* = T$ and $T^* = S$.
\end{proposition}
\begin{proof}
Notice that $S\cap T^*$ and $T\cap S^*$ are linear operators that are adjoints of each other.

We claim that they are densely defined: take $x\in \dom(S\cap T^*)^\perp$. Then there exists some $y\in H$ such that $x = (T\cap S^*)y$ because of surjectivity. Now for all $z\in \dom(S\cap T^*)$
\[ 0 = \inner{z,x} = \inner{z, (T\cap S^*)y} = \inner{(S\cap T^*)z, y}, \]
so $\inner{z',y} = 0$ for all $z'\in H$, by surjectivity. This means, by \ref{elementaryOrthogonality}, that $y=0$ and thus also $x = (T\cap S^*)y = 0$. We conclude that $\dom(S\cap T^*)^\perp = \{0\}$, meaning $(S\cap T^*)$ is densely defined. The argument for $(T\cap S^*)$ is similar.

It follows that $S$ and $T$ must be densely defined. We have, by \ref{kernelImageAdjoint},
\[ \ker(S) = \im(S^*)^\perp \subseteq \im(T\cap S^*)^\perp = \{0\}. \]
Similarly $\ker(T) = \ker(S^*) = \ker(T^*) = \{0\}$.

So we have $\ker(S) = \ker(T^*)$, $\im(S)\subseteq \im(S\cap T^*)$ and $\im(T^*)\subseteq \im(S\cap T^*)$. The equality $S = T^*$ follows from \ref{partialFunctionSubset}. The equality $T = S^*$ is similar.
\end{proof}


\begin{proposition} \label{kernelImageAdjoint}
Let $T: H\not\to K$ be an operator between Hilbert spaces. Then
\[ \forall v\in K: \; (v,0)\in T^* \iff v\in \im(T)^\perp. \]
If $T^*$ is densely defined, this reduces to
\begin{enumerate}
\item $\ker(T^*) = \im(T)^\perp$;
\item $\ker(T) \subseteq \im(T^*)^\perp$;
\item if $T$ is closed, then $\ker(T) = \im(T^*)^\perp$
\end{enumerate}
\end{proposition}
\begin{proof}
(1) Because $\dom(T)$ is dense in $H$, we have $\dom(T)^\perp = \{0\}$ by \ref{orthogonalComplementDenseSpace}. Take $v\in K$. We have the equivalences
\begin{align*}
v\in \im(T)^\perp &\iff \forall x \in\dom(T): \inner{v, T(x)} = 0 \\
&\iff \forall x \in\dom(T): \inner{v, T(x)} = \inner{v, 0} \\
&\iff (v,0)\in T^*,
\end{align*}
using \ref{adjointRelationLemma}.

Point (1) is a direct translation in the case that $T^*$ is a function.

For point (2) note that $T\subseteq T^{**}$ (by \ref{adjointDenselyDefinedClosable}) implies that $(v,0)\in T \implies (v,0)\in T^{**}$.

For point (3): in this case $\ker(T) = \ker(T^{**}) = \im(T^*)^\perp$.
\end{proof}
\begin{corollary}[Closed range theorem for Hilbert spaces]
Let $T$ be a closed, densely defined operator between Hilbert spaces. Then the following are equivalent:
\begin{enumerate}
\item $\im(T)$ is closed;
\item $\im(T^*)$ is closed;
\item $\im(T) = \ker(T^*)^\perp$;
\item $\im(T^*) = \ker(T)^\perp$.
\end{enumerate}
\end{corollary}
\begin{proof}
By the proposition and \ref{orthogonalComplementClosed}, we have $\overline{\im(T)} = \ker(T^*)^\perp$. This shows $(1) \Leftrightarrow (3)$ and $(2) \Leftrightarrow (4)$.

TODO equivalence $(1)\Leftrightarrow (2)$.
\end{proof}
TODO ref closed range theorem for Banach spaces. This is, e.g., the case when $T$ is bounded below, see \ref{boundedBelowClosedRange}.

\begin{proposition}
Let $T: H\not\to K$ be a densely defined operator between Hilbert spaces. Then
\begin{enumerate}
\item $\im(T)$ is dense in $K$ \textup{if and only if} $T^*$ is injective;
\item if $T$ and $T^*$ are injective, then $(T^*)^{-1} = (T^{-1})^*$;
\item if $T$ is closable and $\overline{T}$ is injective, then $\overline{T}^{\,-1} = \overline{T^{-1}}$.
\end{enumerate}
\end{proposition}
\begin{proof}
(1) This is immediate from \ref{kernelImageAdjoint} and \ref{injectivityKernelTriviality}:
\[ \text{$\im(T)$ is dense} \quad\iff\quad \{0\} = \im(T)^\perp = \ker(T^*). \]

(2) We have $\graph(T^{-1}) = \begin{pmatrix}
0 & \id \\ \id & 0
\end{pmatrix}\graph(T)$. Also note that $\begin{pmatrix}
0 & \id \\ \id & 0
\end{pmatrix}$ and $\begin{pmatrix}
0 & -\id \\ \id & 0
\end{pmatrix}$ commute. Then we compute using \ref{adjointGraph}:
\begin{align*}
\graph((T^*)^{-1}) &= \begin{pmatrix}
0 & \id \\ \id & 0
\end{pmatrix}\begin{pmatrix}
0 & -\id \\ \id & 0
\end{pmatrix}\graph(T)^\perp \\
&= \begin{pmatrix}
0 & -\id \\ \id & 0
\end{pmatrix}\begin{pmatrix}
0 & \id \\ \id & 0
\end{pmatrix}\graph(T)^\perp \\
&= \begin{pmatrix}
0 & -\id \\ \id & 0
\end{pmatrix}\left(\begin{pmatrix}
0 & \id \\ \id & 0
\end{pmatrix}\graph(T)\right)^\perp = \graph((T^{-1})^*).
\end{align*}
The penultimate equality follows from \ref{perpUnderIsometry}, using the fact that $\begin{pmatrix}
0 & \id \\ \id & 0
\end{pmatrix}$ is a surjective isometry.
\end{proof}

\subsubsection{Adjoints of bounded operators}
\begin{proposition}
Let $T: H\to K$ be a densely defined operator between Hilbert spaces. Then
\begin{enumerate}
\item if $T\in\Bounded(H,K)$, then $T^*\in\Bounded(K,H)$;
\item if $T^*\in\Bounded(K,H)$, then $T$ is bounded. If $T$ is closed, then $T$ is defined everywhere.
\end{enumerate}
Assume $T\in\Bounded(H,K)$. Then
\begin{enumerate} \setcounter{enumi}{2}
\item $\norm{T} = \norm{T^*}$;
\item $T^* = C_H^{-1}T^tC_K$, where $C_K$ is the Riesz isometry from \ref{RieszIsometry}.
\end{enumerate}
\end{proposition}
\begin{proof}
(1) Assume $T\in\Bounded(H,K)$. Then $u\mapsto \inner{x,Tu}$ is a bounded functional for all $x\in K$, so $\dom(T^*) = K$ by \ref{adjointDomain}. Also $T^*$ is closed by \ref{adjointGraph}, so it is bounded by the closed graph theorem \ref{closedGraphTheorem}.

(2) Assume $T^*\in\Bounded(K,H)$. By the previous argument $T \subseteq \overline{T} = T^{**}\in\Bounded(H,K)$.

(3) The function $(x,u)\mapsto \inner{x,Tu}$ is a bounded sesquilinear form. By proposition \ref{sesquilinearRepresentation}, $T^*$ must be the unique $S$ from the proposition, which has norm $\norm{T}$.

(4) Finally we note that $C_H^{-1}T^tC_K$ is an adjoint with domain $K$ and conclude by \ref{everywhereDefinedAdjointLemma}.
\end{proof}

\begin{lemma}
The adjoint defines a map $*:\Bounded(H,K)\to \Bounded(K,H)$ that is anti-linear and continuous in the weak and uniform operator topologies. It is continuous in the strong operator topology \textup{if and only if} finite dimensional.
\end{lemma}
\begin{proof}
By the proposition the adjoint map is anti-linear. It is also bounded with norm $1$. Then by corollary \ref{boundedAntiLinearMaps} it must be bounded.

TODO
\end{proof}

\begin{proposition}
Let $H,K$ be Hilbert spaces and $T:H\to K$ a bijective bounded linear operator with bounded inverse. Then $(T^*)^{-1}$ exists and
\[ (T^*)^{-1} = (T^{-1})^*. \]
\end{proposition}
\begin{proof}
We prove $(T^{-1})^*$ is both a left- and a right-inverse of $T^*$: $\forall x\in H, y\in K$
\begin{align*}
\inner{T^*(T^{-1})^*x,y} &= \inner{x,T^{-1}Ty} = \inner{x,y} \\
\inner{x,(T^{-1})^*T^*y} &= \inner{TT^{-1}x,y} = \inner{x,y}
\end{align*}
So, by lemma \ref{elementaryOrthogonality}, $T^*(T^{-1})^* = \id_H$ and $(T^{-1})^*T^* = \id_K$.
\end{proof}

\begin{proposition} \label{normOfSquare}
Let $T\in \Bounded(H,K)$ with $H,K$ Hilbert spaces. Then
\[ \norm{T^*T}= \norm{T}^2 = \norm{TT^*}. \]
\end{proposition}
\begin{proof}
For $\norm{T^*T}= \norm{T}^2$ first observe that
\[ \norm{T^*T} \leq \norm{T^*}\cdot\norm{T} = \norm{T}^2. \]
Conversely, $\forall x\in H$:
\[ \norm{T(x)}^2 = \inner{Tx,Tx} = \inner{T^*Tx,x} \leq \norm{T^*Tx}\cdot \norm{x} \leq \norm{T^*T}\cdot\norm{x}^2. \]
The other equality follows by applying the first to $T^*$ and using $\norm{T^*}=\norm{T}$.
\end{proof}


\subsection{Mackey topology}

\begin{theorem}[Mackey-Arens]
\end{theorem}

\subsection{Duality sets for normed spaces}
\begin{definition}
Consider a dual system $\sSet{V, W, \pair{\cdot,\cdot}}$ where $V,W$ are normed spaces. For all $v\in V$ we have the \udef{duality set} of $v$
\[ \mathfrak{d}(v) \defeq \setbuilder{w\in W}{\norm{v}^2 = \pair{v,w} = \norm{w}^2}. \]
Similarly the duality set of $w\in W$ is
\[ \mathfrak{d}(w) \defeq \setbuilder{v\in V}{\norm{v}^2 = \pair{v,w} = \norm{w}^2}. \]
\end{definition}

\begin{proposition}
Consider the dual pair $\sSet{V^*, V}$ for some normed space $V$. Then for all $v\in V$, $\mathfrak{d}(v)$ is not empty.
\end{proposition}
\begin{proof}
We have $\norm{v}\cdot \omega_v\in \mathfrak{d}(v)$, where $\omega_v$ is defined in Corollary \ref{existenceBoundedFunctionalOfSameNorm} of the Hahn-Banach extension theorem.
\end{proof}

\section{Operators on topological vector spaces}

\subsection{Continuous operators}
\subsubsection{Closed graph theorem}

\begin{theorem}[Closed graph theorem] \label{closedGraphTheorem}
Let $f:X\to Y$ be a map from a topological space $X$ into a Hausdorff space $Y$.
\begin{enumerate}
\item if $f$ is continuous, then $f$ has closed graph;
\item if $X$ is compact, then the converse also holds.
\end{enumerate}
\end{theorem}
\begin{proof}
TODO
\end{proof}
TODO: for Banach spaces $X$ complete enough!!!!!!


\subsection{Compact operators}
\begin{definition}
A linear operator $T:X\to Y$ between TVSs is \udef{compact} if it maps a neighbourhood of the origin to a precompact set, i.e.\ 
\[ \exists U \in \neighbourhood(0): \;  \text{$\overline{T[U]}$ is compact.} \]
The set of compact linear operators in $(X\to Y)$ is denoted $\Compact(X,Y)$.
\end{definition}
TODO: doesn't the neighbourhood need to be bounded in some way?????

\begin{proposition}
Let $X$ be a normed space and $Y$ a TVS and $T:X\to Y$ a linear operator. Then the following are equivalent:
\begin{enumerate}
\item $T$ is a compact operator;
\item there exists a neighbourhood $U \subset X$ of the origin and a compact set $V\subset Y$ such that $T[U] \subset V$;
\item the image of the unit ball of $X$, $T[B(\vec{0},1)]$, is precompact in $Y$;
\item the image of any bounded set in $X$ is precompact in $Y$.
\end{enumerate}
If $Y$ is a normed space, these are also equivalent to
\begin{enumerate} \setcounter{enumi}{4}
\item for any bounded sequence $(x_{n})_{n\in \mathbb{N}}$ in $X$, the sequence $(Tx_{n})_{n\in \mathbb{N} }$ contains a converging subsequence.
\end{enumerate}
\end{proposition}
\begin{proof}
TODO
\end{proof}


\begin{lemma}
Let $X,Y$ be TVSs.
\begin{enumerate}
\item Then $\Compact(X, Y)$ is a vector space.
\item If $X,Y$ are normed spaces, then $\Compact(X, Y)$ is a subspace of $\Bounded(X, Y)$.
\end{enumerate}
\end{lemma}
\begin{proof}
(1) Let $K,K':X\to Y$ be compact operators. Then, by \ref{closureGroupOperation} (TODO opposite inclusion!),
\[ \overline{K[B(0, 1)]+K'[B(0, 1)]} \subseteq \overline{K[B(0, 1)]}+\overline{K'[B(0, 1)]}, \qquad \overline{K[\lambda B(0, 1)]} = \lambda\overline{K[B(0, 1)]}. \]

(2) Let $K\in\Compact(X, Y)$. Then the image of the unit ball is precompact, meaning it is bounded. So $K$ is bounded by \ref{existenceOperatorNorm}.
\end{proof}

\begin{lemma}
Let $T:V\to W$ be a bounded operator. If $W$ has the Heine-Borel property, then $T$ is compact.
\end{lemma}
\begin{proof}
The set $T[B(\vec{0},1)]$ is bounded because $T$ is. By the Heine-Borel (TODO ref) property of $W$, $\overline{B(\vec{0},1)}$ is compact.
\end{proof}
\begin{corollary}
Bounded operators with as image a finite dimensional normed space are compact.
\end{corollary}
\begin{corollary}
The identity on a normed space $X$ is compact \textup{if and only if} $X$ is finite-dimensional.
\end{corollary}
\begin{proof}
TODO ref. 
\end{proof}

\begin{proposition}
Compact operators map weakly convergent sequences to strongly convergent sequences. TODO! + remove from Hilbert section.
\end{proposition}
\begin{corollary} \label{limitCompactImageOrthonormalSequence}
Let $V$ be an inner product space and $\seq{e_n}$ a sequence of orthonormal vectors in $V$. If $K$ is a compact operator, then $\lim_{n\to\infty}Ke_n = 0$.
\end{corollary}
\begin{proof}
Any sequence of orthonormal vectors $\seq{e_n}$ converges weakly to $0$. Because $K$ is compact, $\seq{Ke_n}$ converges strongly to zero. TODO ref.
\end{proof}
\begin{corollary}
If $V$ is infinite-dimensional and $K$ is invertible, then its inverse is unbounded.
\end{corollary}
\begin{proof}
Due to $\lim_{n\to\infty}Ke_n = 0$ the operator $K$ cannot be bounded below, so $K^{-1}$ is not bounded by \ref{boundedBelow}.
\end{proof}

\section{Continuity}
\url{https://en.wikipedia.org/wiki/Bilinear_map#Continuity_and_separate_continuity}













\chapter{Normed and Banach spaces}
In this chapter we will always use either $\mathbb{F} = \R$ or $\mathbb{F} = \C$.

\begin{definition}
\begin{itemize}
\item A \udef{normed space} is a vector space equipped with a norm.
\item A \udef{Banach space} is a normed vector space that is complete as a metric space.
\end{itemize}
\end{definition}

A finite-dimensional normed space is automatically a Banach by proposition \ref{finiteDimComplete}.

Every proper subspace $U$ of a normed vector space $V$ has empty interior.
A nice consequence of this is that any closed proper subspace is necessarily nowhere dense. So if V is a Banach space, the Baire category theorem implies that V cannot be a countable union of closed proper subspaces. In particular, an infinite dimensional Banach space cannot be a countable union of finite dimensional subspaces. This means, for example, that a vector space of countable dimension (e.g\ the space of polynomials) cannot be equipped with a complete norm.

The space $\Bounded(V,W)$ is a Banach space.


Complemented subspace problem: \url{https://arxiv.org/pdf/math/0501048v1.pdf}


TODO: \url{https://math.stackexchange.com/questions/2151779/normed-vector-spaces-over-finite-fields/2568231}

\section{Normed spaces}

\begin{lemma}
A subspace of a normed vector space is a normed space, with the norm given by the restriction of the norm in the larger space.
\end{lemma}

\begin{definition}
A vector with norm 1 is called a \udef{unit vector}. Unit vectors are often written with a hat:
\[ \norm{\vhat{v}} = 1. \]
\end{definition}

\subsection{Uniform norm}
\begin{lemma} \label{vectorSpaceUniformNorm}
Let $\sSet{V,\norm{\cdot}}$ be a normed vector space and $\sSet{X,d}$ a metric space. The uniform norm on $(X\to V)$ is a norm.
\end{lemma}
By \ref{groupUniformNorm}, the norm convergence of this norm is the uniform convergence.
\begin{proof}
It is a group norm by \ref{groupUniformNorm}. We just need to show that it is positively homogeneous:
\[ \norm{\lambda f}_u = \sup_{x\in X}\norm{\lambda f(x)} = \sup_{x\in X}|\lambda|\norm{f(x)} = |\lambda|\sup_{x\in X}\norm{f(x)} = |\lambda|\norm{f}_u. \]
\end{proof}

\subsection{TODO from TVS theory}
\begin{proposition} \label{dualNormTopologyStrong}
Let $\sSet{V, \norm{\cdot}}$ be a normed space. The norm topology on $\dual{V}$ is equal to the strong topology $\beta(\dual{V}, V)$.
\end{proposition}
\begin{proof}
TODO!
\end{proof}

\begin{lemma} \label{polarOfBall}
Let $\sSet{V, \norm{\cdot}}$ be a normed space and $\epsilon >0$. Then $\cball_V(0,\epsilon)^\pol = \cball_{\dual{V}}(0,\epsilon)$.
\end{lemma}
\begin{proof}
We have
\begin{align*}
f\in \cball_V(0,\epsilon)^\pol &\iff \forall x\in \cball_V(0,\epsilon): \; \abspair{f,x} \leq 1 \\
&\iff \forall x\in \cball_V(0,1): \; \abspair{f,\epsilon^{-1}x} \leq 1 \\
&\iff \forall x\in \cball_V(0,1): \; \abspair{f,x} \leq \epsilon \\
&\iff \norm{f} \leq \epsilon \\
&\iff f\in \cball_{\dual{V}}(0,\epsilon).
\end{align*}
\end{proof}

\begin{proposition}
Let $\sSet{V, \norm{\cdot}}$ be a normed space. Then $\cball_{\dual{V}}(0, 1)$ is pointwise compact.
\end{proposition}
\begin{proof}
By \ref{polarOfBall} and \ref{alaogluTheorem}.
\end{proof}

\subsection{The topology of a normed space}
\begin{definition}
Let $\sSet{V,\norm{\cdot}}$ be a normed space. The initial vector space convergence w.r.t. the norm is called the \udef{norm convergence}.
\end{definition}
The norm convergence is topological TODO ref(!). Its topology is called the \udef{norm topology}.

\begin{lemma}
The norm convergence is \emph{not} the initial convergence w.r.t. to the norm.
\end{lemma}
\begin{proof}
In the initial convergence w.r.t. to the norm, all vectors of the same norm are indistinguishable, so this convergence space is not $T_0$.

On the other hand, $\{0\}$ is closed in $\R$ and thus its preimage $\{0\} \subset V$ is closed in the norm topology (TODO ref preimage closed is closed). By \ref{HausdorffCriterionConvergenceGroup}, we have that the norm convergence must be Hausdorff, or $T_2$.
\end{proof}

\begin{proposition}
The norm convergence is topological and metric.

Every normed space can be viewed as a metric space with the metric $d:V\times V \to [0,\infty[$ given by
\[ d(x,y) = \norm{x-y}. \]
This metric has the properties of
\begin{itemize}[leftmargin=6cm]
\item[\textbf{Translation invariance}] $d(x+a, y+a) = d(x,y)$;
\item[\textbf{Scaling}] $d(\lambda x, \lambda y) = |\lambda|d(x,y)$.
\end{itemize}
Conversely, any metric with translation invariance and scaling determines a norm:
\[ \norm{x} = d(x,\vec{0}). \]
Passing from norm to metric back to norm, we recover the original norm.
\end{proposition}
\begin{lemma}
A linear map $L:V\to W$ between normed spaces is an isometry for the metric \textup{if and only if} it preserves the norm, i.e.\
\[ \forall v\in V: \quad \norm{v}_V = \norm{L(v)}_W. \]
\end{lemma}
\begin{proof}
Assume $L$ is an isometry, then
\[ \norm{v} = d(v,\vec{0}) = d(L(v),L(\vec{0})) = \norm{L(v) - L(\vec{0})} = \norm{L(v) - \vec{0}} = \norm{L(v)}. \]
Assume $L$ preserves the norm, then
\[ d(L(v_1), d(v_2)) = \norm{L(v_1)-L(v_2)} = \norm{L(v_1-v_2)} = \norm{v_1-v_2} = d(v_1,v_2). \]
\end{proof}

\begin{proposition}
Let $V$ be a normed vector space, then the norm $\norm{\cdot}:V\to \R$ is a continuous map.
\end{proposition}
\begin{proof}
The reverse triangle inequality, $|\norm{v}-\norm{w}| \leq \norm{v-w}$, implies that the norm is Lipschitz continuous with Lipschitz constant $1$, so we can use \ref{LipschitzcontinuousContinuous}.
\end{proof}

\subsubsection{Continuous operators}
\begin{theorem} \label{boundedLinearMaps}
Let $L$ be a linear operator between normed spaces $V,W$. The following are equivalent:
\begin{enumerate}
\item $L$ is continuous;
\item $L$ is continuous at $0$;
\item $L$ is uniformly continuous;
\item $L$ is Lipschitz continuous.
\end{enumerate}
\end{theorem}
\begin{proof}
The equivalences $(1) \Leftrightarrow (2) \Leftrightarrow (3)$ are given by \ref{uniformContinuityGroupHomomorphism}. The implication $(4)\Rightarrow (3)$ is given by \ref{LipschitzcontinuousContinuous}.

Finally we prove $(2)\Rightarrow (4)$. From continuity at zero, there exists a $\delta>0$ such that $\norm{L(h)} = \norm{L(h)-L(0)} \leq 1$ for all $h\in \dom(L)$ with $\norm{h}\leq \delta$. Thus for all nonzero $v\in \dom(L)$
\[ \norm{L(v)} = \norm{\frac{\norm{v}}{\delta}L(\delta \frac{v}{\norm{v}})} = \frac{\norm{v}}{\delta}\norm{L(\delta \frac{v}{\norm{v}})}\leq \frac{\norm{v}}{\delta}. \]
\end{proof}
\begin{corollary}
A linear operator $L:V\to W$ between normed spaces is a homeomorphism \textup{if and only if} there exists $C_1,C_2> 0$ such that
\[ \forall x\in V:\qquad C_1\norm{x}\leq \norm{L(x)} \leq C_2\norm{x}. \]
\end{corollary}

\begin{definition}
An operator $L$ between normed vector spaces is called \udef{bounded} if it is (Lipschitz) continuous.

The set of bounded operators from $V$ to $W$ is denoted $\Bounded(V,W)$. If $V=W$, we write $\Bounded(V)$.
\end{definition}
In other words, bounded means there exists an $M>0$ such that $\forall v\in \dom(L)$
\[ \norm{L(v)} \leq M \norm{v}. \]

\begin{proposition} \label{boundedAntiLinearMaps}
An anti-linear map between complex vector spaces is continuous \textup{if and only if} it is bounded.
\end{proposition}
\begin{proof}
An anti-linear map $A:V\to W$ is an $\R$-linear map $A:V_\R\to W_\R$. Now $V_\R, W_\R$ have the same norms as $V,W$ and thus the same topology. So $A:V\to W$ is continuous if and only if $A:V_\R\to W_\R$ is continuous.
\end{proof}


\subsubsection{Equicontinuity}
\begin{proposition}
Let $\mathcal{F}$ be a set of linear operators in $(V\to W)$. Then $\mathcal{F}$ is equicontinuous at $x_0\in V$ \textup{if and only if} there exists $C>0$ such that
\[ \forall f\in\mathcal{F}:\qquad \norm{f(x_0)} \leq C\norm{x_0}. \]
\end{proposition}

\subsubsection{Comparison of norm topologies}
\begin{definition}
Let $V$ be a vector space and $\norm{\cdot}_1$, $\norm{\cdot}_2$ two norms on $V$. We say
\begin{itemize}
\item $\norm{\cdot}_1$ is \udef{bounded} by $\norm{\cdot}_2$ if there exists $C\in \R$ such that $\forall v\in V: \norm{v}_1 \leq C\norm{v}_2$;
\item $\norm{\cdot}_1$ and $\norm{\cdot}_2$ are \udef{equivalent} if each is bounded by the other.
\end{itemize}
\end{definition}

\begin{lemma}
Let $V$ be a vector space and $\norm{\cdot}_1$, $\norm{\cdot}_2$ two norms on $V$. These norms are equivalent \textup{if and only if} there exists a $C>0$ such that
\[ \frac{1}{C}\norm{\cdot}_1 \leq \norm{\cdot}_2 \leq C\norm{\cdot}_1. \]
\end{lemma}

\begin{proposition} \label{normComparison}
Let $V$ be a vector space and $\norm{\cdot}_1$, $\norm{\cdot}_2$ two norms on $V$. Then the following are equivalent:
\begin{enumerate}
\item $\norm{\cdot}_2$ is bounded by $\norm{\cdot}_1$;
\item $\id_V: \sSet{V,\norm{\cdot}_1} \to \sSet{V,\norm{\cdot}_2}$ is uniformly continuous;
\item the norm topology of $\norm{\cdot}_1$ is finer than the norm topology of $\norm{\cdot}_2$.
\end{enumerate}
\end{proposition}
\begin{proof}
$(1) \Leftrightarrow (2)$ By \ref{boundedLinearMaps} both are equivalent to
\[ \exists C>0: \forall v\in V: \qquad \norm{v}_2 = \norm{\id_V((v))}_2 \leq C\norm{v}_1. \]

$(2) \Leftrightarrow (3)$ Follows straight from \ref{identityContinuity}.
\end{proof}
\begin{corollary}
Equivalent norms induce the same topology.
\end{corollary}

\subsubsection{Subspaces of normed spaces}

\begin{lemma}
Every proper subspace $U$ of a normed vector space $V$ has empty interior.
\end{lemma}
\begin{proof}
Suppose $U$ has a non-empty interior. Then it contains some ball $B(u,\epsilon)$. Now every vector in $V$ can be translated and rescaled to fit inside the ball $B(u,\epsilon)$. Indeed let $v\in V$ and set $u' = u+ \frac{\epsilon}{2\norm{v}}v \in B(u,\epsilon)$. Then, since $U$ is a subspace $v = \frac{2\norm{v}}{\epsilon}(u'-u)\in U$. So $U=V$.
\end{proof}

\begin{lemma}[Riesz's lemma] \label{RieszsLemma}
Let $V$ be a normed vector space. Given a non-dense subspace $X$ and a number $\theta<1$, there exists a unit vector $v\in V$ such that
\[ \theta \leq d(X,v) = \inf_{x\in X}\norm{x-v}. \]
\end{lemma}
\begin{proof}
Take a vector $v_1$ not in the closure of $X$ and put $a = \inf_{x\in X}\norm{x-v_1}$. Then $a>0$ by lemma \ref{sequencesSupInf}. For $\epsilon > 0$, let $x_1\in X$ be such that $\norm{x_1+v_1}<a+\epsilon$. Then take
\[ v = \frac{v_1 - x_1}{\norm{v_1-x_1}} \qquad \text{so} \qquad \norm{v}=1. \]
And
\[ \inf_{x\in X}\norm{x-v} = \inf_{x\in X}\norm{x-\frac{v_1 - x_1}{\norm{v_1-x_1}}} = \inf_{x\in X}\norm{\frac{x-v_1 + x_1}{\norm{v_1-x_1}}} = \frac{\inf_{x\in X}\norm{x-v_1}}{\norm{v_1-x_1}} \geq \frac{a}{a+\epsilon}. \]
By choosing $\epsilon >0$ small, $a/(a+\epsilon)$ can be made arbitrarily close to $1$.
\end{proof}
For finite-dimensional spaces we can even take $\theta=1$.

\subsection{Linear independence and bases in normed spaces}
\url{https://math.stackexchange.com/questions/1518029/are-uncountable-schauder-like-bases-studied-used}

\subsection{Finite-dimensional normed (sub)spaces}

\begin{lemma} \label{coordinateContinuity}
Let $V$ be a normed vector space and $\{x_1, \ldots, x_n\}$ a linearly independent set of vectors. There exists a $c>0$ such that $\forall \alpha_1,\ldots, \alpha_n \in \mathbb{F}$:
\[ \norm{\alpha_1x_1 + \ldots + \alpha_nx_n} \geq c(|\alpha_1|+\ldots+|\alpha_n|) . \]
\end{lemma}
\begin{proof}
TODO ref locally convex spaces? Local compactness?
\end{proof}
TODO This is equivalent with continuity of coordinate functions.

\begin{proposition} \label{finiteDimComplete}
Every finite-dimensional subspace of a normed vector space is complete.
\end{proposition}
\begin{proof}
Take a basis $\{e_i\}_{i=1}^n$ and let $c$ be as in lemma \ref{coordinateContinuity}. Consider an arbitrary Cauchy sequence $(v_k)_{k\in\N}$. We can write
\[ v_k = \alpha_{k,1}e_1 + \ldots + \alpha_{k,n}e_n. \]
We claim that $(\alpha_{k,i})_{k\in\N}$ is Cauchy in $\mathbb{F}$ for all $1\leq i\leq n$. Indeed, take an $\epsilon>0$. By the Cauchy nature of $(v_k)_{k\in\N}$ we can find a $k_0$ such that $\forall k', k''>k_0:$
\[ c\epsilon > \norm{v_{k'} - v_{k''}} \geq \norm{\sum_{i=1}^n (\alpha_{k',i}-\alpha_{k'',i})e_i}\geq c\sum_{i=1}^n |\alpha_{k',i}-\alpha_{k'',i}| \geq c |\alpha_{k',i}-\alpha_{k'',i}|. \]
Dividing left and right by $c$ gives exactly the Cauchy condition for each $1\leq i\leq n$. By the completeness of $\R$ or $\C$, each of these sequences has a limit $\alpha_i$.
Then $v= \sum_{i=1}^n\alpha_ie_i$ is an element of the subspace. The sequence $(v_k)$ converges to $v$ because
\[ \norm{v_k-v} = \norm{\sum_{i=1}^n (\alpha_{k,i}-\alpha_i)e_i} \leq \sum_{i=1}^n |\alpha_{k,i}-\alpha_i|\norm{e_i} \]
and the right-hand side goes to zero as $k\to \infty$.
\end{proof}
\begin{corollary} \label{finiteDimClosed}
Every finite-dimensional subspace of a normed vector space is closed.
\end{corollary}
TODO ref for proof.

\begin{proposition}
On a finite-dimensional vector space all norms are equivalent.
\end{proposition}
\begin{proof}
Let $\{e_i\}_{i=1}^n$ be a basis and take an arbitrary vector $v = \sum_{i=1}^nv_ie_i$. Let $\norm{\cdot}_1$ and $\norm{\cdot}_2$ be two norms.
We calculate
\[ \norm{v}_1 \leq \sum_{i=1}^n|v_i|\norm{e_i}_1 \leq k\sum_{i=1}^n|v_i| \leq \frac{k}{c_2}\norm{v}_2 \]
where the first inequality is the triangle inequality, the second comes from $k=\max\norm{e_i}_1$ and the third is lemma \ref{coordinateContinuity}. A similar calculation gives the other necessary inequality.
\end{proof}

\begin{proposition}
In a finite-dimensional normed space $V$, any subset $M \subseteq V$ is compact if and only if $M$ is closed and bounded.
\end{proposition}
\begin{proof}
TODO + ref Heine Borel property
\end{proof}


TODO: move up?
\begin{proposition} \label{compactnessUnitBall}
The closed unit ball of a vector space is compact \textup{if and only if} the vector space is finite-dimensional.
\end{proposition}
\begin{proof}
One direction is given by the previous proposition. For the other direction, we show the contrapositive: let the vector space be infinite-dimensional.
We define a sequence of unit vectors $(e_i)_{i\in\N}$ recursively as follows:
\begin{itemize}
\item $e_1$ is just a unit vector;
\item for $e_{n+1}$ apply Riesz's lemma \ref{RieszsLemma} to the subspace $\Span\{e_i\}_{i=1}^n$ and $\theta = 1/2$. This subspace cannot be dense, because it is a closed (by corollary \ref{finiteDimClosed}) finite-dimensional subspace of an infinite-dimensional vector space.
\end{itemize}
This yields a sequence such that for all $m,n$
\[ \norm{e_m - e_n}\geq \frac{1}{2}. \]
This sequence is not Cauchy and thus not convergent.
\end{proof}





\section{Bounded operators}
Bounded operators are Lipschitz continuous operators.

An operator between normed spaces is bounded iff it is continuous, see \ref{boundedLinearMaps}.

\begin{lemma}
Let $V,W$ be normed spaces and $T:V\to W$ a linear operator. Then $T$ is bounded \textup{if and only if} $T^{\preimf}[\ball(\vec{0},1)]$ has non-empty interior.
\end{lemma}
\begin{proof}
Assume $T$ is bounded, then $\ball(\vec{0}, \norm{T}^{-1}) \subseteq T^{\preimf}[\ball(\vec{0},1)]$, indeed for all $v\in \ball(\vec{0}, \norm{T}^{-1})$ we have
\[ \norm{Tv} \leq \norm{T}\;\norm{v} < \norm{T}\;\norm{T}^{-1} = 1.  \]
As $\ball(\vec{0}, \norm{T}^{-1})$ is open, it is contained in the interior, which is thus non-empty.

Now assume $T^{\preimf}[\ball(\vec{0},1)]$ has non-empty interior, then we can pick some $x$ in the interior and $T^{-\preimf}[\ball(\vec{0},1)]$ is a neighbourhood of $x$. Then we have that
\[ T^{\preimf}[\ball(\vec{0},1)] - x = T^{\preimf}[\ball(\vec{0},1)- T(x)] \]
is a neighbourhood of $0$. Now $T$ is bounded on $T^{\preimf}[\ball(\vec{0},1)- T(x)]$, so it is continuous by \ref{boundedOnVicinityImpliesContinuous}.
\end{proof}

\begin{lemma} \label{kerClosed}
Let $T$ be a bounded linear operator. Then $\ker(T)$ is closed.
\end{lemma}
\begin{proof}
Suppose $T$ bounded and thus continuous. Then $\ker L = L^{-1}[\{0\}]$ and thus closed, by proposition \ref{continuity}.
\end{proof}
\begin{proof}
Let $v\in \overline{\ker(T)}$. Then find a sequence $(v_n)$ in $\ker(T)$ that converges to $v$. Then by continuity $(Tv_n)$ converges to $Tv$, but for all $n\in\N: Tx_n = 0$, so the limit is $Tv=0$. Thus $v\in\ker(T)$, making it closed.
\end{proof}

\begin{proposition}\label{continuousMapCriterion}
Let $L:V\to W$ be a linear map between normed spaces.
\begin{enumerate}
\item If $V$ is finite-dimensional, then $L$ is continuous.
\item If $W$ is finite-dimensional, then $L$ is continuous \textup{if and only if} $\ker L$ is closed.
\end{enumerate}
\end{proposition}
TODO: true for general TVS
\begin{proof}
\begin{enumerate}
\item This follows from a consideration of the graph norm $\norm{v}_L = \norm{v}+\norm{Lv}$ and the fact that on a finite-dimensional space any two norms are equivalent: for all $v$ we can choose an $M$ such that
\[ \norm{Lv}\leq \norm{v}_L \leq M\norm{v}. \]
\item Assume $W$ finite-dimensional. Consider the map $\bar{L}:V/\ker L\to W: v+\ker{L}\mapsto L(v)$, defined in proposition \ref{splittingMap}. Then $V/\ker L$ is isomorphic to a subspace of $W$ and thus is finite-dimensional. By the first point, $\bar{L}$ must be continuous. Let $\pi: V\to V/\ker L$ denote the quotient map, which is continuous (TODO is this where closure of $\ker L$ is used?). Then $L = \bar{L}\circ \pi$ is a composition of continuous maps and thus continuous.

Conversely, we have the lemma \ref{kerClosed}.
\end{enumerate}
\end{proof}

\begin{example}
Let $\seq{e_n}$ be a basis of unit vectors of an infinite dimensional real vector space. Then consider the map $e_n\mapsto n$ and extend by linearity. This is an unbounded linear operator with finite dimensional codomain.
\end{example}


\subsection{The algebra of bounded operators}
\begin{lemma} \label{existenceOperatorNorm}
Let $(V,\norm{\cdot}_V)$ and $(W,\norm{\cdot}_W)$ be normed spaces and $L\in\Lin(V, W)$. Then $L$ is bounded \textup{if and only if}
\[ \sup\setbuilder{\frac{\norm{Lx}_W}{\norm{x}_V}}{x\in V\setminus\{0\}} \] 
is finite.
\end{lemma}
\begin{definition}
Let $(V,\norm{\cdot}_V)$ and $(W,\norm{\cdot}_W)$ be normed spaces and $L\in\Lin(V, W)$ bounded. Then
\[ \norm{L} \defeq \sup\setbuilder{\frac{\norm{Lx}_W}{\norm{x}_V}}{x\in V\setminus\{0\}} \]
is called the \udef{operator norm} of $L$.
\end{definition}

\begin{proposition} \label{operatorNorm}
Let $L\in\Bounded(V,W)$ be a bounded operator and let $B(\vec{0},\epsilon)$ be an open ball centered at $\vec{0}$. Then
\begin{align*}
\norm{L} &= \frac{\sup L[B(\vec{0},\epsilon)]}{\epsilon} \\
&= \frac{\sup L[\overline{B}(\vec{0},\epsilon)]}{\epsilon} \\
&= \sup\setbuilder{\norm{Lx}}{\norm{x} = 1}.
\end{align*}
\end{proposition}
\begin{proof}
TODO
\end{proof}

\begin{proposition} \label{operatorNormIsNorm}
Let $S,T$ be compatible bounded operators. Then
\begin{enumerate}
\item $\norm{\lambda S} = |\lambda|\;\norm{S}$ for all $\lambda\in \F$;
\item $\norm{S+T} \leq \norm{S}+\norm{T}$;
\item $\norm{ST} \leq \norm{S}\norm{T}$.
\end{enumerate}
\end{proposition}
\begin{proof}
(1) We calculate
\[ \norm{\lambda S} = \sup_{\norm{x}=1}\norm{\lambda Sx} = \sup_{\norm{x}=1}|\lambda|\; \norm{Sx} = |\lambda| \sup_{\norm{x}=1}\norm{Sx} = |\lambda| \norm{S}. \]

(2) We calculate
\[ \norm{S+T} = \sup_{\norm{x}=1}\norm{Sx+Tx} \leq \sup_{\norm{x}=1}\big(\norm{Sx}+\norm{Tx}\big)\leq \sup_{\norm{x}=1}\big(\norm{S} + \norm{Tx}\big) = \norm{S} + \norm{T}. \]

(3) We calculate
\[ \norm{ST} = \sup_{\norm{x}=1}\norm{STx} \leq \sup_{\norm{x}=1}\big(\norm{S}\;\norm{Tx}\big) = \norm{S}\;\norm{T}. \]
\end{proof}
\begin{corollary} \label{BoundedSpace}
Let $(V,\norm{\cdot}_V)$ and $(W,\norm{\cdot}_W)$ be normed spaces. Then the set $\Bounded(V,W)$ of bounded linear maps is a normed subspace of $\Lin(V,W)$ when equipped with the operator norm.
\end{corollary}
\begin{proof}
By \ref{linearMapsVectorSpace}, $\Lin(V,W)$ is a vector space. Closure under addition and scalar multiplication follows from the proposition and \ref{operatorNorm}.

The proposition also immediately gives that the operator norm is a seminorm.

To show that it is in fact a norm, we just need to show that it is point-separating. Assume $S\in\Bounded(V,W)$ is such that $\norm{S} = 0$. Then $\sup_{\norm{x}=1}\norm{Sx} = 0$, which implies that $\norm{Sx} = 0$ for all unit vectors. Because the norm on $W$ is point-separating, this means that $Sx = 0$ for all unit vectors. Now for arbitrary $v\in V$, we have $Sv = \norm{v}\cdot S\left(\frac{v}{\norm{v}}\right) =\norm{v} \cdot 0 = 0$. Thus $S = 0$.
\end{proof}

\begin{proposition} \label{boundedOperatorsFormBanachSpace}
Let $(V,\norm{\cdot}_V)$ and $(W,\norm{\cdot}_W)$ be normed spaces. Then $\Bounded(V,W)$ is a Banach space \textup{if and only if} $W$ is a Banach space.
\end{proposition}
\begin{proof}
$\boxed{\Leftarrow}$ Assume $W$ is a Banach space and take an arbitrary Cauchy sequence $\seq{T_n}$ in $\Bounded(V,W)$. For all $v\in V$, we have
\[ \norm{T_mv - T_nv} \leq \norm{T_m-T_n}\;\norm{v} \to 0, \]
so $\seq{T_nv}$ is a Cauchy sequence in $W$. It has a limit that we denote $Tv$. It is clear that $T$ is linear (by continuity of addition and scalar multiplication).

Fix $\epsilon >0$. Then there exists $N\in\N$ such that $\norm{T_m-T_n}\leq \epsilon$ for all $m,n\geq N$. Take arbitrary $v\in V$ with $\norm{v} = 1$ and $n\geq N$. Then (by continuity of the norm)
\[ \norm{(T-T_n)(v)} = \lim_{m\to\infty}\norm{(T_m-T_n)(v)} \leq \lim_{m\to\infty}\norm{T_m - T_n} \leq \epsilon. \]
As $v$ was chosen arbitrarily, we have $\sup_{\norm{v} = 1}\norm{(T-T_n)(v)} \leq \epsilon$, so $\norm{T-T_n} \leq \epsilon$. Thus $\seq{T_n}$ converges to $T$ in norm.

This also immediately shows that $T$ is bounded:
\[ \sup_{\norm{v}=1}\norm{T(v)}\leq \sup_{\norm{v}=1}\norm{T(v)}\norm{T-T_n} + \sup_{\norm{v}=1}\norm{T(v)}\norm{T_n} = \norm{T-T_n} + \norm{T_n}. \]

$\boxed{\Rightarrow}$ TODO
\end{proof}
Also: TODO make the first part nicer.
\begin{corollary}
Let $V$ be a normed space. The continuous dual $X'$ is a Banach space.
\end{corollary}
\begin{corollary}
Topologically reflexive spaces are Banach spaces.
\end{corollary}

\begin{proposition}[Bounded linear extension] \label{BLT}
Let $T:\dom(T)\subseteq X\to Y$ be a bounded operator between normed spaces. Then $T$ has a unique extension
\[ \widetilde{T}:\overline{\dom(T)}\to Y \]
where $\widetilde{T}$ is a bounded operator with $\norm*{\widetilde{T}} = \norm{T}$.
\end{proposition}
\begin{proof}
Normed vector spaces have the unique extension property because they are Hausdorff, \ref{uniqueExtensionHausdorff}. We just need to show the norm stays the same:

Clearly $\norm*{\tilde{T}} \geq \norm{T}$. For the converse take any $x\in X$. As $\overline{\dom(T)} = X$, there exists a sequence $\seq{x_i}\subset \dom(T)$ that converges to $x$. Then
\[ \norm*{\tilde{T}(x)}_Y = \norm{T\left(\lim_{i\to\infty}x_i\right)}_Y = \lim_{i\to\infty}\norm{T(x_i)}_Y \leq \lim_{i\to\infty}\norm{T}\;\norm{x_i}_X = \norm{T}\;\norm{x}_X. \]
\end{proof}


\begin{proposition}
Let $T\in\Bounded(V,W)$. Then the adjoint $T^*$ is a bounded operator in $\Bounded(W,V)$ with $\norm{T^*} = \norm{T}$.
\end{proposition}
TODO: clean up proof.
\begin{proof}
The operator $T^t$ is linear since $\forall f_1,f_2\in \tdual{W}, \forall a\in\mathbb{F}, \forall x\in V:$
\[ (T^t(af_1 + f_2))(x) = (af_1 + f_2)(Tx) = af_1(Tx) + f_2(Tx) = a(T^tf_1)(x) + (T^tf_2)(x). \]
For the equality of norms, we prove two inequalities. First $\forall x\in V, f\in \tdual{W}$
\[ |f(Tx)|\leq \norm{f}\norm{Tx}\leq \norm{f}\norm{x}\norm{T} \implies \frac{|f(Tx)|}{\norm{x}} \leq \norm{f}\norm{T}. \]
taking the supremum over $x\in V$, we get $\norm{T^tf} = \norm{f\circ T}\leq \norm{f}\norm{T}$ and taking the supremum over $f\in \tdual{W}$ gives $\norm{T^t}\leq \norm{T}$. This shows that $T^t$ is bounded.

For the other inequality, we use corollary \ref{existenceBoundedFunctionalOfSameNorm} to the Hahn-Banach theorem: for every $x\in V$, there exists a bounded functional $\omega_x$ such that $\norm{\omega_x}=1$ and $\omega_x(x) = \norm{x}$. Then we can calculate:
\begin{align*}
\norm{Tx} = \omega_{Tx}(Tx) = (T^t\omega_{Tx})(x) \leq \norm{T^t\omega_{Tx}}\norm{x} \leq \norm{T^t}\norm{\omega_{Tx}}\norm{x} = \norm{T^t}\norm{x}
\end{align*}
So $\norm{T}\leq\norm{T^t}$. Combining gives $\norm{T^t}=\norm{T}$.
\end{proof}
\begin{corollary}
The map $T\mapsto T^*$ is an isometric isomorphism in $(\Bounded(X,Y)\to \Bounded(\dual{Y}, \dual{X}))$.
\end{corollary}

\subsection{The uniform boundedness principle}
TODO: \url{https://arxiv.org/pdf/1005.1585.pdf}

\begin{theorem}[Uniform boundedness principle] \label{uniformBoundednessPrinciple}
Let $\mathcal{F}\subset \Bounded(X,Y)$ be a family of bounded operators where $X$ is a Banach space and $Y$ a normed space, such that
\[ \sup\setbuilder{\norm{Tx}}{T\in\mathcal{F}} < \infty \qquad \text{for all $x\in X$}. \]
Then $\sup\setbuilder{\norm{T}}{T\in\mathcal{F}} < \infty$.
\end{theorem}
\begin{proof}
The proof is an application of the Baire category theorem. Define the closed subsets $K_n$ as
\[ K_n = \setbuilder{x\in X}{\forall T\in\mathcal{F}: \norm{Tx}\leq n}. \]
These are closed because the functional $f_T: X\to \R: x\mapsto \norm{Tx}$ is bounded and
\[ K_n = \bigcap_{T\in\mathcal{F}}f_T^{-1}[\,[0,n]\,]. \]
By assumption, $X=\bigcup_{n\in\N} K_n$. As $X$ is a Banach space, and thus a complete metric space, we can apply the Baire category theorem, \ref{BaireCategory}, to conclude that there is a $K_n$ with non-empty interior (by contraposition of the Baire condition). Take $x_0\in K_n^\circ$, then $-x_0+K_n^\circ \subset K_{n2}$. So $\vec{0}\in (K_{2n})^\circ$ and we can find a $\rho$ such that $B(\vec{0},\rho)\subset K_{2n}$. By proposition \ref{existenceOperatorNorm} we have $\norm{T}\leq 2n/\rho$ for all $T\in\mathcal{F}$.
\end{proof}
\begin{corollary}[Banach-Steinhaus] \label{BanachSteinhaus}
Let $X$ be a Banach space and $Y$ a normed space. Let $T_n: X\to Y$ be a sequence of bounded operators that converges pointwise to $T$. Then
\begin{enumerate}
\item $T$ is bounded;
\item $\seq{T_n}$ converges in the continuous convergence.
\end{enumerate}
\end{corollary}
\begin{proof}
For all $x\in X$ we have $\norm{T_nx} \to \norm{Tx}$, so $\seq{\norm{T_nx}}$ is a bounded sequence and $\sup_{n}\norm{T_nx} <\infty$. By the uniform boundedness principle, $\sup_{n}\norm{T_n}$ is bounded by some constant $M$.

Now for all $x\in X$ we have $\norm{T_nx} \leq M$ for all $n\in\N$ and thus $\norm{Tx}\leq M$. This means that $T$ is bounded and $\norm{T} \leq M$.
\end{proof}
This does not imply that $\seq{T_n}$ converges to $T$ in norm!

\subsection{Open mapping and closed graph theorems}

\begin{proposition} \label{openUnitBall}
Let $X,Y$ be Banach spaces and $T:X\to Y$ a surjective bounded operator.  Then the image of the open unit ball $B(\vec{0},1)\subset X$ contains an open ball about $\vec{0}\in Y$.
\end{proposition}
\begin{proof}
We first prove $0\in \overline{T[B(\vec{0},r)]}^\circ$ for every $r>0$: (TODO: make computations lemma.)
\begin{itemize}
\item Using $X = \bigcup_{n=1}^\infty B(\vec{0},n)$, we see by surjectivity
\[ Y = T[X] = T\left[\bigcup_{n=1}^\infty B(\vec{0},n)\right] = \bigcup_{n=1}^\infty T[B(\vec{0},n)]. \]
Because $Y$ has the Baire property (theorem \ref{BaireCategory}) and $Y$ is both open and non-empty, it may not be meagre, by lemma \ref{BaireEquivalents}. So for some $n\in\N$, $T[B(\vec{0},n)]$ is non-rare, meaning that $\overline{T[B(\vec{0},n)]}$ has non-empty interior.
\item Because
\[ \overline{T[B(\vec{0},n)]} = \overline{2nT[B(\vec{0},1/2)]} = 2n\overline{T[B(\vec{0},1/2)]}, \]
$\overline{T[B(\vec{0},1/2)]}$ must have non-empty interior. Let $B(y_0,\epsilon)\subset \overline{T[B(\vec{0},1/2)]}$.
\item Note $B(0,\epsilon) = y_0 - B(y_0,\epsilon) \subset \overline{T[B(\vec{0},1)]}$ and thus $B(0,r\epsilon) \subset \overline{T[B(\vec{0},r)]}$.
\end{itemize}
We then prove $\overline{T[B(\vec{0},1/2)]} \subset T[B(\vec{0}, 1)]$, proving the proposition.
\begin{itemize}
\item Choose some $y_0\in \overline{T[B(\vec{0},1/2)]}$. Then every neighbourhood $B(y_0,\epsilon/4)$ intersects $T[B(\vec{0},1/2)]$.
\item Then
\[ B(y_0,\epsilon/4) = y_0 - B(\vec{0},\epsilon/4) \subset y_0 - \overline{T[B(\vec{0},1/4)]}, \]
so $y_0 - \overline{T[B(\vec{0},1/4)]}$ intersects $T[B(\vec{0},1/2)]$. Take a $y_1 \in \overline{T[B(\vec{0},1/4)]}$ such that $y_0-y_1$ is in this intersection. Then we have an $x_0\in B(\vec{0},1/2)$ such that $T(x_0) = y_0-y_1$.
\item We can continue recursively choosing $y_{n+1}\in \overline{T[B(\vec{0}, 2^{-(n+1)})]}$ and $x_n \in B(\vec{0}, 2^{-n})$ such that $y_n-y_{n+1} = T(x_n)$.
\item Consider the sequence $\sum_{k=0}^nx_k$. It is a Cauchy sequence in $X$. Call its limit $x$. Then $x\in B(\vec{0},1)$.
\item Because $\norm{y_n}\leq 2^{-n}\norm{T}$, $(y_n)$ converges to zero. Then
\[ \left( T\left(\sum^n_{k=1}x_k\right) \right)_{n\in\N} = \left( y_0-y_{n+1} \right)_{n\in\N} \]
converges to $y_0$. Thus $T(x) = y_0 \in T[B(\vec{0},1)]$.
\end{itemize}
\end{proof}

\begin{proposition} \label{zeroInInteriorOfImageImpliesOpen}
Let $X,Y$ be normed spaces and $T: X\to Y$ a linear map. If $\vec{0}$ lies in the interior of $T[B(\vec{0},r)]$ for some $r>0$, then $T$ is open.
\end{proposition}
\begin{proof}
TODO: make computations lemma.
Given the assumption, $0$ lies in the interior of $T[B(\vec{0},\epsilon)]$ for all $\epsilon>0$.
Because $T[B(x,\epsilon)] = T(x) + T[B(\vec{0},\epsilon)]$, $T(x)$ lies in the interior of $T[B(x,\epsilon)]$, for all $x\in X$.
Thus for all neighbourhoods $U(x)\subset X$, $T(x)\subset T[U]^\circ$ and so $T[U] \subset T[U]^\circ$, so $T[U]$ is open.
\end{proof}

\begin{theorem}[Open mapping]
Let $X,Y$ be Banach spaces and $T:X\to Y$ a surjective bounded operator. Then $T$ is an open map.
\end{theorem}
\begin{proof}
This is the consequence of propositions \ref{openUnitBall} and \ref{zeroInInteriorOfImageImpliesOpen}.
\end{proof}
\begin{corollary}[Bounded inverse theorem] \label{boundedInverse}
Let $X,Y$ be Banach spaces. If $T:X\to Y$ is is continuous, linear and bijective, then $T$ is a homeomorphism.
\end{corollary}


\begin{proposition}
Let $T: \dom(T)\subset X\to Y$ be a bounded linear operator. Then
\begin{enumerate}
\item if $\dom(T)$ is a closed subset of $X$, then $T$ has closed graph;
\item if $T$ has closed graph and $Y$ is complete, then $\dom(T)$ is a closed subset of $X$.
\end{enumerate}
\end{proposition}
\begin{proof}
We use proposition \ref{closedGraphEquivalence} twice: First assume $(x_n)$ and $(Tx_n)$ converge to $x$ and $y$, respectively. Then $x\in\dom(T)$ by closure and $y = Tx$ by continuity.

Now assume $T$ has closed graph and $Y$ is complete. Take $x\in\overline{\dom(T)}$ and $(x_n)\subset \dom(T)$ converging to $x$. Since $T$ is bounded:
\[ \norm{Tx_n - Tx_m} = \norm{T(x_n-x_m)} \leq \norm{T}\norm{x_n-x_m}, \]
so $(Tx_n)$ is Cauchy by \ref{CauchyCriterion} and thus by completeness has a limit, say $y$. Then $Tx=y$ by continuity. Since $T$ has closed graph, $x\in\dom(T)$. So $\overline{\dom(T)}\subseteq \dom(T)$ and $\dom(T)$ is closed. 
\end{proof}

For closed graph theorem, see TVS.



\subsection{Compact operators}
\begin{definition}
A linear map $L:V\to W$ between normed spaces is called \udef{compact} if $L[\overline{B}(\vec{0}, 1)]$ is relatively compact.

i.e.\ the image of the closed unit ball has compact closure.

The space of compact maps from $V$ to $W$ is denoted $\mathcal{K}(V,W)$.
\end{definition}

These operators were introduced to study equations of the form
\[ (T-\lambda I)x(t) = p(t). \]

\begin{proposition}
Let $L\in\Lin(V,W)$. The following are equivalent:
\begin{enumerate}
\item $L$ is compact;
\item the image of any bounded subset of $V$ is relatively compact in $W$;
\item there exists a neighbourhood $U$ of $0$ in $V$ such that the image of $U$ is a subset of a compact set in $W$;
\item for any bounded sequence $(x_n)_{n\in\N} \subseteq V$, then sequence $(Lx_n)_{n\in\N}$ contains a converging subsequence.
\end{enumerate}
\end{proposition}
\begin{proof}
TODO
\end{proof}
\begin{corollary}
All maps of finite rank are compact.
\end{corollary}
\begin{proof}
Closed balls in $\C^n$ are compact.
\end{proof}

\begin{proposition}
Let $V$ be a normed space. Then $\mathcal{K}(V)$ is a closed two-sided ideal in $\Bounded(V)$.
\end{proposition}

\begin{lemma}
The identity map on $X$ is compact \textup{if and only if} $X$ is finite-dimensional.
\end{lemma}
\begin{proof}
The unit ball is compact iff $X$ is finite-dimensional, by \ref{compactnessUnitBall}.
\end{proof}
\begin{corollary}
Let $T\in\Compact(X,Y)$. If $T$ is injective and $T^{-1}$ bounded, then $X$ is finite-dimensional.
\end{corollary}
\begin{proof}
In this case $\id_X = T^{-1}T$ is compact by TODO ref.
\end{proof}

\subsubsection{Compact operators on Banach spaces}
\begin{proposition}
Let $L\in\Hom(V,W)$ with $V,W$ Banach spaces. Then $L$ is compact \textup{if and only if} the image of any bounded subset of $V$ under $L$ is totally bounded in $W$.
\end{proposition}
TODO proof

\begin{lemma} \label{identityCompactFiniteDimensional}
Let $X$ be a Banach space. Then $\id_X$ is compact \textup{if and only if} $X$ is finite dimensional.
\end{lemma}

\begin{theorem}[Schauder's theorem] \label{SchaudersTheorem}
Let $X,Y$ be Banach spaces and $T\in\Bounded(X,Y)$. Then the following are equivalent:
\begin{enumerate}
\item $T: X\to Y$ is compact;
\item $T^*: Y^* \to X^*$ is compact.
\end{enumerate}
\end{theorem}
TODO: more general setting: $X,Y$ normed for $(1) \Rightarrow (2)$ and $X$ normed, $Y$ Banach for $(2)\implies (1)$?
\begin{proof}
TODO \ref{https://arxiv.org/pdf/1010.1298.pdf} and \url{https://math.stackexchange.com/questions/41432/easy-proof-adjointcompact-compact}
\end{proof}

\subsubsection{Calkin algebra}
\begin{proposition}
Let $X$ be a Banach space. Then $\Compact(X)$ is a closed two-sided ideal in $\Bounded(X)$.
\end{proposition}
\begin{proof}
TODO + $*$-ideal for Hilbert spaces.
\end{proof}
\begin{corollary}
An invertible operator $T$ on $X$ is compact \textup{if and only if} $X$ is finite-dimensional.
\end{corollary}
\begin{proof}
If $T$ is compact, then so is $TT^{-1} = \id_X$, meaning that $X$ is finite-dimensional by \ref{identityCompactFiniteDimensional}. Conversely, all operators on a finite-dimensional Banach space are compact.
\end{proof}

\begin{definition}
Let $X$ be a Banach space. The \udef{Calkin algebra} is the quotient $\Bounded(X)/\Compact(X)$.
\end{definition}
TODO: quotient algebra ($[A][B] = [AB]$)

\begin{proposition}
Let $[T]\in\Bounded(X)/\Compact(X)$. Then the following are equivalent:
\begin{enumerate}
\item $[T]$ is invertible in the Calkin algebra;
\item $\exists S\in\Bounded(X):$ both $\vec{1}-TS$ and $\vec{1}-ST$ are compact;
\item $T$ has closed range and finite-dimensional kernel and cokernel. 
\end{enumerate}
\end{proposition}
\begin{proof}
Point 1. and 2. are easily equivalent: $[S]$ is an inverse of $[T]$ if and only if $[\vec{1}] = [S][T] = [ST]$ and $[\vec{1}] = [T][S] = [TS]$. Then
\[ [\vec{1}] = [ST] \iff [ST - \vec{1}] = [0] \qquad [\vec{1}] = [TS] \iff [TS - \vec{1}] = [0] \]
and $[F]=[0]$ if and only if $F$ is compact.

TODO
\end{proof}


\section{Completions and constructions}

\begin{lemma} \label{embeddingInCompletionLinear}
Let $X$ be a normed vector space with completion $\hat{X}$. Then
\begin{enumerate}
\item there exist unique operations of addition and scalar multiplication on $\hat{X}$ that make $\hat{X}$ a Banach space;
\item the embedding $\hat{}: X \hookrightarrow \hat{X}$ is linear.
\end{enumerate}
\end{lemma}
\begin{proof}

\end{proof}


\begin{proposition}
Let $X$ be a normed vector space with completion $\hat{X}$ and $\sSet{Y, \xi}$ a complete vector space. Let $T: X\to Y$ be a bounded linear operator. Then $T$ has a unique continuous extension $\hat{T}: \hat{X} \to Y$ and
\begin{enumerate}
\item $\hat{T}$ is linear;
\item $\norm{\hat{T}} = \norm{T}$.
\end{enumerate}
\end{proposition}
\begin{proof}
We have that $T$ is uniformly continuous by \ref{uniformContinuityGroupHomomorphism}. Thus the continuous extension $\hat{T}$ exists and is unique by \ref{uniformlyContinuousExtensionToCompletion}.

(1) Take $a,b\in \hat{X}$ and $\lambda \in \F$. Then, by \ref{sequentialLemma} and \ref{sequentialInherenceAdherence}, we can find sequences $\seq{a_n}, \seq{b_n}$ such that $\seq{\hat{a}_n}, \seq{\hat{b}_n}$ converge to $a$ and $b$ in the completion.

For all $x,y\in X$, we have $\widehat{x+\lambda y} = \hat{x} + \lambda \hat{y}$ by \ref{embeddingInCompletionLinear}, so
\[ \hat{T}\big(\hat{x} + \lambda \hat{y}\big) = \hat{T}\big(\widehat{x+\lambda y}\big) = T(x+\lambda y) = T(x) + \lambda T(y) = \hat{T}(\hat{x}) + \lambda \hat{T}(\hat{y}). \]

Now we can compute
\begin{align*}
\hat{T}(a+ \lambda b) &= \lim_{n\to \infty}\hat{T}\big(\hat{a_n} + \lambda \hat{b_n}\big) \\
&= \lim_{n\to \infty}\hat{T}(\hat{a_n}) + \lambda \hat{T}(\hat{b_n}) \\
&= \lim_{n\to \infty}\hat{T}(\hat{a_n}) + \lambda \lim_{n\to \infty}\hat{T}(\hat{b_n}) \\
&= \hat{T}(a) + \lambda \hat{T}(\hat{b}).
\end{align*}
This shows linearity.
\end{proof}


\begin{proposition}
The completions of a space with respect to two different norms are isomorphic \textup{if and only if} the norms are equivalent.
\end{proposition}

TODO move down
\subsection{Tensor products}
TODO Ryan
\url{https://math.stackexchange.com/questions/2712906/does-mathcalb-mathcalh-mathcalh-otimes-mathcalh-in-infinite-dime}
\url{https://math.stackexchange.com/questions/35191/operator-norm-and-tensor-norms?noredirect=1&lq=1}

\subsection{Direct sums}
\subsubsection{Algebraic direct sum}

\subsubsection{Topological direct sum}
For arbitrary direct sums we can generalise: now that we have a concept of limits, we can relax the requirement that all but finitely many terms be zero. Instead we require that the sequence of norms is bounded in some way. This gives a whole family of related concepts of direct sum, named for which sequence space the sequence of norms belongs to.
\begin{definition}
Let $\{V_i\}_{i\in I}$ be an arbitrary family of Banach spaces over a field $\F$ and let $\ell(I,\F)$ be a space of sequences in $\F$ indexed by $I$. Then the \udef{$\ell$-direct sum} is the vector space with as field
\[ \bigoplus_{i\in I}^\ell V_i = \setbuilder{(v_i)_{i\in I}}{\forall i\in I: v_i\in V_i \quad\text{and}\quad (\norm{v_i}_{V_i})_{i\in I}\in \ell(I,\F) }. \]
In particular we have, for all $1\leq p<\infty$, the \udef{$\ell^p$-direct sum}
\[ \bigoplus_{i\in I}^p V_i \defeq \setbuilder{(v_i)_{i\in I}}{\forall i\in I: v_i\in V_i \quad\text{and}\quad \sqrt[p]{\sum_{i\in I}\norm{v_i}_{V_i}^p}<\infty} \]
and the \udef{$\ell^\infty$-direct sum}
\[ \bigoplus_{i\in I}^\infty V_i \defeq \setbuilder{(v_i)_{i\in I}}{\forall i\in I: v_i\in V_i \quad\text{and}\quad \sup_{i\in I}\norm{v_i}_{V_i}<\infty}. \]
\end{definition}

\begin{proposition}
For any sequence space that is a Banach space the direct sum is a Banach space. TODO: in particular algebraic direct sum as $c_{00}$? (one possible norm)? and finite direct sums?
\end{proposition}

\subsubsection{Direct sum of identical spaces}
\begin{proposition}
Let $V$ be a Banach space over $\F$, $I$ an arbitrary index set and $\ell(I,\F)$ a banach sequence space.
\[ \bigoplus_{i\in I}^\ell V \cong \ell\otimes V \]
\end{proposition}


\subsection{Quotients of normed spaces}
\begin{proposition}
Let $V$ be a normed space and $U\subseteq V$ a subspace. Then
\[ \norm{x+U}_{V/U} \defeq \inf\setbuilder{\norm{x+u}}{u\in U} \]
is a seminorm on $V/U$. It is a norm \textup{if and only if} $U$ is closed in $V$.
\end{proposition}
\begin{proof}
For absolute homogeneity, take $\lambda\in\F\setminus\{0\}$. Then $\lambda x+U = \lambda(x+U)$, so
\[ \norm{\lambda x}_{V/U} = \inf\setbuilder{\norm{\lambda x+u}}{u\in U} = \inf\setbuilder{|\lambda|\norm{(x+u)}}{u\in U} = |\lambda| \norm{x}_{V/U}. \]
For subadditivity, we have
\begin{align*}
\norm{x+y}_{V/U} &= \inf\setbuilder{\norm{x+y+u}}{u\in U} \\
&= \inf\setbuilder{\norm{x+y+u+v}}{u\in U, v\in U} \\
&\leq \inf\setbuilder{\norm{x+u}+\norm{y+v}}{u,v\in U} \\
&= \norm{x}_{V/U} + \norm{y}_{V/U}.
\end{align*}

Now $\norm{x}_{V/U} = 0$ iff there exists a sequence $\seq{u_n}$ in $U$ such that $u_n\to x$ in $V$. Also $[x]_U = 0$ is equivalent to $x\in U$. Thus point-separation (i.e. $\norm{x}_{V/U} = 0$ implies $[x]_U = 0$) is equivalent to the statement ``if there exists a sequence $\seq{u_n}$ in $U$ such that $u_n\to x$ in $V$, then $x\in U$'', which is equivalent to the statement that $U$ is closed.
\end{proof}

\begin{proposition}
Let $V$ be a normed space and $U\subseteq V$ a subspace. Then
\begin{enumerate}
\item the quotient map $[\cdot]_U$ is bounded and $\norm{[\cdot]_U}\leq 1$;
\item the topology generated by $\norm{\cdot}_{V/U}$ is the quotient topology on $V/U$, i.e.\ the final topology w.r.t.\ the quotient map $[\cdot]_U$;
\item $[\cdot]_U$ is an open map.
\end{enumerate}
\end{proposition}
\begin{proof}
(1) We have
\[ \norm{[x]_U}_{V/U} = \inf\setbuilder{\norm{x+u}}{u\in U} \leq \inf\setbuilder{\norm{x+u}}{u\in \{0\}} = \norm{x+0} = \norm{x}. \]

(2), (3) TODO Conway p.71
\end{proof}

\begin{proposition} \label{quotientBanachSpace}
Let $V$ be a Banach space and $U\subseteq V$ a closed subspace. Then $V/U$ is a Banach space.
\end{proposition}
\begin{proof}
TODO Conway p.71
\end{proof}

\section{Unbounded operators}
Should be: not-necessarily-bounded operators.
\subsection{Operators bounded below}
TODO: also unbounded operators!
\begin{definition}
Let $T$ be a linear operator. We say $T$ is \udef{bounded below} if
\[ \exists b>0:\forall v\in \dom(T): \quad \norm{Tv}\geq b\norm{v} \]
\end{definition}

\begin{proposition} \label{boundedBelow}
Let $T\in \Lin(V, W)$ be an operator. Then $T$ has a bounded inverse $T^{-1}: \im(T)\to V$ \textup{if and only if} $T$ is bounded below by some constant $b$.

In this case
\[ \displaystyle\norm{T^{-1}} = \left(\inf_{x\neq 0}\frac{\norm{Tx}}{\norm{x}}\right)^{-1} \leq \frac{1}{b}. \]
\end{proposition}
\begin{proof}
First assume $T$ bounded below.
To show $T$ is injective, take $x_1,x_2\in \dom T$ such that $Tx_1 = Tx_2$. Then
\[ 0 = \norm{Tx_1 - Tx_2} = \norm{T(x_1 - x_2)} \geq b\norm{x_1 - x_2} \geq 0. \]
So $\norm{x_1 - x_2} = 0$ and thus $x_1=x_2$.
The existence of $T^{-1}$ is then clear. For boundedness notice that $T^{-1}y \in \dom(T)$, so because $T$ is bounded below,
\[ b\norm{T^{-1}y} \leq \norm{TT^{-1}y} = \norm{y} \quad\implies\quad \norm{T^{-1}y} \leq \frac{1}{b}\norm{y}. \]

This also shows that $\norm{T^{-1}} \leq 1/b$ for all lower bounds $b$. In other words $1/\norm{T^{-1}} \geq \inf_{x\neq 0}\norm{Tx}/\norm{x}$.

Now assume $T^{-1}$ bounded. Then for all $x\in\dom(T)$: $\norm{x} = \norm{T^{-1}Tx} \leq \norm{T^{-1}}\norm{Tx}$, so $T$ is bounded below by $1/\norm{T^{-1}}$.

This also shows that $1/\norm{T^{-1}}$ is a lower bound, so $1/\norm{T^{-1}} \leq \inf_{x\neq 0}\norm{Tx}/\norm{x}$.
\end{proof}



\subsection{Closed operators and graph norm}
\begin{definition}
Let $T:\dom(T)\subseteq X\to Y$ be an operator. Then $T$ is a \udef{closed operator} if $\graph(T)$ is closed in $X\oplus Y$.
\end{definition}
This is not the same as a closed map in the topological sense!

\subsubsection{The graph norm}
Let $L:V\to W$ be a linear map between normed spaces. The graph of $L$
\[ \setbuilder{(v,w)\in V\oplus W}{w = Lv} \]
has a natural norm inherited from the direct sum:
\[ \norm{(v,Lv)} = \norm{v}_V + \norm{Lv}_W. \]
This norm can also be seen as a norm on $V$: the \udef{graph norm} induced by $L$ is defined as
\[ \norm{v}_L := \norm{v}_V + \norm{Lv}_W. \]

\begin{proposition}
Let $L: \sSet{V, \norm{\cdot}_V}\to \sSet{W, \norm{\cdot}_W}$ be a linear map between normed spaces. Then $L: \sSet{V, \norm{\cdot}_L}\to \sSet{W, \norm{\cdot}_W}$ is bounded with norm $K$.
Also
\begin{enumerate}
\item $K \leq 1$;
\item $K < 1$ \textup{if and only if} $L: \sSet{V, \norm{\cdot}_V}\to \sSet{W, \norm{\cdot}_W}$ is bounded.
\end{enumerate}
\end{proposition}
\begin{proof}
Take any $v\in V$. Then
\[ \norm{L(v)}_W \leq \norm{L(v)}_W + \norm{v}_V = \norm{v}_L. \]
This shows the $L$ is bounded and the norm is less than or equal to $1$.

Now we can write
\[ K = \sup_{v\in V\setminus\{0\}}\frac{\norm{L(v)}_W}{\norm{v}_L} = \sup_{v\in V\setminus\{0\}}\frac{\norm{L(v)}_W}{\norm{v}_V + \norm{L(v)}_W} \defeq \sup_{v\in V\setminus\{0\}}K_v, \]
where we have set $K_v \defeq \frac{\norm{L(v)}_W}{\norm{v}_V + \norm{L(v)}_W}$.

First assume $L: \sSet{V, \norm{\cdot}_V}\to \sSet{W, \norm{\cdot}_W}$ is bounded. Then, from $\norm{L_v}_W = K_v \big(\norm{L(v)}_W + \norm{v}_V\big)$, we calculate
\[ K_v\norm{v}_V = \norm{L(v)}_W(1-K_v) \leq \norm{L}\;\norm{v}_V(1-K_v), \]
which implies $K_v \leq \norm{L}(1-K_v)$. This can be written as $K_v \leq \frac{\norm{L}}{1+ \norm{L}}$.
Thus $K = \sup_v K_v \leq \frac{\norm{L}}{1+ \norm{L}} < 1$.

Now assume $L: \sSet{V, \norm{\cdot}_V}\to \sSet{W, \norm{\cdot}_W}$ is unbounded. We write
\[ K_v = \frac{\norm{L(v)}_W}{\norm{v}_V + \norm{L(v)}_W} = \frac{\norm{L(v)}_W + \norm{v}_V - \norm{v}_V}{\norm{v}_V + \norm{L(v)}_W} = 1 - \frac{\norm{v}_V}{\norm{v}_V + \norm{L(v)}_W}. \]
We can find a sequence $\seq{v_n}$ of unit vectors such that $\norm{L(v_n)}_W \to \infty$. Then $K_{v_n} = 1 - \frac{1}{1+\norm{L(v_n)}_W} \to 1$. Thus
\[ K = \sup_{v\in V}K_v \geq \sup_{v\in \seq{v_n}}K_v \geq \limsup_{n}K_{v_n} \geq \lim_{n}K_{v_n} = 1, \]
so $K = 1$.
\end{proof}


\begin{lemma} \label{graphNormConvergenceLemma}
Let $T: X\not\to Y$ be an operator between normed spaces and $\seq{x_n}$ a sequence in $\dom(T)$. Then the following are equivalent:
\begin{enumerate}
\item $x_n \overset{\norm{\cdot}_T}{\longrightarrow} x$;
\item $(x_n, Tx_n) \overset{\norm{\cdot}_{X\oplus Y}}{\longrightarrow} (x, Tx)$;
\item $x_n\overset{\norm{\cdot}_X}{\longrightarrow} x$ and $Tx_n\overset{\norm{\cdot}_Y}{\longrightarrow} Tx$.
\end{enumerate}
\end{lemma}
\begin{proof}
We have the equivalences
\begin{align*}
x_n \overset{\norm{\cdot}_T}{\longrightarrow} x &\iff \norm{x_n - x}_T \longrightarrow 0 \\
&\iff \norm{x_n - x}_X + \norm{Tx_n - Tx}_Y \longrightarrow 0 \\
&\iff \norm{(x_n - x, Tx_n - Tx)}_{X\oplus Y} \longrightarrow 0 \\
&\iff \norm{(x_n, Tx_n) - (x, Tx)}_{X\oplus Y} \longrightarrow 0 \\
&\iff (x_n, Tx_n) \overset{\norm{\cdot}_{X\oplus Y}}{\longrightarrow} (x, Tx).
\end{align*}
Now if $x_n\overset{\norm{\cdot}_X}{\longrightarrow} x$ and $Tx_n\overset{\norm{\cdot}_Y}{\longrightarrow} Tx$, then $\norm{x_n - x}_X + \norm{Tx_n - Tx}_Y \longrightarrow 0$. 

Conversely, from
\[ 0 \leq \norm{x_n - x}_X\leq \norm{x_n - x}_X + \norm{Tx_n - Tx}_Y, \]
we get $x_n\overset{\norm{\cdot}_X}{\longrightarrow} x$ by the squeeze theorem (TODO ref). We similarly get $Tx_n\overset{\norm{\cdot}_Y}{\longrightarrow} Tx$.
\end{proof}
\begin{corollary}
The graph norm is strong than then original norm. Both norms are equivalent on $\dom(T)$ \textup{if and only if} $T$ is bounded.
\end{corollary}
\begin{corollary}
Let $T: X\not\to Y$ be an operator between normed spaces. Then the topology induced by the graph norm is equal to the initial topology w.r.t. $\{\id_X: X\to \sSet{X,\norm{\cdot}_X}, T\}$.
\end{corollary}

\subsubsection{Closed operators}

The most important property of closed operators is given by the following proposition. It is sometimes taken as the definition.
\begin{proposition} \label{closedGraphEquivalence}
Let $X,Y$ be normed spaces and $T: \dom(T)\subset X \to Y$ be a linear operator. Then
the following are equivalent:
\begin{enumerate}
\item $T$ is a closed operator;
\item if $(x_n)_{n\in\N}\subset \dom(T)$ converges to $x\in X$ and $(Tx_n)_{n\in\N}$ converges to $y$, then $x\in\dom(T)$ and $Tx = y$;
\item $\dom(T)$ is complete w.r.t. the graph norm.
\end{enumerate}
\end{proposition}
TODO:  (? If domain is closed?) I.e. does this work outside the realm of Banach operators??
\begin{corollary}
All bounded operators have closed graph. (? If domain is closed?)
\end{corollary}
The converse is not true in general.

\url{https://en.wikipedia.org/wiki/Unbounded_operator#Closed_linear_operators}
\url{https://en.wikipedia.org/wiki/Closed_graph_theorem_(functional_analysis)}

\begin{proposition} \label{algebraClosedOperators}
Let $T$ be a closed and $S$ a bounded operator, then
\begin{enumerate}
\item $S+T$ is closed;
\item $TS$ is closed;
\item if $T$ is injective, then $T^{-1}: \im(T) \to \dom(T)$ is closed.
\end{enumerate}
\end{proposition}
\begin{proof}
(1) TODO

(2) TODO

(3) We use \ref{closedGraphEquivalence}. Take $\seq{y_n}\subset \dom(T^{-1})$ such that $y_n\to y$ and $T^{-1}y_n\to x$. Set $x_n = T^{-1}y_n$, so then $Tx_n = TT^{-1}y_n = y_n\to y$. Because $T$ is closed it follows that $Tx = y$, so $T^{-1}y = x$, meaning $T^{-1}$ is closed.
\end{proof}
TODO example $ST$ need not be closed.

\begin{lemma} \label{closedOperatorKernelClosed}
Let $T$ be a closed operator, then $\ker(T)$ is closed.
\end{lemma}
\begin{proof}
Let $\seq{x_n}\subset\ker(T)$ be a convergent sequence. Then $\seq{Tx_n}$ is identically zero and thus converges to $0$. By closedness of $T$, $Tx = 0$ and thus $x\in\ker(T)$. 
\end{proof}
\begin{proof}[Alternate proof.]
Let $T: X\to Y$. Then $\ker(T)\times\{0\} = \graph(T)\cap X\times\{0\}$. As $\graph(T)$ is closed and $X\times\{0\}$ is closed by \ref{productOpenClosed}, we have that $\ker(T)$ is closed by \ref{productOpenClosed}.
\end{proof}
We have already proven this for bounded operators, see \ref{kerClosed}.

\begin{proposition} \label{boundedBelowClosedRange}
Let $T\in \Lin(X, Y)$ be a closed operator between Banach spaces that is bounded below. Then $\im(T)$ is closed.
\end{proposition}
\begin{proof}
Let $T$ be bounded below by $b$ and let $\seq{Tx_n}$ be a Cauchy sequence in $\im(T)$. Then $\norm{x_m - x_n} \leq \frac{1}{b}\norm{T(x_m - x_n)}$, so $\seq{x_n}$ is also Cauchy by \ref{CauchyCriterion}.

So we can find $x\in X,y\in Y$ such that $x_n\to x$ and $Tx_n\to y$. By closedness of $T$, we have $Tx = y$ and thus $y\in\im(T)$.
\end{proof}

\begin{proposition}
Let $X,Y$ be Banach spaces and $S,T\in \Lin(X,Y)$ with $\dom(S) = \dom(T)$. If $S$ is a closed operator and there exist $\alpha,\beta,\gamma\in \R^+$ such that $0 < \gamma \leq 1$ and $\beta < 1/\gamma$ and
\[ \norm{(S-T)u} \leq \alpha \norm{u} + \beta\norm{Su}^\gamma{u}^{1-\gamma} \qquad\text{for all $u\in \dom(S) = \dom(T)$,} \]
then $T$ is also closed.
\end{proposition}
\begin{proof}
TODO Jeribi.
\end{proof}

\subsubsection{Closable operators}
\begin{definition}
A linear operator is called \udef{closable} if it has closed extension.
\end{definition}

\begin{proposition} \label{closableCriterion}
A linear operator $T$ is closable \textup{if and only if} for all sequences $\seq{x_n}\subset\dom(T)$
\[ \left(x_n\to 0 \land T(x_n)\to v\right) \quad\implies\quad v = 0. \]
\end{proposition}
\begin{proof}
TODO
\end{proof}

\begin{lemma}
A closable operator $T$ has a minimal closed extension $\overline{T}$, which is given by the closure of the graph of $T$.
\end{lemma}
\begin{proof}
TODO
\end{proof}

\begin{lemma} \label{domImClosureOperator}
Let $T$ be a closable operator. Then
\begin{enumerate}
\item $\dom(T)$ is dense in $\dom(\overline{T})$;
\item $\im(T)$ is dense in $\im(\overline{T})$.
\end{enumerate}
\end{lemma}
\begin{proof}
We have $(x,\overline{T}x)\in\graph(\overline{T})$ iff there exists a sequence $\seq{x_n}$ in $\dom(T)$ such that $\seq{x_n,Tx_n}\overset{\norm{\cdot}_{X\oplus Y}}{\longrightarrow} (x,\overline{T}x)$. So we can conclude using \ref{graphNormConvergenceLemma}.
\end{proof}

\subsubsection{Domain and core}
\begin{definition}
Let $T: X\not\to Y$ be a closed operator between normed spaces and $D\subseteq \dom(T)$ a subspace. We call $D$ a \udef{core} or \udef{essential domain} for $T$ if $\setbuilder{(x,Tx)}{x\in D}$ is dense in $\graph(T)\subseteq X\oplus Y$.
\end{definition}

\begin{proposition} \label{operatorCoreCriterion}
Let $T: X\not\to Y$ be a closed operator between normed spaces and $D\subseteq \dom(T)$ a subspace. Then $D$ is a core of $T$ \textup{if and only if} $D$ is dense in $\dom(T)$ w.r.t. the graph norm $\norm{\cdot}_T$ of $T$.
\end{proposition}
Note that the norm is bounded by the graph norm, so the graph norm topology is finer than the norm topology by \ref{normComparison}. Thus $\closure_{\norm{\cdot}_T}(D) \subseteq \closure_{\norm{\cdot}}(D)$ and it is not enough for $D$ to be norm dense in $\dom(T)$.
\begin{proof}
Immediate by \ref{graphNormConvergenceLemma}.
\end{proof}






\chapter{Differentiation}
\url{file:///C:/Users/user/Downloads/978-1-4614-3894-6.pdf}
\url{file:///C:/Users/user/Downloads/2011_Bookmatter_TheRicciFlowInRiemannianGeomet.pdf}

\section{Derivatives of functions between normed groups}
\begin{definition}
Let $G, H$ be normed groups, $f:G\to H$ a function and $x_0\in G$. We call $f$ \udef{differentiable} if there exists a continuous homomorphism $A_{x_0}$ such that
\[ \lim_{x\to 1}\frac{\norm{f(xx_0)f(x_0)^{-1}A_{x_0}(x)^{-1}}}{\norm{x}} = 0. \]
We call $A_{x_0}$ a \udef{derivative} of $f$ at $x_0$.
\end{definition}

\begin{proposition}
Let $G, H$ be normed groups, $f:G\to H$ a function and $x_0\in G$. There exists at most one derivative of $f$ at $x_0$.
\end{proposition}
\begin{proof}
TODO
\end{proof}

\subsection{Fréchet derivatives on normed vector spaces}

\section{Directional derivatives}

\section{For real normed vector spaces}
TODO: directional / Gateaux derivative for locally convex TVSs?
\subsection{Directional derivatives}
\begin{definition}
Let $V,W$ be normed vector spaces and $f:U\subseteq V\to W$ a function defined on an open subset $U$. For $a,u\in V$, we call
\[ \partial_u f|_a \defeq \lim_{t\to 0} \frac{f(a+tu) - f(a)}{t} \]
the \udef{directional derivative} of $f$ at $a$ in the direction $u$,if it exists.

\begin{itemize}
\item If $V= \R^n$, then we define $\pd{f}{x^i}f \defeq \partial_{\vec{e}_i}f$, where $\mathcal{E} = \seq{\vec{e}_i}_{i=1}^n$ is the standard basis of $\R^n$. These directional derivatives are called the \udef{partial derivatives} w.r.t. the basis $\mathcal{E}$.
\item If $V = \R$, then there is, up to scalar multiplication, only one direction $u$. We denote the directional derivative $f'(a) \defeq \partial_u f|_a$.
\end{itemize}
\end{definition}
For a given function $f:V\to W$, the directional derivative is a partial function of both a direction and a point:
\[ (V\times V) \not\to W:\quad (u,a) \mapsto \partial_u f(a)  \]

Partial application in the first argument gives a function
\[ \partial_u f:\; V\not\to W:\; a\mapsto \partial_u f(a) \defeq \partial_u f|_a \]
that is also referred to as the \udef{directional derivative} of $f$ in the direction $u$.

\begin{lemma}
Let $f,g: V\to W$, $u\in V$ and $\lambda\in\F$, then
\begin{enumerate}
\item $\partial_u(f+g) = \partial_uf + \partial_u g$;
\item $\partial_u(fg) = (\partial_uf)g + f(\partial_u g)$;
\item $\partial_u(\lambda f) = \lambda \partial_uf$.
\end{enumerate}
\end{lemma}

\begin{proposition} \label{derivativeBilinearForm}
Let $B: V_1 \oplus V_2 \to W$ be a bilinear form. Then, for $(x,y),(a,b)\in V_1\oplus V_2$
\[ \partial_{(x,y)}B|_{(a,b)} = B(x,b) + B(a,y). \]
\end{proposition}
\begin{proof}
We calculate
\begin{align*}
\partial_{(x,y)}B|_{(a,b)} &= \lim_{t\to 0} \frac{B(a+tx, b+ty) - B(a,b)}{t} \\
&= \lim_{t\to 0} \frac{1}{t} (B(a,b) + tB(a,y) + tB(x,b) + t^2B(x,y) - B(a,b)) \\
&=B(x,b) + B(a,y) + \lim_{t\to 0} tB(x,y) \\
&= B(x,b) + B(a,y).
\end{align*}
\end{proof}

\subsubsection{Partial derivatives}
TODO notation $D^\alpha$ for multiindex $\alpha$. Also $|\alpha| = \sum_i \alpha_i$.

\subsubsection{Gateaux derivative}
\begin{definition}
Partial application of the directional derivative in the second argument gives a function
\[ \diff{_af}: V\not\to W: u\mapsto \diff{_af}(u) \defeq \partial_u f|_a = \lim_{t\to 0} \frac{f(a+tu) - f(a)}{t} \]
that is referred to as the \udef{Gateaux differential} of $f$ at the point $a$.

If $\diff{_af}: V\not\to W$ is a bounded linear map, we will refer to it as the \udef{Gateaux derivative}.
\end{definition}
The Gateaux differential is homogeneous even if it is not linear:
\begin{lemma}
Let $f:V\to W$ be a function between normed spaces and $a,u\in V$. If $\partial_u f$ is defined at $a$, then
\[ \diff{_a f}(\lambda u) = \partial_{\lambda u}f(a) = \lambda\partial_u f(a) = \lambda \diff{_a f}(u) \qquad \forall \lambda\in\F. \]
\end{lemma}
\begin{proof}
$\partial_{\lambda u}f(a) = \lim_{t\to 0} \frac{f(a+t\lambda u) - f(a)}{t} = \lim_{t\lambda\to 0} \frac{f(a+t\lambda u) - f(a)}{t \lambda / \lambda} = \lambda\partial_u f(a)$.
\end{proof}

TODO mean value theorem?

\subsection{Hadamard derivative}

\subsection{Fréchet derivative}
\begin{definition}
If a function has a (bounded linear) Gateaux derivative at $a$ and the limit in the definition of the derivative
\[ \diff{_af}: V\not\to W: u\mapsto \diff{_af}(u) \defeq \partial_u f|_a = \lim_{t\to 0} \frac{f(a+tu) - f(a)}{t} \]
is uniform in all $u$ on the $S(\vec{0},1)$, then we say the function is \udef{(Fréchet) differentiable} at $a$ and has \udef{Fréchet derivative} $\diff{_af}$.

We may also write $\diff{f}$, leaving the $a$ implicit.
\end{definition}

\begin{proposition}
Let $V,W$ be normed vector spaces and $f:U\subseteq V\to W$ a function defined on an open subset $U$. Let $a\in V$.

Then $f$ is Fréchet differentiable at $a$ \textup{if and only if} there exists a bounded linear map $A: V\to W$ such that $f(a+x)$ can be written as
\[ f(a+x) = f(a) + A(x) + o(x) \qquad \text{as} \qquad x\to 0. \]
In this case $A = \diff{_af}$.
\end{proposition}
\begin{proof}
First assume $f$ is Fréchet differentiable at $a$. Then
\begin{multline*}
\forall \varepsilon>0:\exists \delta>0: \; \forall u\in S(\vec{0},1): \forall t\in\R: \; t< \delta \implies \varepsilon > \\ \norm{\frac{f(a+tu) - f(a)}{t} - \diff{_af}(u)} = \frac{\norm{f(a+tu) - f(a)- \diff{_af}(tu)}}{|t|} = \frac{\norm{f(a+tu) - f(a)- \diff{_af}(tu)}}{\norm{tu}}.
\end{multline*}

Now each vector $x$ in $V$ can be written as $tu$ for some $t\in\R$ and $u\in S(\vec{0},1)$, so this can be written as
\[ \forall \varepsilon>0:\exists \delta>0: \; \forall x\in V: \; \norm{x}< \delta \implies  \varepsilon > \frac{\norm{f(a+x) - f(a)- \diff{_af}(x)}}{\norm{x}} \]
which is exactly the statement $f(a+x) = f(a) + \diff{_af}(x) + o(x)$ as $x\to 0$.

The logic can be reversed to obtain the equivalence.
\end{proof}

\begin{proposition}
If a function is Fréchet differentiable at a point $a$, then it is continuous at $a$.
\end{proposition}
\begin{proof}
Assume $f$ is has Fréchet derivative $A$. Then
\[ 0 = \lim_{x\to a} \norm{f(x) - f(a) - \diff{_af}(x-a)} = \norm{\lim_{x\to a}f(x) - f(a) - \diff{_af}(\lim_{x\to a} x-a)} = \norm{\lim_{x\to a}f(x) - f(a)}. \]
\end{proof}

\begin{lemma}
The Fréchet derivative is the same for equivalent norms.
\end{lemma}

\subsubsection{Link with Gateaux derivative}
\url{https://link.springer.com/content/pdf/bbm%3A978-3-642-16286-2%2F1.pdf}
\url{http://www.m-hikari.com/ams/ams-password-2008/ams-password17-20-2008/behmardiAMS17-20-2008.pdf}
\begin{proposition}
If a function between subsets of normed spaces is Fréchet differentiable, it is also Gateaux differentiable and the Fréchet derivative is equal to the Gateaux derivative.
\end{proposition}
\begin{proof}
Let $A$ be the Fréchet derivative of $f: U\subseteq V\to W$. Then for all $u\in V$
\begin{align*}
0 &= \lim_{t\to 0} \frac{\norm{f(a+tu) - f(a) - A(tu)}}{\norm{tu}} = \lim_{t\to 0} \frac{\norm{(f(a+tu) - f(a))/t - A(u)}}{\norm{u}} \\
&= \frac{\norm{\lim_{t\to 0}(f(a+tu) - f(a))/t - A(u)}}{\norm{u}} = \frac{\norm{\diff{_af}(u) - A(u)}}{\norm{u}}. 
\end{align*}
\end{proof}
For this reason we will also denote the Fréchet derivative of $f$ at $a$ as $\diff{_a f}$. We will sometimes also write $f'(a)$.

\begin{example}
TODO!

There are functions that have a Gateaux derivative, but not a Fréchet derivative at certain points. For example
\[ f: \R^2\to \R: (x,y) \mapsto \begin{cases}
\frac{xy}{x^2+y^2} & (x,y)\neq (0,0) \\0 & (x,y) = (0,0)
\end{cases} \]
which has $\partial_{\vec{u}}f(\vec{0}) = 0$ for all $\vec{u}\in \R^2$ and thus the Gateaux derivative at zero is $\diff{f} = 0$.

Composing $f$ with $t\mapsto (t,t^2)$ yields the function $t\mapsto \begin{cases}
t^{-2} & t\neq 0 \\ 0 & t=0
\end{cases}$, which is not continuous at $0$. So $f$ is not continuous at zero and a fortiori is not Fréchet differentiable.
\end{example}

\begin{proposition}
If there exists a basis $\beta$ of $V$ such that the partial derivatives of $f:U\subseteq V\to W$ w.r.t. $\beta$ exist and are continuous in $a\in V$, then $f$ is Fréchet differentiable in $a$.
\end{proposition}
\begin{proof}

\end{proof}
TODO for finite dimensions! Expand to criterion for Gateaux to Fréchet.
\begin{example}

\end{example}

\subsubsection{The Jacobian}
\begin{definition}
Let $f:U\subseteq\R^m\to\R^n$ be a function. Then $A_{\diff{f}}$ is a matrix with
\[ [A_{\diff{f}}]_{ij} = [\diff{f}\vec{e}_j]_i = \left[\pd{f}{x^j}\right]_i. \]
This matrix is called the \udef{Jacobian} $J_f$.
\end{definition}


\subsection{Differentiation in a convergence algebra}

\begin{proposition}[Leibniz rule]
Let $A$ be a normed algebra and $a,b\in (\R \to A)$ elements that have derivatives. Then
\[ (ab)' = a'b + ab'. \]
\end{proposition}
\begin{proof}
We calculate
\begin{align*}
0 &= 0\cdot a'(t)b'(t) = \lim_{\epsilon \to 0} \epsilon a'(t)b'(t) \\
&= \lim_{\epsilon \to 0} \epsilon \frac{a(t+\epsilon) - a(t)}{\epsilon}\frac{b(t+\epsilon) - b(t)}{\epsilon} \\
&= \lim_{\epsilon \to 0}\frac{a(t+\epsilon)b(t+\epsilon) - a(t+\epsilon)b(t) - a(t)b(t+\epsilon) + a(t)b(t)}{\epsilon} + \frac{a(t)b(t)}{\epsilon} - \frac{a(t)b(t)}{\epsilon} \\
&= \lim_{\epsilon \to 0} \frac{a(t+\epsilon)b(t+\epsilon) - a(t)b(t)}{\epsilon} - \frac{a(t+\epsilon) - a(t)}{\epsilon}b(t) - a(t)\frac{b(t+\epsilon) - b(t)}{\epsilon} \\
&= (ab)' - a'b - ab'.
\end{align*}
\end{proof}

\begin{proposition} \label{derivativeIdempotent}
Let $A$ be an algebra and $p\in A$ such that $p^2 = p$ and $p'$ exists. Then
\begin{enumerate}
\item $p' = pp'+ p'p$;
\item $pp'p = 0$;
\item $(p')^2 = p'pp' + p(p')^2p$;
\item $p^{\prime\prime} = 2(p')^2 + pp^{\prime\prime} + p^{\prime\prime}p$;
\item $pp^{\prime\prime}p = -2p(p')^2p$.
\end{enumerate}
\end{proposition}
\begin{proof}
(1) We calculate $p' = (p^2)' = pp'+ p'p$.

(2) Multiply (1) by $p$ on the left and right.

(3) We calculate $(p')^2 = (pp'+ p'p)(pp'+ p'p) = pp'pp' + pp'p'p + p'ppp' + p'pp'p = 0p + pp'p'p + p'pp' + p'0$.

(4) Take derivative of (1).

(5) Multiply (3) by $p$ on the left and right.
\end{proof}
\begin{corollary}
Let $\Tr$ be a trace functional on $A$ and $p\in A$ as before. Then $\Tr(p') = 0$.
\end{corollary}
\begin{proof}
$\Tr(p') = \Tr(pp'+ p'p) = \Tr(p^2p')+ \Tr(p'p^2) = \Tr(pp'p) + \Tr(pp'p) = 2Tr(0) = 0$.
\end{proof}
\begin{proposition}
Let $p_0,p_1$ be differentiable idempotents such that $p_0p_1 = 0 = p_1p_0$. Then $p_0'p_1 = -p_0p_1'$.
\end{proposition}
\begin{proof}
We have $0 = p_0p_1$, so $0 = 0' = p_0'p_1 + p_0p_1'$.
\end{proof}
\begin{corollary}
Let $p_0,p_1$ be differentiable idempotents such that $p_0p_1 = 0 = p_1p_0$. Then
\begin{enumerate}
\item $p_1p_0'p_0 = -p_1p_1'p_0$;
\item $p_1p_0'p_1 = 0$;
\item $p_1(p_0')^2p_1 = p_1p_1'p_0p_1'p_1$;
\item $p_0(p_0')^2p_1 = 0$;
\item $p_1p^{\prime\prime}_0p_1 = 2p_1(p_0')^2p_1$.
\end{enumerate}
If in addition $p_2$ is a differentiable idempotent such that $p_0p_2 = 0 p_2p_0$ and $p_1p_2 = 0 p_2p_1$, then
\begin{enumerate} \setcounter{enumi}{5}
\item $p_1p_0'p_2 = 0$.
\end{enumerate}
\end{corollary}
\begin{proof}
(1) We have
\[ p_1p_0'p_0 = p_1(p_1p_0')p_0 = -p_1(p_1'p_0)p_0 = -p_1p_1'p_0. \]

(2) We have $p_1p_0'p_1 = -(p_1'p_0)p_1 = -p_1'(p_0p_1) = 0$.

(3) We have $p_1(p_0')^2p_1 = (p_1p_1p_0')(p_0'p_1p_1) = (p_1p_1'p_0)(p_0p_1'p_1)$.

(4) We have $p_0(p_0')^2p_1 = (p_0p_0')(p_0'p_1) = -(p_0p_0')(p_0p_1') = -(p_0p_0'p_0)p_1' = 0$.

(5) We have, using \ref{derivativeIdempotent}, $p_1p^{\prime\prime}_0p_1 = p_1(2(p'_0)^2 + p_0p_0^{\prime\prime} + p_0^{\prime\prime}p_0)p_1 = 2p_1(p_0')^2p_1$.

(6) We have $(p_1p_0')p_2 = - (p_1'p_0)p_2 = - p_1'(p_0p_2) = 0$.
\end{proof}
\begin{corollary} \label{derivativeIdempotentOffDiagonal}
Let $p_0,p_1$ be differentiable idempotents such that $p_0p_1 = 0 = p_1p_0$ and $p_0 + p_1 = 1$. Then
\begin{enumerate}
\item $p_0' = p_0p_0'p_1 + p_1p_0'p_0$;
\item $p_0p_0' = p_0'p_1$ and $p_0'p_0 = p_1p_0'$.
\end{enumerate}
\end{corollary}
\begin{proof}
(1) We have
\[ p_0' = (p_0 + p_1)p_0'(p_0 + p_1) = \cancel{p_0p_0'p_0} + p_0p_0'p_1 + p_1p_0'p_0 + \cancel{p_1p_0'p_1}. \]

(2) We have, using point (1),
\[ p_0p_0' = p_0(p_0p_0'p_1) + p_0(p_1p_0'p_0) = p_0p_0'p_1 = (p_0p_0'p_1)p_1 + (p_1p_0'p_0)p_1 = p_0'p_1. \]
The other equation is similar.
\end{proof}


\begin{proposition}
Let $A$ be a convergence algebra and $a: U\subseteq\C \to A$.

If $a$ is differentiable and $a(t)^{-1}$ exists for all $t\in \R$, then
\[ \od{a^{-1}}{t} = -a^{-1}a'a^{-1}. \]
\end{proposition}
\begin{proof}
We have
\begin{align*}
\od{}{t}a^{-1} &= \lim_{h\to 0}\frac{a(t+h)^{-1} - a(t)^{-1}}{h} \\
&= \lim_{h\to 0} -\frac{a(t)^{-1}a(t+h)a(t+h)^{-1} - a(t)^{-1}a(t)a(t+h)^{-1}}{h} \\
&= \lim_{h\to 0} -\frac{a(t)^{-1}\Big(a(t+h) - a(t)\Big)a(t+h)^{-1}}{h} \\
&= -a(t)^{-1}\left(\lim_{h\to 0}\frac{\big(a(t+h) - a(t)\big)a(t+h)^{-1}}{h}\right)\Big(\lim_{h\to 0}a(t+h)^{-1}\Big) \\
&= -a(t)^{-1}a'a^{-1}.
\end{align*}
\end{proof}

\section{Analytic functions}
TODO: multiindex notation
\begin{definition}
Let $M$ be a convergence module over a ring $R$ and $n\in\N$. A function $f: R^n\to M$ is called \udef{analytic} if
\[ f(x) = \sum_{I\in\N^n}c_I(x-x_0)^I \]
where $c_I\in M^n$ and $x_0\in R^n$.
\end{definition}

\subsection{Taylor expansion}
Radius of convergence

\subsection{Properties of analytic functions}
\begin{proposition}
Let $f$ be an analytic function. Then
\begin{enumerate}
\item $f$ is continuous;
\item $f$ is differentiable.
\end{enumerate}
\end{proposition}

\section{Classification of spaces}
\begin{definition}
Let $X,Y$ be subsets of normed vector spaces and $X$ be open. We call a function $f: X\to Y$
\begin{itemize}
\item \udef{smooth} at $x_0\in V$ if all derivatives of $f$ at $x_0$ exist;
\item \udef{analytic} at $x_0\in V$ if the Taylor series of $f$ at $x_0$ exists and has non-zero radius of convergence.
\end{itemize}
\end{definition}
\begin{lemma}
Let $f: X\to Y$ be a smooth function. Then all derivatives are continuous.
\end{lemma}

\begin{definition}
Let $X,Y$ be subsets of normed vector spaces and $X$ be open.
\begin{itemize}
\item $\cont^r(X,Y)$ is the space of functions in $(X \to Y)$ whose first $r$ derivatives exist and are continuous;
\item $\cont^\infty(X,Y)$ is the space of functions in $(X \to Y)$ that are smooth at all points in $X$;
\item $\cont^\omega(X,Y)$ is the space of functions in $(X \to Y)$ that are analytic at all points in $X$.
\end{itemize}
If $Y = \C$, we write $\cont^r(X), \cont^\infty(X)$ and $\cont^\omega(X)$. We can also use subscripts $_0$ and $_c$ to denote the extra conditions of vanishing at infinity and having compact support.
\end{definition}







\chapter{Banach algebras}
In this part we set $\F \in \{\R, \C\}$. Usually operator algebras are assumed to be complex. We will attempt to give results for real algebras where possible.
\begin{definition}
A \udef{normed algebra} is an associative algebra $A$ over $\F$ with norm $\norm{\cdot}$ such that $(\F, A,+, \norm{\cdot})$ is a normed space and we have \udef{submultiplicativity}, i.e.
\[ \forall x,y\in A: \quad \norm{xy}\leq\norm{x}\norm{y}. \]
We say $A$ is \udef{unital} if there exists a unit element $\vec{1}\in A$ such that
\[ \forall x\in A: \vec{1}\cdot x = x = x\cdot \vec{1} \qquad \text{and} \qquad \norm{\vec{1}} = 1. \]
\end{definition}
TODO: which results also hold for normed algebras?
\begin{definition}
A \udef{Banach algebra} is a normed algebra that is also a Banach space.
\end{definition}

\begin{proposition}
Let $X,Y$ be normed spaces. Then $\Bounded(X,Y)$ is a unital normed algebra.

The algebra $\Bounded(X,Y)$ is a Banach algebra \textup{if and only if} $Y$ is a Banach space.
\end{proposition}
\begin{proof}
We have $\Bounded(X,Y)\subseteq \Lin(X,Y)$. which is an algebra by \ref{linearMapsAlgebra}. Closure under multiplication follows from \ref{operatorNormIsNorm} and \ref{existenceOperatorNorm}. Submultiplicativity is also given by \ref{operatorNormIsNorm}.

The algebra is unital because the identity operator is bounded.

The condition for $\Bounded(X,Y)$ to be a Banach algebra is given by \ref{boundedOperatorsFormBanachSpace}.
\end{proof}

\begin{lemma} \label{multiplicationContinuous}
Let $A$ be a Banach algebra. The multiplication map $\cdot: A\times A \to A: (x,y)\mapsto xy$ is continuous.
\end{lemma}
\begin{proof}
Because $A\times A$ is a metric space, we can combine \ref{sequentialContinuity} and \ref{convergenceFiniteProductTopology} to conclude that the multiplication map is continuous iff $x_ny_n \to xy$ whenever $x_n \to x$ and $y_n \to y$.

Assume $x_n \to x$ and $y_n \to y$. Then
\begin{align*}
\norm{x_ny_n - xy} &= \norm{x_ny_n - xy_n + xy_n - xy} \leq \norm{(x_n-x)y_n}+ \norm{x(y_n-y)}\\ 
&\leq \norm{x_n-x}\cdot\norm{y_n}+ \norm{x}\cdot\norm{y_n-y} = \norm{x_n-x}\cdot\norm{y_n-y+y}+ \norm{x}\cdot\norm{y_n-y}\\
&\leq \norm{x_n-x}\cdot(\norm{y_n-y} + \norm{y})+ \norm{x}\cdot\norm{y_n-y} \to 0
\end{align*}
\end{proof}
As a consequence multiplication by a fixed factor, $x\mapsto cx$ or $x\mapsto xc$ for some $c$, is also continuous, by \ref{productInclusionsContinuous}. This is also immediate from the boundedness of multiplication $\norm{xy}\leq\norm{x}\norm{y}$ and \ref{boundedLinearMaps}.

\begin{lemma}
Let $A$ be a Banach algebra and $D\subset A$ a subset. Suppose $a\in A$ commutes with all elements of $D$, then $a$ commutes with the closure $\overline{D}$.
\end{lemma}
\begin{proof}
Take an arbitrary element $d\in \overline{D}$. Take an arbitrary $\epsilon >0$. Then we can find an $x\in D$ such that $\norm{x-d}\leq \epsilon$. Then, using that $a$ and $x$ commute,
\begin{align*}
\norm{ad - da} &= \norm{a(d+x-x) - (d+x-x)} \\
&= \norm{a(d-x) - (d-x)a} \leq 2\epsilon \norm{a}.
\end{align*}
Because we can choose $\epsilon$ arbitrarily small, $\norm{ad - da}$ must be zero.
\end{proof}

\begin{proposition} \label{smallestBanachAlgebra}
Let $A$ be a Banach algebra and $S\subset A$ a subset. Then
\[ \mathcal{B}(S) \defeq \overline{\Span}\setbuilder{s_1\cdot s_2 \cdot \ldots \cdot s_k}{k\geq 1, s_1,\ldots, s_k \in S} \]
is the smallest Banach subalgebra in $A$ that contains $S$.
\end{proposition}

\section{Unitisation}
\begin{definition}
Let $A$ be a Banach algebra. Then the \udef{unitisation} of $A$ is the algebra $A^\dagger = A\oplus \F$ with multiplication
\[ (x,\lambda)\cdot (y,\mu) = (xy+\lambda y + \mu x, \lambda\mu) \]
and a norm that extends the norm $\norm{\cdot}$ on $A$ to a norm on $A^\dagger$. In other words, there is an isometric embedding
\[ A \hookrightarrow A^\dagger: x\mapsto (x,0). \]
\end{definition}
TODO: is $A^\dagger$ necessarily complete?
\begin{lemma}
For any Banach algebra $A$, $A^\dagger$ is a unital Banach algebra with unit $\vec{1} = (0,1)$.
\end{lemma}
\begin{proof}
TODO: is $A^\dagger$ necessarily complete?
\end{proof}
It is possible to use multiple norms for the unitisation.
\begin{proposition} \label{normsOfUnitisation}
Let $A$ be a Banach algebra. Of the possible norms for $A^\dagger$, the $1$-norm
\[ \norm{(x,\lambda)}_1 = \norm{x}+|\lambda| \]
is minimal and the operator norm
\[ \norm{(x,\lambda)}_{op} = \sup\setbuilder{\norm{xa + \lambda a}}{a\in A \land \norm{a}\leq 1} \]
is maximal. All possible norms are equivalent.
\end{proposition}
\begin{proof}
TODO: prove the operator norm is actually a norm and isometric.
\end{proof}

\begin{definition}
We define
\[ \tilde{A} \defeq \begin{cases}
A & \text{if $A$ unital} \\
A^\dagger & \text{if $A$ non-unital.}
\end{cases} \]
If a Banach algebra $A$ is unital, we can identify $\F$ with $\F\cdot \vec{1} \subseteq A$.
\end{definition}

Alternatively we could define $\tilde{A}$ as the smallest unital Banach algebra containing $A$.

\begin{lemma}
Let $A$ be a Banach algebra. Then $A$ is an ideal of $A^\dagger$.
\end{lemma}

\begin{lemma}
Let $A$ be a Banach algebra. We have the split exact sequence
\[ \begin{tikzcd}
0 \rar & A \rar[hook, "\iota"] & A^\dagger \rar[shift left, "\pi_2"] & \lar[hook, shift left, "\lambda"] \F \rar & 0.
\end{tikzcd} \]
\end{lemma}

\begin{lemma}
Let $A,B$ be Banach algebras. Every algebra homomorphism $\Psi:A\to B$ extends uniquely to a unital homomorphism $\Psi^\dagger: A^\dagger \to B^\dagger$:
\[ \Psi^\dagger: A^\dagger \to B^\dagger: (a,\lambda) \mapsto (\Psi(a),\lambda). \]
\end{lemma}
\begin{proof}
We want $\Psi^\dagger((a,0)) = (\Psi(a),0)$ for all $a\in A$. Because $\Psi$ is unital, we have $\Psi^\dagger((\vec{0},1)) = (\vec{0},1)$. So
\[ \Psi^\dagger((a,\lambda)) = \Psi^\dagger((a,0))+\lambda \Psi^\dagger((\vec{0},1)) = (\Psi(a),0) + \lambda(\vec{0},1) = (\Psi(a),\lambda). \]
\end{proof}
\begin{corollary} \label{projectionOnACommutes}
Let $\pi_1: A^\dagger \to A$ be the projection on the first component: $\pi_1(a,\alpha) = a$.

The unital extension $\Psi^\dagger$ commutes with $\pi_2$:
\[ \pi_2\circ\Psi^\dagger = \Psi^\dagger \circ \pi_2 = \Psi\circ \pi_2. \]
Restricted to $A$, this is equal to $\Psi$.
\end{corollary}

\begin{definition}
As before we set, for $\Psi: A \to B$ an algebra homomorphism
\[ \tilde{\Psi} = \begin{cases}
\Psi & \text{if $A$ unital} \\
\Psi^\dagger & \text{if $A$ non-unital.}
\end{cases} \]
Thus $\tilde{\Psi}$ is a function on $\tilde{A}$.
\end{definition}

\begin{lemma} \label{DaggerMorphismProperties}
Let $A,B$ be Banach algebras and $\Psi:A\to B$ and algebra homomorphism. Then
\begin{enumerate}
\item $\im(\Psi^\dagger) = (\im\Psi)^\dagger$;
\item $\ker(\Psi^\dagger) = \ker(\Psi)\oplus\{0\}$;
\item $\Psi^\dagger$ is injective \textup{if and only if} $\Psi$ is injective;
\item $\Psi^\dagger$ is surjective \textup{if and only if} $\Psi$ is surjective;
\item $\norm{\Psi^\dagger} = \max\{\norm{\Psi},1\}$;
\item $\Psi^\dagger$ is isometric \textup{if and only if} $\Psi$ is isometric.
\end{enumerate}
\end{lemma}
\begin{proof}
The third point follows from the second and \ref{injectivityKernelTriviality}.
\end{proof}

\begin{definition}
Let $A$ be a Banach algebra. We define the \udef{scalar mapping} to be
\[ s = \lambda\circ \pi: A^\dagger \to A^\dagger: (a,\lambda) \mapsto (0,\lambda). \]
\end{definition}
Notice that $\pi\circ s = \pi$.

\subsection{Approximate units}
\begin{definition}
Let $A$ be a Banach algebra. A net $(e_\lambda)_{\lambda\in\Lambda}$ is an \udef{approximate unit} if
\begin{enumerate}
\item $\norm{e_\lambda}\leq 1$ for all $\lambda$;
\item $a = \lim_{\lambda\to \infty} e_\lambda \cdot a = \lim_{\lambda\to \infty} a \cdot e_\lambda$.
\end{enumerate}
We call $(e)_\lambda$ is an \udef{increasing approximate unit} if $\lambda_0 \leq \lambda_1$ implies $0\leq e_{\lambda_0} \leq e_{\lambda_1}$.
\end{definition}
\begin{lemma}
If $A$ is unital, any approximate unit in $A$ converges to $\vec{1}$.
\end{lemma}
\begin{proof}
We have $\vec{1} = \lim_{\lambda\to\infty}e_\lambda\cdot \vec{1} = \lim_{\lambda\to\infty}e_\lambda$.
\end{proof}

\section{Neumann series}
\begin{proposition}[Neumann series] \label{NeumannSeries}
Let $A$ be a unital Banach algebra and $x\in A$. 
If $\norm{x}<1$, then $\vec{1}-x$ is invertible with inverse
\[ (\vec{1}-x)^{-1} = \sum_{n=0}^\infty x^n \qquad\text{and}\qquad \norm{(\vec{1} - x)^{-1}} \leq \frac{1}{1-\norm{x}}. \]
Equivalently, if $\norm{\vec{1}-x}< 1$, then $x$ is invertible with inverse
\[ x^{-1} = \sum_{n=0}^\infty(\vec{1}-x)^n. \]
\end{proposition}
\begin{proof}
Since $\norm{x^n}\leq \norm{x}^n$ for all $n\geq 1$ and $\sum \norm{x}^n$ is a convergent geometric series, the series $\sum x^n$ is convergent by \ref{absoluteUnconditionalConvergenceBanach}.

Also
\[ \norm{(\vec{1} - x)^{-1}} = \norm{\sum_{i=0}^\infty x^i} \leq \sum_{i=0}^\infty \norm{x}^i = \frac{1}{1-\norm{x}} \]
by the geometric series.
\end{proof}
We can in fact weaken the requirement of $\norm{x}<1$ to $\exists k\in\N: \norm{x^k}<1$:
\begin{corollary} \label{NeumannSeriesEventuallyContractive}
Let $A$ be a unital Banach algebra and $x\in A$ such that $\norm{x^k}<1$ for some $k>0$. Then $\vec{1} - x$ is invertible and $(\vec{1} - x)^{-1} = \sum_{i=0}^\infty x^i$.
\end{corollary}
\begin{proof}
We know that the Neumann series $\sum_{i=0}^\infty(x^k)^i$ converges. So
\[ \sum_{i=0}^\infty x^i = (\vec{1} + x + x^2 +\ldots + x^{k-1})\sum_{i=0}^\infty(x^k)^i \]
converges.

To show this convergent sequence acturally gives the correct inverse, we calculate
\begin{align*}
(\vec{1}-x)(\vec{1} + x + x^2 +\ldots + x^{k-1})(\vec{1} - x^k)^{-1} &= \Big(\vec{1} + x + x^2 +\ldots + x^{k-1} - x - x^2 -\ldots - x^{k}\Big)(\vec{1} - x^k)^{-1} \\
&= (\vec{1} - x^k)(\vec{1} - x^k)^{-1} = \vec{1}.
\end{align*}
This shows that $\sum_{i=0}^\infty x^i$ is the correct right inverse. To show it is also a left inverse, we expand $(\vec{1} - x^k)^{-1}(\vec{1} + x + x^2 +\ldots + x^{k-1})(\vec{1}-x)$.
\end{proof}


\begin{lemma}
Let $A$ be a unital Banach algebra and $a,b\in A$ with $\norm{a} < 1$. Then $\sum_{n=0}^\infty a^nb$ is the unique fixed point of $x\mapsto ax+b$.
\end{lemma}
\begin{proof}
The function $x\mapsto ax+b$ is a contraction if and only if $\norm{x}<1$. So it has a unique fixed point (TODO). Starting the fixed point iteration at $b$ yields the series:
\begin{align*}
b &\mapsto  ab+b \\
ab+b &\mapsto a^2b + ab + b \\
&\hdots.
\end{align*}
\end{proof}
This gives an alternate proof of the convergence of the Neumann series.

\begin{proposition} \label{openSetInvertibles}
Let $x\in\GL(A)$ and $y\in A$ such that $\norm{y} < \norm{x^{-1}}^{-1}$, then
\begin{enumerate}
\item $(x-y)^{-1} = x^{-1}\sum_{i=0}^\infty(x^{-1}y)^i$ for all $y\in A$ such that $\norm{y}\leq \norm{x^{-1}}^{-1}$;
\item $\ball(x,\norm{x^{-1}}^{-1})\subset \GL(A)$;
\item the invertible elements $\GL(A)$ form an open subset of $A$.
\end{enumerate}
\end{proposition}
\begin{proof}
(1) From $\norm{x^{-1}y} \leq \norm{x^{-1}}\,\norm{y} < \norm{x^{-1}}\,\norm{x^{-1}}^{-1} = 1$, we have that $(\vec{1} - x^{-1}y)$ is invertible with a Neumann series expansion. We then have
\[ x^{1}\sum_{i=0}^\infty (x^{-1}y)^i = x^{-1}(\vec{1} - x^{-1}y)^{-1} = (x-y)^{-1}. \]

(2) Any element in  $\ball(x,\norm{x^{-1}}^{-1})$ is of the form $x-z$, where $\norm{z} < \norm{x^{-1}}^{-1}$.

(3) This follows from (2) by \ref{interior}.
\end{proof}
TODO: this also works if $x$ is closed, bijective linear operator on a Banach space (i.e.\ not necessarily bounded).

\begin{proposition} \label{inverseMapContinuous}
The map $^{-1}: \GL(A)\to\GL(A): x\mapsto x^{-1}$ is continuous.
\end{proposition}
\begin{proof}
Take a convergent sequence $(x_n)\subset\GL(A)$ with limit $x$. We wish to prove $(x_n^{-1})$ converges to $x^{-1}$, because then the map is continuous by \ref{sequentialContinuity}. We can choose an $n_0$ such that $\forall n\geq n_0: x_n \in B(x,\norm{x^{-1}}^{-1})$. From now on we consider only the tails $(x_n)_{n=n_0}^\infty$ and $(x_n^{-1})_{n=n_0}^\infty$, which have the same limits. Then
\[ \norm{x^{-1}}\cdot\norm{x-x_n} < \norm{x^{-1}}\cdot\norm{x^{-1}}^{-1} = 1. \]
Also
\[ \norm{\vec{1} - x^{-1}x_n} = \norm{x^{-1}(x-x_n)} \leq \norm{x^{-1}}\cdot\norm{x-x_n} < 1. \]
We calculate, using the inequalities to apply the Neumann series formula and geometric series formula:
\begin{align*}
\norm{x_n^{-1} - x^{-1}} &= \norm{(x_n^{-1}x - \vec{1})x^{-1}} = \norm{((x^{-1}x_n)^{-1} - \vec{1})x^{-1}} \\
&= \norm{\left(\sum_{k=0}^\infty[\vec{1} - x^{-1}x_n]^k - \vec{1}\right)x^{-1}} = \norm{\left(\sum_{k=1}^\infty[\vec{1} - x^{-1}x_n]^k\right)x^{-1}} \\
&\leq \sum_{k=1}^\infty\norm{\vec{1} - x^{-1}x_n}^k\cdot\norm{x^{-1}} = \sum_{k=1}^\infty\norm{x^{-1}(x - x_n)}^k\cdot\norm{x^{-1}} \\
&\leq \norm{x^{-1}}\sum_{k=1}^\infty\norm{x - x_n}^k\cdot\norm{x^{-1}}^{k} = \norm{x^{-1}}\sum_{k=0}^\infty\norm{x - x_n}^k\cdot\norm{x^{-1}}^{k} - \norm{x^{-1}} \\
&= \frac{\norm{x^{-1}}}{1-\norm{x - x_n}\cdot\norm{x^{-1}}}-\norm{x^{-1}} = \frac{\norm{x - x_n}\cdot\norm{x^{-1}}^2}{1-\norm{x - x_n}\cdot\norm{x^{-1}}}.
\end{align*}
As the right-hand side converges to $0$, so must the left-hand side. Thus $(x_n^{-1})$ converges to $x^{-1}$.
\end{proof}


\subsection{The exponential}
TODO: use functional calculus??

\begin{proposition}
Let $A$ be a Banach algebra and $a\in A$. Then the series
\[ \sum_{i=1}^\infty \frac{a^i}{i!} \]
converges. We denote its limit $\exp(a)-1$ or $e^a-1$.
\end{proposition}
\begin{proof}
By
\[ \norm{\sum_{i=1}^N \frac{a^i}{i!}} \leq \sum_{i=1}^N \frac{\norm{a}^i}{i!} \]
it is absolutely convergent and thus convergent, by \ref{absoluteUnconditionalConvergenceBanach}.
\end{proof}
The function $a\mapsto \exp(a) = \vec{1} + \sum_{i=1}^\infty \frac{a^i}{i!}$ is the \udef{exponential mapping}.

Note that $\exp(0) = \vec{1}$.

\begin{lemma} \label{continuityExp}
The exponential mapping is continuous.
\end{lemma}
\begin{proof}
TODO - see Coleman
\end{proof}

\begin{proposition} \label{factorisationCommutingExponentials}
Let $A$ be a unital Banach algebra and $a,b\in A$. If $a$ and $b$ commute, then
\[ \exp(a+b) = \exp(a)\exp(b). \]
\end{proposition}
\begin{proof}
TODO - Coleman
\end{proof}
\begin{corollary}
Let $a\in A$. Then $\exp(a)\in \GL(A)$ and $\exp(a)^{-1} = \exp(-a)$.
\end{corollary}
\begin{proof}
As $a$ and $-a$ commute, we have $\exp(-a)\exp(a) = \exp(0) = \vec{1}$.
\end{proof}

\begin{lemma}
Let $A$ be a Banach algebra and $a\in A$. Then
\[ \exp(a) = \lim_{n\to\infty} \left(\vec{1} + \frac{a}{n}\right)^n. \]
\end{lemma}
\begin{proof}
TODO
\end{proof}

\begin{lemma}
Let $A$ be a Banach algebra and $p\in A$ an idempotent. Then
\[ e^p = \vec{1} + (e-1)p. \]
\end{lemma}
\begin{proof}
We calculate
\[ e^p = \vec{1} + \sum_{k=1}^\infty \frac{p^k}{k!} = \vec{1} + \sum_{k=1}^\infty \frac{p}{k!} = \vec{1} + p\sum_{k=1}^\infty \frac{1}{k!} = \vec{1} + (e-1)p. \]
\end{proof}

\begin{lemma}
Let $A$ be a Banach algebra, $a\in A$ and $b\in \GL(A)$. Then $\exp(bab^{-1}) = b\exp(a)b^{-1}$.
\end{lemma}
\begin{proof}
TODO
\end{proof}

TODO: $\exp(a^*) = \exp(a)^*$ is Banach-$*$-algebra.

TODO: correct setting for this:
\begin{proposition}
$\det(e^a) = e^{\Tr(a)}$.
\end{proposition}

\section{Quotient algebras}
\begin{proposition}
Let $A$ be a Banach algebra and $J\subset A$ a closed (two-sided) ideal. Then $A/J$ is a Banach algebra.
\end{proposition}
\begin{proof}
We know that $A/J$ is a Banach space by \ref{quotientBanachSpace} and an algebra by (TODO ref). We just need to check that the quotient norm is submultiplicative.

Take $a,b\in A$ and $\epsilon>0$. Then there exist $x,y\in J$ such that $\norm{a+J} + \epsilon > \norm{a+x}$ and $\norm{b+J} + \epsilon > \norm{b+y}$. Then
\begin{align*}
\big(\norm{a+J} + \epsilon\big)\big(\norm{b+J} + \epsilon\big) &> \norm{a+x}\;\norm{b+y} \\
&\geq \norm{(a+x)(b+y)} = \norm{ab + (ay+xb+xy)} \geq \norm{ab + J}.
\end{align*}
Taking $\epsilon\to 0$, gives $\norm{a+J}\;\norm{b+J} \geq \norm{ab + J}$.
\end{proof}

\begin{proposition}
Let $A$ be a unital Banach algebra and $J\subseteq A$ a proper ideal. Then $\norm{\vec{1}}_{A/J} = 1$.
\end{proposition}
\begin{proof}
Since $0\in J$, we have $\norm{\vec{1}}_{A/J} \leq 1$ and it is enough to show that $\norm{\vec{1}+z} \geq 1$ for all $z\in J$. Suppose, towards a contradiction, that $\norm{\vec{1}+z} < 1$, so $z\in \ball(\vec{1},1)$. This means that $z$ is invertible by \ref{openSetInvertibles} and thus that $J$ is not proper by \ref{properIdealNoUnit}.
\end{proof}
\begin{corollary}
Let $A$ be a unital Banach algebra and $J$ a proper ideal. Then
\begin{enumerate}
\item $\vec{1} \notin \overline{J}$;
\item if $J$ is maximal, then it is closed.
\end{enumerate}
\end{corollary}
\begin{proof}
(1) We have $\vec{1}\notin J$ by \ref{properIdealNoUnit}. There does not exist a sequence in $J$ that converges to $\vec{1}$ because $1 = \norm{\vec{1}}_{A/J} = \inf_{z\in J}\norm{\vec{1}-z}$.

(2) Let $J\subseteq A$ be a maximal ideal in $A$. It is enough to show that the closure $\overline{J}$ is also an ideal. Since $\vec{1}\notin \overline{J}$ this ideal is proper.

The fact that $\overline{J}$ is an ideal follows straight from the continuity of addition and multiplication.
\end{proof}


\section{Finite elements}
elements of the socle. \url{https://link.springer.com/content/pdf/10.1023/A:1009717500980.pdf}

have finite spectrum 

\url{http://matwbn.icm.edu.pl/ksiazki/sm/sm104/sm10431.pdf}

\section{Real and complex Banach algebras}
TODO: define $A_\R$ and $A_\C$.

\begin{proposition} \label{preservationAlgebraicPropertiesComplexificationRealification}
$\GL(A) = \GL(A_\C)\cap A$ etc.
\end{proposition}

\section{The spectrum}
TODO: remove unital requirement.
\begin{definition}
Let $A$ be a complex Banach algebra. The \udef{spectrum} of an element $x\in A$ is defined as
\[ \spec(x) = \spec_A(x) \defeq \setbuilder{\lambda\in\C}{\lambda\cdot \vec{1} - x \in \tilde{A} \;\text{is not invertible}}. \]
If $A$ is a real Banach algebra, then the spectrum of $x\in A$ is defined as
\[ \spec_A(x) \defeq \spec_{A_\C}(x) = \setbuilder{\lambda\in\C}{\lambda\cdot \vec{1} - x \in \widetilde{A_\C} \;\text{is not invertible}}.  \]
The \udef{resolvent set} of an element $x\in A$ is
\[ \res(x) = \C\setminus\spec(x) \]
and its \udef{resolvent map} is
\[ R_x: \res(x) \to A : \lambda\mapsto (\lambda\cdot\vec{1}  - x)^{-1}. \]
The \udef{spectral radius} of $x\in A$ is
\[ \spr(x) = \sup\setbuilder{|\lambda|}{\lambda\in\spec(x)}. \]
\end{definition}
As we will later show that the spectrum is compact (\ref{spectrumCompact}), we may equivalently write
\[ \spr(x) = \max\setbuilder{|\lambda|}{\lambda\in\spec(x)}. \]

\begin{lemma}
Let $A$ be a non-unital Banach algebra. Then $0\in\spec_A(a)$ for all $a\in A$.
\end{lemma}
\begin{proof}
Because $A\subset A^\dagger$ as an ideal, $(a,0)\in A^\dagger$ is not invertible.
\end{proof}

\begin{lemma}
Let $A$ be a real Banach algebra. Then for all $a\in A$ and $\mu_1,\mu_2\in \R$:
\begin{enumerate}
\item $\spec(a) = \overline{\spec(a)}$;
\item $\mu_1 + \mu_2 i \in \spec(a)$ \textup{if and only if} $(a-\mu_1)^2+\mu_2^2$ is not invertible in $\tilde{A}$.
\end{enumerate}
\end{lemma}
\begin{proof}
(1) Assume $\lambda \notin \spec(a)$, so $(\lambda-a)^{-1}$ exists. Then
\[ \vec{1} = \overline{\vec{1}} = \overline{(\lambda-a)}\overline{(\lambda-a)^{-1}} = (\overline{\lambda}-a)\overline{(\lambda-a)^{-1}}, \]
so $\overline{\lambda}-a$ is invertible and $\overline{\lambda}\notin \spec(a)$. The converse is identical, using $\overline{\lambda}$.

(2) By (1), $(\mu_1 + \mu_2 i)\vec{1}- a$ is invertible if and only if $(\mu_1 - \mu_2 i)\vec{1}-a$ is invertible. Because $(\mu_1 + \mu_2 i)\vec{1}-a$ and $(\mu_1 - \mu_2 i)\vec{1}-a$ commute, this is equivalent to saying
\[ \big((\mu_1 + \mu_2 i)-a\big)\big((\mu_1 - \mu_2 i)-a\big) = (\mu_1-a)^2 + \mu_2^2 \]
is invertible in $\widetilde{A_\C}$, by \ref{productInvertibility} and thus also in $A$ by \ref{preservationAlgebraicPropertiesComplexificationRealification}.
\end{proof}

\begin{proposition}
Let $B$ be a complex Banach algebra and $A = B_\R$, then for all $a\in A$
\[ \spec_A(a) = \spec_B(a) \cup \overline{\spec_B(a)}. \]
\end{proposition}
\begin{proof}
TODO
\end{proof}

\begin{proposition} \label{spectrumCompact}
For any $x\in A$, the spectrum $\spec(x)$ is a compact subset of $\setbuilder{\lambda\in\C}{|\lambda|\leq \norm{x}}$.

In particular, $\spr(x) \leq \norm{x}$.
\end{proposition}
\begin{proof}
Let $\lambda\in \C$ be such that $|\lambda|>\norm{x}$, then
\[ 1 > \frac{\norm{x}}{|\lambda|} = \frac{\lambda - (\lambda - \norm{x})}{|\lambda|} = \norm{\vec{1} - \left(\vec{1} - \frac{x}{\lambda}\right)}. \]
By \ref{NeumannSeries}, $\vec{1} - x/\lambda$ is invertible and thus so is $\lambda-x$.

It is then enough to show that $\spec(x)$ is closed. By \ref{openSetInvertibles}, $\GL(A)$ is open and the set of non-invertibles $A\setminus \GL(A)$ is closed. Consider $f: \C \to A: \lambda \mapsto \lambda - x$. Then $\spec(x) = f^{-1}[A\setminus \GL(A)]$ is the preimage of a closed set under a continuous map, and hence is closed.
\end{proof}

\begin{lemma}[Polynomial spectral mapping] \label{polynomialSpectralMapping}
Let $A$ be a Banach algebra and $p$ a complex polynomial. Then
\[ p^{\imf}\big(\spec(x)\big) \subseteq \sigma\big(p(x)\big). \]
\end{lemma}
\begin{proof}
Take $\lambda\in\spec(x)$ so $p(\lambda)\in p^{\imf}\big(\spec(x)\big)$. Then $y\mapsto p(\lambda)-p(y)$ is a polynomial with zero at $\lambda$, so we can factorise it as $(\lambda - y)q(y)$ for some other polynomial $q$ by the fundamental theorem of algebra (TODO ref).

Now suppose, towards a contradiction, that $p(\lambda)\notin \sigma\big(p(x)\big)$. Then $p(\lambda) - p(x) = (\lambda - x)q(x) = q(x)(\lambda - x)$ has an inverse, so $(\lambda - x)$ is invertible by \ref{productInvertibility}. This contradicts $\lambda\in\spec(x)$.
\end{proof}

\begin{proposition}
Let $A$ be a unital Banach algebra and $x,y\in A$. Then $\vec{1} - xy$ is invertible \textup{if and only if} $\vec{1} - yx$ is invertible.
\end{proposition}
\begin{proof}
Assume $\vec{1} - xy$ invertible. Then the inverse of $\vec{1} - yx$ is
\[ y(\vec{1} - xy)^{-1}x + \vec{1}. \]
\end{proof}
\begin{corollary}
Let $A$ be a unital Banach algebra and $x,y\in A$. Then
\[ \spec(xy)\cup\{0\} = \spec(yx)\cup\{0\}. \]
\end{corollary}
\begin{proof}
Assuming $\lambda \neq 0$, we have $\lambda\in\spec(xy) \iff \frac{xy}{\lambda} - \vec{1}$ is invertible.
\end{proof}
It is important to include $0$: there are cases when $0\in \spec(xy)$, but $0\notin \spec(yx)$.

\begin{lemma} \label{spectrumOfImage}
Let $A,B$ be unital Banach algebras and $\Psi: A\to B$ a unital algebra homomorphism. Then for all $x\in A$: $\spec(\Psi(x)) \subseteq \spec(x)$ and hence $\spr(\Psi(x)) \leq \spr(x)$.
\end{lemma}
\begin{proof}
By contraposition: Assume $\lambda\notin\spec(x)$, then $x-\lambda$ has an inverse, call it $a$. Then $(\Psi(x) - \lambda)$ has an inverse by
\[ (\Psi(x) - \lambda)\Psi(a) = \Psi(x-\lambda)\Psi(a) = \Psi((x-\lambda)a) = \Psi(\vec{1}) = \vec{1},\]
meaning $\lambda \notin \spec(\Psi(x))$.
\end{proof}

In general if $B$ is a subalgebra of a Banach algebra $A$, then for any $x\in B$, $\spec_B(x) \supseteq \spec_A(x)$.

\begin{proposition}
Let $A$ be a unital Banach algebra and suppose that $S\subset A$ is a set of pairwise commuting elements. Then there exists a unital commutative Banach subalgebra $C$ such that $S\subset C\subset A$ and
\[ \spec_A(s) = \spec_C(s) \qquad \text{for all $s\in S$.} \]
\end{proposition}
\begin{proof}
TODO
\end{proof}


\subsection{Resolvents and pseudoresolvents}

\begin{proposition} \label{secondNeumannSeries}
Let $A$ be a Banach algebra, $x\in A$ and $|\lambda| > \liminf_{n\to \infty}\norm{x^n}^{1/n}$, then
\begin{enumerate}
\item $\lambda\in\res(x)$;
\item $R_x(\lambda) = \sum_{n=0}^\infty\frac{x^n}{\lambda^{n+1}}$;
\item $\norm{R_x(\lambda)} \leq \frac{1}{|\lambda|-\norm{x}}$.
\end{enumerate}
\end{proposition}
\begin{proof}
Take arbitrary $\epsilon > 0$. We can find an $n\in\N$ such that $|\lambda| > \norm{x^n}^{1/n}$, so $1 > \norm{\left(\frac{x}{\lambda}\right)^n}$. By \ref{NeumannSeriesEventuallyContractive}, we have that $(\vec{1} - \frac{x}{\lambda})$ is invertible and
\[ R_x(\lambda) = \lambda^{-1} \left(\vec{1} - \frac{x}{\lambda}\right)^{-1} = \lambda \sum_{n=0}^\infty\left(\frac{x}{\lambda}\right)^n = \sum_{n=0}^\infty\frac{x^n}{\lambda^{n+1}}. \]
Finally $\norm{R_x(\lambda)} \leq \sum_{n=0}^\infty\frac{\norm{x}^n}{|\lambda|^{n+1}} = \frac{1}{|\lambda|(1-\frac{\norm{x}}{\lambda})} = \frac{1}{|\lambda|-\norm{x}}$.
\end{proof}

\subsubsection{Pseudoresolvents and first resolvent identity}
\begin{definition}
Let $A$ be a Banach algebra. A function $\mathcal{R}:\Lambda \subseteq \C \to A$ is called a \udef{pseudoresolvent} if, for all $\lambda,\mu\in\Lambda$
\[ \mathcal{R}(\lambda) - \mathcal{R}(\mu) = (\mu-\lambda)\mathcal{R}(\lambda)\mathcal{R}(\mu). \]
This equation is known as the (first) \udef{resolvent identity}.
\end{definition}

Note that if a pseudoresolvent $\mathcal{R}$ is zero anywhere, it is identically zero.

\begin{lemma}
Let $\mathcal{R}:\Lambda \subseteq \C \to A$ be a pseudoresolvent on a Banach algebra $a$ and $\lambda,\mu\in\Lambda$. Then $\mathcal{R}(\lambda)\mathcal{R}(\mu) = \mathcal{R}(\mu)\mathcal{R}(\lambda)$.
\end{lemma}
\begin{proof}
If $\lambda = \mu$, then the result is immediate. If $\lambda \neq \mu$, then
\[ \mathcal{R}(\lambda)\mathcal{R}(\mu) = (\mu-\lambda)^{-1}\big(\mathcal{R}(\lambda) - \mathcal{R}(\mu)\big) = (\lambda - \mu)^{-1}\big(\mathcal{R}(\mu) - \mathcal{R}(\lambda)\big) = \mathcal{R}(\mu)\mathcal{R}(\lambda). \]
\end{proof}

\begin{proposition} \label{firstNeumannSeries}
Let $\mathcal{R}:\Lambda \subseteq \C \to A$ be a pseudoresolvent and $\lambda_0,\lambda\in\Lambda$ such that $|\lambda-\lambda_0|\,\norm{\mathcal{R}(\lambda_0)} < 1$. Then
\[ \mathcal{R}(\lambda) = \sum_{n=0}^\infty(\lambda_0 - \lambda)^n \mathcal{R}(\lambda_0)^{n+1} \qquad\text{and}\qquad \norm{\mathcal{R}(\lambda)} \leq \frac{1}{\norm{\mathcal{R}(\lambda_0)}^{-1} - |\lambda_0-\lambda|}. \]
In particular $\mathcal{R}$ is analytic and can be analytically continued to $\Lambda \cup \ball(\lambda_0, \norm{\mathcal{R}(\lambda_0)}^{-1})$. The analytic continuation still satisfies the resolvent identity.
\end{proposition}
\begin{proof}
By assumption we have $(\lambda - \lambda_0)\mathcal{R}(\lambda_0)$ is a contraction, so $(\id_X - (\lambda - \lambda_0)\mathcal{R}(\lambda_0))^{-1}$ exists and has a Neumann series expansion by \ref{NeumannSeries}. From the resolvent identity, we get
\[ \mathcal{R}(\lambda)\big(\id_X - (\lambda_0 - \lambda)\mathcal{R}(\lambda_0)\big) = \mathcal{R}(\lambda_0), \]
so, using the Neumann series expansion,
\begin{align*}
\mathcal{R}(\lambda) &= \big(\id_X - (\lambda_0 - \lambda)\mathcal{R}(\lambda_0)\big)^{-1}\mathcal{R}(\lambda_0) \\
&= \sum_{n=0}^\infty(\lambda_0 - \lambda)^n \mathcal{R}(\lambda_0)^{n+1}.
\end{align*}
This series converges in norm for all $\lambda\in \C$ such that $|\lambda-\lambda_0| < \norm{\mathcal{R}(\lambda_0)}^{-1}$. Running the equalities in reverse gives the resolvent identity.

The norm estimate is also given by \ref{NeumannSeries}:
\begin{align*}
\norm{\mathcal{R}(\lambda)} &\leq \norm{\big(\id_X - (\lambda_0 - \lambda)\mathcal{R}(\lambda_0)\big)^{-1}}\norm{\mathcal{R}(\lambda_0)} \\
&\leq \frac{1}{1 - |\lambda_0-\lambda|\,\norm{\mathcal{R}(\lambda_0)}}\norm{\mathcal{R}(\lambda_0)} \\
&= \frac{1}{\norm{\mathcal{R}(\lambda_0)}^{-1} - |\lambda_0-\lambda|}.
\end{align*}
\end{proof}
\begin{corollary} \label{derivativePseudoresolvent}
Let $\mathcal{R}:\Lambda \subseteq \C \to A$ be a pseudoresolvent. Then
\begin{enumerate}
\item $\mathcal{R}'(\lambda) = -\mathcal{R}(\lambda)^2$ for all $\lambda\in\Lambda$;
\item $\mathcal{R}^{(n)}(\lambda) = n!(-1)^n \mathcal{R}(\lambda)^{n+1}$ for all $n\in \N$.
\end{enumerate}
In particular the map $\mathcal{R}$ is holomorphic on its domain of definition.
\end{corollary}
\begin{proof}
We calculate
\begin{align*}
\mathcal{R}'(\lambda) &= \lim_{\mu\to\lambda} \frac{\mathcal{R}(\mu) - \mathcal{R}(\lambda)}{\mu-\lambda} \\
&= \lim_{\mu\to\lambda} \frac{\mathcal{R}(\mu) - \mathcal{R}(\lambda)}{\mu-\lambda} \\
&= \lim_{\mu\to\lambda} -\mathcal{R}(\lambda)\mathcal{R}(\mu) = -\mathcal{R}(\lambda)\lim_{\mu\to\lambda} \mathcal{R}(\mu) = -\mathcal{R}(\lambda)^2.
\end{align*}
For the last equality we have used the fact that $\mathcal{R}$ is continuous, which follows from its analyticity.
\end{proof}

\begin{proposition} \label{firstResolventIdentity}
Let $A$ be a Banach algebra and $x\in A$. Then the resolvent map
\[ R_x: \res(x)\to A: \lambda \mapsto (\lambda\cdot\vec{1} - x)^{-1} \]
is a pseudoresolvent.
\end{proposition}
\begin{proof}
We first note that $R_x(\lambda), R_x(\mu)$ commute for any $\lambda,\mu\in\spec(x)$, by \ref{commutationInverse}.

We then calculate
\begin{align*}
R_x(\lambda) - R_x(\mu) &= R_x(\lambda)(\mu - x)R_x(\mu) - R_x(\lambda)(\lambda - x)R_x(\mu) \\
&= \mu R_x(\lambda)R_x(\mu) - R_x(\lambda)xR_x(\mu) - \lambda R_x(\lambda)R_x(\mu) + R_x(\lambda)xR_x(\mu) \\
&= \mu R_x(\lambda)R_x(\mu) - \cancel{R_x(\lambda)xR_x(\mu)} - \lambda R_x(\lambda)R_x(\mu) + \cancel{R_x(\lambda)xR_x(\mu)} \\
&= (\mu - \lambda)R_x(\lambda)R_x(\mu).
\end{align*}
\end{proof}

\subsubsection{Second resolvent identity}
\begin{proposition}[Second resolvent identity] \label{secondResolventIdentity}
Let $A$ be a Banach algebra and $x,y\in A$. Then for all $\lambda \in \res(x)\cap \res(y)$ we have
\[ R_x(\lambda) - R_y(\lambda) = R_x(\lambda)\big(x-y\big)R_y(\lambda). \]
\end{proposition}
\begin{proof}
We have
\begin{align*}
R_x(\lambda)\big(x-y\big)R_y(\lambda) &= R_x(\lambda)\big(\lambda\vec{1}-y - (\lambda\vec{1} - x)\big)R_y(\lambda) \\
&= R_x(\lambda)\cancel{(\lambda\vec{1}-y)R_y(\lambda)} - \cancel{R_x(\lambda)(\lambda\vec{1} - x)}R_y(\lambda) \\
&= R_x(\lambda) - R_y(\lambda).
\end{align*}
\end{proof}
We can obtain the first resolvent identity from the second by setting $y = (\lambda-\mu)\vec{1} + x$. Then $R_{(\lambda-\mu)\vec{1} + x}(\lambda) = R_x(\mu)$, so
\begin{align*}
R_x(\lambda) - R_x(\mu) &= R_x(\lambda) - R_{(\lambda-\mu)\vec{1} + x}(\lambda) \\
&= R_x(\lambda)(x - (\lambda-\mu)\vec{1} + x)R_{(\lambda-\mu)\vec{1} + x}(\lambda) \\
&= (\lambda - \mu)R_x(\lambda)R_{(\lambda-\mu)\vec{1} + x}(\lambda) \\
&= (\lambda - \mu)R_x(\lambda)R_x(\mu).
\end{align*}

\begin{corollary}
Let $x,y\in A$ and $\lambda\in \res(x)\cap \res(y)$ be such that $\norm{R_x(\lambda)(y-x)}<1$. Then 
\[ R_y(\lambda) = \sum_{n=0}^\infty \big(R_x(\lambda)(y-x)\big)^nR_x(\lambda) \quad\text{and}\quad \norm{R_y(\lambda)} \leq \frac{1}{\norm{R_x(\lambda)}^{-1} - \norm{y-x}}. \]
\end{corollary}
\begin{proof}
We have
\[ R_x(\lambda) = R_y(\lambda) + R_x(\lambda)(x-y)R_y(\lambda) = \big(\vec{1} + R_x(\lambda)(x-y)\big)R_y(\lambda) = \big(\vec{1} - R_x(\lambda)(y-x)\big)R_y(\lambda). \]
Because $R_x(\lambda)(y-x)$ was assumed a contraction, $\vec{1} + R_x(\lambda)(x-y)$ has an inverse given by its Neumann series, \ref{NeumannSeries}, so
\begin{align*}
R_y(\lambda) &= \big(\vec{1} - R_x(\lambda)(y-x)\big)^{-1} R_x(\lambda) \\
&= \sum_{n=0}^\infty \big(R_x(\lambda)(y-x)\big)^n R_x(\lambda).
\end{align*}
The norm estimate is also given by \ref{NeumannSeries}:
\begin{align*}
\norm{R_y(\lambda)} &\leq \norm{\big(\vec{1} - R_x(\lambda)(y-x)\big)^{-1}}\norm{R_x(\lambda)} \\
&\leq \frac{1}{1 - \norm{R_x(\lambda)(y-x)}}\norm{R_x(\lambda)} \\
&\leq \frac{1}{1 - \norm{R_x}\,\norm{(\lambda)(y-x)}}\norm{R_x(\lambda)} \\
&= \frac{1}{\norm{R_x(\lambda)}^{-1} - \norm{(\lambda)(y-x)}}.
\end{align*}
\end{proof}

\subsubsection{Properties of the spectrum}

\begin{proposition}
Let $A$ be a Banach algebra and $x\in A$. Then $\spec(x)\neq \emptyset$.
\end{proposition}
\begin{proof}
Assume, towards a contradiction, that $\res(x) = \C$. Then the resolvent norm $\lambda\mapsto \norm{R_x(\lambda)}$ is an entire function by \ref{derivativePseudoresolvent}. By \ref{secondNeumannSeries} we have $\lim_{|\lambda|\to\infty}\norm{R_x(\lambda)} = 0$. By Liouville's theorem \ref{liouvilleTheoremAnalysis}, we have that $\norm{R_x(\lambda)}$ is identically zero.

Now we have
\[ 1 = \norm{\vec{1}} = \norm{R_x(\lambda)(\lambda-x)} \leq \norm{R_x(\lambda)}\,\norm{\lambda - x} = 0. \]
This is a contradiction.
\end{proof}
\begin{corollary}[Gelfand-Mazur] \label{GelfandMazur}
Let $A$ be a unital complex Banach algebra. If every non-zero element is invertible, then $A=\C\cdot \vec{1}$.
\end{corollary}
\begin{proof}
Suppose $x\in A\setminus (\C\cdot\vec{1})$. Then $\spec(x) = \emptyset$, contradicting the theorem.
\end{proof}
In other words, $\C$ is the only normed complex division algebra.


\begin{proposition}[Spectral radius formula] \label{spectralRadiusFormula}
Let $A$ be a Banach algebra and $x\in A$. Then
\[ \spr(x) = \lim_{n\to\infty}\norm{x^n}^{1/n} = \inf_{n\in\N}\norm{x^n}^{1/n}. \]
\end{proposition}
\begin{proof}
The contraposition of (1) in \ref{secondNeumannSeries} gives that $\lambda\in\spec(x)$ implies $|\lambda| \leq \liminf_{n\to \infty}\norm{x^n}^{1/n}$. Thus $\spr(x) \leq \liminf_{n\to \infty}\norm{x^n}^{1/n}$.

If we can now prove $\limsup_{n\to\infty}\norm{x^n}^{1/n}\leq \spr(x)$, then
\[ \spr(x) \leq \inf_{n\in\N}\norm{x^n}^{1/n} \leq \liminf_{n\in\N}\norm{x^n}^{1/n} \leq \limsup_{n\in\N}\norm{x^n}^{1/n} \leq \spr(x), \]
which means all terms in the inequality are equal. As the liminf and limsup are equal, they are equal to the limit.

TODO prove $\limsup_{n\to\infty}\norm{x^n}^{1/n}\leq \spr(x)$.
\end{proof}



\subsection{Quasinilpotent operators}
\begin{definition}
Let $A$ be a Banach algebra and $x\in A$. If $\spec(x) = \{0\}$, then $x$ is called \udef{quasinilpotent}
\end{definition}
\url{https://www.jstor.org/stable/2042882?seq=1}
\url{https://www.researchgate.net/profile/Zbigniew-Slodkowski/publication/265547661_A_note_on_quasinilpotent_elements_of_a_Banach_algebra/links/5e7e8f94458515efa0b0fe83/A-note-on-quasinilpotent-elements-of-a-Banach-algebra.pdf?origin=publication_detail}

\url{https://www.cambridge.org/core/services/aop-cambridge-core/content/view/AC3CBD3000D16515D0BD83C07B703186/S0013091500015352a.pdf/finite_dimensionality_nilpotents_and_quasinilpotents_in_banach_algebras.pdf}

\url{https://www.cambridge.org/core/services/aop-cambridge-core/content/view/C8F26DDF45A29D689C726A29D8F0BC2A/S0017089500008429a.pdf/algebraic-ideals-of-semiprime-banach-algebras.pdf}

\begin{proposition} \label{nilpotentQuasinilpotent}
Every nilpotent element is quasinilpotent. The converse holds for finite elements in semisimple Banach algebras (TODO correct version of finite).
\end{proposition}
\begin{proof}
Let $x$ be a nilpotent element in a (unital) Banach algebra $A$.
By the spectral radius formula \ref{spectralRadiusFormula}, we have
\[ \spr(x) = \lim_{n\to\infty}\norm{x^n}^{1/n} = 0. \]
This implies $\spec(x) = \{0\}$.

Now assume $x$ a finite quasimilpotent element. Then $\dim(xAx) = n$ and so $x^2, \ldots x^{n+2}$ a linearly dependent and thus there exists a polynomial $p$ of degree at most $n+2$ such that $p(x) = 0$. We can factorise $p(x) = x^kq(x)$ where $q$ is some polynomial such that $q(0) \neq 0$ and $k\in \N$. Then by spectral mapping, we have that $\spec(q(x)) = q(\spec(x)) = q(\{0\}) \neq \{0\}$. Thus $q(x)$ is invertible and we have
\[ x^k = x^kq(x)q(x)^{-1} = 0\cdot q(x)^{-1} = 0. \]
This means $x$ is nilpotent.
\end{proof}

\section{Characters}
\begin{definition}
Let $A$ be a Banach algebra. A \udef{character} on $A$ is a non-zero algebra homomorphism $A\to\C$.
\end{definition}
In other words, a character on $A$ is a non-zero multiplicative linear functional $A\to \C$.
In particular, if $A$ is a real algebra, a character is still complex valued, but now  an $\R$-linear functional on $A$. 

\begin{lemma}
Let $A$ be a real Banach algebra. Let $\varphi$ be a character on $A$. Then
\[ \varphi \in \tdual{A} \iff \varphi = \overline{\varphi} \iff \varphi[A] \subset \R. \]
\end{lemma}

\begin{proposition} \label{charactersUnital}
Let $A$ be a Banach algebra and $\varphi$ a character on $A$, then $\varphi$ is continuous and
\begin{enumerate}
\item $\norm{\varphi} = 1 = \varphi(\vec{1})$ if $A$ is unital;
\item $\norm{\varphi}\leq 1$;
\item $\norm{\varphi} = 1$ if $A$ contains an approximate unit.
\end{enumerate}
\end{proposition}
\begin{proof}
(1) We prove that $\varphi$ is unital if $A$ is unital: As $\varphi(x) = \varphi(x\cdot\vec{1}) = \varphi(x)\varphi(\vec{1})$ for all $x\in A$ and $\varphi \neq 0$, it follows that $\varphi(\vec{1}) = 1$.

(2) If we can show $\norm{\tilde{\varphi}} \leq 1$, then $\norm{\varphi}\leq 1$ follows by \ref{DaggerMorphismProperties}. To this end suppose that for some $x\in \tilde{A}$, $|\tilde{\varphi}(x)|>\norm{x}$. Then $x-\tilde{\varphi}(x)$ is invertible by corollary \ref{spectrumCompact}. Thus
\[ 1 = \tilde{\varphi}(\vec{1}) = \tilde{\varphi}((x-\tilde{\varphi}(x))^{-1})\tilde{\varphi}(x-\tilde{\varphi}(x)) = \tilde{\varphi}((x-\tilde{\varphi}(x))^{-1})[\tilde{\varphi}(x)-\tilde{\varphi}(x)] = \tilde{\varphi}((x-\tilde{\varphi}(x))^{-1})\cdot 0 = 0 \]
which is a contradiction. Then $\norm{\tilde{\varphi}} \leq 1$ and thus $\norm{\varphi} \leq 1$.

(3) By \ref{boundedLinearMaps}, $\varphi$ is continuous. Let $\seq{e_\lambda}_{\lambda\in\Lambda}$ be an approximate unit.
Then, for all $x\in A$, we have
\[ \varphi(x) = \varphi(\lim_\lambda)\varphi(x\cdot e_\lambda) = \lim_\lambda \varphi(x)\cdot \varphi(e_\lambda). \]
Since $\varphi\neq 0$, we can take $x$ such that $\varphi(x)\neq 0$. Then
\[ 1 = \varphi(x)^{-1}\varphi(x) = \varphi(x)^{-1}\cdot\big(\lim_\lambda \varphi(x)\cdot \varphi(e_\lambda)\big) = \lim_\lambda \varphi(x)^{-1}\varphi(x)\cdot \varphi(e_\lambda) = \lim_\lambda \varphi(e_\lambda). \]
This implies $\sup_{\lambda}|\varphi(e_\lambda)| \geq 1$. Because $\norm{e_\lambda}\leq 1$ for all $\lambda\in\Lambda$, we have
\[ 1 \leq \sup_{\lambda}|\varphi(e_\lambda)| \leq \sup_{\lambda}\frac{|\varphi(e_\lambda)|}{\norm{e_\lambda}} \leq \norm{\varphi}. \]
The other inequality is given by (2).
\end{proof}


\subsection{Commutative Banach algebras}
\subsubsection{The character space}
\begin{definition}
Let $A$ be a commutative Banach algebra. The \udef{character space} (or \udef{(Gelfand) spectrum}) $\hat{A}$ is the set of characters on $A$, equipped with the weak-$*$ topology $\sigma^*(\widehat{A}, A)$.
\end{definition}

All elements of the Gelfand spectrum are continuous, by \ref{charactersUnital}. If $A$ is unital, then all characters are unital.

If $A$ is a real algebra, then
\[ \hat{A} = \setbuilder{\varphi|_A}{\varphi\in \widehat{A_\C}}. \]

\begin{lemma}
The weak-$*$ topology $\sigma^*(\widehat{A}, A)$ is the same as the the subspace topology derived from the weak-$*$ topology $\sigma^*(A^*, A)$ on the dual $A^*$.
\end{lemma}
\begin{proof}

\end{proof}

\begin{proposition}
Let $A$ be a commutative Banach algebra. Then the character space $\hat{A}$ is a locally compact Hausdorff space.

Also $\hat{A}$ compact if and only if $A$ unital.
\end{proposition}
\begin{proof}

\end{proof}

\begin{proposition} \label{commutativeSameSpectrum}
Let $A$ be  a  unital  Banach  algebra,  and  suppose  that $S\subseteq A$ is a subset  of  pairwise commuting elements.  Then there exists a unital commutative Banach subalgebra $C\subseteq A$ with $S\subseteq C$ such that
\[ \forall s\in S: \quad \spec_A(s) = \spec_C(s). \]
\end{proposition}
\begin{proof}
TODO
\end{proof}

\begin{lemma} \label{commutativeBanachAlgebraIdeals}
Let $A$ be a commutative unital Banach algebra and $\mathcal{J}$ is a maximal ideal. Then 
\begin{enumerate}
\item if $A$ is complex, then $A/\mathcal{J} \cong \C$;
\item if $A$ is real, then $A/\mathcal{J} \cong \R$ or $A/\mathcal{J} \cong \C$.
\end{enumerate}
\end{lemma}
\begin{proof}
(1) If $A$ is commutative, $A/\mathcal{J}$ is a field by TODO ref. It is also a unital Banach algebra (TODO), so $A/\mathcal{J} \cong \C$ by the Gelfand-Mazur theorem, \ref{GelfandMazur}.

(2) TODO
\end{proof}

\begin{proposition} \label{characterMaximalIdealsComplex}
Let $A$ be a complex unital commutative Banach algebra. Then we have a bijection
\[ \ker: \hat{A} \twoheadrightarrowtail \{\text{maximal ideals in $A$}\}: \varphi \mapsto \ker(\varphi).  \]
\end{proposition}
\begin{proof}
First we verify that for each character $\varphi$ the kernel is a maximal ideal. Indeed applying \ref{splittingMap} to $\varphi$ we get an isomorphism $A/\ker\varphi \cong \im\varphi = \C$, meaning $\ker(\varphi)$ has codimension 1 and thus is a maximal proper subspace. By \ref{kernelIsIdeal}, $\ker(\varphi)$ is an ideal.

To prove $\ker$ is injective: let $\ker(\varphi) = \ker(\psi)$. Take some $a\in A$, which we can uniquely write as $\lambda+x$, with $x\in\ker(\varphi) = \ker(\psi)$, as $A = \Span(\vec{1})\oplus \ker(\varphi)$. Then
\[ \varphi(a) = \varphi(\lambda\cdot\vec{1} + x) = \lambda\varphi(\vec{1}) + 0 = \lambda = \lambda\psi(\vec{1}) + 0 = \psi(\lambda\cdot\vec{1} + x) = \psi(a), \]
so $\varphi = \psi$.

For surjectivity, take a maximal ideal $\mathcal{J}$.
By \ref{commutativeBanachAlgebraIdeals}, $A/\mathcal{J}\cong \C$, so the quotient map $A\to A/\mathcal{J}\cong \C$ can be seen as a character with kernel $\mathcal{J}$.
\end{proof}

TODO characters in real Banach algebra.


\subsubsection{The Gelfand transform}
\begin{definition}
Let $A$ be a unital commutative Banach algebra. The \udef{Gelfand transform} of $A$ is the map
\[ \evalMap : A\to C(\hat{A}): x\mapsto \evalMap_x \]
where $\evalMap_x(\varphi) = \varphi(x)$ for all $x\in A, \varphi\in\hat{A}$.
\end{definition}
\begin{lemma} \label{GelfandTransformHomomorphism}
Let $A$ be a unital commutative Banach algebra with Gelfand transform $\evalMap : A\to C(\hat{A})$. Then
\begin{enumerate}
\item $\evalMap$ is well-defined, in the sense that $\hat{x} \in C(\hat{A})$;
\item $\evalMap$ is linear and multiplicative;
\item $\norm{\evalMap_{x}} = \spr(x) \leq \norm{x}$.
\end{enumerate}
Thus the Gelfand transform is a unital, norm-contractive Banach algebra homomorphism.
\end{lemma}
\begin{proof}
We prove in turn:
\begin{enumerate}
\item We have equipped $\hat{A}$ with the weak-$*$ topology. Then each $\hat{x}$ is continuous.
\item $\hat{x}(\lambda \varphi + \psi) = \lambda \varphi(x) + \psi(x) = \lambda\hat{x}(\varphi) + \hat{x}(\psi)$ and $\hat{x}(\varphi\psi) = \varphi(x)\psi(x) = \hat{x}(\varphi)\hat{x}(\psi)$.
\item For all $x\in A$:
\[ \norm{\hat{x}} = \sup_{\varphi\in\hat{A}}\left( \frac{|\varphi(x)|}{\norm{\varphi}} \right) = \sup_{\varphi\in\hat{A}}(|\varphi(x)|) = \spr(x) \leq \norm{x} \]
where we have used that $\norm{\varphi} = 1$ by \ref{charactersUnital}, $\spec(x) = \setbuilder{|\varphi(x)|}{\varphi\in\hat{A}}$ by \ref{spectrumFromSpectrum} and the inequality is from \ref{spectrumCompact}.
\end{enumerate}
\end{proof}



\begin{proposition} \label{spectrumFromSpectrum}
Let $A$ be a unital commutative Banach algebra and $x\in A$. Then
\[ \spec(x) = \setbuilder{\varphi(x)}{\varphi\in\hat{A}} = \hat{x}^{\imf}(\hat{A}). \]
\end{proposition}
TODO: introduce notation $\hat{x}$ earlier?
\begin{proof}
Suppose $\lambda = \varphi(x)$ for some $\varphi\in\hat{A}$. Then $x-\lambda\in\ker\varphi$, which is a proper ideal.
So $x-\lambda \notin\GL(A)$ and $\lambda \in\spec(x)$.

Suppose $\lambda \in \spec(x)$. Because $x-\lambda$ is non-invertible, the ideal generated by it is proper, by \ref{nonInvertibleGeneratedIdeals}, and $x-\lambda$ lies in a maximal ideal, by \ref{idealLatticeCoatomic}. By \ref{characterMaximalIdealsComplex}, this means $x-\lambda \in \ker\varphi$ for some $\varphi\in\hat{A}$ and then $\lambda = \varphi(x)$.
\end{proof}

\section{Tensor products}
\subsection{Algebraic tensor product}
\begin{proposition}
Let $A$ be a Banach algebra, $a\in A$ and $p\in A$ an idempotent. Then
\begin{enumerate}
\item $e^{p\otimes a} = (\vec{1}-p)\otimes \vec{1} + p\otimes e^a$;
\item $e^{a\otimes p} = \vec{1}\otimes (\vec{1}-p) + e^a\otimes p$.
\end{enumerate}
In particular, $e^{\vec{1}\otimes a} = \vec{1}\otimes a$ and $e^{a\otimes \vec{1}} = e^a\otimes a$.
\end{proposition}
\begin{proof}
We calculate
\begin{align*}
e^{p\otimes a} &= \sum_{k=0}^\infty \frac{(p\otimes a)^k}{!k} \\
&= \sum_{k=0}^\infty \frac{p^k\otimes a^k}{!k} \\
&= \vec{1}\otimes \vec{1} + \sum_{k=1}^\infty \frac{p\otimes a^k}{!k} \\
&= \vec{1}\otimes \vec{1} + p\otimes\left(\sum_{k=1}^\infty \frac{a^k}{!k}\right) \\
&= \vec{1}\otimes \vec{1} + p\otimes\left(\sum_{k=0}^\infty \frac{a^k}{!k}\right) - p\otimes \vec{1} \\
&= (\vec{1} - p)\otimes \vec{1} + p\otimes e^a,
\end{align*}
where we have used the continuity of $p\otimes -$ (which is bounded by $\norm{p}$).
\end{proof}
\begin{corollary}
Let $A$ be a Banach algebra and $a,b\in A$. Then $e^a\otimes e^b = e^{a\otimes \vec{1} + \vec{1}\otimes b}$.
\end{corollary}
\begin{proof}
We calculate
\[ e^a\otimes e^b = (e^a\otimes \vec{1})(\vec{1}\otimes e^b) = e^{a\otimes \vec{1}}e^{\vec{1}\otimes b} = e^{a\otimes \vec{1} + \vec{1}\otimes b}, \]
using \ref{factorisationCommutingExponentials} with the fact that $a\otimes \vec{1}$ and $\vec{1}\otimes b$ commute.
\end{proof}











\chapter{Spectral theory and functional calculus}
\section{Invariant subspaces}
\begin{definition}
Let $L\in \Hom(V)$ be an endomorphism. A subspace $U$ of $V$ is \udef{invariant} under $L$ if $T|_U$ is an endomorphism on $U$. In other words, $u\in U$ implies $Tu\in U$.
\end{definition}
Clearly this definition only works for endomorphisms, not for linear maps in general. This is true for the rest of the theory about eigenvalues and eigenvectors.
\begin{example}
Let $L\in \Hom(V)$. The following are invariant under $L$:
\begin{itemize}
\item $\{0\}$;
\item $\ker L$;
\item $\im L$.
\end{itemize}
\end{example}

\section{The spectrum}
TODO: eigenvalue problem $Lx = \lambda x$

generalised eigenvalue problem $Lx = \lambda T x$

nonstandard eigenvalue problem $A(\gamma)x = 0$.

TODO: consistency $\lambda \id - L$, not $L-\lambda \id$.
TODO: everything is now in $\C$.

\begin{definition}
Let $L: \dom(L)\subset V \to V$ be an operator on a complex normed vector space $V$.

For $\lambda\in\C$ the \udef{resolvent} $R_L(\lambda): \im(\lambda \id_V - L)\to\dom(L)$ is the left inverse of $\lambda \id_V - L$, if this inverse exists (i.e.\ if $\lambda \id_V - L$ is injective).
\begin{itemize}
\item The \udef{resolvent set} $\res(L)$ is the set
\begin{align*}
\res(L) &\defeq \setbuilder{\lambda\in \C}{R_L(\lambda)\in\Bounded(V, \dom(L))} \\
&= \setbuilder{\lambda\in \C}{\text{$R_L(\lambda)$ exists, has domain $V$ and is bounded}}.
\end{align*}
\item The \udef{spectrum} of $L$ is the complement of the resolvent set: $\spec(L) \defeq \C\setminus\rho(L)$.
\item The \udef{spectral radius} $\spr(L)$ is $\sup_{\lambda\in\spec(L)} |\lambda|$.
\end{itemize}
\end{definition}

\begin{lemma}
Let $T\in\Lin(V)$ be an operator and $\lambda\in\C$ such that $\lambda\id_V - T$ is injective. Then $\im(R_T(\lambda)) = \dom(T)$.
\end{lemma}
\begin{proof}
For all $x\in \dom(T)$ we have $x = R_T(\lambda)(\lambda\id_V - T)x$.
\end{proof}

\begin{lemma} \label{elementResolventSetNormedSpace}
Let $T$ be an operator on a normed vector space $V$. Then $\lambda \in \res(T)$ \textup{if and only if} $\lambda \id_V - T$ is surjective and bounded from below.
\end{lemma}
\begin{proof}
By \ref{boundedBelow}, $\lambda \id_V - T$ has a bounded inverse $(\lambda \id_V - T)^{-1}: \im(\lambda \id_V - T)\to V$ if and only if it is bounded below. In order for $\lambda$ to be in the resolvent set, we need $(\lambda \id_V - T)^{-1}$ to be defined everywhere, i.e. $\im(\lambda \id_V - T) = V$.
\end{proof}

\begin{lemma} \label{densityCoreLemma}
Let $A$ be a closed operator on a Banach space $X$, $D$ a core for $A$ and $\lambda\in\res(A)$. Then $(\lambda\id_X - A)D$ is dense in $X$.
\end{lemma}
\begin{proof}
Take any $x\in X$. Then $R_A(\lambda)x\in \dom(A)$ and we can find a sequence $\seq{x_n}\subseteq \dom(A)$ that converges in graph norm to $R_A(\lambda)x$. Then $Ax_n \to AR_A(\lambda)x$ by \ref{graphNormConvergenceLemma} and so
\[ (\lambda\id_X - A)D \supseteq \seq{(\lambda\id_X - A)x_n} \to (\lambda\id_X - A)R_A(\lambda)x = x. \]
\end{proof}

\subsection{The three-way classification of the spectrum}
\begin{definition}
Let $L: \dom(L)\subset V \to V$ be an operator on a complex vector space $V$.

\begin{itemize}
\item The \udef{point spectrum} or \udef{discrete spectrum} $\pspec(L)$ contains the values of $\lambda$ where $\lambda \id_V - L$ fails to be injective, so the resolvent fails to exist. These values are called the \udef{eigenvalues} of $L$.

We call
\begin{itemize}
\item $\ker(\lambda \id_V - L)$ the \udef{multiplicity space} or \udef{geometric eigenspace} of $\lambda$; and
\item $\dim\ker(\lambda \id_V - L)$ the \udef{(geometric) multiplicity} of $\lambda$.
\end{itemize}
\item The \udef{continuous spectrum} $\cspec(L)$ is the set of all values of $\lambda\in\spec(L)$ such that the resolvent $R_L(\lambda)$ exists and is densely defined.
\item The \udef{residual spectrum} $\rspec(L)$ is the set of all values of $\lambda\in\spec(L)$ such that the resolvent $R_L(\lambda)$ exists, but is not densely defined.

We call
\begin{itemize}
\item $\im(\lambda \id_V - L)^\perp$ the \udef{deficiency subspace} of $\lambda$; and 
\item $\dim(\im(\lambda \id_V - L)^\perp)$ the \udef{deficiency} of $\lambda$.
\end{itemize}
\end{itemize}
The sets $\pspec(T), \cspec(T)$ and $\rspec(T)$ are disjoint.
\end{definition}
In finite dimensions we know that
\[ \text{$\lambda \id_V - L$ is surjective} \quad\iff\quad \text{$\lambda \id_V - L$ is injective} \]
and all linear operator are bounded.
So in this case there can only ever be a point spectrum.

\begin{proposition} \label{spectrumNonClosedOperator}
If $T$ is an operator on a Banach space that is not closed, then $\spec(T) = \C$.
\end{proposition}
\begin{proof}
We can find a sequence $x_n \to x$ such that $Tx_n \to y$, but $Tx \neq y$. Then for all $\lambda\in\C$ we have $z_n = (\lambda\id - T)x_n \to \lambda x - y$. If $R_T(\lambda)$ was a bounded inverse of $(\lambda\id - T)$, then $R_T(\lambda)\circ(\lambda\id - T)x_n \to R_T(\lambda)(\lambda x - y)$. We need to show that $R_T(\lambda)(\lambda x - y) \neq x$. Indeed
\begin{align*}
R_T(\lambda)(\lambda x - y) &= R_T(\lambda)(\lambda x - Tx + Tx - y) \\
&= R_T(\lambda)(\lambda x - Tx) + R_T(\lambda)(Tx - y) \\
&= x + R_T(\lambda)(Tx - y),
\end{align*}
and $R_T(\lambda)(Tx - y) \neq 0$, because $Tx - y \neq 0$ and the kernel of $R_T(\lambda)$ is trivial because it is injective. 
\end{proof}

\begin{example}
Closed operators may also have empty resolvent set. \url{https://math.stackexchange.com/questions/3262168/closed-operator-with-trivial-resolvent-set}
\end{example}

So spectral theory is only interesting for closed operators. In this case the three-way classification exhausts the possibilities: (only on Banach spaces??)

\begin{proposition} \label{closedOperatorBanachSpaceSpectrumCriterion}
Let $X$ be a Banach space and $T$ a closed linear operator on $X$. Then $\lambda \in \spec(T)$ \textup{if and only if} $\lambda \id_X - T: \dom(T) \to V$ is not bijective.
\end{proposition}
\begin{proof}
If $\lambda \id_X - T$ is not bijective, then clearly $\lambda \in \spec(T)$.

Conversely, assume $\lambda \id_X - T$ is bijective. Then $(\lambda \id_X - T)^{-1}: X\to \dom(T)$ is closed by \ref{algebraClosedOperators} and has as domain a Banach space, so it is bounded by the closed graph theorem \ref{closedGraphTheorem}.
\end{proof}
\begin{corollary}
Let $T$ a closed operator on a Banach space. Then
\[ \spec(T) = \pspec(T) \cup \cspec(T) \cup \rspec(T). \]
\end{corollary}


\begin{proposition}
Let $T:X\to X$ be an operator on a Banach space and $\lambda\in\cspec$, then $R_\lambda(T)$ is unbounded.
\end{proposition}
\begin{proof}
If $R_\lambda(T)$ is bounded, $\lambda \id_V - T$ then is bounded below by lemma \ref{boundedBelow} and has closed range by proposition \ref{boundedBelowClosedRange}. Then because $\im(\lambda \id_V - T)$ is dense, this means $T$ is surjective, which is a contradiction because then $\lambda\in\res(T)$.
\end{proof}

\subsection{Resolvents}

\begin{lemma}
Let $T$ be a linear operator on a Banach space $X$ and $\lambda\in\C$. Then
\[ TR_T(\lambda) = \lambda R_T(\lambda) - \id_X. \]
\end{lemma}
\begin{proof}
We have $\id_X = (\lambda \id_X - T)R_T(\lambda) = \lambda R_T(\lambda) - TR_T(\lambda)$.
\end{proof}

\begin{lemma}
Let $T$ be a linear operator and $\lambda,\mu\in\C$. Assume $\lambda\in \spec(T)$. Then $\mu\lambda\in \spec(\mu T)$ and
\[ R_T(\lambda) = \mu R_{\mu T}(\mu \lambda). \]
\end{lemma}
\begin{proof}
TODO
\end{proof}


\subsubsection{Pseudoresolvents}

\begin{lemma} \label{imageRangePseudoresolvents}
Let $\mathcal{R}:\Lambda \subseteq \C \to \Bounded(X)$ be a pseudoresolvent on a Banach space $X$ and $\lambda,\mu\in\Lambda$. Then
\begin{enumerate}
\item $\ker\mathcal{R}(\lambda) = \ker\mathcal{R}(\mu)$;
\item $\im\mathcal{R}(\lambda) = \im\mathcal{R}(\mu)$.
\end{enumerate}
In particular this means that $\mathcal{R}(\lambda)$ is injective \textup{if and only if} $\ker\mathcal{R}(\mu)$ is injective.
\end{lemma}
\begin{proof}
From
\[ \mathcal{R}(\lambda) = \mathcal{R}(\mu)\big(\id_X + (\mu-\lambda)\mathcal{R}(\lambda)\big) = \big(\id_X + (\mu-\lambda)\mathcal{R}(\lambda)\big)\mathcal{R}(\mu), \]
we see that $\im\mathcal{R}(\lambda) \subseteq \im\mathcal{R}(\mu)$ and $\ker\mathcal{R}(\lambda) \supseteq \ker\mathcal{R}(\mu)$. Swapping $\lambda$ and $\mu$ gives the result.
\end{proof}


\begin{proposition}
Let $\mathcal{R}:\Lambda \subseteq \C \to \Bounded(X)$ be a pseudoresolvent. Then $\mathcal{R} = R_T|_\Lambda$ for some operator $T$ \textup{if and only if} $\mathcal{R}(\lambda)$ is injective for some $\lambda\in\Lambda$.

In this case $\im(\mathcal{R}(\lambda)) = \dom(T)$ for all $\lambda\in \Lambda$ and the operator $T$ is unique.
\end{proposition}
\begin{proof}
$\boxed{\Rightarrow}$ Because in particular $\Lambda = \dom(\mathcal{R}) = \dom(R_T|_\Lambda)$, we need to have that $\Lambda \subseteq \res(T)$. The resolvent $R_T(\lambda): X\to \dom(T)$ is bijective for all $\lambda\in \res(T)\subseteq \Lambda$. It is in particular injective.

$\boxed{\Leftarrow}$ By \ref{imageRangePseudoresolvents}, $\mathcal{R}(\lambda)$ is injective for all $\lambda\in\Lambda$. By restricting the codomain of $\mathcal{R}(\lambda)$ to its image, $\mathcal{R}(\lambda)$ becomes invertible. We can define $T: \im(\mathcal{R}(\lambda)) \to X = \lambda\id - \mathcal{R}(\lambda)^{-1}$. Then
\[ \mathcal{R}(\lambda)(\lambda\id - T) = \mathcal{R}(\lambda)\mathcal{R}(\lambda)^{-1} = \id_{\im(\mathcal{R}(\lambda))} = \id_{\dom(T)}. \]
Thus $\mathcal{R}(\lambda)$ is the resolvent of $T$ at $\lambda$.
The definition of $T$ is the only one that makes $\mathcal{R}(\lambda)$ a resolvent of $T$ at $\lambda$, so $T$ is unique.
\end{proof}
\begin{corollary}
Let $\mathcal{R}:\Lambda \subseteq \C \to \Bounded(X)$ be a pseudoresolvent and assume that $\Lambda$ contains an unbounded sequence $\seq{\lambda_n}$. If either
\begin{enumerate}
\item $\lim_{n\to\infty}\lambda_n\mathcal{R}(\lambda_n)x = x$ for all $x\in X$; or
\item $\im\mathcal{R}(\lambda)$ is dense in $X$ for some $\lambda\in \Lambda$ and $\norm{\lambda_n \mathcal{R}(\lambda_n)} \leq M$ for some $M\geq 0$ and all $n\in\N$;
\end{enumerate}
then $\mathcal{R}$ is the resolvent map of a densely defined operator.
\end{corollary}
\begin{proof}
(1) If $x\in \ker\mathcal{R}(\lambda)$ for some $\lambda\in\Lambda$, then it is in the kernel for all $\lambda\in\Lambda$. Thus $\lim_{n\to\infty}\lambda_n\mathcal{R}(\lambda_n)x = 0 = x$, meaning that $\ker\mathcal{R}(\lambda) = \{0\}$ and $\mathcal{\lambda} = R_T(\lambda)$ for some operator $T$. Also
\[ X = \overline{\bigcup_{n\in\N}\im\mathcal{R}(\lambda_n)} = \overline{\im\mathcal{R}(\lambda)} = \overline{\dom(T)}, \]
meaning $T$ is densely defined.

(2) From the resolvent identity we have, for some $\lambda\in \Lambda$,
\[ \lambda_n\mathcal{R}(\lambda_n)\mathcal{R}(\lambda) - \mathcal{R}(\lambda) = \lambda\mathcal{R}(\lambda_n)\mathcal{R}(\lambda) - \mathcal{R}(\lambda_n) \]
and thus
\begin{align*}
\norm{(\lambda_n\mathcal{R}(\lambda_n) - \id)\mathcal{R}(\lambda)} &= \norm{\lambda\mathcal{R}(\lambda_n)\mathcal{R}(\lambda) - \mathcal{R}(\lambda_n)} \\
&\leq |\lambda|\,\norm{\mathcal{R}(\lambda_n)}\,\norm{\mathcal{R}(\lambda)} + \norm{\mathcal{R}(\lambda_n)} \to 0
\end{align*}
because $\norm{\mathcal{R}(\lambda_n)} \leq |\lambda_n^{-1}|\,M \to 0$. So for all $x\in\im\mathcal{R}(\lambda)$, we have $\lim_{n\to\infty}\lambda_n\mathcal{R}(\lambda_n)x = x$.

Now take $x\in X$. Because of density, we can find a sequence $\seq{x_n}$ in $\im\mathcal{R}(\lambda)$ that converges to $x$. By the continuity of the maps $\lambda_n\mathcal{R}(\lambda_n)$ and their uniform bound in conjunction with (TODO ref!!!), we get
\begin{align*}
\lim_{n\to\infty}\lambda_n\mathcal{R}(\lambda_n)x &= \lim_{n\to\infty}\lambda_n\mathcal{R}(\lambda_n)\lim_{k\to\infty}x_k \\
&= \lim_{n\to\infty}\lim_{k\to\infty}\lambda_n\mathcal{R}(\lambda_n)x_k \\
&= \lim_{k\to\infty}\lim_{n\to\infty}\lambda_n\mathcal{R}(\lambda_n)x_k \\
&= \lim_{k\to\infty}x_k = x.
\end{align*}
We conclude with point (1).
\end{proof}

\subsubsection{Properties of the spectrum}

\begin{proposition} \label{resolventNormDistanceToSpectrum}
For all $\lambda\in\res(T)$, we have $d(\lambda, \spec(T)) \geq \norm{R_T(\lambda)}^{-1}$.
\end{proposition}
\begin{proof}
For all $\mu\in \ball(\lambda, \norm{R_T(\lambda)}^{-1})$ we can define $R_T(\mu)$ by analytic continuation as in \ref{firstNeumannSeries}. By \ref{imageRangePseudoresolvents} we have that $R_T(\mu): X\to \dom(T)$ is bijective and bounded. We just need to show that it is a left inverse of $\mu\id_X - T$. We calculate
\begin{align*}
\mathcal{R}(\mu)(\mu\id_X - T) &= \big(\id_X + (\mu - \lambda)\mathcal{R}(\lambda)\big)^{-1}\mathcal{R}(\lambda)(\mu\id_X - T) \\
&= \big(\id_X + (\mu - \lambda)\mathcal{R}(\lambda)\big)^{-1}\mathcal{R}(\lambda)\big((\mu - \lambda)\id_X + (\lambda\id_X - T)\big) \\
&= \big(\id_X + (\mu - \lambda)\mathcal{R}(\lambda)\big)^{-1}\big((\mu - \lambda)\mathcal{R}(\lambda) + \id_X\big) \\
&= \id_X.
\end{align*}
\end{proof}
\begin{corollary}
The resolvent set $\res(T)$ is open. The spectrum $\spec(T)$ is closed.
\end{corollary}
This is stronger than \ref{spectrumCompact}, because $T$ is not assumed closed.


\begin{example}
Operator with empty spectrum. TODO \url{https://math.stackexchange.com/questions/1344287/example-operator-with-empty-spectrum}.
\end{example}

\begin{proposition}
Let $T$ be an injective operator with dense range. Then for all $\lambda\neq 0$
\[ R_{T^{-1}}(\lambda^{-1}) = -\lambda T R_{T}(\lambda) = \lambda -\lambda^2 R_T(\lambda). \]
\end{proposition}
\begin{proof}
This is a reformulation of the calculation
\[ \frac{1}{\lambda^{-1} - T^{-1}} = \frac{\lambda T}{\lambda T}\frac{1}{\lambda^{-1} - T^{-1}} = \frac{\lambda T}{T - \lambda} = \frac{\lambda T - \lambda^2 + \lambda^2}{T - \lambda} = \frac{\lambda\cancel{(T - \lambda)}}{\cancel{T - \lambda}} + \frac{\lambda^2}{T - \lambda} = \lambda - \lambda^2 R_T(\lambda). \]
TODO: make rigourous!!
\end{proof}
\begin{corollary}
Let $T$ be an injective operator with dense range. Then for all $\lambda\neq 0$
\begin{enumerate}
\item $\spec(T^{-1})\setminus\{0\} = (\spec(T)\setminus \{0\})^{-1}$;
\item $\pspec(T^{-1})\setminus\{0\} = (\pspec(T)\setminus \{0\})^{-1}$.
\end{enumerate}
\end{corollary}


\subsection{Parts of the spectrum}

\subsubsection{The point spectrum: eigenvalue and eigenvectors}
In this section we study invariant subspaces with dimension $1$, i.e.\ subspaces $U= \Span\{v\}$ such that
\[ Lv = \lambda v. \]
\begin{definition}
Suppose $L\in \Hom_{\mathbb{F}}(V)$.
\begin{itemize}
\item  A scalar $\lambda\in \mathbb{F}$ is called an \udef{eigenvalue} of $L$ if there exists a $v\in V$ such that $v\neq 0$ and $Lv = \lambda v$.
\item Such a vector $v$ is called an \udef{eigenvector}.
\item The set of all eigenvectors associated with an eigenvalue $\lambda$ is called the \udef{eigenspace} $E_\lambda(L)$. Because
\[ E_\lambda(L) = \ker(L-\lambda \id_V) \]
it is indeed a vector space.

The dimension of $E_\lambda(L)$ is the \udef{geometric multiplicity} of $\lambda$.
\end{itemize}
\end{definition}
If $L$ is a closed operator, then its eigenspaces are closed by \ref{closedOperatorKernelClosed}.

For a bounded operator $T$, we have $\pspec(T)\subset \cball(0, \norm{T})$ by \ref{spectrumCompact}. For the point spectrum a simpler argument also leads to $\pspec(T)\subset \cball(0, \norm{T})$: let $\lambda$ be an eigenvalue with eigenvector $x$. Then
\[ |\lambda|\;\norm{x} = \norm{\lambda x} = \norm{Tx} \leq \norm{T}\;\norm{x}. \]

\begin{proposition}
Let $L\in \Hom_\mathbb{F}(V)$ and $\lambda\in \mathbb{F}$, then
\[ \text{$\lambda$ is an eigenvalue of $L$} \qquad \iff \qquad \text{$\lambda$ is in the point spectrum $\pspec(L)$.} \]
\end{proposition}
\begin{proof}
The equation $Lv = \lambda v$ is equivalent to $(L-\lambda \id_V)v = 0$.
\end{proof}

\begin{proposition}
Let $L\in\Hom(V)$ be an operator on some vector space. Suppose $\lambda_1, \ldots, \lambda_m$ are distinct eigenvalues of $L$ and $v_1,\ldots, v_m$ are corresponding eigenvectors. Then $\{v_1,\ldots, v_m\}$ is linearly independent.
\end{proposition}
\begin{proof}
The proof goes by contradiction. Assume $\{v_1,\ldots, v_m\}$ is linearly dependent. Let $k$ be the smallest positive integer such that
\[ v_k \in \Span\{v_1,\ldots, v_{k-1}\}. \]
So there exists a nontrivial linear combination
\[ v_k = a_1v_1+\ldots +a_{k-1}v_{k-1}. \]
Applying $L$ to both sides gives
\[ \lambda_kv_k = a_1\lambda_kv_1+\ldots +a_{k-1}\lambda_kv_{k-1}. \]
Multipliying the previous combination by $\lambda_k$ and subtracting both equations gives
\[ 0= a_1(\lambda_k-\lambda_1)v_1 +\ldots + a_{k-1}(\lambda_k - \lambda_{k-1})v_{k-1}. \]
By assumption of linear independence of $\{v_1,\ldots, v_{k-1}\}$ this combination must be trivial, however none of the $(\lambda_k-\lambda_i)$ can be zero, so all the $a_i$ must be zero. This is a contradiction with the assumption of linear dependence.
\end{proof}
\begin{corollary}
For each operator on $V$, the set of distinct eigenvalues has at most cardinality $\dim V$.
\end{corollary}
\begin{corollary}
Let $L\in\Hom(V)$. Suppose $\lambda_1, \ldots, \lambda_m$ are distinct eigenvalues of $L$. Then
\[ E_{\lambda_1}(L) \oplus \ldots \oplus E_{\lambda_m}(L) \]
is a direct sum. Furthermore, the sum of geometric multiplicities is less than or equal to the dimension of $V$:
\[ \dim E_{\lambda_1}(L) + \ldots + \dim E_{\lambda_m}(L) \leq \dim V. \]
\end{corollary}

\subsubsection{Approximate spectrum and Weyl sequences}
\begin{definition}
The set of all $\lambda$ such that $T-\lambda \id_V$ is not bounded from below is called the \udef{approximate point spectrum} $\apspec$.

If $\lambda\in\apspec(T)$, then $\lambda$ is an \udef{approximate eigenvalue} of $T$.
\end{definition}

\begin{proposition} \label{approximateSpectrum}
Let $T$ be an operator. Then
\begin{enumerate}
\item $\apspec(T) \subset \spec(T)$;
\item if $T$ is closed, then $\pspec(T)\cup\cspec(T)\subset\apspec(T)$.
\end{enumerate}
\end{proposition}
\begin{proof}
(1) Assume $\lambda \notin \spec(T)$. Then $(T-\lambda \id_V)^{-1}$ is bounded, so its inverse $T-\lambda \id_V$ is bounded below by \ref{boundedBelow} and $\lambda\in \apspec(T)$.

(2) Assume $\lambda\notin \apspec(T)$, 
so $T-\lambda \id_V$ is bounded below. Then $T-\lambda \id_V$ is injective by \ref{boundedBelow} and $\lambda\notin\pspec(T)$. By proposition \ref{boundedBelowClosedRange} the range $\im(T-\lambda \id_V)$ is closed, so it cannot be a proper dense subset of $X$ and $\lambda\notin\cspec(T)$.
\end{proof}

\begin{proposition}[Weyl sequences] \label{WeylSequence}
Let $T$ be an operator on a normed vector space $V$. Then $\lambda \in \apspec(T)$ \textup{if and only if} there exists a sequence of unit vectors $(e_n)_{n\in\N}$ for which
\[ \lim_{n\to\infty}\norm{\lambda e_n - Te_n} = 0. \]
\end{proposition}
\begin{proof}
Assume there is such a sequence $(e_n)_{n\in\N}$. Then for all $\epsilon>0$, we can find a unit  vector $e_k$ such that $\norm{(\lambda \id_V - T)e_n} \leq \epsilon = \epsilon \norm{e_n}$. This is clearly not bounded below.

This other direction is just an inversion of this argument.
\end{proof}
A sequence as described in \ref{WeylSequence} is called a \udef{Weyl sequence} for $\lambda$. This gives meaning to the name ``approximate eigenvalue''.

\begin{corollary}
Let $T$ be an operator. Then $\sigma(T)\cap \overline{\res(T)} \subseteq \apspec(T)$.
\end{corollary}
\begin{proof}
Let $\lambda \in \sigma(T)\cap \overline{\res(T)}$. We show there is a Weyl sequence for $\lambda$.

We can find a sequence $\seq{\lambda_n}\subseteq \res(T)$ such that $\lambda_n \to \lambda$.
Now $d(\lambda_n, \spec(T)) \to 0$, so by \ref{resolventNormDistanceToSpectrum}, we can find a sequence of unit vectors $\seq{x_n}$ such that $\norm{R_T(\lambda_n)x_n} \to \infty$. Now we can rescale $\seq{x_n}$ such that $\norm{R_T(\lambda_n)x_n} = 1$.

Then $\norm{x_n}\to 0$, and hence
\begin{align*}
\norm{(\lambda\id - T)R_T(\lambda_n)x_n} &= \norm{\frac{\lambda\id - T}{\lambda_n\id - T}x_n} \\
&= \norm{\frac{(\lambda\id - T) + (\lambda_n\id - T) - (\lambda_n\id - T)}{\lambda_n\id - T}x_n} \\
&= \norm{\left(\id + \frac{(\lambda\id - T) - (\lambda_n\id - T)}{\lambda_n\id - T}\right)x_n} \\
&= \norm{\left(\id + \frac{\lambda\id - \lambda_n\id}{\lambda_n\id - T}\right)x_n} \\
&= \norm{x_n + (\lambda\id - \lambda_n\id)R_T(\lambda_n)x_n} \\
&\leq \norm{x_n} + |\lambda - \lambda_n|\;\norm{R_T(\lambda_n)x_n} \to 0.
\end{align*}
Thus $\seq{R_T(\lambda_n)x_n}$ is the kind of sequence we were looking for.
\end{proof}




\subsubsection{Compression spectrum}
\begin{definition}
The set of $\lambda$ for which $T-\lambda I$ does not have dense range is the \udef{compression spectrum} $\cpspec(T)$ of $T$.
\end{definition}
Then $\rspec(T) = \cpspec(T)\setminus\pspec(T)$.

\subsubsection{The essential spectrum}
TODO \url{https://en.wikipedia.org/wiki/Spectrum_(functional_analysis)#Classification_of_points_in_the_spectrum}


\subsection{The spectral radius}
\begin{definition}
The \udef{spectral radius} $\spr(T)$ of a operator $T$ is given by
\[ \spr(T) \defeq \sup_{\lambda\in\spec(T)}|\lambda|. \]
\end{definition}


\section{Spectral theory for types of operators}
\subsection{Compact operators}

\begin{proposition}
Let $K$ be a compact operator on a Banach space. Then
\[ \spec(K)\setminus\{0\} = \pspec(K)\setminus\{0\}. \]
\end{proposition}
\begin{proof}
For all $\lambda\neq 0$, we have that $\lambda\id - K$ is Fredholm with index zero (and thus bounded). Then by the Fredholm alternative \ref{FredholmAlternative} $\lambda\id - K$ is either bijective or neither injective nor surjective, meaning $\lambda$ is either in $\rho(T)$ or in $\pspec(T)$. 
\end{proof}

\begin{proposition} \label{spectrumCompactOperator}
Let $K$ be a compact operator on a Banach space $X$. Then
\begin{enumerate}
\item for all $\lambda\in\spec(K)\setminus\{0\}$ there exists a least $m$ such that $\ker(\lambda\id- K)^m = \ker(\lambda\id- K)^{m+1}$. This space is finite dimensional and reducing for $K$;
\item for $\alpha > 0$ the number of eigenvalues $\lambda$ such that $|\lambda|\geq \alpha$ is finite;
\item $0$ is the only accumulation point; if $X$ is infinite dimensional, then $0\in\spec(K)$;
\item $\spec(K)$ is at most countably infinite;
\item every $\lambda \in \spec(K)\setminus \{0\}$ is a pole of the resolvent $R_K$.
\end{enumerate}
\end{proposition}
\begin{proof}
\url{https://en.wikipedia.org/wiki/Spectral_theory_of_compact_operators}
\end{proof}



\subsection{Multiplication operators}
\begin{definition}
Let $(\Omega, \mathcal{A}, \mu)$ be a measure space. A \udef{multiplication operator} is an operator of the form
\[ T: L^p(\Omega, \mu) \to L^p(\Omega, \mu): u(x) \mapsto a(x)u(x) \]
for some $a\in L^\infty(\Omega,\mu)$
\end{definition}

\begin{proposition}
Let $T: L^p(\Omega, \mu) \to L^p(\Omega, \mu): u \mapsto a\cdot u$ be a multiplication operator. Then
\[ \norm{T} = \norm{a}_{L^\infty}. \]
\end{proposition}
\begin{proof}
From the inequality $\norm{Tu}_{L^p}\leq \norm{a}_{L^\infty}\norm{u}_{L^p}$ we get $\norm{T} \leq \norm{a}_{L^\infty}$.

TODO
\end{proof}

\begin{lemma}
Let $T: L^2(\Omega, \mu) \to L^2(\Omega, \mu): u \mapsto a\cdot u$ be a multiplication operator with $a\in L^\infty(\Omega,\mu)$. Then $T^*$ is the multiplication operator
\[ T^*: L^2(\Omega, \mu) \to L^2(\Omega, \mu): u \mapsto \overline{a}\cdot u. \]
\end{lemma}
\begin{proof}
From 
\[ \inner{Tu,v} = \int_\Omega a\cdot u \cdot \overline{v}\diff{\mu} = \int_\Omega u \cdot \overline{\overline{a}\cdot v}\diff{\mu} \]
it follows that $T^*v = \overline{a}\cdot v$.
\end{proof}
\begin{corollary}
Then
\begin{enumerate}
\item $T$ is self-adjoint if $a$ is real-valued;
\item $T$ is skew-adjoint if $a$ is purely imaginary;
\item $T$ is unitary if $|a(x)| \equiv 1$.
\end{enumerate}
\end{corollary}

Let $E_\lambda$ be the level set
\[ E_\lambda = \setbuilder{x\in\Omega}{a(x) = \lambda} \]

\begin{proposition}
Let $T: L^2(\Omega, \mu) \to L^2(\Omega, \mu): u\mapsto a\cdot u$ be a multiplication operator with $a\in \cont(\Omega)$. Then
\begin{enumerate}
\item $\pspec(T) = \setbuilder{\lambda\in \im(a)}{\mu(E_\lambda)>0}$;
\item $\cspec(T) = \setbuilder{\lambda\in \overline{\im(a)}}{\mu(E_\lambda) = 0}$;
\item $\rspec(T) = \emptyset$;
\item $\rho(T) = \C\setminus \overline{\im(T)}$.
\end{enumerate}
\end{proposition}
\begin{proof}
TODO
\end{proof}

\subsection{Dissipative operators}
\begin{definition}
Let $T\in \Lin(V, W)$ be a linear operator between Banach spaces. Then $T$ is called \udef{dissipative} if $\lambda\id-T$ is bounded below by $\lambda$ for all $\lambda>0$:
\[ \norm{(\lambda\id-T)x} \geq \lambda\norm{x} \]
for all $x\in\dom(T)$.
\end{definition}

\begin{lemma} \label{dissipativeResolventBound}
Let $T\in \Lin(V, W)$ be an operator between Banach spaces. Then $T$ is dissipative \textup{if and only if} for all $\lambda>0$ the resolvent $R_T(\lambda): \im(T)\to V$ exists and is bounded by $\norm{R_T(\lambda)} \leq \lambda^{-1}$.
\end{lemma}
\begin{proof}
If $T$ is dissipative, then the result is given by \ref{boundedBelow}.

Assume $R_T(\lambda): \im(T)\to V$ exists. Then
\[ \lambda\norm{x} = \lambda\norm{R_T(\lambda)(\lambda\id-T)x} \leq \lambda \norm{R_T(\lambda)}\,\norm{(\lambda\id-T)x} \leq \lambda\lambda^{-1}\norm{(\lambda\id-T)x} = \norm{(\lambda\id-T)x}. \]
\end{proof}

Thus $\lambda>0$ is in $\res(T)$ if and only if $\lambda\id - T$ is surjective.


\begin{proposition} \label{spectrumDissipativeOperator}
Let $T\in \Lin(V, W)$ be a dissipative operator between Banach spaces. Then either $]0,+\infty[\,\perp \res(T)$ or $]0,+\infty[\,\subseteq \res(T)$.
\end{proposition}
\begin{proof}
By \ref{dissipativeResolventBound}, it is enough to show that if $\lambda\in\res(T)$ for some $\lambda >0$, then $\lambda\id - T$ is surjective for all $\lambda>0$.

Assume $\lambda\in\res(T)$ for some $\lambda >0$. Then \ref{dissipativeResolventBound} and \ref{firstNeumannSeries} combine the give $]0, 2\lambda[ \subseteq \res(T)$. We can repeat this to cover the whole of $]0,+\infty[$.
\end{proof}
\begin{corollary} \label{rangeDisjunctionDissipativeOperator}
Either
\begin{enumerate}
\item $\lambda\id-T$ is surjective for no $\lambda > 0$; or
\item $\lambda\id-T$ is surjective for all $\lambda > 0$.
\end{enumerate}
\end{corollary}

\begin{proposition} \label{closureDissipativeOperator}
Let $T\in \Lin(V, W)$ be a dissipative operator between Banach spaces. Then the following are equivalent:
\begin{enumerate}
\item $T$ is closed;
\item $\im(\lambda\id - T)$ is closed for some $\lambda > 0$;
\item $\im(\lambda\id - T)$ is closed for all $\lambda > 0$.
\end{enumerate}
\end{proposition}
\begin{proof}
For all $\lambda\in\R$, we have that $T$ is closed iff $\lambda\id - T$ is closed iff $(\lambda\id - T)^{-1}: \im(\lambda\id - T) \to V$ is closed by \ref{algebraClosedOperators}.

Now closedness of $\lambda\id - T$ implies $\im(\lambda\id - T)$ is closed by \ref{boundedBelowClosedRange}. Conversely, if $\im(\lambda\id - T)$ is closed, then $\lambda\id - T$ is closed by the closed graph theorem \ref{closedGraphTheorem}.
\end{proof}

\begin{proposition} \label{dissipativeOperatorClosable}
Let $T\in \Lin(V)$ be a dissipative operator on a Banach space $V$. If $\im(T)\subseteq \overline{\dom(T)}$, then
\begin{enumerate}
\item $T$ is closable;
\item its closure $\overline{T}$ is dissipative;
\item $\im(\lambda\id - \overline{T}) = \overline{\im(\lambda\id - T)}$ for all $\lambda >0$.
\end{enumerate}
\end{proposition}
In particular $\im(T)\subseteq \overline{\dom(T)}$ holds whenever $T$ is densely defined.
\begin{proof}
(1) We use \ref{closableCriterion}. Assume $\seq{x_n}\to 0$ and $\seq{Tx_n}\to v$. We need to show that $v=0$. Because $T$ is dissipative, we have
\[ \norm{\lambda(\lambda\id-T)x_n + (\lambda\id-T)w} = \norm{(\lambda\id-T)(\lambda x_n -w)} \geq \lambda\norm{\lambda x_n + w} \]
for all $w\in \in\dom(T)$ and all $\lambda>0$. Taking the limit $n\to \infty$ gives
\[ \norm{-\lambda v +(\lambda\id- T)w} \geq \lambda\norm{w}, \qquad\text{and hence}\qquad \norm{w - v - \frac{1}{\lambda}Tw} \geq w. \]
Taking the limit $\lambda \to \infty$ gives $\norm{w-v}\geq \norm{w}$. Now $y\in \overline{\im(T)} \subseteq \overline{\dom(T)}$. Thus we can find a sequence $\seq{w_n}\to y$ in $\dom(T)$. This sequence then satisfies $\norm{w_n-y} \geq \norm{w_n}$. Taking the limit gives $0\geq \norm{y}$, so $y = 0$.

(2) For all $x\in\dom(\overline{T})$ there exists a sequence $\seq{x_n}\to x$ in $\dom(T)$ such that $\seq{Tx_n} \to \overline{T}x$ by \ref{graphNormConvergenceLemma}. Now for all $n\in \N$,
\[ \norm{(\lambda\id-T)x_n} \geq \lambda\norm{x_n}. \]
Taking the limit $n\to\infty$ gives $\norm{(\lambda\id-\overline{T})x} \geq \lambda\norm{x}$, meaning $\overline{T}$ is dissipative.

(3) By \ref{domImClosureOperator}, $\im(\lambda\id - T)$ is dense in $\im(\lambda\id-\overline{T})$ and by \ref{closureDissipativeOperator}, $\im(\lambda\id-\overline{T})$ is closed.
\end{proof}

\begin{proposition}
Let $T\in\Lin(V,W)$ be an operator between Banach spaces. Then $T$ is dissipative \textup{if and only if} for all $x\in V$, there exists an $x'\in \mathfrak{d}(x)$ such that
\[ \Re\pair{x', Tx} \leq 0. \]
\end{proposition}
\begin{proof}
$\boxed{\Leftarrow}$ Take $x\in V$ and fix some $x'\in \mathfrak{d}(x)$ for which the inequality holds. By definition of the duality set, we have $\norm{x}^2 = x'(x) = \norm{x'}^2$. To verify dissipativity, we calculate
\begin{align*}
\norm{(\lambda\id-T)x}\,\norm{x'} &\geq |x'\big((\lambda\id-T)x\big)| \\
&\geq \Re x'\big((\lambda\id-T)x\big) = \Re x'(\lambda x) - \Re x'(Tx) \\
&\geq \lambda \Re x'(x) = \lambda \norm{x}^2.
\end{align*}
Noting $\norm{x'} = \norm{x}$ and dividing both sides by $\norm{x}$ yields dissipativity.

$\boxed{\Rightarrow}$ TODO
\end{proof}

\section{The spectral theorem}
\url{https://link.springer.com/content/pdf/10.1007%2F978-1-4614-7116-5.pdf}

\url{http://individual.utoronto.ca/jordanbell/notes/SVD.pdf}
\url{https://digitalcommons.mtu.edu/cgi/viewcontent.cgi?article=2133&context=etdr}

\url{https://web.ma.utexas.edu/mp_arc/c/09/09-32.pdf}


\section{Functional calculus}
\subsection{Holomorphic functional calculus}

\begin{theorem}[Holomorphic functional calculus] 
\label{holomorphicFunctionalCalculus} \label{holomorphicSpectralMapping}
Let $A$ be a Banach algebra and $x\in A$. Consider the function
\[ \Phi_x: \cont^\infty(\spec(x),\C) \to A: f\mapsto f(x)\defeq \oint_\Gamma f(z)R_x(z)\diff{z}. \]
Here $\Gamma$ is any simple Jordan curve that contains $\spec(x)$ such that $f$ is holomorphic in a region that contains $\Gamma$ and its interior. Then
\begin{enumerate}
\item $\Phi_x$ is well-defined: it does not depend on the particular curve $\Gamma$;
\item $\Phi_x$ is a homomorphism;
\item for any polynomial $p\in \C[X]$, we have $\Phi_x(p) = p(x)$; \\
in particular $\Phi_x(\id_\C) = x$ and $\Phi_x(\underline{1}) = \id_A$;
\item $\spec(\Phi_x(f)) = f[\spec(x)]$;
\item $\Phi_x$ is continuous if $\cont^\infty(\spec(x),\C)$ is equipped with continuous convergence (?).
\end{enumerate}
\end{theorem}
TODO: $\cont^\infty(\spec(x))$ should be the space of functions that are analytic in some neighbourhood of $\spec(x)$. Is it??
\begin{proof}
TODO
\end{proof}

TODO unbounded operators

\subsubsection{Riesz eigenprojections}
Holomorphic functional calculus applied to
\[ \chi_{S,\delta}: A\to \{0,1\}: x\mapsto \begin{cases}
1 & d(x,S) \leq \delta \\
0 & \text{otherwise}.
\end{cases} \]

TODO: spectral measure with only disconnected parts in $\sigma$-algebra??

TODO: $P_\Delta$ and $E_\Delta \defeq \im P_\Delta$.

\begin{lemma}
$\spec(T|_{E_\Delta}) = \spec(T)\cap\Delta$.
\end{lemma}

\begin{definition}
We call $\dim E_\lambda$ the \udef{algebraic multiplicity} of $\lambda$.
\end{definition}

\subsubsection{Frobenius covariants}
TODO $P_\lambda$ is a Frobenius covariant. \url{https://en.wikipedia.org/wiki/Frobenius_covariant}

TODO cfr. Lagrange polynomial??

\section{Jordan decomposition}
\subsection{Eigennilpotent}
\begin{definition}
Let $a$ be a finite element in a semisimple Banach algebra and $\lambda\in \spec(a)$. The \udef{eigennilpotent operator} of $a$ at $\lambda$ is defined as
\[ D_{\lambda} \defeq (a-\lambda)P_{\lambda}. \]
\end{definition}
This definition works because we can find a $\delta < d(\lambda, \spec(a)\setminus\{\lambda\})$.

\begin{lemma}
Let $a$ be a finite element in a semisimple Banach algebra and $\lambda\in \spec(a)$. The eigennilpotent operator $D_\lambda$ is nilpotent.
\end{lemma}
\begin{proof}
By spectral mapping \ref{holomorphicSpectralMapping}, $D_\lambda$ is quasinilpotent. Because $a$ is finite, it is nilpotent by \ref{nilpotentQuasinilpotent}.
\end{proof}



\subsection{Jordan vectors}
\begin{definition}
Let $V$ be a finite dimensional vector space and $T$ an operator on $V$. A \udef{Jordan vector} of $T$ belonging to the eigenvalue $\lambda$ is a vector $x\in V$ such that
\[ (\lambda\id_V - T)^kx = 0 \]
for some $k\in \N$. The least such $k$ is called the \udef{degree} of $x$ and is denoted $\deg_J(x)$.
\end{definition}
Eigenvectors are Jordan vectors of degree $1$.

\begin{proposition}
Let $V$ be a finite dimensional vector space, $T$ an operator on $V$ $\lambda\in\spec(T)$ and $x\in V$. Then $x$ is a Jordan vector of $T$ belonging to the eigenvalue $\lambda$ \textup{if and only if} $x\in E_\lambda$.
\end{proposition}
\begin{proof}
Let $x\in E_\lambda$. Then $x = P_\lambda x$ and thus
\[ (\lambda\id_V - T)^kx = (\lambda\id_V - T)^kP_\lambda x = \big((\lambda\id_V - T)P_\lambda\big)^k x = D_\lambda^k x, \]
which is zero for some $k$ because $D_\lambda$ is nilpotent.

Conversely, assume $x$ is a Jordan vector of $T$ belonging to the eigenvalue $\lambda$. We can write $x = x_1+x_2 \in E_{\lambda}\oplus E_{\C\setminus\{\lambda\}}$.
Then (because $E_\lambda$ is reducing for $T-\lambda\id_V$)
\[ 0 = (\lambda\id_V - T)^kx = (\lambda\id_V - T)^kx_1 + (\lambda\id_V - T)^kx_2 \in E_{\lambda}\oplus E_{\C\setminus\{\lambda\}} \]
Thus we have $(\lambda\id_V - T)^kx_1 = 0$ and $(\lambda\id_V - T)^kx_2 = 0$ separately.
Now $T-\lambda\id_V$ is invertible on $E_{\C\setminus\{\lambda\}}$, so $x_2 = 0$ (TODO ref). This means that $x = x_1 \in E_\lambda$.
\end{proof}

\begin{definition}
Let $m = \deg_N(D_\lambda)$. Then we have
\[ \{0\} \subsetneq \ker(\lambda\id_V - T) \subsetneq \ker(\lambda\id_V - T)^2 \subsetneq \ldots \subsetneq \ker(\lambda\id_V - T)^{m-1} \subsetneq \ker(\lambda\id_V - T)^m = V. \]
We define $E^k_\lambda \defeq \ker(\lambda\id_V - T)^k$. In particular
\begin{itemize}
\item $E^1_\lambda$ is the \udef{geometric eigenspace};
\item $E^{m-1}_\lambda$ is the \udef{algebraic eigenspace}.
\end{itemize}
\end{definition}

\begin{lemma}
Let $V$ be a finite dimensional vector space, $T$ an operator on $V$, $\lambda\in\spec(T)$ and $x\in E_\lambda$. Then
\begin{enumerate}
\item $1 \leq \dim\ker(\lambda\id_V - T) \leq \dim E_\lambda$;
\item $1 \leq \deg_J(x) \leq \dim_E\lambda$.
\end{enumerate}
\end{lemma}
The lemma says the geometric multiplicity is smaller than the algebraic multiplicity.
\begin{proof}
Every eigenvector is a Jordan vector, so $\ker(\lambda\id_V - T) \subseteq E_\lambda$.

For all $k\in\N$ smaller then the degree of $x$, $(\lambda\id_V - T)^kx$ is a Jordan vector and thus in $E_\lambda$. TODO all $(\lambda\id_V - T)^kx$ are linearly independent (like in \ref{nilpotentQuasinilpotent})
\end{proof}

\begin{definition}
Let $V$ be a finite dimensional vector space, $T$ an operator on $V$ and $\lambda\in\spec(T)$. The eigenvalue $\lambda$ is called
\begin{itemize}
    \item \udef{simple} if the algebraic multiplicity is $1$;
    \item \udef{semisimple} if every Jordan vector in $E_\lambda$ has degree $1$;
    \item \udef{prime} if the geometric multiplicity is $1$.
\end{itemize}
If all eigenvalues of $T$ are semisimple, then $T$ is called a \udef{diagonal operator}.
\end{definition}

\begin{lemma}
An operator $T$ is diagonal iff $T$ is of the form $\sum_j a_jP_j$, where $a_j\in \F$ and $P_j$ are projectors that commute pairwise.
\end{lemma}

\subsection{Characteristic polynomial and equation}
\begin{definition}
Let $V$ be a finite dimensional vector space and $T$ an operator on $V$. The \udef{characteristic polynomial} $p_T(x)$ of $T$ is the polynomial
\[ p_T(x) \defeq \det(x\id_V - T). \]
\end{definition}

\begin{proposition}
Let $V$ be a finite dimensional vector space, $T$ an operator on $V$ and $\spec(T) = \{\lambda_j\}_{j=1}^r$. Then
\[ p_T(x) = \prod_{j=1}^r(x - \lambda_j)^{\dim E_{\lambda_j}}. \]
\end{proposition}
\begin{proof}
TODO
\end{proof}
\begin{corollary}
A number $\lambda\in \C$ is an eigenvalue of $T$ \textup{if and only if} it is a root of $p_T(x)$.
\end{corollary}

\begin{definition}
The equation $p_T(x) = 0$ is the \udef{characteristic equation} of $T$.
\end{definition}

\subsection{Spectral representation}
\begin{proposition}
Let $V$ be a finite dimensional complex vector space and $T$ an operator on $V$. There exists a unique decomposition $T = S + D$ such that
\begin{itemize}
\item $S$ is diagonal;
\item $D$ is nilpotent;
\item $SD = DS$.
\end{itemize}
If $\spec(T) = \{\lambda_j\}_{j=1}^r$, this decomposition is given by
\[ T = \sum_{j=1}^r \lambda_r P_{\lambda_r} + \sum_{j=1}^r D_{\lambda_r}. \]
\end{proposition}

\subsection{Partial fraction decomposition of the resolvent}
For any operator $T$ on a vector space $V$ with eigenvalue $\lambda_0$, the resolvent $R_T(\lambda)$ has a pole at $\lambda_0$.

\begin{proposition}
Let $T$ be an operator on a finite dimensional vector space $V$ and $\lambda_0\in\spec(T)$. Then the Laurent expansion of $R_T(\lambda)$ around $\lambda_0$ is of the form
\[ R_T(\lambda) = \frac{P_0}{\lambda-\lambda_0} + \sum_{n=1}^{\deg_N(D_0)-1}\frac{D_0^{n}}{(\lambda - \lambda_0)^{n+1}} + \sum_{n=0}^\infty(-1)^n S_0^{n+1}(\lambda - \lambda_0)^n, \]
where $P_0\defeq P_{\lambda_0}, D_0\defeq D_{\lambda_0}$ and $S_0$ is some fixed operator.
\end{proposition}
\begin{proof}
TODO
\end{proof}

\begin{definition}
The holomorphic part of the Laurent expansion of $R_T(\lambda)$ at $\lambda_0$ is called the \udef{reduced resolvent} of $T$ w.r.t. $\lambda_0$:
\[ S_{T,\lambda_0}(\lambda) \defeq \sum_{n=0}^\infty(-1)^n S_0^{n+1}(\lambda - \lambda_0)^n = R_T(\lambda) - \left(\frac{P_0}{\lambda-\lambda_0} + \sum_{n=1}^{\deg_N(D_0)-1}\frac{D_0^{n}}{(\lambda - \lambda_0)^{n+1}}\right). \]
\end{definition}

\begin{proposition}
Let $T$ be an operator on a finite dimensional vector space $V$ and $\lambda_0\in\spec(T)$. Then
\[ R_{T|_{(\id_V-P_0)}}(\lambda) = S_{T,\lambda_0}|_{\id_V-P_0}(\lambda). \]
\end{proposition}

\begin{proposition}
Let $T$ be an operator on a finite dimensional vector space $V$ with $\spec(T) = \{\lambda_j\}_{j=1}^r$. The partial fraction decomposition of $R_T(\lambda)$ is given by
\[ R_T(\lambda) = \sum_{j=1}^r\left(\frac{P_{\lambda_j}}{\lambda - \lambda_j} +\sum_{n=1}^{\deg_N(D_{\lambda_j})-1}\frac{D_{\lambda_j}^n}{(\lambda - \lambda_j)^{n+1}}\right). \]
The partial fraction decomposition of $S_{T,\lambda_k}(\lambda)$ is given by
\[ S_{T,\lambda_k}(\lambda) = \sum_{\substack{j=1 \\ j\neq k}}^r\left(\frac{P_{\lambda_j}}{\lambda - \lambda_j} +\sum_{n=1}^{\deg_N(D_{\lambda_j})-1}\frac{D_{\lambda_j}^n}{(\lambda - \lambda_j)^{n+1}}\right). \]
\end{proposition}
\begin{proof}
The poles of $R_T(\lambda)$ are exactly the eigenvalues of $T$. There are finitely many of them, so we can use partial fraction decomposition, \ref{partialFractionDecomposition}. We just need to show that the holomorphic part is zero. For that we note that $\lim_{\lambda \to \infty} R_T(\lambda) = 0$ and all principal parts tend to $0$ at infinity as well. Thus the holomorphic part also tends to $0$, making it bounded. By Liouville's theorem, \ref{liouvilleTheoremAnalysis}, we get that it is identically zero.
\end{proof}
\begin{corollary}[Sylvester-Lagrange formula]
Let $f$ be a holomorphic function on an open set that contains $\spec(T)$. Then
\[ f(T) = \sum_{j=1}^r\left(f(\lambda_j)P_{\lambda_j} +\sum_{n=1}^{\deg_N(D_{\lambda_j})-1}\frac{f^{(n)}(\lambda_j)D_{\lambda_j}^n}{n!}\right). \] 
\end{corollary}
\begin{proof}
We have
\[ f(T) = \oint_\Gamma f(\lambda)R_T(\lambda)\diff{\lambda} = 2\pi i\sum_{j=1}^r \Res_{\lambda_j}f(\lambda)R_T(\lambda) \]
by the residue theorem (TODO ref for operators).
\end{proof}
\begin{corollary}[Cayley-Hamilton]
Let $p_T(x)$ be the characteristic polynomial of $T$. Then $p_T(T) = 0$.
\end{corollary}
\begin{proof}
Since $p_T(x) = \prod_{j=1}^r(x - \lambda_j)^{\dim E_{\lambda_j}}$ and $\dim E_{\lambda_j} \geq \deg_N(D_{\lambda_j})$, we see that $p_T(\lambda)R_T(\lambda)$ has no poles and is holomorphic, meaning that $oint_\Gamma f(\lambda)R_T(\lambda)\diff{\lambda} = 0$ by Cauchy's theorem (TODO ref for operators).
\end{proof}

\subsection{Normal operators}


\begin{proposition}
If $T$ is a normal operator, then $P_\lambda = P^*_\lambda$ and $D_\lambda = D^*_\lambda = 0$.
\end{proposition}
This means normal operators are diagonalisable.
\begin{proof}
TODO
\end{proof}
\begin{corollary}
Let $V$ be a finite dimensional complex vector space and $T$ a normal operator
on $V$ with $\spec(T) = \{\lambda_j\}_{j=1}^r$.
\begin{enumerate}
\item We have the spectral decompositions
\[ T = \sum_{j=1}^r \lambda_r P_{\lambda_r} \qquad\text{and}\qquad T^* = \sum_{j=1}^r \overline{\lambda_r} P_{\lambda_r}. \]
\item We have
\[ R_T(\lambda) = \sum_{j = 1}^r \frac{P_{\lambda_j}}{\lambda - \lambda_j} \qquad \text{and} \qquad S_{T,\lambda_k}(\lambda) = \sum_{\substack{j = 1 \\ j\neq k}}^r \frac{P_{\lambda_j}}{\lambda - \lambda_j} \]
\end{enumerate}
\end{corollary}

\subsection{Jordan decomposition}
TODO matrix representation + matrix representation of Lagrange-Sylvester. See Baumgärtel



