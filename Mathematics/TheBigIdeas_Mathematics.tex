\documentclass{report}
\usepackage{imakeidx}
\usepackage[greek, english]{babel}
%\usepackage[T1]{fontenc}
\usepackage[utf8x]{inputenc}
\usepackage{amsthm}
%\usepackage{mbboard/texinputs/mbboard}
\usepackage{amssymb}
\usepackage{amsmath}
\usepackage{stmaryrd}
\usepackage{thmtools}
\usepackage{mathtools}
\usepackage{etoolbox}
\usepackage{scalerel}
\usepackage{siunitx}
\usepackage{tikz}
\usepackage{ulem}
\usepackage{contour}
\usepackage{xcolor}
\usepackage{bbold}
\usepackage{url}
\usepackage{slashed}
\usepackage{simpler-wick}
\usepackage{tikz-feynman}
\usepackage{../tikz-uml}
\usepackage{cancel}
\usepackage{graphicx}
\usepackage{wrapfig}
\usepackage{csquotes}
%\usepackage{commath}
\usepackage{subcaption}
\usepackage{pgfplots}
\usepackage{tensor}
\usepackage{verbatim}
\usepackage{esint}
\usepackage[shortlabels]{enumitem}
\usepackage[skins, breakable]{tcolorbox}
%\usepackage{bbm}
%\usepackage{bm}
%\usepackage{autonum}
\usepackage{diagbox}


\usepackage{listings}
\usepackage[ruled, linesnumbered]{algorithm2e}

\usepackage[margin=1.4in]{geometry}
\usepackage[hidelinks, hypertexnames=false]{hyperref}
\usepackage[user, xr]{zref}

\makeindex[name=definition,title={Index of definitions}]

\tcbset{breakable}

\usetikzlibrary{cd, fit, patterns, snakes, decorations.markings, trees}
\usepgfmodule{nonlineartransformations}


% --- Theorems and such ---
\newtheorem{theorem}{Theorem}[part]
\newtheorem{corollary}{Corollary}[theorem]
\newtheorem{lemma}[theorem]{Lemma}
\newtheorem{proposition}[theorem]{Proposition}
\newtheoremstyle{sublemma}% 〈name〉
{3pt}% 〈Space above〉1
{3pt}% 〈Space below 〉1
{}% 〈Body font〉
{}% 〈Indent amount〉2
{\itshape}% 〈Theorem head font〉
{:}% 〈Punctuation after theorem head 〉
{.5em}% 〈Space after theorem head 〉3
{\undline{\thmname{#1}}\thmnumber{ #2}\thmnote{(#3)}}% 〈Theorem head spec (can be left empty, meaning ‘normal’ )〉
\theoremstyle{sublemma}
\newtheorem*{lemma*}{Lemma}
% --- New commands ---
\newcommand{\R}{\mathbb{R}}
\newcommand{\N}{\mathbb{N}}
\newcommand{\Z}{\mathbb{Z}}
\newcommand{\Q}{\mathbb{Q}}
\newcommand{\C}{\mathbb{C}}
\newcommand{\T}{\mathbb{T}}
%\newcommand{\H}{\mathbb{H}}
\newcommand{\F}{\mathbb{F}}
\newcommand{\G}{\mathbb{G}}

\DeclarePairedDelimiter\ceil{\lceil}{\rceil}
\DeclarePairedDelimiter\floor{\lfloor}{\rfloor}

\DeclareMathOperator{\fractional}{frac}
\DeclareMathOperator{\integer}{int}

\newcommand{\ttransp}{t}
\newcommand{\transp}{\mathrm{T}}

% --- Inner products / brackets ---
\newcommand{\inner}[1]{\left\langle #1 \right\rangle}
\newcommand{\ket}[1]{\left| #1 \right\rangle}
\newcommand{\bra}[1]{\left\langle #1 \right|}
\newcommand{\braket}[3][\null]{%    %TOreDO!
  \ifx#1\null
       \langle#2|#3\rangle%
    \else%
       \langle#2|#1|#3\rangle%
    \fi}
\newcommand{\sbraket}[3][\null]{%    %TOreDO!
  \ifx#1\null
       \left\langle#2\vphantom{#3}\right.\left|#3\vphantom{#2}\right\rangle%
    \else%
         \left\langle#2\vphantom{#1}\vphantom{#3}\right.\left|#1\vphantom{#2}\vphantom{#3}\right|\left.#3\vphantom{#1}\vphantom{#2}\right\rangle%
    \fi}
\newcommand{\ketbra}[2]{|#1\rangle\langle#2|}
\newcommand{\sketbra}[2]{\left|#1\vphantom{#2}\right\rangle\left\langle#2\vphantom{#1}\right|}

\newcommand{\grade}[1]{\left\langle #1 \right\rangle}
% --- Set builder notation ---
\newcommand{\setbuilder}[2]{ \left\{ #1 \;\middle|\; #2 \right\} }
% --- Script r ---
\def\rcurs{{\mbox{$\resizebox{.09in}{.08in}{\includegraphics[trim= 1em 0 14em 0,clip]{ScriptR}}$}}}
\def\brcurs{{\mbox{$\resizebox{.09in}{.08in}{\includegraphics[trim= 1em 0 14em 0,clip]{BoldR}}$}}}
\def\hrcurs{{\mbox{$\hat \brcurs$}}}
% --- Vector style ---
\let\point\vec
\renewcommand{\vec}[1]{\boldsymbol{\mathrm{#1}}}
%\let\ihat\hat
%\let\operator\hat
\newcommand{\vhat}[1]{\vec{\hat{#1}}}
% --- Complex vectors ---
\newcommand{\vbar}[1]{\vec{\bar{#1}}}
% --- Norm ---
\makeatletter
\DeclareDocumentCommand\braces{}{{\ifnum\z@=`}\fi\@braces}
\DeclareDocumentCommand\@braces{ s t\big t\Big t\bigg t\Bigg m m m }
{ % General braces with automatic and manual sizing
	\IfBooleanTF{#1}
	{\left#6\smash{#8}\right#7\vphantom{#8}}
	{
		\IfBooleanTF{#2}{\bigl#6{#8}\bigr#7}{
			\IfBooleanTF{#3}{\Bigl#6{#8}\Bigr#7}{
				\IfBooleanTF{#4}{\biggl#6{#8}\biggr#7}{
					\IfBooleanTF{#5}{\Biggl#6{#8}\Biggr#7}{\left#6{#8}\right#7}
				}
			}
		}
	}
	\ifnum\z@=`{\fi}
}
%\DeclareDocumentCommand\norm{ l m }{\braces#1{\lVert}{\rVert}{#2}} % Norm
\makeatother

\newcommand\swapifbranches[3]{#1{#3}{#2}}
\makeatletter
\MHInternalSyntaxOn
\patchcmd{\DeclarePairedDelimiter}{\@ifstar}{\swapifbranches\@ifstar}{}{}
\MHInternalSyntaxOff
\makeatother

\DeclarePairedDelimiter{\norm}{\lVert}{\rVert}
% --- Algorithms ---
\let\oldnl\nl% Store \nl in \oldnl
\newcommand{\nonl}{\renewcommand{\nl}{\let\nl\oldnl}}% Remove line number for one line
\let\oldKwSty\KwSty
\renewcommand{\KwSty}[1]{\nonl\textnormal{\textbf{#1}}\unskip}
\newenvironment{centeredAlgorithm}[1][0.5]
{\par\centering
\begin{minipage}{.5\linewidth}
  \begin{algorithm}[H]
}
{
  \end{algorithm}
\end{minipage}
\par
}
\SetKw{Dots}{\hspace{1em}$\hdots$ \\}
\SetKwBlock{Subroutine}{\vspace{-1em}}{}
% --- Defining quantities ---
\newcommand{\defeq}{\coloneqq}
\newcommand{\eqdef}{\eqqcolon}
\newcommand{\defequiv}{\quad\Leftrightarrow_{\text{def}}\quad}
% --- Maths operators ---
\DeclareMathOperator{\Class}{Class}
\DeclareMathOperator{\Element}{Element}
\DeclareMathOperator{\ProperClass}{ProperClass}
\DeclareMathOperator{\Set}{Set}
\DeclareMathOperator{\Pair}{Pair}
\DeclareMathOperator{\acc}{acc}
\DeclareMathOperator\powerset{\mathcal{P}}
\DeclareMathSymbol\mesh{\mathrel}{operators}{`\#}
\newcommand\aset[1]{\mathrel{\widehat{#1}}}
\newcommand\amesh{\aset{\#}}
\newcommand\aperp{\aset{\perp}}
\newcommand{\symdiff}{\mathbin{\Delta}}

\newcommand{\sSet}[1]{\left( #1 \right)}

\newcommand\inj{\rightarrowtail}
\newcommand\surj{\twoheadrightarrow}
\newcommand\bij{\twoheadrightarrowtail}

\DeclareMathOperator\relconvex{conv}

\DeclareMathOperator\Card{Card}
\DeclareMathOperator\Ord{Ord}

\DeclareMathOperator{\graph}{graph}
\DeclareMathOperator{\len}{len}
\newcommand\seq[1]{\left\langle #1 \right\rangle}

\newcommand\cat[1]{\mathsf{#1}}
\DeclareMathOperator{\ob}{ob}
\DeclareMathOperator{\mor}{mor}

\DeclareMathOperator\id{id}
\DeclareMathOperator\dom{dom}
\DeclareMathOperator\codom{codom}
\DeclareMathOperator\co{co}
\DeclareMathOperator\im{im}
\DeclareMathOperator\preim{preim}
\newcommand\imf{\downarrow}
\newcommand\preimf{{\text{\hspace{0.15em}-}\downarrow}}
\DeclareMathOperator\sunp{\Rsh}

\newcommand\syq{\mathbin{\obar}}

\DeclareMathOperator\swap{swap}
\DeclareMathOperator\curry{curry}
\newcommand\constant\underline

\DeclareMathOperator\Fixedpoints{Fp}

\newcommand\leftconnections[1]{\mathop{\swarrow}#1}
\newcommand\rightconnections[1]{#1\mathop{\searrow}}

\DeclareMathSymbol\greensL{\mathrel}{symbols}{"4C}
\DeclareMathSymbol\greensR{\mathrel}{symbols}{"52}
\DeclareMathSymbol\greensH{\mathrel}{symbols}{"48}
\DeclareMathSymbol\greensD{\mathrel}{symbols}{"44}

\DeclareMathOperator\evalMap{ev}

\newcommand\commute{\leftrightarrow}

\DeclareMathOperator\Hom{Hom}

\DeclareMathOperator\upset{\uparrow}
\DeclareMathOperator\downset{\downarrow}
\newcommand\from\leftarrow

\newcommand\proj{\pi}
\newcommand\pbCorner{\arrow[dr, phantom, "\ulcorner", very near start]}

\DeclareMathOperator{\Closure}{Cl}

\DeclareMathOperator{\lcm}{lcm}
%\DeclareMathOperator{\gcd}{gcd}

\DeclareMathOperator\directed{\mathcal{D}}
\newcommand{\dirvee}{\mathop{\mathchoice{%
\setlength{\unitlength}{0.95em}\linethickness{0.2mm}\raisebox{-.5em}% Display
    {\begin{picture}(1,1.5)\put(.5,0){\line(-1,3){.48}}
    \put(.5,0){\vector(1,3){.5}}\end{picture}}}{%
\setlength{\unitlength}{.7em}\linethickness{0.18mm}\raisebox{-.2em}% Text
    {\begin{picture}(1,1.5)\put(.5,0){\line(-1,3){.48}}
    \put(.5,0){\vector(1,3){.5}}\end{picture}}}{%
\setlength{\unitlength}{.7em}\raisebox{-.2em}% Script
    {\begin{picture}(1,1.5)\put(.5,0){\line(-1,3){.48}}
    \put(.5,0){\vector(1,3){.5}}\end{picture}}}{%
\setlength{\unitlength}{.7em}\raisebox{-.2em}% Scriptscript
    {\begin{picture}(1,1.5)\put(.5,0){\line(-1,3){.48}}
    \put(.5,0){\vector(1,3){.5}}\end{picture}}}}%
}

\DeclareMathOperator\ideals{\mathcal{I}}
\DeclareMathOperator\filters{\mathcal{F}}

\DeclareMathOperator\joinIr{\mathcal{J}}
\DeclareMathOperator\meetIr{\mathcal{M}}

\DeclareMathOperator\atoms{\mathcal{A}}
\DeclareMathOperator\coatoms{\mathcal{CA}}

\DeclareMathOperator\ultrafilters{\mathbb{U}}

\newcommand\domain{\mathcal{D}}

\newcommand\pfilter{\dot}
\DeclareMathOperator\powerfilters{\mathcal{FP}}
\DeclareMathOperator\powerideals{\mathcal{IP}}
\DeclareMathOperator\powerultrafilters{\mathbb{U}\mathcal{P}}
\DeclareMathOperator\powerdirected{\mathcal{DP}}

\DeclareMathOperator\lmax{lmax}
\DeclareMathOperator\lmin{lmin}
\DeclareMathOperator\llmax{llmax}
\DeclareMathOperator\llmin{llmin}

\newcommand\initSeq{\sqsubseteq}
\newcommand\concat{\star}
\newcommand{\reverse}[1]{{#1}^\mathrm{R}}

\DeclareMathOperator\strInterleave{interleave}
\DeclareMathOperator\strConcat{concat}
\DeclareMathOperator\tailsOf{tailsOf}
\DeclareMathOperator\splitTerms{splitTerms}
\DeclareMathOperator\syntaxTree{syntaxTree}
\DeclareMathOperator\positions{pos}
\DeclareMathOperator\treeToTerm{treeToTerm}
\DeclareMathOperator\leafAt{leafAt}
\DeclareMathOperator\varSubs{varSubs}

\DeclareMathOperator\branch{branch}

\newcommand{\joins}{\mathrel{\downarrow}}
\newcommand{\normalform}\breve

\newcommand\LB{\scalebox{0.5}[1]{LB}}
\newcommand\RB{\scalebox{0.5}[1]{RB}}
\newcommand\SEP{\scalebox{0.5}{SEP}}

\DeclareMathOperator\neighbourhood{\mathcal{N}}
\DeclareMathOperator\vicinity{\mathcal{V}}
\DeclareMathOperator\adh{adh}
\DeclareMathOperator\inh{inh}
\DeclareMathOperator{\closure}{cl}
\DeclareMathOperator{\interior}{int}
\DeclareMathOperator\topology{\mathcal{T}}
\DeclareMathOperator\Tails{Tails}
\DeclareMathOperator\TailsFilter{TailsFilter}

\DeclareMathOperator\Seq{Seq}

\DeclareMathOperator\ball{B}
\DeclareMathOperator\cball{\overline{B}}
\DeclareMathOperator\sphere{S}

\DeclareMathOperator\entourage{\mathcal{E}}

\DeclareMathOperator\diam{diam}
\DeclareMathOperator\rad{rad}

\DeclareMathOperator\NF{NF}

\DeclareMathOperator{\st}{st}

\newcommand\genIdeal[1]{\left(\left(#1\right)\right)}
\newcommand\genIdealBuilder[2]{\left(\left(#1\middle|#2\right)\right)}
\DeclareMathOperator{\group}{gp}
\DeclareMathOperator\Inn{Inn}

\let\Im\relax
\DeclareMathOperator\Im{\mathfrak{I}m}
\let\Re\relax
\DeclareMathOperator\Re{\mathfrak{R}e}
\DeclareMathOperator\supp{supp}

%\renewcommand{\Re}{\operatorname{Re}}
%\renewcommand{\Im}{\operatorname{Im}}

\DeclareMathOperator{\SF}{SF}

\DeclareMathOperator{\GL}{GL}
\DeclareMathOperator{\SL}{SL}
\DeclareMathOperator{\Ogroup}{O}
\DeclareMathOperator{\SO}{SO}
\DeclareMathOperator{\SU}{SU}
\DeclareMathOperator{\U}{U}

\DeclareMathOperator{\Iso}{Iso}

\DeclareMathOperator{\Pin}{Pin}
\DeclareMathOperator{\Spin}{Spin}

\DeclareMathOperator{\glAlg}{\mathfrak{gl}}
\DeclareMathOperator{\slAlg}{\mathfrak{sl}}
\DeclareMathOperator{\uAlg}{\mathfrak{u}}
\DeclareMathOperator{\suAlg}{\mathfrak{su}}
\DeclareMathOperator{\oAlg}{\mathfrak{o}}
\DeclareMathOperator{\soAlg}{\mathfrak{so}}

\DeclareMathOperator{\diag}{diag}
\DeclareMathOperator{\Ad}{Ad}
\DeclareMathOperator{\ad}{ad}
\DeclareMathOperator{\sgn}{sgn}
\DeclareMathOperator\atanh{arctanh}
\DeclareMathOperator\sech{sech}
\DeclareMathOperator\csch{csch}

\DeclareMathOperator\Span{span}
\DeclareMathOperator\Lin{\mathcal{L}}
\DeclareMathOperator\codim{codim}
\DeclareMathOperator\coker{coker}

\DeclareMathOperator\convex{conv}
\DeclareMathOperator\conic{cone}
\DeclareMathOperator\affine{aff}
\DeclareMathOperator\balanced{bal}
\DeclareMathOperator\balancedCore{balcore}
\DeclareMathOperator\semibalanced{semibal}
\DeclareMathOperator\semibalancedCore{semibalcore}
\DeclareMathOperator\disked{cobal}

\DeclareMathOperator\epigraph{epi}

\newcommand{\pair}[1]{\left\langle #1 \right\rangle}
\newcommand{\abspair}[1]{\left|\left\langle #1 \right\rangle\right|}
\let\pol\oslash
%\DeclareMathSymbol\pol{\mathrel}{operators}{`\oslash}

\DeclareMathOperator\meromorphic{\mathcal{M}}

\newcommand\res{\rho}
\newcommand\spec{\sigma}
\newcommand\pspec{\sigma_\text{p}}
\newcommand\cspec{\sigma_\text{c}}
\newcommand\rspec{\sigma_\text{r}}
\newcommand\cpspec{\sigma_\text{cp}}
\newcommand\apspec{\sigma_\text{ap}}
\DeclareMathOperator\spr{spr}

\DeclareMathOperator\NumRange{W}
\DeclareMathOperator\nr{nr}

\DeclareMathOperator\Row{row}
\DeclareMathOperator\Col{col}
\DeclareMathOperator\Null{null}
\DeclareMathOperator\Rank{rank}
\DeclareMathOperator\KruskalRank{K}

\newcommand\Expval[1]{\mathbb{E}\left[#1\right]}
\newcommand\expval[1]{\left\langle #1 \right\rangle}
\DeclareMathOperator\E{E}
\DeclareMathOperator\Var{Var}

\DeclareMathOperator\vectorisation{vec}
\DeclareMathOperator\Tr{Tr}
\DeclareMathOperator\adj{adj}
\DeclareMathOperator\End{End}
\DeclareMathOperator\Aut{Aut}
\DeclareMathOperator\Bounded{\mathcal{B}}
\DeclareMathOperator\Compact{\mathcal{K}}
\DeclareMathOperator\Fred{\mathcal{F}}

\DeclareMathOperator\meas{\mathcal{M}}
\DeclareMathOperator\cont{\mathcal{C}}
\DeclareMathOperator\ucont{\mathcal{UC}}

\DeclareMathOperator\HLmax{\mathcal{M}}

\DeclareMathOperator\testFuncs{\mathcal{D}}
\DeclareMathOperator\dists{\mathcal{D}^\prime}

\DeclareMathOperator\Cl{Cl}
\DeclareMathOperator\cCl{\mathbb{C}l}

\DeclareMathOperator\Ind{Ind}
\DeclareMathOperator\Index{idx}

\NewDocumentCommand{\grad}{}{%
  \mathop{}\!% \mathop for good spacing before \nabla
  \nabla}
\NewDocumentCommand{\curl}{}{%
  \mathop{}\!% \mathop for good spacing before \nabla
  \nabla\times}
\DeclareMathOperator\vnabla{\vec{\nabla}}

\DeclareMathOperator\Res{Res}

\newcommand\Normals{\mathop{\mathcal{N}}\nolimits}
\newcommand\SelfAdjoints{\mathop{\mathcal{S}\hspace{-0.15em}\mathcal{A}}\nolimits}
\newcommand\Projections{\mathop{\mathcal{P}}\nolimits}
\newcommand\Unitaries{\mathop{\mathcal{U}}\nolimits}

\DeclareMathSymbol{\mlq}{\mathord}{operators}{``}
\DeclareMathSymbol{\mrq}{\mathord}{operators}{`'}

\newcommand\adual[1]{{#1}^*}
\newcommand\abidual[1]{{#1}^{**}}
\newcommand\dual[1]{{#1}^*}
\newcommand\tdual[1]{{#1}^\prime}
\newcommand\tbidual[1]{{#1}^{\prime\prime}}
\newcommand\comm[1]{{#1}\raisebox{-0.15em}{$\scaleobj{1.3}{\mrq}$}}

\DeclareMathOperator\Bernoulli{Bernoulli}
\DeclareMathOperator\Binomial{Bin}
\DeclareMathOperator\Poisson{Poisson}
\DeclareMathOperator\GammaDist{Gamma}
\DeclareMathOperator\Erlang{Erlang}
\DeclareMathOperator\Exponential{Exp}

% --- Intervals ---
\newcommand\interval[2][c]{%
\edef\inp{#1}%
\def\cCase{c}%
\def\oCase{o}%
\def\ocCase{oc}%
\def\coCase{co}%
\ifx\inp\cCase{\mathopen[#2\mathclose]}
\else \ifx\inp\oCase{\mathopen]#2\mathclose[}
\else \ifx\inp\ocCase{\mathopen]#2\mathclose]}
\else \ifx\inp\coCase{\mathopen[#2\mathclose[}
\fi\fi\fi\fi}

% --- Define custom environments ---
\newtcolorbox{note}{enhanced,sharp corners=all,colback=white,colframe=black,toprule=-1pt,bottomrule=-1pt,leftrule=1pt,rightrule=-1pt, overlay unbroken={
\draw[black,line width=1pt] (frame.north west) -- ++(.2,0);
\draw[black,line width=1pt] (frame.south west) -- ++(.2,0);
}, overlay first={
\draw[black,line width=1pt] (frame.north west) -- ++(.2,0);
}, overlay last={
\draw[black,line width=1pt] (frame.south west) -- ++(.2,0);
}, left=2mm, top=2mm, bottom=2mm}

%\newtcolorbox{practical}{enhanced,sharp corners=all,colback=white,colframe=black,toprule=0pt,bottomrule=0pt,leftrule=1pt,rightrule=0pt, overlay unbroken={
%\draw[black,line width=1pt] (frame.north west) -- ++(.2,0);
%\draw[black,line width=1pt] (frame.south west) -- ++(.2,0);
%}, overlay first={
%\draw[black,line width=1pt] (frame.north west) -- ++(.2,0);
%}, overlay last={
%\draw[black,line width=1pt] (frame.south west) -- ++(.2,0);
%}, left=2mm, top=2mm, bottom=2mm}

\newtcolorbox{example}{enhanced,sharp corners=all,colback=white,colframe=black,toprule=-1pt,bottomrule=-1pt,leftrule=1pt,rightrule=-1pt, overlay unbroken={
\draw[black,line width=1pt] (frame.north west) -- ++(.2,0);
\draw[black,line width=1pt] (frame.south west) -- ++(.2,0);
}, overlay first={
\draw[black,line width=1pt] (frame.north west) -- ++(.2,0);
}, overlay last={
\draw[black,line width=1pt] (frame.south west) -- ++(.2,0);
}, title={\underline{Example}}, attach boxed title to top,
boxed title style={empty,size=minimal,toprule=2pt,top=4pt},
coltitle=black, left=2mm, top=2mm, bottom=2mm}

\newtcolorbox{problem}{enhanced,sharp corners=all,colback=white,colframe=black,toprule=-1pt,bottomrule=-1pt,leftrule=1pt,rightrule=-1pt, overlay unbroken={
\draw[black,line width=1pt] (frame.north west) -- ++(.2,0);
\draw[black,line width=1pt] (frame.south west) -- ++(.2,0);
}, overlay first={
\draw[black,line width=1pt] (frame.north west) -- ++(.2,0);
}, overlay last={
\draw[black,line width=1pt] (frame.south west) -- ++(.2,0);
}, title={\underline{Problem statement}}, attach boxed title to top,
boxed title style={empty,size=minimal,toprule=2pt,top=4pt},
coltitle=black, left=2mm, top=2mm, bottom=2mm}

\newtcolorbox{params}{enhanced,sharp corners=all,colback=white,colframe=black,toprule=-1pt,bottomrule=-1pt,leftrule=1pt,rightrule=-1pt, overlay unbroken={
\draw[black,line width=1pt] (frame.north west) -- ++(.2,0);
\draw[black,line width=1pt] (frame.south west) -- ++(.2,0);
}, overlay first={
\draw[black,line width=1pt] (frame.north west) -- ++(.2,0);
}, overlay last={
\draw[black,line width=1pt] (frame.south west) -- ++(.2,0);
}, left=2mm, top=2mm, bottom=2mm}

\newtcolorbox{definition}{enhanced,sharp corners=all,colback=white,colframe=red,toprule=-1pt,bottomrule=-1pt,leftrule=1pt,rightrule=-1pt, overlay unbroken={
\draw[red,line width=1pt] (frame.north west) -- ++(.2,0);
\draw[red,line width=1pt] (frame.south west) -- ++(.2,0);
}, overlay first={
\draw[red,line width=1pt] (frame.north west) -- ++(.2,0);
}, overlay last={
\draw[red,line width=1pt] (frame.south west) -- ++(.2,0);
}, %title={DEF}, attach boxed title to top, boxed title style={empty,size=minimal,toprule=2pt,top=4pt}, coltitle=red, left=2mm, top=2mm, bottom=2mm
}

\newtcolorbox{eigenschap}{enhanced,sharp corners=all,colback=white,colframe=green,toprule=0pt,bottomrule=0pt,leftrule=1pt,rightrule=0pt, overlay unbroken={
\draw[green,line width=1pt] (frame.north west) -- ++(.2,0);
\draw[green,line width=1pt] (frame.south west) -- ++(.2,0);
}, overlay first={
\draw[green,line width=1pt] (frame.north west) -- ++(.2,0);
}, overlay last={
\draw[green,line width=1pt] (frame.south west) -- ++(.2,0);
}, left=2mm, top=2mm, bottom=2mm}
% --- Define underlines ---
\renewcommand{\ULdepth}{1.8pt}
\contourlength{0.9pt}
\renewcommand{\ULthickness}{.7pt}
\newcommand{\udef}[1]{%
\textcolor{red}{\uline{\phantom{#1}}}%
  \llap{\contour{white}{#1}}%
\index[definition]{#1}%
}
\newcommand{\ueig}[1]{%
\textcolor{green}{\uline{\phantom{#1}}}%
  \llap{\contour{white}{#1}}%
}
\newcommand{\undline}[1]{%
\uline{\phantom{#1}}%
  \llap{\contour{white}{#1}}%
}
% --- Important remark ---
\newcommand{\remark}[1]{\begin{center}\textbf{#1}\end{center}}
% --- Extra symbols ---
\newcommand*{\twoheadrightarrowtail}{\mathrel{\rightarrowtail\kern-1.9ex\twoheadrightarrow}}

\DeclareFontFamily{U}{mathb}{\hyphenchar\font45}
\DeclareFontShape{U}{mathb}{m}{n}{
      <5> <6> <7> <8> <9> <10> gen * mathb
      <10.95> mathb10 <12> <14.4> <17.28> <20.74> <24.88> mathb12
      }{}
\DeclareSymbolFont{mathb}{U}{mathb}{m}{n}
\DeclareFontSubstitution{U}{mathb}{m}{n}
\DeclareMathSymbol{\sqsubsetneq}    {3}{mathb}{"88}


\DeclareFontFamily{OT1}{mbb}{\hyphenchar\font45}
%
\DeclareFontShape{OT1}{mbb}{m}{n}{
      <5> <6> <7> <8> <9> <10> gen * mbb
      <10.95> mbb10 <12> <14.4> mbb12 <17.28> <20.74> <24.88> mbb17
      }{}
\DeclareSymbolFont{mbb}{OT1}{mbb}{m}{n}
%
\DeclareFontShape{OT1}{mbb}{bx}{n}{
      <5> <6> <7> <8> <9> <10> gen * mbb
      <10.95> mbb10 <12> <14.4> mbb12 <17.28> <20.74> <24.88> mbb17
      }{}
\DeclareSymbolFont{mbb}{OT1}{mbb}{bx}{n}


\DeclareMathSymbol{\bbomega}{0}{mbb}{"B8}
% --- Defines pictures and graphics ---
\makeatletter
\def\circletransformation{%
\pgfmathsetmacro{\myX}{\pgf@x*sin(\pgf@y)*.6}
\pgfmathsetmacro{\myY}{\pgf@x*cos(\pgf@y)*.6}
\setlength{\pgf@x}{\myX pt}
\setlength{\pgf@y}{\myY pt}
}
\makeatother

\tikzset{
  apple/.pic={
    \draw (0,1) .. controls (-.6,1.8) and (-1.3,.8) .. (-1.1,0) .. controls (-.9,-.8) and (-.3,-1.5) .. (0,-1) .. controls (.3,-1.5) and (.9,-.8) .. (1.1,0) .. controls (1.3,.8) and (.6,1.8) .. (0,1) -- (0,1.6);
  }
}
% --- More settings ---
\newlength\tindent
\setlength{\tindent}{\parindent}
\setlength{\parindent}{0pt}
\renewcommand{\indent}{\hspace*{\tindent}}

% Start at chapter zero:
\setcounter{chapter}{-1}

% Include \paragraph in ToC:
\setcounter{tocdepth}{5}
% Number \subsubsection:
\setcounter{secnumdepth}{3}

\graphicspath{ {./images/} }

% =========================== Commath clone ========================
% Differential (upface d)
\DeclareMathOperator{\diff}{d \!}
% Derivative (upface D)
\DeclareMathOperator{\Diff}{D \!}

% Command for partial derivatives. The first argument denotes the function and the second argument denotes the variable with respect to which the derivative is taken. The optional argument denotes the order of differentiation. The style (text style/display style) is determined automatically
\providecommand{\pd}[3][]{\ensuremath{
\frac{\partial{^{#1}}#2}{\partial{{#3}^{#1}}}
}}

% \tpd[2]{f}{k} denotes the second partial derivative of f with respect to k
% The first letter t means "text style"
\providecommand{\tpd}[3][]{\ensuremath{\mathinner{
\tfrac{\partial{^{#1}}#2}{\partial{{#3}^{#1}}}
}}}
% \dpd[2]{f}{k} denotes the second partial derivative of f with respect to k
% The first letter d means "display style"
\providecommand{\dpd}[3][]{\ensuremath{\mathinner{
\dfrac{\partial{^{#1}}#2}{\partial{{#3}^{#1}}}
}}}

% mixed derivative - analogous to the partial derivative command
% \md{f}{5}{x}{2}{y}{3}
\providecommand{\md}[6]{\ensuremath{
\ifinner
\tfrac{\partial{^{#2}}#1}{\partial{{#3}^{#4}}\partial{{#5}^{#6}}}
\else
\dfrac{\partial{^{#2}}#1}{\partial{{#3}^{#4}}\partial{{#5}^{#6}}}
\fi
}}

% \tpd[2]{f}{k} denotes the second partial derivative of f with respect to k
% The first letter t means "text style"
\providecommand{\tmd}[6]{\ensuremath{\mathinner{
\tfrac{\partial{^{#2}}#1}{\partial{{#3}^{#4}}\partial{{#5}^{#6}}}
}}}
% \dpd[2]{f}{k} denotes the second partial derivative of f with respect to k
% The first letter d means "display style"
\providecommand{\dmd}[6]{\ensuremath{\mathinner{
\dfrac{\partial{^{#2}}#1}{\partial{{#3}^{#4}}\partial{{#5}^{#6}}}
}}}


% ordinary derivative - analogous to the partial derivative command
\providecommand{\od}[3][]{\ensuremath{
\ifinner
\tfrac{\diff{^{#1}}#2}{\diff{{#3}^{#1}}}
\else
\dfrac{\diff{^{#1}}#2}{\diff{{#3}^{#1}}}
\fi
}}

\providecommand{\tod}[3][]{\ensuremath{\mathinner{
\tfrac{\diff{^{#1}}#2}{\diff{{#3}^{#1}}}
}}}
\providecommand{\dod}[3][]{\ensuremath{\mathinner{
\dfrac{\diff{^{#1}}#2}{\diff{{#3}^{#1}}}
}}}

% functional derivative - analogous to the partial derivative command
\providecommand{\fd}[3][]{\ensuremath{
\ifinner
\tfrac{\delta{^{#1}}#2}{\delta{{#3}^{#1}}}
\else
\dfrac{\delta{^{#1}}#2}{\delta{{#3}^{#1}}}
\fi
}}

\providecommand{\tfd}[3][]{\ensuremath{\mathinner{
\tfrac{\delta{^{#1}}#2}{\delta{{#3}^{#1}}}
}}}
\providecommand{\dfd}[3][]{\ensuremath{\mathinner{
\dfrac{\delta{^{#1}}#2}{\delta{{#3}^{#1}}}
}}}

% --- Quantum stuff ---
\newcommand\hilbert{\mathcal{H}}

% --- Code style ---
\lstdefinestyle{program}{numbers=left}
\lstdefinestyle{snippet}{}
\usepackage{courier}
\definecolor{verylightgray}{gray}{0.95}
\lstset{basicstyle=\selectfont\ttfamily, backgroundcolor=\color{verylightgray}}
% --- Syntax environment ---
\newenvironment{syntax}{\begin{center}\ttfamily \small}{\end{center}}
\newcommand{\opt}[1]{\textbf{(}#1\textbf{)?}}
\newcommand{\optnb}[1]{#1\textbf{?}}
\newcommand{\mult}[1]{\textbf{(}#1\textbf{)*}}
\newcommand{\multnb}[1]{#1\textbf{*}}
\newcounter{index}
\newcommand\opts[1]{%
  \getargsC{#1}%
  \textbf{(} \argi%
  \setcounter{index}{1}
  \whiledo{\value{index} < \narg}{%
    \stepcounter{index}%
     \textbf{|} \csname arg\romannumeral\value{index}\endcsname \hspace{.8em}%
  }\textbf{)}%
}



% --- End setup ---


\makeatletter
\@fileswtrue
\makeatother

\title{Some of the Big Ideas in Mathematics}
\author{Joseph Cunningham}
\date{}

\addbibresource{mathematics.bib}

\begin{document}
\maketitle
\tableofcontents

TODO: Bertrand paradox (probability); Routh-Hurwitz; Marden's theorem, Sylvester-Gallai; \url{https://mathoverflow.net/questions/28997/does-anyone-know-an-intuitive-proof-of-the-birkhoff-ergodic-theorem}

\url{https://grossack.site/}

\chapter{Preface}
The text is a collection of interesting results. If a proof is not supplied, it it probably fairly trivial, but it may also be an oversight. This text starts with set theory and attempts to develop as much mathematics as possible from there. Only a basic knowledge of logic is assumed.

In particular the following logical symbols will be used and not introduced:
\begin{center}
\begin{tabular}{ c c }
Symbol & Meaning \\
 \hline
$\land$ & logical and \\
$\lor$ & logical or \\
$\neg$ & logical not \\
$\forall$ & universal quantifier \\
$\exists$ & existential quantifier \\
$\exists!$ & uniqueness quantifier \\
$(\,), [\,]$ & parentheses \\
$=$ & equality \\
$\neq$ & non-equality
\end{tabular}
\end{center}

One important note for proving things: if we say $\exists x: P(x)$, we can take this $x$ and continue the proof using it.

\section{I am a formalist}
I think it is very hard to argue that many of the mathematical objects discussed here \textit{actually} exist. You will not find a real number with infinite precision floating around in the universe. Most real numbers, i.e.\ the ones that do not have a simple representation or construction procedure, cannot even be thought of.

Still, they seem to be very useful to model many things in this world, so we want a way to study them. Since I do not really believe the physical existence, what am I studying? Some abstract thought? A meme (in the original meaning)? If so, how do I know that I am talking about the `same' numbers as everyone else? This is essentially the second horm of Benacerraf's dilemma.

Also, what does it mean to be wrong about a mathematical object? Say I had convinced myself that all numbers larger than one hundred are even and then somebody tells me I am wrong. Of course I ask for proof and they say ``Look, 101 is larger than 100 and not even''. Why should I accept this? I have two lines of reasoning, leading to opposite conclusions. Which is more trustworthy? And why?
In this case, clearly the counterexample is generally considered more trustworthy, but in principle I could have rejected that 101 was even a proper number. Since they live in my head, I can do what I want. This feels very wrong, because we have so many physical analogues of natural numbers all around us. It is not too difficult to assemble 101 marbles and use that as a justification for the existence of 101.

For natural numbers that are small enough we have good justification and we have a robust way to communicate observations about them. We can be fairly sure we are actually thinking of the same things. For large numbers this becomes a problem. Consider $9^{10^{11^{12}}}$ and suppose I want to know what the middle digit of the decimal expansion is. There is no way I can write down the full decimal expansion. If I have an argument to say it is 1 and my friend has one to say that it is 2, on what grounds should we believe one over the other? The situation becomes even more complicated when considering real numbers like $\pi+e+\sqrt{2}$.

Of course, in practice, mathematicians have a way to sort good arguments from bad ones. They use formal logic. If me and my friend go to a mathematician and ask who is correct, the mathematician will presumably ask us the explain our proofs step by step and will at the end declare there to be an error in one (or both) of them.

The mathematician is judging the steps by comparing them to a finite set of finite rules. It is exactly this abstract machinery that is used to share the idea of, say, real numbers. The abstract machine is finite and we can easily build physical models of it (by manipulating symbols on a piece of paper or by programming a computer). Given these physical models, we can validate the ideas we send and receive. The idea of the real numbers can then be shared by sharing the idea of the machine together with a (finite) recipe for constructing the real numbers in it.

We can trust the abstract machine in a way that we cannot trust very large of very precise numbers. In this way I am a finitist. But I am a finitist when it comes to the supporting machine, not when it comes to the numbers themselves. I see this as having all the advantages of finitism with fewer of the downsides. For example, a (strict) finitist cannot say much about all natural numbers. They cannot say that all natural numbers are even or odd. Rather they must say that all natural numbers less that a million are even or odd. Or a billion. Whatever they can verify. This can be quite cumbersome. With the support of the machine, I can easily say all natural numbers are even or odd.

Another big advantage is that the machine can be adapted to suit many different situations. Given a nice calculation, we do not need to ponder the metaphysical reality of what has been written. Rather, it is usually relatively easy to make the necessary definitions accessible to the machine such that we can prove this theorem, say about lattice-valued groups or some other relatively esoteric mathematical `object'.

I value arguments and calculations. I will invent any context necessary that naturally supports an interesting argument. I have the feeling many mathematicians feel the same way.

TODO: this also means mathematics can yield good models for science: just ask the machine and it will give you an answer. The computation may use esoteric mathematical `objects', but the output should be testable.

\part{Set Theory}
\setcounter{chapter}{0} % Reset chapter counter
\chapter{Set theory}
\url{math.colorado.edu/~monkd}

\url{philsci-archive.pitt.edu/1372/1/SetClassCat.PDF}

\url{https://mathoverflow.net/questions/22635/can-we-prove-set-theory-is-consistent}

Rethinking set theory - Leinster




\section{Some initial ideas}
\begin{enumerate}
\item Sets are defined by what is in them (extensionality);
\item Sets can be created by specifying a condition: for any definite condition $P$ there is a set
\[ A = \{x\;|\;P(x)\} \]
defined by $x\in A \iff P(x)$. This is de \udef{general comprehension principle}.
\end{enumerate}
\subsection{Russell's paradox}
Russell showed that the general comprehension principle cannot be valid. Consider
\[ R = \{x\;|\; x\;\text{is a set and}\;x\notin x\}. \]
Then $R\in R$ if and only if $R\notin R$.

There have been several proposed solutions:
\begin{enumerate}
\item We can restrict the general comprehension principle to only separation, which means that we can only apply comprehension to elements that are already in a set. In other words, we do not write
\[ \setbuilder{x}{P(x)} \qquad \text{but instead}\qquad \setbuilder{x\in B}{P(x)}\qquad \text{for some set $B$}. \]
For this to work clearly there must not be a set containing all objects in the universe. We could say the universe is too big to fit into a set.
\item We can restrict the type of objects that are put into the same set: We can put two objects that are not sets in the same set, but not a set and an object that is not a set. Such objects have type $0$. Sets of these objects have type $1$. Sets of these sets have type $2$ etc. We then say we can only form sets containing only objects of the same type. Such a set is of a type one higher. This is the essential idea behind type theory.
\end{enumerate}
In these notes we will use the first solution.

\subsection{The axiomatic setup}
For the setup of set theory we have:
\begin{enumerate}
\item A \udef{domain} or \udef{universe} $\mathcal{W}$ of objects. Some of these objects are sets. Some of these object are not sets. These are called \udef{atoms} or \udef{urelements}. A universe is called \udef{pure} if it contains only sets.
\item We have a (logical) language (usually first order logic) which allows us to express definite conditions using which we can define sets using comprehension. We assume this logical language has
\begin{enumerate}
\item a notion of identity, i.e.\ $=$;
\item a definite predicate giving sethood:
\[ \Set(x) \iff \text{$x$ is a set}; \]
\item a definite binary predicate giving membership, denoted $\in$:
\[ x\in y \iff \text{$\Set(y)$ and $x$ is a member of $y$}. \]
\end{enumerate}
\end{enumerate}

TODO purity?

\begin{note}
We abbreviate
\begin{itemize}
\item $\forall x: x\in X \implies (\ldots)$ by $\forall x\in X: (\ldots)$;
\item $\exists x: x\in X \land (\ldots)$ by $\exists x\in X: (\ldots)$.
\end{itemize}
This can be generalised to the abbreviation, for some definite condition $P$, of
\begin{itemize}
\item $\forall x: P(x) \implies (\ldots)$ by $\forall P(x): (\ldots)$;
\item $\exists x: P(x) \land (\ldots)$ by $\exists P(x): (\ldots)$.
\end{itemize}
For formal proofs it is often useful to write out the abbreviations.
\end{note}

\subsubsection{Definitions of some set-theoretic operations}
TODO: setbuilder (with expressions in creation part) and $\{\}$.
\begin{definition}
Let $A,B$ be sets. We define
\begin{itemize}
\item the \udef{emptyset}
\[ \emptyset \defeq \setbuilder{x}{x\neq x}; \]
\item the \udef{set difference} of $A$ and $B$ or \udef{relative complement} of $B$ in $A$
\[ A\setminus B \defeq \setbuilder{x}{x\in A \land x\notin B}; \]
\item the \udef{powerset} of $A$
\[ \powerset(A) \defeq \setbuilder{X}{X\subseteq A}; \]
\item the \udef{union} of $A$
\[ \bigcup A \defeq \setbuilder{x}{\exists X\in A: x\in X}; \]
\item the \udef{intersection} of $A$
\[ \bigcap A \defeq \setbuilder{x}{\forall X\in A: x\in X}; \]
\end{itemize}
We introduce some abbreviations for the union and intersection:
\begin{itemize}
\item $A\cup B \defeq \bigcup\{A,B\}$;
\item $\bigcup_\Phi \sigma \defeq \bigcup\setbuilder{\sigma(x)}{\Phi(x)}$.
\end{itemize}
And similarly for the intersection.
\end{definition}

\begin{lemma} \label{elementSubsetUnion}
Let $A,B$ be collections. Then
\[ A \in B \implies A\subseteq \bigcup B \]
\end{lemma}

\section{The Zermelo axioms}
The \udef{Zermelo axioms} are the following, except for the axiom of choice which will be discussed later:
\begin{enumerate}[(I)]
\item \textbf{Axiom of extensionality}: for any two sets $A,B$:
\[ A=B \;\iff\; \left[\forall x:x\in A\iff x\in B\right]. \]
\item \textbf{Axiom of elementary sets} or the \textbf{emptyset and pairset axioms}:
\begin{enumerate}[(1)] \setcounter{enumii}{-1}
\item There exists a set $\emptyset$ that has no members.
\item For any object $x$ in the domain, there exists a set $\{x\}$ containing only $x$.
\item For any two objects $x,y$ in the domain, there exists a set $A = \{x,y\}$ containing only $x$ and $y$:
\[ t\in A \iff [t=x\;\lor t=y]. \]
\end{enumerate}
\begin{note}
A set with only one element is a \udef{singleton}. A set with two elements is a \udef{doubleton}. The existence of singletons follows from the existence of doubletons by setting $x=y$, so part (1) is superfluous.
\end{note}
\item \textbf{Axiom of separation}: for each set $A$ and each unitary predicate $P$, there exists a set $B$ such that
\[ x\in B \iff [x\in A \land P(x)]. \]
We write
\[ B = \{x\in A\;|\; P(x)\}. \]
\begin{note}
When working in a first-order system we do not have variables for predicates and the axiom of separation becomes more properly an \emph{axiom schema}: for every property definable by a first-order formula we add a new axiom with this formula substituted for $P$.
\end{note}
\begin{note}
\begin{proposition}
Let $A$ be a set. The set
\[ r(A) \defeq \{ x\in A\;|\; x\notin x \} \]
is not a member of $A$.
\end{proposition}
\begin{corollary}
There is no set of sets.
\end{corollary}
\end{note}
\item \textbf{Power set axiom}: for each set $A$ there exists a set $\powerset(A)$ whose members are the subsets of $A$. This set is called the \udef{power set} of $A$.
\begin{note}
For this we need the definition of a \udef{subset}. We define
\[ X\subseteq A \quad \Leftrightarrow_\text{def} \quad \forall t:t\in X\implies t\in A  \]
and say that $X$ is a subset of $A$. Then $A$ is a \udef{superset} of $X$. We also write $X\subset A$. If we want to emphasise that $X\neq A$, we write $X\subsetneq A$. In this case $X$ is a \udef{proper subset}.
\begin{lemma}
Let $A$ and $B$ be sets. Then $A = B$ \textup{if and only if}
\[ (A\subseteq B) \land (B\subseteq A). \]
\end{lemma}
\end{note}
Then we can define the power set of $A$ as
\[ \powerset(A) \defeq \{X \;|\; \Set(X) \land (X\subseteq A)\}. \]
\item \textbf{Union set axiom}: for every object $\mathcal{E}$, there exists a set $\bigcup \mathcal{E}$ whose members are the members of $\mathcal{E}$:
\[ t\in \bigcup\mathcal{E} \iff \exists X\in \mathcal{E}:t\in X \]
\begin{note}
\begin{lemma}
Let $a$ be an atom. Then
\[ \bigcup a = \emptyset. \]
Also
\[ \bigcup \emptyset = \emptyset. \]
\end{lemma}
\end{note}
\begin{note}
Given two objects $A,B$, we define
\[ A\cup B \defeq \bigcup \{ A,B \}. \]
\end{note}
\item \textbf{Axiom of infinity}: there exists a set $I$ which contains
\begin{itemize}
\item the empty set $\emptyset$ and
\item the singleton of each of its members:
\[ \forall x: \quad x\in I \implies \{x\}\in I. \]
\end{itemize}
\begin{note}
A possible version of the set $I$ is
\[ I_0 = \{ \emptyset, \{\emptyset\}, \{\{\emptyset\}\}, \{\{\{\emptyset\}\}\}, \ldots \} \]
and indeed we need $I_0\subset I$, but we have not excluded the possibility that $I$ contains other elements.
\end{note}
\end{enumerate}

\section{Von Neumann-Bernays-Gödel and Morse-Kelley set theory}
NBG / MK are axiomatic systems in first-order predicate logic with equality (TODO: remove equality, made definition (?), in which case make equivalence relation a lemma. Check definition of empty class and universe class.), whose only primitive notion is the membership relation $\in$. (TODO sometimes class is called primitive notion as well (?))

TODO: replacement, infinity and foundation.

\subsection{Classes}
\begin{definition}
We consider a universe $\mathcal{W}$. Let $x$ be an object in $\mathcal{W}$. Then we call $x$
\begin{itemize}
\item a \udef{class} if $\exists y: y\in x$;
\item an \udef{element} if $\exists y: x\in y$;
\item a \udef{set} if it is both a class and an element;
\item a \udef{proper class} if it is a class and not an element (or, equivalently, a class and not a set).
\end{itemize}
We define the predicates
\begin{itemize}
\item $\Class(x) \defequiv \exists y: y\in x$;
\item $\Element(x) \defequiv \exists y: x\in y$;
\item $\Set(x) \defequiv \exists y,z: y\in x \land x\in z$;
\item $\ProperClass(x) \defequiv (\exists y: y \in x)\land \neg (\exists z: x\in z)$.
\end{itemize}
\end{definition}

TODO: emptyset is not class or set?

\begin{enumerate}[(A)]
\item \textbf{Axiom of extensionality}: For any two classes $A,B$:
\[ A=B \;\iff\; \left[\forall x:x\in A\iff x\in B\right]. \]
\item \textbf{Axiom of class existence}: Let $\phi(x)$ be a proposition with free variable $x$. Then there exists a class $\setbuilder{x}{\phi(x)}$ such that
\[ \forall y: \;\; \Element(y) \;\implies\; \Big( y\in \setbuilder{x}{\phi(x)} \iff \phi(y) \Big). \]
\end{enumerate}
This is for MK. In NBG the quantifiers in $\phi$ must be bound to elements.

Let $A$ be a class and $\phi(x)$ a proposition with free variable $x$. We often abbreviate $\setbuilder{x}{x\in A \land \phi(x)}$ by $\setbuilder{x\in A}{\phi(x)}$.


\begin{proposition}[Russel's paradox] \label{russelParadox}
Let $A$ be a set. The set
\[ r(A) \defeq \setbuilder{x\in A}{ x\notin x } \]
is not a member of $A$.
\end{proposition}
\begin{corollary} \label{setOfSets}
The class of sets, $\setbuilder{x}{\Set(x)}$, is not a set.
\end{corollary}
\begin{corollary} \label{properClassExistence}
There exists a proper class.
\end{corollary}


\begin{proposition}
There exists a proper
\end{proposition}

\subsubsection{Subclasses}
\begin{definition}
Let $A,B$ be classes. We say $B$ is a \udef{subclass} of $A$, denoted $B\subseteq A$, if
\[ \forall x: \; x\in A \implies x\in B. \]
If $B$ is a subclass of $A$ and $A \neq B$, then we call $B$ a \udef{strict subclass} of $A$, denoted $B \subset A$ or $B \subsetneq A$.
\end{definition}

\begin{lemma}
Let $A,B,C$ be classes. Then
\begin{enumerate}
\item $A = B$ \textup{if and only if} $A \subseteq B$ and $B \subseteq A$;
\item if $A \subseteq B$ and $B\subseteq C$, then $A \subseteq C$.
\end{enumerate}
\end{lemma}

\begin{enumerate}[(A)] \setcounter{enumi}{2}
\item \textbf{Axiom of separation}: Any subclass of a set is a set.
\end{enumerate}
TODO: consequence of replacement.

In particular, for any set $A$ and proposition $\phi(x)$, the class $\setbuilder{x\in A}{\phi(x)}$ is a set.

\subsubsection{Class construction}
We can use the axiom of class construction to build some basic classes.

\begin{definition}
Let $A,B$ be classes. We define
\begin{itemize}
\item the \udef{union} of $A$ and $B$ as $A \cup B \defeq \setbuilder{x}{x\in A \lor x\in B}$;
\item the \udef{intersection} of $A$ and $B$ as $A \cap B \defeq \setbuilder{x}{x\in A \land x\in B}$;
\item the \udef{complement} of $A$ as $A^c \defeq \setbuilder{x}{x\notin A}$;
\item the \udef{power set} of $A$ as $\powerset(A) \defeq \setbuilder{B}{B \subseteq A}$.
\end{itemize}
Let $\mathcal{E}$ be a class. We define
\begin{itemize}
\item the \udef{union} of $\mathcal{E}$ as
\[ \bigcup \mathcal{E} \defeq \setbuilder{x}{\exists A\in \mathcal{E}: \; x\in A}; \]
\item the \udef{intersection} of $\mathcal{E}$ as
\[ \bigcap \mathcal{E} \defeq \setbuilder{x}{\forall A\in \mathcal{E}: \; x\in A}. \]
\end{itemize}
We also define
\begin{itemize}
\item the \udef{empty class} $\emptyset \defeq \setbuilder{x}{x \neq x}$;
\item the \udef{universe class} $\mathcal{U} \defeq \setbuilder{x}{x = x}$.
\end{itemize}
\end{definition}

\begin{lemma}
Let $x$ be an element and $A$ a class. Then
\begin{enumerate}
\item $x\notin \emptyset$ and $x\in \mathcal{U}$;
\item $\emptyset \subseteq A$ and $A \subseteq \mathcal{U}$.
\end{enumerate}
\end{lemma}

\begin{lemma}
The empty class is a set \textup{if and only if} there exists a set.
\end{lemma}
\begin{proof}
The direction $\Rightarrow$ is clear.

For the direction $\Leftarrow$, assume there exists a set $A$. Then $\emptyset \subseteq A$. By the axiom of separation, this means $\emptyset$ is a set.
\end{proof}

\begin{enumerate}[(A)] \setcounter{enumi}{3}
\item \textbf{Set existence}: There exists a set.
\end{enumerate}
TODO: is implied by the axiom of infinity (TODO later!)

\begin{lemma}
Let $A,B, \mathcal{E}$ be classes. Then
\begin{enumerate}
\item $\bigcap\mathcal{E}$ is either a set or $\mathcal{U}$;
\item if either $A$ or $B$ is a set, the intersection $A \cap B$ is a set.
\end{enumerate}
\end{lemma}
\begin{proof}
(1) Assume $\bigcap\mathcal{E} \neq \mathcal{U}$. Then there exists an element $x\notin \bigcap\mathcal{E}$, meaning there exists a set $A\in \mathcal{E}$ such that $x\notin A$. Now $\bigcap\mathcal{E} \subseteq A$, so it is a set by separation. 

(2) This follows from the axiom of separation and the fact that $A \cap B$ is a subclass of both $A$ and $B$.
\end{proof}

We also assert some other classes constructed from sets are sets.

\begin{enumerate}[(A)] \setcounter{enumi}{4}
\item \textbf{Power set axiom}: Let $A$ be a set. Then $\powerset(A)$ is a set.
\item \textbf{Union set axiom}: Let $\mathcal{E}$ be a set. Then $\bigcup\mathcal{E}$ is a set.
\end{enumerate}

\begin{lemma} \mbox{} \label{distinctElements}
\begin{enumerate}
\item $\emptyset \neq \powerset(\emptyset)$;
\item for any object $a$, we can find an object $b$ such that $a \neq b$.
\end{enumerate}
\end{lemma}
\begin{proof}
(1) We have $\emptyset \in \powerset(\emptyset)$, but $\emptyset \notin \emptyset$.

(2) If $a = \emptyset$, we may take $\powerset(\emptyset)$. Otherwise take $\emptyset$.
\end{proof}

\begin{definition}
Let $a,b, c \ldots$ be objects. Then
\[ \{a,b, c \ldots\} \defeq \setbuilder{x}{x = a \lor x = b \lor x = c \lor \ldots}. \]
\end{definition}

\begin{enumerate}[(A)] \setcounter{enumi}{6}
\item \textbf{Doubleton existence axiom}: Let $a,b$ be objects. Then $\{a,b\}$ is a set.
\end{enumerate}

\begin{lemma} \label{binaryUnionIntersection}
Let $A,B$ be sets. Then
\begin{enumerate}
\item $A \cup B = \bigcup\{A, B\}$;
\item $A \cap B = \bigcap\{A, B\}$;
\end{enumerate}
In particular $A\cup B$ is a set.
\end{lemma}

\begin{lemma}
Let $a,b, c \ldots$ be an object. Then
\begin{enumerate}
\item $\{a\} = \{a, a\}$;
\item $\{a\}$ is a set;
\item $a\in \{a\}$ \textup{if and only if} $\Element(a)$;
\item $\{b,c \ldots\} \subseteq \{a,b,c \ldots\}$;
\item $\{a,b, c \ldots\}$ is a set.
\end{enumerate}
\end{lemma}
\begin{proof}
(1) $\setbuilder{x}{x = a \lor x = a} = \setbuilder{x}{x = a}$.

(2) By the doubleton existence axiom.

(3) By definition.

(4) Clear.

(5) By definition we can find sets $A,B,C \ldots$ such that $a\in A, b\in B, c\in C \ldots$ Then $\{a,b, c \ldots\} \subseteq A\cup B\cup C \cup \ldots$ and thus a set by \ref{binaryUnionIntersection} and the axiom of separation.
\end{proof}

\section{Finite sets}
Nex axiomatised element: finite sets. Classes may be members of finite sets.

\section{Potter-Scott set theory}
\begin{definition}
Let $\mathcal{W}$ be a universe with urelements $\mathcal{U}$.
\begin{itemize}
\item The \udef{accumulation} of an entity $a$ is
\[ \acc(a) \defeq \setbuilder{x}{x\in \mathcal{U} \lor \left[\exists b\in a: x\in b \lor x\subseteq b\right]}. \]
\item A collection $V$ is called a \udef{history} if
\[ \forall v\in V: v = \acc(V\cap v). \]
\item The accumulation of a history is called a \udef{level}.
\end{itemize}
\end{definition}

We can write the accumulation as
\[ \acc(a) = \mathcal{U}\cup \left(\bigcup a\right) \cup \left(\bigcup\setbuilder{\powerset(b)}{b\in a}\right). \]

Lemma \ref{elementSubsetUnion} translates to:
\begin{lemma} \label{elementsSubsetAccumulation}
Let $a,b$ be collections. If $b\in a$, then $b\subseteq \acc(a)$.
\end{lemma}

\begin{example}
Let $\mathcal{W}$ be a universe with urelements $\mathcal{U}$.
\begin{enumerate}
\item $\emptyset$ is (trivially) a history (assuming it exists). Then
\[ \acc(\emptyset) = \mathcal{U} \]
is a level.
\item $V = \{\mathcal{U}\}$ is a history:
\begin{itemize}
\item $\acc(V\cap \mathcal{U}) = \acc(\emptyset) = \mathcal{U}$.
\end{itemize}
Then $\acc(\{\mathcal{U}\}) = \mathcal{U}\cup \powerset(\mathcal{U}) \eqdef L_1$ is a level.
\item $V = \{\mathcal{U}, L_1\}$ is a history:
\begin{itemize}
\item $\acc(V \cap \mathcal{U}) = \acc(\emptyset) = \mathcal{U}$;
\item $\acc(V \cap L_1) = \acc(\{\mathcal{U}\}) = L_1$.
\end{itemize}
Then $\acc(\{\mathcal{U}, L_1\}) = \mathcal{U}\cup \powerset(\mathcal{U})\cup \powerset(\powerset(\mathcal{U})) \eqdef L_2$ is a level.
\item $V = \{\mathcal{U}, L_1, L_2\}$ is a history:
\begin{itemize}
\item $\acc(V \cap \mathcal{U}) = \acc(\emptyset) = \mathcal{U}$;
\item $\acc(V \cap L_1) = \acc(\{\mathcal{U}\}) = L_1$;
\item $\acc(V \cap L_2) = \acc(\{\mathcal{U}, L_1\}) = L_2$.
\end{itemize}
Then $\acc(\{\mathcal{U}, L_1, L_2\}) = \mathcal{U}\cup \powerset(\mathcal{U})\cup \powerset(\powerset(\mathcal{U}))\cup \powerset(\powerset(\powerset(\mathcal{U}))) \eqdef L_3$ is a level.
\end{enumerate}
\end{example}

A level is supposed to contain all collections up to a certain ``depth''.

A history $V$ is supposed to be a set of levels such that for each level $L$ in $V$ all previous levels contained in $L$ are also in $V$.

Formally:
\begin{proposition}
\begin{enumerate}
\item Let $V$ be a history, then every element $l$ of $V$ is a level with history $V\cap l$.

\item Let $l,L$ be levels and $V$ a history of $L$. If $l\in L$, then $l\in V$.

\item Each level has a unique history.

\item Let $L, L'$ be levels such that $L' \subseteq L$. If $V$ is a history and $L\in V$, then $L'\in V$.
\item Let $L, L'$ be levels then $L'\subseteq L$ \textup{if and only if} $L' = L$ or $L'\in L$.
\end{enumerate}
\end{proposition}
\begin{proof}
(1) Because $V$ is a history, we have $l = \acc(V\cap l)$. So we just need to show $V\cap l$ is a history. Indeed take $A\in (V\cap l)$, then $A \subseteq \acc(V\cap l) = l$ (by \ref{elementsSubsetAccumulation}). So $A = A\cap l$ and
\[ \acc((V\cap l)\cap A) = \acc(V\cap A) = A. \]

(2) Let $L$ be a level. Assume there exist histories $V_1, V_2$ such that $\acc(V_1) = L = \acc(V_2)$.

(3)
\end{proof}




\subsection{Sets}
\begin{definition}
A collection is called a \udef{set} or \udef{grounded} if it is a subcollection of some level.
\end{definition}

\subsection{Axiom scheme of separation}
TODO $A$ level??
\begin{enumerate}[(I)]
\item \textbf{Axiom of separation}: let $P$ be a unary predicate, then for each set $A$, the collection $\setbuilder{x\in A}{P(x)}$ exists.
\end{enumerate}
These collections are automatically sets.

Only those sets describable by predicates (a countable number). (cfr. second order).

\subsection{Theory of levels}

\section{Working with sets}
\subsection{Venn diagrams}
Much reasoning about sets can be simplified by drawing sets as circles. Many set-theoretic operations can be described in this way. The resulting pictures are called \udef{Venn diagrams}.

For any given set $A$, all objects are either in $A$ or not in $A$. So we divide the paper into a region inside $A$ and a region outside $A$:
\[\begin{tikzpicture}[thick]
% Circle with label
\node[draw,
    circle,
    minimum size =2cm,
    label=135:$A$] (circle1) at (0,0){};
\end{tikzpicture} \]

Given two sets $A,B$ there are now four possibilities for all objects in the universe:
\begin{enumerate}
\item outside both $A$ and $B$;
\item inside $A$, but not inside $B$;
\item inside $B$, but not inside $A$;
\item inside both $A$ and $B$.
\end{enumerate}
We correspondingly divide the paper into four regions:
\[ \begin{tikzpicture}[thick]
% Set A
\node [draw,
    circle,
    minimum size =2cm,
    label={135:$A$}] (A) at (0,0){};
% Set B
\node [draw,
    circle,
    minimum size =2cm,
    label={45:$B$}] (B) at (1.2,0){};
\end{tikzpicture} \]

For three sets $A,B,C$ the picture becomes:
\[ \begin{tikzpicture}[thick]
% Set A
\node[draw,circle,minimum size =2cm,label={135:$A$}] (A) at (0,0) {};

% Set B
\node[draw,circle,minimum size =2cm,label={45:$B$}] (B) at (1.2,0) {};

% Set C
\node[draw,circle,minimum size =2cm,label=$C$] (C) at (0.6,1.08) {};
\end{tikzpicture} \]

If we want to show that one set is a subset of another set, e.g\ $B\subset A$, then we can represent this as follows:
\[ \begin{tikzpicture}[thick]
% Set A
\node [draw,
    rectangle,
    minimum width =4cm,
	minimum height = 2.2cm,
    label={135:$A$}] (A) at (0,0){};
% Set B
\node [draw,
    circle,
    minimum size =1.8cm,
    label={45:$B$}] (B) at (0,0){};
\end{tikzpicture} \]

Expressions talking about sets can be expressed by shading regions of Venn diagrams. For example $A\cup B$:
\[ \begin{tikzpicture}[thick,
    set/.style = {circle,
        minimum size = 2cm,
        fill=black!30}]

% Set A
\node[set,label={135:$A$}] (A) at (0,0) {};

% Set B
\node[set,label={45:$B$}] (B) at (1.2,0) {};

% Circles outline
\draw (0,0) circle(1cm);
\draw (1.2,0) circle(1cm);
\end{tikzpicture} \]

\subsection{Operations on sets}
\subsubsection{Intersection}
\begin{definition}
The \udef{intersection} of an object $\mathcal{E}$ is a set $\bigcap \mathcal{E}$ defined by
\[ \bigcap \mathcal{E} \defeq \left\{ x\in \bigcup\mathcal{E} \;|\; \forall X\in \mathcal{E}: x\in X \right\}. \]
\end{definition}
As before, for the union, we define
\[ A\cap B \defeq \bigcap \{A,B\}. \]

The intersection $A\cap B$ can be represented in a Venn diagram as follows:
\[ \begin{tikzpicture}[thick,
    set/.style = {circle,
        minimum size = 2cm}]

% Set A
\node[set,label={135:$A$}] (A) at (0,0) {};

% Set B
\node[set,label={45:$B$}] (B) at (1,0) {};

% Intersection
\begin{scope}
    \clip (0,0) circle(1cm);
    \clip (1,0) circle(1cm);
    \fill[black!30](0,0) circle(1cm);
\end{scope}

% Circles outline
\draw (0,0) circle(1cm);
\draw (1,0) circle(1cm);

% Set intersection label
\node at (0.5,0) {$A\cap B$};
\end{tikzpicture} \]

\begin{proposition}
Let $A, B$ be non-empty sets. Then
\begin{enumerate}
\item $A \subseteq B \implies \bigcap B\subseteq \bigcap A$;
\item $\bigcap(A\cup B) = (\bigcap A)\cup (\bigcap B)$.
\end{enumerate}
These statements hold for all sets $A,B$ iff we use an unrelativised intersection (TODO!).
\end{proposition}

\begin{definition}
Let $A,B$ be sets. We call $A$ and $B$ \udef{disjoint} if $A\cap B = \emptyset$. We write $A\perp B$.

A family of sets $\mathcal{E}$ is called \udef{pairwise disjoint} if $\forall A,B\in\mathcal{E}: A\perp B$.
\end{definition}
The notation $A\perp B$ is not standard in general set theory, but is somewhat standard for disjoint elements in lattice theory.

\begin{definition}
Let $A,B$ be sets. We call the union $A\cup B$ a \udef{(inner) disjoint union} if $A$ and $B$ are disjoint. We may write $A\uplus B$ for the union if it is disjoint.

If a family of sets $\mathcal{E}$ is pairwise disjoint, we may denote its union $\biguplus \mathcal{E}$.
\end{definition}

\begin{definition}
Let $\mathcal{A},\mathcal{B}$ be families of sets. We say $\mathcal{A}$ and $\mathcal{B}$ \udef{mesh}, denoted $\mathcal{A} \# \mathcal{B}$ if $A$ and $B$ are not disjoint, $A\cap B \neq \emptyset$, for all $A\in \mathcal{A}$ and $B\in\mathcal{B}$.

We write $A \# \mathcal{B}$ for $\{A\}\# \mathcal{B}$ and $A\# B$ for $\{A\}\#\{B\}$.
\end{definition}

\subsubsection{Difference}
\begin{definition}
Given two objects $A,B$, we define the \udef{difference} as
\[A\setminus B \defeq \setbuilder{x\in A}{x\notin B}. \]
\end{definition}
The difference $A\setminus B$ can be represented in a Venn diagram as follows:
\[ \begin{tikzpicture}[thick,
    set/.style = {circle,
        minimum size = 2cm}]

% Set A
\node[set,draw,fill=black!30,label={135:$A$}] (A) at (0,0) {};

% Set B
\node[set,draw,fill=white,label={45:$B$}] (B) at (1.2,0) {};

% Circles outline
\draw (0,0) circle(1cm);

% Set label
\node at (-0.36,0) {$A\setminus B$};
\end{tikzpicture} \]

\begin{proposition}[De Morgan's laws]
Let $A,B,C$ be sets. Then
\begin{align*}
C\setminus (A\cap B) &= (C\setminus A)\cup(C\setminus B) \\
C\setminus (A\cup B) &= (C\setminus A)\cap(C\setminus B)
\end{align*}
This can be extended to arbitrary families of sets:
\begin{align*}
C\setminus\left(\bigcup \mathcal{E}\right) &= \bigcap\setbuilder{C\setminus A}{A\in\mathcal{E}} \\
C\setminus\left(\bigcap \mathcal{E}\right) &= \bigcup\setbuilder{C\setminus A}{A\in\mathcal{E}}
\end{align*}
where $\mathcal{E}$ is a family of sets.
\end{proposition}
\begin{lemma}
Let $\mathcal{E}$ be a family of sets and $A$ a set. Then
\[ \bigcup \mathcal{E} \setminus A = \bigcup\setbuilder{X\setminus A}{X\in\mathcal{E}} \qquad\text{and}\qquad \bigcap \mathcal{E} \setminus A = \bigcap\setbuilder{X\setminus A}{X\in\mathcal{E}}. \]
\end{lemma}

\begin{lemma} \label{differenceProperties}
Let $A,B,C$ be sets. Then
\begin{enumerate}
\item $(A\setminus B)\setminus C = A\setminus (B\cup C)$;
\item $A\setminus (B\setminus C) = (A\setminus B) \cup (A\cap C)$;
\end{enumerate}
and
\begin{enumerate} \setcounter{enumi}{2}
\item $(A\setminus B)\cap C = (A\cap C)\setminus B = A\cap (C\setminus B)$;
\item $(A\setminus B)\cup C = (A\cup C)\setminus (B\setminus C)$;
\end{enumerate}
and
\begin{enumerate} \setcounter{enumi}{4}
\item $A\setminus A = \emptyset$;
\item $\emptyset\setminus A = \emptyset$;
\item $A\setminus \emptyset = A$.
\end{enumerate}
\end{lemma}

\subsubsection{Symmetric difference}
\begin{definition}
We define the \udef{symmetric difference} of two sets $A,B$ as
\[ A \symdiff B \defeq (A\setminus B)\cup(B\setminus A). \]
This is equivalent to $A\symdiff B = \setbuilder{x\in A\cup B}{(x\in A)\oplus (x\in B)}$.
\end{definition}

The symmetric difference $A\symdiff B$ can be represented in a Venn diagram as follows:
\[ \begin{tikzpicture}[thick,
    set/.style = {circle,
        minimum size = 2cm}]

% Set A
\node[set,draw,fill=black!30,label={135:$A$}] (A) at (0,0) {};

% Set B
\node[set,draw,fill=black!30,label={45:$B$}] (B) at (1.2,0) {};

\begin{scope}
    \clip (0,0) circle(1cm);
    \clip (1.2,0) circle(1cm);
    \fill[white](0,0) circle(1cm);
\end{scope}

% Circles outline
\draw (0,0) circle(1cm);
\draw (1.2,0) circle(1cm);
\end{tikzpicture} \]

\begin{lemma}
Let $A,B,C$ be sets. Then
\begin{align*}
A\symdiff \emptyset &= A \\
A\symdiff B = \emptyset &\iff A = B
\end{align*}
and
\begin{align*}
A \symdiff B &= B \symdiff A \\
(A \symdiff B) \symdiff C &= A \symdiff (B \symdiff C) \\
A \symdiff C &= (A \symdiff B)\symdiff(B \symdiff C).
\end{align*}
\end{lemma}

\subsection{Identities and equivalences}
\subsubsection{Identities involving families of sets}
\begin{lemma}
Let $\mathcal{F}, \mathcal{G}$ be families of sets such that $\mathcal{F}\subseteq \mathcal{G}$. Then
\begin{enumerate}
\item $\bigcup \mathcal{F} \subseteq \bigcup \mathcal{G}$;
\item $\bigcap \mathcal{F} \supseteq \bigcup \mathcal{G}$.
\end{enumerate}
\end{lemma}

\subsubsection{Set operations and statements characterised}
\begin{lemma}
Let $A,B$ be sets. Then
\begin{enumerate}
\item $\begin{aligned}[t]
A\cap B &= A\setminus (A\setminus B) \\
&= B\setminus (A\symdiff B) \\
&= A\symdiff (A\setminus B);
\end{aligned}$
\item $\begin{aligned}[t]
A\cup B &= (A\symdiff B)\cup A \\
&= (A\symdiff B)\symdiff(A\cap B) \\
&= (A\setminus B)\cup B;
\end{aligned}$
\item $\begin{aligned}[t]
A\symdiff B &= (A\cup B)\setminus (A\cap B) \\
&= (A\symdiff C)\symdiff(C\symdiff B)
\end{aligned}$

for any set $C$;
\item $\begin{aligned}[t]
A\setminus B &= A\setminus (A\cap B) \\
&= A\cap(A\symdiff B) \\
&= (A\cup B)\symdiff B \\
&= A\symdiff (A\cap B);
\end{aligned}$
\end{enumerate}
\end{lemma}

\begin{lemma}
Let $A$ be a set. Then the following are equivalent:
\begin{enumerate}
\item $A$ is empty;
\item $A\cup B\subseteq B$ for every set $B$;
\item $A\subseteq B$ for every set $B$;
\item $A\subseteq (B\setminus A)$ for some set $B$;
\item $A\subseteq (B\setminus A)$ for every set $B$;
\item $\emptyset \setminus A= A$.
\end{enumerate}
\end{lemma}

\begin{lemma} \label{inclusionCriteria}
Let $A,B$ be sets. Then following are equivalent:
\begin{enumerate}
\item $A\subseteq B$;
\item $A\cap B = A$;
\item $A\cup B = B$;
\item $A\symdiff B = B\setminus A$;
\item $A\symdiff B \subseteq B\setminus A$
\item $A\setminus B = \emptyset$.
\end{enumerate}
For any universe set $\Omega$ such that $A,B\subset \Omega$ the above is also equivalent to
\begin{enumerate}\setcounter{enumi}{4}
\item $\Omega\setminus B \subseteq \Omega\setminus A$;
\item $(\Omega\cap A)\setminus B = \emptyset$;
\item $(\Omega\setminus A)\cup B = \Omega$.
\end{enumerate}
\end{lemma}
\begin{corollary}
Let $A,B$ be sets. Then following are equivalent:
\begin{enumerate}
\item $A=B$;
\item $A\symdiff B = \emptyset$;
\item $A\setminus B = B\setminus A$.
\end{enumerate}
\end{corollary}
\begin{corollary} \label{setPerpInequality}
Let $A,B\subseteq \Omega$ be sets. Then following are equivalent:
\begin{enumerate}
\item $A\perp B$;
\item $A \subseteq \Omega\setminus B$;
\item $B \subseteq \Omega\setminus A$.
\end{enumerate}
\end{corollary}

\begin{lemma} \label{disjointSetDifference}
Let $A,B,C$ be sets. Then
\[ A\setminus B \perp C \qquad\iff\qquad A\cap C \subseteq B. \]
\end{lemma}
\begin{proof}
We have
\begin{align*}
A\setminus B \perp C &\iff (A\setminus B) \cap C = \emptyset \\
&\iff (A\cap C)\setminus B = \emptyset \\
&\iff A\cap C \subseteq B.
\end{align*}
\end{proof}

\subsubsection{Distributivity}
TODO: tables complete / correct?? TODO: only works for relativised intersection.
\begin{lemma}
We say $*$ left distributes over $\bullet$ if
\[ A*(B\bullet C) = (A*B)\bullet (A*C) \qquad\text{for all sets $A,B,C$.} \]
This gives the table
\[ \begin{array}{c | c c c c c}
\text{\diagbox{$*$}{$\bullet$}} & \cup & \cap & \symdiff & \setminus & \times \\ \hline
\cup & \checkmark & \checkmark &  & & \\
\cap & \checkmark & \checkmark & \checkmark & \checkmark & \\
\symdiff &  &  &  & & \\
\setminus &  &  &  & & \\
\times & \checkmark & \checkmark &  & \checkmark &
\end{array} \]
\end{lemma}

\begin{lemma}
We say $*$ right distributes over $\bullet$ if
\[ (A\bullet B)*C = (A*C)\bullet (B*C) \qquad\text{for all sets $A,B,C$.} \]
This gives the table
\[ \begin{array}{c | c c c c c}
\text{\diagbox{$*$}{$\bullet$}} & \cup & \cap & \symdiff & \setminus & \times \\ \hline
\cup & \checkmark & \checkmark &  & & \\
\cap & \checkmark & \checkmark & \checkmark & \checkmark & \\
\symdiff &  &  &  & & \\
\setminus & \checkmark & \checkmark & \checkmark & \checkmark & \\
\times & \checkmark & \checkmark &  & \checkmark &
\end{array} \]
\end{lemma}

TODO: intersection distributes over disjoint union.

\begin{lemma} \label{unionIntersectionLabelSet}
Let $\mathcal{I}$ be a family of index sets and let $A_i$ be a set for all $i\in \bigcup \mathcal{I}$. Then
\begin{enumerate}
\item $\bigcup_{i\in \bigcup \mathcal{I}} A_i = \bigcup_{I\in \mathcal{I}}\bigcup_{i\in I} A_i$;
\item $\bigcap_{i\in \bigcap \mathcal{I}} A_i = \bigcap_{I\in \mathcal{I}}\bigcap_{i\in I} A_i$;
\item $\bigcup_{i\in \bigcap \mathcal{I}} A_i \subseteq \bigcap_{I\in \mathcal{I}}\bigcup_{i\in I} A_i$;
\item $\bigcap_{i\in \bigcup \mathcal{I}} A_i \supseteq \bigcup_{I\in \mathcal{I}}\bigcap_{i\in I} A_i$.
\end{enumerate}
\end{lemma}
\begin{proof}
(1) We calculate
\begin{align*}
x\in \bigcup_{i\in \bigcup \mathcal{I}} A_i &\iff \exists i\in \bigcup \mathcal{I}: x\in A_i \\
&\iff \exists i: (i\in \bigcup \mathcal{I}) \land (x\in A_i) \\
&\iff \exists i: (\exists I \in \mathcal{I}: i\in I) \land (x\in A_i) \\
&\iff \exists i: \exists I: (I \in \mathcal{I}) \land (i\in I) \land (x\in A_i) \\
&\iff \exists I\in \mathcal{I}: \exists i \in I: x\in A_i \\
&\iff x\in \bigcup_{I\in \mathcal{I}}\bigcup_{i\in I} A_i
\end{align*}

(2) Replace $\exists$ by $\forall$ and $\land$ by $\Rightarrow$ in the proof of (1).

(3) We calculate
\begin{align*}
x\in \bigcup_{i\in \bigcap \mathcal{I}} A_i &\iff \exists i\in \bigcap \mathcal{I}: x\in A_i \\
&\iff \exists i: (i\in \bigcap \mathcal{I}) \land (x\in A_i) \\
&\iff \exists i: (\forall I \in \mathcal{I}: i\in I) \land (x\in A_i) \\
&\iff \exists i: (\forall I: (I \in \mathcal{I}) \Rightarrow (i\in I)) \land (x\in A_i) \\
&\implies \exists i: \forall I: (I \in \mathcal{I}) \Rightarrow ((i\in I) \land (x\in A_i)) \\
&\implies \forall I: \exists i: (I \in \mathcal{I}) \Rightarrow ((i\in I) \land (x\in A_i)) \\
&\iff \forall I: (I \in \mathcal{I}) \Rightarrow (\exists i:(i\in I) \land (x\in A_i)) \\
&\iff \forall I\in \mathcal{I}: \exists i \in I: x\in A_i \\
&\iff x\in \bigcap_{I\in \mathcal{I}}\bigcup_{i\in I} A_i
\end{align*}

(4) TODO
\end{proof}

For more identities, see \url{https://en.wikipedia.org/wiki/List_of_set_identities_and_relations}.


\chapter{Relations and functions}
\section{Pairs}
\begin{definition}
A definition of $(a,b)$ for all $a,b$ is called an \udef{(ordered) pair operation} if it satisfies
\begin{itemize}
\item $(a,b) = (x,y) \iff (a=x)\land (b=y)$;
\item for all sets $A,B$, $\setbuilder{(a,b)}{a\in A\land b\in B}$ is a set.
\end{itemize}
We call $\setbuilder{(a,b)}{a\in A\land b\in B}$ the \udef{Cartesian product} of classes $A$ and $B$ and denote it $A\times B$.
\end{definition}
Note that for the second condition it is enough to check that $A\times B$ is a subset of some set $S$. Then, by separation, $A\times B$ is the set
\[ A\times B = \setbuilder{x\in S}{\exists a\in A:\exists b\in B: x = (a,b)}. \]

\begin{lemma}
Let $A$ be a class. Then
\[ A\times \emptyset = \emptyset = \emptyset\times A. \]
\end{lemma}

\begin{lemma} \label{productUnionIntersection}
Let $A,B,C,D$ be classes. Then
\begin{enumerate}
\item $(A\cup B)\times C = A\times C \cup B\times C$;
\item $(A\uplus B)\times C = A\times C \uplus B\times C$;
\item $(A\cap B)\times C = A\times C\cap B\times C$;
\item $(A\times B)\cap (C\times D) = (A \cap C)\times (B\cap D)$.
\end{enumerate}
\end{lemma}

\begin{definition}
Let $p = (a,b)$ be a pair, we use $\pi_1(p)$ to denote the first element of $p$ and $\pi_2(p)$ to denote the second element:
\[ (a,b) = p = (\pi_1(p),\pi_2(p)). \]
We can convert a pair to a set:
\[ \operatorname{set}((a,b)) = \{a,b\}. \]
\end{definition}


\subsection{Axiomatising pairs}
Consider the ternary proposition $A = (B,C)$ as primitive. 
\begin{definition}
Consider a universe $\mathcal{W}$. Let $p$ be an object in $\mathcal{W}$. We call $p$ a \udef{pair} if $\exists x,y: p = (x,y)$.

We define the predicate
\begin{itemize}
\item $\Pair(p) \defequiv \exists x,y: \; p = (x, y)$.
\end{itemize}
\end{definition}

\begin{enumerate}[(A)] \setcounter{enumi}{0}
\item \textbf{Axiom of pair identity}: Let $p_1, p_2,x_1,x_2,y_1,y_2$ be objects such that
\[ p_1 = (x_1, y_1) \qquad \text{and}\qquad p_2 = (x_2, y_2). \]
Then
\[ p_1 = p_2 \quad\iff\quad \big[x_1 = x_2\big] \;\land\; \big[y_1 = y_2\big]. \]
\item \textbf{Axiom of pair existence}: Let $x,y$ be objects. Then $\exists p: \; p = (x,y)$ with $p\neq x $ and $p \neq y$.
\end{enumerate}

\begin{lemma}
Let $x,y$ be objects. Then $\exists! p: \; p = (x,y)$ with $p\neq x $ and $p \neq y$.
\end{lemma}
\begin{proof}
Existence is given by the axiom of pair existence. We just need to show uniqueness. Assume there exist $p_1, p_2$ such that $p_1 = (x,y)$ and $p_2 = (x,y)$. Then $p_1 = p_2$ by the axiom of pair identity.
\end{proof}

Usually an axiomatic system of pairs is integrated with a set theory.
\begin{definition}
Let $\mathcal{W}$ be a set theoretic universe satisfying the pair axioms. Let $A,B$ be classes. The \udef{Cartesian product} of $A$ and $B$ is defined as
\[ A\times B \defeq \setbuilder{p}{\exists a\in A: \exists b\in B: p = (a,b)}. \]
\end{definition}

In this case we impose the following additional axioms:
\begin{enumerate}[(A)] \setcounter{enumi}{2}
\item \textbf{Axiom of element pairs}: Let $a,b$ be elements and $p=(a,b)$. Then $p$ is an element.
\item \textbf{Cartesian product axiom}: Let $A,B$ be sets. Then $A \times B$ is a set.
\end{enumerate}

\begin{lemma}
Let $A, B$ be classes $a\in A, b\in B$ and $p = (a,b)$. Then $p\in A\times B$.
\end{lemma}
\begin{proof}
In this case $a,b$ are elements. So $p$ is an element by the axiom of element pairs and thus $p\in A\times B$ by class comprehension.
\end{proof}


\subsection{Defining pairs}
TODO: pair axioms conservative extension of ZFC.

\subsubsection{The Kuratowski pair}
\begin{definition}
The \udef{Kuratowski pair} is defined as
\[ (a,b) \defeq \{\{a\}, \{a,b\}\} \]
\end{definition}
Note that when $a=b$ we have
\[ (a,a) = \{\{a\},\{a,a\}\} = \{\{a\},\{a\}\} = \{\{a\}\}. \]
\begin{lemma}
Given a Kuratowski pair $p = (a,b)$, we can extract the first element $\pi_1(p)$ and the second element $\pi_2(p)$ as follows:
\begin{align*}
\pi_1(p) &= \bigcup\bigcap p; \\
\pi_2(p) &= \bigcup\left\{ x\in\bigcup p\;|\; \left[\bigcup p \neq \bigcap p\right] \implies \left[ x\notin \bigcap p \right] \right\}.
\end{align*}
The Kuratowski pair can be converted to a set by
\[ \operatorname{set}(p) = \bigcup p. \]
\end{lemma}
The construction ``$\left[\bigcup p \neq \bigcap p\right] \implies$'' in the formula for $\pi_2(p)$ is there so that it still works in case the first and second elements are the same.
\begin{proposition}
The Kuratowski pair is adequate, in that is satisfies
\[ (a,b) = (x,y) \iff (a=x)\land (b=y) \]
and the Cartesian product is a set.
\end{proposition}
\begin{proof}
If $(a=x)\land (b=y)$, then
\[ \{\{a\},\{a,b\}\} = \{\{x\},\{x,y\}\} \]
and thus $(a,b) = (x,y)$.

Now assume $(a,b) = (x,y)$. We consider two cases: $a=b$ and $a\neq b$.
\begin{itemize}[leftmargin=2cm]
\item[$\boxed{a=b}$] Then $(a,b) = \{\{a\},\{a,a\}\} = \{\{a\}\} = (x,y)$ and thus $\{x\} = \{x,y\} = \{a\}$ by extensionality. This implies $a=x=y$ and thus $(a=x)\land (b=y)$.
\item[$\boxed{a\neq b}$] Now $(a,b) = (x,y)$ implies
\[ \{\{a\},\{a,b\}\} = \{\{x\},\{x,y\}\}. \]
By extensionality, either $\{x\} = \{a\}$ or $\{x\} = \{a,b\}$. In the second option $a=x=b$ and thus $a=b$ which is a contradiction. So $x=a$.

Again by extensionality, either $\{x,y\} = \{a\}$ or $\{x,y\} = \{a,b\}$. In the first case $(x,y)$ would be a singleton and then by equality so would $(a,b)$, yielding a contradiction. Thus $\{x,y\} = \{a,b\}$. We know $a=x$ and $b\neq a$, so by extensionality $b=y$.
\end{itemize}
To prove the Cartesian product $A\times B$ is a set, notice that
\begin{align*}
a\in A, b\in B &\implies \{a\},\{a,b\}\subseteq A\cup B \implies \{a\},\{a,b\}\in \powerset(A\cup B)\\
&\implies \{\{a\},\{a,b\}\}\subseteq \powerset(A\cup B) \implies \{\{a\},\{a,b\}\}\in \powerset(\powerset(A\cup B)) \\
&\implies (a,b) \in \powerset(\powerset(A\cup B)).
\end{align*}
and $\powerset(\powerset(A\cup B))$ is a set. So
\[ A\times B = \{ x\in \powerset(\powerset(A\cup B))\;|\; \exists a\in A:\exists b\in B: x = (a,b) \} \]
is a set.
\end{proof}
\subsubsection{The short variant}
We can also define a pair as
\[ (a,b) \defeq \{a,\{a,b\}\} \]
The advantage of this short definition is fewer braces. Some disadvantages include:
\begin{enumerate}
\item If $a$ and $b$ have the same type, $a$ and $\{a,b\}$ do not have the same type.
\item In order to prove adequacy, we need a new axiom, the axiom of regularity.
\end{enumerate}
\subsubsection{Using $0,1$}
Suppose we have decided on two special, distinct objects $0,1$. Then we can define
\[ (a,b) \defeq \{\{0,a\},\{1,b\}\}. \]
\begin{proposition}
This definition satisfies the requirements for a pair.
\end{proposition}
\begin{proof}
Assuming $(a=x)\land (b=y)$, it is clear that $(a,b)=(x,y)$ as sets by extensionality.

Now assume $(a,b)=(x,y)$. In this implementation a pair always has two elements as a set. The reasoning by extensionality is simple and only slightly more difficult if $a,b,x,y$ are equal to $0$ or $1$.

The Cartesian product is a subset of $\powerset(\powerset(A\cup B\cup \{0,1\}))$ and thus a set.
\end{proof}
\subsubsection{Wiener pair}
The Wiener definition of a pair is
\[ (a,b) \defeq \{\{\emptyset,\{a\}\},\{\{b\}\}\}. \]
\begin{proposition}
This definition satisfies the requirements for a pair.
\end{proposition}

\subsection{Structured classes and sets}
\begin{definition}
A \udef{structured class} is a pair $U = (A,S)$ where $A$ is a class and $S$ is an arbitrary object.
\begin{itemize}
\item $A$ is the \udef{field} or \udef{space} of $U$, written $\operatorname{Field}(U)$;
\item $S$ is the \udef{frame} of $U$.
\end{itemize}
If we write $x\in U$, we mean $x\in \operatorname{Field}(U)$.

In particular if $A$ is a set, then we call $U$ a \udef{structured set}.
\end{definition}

Often the frame $S$ is a $n$-tuple. In this case we may write the structured class as an $n+1$-tuple by concatenating the field and the frame. e.g\ $(A,(S_1,S_2,S_3))$ becomes $(A,S_1,S_2,S_3)$.

\section{Relations}
\begin{definition}
Let $A,B$ be classes and $G$ any subclass of the Cartesian product $A\times B$. A \udef{(binary) relation} $R$ on $(A, B)$ is a tuple $(G,(A,B))$. The \udef{graph} of the binary relation $R$ is the class $\graph(R) \defeq G$.

We write
\[ xRy \defequiv (x,y)\in \graph(R) \]
and say $x$ is \udef{left related} to $y$ or $y$ is \udef{right related} to $x$.

We call
\begin{itemize}
\item $\dom(R) \defeq A$ the \udef{domain} of the relation;
\item $\codom(R) \defeq B$ the \udef{codomain} of the relation.
\end{itemize}
A relation $R$ is \udef{homogeneous} or an \udef{endorelation} if $\dom(R) = \codom(R)$.
If we say $R$ is a (homogenous) relation on $A$, we mean $\dom(R) = A = \codom(R)$. 

A relation is \udef{heterogeneous} is the domain and codomain are different.
\end{definition}

Often we will write $R \subseteq S$ as a shorthand for $\graph(R)\subseteq \graph(S)$.

\begin{lemma}
Let $R,S$ be relations on $(A,B)$. Then
\begin{enumerate}
\item $R\cup S \defeq \sSet{\graph(R)\cup \graph(S), (A,B)}$ is a relation on $(A,B)$;
\item $R\cap S \defeq \sSet{\graph(R)\cap \graph(S), (A,B)}$ is a relation on $(A,B)$.
\end{enumerate}
\end{lemma}

\begin{definition}
Let $A,B$ be classes. We have the following relations:
\begin{itemize}
\item the \udef{empty relation} $E_{A,B}$ on $(A, B)$ has graph $\emptyset$;
\item the \udef{universal relation} $U_{A,B}$ on $(A, B)$ has graph $A\times B$;
\item the \udef{identity relation} $\id_A$ on $A$ has graph $\setbuilder{(x,y)\in A\times A}{x=y}$.
\end{itemize}
We may also write $U_A$ instead of $U_{A,A}$ and $E_A$ instead of $E_{A,A}$.
\end{definition}
The identity relation on $A$ is also known as the \udef{diagonal relation} on $A$.

\begin{lemma}
Let $A,B,C,D$ be classes. Then
\begin{enumerate}
\item $\id_{A\cup B} = \id_A\cup \id_B$ and $\id_{A\cap B} = \id_A\cap \id_B$;
\item $E_{A\cup C, B\cup D} = E_{A,B}\cup E_{C,D}$ and $E_{A\cap C, B\cap D} = E_{A,B}\cap E_{C,D}$;
\item $U_{A\cap C, B\cap D} = U_{A,B}\cap U_{C,D}$.
\end{enumerate}
\end{lemma}
Note that the equalities mean the graphs are equal. The relations are not the same as they have different domains and codomains.
\begin{proof}
(1) Assume $(x,x)\in \id_{A\cup B}$. We have the equivalences
\[ (x\in A)\lor (x\in B) \iff \Big((x,x)\in \id_A\Big) \lor \Big((x,x)\in\id_B\Big) \iff (x,x)\in \id_A\cup \id_B. \]
For the second part replace $\lor$ with $\land$.

(2) Trivial because all sets are $\emptyset$.

(3) Take $(x,y)\in U_{A\cap C, B\cap D}$. We have the equivalences
\begin{align*}
(x\in A\cap C) \land (y\in B\cap D) &\iff (x\in A)\land (y\in B)\land(x\in C)\land(y\in D) \\
&\iff \big((x,y)\in U_{A, B})\big)\land\big((x,y)\in U_{A, B}\big) \\
&\iff (x,y)\in U_{A,B}\cap U_{C,D}.
\end{align*}
\end{proof}


\begin{definition}
Let $R$ be a binary relation. We say
\begin{itemize}
\item $x$ is \udef{related to} or \udef{comparable with} $y$ if $xRy$ or $yRx$; we denote this $x\nparallel_R y$ or just $x\nparallel y$;
\item $x$ is \udef{unrelated to}, \udef{incomparable with} or \udef{parallel with} $y$ if neither $xRy$ nor $yRx$; we denote this $x\parallel_R y$ or just $x\parallel y$.
\end{itemize}
\end{definition}

\subsection{Relations and subclasses}
\subsubsection{Images and preimages}
\begin{definition}
Let $R$ be a relation on $(A, B)$.
\begin{itemize}
\item The \udef{image} of a subclass $X\subset A$ under $R$ is the class
\[ X_R \defeq \setbuilder{b\in B}{\exists x\in X: xRb}. \]
\item The \udef{preimage} of a subclass $Y\subset B$ under $R$ is the class
\[ _RY \defeq \setbuilder{a\in A}{\exists y\in Y: aRy}. \]
\end{itemize}
In particular for $X=A$ and $Y=B$:
\begin{enumerate}
\item The class $A_R$ is the \udef{active codomain}, \udef{codomain of definition}, \udef{image} or \udef{range} of the relation, also denoted $\im(R)$.
\item The class $_RB$ is the \udef{active domain}, \udef{domain of definition}, \udef{preimage} or \udef{prerange} of the relation, also denoted $\preim(R)$.
\end{enumerate}
In particular for $X = \{x\}$ and $Y = \{y\}$ we define:
\begin{itemize}
\item $xR \defeq \{x\}_R$;
\item $Ry \defeq {_R\{x\}}$.
\end{itemize}
We call such images and preimages \udef{principle} images and preimages.
\end{definition}

A relation is completely characterised by its principal images.
\begin{lemma} \label{relationFromPrincipalImages}
Let $R$ be a relation on $(A, B)$, $x\in A$ and $y\in B$. Then
\[ x\in Ry \iff xRy \iff xR \ni y. \]
\end{lemma}
Principal images atoms in lattice of images??

\begin{lemma}
Let $R$ be a relation on $(A, B)$, $X\subseteq A$ and $Y\subseteq B$. Then
\begin{enumerate}
\item $X_R = {_{R^\transp}X}$;
\item $_RX = X_{R^\transp}$.
\end{enumerate}
In particular $xR = R^\transp x$ and $Ry = yR^\transp$ for all $x\in A$ and $y\in B$.
\end{lemma}

TODO: atomicity / atomistic ??
\begin{lemma}
Let $R$ be a relation on $(A, B)$, $X\subseteq A$ and $Y\subseteq B$. Then
\begin{enumerate}
\item $X_R = \displaystyle\bigcup_{x\in X}xR = \bigcup \setbuilder{xR}{x\in X}$;
\item $_RY = \displaystyle\bigcup_{y\in Y}Ry = \bigcup \setbuilder{Ry}{y\in Y}$.
\end{enumerate}
\end{lemma}


\begin{corollary} \label{monotonicityImage}
Let $R$ be a relation on $(A, B)$ and $X,Y\subset A$. Then
\begin{enumerate}
\item if $X\subseteq Y$, then $X_R \subseteq Y_R$;
\item if $X\subseteq Y$, then ${_RX} \subseteq {_RY}$.
\end{enumerate}
\end{corollary}
\begin{proof}
(1) We can write $Y = X \cup (Y\setminus X)$. Then
\[ Y_R = \bigcup \setbuilder{xR}{x\in Y} = \bigcup \setbuilder{xR}{x\in X}\cup\setbuilder{xR}{x\in (Y\setminus X)} = X_R \cup (Y\setminus X)_R \supseteq X_R. \]

(2) Similar.
\end{proof}
\begin{corollary} \label{imageRelation} \label{preimageRelation}
Let $R$ be a relation on $(A, B)$, $X,Y\subset A$ and $Z,W\subset B$. Then
\begin{enumerate}
\item $(X\cup Y)_R = X_R\cup Y_R$;
\item $(X\cap Y)_R \subseteq X_R\cap Y_R$;
\item $(X\setminus Y)_R \supseteq X_R\setminus Y_R$;
\item $(X\symdiff Y)_R \supseteq X_R\symdiff Y_R$.
\end{enumerate}
and
\begin{enumerate}
\item $_R(Z\cup W) = {_RZ}\cup {_RW}$;
\item $_R(Z\cap W) \subseteq {_RZ}\cap {_RW}$;
\item $_R(Z\setminus W) \supseteq {_RZ}\setminus {_RW}$;
\item $_R(Z\symdiff W) \supseteq {_RZ}\symdiff {_RW}$.
\end{enumerate}
\end{corollary}
\begin{proof}\mbox{}

(1) $\begin{aligned}[t]
(X\cup Y)_R &= \bigcup \setbuilder{xR}{x\in (X\cup Y)} = \bigcup \setbuilder{xR}{x\in X}\cup \setbuilder{xR}{x\in Y} \\
&= \left(\bigcup \setbuilder{xR}{x\in X}\right)\cup \left(\bigcup \setbuilder{xR}{x\in Y}\right) = X_R \cup Y_R.
\end{aligned}$

(2) We have both $X\cap Y \subseteq X$ and $X\cap Y \subseteq Y$, so $(X\cap Y)_R \subseteq X_R$ and $(X\cap Y)_R \subseteq Y_R$ by \ref{monotonicityImage}. Thus $(X\cap Y)_R \subseteq X_R\cap Y_R$.

TODO: use lattice properties

(3), (4) TODO
\end{proof}

\subsubsection{$R$-closed subsets}
\begin{definition}
Let $R$ be a homogeneous relation on $A$. A set $X\subseteq A$ is called \udef{$R$-closed} if $X_R \subseteq X$.
\end{definition}
TODO relate to ``Functions on ordered sets''.

\begin{lemma} \label{meetJoinRClosedSets}
Let $\sSet{A, R}$ be a relational structure and $\mathcal{E} \subseteq \powerset(A)$ a set of $R$-closed subsets. Then
\begin{enumerate}
\item $\bigcup \mathcal{E}$ is $R$-closed;
\item $\bigcap \mathcal{E}$ is $R$-closed.
\end{enumerate}
\end{lemma}
\begin{proof}
(1) We have
\[ \left(\bigcup \mathcal{E}\right)_R = \bigcup_{E\in\mathcal{E}} E_R \subseteq \bigcup_{E\in\mathcal{E}} E = \bigcup \mathcal{E}. \]
TODO ref.

(2) We have
\[ \left(\bigcap \mathcal{E}\right)_R \subseteq \bigcap_{E\in\mathcal{E}} E_R \subseteq \bigcap_{E\in\mathcal{E}} E = \bigcap \mathcal{E}.\]
TODO ref.
\end{proof}

\subsubsection{Left and right bounds}
\begin{definition}
Let $R$ be a relation on $(A, B)$, $X\subseteq A$ and $Y\subseteq B$.
\begin{itemize}
\item A \udef{right bound} (or \udef{upper bound}) of $X$ under $R$ is an element $b\in B$ such that $b$ is right related to all $x\in X$. We denote the class of right bounds by
\[ X^R \defeq \setbuilder{b\in B}{\forall x\in X: xRb}. \]
\item A \udef{left bound} (or \udef{lower bound}) of $Y$ under $R$ is an element $a\in A$ such that $a$ is left related to all $y\in Y$. We denote the class of left bounds by
\[ ^RY \defeq \setbuilder{a\in A}{\forall y\in Y: aRy}. \]
\end{itemize}
The classes $X^R$ and $^RY$ are also called \udef{polars}.\footnote{According to Birkhoff (TODO ref), the term ``polar'' was chosen due to the link with the polars of conic sections.}

In particular for $X=A$ and $Y=B$:
\begin{enumerate}
\item The class $A^R$ is referred to as the \udef{top} of $\sSet{R,(A,B)}$.
\item The class $^RB$ is the \udef{bottom} of $\sSet{R,(A,B)}$.
\end{enumerate}
\end{definition}
Note that the definitions of the polars is similar to the definition of the image/preimage. The only difference is that ``$\exists$'' is replaced by ``$\forall$''.

Consequently, we can state results similar to the ones above.

\begin{lemma}
Let $R$ be a relation on $(A, B)$, $X\subset A$ and $Y\subset B$. Then
\begin{enumerate}
\item $X^R = {^{R^\transp}X}$;
\item $^RX = X^{R^\transp}$.
\end{enumerate}
\end{lemma}

\begin{lemma} \label{boundsFromPrincipalImages}
Let $R$ be a relation on $(A, B)$, $X\subset A$ and $Y\subset B$. Then
\begin{enumerate}
\item $X^R = \displaystyle\bigcap_{x\in X}xR = \bigcap \setbuilder{xR}{x\in X}$;
\item $^RY = \displaystyle\bigcap_{y\in Y}Ry = \bigcap \setbuilder{Ry}{y\in Y}$.
\end{enumerate}
In particular for $x\in A$ and $y\in B$:
\begin{enumerate}
\item $\{x\}^R = xR = \{x\}_R$;
\item $^R\{y\} = Ry = {_R\{y\}}$.
\end{enumerate}
\end{lemma}
TODO this only works for empty sets if we relativise the intersection!

\begin{lemma} \label{polarsCartesianProduct}
Let $R$ be a relation on $(A,B)$ and $X\subseteq A, Y\subseteq B$. Then
\[ Y\subseteq X^R \iff X\times Y \subseteq R. \]
\end{lemma}

\begin{corollary} \label{antitonicityPolars}
Let $R$ be a relation on $(A, B)$ and $X,Y\subset A$. Then
\begin{enumerate}
\item if $X\subseteq Y$, then $X^R \supseteq Y^R$;
\item if $X\subseteq Y$, then ${^RX} \supseteq {^RY}$.
\end{enumerate}
\end{corollary}
\begin{proof}
(1) We can write $Y = X \cup (Y\setminus X)$. Then
\[ Y^R = \bigcap \setbuilder{xR}{x\in Y} = \bigcap \setbuilder{xR}{x\in X}\cap\setbuilder{xR}{x\in (Y\setminus X)} = X^R \cap (Y\setminus X)^R \subseteq X^R. \]

(2) Similar.
\end{proof}
\begin{corollary} \label{polarasRelation}
Let $R$ be a relation on $(A, B)$, $X,Y\subset A$ and $Z,W\subset B$. Then
\begin{enumerate}
\item $(X\cup Y)^R \subseteq X^R\cap Y^R$;
\item $(X\cap Y)^R \supseteq X^R\cup Y^R$;
\item $(X\setminus Y)^R ? X^R\setminus Y^R$;
\item $(X\symdiff Y)^R ? X^R\symdiff Y^R$.
\end{enumerate}
and
\begin{enumerate}
\item $^R(Z\cup W) \subseteq {^RZ}\cup {^RW}$;
\item $^R(Z\cap W) \supseteq {^RZ}\cap {^RW}$;
\item $^R(Z\setminus W) ? {^RZ}\setminus {^RW}$;
\item $^R(Z\symdiff W) ? {^RZ}\symdiff {^RW}$.
\end{enumerate}
\end{corollary}
\begin{proof}\mbox{}
(1) We have both $X\cup Y \supseteq X$ and $X\cup Y \supseteq Y$, so $(X\cup Y)^R \subseteq X^R$ and $(X\cup Y)^R \subseteq Y^R$ by \ref{monotonicityImage}. Thus $(X\cup Y)^R \subseteq X^R\cap Y^R$.

(2) TODO ref order reversing function on lattice.

(3), (4) TODO
\end{proof}

\subsubsection{Extending the relation to powersets}
\begin{definition}
Let $R$ be a relation on $(A,B)$. Let $X\subseteq A$ and $Y\subseteq B$ be classes. Then we write $X\aset{R}Y$ if
\[ \forall x\in X: \forall y\in Y: \; xRy. \]
\end{definition}

\begin{lemma} \label{polarsSetRelation}
Let $R$ be a relation on $(A,B)$. Let $X\subseteq A$ and $Y\subseteq B$ be classes. Then
\[ X \aset{R} Y \quad\iff\quad X \subseteq {^RY} \quad\iff\quad Y \subseteq X^R. \]
\end{lemma}

\subsubsection{Greatest and least elements}
\begin{definition}
Let $\sSet{A, R}$ be a relational structure and $X\subseteq A$ a subclass.
\begin{itemize}
\item A \udef{greatest element}, \udef{largest element} or \udef{maximum} of $X$ is an upper bound of $X$ that is an element of $X$. We denote the class of maxima by $\max(X) \defeq X^R\cap X$.
\item A \udef{least element}, \udef{smallest element} or \udef{minimum} of $X$ is a lower bound of $X$ that is an element of $X$. We denote the class of minima by $\min(X) \defeq {^RX}\cap X$.
\end{itemize}
We also call
\begin{itemize}
\item an element of $\sup(S) \defeq \min(X^R) = X^R \cap (X^R)^{R^\transp}$ a \udef{least upper bound}, \udef{supremum}, or \udef{join};
\item an element of $\inf(S) \defeq \max({^RX}) = X^{R^\transp} \cap (X^{R^\transp})^R$ a \udef{greatest lower bound}, \udef{infimum}, or \udef{meet}.
\end{itemize}
\end{definition}

\begin{lemma} \label{minMaxSingletons}
If $\sSet{A, R}$ is an antisymmetric relational structure and $X\subseteq A$, then $\max(X), \min(X), \sup(X)$ and $\inf(X)$ are either singletons or empty.
\end{lemma}
\begin{proof}
We prove for $\max(X)$. The other cases follow dually or a fortiori. Let $x,y\in \max(X)$. Then $xRy$ and $yRx$, so $x=y$ by antisymmetry.
\end{proof}
In this case we use $\max/\min/\sup/\inf$ to denote the contents of the singleton rather than the singleton itself. If the set is empty, we say the $\max/\min/\sup/\inf$ does not exist.

\begin{lemma} \label{greatestLeastElementsSubsetPoset}
Let $\sSet{P, \precsim}$ be a poset and $R\subseteq S\subseteq P$.
\begin{enumerate}
\item If $\max(R)$ and $\max(S)$ exists, then $\max(R) \precsim \max(S)$.
\item If $\min(R)$ and $\min(S)$ exists, then $\min(R) \succsim \min(S)$.
\end{enumerate}
\end{lemma}
\begin{proof}
By definition $\max(S) \succsim x$ for all $x \in S$. Now $\max(R)\in R\subseteq S$, so in particular $\max(R) \precsim \max(S)$.
\end{proof}

TODO: set to class?
\begin{lemma} \label{maxSupMinInf}
If $\sSet{P, \Yleft}$ is a relational structure and $S\subseteq P$, then
\begin{enumerate}
\item $\max(S)\subset \sup(S)$;
\item $\min(S)\subset \inf(S)$.
\end{enumerate}
\end{lemma}
\begin{proof}
For (1) we calculate $\max(S) = S \cap S^u \subseteq (S^u)^l \cap S^u = \sup(S)$; (2) is dual.

TODO: post Galois??
\end{proof}

\subsubsection{Maximal and minimal elements}
\begin{definition}
Let $\sSet{A, R}$ be a relational structure and $X\subseteq A$ a subclass. We say
\begin{itemize}
\item $x\in X$ is \udef{strictly maximal} if $xR\cap X = \emptyset$;
\item $x\in X$ is \udef{maximal} if $xR\cap X \subseteq \{x\}$;
\item $x\in X$ is \udef{loosely maximal} if $xR\cap X \subseteq Rx$;
\end{itemize}
and
\begin{itemize}
\item $x\in X$ is \udef{strictly minimal} if $Rx\cap X = \emptyset$;
\item $x\in X$ is \udef{minimal} if $Rx\cap X \subseteq \{x\}$;
\item $x\in X$ is \udef{loosely minimal} if $Rx\cap X \subseteq xR$.
\end{itemize}
We write
\begin{itemize}
\item $\lmax(X)$ for the class of maximal elements in $X$;
\item $\llmax(X)$ for the class of loosely maximal elements in $X$;
\item $\lmin(X)$ for the class of minimal elements in $X$;
\item $\llmin(X)$ for the class of loosely minimal elements in $X$.
\end{itemize}
\end{definition}

\begin{lemma} \label{maximalMinimalEquivalents}
Let $\sSet{A, R}$ be a relational structure and $X\subseteq A$ a subclass. Then
\begin{enumerate}
\item $x\in X$ is strictly maximal in $X$ \textup{if and only if} $xR \perp X$;
\item $x\in X$ is maximal in $X$ \textup{if and only if} $xR\setminus\{x\} \perp X$;
\item $x\in X$ is loosely maximal in $X$ \textup{if and only if} $xR\setminus Rx \perp X$;
\end{enumerate}
and
\begin{enumerate}
\item $x\in X$ is strictly minimal in $X$ \textup{if and only if} $Rx \perp X$;
\item $x\in X$ is minimal in $X$ \textup{if and only if} $Rx\setminus\{x\} \perp X$;
\item $x\in X$ is loosely minimal in $X$ \textup{if and only if} $Rx\setminus xR \perp X$.
\end{enumerate}
\end{lemma}
\begin{proof}
Immediate from \ref{disjointSetDifference}.
\end{proof}

\begin{lemma} \label{maximalMinimalImplications}
Let $\sSet{A, R}$ be a relational structure, $X\subseteq A$ a subclass and $x\in X$. Then
\[ \text{$x$ is strictly maximal in $X$} \;\implies\; \text{$x$ is maximal in $X$} \;\implies\; \text{$x$ is loosely maximal in $X$}; \]
and
\[ \text{$x$ is strictly minimal in $X$} \;\implies\; \text{$x$ is minimal in $X$} \;\implies\; \text{$x$ is loosely minimal in $X$}. \]
\end{lemma}

\begin{lemma} \label{maximumIsMaximal}
Let $R$ be a relation on $A$ and $X\subseteq A$. Then
\begin{enumerate}
\item every maximum of $X$ is loosely maximal in $X$;
\item every minimum of $X$ is loosely minimal in $X$.
\end{enumerate}
If $\sSet{X, R|_X^X}$ is connex, then the converses also hold.
\end{lemma}
\begin{proof}
(1) Let $x\in X$ be a maximum. Then $x\in X^R$, so
\[ \{x\} \subseteq X^R \iff X\subseteq {^R\{x\}} = Rx \implies xR \cap X \subseteq Rx, \]
which means that $x$ is loosely maximal.

(2) Dual to (1).

(Converses for connex subsets) Let $x\in X$ be loosely maximal. For all $y\in X$ we have either $x\mathrel{R}y$ or $y\mathrel{R}x$. In the first case, $y\in xR\cap X$, so $y\in Rx$ by definition of loose maximality. Thus $y\mathrel{R}x$ in both cases, so $x\in X^R\cap X = \max(X)$.

The other converse is dual.
\end{proof}

\begin{lemma} \label{minimalMaximalSubset}
Let $\sSet{A,R}$ be a relational structure, $X, Y\subseteq A$ subsets and $x\in X$.
\begin{enumerate}
\item If $x$ is loosely minimal / minimal / strictly minimal in $Y$, then $x$ is loosely minimal / minimal / strictly minimal in $X$.
\item If $x$ is loosely maximal / maximal / strictly maximal in $Y$, then $x$ is loosely maximal / maximal / strictly maximal in $X$.
\end{enumerate}
\end{lemma}

\subsection{Converse relation}
\begin{definition}
Let $R$ be a relation on $(A, B)$. The \udef{converse} $R^\transp$ of $R$ is the relation on $B\times A$ with graph
\[ \graph(R^{\transp}) = \setbuilder{(y,x)}{(x,y)\in \graph(R)} \subset B\times A. \]
It is also known as the \udef{inverse}, \udef{transpose}, \udef{reciprocal}, \udef{opposite} or \udef{dual} of $R$.
\end{definition}

\begin{lemma}
Let $R,S$ be relations on $(A, B)$, $X\subseteq A$ and $Y\subseteq B$. Then
\begin{enumerate}
\item $(R^\transp)^\transp = R$;
\item $(R \cup S)^\transp = R^\transp \cup S^\transp$;
\item $(R \cap S)^\transp = R^\transp \cap S^\transp$;
\item $\dom(R^\transp) = \codom(R)$ and $\codom(R^\transp) = \dom(R)$;
\item $_{R^\transp}X = X_R$ and $Y_{R^\transp} = {_RY}$;
\item $\im(R^\transp) = \preim(R)$;
\item if $R\subseteq S$, then $R^\transp \subseteq S^\transp$.
\end{enumerate}
\end{lemma}

\begin{lemma}
Let $A,B$ be classes. Then
\begin{enumerate}
\item $U_{A,B}^\transp = U_{B,A}$;
\item $E_{A,B}^\transp = E_{B,A}$;
\item $\id_A^\transp = \id_A$.
\end{enumerate}
\end{lemma}

\subsection{Complementary relation}
\begin{definition}
Let $R$ be a binary relation on $(A, B)$. The \udef{complementary relation} $\overline{R}$ of $R$ is the relation on $(A, B)$ with graph
\[ \graph(\overline{R}) = \setbuilder{(x,y)}{\neg xRy}. \]
\end{definition}
\begin{lemma} \label{relationalComplementProperties}
Let $R,S$ be binary relations.
\begin{enumerate}
\item $\overline{\overline{R}} = R$;
\item $\overline{R^\transp} = \overline{R}^\transp$;
\item $\overline{R\cup S} = \overline{R}\cap \overline{S}$;
\item $\overline{R\cap S} = \overline{R}\cup \overline{S}$;
\item $U = R \cup \overline{R}$;
\item if $R \subseteq S$, then $\overline{R} \supseteq \overline{S}$.
\end{enumerate}
\end{lemma}


\begin{lemma} \label{imageComplementaryRelation}
Let $R$ be a relation on $(A,B)$, $x\in A$ and $y\in B$. Then
\begin{enumerate}
\item $x\overline{R} = B\setminus (xR)$;
\item $\overline{R}y = A\setminus (Ry)$.
\end{enumerate}
If $X\subseteq A$ and $Y\subseteq B$. Then
\begin{enumerate} \setcounter{enumi}{2}
\item $X_{\overline{R}} = B\setminus X^R$;
\item $_{\overline{R}}Y = A\setminus {^RY}$.
\end{enumerate}
\end{lemma}
\begin{proof}
(1) We calculate
\[ y \in x\overline{R} \iff \neg xRy \iff \neg (y\in xR) \iff y\in B\setminus (xR). \]

(2) Similar.

(3) We calculate, using (1),
\[ X_{\overline{R}} = \bigcup_{x\in X}x\overline{R} = \bigcup_{x\in X}B\setminus (xR) = B\setminus \left(\bigcap_{x\in X}xR\right) = B\setminus X^R. \]

(4) Similar.
\end{proof}
\begin{corollary}
Let $X\subseteq A$ be classes. Then
\[ X^c = X^{\overline{\id_A}}, \]
where the complement is taken with respect to $A$.
\end{corollary}
\begin{proof}
We have $X = X_{\id_A} = (X^{\overline{\id_A}})^c$.
\end{proof}

\subsection{Composition of relations}
\begin{definition}
Let $R$ be a relation on $(A, B)$ and $S$ a relation on $(B, C)$. Then the \udef{composition} of $R$ and $S$ is a new relation $R;S$ on $(A, C)$ with graph
\[ \graph(R;S) = \setbuilder{(x,z)\in A\times C}{\exists y\in B: xRy \land ySz}. \]
If $R$ and $S$ are relations such that the codomain of one is the domain of the other, they are called \udef{composable}.

If $R$ and $S$ are composable we also define the notation
\[ S\circ R \defeq R;S. \]
\end{definition}
\begin{lemma} \label{relationalComposition}
Let $R,S,T$ be composable relations.
\begin{enumerate}
\item The composition is associative: $R;(S;T) = (R;S);T$.
\item $(R;S)^\transp = S^\transp ; R^\transp$;
\item $(R\cup S);T = (R;T) \cup (S;T)$;
\item $(R\cap S);T \subseteq R;T \cap S;T$.
\end{enumerate}
\end{lemma}
\begin{proof}
TODO
\end{proof}
TODO: equality for $\cap$ with functions!!

\begin{lemma} \label{setOfRelationComposition}
Let $R,S$ be composable relations. Then for all $x,y$
\begin{enumerate}
\item $x(R;S)y \iff xR \mesh Sy$;
\item $x(\overline{R;S})y \begin{aligned}[t]
&\iff xR\perp Sy \\
&\iff xR \subseteq \overline{S}y;
\end{aligned}$
\item $x(\overline{R;\overline{S}})y \iff xR \subseteq Sy$.
\end{enumerate}
\end{lemma}
\begin{proof}
TODO
\end{proof}
\begin{corollary}
Let $R,S$ be composable relations. If $R$ is right unique, then $\overline{R;\overline{S}} = R;S$.
\end{corollary}


\begin{lemma}
Let $A,B,C,D$ be relations such that $A \subseteq B$ and $C\subseteq D$ and both $A,B$ and $C,D$ are composable, then $A;C\subseteq B;D$.
\end{lemma}
\begin{proof}
Assume the hypotheses of the lemma and let $(x,y) \in \graph(A;C)$. Then there exists a $z$ such that $xAz$ and $zCy$. By hypothesis this means $xBz$ and $zDy$, so $x(B;D)y$.
\end{proof}

\begin{lemma} \label{universalQuantificationForCompositionSuperset}
Let $R,S$ be composable and $T$ a relation. Then
\[ R;S \subseteq T \qquad\iff\qquad \forall x,y,z:\; xRy \land ySz \implies xTz. \]
\end{lemma}

\begin{lemma} \label{compositionCanonicalRelations}
Let $A,B,X, Y$ be classes and $R$ a relation on $(A, B)$. Then
\begin{enumerate}
\item $\id_A;R = R = R;\id_B$;
\item if $S,T\subseteq A$, then $\id_{S};\id_{T} = \id_{S\cap T}$;
\item $E_{X,A};R = E_{X,B}$ and $R; E_{B,Y} = E_{A,Y}$;
\item $U_{X,A};R = \U_{X,A_R}$ and $R; U_{B,Y} = U_{_RB, Y}$;
\item $U_{X,A};R;U_{B,Y} = \begin{cases}
U_{X,Y} & \graph(R) \neq \emptyset \\
E_{X,Y} & \graph(R) = \emptyset.
\end{cases}$
\end{enumerate}
\end{lemma}

\begin{lemma} \label{kernelInclusions}
Let $R$ be a relation. Then
\begin{enumerate}
\item $\id_{\preim(R)} \subseteq R;R^\transp \subseteq U_{\preim(R)}$;
\item $\id_{\im(R)} \subseteq R^\transp;R \subseteq U_{\im(R)}$.
\end{enumerate}
\end{lemma}
\begin{corollary}
Let $R$ be a relation on $(A,B)$, then
\begin{enumerate}
\item $R \subseteq R;R^\transp;R$;
\item $(\id_A \cap R;R^\transp);R = R = R;(\id_{B}\cap R^\transp;R)$.
\end{enumerate}
\end{corollary}
\begin{corollary}
Let $R$ be a relation on $(A,B)$. Then
\begin{enumerate}
\item $\id_A \;\subseteq\; R;R^\transp \cup \overline{R};\overline{R}^\transp$;
\item $\id_A \;\subseteq\; R^\transp;R \cup \overline{R}^\transp;\overline{R}$.
\end{enumerate}
\end{corollary}

\begin{lemma}
Let $R$ be a relation on $(A, B)$ and $S$ a relation on $(B, C)$. Let $X\subset A$ and $Y\subset C$. Then
\begin{enumerate}
\item $X_{R;S} = (X_R)_S = X_{S\circ R}$;
\item $_{R;S}Y = {_R({_SY})} = {_{S\circ R}Y}$.
\end{enumerate}
\end{lemma}

\begin{lemma}
Let $R$ be a relation on $(A,B)$ and $S$ a relation on $(B,C)$. Let $X\subseteq A$. Then $(X^R)_S \subseteq X^{R;S}$
\end{lemma}
\begin{proof}
We have
\[ z\in (X^R)_S \iff \Big[\exists y\in B: \forall x\in X: xRy \land ySz\Big] \implies \Big[\forall x\in X:\exists y\in B: xRy \land ySz\Big] \iff \Big[\forall x\in X: x(R;S)z\Big] \iff z\in X^{R;S}. \] 
\end{proof}

\begin{proposition}[Dedekind formula] \label{DedekindFormula}
Let $R,S,T$ be compatible relations. Then
\[ (R;S)\cap T \subseteq (R\cap (T; S^\transp));(S\cap (R^\transp;T)). \]
\end{proposition}
\begin{proof}
Take $(x,z)\in \graph((R;S)\cap T)$. Then $xTy$ and $xR \mesh Sy$, meaning we can take a $z\in xR\cap Sy$, i.e.\ satisfying $xRz$ and $zSy$. It is then easy to show that $x(R\cap (T; S^\transp))z$ and $z(S\cap (R^\transp;T))y$.
\end{proof}

\begin{lemma}
Let $R,S,T, Q$ be relations. Then $R^\transp; S \subseteq T$ implies $R;Q\cap S \subseteq R; (Q\cap T)$.
\end{lemma}

\begin{definition}
Let $R$ be a homogeneous relation on a class $A$. Then we can define $R^n$ as the $n$-fold composition of $R$:
\[ R^n \defeq \underbrace{R;R; \ldots ;R}_{\text{$n$ times}}. \]
We call $R$ \udef{idempotent} if $R^2 = R$.
\end{definition}

TODO: refine
\begin{proposition}
\begin{itemize}
\item transitive equivalent with $\graph(R^2) \subseteq \graph(R)$
\item reflexive implies $\graph(R) \subseteq \graph(R^2)$
\end{itemize}
\end{proposition}
so preorder sufficient, but not necessary for idempotent.

\subsubsection{Left and right residuals}
\begin{definition}
Let $R,S$ be relations.
\begin{itemize}
\item If $R,S$ have the same codomain, we define the \udef{right residual} as $R\diagup S \defeq \overline{\overline{R}; S^\transp}$.
\item If $R,S$ have the same domain, we define the \udef{left residual} as $S\diagdown R \defeq \overline{S^\transp; \overline{R}}$.
\end{itemize}
\end{definition}

\begin{lemma} \label{residuals}
Let $R,S,T$ be relations. Then
\begin{enumerate}
\item $(R\diagdown S)^\transp = R^\transp\diagup S^\transp$ and $(R\diagup S)^\transp = R^\transp\diagdown S^\transp$;
\item $R^\transp\diagdown S = \overline{R}\diagup \overline{S}^\transp$ and $R^\transp\diagup S = \overline{R}\diagdown \overline{S}^\transp$;
\item $R \diagdown (T\cap S) = R\diagdown T \cap R\diagdown S$ and $(T\cap S)\diagup R = T\diagup R \cap S\diagup R$;
\item $R \diagdown (T\cup S) = R\diagdown T \cup R\diagdown S$ and $(T\cap S)\diagup R = T\diagup R \cup S\diagup R$;
\item $(R\cap S) \diagdown T = R\diagdown T \cup S\diagdown T$ and $T\diagup (R\cap S) = T\diagup R \cup T\diagup S$;
\item $(R\cup S) \diagdown T = R\diagdown T \cap S\diagdown T$ and $T\diagup (R\cup S) = T\diagup R \cap T\diagup S$;
\item if $R\subseteq T$, then $R\diagup S \subseteq T\diagup S$ and $S\diagdown R \subseteq S\diagdown T$;
\item if $S\subseteq T$, then $R\diagup S \supseteq R\diagup T$ and $S\diagdown R \supseteq T\diagdown R$.
\end{enumerate}
\end{lemma}

As an aide-mémoire: the residuals are monotone in the ``numerator'' and antitone in the ``denominator'', where the numerator and denominator refer the the relation above, resp. below, the line in both residuals. The left residual has the denominator on the left; the right residual has it on the right.

\begin{proposition}[Schröder rule] \label{SchroderRule}
Let $R,S,T$ be relations. Then
\begin{align*}
R;S \subseteq T &\quad\iff\quad \overline{R} \supseteq \overline{T};S^\transp \quad\iff\quad R \subseteq T\diagup S \\
&\quad\iff\quad \overline{S} \supseteq R^\transp ; \overline{T} \quad\iff\quad S \subseteq R\diagdown T
\end{align*}
\end{proposition}
\begin{proof}
The second line follows from the first by applying the first to $S^\transp;R^\transp \subseteq T^\transp$. The second equivalence is immediate by \ref{relationalComplementProperties} and $\overline{\overline{T};S^\transp} = T\diagup S$. For the first equivalence we only need to prove the direction $\Rightarrow$: applying this implication to $\overline{T};S^\transp \subseteq \overline{R}$ gives $\overline{\overline{T}}\supseteq \overline{\overline{R}};S^{\transp\transp}$. i.e.\ $R;S \subseteq T$.

So assume $R;S \subseteq T$. By \ref{universalQuantificationForCompositionSuperset} this is equivalent to
\[ \forall x,y,z:\; xRy \land ySz \implies xTz. \]
Fix arbitrary $x,y,z$. Assume $ySz$ and $\neg xTz$. This means we must have $\neg xRy$, or we could also derive $xTz$, leading to a contradiction. Thus $ySz \land x\overline{T}z$ imply $x\overline{R}y$. Using \ref{universalQuantificationForCompositionSuperset} again, we get $\overline{T};S^\transp \subseteq \overline{R}$.
\end{proof}
We can also give a proof using image and preimage classes.
\begin{proof}
As before it is enough to prove the first implication. So assume $R;S \subseteq T$; we want to prove $R \subseteq T\diagup S = \overline{\overline{T}; S^\transp}$.

Take $x,y$ such that $xRy$. Then $yS \subseteq xT$, because
\[ \forall z: \quad ySz \implies xRy\land ySz \implies x(R;S)z \implies xTz. \]
We have the following equivalences:
\[ yS \subseteq xT \iff S^\transp y \subseteq xT \iff \overline{S^\transp}y \supseteq \overline{xT} \iff x(\overline{\overline{T}; S^\transp})y, \]
using \ref{setOfRelationComposition} for the last equivalence.
\end{proof}
\begin{corollary} \label{GaloisConnectionFromSchroderRule}
Let $T,S$ be relations. Then
\begin{enumerate}
\item $(R\diagup S);S \subseteq R$ and $S;(S\diagdown R) \subseteq R$;
\item $(R;S)\diagup S \supseteq R$ and $R\diagdown (R;S) \supseteq S$;
\item $((R;S)\diagup S);S = R;S$ and $R;(R\diagdown(R;S)) = R;S$.
\end{enumerate}
\end{corollary}
\begin{proof}
(1) Setting $R$ in the proposition to $R\diagup S$, we get that the truth $R\diagup S \subseteq R\diagup S$ implies $(R\diagup S);S \subseteq R$. The case for $S\diagdown R$ is similar.

(2) Now we set $T$ in the proposition to $R;S$.

(3) This is a combination of (1) and (2): Set the $R$ in (1) to $R;S$ to get $((R;S)\diagup S);S \subseteq R;S$. From (2) we see that $(R;S)\diagup S \supseteq R$, so $((R;S)\diagup S);S \supseteq R;S$.
\end{proof}
\begin{corollary}
Let $R,S,T$ be relations. Then
\[ \overline{R}^\transp;\overline{S}^\transp \subseteq T \quad\iff\quad \overline{S}^\transp;\overline{T}^\transp \subseteq R \quad\iff\quad \overline{T}^\transp;\overline{R}^\transp \subseteq S. \]
\end{corollary}

Suppose we have relations $R$ and $T$ with the same domain and we are interested in finding a relation $X$ such that
\[ R;X = T. \]
It will not always be possible to find such an $X$. It is, however, always possible to find an $X$ such that $R;X \subseteq T$ (for example we could take the empty relation). The Schröder rule says that $R\diagdown T$ is the largest $X$ satisfying this inequality (i.e.\ for all such $X$ we have $X\subseteq R\diagdown T$).

So, if the equation $R;X = T$ has a solution, then it must be the left residual $X = R\diagdown T$. There is a similar result for the right residual.

\begin{corollary}
Let $R,T$ be relations and suppose they have the same domain. Then
\begin{align*}
\text{There exists an $X$ such that $R;X = T$} \quad&\iff\quad R;(R\diagdown T) = T \quad\iff\quad R;(R\diagdown T) \supseteq T \\
&\implies\quad X = R\diagdown T.
\end{align*}
Suppose $R$ and $T$ have the same codomain, then
\begin{align*}
\text{There exists an $X$ such that $X;R = T$} \quad&\iff\quad (T\diagup R);R = T \quad\iff\quad (T\diagup R);R \supseteq T \\
&\implies\quad X = T\diagup R.
\end{align*}
\end{corollary}

\subsubsection{Symmetric quotient}
\begin{definition}
Let $R,S$ be composable relations. We define the \udef{symmetric quotient} of $R$ and $S$ as
\[ R \syq S \defeq \overline{R;\overline{S}} \;\cap\; \overline{\overline{R};S}. \]
\end{definition}
Usually in the literature the first argument of the symmetric quotient transposed, i.e.\ it is defined as $R^\transp \syq S$.

\begin{lemma}
Let $R,S$ be composable relations. Then
\begin{align*}
R\syq S &= R^\transp\diagdown S \cap R\diagup S^\transp \\
&= \overline{R}\diagup \overline{S}^\transp \cap \overline{R}^\transp \diagdown \overline{S}.
\end{align*}
\end{lemma}

\begin{lemma}
Let $R,S$ be composable relations. Then
\[ x(R\syq S)y \qquad\iff\qquad xR = Sy. \]
\end{lemma}
\begin{proof}
Immediate from \ref{setOfRelationComposition}.
\end{proof}

\subsection{Restrictions and extensions}
\begin{definition}
Let $R$ be a relation on $(A, B)$, $X\subseteq A$ and $Y \subseteq B$. The \udef{restriction} of $R$ to $(X,Y)$ is the relation $R|^Y_X$ on $(X,Y)$ with graph
\[ \graph(R|^Y_X) = \graph(R)\cap (X\times Y). \]
\begin{itemize}
\item If $Y = B$, then the restriction is called the \udef{left-restriction} of $R$ to $X$ and denoted $\left.R\right|_X$.
\item If $X = A$, then the restriction is called the \udef{right-restriction} of $R$ to $Y$ and denoted $\left.R\right|^Y$.
\end{itemize}
If $S$ is a restriction of $R$, then $R$ is called an \udef{extension} of $S$.
\end{definition}

\begin{lemma}
Let $R$ be a relation on $(A, B)$, $X\subseteq A$ and $Y \subseteq B$. Then
\[ R|^Y_X = \id_X;R;\id_Y. \]
\end{lemma}
\begin{corollary}
Let $R$ be a relation on $(A, B)$, $X_1,X_2\subseteq A$ and $Y_1,Y_2 \subseteq B$. Then
\[ \left.\left(R|_{X_1}^{Y_1}\right)\right|_{X_2}^{Y_2} = R|_{X_1\cap X_2}^{Y_1\cap Y_2}. \]
\end{corollary}
\begin{proof}
Use \ref{compositionCanonicalRelations}.
\end{proof} 

\begin{lemma}
Let $R$ be a relation on $(A,B)$, $X\subseteq A$ and $Y\subseteq B$. Then
\begin{enumerate}
\item $X_R = \im(R|_X) = \im(\id_X;R)$;
\item $_RY = \preim(R|^Y) = \preim(R;\id_Y)$.
\end{enumerate}
\end{lemma}

\subsection{Galois connections}
\begin{proposition}
Consider a monoid $M$ of relations under composition. Then for $R\in M$ the following maps form a Galois connection:
\[ \rho_R: M\to M: S \mapsto S;R \qquad\text{and}\qquad \rho_R^+: M\to M: S \mapsto S\diagup R \]
as do the following:
\[ \lambda_R: M\to M: S \mapsto R;S \qquad\text{and}\qquad \lambda_R^+: M\to M: S \mapsto R\diagdown S. \]
\end{proposition}
\begin{proof}
The is just a restatement of \ref{GaloisConnectionFromSchroderRule}.
\end{proof}

\begin{proposition}
We can order relations by image, $\leq_i$ or by preimage $\leq_p$:
\[ R \leq_i S \defequiv \im(R) \subseteq \im(S) \qquad R \leq_p S \defequiv \preim(R) \subseteq \preim(S). \]
These orders are preorders, but not partial orders.
\begin{enumerate}
\item If we order relations by image, then $X\mapsto \id_X$ and $\im$ form a Galois connection;
\item If we order relations by preimage, then $X\mapsto \id_X$ and $\preim$ form a Galois connection.
\end{enumerate}
\end{proposition}

\begin{lemma}
Let $R$ be a relation on $(A,B)$, $X\subseteq A$ and $Y\subseteq B$, then
\begin{enumerate}
\item $X_R = \im(\id_X; R)$ and $_RX = \preim(R;\id_X)$;
\item $X^R = B\setminus \im(\overline{\id_X}; R)$ and $^RX = A\setminus \preim(R;\overline{\id}_X)$.
\end{enumerate}
\end{lemma}

\begin{proposition}
Let $R$ be a relation on $(A,B)$. Then
\begin{enumerate}
\item $\powerset(A)\to \powerset(B): X\to X^R$ and $\powerset(B)\to \powerset(A): X\to {^RX}$ form a Galois connection;
\item $\powerset(A)\to \powerset(B): X\to X^{\overrightarrow{R}}$ and $\powerset(B)\to \powerset(A): X\to {^{\overleftarrow{R}}X}$ form a Galois connection.
\end{enumerate}
\end{proposition}

\subsubsection{Closures}
TODO use Galois theory
\begin{definition}
Let $R$ be a homogeneous relation on a class $A$.
\begin{itemize}
\item The \udef{reflexive closure} of $R$ is the relation $R^=$ on $A$ with graph
\begin{align*}
\graph(R^=) &= \bigcap\setbuilder{\graph(R')}{\text{$R'$ extends $R$ and is reflexive}} \\
&= \setbuilder{(x,x)\in A\times A}{x\in A}\cup \graph(R); \\
&= \graph(\id_A)\cup \graph(R).
\end{align*}
\item The \udef{reflexive reduction} of $R$ is the relation $R^{\neq}$ on $A$ with graph
\[ \graph{R^{\neq}} = \graph(R)\setminus \setbuilder{(x,x)\in A\times A}{x \in A}. \]
\item The \udef{transitive closure} of $R$ is the relation $R^{+}$ on $A$ with graph
\[ \graph(R^+) = \bigcap\setbuilder{\graph(R')}{\text{$R'$ extends $R$ and is transitive}}. \]
\item The \udef{symmetric closure} of $R$ is the relation $R^{\leftrightarrow}$ on $A$ with graph
\begin{align*}
\graph(R^\leftrightarrow) &= \bigcap\setbuilder{\graph(R')}{\text{$R'$ extends $R$ and is symmetric}} \\
&=  \graph(R)\cup \graph(R^\transp).
\end{align*}
\end{itemize}
\end{definition}

TODO: $R^?$ for reflexive closure?

\begin{lemma}
Let $R$ be a homogeneous relation on a class $A$.
\begin{enumerate}
\item The reflexive closure $R^=$ is the smallest reflexive relation on $A$ that extends $R$.
\item The reflexive reduction $R^{\neq}$ is the largest irreflexive relation on $A$ that is a restriction of $R$.
\item The transitive closure $R^{+}$ is the smallest transitive relation on $A$ that extends $R$.
\item The symmetric closure $R^{\leftrightarrow}$ is the smallest symmetric relation on $A$ that extends $R$.
\end{enumerate}
\end{lemma}

\begin{lemma}
The closures (and reduction) are monotone: if $R$ extends $S$, then $R^a$ extends $S^a$ for all $a,b\in\{=,+,\leftrightarrow,\neq\}$.
\end{lemma}

\begin{lemma}
Let $R$ be a homogeneous relation over a class $A$. Then $\big(R^=\big)^\leftrightarrow = \big(R^\leftrightarrow\big)^=$.
\end{lemma}
\begin{proof}
We have $\big(R^=\big)^\leftrightarrow = R^=\cup \big(R^=\big)^\transp = (R\cup \id)\cup \big(R \cup \id\big)^\transp = R\cup R^\transp \cup \id = \big(R^\leftrightarrow\big)^=$.
\end{proof}

\begin{lemma}
Let $R$ be a homogeneous relation over a class $A$.
\begin{enumerate}
\item The reflexive transitive closure $R^*$ is the smallest preorder containing $R$.
\item The reflexive transitive symmetric closure $\equiv_R$ is the smallest equivalence relation containing $R$.
\end{enumerate}
\end{lemma}
TODO: equivalence relations are defined below.

\begin{definition}
The reflexive transitive closure $R^*$. The reflexive transitive symmetric closure $\equiv_R$.
\end{definition}

\subsubsection{Transitive closure}

\begin{proposition} \label{transitiveClosureConnection}
Let $R$ be a homogeneous relation over a class $A$ and $x,y\in A$. Then $x\mathrel{R^+}y$ \textup{if and only if} there exists a finite sequence $\seq{z_0, \ldots, z_n}$ such that $x=z_0$, $z_n=y$ and $\forall i<n: z_iRz_{i+1}$.
\end{proposition}
\begin{proof}
The proof of the direction $\Leftarrow$ is by induction on $n$.

For the direction $\Rightarrow$ TODO (+relate to $T_R$)
\end{proof}

\begin{lemma} \label{transpositionTransitiveClosure}
Let $R$ be a homogeneous relation over a class $A$. Then $(R^+)^\transp = (R^\transp)^+$.
\end{lemma}
\begin{proof}
TODO
\end{proof}

\begin{proposition}
Let $R$ be a homogeneous relation over a class $A$. Then
\begin{enumerate}
\item $\big(R^=\big)^+ = \big(R^+\big)^=$;
\item $\big(R^\leftrightarrow\big)^+ \supseteq \big(R^+\big)^\leftrightarrow$.
\end{enumerate}
\end{proposition}
\begin{proof}
(1) Take arbitrary $x,y\in A$. First assume $x\mathrel{\big(R^+\big)^=}y$. Then either $x=y$ or $x\mathrel{R^+}y$. In the first case we clearly have $x\mathrel{R^=}y$ and thus $x\mathrel{\big(R^=\big)^+}y$. In the second, we have $R^+ \subseteq \big(R^=\big)^+$ by monotonicity of the closure, so $x\mathrel{\big(R^=\big)^+}y$.

Now assume $x\mathrel{\big(R^=\big)^+}y$. Then consider a finite sequence as defined in \ref{transitiveClosureConnection}. Remove all repeated adjacent elements in the sequence. Then all adjacent elements are $R$-related and the initial and final elements are the same. These elements are related by $R^+$ if the sequence consists of at least two elements. If it consists of a single element, then $x=y$ and $x\mathrel{\big(R^+\big)^=}$.

(2) We have
\[ \big(R^\leftrightarrow\big)^+ = \big(R\cup R^\transp\big)^+ \supseteq R^+ \cup \big(R^\transp\big)^+ = R^+ \cup \big(R^+\big)^\transp = \big(R^+\big)^\leftrightarrow, \]
using \ref{orderPreservingFunctionLatticeOperations} and \ref{transpositionTransitiveClosure}.
\end{proof}
TODO: this uses references from later theory.

\begin{example}
There exists a relational structure $\sSet{A,R}$ such that
\[ \big(R^\leftrightarrow\big)^+ \neq \big(R^+\big)^\leftrightarrow. \]
Consider the relation $R = \to$ on $\{a,b,c\}$ with relationships
\[ \begin{tikzcd}
a \ar[dr] & {} & \ar[dl] c \\
{} & b & {}
\end{tikzcd}. \]
Then $a \mathrel{\big(R^\leftrightarrow\big)^+}b$ but not $a \mathrel{\big(R^+\big)^\leftrightarrow}b$.
\end{example}



\subsubsection{Convex hull}
\begin{definition}
Let $\sSet{A,R}$ be a relational structure and $X\subseteq A$ a subset. The \udef{$R$-convex hull} of $X$ is
\begin{align*}
\relconvex_R(X) &\defeq \setbuilder{a\in A}{\exists b,c\in X:\; bR^*a \land aR^*c} \\
&= \setbuilder{a\in A}{\exists b,c\in X:\; bR^+a \land aR^+c} \cup S.
\end{align*}
The subset $X$ is called \udef{$R$-convex} if $X = \relconvex_R(X)$.
\end{definition}

\begin{lemma}
Let $\sSet{A,R}$ be a relational structure and $X\subseteq A$ a subset. Then $\relconvex_R(X) = \relconvex_{R^+}(X) = \relconvex_{R^*}(X)$.
\end{lemma}

\begin{lemma} \label{transitiveClosureConnectionInSet}
Let $\sSet{A,R}$ be a relational structure and $X\subseteq A$ an $R$-convex subset. Then for all $x,y\in X$ such that $x\mathrel{R^+}y$, there exists a finite sequence $\seq{z_0, \ldots, z_n}$ in $X$ such that $x = z_0$, $z_n = y$ and $\forall i<n: z_iRz_{i+1}$.
\end{lemma}
\begin{proof}
TODO \ref{transitiveClosureConnection}
\end{proof}

\begin{lemma} \label{addedElementsRConvexHull}
Let $\sSet{A,R}$ be a relational structure and $X\subseteq A$. Then for all $x\in \relconvex_R(X)\setminus X$, there exist $y,z\in \relconvex_R(X)$ such that $y\neq x\neq z$  and $yRxRz$.

In particular none of the elements of $\relconvex_R(X)\setminus X$ are (strictly) maximal / minimal in $\relconvex_R(X)$.
\end{lemma}
\begin{proof}
TODO
\end{proof}

\begin{proposition} \label{maximalityMinimalityRConvexSets}
Let $\sSet{A,R}$ be a relational structure, $X\subseteq A$ an $R$-convex subset and $x\in X$. Then 
\begin{enumerate}
\item $x$ is $R$-maximal in $X$ \textup{if and only if} $x$ is $R^+$-maximal in $X$;
\item $x$ is strictly $R$-maximal in $X$ \textup{if and only if} $x$ is strictly $R^+$-maximal in $X$;
\item $x$ is $R$-minimal in $X$ \textup{if and only if} $x$ is  $R^+$-minimal in $X$;
\item $x$ is strictly $R$-minimal in $X$ \textup{if and only if} $x$ is strictly $R^+$-minimal in $X$.
\end{enumerate}
\end{proposition}
For (strict) minimality / maximality, the converse also holds as 
\begin{proof}
(1) First assume $x$ is $R$-maximal, so $xR \cap X \subseteq \{x\}$. Now take $y\in xR^+\cap X$. By \ref{transitiveClosureConnectionInSet}, there exists $\seq{z_0, \ldots, z_n}$ in $X$ such that $x = z_0$, $z_n = y$ and $\forall i<n: z_iRz_{i+1}$. WLOG we may assume that no adjacent elements are equal. If the sequence has length one, then set $y = x$. Otherwise set $y' = z_1$. We have $xRy'$, so $x=y'$. As $z_1$ must be different from $x$, we have that this is impossible. Thus $xR^+\cap X \subseteq \{x\}$.

The other direction ss immediate from $R \subseteq R^+$.

(2) Now assume $x$ is strictly $R$-maximal. The proof is similar, except a sequence of lenth one is now impossible, so $xR^+\cap X = \emptyset$.

(3, 4) Dual to (1,2).
\end{proof}

\begin{example}
Neither direction of \ref{maximalityMinimalityRConvexSets} holds in general for loosely minimal / maximal elements.
\begin{itemize}
\item Consider the relation $R$ on $\{x,y,z\}$ determined by
\[ \begin{tikzcd}[column sep=0.8em]
x \arrow[rr, bend left] & {} & y \arrow[dl, bend left] \\
{} & z \arrow[ul, bend left] & {}
\end{tikzcd}. \]
All three elements are loosely $R^+$-maximal in the $R$-convex set $\{x,y,z\}$, but none are loosely $R$-maximal.
\item Consider the relation $R$ on $\{x,y,z\}$ determined by
\[ \begin{tikzcd}
x \arrow[r, bend left] & \arrow[l, bend left] y \arrow[r] & z.
\end{tikzcd} \]
Then $x$ is loosely $R$-maximal in the $R$-convex set $\{x,y,z\}$, but it is not loosely $R^+$-maximal.
\end{itemize}
\end{example}

\subsection{Direct product}
TODO: extend: heterogeneous relations + direct product of two different relations.
\begin{definition}
Let $R$ be a homogeneous relation over a class $A$. We can turn $R$ into a relation over $(A, A)$ as follows:
\[ (a,b)R(c,d) \iff aRc \land bRd. \]
\end{definition}
\begin{lemma} \label{relationPropertiesDirectProduct}
The following properties of binary endorelations are conserved under taking the direct product:
\begin{enumerate}
\item all forms of reflexivity;
\item all forms of symmetry;
\item all forms of transitivity;
\item left and right Euclideanness;
\item density.
\end{enumerate}
The following properties are not necessarily conserved:
\begin{enumerate}
\item connexity;
\item semi-connexity;
\item trichotomy.
\end{enumerate}
\end{lemma}

\subsection{Homogeneous relations}
\begin{lemma} \label{selfRelatedElements}
Let $R$ be a homogeneous relation on $A$. Then
\begin{enumerate}
\item $R\cap \id_A = R^\transp\cap \id_A \subseteq R^\transp$;
\item $\id_A \cap R \subseteq R\cap R^\transp$;
\item $\id_A \cap R\cap \overline{R}^\transp \subseteq E_A$;
\item $R\cap \overline{R}^\transp \subseteq \overline{\id_A}$.
\end{enumerate}
\end{lemma}
\begin{proof}
(1) Assume $x(R\cap \id_A)y$, then $x=y$ and $xRy$. So $xRx$, meaning $xR^\transp x$ and thus $xR^\transp y$.

(2) Clearly $\id \cap R \subseteq R$. Combining this with (1) gives the result.

(3, 4) Follow from \ref{setPerpInequality}.
\end{proof}

\subsubsection{Reflexivity and fixed points}
\begin{definition}
Let $R$ be a homogeneous binary relation on a class $A$. The \udef{fixed points} of $R$ are the elements of the class
\[ \Fixedpoints(R) \defeq \setbuilder{a\in A}{aRa}. \]
\end{definition}


\begin{definition}
Let $R$ be a homogeneous binary relation on a class $A$. We say
\begin{itemize}
\item $R$ is \udef{reflexive} if $\Fixedpoints(R) = A$;
\item $R$ is \udef{irreflexive} or \udef{anti-reflexive} if $\Fixedpoints(R) = \emptyset$;
\item $R$ is \udef{quasi-reflexive} if every element that is related to some element is related to itself;
\item $R$ is \udef{left quasi-reflexive} if every element that is left related to some element is related to itself;
\item $R$ is \udef{right quasi-reflexive} if every element that is right related to some element is related to itself;
\item $R$ is \udef{coreflexive} if $xRy$ implies $x=y$.
\end{itemize}
\end{definition}

\begin{lemma} \label{relationalReflexivityEquivalents}
Let $R$ be a homogeneous relation on $A$. Then $R$ is
\begin{enumerate}
\item reflexive $\begin{aligned}[t]
&\text{\textup{if and only if}}\; \id_A \subseteq R \\
&\text{\textup{if and only if}}\; \id_A \perp \overline{R};
\end{aligned}$
\item irreflexive $\begin{aligned}[t]
&\text{\textup{if and only if}}\; \id_A \subseteq \overline{R} \\
&\text{\textup{if and only if}}\; \id_A \perp R;
\end{aligned}$
\item left quasi-reflexive \textup{if and only if} $\id_{\preim(R)} \subseteq R$;
\item right quasi-reflexive \textup{if and only if} $\id_{\im(R)} \subseteq R$;
\item quasi-reflexive \textup{if and only if} $\id_{\preim(R)\cup \im(R)} \subseteq R$;
\item coreflexive \textup{if and only if} $\begin{aligned}[t]
&\text{\textup{if and only if}}\; R \subseteq \id_A \\
&\text{\textup{if and only if}}\; R \perp \overline{\id}_A.
\end{aligned}$
\end{enumerate}
\end{lemma}

\begin{lemma}
Let $R$ be a homogeneous relation on $A$.
\begin{enumerate}
\item If $R$ is left quasi-reflexive, then $R^\transp$ is right quasi-reflexive.
\item If $R$ is right quasi-reflexive, then $R^\transp$ is left quasi-reflexive.
\item If $R$ is reflexive / irreflexive / coreflexive, then $R^\transp$ is too.
\end{enumerate}
\end{lemma}
\begin{proof}
(1) Assume $\id_{\preim(R)} \subseteq R$, then $\id_{\preim(R)} = \id_{\im(R^\transp)} = \id_{\im(R^\transp)}^\transp$. So $\id_{\im(R^\transp)} \subseteq R^\transp$.

(2) Similar.

(3) Follow simply because the converse preserves inclusions.
\end{proof}

\begin{lemma} \label{reflexiveIrreflexive}
Let $R$ be a homogeneous relation on $A$. Then $R$ is reflexive \textup{if and only if} $\overline{R}$ is irreflexive.
\end{lemma}

\begin{lemma}
Let $R$ be a relation. Then $R\diagup R$ and $R\diagdown R$ are reflexive.
\end{lemma}
\begin{proof}
We have $\id; R = R \subseteq R$, so $\id \subseteq R\diagup R$ by the Schröder rule \ref{SchroderRule}. The other case is similar.
\end{proof}

\subsubsection{Transitivity}
\begin{definition}
Let $R$ be a homogeneous binary relation on a class $A$. We say
\begin{itemize}
\item $R$ is \udef{transitive} if $\forall x,y,z\in A: [xRy \land yRz] \implies xRz$;
\item $R$ is \udef{intransitive} if it is not transitive;
\item $R$ is \udef{anti-transitive} if it is never transitive:
\[ \forall x,y\in A: (xRy\land yRz) \implies \neg xRz.\]
\end{itemize}
\end{definition}

\begin{lemma}
Let $R$ be a homogeneous relation on $A$. Then $R$ is
\begin{enumerate}
\item transitive \textup{if and only if} $R; R \subseteq R$;
\item anti-transitive \textup{if and only if} $R; R \subseteq \overline{R}$.
\end{enumerate}
\end{lemma}

\begin{lemma}
Let $R$ be a homogeneous relation on $A$. If $R$ is transitive / anti-transitive, then $R^\transp$ is too.
\end{lemma}


\begin{lemma}
\begin{enumerate}
\item An anti-transitive relation is always irreflexive.
\item An irreflexive and left- (or right-) unique relation is always anti-transitive.
\item An anti-transitive relation on a class of more than four elements elements is never connex.
\end{enumerate}
\end{lemma}

\subsubsection{Symmetry}
\begin{definition}
Let $R$ be a homogeneous binary relation on a class $A$. We say
\begin{itemize}
\item $R$ is \udef{symmetric} if $\forall x,y\in A: xRy \implies yRx$;
\item $R$ is \udef{asymmetric} if $\forall x,y\in A: xRy \implies \neg yRx$;
\item $R$ is \udef{antisymmetric} if $\forall x,y\in A: (xRy\land yRx) \implies x=y$.
\end{itemize}
\end{definition}

\begin{lemma} \label{relationalSymmetryEquivalents}
Let $R$ be a homogeneous relation on $A$. Then $R$ is
\begin{enumerate}
\item symmetric \textup{if and only if} $R = R^\transp$;
\item asymmetric $\begin{aligned}[t]
&\text{\textup{if and only if}}\; R \subseteq \overline{R}^\transp \\
&\text{\textup{if and only if}}\; R = R \cap \overline{R}^\transp \\
&\text{\textup{if and only if}}\; R \cap R^\transp = E_A;
\end{aligned}$
\item antisymmetric $\begin{aligned}[t]
&\text{\textup{if and only if}}\; R\cap R^\transp \subseteq \id_A \\
&\text{\textup{if and only if}}\; R^\transp\subseteq \overline{R}\cup \id_A.
\end{aligned}$
\end{enumerate}
\end{lemma}
\begin{proof}
(1) The inclusion $R\subseteq R^\transp$ follows straight from the definition. The other inclusion is obtained by taking the converse.

(2) The third equation is a consequence of \ref{setPerpInequality}.

(3) For the second equivalence we calculate using \ref{setPerpInequality}:
\[ R\cap R^\transp \subseteq \id_A \iff R\cap R^\transp \cap\overline{\id_A} = E_A \iff R^\transp \subseteq \overline{R \cap \overline{\id_A}} = \overline{R} \cup \id_A. \]
\end{proof}
\begin{corollary} \label{asymmetryAntisymmetry}
Asymmetry implies antisymmetry.
\end{corollary}

\begin{lemma}
Let $R$ be a homogeneous relation on $A$.
\begin{enumerate}
\item If $R$ is symmetric / asymmetric / antisymmetric, then $R^\transp$ is too.
\item If $R$ is symmetric / asymmetric, then $\overline{R}$ is too.
\end{enumerate}
\end{lemma}

\begin{lemma} \label{asymmetricIrreflexive}
Let $R$ be a relation. Then
\begin{enumerate}
\item if $R$ is asymmetric \textup{if and only if} $R$ is irreflexive and antisymmetric;
\item if $R$ is transitive, then $R$ is asymmetric iff $R$ is irreflexive.
\end{enumerate}
\end{lemma}
\begin{proof}
(1) Assume $R$ asymmetric. Then  $E_A = R \cap R^\transp \supseteq \id_A\cap R \supseteq E_A$ from \ref{selfRelatedElements}. Thus $\id_A\cap R = E_A$, which means $R$ is irreflexive.

Also $R\cap R^\transp = E_A \subseteq \id_A$, so $R$ is antisymmetric by \ref{relationalSymmetryEquivalents}.

Now assume $R$ antisymmetric and irreflexive. Then $R\cap R^\transp \subseteq \id_A$, so $R\cap R^\transp = \id_A \cap R\cap R^\transp$, but $\id_A\cap R = \emptyset$ by \ref{relationalReflexivityEquivalents}, so $R\cap R^\transp = E_A$ and thus $R$ is asymmetric by \ref{relationalSymmetryEquivalents}.

(2) Assume $R$ transitive and irreflexive. Then $R;R \subseteq R \subseteq \overline{\id}_A$. By Schröder's rule, \ref{SchroderRule}, we have $R;\id_A \subseteq \overline{R}^\transp$ and thus $R\perp R^\transp$ by \ref{setPerpInequality}.
\end{proof}

\begin{proposition} \label{symmetricAsymmetricDecomposition}
Let $R$ be a homogeneous relation. Then $R$ can be decomposed as $R = R_S \cup R_A$ where
\begin{enumerate}
\item $R_S \defeq R \cap R^\transp$ is symmetric; and
\item $R_A \defeq R \cap \overline{R}^\transp$ is asymmetric.
\end{enumerate}
\end{proposition}
\begin{proof}
We have
\[ R = R\cap U = R\cap (R^\transp \cup \overline{R}^\transp) = (R \cap R^\transp) \cup (R \cap \overline{R}^\transp) = R_S \cup R_A. \]
(1) From $R_S^\transp = (R \cap R^\transp)^\transp = R^\transp \cap R = R_S$, we see that $R_S$ is symmetric.

(2) From
\[ R_A \cap R_A^\transp = R \cap \overline{R}^\transp \cap R^\transp \cap \overline{R} = (R \cap \overline{R}) \cap (R^\transp \cap \overline{R}^\transp) = E_A, \]
we see that $R_A$ is asymmetric.
\end{proof}\

Using this decomposition we can rephrase the antisymmetry property as $R_S \subseteq \id_A$.

\begin{lemma} \label{symmetricMaximalityMinimality}
Let $R$ be a relation on $A$, $X\subseteq A$ and $x\in X$.
\begin{enumerate}
\item If $R$ is antisymmetric, then
\begin{enumerate}
\item $x$ is maximal in $X$ \textup{if and only if} $x$ is loosely maximal in $X$;
\item $x$ is minimal in $X$ \textup{if and only if} $x$ is loosely minimal in $X$.
\end{enumerate}
\item If $R$ is asymmetric, then
\begin{enumerate}
\item strictly maximality, maximality and loose maximality in $X$ are equivalent;
\item strictly minimality, minimality and loose minimality in $X$ are equivalent.
\end{enumerate}
\end{enumerate}
\end{lemma}
\begin{proof}
(1) We prove (a); point (b) is dual.

We only need to prove that loose maximality implies maximality. Let $x$ be loosely maximal. Then $xR \cap X \subseteq Rx$, which implies $xR\cap X \subseteq xR \cap Rx \subseteq \{x\}$, where the last inclusion follows from antisymmetry. Thus $x$ is maximal.

(2) We prove (a); point (b) is dual.

We only need to prove that loose maximality implies strict maximality. Let $x$ be loosely maximal. Then $xR \cap X \subseteq Rx$, which implies $xR\cap X \subseteq xR \cap Rx \subseteq \emptyset$, where the last inclusion follows from asymmetry. Thus $x$ is strictly maximal.
\end{proof}

\subsubsection{Connexity}
\begin{definition}
Let $R$ be a homogeneous binary relation on a class $A$. We say
\begin{itemize}
\item $R$ is \udef{connex} (or \udef{connected} or \udef{complete}) if $\forall x,y\in A: xRy \lor yRx$;
\item $R$ is \udef{semi-connex} (or \udef{weakly connected} or \udef{total}) if $\forall x,y\in A: xRy \lor yRx \lor x=y$;
\item $R$ is \udef{trichotomous} if $\forall x,y\in A$, exactly one of $xRy, yRx$ or $x=y$ holds.
\end{itemize}
\end{definition}

\begin{lemma}
Let $R$ be a homogeneous relation on $A$. Then
\begin{enumerate}
\item $\begin{aligned}[t]
R \;\text{is connex} &\iff U_A = R\cup R^\transp \iff E_A = \overline{R}\cap \overline{R}^\transp\\
&\iff \overline{R}\subseteq R^\transp \iff \overline{R}\; \text{is asymmetric};
\end{aligned}$
\item $\begin{aligned}[t]
R\; \text{is semi-connex} &\iff U_A = R\cup R^\transp \cup \id_A \iff \overline{\id}_A\subseteq R\cup R^\transp \\
&\iff \overline{R}\cap \overline{R}^\transp \subseteq \id_A \iff \overline{R}\; \text{is antisymmetric}
\end{aligned}$;
\item $\begin{aligned}[t]
R\; \text{is trichotomous} &\iff U_A = R\symdiff R^\transp \symdiff \id_A \\
&\iff \begin{cases}
U_A = R\cup R^\transp \cup \id_A \\
E_A = R\cap R^\transp \cap \id_A
\end{cases} \iff R\cup R^\transp = \overline{\id}_A \\
&\iff \begin{cases}
U_A = R\cup R^\transp \cup \id_A \\
E_A = R\cap \id_A \\
E_A = R \cap R^\transp
\end{cases} \iff R \; \text{is}\;\begin{cases}
\text{semi-connex} \\ \text{irreflexive} \\ \text{asymmetric}
\end{cases} \\
&\iff R \; \text{is}\;\begin{cases}
\text{semi-connex} \\ \text{asymmetric}
\end{cases} \iff \overline{R} \; \text{is}\;\begin{cases}
\text{antisymmetric} \\ \text{connex}
\end{cases}.
\end{aligned}$
\end{enumerate}
\end{lemma}

\begin{corollary} \label{connexityConsequences}
Let $R$ be a homogeneous relation. Then
\begin{enumerate}
\item if $R$ is connex, then $R$ is reflexive.
\end{enumerate}
\end{corollary}
\begin{proof}
(1) We have
\[ R \; \text{connex} \iff \overline{R} \; \text{asymmetric} \implies \overline{R} \; \text{irreflexive} \iff R \; \text{reflexive}, \]
using \ref{asymmetricIrreflexive} and \ref{reflexiveIrreflexive}. 
\end{proof}

\begin{lemma}
Let $R$ be a homogeneous relation on $A$. If $R$ is connex / semi-connex / trichotomous, then $R^\transp$ is too.
\end{lemma}

\subsubsection{Euclideanness}
\begin{definition}
Let $R$ be a homogeneous binary relation on a class $A$. We say
\begin{itemize}
\item $R$ is \udef{(right) Euclidean} if $\forall x,y,z\in A: xRy \land xRz \implies yRz$;
\item $R$ is \udef{left Euclidean} if $\forall x,y,z\in A: yRx \land zRx \implies yRz$.
\end{itemize}
\end{definition}

\begin{lemma}
Let $R$ be a homogeneous relation on $A$. Then $R$ is
\begin{enumerate}
\item Euclidean \textup{if and only if} $R^\transp; R \subseteq R$;
\item left Euclidean \textup{if and only if} $R; R^\transp \subseteq R$.
\end{enumerate}
\end{lemma}
\begin{lemma}
Let $R$ be a homogeneous relation on $A$. If
\begin{enumerate}
\item $R$ is right Euclidean, then $R^\transp$ is left Euclidean;
\item $R$ is left Euclidean, then $R^\transp$ is right Euclidean.
\end{enumerate}
\end{lemma}

\subsubsection{Density}
\begin{definition}
Let $R$ be a homogeneous binary relation on a class $A$. We say
\begin{itemize}
\item $R$ is \udef{dense} if $\forall x,y\in A: xRy \implies [\exists z\in A: xRz \land zRy]$.
\end{itemize}
\end{definition}

\begin{lemma}
A homogeneous relation $R$ is dense \textup{if and only if} $R \subseteq R;R$.
\end{lemma}
\begin{lemma}
If $R$ is reflexive, then $R$ is dense.
\end{lemma}
\begin{proof}
Assume $R$ reflexive, i.e.\ $\id_A\subseteq R$. Then $R = \id_A ;R \subseteq R;R$.
\end{proof}

\begin{lemma}
Let $R$ be a homogeneous relation on $A$. If $R$ is dense, then $R^\transp$ is dense.
\end{lemma}

\subsection{Equivalence relations and generalisations}
\begin{definition}
An \udef{equivalence relation} is a binary relation that is reflexive, symmetric and transitive.
\end{definition}

\begin{lemma}
Let $A,B$ be classes. Then the empty relation on $(A,B)$, the universal relation $U_{A,B}$ and the identity relation $\id_A$ are equivalence relations.
\end{lemma}

\subsubsection{Equivalence classes and partitions}
\begin{definition}
Let $\sim$ be an equivalence relation on $A$. Let $x\in A$. The \udef{equivalence class} of $x$ is the set
\[ [x]_\sim \defeq x\sim \; = \; \sim x = \setbuilder{y\in A}{x\sim y}. \]
Then $x$ is called a \udef{representative} of this equivalence class.

If $A$ is a \emph{set}, there is a set of all equivalence classes. This is called the \udef{quotient set} of $A$ by $\sim$
\[ A/\sim \defeq \{ [x]_\sim \in \powerset(A) \;|\; x\in A \}. \]
\end{definition}
So the equivalence classes are the principle images / preimages of the equivalence relation.

\begin{proposition}
Let $\sim$ be an equivalence relation on a set $A$. The quotient set of $A$ by $\sim$ defines a \udef{partition} of $A$:
\begin{enumerate}
\item Every equivalence class is non-empty (each equivalence class has a representative);
\item every element of $A$ is in an equivalence class, i.e.\
\[ A = \bigcup (A/\sim); \]
\item any two equivalence classes are either disjoint or the same:
\[ \forall x,y\in A: \qquad[x]_\sim \perp [y]_\sim \quad \lor \quad [x]_\sim = [y]_\sim. \]
\end{enumerate}
Conversely, every partition defines an equivalence relation.
\end{proposition}

\subsubsection{Egg-box diagrams and multiple equivalences}
\begin{lemma}
Let $A$ be a class and $R_1, R_2$ equivalence relations on $A$. Then $R_1 \cap R_2$ is an equivalence relation.
\end{lemma}

\begin{definition}
Let $A$ be a class and $R_1, R_2$ two equivalence relations. We can draw the elements of $A$ in a grid such that
\begin{itemize}
\item two elements are in the same column \textup{if and only if} they are $R_1$-equivalent;
\item two elements are in the same row \textup{if and only if} they are $R_2$-equivalent.
\end{itemize}
The cells in the grid are then $R_1\cap R_2$-equivalence classes.

Such a diagram is called an \udef{egg-box diagram}.
\end{definition}

Usually the term ``egg-box diagram'' specifically refers to the case when $R_1, R_2$ are the Green's relations $\greensL$ and $\greensR$ of a semigroup.

\begin{example}
Consider the set $\powerset(\{1,2,3,4\})$ with equivalence relations
\begin{itemize}
\item $aR_1 b$ if subsets $a$ and $b$ have the number of elements;
\item $aR_2 b$ if subsets $a$ and $b$ have the number of even elements.
\end{itemize}
Then the egg-box diagram of $R_1,R_2$ is
\[ \begin{array}{|c|c|c|c|c|}
\hline
\emptyset & \{1\}, \{3\} & \{1,3\} && \\ \hline
& \{2\}, \{4\} & \{1,2\}, \{1,4\}, & \{1,2,3\}, \{1,3,4\} & \\
&  & \{2,3\}, \{3,4\} &  & \\ \hline
&& \{2,4\} & \{1,2,4\}, \{2,3,4\}  & \{1,2,3,4\}    \\ \hline
\end{array} \]
\end{example}

\begin{proposition} \label{commutingEquivalenceRelations}
Let $A$ be a class and $R_1, R_2$ equivalence relations on $A$. Then the following are equivalent:
\begin{enumerate}
\item $R_1;R_2$ is symmetric;
\item $R_1;R_2 = R_2;R_1$;
\item $R_1;R_2$ is an equivalence relation;
\item the egg-box diagram of $R_1, R_2$ decomposes into blocks, i.e.\
\begin{itemize}
\item a block consists of a rectangular region of cells;
\item each cell in the block is occupied;
\item no cell not included in the block, but in the same row / column as a cell in the block is occupied;
\end{itemize}
these blocks are $R_1;R_2$-equivalence classes;
\item for any four cells that are the corners of a rectangle in the egg-diagram: if any three of these cells are occupied, all four cells are occupied.
\end{enumerate}
In this case $R_1;R_2$ is the smallest equivalence relation containing $R_1\cup R_2$.
\end{proposition}
TODO $R_1; R_2$ is transitive closure of $R_1\cup R_2$.
\begin{proof}
$(1)\Rightarrow (2)$ We have $R_1;R_2 = (R_1;R_2)^\transp = R_2^\transp; R_1^\transp = R_2;R_1$.

$(2)\Rightarrow (3)$ We verify
\begin{itemize}
\item \emph{reflexivity}: from $\id_A \subseteq R_1$ and $\id_A \subseteq R_2$, we get $\id_A = \id_A^2 \subseteq R_1;R_2$;
\item \emph{symmetry}: we have $(R_1;R_2)^\transp = R_2^\transp; R_1^\transp = R_2;R_1 = R_1;R_2$;
\item \emph{transitivity}: we have
\[ (R_1;R_2);(R_1; R_2) = R_1;R_1;R_2;R_1 = R_1;R_2. \]
\end{itemize}

$(3)\Rightarrow (1)$ Immediate.

$(2)\Leftrightarrow (4, 5)$ On reflection, one may understand that (4) and (5) are equivalent.

First assume (2) and take any three occupied cells that lie on the corners of a rectangle. Take $a,b$ from the diagonally opposing cells. Then $a(R_1;R_2)b$, because there exists a $c$ in the third cell. So also $a(R_2;R_1)b$, meaning there exists an element in the fourth sell, so (5) holds.

We can run the argument in reverse for the converse.

(Final observation) We first show that $R_1\cup R_2 \subseteq R_1;R_2$. Because $\id_A \subseteq R_1, R_2$, we have
\[ R_1\cup R_2 \subseteq R_1;(R_1\cup R_2);R_2 = R_1;R_1;R_2 \cup R_1;R_2;R_2 = R_1;R_2. \]
Now assume $R$ is an equivalence class containing $R_1\cup R_2$, then
\[ R \supseteq (R_1\cup R_2);(R_1\cup R_2) = R_1 \cup R_1;R_2 \cup R_2;R_1 \cup R_2 = R_1;R_2. \]
\end{proof}

\subsubsection{Tolerance relations}
\begin{definition}
A \udef{tolerance relation} is a binary relation that is reflexive and symmetric.

If $T$ is a tolerance relation on a class $X$, then $\sSet{X,T}$ is called a \udef{tolerance space}.
\end{definition}
A tolerance relation expresses the idea of `resembling' or `being within tolerance'.

\subsubsection{Quasi-equivalence relations}
\begin{definition}
A \udef{quasi-equivalence relation} is a binary relation $R$ such that
\begin{itemize}
\item $R$ is symmetric;
\item $R$ is transitive.
\end{itemize}
\end{definition}

\begin{lemma} \label{quasiEquivalenceLemma}
Let $R$ be a quasi-equivalence relation on a class $A$ and $x\in A$. If $xR$ is not empty, then $xRx$.
\end{lemma}
\begin{proof}
If $xR$ is not empty, then there exists an $a\in xR$, i.e.\ $xRa$. By symmetry and transitivity, we have $xRaRx$, so $xRx$.
\end{proof}

\begin{definition}
Let $X$ be a set. We call $P\in\powerset^2(X)$ a \udef{quasi-partition} if $\emptyset\notin P$ and any two elements are either disjoint or the same:
\[ \forall A,B\in P: \quad A\perp B \quad\lor\quad A=B. \]
\end{definition}

\begin{proposition}
Let $X$ be a set. The function
\[ \operatorname{QP}: \{\text{quasi-equivalence relations on $X$}\} \to \{\text{quasi-partitions on $X$}\}: R\mapsto \setbuilder{xR}{x\in X} \]
is bijective and its inverse is given by
\[ \operatorname{QP}^{-1}(P) = \setbuilder{(x,y)\in X^2}{\exists A\in P: \{x,y\}\subseteq A}. \]
\end{proposition}
\begin{proof}
We first prove that $\operatorname{QP}$ is well-defined, i.e.\ $\operatorname{QP}(R)$ is a quasipartition on $X$ for each quasi-equivalence relation $R$ on $X$.

Take $xR,yR\in \operatorname{QP}(R)$ and assume that $xR\mesh yR$. We will show that in this case $xR = yR$. By assumption we can find a $z\in xR\cap yR$. Take $a\in xR$, so that we have $aRx, zRx$ and $zRy$. Then by symmetry and transitivity, we have $aRy$, because $aRxRzRy$. This gives us $xR\subseteq yR$. Similarly $yR\subseteq xR$, so $xR=yR$.

Next we show that $\operatorname{QP}^{-1}$ is a left-inverse.
Let $R$ be a quasi-equivalence relation on $X$ and take $(a,b)\in R$. Then $\{a,b\}\in aR$ by \ref{quasiEquivalenceLemma} and thus $(a,b)\in \operatorname{QP}^{-1}(\operatorname{QP}(R))$.
Now take $(a,b)\in \operatorname{QP}^{-1}(\operatorname{QP}(R))$. Then there exists $z\in X$ such that $\{a,b\}\subseteq zR$. Thus $aRzRb$ and so $(a,b)\in R$.

Finally we show that $\operatorname{QP}^{-1}$ is a right-inverse. Take a quasi-partition $P$ on $X$, $A\in P$ and $a\in A$ (by definition $A\neq \emptyset$). Then
\begin{align*}
P \ni A &= a\setbuilder{(a,b)}{b\in A} \\
&= a\setbuilder{(a,b)}{\exists A\in P: b\in A} \\
&= a\setbuilder{(a,b)}{\exists A\in P: \{a,b\}\subseteq A} \\
&= a\operatorname{QP}^{-1}(P) \in \operatorname{QP}(\operatorname{QP}^{-1}(P)).
\end{align*}
This shows that $P = \operatorname{QP}(\operatorname{QP}^{-1}(P))$.
\end{proof}

\subsection{Uniqueness and totality}
\begin{definition}
Let $R$ be a relation on $(A, B)$. We call $R$
\begin{itemize}
\item \udef{left-total}, \udef{serial} or simply \udef{total} if $A = {_RB}$;
\item \udef{right-total} or \udef{surjective} or \udef{onto} if $A_R = B$;
\item \udef{left-unique} or \udef{injective} if
\[ \forall x_1,x_2\in A: \forall y\in B: \; x_1Ry \land x_2Ry \implies x_1=x_2; \]
\item \udef{right-unique} or \udef{functional} if
\[ \forall x\in A: \forall y_1,y_2\in B: \; xRy_1 \land xRy_2 \implies y_1=y_2. \]
\end{itemize}
We may also call a binary relation
\begin{itemize}
\item \textbf{one-to-one} if it is injective and functional;
\item \textbf{one-to-many} if it is injective and not functional;
\item \textbf{many-to-one} if it is not injective and functional;
\item \textbf{many-to-many} if it is not injective and not functional.
\end{itemize}
\end{definition}

\begin{lemma}
Let $R$ be a relation. Then
\begin{enumerate}
\item $R$ is right-unique \textup{if and only if} $R^\transp$ is left-unique;
\item $R$ is right-total \textup{if and only if} $R^\transp$ is left-total.
\end{enumerate}
\end{lemma}

\subsubsection{Totality}
\begin{lemma} \label{totalityEquivalences}
Let $R$ be a relation on $(A, B)$. Then
\begin{enumerate}
\item the following are equivalent to $R$ being left-total:
\begin{enumerate}
\item $U_{A,B} = R;U_{B,B}$
\item $\id_A \subseteq R;R^\transp$;
\item $\overline{R}\subseteq R;\overline{\id}_B$;
\item for all relations $S$: $S;R = E \implies S = E$;
\end{enumerate}
\item the following are equivalent to $R$ being right-total:
\begin{enumerate}
\item $U_{A,B} = U_{A,A};R$
\item $\id_B \subseteq R^\transp;R$;
\item $\overline{R}\subseteq \overline{\id}_A;R$;
\item for all relations $S$: $R;S = E \implies S = E$.
\end{enumerate}
\end{enumerate}
\end{lemma}
\begin{proof}
$(a \Rightarrow b)$ We calculate using the Dedekind rule:
\[ \id = U\cap \id = R;U\cap \id \subseteq (R\cap \id;U^\transp);(U\cap R^\transp;\id) = R;R^\transp. \]
\end{proof}

\subsubsection{Uniqueness}
\begin{lemma} \label{relationTimesTransposeSubsetIdentity}
Let $R$ be a relation. Then
\begin{enumerate}
\item $R$ is functional \textup{if and only if} $R^\transp; R \subseteq \id_{\codom(R)}$;
\item $R$ is injective \textup{if and only if} $R; R^\transp \subseteq \id_{\dom(R)}$.
\end{enumerate}
We also have
\begin{enumerate} \setcounter{enumi}{2}
\item $R$ is functional \textup{if and only if} $R^\transp; R = \id_{\im(R)}$;
\item $R$ is injective \textup{if and only if} $R; R^\transp = \id_{\preim(R)}$.
\end{enumerate}
\end{lemma}
\begin{proof}
(1) Assume that there exist $x,y_1,y_2$ such that $xRy_1$ and $xRy_2$. This means $y_1R^\transp x$ and $xRy_2$, so $y_1(R^\transp; R) y_2$. Then $R$ is functional iff $y_1 = y_2$ iff $R^\transp; R \subseteq \id_{\codom(R)}$.

(2) Similar.

(3, 4) Due to the inclusions in \ref{kernelInclusions}.
\end{proof}
\begin{corollary}
Let $R, S$ be relations. Then
\begin{enumerate}
\item if $R$ and $S$ are left/right-unique, then $R;S$ is left/right-unique;
\item $R$ is right-unique \textup{if and only if} $R \subseteq R^\transp \diagdown \id$;
\item $R$ is left-unique \textup{if and only if} $R\subseteq \id \diagup R^\transp$.
\end{enumerate}
\end{corollary}
\begin{proof}
(1) Assume $R$ and $S$ are right-unique. Then
\[ (R;S)^\transp; (R;S) = S^\transp;R^\transp;R;S \subseteq S^\transp;\id_{\dom(S)};S = S^\transp;S \subseteq \id_{\codom(S)} = \id_{\codom(R;S)}. \]

(2, 3) Applications of Schröder's rule.
\end{proof}

\begin{lemma}
Let $R,S, T$ be composable relations. Then
\begin{enumerate}
\item if $R$ is right-unique, then $\begin{aligned}[t]
R;(S\cap T) &= R;S \cap R;T\\
S\cap T;R &= (S;R^\transp \cap T);R
\end{aligned}$
\item if $R$ is left-unique, then $\begin{aligned}[t]
(S\cap T);R &= S;R \cap T;R \\
R;S\cap T &= R;(S\cap R^\transp;T) .
\end{aligned}$
\end{enumerate}
\end{lemma}
\begin{proof}
(1a) The inclusion $\subseteq$ is in \ref{relationalComposition}. For the other inclusion we can use the Dedekind formula \ref{DedekindFormula}:
\begin{align*}
R;S \cap R;T &\subseteq (R\cap (R;T;S^\transp));(S\cap R^\transp;R;T) \\
&\subseteq R;(S\cap R^\transp;R;T) \\
&\subseteq R;(S\cap T),
\end{align*}
where we have used the right-uniqueness for the last inclusion: $R^\transp;R;T \subseteq T$.

(1b) We use the Dedekind formula twice:
\begin{align*}
S\cap T;R &\subseteq (T\cap S;R^\transp);(R\cap T^\transp;S) \subseteq (T\cap S;R^\transp);R \\
&\subseteq (S\cap T;R);(R^\transp\cap S^\transp;T);R \subseteq (S\cap T;R);R^\transp;R \\
&\subseteq S\cap T;R,
\end{align*}
where we have used the right-uniqueness for the last inclusion: $(S\cap T;R);R^\transp;R \subseteq (S\cap T;R)$.

(2a) The calculation is similar to (1a):
\begin{align*}
S;R \cap T;R &\subseteq (S\cap (T;R;R^\transp));(R\cap S^\transp;T;R) \\
&\subseteq (S\cap (T;R;R^\transp));R \\
&\subseteq (S\cap T);R.
\end{align*}

(2b) The calculation is similar to (1b):
\begin{align*}
R;S\cap T &\subseteq (R\cap T;S^\transp);(S\cap R^\transp;T) \subseteq R;(S\cap R^\transp;T) \\
&\subseteq R;(R^\transp\cap S;T^\transp);(T\cap R;S) \subseteq R;R^\transp;(T\cap R;S) \\
&\subseteq T\cap R;S.
\end{align*}
\end{proof}

\begin{lemma} \label{uniquenessResiduals}
Let $R,S$ be composable relations.
\begin{enumerate}
\item If $R$ is right-unique, then $\begin{aligned}[t]
\overline{R;S} &= R;\overline{S} \cup \overline{R;U} \\
R;\overline{S} &= R;U\cap \overline{R;S} \\
\overline{R;\overline{S}} &= \overline{R;U} \cup R;S.
\end{aligned}$
\item If $S$ is left-unique, then $\begin{aligned}[t]
\overline{R;S} &= \overline{R};S \cup \overline{U;S} \\
\overline{R};S &= U;S\cap \overline{R;S} \\
\overline{\overline{R};S} &= \overline{U;S}\cup R;S.
\end{aligned}$
\end{enumerate}
\end{lemma}

\begin{lemma} \label{imagePreimageUniqueness}
Let $R$ be a relation on $(A, B)$, $X,Y\subset A$ and $Z,W\subset B$. If $R$ is injective, then
\begin{enumerate}
\item $(X\cup Y)_R = X_R\cup Y_R$;
\item $(X\cap Y)_R = X_R\cap Y_R$;
\item $(X\setminus Y)_R = X_R\setminus Y_R$;
\item $(X\symdiff Y)_R = X_R\symdiff Y_R$.
\end{enumerate}
If $R$ is functional, then
\begin{enumerate} \setcounter{enumi}{4}
\item $_R(Z\cup W) = {_RZ}\cup {_RW}$;
\item $_R(Z\cap W) = {_RZ}\cap {_RW}$;
\item $_R(Z\setminus W) = {_RZ}\setminus {_RW}$;
\item $_R(Z\symdiff W) = {_RZ}\symdiff {_RW}$.
\end{enumerate}
If $R$ is injective and surjective, then
\begin{enumerate} \setcounter{enumi}{8}
\item $(X^c)_R = (X_R)^c$.
\end{enumerate}
If $R$ is total and functional, then
\begin{enumerate} \setcounter{enumi}{8}
\item ${_R}(X^c) = ({_RX})^c$.
\end{enumerate}
\end{lemma}
TODO: similar for totality and polars?

\subsubsection{Kernels}

\begin{definition}
Let $R$ be a relation on $(A, B)$.
\begin{itemize}
\item The \udef{kernel} of $R$ is $\ker R \defeq \setbuilder{(x,y)\in A\times A}{xR \mesh yR}$.
\item The \udef{cokernel} of $R$ is $\coker R \defeq \setbuilder{(x,y)\in B\times B}{Rx \mesh Ry}$.
\end{itemize}
\end{definition}

\begin{lemma}
Let $R$ be a relation on $(A, B)$. Then
\begin{enumerate}
\item $\ker R$ is a symmetric relation on $A$;
\item if $R$ is left total, then $\ker$ is a tolerance relation on $A$;
\item $\ker R = R;R^\transp = R^{\transp}\circ R$.
\end{enumerate}
\end{lemma}


\subsection{Constant relations and points}
\begin{definition}
Let $A,B$ be classes and $R$ a relation on $A,B$. We call $R$
\begin{itemize}
\item \udef{right-constant} if $\forall x\in A, \forall y,z \in B: \; xRy \iff xRz$;
\item \udef{left-constant} if $\forall x, y\in A, \forall z \in B: \; xRz \iff yRz$;
\item a \udef{point} if $R = \{x\}\times B$ for some $x\in A$.
\end{itemize}
Let $x\in A$. We define $\point{x} \defeq \{x\}\times A$.
\end{definition}

\begin{lemma}
Let $R$ be a relation. Then $R$ is right-constant \textup{if and only if} $R^\transp$ is left-constant.
\end{lemma}

\begin{lemma}
Let $R$ be a relation on $(A, B)$. Then
\begin{enumerate}
\item $R$ is right-constant \textup{if and only if} $R = R;U_{B,B}$;
\item $R$ is left-constant \textup{if and only if} $R = U_{A,A};R$;
\item $R$ is a point \textup{if and only if} $R$ is right-constant, non-zero and injective.
\end{enumerate}
\end{lemma}

\begin{lemma}
Let $R$ be a relation on $(A,B)$, $x\in A$ and $y\in B$. Then
\begin{enumerate}
\item $R;\point{y}$ is a point;
\item $\point{x}^\transp; R$ is the converse of a point;
\item $Ry = \preim(R;\point{y})$;
\item $xR = \im(\point{x}^\transp; R)$.
\end{enumerate}
\end{lemma}

\section{Functions}
TODO: equiv for equality!

\begin{definition}
A \udef{function} is a binary relation that is functional and serial.
\end{definition}
If a relation $f$ on $(A, B)$ is a function, then for every $x\in A$, there is a unique $y\in B$ such that $xfy$. We write $f(x)$ to denote this unique element.

Conceptually we can rephrase this as saying that for each ``input'' in the domain $A$, the function $f$ produces a unique ``output'' in the codomain $B$.

\begin{note}
We write
\[f:A \to B: x\mapsto f(x) \]
to show $f$ is a function with domain $A$ and codomain $B$ that maps $x\in A$ to $f(x)\in B$. We say $f$ is a function from $A$ to $B$ or $f$ maps $A$ to $B$. 

If $A, B$ are sets, we can consider the  set of all functions from $A$ to $B$. This set is denoted $(A\to B)$.
\end{note}

Depending on the context, functions may also be called \udef{maps} or \udef{transformations}. These are synonyms with slightly different connotations.

\begin{lemma}
Let $f_1: A_1\to B_1$ and $f_2: A_2\to B_2$ be functions. Then $f_1$ and $f_2$ are the same functions \textup{if and only if}
\begin{enumerate}
\item $A_1 = A_2$ and $B_1 = B_2$;
\item $\forall x\in A_1: f_1(x) = f_2(x)$.
\end{enumerate}
\end{lemma}

\begin{lemma}
Let $f$ be a relation. Then $f$ is a function \textup{if and only if} $f;\overline{\id} = \overline{R}$.
\end{lemma}

\begin{lemma} \label{functionEqualityIdComparison}
Let $f,g: A\to B$ be functions. Then $f = g$ \textup{if and only if} $\id_A \subseteq f;g^\transp$.
\end{lemma}

\subsection{Image and preimage}
\subsubsection{Functions associated to a relation}
We can associate to any relation $R$ on $(A,B)$
\begin{itemize}
\item an \udef{image function} $R^{\imf}: \powerset(A) \to \powerset(B): X \mapsto X_R$;
\item a \udef{preimage function} $R^{\preimf}: \powerset(B) \to \powerset(A): X \mapsto {_RX}$;
\item a \udef{right bounds function} $\powerset(A) \to \powerset(B): X \mapsto X^R$;
\item a \udef{left bounds function} $\powerset(B) \to \powerset(A): X \mapsto {^RX}$.
\end{itemize}

Clearly $R^{\preimf} = (R^\transp)^\imf$.

\begin{lemma}
Let $R,S$ be composable relations. Then $(R;S)^\imf = R^\imf;S^\imf$.
\end{lemma}
\begin{proof}
Take some arbitrary $A\subseteq \dom(R)$. Then for all $y\in \codom(S)$ we have
\begin{align*}
y \in (R;S)^\imf(A) &\iff \exists x\in A:\; x(R;S)y \\
&\iff \exists x\in A:\exists z\in \dom(S) = \codom(R): \; xRz\;\wedge\; zSy \\
&\iff \exists z\in \dom(S) = \codom(R):\exists x\in A: \; xRz\;\wedge\; zSy \\
&\iff \exists z\in R^\imf(A): zSy \\
&\iff y\in S^\imf(R^\imf(A)) = (R^\imf;S^\imf)(A).
\end{align*}
\end{proof}
\begin{corollary}
For any relation $R$, $\ker(R)^\imf = R^\imf;R^\preimf = R^\preimf\circ R^\imf$.
\end{corollary}

\subsubsection{Applying functions inside sets}
As functions are relations they have images, preimages and associated image and preimage functions.

\begin{lemma}
Let $A,B$ be classes and $f:A\to B$ a function. Then
\begin{enumerate}
\item $f^{\imf}: \powerset(A) \to \powerset(B): X\mapsto \setbuilder{f(x)\in B}{x\in X}$;
\item $f^{\preimf}: \powerset(B) \to \powerset(A): Y\mapsto \setbuilder{x\in A}{f(x) \in Y}$.
\end{enumerate}
\end{lemma}

Image and preimage functions are also functions and thus also have images, preimages and associated image and preimage functions.

We have, for a function $f:A\to B$,
\begin{itemize}
\item $f^{\imf\imf}\defeq (f^{\imf})^\imf: \powerset^2(A) \to \powerset^2{B}$;
\item $f^{\preimf\preimf}\defeq (f^{\preimf})^\preimf: \powerset^2(A) \to \powerset^2{B}$;
\item $f^{\imf\preimf}\defeq (f^{\imf})^\preimf: \powerset^2(B) \to \powerset^2{A}$;
\item $f^{\preimf\imf}\defeq (f^{\preimf})^\imf: \powerset^2(B) \to \powerset^2{A}$;
\end{itemize}

Note that in general $f^{\imf\preimf} \neq f^{\preimf\imf}$ and $f^{\preimf\preimf} \neq f^{\imf\imf}$.

\begin{example}
Let $A = \{a,b\}$ and $B = \{c\}$. Consider the unique (constant) function in $(A\to B)$. Then
\begin{itemize}
\item $\begin{aligned}[t]
f^{\preimf\imf}(\{\{c\}\}) &= \setbuilder{f^\preimf(X)}{X\in \{\{c\}\}} \\
&= \{f^\preimf(\{c\})\} \\
&= \{\{a,b\}\};
\end{aligned}$
\item $\begin{aligned}[t]
f^{\imf\preimf}(\{\{c\}\}) &= \setbuilder{X}{f^\imf(X)\in \{\{c\}\}} \\
&= \setbuilder{X}{f^\imf(X) = \{c\}} \\
&= \{ \{a\}, \{b\}, \{a,b\} \}.
\end{aligned}$
\end{itemize}
\end{example}

\begin{lemma}
Let $A,B$ be classes, $f:A\to B$ a function, $x\in A$ and $Y\subseteq B$. Then $x\in f^\preimf(Y) \iff f(x)\in Y$.
\end{lemma}
TODO Galois connection.

\begin{proposition}
Let $A,B$ be classes and $f:A\to B$ a function. Then
\begin{enumerate}
\item $(f^\imf,f^\preimf)$ is a Galois connection between $\sSet{\powerset(A), \subseteq}$ and $\sSet{\powerset(B), \subseteq}$.
\end{enumerate}
Also $(f^\imf\circ f^{\preimf})(Y) = Y\cap \im(f)$.
\end{proposition}
\begin{proof}
We have
\[ f^\imf(X) \subseteq Y \iff \forall x\in X: f(x) \in Y \iff X \subseteq f^{\preimf}(Y). \]

Take $Y\in \powerset(B)$. Then
\begin{align*}
y\in (f^\imf\circ f^{\preimf})(Y) &\iff \exists x\in f^{\preimf}:\; f(x) = y \\
&\iff \exists x\in A: \;f(x)\in Y \land f(x) = y \\
&\iff \exists x\in A: \;y \in Y \land f(x) = y \\
&\iff y\in Y\cap \im(f).
\end{align*}
\end{proof}





\begin{lemma}
Let $f$ be a function. Then $f^{\preimf}\big(\biguplus \mathcal{E}\big) = \biguplus f^\preimf(\mathcal{E})$.
\end{lemma}

\begin{lemma} \label{functionUpperbound}
Let $A,B,C$ be classes, $f: A\to B$ a function and $R$ a relation on $(B,C)$. Let $X \subseteq A$. Then $f^\imf(X)^R = X^{f;R}$.
\end{lemma}
\begin{proof}
We calculate
\[ (X_f)^{R} = (X_{f;\overline{R}})^c = X^{\overline{f;\overline{R}}} = X^{f;R \cup \overline{f;U_{B,C}}} = X^{f;R}, \]
where we have used the uniqueness and totality of $f$ in \ref{uniquenessResiduals} and \ref{totalityEquivalences}.

Alternatively we can give the following calculation:
\[ y\in f^\imf(X)^R \iff \forall x\in X: f(x)Ry \iff \forall x\in X: x(f;R)y \iff y\in X^{f;R}. \]
\end{proof}

\subsection{Injectivity, surjectivity and bijectivity}
\begin{definition}
The terms injective and surjective are commonly applied to functions. A function that is both injective and surjective is \udef{bijective}.
\end{definition}
In the context of functions the notions of injectivity and one-to-one coincide.
\begin{lemma}
Let $f:A\to B$ be a function. We say
\begin{itemize}
\item $f$ is injective, denoted $f: A\rightarrowtail B$, if
\[ \forall x_1,x_2\in A: f(x_1) = f(x_2) \implies x_1 = x_2; \]
\item $f$ is surjective, denoted $f: A\twoheadrightarrow B$, if
\[ \forall y\in B: \exists x\in A: f(x) = y; \]
\item $f$ is bijective, denoted $f: A\twoheadrightarrowtail B$, if
\[ \forall y\in B: \exists! x\in A: f(x) = y; \]
\end{itemize}
We also say $f$ is an injection, a surjection or a bijection if it is injective, surjective or bijective, respectively.

We will also sometimes write $A \leftrightarrow B$, instead of $A\twoheadrightarrowtail B$, to denote a bijection between $A$ and $B$.
\end{lemma}

\begin{lemma}
If $A\subseteq B$ classes, then there exists a canonical injection $\iota: A\to B$, the \udef{inclusion map}
\[ \iota: A\to B: a\mapsto a. \]
The inclusion map is often denoted $A\hookrightarrow B$.
\end{lemma}

\begin{lemma}
Let $f:A\to B$ be a function. Then $\ker f$ is an equivalence relation.
\end{lemma}

\begin{definition}
Let $f: A\to B$ be a function between sets. An equivalence class $[x]_{\ker f}$ is called a \udef{fibre} of $f$.
\end{definition}

\begin{proposition}
Let $f:A\to B$ be a function between sets. We can associate to $f$
\begin{enumerate}
\item a surjective function $f': A\to f[A]: x\mapsto f(x)$;
\item an injective function $f'': A/(\ker f) \;\to B: [x]_{\ker f}\mapsto f(x)$;
\item a bijective function $f''': A/(\ker f) \;\to f[A]: [x]_{\ker f}\mapsto f(x)$.
\end{enumerate}
\end{proposition}
Whenever a function on a quotient set defines an image of an equivalence class using an element of said equivalence class, we need to verify this definition is \emph{well-defined}, i.e.\ it does not depend on the chosen element in the equivalence class. (In other words it is properly a function of the equivalence class, not of the elements of the equivalence classes.)
\begin{proof}
We show $f''$ is well-defined. Let $[x]_{\ker f}\in A/(\ker f)$ and let $x_1,x_2\in [x]_{\ker f}$. Then $f(x) = f(x_1) = f(x_2)$ and
\[ f''([x_1]_{\ker f}) = f(x_1) = f(x_2) = f''([x_2]_{\ker f}), \]
so $f''$ is well-defined.
\end{proof}



\subsection{Constructing new functions}
Constructions defined for relations are in particular applicable to functions.

\subsubsection{Restriction and extension}
\begin{lemma}
Let $f: A\to B$ be a function and $S\subset A$. Then $f|_S$ is a function.
\end{lemma}
When talking about the restriction of a function, this left restriction is always the one that is meant. Right restrictions $f|^K$ are sometimes called \udef{corestrictions} and are in general not functions.

\begin{definition}
Let $A\subset B$ and $C$ be classes. Let $f: A\to C$ be a function. A function $\tilde{f}: B\to C$ such that $\tilde{f}|_A = f$ is called an \udef{extension} of $f$.
\end{definition}
Given a function, a different function with as codomain a superset of the original codomain, which is otherwise identical, is sometimes called a \udef{coextension} of the function.

When given a function prescription, we may need to verify the function is \emph{well-defined} in the sense that the prescription always gives an element in the codomain for every element in the domain.

\subsubsection{Composition}
\begin{lemma}
The functions $f:A\to B$ and $g:B\to C$ are composable as relations and the relation
\[ f;g = g\circ f \]
is a function. In particular it has functional form
\[ g\circ f: A\to C: x\mapsto g(f(x)). \]
\end{lemma}
Because of the simplicity of $(g\circ f)(x) = g(f(x))$, the notation $\circ$ is almost exclusively used when dealing with functions.

\begin{definition}
Let $f, g:A\to A$ be functions. We say $f$ and $g$ \udef{commute} if $f\circ g = g\circ f$.
\end{definition}

\begin{note}
Let $X,Y,Z$ be classes. Using composition we can view any function $f: X\to Y$ also as a function
\[ f: (Z\to X)\to (Z\to Y): g\mapsto f(g) = (z\mapsto (f\circ g)(z)). \]
\end{note}

\subsubsection{Pre- and post-composition}
\begin{definition}
Let $A,B,C$ be sets and $f\in (A\to B)$. Then we define
\begin{itemize}
\item the \udef{post-composition function} of $f$ as $f_*: (C\to A) \to (C\to B): g\mapsto f\circ g$.
\item the \udef{pre-composition function} of $f$ as $f^*: (B\to C) \to (A\to C): g\mapsto g\circ f$.
\end{itemize}
The post-composition function is also known as the \udef{pointwise} application or extension of $f$.
\end{definition}

\begin{lemma} \label{covarianceContravarianceComposition}
Let $A,B,C, D$ be sets, $f\in (A\to B)$ and $g\in(B\to C)$. Then
\begin{enumerate}
\item $(g\circ f)_* = g_*\circ f_*$;
\item $(g\circ f)^* = f^*\circ g^*$.
\end{enumerate}
\end{lemma}
\begin{proof}
(1) Let $h\in (D\to A)$. Then
\[(g\circ f)_*(h) = (g\circ f)\circ h = g\circ(f\circ h) = g_*(f_*(h)) = (g_*\circ f_*)(h). \]

(2) Let $h\in (C\to D)$. Then, similarly, 
\[ (g\circ f)^*(h) = h\circ (g\circ f) = (h\circ g)\circ f = (g^*(h))\circ f = f^*(g^*(h)) = (f^*\circ g^*)(h). \]
\end{proof}

\subsubsection{Inverses of functions}
All identity relations are functions.

\begin{definition}
Let $f:X\to Y$ be a function.
\begin{itemize}
\item A \udef{left inverse} (or \udef{retraction}) of $f$ is a function $g: Y\to X$ such that
\[ g\circ f = \id_X. \]
\item A \udef{right inverse} (or \udef{section}) of $f$ is a function $h: Y\to X$ such that
\[ f\circ h = \id_Y. \]
\item A \udef{(two-sided) inverse} of $f$ is a function that is both a left and a right inverse.
\end{itemize}
\end{definition}

\begin{lemma} \label{injectiveInverse}
Let $f:X\to Y$ be a function. The following are equivalent:
\begin{enumerate}
\item $f$ has a left inverse $g$;
\item $f^\transp$ is a partial function from $X$ to $Y$; 
\item $\graph(f^\transp) \subseteq \graph(g)$;
\item $f$ is injective.
\end{enumerate}
\end{lemma}
There is also a result linking the existence of a right inverse and surjectivity, but this in general only holds assuming the axiom of choice. (TODO ref)

\begin{lemma} \label{leftRightInverse}
If $f$ has a left inverse $g$ and a right inverse $h$, then $g=h$.
\end{lemma}
\begin{proof}
By the simple calculation
\[ g = g\circ (f\circ h) = (g\circ f) \circ h = h. \]
\end{proof}
Thus the two-sided inverse of $f$ is unique and exists \textup{if and only if} $f$ has a left and a right inverse. The unique two-sided inverse is denoted $f^{-1}$.

\begin{lemma}
Let $f:X\to Y$ be a function. The following are equivalent:
\begin{enumerate} 
\item the inverse exists;
\item the inverse exists and $f^{-1} = f^\transp$;
\item $f^\transp$ is a function from $X$ to $Y$;
\item $f$ is a bijection;
\item $f^\transp$ is a bijective function from $X$ to $Y$.
\end{enumerate}
\end{lemma}

\begin{lemma} \label{commutationInverse}
Let $f, g:A\to A$ be commuting and bijective functions. Then $f^{-1}$ and $g^{-1}$ commute.
\end{lemma}
\begin{proof}
We calculate
\[ f^{-1}\circ g^{-1} = f^{-1}\circ g^{-1}\circ (f\circ g \circ g^{-1} \circ f^{-1}) = f^{-1}\circ g^{-1}\circ (g\circ f) \circ g^{-1} \circ f^{-1} = g^{-1} \circ f^{-1} \]
\end{proof}

\subsubsection{Constant functions}
\begin{definition}
Let $X,Y$ be classes. A function $f:X\to Y$ is called \udef{constant} if there exists a $y_0$ such that $\forall x\in X: f(x) = y_0$.

We denote this function $\constant{y_0}$.
\end{definition}

A function is constant if and only if its range is a singleton.

\begin{lemma}
Let $X,Y$ be classes and $y\in Y$. Then
\[ \graph(\constant{y}) = X\times \{y\}. \]
\end{lemma}

\subsection{Partial functions}
\begin{definition}
A \udef{partial function} is a relation that is functional (but not necessarily serial). If we wish to emphasise a function is both functional and serial, we may call it a \udef{total function}.
\end{definition}
If $f\subset A\times B$ is a partial function, we write $A\not \to B$. The set of all partial functions from $A$ to $B$ is
\[ (A\not \to B) = \bigcup _{S\subseteq A}(S\to B). \]

For all $a\in A$ we write
\[ f(a)\downarrow \defequiv a\in\dom(f),\qquad f(a)\uparrow \defequiv a\notin\dom(f) \]
We can read $f(a)\downarrow$ as ``$f$ converges at $a$'' and $f(a)\uparrow$ as ``$f$ diverges at $a$''.

\begin{lemma} \label{partialFunctionExtension}
Let $f: A \not\to B$ be a partial function. Then we can construct a total function
\[ \widehat{f}: A \to B^+ \]
where $B^+ = B\cup \{e\}$ for some $e\notin B$, such that $\widehat{f}|_{\preim(f)}^B = f|_{\preim(f)}$.
\end{lemma}
\begin{proof}
We can always find an element $e$ by \ref{russelParadox}. Then we let $\widehat{f}$ map all elements in $A\setminus\preim(f)$ to $e$.
\end{proof}

\begin{proposition} \label{partialFunctionSubset}
Let $A,B$ be sets and $f,g\in(A\not\to B)$. Then the following are equivalent:
\begin{enumerate}
\item $f\subseteq g$;
\item $\ker(f) \subseteq \ker(g)$ and $\im(f)\subseteq \im(f\cap g)$;
\item $\ker(f\cup g) \subseteq \ker(g)$ and $\im(f)\subseteq \im(g)$.
\end{enumerate}
TODO: give proper meaning to $\ker(f\cup g)$
\end{proposition}
\begin{proof}
(1) clearly implies (2) and (3).

$(2) \Rightarrow (1)$. Take $(x,f(x)) \in f$. Now $f(x)\in \im(f)$, so $f(x)\in\im(f\cap g)$, meaning there exists a $u\in A$ such that $(u,f(x))\in f\cap g$. Because $(u,x)\in \ker(f)\subseteq \ker(g)$, we have $(x,f(x))\in g$.

$(3) \Rightarrow (1)$. Take $(x,f(x)) \in f$. Now $f(x)\in \im(f)$, so $f(x)\in\im(g)$ and we can find a $u\in A$ such that $(u,f(x))\in g$. Then $(x,f(x)) \in f\cup g$ and $(u,f(x))\in f\cup g$, so $(x,u)\in \ker(f\cup g)$ and thus $(x,u)\in\ker(g)$. This means $(x,f(x))\in g$.
\end{proof}

\subsubsection{Singleton unpacking}
\begin{definition}
Let $A$ be a class. We define the partial function
\[ \sunp_A: \powerset{A} \not\to A: X \mapsto \begin{cases}
x & \text{if $X = \{x\}$} \\
\text{undefined} & \text{otherwise}.
\end{cases} \]
We may also write simply $\sunp$ if the class is clear from context.
\end{definition}

\subsection{Functional dependence}
\begin{definition}
Let $A,B$ be classes and $\mathcal{F} \subseteq (A\to B)$ a subclass.
\begin{itemize}
\item The relation $[\mathcal{F}]$ on $(A,B)$ is defined by
\[ \forall a\in A, b\in B: \qquad a[\mathcal{F}]b \iff \exists f\in \mathcal{F}: \; a = f(b). \]
We say $a$ \udef{depends on} $b$ under $\mathcal{F}$.
\item The relation $[\mathcal{F}]^!$ on $(A,B)$ is defined by
\[ \forall a\in A, b\in B: \qquad a[\mathcal{F}]b \iff \exists! f\in \mathcal{F}: \; a = f(b). \]
We say $a$ \udef{uniquely depends on} $b$ under $\mathcal{F}$.
\end{itemize}
\end{definition}

Note that $[\mathcal{F}] = \bigcup\mathcal{F}$.

\section{Binary functions}
We can easily give a function multiple inputs from multiple domains by first joining them into an $n$-tuple, i.e.\ considering the Cartesian product of the domains as the domain of the function.

\begin{example}
A function $f$ that takes an input in $A$ and one in $B$ to generate an output in $C$ can be written as
\[ f: A\times B \to C: (a,b)\mapsto f(a,b). \]
\end{example}

We can consider the class of binary functions between sets.
\begin{lemma}
There does not exist a class of binary functions between classes.
\end{lemma}
\begin{proof}
There exists a proper class $\mathcal{U}$ by \ref{properClassExistence}. Then
\[ \mathcal{U}\times \{\emptyset\}\to \mathcal{U}: (x,\emptyset) \mapsto x \]
is a proper class by replacement (TODO ref! + necesary?) and thus an element. However, it would have to be an element of a class of binary functions between classes. This is a contradiction.
\end{proof}

\subsection{Input permutation}

Let $A,B$ be classes. Then we can define a swap function as follows:
\[ \swap_{A,B}: A\times B \to B\times A: (a,b) \mapsto (b,a). \]
We may write $\swap$ instead of $\swap_{A,B}$ is the classes are clear from the context.

\begin{lemma}
Let $A, B$ be classes. Then $\swap_{B,A}\circ \swap_{A,B} = \id_{A\times B}$.
\end{lemma}

\begin{definition}
Let $A, B, C$ be classes and $f: A\times B \to C$ a binary function. We define the \udef{dual} function $f^d: B\times A \to C$ of $f$ as $f^d \defeq f\circ \swap$. 
\end{definition}

If we restrict ourselves to the class of binary functions between sets, then $\swap$ can be considered a function.

\begin{lemma}
Let $A, B, C$ be classes and $f: A\times B \to C$ a binary function. Then $(f^d)^d = f$.
\end{lemma}

\subsection{Currying and evaluation}
\subsubsection{Currying}
TODO cfr residuation

\begin{definition}
Let $A,B,C$ be \emph{sets}.

Given a function $f: A\times B \to C$, we can \udef{curry} it in the first argument to obtain a new function
\[ \curry_1(f): A \to (B\to C): a\mapsto f(a,-) \qquad \text{where $f(a,-): B \to C: b\mapsto f(a,b)$} \]
or in the second argument to obtain
\[ \curry_2(f): B \to (A\to C): b\mapsto f(-,b) \qquad \text{where $f(-,b): A \to C: a\mapsto f(a,b)$}. \]
\end{definition}
\begin{lemma}
For given sets $A,B,C$ the act of currying defines two bijective functions
\begin{align*}
&\curry_1: (A\times B \to C) \twoheadrightarrowtail (A \to (B\to C)); \\
&\curry_2: (A\times B \to C) \twoheadrightarrowtail (B \to (A\to C))
\end{align*}
\end{lemma}

\subsubsection{The evaluation map}
\begin{definition}
Let $A,B$ be sets. We define the \udef{evaluation map}
\[ \evalMap: (A\to B)\times A \to B: (f,x) \mapsto f(x). \]
Often we will consider partial applications of the evaluation map in the second argument, i.e.\ for $x\in A$
\[ \evalMap_x: (A\to B) \to B: f\mapsto f(x), \]
which is also called an evaluation map.
\end{definition}
We have $\evalMap_x = \curry_2(\evalMap)(x)$.

\begin{lemma}
Let $A,B,C$ be sets and $f: A\to (B\to C)$. Then
\begin{enumerate}
\item $\curry_1^{-1}(f) = \evalMap\circ (f, \id_B)$;
\item $\curry_2^{-1}(f) = \evalMap\circ (f, \id_B)\circ \transp$.
\end{enumerate}
\end{lemma}

\subsubsection{Functoriality}
\begin{proposition}
Let $f: A\to B$ be a function between sets and $X$ a set. Then there exists a unique $f^X: (X\to A)\to (X\to B)$ such that
\[ \curry_1(f\circ g) = f^X\circ \curry_1(g) \qquad \forall g\in \big(Y\times X \to A\big). \]
This is given by $f^X = f\circ -$.
\end{proposition}
\begin{proof}
Let $\evalMap_1$ be the evaluation map $\evalMap_1: (X\to A)\times X \to A$ and $\evalMap_2$ the evaluation map $\evalMap_2: (A\to B) \times A \to B$.
then we can calculate
\begin{align*}
f \circ g &= f \circ \curry_1^{-1}\big(\curry_1(g)\big) \\
&= f \circ \evalMap_1 \circ \big(\curry_1(g), \id_X\big) \\
&= \curry_1^{-1}\big(\curry_1(f \circ \evalMap_1)\big) \circ \big(\curry_1(g), \id_X\big) \\
&= \evalMap_2\circ \big(\curry_1(f \circ \evalMap_1), \id_X\big) \circ \big(\curry_1(g), \id_X\big) \\
&= \evalMap_2\circ \big(\curry_1(f \circ \evalMap_1) \circ \curry_1(g), \id_X\big) \\
&= \curry_1^{-1}\big(\curry_1(f \circ \evalMap_1) \circ \curry_1(g)\big).
\end{align*}
Setting $f^X \defeq \curry_1(f \circ \evalMap_1) = f\circ -$, we clearly have existence. For uniqueness, note that $f\circ h = g \circ h$ for arbitrary $h$ implies $f=g$. And we can choose $\curry_1(g)$ arbitrarily.
\end{proof}
\begin{corollary}
Let $f: A\to B$ and $g: B\to C$ be functions between sets and $X$ some other set. Then $(g\circ f)^X = g^X\circ f^X$.
\end{corollary}
\begin{proof}
Take arbitrary $h: Y\times X \to A$. Then
\[ (g\circ f)^X\circ \curry_1(h) = \curry_1(g\circ f\circ h) = g^X\circ \curry_1(f\circ h) = g^X \circ f^X \circ \curry_1(h). \]
\end{proof}

\begin{proposition}
Let $f: A\to B$ be a function between sets and $X$ a set. Then there exists a unique $X^f: (B\to X) \to (A\to X)$ such that
\[ \curry_1\big(g\circ (\id_Y, f)\big) = X^f\circ \curry_1(g) \qquad \forall g\in \big(Y\times B \to X\big). \]
This is given by $X^f = -\circ f$.
\end{proposition}
\begin{proof}
Let $\evalMap_1$ be the evaluation map $\evalMap_1: (B\to X)\times B \to X$ and $\evalMap_2$ the evaluation map $\evalMap_2: (A\to X) \times A \to X$.
then we can calculate
\begin{align*}
g\circ (\id_Y, f) &= \curry_1^{-1}\big(\curry_1(g)\big)\circ (\id_Y, f) \\
&= \evalMap_1 \circ \big(\curry_1(g), \id_B\big)\circ (\id_Y, f) \\
&= \evalMap_1 \circ (\id_{B\to X}, f) \circ \big(\curry_1(g), \id_A\big) \\
&= \curry_1^{-1}\Big(\curry_1\big(\evalMap_1 \circ (\id_{B\to X}, f)\big)\Big) \circ \big(\curry_1(g), \id_A\big) \\
&= \evalMap_2\circ\Big(\curry_1\big(\evalMap_1 \circ (\id_{B\to X}, f)\big), \id_A\Big) \circ \big(\curry_1(g), \id_A\big) \\
&= \evalMap_2\circ\Big(\curry_1\big(\evalMap_1 \circ (\id_{B\to X}, f)\big)\circ \curry_1(g), \id_A\Big) \\
&= \curry_1^{-1}\Big(\curry_1\big(\evalMap_1 \circ (\id_{B\to X}, f)\big)\circ \curry_1(g) \Big).
\end{align*}
Setting $X^f \defeq \curry_1(\evalMap_1\circ (\id_{B\to X}, f)) = -\circ f$, we clearly have existence. For uniqueness, note that $f\circ h = g \circ h$ for arbitrary $h$ implies $f=g$. And we can choose $\curry_1(g)$ arbitrarily.
\end{proof}
\begin{corollary}
Let $f: A\to B$ and $g: B\to C$ be functions between sets and $X$ some other set. Then $X^{g\circ f} = X^f\circ X^g$.
\end{corollary}
\begin{proof}
Take arbitrary $h: Y\times B \to X$. Then
\begin{align*}
X^{g\circ f}\circ \curry_1(h) &= \curry_1\big(h \circ (\id_Y, g\circ f)\big) \\
&= \curry_1\big(h \circ (\id_Y, g)\circ (\id_Y, f)\big) \\
&= X^f\circ \curry_1\big(h \circ (\id_Y, g)\big) \\
&= X^f\circ X^g \circ \curry_1(h).
\end{align*}
\end{proof}

\subsubsection{Partial application}
\begin{definition}
Let $f: A\times B \to C$ be a function, $a\in A$ and $b\in B$. We write
\begin{itemize}
\item $f(a, -)$ to mean $\curry_1(f)(a)$; and
\item $f(-, b)$ to mean $\curry_2(f)(b)$.
\end{itemize}
Any function of the form $f(a, -)$ or $f(-,b)$ is called a \udef{partial application} of $f$.
\end{definition}

\subsection{Homogenous binary operators}
\begin{definition}
Let $A$ be a class and $f: A\times A \to A$ a binary function. We call $f$
\begin{itemize}
\item \udef{associative} if $\forall x,y, z\in A: \; f(f(x,y),z) = f(x,f(y,z))$;
\item \udef{commutative} if $\forall x,y\in A: \; f(x,y) = f(y,x)$;
\item \udef{idempotent} if $\forall x\in A:\; f(x,x) = x$.
\end{itemize}
If something is equal to an undefined quantity, we require it to be undefined.
\end{definition}

\subsubsection{Duality}
Let $A$ be a class and $f: A\times A \to A$ a homogeneous binary function. Then $f^d: A\times A \to A$ is also a homogeneous binary function.

Often a property of $f$ can be translated to a different property of $f^d$. In this case we say the properties are \udef{dual} to each other.

\begin{proposition}
If a certain logical dependence holds between certain properties of binary homogenous functions, for all binary homogenous functions, then the same logical dependence holds between the duals of these properties.
\end{proposition}
\begin{proof}
If the logical dependence holds for all functions, it in particular holds for all functions of the form $f^d$.
\end{proof}

\subsubsection{Notation for binary operators}
Prefix, infix, postfix, Polish, necessity of brackets.

\subsubsection{Identity and absorbing elements}
\begin{definition}
Let $A$ be a class and $f: A\times A \to A$ a binary function. 
We say
\begin{itemize}
\item $f$ has a \udef{left-identity} $e_L$ if $\forall x\in A:\; f(e_L, x) = x$;
\item $f$ has a \udef{right-identity} $e_R$ if $\forall x\in A:\; f(x, e_R) = x$;
\item $f$ has an \udef{identity} $e$ if $e$ is both a left- and a right-identity of $f$.
\end{itemize}
We say
\begin{itemize}
\item $f$ has a \udef{left-absorbing element} $u_L$ if $\forall x\in A:\; f(u_L, x) = u_L$;
\item $f$ has a \udef{right-absorbing element} $u_R$ if $\forall x\in A:\; f(x, u_R) = u_R$;
\item $f$ has an \udef{absorbing element} $u$ if $u$ is both a left- and a right-absorbing element of $f$.
\end{itemize}
\end{definition}

Left and right identity are dual. Left and right absorbing are also dual.

TODO require that absorbing element is no identity?

\begin{lemma} \label{leftRightIdentity}
Let $A$ be a class and $f: A\times A \to A$ a binary operator.
\begin{enumerate}
\item If $f$ has both a left-identity $e_L$ and a right-identity $e_R$, then $f$ has an identity $e$ and
\[ e = e_L = e_R. \]
\item If $f$ has both a left-absorbing element $u_L$ and a right-absorbing element $u_R$, then $f$ has an absorbing element $u$ and
\[ u = u_L = u_R. \]
\end{enumerate}
\end{lemma}
\begin{proof}
(1) Assume $f$ has a left- and a right-identity. Then $e_L = f(e_L, e_R) = e_R$.

(2) Assume $f$ has a left- and a right-absorbing element. Then $u_L = f(u_L, u_R) = u_R$.
\end{proof}
\begin{corollary}
A binary operator may have multiple left-identities or multiple right-identities, but if it has both, then the identity is unique.

An absorbing element is similarly unique.
\end{corollary}

\begin{definition}
Let $A$ be a class and $f: A\times A \to A$ a binary operator. We define
\[ \widetilde{A} \defeq \begin{cases}
A & \text{if $f$ has an identity} \\
A\uplus \{e\} & \text{if $f$ has no identity.}
\end{cases} \]
and also
\[ \widetilde{f} \quad\defeq\quad \widetilde{A}\times \widetilde{A}\to \widetilde{A}: (a,b) \mapsto \begin{cases}
f(a, b) & a,b\in A \\
b & a = e \\
a & b = e.
\end{cases} \]
\end{definition}

\begin{definition}
Let $A$ be a class and $f: A\times A \to A$ a binary operator. We define
\[ \widehat{A} \defeq \begin{cases}
A & \text{if $A$ has an absorbing element} \\
A\uplus \{u\} & \text{if $A$ has no absorbing element.}
\end{cases} \]
and also
\[ \widehat{f} \quad\defeq\quad \widehat{A}\times \widehat{A}\to \widehat{A}: (a,b) \mapsto \begin{cases}
f(a, b) & a,b\in A \\
u & (a = u \lor b = u).
\end{cases} \]
\end{definition}

TODO: should we add absorbing element regardless of prior existance??

\begin{lemma}
Let $A$ be a class and $f: A\times A \to A$ an associative partial binary function with absorbing element $u$. Then we can extend $f$ to an associative total function $\widehat{f}: A \times A \to A$
by setting
\[ (x,y) \in A\times A \setminus \preim(f) \quad\implies\quad \widehat{f}(x,y) = u. \]
\end{lemma}
\begin{proof}
We just need to verify associativity.

We have $\widehat{f}(\widehat{f}(x,y),z) = \widehat{f}(x,\widehat{f}(y,z))$ if $f(f(x,y),z)$ and $f(x,f(y,z))$ are defined.

If $f(f(x,y),z)$ is not defined, we claim $\widehat{f}(\widehat{f}(x,y),z) = u$. Indeed,
\begin{itemize}
\item if $f(x,y)$ is defined, then $f(\widehat{f}(x,y),z) = f(f(x,y),z)$ is undefined and thus equal to $u$;
\item if $f(x,y)$ is undefined, then $\widehat{f}(\widehat{f}(x,y),z) = \widehat{f}(u,z) = f(u,z) = u$.
\end{itemize}

Similarly, if $f(x, f(y, z))$ is not defined, we have $\widetilde{f}(x, \widetilde{f}(y, z)) = u$.
\end{proof}

\subsubsection{Closure}
TODO Galois connection.
\begin{definition}
Let $f:A\times A\to A$ be a binary function and $B\subseteq A$. Then we call $B$ \udef{closed under $f$} if $f(B,B) \subseteq B$.
\end{definition}

\subsubsection{Distributivity}
\begin{definition}
Let $A$ be a class and $f,g: A\times A \to A$ two binary functions. We say
\begin{itemize}
\item $f$ is \udef{left-distributive} over $g$ if
\[ \forall x,y,z\in A: \; f(x,g(y,z)) = g(f(x,y),f(x,z)); \]
\item $f$ is \udef{right-distributive} over $g$ if
\[ \forall x,y,z\in A: \; f(f(x,y), z) = g(f(x,z),f(y,z)); \]
\item $f$ is \udef{distributive} over $g$ if it is left- and right-distributive;
\item $f$ is \udef{self-distributive} if it is distributive over itself.
\end{itemize}
\end{definition}

\subsubsection{The absorption law}
\begin{definition}
Let $A$ be a class and $f,g: A\times A \to A$ two binary functions.
We say $f, g$ are linked by the \udef{absorption law} if
\[ \forall x,y\in A:\; f(x,g(x,y)) = x = g(x,f(x,y)). \]
\end{definition}

\begin{lemma} \label{absorptionIdempotency}
Let $A$ be a class and $f,g: A\times A\to A$ binary functions. If $f$ and $g$ are linked by the absorption law, then they are both idempotent.
\end{lemma}
\begin{proof}
For all $x\in A$ we have $f(x,x) = f(x,g(x,f(x,x))) = x$, where the last equality follows from the absorption law with $y = f(x,x)$.
\end{proof}


\section{Associative classes}
\begin{definition}
Let $A$ be a class and $f: A\times A \not\to A$ an associative binary partial function. We call $\sSet{A,f}$ an \udef{associative class.}

We will often abbreviate $f(x,y)$ by $xy$.
\end{definition}
TODO all functions are $\lambda - \rho$ in larger space.

In particular, the following are equivalent for all $x,y,z\in A$:
\begin{itemize}
\item $f\big(f(x,y),z\big)$ is defined;
\item $f\big(x, f(y, z)\big)$ is defined;
\item $f(x, y)$ and $f(y,z)$ are defined.
\end{itemize}
TODO: proper definition of when composition is defined!

Notice in particular, that the third point implies the first two.

\subsection{Undefined operations}
We can simulate a partial function by adding an absorbing element $u$ and mapping undefined operations to $u$. The problem with this is that elements can then only be cancellative if they can be composed with anything. Indeed, if $a,b\in A$ such that $f(a,b) = u$. Then $f(a,b) = u = f(a,u)$ and the only way $a$ can be left-cancellative is by having $b=u$.

The problem is that we have made any two undefined operations the same, while we really want any two undefined operations to be different.

TODO: modify set theory to allow $u\neq u$??

\subsubsection{Connections}
\begin{definition}
Let $A$ be an associative class and $x,y\in A$. We call
\begin{itemize}
\item $\leftconnections{x} \defeq \setbuilder{y\in A}{\text{$yx$ is defined}}$ the class of \udef{left connections};
\item $\rightconnections{x} \defeq \setbuilder{y\in A}{\text{$xy$ is defined}}$ the class of \udef{right connections}.
\end{itemize}
We write
\begin{itemize}
\item $x \preceq_L y$ if $\leftconnections{x} \subseteq \leftconnections{y}$;
\item $x \preceq_R y$ if $\rightconnections{x} \subseteq \rightconnections{y}$.
\end{itemize}
\end{definition}

\begin{lemma}
Let $A$ be an associative class and $x,y,z\in A$. Then
\begin{enumerate}
\item $\leftconnections{xy} = \leftconnections{x}$;
\item $\rightconnections{xy} = \rightconnections{y}$;
\item $x \preceq_L y \iff xz \preceq_L y \iff x \preceq_L yz$ if the relevant terms are defined;
\item $x \preceq_R y \iff zx \preceq_R y \iff x \preceq_R zy$ if the relevant terms are defined.
\end{enumerate}
\end{lemma}


\subsection{Distinguishability, cancellation, identity and inverses}
\subsubsection{Distinguishability}
\begin{definition}
Let $A$ be an associative class and $x,y\in A$. 
We say
\begin{itemize}
\item $x$ and $y$ are \udef{left-distinguishable} if
\[ x\neq y \quad\implies\quad \exists a\in A: \;\text{$ax$ or $ay$ is defined and $ax \neq ay$}; \]
\item $x$ and $y$ are \udef{right-distinguishable} if
\[ x\neq y \quad\implies\quad \exists a\in A: \; \text{$xa$ or $ya$ is defined and $xa \neq ya$}; \]
\item $x$ and $y$ are \udef{distinguishable} if they are left- \emph{or} right-distinguishable;
\item $x$ and $y$ are \udef{bidistinguishable} if they are left- \emph{and} right-distinguishable.
\end{itemize}
We say
\begin{itemize}
\item $A$ is left-distinguishable if every two elements in $A$ are left-distinguishable;
\item $A$ is right-distinguishable if every two elements in $A$ are right-distinguishable;
\item $A$ is distinguishable if every two elements in $A$ are distinguishable;
\item $A$ is bidistinguishable if every two elements in $A$ are bidistinguishable.
\end{itemize}
\end{definition}

\begin{lemma}
Let $A$ be an associative class and $x,y\in A$. 
Then
\begin{enumerate}
\item $x$ and $y$ are left-distinguishable \textup{if and only if}
\[ x = y \quad\iff\quad \forall a\in A: \;\Big(\text{$ax$ or $ay$ is defined} \implies ax = ay\Big); \]
\item $x$ and $y$ are right-distinguishable \textup{if and only if}
\[ x = y \quad\iff\quad \forall a\in A: \; \Big(\text{$xa$ or $ya$ is defined} \implies xa = ya\Big). \]
\end{enumerate}
In both cases the direction $\Rightarrow$ is automatic.
\end{lemma}

\subsubsection{Cancellation}
\begin{definition}
Let $A$ be an associative class and $x\in A$. 
We call $x$
\begin{itemize}
\item \udef{left-cancellative} or \udef{monic} if $x\cdot -: y\mapsto xy$ is injective;
\item \udef{right-cancellative} or \udef{epic} if $-\cdot x: y\mapsto yx$ is injective.
\end{itemize}
\end{definition}

Left and right cancellative are dual properties.

\begin{lemma}
Let $A$ be an associative class and $x,y$ in $A$.
\begin{enumerate}
\item If $x$ and $y$ are left-(resp. right-)cancellative, then $xy$ is left-(resp. right-)cancellative.
\item If $xy$ is left-cancellative, then $y$ is left-cancellative.
\item If $xy$ is right-cancellative, then $x$ is right-cancellative.
\end{enumerate}
\end{lemma}
\begin{proof}
Let $z_1, z_2\in A$.

(1) Assume that $x$ and $y$ are left-cancellative and $(xy)z_1 = (xy)z_2$. By associativity, we have $x(yz_1) = x(yz_2)$. Thus by injectivity we get first $yz_1 = yz_2$ and then $z_1 = z_2$.

Right-cancellation is similar.

(2) Assume $yz_1 = yz_2$. Then $xyz_1 = xyz_2$, so $z_1 = z_2$ because $xy$ is left-cancellative.

(3) Assume $z_1x = z_2x$. Then $z_1xy = z_2xy$ so $z_1 = z_2$ because $xy$ is right-cancellative.
\end{proof}

\begin{lemma}
Let $A$ be an associative class.
\begin{enumerate}
\item If $\forall x\in A$ there exists a left-cancellative element $a_x$ such that $a_xx$ exists, then $A$ is left-distinguishable.
\item If $\forall x\in A$ there exists a right-cancellative element $a$ such that $ax$ exists, then $A$ is right-distinguishable.
\end{enumerate}
\end{lemma}
\begin{proof}
(1) Take $x,y\in A$ and assume $\forall b\in A: \Big(\text{$bx$ or $by$ is defined} \implies bx = by\Big)$. In particular, this means that $a_xx = a_xy$. By left-cancellation, we have $x=y$.

(2) Dual to (1).
\end{proof}

\subsubsection{Identity and objects}
\begin{definition}
Let $A$ be a class, $f: A\times A \not\to A$ a binary partial function and $e\in A$ an idempotent (i.e.\ $e^2 = e$).

We call $e$
\begin{itemize}
\item a \udef{centre identity} if $xy = xey$ for all $x,y\in A$ such that both sides are defined;
\item a \udef{left identity} if $x = ex$ for all $x\in A$ such that $ex$ is defined;
\item a \udef{right identity} if $x = xe$ for all $x\in A$ such that $xe$ is defined;
\item an \udef{identity} if $e$ is both a left and a right identity.
\end{itemize}

We call $e$
\begin{itemize}
\item a \udef{weak left identity} if $x = ex$ for all $x\in A$ such that $ex$ is defined and $\leftconnections{e} = \leftconnections{x}$;
\item a \udef{weak right identity} if $x = xe$ for all $x\in A$ such that $xe$ is defined and $\rightconnections{e} = \rightconnections{x}$;
\item a \udef{weak identity} if $e$ is both a weak left and a weak right identity.
\end{itemize}
\end{definition}

\begin{lemma}
Let $A$ be an associative class and $e\in A$ an idempotent.
Then
\begin{enumerate}
\item $e$ is a left identity \textup{if and only if} $e$ is left-cancellative;
\item $e$ is a right identity \textup{if and only if} $e$ is right-cancellative.
\end{enumerate}
\end{lemma}
\begin{proof}
(1) First assume $e$ is a left identity. Let $x,y\in A$ be such that $ex = ey$. Then $x = ex = ey = y$.

Now assume that $e$ is left-cancellative and that $ex$ is defined. Then $ex = eex$ by idempotency and thus $x = ex$ by left-cancellation.

(2) Dual.
\end{proof}

\begin{lemma} \label{uniquenessIdentity}
Let $A$ be an associative class and $e,e'\in A$. Then
\begin{enumerate}
\item if $e,e'$ are weak left identities such that $ee' = e'e$, then $e=e'$;
\item if $e,e'$ are weak right identities such that $ee' = e'e$, then $e=e'$.
\end{enumerate}
Also
\begin{enumerate} \setcounter{enumi}{2}
\item if $e$ is a left identity and $e'$ a right identity such that $ee'$ is defined, then $e=e'$;
\end{enumerate}
and
\begin{enumerate} \setcounter{enumi}{3}
\item if $e,e'$ are identities and there exists $x\in A$ such that $ex$ and $e'x$ are both defined, then $e = e'$;
\item if $e,e'$ are identities and there exists $x\in A$ such that $xe$ and $xe'$ are both defined, then $e = e'$.
\end{enumerate}
\end{lemma}
\begin{proof}
(1) We have $\leftconnections{e} = \leftconnections{ee'} = \leftconnections{e'e} = \leftconnections{e'}$, so $e = e'e = ee' = e'$.

(2) Dual.

(3) We have $e = ee' = e'$.

(4) We have $x = ex = ee'x$, so $ee'$ is defined. We conclude with (3).

(5) Dual to (4).
\end{proof}

\begin{lemma} \label{distinguishableIdentity}
Let $A$ be an associative class and $e\in A$.
\begin{enumerate}
\item If $e$ is a left identity and $A$ is right-distinguishable, then $e$ is a weak right identity.
\item If $e$ is a right identity and $A$ is left-distinguishable, then $e$ is a weak left identity.
\end{enumerate}
\end{lemma}
\begin{proof}
(1) Take $x\in A$ such that $xe$ is defined and $\rightconnections{e} = \rightconnections{x}$. Take arbitrary $a\in A$. If $xa$ is defined, then $xea$ is also defined and $xea = xa$. By right-distinguishability, we have $xe = x$. Thus $e$ is a right identity.

(2) Dual.
\end{proof}

\begin{lemma} \label{identityConnection}
Let $A$ be an associative class and $e,x\in A$. Then
\begin{enumerate}
\item if $e$ is a left identity such that $ex$ is defined, then $\leftconnections{e} = \leftconnections{x}$;
\item if $e$ is a right identity such that $xe$ is defined, then $\rightconnections{e} = \rightconnections{x}$;
\end{enumerate}
and
\begin{enumerate} \setcounter{enumi}{2}
\item if $e$ is a left identity such that $xe$ is defined, then $e \preceq_R x$;
\item if $e$ is a right identity such that $ex$ is defined, then $e \preceq_L x$.
\end{enumerate}
\end{lemma}
\begin{proof}
(1) We have $\leftconnections{e} = \leftconnections{ex} = \leftconnections{x}$.

(2) Dual.

(3) Take $a\in \rightconnections{e} = \rightconnections{xe}$. Then $xea$ is defined and $xea = xa$, so $a\in \rightconnections{x}$.

(4) Dual.
\end{proof}


\subsubsection{Inverses}
\begin{definition}
Let $A$ be an associative class and $x\in A$. 
We say an element $y\in A$ is
\begin{itemize}
\item a \udef{left-inverse} of $x$ if $yx$ is a left-identity;
\item a \udef{right-inverse} of $x$ if $xy$ is a right-identity;
\item a \udef{(two-sided) inverse} of $x$ it is both a left- and a right-inverse.
\end{itemize}
We say $(x,y)\in A^2$ is a pair of
\begin{itemize}
\item \udef{mutual left-inverses} if $x$ is a left-inverse of $y$ and $y$ a left-inverse of $x$;
\item \udef{mutual right-inverses} if $x$ is a right-inverse of $y$ and $y$ a right-inverse of $x$.
\end{itemize}
We call $x$ \udef{invertible} if it has a (left/right) mutual inverse.
\end{definition}

\begin{lemma}
Let $A$ be an associative class and $x\in A$.
\begin{enumerate}
\item If $x$ has a left-inverse, then it is left-cancellative.
\item If $x$ has a right-inverse, then it is right-cancellative.
\end{enumerate}
\end{lemma}
\begin{proof}
(1) Let $y\in A$ be a left-inverse of $x$. Take arbitrary $a,b\in A$. Assume $xa = xb$. Then $yxa = yxb$ and so $a = b$.

(2) Dual.
\end{proof}

\begin{proposition}
Let $A$ be an associative class and $x,y_1,y_2,l,r \in A$.
\begin{enumerate}
\item If $A$ is right-distinguishable and both $(x, y_1)$ and $(x,y_2)$ are pairs of mutual left-inverses, then $y_1 = y_2$.
\item If $A$ is left-distinguishable and both $(x, y_1)$ and $(x,y_2)$ are pairs of mutual right-inverses, then $y_1 = y_2$.
\end{enumerate}
Also
\begin{enumerate} \setcounter{enumi}{2}
\item If $l$ is a left-inverse of $x$ and $r$ a right-inverse of $x$, then $l = r$.
\item If $x$ has an inverse, then this inverse is unique.
\end{enumerate}
\end{proposition}
\begin{proof}
(1) We have $y_1 = (y_2x)y_1 = y_2(xy_1)$, so the latter is defined, which implies $xy_1 \preceq_R y_2$ by \ref{identityConnection}. Thus $y_1 \preceq y_2$. Similarly $y_1xy_2$ is defined and this implies $y_2 \preceq y_1$. Thus $\rightconnections{xy_1} = \rightconnections{y_1} = \rightconnections{y_2}$. By \ref{distinguishableIdentity} $xy_1$ is a weak right identity, so $\rightconnections{xy_1} = \rightconnections{y_2}$ implies $y_1 = y_2xy_1 = y_2$.

(2) Dual.

(3) We have $l = l(xr) = (lx)r = r$.

(4) Any inverse of $x$ is both a left- and a right-inverse.
\end{proof}

If $x$ is invertible, we denote the unique inverse by $x^{-1}$.




\begin{lemma}
Let $A$ be an associative class and $x,y \in A$ such that $xy$ is defined.
\begin{enumerate}
\item If $x$ has left-inverse $x^{-L}$ and $y$ has a left-inverse $y^{-L}$, then $xy$ has a left-inverse $y^{-L}x^{-L}$.
\item If $x$ has right-inverse $x^{-R}$ and $y$ has a right-inverse $y^{-R}$, then $xy$ has a right-inverse $y^{-R}x^{-R}$.
\item If $x$ has inverse $x^{-1}$ and $y$ has inverse $y^{-1}$, then $xy$ has inverse $y^{-1}x^{-1}$.
\item If $xy$ has ya left-inverse $z$, then $y$ has a left-inverse $zx$.
\item If $xy$ has a right-inverse $z$, then $x$ has a right-inverse $yz$.
\end{enumerate}
\end{lemma}
\begin{proof}
(1) Set $e = x^{-L}x$. We calculate
\[ y^{-L}x^{-L}xy = y^{-L}ey = y^{-L}y, \]
which is a left identity. Then we just need to show that $xyy^{-L}x^{-L}xy = xy$. Indeed
\[ xyy^{-L}x^{-L}xy = xyy^{-L}ey = x(yy^{-L}y) = xy. \]

(2) Dual to (1).

(3) Follows from (1) and (2).

(4) Set $e = z(xy)$. We calculate
\[ e = z(xy) = (zx)y, \]
so $zx$ is a left-inverse of $y$.

\end{proof}



\begin{lemma}
Let $\sSet{A, f}$ be an associative class and $x,y$ in $A$ with identity $e$. Then
\begin{enumerate}
\item if $x$ has a left inverse, it is left-cancellative;
\item if $x$ has a right inverse, it is right-cancellative;
\item if $x$ has a left inverse and is right-cancellative, it is invertible;
\item if $x$ has a right inverse and is left-cancellative, it is invertible.
\end{enumerate}
\end{lemma}
\begin{proof}
(1) Let $l$ be a left inverse of $x$ and assume $f(x, z_1) = f(x,z_2)$, then $f(l, f(x,z_1)) = f(l, f(x,z_2))$ and thus
\[ z_1 = f(e,z_1) = f(f(l,x), z_1) = f(l, f(x,z_1)) = f(l, f(x,z_2)) = f(f(l,x), z_2) = f(e,z_2) = z_2. \]

(2) Similar.

(3) Let $l$ be a left inverse of $x$. It is enough to show that $l$ is also a right inverse of $x$. We calculate
\[ f(f(x,l), x) = f(x, f(l,x)) = f(x, e) = x = f(e,x). \]
Beacuse $x$ is right-cancellative, this means $f(x,l) = e$ and thus that $l$ is a right inverse.

(4) Similar.
\end{proof}


\subsection{Left and right relation}
\begin{definition}
Let $\sSet{A, f}$ be an associative class and $x, y\in A$. Then
\begin{itemize}
\item let $L$ be the relation defined by $xLy \defequiv \exists a: f(a, x) = y$;
\item let $R$ be the relation defined by $xRy \defequiv \exists a: f(x, a) = y$;
\item let $L'$ be the relation defined by $xL^!y \defequiv  \exists! a: f(a, x) = y$;
\item let $R'$ be the relation defined by $xR^!y \defequiv \exists! a: f(x, a) = y$.
\end{itemize}
\end{definition}

The relations $L$ and $R$ are dual. The relations $L^!$ and $R^!$ are dual.

\begin{lemma}
The relations $L$ and $R$ are transitive.
\end{lemma}
\begin{proof}
Let $\sSet{A, f}$ be an associative class and $x, y, z\in A$ such that $xLy$ and $yLz$. Then there exist $a,b\in A$ such that $f(a,x) = y$ and $f(b,y) = z$. Then
\[ z = f(b,y) = f\big(b,f(a,x)\big) = f\big(f(b,a), x\big), \]
so $xLz$. The statement for $R$ is dual.
\end{proof}

\begin{lemma}
Let $\sSet{A, f}$ be an associative class and $x\in A$. Then the following are equivalent:
\begin{enumerate}
\item $x$ is right-cancellative
\item $xL \subseteq xL^!$;
\item $xL = xL^!$.
\end{enumerate}
As are the following:
\begin{enumerate}
\item $x$ is left-cancellative
\item $xR \subseteq xR^!$;
\item $xR = xR^!$.
\end{enumerate}
\end{lemma}
\begin{proof}
$(1) \Rightarrow (2)$ Take $y\in xL$. Then there exists $a\in A$ such that $f(a,x) = y$. Assume there exists another $b\in A$ such that $f(b,x) = y$. Then $f(a,x) = f(b,x)$, so $a=b$. Thus the $a\in A$ is unique and $y\in xL^!$.

$(2) \Rightarrow (3)$ The inclusion $xL \supseteq xL^!$ is immediate.

$(3) \Rightarrow (1)$ Take $a,b\in A$ and assume $f(a,x) = f(b,x) = y$. Then $xLy$, so $xL^!y$, so $a=b$.

The proof of the second part is dual.
\end{proof}

\begin{lemma}
Let $\sSet{A, f}$ be an associative class. Then
\begin{enumerate}
\item $L = \bigcup_{a\in A}f(a, -)$;
\item $R = \bigcup_{a\in A}f(-, a)$.
\end{enumerate}
\end{lemma}

For given and fixed $f$, we will often write
\begin{itemize}
\item $\lambda_a: A\to A$ for $f(a, -)$;
\item $\rho_a: A\to A$ for $f(-, a)$.
\end{itemize}

\begin{lemma} \label{lambdaRhoCommute}
For all $a,b\in A$, we have $\lambda_a\circ \rho_b = \rho_b \circ \lambda_a$.
\end{lemma}
\begin{proof}
We calculate, for arbitrary $x\in A$,
\[ \lambda_a\big(\rho_b(x)\big) = f\big(a, f(x,b)\big) = f\big(f(a,x), b\big)  = \rho_b\big(\lambda_a(x)\big). \]
\end{proof}
\begin{corollary} \label{LRcommute}
Let $A$ be a class and $f: A\times A \to A$ an associative binary function. Then $L;R = R;L$.
\end{corollary}
\begin{proof}
Let $x,y\in A$. Then $x(L;R)y$ iff there exist $a,b\in A$ such that the top path in
\[ \begin{tikzcd}
x \ar[r, maps to, "\lambda_a"] \ar[d, maps to, swap, "\rho_b"] & f(a, x) \ar[d, maps to, "\rho_b"] \\
f(x, b) \ar[r, maps to, swap, "\lambda_a"] & y
\end{tikzcd} \]
holds. By \ref{lambdaRhoCommute} this is equivalent to the bottom path holding. The bottom path implies $x(R;L)y$.
\end{proof}

\begin{proposition} \label{functionsLeftRightRelations}
Let $g,h$ be functions in the assocative class of functions with composition $\circ$. Then
\begin{enumerate}
\item $gLh$ \textup{if and only if} $\ker g \subseteq \ker h$;
\item $gRh$ \textup{if and only if} $\im g \supseteq \im h$. 
\end{enumerate}
\end{proposition}

\subsubsection{Principal ideals}
\begin{definition}
Let $\sSet{A, f}$ be an associative class and $x, y\in A$. Then
\begin{itemize}
\item the \udef{left principal ideal} generated by $x$ is $f(\widetilde{A}, x) = f(A, x)\cup \{x\}$;
\item the \udef{right principal ideal} generated by $x$ is $f(x, \widetilde{A}) = f(x,A)\cup \{x\}$.
\end{itemize}
\end{definition}

\begin{lemma} \label{idealAbsorption}
Let $\sSet{A, f}$ be an associative class and $x, y\in A$. Then
\begin{enumerate}
\item $f(A, f(x,y)) \subseteq f(A, y)$;
\item $f(f(x,y), A) \subseteq f(x, A)$.
\end{enumerate}
\end{lemma}
\begin{proof}
(1) Take $z\in f(A, f(x,y))$. Then there exists $z'\in A$ such that
\[ z = f(z', f(x,y)) = f(f(z', x), y) \in f(A, y). \]

(2) By duality.
\end{proof}

\begin{lemma}
LLet $\sSet{A, f}$ be an associative class and $x, y\in A$. Then
\begin{enumerate}
\item $f(A, x) \subseteq f(A,y)$ \textup{if and only if} $\forall a\in A: \exists b\in A: \; f(a, x) = f(b, y)$;
\item $f(x, A) \subseteq f(y, A)$ \textup{if and only if} $\forall a\in A: \exists b\in A: \; f(x, a) = f(y, b)$.
\end{enumerate}
If $A$ contains an identity $e$, then
\begin{enumerate} \setcounter{enumi}{2}
\item $f(A, x) \subseteq f(A,y)$ \textup{if and only if} $\exists b\in A: \; x = f(b, y)$;
\item $f(x, A) \subseteq f(y, A)$ \textup{if and only if} $\exists b\in A: \; x = f(y, b)$.
\end{enumerate}
\end{lemma}

\begin{proposition} \label{invertibilityFromPrincipalIdeals}
Let $\sSet{A, f}$ be an associative class. Then $f$ has an identity and every $x\in A$ is invertible \textup{if and only if}
\[ \forall x\in A: \quad f(x, A) = A = f(A,x). \]
\end{proposition}
\begin{proof}
$\Rightarrow$ We clearly have $f(x, A) \subseteq A$. The other inclusion follows from \ref{idealAbsorption}: $A = f(A, e) = f(A, f(x^{-1}, x)) \subseteq f(A, x)$.

$\Leftarrow$ Pick some $x\in A$, so $x\in f(x,a)$, meaning there exists an $a\in A$ such that $x = xa$. We claim $a$ is a right-identity for $f$. Indeed, take arbitrary $y\in A$. Then $y = f(b,x)$ for some $b\in A$ and so
\[ f(y, a) = f(f(b,x), a) = f(b,f(x,a)) = f(b,x) = y. \]
In the same way we can also find a left-identity. So $A$ contains an identity $e \defeq a$ by \ref{leftRightIdentity}.

Now for all $x\in A$ we have $e\in A = f(x,A)$, so we can find a right-inverse of $x$. Similarly, we can find a left-inverse of $x$. This means $x$ is invertible by \ref{leftRightInverse}.
\end{proof}


\subsubsection{Green's relations}
\begin{definition}
Let $\sSet{A, f}$ be an associative class.
The \udef{Green's relations} of $f$ are relations on $A$ defined as
\begin{itemize}
\item $\greensL$ is the reflexive closure of the symmetric part of $L$;
\item $\greensR$ is the reflexive closure of the symmetric part of $R$;
\item $\greensH \;\defeq\; \greensL \cap\greensR$;
\item $\greensD \;\defeq\; \greensL;\greensR$.
\end{itemize}
\end{definition}
The relations $\greensL$ and $\greensR$ are dual.

\begin{lemma}
Let $\sSet{A, f}$ be an associative class. Then the relations $\greensL$ and $\greensR$ are equivalence relations.
\end{lemma}

\begin{lemma}
Let $\sSet{A, f}$ be an associative class and $x, y\in A$. Then the following are equivalent:
\begin{enumerate}
\item $x \greensL y$;
\item $x\big((L\cap L^\transp) \cup \id_A\big)y$
\item $\exists a,b\in \widetilde{A}: (f(a, x) = y) \land (f(b, y) = x)$;
\item $\begin{tikzcd}[sep=large]
x \ar[r, maps to, shift left, "\exists a: \lambda_a"] & y \ar[l, maps to, shift left, "\exists b: \lambda_b"]
\end{tikzcd}$
\item $f(\widetilde{A},x) = f(\widetilde{A}, y)$;
\item $f(A,x) = f(A, y)$;
\end{enumerate}
as are
\begin{enumerate}
\item $x \greensR y$;
\item $x\big((R\cap R^\transp) \cup \id_A\big)y$
\item $\exists a,b\in \widetilde{A}: (f(x, a) = y) \land (f(y, b) = x)$;
\item $\begin{tikzcd}[sep=large]
x \ar[r, maps to, shift left, "\exists a: \rho_a"] & y \ar[l, maps to, shift left, "\exists b: \rho_b"]
\end{tikzcd}$
\item $f(x, \widetilde{A}) = f(y, \widetilde{A})$;
\item $f(x, A) = f(y, A)$.
\end{enumerate}
\end{lemma}

\begin{lemma} \label{lambdaRhoPreservation}
Let $\sSet{A, f}$ be an associative class and $x, y, z\in A$.
\begin{enumerate}
\item If $x\greensL y$, then $\rho_z(x)\greensL \rho_z(y)$.
\item If $x\greensR y$, then $\lambda_z(x)\greensR \lambda_z(y)$.
\end{enumerate}
\end{lemma}



\subsubsection{Egg-box diagrams}
\begin{proposition} \label{greensDequivalence}
The relation $\greensD$ is an equivalence relation.
\end{proposition}
This is equivalent to $\greensL;\greensR = \greensR;\greensL$ by \ref{commutingEquivalenceRelations}. 
\begin{proof}
Assume $x(\greensR;\greensL)z$, meaning $\exists y: x\greensR y$ and $y\greensL z$. Then there exist $a,b,c,d\in \widetilde{A}$ such that
\[ \begin{tikzcd}
x \ar[r, maps to, shift left, "\rho_a"] & y \ar[l, maps to, shift left, "\rho_b"] \ar[r, maps to, shift left, "\lambda_c"] & z \ar[l, maps to, shift left, "\lambda_d"]
\end{tikzcd}. \]
Using \ref{lambdaRhoCommute}, we can rearrange such that we also get the mappings along the left and bottom sides of
\[ \begin{tikzcd}[sep=large]
x \ar[r, maps to, shift left, "\rho_a"] \ar[d, maps to, shift left, "\lambda_c"] & y \ar[l, maps to, shift left, "\rho_b"] \ar[d, maps to, shift left, "\lambda_c"] \\
y' \ar[u, maps to, shift left, "\lambda_d"] \ar[r, maps to, shift left, "\rho_a"] & z \ar[u, maps to, shift left, "\lambda_d"] \ar[l, maps to, shift left, "\rho_b"]
\end{tikzcd} \]
for some $y' \in A$. Thus $x(\greensL;\greensR)y$. The other inclusion is similar.
\end{proof}

From \ref{commutingEquivalenceRelations} we also know that the $\greensL,\greensR$-egg-box diagram decomposes into blocks, which are $\greensD$-equivalence classes. The columns are $\greensL$-equivalence classes, the rows are $\greensR$-equivalence classes, and the cells are $\greensH$-equivalence classes.

\begin{example}
Let $A = (\{1,2,3\} \to \{1,2,3\})$ with the binary function $A\times A \to A: (f,g)\mapsto f;g$. We can represent an element $f$ of $A$ as $(f(1) f(2) f(3))$. We have
\begin{itemize}
\item $f\greensL g$ if $f$ and $g$ have the same image;
\item $f\greensR g$ if $f$ and $g$ have the same kernel.
\end{itemize}
An egg-box diagram can be drawn as follows:
\[ \begin{array}{|c|c|c|c|c|c|c|}
\hline
\mathbf{(1 1 1)} & \mathbf{(2 2 2)} & \mathbf{(3 3 3)} &&& &  \\ \hline
&& & \mathbf{(1 2 2)}, & \mathbf{(1 3 3)}, & (2 3 3), &  \\
&& & (2 1 1)  & (3 1 1)  & (3 2 2)  &  \\ \hline
&& & (2 1 2), & (3 1 3), & \mathbf{(3 2 3)},  &  \\
&& & \mathbf{(1 2 1)}  & (1 3 1)  & (2 3 2)  &  \\ \hline
&& & (2 2 1), & (3 3 1), & (3 3 2),  &  \\
&& & (1 1 2)  & \mathbf{(1 1 3)}  & \mathbf{(2 2 3)}  &  \\ \hline
&& &&& & \mathbf{(1 2 3)}, (2 3 1), (3 1 2)  \\
&& &&& & (1 3 2), (2 1 3), (3 2 1)  \\ \hline
\end{array} \]
The bold elements are idempotents.
\end{example}


\begin{proposition}[Green's lemma] \label{GreensLemma}
Let $\sSet{A, f}$ be an associative class and $x, y\in A$.
\begin{enumerate}
\item If $x\greensL y$ with $\begin{tikzcd}
x \ar[r, maps to, shift left, "\lambda_a"] & y \ar[l, maps to, shift left, "\lambda_b"]
\end{tikzcd}$, then
\begin{enumerate}
\item $\lambda_a|_{[x]_\greensR}: [x]_\greensR \to [y]_\greensR \qquad \text{is a bijection with inverse} \qquad \lambda_b|_{[y]_\greensR}: [y]_\greensR \to [x]_\greensR$;
\item $\lambda_a|_{[x]_\greensH}: [x]_\greensH \to [y]_\greensH \qquad \text{is a bijection with inverse} \qquad \lambda_b|_{[y]_\greensH}: [y]_\greensH \to [x]_\greensH$.
\end{enumerate}
\item If $x\greensR y$ with $\begin{tikzcd}
x \ar[r, maps to, shift left, "\rho_a"] & y \ar[l, maps to, shift left, "\rho_b"]
\end{tikzcd}$, then
\begin{enumerate}
\item $\rho_a|_{[x]_\greensL}: [x]_\greensL \to [y]_\greensL \qquad \text{is a bijection with inverse} \qquad \rho_b|_{[y]_\greensL}: [y]_\greensL \to [x]_\greensL$;
\item $\rho_a|_{[x]_\greensH}: [x]_\greensH \to [y]_\greensH \qquad \text{is a bijection with inverse} \qquad \rho_b|_{[y]_\greensH}: [y]_\greensH \to [x]_\greensH$.
\end{enumerate}
\end{enumerate}
\end{proposition}
\begin{proof}
Take some arbitrary $x'\in [x]_\greensR$. Then there exist $c,d\in \widetilde{A}$ such that $\begin{tikzcd}
x \ar[r, maps to, shift left, "\rho_c"] & x' \ar[l, maps to, shift left, "\rho_d"]
\end{tikzcd}$. Then, by \ref{lambdaRhoCommute},
\[ \begin{tikzcd}
x' \ar[r, maps to, shift left, "\rho_d"] & x \ar[l, maps to, shift left, "\rho_c"] \ar[r, maps to, shift left, "\lambda_a"] & y \ar[l, maps to, shift left, "\lambda_b"]
\end{tikzcd} \qquad\text{implies that} \qquad \begin{tikzcd}
x' \ar[r, maps to, shift left, "\lambda_a"] & y' \ar[l, maps to, shift left, "\lambda_b"] \ar[r, maps to, shift left, "\rho_d"] & y \ar[l, maps to, shift left, "\rho_c"]
\end{tikzcd} \]
Thus, for all $x'\in [x]_\greensR$, we have
\begin{itemize}
\item $\lambda_a(x') \in [y]_\greensR$, meaning that $\lambda_a|_{[x]_\greensR}: [x]_\greensR \to [y]_\greensR$ is well-defined;
\item by similar reasoning, we can see that $\lambda_b|_{[y]_\greensR}: [y]_\greensR \to [x]_\greensR$ is also well-defined;
\item $\lambda_b(\lambda_a(x')) = x'$, so the functions are inverse of each other.
\end{itemize}
If $x'\in [x]_\greensH$, then $\lambda_a(x')\greensL \lambda_a(x) = y$ by \ref{lambdaRhoPreservation}, so $\lambda_a(x')\in [y]_\greensH$. This means that $\lambda_a|_{[x]_\greensH}: [x]_\greensH \to [y]_\greensH$ is well-defined.

Point (2) is dual.
\end{proof}
\begin{corollary} \label{greensDisomorphism}
Let $\sSet{A, f}$ be an associative class and $x, y\in A$ such that $x\greensD y$. Then there exist $a,b,c,d\in \widetilde{A}$ such that
\[ \lambda_a\circ\rho_b|_{[x]_\greensH}: [x]_\greensH \to [y]_\greensH \qquad \text{is a bijection with inverse} \qquad \lambda_c\circ \rho_d|_{[y]_\greensH}: [y]_\greensH \to [x]_\greensH. \]
\end{corollary}
\begin{proof}
There exists a $z\in A$ such that $x\greensL z$ and $z\greensR y$. We then just compose the bijections in Green's lemma, keeping in mind that $\lambda$ and $\rho$ commute.
\end{proof}
\begin{corollary}
Let $\sSet{A, f}$ be an associative class and $x\in A$. If $x$ is an idempotent, then
\[ \lambda_x|_{[x]_\greensR} = \id_{[x]_\greensR} \qquad \text{and} \qquad \rho_x|_{[x]_\greensL} = \id_{[x]_\greensL}. \]
Thus $x$ is a left identity for $[x]_\greensR$ and a right identity for $[x]_\greensL$.
\end{corollary}
\begin{proof}
Take $y\in [x]_\greensR$. Then there exist $a,b\in \widetilde{A}$ such that $\begin{tikzcd}
x \ar[r, maps to, shift left, "\rho_a"] & y \ar[l, maps to, shift left, "\rho_b"]
\end{tikzcd}$. Then we have
\[ xy = xxb = xb = y. \]
The other claim is dual.
\end{proof}

\begin{theorem}[Green's theorem]
Let $\sSet{A, f}$ be an associative class and $H$ an $\mathcal{H}$-class in $A$. Then either
\begin{enumerate}
\item $H^2\perp H$; or
\item $H^2 = H$, $f|_{H\times H}$ has an identity and each $x\in H$ is invertible.
\end{enumerate}
\end{theorem}
\begin{proof}
Suppose $H^2\cap H \neq \emptyset$, then there exist $a,b\in H$ such that $ab = c\in H$. By the Green's lemma \ref{GreensLemma} we have that $\rho_b:H\to H$ and $\lambda_a: H\to H$ are bijections.

Then for all $h\in H$, $\rho_b(h) = hb \in H$. Again by the Green's lemma, this means that $\lambda_h: H\to H$ is a bijection. Similarly $\rho_h: H\to H$ is a bijection for all $h$. So for all $h\in H$ we have $hH = H = Hh$. The result follows by \ref{invertibilityFromPrincipalIdeals}.
\end{proof}
\begin{corollary} \label{GreensTheoremCorollary}
Let $\sSet{A, f}$ be an associative class and $H$ an $\mathcal{H}$-class in $A$.Then
\begin{enumerate}
\item if $x$ is an idempotent in $H$, then we have the second case;
\item no $\mathcal{H}$-class can contain more than one idempotent. 
\end{enumerate}
\end{corollary}

\subsection{Regular elements and generalised inverses}
\begin{definition}
Let $\sSet{A, f}$ be an associative class and $x, y\in A$. We call
\begin{itemize}
\item $x$ \udef{regular} if $\exists a\in A: \; x = f(f(x, a), x)$
\item $x$ and $y$ \udef{generalised inverses} if $x = f(f(x, y), x)$ and $y = f(f(y, x), y)$.
\end{itemize}
\end{definition}

\begin{proposition} 
Let $\sSet{A, f}$ be an associative class. If $x\in A$ is regular, then every element in $[x]_{\greensD}$ is regular.
\end{proposition}
So it makes sense to call a $\greensD$-class \udef{regular} if it consists of regular elements and \udef{irregular} otherwise.
\begin{proof}
Let $x$ be regular with $x = xx'x$ and $x\greensD y$. Then we have $a,b,c,d \in \widetilde{A}$ such that
$\lambda_a\circ\rho_b|_{[x]_\greensH}: [x]_\greensH \to [y]_\greensH$ is a bijection with inverse $\lambda_c\circ \rho_d|_{[y]_\greensH}: [y]_\greensH \to [x]_\greensH$, as in \ref{greensDisomorphism}. Then we have
\[ y = (\lambda_a\circ\rho_b)(x) = (\lambda_a\circ\rho_b)(xx'x) = axx'xb = a(\lambda_c\circ\rho_d)(y)x'(\lambda_c\circ\rho_d)(y)b = ydx'cy. \]
So $y$ is regular.
\end{proof}
\begin{corollary}
If there is an idempotent $x\in [a]_{\greensD}$, then $[a]_{\greensD}$ is regular.
\end{corollary}
\begin{proof}
An idempotent is regular: $x = f(x,x) = f(f(x,x), x)$.
\end{proof}

\begin{proposition}
Let $\sSet{A, f}$ be an associative class. Then $x$ is regular \textup{if and only if} it has a generalised inverse.
\end{proposition}
\begin{proof}
Clearly every element with a generalised inverse is regular. Conversely, assume $x$ regular with $x = xax$. Then $y = axa$ is a generalised inverse of $x$: $x(axa)x = xax = x$ and $(axa)x(axa) = a(xax)axa = axaxa = axa$.
\end{proof}
Note that we do not have that $x = xyx$ implies $y = yxy$.

\begin{proposition} \label{greensRelationsRegularElements}
Let $\sSet{A, f}$ be an associative class and $x\in A$ a regular element with $x = f(f(x, y), x)$. Then
\begin{enumerate}
\item $f(x,y)$ and $f(y,x)$ are idempotent;
\item $f(y,x) \greensL x$ and $x \greensR f(x, y)$.
\end{enumerate}
\end{proposition}
\begin{proof}
(1) We calculate
\[ f(f(x,y), f(x,y)) = f(f(f(x,y), x), y) = f(x, y)\] and \[f(f(y,x), f(y,x)) = f(y, f(x, f(y,x))) = f(y,x). \]

(2) Using \ref{idealAbsorption}, we have
\[ f(A, x) = f(A, f(x,f(y,x))) \subseteq f(A, f(y,x)) \subseteq f(A,x). \]
Thus $f(A, x) = f(A, f(y,x))$. The second part is dual.
\end{proof}
\begin{corollary}
In a regular $\mathcal{D}$-class each $\mathcal{L}$-class and each $\mathcal{R}$-class contains at least one idempotent.
\end{corollary}
\begin{proof}
Let $[x]_\mathcal{L}$ be an $\mathcal{L}$-class in a regular $\mathcal{D}$-class. By regularity there exists a $y\in A$ such that $xyx = x$. From the proposition, we have that $[x]_\mathcal{L}$ contains the idempotent $yx$ and $[x]_\mathcal{R}$ the idempotent $xy$.
\end{proof}
\begin{corollary} \label{idempotentsHclass}
If $x,x'\in A$ are generalised inverses, then $[xx']_\mathcal{H} = [x]_\mathcal{R}\cap [x']_\mathcal{L}$ and $[x'x]_\mathcal{H} = [x']_\mathcal{R}\cap [x]_\mathcal{L}$.
\end{corollary}
\begin{proof}
From $x\greensR (xx')$ and $(xx')\greensL x'$, we get the first equality. The second is dual.
\end{proof}
We can depict the situation in the corollary as follows:
\[ \hspace{-8.4em} \exists a,b,c,d \in \widetilde{A}: \qquad \begin{tikzcd}[sep=large]
x \ar[r, maps to, shift left, "\rho_a"] \ar[d, maps to, shift left, "\lambda_c"] & xx' \ar[l, maps to, shift left, "\rho_b"] \ar[d, maps to, shift left, "\lambda_c"] \\
x'x \ar[u, maps to, shift left, "\lambda_d"] \ar[r, maps to, shift left, "\rho_a"] & x' \ar[u, maps to, shift left, "\lambda_d"] \ar[l, maps to, shift left, "\rho_b"]
\end{tikzcd} \]

So generalised inverses along one diagonal imply idempotents along the other. In fact, the other direction also holds:
\begin{proposition}
Let $\sSet{A, f}$ be an associative class and $e,f$ idempotents in $A$ such that $e\greensD f$. Then there exist $x\in [e]_\greensR\cap [f]_\greensL$ and $x'\in [e]_\greensL\cap [f]_\greensR$ such that
\begin{itemize}
\item $x,x'$ are generalised inverses;
\item $e = xx'$ and $f = x'x$.
\end{itemize}
\end{proposition}
\begin{proof}
Because $e\greensD f$, we can find $x,x'\in A$ such that
\[ \hspace{-8.4em} \exists a,b,c,d \in \widetilde{A}: \qquad \begin{tikzcd}[sep=large]
e \ar[r, maps to, shift left, "\rho_a"] \ar[d, maps to, shift left, "\lambda_c"] & x \ar[l, maps to, shift left, "\rho_b"] \ar[d, maps to, shift left, "\lambda_c"] \\
x' \ar[u, maps to, shift left, "\lambda_d"] \ar[r, maps to, shift left, "\rho_a"] & f \ar[u, maps to, shift left, "\lambda_d"] \ar[l, maps to, shift left, "\rho_b"]
\end{tikzcd} \]
Then
\begin{align*}
xx' &= (df)(fb) = dfb = e \\
x'x &= (ce)(ea) = cea = f \\
xx'x &= ex = eea = ea = x \\
x'xx' &= fx' = ffb = fb = x',
\end{align*}
which completes the proof.
\end{proof}
\begin{corollary}
Let $\sSet{A, f}$ be an associative class, $y\in A$ and $e,f$ idempotents in $A$. Then $e\greensD f$ \textup{if and only if} there exist generalised inverses $x,x'$ such that $e = xx'$ and $f = x'x$.
\end{corollary}
\begin{proof}
The direction $\Rightarrow$ follows from the proposition. The converse from \ref{greensRelationsRegularElements}.
\end{proof}

\begin{lemma}
Let $\sSet{A, f}$ be an associative class, $x\in A$. Then no $\greensH$-class contains more than one generalised inverse of $x$.
\end{lemma}
\begin{proof}
Assume $x$ has two generalised inverses, $x_1'$ and $x_2'$. From \ref{idempotentsHclass} and \ref{GreensTheoremCorollary} we get that $xx_1' = xx_2'$ and $x_1'x = x_2'x$. Thus
\[ x_1' = x_1'(xx_1') = x_1'xx_2' = (x_1'x)x_2' =  x_2'xx_2' = x_2'. \]
\end{proof}

\begin{proposition}
Let $\sSet{A, f}$ be an associative class and $x,y\in A$. Then $xy\in [x]_\greensR \cap [y]_\greensL$ \textup{if and only if} $[x]_\greensL \cap [y]_\greensR$ contains an idempotent.
\end{proposition}
\begin{proof}
First assume $[x]_\greensL \cap [y]_\greensR$ contains an idempotent $e$. We can depict the situation as
\[ \hspace{-8.4em} \exists a,b,c,d \in \widetilde{A}: \qquad \begin{tikzcd}[sep=large]
x \ar[r, maps to, shift left, "\rho_a"] \ar[d, maps to, shift left, "\lambda_c"] &  \ar[l, maps to, shift left, "\rho_b"] \ar[d, maps to, shift left, "\lambda_c"] \\
e \ar[u, maps to, shift left, "\lambda_d"] \ar[r, maps to, shift left, "\rho_a"] & y \ar[u, maps to, shift left, "\lambda_d"] \ar[l, maps to, shift left, "\rho_b"]
\end{tikzcd} \]
Then we can calculate
\[ xy = deea = dea = xa \in [x]_\greensR \cap [y]_\greensL. \]
Now assume $xy\in [x]_\greensR \cap [y]_\greensL$.
We can depict the situation as
\[ \hspace{-8.4em} \exists a,b,c,d \in \widetilde{A}: \qquad \begin{tikzcd}[sep=large]
x \ar[r, maps to, shift left, "\rho_a"] \ar[d, maps to, shift left, "\lambda_c"] & xy \ar[l, maps to, shift left, "\rho_b"] \ar[d, maps to, shift left, "\lambda_c"] \\
e \ar[u, maps to, shift left, "\lambda_d"] \ar[r, maps to, shift left, "\rho_a"] & y \ar[u, maps to, shift left, "\lambda_d"] \ar[l, maps to, shift left, "\rho_b"]
\end{tikzcd} \]
Now we need to show that $e$ is idempotent. Indeed, starting from the three other corners, we see that $e = cx$ and $e = yb$ and $e = c(xy)b = (cx)(yb) = ee$.
\end{proof}

\subsection{Commutation}
\begin{definition}
Let $\sSet{A,f}$ be an associative class and $x,y \in A$. We say $x$ and $y$ \udef{commute} if $f(x,y) = f(y,x)$. We write $x\commute y$.
\end{definition}


\subsubsection{Centraliser or commutant}
\begin{definition}
Let $\sSet{A,f}$ be an associative class and $B\subseteq A$ a subclass. The \udef{centraliser} or \udef{commutant} of $B$ is defined as $Z_A(B) \defeq B^\commute$.

In particular we define the \udef{centre} of $A$ as the centraliser of all of $A$: $Z_A \defeq Z_A(A)$.
\end{definition}
Thus
\[ Z_A(B) = \setbuilder{x\in A}{\forall b\in B:\;f(x,b) = f(b,x)}. \].

Note that taking the commuting forms a Galois connection. In particular $B \subseteq B^{\commute\commute}$.

\begin{lemma}
Let $\sSet{A,f}$ be an associative class and $B\subseteq A$. If $f$ is commutative, then $Z_A(B) = A$.
\end{lemma}

\begin{proposition}
Let $\sSet{A,f}$ be an associative class and $B\subseteq A$. Then $Z_A(B)$ is closed under $f$.
\end{proposition}
\begin{proof}
Take arbitrary $x,y\in Z_A(B)$. Take arbitrary $b\in B$. Then
\[ f(f(x,y),b) = f(x,f(y,b)) = f(x,f(b,y)) = f(f(x,b),y) = f(f(b,x),y) = f(b,f(x,y)), \]
which means that $f(x,y)\in Z_A(B)$.
\end{proof}

\subsection{Normaliser}
\begin{definition}
Let $\sSet{A,f}$ be an associative class and $B\subseteq A$ a subclass. An element $x\in A$ is said to \udef{normalise} $B$ if $f(x,B) = f(B,x)$.

The \udef{normaliser} of $B$ in $A$ is the set of all elements in $A$ that normalise $B$:
\[ N_A(B) \defeq \setbuilder{x\in A}{f(x,B) = f(B,x)}. \]
\end{definition}

\begin{proposition}
Let $\sSet{A,f}$ be an associative class and $B\subseteq A$. Then
\begin{enumerate}
\item $N_A(B)$ is closed under $f$;
\item $Z_A(B) \subseteq N_A(B)$;
\item $Z_A(\{a\}) = N_A(\{a\})$ for all $a\in A$.
\end{enumerate}
\end{proposition}
\begin{proof}
(1) Take arbitrary $x,y\in N_A(B)$. Then
\[ f(f(x,y),B) = f(x,f(y,B)) = f(x,f(B,y)) = f(f(x,B),y) = f(f(B,x),y) = f(B,f(x,y)), \]
which means that $f(x,y)\in N_A(B)$.

(2) If $f(x,b) = f(b,x)$ for all $b\in B$, then $f(x,B) = f(B,x)$.

(3) $f(x,\{a\}) = f(x,a)$ and $f(\{a\}, x) = f(a,x)$.
\end{proof}

\subsection{Composition of relations}
\begin{definition}
Let $\mathfrak{R}$ be the class of all relations, with absorbing element $\mathrm{NULL}$ adjoined.
\end{definition}

\begin{proposition}
Let $R,S \in \mathfrak{R}$ be relations. Then
\begin{enumerate}
\item $R \greensL S$ \textup{if and only if} $R$ and $S$ have the same image and cokernel;
\item $R \greensR S$ \textup{if and only if} $R$ and $S$ have the same preimage and kernel.
\end{enumerate}
\end{proposition}
\begin{proof}
TODO
\end{proof}

\chapter{The natural numbers}
\begin{definition}
A \udef{system of natural numbers} or \udef{Peano system} is a structured set $(N,(0,S)) = (N,0,S)$ which satisfies
\begin{enumerate}
\item $N$ is a set containing $0$;
\item $S$ is an injective function on the set $N$;
\item for all $n\in N: S(n) \neq 0$;
\item the induction principle: $\forall X\subset N$
\[ [(0\in X) \land (\forall n\in N: n\in X \implies Sn \in X)] \implies X = N. \]
\end{enumerate}
We call $0$ the \udef{zero} and $S$ the \udef{successor function}. These axioms are the \udef{Peano axioms}. An object $n\in N$ is called a \udef{successor} if $\exists m\in N: n = S(m)$.
\end{definition}
The most involved axiom is the induction principle. Note that it is not formulated in first order logic. Thus the Peano axioms are not subject to the Löwenheim-Skolem theorem and we can obtain a uniqueness result (despite the fact that, as we will see, $N$ is infinite).

\begin{lemma} \label{successor}
Let $(N,0,S)$ be a Peano system. Then every element $n\neq 0$ is a successor and for each $n\in N: S(n) \neq n$.
\end{lemma}
\begin{proof}
We wish to prove that the set
\[ X = \{n\in N\;|\; (n=0)\lor(\exists m\in N: n = S(m))\} \]
equals $N$. It is obvious that both $0\in X$ and $\forall n\in N: n\in X \implies Sn \in X$, so we can apply the induction principle to obtain $X=N$. The second claim then follows by injectivity.
\end{proof}
\section{Recursion and induction}
\subsection{Recursion}
\begin{theorem}[Recursion theorem]
Assume $(N,0,S)$ is a Peano system, $E$ is some set, $a\in E$, and $h:E\to E$ is some function.

There is exactly one function $f: N\to E$ which satisfies
\[ \begin{cases}
f(0) = a, \\
f(Sn) = h(f(n)) & (n\in N).
\end{cases} \]
\end{theorem}
Functions defined using this theorem are said to be defined \udef{recursively}.\footnote{The terms ``recursion'' and ``induction'' are often used synonymously. We
will usually distinguish recursive definitions from inductive
proofs (using the induction principle).}
\begin{proof}
We consider the set $\mathcal{A}$ of ``approximations'' of the function $f$:
\begin{align*}
\mathcal{A} = \{ p: X\to E \;|\; &(X\subset N)\land \\
&(0\in X)\;\land \\
&[\forall n\in N: Sn\in X \implies (n\in X \land p(Sn) = h(p(n)))] \}.
\end{align*}
In words we may say that $X$ is a downwards closed subset of $N$ and $p$ satisfies the recursion conditions. Some examples of elements of $\mathcal{A}$ include
\begin{align*}
&\{(0,a)\} \\
&\{(0,a), (S0, h(a))\} \\
&\{(0,a), (S0, h(a)), (SS0, h(h(a)))\} \\
&\;\;\ldots
\end{align*}

Any function $f$ with domain $N$ satisfying the recursion conditions, as in the theorem must be an element of $\mathcal{A}$. Thus to prove the theorem we need to prove that $\mathcal{A}$ contains exactly one function with domain $N$.

We prove this using a lemma.
\begin{lemma*}
For all $p,q\in \mathcal{A}$ and $n\in N$,
\[ n\in \dom(p)\cap\dom(q) \implies p(n) = q(n). \]
\end{lemma*}
\begin{proof}[Proof of lemma] \renewcommand{\qedsymbol}{$\dashv$ (Lemma)}
We need to prove that the set of $n\in N$ for which this is true,
\[ Y = \{n\in N \;|\; \forall p,q\in\mathcal{A}:\left[ n\in \dom(p)\cap\dom(q) \implies p(n) = q(n) \right] \} \]
is exactly $N$. We prove this with the principle of induction.

Clearly $0\in Y$, because every $p\in\mathcal{A}$ satisfies $p(0) = a$. Now let $n\in Y$ and $p,q\in\mathcal{A}$ such that $Sn\in \dom(p)\cap\dom(q)$. Then
\[ p(Sn) = h(p(n)) = h(q(n)) = q(Sn) \]
so $Sn\in Y$. By the induction principle $Y = N$.
\end{proof}
The lemma immediately implies there is at most one suitable function $f$ with domain $N$. We show such a function exists. Indeed it is given by $f = \bigcup \mathcal{A}$. Using the lemma we see this must be a function. It is not difficult to see that $f\in \mathcal{A}$. We then just need to verify that $\dom(f) = N$. This is again an application of the induction principle: $0\in \dom(f)$ and if $n\in \dom(f)$, then there exists some $p\in\mathcal{A}$ with $n\in\dom(p)$ and we have
\[ q = p\cup \{(Sn, h(p(n)))\}\in \mathcal{A}, \]
so that $Sn\in\dom(q)\subseteq \dom(f)$.
\end{proof}
\begin{corollary}[Recursion with parameters]
Let $(N,0,S)$ be a Peano system, $Y,E$ sets and functions
\[ g: Y\to E,\qquad h:E\times Y\to E. \]

There is exactly one function $f: N\times Y\to E$ which satisfies
\[ \begin{cases}
f(0,y) = g(y), & (y\in Y) \\
f(Sn,y) = h(f(n,y),y) & (y\in Y, n\in N).
\end{cases} \]
\end{corollary}
\begin{proof}
For any $y\in Y$ we can apply the normal recursion theorem the obtain a function $f_y: N\to E$. Then set $f(n,y) = f_y(n)$.
\end{proof}
\begin{corollary}[Recursion with the argument as parameter]
Let $(N,0,S)$ be a Peano system, $E$ a set, $a\in E$ and $h: E\times N\to E$ a function.

There is exactly one function $f: N\to E$ which satisfies
\[ \begin{cases}
f(0) = a, \\
f(Sn) = h(f(n),n) & (n\in N).
\end{cases} \]
\end{corollary}
\begin{proof}
Consider the function $\phi: N \to N\times E$ which returns both the required result and the successor of its argument. As the argument is returned by the function, it can be used in the normal recursion theorem if we replace $E$ by $N\times E$. Then just define $f(n)$ to be the second component of $\phi(n)$.
\end{proof}
More version of recursion can be cooked up, such as
\begin{itemize}
\item \udef{complete recursion} where at each step of the recursion. $h$ is passed not just the latest function value, but the whole partial function created so far:
\[ h: (\N \not\to E)\to E. \]
\item recursion with both the argument as parameter and other parameters;
\item simultaneous recursion that defines two functions $f_1,f_2$ and the next step depends on the current step of both functions:
\[ f_1(Sn) = h_1(f_1(n),f_2(n)) \qquad \text{and}\qquad f_2(Sn) = h_2(f_1(n),f_2(n)). \]
\end{itemize}

\subsection{Induction}
\begin{lemma}[Mathematical induction]
We can prove a definite statement $P(n)$ holds for all $n\in \N$ by proving
\begin{enumerate}
\item the \udef{base case} $P(0)$; and
\item the \udef{induction step} $\forall k\in \N: P(k)\implies P(Sk)$.
\end{enumerate}
We call ``$\forall k\in \N: P(k)$'' the \udef{induction hypothesis}.
\end{lemma}
\begin{proof}
Consider the set
\[ X = \{ n\in \N\;|\; P(n) \}. \]
The statement $P(n)$ holds for all $n\in \N$ if $X=\N$. Using the base case we have $0\in X$ and using the induction step we have
\[ \forall k\in\N: k\in X \implies Sn\in X. \]
By induction we conclude $X=\N$ and thus $P(n)$ for all $n\in \N$.
\end{proof}
\begin{corollary}
We can prove a definite statement $P(n)$ holds for all $n\geq n_0$ by proving
\begin{enumerate}
\item the base case $P(n_0)$; and
\item the induction step $\forall k\geq n_0: P(k)\implies P(Sk)$.
\end{enumerate}
\end{corollary}

\begin{lemma}[Complete (strong) induction]
We can prove a definite statement $P(n)$ holds for all $n\in \N$ by proving
\begin{enumerate}
\item the base case $P(0)$; and
\item the \udef{strong induction step} $\forall k\in \N: [\forall j\leq k:  P(j)] \implies P(Sk)$.
\end{enumerate}
\end{lemma}
The definition of $\leq$ follows later.
\begin{proof}
A strong inductive proof (i.e.\ proving these two steps) is a normal inductive proof of
\[ Q(n) \defequiv \forall m\leq n: P(m) \]
for all $n$. Clearly $Q(n)$ implies $P(n)$.
\end{proof}
Conversely we can prove nothing new using strong induction: every strong inductive proof is in particular also a (weak) inductive one.

\section{Existence and uniqueness of the natural numbers}
\begin{theorem} \label{existenceUniquenessPeano}
There exists a Peano system $\sSet{N,0,S}$. For any two Peano systems $\sSet{N_1,0_1,S_1}$ and
$(N_2,0_2,S_2)$ there exists a unique bijection $\pi: N_1 \to N_2$ such that
\[ \begin{cases}
\pi(0_1) = 0_2, \\
\pi(S_1n) = S_2\pi(n) & (n\in N_1).
\end{cases} \]
\end{theorem}
Such a bijection is called an \udef{isomorphism} of Peano systems. The theorem says any two Peano systems are (uniquely) isomorphic. So all Peano systems are essentially the same. We denote the all in the same way: $\sSet{N,0,S}$.
\begin{proof}
We first prove existence and then uniqueness.
\begin{itemize}
\item[\textbf{Existence}] By the axiom of infinity we have the set $I$ with properties
\[ \emptyset\in I \qquad \text{and}\qquad \forall n\in I: \{n\}\in I. \]
We then define
\[ \mathcal{J} = \{ X\subseteq I\;|\;[\emptyset\in X] \land [\forall n\in X: \{n\} \in X] \} \]
Then we set
\[ \N = \bigcap \mathcal{J},\qquad 0 = \emptyset, \qquad S: \N\to\N: n\mapsto \{n\}. \]
Note that by constructing $\mathcal{J}$ and then taking the intersection, we have removed possible other elements of $I$ and are left with
\[ \N = \{ \emptyset, \{\emptyset\}, \{\{\emptyset\}\}, \{\{\{\emptyset\}\}\}, \ldots \}. \]

With these definitions verifying the Peano axioms is not hard. The induction principle follows from the intersection.
\item[\textbf{Uniqueness}] By the recursion theorem we can a unique function $\pi$ satisfying the identities. What remains the be shown is that $\pi$ is bijective.
\begin{itemize}
\item[Surjectivity] We need to show that $\pi[N_1] = N_2$. To that end we use the induction principle on $(N_2,0_2, S_2)$.
\begin{itemize}
\item Obviously $0_2\in \pi[N_1]$, since $0_2=\pi(0_1)$.
\item Let $m\in\pi[N_1]$. Then $\exists n\in N_1: m=\pi(n)$. Applying $S_2$ gives
\[ S_2m = S_2\pi(n) = \pi(S_1 n). \]
This implies $S_2m\in \pi[N_1]$.
\end{itemize}
The induction principle then gives $\pi[N_1] = N_2$.
\item[Injectivity] We verify that the set
\[ X = \{ n\in N_1\;|\; \forall m\in N_1: \pi(m)=\pi(n) \implies m=n \} \]
equals the set $N_1$. This is again done using the induction principle.
\begin{itemize}
\item Firstly, if $m\neq 0_1$, then $m=S_1m'$ for some $m'\in N_1$, by lemma \ref{successor}. This implies
\[ \pi(m) = \pi(S_1m') = S_2\pi(m') \neq 0_2 \]
and so $0_1\in X$.
\item We need to show that, if $n\in X$,
\[ \pi(m) = \pi(S_1n) \implies m = S_1n. \]
Assume the antecendent. By hypothesis $\pi(m) = \pi(S_1n) = S_2\pi(n) \neq 0_2$ and thus $m\neq 0_1$. Again by lemma \ref{successor} $\exists m'\in N_1: m = S_1m'$. So
\[ S_2\pi(n) = \pi(m) = \pi(S_1m') = S_2\pi(m') \]
implying $n=m'$ and thus $m=S_1m'=S_1n$.
\end{itemize}
The induction principle then gives $X=N_1$ and thus the surjectivity of $\pi$.
\end{itemize}
\end{itemize}
\end{proof}

\subsection{Zermelo ordinals}
The natural numbers constructed in the existence proof are called \udef{Zermelo ordinals}. They are defined by
\[ 0 = \emptyset \qquad \text{and} \qquad S(a) = \{a\}. \]
Then
\begin{align*}
0 &= \emptyset \\
1 &= \{0\} = \{\emptyset\} \\
2 &= \{1\} =  \{\{\emptyset\}\} \\
3 &= \{2\} = \{\{\{\emptyset\}\}\} \\
&\hdots
\end{align*}
\subsection{(Finite) Von Neumann ordinals}
The \udef{Von Neumann ordinals} are an alternate construction. They are defined by
\[ 0 = \emptyset \qquad \text{and} \qquad S(a) = a\cup\{a\}. \]
Then
\begin{align*}
0 &= \emptyset \\
1 &= 0\cup \{0\} = \{\emptyset\} \\
2 &= 1\cup \{1\} = \{0,1\} = \{\emptyset, \{\emptyset\}\} \\
3 &= 2\cup \{2\} = \{0,1,2\} = \{ \emptyset, \{\emptyset\}, \{\emptyset, \{\emptyset\}\} \} \\
&\hdots
\end{align*}
Unlike the Zermelo ordinals, the Von Neumann ordinals can be readily generalised to infinite ordinals (see later).
\section{Operations and relations on natural numbers}
We can use the recursion theorem to define addition and multiplication.
\begin{definition}
The \udef{addition function} $+:\N\times \N \to \N$ on the natural numbers is defined by the recursion (with parameter)
\[ \begin{cases}
+(0,n) = n \\
+(Sm, n) = S(m+n)
\end{cases}. \]
The \udef{multiplication function} $\cdot:\N\times \N \to \N$ on the natural numbers is defined by the recursion (with parameter)
\[ \begin{cases}
\cdot(0,n) = 0 \\
\cdot(Sm, n) = (m\cdot n)+n
\end{cases}. \]
We will usually write $m+n$ and $m\cdot n$ or $mn$ instead of $+(m,n)$ and $\cdot(m,n)$.
\end{definition}

\begin{proposition}
Addition has the following properties:
\begin{enumerate}
\item It is associative: $(k+n)+m = k+(n+m)$;
\item $0$ is a neutral element: $0+n = n$ and $n+0 = n$;
\item for all $m,n\in \N: Sm+n = m+Sn$;
\item it is commutative: $n+m = m+n$.
\end{enumerate}
\end{proposition}
\begin{proof}
All are proven by induction on $m$. The proofs of later claims make use of earlier ones.
\end{proof}
\begin{lemma}
For all $n\in \N$, the function $\N\to \N: s\mapsto n+s$ is 1-1, so
\[n+s = n+t \implies s=t.\]
\end{lemma}

\begin{definition}
The binary relation $\leq$ defined by
\[ n \leq m \defequiv \exists s\in \N: n+s = m \]
is called the ordering of the natural numbers.

We abbreviate $\neg(m\leq n)$ by $n < m$.
\end{definition}

\begin{lemma} \label{orderingN}
Let $\N$ be ordered by $\leq$. Then $\forall n\in\N$
\begin{enumerate}
\item $0\leq n$;
\item there is no $m$ such that $n<m<n+1$.
\end{enumerate}
\end{lemma}
TODO proof

\begin{proposition} \label{proposition:wellOrderingN}
The endorelation $\leq$ on the natural numbers has the following properties:
\begin{enumerate}
\item it is transitive;
\item it is reflexive;
\item it is antisymmetric;
\item it is connex (i.e.\ any two numbers are comparable);
\item every non-empty subset $S$ of $\N$ contains an element $x\in S$ that is left related to all elements of $S$: $\forall y\in S: x\leq y$.
\end{enumerate}
These are exactly the properties of a well-ordering.
\end{proposition}
\begin{proof}
Take a non-empty set $S\subset \N$ and define the set
\[ L = \setbuilder{n\in\N}{\forall m\in S: n\leq m}. \]
We need to show that $L\cap S$ is non-empty. Assume, towards a contradiction, that $L\cap S = \emptyset$.
Now
\begin{itemize}
\item $0\in L$.
\item If $n\in L$, then $n+1\in L$. Indeed if $n+1 \notin L$, then there exists a $z\in S$ such that $n\leq z<n+1$. But by \ref{orderingN} this would mean that $z=n$ and $n\in L\cap S = \emptyset$.
\end{itemize}
By induction $L=\N$, so $S = L\cap S = \emptyset$. This is a contradiction.
\end{proof}


The element $x$ in the last property is called the least element or minimum of $S$. It is unique by antisymmetry and denoted $\min(S)$.

Some subsets $S$ of $\N$ contain an element $x\in S$ that is right related to all elements of $S$: $\forall y\in S: y\leq x$. If it exists, it is called the greatest element or maximum of $S$. It is again unique by antisymmetry and denoted $\max(S)$.

\begin{lemma} \label{naturalNumbersInequalityInclusion}
Let $m,n\in \N$. Then the following are equivalent:
\begin{enumerate}
\item $n\leq m$; 
\item $\interval[co]{0,n} \subseteq \interval[co]{0,m}$.
\end{enumerate}
\end{lemma}
\begin{proof}
By \ref{orderingInitialSegments}. TODO superfluous?
\end{proof}
TODO: this is Dedekind / Yoneda principle??

\subsection{Intervals}
\begin{definition}
Let $n,m\in \N$. Then
\begin{itemize}
\item $n:m \defeq \interval[co]{n,m}$;
\item $:n \defeq 0:m$;
\item $n: \defeq \interval[co]{n,\infty} = \setbuilder{k\in \N}{k\geq n}$.
\end{itemize}
\end{definition}

\subsection{Enumerating pairs}
It will turn out to be useful to have a bijection
\[ \rho: \N\times \N \to \N. \]
We give two examples, the first due to Gödel, the second due to Zermelo.
\begin{lemma} \label{pairEnumeration}
The functions
\[\rho_1 : \N\times\N \to \N: (m,n)\mapsto \frac{(m+n)(m+n+1)}{2}+m \]
and
\[ \rho_1 : \N\times\N \to \N: (m,n)\mapsto \begin{cases}
(m+1)^2-1 & (m=n) \\
n^2 + m & (m<n) \\
m^2+m+n & (m>n)
\end{cases} \]
are bijections.
\end{lemma}
TODO pictures!

TODO: Stern-Brocot -> all fractions listed in most simplified form without repeating

\section{Sequences }
TODO: lower!
\begin{definition}
A \udef{sequence} in a set $A$ is a function
\[ x:I\subseteq\N\to A. \]
The domain $I$ is called the \udef{index set}. It is usually well-ordered.

We usually write $x_i$ instead of $x(i)$ and we denote the function $x$ as $\seq{x_i}_{i\in I}$ (or $\seq{x_i}$ if the index is clear).

The set of all sequences in $A$ with domain $I$ is denoted $A^I$.

\begin{itemize}
\item If $u \subseteq v$, we say $u$ is a \udef{subsequence} of $v$.
\item If $u$ is a subsequence of $v$ and the index set of $u$ is downwards closed in the index set of $v$, then we call $u$ an \udef{initial segment} of $v$, denoted $u \initSeq v$.
\item If $u$ is a subsequence of $v$ and the index set of $u$ is upwards closed in the index set of $v$, then we call $u$ a \udef{tail} of $v$.
\end{itemize}
Note: often the notion of subsequence is generalised by allowing the index set of $u$ to be order-embedded in the index set of $v$.
\end{definition}
The choice of the index set is irrelevant up to a bijection. Thus which theorems hold depends on the cardinality of the index set (see later).


\subsection{Inverse and complementary sequences}
TODO: Hofstadter Figure-Figure sequence
\subsubsection{Inverse sequences}
\url{https://www.math.hkust.edu.hk/excalibur/v4_n1.pdf}
\url{https://www.jstor.org/stable/2308078}
\begin{definition}
Let $s:\N\to\N$ be a non-decreasing sequence of natural numbers.
\end{definition}


\subsection{Tuples}
\begin{definition}
Let $\{A,B,C \ldots \}$ be a finite set of classes and $a\in A, b\in B, c\in C \ldots$. We define
\begin{itemize}
\item $(a,b,c,\ldots) \defeq (a, (b,(c,\ldots)))$;
\item $A\times B\times C\times\ldots \defeq A\times(B\times (C\times \ldots))$.
\end{itemize}
\end{definition}
Clearly $(a,b,c,\ldots)\in A\times B\times C\times\ldots$.

TODO link tuples - finite sequences.

\subsubsection{Transposition of tuples}
\begin{definition}
Let $\mathcal{E}$ be the class of all elements and $x,X$ elements. Then we define the function $^\ttransp: \mathcal{E}\to \mathcal{E}$ recursively by
\[ x^\ttransp \defeq \begin{cases}
((a,c), (b,d)^t) & \exists a,b,c,d:\; x = ((a,b), (c,d)) \\
\setbuilder{p^\ttransp}{p\in x} & \text{$x$ is a set} \\
x & \text{$x$ is not of these forms.}
\end{cases} \]
\end{definition}
TODO: why valid recursion??
\begin{lemma}
Let $X$ be a set. Then
\[ X^\ttransp \defeq \begin{cases}
(A\times C) \times (B\times D)^t & \exists A,B,C,D:\; X = (A\times B)\times (C\times D) \\
x & \text{$x$ is not of this form.}
\end{cases} \]
\end{lemma}
\begin{proof}
TODO!
\end{proof}

\subsubsection{Tuples of functions}
\begin{definition}
Let $f: A\to B$ and $g: X\to Y$ be functions. Then the \udef{tuple function} of $f$ and $g$ is the function
\[ (f,g): A\times X \to G\times Y: (a,x) \mapsto (f(a), g(x)). \]

In particular if $f=g$, we say the tuple function is the \udef{pointwise application} of $f$. We write $f$ instead of $(f,f)$.
\end{definition}

TODO extend??

\subsubsection{Functions on finite sequences}
Let $f: A^{(m)}\to A^{(n)}$. We write
\[ f_k(a) = f(a)(k) \]
to denote the $k^\text{th}$ component of $f(a)$.
\chapter{Comparing sets}
\section{Equinumerosity: comparing sets in size}
\begin{definition}
Two sets $A,B$ are \udef{equinumerous} or \udef{equal in cardinality} if there exists an bijection between them:
\[ A =_c B \defequiv \exists f: A\twoheadrightarrowtail B. \]
The set $A$ is \udef{less than or equal to $B$ in size} if it is equinumerous with some subset of $B$:
\[ A \leq_c B \defequiv \exists C: C\subseteq B \land A=_c C. \]
\end{definition}
We might think $=_c$ and $\leq_c$ are relations, but there is no set of all sets, so they are not defined on any set. If we restrict $A,B$ to be subsets of a set $U$, they become relations on $\powerset(U)$.
\begin{proposition}
For all sets $A,B,C$:
\begin{enumerate}
\item $A =_c A$;
\item if $A=_c B$, then $B =_c A$;
\item if $A=_c B$ and $B =_c C$, then $A =_c C$.
\end{enumerate}
Restricted to $U$, equinumerosity is an equivalence relation on $\powerset(U)$.
\end{proposition}

Note that $\emptyset =_c \emptyset$, because $\emptyset$ is a function $\emptyset\to \emptyset$ which is bijective.

\begin{lemma} \label{welldefinedCardinalArithemtic}
Let $A_1,A_2,B_1,B_2$ be sets such that $A_1=_c A_2$ and $B_1=_cB_2$. Then
\begin{enumerate}
\item $A_1 \sqcup B_1 =_c A_2\sqcup B_2$;
\item $A_1 \times B_1 =_c A_2\times B_2$;
\item $(A_1\to B_1) =_c (A_2\to B_2)$.
\end{enumerate}
\end{lemma}

\begin{proposition} \label{injectiveSurjectiveMappingCardinality}
Let $A,B$ be sets. Then the following are equivalent
\begin{enumerate}
\item $A\leq_c B$;
\item $\exists f: A\rightarrowtail B$;
\item $\exists g: B\twoheadrightarrow A$.
\end{enumerate}
\end{proposition}
\begin{proof}
TODO
\end{proof}
\begin{corollary}
If $A\subseteq B$, then $A\leq_c B$.
\end{corollary}
This follows from the existence of the inclusion map $A \hookrightarrow B$.

\begin{proposition}
For all sets $A,B,C$
\begin{enumerate}
\item $A\leq_c A$;
\item if $A\leq_c B$ and $B\leq_c C$, then $A\leq_c C$.
\end{enumerate}
Restricted to $U$, $\leq_c$ is a preorder on $\powerset(U)$.
\end{proposition}
Even restricting to $U$, $\leq_c$ is not anti-symmetric and so not generally a partial order, but the Schröder-Bernstein theorem (theorem \ref{SchroederBernstein}) does give a result in this vein (with $=_c$ instead of $=$!).

\begin{theorem}[Schröder-Bernstein] \label{SchroederBernstein}
Let $A, B$ be sets. If there exist injective functions $f: A\rightarrowtail B$ and $g: B\rightarrowtail A$, then there exists a bijective function $h: A\twoheadrightarrowtail B$.
\end{theorem}
\begin{proof}
In general $f[A]\subset B$ and $g[B]\subset A$. First we identify subsets $A^*\subset A$ and $B^*\subset B$ such that $f[A^*] = A^*$ and $g[B^*] = B$. To that end we define $A_n,B_n$ recursively
\[ \begin{cases}
A_0 = A \\ A_{n+1} = (g\circ f)[A_n],
\end{cases} \qquad \begin{cases}
B_0 = B \\ B_{n+1} = (f\circ g)[B_n].
\end{cases} \]
Then we can take
\[ A^* = \bigcap_{n=0}^\infty A_n \qquad B^* = \bigcap_{n=0}^\infty B_n. \]
By induction we can see that we have chains of inclusions:
\[ A_n\subseteq g[B_n] \subseteq A_{n+1}, \qquad B_n\subseteq g[A_n] \subseteq B_{n+1}. \]
Then $B^* = \bigcup_{n=0}^\infty f[A_n]$ by
\[ B^* = \bigcup_{n=0}^\infty B_n \subseteq \bigcup_{n=0}^\infty f[A_n]\subseteq \bigcup_{n=0}^\infty B_{n+1} = B^{*}. \]
So, because $f$ is injective, we have
\[ f[A^*] = f[\bigcup_{n=0}^\infty A_n] = \bigcup_{n=0}^\infty f[A_n] = B^* \]
as required. Similarly $g[B^*] = A^*$.

Then notice that
\begin{align*}
A &= A^* \cup \left[ \bigcup_{n=0}^\infty (A_n\setminus g[B_n])\cup (g[B_n]\setminus A_{n+1}) \right] \\
B &= B^* \cup \left[ \bigcup_{n=0}^\infty (B_n\setminus f[A_n])\cup (f[A_n]\setminus B_{n+1}) \right] \\
\end{align*}
and these are partitions of $A$ and $B$.

Finally we can construct the bijection $\pi: A\twoheadrightarrowtail B$
\[ \pi(x) = \begin{cases}
f(x) & x\in A^* \lor \exists n\in \N: x\in (A_n\setminus g[B_n]) \\
g^{-1}(x) & x\notin A^* \land \exists n\in \N: x\in (g[B_n]\setminus A_{n+1}).
\end{cases} \]
\end{proof}
We can also give a more condensed proof TODO write proof (this is not yet a proof!) and express this as fixed point?:
\begin{proof}
Define $Q = f[A]\setminus (g\circ f)[A]$,
\[ \mathcal{J} = \{ X\in \powerset(A)\;|\; Q\cup (g\circ f)[X] \subseteq X \} \]
and $T = \bigcap \mathcal{J}$. We can show that $T = Q\cup (g\circ f)[T]$. Then
\[ f[A] = Q\cup (g\circ f)[A] = Q\cup (g\circ f)[T]\cup [(g\circ f)[A]\setminus (g\circ f)[T]] = T\cup [(g\circ f)[A]\setminus (g\circ f)[T]] \]
\end{proof}
\begin{corollary}
Let $A,B$ be sets. Then $A =_c B$ \textup{if and only if} $A\leq_c B$ and $B\leq_c A$.
\end{corollary}

\begin{theorem}[Cantor's theorem] \label{Cantor}
For every set $A$,
\[ A <_c \powerset(A) \]
i.e.\ $A\leq_c \powerset(A)$ but $A\neq_c \powerset(A)$.
\end{theorem}
\begin{proof}
That $A\leq_c \powerset(A)$ follows from the existence of the injection $A\to \powerset(A): x\mapsto \{x\}$.

Assume, towards a contradiction, that there exists a surjection
\[ \pi: A\twoheadrightarrow \powerset(A), \]
and define the set $B = \{x \in A \;|\; x \notin \pi(x)\}$.
Now $B$ is a subset of $A$ and $\pi$ is a surjection, so there must exist some $b \in A$ such that $B = \pi(b)$. We get
\[ b \in B \iff b \notin B. \]
A contradiction.
\end{proof}


\subsection{Cantor's paradox}
We can use Cantor's theorem to disprove the existence of a set of all sets. It is similar to Russell's paradox, but predates it.

Assume there was a set $V$ of all sets. Consider the power set $\powerset(V)$. Since every element of $\powerset(V)$ is a set, $\powerset(V) \subseteq V$ and thus $\powerset(V) \leq_c V$. This contradicts Cantor's theorem, yielding a paradox.

Looking at the proof of Cantor's theorem, we see a strong analogy with Russell's paradox.

\subsection{Countability}
\begin{definition}
Let $A$ be a set. We say
\begin{itemize}
\item $A$ is \udef{finite} if there exists some $n\in \N$ such that
\[ A =_c \interval[co]{0, n} = \setbuilder{i\in \N}{i<n}, \]
otherwise $A$ is \udef{infinite}.
\item $A$ is \udef{countable} if $A\leq_c\N$, otherwise it is \udef{uncountable}.
\end{itemize}
\end{definition}
Notice the link between $\interval[co]{0,n}$ and the Von Neumann ordinals.

Also $\interval[co]{0,n} \leq_c \N$ for all $n\in \N$.

If we have a hundred pigeons, a hundred pigeonholes and are not allowed to put two pigeons in the same hole, all pigeonholes must be filled. This principle is formalised in the pigeonhole principle.
\begin{theorem}[Pigeonhole principle] \label{pigeonholePrinciple}
Every injection $f:A\rightarrowtail A$ of a finite set into itself is also a surjection, i.e.\ $f[A] = A$.
\end{theorem}
\begin{proof}
It is enough to prove that every injection $g: \interval[co]{0,m} \inj \interval[co]{0,m}$ is surjective. (By finiteness $A$ is bijectively related to some $\interval[co]{0,m}$.)
The proof is by induction on $m$ to prove the assertion:
\[ \forall g: \left(g: \interval[co]{0,m} \rightarrowtail \interval[co]{0,m}\right) \implies g^{\imf}(\interval[co]{0,m}) = \interval[co]{0,m}. \]
\begin{itemize}[leftmargin=2.5cm]
\item[\textbf{Basis step}] If $m=1$, there is only one such function, namely $0\mapsto 0$, which is bijective.
\item[\textbf{Induction step}] It is easy to see that $\interval[co]{0,Sm} = \interval[co]{0,m}\cup \{m\}$. Taking a $g: \interval[co]{0,m} \rightarrowtail \interval[co]{0,m}$ and letting $h = g\setminus \{(m,g(m))\}$ (which is still injective), we have three cases:
\begin{itemize}[leftmargin=2cm]
\item[$\boxed{m\notin \im(g)}$] By the induction hypothesis, $h^\imf(\interval[co]{0,m}) = \interval[co]{0,m}$. Because also $g(m)\in \interval[co]{0,m}$, $g$ would not be injective, making this case impossible.
\item[$\boxed{g(m) = m}$] Now $h^\imf(\interval[co]{0,m}) = \interval[co]{0,m}$ and $g(m)=m$, so $g^\imf(\interval[co]{0,Sm}) = \interval[co]{0,m}\cup \{m\} = \interval[co]{0,Sm}$. Thus $g$ is a surjection.
\item[$\boxed{\exists u<m: g(u) = m}$] In this case there must be a $v<m$ such that $g(m) = v$. Now we apply the induction hypothesis to
\[ h': \interval[co]{0,m} \to \interval[co]{0,m}: i\mapsto \begin{cases}
g(i) & i\neq u \\
v & i=u.
\end{cases} \]
So $h'$ is surjective and $g^{\imf}(\interval[co]{0,m}) = \interval[co]{0,m}\cup \{m\} = \interval[co]{0,Sm}$, so $g$ is surjective.
\end{itemize}
\end{itemize}
\end{proof}
\begin{corollary}
The set $\N$ of natural numbers is infinite.
\end{corollary}
\begin{proof}
The function $\N\to \N\setminus\{0\}: n\mapsto Sn$ is injective.
\end{proof}
\begin{corollary} \label{infiniteComparisonWithN}
Let $A$ be a set. If $\N\leq_c A$, then $A$ is infinite.

The converse holds assuming countable choice (TODO only weaker assumption necessary).
\end{corollary}
\begin{proof}
We prove the contraposition. Assume $A$ is finite, i.e.\ $A =_c\interval[co]{0,n}$. Assume, towards a contradiction, that $\N\leq_c A$, then $\N \leq_c \interval[co]{0,n}$. As $\interval[co]{0,n} \leq_c \N$, we have $\N =_c \interval[co]{0,n}$, which is the contradiction.

TODO prove converse using countable choice.
\end{proof}
\begin{corollary} \label{sizeOfIntervalsInN}
Let $m,n\in \N$ and $k\leq m$. Then
\begin{enumerate}
\item $\interval[co]{0,m} =_c \interval[co]{0,n}$ \textup{if and only if} $m=n$;
\item $\interval[co]{k,m} =_c \interval[co]{0,n}$ \textup{if and only if} $m-k=n$.
\end{enumerate}
\end{corollary}
\begin{proof}
(1) The direction $\Leftarrow$ is immediate. Now assume $\interval[co]{0,m} =_c \interval[co]{0,n}$, such that there exists a bijection $f: \interval[co]{0,m} \bij \interval[co]{0,n}$. WLOG we may assume $n \leq m$, so that $\interval[co]{0,n} \subseteq \interval[co]{0,m}$ (by \ref{naturalNumbersInequalityInclusion}) and we have the inclusion $\iota: \interval[co]{0,n} \hookrightarrow \interval[co]{0,m}$. Now $\iota \circ f$ is injective, so bijective by the pigeonhole principle. Thus $\iota \circ f\circ f^{-1} = \iota$ is also bijective, which means that $\interval[co]{0,n} = \interval[co]{0,m}$ and thus $m=n$ by \ref{naturalNumbersInequalityInclusion}.

(2) The function $n\mapsto n-k$ is a bijection. Then we can use (1).
\end{proof}
Note that it is important that the inclusion is a bijection, not just that there exists some bijection: there exist strict subsets that are equinumerous to their supersets. (In fact this is the definition of Dedekind infinity). We always have that a surjective inclusion is an identity.
\begin{corollary} \label{finiteSetNumberOfElements}
For each finite set $A$, there exists exactly one $n\in\N$ such that $A =_c \interval[co]{0,n}$. We call this $n$ the number of elements of $A$, $\#(A)$.
\end{corollary}
In other words, $A =_c \interval[co]{0,\#A}$.
\begin{corollary}
Every surjection $f:A\surj A$ of a finite set into itself is also an injection.
\end{corollary}
\begin{proof}
Let $\#(A) = n$ and $f$ be surjective. Fix a bijection $\pi: \interval[co]{0,n} \bij A$. Construct the sets
\[ X = \setbuilder{j \in \N}{\exists i<j: f(\pi(i)) = f(\pi(j))}\qquad \text{and}\qquad Y = \setbuilder{j \in \N}{\forall i<j: f(\pi(i)) \neq f(\pi(j)) }. \]
Then $X\cup Y = \interval[co]{0,n}$. Now $f': A/\sim \to A$ is a bijection and we have a bijection $A/\sim \leftrightarrow Y$ given by
\[ [a] \quad \leftrightarrow \quad \min\setbuilder{i\in \N}{\pi(i)\in [a]}, \]
so we have a bijection $A\leftrightarrow Y$ and thus $\#(Y) = n$.
Assume $f$ not injective, in which case $X$ is not empty and let $\#(X) = m \neq 0$.

Now because $X,Y$ disjunct, $\#(X\cup Y) = \#(X)+\#(Y) = m+n >n$, by the uniqueness of $\#$. This is a contradiction.
\end{proof}

\begin{lemma} \label{sizeEstimationByIntervalInclusion}
Let $A\subseteq \N$. Then
\begin{enumerate}
\item $A\subseteq \interval[co]{0,n} \implies \#(A) \leq n$;
\item $\#(A) > n \implies \exists k \geq n: k\in A$.
\end{enumerate}
\end{lemma}
\begin{proof}
(1) TODO (review definition of order on $\N$)

(2) Contraposition of (1).
\end{proof}

\begin{proposition} \label{countableEquivalents}
The following are equivalent for every set $A$:
\begin{enumerate}
\item $A$ is countable;
\item there is a surjection $\pi:\N\twoheadrightarrow A$;
\item $A$ is finite or equinumerous with $\N$.
\end{enumerate}
We call such a $\pi$ an \udef{enumeration}. Then
\[ A = \pi^\imf(\N) = \{\pi(0),\pi(1),\pi(2) \ldots \}. \]
\end{proposition}
\begin{proof}
The proof is cyclic:
\begin{enumerate}[leftmargin=2cm]
\item[$\boxed{(1)\rightarrow (2)}$] If $A = \emptyset$, any $\pi:\N\to A$ is surjective. Assume $A\neq \emptyset$ and choose an $a_0\in A$. By lemma \ref{injectiveSurjectiveMappingCardinality}, there is an $f:A\rightarrowtail$. Define
\[ \pi: \N\to A: i\mapsto \begin{cases}
a_0 & (i\notin f[A]) \\ f^{-1}(i) & (i\in f[A]).
\end{cases} \]
\item[$\boxed{(2)\rightarrow (3)}$] Assume 2, so we have a surjective $\pi:\N\twoheadrightarrow A$. We need to prove that if $A$ is not finite, it is equinumerous with $\N$. Define the function $f:\N\to A$ recursively by
\[ \begin{cases}
f(0) = \pi(0) \\
f(n+1) = \pi\left(\;\text{the least $m$ such that}\; \pi(m)\neq \{f(0),\ldots,f(n)\}\right).
\end{cases} \]
Note that because $A$ is infinite, the set $\setbuilder{m\in\N}{\pi(m)\notin \{ f(0),\ldots,f(n) \}}$ is not empty. Due to the well-ordering on $\N$, every such set has a least element.

Then $f$ is a bijection. Injectivity is obvious. Surjectivity follows from the fact that $\forall n\in \N: \pi(n)\in f[\N]$.
\item[$\boxed{(3)\rightarrow (1)}$] All intervals $[0, n[$ are subsets of $\N$ and $\N$ is a subset of $\N$.
\end{enumerate}
\end{proof}
\begin{corollary} \label{finiteComparisonWithInterval}
Let $A$ be a set and $n\in \N$. If $A\leq_c (0:n)$, then $A$ is finite.
\end{corollary}
\begin{proof}
We have $A \leq_c (0:n) \leq_c \N$, so $A$ is countable. Then either $A$ is finite or $A =_c \N$. We show that the second option is impossible: then we would have $\N \leq_c A \leq_c \N$, so $(0:n) =_c \N$, i.e.\ $\N$ is finite. But $\N$ is infinite.
\end{proof}
\begin{corollary}
There exists no set $A$ such that $\powerset(A) =_c \N$.
\end{corollary}
\begin{proof}
By Cantor's theorem.
\end{proof}

\begin{lemma}
We have the following characterisation of countable infinity:
\begin{align*}
A =_c \N \quad \Leftrightarrow \quad (\exists \mathcal{E})&[A = \bigcup \mathcal{E} \\
&\& \emptyset \in \mathcal{E} \\
&\& (\forall u\in \mathcal{E})(\exists ! y \notin u)[u\cup \{ y \}\in \mathcal{E}] \\
&\& (\forall Z)[[\emptyset \in Z \& (\forall u \in Z)(\exists ! y \notin u)u\cup \{y\} \in Z \cap \mathcal{E} \Rightarrow \mathcal{E}\subseteq Z]].
\end{align*}
This is directly in terms of the membership relation with no appeal to the defined notions of $\N$ and function.
\end{lemma}

\begin{proposition}[Cantor]
A countable union of countable sets is countable:

For each sequence $A_0, A_1 \ldots$ of countable sets, the union
\[ A = \bigcup_{n=0}^\infty A_n \]
is countable.
\end{proposition}
\begin{proof}
We may assume none of the $A_n$ are empty (otherwise we can just take a superset and if this is countable, the subset will be as well). We can find an enumeration $\pi^n: \N\twoheadrightarrow A_n$ for each $A_n$. Let $\rho^{-1}$ be an enumeration of $\N\times\N$, as in lemma \ref{pairEnumeration}. Then
\[ \N \twoheadrightarrow  \bigcup_{n=0}^\infty A_n: m\mapsto \pi^{x}(y) \]
where $x(m) = \rho^{-1}_0(m)$ and $y(m) = \rho^{-1}_1(m)$, is an enumeration.
\end{proof}

\begin{proposition}
The set of infinite, binary sequences
\[ \Delta = \{ (a_i)_{i\in\N} \;|\; \forall i\in\N: a_i=0\lor a_i=1 \} \]
is uncountable.
\end{proposition}
\begin{proof}
Assume, towards a contradiction, that there exists an enumeration $(\alpha_n)_{n\in\N}$ of $\Delta$. The diagonal argument is then to construct the binary sequence
\[ \alpha_0(0),\alpha_1(1), \alpha_2(2), \alpha_3(3), \alpha_4(4),\ldots  \]
and make every $0$ a $1$ and vice versa. This new is not an element of the enumeration by construction (it is different from every element of the enumeration in at least one digit).
\end{proof}

\subsubsection{Sequences of natural numbers}
\begin{lemma} \label{injectiveSequenceNaturalNumbersMonotoneSubsequence}
Let $\seq{n_k}$ be an injective sequence in $\N$. Then
\begin{enumerate}
\item for all $k\in \N$, there exists $m> k$ such that $n_{m}>n_k$;
\item $\seq{n_k}$ has a monotone subsequence.
\end{enumerate}
\end{lemma}
\begin{proof}
(1) Consider the initial slice $A \defeq \seq{n_k}|_{k:k+n_k+1}$. This is an injective function, so $A \geq_c \interval[co]{k, k+n_k+1}$ by \ref{injectiveSurjectiveMappingCardinality}, $\#(A) = n_k +1$ by \ref{sizeOfIntervalsInN} and thus there exists an element of $A$ that is larger than $n_k$ by \ref{sizeEstimationByIntervalInclusion}.

(2) Apply the axiom of dependent choice. (TODO can this be relaxed to countable choice?).
\end{proof}

\subsubsection{Finite sets}
\begin{proposition} \label{minimalMaximalFiniteSet}
Let $R$ be a relation on a class $A$. Then every finite set $S\subseteq A$ has a minimal and maximal element.
\end{proposition}
\begin{proof}
We prove that $S$ has a minimal element by induction on the size of $S$. The existence of a maximal element follows by duality.

For the base step, note that the unique element of a singelton is a minimal element.

Now assume the induction hypothesis for $k\in \N$, i.e.\ that each set $S'\subseteq A$ such that $\#(S') = k$ has a minimal element.

To prove the induction step, take some arbitrary $S\subseteq A$ such that $\#(S) = k+1$. Now $S$ is non-empty, so we can take $x\in S$ and $\#(S\setminus \{x\}) = k$. Thus $S\setminus \{x\}$ has a minimal element $m$. If $x\overline{R}m$, then $m$ is also a minimal element of $S$ and thus $S$ has a minimal element.
If $xRm$, then $x$ is a minimal element of $S$: for all $y\in S\setminus\{x\}$,  $yRx$ implies $yRm$ by transitivity. By minimality of $m$ in $S\setminus\{x\}$, we have $mRy$, so $xRy$ (again by transitivity).

This proves the induction step.
\end{proof}
Note that we cannot use transfinite induction: removing one element from an infinite set does not lower the cardinality.






\chapter{Replacement}

\undline{Maximal antichain principle}: Every non-empty poset has a maximal antichain.

\chapter{Cardinals and ordinals}
\url{http://euclid.colorado.edu/~monkd/monk11.pdf}

\section{Ordinals}
\begin{definition}
The class of all well-ordered sets is denoted $\Ord$.
\end{definition}

\begin{definition}
An \udef{ordinal assignment} is a function $\operatorname{ord}$ from the class of well-ordered posets to the class of well-ordered posets such that for all well-ordered posets $A,B$,
\[ \operatorname{ord}(A) = \operatorname{ord}(B) \iff A =_o B. \]
Sets in the image of $\operatorname{ord}$ are called \udef{ordinals} or \udef{ordinal numbers}.
\end{definition}
By \ref{wellfoundednessOfWosetComparison} we have:
\begin{lemma}
The inclusion order restricted to ordinals is well-founded.
\end{lemma}

\begin{definition}
A set $S$ is a \udef{(von Neumann) ordinal} if it is transitive and all its members are transitive.
\end{definition}
The class of von Neumann ordinals is a transitive class. TODO: exclude urelements.

\url{https://en.wikipedia.org/wiki/Ordinal_number}

\section{Cardinals}

The idea behind the cardinals is to have a set of objects that witnesses the size, or potency, of sets. These two conditions define the cardinal assignment.
\begin{definition}
Define an operation $A\mapsto |A|$ on the class of sets such that
\begin{itemize}
\item $A=_c |A|$;
\item for each set of sets $\mathcal{E}$, $\{ |X|\;|\; X\in\mathcal{E} \}$ is a set.
\end{itemize}
Such an operation is called a \udef{(weak) cardinal assignment}. The class of \udef{cardinal numbers}(relative to a given
cardinal assignment), denoted $\Card$, is the image of the cardinal assigment:
\[ \kappa \in \Card \defequiv \exists A: \kappa = |A|. \]
If the cardinal assignment also satisfies
\begin{itemize}
\item if $A=_c B$, then $|A|=|B|$,
\end{itemize}
then it is called a \udef{strong cardinal assignment}.
\end{definition}
In particular, notice that cardinals are sets. Indeed the assignment $A\mapsto |A|$ is a weak cardinal assignment, so practically all results in this section hold in particular for sets.

\begin{proposition} \label{cardinalsFromOrdinals}
Given an ordinal assignment $\operatorname{ord}$, we can define a cardinal assignment
\[ |A| = \min\setbuilder{\alpha}{\alpha =_c A}. \]
In this case the cardinals are either finite or limit ordinals.
\end{proposition}
Note that this works even though $\setbuilder{\alpha}{\alpha =_c A}$ may be a class in general.
\begin{proof}
TODO.
\end{proof}

\begin{lemma}
If cardinal numbers are defined using a strong cardinal assignment, then for all cardinals $\kappa,\lambda$:
\[ |\kappa| = \kappa \quad \text{and}\quad \kappa =_c\lambda \iff \kappa = \lambda \]
\end{lemma}

\begin{lemma}
For any cardinal assignment and any two sets $A,B$,
\begin{enumerate}
\item $|A| = |B| \implies A=_c B$;
\item $|\emptyset| = \emptyset$.
\end{enumerate}
\end{lemma}

\subsection{Cardinal arithmetic without choice}
\begin{definition}
Fix a specific, possibly weak, cardinal assignment. We define the arithmetic operations on the cardinal numbers $\kappa, \lambda$ as
\begin{align*}
\kappa + \lambda &\defeq |\kappa \sqcup \lambda | & &=_c \kappa \sqcup \lambda, \\
\kappa \cdot \lambda &\defeq |\kappa \times \lambda | & &=_c \kappa \times \lambda, \\
\kappa^\lambda &\defeq |(\lambda \to \kappa) | & &=_c (\lambda \to \kappa).
\end{align*}
Also
\begin{align*}
\sum_{i\in I}\kappa_i &\defeq \left|\bigsqcup_{i\in I}\kappa_i\right|, \\
\prod_{i\in I}\kappa_i &\defeq \left| \prod_{i\in I}\kappa_i \right|.
\end{align*}
We fix the following symbols for some empty set, singleton and doubleton cardinals:
\[ 0 \defeq |\emptyset|=\emptyset \qquad 1 \defeq |\{0\}| \qquad 2\defeq |\{0,1\}|.   \]
\end{definition}
The definition of $0,1,2$ does not conflict with the natural numbers.

\begin{lemma}
Let $\kappa_1,\kappa_2,\lambda_1,\lambda_2$ be cardinal numbers such that $\kappa_1=_c \kappa_2$ and $\lambda_1=_c\lambda_2$. Then
\begin{enumerate}
\item $\kappa_1 + \lambda_1 =_c \kappa_2 + \lambda_2$;
\item $\kappa_1 \cdot \lambda_1 =_c \kappa_2\cdot \lambda_2$;
\item $\kappa_1^{\lambda_1} =_c \kappa_2^{\lambda_2}$.
\end{enumerate}
\end{lemma}
The proof is the same as for lemma \ref{welldefinedCardinalArithemtic}.

\begin{lemma}
For all sets $A,B$ and thus for all cardinals $\kappa,\lambda$:
\[ \prod_{i\in A}B = (A\to B), \qquad \prod_{i\in\lambda}\kappa = \kappa^\lambda. \]
\end{lemma}
Notice that there is equality, $=$, not just equinumerosity $=_c$.

Cardinal arithmetic has properties reminiscent of a commutative semiring (i.e.\ ring where additive inverses are not guaranteed, see later). Of course cardinal arithmatic is not defined on a set, but on a class, so this is not completely true.
\begin{lemma} \label{cardinalArithmetic}
Let $\kappa,\lambda, \mu$ be cardinal numbers. Then $+$ is like a commutative monoid:
\begin{enumerate}
\item $\kappa + (\lambda+\mu) =_c (\kappa+\lambda)+\mu$;
\item $0+\kappa =_c \kappa+0 =_c \kappa$;
\item $\kappa+\lambda =_c \lambda+\kappa$;
\end{enumerate}
and $\cdot$ is also like a commutative monoid:
\begin{enumerate}
\setcounter{enumi}{3}
\item $\kappa \cdot (\lambda\cdot\mu) =_c (\kappa\cdot\lambda)\cdot\mu$;
\item $1\cdot\kappa =_c \kappa\cdot 1 =_c \kappa$;
\item $\kappa\cdot\lambda =_c \lambda\cdot\kappa$.
\end{enumerate}
Multiplication distributes over addition:
\begin{enumerate}
\setcounter{enumi}{6}
\item $\kappa\cdot(\lambda +\mu) =_c \kappa\cdot \lambda+\kappa\cdot\mu$.
\end{enumerate}
Multiplication by $0$ annihilates:
\begin{enumerate}
\setcounter{enumi}{7}
\item $0\cdot \kappa =_c 0$.
\end{enumerate}
There are no zero divisors:
\begin{enumerate}
\setcounter{enumi}{8}
\item $\kappa\neq 0 \neq \lambda \implies \kappa\cdot\lambda \neq 0$.
\end{enumerate}
\end{lemma}
\begin{lemma}
Let $\kappa_i$ be cardinals for all $i\in I$. Then
\[ \exists i\in I: \kappa_i = 0 \implies \prod_{i\in I}\kappa_i. \]
\end{lemma}
The other implication depends on the axiom of choice!

\begin{lemma} \label{repeatedAdditionMultiplicationCardinals}
Let $\kappa, \lambda$ be cardinals. Then
\begin{enumerate}
\item $\sum_{i\in \lambda}\kappa =_c \kappa\times\lambda$;
\item $\prod_{i\in\lambda}\kappa =_c \kappa^\lambda$.
\end{enumerate}
\end{lemma}
\begin{corollary}
Let $\kappa$ be a cardinal number. Then
\[ \kappa^0 =_c 1, \qquad \kappa^1 =_c \kappa, \qquad \kappa^2 =_c \kappa\cdot \kappa. \]
\end{corollary}

\begin{lemma}
Let $\kappa,\lambda, \mu$ be cardinal numbers. Then
\begin{enumerate}
\item $(\kappa\cdot\lambda)^\mu =_c \kappa^\mu\cdot\lambda^\mu$;
\item $\kappa^{(\lambda+\mu)} =_c \kappa^\lambda\cdot \kappa^\mu$;
\item $\left(\kappa^\lambda\right)^\mu =_c \kappa^{\lambda\cdot \mu}$.
\end{enumerate}
\end{lemma}
By Cantor's theorem \ref{Cantor}, this also gives $\kappa \leq_c 2^\kappa$.

\begin{lemma}
Let $\kappa,\lambda_i$ be cardinals for all $i\in I$. Then
\[ \kappa\cdot \sum_{i\in I}\lambda_i =_c \sum_{i\in I}\kappa\cdot \lambda_i. \]
\end{lemma}
\begin{proof}
We compute
\begin{align*}
\kappa\cdot \sum_{i\in I}\lambda_i &=_c \kappa\times\left(\bigsqcup_{i\in I}\lambda_i\right) = \kappa\times \left(\bigcup_{i\in I}\{i\}\times \lambda_i\right) = \bigcup_{i\in I}\kappa\times(\{i\}\times \lambda_i) \\
&=_c \bigcup_{i\in I}\{i\}\times(\kappa\times \lambda_i) = \bigsqcup_{i\in I}\kappa\times \lambda_i =_c \sum_{i\in I}\kappa\cdot \lambda_i.
\end{align*}
\end{proof}
TODO: expand on such computations?

\begin{lemma}
Let $\kappa,\lambda, \mu$ be cardinal numbers. Then
\begin{enumerate}
\item $\kappa\leq_c \mu \implies \kappa+\lambda \leq_c \mu+\lambda$;
\item $\kappa\leq_c \mu \implies \kappa\cdot\lambda \leq_c \mu\cdot\lambda$;
\item $\kappa\leq_c \mu \implies \kappa^\lambda \leq_c \mu^\lambda$;
\item $\kappa\leq_c \mu \implies \lambda^\kappa \leq_c \lambda^\mu$ if $\lambda\neq 0$.
\end{enumerate}
\end{lemma}
These implications do not necessarily hold for strict inequalities.

For the last implication: if $\lambda = 0$, then
\[ \forall \kappa \in \operatorname{Card}: \lambda^\kappa = (\kappa\to\emptyset) = \begin{cases}
\emptyset & \kappa \neq 0 \\
\{\emptyset\} & \kappa = 0.
\end{cases} \]
So if $\kappa = 0$ and $\mu\neq 0$, there exists an injection $\kappa\to\mu$, namely $\emptyset$ and so $\kappa\leq_c \mu$, but there is no injection (in fact no function) $\{\emptyset\}\to\emptyset$, so $\lambda^\kappa \nleq_c \lambda^\mu$.

\subsection{The cardinality of natural numbers}
\begin{definition}
We define
\[ \aleph_0 \defeq |\N|. \]
This is the cardinality of countably infinite sets. If we have a strong cardinal assignment is is uniquely so.

We also define for each $n\in \N$:
\[ \kappa^n \defeq |\kappa^{(n)}|. \]
\end{definition}
We also define $\aleph_1 \defeq \aleph_0^+,\aleph_2 \defeq \aleph_1^+,\ldots $

If we take the natural numbers to be Von Neumann ordinals, then for all finite sets $A$, $\#(A) =_c A$. Then $\#$ is a strong cardinal assignment on the finite sets and all the previous results apply.


\begin{proposition} \label{stringsInCountableAlphabetCountable}
For each countably infinite set $A$ and each $n>0$,
\[A =_c A\times A =_c A^{(n)} =_c A^*. \]
The equivalent expression in cardinal arithmetic is
\[ \aleph_0 =_c \aleph_0\cdot\aleph_0 =_c \aleph_0^n =_c |\aleph_0^*|. \]
\end{proposition}
\begin{proof}
If all $=_c$ are replaced by $\leq_c$, the claim is trivial, thus, by the Schröder-Bernstein theorem \ref{SchroederBernstein}, it is enough to show $\N^*\leq_c \N$.

Choose a bijection $\rho: \N\times\N\twoheadrightarrowtail \N$ as in lemma \ref{pairEnumeration}.

Define by recursion a function $f:\N\to (\N^{(n+1)}\rightarrowtail \N): n\mapsto \pi_n$, such that
\begin{align*}
\pi_0(u) &= u(0) \\
\pi_{n+1}(u) &= \rho(\pi_n(u|_{[0,n+1[}), u(n+1)).
\end{align*}
Then the function
\[ \pi(u) = (\len(u)-1, \pi_{\len(u)-1}(u)) \]
is injective, proving
\[ \bigcup_{n=0}^\infty \N^{(n+1)}\leq_c \N\times \N. \]

Using $\rho$ we see that $\bigcup_{n=0}^\infty \N^{(n+1)}\leq_c \N$.
\end{proof}

\begin{lemma}
For all cardinals $\kappa$, $2^\kappa \neq_c \aleph_0$.
\end{lemma}
\begin{proof}
We split into two cases $\kappa$ countable and uncountable:
\begin{enumerate}
\item If $\kappa$ is countable, then either $\kappa =_c \aleph_0$ and $2^\kappa \neq_c \aleph_0$ by Cantor's theorem, \ref{Cantor}, or $\kappa$ is finite in which case $2^\kappa =_c \#(2^{\kappa}) = 2^{\#(\kappa)}$ which is finite.
\item If $\kappa$ is uncountable and $2^\kappa =_c \aleph_0$, then $\kappa \leq_c 2^\kappa =_c \aleph_0$ and $\kappa$ would be countable. A contradiction.
\end{enumerate}
\end{proof}
This is an expression of the fact that we can compare cardinals to countable cardinals, even without choice.

\subsection{Cardinal arithmetic with choice}
\[ |\mathbb{F}|^{|\beta|} > |\beta| \]
In this section we assume the axiom of choice.
\begin{definition}
Given any cardinal $\kappa$, we can define the \udef{successor cardinal} as
\[ \kappa^+ \defeq \left|\aleph(\kappa)\right|. \]
\end{definition}

\begin{lemma}
For any cardinal $\kappa$, the cardinal $\kappa^+$ is $\leq_c$-least among the cardinals bigger than $\kappa$.
\end{lemma}
\begin{proof}
By Hartogs' lemma and cardinal comparability, we know $\kappa <_c \kappa^+$. By proposition \ref{proposition:HartogsLeast} $\kappa^+$ is $\leq_o$-least (and thus $\leq_c$-least) with this property.
\end{proof}

\begin{theorem}[Tarski's theorem about choice]
The following are equivalent:
\begin{enumerate}
\item the axiom of choice;
\item $A =_c A\times A$ for every infinite set $A$.
\end{enumerate}
\end{theorem}
\begin{proof}
$\boxed{\Rightarrow}$
We now prove $A =_c A\times A$ by induction on $\Ord$. For the base case, see \ref{stringsInCountableAlphabetCountable}.

Now take $A\in\Ord$. If $A$ has a proper initial segment that has the same cardinality as $A$, then $A =_c A\times A$ is immediate from the induction hypothesis.

Now suppose each proper initial segment of $A$ has a smaller cardinality than $A$ (i.e. $A$ is a limit ordinal, so $A$ has no maximal element). Clearly $A\leq_c A\times A$, so it is enough to prove that there exists a well-ordering on $A\times A$ such that $A\times A \leq_o A$. We claim the order $\prec$ on $A\times A$ defined by
\begin{align*}
(a,b)\prec (c,d) \defequiv &\big(\max\{a,b\} < \max\{c,d\}\big)\;\lor \\ &\Big(\big(\max\{a,b\} = \max\{c,d\}\big)\land(a < c)\Big) \;\lor \\ &\Big(\big(\max\{a,b\} = \max\{c,d\}\big)\land(a = c)\land(b < d)\Big)
\end{align*}
works. It is easy to verify that it is a well-ordering.

Take some, arbitrary, proper initial segment of $A\times A$, say $\downset\{(x,y)\}$, and set $z = \max\{x,y\}$. Then $\downset\{(x,y)\} \subseteq \downset\{(z,z)\} = (\downset\{z\})\times(\downset\{z\})$. Now $\downset\{z\}$ is a proper initial segment of $A$ and thus $\downset\{z\} <_c A$. By induction hypothesis $\downset\{z\} =_c (\downset\{z\})\times(\downset\{z\})$. So we have
\[ \downset\{(x,y)\} \leq_c (\downset\{z\})\times(\downset\{z\}) =_c \downset\{z\} <_c A. \]
Thus, by \ref{ordinalInequalitiesLemma}, we have $\downset\{(x,y)\} \leq_o A$. 

As the initial segment $\downset\{(x,y)\}$ of $A\times A$ was taken arbitrarily, we have $A\times A \leq_o A$ by \ref{ordinalInequalityFromInitialSegments}.

$\boxed{\Leftarrow}$ TODO \url{https://en.wikipedia.org/wiki/Tarski%27s_theorem_about_choice}. and \url{https://math.stackexchange.com/questions/56466/for-every-infinite-s-s-s-times-s-implies-the-axiom-of-choice}
\end{proof}
\begin{corollary} \label{productInfiniteCardinals}
Let $\alpha,\beta$ be cardinals, at least one of which is infinite. Then $\alpha \cdot \beta = \max\{\alpha, \beta\}$. 
\end{corollary}
\begin{corollary} \label{cardinalityStringsFiniteSubsets}
Let $X$ be an infinite set. Then the following have the same cardinality as $X$:
\begin{enumerate}
\item $X^*$;
\item the set of finite subsets of $X$.
\end{enumerate}
\end{corollary}
\begin{proof}
It is clear that both are larger in cardinality than $X$ and $(2) \leq_c (1)$. Thus we just need to prove that $X^* \leq_c X$.

We first prove $X =_c X^{(n)}$ for all $n\in \N$. We prove this by induction on $n$. Assume $X =_c X^{(n-1)}$. Then, using the proposition, we have $X =_c X\times X =_c X\times X^{(n-1)} =_c X^{(n)}$. Now
\[ |X^*| =_c \sum_{n\in \N}|X^{(n)}| =_c \sum_{n\in \N}|X| = \aleph_0\cdot |X| = \max\{\aleph_0, |X|\} = |X|, \]
using \ref{repeatedAdditionMultiplicationCardinals}.
\end{proof}


\subsection{The continuum}
\begin{definition}
We define the \udef{continuum}
\[ \mathfrak{c} \defeq |\powerset(\N)| =_c 2^\aleph_0. \]
\end{definition}
The continuum hypothesis is that there are no cardinals between $\aleph_0$ and $\mathfrak{c}$. This is independent of ZFC.

From cardinal arithmetic we can immediately obtain some results, like
\[ \mathfrak{c}\cdot\mathfrak{c} =_c 2^{\aleph_0}\cdot 2^{\aleph_0} =_c 2^{\aleph_0+\aleph_0} =_c 2^{\aleph_0} =_c \mathfrak{c}\]
and
\[ \mathfrak{c} =_c 2^{\aleph_0} \leq_c \aleph_0^{\aleph_0} \leq_c \mathfrak{c}^{\aleph_0} =_c \left(2^{\aleph_0}\right)^{\aleph_0} =_c 2^{\aleph_0\cdot \aleph_0} =_c \mathfrak{c}.\]




\chapter{Going from two to many}
TODO: move up!!
\section{Functions on ordinals}
\begin{definition}
Let $\alpha$ be an ordinal and $A$ a set. We introduce special notation in this case:
\begin{itemize}
\item $A^\alpha \defeq (\alpha \to A)$;
\item if $a\in A^\alpha$, then we write $a_0, a_1, a_2 \ldots$ instead of $a(0), a(1), a(2) \ldots$
\end{itemize}
\end{definition}

\subsection{Pointwise extensions}
\begin{definition}
Let $\alpha\in\Ord$, $A,B\in\Set$ and $f\in (A\to B)$. Then
\[ f^\alpha: A^\alpha \to B^\alpha: a\mapsto f\circ a \]
is the pointwise extension of $f$ to $A^\alpha$.
\end{definition}
This is a particular instance of post-composition, $f_*$.

\section{Finite Cartesian proucts: Tuples}
\begin{definition}
Let $a_1, \ldots, a_n$ be objects. A definition of $(a_1, \ldots, a_n)$ is called an \udef{$n$-tuple operation} (or just \udef{tuple operation}) if it satisfies
\begin{itemize}
\item $(a_1, \ldots, a_n) = (b_1, \ldots, b_n) \iff \forall i\in (1:n): a_i=b_i$;
\item for all sets $A,\ldots, A_n$, $\setbuilder{(a_1,\ldots, a_n)}{\forall i\in (1:n): a_i \in A_i}$ is a set.
\end{itemize}
We call
\[ \bigtimes_{i=1}^nA_i = \setbuilder{(a_1,\ldots, a_n)}{\forall i\in (1:n): a_i \in A_i} \]
the \udef{Cartesian product} of $A_1,\ldots, A_n$.
\end{definition}

\begin{proposition} \label{pairNTupleDefinition}
Let $a_1, \ldots, a_n$ be objects. Defining $(a_1, \ldots, a_n)$ as the string $\seq{a_1, \ldots, a_n}$ of length $n$ is a valid $n$-tuple operation.
\end{proposition}

We can also directly use a pair operation to define a $n$-tuple operation.
\begin{proposition}
Let $a_1, \ldots, a_n$ be objects. Define the function $f_a$ recursively by
\[ \begin{cases}
f_a(0) = \emptyset \\
f_a(i+1) = \begin{cases}
(f_a(i), a_{i+1}) & i < n \\
(f_a(i), \emptyset) & i \geq n
\end{cases}
\end{cases} \]
Defining $(a_1, \ldots, a_n)$ as $f_a(n)$ is a valid $n$-tuple operation.
\end{proposition}

Informally, this construction can be described as follows:
\begin{itemize}
\item The $0$-tuple is defined as the empty set $\emptyset$;
\item A $1$-tuple containing $a$ is defined as $(\emptyset, a)$;
\item An $n$-tuple, with $n > 1$, is defined as an ordered pair of its last entry and an $(n - 1)$-tuple which contains the preceding entries:
\[ (a_1, \ldots, a_n) = ((a_1, \ldots, a_{n-1}), a_n) = ((\ldots((\emptyset, a_1), a_2), \ldots), a_n). \]
\end{itemize}
For example $(1,2,3,4) = ((((\emptyset, 1), 2), 3), 4)$.

\subsection{Association relations}
TODO: $((a,b),c) \approx (a,(b,c))$.


\subsection{$n$-ary relations}
\begin{definition}
Let $A_1, \ldots, A_n$ be sets and $G \subseteq \bigtimes_{i=1}^nA_i$. We call $R = (G, (A_1, \ldots, A_n))$ an \udef{$n$-ary relation} on $(A_1, \ldots, A_n)$ and $\graph(R) \defeq G$ the \udef{graph} of $R$.
\end{definition}
In particular a \udef{ternary relation} on $(A,B,C)$ is a structured sets $(G,(A,B,C))$ where $G \subset A\times B\times C$.

\begin{proposition}
Let $A_1, \ldots, A_n$ be sets and $G \subseteq \bigtimes_{i=1}^nA_i$. Using the definition of $n$-tuple in \ref{pairNTupleDefinition}, $R = (G, (A_1 \times \ldots \times A_{n-1}), A_n))$ is a binary relation.
\end{proposition}

\section{Operations on sequences of sets}
\begin{definition}
Let $I$ be an arbitrary index set, $A$ some set and let there be a surjective function $a: I\twoheadrightarrow A: i \mapsto a_i$.

Then we say $A$ is \udef{indexed by I} and we write $A = \{a_i\}_{i\in I}$. In this case we call $A$ an \udef{(indexed) family}.

In particular if $A\subseteq \powerset(X)$ for some set $X$, $A$ is an indexed family of sets.
\end{definition}
Notice we do not require the function $a$ to be injective and thus multiple indexed elements may be the same.

Sometimes the notation $(a_i)_{i\in I}$ is used, if $I$ is ordered, to emphasise the ordering of $\{a_i\}_{i\in I}$ by $I$.

\subsection{Union and intersection of indexed families of sets}
Let $\{A_i\}_{i\in I}$ be an indexed family of sets. Then we write
\begin{align*}
\bigcup_{i\in I} A_i &\defeq \bigcup A[I] \\
\bigcap_{i\in I} A_i &\defeq \bigcap A[I]
\end{align*}
If $I = \interval[co]{0,n}$, we write $\{A_i\}_{i=1}^{n} \defeq \{A_i\}_{i\in \interval[co]{0,n}}$,
\[ \bigcup_{i=1}^{n-1} A_i \defeq \bigcup_{i\in \interval[co]{0,n}} A_i \qquad \text{and} \qquad \bigcap_{i=1}^{n-1} A_i \defeq \bigcap_{i\in \interval[co]{0,n}} A_i \]
and if $I=\N$, we write $\{A_i\}_{i=1}^{n} \defeq \{A_i\}_{i\in \N}$,
\[ \bigcup_{i=1}^{\infty} A_i \defeq \bigcup_{i\in \N} A_i \qquad \text{and} \qquad \bigcap_{i=1}^{\infty} A_i \defeq \bigcap_{i\in \N} A_i. \]

\begin{lemma}
Let $A$ be a set and $\{B_i\}_{i\in I}$ an indexed family of sets. Then
\begin{align*}
A\times \bigcup_{i\in I} B_i &= \bigcup_{i\in I}A\times B_i; \\
A\times \bigcap_{i\in I} B_i &= \bigcap_{i\in I}A\times B_i.
\end{align*}
\end{lemma}

\subsubsection{Multiple indices}
If the index set $I$ is a Cartesian product $I=J\times K$, then we also write
\[ \bigcup_{(j,k)\in I} A_{j,k} = \bigcup_{\substack{j\in J \\ k\in K}} A_{j,k} \qquad\text{and}\qquad \bigcap_{(j,k)\in I} A_{j,k} = \bigcap_{\substack{j\in J \\ k\in K}} A_{j,k}. \]

If $\{A_{j,k}\}_{(j,k) \in J\times K}$ is such an indexed family of sets, then $\{A_{j,k'}\}_{j\in J}$ is an indexed family of sets for each $k\in K$by partial application of $k'$ to the second argument. This allows us to apply union and intersection pointwise: we define
\[ \bigcup_{j\in J}A_{j,k} \defeq \left(k'\mapsto \bigcup_{j\in J}A_{j,k'}\right) \qquad\text{and}\qquad \bigcap_{j\in J}A_{j,k} \defeq \left(k'\mapsto \bigcap_{j\in J}A_{j,k'}\right) \]
as well as something similar for the first argument.

\subsubsection{Associativity and commutativity}
\begin{lemma} \label{setAssociativityCommutativity}
Let $\{A_{j,k}\}_{(j,k) \in J\times K}$ be an indexed family of sets. Then
\begin{align*}
\bigcup_{j\in J}\left(\bigcup_{k\in K}A_{j,k}\right) &= \bigcup_{\substack{j\in J \\ k\in K}}A_{j,k} = \bigcup_{k\in K}\left(\bigcup_{j\in J}A_{j,k}\right) \qquad\text{and} \\
\bigcap_{j\in J}\left(\bigcap_{k\in K}A_{j,k}\right) &= \bigcap_{\substack{j\in J \\ k\in K}}A_{j,k} = \bigcap_{k\in K}\left(\bigcap_{j\in J}A_{j,k}\right).
\end{align*}
\end{lemma}
\begin{corollary}
Let $\{A_{i}\}_{i \in I}$ be an indexed family of sets and $B$ a set. Then
\[ \left(\bigcup_{i\in I}A_i\right)\cup B = \bigcup_{i\in I}(A_i\cup B) \qquad\text{and}\qquad \left(\bigcap_{i\in I}A_i\right)\cap B = \bigcap_{i\in I}(A_i\cap B). \]
\end{corollary}
\begin{proof}
Set $J\times K = I\times\{0,1\}$ and $A_{j,k} = \begin{cases}
A_j & (k=0) \\
B & \text{(else)}
\end{cases}$.
\end{proof}
\begin{corollary}
Let $\{A_{i}\}_{i \in I}$ and $\{B_{j}\}_{j \in J}$ be indexed families of sets. Then
\[ \left(\bigcup_{i\in I}A_i\right)\cup \left(\bigcup_{j\in J}B_j\right) = \bigcup_{\substack{i\in I\\j\in J}}(A_i\cup B_j) \qquad\text{and}\qquad \left(\bigcap_{i\in I}A_i\right)\cap \left(\bigcap_{j\in J}B_j\right) = \bigcap_{\substack{i\in I\\j\in J}}(A_i\cap B_j). \]
If both families are indexed by the same index set $I$, we may take the union/intersection over just $I$, not $I\times I$.
\end{corollary}

\subsubsection{Distributivity}
\begin{lemma} \label{setDistributivity}
Let $\{A_{i}\}_{i \in I}, \{B_{j}\}_{j \in J}$ be indexed families of sets and $C$ a set. Then
\[ \left(\bigcup_{i\in I}A_i\right)\cap B = \bigcup_{i\in I}(A_i\cap B) \qquad\text{and}\qquad \left(\bigcap_{i\in I}A_i\right)\cup B = \bigcap_{i\in I}(A_i\cup B). \]
Also
\[ \left(\bigcup_{i\in I}A_i\right)\cap \left(\bigcup_{j\in J}B_j\right) = \bigcup_{\substack{i\in I\\j\in J}}(A_i\cap B_j) \qquad\text{and}\qquad \left(\bigcap_{i\in I}A_i\right)\cup \left(\bigcap_{j\in J}B_j\right) = \bigcap_{\substack{i\in I\\j\in J}}(A_i\cup B_j). \]
\end{lemma}

\begin{proposition}
Let $\{A_{j,k}\}_{(j,k) \in J\times K}$ be an indexed family of sets. Then
\begin{align*}
\bigcap_{j\in J}\bigcup_{k\in K}A_{j,k} &= \bigcup_{f \in K^J}\bigcap_{j\in J}A_{j,f(j)} \\
\bigcup_{j\in J}\bigcap_{k\in K}A_{j,k} &= \bigcap_{f \in K^J}\bigcup_{j\in J}A_{j,f(j)}
\end{align*}
\end{proposition}
\begin{proof}
We have
\begin{align*}
x\in \bigcap_{j\in J}\bigcup_{k\in K}A_{j,k} &\iff \forall j\in J:\exists k\in K: x\in A_{j,k} \\
&\iff \exists f\in J^K: \forall j\in J: x\in A_{j, f(j)} \\
&\iff \bigcup_{f \in K^J}\bigcap_{j\in J}A_{j,f(j)}.
\end{align*}
\end{proof}
The $f\in J^K$ encodes which $k$ is the ``good one'' for each $j$.
\begin{corollary}
Let $\{A_{j,k}\}_{(j,k) \in J\times K}$ be an indexed family of sets. Then
\[ \bigcup_{j\in J}\left(\bigcap_{k\in K}A_{j,k}\right) \subseteq \bigcap_{k\in K}\left(\bigcup_{j\in J}A_{j,k}\right). \]
\end{corollary}
\begin{proof}
We restrict the $f$ in the proposition to the set of constant functions.
\end{proof}
In general these two sets are not equal!

\begin{example}
For example, set $A_{j,k} = \interval[o]{0,\frac{j}{k}} \subseteq \R$, where $j,k\in \N$. Then $\bigcap_{k\in \N} A_{j,k} = \emptyset$ for all $j$ and $\bigcup_{j\in \N}A_{j,k} = \interval[o]{0,\infty}$ for all $k\in \N$. So
\[ \bigcup_{j\in \N}\bigcap_{k\in\N} A_{j,k} = \bigcup_{j\in \N}\emptyset = \emptyset \quad \subseteq \quad \interval[o]{0,\infty} = \bigcap_{k\in \N}\interval[o]{0,\infty} = \bigcap_{k\in \N}\bigcap_{j\in \N}A_{j,k}. \]
\end{example}


\subsubsection{Union and intersection of index sets}

\begin{lemma} \label{unionIntersectionLabelSet}
Let $\mathcal{I}$ be a family of index sets and let $A_i$ be a set for all $i\in \bigcup \mathcal{I}$. Then
\begin{enumerate}
\item $\bigcup_{i\in \bigcup \mathcal{I}} A_i = \bigcup_{I\in \mathcal{I}}\bigcup_{i\in I} A_i$;
\item $\bigcap_{i\in \bigcap \mathcal{I}} A_i = \bigcap_{I\in \mathcal{I}}\bigcap_{i\in I} A_i$;
\item $\bigcup_{i\in \bigcap \mathcal{I}} A_i \subseteq \bigcap_{I\in \mathcal{I}}\bigcup_{i\in I} A_i$;
\item $\bigcap_{i\in \bigcup \mathcal{I}} A_i \supseteq \bigcup_{I\in \mathcal{I}}\bigcap_{i\in I} A_i$.
\end{enumerate}
\end{lemma}
\begin{proof}
(1) We calculate
\begin{align*}
x\in \bigcup_{i\in \bigcup \mathcal{I}} A_i &\iff \exists i\in \bigcup \mathcal{I}: x\in A_i \\
&\iff \exists i: (i\in \bigcup \mathcal{I}) \land (x\in A_i) \\
&\iff \exists i: (\exists I \in \mathcal{I}: i\in I) \land (x\in A_i) \\
&\iff \exists i: \exists I: (I \in \mathcal{I}) \land (i\in I) \land (x\in A_i) \\
&\iff \exists I\in \mathcal{I}: \exists i \in I: x\in A_i \\
&\iff x\in \bigcup_{I\in \mathcal{I}}\bigcup_{i\in I} A_i
\end{align*}

(2) Replace $\exists$ by $\forall$ and $\land$ by $\Rightarrow$ in the proof of (1).

(3) We calculate
\begin{align*}
x\in \bigcup_{i\in \bigcap \mathcal{I}} A_i &\iff \exists i\in \bigcap \mathcal{I}: x\in A_i \\
&\iff \exists i: (i\in \bigcap \mathcal{I}) \land (x\in A_i) \\
&\iff \exists i: (\forall I \in \mathcal{I}: i\in I) \land (x\in A_i) \\
&\iff \exists i: (\forall I: (I \in \mathcal{I}) \Rightarrow (i\in I)) \land (x\in A_i) \\
&\implies \exists i: \forall I: (I \in \mathcal{I}) \Rightarrow ((i\in I) \land (x\in A_i)) \\
&\implies \forall I: \exists i: (I \in \mathcal{I}) \Rightarrow ((i\in I) \land (x\in A_i)) \\
&\iff \forall I: (I \in \mathcal{I}) \Rightarrow (\exists i:(i\in I) \land (x\in A_i)) \\
&\iff \forall I\in \mathcal{I}: \exists i \in I: x\in A_i \\
&\iff x\in \bigcap_{I\in \mathcal{I}}\bigcup_{i\in I} A_i
\end{align*}

(4) TODO
\end{proof}

\subsection{Arbitrary Cartesian products}
\begin{definition}
Let $\{A_i\}_{i\in I}$ be an arbitrary indexed family of sets, then we define the \udef{Cartesian product} of $\{A_i\}_{i\in I}$ to be
\[ \prod_{i\in I}A_i \defeq \left\{ f\in \left(\left. I \to \bigcup_{i\in I}A_i \right) \; \right| \; \forall i\in I: f(i) \in A_i \right\}. \]
For each $j\in I$, the function
\[ \pi_j : \prod_{i\in I}A_i \to A_j: f\mapsto f(j) \]
is called the \udef{$j^\text{th}$ projection map}.
\end{definition}

\begin{definition}
A Cartesian product of an indexed family of sets $\{A_i\}_{i\in I}$ is called a \udef{Cartesian power} of $A$ if for all $i\in I$, $A_i$ is the same set $A$. This is denoted $A^I$.
\end{definition}
If $I = \interval[co]{0,n}$ for some $n\in \N$, we write $A^n = A^I$.

Note that
\[ A^I = \prod_{i\in I} A = (I\to A).  \]

TODO IMPORTANT $\uparrow$!

\begin{lemma}
There exists a bijection $A_0\times A_1 \leftrightarrow \prod_{i\in\{0,1\}} A_i$.
\end{lemma}
\begin{proof}
The bijection is given by
\[ (a,b) \quad\leftrightarrow\quad \{(0,a),(1,b)\} \qquad \forall a\in A_0, b\in A_1. \]
\end{proof}

\subsubsection{Distributing over unions and intersections}
\begin{lemma}
Let $\{A_{i}\}_{i \in I}$ and $\{B_{j}\}_{j \in J}$ be indexed families of sets, indexed over the same index family $I$. Then
\[ \left(\prod_{i\in I}A_i\right)\cap\left(\prod_{i\in I}B_i\right) = \prod_{i\in I}(A_i\cap B_i) \qquad\text{but}\qquad \left(\prod_{i\in I}A_i\right)\cup\left(\prod_{i\in I}B_i\right) \subset \prod_{i\in I}(A_i\cup B_i). \]
\end{lemma}

\begin{lemma}
Let $\{A_{i,j}\}_{(i,j) \in I\times J}$ be an indexed family of sets. Then
\[ \bigcap_{i\in I}\left(\prod_{j\in J}A_{i,j}\right) = \prod_{j\in J}\left(\bigcap_{i\in I}A_{i,j}\right) \qquad\text{but}\qquad \bigcup_{i\in I}\left(\prod_{j\in J}A_{i,j}\right) \subset \prod_{j\in J}\left(\bigcup_{i\in I}A_{i,j}\right). \]
\end{lemma}

\subsection{Disjoint union}
\begin{definition}
The \udef{(outer) disjoint union} of a family of sets $\{A_i\}_{i\in I}$ is defined as
\[ \bigsqcup_{i\in I}A_i \defeq \bigcup_{i\in I}\{i\}\times A_i. \]
If $A,B$ are sets, then we define
\[ A\sqcup B \defeq (\{0\}\times A)\cup (\{1\}\times B). \]
\end{definition}

Notice the difference between the inner and outer disjoint union: one is a normal union that happens to be disjoint, while the other is a separate operation that may be applied to any indexed family of sets.

\begin{lemma}
Let $\{A_i\}_{i\in I}$ be an indexed family of sets. If $\{A_i\}_{i\in I}$ is pairwise disjoint, then
\[ \bigsqcup_{i\in I}A_i \twoheadrightarrowtail \biguplus_{i\in I}A_i: (i, a) \mapsto a. \]
is a bijection.
\end{lemma}
\begin{proof}
The function is clearly surjective. Now assume, towards a contradiction, that it is not injective. Then there exist distinct $(i,a)$ and $(j,a)$ in $\bigsqcup_{i\in I}A_i$, which means $a\in A_i$ and $a\in A_j$. So $a\in A_i\cap A_j$ and $\{A_i\}_{i\in I}$ is not pairwise disjoint.
\end{proof}

\begin{lemma}
Let $\{A_i\}_{i\in I}$ be a family of sets.
\[ \bigsqcup_{i\in I}A_i = \left\{ (i,a)\in I\times \bigcup_{i\in I}A_i \;|\; a\in A_i \right\} \]
\end{lemma}

\subsection{Images and preimages}
\begin{lemma}
Let $R \subseteq A\times B$ be a relation, $\{X_i\}_{i\in I}$ a family of subsets of $A$ and $\{Y_j\}_{j\in J}$ a familiy of subsets of $B$. Then
\begin{enumerate}
\item $R\left(\bigcup_{j\in J} Y_j\right) = \bigcup_{j\in J} RY_j$;
\item $R\left(\bigcap_{j\in J} Y_j\right) \subseteq \bigcap_{j\in J} RY_j$;
\item $\left(\bigcup_{i\in I} X_i\right)R = \bigcup_{i\in I} X_iR$;
\item $\left(\bigcap_{i\in I} X_i\right)R \subseteq \bigcap_{i\in I} X_iR$.
\end{enumerate}
also
\begin{enumerate} \setcounter{enumi}{4}
\item if $R$ is functional, then $R\left(\bigcap_{j\in J} Y_j\right) = \bigcap_{j\in J} RY_j$;
\item if $R$ is injective, then $\left(\bigcap_{i\in I} X_i\right)R = \bigcap_{i\in I} X_iR$.
\end{enumerate}
\end{lemma}
In particular these result hold for functions.


\part{Formal Languages and Term Rewriting}
\setcounter{chapter}{0} % Reset chapter counter
\chapter{Abstract rewriting systems}
\section{Definitions}
\begin{definition}
An \udef{(abstract) rewriting system} is a pair $\sSet{X, R}$ where
\begin{itemize}
\item $X$ is a set;
\item $R$ is a binary relation on $X$. The elements of $R$ are called \udef{(rewriting) rules}.
\end{itemize}
Let $x,y\in X$. Then we call
\begin{itemize}
\item $y$ a \udef{successor} of $x$ if $xR^+y$;
\item $y$ a \udef{direct successor} of $x$ if $xRy$.
\end{itemize}
We say
\begin{itemize}
\item $x$ is \udef{reducible} if there exists $y$ such that $x\mathrel{R}y$;
\item $x$ is \udef{irreducible} or \udef{in normal form} if $x$ is not reducible;
\item $y$ is a \udef{normal form of} $x$ if $xR^*y$ and $y$ is in normal form.
\end{itemize}
We say
\begin{itemize}
\item $x$ and $y$ are \udef{joinable} if $x\big(R^*;(R^\transp)^*\big)y$. We write $x \joins y$.
\end{itemize}
\end{definition}

\subsection{Properties of rewriting systems}
\begin{definition}
Let $\sSet{X, R}$ be a rewriting system. This system is called
\begin{itemize}
\item \udef{terminating} if $R^+$ satisfies the ascending chain condition;
\item \udef{normalising} if every element has a normal form;
\end{itemize}
it is said to
\begin{itemize}
\item have the \udef{Church-Rosser} property if $\forall x,y\in X: \; x\mathrel{\equiv_R}y \implies x \joins y$;
\item be \udef{confluent} if $\forall x,y_1, y_2\in X: \; \big(x\mathrel{R^*}y_1\big) \land \big(x\mathrel{R^*}y_2\big) \implies y_1 \joins y_2$;
\item be \udef{semi-confluent} if $\forall x,y_1, y_2\in X: \; \big(x\mathrel{R}y_1\big) \land \big(x\mathrel{R^*}y_2\big) \implies y_1 \joins y_2$;
\end{itemize}
\end{definition}

\begin{lemma}

\end{lemma}
By \ref{welfoundedACC}, we have that 

\chapter{Strings and term rewriting}
\section{Strings}
\begin{definition}
A \udef{string} in an \udef{alphabet} $A$ is a sequence whose domain is a set of the form $\interval[co]{0,n}$ and whose codomain is the set $A$. The elements of $A$ are called \udef{symbols}.

The size $n$ of the domain is the \udef{length} of the sequence, denoted $\len(x)$. Thus $x: \interval[co]{0,\len(x)}\to A$.


We identify $n$ with the corresponding von Neumann ordinal to get
\begin{align*}
A^{n} &\defeq A^{\interval[co]{0,n}} \\
A^* &\defeq \bigcup\setbuilder{ A^{n}}{n\in \N}.
\end{align*}
We call $A^*$ the \udef{Kleene closure} of $A$.

Let $a_0,\ldots, a_{n-1}\in A$. Then we have the string
\[ \seq{a_0,\ldots, a_{n-1}} \defeq \{(0,a_0),\ldots, (n-1,a_{n-1})\} \in A^{n}. \]

Let $u\in A^*$ be a string. We take all indices modulo $\len(x)$. In particular negative indices can be used to count back from the end of the string.

\begin{itemize}
\item We allow the domain of a string to be $\emptyset = \interval[co]{0,0}$. There is a unique string with this domain, the \udef{empty string} $\seq{}$. We have $\seq{}\in A^*$.
\item Let $u\in A^{n}, v\in A^{m}$ be strings. The \udef{concatenation} of $u$ and $v$ is the string
\[ u\star v \defeq \seq{u_0,\ldots, u_{n-1},v_0,\ldots, v_{m-1}} \in A^{n+m}. \]
\item Let $u\in A^{n}$ be a string. The \udef{reverse} string $\reverse{u}$ is defined by
\[ \reverse{u} \defeq \seq{u_{n-1}, \ldots, u_0}. \]
\end{itemize}
We call a subsequence a \udef{substring} if its domain is an interval subset.
\end{definition}
Notice that we can view tuples as strings of length two (i.e.\ there is a bijection $A\times A \leftrightarrow A^{2}$ for all sets $A$). Similarly an $n$-tuple can be seen as a string of length $n$. So sometimes we write
\[ A^* = \bigcup^\infty_{n=0}A^n = A^{<\omega}\]
where $A^{<\omega}$ is just a notational equivalent.

We can think of $A^*$ as a generalisation of $\N$, with $\seq{}$ instead of $0$ and appending operators
\[ S_a(u) = u\concat \seq{a} \]
for all $a\in A$. In this way $\N \cong \{1\}^*$.


\begin{lemma} \label{stringPigeonholePrinciple}
Let $A$ be a finite alphabet and $u\in A^*$. If $\#A < \len(u)$, then $u$ has a repeating character.
\end{lemma}
\begin{proof}
Assume $\#A < \len(u)$. Then we have the inclusion $\interval[co]{0,\#A} \hookrightarrow \interval[co]{0,\len(u)}$ by \ref{naturalNumbersInequalityInclusion}. Also $A =_c \interval[co]{0,\#A}$ by definition.


We need to show that $u$ is not an injection. Assume, towards a contradiction, that it is injective. 
Then $\begin{tikzcd}
A \ar[r, "{=_c}"] &\interval[co]{0,\#A} \ar[r,hook] & \interval[co]{0,\len(u)} \ar[r, "u"] & A
\end{tikzcd}$
is injective. By the pigeonhole principle \ref{pigeonholePrinciple}, this map is bijective. Thus $u$ is surjective, so $\interval[co]{0,\len(u)} =_c A$, which means that $\len(u) = \#A$ by the unicity in \ref{finiteSetNumberOfElements}. This is a contradiction.
\end{proof}

\begin{theorem}[String recursion theorem] \label{stringRecursion}
Let $A,E$ be sets, $a\in E$, and $h:E\times A\to E$ some function.

There is exactly one function $f: A^*\to E$ which satisfies
\[ \begin{cases}
f(\seq{}) = a, \\
f(u\concat \seq{x}) = h(f(u), x) & (u\in A^*, x\in A).
\end{cases} \]
\end{theorem}
\begin{proof}
Define a function $\phi: \N \times A^*\to E$ recursively such that it satisfies
\begin{align*}
\phi(0,u) &= a \\
\phi(n+1,u) &= h(\phi(n,u), u(n))
\end{align*}
and set $f(u) = \phi(\len(u), u)$. Proving that $f$ satisfies the second equality and the uniqueness of $f$ goes by induction on $\len(u)$.
\end{proof}
Like before, a version of the theorem can also be stated for recursion with parameters.

\subsection{Ordering string}
\begin{definition}
Let $\sSet{A,\leq}$ be a linearly ordered alphabet. Then
\begin{itemize}
\item the \udef{lexicographic order} $<_l$ on $A^*$ is defined by
\[ u <_l v \qquad\iff\qquad u_k < v_k \;\text{where $k = \min\setbuilder{n\in\N}{n \leq \len(u), n\leq \len(v), u_n \neq v_n}$}. \]
\item the \udef{shortlex order} or \udef{strong order} $<_{s}$ on $A^*$ is defined by
\[ u <_l v \qquad\iff\qquad \begin{cases}
\len(u) < \len(v) & \text{or} \\
\big(\len(u) = \len(v)\big) \land (u <_l v).
\end{cases}\]
\end{itemize}
\end{definition}

\subsection{Slices and substrings}
\begin{definition}
Let $A$ be an alphabet, $u$ a string in $A$ and $S\subseteq \dom(u)$. Then the $S$-\udef{slice} of $u$ is the string
\[ u[S] \defeq \begin{tikzcd}
\interval[co]{0,\len(S)} \ar[r, "\cong"] & S \ar[r, "u|_S"] & A,
\end{tikzcd} \]
where $\interval[co]{0,\len(S)} \cong S$ is the unique order similarity of \ref{WOSetsUniqueSimilarity} (where $S$ has the subspace order).

We also allow $S$ to not be a subset of $\dom(u)$. In this case it is first replaced by $S\cap \dom(u)$.
\end{definition}

\section{Term rewriting}
\begin{definition}
A \udef{string rewriting system} or \udef{semi-Thue system} is a pair $\sSet{\Sigma, R}$ where
\begin{itemize}
\item $\Sigma$ is a set (the \udef{alphabet});
\item $R$ is a binary relation on $\Sigma^*$. The elements of $R$ are called \udef{(rewriting) rules}.
\end{itemize}
\end{definition}

\chapter{Languages}
\begin{definition}
A \udef{language} is a set of strings.
\end{definition}

\begin{definition}
Let $L$ be a language in an alphabet $A$. Then we define the \udef{Kleene closure} of the language $L$ as
\[ L^* \defeq \Closure_\concat(L) \cup \{\seq{}\}. \]
\end{definition}
Note that $A^* = \setbuilder{\seq{a}}{a\in A}^*$.

\section{Grammars}
\begin{definition}
A \udef{(formal) grammar} is a tuple $G = \sSet{\Sigma, N, R, S}$ where
\begin{itemize}
\item $\Sigma$ is a finite set, called the \udef{alphabet} or set of \udef{terminal symbols};
\item $N$ a finite set, disjoint from $\Sigma$, of \udef{nonterminal symbols};
\item $R$ is relation on $(\Sigma \uplus V)^*$ that is a finite set; it is also called the set of \udef{production rules};
\item $S\in N$.
\end{itemize}
\end{definition}

\subsection{Chomsky hierarchy}
\subsubsection{Regular}
\subsubsection{Context-free}
\subsubsection{Context-sensitive}
\subsubsection{Recursively enumerable}


\section{Automata}
\begin{definition}
An \udef{automaton} is a $5$-tuple $M = \sSet{\Sigma, Q, \delta, q_0, F}$, where
\begin{itemize}
\item $\Sigma$ is a finite set, called the \udef{input alphabet} of the automaton,
\item $Q$ is a set of \udef{states},
\item $\delta: Q\times \Sigma\to Q$ is a function, called the \udef{transition function},
\item $q\in Q$ is the \udef{start state}, and
\item $F\subseteq Q$ is the set of \udef{accept states} or \udef{final states}.
\end{itemize}
We call the automaton \udef{finite} if $Q$ is finite.

The automaton $M$ defines a function $e_M: \Sigma^* \to Q^*$ recursively by
\[ \begin{cases}
e_M(\seq{}) = \seq{q_0}, \\
e_M(u\concat \seq{x}) = e_M(u)\concat \delta(e(u)_{-1}, x) & (u\in \Sigma^*, x\in \Sigma).
\end{cases} \]

We say $M$ \udef{accepts} a string $u\in \Sigma^*$ if $e_M(u)_{-1} \in F$.

The set of all strings in $\Sigma^*$ that are accepted by the automaton is the language \udef{recognised} by $M$, or just the \udef{language of $M$}:
\[ L_M \defeq \setbuilder{u\in\Sigma^*}{e_M(u)_{-1} \in F}. \]
\end{definition}

\begin{lemma} \label{automatonExecutionLength}
Let $M = \sSet{\Sigma, Q, \delta, q_0, F}$ be an automaton and $u\in \Sigma^*$. Then $\len(e_M(u)) = \len(u)+1$.
\end{lemma}

\begin{proposition}
Let $M_1 = \sSet{\Sigma, Q_1, \delta_1, q_{0,1}, F_1}$ and $M_2 = \sSet{\Sigma, Q_2, \delta_2, q_{0,2}, F_2}$ be automata that accept the languages $L_{M_1}$ and $L_{M_2}$ in the same alphabet $\Sigma$. Define
\[ \delta = \curry_2^{-1}\Big(a\mapsto (\delta_1(-, a), \delta_2(-, a))\Big). \]
Then
\begin{enumerate}
\item $M_{\cup} = \sSet{\Sigma, Q_1\times Q_2, \delta, (q_{0,1}, q_{0,2}), (F_1\times Q_1) \cup (F_2\times Q_2)}$ is an automaton that recognises $L_{M_1}\cup L_{M_2}$;
\item $M_{\cap} = \sSet{\Sigma, Q_1\times Q_2, \delta, (q_{0,1}, q_{0,2}), F_1 \times F_2}$ is an automaton that recognises $L_{M_1}\cap L_{M_2}$.
\end{enumerate}
\end{proposition}

\subsection{State diagrams}
TODO. Double circle for accept states.

\subsection{Constructions}
\subsubsection{Changing initial state}
\begin{definition}
Let $M = \sSet{\Sigma, Q, \delta, q_{0}, F}$ be an automaton and $q\in Q$. Then $M|_{q}$ is the automaton $\sSet{\Sigma, Q, \delta, q, F}$, i.e.\ the same automaton except the initial state has been replaced by $q$.
\end{definition}

\begin{lemma} \label{automatonRunFactorisation}
Let $M = \sSet{\Sigma, Q, \delta, q_{0}, F}$ be an automaton and $x,y\in \Sigma^*$. Then
\[ e_M(x \concat y) = e_M(x)\concat \big(e_{M|_{q_x}}(y)[1:]\big), \]
where $q_x = e_M(x)_{-1}$.
\end{lemma}
\begin{proof}
Consider the functions $y\mapsto e_M(x \concat y)$ and $y\mapsto e_M(x)\concat e_{M|_{q_x}}(y)$. They satisfy the same recursion relation and thus are the same by \ref{stringRecursion}.
\end{proof}

\subsection{Non-deterministic automata}
\begin{definition}
A \udef{non-deterministic automaton} is a $5$-tuple $M = \sSet{\Sigma, Q, \delta, Q_0, F}$, where
\begin{itemize}
\item $\Sigma$ is a finite set of symbols, called the \udef{input alphabet} of the automaton,
\item $Q$ is a set of \udef{states},
\item $\delta: Q\times \Sigma \to \powerset(Q)$ is a function, called the \udef{transition function},
\item $Q_0\subseteq Q$ is the \udef{start state}, and
\item $F\subseteq Q$ is the set of \udef{accept states} or \udef{final states}.
\end{itemize}
We call the non-deterministic automaton \udef{finite} if $Q$ is finite.

The automaton $M$ defines a function $e_M: \Sigma^* \to \powerset(Q^*)$ recursively by
\[ \begin{cases}
e(\seq{}) = \setbuilder{\seq{q_0}}{q_0\in Q_0}, \\
e(u\concat \seq{x}) = \setbuilder{v\concat q}{v\in e(u), q\in \delta(v_{-1}, x)} & (u\in \Sigma^*, x\in \Sigma).
\end{cases} \]

We say $M$ \udef{accepts} a string $u\in \Sigma^*$ if $\exists r\in e_M(u): r_{-1} \in F$.

The set of all strings in $\Sigma^*$ that are accepted by the automaton is the language \udef{recognised} by $M$, or just the \udef{language of $M$}:
\[ L_M \defeq \setbuilder{u\in\Sigma^*}{\exists r\in e_M(u): r_{-1} \in F}. \]
\end{definition}

\begin{lemma} \label{automatonAsNDAutomaton}
Let $M = \sSet{\Sigma, Q, \delta, q_0, F}$ be an automaton. Consider the non-deterministic automaton $M' = \sSet{\Sigma, Q, \delta', Q_0', F}$, where
\begin{itemize}
\item $Q_0' = \{q_0\}$; and
\item $\delta' = \{\delta(-)\}$.
\end{itemize}
Then
\begin{enumerate}
\item $e_{M} = \{e_{M'}(-)\}$;
\item $M$ and $M'$ accept the same language.
\end{enumerate}
\end{lemma}
\begin{proof}
Point (2) follows straight from point (1) and point (1) is easy to prove by induction on input string length.
\end{proof}

\begin{proposition}
Let $M_1 = \sSet{\Sigma, Q_1, \delta_1, Q_{0,1}, F_1}$ and $M_2 = \sSet{\Sigma, Q_2, \delta_2, Q_{0,2}, F_2}$ be non-deterministic automata that accept the languages $L_{M_1}$ and $L_{M_2}$ in the same alphabet $\Sigma$.
Then
\begin{enumerate}
\item $M_{\cup} = \sSet{\Sigma, Q_1\sqcup Q_2, (\delta_1 \bbslash \delta_2), Q_{0,1}\sqcup Q_{0,2}, F_1 \sqcup F_2}$ is a non-deterministic automaton that recognises $L_{M_1}\cup L_{M_2}$;
\item $M_\star = \sSet{\Sigma, (Q_1\setminus F_1) \sqcup (F_1\times Q_{0,2}) \sqcup (Q_2\setminus Q_{0,2}), \delta', Q_{0,1}, F_2}$, where
\[ \delta': (q,s)\mapsto \begin{cases}
\big(\delta_1(q,s)\setminus F_1\big) \sqcup \Big(\big(\delta_1(q,s)\cap F_1\big)\times Q_{0,2}\Big) & (q\in Q_1\setminus F_1) \\
\big(\delta_1(\proj_1(q),s)\setminus F_1\big) \sqcup \Big(\big(\delta_1(\proj_1(q),s)\cap F_1\big)\times Q_{0,2}\Big) \sqcup \delta_2\big(\proj_2(q), s\big) & (q\in F_1\times Q_{0,2}) \\
\delta_2(q,s) & (q\in Q_2\setminus Q_{0,2})
\end{cases} \]
is a non-deterministic automaton that recognises $L_{M_1}\concat L_{M_2}$.
\end{enumerate}
\end{proposition}

TODO: define (better syntax!!!!!!) + picture gluing of automata.

\begin{proposition}
Let $M = \sSet{\Sigma, Q, \delta, Q_{0}, F}$ be a non-deterministic automata that accept the languages $L_{M}$.
Then consider the non-deterministic automaton $M' = \sSet{\Sigma, Q', \delta', Q_{0}, F'}$, where
\begin{itemize}
\item $Q' = Q_0 \sqcup (Q\setminus F) \sqcup (F\times Q_0)$;
\item $\delta': (q,s) \mapsto \begin{cases}
\big(\delta(q,s)\setminus F\big) \sqcup \Big(\big(\delta(q,s)\cap F\big) \times Q_0\Big) & (\text{$q\in Q_0$ or $q\in Q\setminus F$}) \\
\begin{multlined}\big(\delta(\proj_1(q),s)\setminus F\big) \sqcup \Big(\big(\delta(\proj_1(q),s)\cap F\big) \times Q_0\Big) \vspace{-1em} \\ \cup \big(\delta(\proj_2(q),s)\setminus F\big) \sqcup \Big(\big(\delta(\proj_2(q),s)\cap F\big) \times Q_0\Big) \end{multlined}& (q\in F\times Q_0);
\end{cases}$
\item $F' = Q_0 \sqcup (F\times Q_0)$.
\end{itemize}
Then $M'$ recognises $L_{M}^*$.
\end{proposition}
Note the use of $\cup$ vs $\sqcup$.



\subsection{Finite automata}
\begin{definition}
A language is called a \udef{regular language} if it is recognised by a finite automaton.
\end{definition}

\begin{proposition}
Let $L$ be a language in an alphabet $\Sigma$. Then $L$ is recognised by a finite automaton \textup{if and only if} it is recognised by a finite non-deterministic automaton.
\end{proposition}
\begin{proof}
The direction $\Rightarrow$ is immediate from \ref{automatonAsNDAutomaton}.

For the converse, take a finite non-deterministic automaton $M = \sSet{\Sigma, Q, \delta, Q_0, F}$. Consider the automaton $M' = \sSet{\Sigma, \powerset{Q}, \delta', Q_0, F'}$, where
\begin{itemize}
\item $\delta': \powerset{Q} \times \Sigma \to \powerset{Q}: (A,s) \mapsto \bigcup_{q\in A}\delta(q, s)$;
\item $F' = \setbuilder{A\in \powerset(Q)}{A\mesh F}$.
\end{itemize}
Then we need to prove $L_{M} = L_{M'}$. Indeed we can prove by induction that, for all $u\in \Sigma^*$,
\[ \setbuilder{q_{-1}}{q\in e_M(u)} = e_{M'}(u)_{-1}. \]
Call this set $A$. We have that $M$ accepts $u$ iff $F\mesh A$ and $M'$ accepts $u$ iff $A\in F'$. By definition of $F'$, these cases are the same.
\end{proof}


\subsubsection{The pumping lemma}
\begin{theorem}[Pumping lemma]
Let $L$ be a regular language. Then there exists $p\in \N$ such that for all $s \in L$, we can write $s = x\concat y \concat z$ with
\begin{enumerate}
\item $x\concat y^k \concat z \in L$ for all $k\in\N$;
\item $\len(y) \geq 1$;
\item $\len(x\concat y) \leq p$.
\end{enumerate}
\end{theorem}
We also allow $k=0$, so $x\concat z$ must be an element of $L$.
\begin{proof}
Let $M = \sSet{\Sigma, Q, \delta, q_0, F}$ be a finite automaton that recognises $L$. Set $p = \#Q$. If $\len(s) \leq p$, then we can set $x=\seq{}, y= s$ and $z=\seq{}$.

Now assume $\len(s) > p$ and consider $e_M(s)$, which is a string in $Q$. By \ref{automatonExecutionLength}, we have
\[ \len(e_M(s)) = \len(s)+1 > p +1 = \#Q+1. \]
By \ref{stringPigeonholePrinciple} there must be a state in $e_M(s)[0:\#Q+1]$ that repeates. Let $k_1, k_2$ be the two different indices of this state in $e_M(s)$.

Then we may set $x = s[0:k_1]$, $y = s[k_1:k_2]$ and $z = s[k_2:]$.

Point (3) is satisfied as $\len(x\concat y) = k_2 \leq \#Q = p$.

Point (2) is satisfied as $k_1\neq k_2$.

For point (1), define the automaton $M' = \sSet{\Sigma, Q, \delta, e_M(x)_{-1}, F}$ and $M^{\prime\prime} = \sSet{\Sigma, Q, \delta, e_M(x\concat y)_{-1}, F}$.
It is enough to prove that
\[ e_M(x\concat y^k \concat z) = e_M(x)\concat \big(e_{M'}(y)[1:]\big)^k \concat \big(e_{M^{\prime\prime}}(z)[1:]\big). \]
We use \ref{automatonRunFactorisation}. We then only need to prove that $\big(e_{M'}(y^k)[1:]\big) = \big(e_{M'}(y)[1:]\big)^k$. This is evident by induction on $k$, using \ref{automatonRunFactorisation}.
\end{proof}

\begin{example}
The following languages are not regular:
\begin{itemize}
\item $\setbuilder{\seq{0}^n\seq{1}^n}{n\geq 0}$;
\item $\setbuilder{u\in \{0, 1\}^*}{\text{$u$ contains an equal number of $1$s and $0$s}}$;
\item $\setbuilder{u\concat u}{u\in \{0,1\}^*}$;
\item $\setbuilder{\seq{1}^{n^2}}{n\in \N}$;
\item $\setbuilder{\seq{0}^i\seq{1}^j}{i>j \in \N}$. We prove this by contradiction. Assume $p$ is the pumping length and consider $\seq{0}^{p+1}\seq{1}^p$, which is a string in the language. Then $y$ must consist of a non-empty string of zeros. Now the string $xy^0z = xz$ must be in the language by the pumping lemma. However in this case $xz = \seq{0}^l\seq{1}^j$ where $l\leq j$, which means that $xz$ is not an element of the language. This is a contradiction.
\end{itemize}
\end{example}

\part{Order Theory}
\setcounter{chapter}{0} % Reset chapter counter
\chapter{Ordered sets}
\section{Order relations}
\begin{definition}
All types of orders are transitive.
\begin{itemize}
\item A \udef{preorder} or \udef{quasiorder} $\precsim$ is also reflexive.
\item A \udef{total preorder} is also connex.
\item A \udef{partial order} $\preceq$ is also reflexive and anti-symmetric.
\item A \udef{strict partial order} $\prec$ is also irreflexive (or, equivalently, asymmetric).
\item A \udef{total order} $\leq$ is also reflexive, anti-symmetric and connex. Connex means all elements are comparable.
\item A \udef{strict total order} $<$ is also trichotomous.
\end{itemize}
An \udef{ordered set} is a pair $\sSet{P,\Yleft}$ such that $P$ is a set and $\Yleft$ is an order on $P$. We call this ordered set
\begin{itemize}
\item a \udef{proset} if $\Yleft$ is a preorder;
\item a \udef{poset} if $\Yleft$ is a partial order;
\end{itemize}
\end{definition}

A total preorder is necessarily also reflexive by \ref{connexityConsequences}.

A strict partial order is necessarily also anti-symmetric by \ref{asymmetricIrreflexive} and \ref{asymmetryAntisymmetry} (which is why there is no separate notion of ``strict preorder'').

A strict total order is like a total order that is strict (i.e. irreflexive instead of reflexive).  However by \ref{connexityConsequences}, no irreflexive relation can be connex, so we relax the requirement to semi-connexity:
\[ \begin{cases}
\text{transitive} \\ \text{trichotomous}
\end{cases} \iff \begin{cases}
\text{transitive} \\ \text{irreflexive} \\ \text{\textit{semi}-connex}
\end{cases}. \]

If we say $\sSet{P,\Yleft}$ is an ordered set without any other qualifiers, we only assume $\Yleft$ is transitive.

\begin{example}
\begin{itemize}
\item Every equivalence relation is a preorder. Conversely, symmetric preorders are equivalence relations.
\item Let $X$ be a set. Then $\sSet{\powerset(X), \subseteq}$ is a poset.
\item Let $A$ be a set and $\id_A$ the identity relation on $A$. Then $\sSet{A,\id_A}$ is a poset. Such posets are called \udef{discrete posets}. Two elements $x,y\in A$ are comparable if and only if $x=y$.
\item Let $B$ be a set and $\bot\in B$. Then the order $\preceq$ defined by
\[ x \preceq y \defequiv (x=\bot) \lor (x = y) \]
is a partial order. Such posets are called \udef{flat posets}.
\end{itemize}
\end{example}

\begin{lemma}
For any binary homogeneous relation $R$,
\begin{enumerate}
\item the reflexive transitive closure, $R^{+=}$, is a preorder;
\item the left residual, $R\backslash R = \overline{R^\transp;\overline{R}}$ is a preorder.
\end{enumerate}
\end{lemma}

\begin{lemma} \label{preorderEquivalence}
Let $(P, \precsim)$ be a proset. Then $\precsim \cap \precsim^\transp$ is an equivalence relation.
\end{lemma}


We can decompose $\Yleft = \Yleft_S \cup \Yleft_A$, as in \ref{symmetricAsymmetricDecomposition}, into a symmetric and asymmetric part.
\begin{itemize}
\item By \ref{asymmetricIrreflexive} an order is strict (i.e. irreflexive) if and only if it is asymmetric: $\Yleft_S = E$ and $\Yleft_A = \Yleft$.
\item For non-strict partial and total orders we have $\Yleft_S = \id$.
\end{itemize}

So we can turn a partial / total order into a strict partial / total order by removing $\id$. Conversely, we can turn a strict partial / total order into a partial / total order by adding $\id$.

\begin{proposition}
\mbox{}
\begin{enumerate}
\item Let $(P, \precsim)$ be a proset. The relation $\precsim \cap\; \overline{\precsim}^\transp$
is a strict partial order.
\item Let $(P, \preceq)$ be a poset. The relation $\preceq \cap \;\overline{\preceq}^\transp = \preceq \cap\; \overline{\id}_P$
is a strict partial order.
\item Let $(P, \prec)$ be a strict partial order. Then $\prec \cup \id_P$ is a partial order.
\item Let $(P, \leq)$ be a totally ordered set. The relation $\leq \cap\; \overline{\leq}^\transp = \leq \cap\; \overline{\id}_P = \overline{\leq}^\transp$
is a strict total order.
\item Let $(P, <)$ be a strict total order. Then $< \cup \id_P$ is a total order.
\end{enumerate}
Multiple different preorders are associated to the same strict partial orders. For partial and total orders the association is one-to-one.
\end{proposition}

\section{Covering relations}
\begin{definition}
Let $(P,\Yleft)$ be a an order relation and $x,y\in P$. We say $y$ \udef{covers} $x$ if
\begin{itemize}
\item $x\Yleft y$;
\item $x\neq y$
\item $\nexists z\in P\setminus\{x,y\}: x\Yleft z \Yleft y$).
\end{itemize}
We write $x\lessdot y$.
\end{definition}

\begin{lemma}
Let $(P,\Yleft)$ be an ordered set. Let $\Yleft_A$ be the asymmetric part of $\Yleft$. Then $\lessdot = \Yleft_A \cap \overline{\Yleft_A^2}$.
\end{lemma}

\begin{lemma}
Let $P$ be a preordered set. Then $y$ covers $x$ \textup{if and only if} $x\neq y$ and $[x,y] = \{x,y\}$.
\end{lemma}

\subsection{Hasse diagrams}
A \udef{Hasse diagram} is a graphical depiction of an order relation. Each element of the ordered set is a point and points are connected such that:
\begin{enumerate}
\item if $x \Yleft y$, then the point for $x$ is drawn lower than the point for $y$;
\item two elements $x,y$ are connected if $y$ covers $x$ or $x$ covers $y$.
\end{enumerate}

TODO is transitive (reflective) reduction! (+ link Galois)

\begin{lemma}
The Hasse diagrams of partial orders are acyclic due to anti-symmetry.
\end{lemma}

\begin{example}
The power set $\powerset\{a,b,c\}$ can be ordered by the inclusion relation $\subseteq$. The following is a Hasse diagram for this ordered set:
\begin{center}
\begin{tikzcd}
 & \{a,b,c\} & \\
\{a,b\}\ar[ru, dash] & \{a,c\}\ar[u,dash] & \{b,c\}\ar[lu,dash] \\
\{a\} \ar[u,dash] \ar[ru, dash] & \{b\} \ar[lu, dash] \ar[ru, dash] & \{c\} \ar[lu,dash] \ar[u, dash] \\
 & \emptyset \ar[lu, dash] \ar[u, dash] \ar[ru, dash] &
\end{tikzcd}
\end{center}
\end{example}

\section{The dual of an ordered set}
\begin{definition}
For any ordered set $(P, \Yleft)$ the \udef{dual} of $P$ is the ordered set $(P^o, \Yleft^o)$, where $P^o \defeq P$ and $\Yleft^o \defeq \Yleft^\transp = \Yright$.
\end{definition}
This means $\forall a,b\in P: a\Yleft b \iff b\Yleft^o a$.

\begin{lemma}
The dual of an ordered set of a particular type is an ordered set of the same type.
\end{lemma}
\begin{corollary}
All statements about all ordered sets also hold for all duals of ordered sets.
\end{corollary}

\section{Functions on ordered sets}
TODO define for all relations
\begin{definition}
Let $(P, \Yleft_P)$ and $(Q, \Yleft_Q)$ be prosets and $f: P\to Q$ a function. We say
\begin{itemize}
\item $f$ is \udef{order-preserving}, \udef{isotone}, \udef{monotonically increasing} or just \udef{increasing} if
\[ \forall x,y\in P: x\Yleft_P y \implies f(x)\Yleft_Q f(y); \]
\item $f$ is \udef{order-reflecting} if
\[ \forall x,y\in P: f(x)\Yleft_Q f(y) \implies x\Yleft_P y; \]
\item $f$ is an \udef{order embedding} if it is order-preserving and order-reflecting:
\[ \forall x,y\in P: x\Yleft_P y \iff f(x)\Yleft_Q f(y); \]
\end{itemize}
and, dually, we say
\begin{itemize}
\item $f$ is \udef{order-reversing}, \udef{antitone}, \udef{monotonically decreasing} or just \udef{decreasing} if
\[ \forall x,y\in P: x\Yleft_P y \implies f(x) \Yright_Q f(y); \]
\item $f$ is \udef{reverse order-reflecting} if
\[ \forall x,y\in P: f(x)\Yleft_Q f(y) \implies x\Yright_P y; \]
\item $f$ is a \udef{reverse order embedding} if it is order-reversing and reverse order-reflecting:
\[ \forall x,y\in P: x\Yleft_P y \iff f(x)\Yright_Q f(y). \]
\end{itemize}
Finally
\begin{itemize}
\item $f$ is \udef{monotone} or \udef{monotonic} if it is either order-preserving or order-reversing.
\end{itemize}

The adjectives ``strict'' and ``weak'' can be added to any of these cases to identify whether the (weak) order or the associated strict order is meant.
\end{definition}
An order isomorphism, or \udef{similarity}, is a bijective order embedding. Let $U,V$ be ordered sets. We write $U =_o V$ if $U$ and $V$ are order isomorphic and $U \neq_o V$ if not.

\begin{lemma}
Let $P,Q$ be ordered sets and $f: P\to Q$ order-reflecting. Then
\[ \forall y,z \in f^{-1}[f(x)]: y\Yleft z \land z\Yleft y. \]
\end{lemma}
\begin{proof}
We calculate
\[ y\in f^{-1}[f(x)] \iff f(y) = f(x) \implies y\Yleft x \land x\Yleft y. \]
The same is true for $z \in f^{-1}[f(x)]$, so we conclude by transitivity.
\end{proof}
TODO: make corollary:
\begin{lemma} \label{orderReflectionIsInjective} \label{strictOrderPreservationIsInjective}
Let $P,Q$ be posets and $f: P\to Q$. The following are sufficient conditions for $f$ to be injective:
\begin{enumerate}
\item $f$ is order-reflecting;
\item $f$ is strictly order-preserving and $P$ is totally ordered.
\end{enumerate}
\end{lemma}
\begin{proof}
(1) If $f(x) = f(y)$, then $x\preceq y$ and $y\preceq x$, so $x=y$ by anti-symmetry.

(2) Let $f$ be the strict order-preserving function and $x,y\in P$. Assume $f(x)=f(y)$. Assume, towards a contradiction, that $x\neq y$, then either (by totality) $x< y$ and $f(x) < f(y)$ or $y < x$ and $f(y)<f(x)$. Both cases contradict $f(x)=f(y)$.
\end{proof}

\begin{lemma}
Let $f:P\to Q$ be a function between posets. Then
\begin{enumerate}
\item If $f$ preserves strict order, then it is order-preserving.
\item If $f$ is order-reflecting, then it reflects strict order.
\end{enumerate}
If $f$ is injective, the opposite implications also hold.
\end{lemma}
\begin{corollary}
Let $f: P\to Q$ be a function between posets.
\begin{enumerate}
\item If $f$ is an order embedding, then it is a strict order embedding.
\item If $P$ is totally ordered, the converse implication holds as well.
\end{enumerate}
\end{corollary}
\begin{proof}
Use \ref{orderReflectionIsInjective}.
\end{proof}

The converse does not hold in general if $P$ is a poset because a strict order embedding between posets is not necessarily injective. A counterexample is
\[
\begin{tikzcd}
& \top & \\
x \ar[ur, dash] & & y \ar[ul, dash] \\
& \bot \ar[ul, dash] \ar[ur, dash] &
\end{tikzcd} \qquad \begin{tikzcd} {} \arrow[rr, "f"] & {}& {}\end{tikzcd} \qquad \begin{tikzcd}
f(\top) \\ f(x) = f(y) \ar[u,dash] \\ f(\bot) \ar[u,dash]
\end{tikzcd}.
\]

\begin{lemma} \label{equivalenceOrderPreservingReflecting}
Let $f:P\to Q$ be a function between totally ordered sets. Then
\begin{enumerate}
\item If $f$ preserves strict order, then it is order-preserving.
\item If $f$ is order-reflecting, then it reflects strict order.
\end{enumerate}
If $f$ is injective, the opposite implications also hold.
\end{lemma}
\begin{proof}
(1) Assume $f$ order-reflecting. Assume $x\leq y$. Either $f(x)\leq f(y)$ or $f(y)\leq f(x)$. If $f(y)\leq f(x)$, then
\[ y\leq x \implies x=y \implies f(x)=f(y) \implies f(x)\leq f(y). \]
So in both cases $f(x)\leq f(y)$.

(2) Assume $f$ preserves strict order. Assume $f(x) < f(y)$. Either $x<y, y<x$ or $x=y$. If either $y<x$ or $x=y$ hold, preservation of strict order yields a contradiction. So $x<y$.

(1') Assume $f$ order-preserving and injective. Assume $f(x)\leq f(y)$. Either $x\leq y$ or $y\leq x$. If $y\leq x$, then
\[ f(y)\leq f(x) \implies f(x)=f(y) \implies x= y \implies x\leq y. \]
So in both cases $x\leq y$.

(2') Assume $f$ is injective reflects strict order. Assume $x < y$. Either $f(x)<f(y), f(y)<f(x)$ or $f(x)=f(y)$. If $f(y)<f(x)$, preservation of strict order yields a contradiction. If $f(x)=f(y)$, injectivity yields a contradiction. So $f(x)<f(y)$.
\end{proof}


\begin{proposition}
Let $\sSet{P,R}$ and $\sSet{Q,\Yleft}$ be relational structures and $f:P\to Q$ a function. Assume $\Yleft$ is a preorder. Then the following are equivalent:
\begin{enumerate}
\item $f$ is relation-preserving;
\item $R;f;\Yleft \subseteq f;\Yleft$;
\item the inverse image of every principle down-set is closed under $R^\transp$;
\item $\Yleft;f^\transp; R \subseteq \Yleft;f^\transp$
\item the inverse image of every principle up-set is closed under $R$.
\end{enumerate}
\end{proposition}
\begin{proof}
$(1 \Rightarrow 2)$ By \ref{relationPreserving} we have $R;f \subseteq f;\Yleft$. Then $R;f;\Yleft \subseteq f;\Yleft^2 \subseteq f;\Yleft$ by transitivity.

$(2 \Rightarrow 1)$ By reflexivity we have $\id_Q \subseteq \Yleft$ and so $R;f \subseteq R;f;\Yleft \subseteq f;\Yleft$. Thus $f$ is relation-preserving by \ref{relationPreserving}.

$(3)$ Is simply a translation of $\forall x: R;f;Sx \subseteq f;Sx$.

$(4)$ Similar to (2), except using a different part of \ref{relationPreserving}.

$(3)$ Is simply a translation of $\forall x: xS;f^\transp; R \subseteq xS;f^\transp;f^\transp$.
\end{proof}


\begin{definition}
A function $f: P\to P$ on an ordered set $(P,\Yleft)$ into itself is
\begin{itemize}
\item \udef{expansive} if $\forall x\in P: x\Yleft f(x)$;
\item \udef{contractive} if $\forall x\in P: f(x)\Yleft x$.
\end{itemize}
\end{definition}

\begin{lemma}
Let $f:P\to P$ be a function on an ordered set $(P,\Yleft)$. Consider $(P\to P)$ to be ordered by pointwise order. Then the following are equivalent:
\begin{enumerate}
\item $f$ is expansive;
\item $\id_P \Yleft f$;
\item $\graph(f) \subseteq \graph(\Yleft)$.
\end{enumerate}
The following are also equivalent:
\begin{enumerate}
\item $f$ is contractive;
\item $f \Yleft \id_P$;
\item $\graph(f) \subseteq \graph(\Yleft^\transp)$.
\end{enumerate}
\end{lemma}

\subsection{Pointwise order}
TODO

\subsection{Poset Galois connections}
\url{file:///C:/Users/user/Downloads/(Mathematics%20and%20Its%20Applications%20565)%20Marcel%20Ern%C3%A9%20(auth.),%20K.%20Denecke,%20M.%20Ern%C3%A9,%20S.%20L.%20Wismath%20(eds.)%20-%20Galois%20Connections%20and%20Applications-Springer%20Netherlands%20(2004).pdf}

TODO \url{https://en.wikipedia.org/wiki/Residuated_mapping}

\url{https://www.logicmatters.net/resources/pdfs/Galois.pdf}

\url{https://sciendo.com/pdf/10.2478/ausm-2014-0019}

\begin{proposition}
Let $\sSet{P,\leq_P}$ and $\sSet{Q,\leq_Q}$ be posets and let $f: P\to Q$, $g:Q\to P$ be order-preserving functions. Then the following are equivalent:
\begin{enumerate}
\item $(f,g)$ is a Galois connection;
\item $f;\leq_Q \;=\; \leq_P;g^\transp$;
\item $f;g \;\subseteq\; \leq_P$ and $g;f \;\subseteq\; \leq_Q^\transp$.
\end{enumerate}
\end{proposition}
\begin{proof}
$\boxed{(1) \Rightarrow (2)}$ and $\boxed{(2) \Rightarrow (3)}$ Are evident from \ref{GaloisIdentity} and \ref{reflexiveGaloisCorollary}.

We now just need to show $\boxed{(3) \Rightarrow (1)}$. We have $f;g;\leq_P \;\subseteq\; \leq_P^2 \;\subseteq\; \leq_P$, where the last inclusion follows from transitivity. The case for $g;f;\leq_Q^\transp \;\subseteq\; \leq_Q^\transp$ is similar.

Finally we show $f,g$ are generalised inverses. We have, using \ref{relationPreserving},
\[ f;g;f \subseteq \;\leq_P; f \subseteq f;\leq_Q \qquad\text{and}\qquad f;g;f \subseteq f;\leq_Q^\transp. \]
So for all $x\in P$ we have $(f;g;f)(x) \leq_Q f(x)$ and $(f;g;f)(x) \geq_Q f(x)$, meaning $(f;g;f)(x) = (f\circ g\circ f)(x) = f(x)$.

The case for $gfg$ is similar.
\end{proof}

\begin{proposition}
Let Let $\sSet{P,\leq_P}$ and $\sSet{Q,\leq_Q}$ be posets and $f: P\to Q$ an order-preserving function. There is at most one function $g: Q\to P$ such that $(f,g)$ is a Galois connection.
\end{proposition}

\begin{definition}
Let $\sSet{P,\leq_P}$ and $\sSet{Q,\leq_Q}$ be posets. Let $f: P\to Q$ and $g: Q \to P$ be functions such that $(f,g)$ is 
\begin{itemize}
\item The pair $(\rhd, \lhd)$ is called a \udef{(monotone) Galois connection} between $P$ and $Q$.
\item The map $\rhd$ is called the \udef{lower adjoint} of $\lhd$ and the map $\lhd$ the \udef{upper
adjoint} of $\rhd$.
\item We may also call $\rhd$ a \udef{residuated map}. Then $\lhd$ is called the \udef{residual} of $\rhd$.
\end{itemize}

TODO notation: $\rhd = f^\to$ and $\lhd = f^\from$

Now assume the functions satisfy
\[ \forall p\in P, q\in Q: \; p^\rhd \Yleft_Q q \iff p \Yleft_P q^\lhd. \]
\end{definition}

\begin{lemma}
Let $P$ and $Q$ be ordered sets. If $(\rhd, \lhd)$ is a Galois connection between $P$ and $Q$, then $(\lhd, \rhd)$ is a Galois connection between $Q^o$ and $P^o$.
\end{lemma}

\begin{example}
\begin{itemize}
\item Let $P$ and $Q$ be discretely ordered sets. Then $\rhd: P\to Q$ and $\lhd: Q \to P$ form a Galois connection if and only if they are invertible and $\rhd = \lhd^{-1}$.
\item Let $P$ be an ordered set and $A\subseteq P$. Then
\[ A^\rhd = P\setminus \downset A \qquad \text{and} \qquad A^\lhd = P\setminus \upset A \]
defines a Galois connection $(\rhd, \lhd)$ between $\powerset(P)$ and $\powerset(P)^o$.
\end{itemize}
\end{example}

\begin{proposition}
Let $P$ and $Q$ be ordered sets. Consider functions $\rhd: P\to Q$ and $\lhd: Q \to P$. Then the following are equivalent:
\begin{enumerate}
\item $(\rhd, \lhd)$ is a Galois connection between $P$ and $Q$;
\item for all $q\in Q: \; \rhd^{-1}[\downset q] = \downset q^\lhd$;
\item for all $p\in P: \; \lhd^{-1}[\upset p] = \upset p^\rhd$;
\item $\rhd^{-1}: \powerset(Q) \to \powerset(P)$ maps principle down sets to principle down sets;
\item $\lhd^{-1}: \powerset(P) \to \powerset(Q)$ maps principle up sets to principle up sets;
\item for all $p\in P$ and all $q\in Q$
\begin{enumerate}
\item $\rhd$ and $\lhd$ are order-preserving;
\item $p \Yleft p^{\rhd\lhd}$ and $q^{\lhd\rhd} \Yleft q$; in other words $\rhd\lhd$ is expansive and $\lhd\rhd$ is contractive;
\end{enumerate}
\end{enumerate}
\end{proposition}
\begin{proof}
$(1 \Rightarrow 2)$ We calculate
\[ x\in \rhd^{-1}[\downset q] \iff x^\rhd \in \downset q \iff x^\rhd \Yleft q \iff x \Yleft q^\lhd \iff x\in \downset q^\lhd. \]

$(1 \Rightarrow 3)$ Similarly we calculate
\[ x\in \lhd^{-1}[\upset p] \iff x^\lhd \in \upset p \iff p \Yleft x^\lhd \iff p^\rhd \Yleft x \iff x\in \upset p^\rhd. \]

$(2 \Rightarrow 4)$ and $(3 \Rightarrow 5)$ are clear.

$(4 \Rightarrow 6)$ We fist note that $\rhd$ must be isotone by If $\rhd^{-1}: \powerset(Q) \to \powerset(P)$ maps principle down sets to principle down sets, then there must exist a function $f: Q\to P$ such that
\[ \rhd^{-1}[\downset q] = \downset f(q). \]
We claim 

$(5 \Rightarrow 6)$ 

$(6 \Rightarrow 1)$ First assume $p^\rhd \Yleft q$. From (a) we have $p^{\rhd\lhd} \Yleft q^\lhd$ and from (b) we have $p \Yleft p^{\rhd\lhd}$. Combining these gives $p\Yleft q^\lhd$.

For the converse, assume $p \Yleft q^\lhd$. From (a) we have $p^{\rhd} \Yleft q^{\lhd\rhd}$ and from (b) we have $q^{\lhd\rhd} \Yleft q$. Combining these gives $p^{\rhd} \Yleft q$.
\end{proof}
\begin{corollary}
Let $P$ and $Q$ be ordered sets and $(\rhd, \lhd)$ a Galois connection between them. Then
\[ \rhd = \rhd\circ\lhd\circ\rhd \qquad\text{and}\qquad \lhd = \lhd\circ\rhd\circ\lhd \]
\end{corollary}
\begin{proof}

\end{proof}

\begin{lemma}
Composition of Galois connections is Galois connection.
\end{lemma}

\begin{corollary}
Let $(P,\Yleft)$ be an ordered set and $A,B$ subsets of $P$. Then
\begin{enumerate}
\item $A\subseteq (A^l)^u$ and $A\subseteq (A^u)^l$;
\item if $A\subseteq B$, then $B^u\subseteq A^u$ and $B^l \subseteq A^l$;
\item $A^l = ((A^l)^u)^l$ and $A^u = ((A^u)^l)^u$.
\end{enumerate}
\end{corollary}
\begin{proof}
(1) $A^l \subseteq A^l \implies A = (A^l)^u$.

(2) $A\subseteq B \implies A \subseteq (B^u)^l \implies B^u\subseteq A^u$.

(3) $(A^l)^u \subseteq (A^l)^u \implies A^l \subseteq ((A^l)^u)^l$ and by 1. and 2. $A\subseteq (A^l)^u \implies ((A^l)^u)^l \subseteq A^l$.
\end{proof}

\begin{proposition}
Image / preimage Galois connection.
\end{proposition}


\subsubsection{Closure}

\begin{definition}
Let $A$ be a subset of an ordered set $P$. If $A = (A^u)^l$, the $A$ is called \udef{saturated}.
\end{definition}

The family of saturated sets is a Moore family. The For any $x\in P$, the downset $\downset x$ is saturated.

\begin{definition}
Let $\sSet{P,\Yleft}$ be an ordered set and let $f:P \to P$ be a function. We say $f$ is a \udef{(Moore) closure} on $P$ if it is
\begin{itemize}
\item extensive: $x \Yleft f(x)$;
\item monotone: if $A \Yleft B$, then $\closure(A) \Yleft \closure(B)$;
\item idempotent: $\closure(\closure(A)) = \closure(A)$.
\end{itemize}
We say $f$ is a \udef{dual closure} if it is
\begin{itemize}
\item contractive: $f \leq \id$
\item monotone
\item idempotent
\end{itemize}
\end{definition}

\begin{proposition}
Let $L$ be a lattice and $f$ a closure. Then $f[L]$ is a lattice with lattice operations given by
\[ a\wedge b = a\wedge_L b, \qquad a\vee b = f(a\vee_L b). \]
\end{proposition}
\begin{proof}
Take $a,b\in f[L]$. Because $f$ is extensive, we have $a\wedge_L b \leq f(a\wedge_L b)$ and because $f$ is monotone, we have $f(a\wedge_L b)\leq f(a)\wedge_L f(b)$ by \ref{orderPreservingFunctionLatticeOperations}, so
\[ f(a)\wedge_L f(b) = a\wedge_L b \leq f(a\wedge_L b) \leq f(a)\wedge_L f(b), \]
meaning that $a\wedge_L b \in f[L]$ and thus $f[L]$ is a $\wedge$-subsemilattice.

From $a\vee_L b \leq f(a\vee_L b)$, it is clear that $f(a\vee_L b)$ is an upper bound of $\{a,b\}$. Let $c$ be any other upper bound in $f[L]$. Clearly $a\vee_L b \leq c$, so $f(a\vee_L b) \leq f(c) = c$, meaning $f(a\vee_L b)$ is the least upper bound in $f[L]$.
\end{proof}
\begin{proposition} \label{completeLatticeOperationsUnderClosure}
Let $L$ be a complete lattice and $f$ a closure. Then $f[L]$ is a complete lattice with lattice operations given by
\[ \bigwedge A = {\bigwedge}_L A, \qquad \bigvee A = f({\bigvee}_L A). \]
\end{proposition}
\begin{proof}
TODO
\end{proof}

\begin{theorem}[Dedekind-MacNeille]
Every ordered set $E$ can be embedded in a Dedekind complete lattice $L$ such that meets and joins that exist in $E$ are preserved in $L$.
\end{theorem}
\begin{proof}
We use the closure $f: \powerset(E) \to \powerset(E): A\mapsto A^{ul}$.
\end{proof}

\subsubsection{Closure under a relation}

\begin{proposition}
Let $\sSet{P, \leq}$ be a partially ordered set and $\sSet{Q, \leq}$ a complete meet semi-sublattice. Then $\upset: P \to \powerset(Q)$ and $\bigwedge: \powerset(Q) \to P$ are antitone generalised inverses. 
\end{proposition}


\begin{definition}
Let $R$ be a homogeneous binary relation on a set $X$. Let $A\subseteq X$ be a subset.
\begin{itemize}
\item We call $A$ \udef{$R$-closed} if $AR \subseteq A$.
\item We define the \udef{$R$-closure} of $A$ in $X$ as
\[ \closure_R(A) \defeq \bigcap \setbuilder{B}{A \subseteq B \subseteq X \land \;\text{$B$ is $R$-closed}}. \]
\end{itemize}
\end{definition}

\begin{proposition} \label{RclosureIsClosure}
Let $R$ be a homogeneous binary relation on a set $X$ and $A\subseteq X$ a subset. Then $\closure_R$ is a proper closure operator:
\begin{enumerate}
\item $A \subseteq \closure_R(A)$;
\item if $A\subseteq B$, then $\closure_R(A) \subseteq \closure_R(B)$;
\item $\closure(\closure(A)) = \closure(A)$;
\end{enumerate}
and
\begin{enumerate} \setcounter{enumi}{3}
\item $\closure_R(A)$ is $R$-closed;
\item $\closure_R(A)$ is the smallest $R$-closed superset of $A$ in the poset $\sSet{\powerset(X),\subseteq}$;
\item $A$ is $R$-closed \textup{if and only if} $A = \closure_R(A)$.
\end{enumerate}
\end{proposition}
\begin{proof}
(1) This is clear.

(2) This follows because $\setbuilder{C}{A \subseteq C \subseteq X \land \;\text{$C$ is $R$-closed}} \supseteq \setbuilder{C}{B \subseteq C \subseteq X \land \;\text{$C$ is $R$-closed}}$.

(3) This follows because $\closure_R(A) \in \setbuilder{C}{\closure_R(A) \subseteq C \subseteq X \land \;\text{$C$ is $R$-closed}}$.

(4) We calculate
\begin{align*}
\closure_R(A)R &= \left(\bigcap \setbuilder{B}{A \subseteq B \subseteq X \land \;\text{$B$ is $R$-closed}}\right)R \\
&\subseteq \bigcap \setbuilder{BR}{A \subseteq B \subseteq X \land \;\text{$B$ is $R$-closed}} \\
&\subseteq \bigcap \setbuilder{B}{A \subseteq B \subseteq X \land \;\text{$B$ is $R$-closed}} = \closure_R(A).
\end{align*}

(5) Intersection is infimum in $\sSet{\powerset(X),\subseteq}$. (TODO terminology higher??)

(6) The direction $\Leftarrow$ is clear because $\closure_R(A)$ is $R$-closed. The converse follows from (5).
\end{proof}

\begin{lemma}
Let $R$ be a homogeneous binary relation on a set $X$ and $A\subseteq X$ a subset. Then
\begin{enumerate}
\item $\closure_R(AR) \subseteq \closure_R(A)$;
\item $\closure_R(A) = A \cup \closure_R(AR)$;
\item $\closure_R(AR) = \closure_R(A)R$.
\end{enumerate}
\end{lemma}
\begin{proof}
(1) We calculate $AR \subset \closure_R(A)R \subseteq \closure_R(A)$, using \ref{monotonicityImage} and the fact that $\closure_R(A)$ is $R$-closed.

(2) The inclusion $\closure_R(A) \supseteq A \cup \closure_R(AR)$ is given by \ref{RclosureIsClosure} and point (1).

For the converse it is enough to see that $A \cup \closure_R(AR)$ is $R$-closed:
\[ \big(A \cup \closure_R(AR)\big)R = AR \cup \closure_R(AR)R \subseteq \closure_R(AR) \subseteq A \cup \closure_R(AR), \]
where we have used that $\closure_R(AR)$ is $R$-closed.

(3) First we calculate
\[ \closure_R(A)R = \big(A \cup \closure_R(AR)\big)R = AR \cup \closure_R(AR)R \subseteq \closure_R(AR)R \subseteq \closure_R(AR) \]
where we have used point (2) and the fact that $\closure_R(AR)$ is closed.

For the converse it is enough to prove that $AR \subseteq \closure_R(A)R$ and $\closure_R(A)R$ is $R$-closed. The first follows from \ref{monotonicityImage} as does the second, with
\[ \closure_R(A)R \subseteq \closure_R(A) \implies \big(\closure_R(A)R\big)R \subseteq \closure_R(A)R. \]
\end{proof}

\begin{lemma}
Let $R$ be a homogeneous binary relation on a set $X$. Let $\im_R$ denote the function $\powerset(X) \to \powerset(X): A \mapsto AR$. Then for all $A\subseteq X$:
\[ \closure_R(A) = \bigcup \closure_{\im_R}(\powerset(A)). \]
IS THIS TRUE?
\end{lemma}


\subsubsection{Closure under a function}

\begin{lemma}
$\closure_f(A) = \setbuilder{f(a)}{a\in A}$.

$\closure_g(A) = \setbuilder{g(a,b)}{a,b\in A}$.
\end{lemma}

\section{Subsets of ordered sets}
\subsection{Up and down sets}
\begin{definition}
Let $\sSet{P, \Yleft}$ be an ordered set and $Q\subseteq P$. We call
\begin{itemize}
\item $Q$ \udef{upwards closed} or an \udef{up set} if $\forall x\in Q: \forall y\in P: x\Yleft y \implies y\in Q$;
\item $Q$ \udef{downwards closed} or a \udef{down set} if $\forall x\in Q: \forall y\in P: y\Yleft x \implies y\in Q$.
\end{itemize}
We define
\begin{itemize}
\item the \udef{upward closure} of $Q$ as $\upset Q \defeq \setbuilder{y\in P}{\exists x\in Q: x\Yleft y}$;
\item the \udef{downward closure} of $Q$ as $\downset Q \defeq \setbuilder{y\in P}{\exists x\in Q: y\Yleft x}$.
\end{itemize}
For $x\in P$,
\begin{itemize}
\item $\upset x \defeq \upset\{x\}$ is the \udef{principle up set} generated by $x$;
\item $\downset x \defeq \downset\{x\}$ is the \udef{principle down set} generated by $x$.
\end{itemize}
\end{definition}


\begin{lemma} \label{definitionUpsetDownset}
Let $\sSet{P, \Yleft}$ be an ordered set and $Q\subseteq P$. Then
\begin{enumerate}
\item $Q$ an up set iff 
\[ \forall y\in P: \Big(\exists x\in Q:  x\Yleft y\Big) \implies y\in Q; \]
\item $Q$ a down set iff 
\[ \forall y\in P: \Big(\exists x\in Q:  y\Yleft x\Big) \implies y\in Q. \]
\end{enumerate}
\end{lemma}
\begin{proof}
The following propositions are equivalent $\forall y\in P$:
\begin{align*}
&\forall x\in Q: x\Yleft y \implies y\in Q \\
&\forall x\in Q: \neg(x\Yleft y) \lor (y\in Q) \\
\Big(&\forall x\in Q: \neg(x\Yleft y)\Big) \lor (y\in Q) \\
\neg \Big(&\exists x\in Q: x\Yleft y \Big) \lor (y\in Q) \\
\Big(&\exists x\in Q:  x\Yleft y\Big) \implies y\in Q.
\end{align*}
TODO justify (in particular line 2 to 3).
\end{proof}

\begin{lemma} \label{QsubseteqDownQ} \label{upwardDownwardClosure}
Let $\sSet{P, \Yleft}$ be an ordered set and $Q\subseteq P$. Then
\begin{enumerate}
\item $Q$ is upwards closed \textup{if and only if} $\upset Q \subseteq Q$;
\item $Q$ is downwards closed \textup{if and only if} $\downset Q \subseteq Q$;
\end{enumerate}
also
\begin{enumerate} \setcounter{enumi}{2}
\item if $\Yleft$ is a preorder, then $Q \subseteq \upset Q$ and $Q \subseteq \downset Q$;
\item $\upset\upset Q \subseteq \upset Q$ and $\downset\downset Q \subseteq \downset Q$;
\item if $\Yleft$ is a preorder, then $\upset\upset Q = \upset Q$ and $\downset\downset Q = \downset Q$.
\end{enumerate}
\end{lemma}
\begin{proof}
(1,2) By \ref{definitionUpsetDownset} both statements are equivalent to
\[ \forall y\in P: y\in \upset Q \implies y\in Q. \]

(3) Reflexivity.

(4) First take $x\in \upset\upset Q$. Then there exists $y\in \upset Q$ such that $y\Yleft x$ and $q\in Q$ such that $q\Yleft y$. By transitivity $q \Yleft x$, so $x\in \upset Q$. The other statement is dual.

(5) Combine (3) and (4).
\end{proof}

\begin{lemma}
Let $P$ be an ordered set. Then $\downset\emptyset = \emptyset = \upset\emptyset$.
\end{lemma}

\begin{lemma}
Let $\sSet{P, \Yleft}$ be an ordered set and $Q\subseteq P$. Then
\begin{enumerate}
\item $\upset x = \setbuilder{y\in P}{x\Yleft y}$;
\item $\downset x = \setbuilder{y\in P}{y\Yleft x}$;
\end{enumerate}
and
\begin{enumerate} \setcounter{enumi}{2}
\item $\upset Q = \bigcup_{x\in Q} \upset x$;
\item $\downset Q = \bigcup_{x\in Q} \downset x$.
\end{enumerate}
\end{lemma}
\begin{corollary} \label{upDownsetUnionIntersection}
Let $\sSet{P,\Yleft}$ be an ordered set and $\mathcal{A}\subseteq \powerset(P)$. Then
\begin{enumerate}
\item $\upset \bigcup \mathcal{A} = \bigcup_{A\in \mathcal{A}} \upset A$ and $\downset \bigcup \mathcal{A} = \bigcup_{A\in \mathcal{A}} \downset A$;
\item $\upset \bigcap \mathcal{A} \subseteq \bigcap_{A\in \mathcal{A}} \upset A$ and $\downset \bigcap \mathcal{A} \subseteq \bigcap_{A\in \mathcal{A}} \downset A$;
\item if $\Yleft$ is a preorder, then
\[ \upset \bigcap_{A\in \mathcal{A}} \upset A = \bigcap_{A\in \mathcal{A}} \upset A \quad\text{and} \downset \bigcap_{A\in \mathcal{A}} \downset A = \bigcap_{A\in \mathcal{A}} \downset A \]
\end{enumerate}
\end{corollary}
\begin{proof}
(1) We calculate using \ref{unionIntersectionLabelSet}:
\[ \upset \bigcup \mathcal{A} = \bigcup_{x\in \bigcup\mathcal{A}} \upset x = \bigcup_{A\in \mathcal{A}}\bigcup_{x\in A}\upset x = \bigcup_{A\in \mathcal{A}} \upset A. \]

(2) Again we calculate using \ref{unionIntersectionLabelSet}:
\[ \upset \bigcap \mathcal{A} = \bigcup_{x\in \bigcap\mathcal{A}} \upset x \subseteq \bigcap_{A\in \mathcal{A}}\bigcup_{x\in A}\upset x = \bigcap_{A\in \mathcal{A}} \upset A. \]

(3) We calculate using (2) and \ref{upwardDownwardClosure}:
\[ \bigcap_{A\in \mathcal{A}} \upset A = \bigcap_{A\in \mathcal{A}} \upset\upset A \supseteq \upset \bigcap_{A\in \mathcal{A}} \upset A \supseteq \bigcap_{A\in \mathcal{A}} \upset A. \]
\end{proof}
\begin{corollary} \label{unionIntersectionDownUpSets}
Let $P$ be an ordered set and $\{Q_i\}_{i\in I}$ be a set of down sets in $P$. Then
\begin{enumerate}
\item $\bigcup Q_i$ is a down set;
\item $\bigcap Q_i$ is a down set.
\end{enumerate}
The same is true for up sets.
\end{corollary}
\begin{proof}
This follows from $\downset \bigcup Q_i = \bigcup \downset Q_i = \bigcup Q_i$ and $\downset \bigcap Q_i \subseteq \bigcap \downset Q_i = \bigcap Q_i$ by \ref{upwardDownwardClosure}.
\end{proof}


\begin{lemma} \label{upDownsetInclusion}
Let $\sSet{P,\Yleft}$ be an ordered set and $Q,R\subseteq P$. Then
\begin{enumerate}
\item $\downset$ and $\upset$ are isotone:
\[Q \subseteq R \implies \downset Q \subseteq \downset R\qquad\text{and}\qquad Q \subseteq R \implies \upset Q \subseteq \upset R; \]
\item if $\Yleft$ is a preorder, then $\downset$ and $\upset$ are order embeddings:
\[ \downset Q \subseteq \downset R \iff Q \subseteq R \iff \upset Q \subseteq \upset R. \]
\end{enumerate}
Let $x, y \in P$. Then
\begin{enumerate} \setcounter{enumi}{2}
\item $x\in \downset y \iff x\Yleft y \iff y \in \upset x$;
\item $x\Yleft y \implies \downset x \subseteq \downset y$;
\item if $\Yleft$ is a preorder, then $x\Yleft y \iff \downset x \subseteq \downset y$.
\end{enumerate}
\end{lemma}


\begin{lemma}
Let $(P, \Yleft_P)$ and $(Q, \Yleft_Q)$ be ordered sets and $f: P\to Q$ a function. Then
\begin{enumerate}
\item the following are equivalent:
\begin{enumerate}
\item $f$ is order-preserving;
\item for all $x\in P$: $f[\upset x] \subseteq \upset f(x)$;
\item for all $x\in P$: $f[\downset x] \subseteq \downset f(x)$;
\item for all $x\in P$: $f^{-1}[\downset x]$ is downwards closed;
\item for all $x\in P$: $f^{-1}[\upset x]$ is upwards closed;
\end{enumerate}
\item as are
\begin{enumerate}
\item $f$ is order-reversing;
\item for all $x\in P$: $f[\upset x] \subseteq \downset f(x)$;
\item for all $x\in P$: $f[\downset x] \subseteq \upset f(x)$;
\item for all $x\in P$: $f^{-1}[\downset x]$ is upwards closed;
\item for all $x\in P$: $f^{-1}[\upset x]$ is downwards closed.
\end{enumerate}
\item If 
\end{enumerate}
\end{lemma}
\begin{proof}
All statements pertaining to images are variations on the calculation
\[ y\in \upset x \iff x\Yleft y \implies f(x) \Yleft f(y) \iff f(y) \in \upset f(x). \]

Now assume $f$ order-preserving. Then we have the equivalences
\[ a\in f^{-1}[\downset x] \iff f(a) \in \downset x \iff f(a) \Yleft x. \]
Let $b \in \downset f^{-1}[\downset x]$. Then $\exists a \in f^{-1}[\downset x]: b \Yleft a$. Because $f$ is order-preserving, we have $f(b) \Yleft f(a) \Yleft x$. Thus $b \in f^{-1}[\downset x]$ and so $f^{-1}[\downset x]$ is downwards closed.

Conversely, assume $f^{-1}[\downset x]$ is downwards closed for all $x\in Q$. Take $a \Yleft b \in P$.
\end{proof}
\begin{lemma} \label{imageUpDownsets}
Let $(P, \Yleft_P)$ and $(Q, \Yleft_Q)$ be ordered sets, $f: P\to Q$ a function and $S\subseteq P$ a subset.
\begin{enumerate}
\item If $f$ is order-preserving, then $f[\upset S] \subseteq \upset f[S]$ and $f[\downset S] \subseteq \downset f[S]$.
\item If $f$ is order-reversing, then $f[\upset S] \subseteq \downset f[S]$ and $f[\downset S] \subseteq \upset f[S]$.
\item If $f$ is order-reflecting and surjective, then $f[\upset S] \supseteq \upset f[S]$ and $f[\downset S] \supseteq \downset f[S]$.
\item If $f$ is reverse order-reflecting and surjective, then $f[\downset S] \supseteq \upset f[S]$ and $f[\upset S] \supseteq \downset f[S]$.
\end{enumerate}
Also:
\begin{enumerate}
\item If $f$ is order-preserving, then $\upset f^{-1}[S] \subseteq \upset f[S]$ and $f[\downset S] \subseteq \downset f[S]$.
\item If $f$ is order-reversing, then $f[\upset S] \subseteq \downset f[S]$ and $f[\downset S] \subseteq \upset f[S]$.
\item If $f$ is order-reflecting and surjective, then $f[\upset S] \supseteq \upset f[S]$ and $f[\downset S] \supseteq \downset f[S]$.
\item If $f$ is reverse order-reflecting and surjective, then $f[\downset S] \supseteq \upset f[S]$ and $f[\upset S] \supseteq \downset f[S]$.
\end{enumerate}
\end{lemma}
\begin{proof}
(1) Assume $f$ is order-preserving and take $y\in f[\upset S]$. Then $\exists x\in \upset S: f(x) = y$ and $\exists s\in S: s\Yleft_P x$. So $f(s) \Yleft_Q f(x) = y$, meaning $y\in \upset f[S]$. The second part is dual.

(2) Like the previous point, except now $y = f(x) \Yleft_Q f(s)$, so $y \in \downset f[S]$.

(3) Assume $f$ is order-reflecting and surjective and take $y\in \upset f[S]$. Then $\exists t\in f[S]: t \Yleft_Q y$ and $\exists s\in S: f(s) = t$. Now because $f$ is surjective, we can find an $x\in P$ such that $f(x) = y$. So $t \Yleft_Q y$ is equivalent to $f(s) \Yleft_Q f(x)$, which implies $s \Yleft_P x$, meaning $x\in \upset S$ and so $y = f(x) \in f[\upset S]$. The second part is dual.

(4) Like the previous point, except now $x\Yleft_P s$, meaning $x\in \downset S$ and so $y = f(x) \in f[\downset S]$.
\end{proof}
TODO: reverse image.

\subsection{Upper and lower bounds}
TODO: picture!
\begin{definition}
Let $(P,\Yleft)$ be an ordered set and $S$ a subset of $P$.
\begin{itemize}
\item An \udef{upper bound} of $S$ is an element $x\in P$ that is greater than or equal to every element in $S$:
\[\forall y\in S: y\Yleft x. \]
\item A \udef{greatest element}, \udef{largest element} or \udef{maximum} of $S$ is an upper bound of $S$ that is an element of $S$.
\item A \udef{maximal element} of $S$ is an element $x\in S$ such that no element $y$ is strictly greater than $x$:
\[ \forall y\in S: x\Yleft y \implies y \Yleft x. \]
\end{itemize}
And dually we have the following:
\begin{itemize}
\item A \udef{lower bound} of $S$ is an element $x\in P$ that is smaller than or equal to every element in $S$:
\[ \forall y\in S: x\Yleft y. \]
\item A \udef{least element}, \udef{smallest element} or \udef{minimum} of $S$ is a lower bound of $S$ that is an element of $S$.
\item A \udef{minimal element} of $S$ is an element $x\in S$ such that no element $y$ is strictly smaller than $x$:
\[ \forall y\in S: y\Yleft x \implies x \Yleft y. \]
\end{itemize}
We denote
\begin{itemize}
\item the set of all upper bounds of $S$ as $S^u$;
\item the set of all lower bounds of $S$ as $S^l$;
\item the set of greatest elements of $S$ as $\max(S)$;
\item the set of least elements of $S$ as $\min(S)$.
\end{itemize}
We say $S$ is
\begin{itemize}
\item \udef{bounded above} if $S^u \neq \emptyset$;
\item \udef{bounded below} if $S^l \neq \emptyset$.
\end{itemize}
We also define:
\begin{itemize}
\item $\sup(S) = \min(S^u)$; an element of $\sup(S)$ is a \udef{least upper bound}, \udef{supremum}, or \udef{join};
\item $\inf(S) = \max(S^l)$; an element of $\inf(S)$ is a \udef{greatest lower bound}, \udef{infimum}, or \udef{meet}.
\end{itemize}
\end{definition}

\begin{lemma}
Let $(P,\Yleft)$ be an ordered set and $S$ a subset of $P$. Then
\begin{enumerate}
\item $\max(S) = S\cap S^u$;
\item $\min(S) = S\cap S^l$;
\item $\sup(S) = S^u\cap (S^u)^l$;
\item $\inf(S) = S^l\cap (S^l)^u$.
\end{enumerate}
\end{lemma}


\begin{lemma} \label{upperBoundUpsetlowerBoundDownset}
Let $(P,\Yleft)$ be an ordered set and $S$ a non-empty subset of $P$, then
\begin{enumerate}
\item $\upset S^u \subseteq S^u$ and $(\downset S)^u \subseteq S^u$;
\item $\upset S^u \subseteq S^u$ and $(\upset S)^l \subseteq S^l$.
\end{enumerate}
\end{lemma}
In particular $S^u$ is an up set and $S^l$ a down set.
\begin{corollary} \label{minMaxUpsetDownset}
Let $(P,\Yleft)$ be a preordered set, $x\in P$ and $S$ a non-empty subset of $P$, then
\begin{enumerate}
\item $x \in \max(\downset x)$;
\item $x \in \min(\upset x)$;
\item if $\max(S)\neq \emptyset$, then $\downset S = \downset \max(S)$;
\item if $\min(S)\neq \emptyset$, then $\upset S = \upset \min(S)$.
\end{enumerate}
\end{corollary}

\begin{lemma} \label{boundsFromUpDownSets}
Let $(P,\Yleft)$ be an ordered set and $S$ a non-empty subset of $P$. Then
\begin{enumerate}
\item $S^u = \bigcap_{x\in S}\upset x$;
\item $S^l = \bigcap_{x\in S}\downset x$.
\end{enumerate}
In particular $\{x\}^u = \upset x$ and $\{x\}^l = \downset x$.

This also holds for empty $S$, if we relativise our intersection to $P$ (TODO!!).

If $S$ is empty, then we have
\begin{enumerate}
\item $\emptyset^u = P$;
\item $\emptyset^l = P$.
\end{enumerate}
\end{lemma}

\begin{lemma}
Let $(P,\Yleft)$ be an ordered set and $A,B$ subsets of $P$. Then
\[ A\subseteq B^l \iff \Big[\forall a\in A, b\in B: a\Yleft b \Big] \iff B\subseteq A^u. \]
\end{lemma}
This lemma identifies $(^u,^l)$ as a Galois connection. (TODO ref)
\begin{corollary}
Let $(P,\Yleft)$ be an ordered set and $A,B$ subsets of $P$. Then
\begin{enumerate}
\item $A\subseteq (A^l)^u$ and $A\subseteq (A^u)^l$;
\item if $A\subseteq B$, then $B^u\subseteq A^u$ and $B^l \subseteq A^l$;
\item $A^l = ((A^l)^u)^l$ and $A^u = ((A^u)^l)^u$.
\end{enumerate}
\end{corollary}
\begin{proof}
(1) $A^l \subseteq A^l \implies A = (A^l)^u$.

(2) $A\subseteq B \implies A \subseteq (B^u)^l \implies B^u\subseteq A^u$.

(3) $(A^l)^u \subseteq (A^l)^u \implies A^l \subseteq ((A^l)^u)^l$ and by 1. and 2. $A\subseteq (A^l)^u \implies ((A^l)^u)^l \subseteq A^l$.
\end{proof}
\begin{corollary} \label{maxSupMinInf}
If $\sSet{P, \Yleft}$ is an ordered set and $S\subseteq P$, then
\begin{enumerate}
\item $\max(S)\subset \sup(S)$;
\item $\min(S)\subset \inf(S)$.
\end{enumerate}
\end{corollary}
\begin{proof}
For (1) we calculate $\max(S) = S \cap S^u \subseteq (S^u)^l \cap S^u = \sup(S)$; (2) is dual.
\end{proof}
\begin{corollary}
If $\sSet{P, \Yleft}$ is an ordered set and $S\subseteq P$, then
\begin{enumerate}
\item $(A\cup B)^u \subseteq A^u\cap B^u$ and $(A\cup B)^l \subseteq A^l\cap B^l$;
\item $(A\cap B)^u \supseteq A^u\cup B^u$ and $(A\cap B)^l \supseteq A^l\cup B^l$.
\end{enumerate}
\end{corollary}
\begin{proof}
From $A\subseteq A\cup B$, we get $(A\cup B)^u \subseteq A^u$. Similary $(A\cup B)^u \subseteq B^u$, so $(A\cup B)^u \subseteq A^u\cap B^u$. The other cases are similar.
\end{proof}

\begin{lemma} \label{minMaxSingletons}
If $\sSet{P, \precsim}$ is a poset and $S\subseteq P$, then $\max(S), \min(S), \sup(S), \inf(S)$ are either singletons or empty.
\end{lemma}
\begin{proof}
We prove for $\max(S)$. The other cases follow dually or a fortiori. Let $x,y\in \max(S)$. Then $x\precsim y$ and $y\precsim x$, so $x=y$ by anti-symmetry.
\end{proof}
For posets we sometimes use $\max/\min/\sup/\inf$ to denote the contents of the singleton rather than the singleton itself. If the set is empty, we say the $\max/\min/\sup/\inf$ does not exist.

\begin{lemma} \label{greatestLeastElementsSubsetPoset}
Let $\sSet{P, \precsim}$ be a poset and $R\subseteq S\subseteq P$.
\begin{enumerate}
\item If $\max(R)$ and $\max(S)$ exists, then $\max(R) \precsim \max(S)$.
\item If $\min(R)$ and $\min(S)$ exists, then $\min(R) \succsim \min(S)$.
\end{enumerate}
\end{lemma}
\begin{proof}
By definition $\max(S) \succsim x$ for all $x \in S$. Now $\max(R)\in R\subseteq S$, so in particular $\max(R) \precsim \max(S)$.
\end{proof}

\begin{lemma}
Let $\sSet{P, \Yleft}$ be an ordered set and $R,S \subseteq P$ subsets. Then the following are equivalent
\begin{enumerate}
\item $\forall x\in S: \exists y\in T: x \Yleft y$;
\item $S \subseteq \downset T$.
\end{enumerate}
If $P$ is a poset, then these are also equivalent to
\begin{enumerate} \setcounter{enumi}{2}
\item $S^u \supseteq T^u$.
\end{enumerate}
If $P$ is a poset and $\sup(S), \sup(T)$ exist, then these statements are also equivalent to
\begin{enumerate} \setcounter{enumi}{3}
\item $\sup(S) \Yleft \sup(T)$.
\end{enumerate}
\end{lemma}
\begin{proof}
$(1) \Leftrightarrow (2)$ holds by definition.

$(2) \Rightarrow (3)$ $S \subseteq \downset T \implies S^u \supseteq (\downset T)^u = T^u$.

$(3) \Rightarrow (1)$ Assume, towards a contradiction, that (1) does not hold. TODODODO!
\end{proof}
{greatestLeastElementsSubsetPoset}

\begin{lemma} \label{orderPreservingReversingBounds}
Let $(P, \Yleft_P)$ and $(Q, \Yleft_Q)$ be ordered sets, $f: P\to Q$ a function and $S\subseteq P$ a subset.
\begin{enumerate}
\item If $f$ is order-preserving, then
\begin{enumerate}
\item $f[S^u] \subseteq f[S]^u$;
\item $f[S^l] \subseteq f[S]^l$;
\item $f[\max(S)] \subseteq \max(f[S])$;
\item $f[\min(S)] \subseteq \min(f[S])$.
\end{enumerate}
\item If $f$ is order-reversing, then
\begin{enumerate}
\item $f[S^u] \subseteq f[S]^l$;
\item $f[S^l] \subseteq f[S]^u$;
\item $f[\max(S)] \subseteq \min(f[S])$;
\item $f[\min(S)] \subseteq \max(f[S])$.
\end{enumerate}
\item If $f$ is a bijective order-embedding, then
\begin{enumerate}
\item $f[S^u] = f[S]^u$;
\item $f[S^l] = f[S]^l$;
\item $f[\max(S)] = \max(f[S])$;
\item $f[\min(S)] = \min(f[S])$;
\item $f[\sup(S)] = \sup(f[S])$;
\item $f[\inf(S)] = \inf(f[S])$.
\end{enumerate}
\item If $f$ is a bijective reverse order-embedding, then
\begin{enumerate}
\item $f[S^u] = f[S]^l$;
\item $f[S^l] = f[S]^u$;
\item $f[\max(S)] = \min(f[S])$;
\item $f[\min(S)] = \max(f[S])$;
\item $f[\sup(S)] = \inf(f[S])$;
\item $f[\inf(S)] = \sup(f[S])$.
\end{enumerate}
\end{enumerate}
\end{lemma}
\begin{proof}
We prove the properties of the order-preserving function.

(a) We calculate, using \ref{boundsFromUpDownSets} and \ref{imageUpDownsets},
\[ f[S^u] = f\left[\bigcap_{x\in S}\upset x\right] \subseteq \bigcap_{x\in S}f[\upset x] \subseteq \bigcap_{x\in S}\upset f(x) = \bigcap_{y\in f[S]}\upset y =  f[S]^u. \]

(b) Dual to (a).

(c) We calculate $f[\max(S)] = f[S\cap S^u] \subseteq f[S]\cap f[S^u] \subseteq f[S]\cap f[S]^u = \max(f[S])$.

(d) Dual to (c).

The properties of the embedding are proved by replacing $\subseteq$ with equality. For the supremum and infimum we have:

(e) $f[\sup(S)] = f[S^u\cap (S^u)^l] = f[S]^u \cap (f[S]^u)^l = \sup(f[S])$ and (f) dual to (e).

The properties of the order reversing functions are similarly proved.
\end{proof}
We cannot say anything about the supremum or infimum for general order-preserving (or order-reversing) functions, because $f[S^u] \subseteq f[S]^u$ implies $f[S^u]^l \supseteq (f[S]^u)^l$, so the calculation would be
\[ f[\sup(S)] = f[S^u\cap (S^u)^l] \subseteq f[S]^u \cap f[(S^u)]^l \supseteq f[S]^u \cap (f[S]^u)^l = \sup(f[S]), \]
from which we cannot conclude anything.

\subsection{Chains}
\begin{definition}
Let $(P,\Yleft)$ be an ordered set. A \udef{chain} in $P$ is a linearly ordered subset $S$ of $P$, i.e.
\[ \forall x,y\in S: x\Yleft y \lor y\Yleft x. \]

An \udef{antichain} is a subset $A$ such that no two elements of $A$ are comparable.
\end{definition}

\begin{example}
For every $n\in \N$ there exists a chain $\mathbb{n}$ of $n$ elements and an antichain $\overline{\mathbb{n}}$ of $n$ elements:
\[ \mathbb{1}: \begin{tikzcd}
\circ
\end{tikzcd} \qquad \mathbb{2}: \begin{tikzcd}
\circ \ar[d, dash] \\ \circ
\end{tikzcd} \qquad \mathbb{3}: \begin{tikzcd}
\circ \ar[d, dash] \\ \circ \ar[d, dash] \\ \circ
\end{tikzcd} \qquad \hdots \]
\[ \overline{\mathbb{1}}: \begin{tikzcd}
\circ
\end{tikzcd} \qquad \overline{\mathbb{2}}: \begin{tikzcd}
\circ & \circ
\end{tikzcd} \qquad \overline{\mathbb{3}}: \begin{tikzcd}
\circ & \circ & \circ
\end{tikzcd} \qquad \hdots \]
\end{example}

\begin{lemma} \label{boundsFiniteChain}
Let $P$ be an ordered set. If $C$ is a non-empty, finite chain in $P$, then
\begin{enumerate}
\item $\sup(C) \supseteq \max(C) \neq \emptyset$;
\item $\inf(C) \supseteq \min(C) \neq \emptyset$.
\end{enumerate}
\end{lemma}
\begin{proof}
The $\supseteq$-relation is due to \ref{maxSupMinInf}.

The proof that $\max(C)$ is non-empty is by induction on the cardinality of $C$. For the base case, assume $C$ has one element $x$. Then $x \leq x$, so $x\in \max(C)$.

Now assume all chains with one fewer element than $C$ have a maximal element. Pick some $x_0\in C$, then $y = \max(C\setminus \{x_0\})$ exists. We can compare $x_0$ and $y$ because $C$ is a chain. If $x_0 \leq y$, then $\max(C) = y$. If $y \leq x_0$, then $\max(C) = x_0$. This exhausts the possibilities.
\end{proof}

\begin{definition}
An ordered set $P$
\begin{itemize}
\item is \udef{chain-complete} or \udef{inductive} if every chain in $P$ has a least upper bound;
\item satisfies the \udef{ascending chain condition} (ACC) if each chain in $P$ that has a least element is finite;
\item satisfies the \udef{descending chain condition} (DCC) if each chain in $P$ that has a greatest element is finite.
\end{itemize}
\end{definition}

\begin{lemma}
Every inductive poset has a least element.
\end{lemma}
\begin{proof}
The empty set $\emptyset$ is a chain in any poset $P$. Then $\sup(\emptyset) = \min(\emptyset^u) = \min(P)$.
\end{proof}

\begin{lemma}
Let $P$ be an ordered set.
\begin{enumerate}
\item If $P$ satisfies to ascending chain condition, then for each non-empty chain $C$ in $P$ we have $\max(C) \neq \emptyset$;
\item If $P$ satisfies to descending chain condition, then for each non-empty chain $C$ in $P$ we have $\min(C) \neq \emptyset$,
\end{enumerate}
\end{lemma}
\begin{proof}
Assume $P$ satisfies the ascending chain condition and take a non-empty chain $C\subset P$. Take some $x_0\in C$ and define $C' = \setbuilder{c\in C}{x_0 \Yleft c}$. Clearly $x_0$ is a leat element of $C'$, so $C'$ is finite by ACC and $\max(C) = \max(C') \neq \emptyset$ by \ref{boundsFiniteChain}.
\end{proof}
\begin{corollary}
Let $P$ be an ordered set. If $P$ satisfies the ascending chain condition and has a least element, then $P$ is inductive.
\end{corollary}
\begin{proof}
We have $\sup(C) \supseteq \max(C)$ by \ref{maxSupMinInf}.
\end{proof}

\begin{example}
The set $\N$ is not inductive, because $\N$ itself does not have a least upper bound. It therefore does also not satisfy the ascending chain condition.

It does satisfy the descending chain condition.
\end{example}


\begin{proposition} \label{inductive}
Let $A,B$ be sets and $P$ an ordered set. Then the following posets are inductive:
\begin{enumerate}
\item $\powerset(A)$, ordered by inclusion;
\item $(A \not\to B)$, ordered by inclusion;
\item the set of chains in $P$, ordered by inclusion.
\end{enumerate}
\end{proposition}
\begin{proof}
In all cases the least upper bound of a chain $S$ is given by $\bigcup S$.
\end{proof}

\subsection{Intervals}
\begin{definition}
Let $(P,\Yleft)$ be an ordered set and $m,n\in P$. We define
\begin{itemize}
\item the \udef{closed interval} $[m,n] \defeq \setbuilder{k\in P}{m\Yleft k \land k\Yleft n}$;
\item the \udef{open interval} $]m,n[ \defeq [m,n]\setminus\{m,n\}$;
\item the \udef{half-open intervals}
\begin{align*}
[m,n[ &\defeq [m,n]\setminus\{n\}; \\
]m,n] &\defeq [m,n]\setminus\{m\}.
\end{align*}
\end{itemize}
\end{definition}

\begin{lemma} \label{emptyInterval}
Let $(P,\Yleft)$ be an ordered set and $m,n\in P$.
\begin{enumerate}
\item If $m \cancel\Yleft n$, then $[m,n] = \emptyset$.
\item If $\Yleft$ is a preorder, then
\begin{enumerate}
\item $m \cancel\Yleft n$ \textup{if and only if} $[m,n] = \emptyset$;
\item either $[m,n] = \emptyset$ or $\{m,n\} \subseteq [m,n]$.
\end{enumerate}
\end{enumerate}
\end{lemma}
\begin{proof}
(1) We prove by contraposition. Assume $k\in [m,n]$, then $m\Yleft k$ and $k\Yleft n$, so $m\Yleft n$ by transitivity.

(2) (a) The direction $\Rightarrow$ is given by \ref{emptyInterval}. Now assume $m \Yleft n$. Also $m\Yleft m$ and $n \Yleft n$ by reflexivity. So $m,n\in [m,n]$ by definition. This also gives point (b).
\end{proof}


\section{Completeness}
\begin{definition}
An ordered set $P$ is 
\begin{itemize}
\item \udef{order complete} (or simply \udef{complete}) if each subset has a supremum and an infimum;
\item \udef{order $\sigma$-complete} if each countable subset has a supremum and an infimum;
\item \udef{finitely order complete} if each finite subset has a supremum and an infimum;
\item \udef{Dedekind complete} if
\begin{itemize}
\item each non-empty subset that is bounded above has a supremum; and
\item each non-empty subset that is bounded below has an infimum;
\end{itemize}
\item \udef{Dedekind $\sigma$-complete} if
\begin{itemize}
\item each non-empty countable subset that is bounded above has a supremum; and
\item each non-empty countable subset that is bounded below has an infimum.
\end{itemize}
\item \udef{finitely Dedekind complete} if
\begin{itemize}
\item each non-empty finite subset that is bounded above has a supremum; and
\item each non-empty finite subset that is bounded below has an infimum.
\end{itemize}
\end{itemize}
\end{definition}
Some authors use ``order completeness'' to mean what we have called Dedekind completeness.

\begin{example}
The closed unit disk in $\R^2$ with coordinatewise ordering is Dedekind complete, but not order complete and is not a lattice.
\end{example}


\begin{proposition} \label{existenceSupremaInfima}
Let $(P,\precsim)$ be an ordered set. Then the following are equivalent:
\begin{enumerate}
\item $P$ is order complete;
\item each subset of $P$ has a supremum;
\item each subset of $P$ has an infimum;
\item each non-empty subset of $P$ has a supremum and $P$ has a least element;
\item each non-empty subset of $P$ has an infimum and $P$ has a greatest element.
\end{enumerate}
The following are also equivalent:
\begin{enumerate}
\item $P$ is Dedekind complete
\item each non-empty set that is bounded above has a supremum;
\item each non-empty set that is bounded below has an infimum.
\end{enumerate}
\end{proposition}
\begin{proof}
We show that (2) and (3) are equivalent.

Suppose each subset has a supremum and let $S$ be a subset. Then in particular $S^l$ has a supremum. This supremum of $S^l$ is an infimum of $S$:
\[ \sup(S^l) = (S^l)^u\cap((S^l)^u)^l = (S^l)^u\cap S^l = \inf(S).  \]

The converse is similar.

For equivalence with (3) and (4), we just need to remark that $\sup(\emptyset) = \min(P)$ and $\inf(\emptyset) = \max(P)$.

For Dedekind completeness the argument is similar, we just need to remark that if $S$ is non-empty and bounded below, then $S^l$ is non-empty and bounded above.
\end{proof}

\begin{lemma}
Let $P$ be an ordered set that contains a least and a greatest element. Then $P$ is order complete \textup{if and only if} $P$ is Dedekind complete.
\end{lemma}
\begin{proof}
Every set is bounded (by the least and greatest elements).
\end{proof}

\begin{lemma}
The ordered natural numbers $\sSet{\N,\leq}$ are complete.
\end{lemma}
\begin{proof}
Any set of natural numbers has a least element by \ref{proposition:wellOrderingN}. Any non-empty set $S$ of natural numbers that is bounded above, we can take $\max(S) = \min(\N\setminus S)-1$.
\end{proof}

\subsection{Directed sets}
\begin{definition}
An ordered set $(D,\Yleft)$ is called
\begin{itemize}
\item \udef{(upward) directed} if every finite subset has an upper bound;
\item \udef{downward directed} if every finite subset has a lower bound.
\end{itemize}
If $(D,\Yleft)$ is a directed set, we call the relation $\Yleft$ a \udef{direction}.
We call 
\end{definition}
TODO: use $\closure_\vee$ and $\closure_\wedge$.

TODO: important note: $\closure_\vee$ only implies closure under finite $\vee$

\begin{example}
\begin{itemize}
\item Any non-empty chain is directed.
\item An antichain is directed iff it is a singleton.
\end{itemize}
\end{example}

TODO: move:
\begin{proposition}
Let $\{(D_i,\Yleft_i)\}_{i\in I}$ be a family of directed sets. Then $D= \prod_{i\in I}D_i$ is a directed set with direction defined by
\[ (a_i)_{i\in I} \Yleft (b_i)_{i\in I} \iff \forall i\in I: a_i\Yleft_i b_i. \]
The directed set $(D,\Yleft)$ is called a \udef{product direction}.
\end{proposition}


\section{Ordered sets of subsets}
\subsection{The ``finer than / coarser than'' relation}
\begin{definition}
Let $\sSet{P,\Yleft}$ be an ordered set and $A,B \subseteq P$. We say $A$ is \udef{finer} than $B$ (or $B$ is \udef{coarser} than $A$) if $B\subseteq \upset A$.

If $A$ is finer than $B$ and $B$ is finer than $A$, we write $A \approx B$ and say $A$ and $B$ are \udef{equally fine} or \udef{equally coarse}.
\end{definition}

Alternatively, we say $A$ is finer than $B$ iff
\[ \forall b\in B: \exists a\in A: \; a \Yleft b. \]

For up upwards closed sets, ``finer than'' is $\supseteq$ and ``coarser than'' is $\subseteq$.

\begin{lemma}
Let $\sSet{P,\Yleft}$ be an ordered set and $A,B,C \subseteq P$. Then
\begin{enumerate}
\item if $A$ is finer than $B$ and $B$ is finer than $C$, then $A$ is finer than $C$;
\item if $\Yleft$ is a preorder, then $A$ is finer than $A$.
\end{enumerate}
\end{lemma}
\begin{proof}
(1) Assume $A$ is finer than $B$ and $B$ is finer than $C$. Then $B \subseteq \upset A$ and $C\subseteq \upset B$, so $C \subseteq \upset\upset A \subseteq \upset A$.

(2) from \ref{QsubseteqDownQ}.
\end{proof}
\begin{corollary}
Let $\sSet{P,\Yleft}$ be a preordered set.
\begin{enumerate}
\item The ``finer than'' relation is a preorder on $\powerset(P)$.
\item The equivalence derived from this relation is an equivalence relation.
\end{enumerate}
\end{corollary}
The equivalence relation is the equivalence from \ref{preorderEquivalence}.

\begin{lemma}
Let $\sSet{P,\Yleft}$ be an ordered set and $A \subseteq P$. Then
\begin{enumerate}
\item $A$ is finer than $\upset A$;
\item if $\Yleft$ is a preorder, then $A \approx \upset A$;
\item $A \approx B$ implies $\upset A = \upset B$;
\item if $\Yleft$ is a preorder, then $A \approx B$ \textup{if and only if} $\upset A = \upset B$.
\end{enumerate}
\end{lemma}
\begin{proof}
(1) is equivalent to $\upset A \subseteq \upset A$.

(2) We need to show that $\upset A$ is finer than $A$, i.e. $A\subseteq \upset\upset A$. This follows from \ref{QsubseteqDownQ}.

(3) Assume $A \approx B$. Then $A \subseteq \upset B$ and $B \subseteq \upset A$. This implies $\upset A \subseteq \upset\upset B \subseteq \upset B$ and $\upset B \subseteq \upset\upset A \subseteq \upset A$.

(4) Assume $\upset A = \upset B$. Then $A$ is finer than $\upset B$ and $B$ is finer than $\upset A$. The result then follows using (2).
\end{proof}


\subsubsection{Refinement}
\begin{definition}
Let $\sSet{P,\Yleft}$ be an ordered set and $A,B \subseteq P$. We say $A$ \udef{refines} $B$ (or $A$ is a \udef{refinement} of $B$) if $A\subseteq \downset B$.
\end{definition}

\begin{lemma}
Let $f:P\to Q$ be an order reversing function between ordered sets. Let $A,B\in P$. Then
\begin{enumerate}
\item if $A$ refines $B$, then $f[B]$ is finer than $f[A]$;
\item if $A$ is finer than $B$, then $f[B]$ refines $f[A]$.
\end{enumerate}
\end{lemma}
\begin{proof}
(1) $A \subseteq \downset B$ implies $f[A] \subseteq f[\downset B] \subseteq \upset f[B]$.

(2) $B \subseteq \upset A$ implies $f[B] \subseteq f[\upset A] \subseteq \downset f[A]$.
\end{proof}

\subsection{The ordered set of downsets}
\begin{definition}
Let $P$ be an ordered set. The set of down sets in $P$ is denoted $\downsets(P)$.
\end{definition}

\begin{lemma}
Let $P$ be an ordered set.
If $P$ is a discrete poset, then $\downsets(P) = \powerset(P)$.
\end{lemma}

\begin{proposition} \label{orderedSetPowerset}
Let $P$ be a preordered ordered set. Then $x\mapsto \downset x$ is an order embedding between $P$ and the principle down sets in $P$.

If $P$ is a partial order, then the map is bijective and thus an order isomorphism.
\end{proposition}
\begin{proof}
The first part is just \ref{upDownsetInclusion}.

For the second part, $x\mapsto \downset x$ is surjective by definition. For injectivity: the inverse map is given by $Q\mapsto \max(Q)$. This is a function by \ref{minMaxSingletons} and an inverse by \ref{minMaxUpsetDownset}.
\end{proof}

\begin{lemma}
Let $P_1,P_2$ be prosets and $\mathbb{1}$ a singleton. Then
\begin{enumerate}
\item $\downsets(P)^o \cong \downsets(P^o)$;
\item $\downsets(P\oplus \mathbb{1}) \cong \downsets(P)\oplus \mathbb{1}$;
\item $\downsets(\mathbb{1}\oplus \mathbb{1}) \cong \mathbb{1}\oplus \downsets(P)$;
\item $\downsets(P_1\sqcup P_2)\cong \downsets(P_1)\times \downsets(P_2)$.
\end{enumerate}
\end{lemma}

\section{Join- and meet-density}
\begin{definition}
Let $P$ be a poset and let $Q\subset P$ be a subset. Then $Q$ is called \udef{join-dense} in $P$ if for every $x\in P$, there exists a subset $S\subset Q$ such that $x= \bigvee_P S$.

The dual of join-dense is \udef{meet-dense}.
\end{definition}

\section{Atoms}
\subsection{Atomic elements}
\begin{definition}
Let $\sSet{P,\leq}$ be a poset.
\begin{itemize}
\item An element $a\in P$ is called an \udef{atom} if $a$ is minimal in $P\setminus\min(P)$.
\item An element $a\in P$ is called a \udef{coatom} if $a$ is maximal in $P\setminus\max(P)$.
\end{itemize}
We denote the set of atoms of $P$ as $\atoms(P)$. The set of coatoms is denoted $\coatoms(P)$.
\end{definition}
If $P$ has a least element $\bot$, then $a$ is an atom if and only if it covers $\bot$.



\subsection{Atomic posets}
\begin{definition}
Let $\sSet{P,\leq}$ be a poset. Then
\begin{itemize}
\item $P$ is \udef{atomic} if $\forall x\in L\setminus\min(L): \exists a\in\atoms(L): a\leq x$;
\item $P$ is \udef{coatomic} if $\forall x\in L\setminus\min(L): \exists a\in\coatoms(L): a\geq x$;
\end{itemize}
also
\begin{itemize}
\item $P$ is \udef{strongly atomic} if $\forall x,y\in L$ the poset $[x,y]$ is atomic;
\item $P$ is \udef{strongly coatomic} if $\forall x,y \in L$ the poset $[x,y]$ is coatomic;
\end{itemize}
and
\begin{itemize}
\item $P$ is \udef{atomistic} if $\atoms(L)$ is join-dense in $L$;
\item $P$ is \udef{coatomistic} if $\atoms(L)$ is meet-dense in $L$.
\end{itemize}
\end{definition}

\url{https://www.emis.de/journals/PM/51f4/pm51f409.pdf}


\section{Filters and ideals}
\begin{definition}
Let $\sSet{P,\Yleft}$ be an ordered set. A subset $F\subseteq P$ is called
\begin{itemize}
\item an \udef{order filter} if
\begin{itemize}
\item $F$ is downward directed;
\item $F$ is an up set in $P$;
\end{itemize}
\item an \udef{order ideal} if
\begin{itemize}
\item $F$ is upward directed;
\item $F$ is a down set in $P$.
\end{itemize}
\end{itemize}
In addition $F$ is called \udef{proper} if $F \neq P$.

The set of ideals on $P$ is denoted $\ideals(P)$ and the set of filters on $P$ is denoted $\filters(P)$.
\end{definition}
Many authors also require filters and ideals to be non-empty.

\begin{lemma}
Let $P$ be an ordered set and $x\in P$.
\begin{enumerate}
\item The principle up set $\upset x$ is an order filter.
\item The principle down set $\downset x$ is an order ideal.
\end{enumerate}
\end{lemma}
If the order is not a preorder, then a principle filter/ideal may be empty.

\begin{example}
Let $X$ be a set.
\begin{itemize}
\item The set of finite subsets of $X$ is an ideal.
\item The set of countable subsets $X$ is an ideal.
\item A subset $A$ of $X$ is called \udef{cofinite} if $X\setminus A$ is finite. The set of cofinite subsets of $X$ is a filter.
\item A subset $A$ of $X$ is called \udef{cocountable} if $X\setminus A$ is countable. The set of cocountable subsets of $X$ is a filter.
\end{itemize}
\end{example}

\subsection{Maximal filters and ideals}
\begin{definition}
Let $\sSet{P,\Yleft}$ be an ordered set.
\begin{itemize}
\item Let $I$ be a proper ideal, then $I$ is called \udef{maximal} if there is no proper ideal $J$ such that $I \subsetneq J$.
\item Let $F$ be a proper filter, then $F$ is called \udef{maximal} if there is no proper filter $G$ such that $F \subsetneq G$.
\end{itemize}
\end{definition}

\subsection{Filter bases and subbases}

\begin{lemma}
Let $\sSet{P,\Yleft}$ be an ordered set and $\{F_i\}_{i\in I}$ a family of subsets.
\begin{enumerate}
\item If $F_i$ is a filter for all $i\in I$, then $\bigcap_{i\in I}F_i$ is a filter;
\item If $F_i$ is an ideal for all $i\in I$, then $\bigcap_{i\in I}F_i$ is an ideal.
\end{enumerate}
\end{lemma}

\begin{definition}
Let $\sSet{P,\Yleft}$ be an ordered set and $B\subseteq P$ a subset. Then
\begin{itemize}
\item the \emph{filter generated by $B$} is
\[ \mathfrak{F}\{B\} \defeq \bigcap\setbuilder{B \subseteq S\subseteq P}{\text{$S$ is a filter}} \]
we call $B$ a \udef{filter subbasis} of $F$. If $B$ is downward directed, it is called a \udef{filter basis} of $F$;
\item the \emph{filter basis generated by $B$} as
\[ \mathfrak{FB}\{B\} \defeq \bigcap\setbuilder{B \subseteq S\subseteq P}{\text{$S$ is downward directed}}. \]
\end{itemize}
\end{definition}

If $P$ is not downward directed, there may not exists any $S \supseteq B$ that is downward directed, in which case $\mathfrak{FB}\{B\} =$ TODO.

\begin{proposition}
Let $\sSet{P,\Yleft}$ be an ordered set and $B\subseteq P$ a subset. Then
\begin{enumerate}
\item if $P$ is downward directed, then $\mathfrak{FB}\{B\}$ is a filter basis;
\item $\mathfrak{F}\{B\} = \upset\mathfrak{FB}\{B\}$.
\end{enumerate}
\end{proposition}
In particular, if $S$ is a filter base, then $\upset S$ is a filter.

\begin{lemma}
Let $\sSet{P,\Yleft}$ be an ordered set.

If $P$ is downward directed and $\exists \bot \in \min(P)$ and $B \subseteq P$ a filter basis, then $\mathfrak{F}\{B\}$ is a proper filter \textup{if and only if} $\bot \notin B$; 
\end{lemma}

\begin{proposition}
Let $\sSet{P,\Yleft}$ be an ordered set. Then for all filter bases $B,C$ in $P$: $B$ is finer than $C$ \textup{if and only if} $\mathfrak{F}\{B\} \supseteq \mathfrak{F}\{C\}$.
\end{proposition}


\section{Combining ordered sets}
\begin{definition}
Let $P,Q$ be ordered sets. Then
\begin{itemize}
\item the \udef{disjoint union} $P\sqcup Q$ is ordered by
\[ x\precsim y \iff \begin{cases}
x,y\in P \;\text{and}\; x\precsim y \\
x,y\in Q \;\text{and}\; x\precsim y
\end{cases} \]
\item the \udef{linear sum} $P\oplus Q$ is the disjoint union ordered by
\[ x\precsim y \iff \begin{cases}
x,y\in P \;\text{and}\; x\precsim y \\
x,y\in Q \;\text{and}\; x\precsim y \\
x\in P \;\text{and}\; y\in Q
\end{cases} \]
\end{itemize}
\end{definition}

\begin{example}
We define $\mathbf{M}_n \defeq \mathbb{1}\oplus \overline{\mathbb{n}} \oplus \mathbb{1}$:
\[ \mathbf{M}_1: \begin{tikzcd}
\circ \ar[d, dash] \\ \circ \ar[d, dash] \\ \circ
\end{tikzcd} \qquad \mathbf{M}_2: \begin{tikzcd}
&\circ \ar[ld, dash] \ar[rd, dash] & \\ \circ \ar[rd, dash] && \circ \ar[ld, dash] \\ &\circ &
\end{tikzcd} \qquad \mathbf{M}_3: \begin{tikzcd}
&\circ \ar[ld, dash] \ar[d,dash] \ar[rd, dash] &\\ \circ \ar[rd, dash]& \circ \ar[d, dash] & \circ \ar[ld, dash] \\ &\circ &
\end{tikzcd} \qquad \hdots \]
\end{example}

\begin{lemma}
For ordered sets taking the disjoint union or linear sum is associative.
\end{lemma}

\begin{lemma}
Let $m,n\in \N$. If $p = m+n$, then $\mathbb{m}\oplus\mathbb{n} = \mathbb{p}$.
\end{lemma}

\begin{definition}
Let $P,Q$ be ordered sets. Then
\begin{itemize}
\item  the standard order on $P\times Q$ is defined by
\[ (x_1,x_2) \precsim (y_1,y_2) \iff x_1\precsim y_1 \;\text{and}\; x_2\precsim y_2 \]
\item the \udef{lexicographic order} on $P\times Q$ is defined by
\[ (x_1,x_2) \precsim (y_1,y_2) \iff (x_1\precsim y_1) \;\text{or}\; (x_1=y_1 \;\text{and}\; x_2\precsim y_2). \]
\end{itemize}
\end{definition}


\chapter{Lattices}
\section{Semilattice}
\begin{definition}
A \udef{semilattice} is an algebraic structure $\sSet{S,\wedge}$ where $\wedge$ is a binary operation on the set $S$ satisfying
\begin{itemize}[leftmargin=2.5cm]
\item[\textbf{Associativity}] $x\wedge (y\wedge z) = (x\wedge y) \wedge z$;
\item[\textbf{Commutativity}] $x \wedge y = y \wedge x$;
\item[\textbf{Idempotency}] $x\wedge x = x$.
\end{itemize}
We call a semilattice \udef{bounded} if it contains an identity $\top$ for $\wedge$. We write $\sSet{S, \wedge, \top}$.
\end{definition}
In other words a semilattice is a commutative band.

\begin{proposition}
Let $(P,\leq)$ be a poset and let $\{x,y,z\}\subseteq P$ be such that each subset has a supremum. Then
\[ \sup\{\sup\{x,y\},z\} = \sup\{x,y,z\} = \sup\{x,\sup\{y,z\}\} \]
and, dually,
\[ \inf\{\inf\{x,y\},z\} = \inf\{x,y,z\} = \inf\{x,\inf\{y,z\}\}. \]
\end{proposition}
\begin{corollary}
Let $(P,\leq)$ be a poset.
\begin{enumerate}
\item If $\sup\{x,y\}$ exists for all $x,y\in P$, then $(P,\sup)$ is a semilattice.
\item If $\inf\{x,y\}$ exists for all $x,y\in P$, then $(P,\inf)$ is a semilattice.
\end{enumerate}
\end{corollary}
The converse to this corollary also holds:
\begin{proposition} \label{orderSemilattice}
Let $\sSet{L,\wedge}$ be a semilattice. Define the relation $\leq$ on $L$ by
\[ \forall x,y\in L:\; x\leq y \qquad \iff \qquad x = x\wedge y. \]
Then $\sSet{L,\leq}$ is a poset such that $\forall x,y\in L: x\wedge y = \inf\{x,y\}$.
\end{proposition}
\begin{proof}
First we prove $\sSet{L,\leq}$ is a poset:
\begin{itemize}
\item Reflexivity follows from idempotency.
\item Antisymmetry follows from commutativity:
\[ x\leq y, y\leq x \quad\implies\quad x = x\wedge y = y \wedge x = y. \]
\item For transitivity: assume $x\leq y$ and $y\leq z$. This implies $x = x\wedge y$ and $y = y\wedge z$, and so $x = x\wedge (y\wedge z) = (x\wedge y) \wedge z = x\wedge z$. This implies $x\leq z$.
\end{itemize}
If $x,y$ are comparable, the $x \wedge y$ is clearly $\inf\{x,y\}$. Because $x\wedge(x\wedge y) = (x\wedge x)\wedge y = x\wedge y$, we have $(x\wedge y) \leq x$. So $x\wedge y$ is a lower bound of $\{x,y\}$. Let $u$ be a lower bound of $\{x,y\}$. Then $u = x\wedge u = x\wedge (y\wedge u) = (x\wedge y)\wedge u$. So $u \leq x\wedge y$, meaning $x\wedge y$ is the infimum.
\end{proof}
Dually we also have:
\begin{corollary}
Let $\seq{L,\vee}$ be a semilattice. Define the relation $\leq$ on $L$ by
\[ x\leq y \qquad \iff \qquad x\vee y = y. \]
Then $\seq{L,\leq}$ is a poset such that $\forall x,y\in L: x\vee y = \sup\{x,y\}$.
\end{corollary}

\subsection{Complete semilattices}
\begin{lemma}
Let $\sSet{S,\wedge}$ be a semilattice. For every finite set $D\subset S$, $\inf(D)$ exists.
\end{lemma}
\begin{definition}
Let $\sSet{S,\wedge}$ be a semilattice. We call $S$ a \udef{complete semilattice} if each subset $D\subseteq S$ has a supremum. We write
\[ \sup(D) = \bigvee D. \]
If we want to emphasise that the supremum of $D$ is taken as a subset of $S$, we write $\bigvee_S D$.
\end{definition}

We of course have dual definitions for $\vee$-semilattices. In this case we consider suprema and write $\sup(D) = \bigvee D$.

\begin{lemma} \label{supInfFiniteSubsetsLattice}
Let $\sSet{S,\wedge}$ be a semilattice and $F\subseteq S$ a finite subset. Then $\bigvee S$ exists. 
\end{lemma}
In particular every finite semilattice is complete.


\begin{theorem}[Knaster-Tarski fixed-point theorem for semilattices] \label{KnasterTarskiSemilattice}
Let $\sSet{S,\wedge}$ be a complete semilattice and $f:S\to S$ an order-preserving map. Then 
\[ \bigwedge \setbuilder{x\in S}{f(x) \leq x} \]
is the least fixed point of $f$.
\end{theorem}
\begin{proof}
Let $P$ be the set of fixed points of $f$. Set $H = \setbuilder{x\in L}{f(x)\leq x}$ and $\alpha = \bigwedge H$. It is clear that $\alpha$ is a lower bound of $P$ because $P \subset H$. So we just need to show that $\alpha$ is a fixed point.

Now $f(\alpha)$ is a lower bound of $H$ due to $f$ being order preserving: for all $x\in H$ we have $\alpha\leq x$, so $f(\alpha) \leq f(x)$ and also $f(\alpha) \leq f(x) \leq x$. So $f(\alpha)\leq \alpha$ because $\alpha$ is the greatest lower bound.

Conversely, $f(f(\alpha)) \leq f(\alpha)$ because $f$ is order preserving. This means $f(\alpha)\in H$, so $f(\alpha)\leq \alpha$. We have thus shown that $\alpha = f(\alpha)$.
\end{proof}

\subsection{Semilattice homomorphisms}
\begin{proposition} \label{orderPreservingFunctionLatticeOperations}
Let $\sSet{P,\leq}$, $\sSet{Q,\leq}$ be posets and $f: P\to Q$ an order-preserving function.
\begin{enumerate}
\item If $P,Q$ are $\wedge$-semilattices, then
\[  \forall x,y\in P: f(x\wedge y) \leq f(x)\wedge f(y) \quad\iff\quad \text{$f$ is order-preserving}. \]
\item If $P,Q$ are complete $\wedge$-semilattices, then
\[  \forall D\subseteq P: f(\bigwedge D) \leq \bigwedge f[D] \quad\iff\quad \text{$f$ is order-preserving}. \]
\item If $P,Q$ are $\vee$-semilattices, then
\[  \forall x,y\in P: f(x\vee y) \geq f(x)\vee f(y) \quad\iff\quad \text{$f$ is order-preserving}. \]
\item If $P,Q$ are complete $\vee$-semilattices, then
\[  \forall D\subseteq P: f(\bigvee D) \geq \bigvee f[D] \quad\iff\quad \text{$f$ is order-preserving}. \]
\end{enumerate} 
\end{proposition}
\begin{proof}
(1) Assume $f(x\wedge y) \leq f(x)\wedge f(y)$. Then $x\leq y$ implies $x = x\wedge y$, which implies $f(x) = f(x\wedge y) \leq f(x)\wedge f(y) \leq f(y)$.

For the converse: From $x\wedge y \leq x$ we get $f(x\wedge y) \leq f(x)$. Similarly $f(x\wedge y) \leq f(y)$. Together this gives $f(x\wedge y) \leq f(x)\wedge f(y)$.

(2) The direction $\Rightarrow$ follows from (1). Assume $f$ order-preserving. Take arbitrary $D\subseteq P$. For all $x\in D$ we have $\bigwedge D \leq x$, so $f(\bigwedge D) \leq f(x)$. This means $f(\bigwedge D)$ is a lower bound of $f[D]$, so $f(\bigwedge D) \leq \bigwedge f[D]$.

(3,4) Similar.
\end{proof}
\begin{corollary}
Let $\sSet{S,\wedge}, \sSet{T, \wedge}$ be a semilattices and $f: S\to T$ a function. If $f$ is a semilattice homomorphism, then it is order-preserving. The converse does not hold in general.
\end{corollary}
\begin{proof}
We just need to show the converse does not hold. TODO
\end{proof}

\begin{lemma}
Let $\sSet{P,\leq}$, $\sSet{Q,\leq}$ be posets and $f: P\to Q$ a bijective order-embedding.
Then $f$ preserves all meets and joins, i.e. $\forall D\subseteq P$:
\[ f(\bigwedge D) = \bigwedge f[D] \qquad\text{and}\qquad f(\bigvee D) = \bigvee f[D] \]
if the relevant meets and joins exist.
\end{lemma}
TODO does existence of one side of an equals mean existence of the other???
\begin{proof}
We have already proved an inequality in both cases. By bijectivity we can find a $c\in P$ such that $f(c) = \bigwedge f[D]$. By order-reflection, we have that $c$ is a lower bound of $D$, so $c\leq \bigwedge D$. By order-preservation we have $\bigwedge f[D] = f(c) \leq f(\bigwedge D)$.

The other equality is dual.
\end{proof}
\begin{corollary} \label{orderReversingSimilarityLatticeOperations}
Let $\sSet{P,\leq}$, $\sSet{Q,\leq}$ be posets and $f: P\to Q$ a reverse bijective order-embedding. Then $\forall D\subseteq P$
\[ f(\bigwedge D) = \bigvee f[D] \qquad\text{and}\qquad f(\bigvee D) = \bigwedge f[D] \]
if the relevant meets and joins exist.
\end{corollary}

\begin{proposition}
Let $\sSet{S,\wedge}$ be a semilattice and $X$ a set. Then the pointwise ordering of $(X\to S)$ makes it a semilattice. For all $f,g\in (X\to S)$, the function $f\wedge g$ is defined by
\[ f\wedge g: X\to S: x\mapsto (f\wedge g)(x) = f(x) \wedge g(x). \]
If $S$ is a complete semilattice, then so is $(X\to S)$.
\end{proposition}

\begin{proposition} \label{semilatticeOfSemilatticeHomomorphisms}
Let $\sSet{S,\wedge}, \sSet{T, \wedge}$ be (complete) semilattices. The set of (complete) semilattice homomorphisms is a (complete) subsemilattice of $(S\to T)$.
\end{proposition}
\begin{proof}
We calculate for all $x,y\in S$:
\begin{align*}
(f\wedge g)(x\wedge y) &= f(x\wedge y)\wedge g(x\wedge y) = f(x)\wedge f(y) \wedge g(x) \wedge g(y) \\
&= (f(x)\wedge g(x))\wedge(f(y)\wedge g(y)) = (f\wedge g)(x) \wedge (f\wedge g)(y).
\end{align*}
TODO complete notation.
\end{proof}
TODO link universal algebra.

\subsection{Join- and meet-irreducible elements}
\begin{definition}
Let $L$ be a $\vee$-semilattice. We call $x\in L$ \udef{join-irreducible} if
\begin{itemize}
\item $x$ is not a least element of $L$,
\item for all $a,b\in L$: $x= a\vee b$ implies $x=a$ or $x=b$.
\end{itemize}
The definition of \udef{meet-irreducible} is dual on a $\wedge$-semilattice.

We denote the set of join-irreducible elements in $L$ as $\joinIr(L)$ and the set of meet-irreducible elements in $L$ as $\meetIr(L)$.
\end{definition}

\begin{lemma} \label{joinIrreducibleLemma}
Let $S$ be a $\vee$-semilattice and $x\in S$ not a least element. Then the following are equivalent:
\begin{enumerate}
\item $x$ is join-irreducible;
\item for all $a,b\in S$: $x= a\vee b$ implies $x \leq a$ or $x \leq b$;
\item for all $a,b\in S$: $x > a$ and $x > b$ implies $x > a\vee b$;
\item for all finite $F\subseteq S$: $x = \bigvee F$ implies $x\in F$.
\end{enumerate}
\end{lemma}

\begin{lemma}
In a finite lattice $L$, an element is join-irreducible if and only if it
has exactly one lower cover.
\end{lemma}

\begin{example}
Consider the lattice $\seq{\N,\lcm, \gcd}$. A non-zero element of $\N$ is join irreducible if and only if it is of the form $p^r$ for some prime $p$ and $r\in\N$.
\end{example}

\begin{lemma} \label{atomsJoinIrriducible}
Let $\sSet{P,\leq}$ be a poset. Then
\begin{enumerate}
\item $\atoms(P) \subseteq \joinIr(P)$;
\item $\coatoms(P) \subseteq \meetIr(P)$;
\end{enumerate}
\end{lemma}
\begin{proof}
(1) Assume $a\in \atoms(P)$. Then $a$ is not a least element of $P$. Assume there exist $x,y\in P$ such that $a = x\vee y$. Then $x,y\in \downset a \subseteq \{\bot, a\}$. If $x= \bot = y$, then $a = \bot$, which is not an atom. So either $x = a$ or $y = a$.

(2) Dual.
\end{proof}


\subsection{Generalised inverse of inf}
\begin{proposition}
Let $\sSet{P,\leq}$ be a poset and $S\subseteq P$ a subset such that $\sSet{S,\leq}$ is a complete $\wedge$-semilattice. Then
\[ f: P \to \powerset(S)^o: x\mapsto \upset x \cap S \qquad\text{and}\qquad g: \powerset(S)^o\to P: D\mapsto \bigwedge_S D \]
are order-preserving generalised inverses. Additionally, they form a Galois connection \textup{if and only if} for all $D\subseteq S$: $\bigwedge_P D$ exists and is equal to $\bigwedge_S D$.
\end{proposition}
\begin{proof}
TODO first part.

Assume for all $D\subseteq S$: $\bigwedge_P D$ exists and $\bigwedge_S D = \bigwedge_P D$. It is enough to verify, for all $x\in P$ and $B\subseteq S$
\[ (\upset x)\cap S \supseteq B \iff x\leq \bigwedge_S B. \]
Assume $(\upset x)\cap S \supseteq B$. Then $\upset x \supseteq B$, so $x \bigwedge_P \leq \bigwedge_P B = \bigwedge_S B$. Conversely, assume $x\leq \bigwedge_S B$. Then $B\subseteq \upset_S\bigwedge_S B \subseteq \upset_S x = \upset x\cap S$.

Assume these functions form a Galois connection. In order to obtain a contradiction, assume there exists a $D\subseteq S$ such that $\bigwedge_P D$ does not exist or is larger than $\bigwedge_S D$ (it can never be smaller). In both cases this means there exists a lower bound $y$ of $D$ in $P$ that is larger than $\bigwedge_S D$.

Then $(\upset y)\cap S \supseteq (\upset \bigwedge_S D)\cap S \supseteq D$, but $y\geq \bigwedge_S D$.
\end{proof}

\subsubsection{Projection onto a complete subsemilattice}
\begin{definition}
Let $\sSet{P,\leq}$ be a poset and $S\subseteq P$ a subset such that $\sSet{S,\leq}$ is a complete $\wedge$-semilattice. Then the \udef{(upper) order projection} on $S$ is the function
\[ P_S^u: P \to S: x\mapsto \bigwedge(\upset x\cap S). \]
\end{definition}

\begin{lemma}
Let $\sSet{P,\leq}$ be a poset and $S\subseteq P$ a subset that is a complete $\wedge$-semilattice. The upper order projection $P_S^u$ is
\begin{enumerate}
\item idempotent;
\item order-preserving.
\end{enumerate}
Also $P_S^u$ is expansive \textup{if and only if} for all $D\subseteq S$: $\bigwedge_S D = \bigwedge_P D$.
\end{lemma}
\begin{proof}
(1) Assume $x\in S$. Then $x\in (\upset x \cap S)$ and $x$ is a lower bound of $(\upset x \cap S)$, so $P_S^u(x) = x$.

(2) Assume $x \leq y$. Then $\upset x \supseteq \upset y$, so $\upset x\cap S \supseteq \upset y\cap S$ and $\bigwedge(\upset x\cap S) \subseteq \bigwedge(\upset y\cap S)$.


\end{proof}

\section{Lattices}
\url{file:///C:/Users/user/Downloads/Gr%C3%A4tzer,%20George%20-%20General%20lattice%20theory-Birkh%C3%A4user%20(2007).pdf}
\url{file:///C:/Users/user/Downloads/R.%20Padmanabhan,%20S.%20Rudeanu%20-%20Axioms%20for%20lattices%20and%20Boolean%20algebras-World%20Scientific%20(2008).pdf}
\begin{definition}
A \udef{lattice} is an algebraic structure $\seq{L,\vee, \wedge}$, where $\vee, \wedge$ are binary operations on the set $L$ such that $\seq{L,\vee}$ and $\seq{L,\wedge}$ are semilattices and $\vee,\wedge$ are linked by the absorption law:
\[ \forall a,b\in L: \; a \vee (a \wedge b) = a = a \wedge (a\vee b). \]
We call
\begin{itemize}
\item $\sSet{L,\vee}$ the \udef{join-semilattice} and $a\vee b$ the \udef{join} of $a$ and $b$;
\item $\sSet{L,\wedge}$ the \udef{meet-semilattice} and $a\wedge b$ the \udef{meet} of $a$ and $b$.
\end{itemize}
We call a lattice \udef{bounded} if both the join- and the meet-semilattice are bounded. We denote
\begin{itemize}
\item the identity of the join-semilattice by $\bot$;
\item the identity of the meet-semilattice by $\top$.
\end{itemize}
We denote the bounded lattice $\sSet{L, \vee, \wedge, \bot, \top}$.
\end{definition}
By \ref{absorptionIdempotency} the absortion law renders the axiom of  idempotency of the semilattices redundant. So we just need that $\seq{L,\vee}$ and $\seq{L,\wedge}$ are commutative semigroups that are linked by the absorption law.

\begin{proposition}
Observations from universal algebra:
\begin{enumerate}
\item Subset closed under $\vee,\wedge$ is sublattice.
\item Product lattices
\item Inverse of homomorphism is homomorphism
\end{enumerate}
\end{proposition}

\begin{example}
\begin{itemize}
\item Let $X$ be a set. Then $\mathcal{P}(X)$ ordered by inclusion is a lattice and $\cup, \cap$ are the corresponding join and meet operations.

In particular, for all $A,B\in \mathcal{P}(X)$:
\begin{enumerate}
\item $\inf\{A,B\} = A\cap B$;
\item $\sup\{A,B\} = A\cup B$.
\end{enumerate}
\item The natural numbers forms a lattice if ordered by division. The meet and join are given by
\[ m\vee n = \lcm\{m,n\} \qquad \text{and}\qquad m\wedge n = \gcd\{m,n\}. \]
\end{itemize}
\end{example}

\subsection{Lattices and order}
As for semilattices, we can equivalently characterise lattices as posets with certain conditions.
\begin{proposition}
Let $L$ be a set.
\begin{enumerate}
\item If $\seq{L,\vee, \wedge}$ is a lattice, then $\seq{L,\leq}$ is a poset such that every two element set has a supremum and an infimum, where
\[ \forall x,y\in L:\; x\leq y \qquad \iff \qquad x\vee y = y \]
or, equivalently,
\[ \forall x,y\in L:\; x\leq y \qquad \iff \qquad x\wedge y = x. \]
\item If $\seq{L,\leq}$ is a poset such that every two element set has a supremum and an infimum, then $\seq{L,\vee, \wedge}$ is a lattice, where
\[ \forall x,y\in L: \; x\vee y = \sup\{x,y\} \quad \text{and} \quad x\wedge y = \inf\{x,y\}. \]
\end{enumerate}
The order can also be defined by
\[ \forall x,y\in L:\; x\leq y \qquad \iff \qquad x\wedge y = x. \]
\end{proposition}
\begin{proof}
Mostly this follows from \ref{orderSemilattice}. We just need to show the two definitions of order are equivalent. This follows from the absorption law:
\begin{align*}
x\vee y &= y \implies x\wedge y = x\wedge (x\vee y) = x \\
x\wedge y &= x \implies x\vee y = (x\wedge y) \vee y = y.
\end{align*}
\end{proof}

\begin{lemma} \label{orderLatticeLemma}
Let $L$ be a lattice and $a,b,c,d\in L$. If $a\leq b$ and $c\leq d$, then
\[ a\vee c\leq b\vee d \qquad\text{and}\qquad a\wedge c \leq b\wedge d. \]
\end{lemma}
\begin{proof}
If $a\leq b$ and $c\leq d$, then we have
\[ \begin{cases}
a = a\wedge b \\ c = c\wedge d
\end{cases} \quad \text{and} \quad \begin{cases}
b = a\vee b \\ d = c\vee d.
\end{cases}\]
We calculate
\[ (a\vee c)\wedge (b\vee d) = (a\vee c)\wedge (a\vee b\vee c\vee d) = (a\vee c)\wedge ((a\vee c)\vee (b\vee d)) = a\vee c, \]
so $a\vee c\leq b\vee d$.
Similarly
\[ (a\wedge c)\vee (b\wedge d) = (a\wedge b\wedge c\wedge d)\vee (b\wedge d) = ((a\wedge c)\wedge (b\wedge d))\vee (b\wedge d) = b\wedge d, \]
so $a\wedge c \leq b\wedge d$.
\end{proof}
\begin{corollary} \label{orderLatticeCorollary}
Let $L$ be a lattice and $a,b,c,d\in L$. Then
\begin{enumerate}
\item if $a\leq b$ then $a\vee c \leq b \vee c$ and $a\wedge c \leq b\wedge c$;
\item if $a\leq c$ and $b\leq c$, then $a\vee b \leq c$;
\item if $a\leq b$ and $a\leq c$, then $a\leq b\wedge c$.
\end{enumerate}
\end{corollary}

\begin{lemma} \label{cancellationGeneralLattices}
Let $L$ be a lattice and $a,b\in L$. If for all $x\in L$: $a\vee x = b\vee x$ and $a\wedge x = b\wedge x$, then $a = b$.
\end{lemma}
\begin{proof}
Setting $x = a$ gives $a = a\vee b$ and $a = a\wedge b$. Thus $a \leq b$ and $b \leq a$, meaning $a = b$.
\end{proof}

\begin{proposition}[Mini-max theorem]
Let $L$ be a lattice and let $\seq{a_{i,j}}\subset L$ be indexed by $i,j\in \N$. Then
\[ \bigvee_{j=1}^n \left(\bigwedge_{i=1}^m a_{i,j}\right) \leq \bigwedge_{i=1}^m \left(\bigvee_{j=1}^n a_{i,j}\right). \]
\end{proposition}
\begin{proof}
For all $k,l$ we have $a_{k,l}\leq \bigvee_{j=1}^n a_{k,j}$. This implies $\bigwedge_{i=1}^m a_{i,l} \leq \bigwedge_{i=1}^m \left(\bigvee_{j=1}^n a_{i,j}\right)$ for all $l$. Taking the supremum over $l$ gives the result.
\end{proof}
\begin{corollary}[Median inequality]
Let $L$ be a lattice and $a,b,c \in L$, then
\[ (a\wedge b) \vee (b\wedge c) \vee (c\wedge a) \leq (a\vee b)\wedge (b\vee c) \wedge (c\vee a). \]
\end{corollary}
\begin{proof}
Use $a_{i,j} = \begin{pmatrix}
a & b & a \\ b & b & c \\ a & c & c
\end{pmatrix}$.
\end{proof}
\begin{corollary}[Distributive inequalities] \label{distributiveInequality}
Let $L$ be a lattice and $a,b,c \in L$, then
\begin{align*}
(a\wedge b)\vee (a\wedge c) &\leq a\wedge (b \vee c); \\
(a\vee b)\wedge (a\vee c) &\geq a\vee (b \wedge c).
\end{align*}
In particular this also means
\[ c \leq a \implies (a\wedge b)\vee c \leq a\wedge (b\vee c).  \]
\end{corollary}
\begin{proof}
Use $a_{i,j} = \begin{pmatrix}
a & a \\ b & c 
\end{pmatrix}$ and $a_{i,j} = \begin{pmatrix}
a & b \\ a & c 
\end{pmatrix}$. The particular cases follow because in this case $a\wedge c = c$. This statement is self-dual.
\end{proof}
The distributive inequalities are fairly elementary and do not need to be derived from mini-max theorem. 
In fact they are corollary to the following lemma which can be obtained by more elementary means:
\begin{lemma} \label{UpperLowerBoundsMeetJoin}
Let $L$ be a lattice $x\in L$ and $S\subseteq L$ a subset. Then
\begin{enumerate}
\item $S^u\vee x \subseteq (S\vee x)^u$ and $S^l\vee x \subseteq (S\vee x)^l$;
\item $S^u\wedge x \subseteq (S\wedge x)^u$ and $S^l\wedge x \subseteq (S\wedge x)^l$.
\end{enumerate}
\end{lemma}
\begin{proof}
The maps $y\mapsto y\vee x$ and $y\mapsto y \wedge x$ are order-preserving, so we can use \ref{orderPreservingReversingBounds}.
\end{proof}
\begin{corollary}[Infinite distributive inequalities] \label{infiniteDistributiveInequalities}
Let $L$ be a lattice $x\in L$ and $S\subseteq L$ a subset. Assume all relevant suprema exist, then
\begin{enumerate}
\item $\sup(S)\wedge x \geq \sup(S \wedge x)$;
\item $\inf(S)\vee x \leq \inf(S \vee x)$.
\end{enumerate}
\end{corollary}
\begin{proof}
Because $\sup(S)\in S^u$, we have $\sup(S)\wedge x \in S^u\wedge x$, so $\sup(S)\wedge x \in (S\wedge x)^u$ and $\sup(S)\wedge x \geq \min((S\wedge x)^u) = \sup(S\wedge x)$. The second part is dual.
\end{proof}
In particular if $S$ has two elements, we recover the distributive inequalities.


\subsection{Complete lattices}
\begin{lemma}
Let $L$ be a lattice. For every finite set $S\subset L$, $\sup(S)$ and $\inf(S)$ exist.
\end{lemma}
\begin{definition}
Let $L$ be a lattice. We call $L$ a \udef{complete lattice} if each subset $S\subseteq L$ has both a supremum and an infimum. We write
\[ \sup(S) = \bigvee S \qquad \inf(S) = \bigwedge S. \]
If we want to emphasise that the supremum/infimum of $S$ is taken as a subset of $L$, we write $\bigvee_L S$ and $\bigwedge_L S$.
\end{definition}
Clearly every finite lattice is complete.

\begin{lemma} \label{supInfFiniteSubsetsLattice}
Let $L$ be a lattice and $F\subseteq L$ a finite subset. Then $\bigvee F$ and $\bigwedge F$ exist. 
\end{lemma}

\begin{example}
For any set $X$, $\powerset(X)$ is a complete lattice.
\end{example}

\begin{lemma}
Let $P$ be an ordered set such that all relevant suprema and infima exist and $S,T\subseteq P$. Then
\begin{enumerate}
\item $\bigvee S \leq \bigwedge T$ if and only if $s\leq t$ for all $s\in S,t\in T$;
\item if $S\subseteq T$, then $\bigvee S \leq \bigvee T$ and $\bigwedge S \geq \bigwedge T$;
\item $\bigvee(S\cup T) = \left(\bigvee S\right)\vee \left(\bigvee T\right)$ and $\bigwedge(S\cup T) = \left(\bigwedge S\right)\wedge \left(\bigwedge T\right)$.
\end{enumerate}
\end{lemma}

\begin{proposition}[Mini-max theorem for complete lattices]
Let $L$ be a complete lattice, $I,J$ index sets and $\setbuilder{a_{i,j}\in L}{i\in I, j\in J}$. Then
\[ \bigvee_{j\in J}\bigwedge_{i\in I}a_{i,j} \leq \bigwedge_{i\in I}\bigvee_{j\in J} a_{i,j}. \]
\end{proposition}
\begin{proof}
For all $k,l$ we have $a_{k,l}\leq \bigvee_{j\in J} a_{k,j}$. This implies $\bigwedge_{i\in I} a_{i,l} \leq \bigwedge_{i\in I}\bigvee_{j\in J} a_{i,j}$ for all $l$. Taking the supremum over $l$ gives the result.
\end{proof}

\begin{proposition} \label{completeLatticeBasic}
Let $P$ be a non-empty ordered set. Then the following are equivalent:
\begin{enumerate}
\item $P$ is a complete lattice;
\item $\bigvee S$ exists for all subsets $S\subseteq P$;
\item $\bigwedge S$ exists for all subsets $S\subseteq P$;
\item $P$ has a bottom element $\bot$ and $\bigvee S$ exists for all non-empty $S\subseteq P$;
\item $P$ has a top element $\top$ and $\bigwedge S$ exists for all non-empty $S\subseteq P$;
\item for all $x\in P$ both $\upset x$ and $\downset x$ are complete lattices.
\end{enumerate}
\end{proposition}
\begin{proof}
The only difficult implication is $(5)\Rightarrow (1)$. All infima exist because $\inf(\emptyset) = \top \in P$. New each non-empty set $S$ in $P$ has an upper bound, $\top$, so $\bigvee S$ exists in $P$ by \ref{existenceSupremaInfima}. Finally $\bigvee \emptyset = \bot = \bigwedge P \in P$.
\end{proof}

\begin{theorem}[Knaster-Tarski fixed-point theorem]
Let $L$ be a complete lattice and $f:L\to L$ an order-preserving map. Then 
\[ \bigvee \setbuilder{x\in L}{x\leq f(x)} \qquad\text{and}\qquad \bigwedge \setbuilder{x\in L}{x\geq f(x)} \]
are, resp., the greatest and the least fixed point of $f$.
\end{theorem}
\begin{proof}
This is just two applications of the semilattice formulation of the Knaster-Tarski theorem, \ref{KnasterTarskiSemilattice}, once for the join-semilattice and once for the meet-semilattice.
\end{proof}
\begin{corollary}
The set of fixed points of $f$ forms a complete lattice.
\end{corollary}
\begin{proof}
Let $P$ be the set of fixed points. To show $P$ is a complete lattice, take any subset $S\subset P$.
Set $w = \bigvee S$, where $S$ is considered as a subset of $L$. For all $x\in W$: $x\leq w$, which implies $x=f(x)\leq f(w)$. As $w$ is the least upper bound, we have $w\leq f(w)$. This implies $f[\upset w]\subseteq \upset w$, meaning we can view $f$ as a function on the complete lattice $\upset w$. In particular $f|_{\upset w}$ has a least fixed point by the theorem, so $S$ has a supremum in $P$. The existence of the infimum is dual.
\end{proof}
\begin{corollary}[Banach decomposition theorem]
Let $X,Y$ be sets and $f:X\to Y$ and $g:Y\to X$ functions. There exist partitions $X_1,X2$ and $Y_1,Y_2$ of $X$ and $Y$ such that
\[ f[X_1] = Y_1 \qquad\text{and}\qquad g[Y_2] = X_2. \]
\end{corollary}
\begin{proof}
Consider the map $F: \powerset(X)\to\powerset(X): S\mapsto X\setminus g[Y\setminus f[S]]$. By the theorem this map has a fixed point, which we call $X_1$. We then need to set $Y_1 = f[X_1], X_2 = X\setminus X_1$ and $Y_2 = Y\setminus Y_1$. The fact $X_1$ is a fixed point means that $X_1 = X\setminus g[Y\setminus f[X_1]] = g[Y\setminus Y_1] = g[Y_2]$.
\end{proof}
\begin{corollary}[Schröder-Bernstein]
Let $X,Y$ be sets and $f:X\rightarrowtail Y$ and $g:Y\rightarrowtail X$ injective functions. Then there exists a bijective function $h: X\twoheadrightarrowtail Y$.
\end{corollary}
\begin{proof}
Use the Banach decomposition theorem to obtain partitions $X_1,X_2$ and $Y_1,Y_2$. Then $f|_{X_1}: X_1\twoheadrightarrowtail Y_1$ and $g|_{Y_2}: Y_2 \twoheadrightarrowtail X_2$ are bijective, so we can construct
\[ h: X\twoheadrightarrowtail Y: x \mapsto \begin{cases}
f(x) & x\in X_1 \\ (g|_{Y_2})^{-1}(x) & x\in X_2
\end{cases}. \]
\end{proof}
The Schröder-Bernstein theorem was already proven in \ref{SchroederBernstein}.

\subsubsection{Upper and lower order projection}
\begin{definition}
Let $L$ be a complete lattice and $S\subseteq L$ a subset. We define
\begin{itemize}
\item the \udef{upper projection} on $S$ as the map $P_S^U: L\to S: x\mapsto \bigwedge (\upset x\cap S)$;
\item the \udef{lower projection} on $S$ as the map $P_S^L: L\to S: x\mapsto \bigvee (\downset x\cap S)$.
\end{itemize}
\end{definition}

\begin{lemma}
Let $L$ be a complete lattice and $S\subseteq L$ a subset. Then the projections $P_S^U, P_S^L$ are
\begin{enumerate}
\item idempotent;
\item order-preserving.
\end{enumerate}
\end{lemma}
\begin{proof}

\end{proof}

\subsubsection{Chain conditions}
The following requires dependent choice:
\begin{proposition} \label{ascendingDescendingChainLattice}
Let $L$ be a lattice. Then
\begin{enumerate}
\item if $L$ satisfies the ascending chain condition, then for all non-empty subsets $S\subset L$ there exists a finite set $F\subset L$ such that $\bigvee S = \bigvee F$;
\item if $L$ satisfies the descending chain condition, then for all non-empty subsets $S\subset L$ there exists a finite set $F\subset L$ such that $\bigwedge S = \bigwedge F$;
\item if $L$ has a bottom element and satisfies the ascending chain condition, then $L$ is complete;
\item if $L$ has a top element and satisfies the descending chain condition, then $L$ is complete;
\item if $L$ has no infinite chains, then $L$ is complete.
\end{enumerate}
\end{proposition}
\begin{proof}
(1) Assume $L$ satisfies the ascending chain condition and let $S\subset L$ be non-empty. Define
\[ B = \setbuilder{\bigvee G}{\text{$G$ is a finite, non-empty subset of $S$}}. \]
This is well-defined by \ref{supInfFiniteSubsetsLattice}. Then $B$ has a maximal element $m = \bigvee F$ for some finite $F$ by \ref{welfoundedACC}.

Now $m$ is an upper bound of $S$. Indeed, let $x\in S$. Then $m= \bigvee F \leq \bigvee (F\cup\{x\})$ because $F\subseteq (F\cup \{x\})$. Since $m$ is maximal in $B$, we have $m = \bigvee (F\cup\{x\}) \geq x$. It is clearly also the least upper bound, otherwise it was not the least upper bound of $F$.

(2) Dual of 1.

(3) This follows from 1. and \ref{completeLatticeBasic}.

(4) Dual of 3.

(5) A lattice with no infinite chains satisfies the ascending chain condition. Also a lattice
with no infinite chains has a bottom element. (TODO: need dependent/countable choice?)
\end{proof}

\begin{proposition} \label{joinIrreducibilityDescendingChainLattice}
Let $L$ be a lattice satisfing the descending chain condition. Then
\begin{enumerate}
\item $\forall a,b\in L:\; a\nleq b \implies \exists x\in \joinIr(L): \; x\leq a$ and $x\nleq b$;
\item $\forall a\in L:\; a = \bigvee\setbuilder{x\in\joinIr(L)}{x\leq a}$.
\end{enumerate}
\end{proposition}
\begin{proof}
(1) Set $S = \setbuilder{x\in L}{\text{$x\leq a$ and $x\nleq b$}}$, which is non-empty and thus contains a minimal element $m$ by \ref{welfoundedACC}. We claim $m$ is join-irreducible. Assume, towards a contradiction, $x = c\vee d$ and $c < x > d$. By minimality of $x$, $c,d\notin S$. As $c,d< x \leq a$, we must have $c,d\leq b$. But this means $x\leq b$, so $x\notin S$ which is a contradiction.

(2) Set $T = \setbuilder{x\in\joinIr(L)}{x\leq a}$. Clearly $a$ is an upper bound of $T$. To see that it is the least upper bound, take a different upper bound $c$. Assume, towards a contradiction, that $a\nleq c$. Then $a\nleq a\wedge c$. By point 1. there exists an $x\in\joinIr(L)$ such that $x\leq a$ (meaning $x\in T$) and $x\nleq a\wedge c$. But if $c$ were an upper bound of $T$, then $x\leq a\wedge c$, which is a contradiction.
\end{proof}

\begin{proposition}
Let $L$ be a lattice.
\begin{enumerate}
\item If $L$ satisfies the descending chain condition, then any subset $Q\supseteq \joinIr(L)$ is join-dense in $L$.
\item If $L$ satisfies the ascending chain condition and $Q$ is join-dense in $L$, then for all $a\in L$ there exists a finite subset $F$ of $Q$ such that $a = \bigvee F$.
\end{enumerate}
\end{proposition}
\begin{proof}
(1) is a corollary of \ref{joinIrreducibilityDescendingChainLattice}. (2) is a corollary of \ref{ascendingDescendingChainLattice}.
\end{proof}
\begin{corollary}
Let $L$ be a lattice with no infinite chains. Then
\begin{enumerate}
\item for each $a \in L$, there exists a finite subset $F$ of $\joinIr(L)$ such that $a = F$.
\item $Q\subseteq L$ is join-dense in $L$ if and only if $Q \supseteq \joinIr(L)$.
\end{enumerate}
\end{corollary}
\begin{proof}
If $L$ has no finite chains, then it satisfies the ascending and descending chain conditions.

Only the $\Rightarrow$ direction of (2) is not immediately obvious. Assume $Q$ is join-dense and let $x\in \joinIr(L)$. By the proposition there exists a finite $F\subseteq Q$ such that $x = \bigvee F$. Since $x$ is join-irreducible, we have $x \in F$ and hence $x \in Q$. Thus, $\joinIr(L) \subseteq Q$.
\end{proof}


\subsection{Filters and ideals}
Since lattices are posets, we can define filters and ideal on them.
We can give alternative characterisations of filters and ideals using the lattice operations.
\begin{lemma}
Let $L$ be a lattice and $J$ a subset. Then
\begin{enumerate}
\item $J$ is an ideal \textup{if and only if}
\begin{enumerate}
\item $a,b\in J \implies a\vee b\in J$;
\item $x\in L, b\in J$ and $x\leq b$ implies $x\in J$;
\end{enumerate}
\textup{if and only if}
\begin{enumerate}
\item $\forall a,b\in J: \qquad a\vee b\in J$;
\item $\forall x\in L, \forall b\in J: \quad x\wedge b \in J$;
\end{enumerate}
\textup{if and only if}
\begin{enumerate}
\item $J$ is a sublattice;
\item $J$ is a down-set;
\end{enumerate}
\item $J$ is a filter \textup{if and only if}
\begin{enumerate}
\item $a,b\in J \implies a\wedge b\in J$;
\item $x\in L, b\in J$ and $x\geq b$ implies $x\in J$;
\end{enumerate}
\textup{if and only if}
\begin{enumerate}
\item $\forall a,b\in J: \qquad a\wedge b\in J$;
\item $\forall x\in L, \forall b\in J: \quad x\vee b \in J$.
\end{enumerate}
\textup{if and only if}
\begin{enumerate}
\item $J$ is a sublattice;
\item $J$ is a up-set.
\end{enumerate}
\end{enumerate}
\end{lemma}

TODO difference in definition between posets and lattices?

\begin{lemma}
Let $L$ be a lattice. Then
\begin{enumerate}
\item if $L$ contains $\top$, then an ideal in $L$ is proper \textup{if and only if} is does not contain $\top$;
\item if $L$ contains $\bot$, then a filter in $L$ is proper \textup{if and only if} is does not contain $\bot$.
\end{enumerate}
\end{lemma}


\begin{proposition}[The lattice of filters]
Let $L$ be a lattice. Consider the set of filters $\filters{L}$ to be ordered by inclusion. Then $\filters{L}$ is a complete bounded lattice with top $L$ and bottom $\emptyset$.

Let $\{F_i\}_{i\in I} \in \filters(L)^I$ be a set of filters. Then
\begin{align*}
\bigwedge \{F_i\}_{i\in I} &= \bigcap_{i\in I}F_i; \\
\bigvee \{F_i\}_{i\in I} &=  \closure_\wedge\left(\bigcup_{i\in I}F_i\right).
\end{align*}
Note that $\closure_\wedge$ only means closure under finite meets.
\end{proposition}
\begin{proof}
We just need to verify that the proposed suprema and infima are indeed filters.

TODO
\end{proof}
\begin{proposition} \label{joinProperFilter}
Let $L$ be a lattice and $\{F_i\}_{i\in I} \in \filters(L)^I$ be set of filters. Then $\bigvee \{F_i\}_{i\in I}$ is a proper filter \textup{if and only if} $\bigvee \{F_i\}_{i\in I}\mesh \bigvee \{F_i\}_{i\in I}$.
\end{proposition}

\subsubsection{Free and principal filters}
\begin{definition}
Let $L$ be a complete lattice and $F\subseteq P$ a filter.
\begin{itemize}
\item The \udef{kernel} of the filter $F$ is $\ker F \defeq \bigwedge F$.
\item A filter is called \udef{principal} if $\ker F \in F$;
\item A filter is called \udef{free} if $\ker F = \bot$.
\end{itemize}
The set of principal filters in $P$ is denoted $\filters_*(P)$ and the set of free filters in $P$ is denoted $\filters_0(P)$.
\end{definition}
The only filter that is both principal and free is $L$. A proper filter cannot be both principal and free.

\begin{lemma}
Let $L$ be a lattice and $F$ a principal filter in $L$. Then $F = \upset \ker F$.
\end{lemma}

\begin{lemma} \label{finiteFiltersPrincipal}
Let $L$ be a finite lattice. Then every filter in $L$ is principal.
\end{lemma}
\begin{proof}
Every filter $F$ in a finite lattice is finite, so $\bigwedge F$ is a finite meet, and thus must be in $F$.
\end{proof}

\begin{proposition}
Let $X$ be a set and $F$ a free filter in $\filters_0(\powerset(X))$. If $A\in F$, then $F$ is finer than the cofinite filter of $A$.
\end{proposition}
\begin{proof}
As filters are upwards closed, ``$F$ finer than the cofinite filter of $A$'' means the cofinite filter of $A$ is a subset of $F$. So take some cofinite subset $C$ of $A$, such that $A\setminus C = \{x_1, \ldots, x_n\}$ is finite. Now for each $x_i$ we can find a set $f_i\in F$ such that $x_i\notin f_i$, because if this was not possible, then $x_i \in \bigcap F$, but $\bigcap F = \emptyset$ due to $F$ being free. Also $\bigcap_{i=1}^n f_i \in F$, because $F$ is closed under finite intersections.

So $(A\setminus C)\cap (\bigcap_{i=1}^n f_i) = \emptyset$, or equivalently $\bigcap_{i=1}^n f_i \subseteq X\setminus (A\setminus C)$, meaning $X\setminus (A\setminus C) \in F$ and thus $C = A\cap (X\setminus (A\setminus C)) \in F$.
\end{proof}
\begin{corollary}
A filter of subsets of $X$ is free \textup{if and only if} it is finer than the cofinite filter of $X$.
\end{corollary}

\begin{proposition}
Let $\sSet{P,\Yleft}$ be a complete ordered set (TODO: do we need Boolean lattice??) and $F\subseteq P$ a filter. Then there exists a unique pair of filters $F_*$ and $F_0$ such that $F_*$ is principal, $F_0$ is free and
\[ F = F_* \wedge F_0 \qquad P = F_* \vee F_0. \]
\end{proposition}
We call $F^*$ the \udef{principal part} and $F^0$ the \udef{free part} of the filter.
\begin{proof}

\end{proof}

\subsubsection{Prime and maximal filters and ideals}
\begin{definition}
Let $L$ be a lattice.
\begin{itemize}
\item a \udef{maximal filter} or \udef{ultrafilter} is a coatom in $\filters(L)$;
\item a \udef{maximal ideal} is a coatom in $\ideals(L)$;
\item a \udef{prime filter} is a meet-irreducible element in $\filters(L)$;
\item a \udef{prime ideal} is a meet-irreducible element in $\ideals(L)$.
\end{itemize}

We denote the set of ultrafilters on $L$ as $\ultrafilters(F)$.
\end{definition}

\begin{lemma}
Let $L$ be a lattice.
\begin{enumerate}
\item A filter $F\in \filters(L)$ is prime \textup{if and only if}
\begin{enumerate}
\item $F$ is proper;
\item $\forall x,y\in L$: $x\vee y \in F \implies x\in F$ or $y\in F$.
\end{enumerate}
\item An ideal $I\in \ideals(L)$ is prime \textup{if and only if}
\begin{enumerate}
\item $I$ is proper;
\item $\forall x,y\in L$: $x\wedge y \in I \implies x\in I$ or $y\in I$.
\end{enumerate}
\end{enumerate}
\end{lemma}
If $L$ is only a $\wedge$-semilattice (resp. $\vee$-semilattice), then only the direction $\Rightarrow$ holds in general.
\begin{proof}
(1) Assume $F$ is a prime filter. Then $F \neq \top = \powerset(L)$, so it is proper. Now assume $x\vee y\in F$, meaning $\upset x\vee y \subseteq F$. Thus
\[ F = \Cl_\wedge (F\cup \upset(x\vee y)) = (F\vee \upset x) \cap (F\vee \upset y). \]
Because $F$ is prime, we have $F = F\vee \upset x$ or $F\vee \upset y$.

Now assume TODO \url{http://www.ascent-journals.com/IJPEM/Vol3No3/1-sagi.pdf} using \ref{joinIrreducibleLemma}.
\end{proof}

\subsubsection{Sequential filters}
\begin{definition}
Let $A$ be a set and $x: I\to A$ a sequence in $A$. The \udef{sequential filter} associated with $x$
\end{definition}

\subsection{Distributive and modular lattices}
\subsubsection{Distributive lattices}
For all lattices $L$ the distributive inequalities, \ref{distributiveInequality}, hold: $\forall a,b,c \in L$:
\begin{align*}
a \vee (b\wedge c) &\leq (a\vee b) \wedge (a\vee c); \\
a\wedge (b \vee c) &\geq (a\wedge b)\vee (a\wedge c).
\end{align*}

The two corresponding equalities are equivalent:
\begin{proposition} \label{equivalenceDistributiveLaws}
Let $L$ be a lattice. Then the following are equivalent:
\begin{enumerate}
\item $\forall a,b,c \in L: \; a \vee (b\wedge c) = (a\vee b) \wedge (a\vee c)$;
\item $\forall a,b,c \in L: \; a\wedge (b \vee c) = (a\wedge b)\vee (a\wedge c)$.
\end{enumerate}
\end{proposition}
These equivalent equalities are known as the \udef{distributive laws}.
\begin{proof}
We show $(1)\Rightarrow (2)$. Then other implication follows by duality.

Assume (1). Then, for all $a,b,c \in L$:
\begin{align*}
(a\wedge b)\vee (a\wedge c) &= ((a\wedge b)\vee a) \wedge ((a\wedge b)\vee c) & \text{by (1)}\\
&= (a\wedge (c\vee (a\wedge b))  & \text{by the absorption law}\\
&= (a \wedge ((c\vee b) \wedge (c\vee a))  & \text{by (1)}\\
&= a\wedge (b\vee c) & \text{by the absorption law.}
\end{align*}
\end{proof}
Note that it is \emph{not} true that
\[ \forall a,b,c \in L: \; a \vee (b\wedge c) = (a\vee b) \wedge (a\vee c) \iff a\wedge (b \vee c) = (a\wedge b)\vee (a\wedge c). \]

\begin{definition}
A lattice $L$ is called \udef{distributive} if it satisfies the distributive laws.
\end{definition}

\begin{lemma}
A lattice is distributive \textup{if and only if} its dual is distributive.
\end{lemma}

\begin{example}
\begin{itemize}
\item Any totally ordered set is a distributive lattice.
\item For any set $X$, the poset $\sSet{\powerset(X), \subseteq}$ is a distributive lattice.
\end{itemize}
\end{example}

\begin{lemma}
Let $L$ be a lattice. The following are equivalent:
\begin{enumerate}
\item $L$ is distributive;
\item for all $x,y,z,w\in L: \quad x\wedge y \leq w \;\text{and}\; x\wedge z\leq w \implies x\wedge (y\vee z) \leq w;$
\item for all $x,y,z,w\in L: \quad x\vee y \geq w \;\text{and}\; x\vee z\geq w \implies x\vee (y\wedge z) \geq w$.
\end{enumerate}
\end{lemma}
\begin{proof}
Point (1) is self-dual and points (2) and (3) are dual, so it is enough to show that (1) and (2) are equivalent.

Assume $L$ distributive. By \ref{orderLatticeCorollary}, $ x\vee y \geq w$ and $x\vee z\geq w$ imply $(x\wedge y)\vee (x\wedge z) \leq w$. By distributivity, we get (2).

Conversely, we can take $w = (x\wedge y)\vee (x\wedge z)$. Then (2) gives $x\wedge (y\vee z) \leq (x\wedge y)\vee (x\wedge z)$ and the distributive inequality \ref{distributiveInequality} gives the other inequality. 
\end{proof}

\begin{proposition}[Cancellation law for distributive lattices] \label{cancellationDistributiveLattices}
Let $L$ be a distributive lattice and $a,b,c\in L$. Then
\[ \text{$a\vee c = b \vee c$ and $a\wedge c = b \wedge c$ implies $a = b$.} \]
\end{proposition}
Note that we only assume the equalities hold for some $c$, not all $c$. If these equalities hold for all $c$, then we have cancellation in all lattices, not just distributive ones. See \ref{cancellationGeneralLattices}.
\begin{proof}
We calculate, using the absorption law and distributivity,
\[ a = a\vee (a\wedge c) = a \vee (b\wedge c) = (a\vee b)\wedge (a\vee c) = (a\vee b) \wedge (b\vee c) = b \vee (a \wedge c) = b \vee (b\wedge c) = b. \]
\end{proof}

\subsubsection{Modular lattices}
For all lattices $L$ the modular inequality holds: $\forall a,b,c \in L$:
\[ a\leq c \implies a \vee (b\wedge c) \leq (a\vee b) \wedge c. \]
See \ref{distributiveInequality}.

\begin{proposition} \label{modularEquivalences}
Let $L$ be a lattice. Then the following are equivalent:
\begin{enumerate}
\item $\forall a,b,c\in L$: $a \vee (b\wedge c) = (a\vee b) \wedge c$ if $a\leq c$;
\item $\forall a,b,c\in L$: $a \vee (b\wedge c) = (a\vee b) \wedge (a\vee c)$ if $a\leq b$ or $a\leq c$; the dual of (2);
\item $\forall a,b,c\in L$: $a\vee (b\wedge (a\vee c)) = (a\vee b)\wedge (a\vee c)$; the dual of (3);
\item \textup{Shearing identity}: $\forall a,b,c\in L$: $(a\vee b) \wedge c = (a\vee (b\wedge (a\vee c)))\wedge c$; the dual of the shearing identity.
\end{enumerate}
\end{proposition}
The first of these is referred to as the \udef{modular law}. Notice that it is self-dual.
\begin{proof}
(1) is equivalent to its dual by replacing $a\leftrightarrow c$. This will imply all statements are equivalent to their duals once the equivalence with (1) has been established. 

$\boxed{(1)\Rightarrow (2)}$ Assume (1). Assume $a\leq b$ or $a\leq c$. By relabelling we can assume $a\leq c$.  Then $a\vee c = c$ and (2) clearly follows from (1).


$\boxed{(2)\Rightarrow (3)}$ Apply (2) to $a\leq a\vee c$.

$\boxed{(3)\Rightarrow (1)}$ Assume $a\leq c$. Then $a\vee c = c$ and $a\vee (b\wedge (a\vee c)) = (a\vee b)\wedge (a\vee c)$ reduces to $a \vee (b\wedge c) = (a\vee b) \wedge c$.

$\boxed{(3)\Rightarrow (4)}$ We calculate, using (3),
\[ (a\vee (b\wedge (a\vee c)))\wedge c = ((a\vee b)\wedge (a\vee c))\wedge c = (a\vee b) \wedge (a\vee c) \wedge c = (a\vee b) \wedge c. \]

$\boxed{(4)\Rightarrow (3)}$ TODO????
\end{proof}

\begin{definition}
A lattice $L$ is called \udef{modular} if it satisfies the modular law.
\end{definition}

\begin{lemma}
If a lattice is distributive, it is also modular.
\end{lemma}
\begin{proof}
If the lattice is distributive, point (2) of \ref{modularEquivalences} holds unconditionally, and so in particular also conditionally.
\end{proof}

\subsubsection{Derived lattices}

\begin{proposition}
(i) If L is a modular (distributive) lattice, then every sublattice of L
is modular (distributive).

(ii) If L and K are modular (distributive) lattices, then L × K is
modular (distributive).

(iii) If L is modular (distributive) and K is the image of L under a
homomorphism, then K is modular (distributive).
\end{proposition}
\begin{corollary}
If a lattice is isomorphic to a sublattice of a product
of distributive (modular) lattices,then it is distributive (modular).
\end{corollary}

\begin{theorem}[$\mathbf{M}_3-\mathbf{N}_5$ theorem]
Let $L$ be a lattice.
\begin{enumerate}
\item $L$ is non-modular \textup{if and only if} $\mathbf{N}_5 \rightarrowtail L$;
\item $L$ is non-distributive \textup{if and only if} $\mathbf{N}_5 \rightarrowtail L$ or $\mathbf{M}_3 \rightarrowtail L$.
\end{enumerate}
\end{theorem}
\begin{proof}
TODO!
\end{proof}


\section{Complementation}
\subsection{Disjoint elements and disjoint complement}
\begin{definition}
Let $L$ be a lattice with a bottom $\bot$.

We say $x,y\in L$ are \udef{disjoint} if $x\wedge y = \bot$. We write $x \perp y$.

Let $S\subset L$. Then the set
\[ S^\perp \defeq \setbuilder{x\in L}{\forall y\in S: x\perp y} \]
is called the \udef{disjoint complement} of $Y$.
\end{definition}

\begin{lemma} \label{disjointComplementIdeal}
Let $L$ be a distributive lattice with a bottom $\bot$ and $S\subset L$. Then $S^\perp$ is an ideal.
\end{lemma}
\begin{proof}
Let $a,b\in S^\perp$. Then for all $x\in S$, we have
\[ (a\vee b)\wedge x = (a\wedge x)\vee (b\wedge x) = \bot\vee \bot = \bot, \]
so $a\vee b\in S^\perp$.

It is also clear $S^\perp$ must be a down set by \ref{orderLatticeCorollary}.
\end{proof}

\subsection{Complementation}
\begin{definition}
Let $L$ be a bounded lattice and $x\in L$. We call $y$ a \udef{complement} of $x$ if
\[ x \vee y = \top \qquad \text{and} \qquad x\wedge y = \bot. \]
\end{definition}

\begin{proposition} \label{distributiveComplementUnique}
Let $L$ be a bounded lattice. If $L$ is distributive, then any $x\in L$ has at most one complement.
\end{proposition}
\begin{proof}
This follows from the cancellation law \ref{cancellationDistributiveLattices}. Let $y,y'$ be complements of $x$. Then $y\vee x = \top = y'\vee x$ and $y\wedge x = \bot = y'\wedge x$, so $y=y'$.
\end{proof}

A lattice element may also have no complements.
\begin{example}
The only elements with a complement in a bounded chain are $\top$ and $\bot$.
\end{example}

\subsubsection{Complemented lattices}
\begin{definition}
A \udef{complemented lattice} is a bounded lattice with a function $c:L\to L$, called the \udef{complementation}, such that $c(x)$ is a complement of $x$ for all $x\in L$.

A lattice in which every element has exactly one complement is called a \udef{uniquely complemented lattice}.

A lattice with the property that every interval (viewed as a sublattice) is complemented is called a \udef{relatively complemented lattice}.
\end{definition}
A distributive complemented lattice is uniquely complemented.

\begin{lemma}
Let $L$ be a complemented lattice with complementation $c:L\to L$. Then $c(\top) = \bot$ and $c(\bot) = \top$.
\end{lemma}
\begin{proof}
For $c(\top)$ to be a complement of $\top$, we need $\top \wedge c(\top) = \bot$. But for all $x\in L$ we have $x\leq \top$, so $\top\wedge x = x$. This means we have $c(\top) = \bot$.
\end{proof}

\begin{lemma} \label{uniqueComplementInvolution}
Let $L$ be a uniquely complemented lattice. Then the complementation $c:L\to L$ is an involution.
\end{lemma}
\begin{proof}
Clearly if $c(x)$ is a complement of $x$, then $x$ is a complement of $c(x)$. By uniqueness $c(c(x)) = x$.
\end{proof}

\subsubsection{Orthocomplemented lattices}
\begin{definition}
Let $L$ be a bounded lattice. An \udef{orthocomplementation} is a function $L \to L: x \to \overline{x}$ that maps each element $x\in L$ to an \udef{orthocomplement} $\overline{x}$ such that
\begin{itemize}
\item $x$ and $\overline{x}$ are complements;
\item $x\mapsto \overline{x}$ is an involution: $\overline{\overline{x}} = x$;
\item $x\mapsto \overline{x}$ is order-reversing: $x\leq y \implies \overline{y} \leq \overline{x}$.
\end{itemize}
An \udef{orthocomplemented lattice} or \udef{ortholattice} is a bounded lattice equipped with an orthocomplementation.
\end{definition}
An orthocomplemented lattice is not necessarily uniquely complemented.

\begin{lemma}
Let $L$ be an ortholattice with completement $c$. Then $c: L \to L$ is a reverse order-isomorphism.
\end{lemma}
\begin{corollary}
The dual lattice of an ortholattice is an ortholattice. We can view an orthocomplementation as an isomorphism between an ortholattice and its dual.
\end{corollary}
\begin{corollary}
Let $A\subseteq L$. Then
\begin{enumerate}
\item $c\left[\bigvee A\right] = \bigwedge c[A]$;
\item $c\left[\bigwedge A\right] = \bigvee c[A]$.
\end{enumerate}
\end{corollary}
\begin{proof}
See \ref{orderReversingSimilarityLatticeOperations}.
\end{proof}

\begin{proposition}
The cardinality of any finite ortholattice is either even or $1$.
\end{proposition}
\begin{proof}
Let $L$ be a finite ortholattice.
The orthocomplementation pairs elements $x,y$ such that $\overline{x} = y$ and $\overline{y} = x$. If for all such pairs we have $x\neq y$, then the cardinality of $L$ is even. Now assume there exists an $x\in L$ such that $\overline{x} = x$. Then 
\[ \bot = x\wedge \overline{x} = x\wedge x = x = x\vee x = x\vee \overline{x} = \top. \]
So $\bot = \top$, which is only possible if $L = \{\bot\}$.
\end{proof}

\begin{example}
\begin{itemize}
\item The lattice $\mathbf{M}_2 = \begin{tikzcd}[column sep={2em,between origins},row sep={2em,between origins}]
&\circ \ar[ld, dash] \ar[rd, dash] & \\ \circ \ar[rd, dash] && \circ \ar[ld, dash] \\ &\circ &
\end{tikzcd}$ admits a unique orthocomplementation.
\item The lattice $\mathbf{M}_3 = \begin{tikzcd}[column sep={2em,between origins},row sep={2em,between origins}]
&\circ \ar[ld, dash] \ar[d,dash] \ar[rd, dash] &\\ \circ \ar[rd, dash]& \circ \ar[d, dash] & \circ \ar[ld, dash] \\ &\circ &
\end{tikzcd}$ admits no orthocomplementations.
\item The lattice $\mathbf{M}_4 = \begin{tikzcd}[column sep={1em,between origins},row sep={2em,between origins}]
&&&\circ \ar[llld, dash] \ar[ld,dash] \ar[rd, dash] \ar[rrrd, dash] &&&\\ \circ \ar[rrrd, dash]&& \circ \ar[rd, dash] && \circ \ar[ld, dash] && \circ \ar[llld, dash] \\ &&&\circ &&&
\end{tikzcd}$ admits three orthocomplementations.
\item The hexagon lattice $\begin{tikzcd}[column sep={1.5em,between origins},row sep={1.7em,between origins}]
& \circ \ar[ld, dash] \ar[rd, dash] & \\
a \ar[d, dash] & & b \ar[d, dash] \\
c \ar[rd, dash] & & d \ar[ld, dash] \\
& \circ &
\end{tikzcd}$ admits a unique orthocomplementation, but it is not uniquely complemented. Indeed both of the functions $f: \begin{tikzcd}[column sep={1.5em,between origins},row sep={1.7em,between origins}]
& \circ \ar[ld, dash] \ar[rd, dash] & \\
a \ar[d, dash] \ar[rr, leftrightarrow] & & b \ar[d, dash] \\
c \ar[rd, dash] \ar[rr, leftrightarrow] & & d \ar[ld, dash] \\
& \circ &
\end{tikzcd}$ and $g: \begin{tikzcd}[column sep={1.5em,between origins},row sep={1.7em,between origins}]
& \circ \ar[ld, dash] \ar[rd, dash] & \\
a \ar[d, dash] \ar[rrd, leftrightarrow] & & b \ar[d, dash] \\
c \ar[rd, dash] \ar[rru, leftrightarrow] & & d \ar[ld, dash] \\
& \circ &
\end{tikzcd}$ are complementations. Only $g$ is an orthocomplementation, because $f$ does not reverse order: $a \geq c$  and $f(a) = b \geq d = f(c)$.
\end{itemize}
\end{example}

\begin{theorem}[De Morgan's laws]
Let $L$ be an ortholattice, then for all $x,y\in L$
\begin{enumerate}
\item $\overline{(x\vee y)} = \overline{x} \wedge \overline{y}$;
\item $\overline{(x\wedge y)} = \overline{x} \vee \overline{y}$.
\end{enumerate}
\end{theorem}
\begin{proof}
From $x\leq x\vee y$ and $y\leq x\vee b$, we have $\overline{(x\vee b)} \leq \overline{x}$ and $\overline{(x\vee b)} \leq \overline{y}$. By \ref{orderLatticeCorollary} we have $\overline{(x\vee y)}\leq \overline{x} \wedge \overline{y}$. For the other inequality we start with $\overline{x} \geq \overline{x} \wedge \overline{y}$ and $\overline{y} \geq \overline{x} \wedge \overline{y}$ to obtain $x\vee y \leq \overline{(\overline{x} \wedge \overline{y})}$, which implies $\overline{x} \wedge \overline{y} \leq \overline{(x\vee y)}$.
\end{proof}

\begin{proposition}
Let $L$ be a complemented lattice.

If the complement is an involution and satisfies either of the de Morgan laws, then $L$ is an ortholattice.
\end{proposition}
\begin{proof}
Assume the de Morgan law $\overline{(x\vee y)} = \overline{x} \wedge \overline{y}$ holds. Assume $x\leq y$. Then $x\vee y = y$, so
\[ \overline{y} = \overline{(x\vee y)} = \overline{x} \wedge \overline{y} \]
meaning $\overline{y} \leq \overline{x}$.
\end{proof}

The requirement that the complement be an involution is important. There are lattices in which the de Morgan laws hold that are not ortholattices.

\begin{example}
The lattice $\mathbf{M}_3 = \begin{tikzcd}[column sep={2em,between origins},row sep={2em,between origins}]
&\top \ar[ld, dash] \ar[d,dash] \ar[rd, dash] &\\ a \ar[rd, dash]& b \ar[d, dash] & c \ar[ld, dash] \\ &\bot &
\end{tikzcd}$ admits a complementation $'$ such that $a' = b$, $b' = c$ and $c' = a$ that is clearly not an orthocomplementation, but does satisfy the de Morgan laws.
\end{example}

TODO Ockham algebras, De Morgan algebras, Kleene algebras, Stone algebras.

\subsection{Boolean lattices}
\begin{definition}
A distributive complemented lattice is called a \udef{Boolean lattice} or \udef{Boolean algebra}.
\end{definition}
We will use $\overline{x}$ to denote the (necessarily unique, \ref{distributiveComplementUnique}) completement of $x$.

\begin{lemma} \label{BooleanComplementLargestDisjoint}
Let $L$ be a Boolean lattice and $x\in L$. Then $\{x\}^\perp = \downset \overline{x}$.
\end{lemma}
\begin{proof}
From the requirement $x\wedge \overline{x} = \bot$, we see that $\overline{x}\in \{x\}^\perp$. Now $\{\overline{x}\} \subseteq \{x\}^\perp$ implies $\downset \overline{x} \subseteq \downset\{x\}^\perp = \{x\}^\perp$. For the last equality we have used the fact that $\{x\}^\perp$ is downwards closed (in fact even an ideal), see \ref{disjointComplementIdeal}.

Now we prove the other inclusion: $\{x\}^\perp \subseteq \downset \overline{x}$.
Take $y\in \{x\}^\perp$. We just need to show that $y\leq \overline{x}$, or, equivalently, $y = y\wedge \overline{x}$. Indeed
\[ y = y \wedge \top = y\wedge (x\vee \overline{x}) = (y\wedge x)\vee (y\wedge \overline{x}) = \bot \vee (y\wedge \overline{x}) = y\wedge \overline{x}. \]
\end{proof}
    \begin{corollary} \label{BooleanInequalities}
Let $L$ be a Boolean lattice and $x,y\in L$. Then the following are equivalent:
\begin{enumerate}
\item $x \leq y$;
\item $x\wedge \overline{y} = \bot$;
\item $\overline{x} \vee y = \top$.
\end{enumerate}
\end{corollary}
\begin{corollary}
Every Boolean lattice is an ortholattice.
\end{corollary}
\begin{proof}
By \ref{distributiveComplementUnique} we know that a Boolean lattice is uniquely complemented, so its complement is an involution by \ref{uniqueComplementInvolution}. We just need to check the complementation reverses order.

Let $x\leq y$. Then $\overline{y} \wedge x \leq \overline{y} \wedge y = \bot$, so $\overline{y} \wedge x = \bot$ and thus $\overline{y}$ is disjoint from $x$. Then $\overline{y} \leq \overline{x}$ follows from \ref{BooleanComplementLargestDisjoint}.
\end{proof}
\begin{corollary}
The laws of de Morgan hold in Boolean lattices.
\end{corollary}

\begin{lemma} \label{BooleanLatticesLemma}
Let $L$ be a Boolean lattice and $x,y\in L$. Then
\begin{enumerate}
\item $y = (x\vee y) \wedge (\overline{x}\vee y)$;
\item $y = (x\wedge y) \vee (\overline{x}\wedge y)$;
\item $x \wedge y = x\wedge (\overline{x}\vee y)$;
\item $x \vee y = x\vee (\overline{x}\wedge y)$.
\end{enumerate}
\end{lemma}
\begin{proof}
(1, 2) We calculate
\[ y = y \wedge \top = y \wedge (x\vee \overline{x}) = (x\vee y) \wedge (\overline{x}\vee y). \]
The second statement is dual.

(3, 4) We calculate $x\wedge (\overline{x}\vee y) = (x\wedge \overline{x})\vee (x\wedge y) = \bot \vee (x\wedge y) = x\wedge y$. The fourth statement is dual.
\end{proof}

\begin{lemma} \label{atomsJoinIrriducibleBoolean}
Let $L$ be a Boolean lattice. Then
\begin{enumerate}
\item $\atoms(L) = \joinIr(L)$;
\item $\coatoms(L) = \meetIr(L)$;
\end{enumerate}
\end{lemma}
\begin{proof}
We have the inclusion $\subseteq$ from \ref{atomsJoinIrriducible}. Assume $x$ is an atom. Take $y\in \downset x\setminus\{x\}$, which is non-empty due to the existence of $\bot \neq x$. In fact we need to prove $y = \bot$. By \ref{BooleanLatticesLemma} we have $x = x\vee y = (x\wedge \overline{y})\vee y$. By join-irreducibility this means that $x = x\wedge \overline{y}$ (because $x\neq y$ by assumption). So $x\leq \overline{y}$, meaning $y = x\wedge y \leq \overline{y}\wedge y = \bot$.
\end{proof}

\begin{lemma}
Let $L$ be a lattice, $x\in L$ and $A\subseteq L$. Then
\begin{enumerate}
\item $\upset x\wedge A^u = (x\wedge A)^u$;
\item $\downset x\vee A^l = (x\vee A)^l$.
\end{enumerate}
\end{lemma}
In any lattice we have the inclusions $x\wedge A^u \subseteq (x\wedge A)^u, \;x\wedge A^l \subseteq (x\wedge A)^l, \;x\vee A^u \subseteq (x\vee A)^u$ and $x\vee A^l \subseteq (x\vee A)^l$. See \ref{UpperLowerBoundsMeetJoin}.
\begin{proof}
Because of \ref{UpperLowerBoundsMeetJoin}, we have $x\wedge A^u \subseteq (x\wedge A)^u$, so $\upset x\wedge A^u \subseteq \upset(x\wedge A)^u = (x\wedge A)^u$.

We just need to prove $\upset x\wedge A^u \supseteq (x\wedge A)^u$. Take some $z\in (x\wedge A)^u$, we need to find a $y\in x\wedge A^u$ such that $y \leq z$. We claim that $y = x\wedge (\overline{x}\vee z)$ is such a $y$. We just need to verify that $y \leq z$ and $\overline{x}\vee z \in A^u$.

The first claim is immediate from $y = x\wedge (\overline{x}\vee z) = x\wedge z$, using \ref{BooleanLatticesLemma}.

For the second claim, take an arbitrary $a\in A$. Then $x\wedge a \leq z$ by definition and so $\overline{x}\vee z \geq \overline{x}\vee (x\wedge a) = \overline{x}\vee a \geq a$.
\end{proof}
\begin{corollary}[Infinite distributive laws]
Let $L$ be a Boolean lattice. Then for all $x\in L$ and all $A\subseteq L$:
\begin{enumerate}
\item if $\bigvee A$ exists, then $x\wedge \bigvee A = \bigvee x\wedge A$;
\item if $\bigwedge A$ exists, then $x\vee \bigwedge A = \bigwedge x\vee A$.
\end{enumerate}
\end{corollary}
The inequalities $x\wedge \bigvee A \geq \bigvee x\wedge A$ and $x\vee \bigwedge A \leq \bigwedge x\vee A$ hold in any lattice, see \ref{infiniteDistributiveInequalities}.
\begin{proof}
We calculate
\begin{align*}
x\wedge \sup(A) &= x\wedge (A^u\cap A^{ul}) \\
&\subseteq (x\wedge A^u)\cap (x\wedge A^{ul}) \\
&\subseteq (x\wedge A)^u\cap (x\wedge A^{u})^l = (x\wedge A)^u\cap (\upset(x\wedge A^{u}))^l = (x\wedge A)^u\cap (x\wedge A))^{ul} = \sup(x\wedge A).
\end{align*}
We have used that $x\wedge (A^\cap A^{ul})$ is an image of the function $y\mapsto x\wedge y$ for the first inclusion and \ref{UpperLowerBoundsMeetJoin} for the second. We have also used {upperBoundUpsetlowerBoundDownset} for $(x\wedge A^{u})^l = (\upset(x\wedge A^{u}))^l$.

The left-hand set is a singleton. The right is either a singleton or empty. Thus both sets are singletons with the same content.

Point (2) is dual.
\end{proof}

\begin{proposition}
Let $L$ be a Boolean lattice with complement and $a,b\in L$. Then $[a,b]$ is a Boolean lattice with complementation $x\mapsto \widetilde{x} = (\overline{x} \wedge b)\vee a$. 
\end{proposition}
\begin{proof}
Clearly $[a,b]$ inherits distributivity from $L$. All we need to show is that for all $x\in [a,b]$ the complement of $x$ in $[a,b]$ is $\widetilde{x}$. We calculate
\begin{align*}
x \wedge \widetilde{x} &= x\wedge ((\overline{x} \wedge b)\vee a) = (x\wedge(\overline{x} \wedge b))\vee (x\wedge a) = (( x\wedge \overline{x}) \wedge b)\vee (x\wedge a) = (\bot \wedge b)\vee a = \bot \vee a = a \\
x \vee \widetilde{x} &= x\vee ((\overline{x} \wedge b)\vee a) = ((x\vee \overline{x}) \wedge (x\vee b))\vee a = (\top \wedge b) \vee a = b\vee a = b.
\end{align*}
\end{proof}

\begin{proposition}
An algebra of sets is a Boolean algebra with as top the unit $\Omega$, as bottom the empty set $\emptyset$ and as complement $A\mapsto A^c = \Omega\setminus A$.
\end{proposition}

\subsubsection{Duality and complementation}
TODO: dual expression can be obtained by taking the complement? Dual statement of equality is equality of complements?

TODO: general for ortholattices?

\subsubsection{Boolean rings}
TODO: define ring above!
\begin{definition}
Let $\sSet{R,+,\cdot, 0, 1}$ be a ring. We call $R$ a \udef{Boolean ring} if each $x\in R$ is idempotent: $x^2 = x$.
\end{definition}

\begin{lemma}
Let $\sSet{R,+,\cdot, 0, 1}$ be a Boolean ring. Then
\begin{enumerate}
\item $x = -x$ for all $x\in R$;
\item $R$ is commutative.
\end{enumerate}
\end{lemma}
\begin{proof}
(1) We calculate
\[ x+x = (x+x)^2 = x^2 + x^2 + x^2 + x^2 = x+x+x+x. \]
Subtracting $x+x$ from both sides gives $x+x = 0$.

(2) Let $x,y\in R$. We then have
\[ x+y = (x+y)^2 = x^2 + xy + yx + y^2 = x + xy + yx + y. \]
So $xy = -yx = yx$, using point (1).
\end{proof}

\begin{lemma}
Every subring of a Boolean ring is a Boolean ring.
\end{lemma}

\begin{proposition}
Let $R$ be a set and $0,1\in R$.
\begin{enumerate}
\item If $\sSet{R, \vee, \wedge, 0, 1, c}$ is a Boolean algebra, then $\sSet{R, +, \cdot, 0 , 1}$ is a Boolean ring with operations defined by
\begin{align*}
&+: (x,y) \mapsto (x\wedge y^c)\vee (x^c \wedge y) \\
&\cdot: (x,y) \mapsto x \wedge y.
\end{align*}
\item If $\sSet{R, +, \cdot, 0 , 1}$ is a Boolean ring, then $\sSet{R, \vee, \wedge, 0, 1, c}$ is a Boolean algebra with operations defined by
\begin{align*}
&\vee: (x,y) \mapsto x+y - x\cdot y \\
&\wedge: (x,y) \mapsto x \cdot y \\
&c: x\mapsto x^c = 1 + x
\end{align*}
\end{enumerate}
\end{proposition}
Note we need the existence of $1$ to define the complement.
\begin{proof}
TODO
\end{proof}

TODO: \url{https://en.wikipedia.org/wiki/Boolean_ring} (Maybe later in ring section?)

\subsubsection{Identities in Boolean algebras}
TODO rewrite!!!!

Given any set $U$ we can form the family $\powerset(U)$ for which $U$ is a universe set.

The set theoretic operations of union, intersection, difference, symmetric difference and complementation can be restricted to $\powerset(U)$. In other words $\powerset(U)$ is closed w.r.t. these operations.

\begin{proposition} \label{setBooleanAlgebra}
Given a set $U$, the operations $\cap, \cup, ^c$ form a Boolean algebra with bottom $\emptyset$ and top $U$: $\forall A,B,C\subset U$: $\powerset(U)$ is closed under $\cap, \cup, ^c$ and
\[ \begin{array}{l c c}
\text{\textbf{Commutativity}} & A\cup B= B\cup A & A\cap B = B\cap A \\
\text{\textbf{Identity}} & A\cup\emptyset = A & A\cap U = A \\
\text{\textbf{Distributivity}} & A\cup(B\cap C) = (A\cup B)\cap(A\cup C) & A\cap(B\cup C) = (A\cap B)\cup(A\cap C) \\
\text{\textbf{Complements}} & A\cup A^c = U & A\cap A^c = \emptyset
\end{array} \]
\end{proposition}
The two columns are duals of each other.

TODO: distributivity for arbitrary union and intersection.

\begin{corollary} \label{BooleanConsequences}
Let $A,B\subseteq U$ be sets. Then
\[ \begin{array}{l c c}
\text{\textbf{Idempotency}} & A\cup A = A & A\cap A = A \\
\text{\textbf{Domination}} & A\cup U = U & A\cap \emptyset = \emptyset \\
\text{\textbf{Absorption}} & A\cup(A\cap B) = A & A\cap(A\cup B) = A \\
\text{\textbf{Associativity}} & A\cup(B\cup C) = (A\cup B)\cup C & A\cap(B\cap C) = (A\cap B)\cap C
\end{array} \]
\end{corollary}
\begin{proof}
Using set theory the proof of these statements is simple. It is also possible to prove the equalities using only the properties of Boolean algebras listed in lemma \ref{setBooleanAlgebra} and the properties derived here. We only prove the first column. The proof of the second column can be obtained easily by duality.

(1) $A = A\cup(A\cap A^c) = (A\cup A)\cap (A\cup A^c) = (A\cup A)\cap U = A\cup A$.

(2) $U = U\cup (U\cap A) = (U\cup U)\cap (U\cup A) = U\cap (U\cup A) = (A\cup U)\cap U = (A\cup U)$.

(3) $A\cup(A\cap B) = (A\cap U)\cup(A\cap B) = A\cap(U\cup B) = A\cap U = A$.

(4) TODO \url{https://proofwiki.org/wiki/Operations_of_Boolean_Algebra_are_Associative}
\end{proof}

\section{Residuated lattices}


\section{Completions}
TODO: Dedekind-MacNeille completion.

\section{Formal concept analysis}
\url{file:///C:/Users/user/Downloads/978-3-540-31881-1.pdf}
\url{file:///C:/Users/user/Downloads/978-3-662-49291-8.pdf}

\begin{definition}
A \udef{context} is a triple $\seq{G,M,I}$ where $G$ is a set of \udef{objects}, $M$ is a set of \udef{attributes} and $I\subseteq G\times M$ is a binary relation.

For $g\in G, m\in M$ we interpret $gIm$ as ``the object $g$ has the attribute $m$''.



A \udef{concept} is a pair $\seq{A,B}$ where $A\subset G$ is a set of objects and $B\subset M$ is a set of attributes such that
\begin{itemize}
\item $A = \setbuilder{g\in G}{\forall m\in B: gIm}$;
\item $B = \setbuilder{m\in M}{\forall g\in A: gIm}$
\end{itemize}
\end{definition}
The letters $G$ and $M$ come from the German: Gegenstände and Merkmale. The $I$ is for ``incidence relation'' (I think).

\chapter{The poset of subsets}
\begin{definition}
Let $U$ be a set. Then $\sSet{\powerset(U), \subseteq}$ is a partially ordered set. We call $U$ the \udef{universe} of this poset.
\end{definition}

Any family of sets $\mathcal{F}$ may be seen as a subset of the poset with universe $\bigcup \mathcal{F}$.

\begin{proposition} \label{posetPowerset}
Let $\mathcal{F}$ be a family of sets. Then $\sSet{\mathcal{F}, \subseteq}$ is a poset.

Conversely, every poset $(P,\preceq)$ is isomorphic to $\sSet{\mathcal{F}, \subseteq}$ for some family of sets $\mathcal{F}$ with universe $P$.
\end{proposition}
\begin{proof}
This is just a reformulation of \ref{orderedSetPowerset}.
\end{proof}
\begin{corollary} \label{MooreFamily}
Let $\mathcal{F}$ be a family of subsets of $U$. If
\begin{itemize}
\item $\mathcal{F}$ is closed under arbitrary intersections; and
\item $U \in \mathcal{F}$;
\end{itemize}
then $\sSet{\mathcal{F},\subseteq}$ is a complete lattice.

Conversely, every complete lattice is isomorphic to such a lattice.
\end{corollary}
Note that in general the join in such a lattice is \emph{not} given by the union, $\bigvee \neq \bigcup$!
\begin{proof}
Such a family of sets is a complete lattice by \ref{completeLatticeBasic}.

For the converse, the isomorphism is given by $x\mapsto \downset x$, as in \ref{orderedSetPowerset}. We just need to show that the meet translates to the intersection, i.e. $\downset \bigwedge Q = \bigcap_{x\in Q}\downset x$. We calculate, using \ref{minMaxUpsetDownset} and \ref{boundsFromUpDownSets}:
\[ \downset \bigwedge Q = \downset \max(Q^l) = \downset Q^l = \downset \bigcap_{x\in Q} \downset x \subseteq \bigcap_{x\in Q} \downset x. \]
For the other inclusion we just note that a lattice order is in particular a  preorder and so we can use \ref{QsubseteqDownQ}.
\end{proof}

\begin{definition}
A family of sets $\mathcal{F}$ satisfying the hypothesis of \ref{MooreFamily}, i.e.
\begin{itemize}
\item $\mathcal{F}$ is closed under arbitrary intersections; and
\item $U \in \mathcal{F}$;
\end{itemize}
is called a \udef{Moore family} or \udef{topped intersection structure}.
\end{definition}

TODO: join is given by union if directed?? (+CPO)??

\section{The complete Boolean lattice of subsets}
\begin{lemma}
Let $U$ be a universe set. Consider the poset $\sSet{\powerset(U), \subseteq}$ and let $\mathcal{F}\subseteq \powerset(U)$. Then
\begin{enumerate}
\item $\sup(\mathcal{F}) = \bigcup \mathcal{F}$;
\item $\inf(\mathcal{F}) = \bigcap \mathcal{F}$.
\end{enumerate}
(Assuming relativised intersection TODO!)
\end{lemma}
\begin{corollary}
Let $U$ be a universe set. Then $\sSet{\powerset(U), \subseteq}$ is a bounded, complete, distributive lattice with top $U$ and bottom $\emptyset$.
\end{corollary}
\begin{proof}
TODO ref distributivity.
\end{proof}

\subsection{Complementation}
\begin{definition}
Let $U$ be a universe and $A\subseteq U$. The \udef{complement} of $A$ w.r.t. $U$ is
\[ A^c \defeq U\setminus A. \]
\end{definition}

\begin{lemma}
The complement $^c$ is a lattice-theoretical complement.
\end{lemma}
\begin{proof}
For all $A\subseteq U$ we have $A\cup A^c = U$ and $A\cap A^c = \emptyset$ from \ref{differenceProperties}.
\end{proof}
\begin{corollary}
Let $U$ be a universe set. Then $\sSet{\powerset(U), \subseteq}$ is a Boolean lattice.
\end{corollary}

In particular de Morgan's laws can be formulated in this context as:
\begin{proposition}
Let $U,A,B$ be sets, then
\begin{align*}
(A\cup B)^c &= (A^c)\cap (B^c); \\
(A\cap B)^c &= (A^c)\cup (B^c).
\end{align*}
Where complementation is with respect to $U$.

This can be extended to arbitrary families of sets:
\begin{align*}
\left(\bigcup \mathcal{E}\right)^c &= \bigcap\setbuilder{A^c}{A\in\mathcal{E}} \\
\left(\bigcap \mathcal{E}\right)^c &= \bigcup\setbuilder{A^c}{A\in\mathcal{E}}
\end{align*}
where $\mathcal{E}$ is a family of sets.
\end{proposition}

\subsection{Expressing set theoretic operations with $\cup,\cap, ^c$}
\begin{proposition}
Let $A,B\subseteq U$ be sets. Then
\begin{enumerate}
\item $A\setminus B = A \cap B^c$;
\item $A\symdiff B = (A\cup B)\cap (A^c\cup B^c)$.
\end{enumerate}
\end{proposition}
\begin{corollary}
Let $A,B \subseteq U$ be sets. Then
\begin{enumerate}
\item $A\setminus B = B^c\setminus A^c$;
\item $A \symdiff B = A^c \symdiff B^c$;
\item $A \symdiff A^c = U$.
\end{enumerate}
\end{corollary}

\section{Grills}
\begin{definition}
Let $X$ be a set. Let $\mathcal{A} \subseteq \powerset(X)$. The \udef{grill} of $\mathcal{A}$ is defined as the commonality $\mathcal{A}^{\overrightarrow{\mesh}}$.
\end{definition}

\begin{lemma}
Let $X$ be a set, $\powerset(X)$ be ordered by inclusion, $A, B\in \powerset(X)$ and $\mathcal{A} \subseteq \powerset(X)$. Then
\begin{enumerate}
\item $\upset A\mesh$ = $A\mesh$;
\item $\upset \mathcal{A}^{\overrightarrow{\mesh}} = \mathcal{A}^{\overrightarrow{\mesh}}$;
\item $(\upset \mathcal{A})^{\overrightarrow{\mesh}} = \mathcal{A}^{\overrightarrow{\mesh}}$;
\item $((\mathcal{A}^{\overrightarrow{\mesh}})^{\overrightarrow{\mesh}})^{\overrightarrow{\mesh}} = \mathcal{A}^{\overrightarrow{\mesh}}$;
\item $(\mathcal{A}^{\overrightarrow{\mesh}})^{\overrightarrow{\mesh}} = \upset \mathcal{A}$.
\end{enumerate}
\end{lemma}
\begin{proof}
(1) If $B$ meshes with $A$, then any superset of $B$ also meshes with $A$. The opposite inclusion is given by \ref{upwardDownwardClosure}.

(2) We calculate using \ref{upDownsetUnionIntersection}
\[ \upset \mathcal{A}^{\overrightarrow{\mesh}} = \upset \bigcap_{A\in\mathcal{A}} A\mesh = \upset \bigcap_{A\in\mathcal{A}} \upset A\mesh = \bigcap_{A\in\mathcal{A}} \upset A\mesh = \bigcap_{A\in\mathcal{A}} A\mesh = \mathcal{A}^{\overrightarrow{\mesh}} \]

(3) TODO

(4) TODO (general for commonalities using $\mesh^transp = \mesh$ (?)).

(5) By \ref{GreensTheoremCorollary} this is a consequence of (2), (3) and (4).
\end{proof}

\section{Indicator functions}
The family $\powerset(U)$ can be bijectively mapped to the family of functions $(U\to \{0,1\})$, by mapping each set $A$ to its indicator function $\chi_A$.

\begin{definition}
Let $U$ be a set and $A\subseteq U$. The \udef{indicator function} or \udef{characteristic function} of $A$ as an element of $\powerset(U)$ is defined as
\[ \chi_A: U\to \{0,1\}: x\mapsto \begin{cases}
1 & x\in A \\ 0 & x\notin A.
\end{cases} \]
\end{definition}
\begin{lemma}
Let $A,B$ be elements of $\powerset(U)$. Then
\begin{enumerate}
\item $\chi_{A\cap B} = \min\{\chi_A,\chi_B\} = \chi_A\cdot \chi_B$;
\item $\chi_{A\cup B} = \max\{\chi_A,\chi_B\} = \chi_A + \chi_B - \chi_A\cdot \chi_B$;
\item $\chi_{A^c} = \underline{1}-\chi_A$; and thus $\chi_A + \chi_{A^c} = \underline{1}$;
\item $\chi_{A\Delta B} \begin{aligned}[t] &= \chi_A + \chi_B - \underline{2}\cdot\chi_A\cdot \chi_B \\
&= |\chi_A - \chi_B| \\
&= \chi_A + \chi_B \mod 2 \defeq \begin{cases}
1 & (\chi_A + \chi_B = 1) \\
0 & \text{(else)}
\end{cases}
\end{aligned}$.
\end{enumerate}
Where all operations are defined point-wise.
\end{lemma}

\begin{proposition}
The indicator functions define a bijection between $\powerset(U)$ and $(U\to \{0,1\})$.
\end{proposition}

\section{Closure under set operations}
We say a family of sets $\mathcal{F}$ is closed under an operation if the result of this operation acting on sets in $\mathcal{F}$ is again in $\mathcal{F}$.

\begin{definition}
A family of sets $\mathcal{F}\subseteq\powerset(U)$ is called
\begin{itemize}
\item \udef{closed under complementation} if $A^c\in\mathcal{F}$ for all $A\in\mathcal{F}$;
\item \udef{closed under relative complements} if $A\setminus B\in\mathcal{F}$ for all $A \supset B\in\mathcal{F}$;
\item \udef{closed under set difference} if $A\setminus B\in\mathcal{F}$ for all $A, B\in\mathcal{F}$;
\end{itemize}
and
\begin{itemize}
\item \udef{closed under finite unions} if $A\cup B \in\mathcal{F}$ for all $A,B\in\mathcal{F}$;
\item \udef{closed under finite intersections} if $A\cap B \in\mathcal{F}$ for all $A,B\in\mathcal{F}$;
\item \udef{closed under disjoint unions} if $\biguplus_{i\in I}A_i \in\mathcal{F}$ for any indexed family of disjoint sets $\{A_i\}_{i\in I}$;
\item \udef{closed under countable monotone unions} if $\bigcup_{i=1}^\infty A_i \in\mathcal{F}$ for any indexed family of sets $\{A_i\}_{i\in \N}$ such that $i\leq j \implies A_i \subseteq A_j$;
\item \udef{closed under countable monotone intersections} if $\bigcap_{i=1}^\infty A_i \in\mathcal{F}$ for any indexed family of sets $\{A_i\}_{i\in \N}$ such that $i\leq j \implies A_i \supseteq A_j$.
\end{itemize}
Our definition of closure under relative complements is \emph{not} the same as closure under set difference! Our definition is non-standard as they are usually taken to be the same thing.
\end{definition}

\subsection{Complementation, relative complementation and set difference}
In general the notions of closure under complementation, relative complementation and set difference are distinct, the only implication being from set difference to relative complementation.

\begin{lemma} \label{complementTypesUnionClosure}
Let $\mathcal{F}$ be a family of sets that is closed under finite (disjoint) unions. Then
\[ \begin{tikzcd}[row sep=0]
\text{$\mathcal{F}$ is closed under set differences} \arrow[dr, Rightarrow] & \\
\text{$\mathcal{F}$ is closed under complementation} \arrow[r,Rightarrow] & \text{$\mathcal{F}$ is closed under relative complements.}
\end{tikzcd} \]
\end{lemma}
\begin{proof}
Assume $A\supset B$, then $A\setminus B = A^c \uplus B$. This is a disjoint union.
\end{proof}

\begin{lemma}
Let $\mathcal{F}$ be a family of sets that is closed under finite intersections. Then
\[ \text{closure under complements} \quad\implies\quad \text{closure under set difference} \quad\iff\quad \text{closure under relative complements.} \]
All three are equivalent if $\mathcal{F}$ contains the universe set.
\end{lemma}
\begin{proof}
$A\setminus B = A\cap B^c$ and $A\setminus B = A\setminus (B\cap A)$.
\end{proof}

\begin{lemma} \label{closureSetDifference}
Let $\mathcal{F}$ be a family of sets. Then
\[ \text{closure under set differences} \quad\implies\quad \text{closure under finite intersections.} \]
\end{lemma}
\begin{proof}
$A\cap B = A\setminus (A\setminus B)$.
\end{proof}

Any non-empty family of sets that is closed under set differences is also an order-theoretic ring.

\subsection{Types of closure for unions and intersections}
\begin{lemma} \label{unionsIntersectionClosureImplications}
Let $\mathcal{F}$ be a family of sets. Then we have the following implications for closure under unions:
\[ \begin{tikzcd}[row sep=tiny]
&& \text{countable disjoint $\uplus$} \\
\text{arbitrary $\cup$} \arrow[r, Rightarrow] & \text{countable $\cup$} \arrow[ur, Rightarrow] \arrow[r, Rightarrow] \arrow[dr, Rightarrow] & \text{countable monotone $\cup$} \\
&& \text{finite $\cup$}
\end{tikzcd} \]
and for closure under intersections:
\[ \begin{tikzcd}[row sep=0em]
&& \text{countable monotone $\cap$} \\
\text{arbitrary $\cap$} \arrow[r, Rightarrow] & \text{countable $\cap$} \arrow[ur, Rightarrow] \arrow[dr, Rightarrow] &  \\
&& \text{finite $\cap$}
\end{tikzcd} \]
\end{lemma}


The implications in \ref{unionsIntersectionClosureImplications} for unions can be simplified, if $\mathcal{F}$ is closed under relative complementation:
\begin{lemma} \label{typesOfUnionsRelativeComplementation}
Let $\mathcal{F}$ be a family of sets that is closed under relative complementation. Then we have the following implications for closure under unions:
\[ \begin{tikzcd}[row sep=0em]
\text{arbitrary $\cup$} \arrow[r, Rightarrow] & \text{countable $\cup$} \arrow[r, Rightarrow] \arrow[dr, Rightarrow] & \text{countable disjoint $\uplus$} \arrow[r, Rightarrow] & \text{countable monotone $\cup$} \\
&& \text{finite $\cup$} &
\end{tikzcd} \]
\end{lemma}
\begin{proof}
We need to prove that closure under countable disjoint unions implies closure under countable monotone unions.

Assume $\mathcal{F}$ closed under countable disjoint unions. Let $\{A_i\}_{i\in \N}$ be a monotonically increasing family of sets. Then we can recursively define a family $\{D_i\}_{i\in \N}$ by $D_0=\emptyset$ and
\[ D_{i+1} = A_{i+1}\setminus D_i. \]
This is allowed because $A_{i+1}\supset A_i \supset D_i$. By induction we see that $\{D_i\}_{i\in \N}$ is a disjoint family and has the same union as $\{A_i\}_{i\in \N}$.
\end{proof}

These implications can be further simplified, if $\mathcal{F}$ is closed under set differences:
\begin{lemma}
Let $\mathcal{F}$ be a family of sets that is closed under set differences. Then we have the following implications for closure under unions:
\[ \begin{tikzcd}[row sep=0em]
\text{arbitrary $\cup$} \arrow[r, Rightarrow] & \text{countable $\cup$} \arrow[r, Leftrightarrow] & \text{countable disjoint $\uplus$} \arrow[r, Rightarrow] \arrow[dr, Rightarrow] & \text{countable monotone $\cup$} \\
&&& \text{finite $\cup$}
\end{tikzcd} \]
\end{lemma}
\begin{proof}
We just need to prove that closure under countable disjoint unions implies closure under countable unions.

This can be done with the same construction of $\{D_i\}_{i\in \N}$ as before because now the assignment $D_{i+1} = A_{i+1}\setminus D_i$ works for arbitrary families $\{A_i\}_{i\in \N}$, not just monotone ones.
\end{proof}

\section{Filters}

\begin{proposition}
Let $X$ be a set and $F\in\filters(\powerset(X))$ a proper filter. Then $F \subseteq F^{\overrightarrow{\mesh}}$.
\end{proposition}
\begin{proof}
Every set in $F$ meshes with every other set in $F$. If not, i.e. $f,g\in F$ such that $f\cap g = \emptyset$, then $\emptyset\in F$ and $F$ would not be proper.
\end{proof}

\subsection{Ultrafilters}
\begin{definition}
Let $X$ be a set and $F\in\filters(\powerset(X))$ a filter. We call $F$ an \udef{ultrafilter} if $F = F^{\overrightarrow{\mesh}}$.
\end{definition}

TODO characterisations. Ultrafilters are maximally fine (-> ~ Zorns lemma).

\begin{proposition}
An ultrafilter is either principal or free.
\end{proposition}

\begin{proposition}
An ultrafilter is sequential \textup{if and only if} it is principal.
\end{proposition}

\section{Monotone classes}
\begin{definition}
A family of sets $\mathcal{F}$ is called a \udef{monotone class} if it is closed under both countable monotone unions and countable monotone intersections.
\end{definition}
\begin{lemma}
Any arbitrary intersection of monotone classes is a monotone class.
\end{lemma}

\subsection{Dynkin systems}
\begin{definition}
A \udef{Dynkin system} of sets (also known as a \udef{$\lambda$-system} or \udef{d-system}) is a pair of a set $\Omega$ and a collection of sets $D\subset\powerset(\Omega)$ such that
\begin{itemize}
\item $\Omega\in D$;
\item if $A\in D$, then $A^c\in D$;
\item if $(A_i)_{i\in\N}$ is a countable sequence of pairwise disjoint sets in $D$, then $\biguplus_{i=1}^\infty A_i\in D$.
\end{itemize}
\end{definition}

\begin{lemma}
A pair of a set $\Omega$ and a family of subsets $D$ is a Dynkin system \textup{if and only if}
\begin{itemize}
\item $\Omega\in D$;
\item $D$ is closed under relative complements: if $A,B\in D$ and $A\supset B$, then $A\setminus B\in D$;
\item $D$ is closed under countable monotone unions.
\end{itemize}
\end{lemma}
\begin{proof}
Call the original set of axioms Ax1 and this set of axioms Ax2.

$\boxed{\text{Ax1}\implies\text{Ax2}}$ Point (2) follows from \ref{complementTypesUnionClosure} and point (3) follows from \ref{typesOfUnionsRelativeComplementation}.

$\boxed{\text{Ax2}\implies\text{Ax1}}$ Point (2) follows immediately. For point (3): let $A,B$ be disjoint sets. Then $A^c \supset B$ and so $A\cup B = (A^c\setminus B)^c \in D$, meaning $D$ is closed under finite unions of disjoint sets. Now let $(A_i)_{i\in\N}$ be a countable sequence of pairwise disjoint sets. Then
\[ \biguplus_{i=1}^\infty A_i = \bigcup_{i=1}^\infty \left(\biguplus_{j=1}^i A_j\right) \]
which is a countable monotone union.
\end{proof}

\begin{lemma}
A Dynkin system is a monotone class.
\end{lemma}
\begin{proof}
If $\bigcap_{i=1}^\infty A_i$ is a countable monotone intersection, then $\left(\bigcap_{i=1}^\infty A_i\right)^c = \bigcup_{i=1}^\infty A_i^c$ is a countable monotone union.
\end{proof}
\begin{lemma}
Any arbitrary intersection of Dynkin systems is a Dynkin system.
\end{lemma}

\section{$\pi$-systems}
TODO: directed set!
\begin{definition}
A \udef{$\pi$-system} is a collection of sets $P$ such that
\begin{itemize}
\item $P$ is not empty;
\item if $A,B\in P$, then $A\cap B\in P$.
\end{itemize}
\end{definition}

\begin{lemma}
Let $\mathcal{F}$ be a collection of sets. If $\mathcal{F}$ is closed under set differences, then it is a $\pi$-system.
\end{lemma}
\begin{proof}
By \ref{closureSetDifference}.
\end{proof}

For $\pi$-systems the intersection implications in \ref{unionsIntersectionClosureImplications} reduce to:
\begin{lemma} \label{piSystemunionsIntersectionClosureImplications}
Let $\mathcal{F}$ be a $\pi$-system. Then we have the following implications for closure under intersections:
\[ \text{arbitrary $\cap$} \quad\implies\quad \text{countable $\cap$} \quad\iff\quad \text{countable monotone $\cap$} \]
\end{lemma}
\begin{proof}
We need to prove that closure under countable monotone intersections implies closure under countable intersections.

Assume $\mathcal{F}$ is a $\pi$-system closed under countable monotone intersections. Let $\{A_i\}_{i\in \N}$ be an indexed family of sets. Then we can define the family $\{B_i\}_{i\in \N}$ recursively by $B_1 = A_1$ and
\[ B_{i+1} = B_i \cap A_{i+1}. \]
By induction we see that $\{B_i\}_{i\in \N}$ is monotone and has the same intersection as $\{A_i\}_{i\in \N}$.
\end{proof}

\subsection{Intersections structures}
\begin{definition}
Let $\mathcal{F} \subseteq \powerset(U)$ be a family of sets. We call $\mathcal{F}$ an \udef{intersection structure} if it is closed under arbitrary intersections.

If an intersection structure contains the universe set, it is called a \udef{topped intersection structure} or \udef{closure system}.
\end{definition}

\subsection{Ring}
\subsubsection{Order-theoretic ring}
TODO: sublattice!
\begin{definition}
An \udef{order-theoretic ring} of sets is a non-empty collection of sets $\mathcal{R}$ such that
\begin{itemize}
\item if $A,B\in \mathcal{R}$, then $A\cup B\in \mathcal{R}$;
\item if $A,B\in \mathcal{R}$, then $A\cap B\in \mathcal{R}$.
\end{itemize}
\end{definition}

\subsubsection{Semi-rings}
\begin{definition}
A \udef{semi-ring} is a non-empty collection of sets $\mathcal{S}$ such that
\begin{itemize}
\item if $A,B\in \mathcal{S}$, then $A\cap B\in \mathcal{S}$;
\item if $A,B\in \mathcal{S}$, then $A\setminus B$ is a finite disjoint union of sets in $\mathcal{S}$.
\end{itemize}
\end{definition}

\begin{lemma}
Let $\mathcal{S}$ be a semi-ring. Then $\emptyset \in \mathcal{S}$.
\end{lemma}
\begin{proof}
Because $\mathcal{S}$ is non-empty, we can take $A\in \mathcal{S}$. Then $A\setminus A = \emptyset$ is a finite disjoint union of sets in $\mathcal{S}$, so $\emptyset \in \mathcal{S}$.
\end{proof}

\subsubsection{Measure-theoretic rings}
\begin{definition}
A \udef{(measure-theoretic) ring} of sets is a non-empty collection of sets $\mathcal{R}$ such that
\begin{itemize}
\item if $A,B\in \mathcal{R}$, then $A\cup B\in \mathcal{R}$;
\item if $A,B\in \mathcal{R}$, then $A\setminus B\in \mathcal{R}$.
\end{itemize}
\end{definition}
By \ref{closureSetDifference} a measure-theoretic ring is in particular a $\pi$-system and an order-theoretic ring.

\begin{lemma}
A non-empty collection of sets $\mathcal{R}$ is a measure-theoretic ring \textup{if and only if}
\begin{itemize}
\item it is closed under finite intersections;
\item it is closed under symmetric differences.
\end{itemize}
\end{lemma}
\begin{proof}
By the identities $A\cup B = (A\symdiff B)\symdiff (A\cap B)$ and $A\setminus B = A\symdiff (A\cap B)$.
\end{proof}

\begin{lemma}
Let $\mathcal{S}$ be a semi-ring. Then the smallest ring $\mathcal{R}$ containing $\mathcal{S}$ is
\[ \mathfrak{R}\{\mathcal{S}\} = \setbuilder{E_1\uplus\ldots\uplus E_n}{\text{$E_i\in \mathcal{S}$ are pairwise disjoint}}. \]
We call this the ring \udef{generated} by $\mathcal{S}$.
\end{lemma}
\begin{proof}
Every ring containing $\mathcal{S}$ must contain $\mathfrak{R}\{\mathcal{S}\}$, so if it is a ring it is automatically the smallest. We just need to show it is a ring. Let $E,F$ be arbitrary elements of $\mathfrak{R}\{\mathcal{S}\}$. We need to show that both $E\setminus F$ and $E\cup F$ are in $\mathfrak{R}\{\mathcal{S}\}$.

Now $E\cup F = (E\setminus F) \uplus F$ can be written as a disjoint union, so we just need to write $E\setminus F$ as a pairwise disjoint union of elements of $\mathcal{S}$. To that end write $E = \biguplus_{i=0}^nE_i$ and $\biguplus_{j= 0}^mF_j$. Then
\[ E\setminus F = \biguplus_{i=0}^n \left[ \Big(\big((E_j\setminus F_1)\setminus F_2\big)\setminus \ldots\Big)\setminus F_m \right]. \]
By industion and using the semi-ring property we can see that this is expressible as a finite pairwise disjoint union of elements of $\mathcal{S}$, and thus is an element of $\mathfrak{R}\{\mathcal{S}\}$.
\end{proof}

\subsubsection{$\sigma$- and $\delta$-rings}
\begin{definition}
A \udef{$\sigma$-ring} of sets is a collection of sets $\mathcal{R}$ such that
\begin{itemize}
\item $\mathcal{R}$ is a measure-theoretic ring;
\item $\mathcal{R}$ is closed under countable unions.
\end{itemize}
A \udef{$\delta$-ring} of sets is a collection of sets $\mathcal{R}$ such that
\begin{itemize}
\item $\mathcal{R}$ is a measure-theoretic ring;
\item $\mathcal{R}$ is closed under countable intersections.
\end{itemize}
\end{definition}

\subsection{Algebras of sets}
\begin{definition}
An \udef{algebra of sets} on a set $\Omega$ (also known as a \udef{field of sets}) is a ring that contains $\Omega$.
\end{definition}

In principle we can thus define
\begin{itemize}
\item order-theoretic algebra;
\item semi-algebra;
\item measure-theoretic algebra;
\item $\sigma$-algebra;
\item $\delta$-algebra.
\end{itemize}

Some of these notions coincide.

\begin{lemma}
Let $\mathcal{A}$ be a family of sets. Then
\begin{enumerate}
\item $\mathcal{A}$ is an order-theoretic algebra \textup{if and only if} it is a measure-theoretic algebra;
\item $\mathcal{A}$ is a $\sigma$-algebra \textup{if and only if} it is a $\delta$-algebra.
\end{enumerate}
\end{lemma}

So we define define \udef{semi-algebra}, \udef{algebra} and \udef{$\sigma$-algebra}.

\begin{lemma} \label{setAlgebraCriteria}
A family of sets $\mathcal{A}$ is a semi-algebra on $\Omega$ \textup{if and only if}
\begin{itemize}
\item $\Omega\in\mathcal{A}$;
\item if $A,B\in \mathcal{A}$, then $A\cap B\in \mathcal{A}$;
\item if $A\in \mathcal{A}$, then $A^c$ is a finite disjoint union of sets in $\mathcal{A}$.
\end{itemize}
A family of sets $\mathcal{A}$ is an algebra on $\Omega$ \textup{if and only if}
\begin{itemize}
\item $\Omega\in\mathcal{A}$;
\item if $A\in \mathcal{A}$, then $A^c\in \mathcal{A}$;
\item if $A,B\in \mathcal{A}$, then $A\cup B\in \mathcal{A}$.
\end{itemize}
A family of sets $\mathcal{A}$ is a $\sigma$-algebra on $\Omega$ \textup{if and only if}
\begin{itemize}
\item $\Omega\in\mathcal{A}$;
\item if $A\in \mathcal{A}$, then $A^c\in \mathcal{A}$;
\item if $(A_i)_{i\in\N}$ is a countable sequence of sets in $\mathcal{A}$, then $\bigcup_{i=1}^\infty A_i\in \mathcal{A}$.
\end{itemize}
\end{lemma}

\begin{example}
For any set $\Omega$, the power set $\powerset(\Omega)$ is a $\sigma$-algebra on $\Omega$.
\end{example}

\begin{lemma} \label{algebraMonotoneClass}
An algebra $\mathcal{A}$ is a $\sigma$-algebra \textup{if and only if} $\mathcal{A}$ is a monotone class.
\end{lemma}
\begin{lemma} \label{DynkinPiSystem}
A Dynkin system is a $\sigma$-algebra \textup{if and only if} it is a $\pi$-system.
\end{lemma}


\begin{lemma}
\begin{enumerate}
\item The countable monotone union of a sequence of $\sigma$-algebras is an algebra, but not necessarily a $\sigma$-algebra.
\item Any arbitrary intersection of $\sigma$-algebras is a $\sigma$-algebra.
\end{enumerate}
\end{lemma}

\begin{lemma}
Let $\mathcal{A}$ be a $\sigma$-algebra on $\Omega$ and $B\subset \Omega$. Then $B\cap \mathcal{A} = \setbuilder{B\cap A}{A\in\mathcal{A}}$ is a $\sigma$-algebra on $B$.
\end{lemma}

\section{Generators}
\begin{definition}
Let $\mathcal{F}$ be a family of subsets of $\Omega$. Then we define
\begin{itemize}
\item the $\sigma$-algebra generated by $\mathcal{F}$, $\sigma\{\mathcal{F}\}$;
\item the monotone class generated by $\mathcal{F}$, $\mathfrak{M}\{\mathcal{F}\}$; and
\item the Dynkin system generated by $\mathcal{F}$, $\mathfrak{D}\{\mathcal{F}\}$;
\end{itemize}
the intersection of all such families $\subseteq \powerset(\Omega)$ that contain $\mathcal{F}$.

In each case we call $\mathcal{F}$ the \udef{generator} of the system.
\end{definition}
These intersections are again $\sigma$-algebras, monotone classes and Dynkin systems, respectively. So these generated families are the smallest such families containing $\mathcal{F}$.

\begin{example}
\begin{itemize}
\item If $\mathcal{A}$ is a $\sigma$-algebra, then $\sigma\{\mathcal{A}\} = \mathcal{A}$.
\item If $\mathcal{A} = \{A\}$, a single set, then $\sigma\{\mathcal{A}\} = \{\emptyset, A,A^c,\Omega\}$.
\end{itemize}
\end{example}

\begin{lemma} \label{unitGeneratedSets}
A universe set for $\mathcal{F}$ is also a universe set for $\sigma\{\mathcal{F}\}$, $\mathfrak{M}\{\mathcal{F}\}$ and $\mathfrak{D}\{\mathcal{F}\}$.
\end{lemma}

\begin{proposition}[Monotone class theorem] \label{monotoneClassTheorem}
Let $\mathcal{A}$ be an algebra. Then
\[ \mathfrak{M}\{\mathcal{A}\} = \sigma\{\mathcal{A}\}. \]
\end{proposition}
\begin{proof}
Every $\sigma$-algebra is a monotone class, so $\mathfrak{M}\{\mathcal{A}\} \subset \sigma\{\mathcal{A}\}$.

For the other inclusion it is enough to show that $\mathfrak{M}\{\mathcal{A}\}$ is an algebra: using \ref{algebraMonotoneClass} we have
\[ \text{$\mathfrak{M}\{\mathcal{A}\}$ is an algebra} \implies \text{$\mathfrak{M}\{\mathcal{A}\}$ is a $\sigma$-algebra} \implies \sigma\{\mathcal{A}\}\subset\mathfrak{M}\{\mathcal{A}\}. \]
In particular, due to \ref{setAlgebraCriteria}, we verify $\Omega\in \mathfrak{M}\{\mathcal{A}\}$, $B^c \in \mathfrak{M}\{\mathcal{A}\}$ and $B\cup C \in \mathfrak{M}\{\mathcal{A}\}$.

$\boxed{\Omega\in \mathfrak{M}\{\mathcal{A}\}}$ By \ref{unitGeneratedSets}.

$\boxed{B^c \in \mathfrak{M}\{\mathcal{A}\}}$ Define
\[ \mathcal{E}_1 = \setbuilder{B\in\mathfrak{M}\{\mathcal{A}\}}{B^c\in\mathfrak{M}\{\mathcal{A}\}} \]
which is a monotone class by De Morgan's laws:
\[ (B_i)_{i=1}^\infty\subset \mathcal{E}_1 \implies (B_i^c)_{i=1}^\infty\subset \mathfrak{M}\{\mathcal{A}\} \implies \bigcup_{i=1}^\infty B_i^c = \left(\bigcap_{i=1}^\infty B_i\right)^c \in\mathfrak{M}\{\mathcal{A}\} \implies \bigcap_{i=1}^\infty B_i \in \mathcal{E}_1. \]
Also $\mathcal{E}_1\subset \mathfrak{M}\{\mathcal{A}\}$, so $\mathcal{E}_1 = \mathfrak{M}\{\mathcal{A}\}$ by minimality, so
\[ B\in\mathfrak{M}\{\mathcal{A}\} \iff B\in\mathcal{E}_1 \implies B^c\in\mathfrak{M}\{\mathcal{A}\}. \]

$\boxed{B\cup C \in \mathfrak{M}\{\mathcal{A}\}}$ Define
\begin{align*}
\mathcal{E}_2 &= \setbuilder{B\in\mathfrak{M}\{\mathcal{A}\}}{\forall C\in\mathcal{A}: B\cup C \in\mathfrak{M}\{\mathcal{A}\}} \\
\mathcal{E}_3 &= \setbuilder{B\in\mathfrak{M}\{\mathcal{A}\}}{\forall C\in\mathfrak{M}\{\mathcal{A}\}: B\cup C \in\mathfrak{M}\{\mathcal{A}\}}.
\end{align*}
Now $\mathcal{E}_2$ and $\mathcal{E}_3$ are monotone classes: by \ref{setAssociativityCommutativity} and \ref{setDistributivity}, for $k=1,2$
\[ (B_i)_{i=1}^\infty\subset \mathcal{E}_k \implies \forall C: (B_i\cup C)_{i=1}^\infty\subset \mathfrak{M}\{\mathcal{A}\} \implies \forall C: \bigcup_{i=1}^\infty B_i\cup C = \left(\bigcup_{i=1}^\infty B_i\right)\cup C \in\mathfrak{M}\{\mathcal{A}\} \implies \bigcup_{i=1}^\infty B_i \in \mathcal{E}_k. \]
Now clearly $\mathcal{A}\subseteq\mathcal{E}_2$, so by minimality $\mathcal{E}_2 = \mathfrak{M}\{\mathcal{A}\}$. Moreover,
\[ D\in\mathcal{A}\implies \forall C\in \mathcal{E}_2: C\cup D\in \mathfrak{M}\{\mathcal{A}\} \implies \forall C\in \mathfrak{M}\{\mathcal{A}\}: D\cup C\in \mathfrak{M}\{\mathcal{A}\} \implies D\in \mathcal{E}_3. \]
So $\mathcal{A}\subseteq\mathcal{E}_3$ and by minimality $\mathcal{E}_3 = \mathfrak{M}\{\mathcal{A}\}$, which means that
\[ B\in\mathfrak{M}\{\mathcal{A}\} \iff B\in\mathcal{E}_3 \implies \forall C\in\mathfrak{M}\{\mathcal{A}\}: B\cup C\in\mathfrak{M}\{\mathcal{A}\}. \]
\end{proof}
\begin{corollary}
Let $\mathcal{A}$ be an algebra and $M$ a monotone class with $\mathcal{A}\subseteq M$, then $\sigma\{\mathcal{A}\}\subseteq M$.
\end{corollary}

\begin{proposition} \label{generatedDynkinSigma}
If $\mathcal{F}$ is a $\pi$-system on $\Omega$, then
\[ \mathfrak{D}\{\mathcal{F}\} = \sigma\{\mathcal{F}\}. \]
\end{proposition}
\begin{proof}
Every $\sigma$-algebra is a Dynkin system, so $\mathfrak{D}\{\mathcal{F}\} \subset \sigma\{\mathcal{F}\}$.

For the other inclusion it is enough to show that $\mathfrak{D}\{\mathcal{F}\}$ is a $\pi$-system: using \ref{DynkinPiSystem}, we have
\[ \text{$\mathfrak{D}\{\mathcal{F}\}$ is a $\pi$-system} \implies \text{$\mathfrak{D}\{\mathcal{F}\}$ is a $\sigma$-algebra} \implies \sigma\{\mathcal{F}\}\subset\mathfrak{D}\{\mathcal{F}\}. \]

To this end we define
\[ \mathcal{D}_B = \setbuilder{A\subset\Omega}{A\cap B\in\mathfrak{D}\{\mathcal{F}\}} \qquad \text{for some $B\in\mathfrak{D}\{\mathcal{F}\}$,} \]
which we claim is a Dynkin system.
\begin{itemize}[leftmargin=2.5cm]
\item[$\boxed{\Omega\in\mathcal{D}_B}$] Because $\Omega\cap B = B$.
\item[$\boxed{A^c\in\mathcal{D}_B}$] Let $A\in\mathcal{D}_B$. Then
\[ A^c\cap B = (\Omega\setminus A)\cap B = (\Omega\cap B)\setminus(A\cap B) \in \mathfrak{D}\{\mathcal{F}\}, \]
so $A^c\in\mathcal{D}_B$.
\item[$\boxed{\biguplus_{i\in \N}A_i \in \mathcal{D}_B}$] Let $(A_i)_{i=1}^\infty$ be a disjoint family of sets in $\mathcal{D}_B$. Then, using \ref{setDistributivity},
\[ \left(\bigcup_{i=1}^\infty A_i\right)\cap B = \bigcup_{i=1}^\infty (A_i\cap B) \in \mathfrak{D}\{\mathcal{F}\}, \]
so $\bigcup_{i=1}^\infty A_i\in\mathcal{D}_B$.
\end{itemize}
Now because $\mathcal{F}$ is a $\pi$-system, we have $\mathcal{F}\subset\mathcal{D}_B$ and thus $\mathcal{D}_B\subset\mathfrak{D}\{\mathcal{F}\}$.

Now for all $B\in\mathcal{F}$, we have
\[ A\in\mathcal{F}\implies A\cap B\in\mathcal{F}\implies A\cap B\in\mathfrak{D}\{\mathcal{F}\} \implies A\in \mathcal{D}_B. \]
So $\mathcal{F}\subset \mathcal{D}_B$ if $B\in\mathcal{F}$. In this case we then also have $\mathfrak{D}\{\mathcal{F}\}\subset \mathcal{D}_B$.

In fact this holds for all $B\in\mathfrak{D}\{\mathcal{F}\}$:
\[ B\in\mathfrak{D}\{\mathcal{F}\} \implies \forall A\in\mathcal{F}: B\in\mathcal{D}_A \implies  \forall A\in\mathcal{F}: B\cap A \in \mathfrak{D}\{\mathcal{F}\} \implies \mathcal{F}\subset\mathcal{D}_B \implies \mathfrak{D}\{\mathcal{F}\}\subset \mathcal{D}_B. \]
Consequently,
\[ B,C\in\mathfrak{D}\{\mathcal{F}\} \implies C\in\mathcal{D}_B \implies C\cap B\in\mathfrak{D}\{\mathcal{F}\}, \]
meaning $\mathfrak{D}\{\mathcal{F}\}$ is a $\pi$-system.
\end{proof}
\begin{corollary}[$\pi$-$\lambda$ theorem] \label{piLambdaTheorem}
Let $P$ be a $\pi$-system and $D$ a Dynkin system with $P\subseteq D$, then $\sigma\{P\} \subseteq D$.
\end{corollary}
\begin{corollary}
If $\mathcal{A}$ is an algebra, then $\mathfrak{M}\{\mathcal{A}\} = \mathfrak{D}\{\mathcal{A}\} = \sigma\{\mathcal{A}\}$.
\end{corollary}

\subsection{Product $\sigma$-algebras}
\subsubsection{Finite products}
\begin{definition}
Let $X,Y$ be sets and $\mathcal{A}, mathcal{B}$ $\sigma$-algebras on $X,Y$, respectively. Then the \udef{product $\sigma$-algebra} of $\mathcal{A}$ and $\mathcal{B}$ is
\[ \mathcal{A}\otimes\mathcal{B} \defeq \sigma\setbuilder{A\times B}{A\in \mathcal{A}, B\in \mathcal{B}}. \]
\end{definition}
Notice that in general for $\sigma$-algebras $\mathcal{A},\mathcal{B}$, to set $\setbuilder{A\times B}{A\in \mathcal{A}, B\in\mathcal{B}}$ is not a $\sigma$-algebra. It is necessary to take the closure.

\begin{example}
Take $\Omega = \{a,b,c\}$. Define the $\sigma$-algebras $\mathcal{A},\mathcal{B}$ on $\Omega$ by
\begin{align*}
\mathcal{A} &= \{\emptyset, \{a\}, \{b,c\}, \Omega\} \\
\mathcal{B} &= \{\emptyset, \{a, b\}, \{c\}, \Omega\}.
\end{align*}
Then $\{a\}\times\{a,b\} \in \setbuilder{A\times B}{A\in \mathcal{A}, B\in\mathcal{B}}$, but
\[ (\{a\}\times\{a,b\})^c = \Big(\{b,c\}\times \{a,b,c\}\Big) \cup \Big(\{a, b ,c\}\times\{c\}\Big) \]
is not in $\setbuilder{A\times B}{A\in \mathcal{A}, B\in\mathcal{B}}$, so it is not a $\sigma$-algebra.
\end{example}

\begin{lemma}
Let $X,Y$ be sets and $A\subseteq X, B\subseteq Y$ subsets. Then
\[ (A\times B)^c = A^c\times B \uplus A\times B^c \uplus A^c\times B^c = A^c\times Y \cup X\times B^c. \]
\end{lemma}
\begin{proof}
We have $X\times Y = (A\uplus A^c)\times (B\uplus B^c)$ and the lemma follows by \ref{productUnionIntersection}.
\end{proof}

\subsubsection{Infinite products}
\begin{definition}
Let $I$ be an index set, $\{X_i\}_{i\in I}$ a set of sets and $\{\mathcal{A}_i\}_{i\in I}$ a set of $\sigma$-algebras such that for all $i\in I$, $\mathcal{A}_i$ is a $\sigma$-algebra on $X_i$. We define the \udef{product $\sigma$-algebra} $\bigotimes_{i\in I}\mathcal{A}_i$ on $\prod_{i\in I}X_i$ as
\[ \bigotimes_{i\in I}\mathcal{A}_i \defeq \sigma\setbuilder{\prod_{i\in I}A_i}{\forall i\in I: A_i\in\mathcal{A}_i\; \text{and}\; \setbuilder{i\in I}{A_i\neq X_i}\;\text{is finite}}. \]
\end{definition}

\begin{proposition}
We may replace ``is finite'' with ``is countable'' in the definition.
\end{proposition}
\begin{proof}
TODO (can we??)
\end{proof}

\chapter{Well-founded ordered sets}
\begin{definition}
A poset is called
\begin{itemize}
\item \udef{well-founded} if every non-empty subset has a minimal element;
\item \udef{converse well-founded} if every non-empty subset has a maximal element.
\end{itemize}

A \udef{well-ordering} on a set $U$ is a total order $\leq$ on $U$ such that $\sSet{U,\leq}$ is well-founded.

A set $A$ is \udef{well-orderable} if it admits a well-ordering.
\end{definition}
It turns out a well-order is what is needed to do recursion and induction.

\begin{lemma}
Every well-ordering has a least element.
\end{lemma}
\begin{proof}
A minimal element for a total order is always a least element.
\end{proof}

\begin{lemma} \label{wellOrderingSubsets}
Let $(U,\leq_U)$ be a well-ordered set and $f: W \rightarrowtail U$ an injection. Then $W$ is well-ordered by
\[ \forall x,y\in W: x\leq_W y \defequiv f(x) \leq_U f(y). \]


In particular, if $W\subseteq U$ is a subset, then $\leq_W$ is the left- and right-restriction of $\leq$ to $W$, $\leq|_W^W$.
\end{lemma}

\section{Succession}
Every well-ordered set $U$ must have a least element and at its low end it looks like $\N$:
\begin{itemize}
\item let $0_U$ denote the least element of $U$;
\item we can define $S_U(x) \defeq \min\{y\in P\;|\;x<y\}$.
\end{itemize}
This successor function is defined for all $x\in U$, except the maximum (if it exists).

\begin{definition}
Let $(U,\leq)$ be a well-ordered set.
\begin{itemize}
\item The values of the partial function $S: U\not\to U$ are the \udef{successor points} of $U$.
\item A \udef{limit point} is an element $x\in U$ that is neither $0_U$ nor a successor. The first limit point (i.e. the least point in the set of limit points) is denoted $\omega$ or $\omega_U$.
\item The points below $\omega$ are called \udef{finite points} and the points above, and including, $\omega$ are the \udef{infinite points} of $U$.
\end{itemize}
\end{definition}

If $P$ is a poset, we can always add a point on top of all the rest: We can take, e.g. the set
\[ t_P = \{ x\in P\;|\;x\notin x \}. \]
This is guaranteed, by proposition \ref{russelParadox}, not to be in $P$.
The poset $P\cup t_P$ is called the \udef{successor} $\operatorname{Succ}(P)$ of $P$.

\section{Initial segments}
TODO: streamline
\begin{definition}
Let $(U,\leq)$ be a well-ordered set. An \udef{initial segment} $I$ of $U$ is a downward closed subset:
\[ \forall y\in I: \forall x\in U: x\leq y \implies x\in I. \]
We write $I \sqsubseteq U$.
\end{definition}
Each element $y$ of $U$ determines a proper initial segment of points strictly below $y$:
\[ \seg(y) \defeq \{ x\in U\;|\; x < y \} \sqsubsetneq U. \]
We have $\seg(S_U(y)) = \seg(y)\cup\{y\}$.

Conversely, each proper initial segment is of the form $\seg(x)$:
\begin{proposition}
Let $(U,\leq)$ be a well-ordered set and $W$ a subset of $U$. Then $W$ is an initial segment \textup{if and only if} either $W=U$ or $\exists! x\in U: W = \seg(x)$.
\end{proposition}
\begin{proof}
The direction $\Rightarrow$ is trivial. For the other direction, assume $W \sqsubsetneq U$ and let $x= \min(U\setminus W)$. Showing that $W=\seg(x)$ is not difficult.
\end{proof}
We may then, in some sense, view $x$ as the length of $\seg(x)$. We identify $t_U$ as the length of $U$. Let $U$ be a well-ordered set. We define $\len_U$ which maps initial segments of $U$ to $\operatorname{Succ}(U)$ by
\[ \len_U(V) = \begin{cases}
x & \exists x\in U: V=\seg(x) \\
t_U & V = U.
\end{cases} \]

Each well-ordered set $U$ can be viewed as a proper initial segment of another:
\[ U = \seg_{\operatorname{Succ}(U)}(t_U)\sqsubsetneq \operatorname{Succ}(U). \]
\begin{lemma} \label{orderingInitialSegments}
Let $(U,\leq)$ be a well-ordered set and $x,y\in U$, then
\begin{align*}
\seg(x) = \seg(y) &\iff x = y; \\
\seg(x) \sqsubseteq \seg(y) &\iff x \leq y; \\
\seg(x) \sqsubsetneq \seg(y) &\iff x < y.
\end{align*}
\end{lemma}
\begin{proposition} \label{wosetIsomorphicToInitialSegments}
Any well-ordered set $(U,\leq)$ is order isomorphic to the set of its proper initial segments ordered by inclusion, $(\seg_U[U], \sqsubseteq)$.
\end{proposition}
\begin{proof}
The function $\seg_U$ is an order embedding by lemma \ref{orderingInitialSegments}. By lemma \ref{orderReflectionIsInjective} it must be injective and thus $U =_o \seg_U[U]$.
\end{proof}
\begin{lemma}
The family of initial segments of a well-ordered set $U$ is
\begin{enumerate}
\item well-ordered by $\sqsubseteq$; and
\item closed under arbitrary unions.
\end{enumerate}
\end{lemma}
\begin{lemma} \label{unionInitialSegments}
Let $(U,\leq)$ be a well-ordered set and $t\in U$. Then $\bigcup\{ \seg(u)\;|\; u<t \}$ is an initial segment and thus equal to $\seg(v)$ for some $v$. Also
\[ \seg(v) \leq \seg(t) \leq \seg(S_U(v)). \]
\end{lemma}
\begin{proof}
For the first inequality: let $x\in \seg(v)$, so $\exists u<t: x\in\seg(u)$ so $x<t$ and $x\in \seg(t)$.

For the second inequality: assume, towards a contradiction, that $\seg(t)>\seg(S_U(v))$. Then $S_U(v) < t$ and so $\seg(S_U(v))\subseteq \seg(v)$ and thus $v\in\seg(v)$, a contradiction.
\end{proof}

\begin{proposition} \label{injectionsExpansive}
Every order-preserving injection $f: U\rightarrowtail U$ of a well-ordered set into itself is expansive.
\end{proposition}
\begin{proof}
Assume $f: U\rightarrowtail U$ injective but not expansive, i.e. $\exists x\in U: f(x)<x$ then let
\[ x^* = \min\{x\in U\;|\;f(x)<x\}. \]
Then $f(x^*)<x^*$ and $f(f(x^*)) < f(x^*)$ by order preservation. Then $f(x^*)$ is a smaller element in the set, yielding a contradiction.
\end{proof}
\begin{corollary} \label{properInitialSegmentNotIsomorphic}
No well-ordered set is isomorphic with one of its proper initial
segments, and hence no two distinct initial segments of a well-ordered set are
isomorphic.
\end{corollary}


\section{Transfinite induction and recursion}
The principles of induction and recursion can be generalised to well-ordered sets.

In general induction and recursion use the predecessor to define / prove a property of the successor. In general well-ordered sets, there are limit points that have no predecessor. For this reason it is easiest to generalize the principles of proof by \textit{complete
induction} and definition by \textit{complete recursion}. Then we take the set of all predecessors, not the one predecessor that may or may not exist.

\begin{theorem}[Transfinite induction]
Let $U$ be a well-ordered set and $P$ a unary definite predicate. We can prove $P(x)$ holds for all $x\in U$ by proving the strong induction step $\forall x\in U: \forall y<x: P(y)\implies P(x)$.

Or, in other symbols,
\[ \text{if}\quad  \forall x\in U: \left[\forall y < x: P(y)\implies P(x)\right] \quad \text{then}\quad \forall x\in U: P(x) \]
\end{theorem}
\begin{proof}
Assume, towards a contradiction, the induction step and that $\exists x\in P: \neg P(x)$. Then the set of all such $x$ has a least element (due to $U$ being well-ordered). Let
\[ x^* = \min\{ x\in U\;|\; \neg P(x) \}, \]
then all elements smaller than $x^*$ must not be in this set: $\forall y<x^*: P(y)$, so that by the induction step $P(x^*)$.
\end{proof}
Notice that the ``base step'' ($\nexists y\in U: y<0$, so $\forall y<0: P(y)\implies P(x)$) is vacuously true.

It is often as easy to repeat this argument as appeal to the theorem.

\begin{theorem}[Transfinite recursion]
Let $U$ be a well-ordered set, $E$ some non-empty set and $h: (U\not\to E) \to E$ some function.

There is exactly one function $f:U\to E$ which satisfies
\[ f(x) = h(f|_{\seg(x)}) \qquad \forall x\in U. \]
\end{theorem}
\begin{proof}
Like when proving recursion on $\N$, we will consider ``approximations'' of the function $f$, i.e. functions $\seg(t) \to E$ which satisfy the requirement for all $x<t$.

This is the subject of the following lemma:
\begin{lemma*}
Let $U$ be a well-ordered set, $E$ some non-empty set and $h: (U\not\to E) \to E$ some function.

For all $t\in U$, there is exactly one function $\sigma_t:\seg(t)\to E$ which satisfies
\[ \sigma_t(x) = h(\sigma_t|_{\seg(x)}) \qquad \forall x < t. \]
\end{lemma*}
\begin{proof}[Proof of lemma] \renewcommand{\qedsymbol}{$\dashv$ (Lemma)}
The proof goes by transfinite induction. Fix an arbitrary $t\in U$. Assume the induction hypothesis:
\[ \forall u<t: \exists! \sigma_u\in (\seg(u)\to E): \forall x<u: \sigma_u(x) = h(\sigma_u|_{\seg(x)}). \]
We need to prove that this implies there exists exactly one $\sigma_t$ satisfying the condition. We consider three cases: $t$ is the least point $0_U$, a successor point or a limit point.
\begin{itemize}[leftmargin=2.5cm]
\item[\boxed{t = 0_U}] Then $\seg(t) = \emptyset$ and we must have $\sigma_t = \emptyset$.
\item[\boxed{t = S_U(v)}] If $t$ is the successor of $v$, we can set
\[ \sigma_t = \sigma_v\cup\{ (v,h(\sigma_v)) \}. \]
\item[\boxed{t\in \operatorname{Limit}(U)}] The set of functions $\{\sigma_u\;|\; u<t\}$ is a chain in the poset $((U\not\to E),\subseteq)$ which is inductive, see \ref{inductive}. Let $\sigma_t$ be the least upper bound.

To prove $\{\sigma_u\;|\; u<t\}$ is a chain, assume not i.e.
\[ x<u<v<t \implies \sigma_u(x) = \sigma_v(x) \]
fails for some $x<u<v$. Take the least such $x$ (we are effectively doing transfinite induction) so then
\[ \sigma_u|_{\seg(x)} = \sigma_v|_{\seg(x)}, \]
and by the induction hypothesis
\[ \sigma_u(x) = h(\sigma_u|_{\seg(x)}) = h(\sigma_v|_{\seg(x)}) = \sigma_v(x). \]
This is a contradiction, proving we do indeed have a chain.

Finally we verify
\begin{itemize}
\item the domain of $\sigma_t$ is $\seg(t)$; indeed
\[ \dom(\sigma_t) = \bigcup\{\dom(\sigma_u)\;|\; u< t \} = \bigcup\{ \seg(u)\;|\; u<t \} \]
which is an initial segment and thus equal to $\seg(v)$ for some $v$. By lemma \ref{unionInitialSegments}
\[ v\leq t \leq S_U(v). \]
Then either $t=v$ or $t=S_U(v)$. The latter is excluded because $t$ was a limit point.
\item $\sigma_t$ satisfies the condition (easily by transfinite induction);
\item $\sigma_t$ is unique (also easily by transfinite induction).
\end{itemize}
\end{itemize}
\end{proof}

Now consider the well-ordered set $\operatorname{Succ}(U)$ and extend $h$ to $h': (\operatorname{Succ}(U)\not\to E) \to E$ by
\[ h'(\sigma) = \begin{cases}
h(\sigma) & \sigma\in (U\not\to E) \\
\text{an arbitrary element of $E$} & \sigma \notin (U\not\to E).
\end{cases} \]
We can then apply the lemma to $\operatorname{Succ}(U)$ and $h'$. Because $\seg(t_U) = U$, this gives a unique function $\sigma_{t_U}: U\to E$. We take this as our $f$.
\end{proof}


\chapter{Fixed points}

\begin{definition}
A monotone mapping $\pi : P \to Q$ on a inductive posets
 is \udef{countably continuous} if for every non-empty, countable chain $S\subseteq P$:
 \[ \pi(\sup S) = \sup\pi[S]. \]
\end{definition}

\begin{definition}
Let $(P,\leq)$ be a poset and $f: P\to P$ a function from $P$ to $P$.
\begin{itemize}
\item A \udef{fixed point} is an element $x^*\in P$ such that
\[ f(x^*) = x^*. \]
\item A \udef{strongly least fixed point} is a fixed point such that
\[ \forall y\in P: f(y)\leq y \implies x^* \leq y. \]
\item The \udef{orbit} of an element $p$ of $P$ is a sequence $\N\to P$ defined recursively:
\begin{align*}
p_0 &= p \\
p_{n+1} &= f(p).
\end{align*}
Sometimes we use orbit to mean the set $\{p_n\in P\;|\; n\in \N\}$.
\end{itemize}
\end{definition}

\begin{theorem}[Continuous least fixed point theorem]
Let $\pi:P\to P$ be a countably continuous, monotone mapping on an inductive poset $(P,\leq)$. Then $\pi$ has a unique strongly least fixed point $x^*\in P$.
\end{theorem}
\begin{proof}
As $P$ is inductive, it has a least element $\bot$. The orbit $\{x_n\in P\;|\; n\in \N\}$ of $\bot$ is a chain: $\bot \leq \pi(\bot)$ and the rest follows by induction on $n$, using the monotonicity of $\pi$. Thus the orbit has a supremum. Let $x^*$ be this supremum.

Then, by countable continuity,
\[ \pi(x^*) = \pi(\sup\{x_n\in P\;|\; n\in \N\}) = \sup\pi[\{x_n\in P\;|\; n\in \N\}] = \sup\{x_{n+1}\in P\;|\; n\in \N\} = x^*. \]

To prove $x^*$ is a strongly least fixed point, let $y\in P$ assume $\pi(y)\leq y$. Then we apply induction on $n$:
\begin{itemize}[leftmargin=3cm]
\item[Basis step] $x_0 = \bot \leq y$.
\item[Induction step] $x_n \leq y \implies x_{n+1} = \pi(x_n)\leq \pi(y) \leq y$.
\end{itemize}
\end{proof}

Iteration lemma.

Fixed point theorem.

Least fixed point theorem.

Hitchhiker's guide:
Knaster-Tarski fixed point; Tarksi fixed point

\chapter{Graphs}
\chapter{Trees}

\part{Operator Structures}
\setcounter{chapter}{0} % Reset chapter counter
\chapter{Operator structures}
\begin{definition}
Let $A$ be a class. An \udef{operator structure} on $A$ is a structured class $\sSet{A, (O, e)}$ where $O$ is a class and $e: O\times A \not\to A$ a partial function such that $\setbuilder{e(f,-)}{f\in O}$ is closed under function composition.

We will often write $e_f$ for the partial application $e(f, -)$.

A \udef{homogeneous operator structure} is an operator structure of the form $\sSet{A, (A, e)}$.
\end{definition}
We have that for all $f,g\in O$, there exists $h\in O$ such that $e_f\circ e_g = e_h$.

\section{Operator structures}
\subsection{Composition functions}
TODO

\subsection{Operator identity}
\begin{definition}
Let $\sSet{A, (O, e)}$ be an operator structure. We call $f\in O$ an \udef{operator identity} if $e(f,x) =x$ for all $x\in A$ such that $e(f,x)$ is defined.
\end{definition}

\subsection{Operator substructures}
\begin{definition}
Let $\sSet{A, (O, e)}$ be an operator structure $B\subseteq A$ and $F\subseteq O$. Then $\sSet{B, (F, e)}$ is an \udef{operator substructure} of $\sSet{A, (O, e)}$ if it is an operator structure, i.e.\ if $e\imf(F\times B) \subseteq B$.
\end{definition}

\subsubsection{Principal ideals}
\begin{definition}
Let $\sSet{A, (O, e)}$ be an operator structure and $a\in A$. The \udef{principal ideal} $I_a$ of $a$ is the smallest operator structure that has $\sSet{\{a\}, (O,e)}$ as an operator substructure.

We say $a,b\in A$ are \udef{equivalent}, denoted $a\sim b$, if they have the same principal ideals.
\end{definition}
By the definition it is clear that $\sim$ is an equivalence relation.

\begin{lemma} \label{operatorStructurePrincipalIdeal}
Let $\sSet{A, (O, e)}$ be an operator structure and $a\in A$. Then the principal ideal of $a$ is $\{a\}\cup e^\imf(O\times\{a\})$.
\end{lemma}
\begin{proof}
Let $I_a$ be the principal ideal of $a$.
Clearly we have $e^\imf(O\times\{a\}) \subseteq B$, so it is enough to prove that the principal ideal is an operator substructure. Take $e_f(a) \in e^\imf(O\times\{a\})$ and $g\in O$. Then there exists $h\in O$ such that $e_h = e_g\circ e_f$, so $(e_g\circ e_f)(a) = e_h(a) \in e^\imf(O\times\{a\})$.
\end{proof}

\begin{lemma} \label{operatorStructureEquivalence}
Let $\sSet{A, (O, e)}$ be an operator structure and $a,b\in A$. Let $I_a, I_b$ be the principal ideals of $a$ and $b$. Then
\begin{enumerate}
\item $I_a \subseteq I_b$ \textup{if and only if} either $a=b$ or $\exists f\in O: e_f(b) = a$;
\item $a\sim b$ \textup{if and only if} either $a=b$ or $\exists f,g\in O: \; \big(e_f(a) = b\big) \land \big(a = e_g(b)\big)$.
\end{enumerate}
\end{lemma}
\begin{proof}
(1) First assume $I_a \subseteq I_b$. Then $a\in I_b$, so either $a=b$ or $a\in e^\imf(O\times\{b\})$ by \ref{operatorStructurePrincipalIdeal}.

Now assume the right-hand side. If $a=b$, then $I_a =I_b$ and the left-hand side holds. Now assume $e_f(b) = a$, then $I_b$ contains $a$ and is an operator substructure, so $I_a \subseteq I_b$.

(2) Immediate from (1).
\end{proof}


\begin{proposition} \label{operatorStructureEquivalenceCongruence}
Let $\sSet{A, (O, e)}$ be an operator structure. Then $\sim$ is a congruence.
\end{proposition}
\begin{proof}
Take $(a, b) \in {\sim}$. We need to prove that $e_h(a) \sim e_h(b)$ for all $h\in O$.

If $a=b$, then clearly $e_h(a) \sim e_h(b)$ for all $h\in O$. Now assume $a\neq b$. Then by \ref{operatorStructureEquivalence} there exist $f,g\in O$ such that $e_f(a) = b$ and $a = e_g(b)$. By closure of the operator set, there exists $f'\in O$ such that $e_{f'} = e_h \circ e_f$. Similarly there exists $g'\in O$ such that $e_{g'} = e_h\circ e_g$. Then
\[ e_{f'}(a) = (e_h \circ e_f)(a) = e_h(b) \qquad\text{and}\qquad e_{g'}(b) = (e_h \circ e_g)(b) = e_h(a), \]
so $e_h(a) \sim e_h(b)$ by \ref{operatorStructureEquivalence}.
\end{proof}


\section{Homogeneous operator structures}
\subsection{Point identity}
\begin{definition}
Let $\sSet{A, (A, e)}$ be a homogeneous operator structure. We call $a\in A$ a \udef{point identity} if $e(x,a) = x$ for all $x\in A$ such that $e(x,a)$ is defined.
\end{definition}

\subsection{Inverses}
\begin{definition}
Let $\sSet{A, (A, e)}$ be a homogeneous operator structure with (TODO point / operator) identity $f$ and $x\in A$. We call $y\in A$ an \udef{inverse} of $x$ if $e(y,x) = f$. (TODO or $e(x,y)$)
\end{definition}

\subsection{Idempotents}
\begin{definition}
Let $\sSet{A, (A, e)}$ be a homogeneous operator. An element $x\in A$ is called \udef{idempotent} if $e(x,x) = x$.
\end{definition}

\chapter{Operator bistructures and operator $n$-structures}
\begin{definition}
Let $A$ be a class. An \udef{operator bistructure} on $A$ is a structured class $\sSet{A, (L, \lambda), (R,\rho)}$ such that
\begin{itemize}
\item $\sSet{A, (L, \lambda)}$ and $\sSet{A, (R,\rho)}$ are operator structures;
\item $\lambda_f \circ \rho_g = \rho_g\circ \lambda_f$ for all $f\in L$ and $g\in R$.
\end{itemize}
An \udef{operator $n$-structure} on $A$ is a structured class $\sSet{A, \seq{(O_i, e_i)}_{i\in(0:n)}}$ such that
\begin{itemize}
\item $\sSet{A, (O_i, e_i)}$ is an operator structure for all $i\in (0:n)$;
\item $e_i(f,e_j(g,-)) = e_j(g,e_i(f,-))$ for all $i\neq j\in (0:n)$, $f\in e_i$ and $g\in e_j$.
\end{itemize}
\end{definition}

TODO notation: $x>y<z$.

Also $a>b>x<c<d$ is parsed as $a>(b>x<c)<d$

\section{Green's relations}
\begin{definition}
Let $\sSet{A, (L, \lambda), (R,\rho)}$ be an operator bistructure. Then
\begin{itemize}
\item we denote the equivalence of $\sSet{A, (L,\lambda)}$ as $\greensL$;
\item we denote the equivalence of $\sSet{A, (R,\rho)}$ as $\greensR$;
\item $\greensH \defeq \greensL \cap \greensR$;
\item $\greensD \defeq \greensL;\greensR$.
\end{itemize}
These equivalence relations are called \udef{Green's relations}.
\end{definition}

\begin{lemma}
Let $\sSet{A, (L, \lambda), (R,\rho)}$ be an operator bistructure. Then $\sim_L;\sim_R = \sim_R;\sim_L$.
\end{lemma}
\begin{proof}
Assume $x(\greensR;\greensL)z$, meaning $\exists y: x\greensR y$ and $y\greensL z$. Then there exist $a,b\in R$ and $c,d\in L$ such that
\[ \begin{tikzcd}
x \ar[r, maps to, shift left, "\rho_a"] & y \ar[l, maps to, shift left, "\rho_b"] \ar[r, maps to, shift left, "\lambda_c"] & z \ar[l, maps to, shift left, "\lambda_d"]
\end{tikzcd}. \]
Using the fact that the $\lambda$s and $\rho$s commute, we can rearrange such that we also get the mappings along the left and bottom sides of
\[ \begin{tikzcd}[sep=large]
x \ar[r, maps to, shift left, "\rho_a"] \ar[d, maps to, shift left, "\lambda_c"] & y \ar[l, maps to, shift left, "\rho_b"] \ar[d, maps to, shift left, "\lambda_c"] \\
y' \ar[u, maps to, shift left, "\lambda_d"] \ar[r, maps to, shift left, "\rho_a"] & z \ar[u, maps to, shift left, "\lambda_d"] \ar[l, maps to, shift left, "\rho_b"]
\end{tikzcd} \]
for some $y' \in A$. Thus $x(\greensL;\greensR)y$. The other inclusion is similar.
\end{proof}

\begin{corollary}
Let $\sSet{A, (L, \lambda), (R,\rho)}$ be an operator bistructure. Then $\greensD$ is indeed an equivalence relation.
\end{corollary}
\begin{proof}
By \ref{commutingEquivalenceRelations}.
\end{proof}

\begin{proposition}[Green's lemma] \label{GreensLemma}
Let $\sSet{A, (L, \lambda), (R,\rho)}$ be an operator bistructure and $x,y \in A$.
\begin{enumerate}
\item If $x\greensL y$ with $\begin{tikzcd}
x \ar[r, maps to, shift left, "\lambda_a"] & y \ar[l, maps to, shift left, "\lambda_b"]
\end{tikzcd}$, then
\begin{enumerate}
\item $\lambda_a|_{[x]_\greensR}: [x]_\greensR \to [y]_\greensR \qquad \text{is a bijection with inverse} \qquad \lambda_b|_{[y]_\greensR}: [y]_\greensR \to [x]_\greensR$;
\item $\lambda_a|_{[x]_\greensH}: [x]_\greensH \to [y]_\greensH \qquad \text{is a bijection with inverse} \qquad \lambda_b|_{[y]_\greensH}: [y]_\greensH \to [x]_\greensH$.
\end{enumerate}
\item If $x\greensR y$ with $\begin{tikzcd}
x \ar[r, maps to, shift left, "\rho_a"] & y \ar[l, maps to, shift left, "\rho_b"]
\end{tikzcd}$, then
\begin{enumerate}
\item $\rho_a|_{[x]_\greensL}: [x]_\greensL \to [y]_\greensL \qquad \text{is a bijection with inverse} \qquad \rho_b|_{[y]_\greensL}: [y]_\greensL \to [x]_\greensL$;
\item $\rho_a|_{[x]_\greensH}: [x]_\greensH \to [y]_\greensH \qquad \text{is a bijection with inverse} \qquad \rho_b|_{[y]_\greensH}: [y]_\greensH \to [x]_\greensH$.
\end{enumerate}
\end{enumerate}
\end{proposition}
\begin{proof}
Take some arbitrary $x'\in [x]_\greensR$. Then there exist $c,d\in R$ such that $\begin{tikzcd}
x \ar[r, maps to, shift left, "\rho_c"] & x' \ar[l, maps to, shift left, "\rho_d"]
\end{tikzcd}$. Then, because $\lambda$ and $\rho$ commute,
\[ \begin{tikzcd}
x' \ar[r, maps to, shift left, "\rho_d"] & x \ar[l, maps to, shift left, "\rho_c"] \ar[r, maps to, shift left, "\lambda_a"] & y \ar[l, maps to, shift left, "\lambda_b"]
\end{tikzcd} \qquad\text{implies that} \qquad \begin{tikzcd}
x' \ar[r, maps to, shift left, "\lambda_a"] & y' \ar[l, maps to, shift left, "\lambda_b"] \ar[r, maps to, shift left, "\rho_d"] & y \ar[l, maps to, shift left, "\rho_c"]
\end{tikzcd} \]
Thus, for all $x'\in [x]_\greensR$, we have
\begin{itemize}
\item $\lambda_a(x') \in [y]_\greensR$, meaning that $\lambda_a|_{[x]_\greensR}: [x]_\greensR \to [y]_\greensR$ is well-defined;
\item by similar reasoning, we can see that $\lambda_b|_{[y]_\greensR}: [y]_\greensR \to [x]_\greensR$ is also well-defined;
\item $\lambda_b(\lambda_a(x')) = x'$, so the functions are inverse of each other.
\end{itemize}
If $x'\in [x]_\greensH$, then $\lambda_a(x')\greensL \lambda_a(x) = y$ by \ref{operatorStructureEquivalenceCongruence}, so $\lambda_a(x')\in [y]_\greensH$. This means that $\lambda_a|_{[x]_\greensH}: [x]_\greensH \to [y]_\greensH$ is well-defined.

Point (2) is dual.
\end{proof}
\begin{corollary} \label{greensDisomorphism}
Let $\sSet{A, (L, \lambda), (R,\rho)}$ be an operator bistructure, $x, y\in A$ such that $x\greensD y$. Then there exist $a,c\in L$ and $b,d\in R$ such that
\[ \lambda_a\circ\rho_b|_{[x]_\greensH}: [x]_\greensH \to [y]_\greensH \qquad \text{is a bijection with inverse} \qquad \lambda_c\circ \rho_d|_{[y]_\greensH}: [y]_\greensH \to [x]_\greensH. \]
\end{corollary}
\begin{proof}
There exists a $z\in A$ such that $x\greensL z$ and $z\greensR y$. We then just compose the bijections in Green's lemma, keeping in mind that $\lambda$ and $\rho$ commute.
\end{proof}

\section{The intersection of operations and elements}
\subsection{Regular elements and generalised inverses}
\begin{definition}
Let $\sSet{A, (L, \lambda), (R,\rho)}$ be an operator bistructure and $x,y\in A\cap L\cap R$. We say
\begin{itemize}
\item $x$ is \udef{regular} if $\exists a\in A: \; x = (\lambda_x\circ\rho_x) (a) = x>a<x$;
\item $x$ and $y$ \udef{generalised inverses} if $x = x>y<x$ and $y = y>x<y$.
\end{itemize}
\end{definition}

\begin{lemma} \label{idempotentsAreRegular}
Let $\sSet{A, (L, \lambda), (R,\rho)}$ be an operator bistructure. Every idempotent is regular.
\end{lemma}
\begin{proof}
Let $x\in A\cap L\cap R$ be an idempotent. Because $x$ is a left-idempotent, we have $x>x = x$. Because $x$ is a right-idempotent, we have $x<x = x$. Combining the two gives $x = x>x<x$.
\end{proof}

\begin{proposition}
Let $\sSet{A, (L, \lambda), (R,\rho)}$ be an operator bistructure. If $x\in A$ is regular, then every element in $[x]_{\greensD}$ is regular.
\end{proposition}
So it makes sense to call a $\greensD$-class \udef{regular} if it consists of regular elements and \udef{irregular} otherwise.
\begin{proof}
Let $x$ be regular with $x = x>x'<x$ and $x\greensD y$. Then we have $a,c \in L$ and $b,d \in R$ such that
$\lambda_a\circ\rho_b|_{[x]_\greensH}: [x]_\greensH \to [y]_\greensH$ is a bijection with inverse $\lambda_c\circ \rho_d|_{[y]_\greensH}: [y]_\greensH \to [x]_\greensH$, as in \ref{greensDisomorphism}. Then we have
\begin{align*}
y &= a>x<b \\
&= a>(x>x'<x)<b \\
&= a . x>x'<x . b \\
&= a>(c>y<d)>x'<(c>y<d)<b \\
&= (a > c >) y > d > x' > c > y (> d > b) \\
&= y > d > x' > c > y.
\end{align*}
So $y$ is regular.
\end{proof}
\begin{corollary}
If there is an idempotent $x\in [a]_{\greensD}$, then $[a]_{\greensD}$ is regular.
\end{corollary}
\begin{proof}
Every idempotent is regular: \ref{idempotentsAreRegular}.
\end{proof}

\begin{proposition}
Let $\sSet{A, f}$ be an associative class. Then $x$ is regular \textup{if and only if} it has a generalised inverse.
\end{proposition}
\begin{proof}
Clearly every element with a generalised inverse is regular. Conversely, assume $x$ regular with $x = xax$. Then $y = axa$ is a generalised inverse of $x$: $x(axa)x = xax = x$ and $(axa)x(axa) = a(xax)axa = axaxa = axa$.
\end{proof}
Note that we do not have that $x = xyx$ implies $y = yxy$.

\begin{proposition} \label{greensRelationsRegularElements}
Let $\sSet{A, f}$ be an associative class and $x\in A$ a regular element with $x = f(f(x, y), x)$. Then
\begin{enumerate}
\item $f(x,y)$ and $f(y,x)$ are idempotent;
\item $f(y,x) \greensL x$ and $x \greensR f(x, y)$.
\end{enumerate}
\end{proposition}
\begin{proof}
(1) We calculate
\[ f(f(x,y), f(x,y)) = f(f(f(x,y), x), y) = f(x, y)\] and \[f(f(y,x), f(y,x)) = f(y, f(x, f(y,x))) = f(y,x). \]

(2) Using \ref{idealAbsorption}, we have
\[ f(A, x) = f(A, f(x,f(y,x))) \subseteq f(A, f(y,x)) \subseteq f(A,x). \]
Thus $f(A, x) = f(A, f(y,x))$. The second part is dual.
\end{proof}
\begin{corollary}
In a regular $\mathcal{D}$-class each $\mathcal{L}$-class and each $\mathcal{R}$-class contains at least one idempotent.
\end{corollary}
\begin{proof}
Let $[x]_\mathcal{L}$ be an $\mathcal{L}$-class in a regular $\mathcal{D}$-class. By regularity there exists a $y\in A$ such that $xyx = x$. From the proposition, we have that $[x]_\mathcal{L}$ contains the idempotent $yx$ and $[x]_\mathcal{R}$ the idempotent $xy$.
\end{proof}
\begin{corollary} \label{idempotentsHclass}
If $x,x'\in A$ are generalised inverses, then $[xx']_\mathcal{H} = [x]_\mathcal{R}\cap [x']_\mathcal{L}$ and $[x'x]_\mathcal{H} = [x']_\mathcal{R}\cap [x]_\mathcal{L}$.
\end{corollary}
\begin{proof}
From $x\greensR (xx')$ and $(xx')\greensL x'$, we get the first equality. The second is dual.
\end{proof}
We can depict the situation in the corollary as follows:
\[ \hspace{-8.4em} \exists a,b,c,d \in \widetilde{A}: \qquad \begin{tikzcd}[sep=large]
x \ar[r, maps to, shift left, "\rho_a"] \ar[d, maps to, shift left, "\lambda_c"] & xx' \ar[l, maps to, shift left, "\rho_b"] \ar[d, maps to, shift left, "\lambda_c"] \\
x'x \ar[u, maps to, shift left, "\lambda_d"] \ar[r, maps to, shift left, "\rho_a"] & x' \ar[u, maps to, shift left, "\lambda_d"] \ar[l, maps to, shift left, "\rho_b"]
\end{tikzcd} \]

So generalised inverses along one diagonal imply idempotents along the other. In fact, the other direction also holds:
\begin{proposition}
Let $\sSet{A, f}$ be an associative class and $e,f$ idempotents in $A$ such that $e\greensD f$. Then there exist $x\in [e]_\greensR\cap [f]_\greensL$ and $x'\in [e]_\greensL\cap [f]_\greensR$ such that
\begin{itemize}
\item $x,x'$ are generalised inverses;
\item $e = xx'$ and $f = x'x$.
\end{itemize}
\end{proposition}
\begin{proof}
Because $e\greensD f$, we can find $x,x'\in A$ such that
\[ \hspace{-8.4em} \exists a,b,c,d \in \widetilde{A}: \qquad \begin{tikzcd}[sep=large]
e \ar[r, maps to, shift left, "\rho_a"] \ar[d, maps to, shift left, "\lambda_c"] & x \ar[l, maps to, shift left, "\rho_b"] \ar[d, maps to, shift left, "\lambda_c"] \\
x' \ar[u, maps to, shift left, "\lambda_d"] \ar[r, maps to, shift left, "\rho_a"] & f \ar[u, maps to, shift left, "\lambda_d"] \ar[l, maps to, shift left, "\rho_b"]
\end{tikzcd} \]
Then
\begin{align*}
xx' &= (df)(fb) = dfb = e \\
x'x &= (ce)(ea) = cea = f \\
xx'x &= ex = eea = ea = x \\
x'xx' &= fx' = ffb = fb = x',
\end{align*}
which completes the proof.
\end{proof}
\begin{corollary}
Let $\sSet{A, f}$ be an associative class, $y\in A$ and $e,f$ idempotents in $A$. Then $e\greensD f$ \textup{if and only if} there exist generalised inverses $x,x'$ such that $e = xx'$ and $f = x'x$.
\end{corollary}
\begin{proof}
The direction $\Rightarrow$ follows from the proposition. The converse from \ref{greensRelationsRegularElements}.
\end{proof}

\begin{lemma}
Let $\sSet{A, f}$ be an associative class, $x\in A$. Then no $\greensH$-class contains more than one generalised inverse of $x$.
\end{lemma}
\begin{proof}
Assume $x$ has two generalised inverses, $x_1'$ and $x_2'$. From \ref{idempotentsHclass} and \ref{GreensTheoremCorollary} we get that $xx_1' = xx_2'$ and $x_1'x = x_2'x$. Thus
\[ x_1' = x_1'(xx_1') = x_1'xx_2' = (x_1'x)x_2' =  x_2'xx_2' = x_2'. \]
\end{proof}

\begin{proposition}
Let $\sSet{A, f}$ be an associative class and $x,y\in A$. Then $xy\in [x]_\greensR \cap [y]_\greensL$ \textup{if and only if} $[x]_\greensL \cap [y]_\greensR$ contains an idempotent.
\end{proposition}
\begin{proof}
First assume $[x]_\greensL \cap [y]_\greensR$ contains an idempotent $e$. We can depict the situation as
\[ \hspace{-8.4em} \exists a,b,c,d \in \widetilde{A}: \qquad \begin{tikzcd}[sep=large]
x \ar[r, maps to, shift left, "\rho_a"] \ar[d, maps to, shift left, "\lambda_c"] &  \ar[l, maps to, shift left, "\rho_b"] \ar[d, maps to, shift left, "\lambda_c"] \\
e \ar[u, maps to, shift left, "\lambda_d"] \ar[r, maps to, shift left, "\rho_a"] & y \ar[u, maps to, shift left, "\lambda_d"] \ar[l, maps to, shift left, "\rho_b"]
\end{tikzcd} \]
Then we can calculate
\[ xy = deea = dea = xa \in [x]_\greensR \cap [y]_\greensL. \]
Now assume $xy\in [x]_\greensR \cap [y]_\greensL$.
We can depict the situation as
\[ \hspace{-8.4em} \exists a,b,c,d \in \widetilde{A}: \qquad \begin{tikzcd}[sep=large]
x \ar[r, maps to, shift left, "\rho_a"] \ar[d, maps to, shift left, "\lambda_c"] & xy \ar[l, maps to, shift left, "\rho_b"] \ar[d, maps to, shift left, "\lambda_c"] \\
e \ar[u, maps to, shift left, "\lambda_d"] \ar[r, maps to, shift left, "\rho_a"] & y \ar[u, maps to, shift left, "\lambda_d"] \ar[l, maps to, shift left, "\rho_b"]
\end{tikzcd} \]
Now we need to show that $e$ is idempotent. Indeed, starting from the three other corners, we see that $e = cx$ and $e = yb$ and $e = c(xy)b = (cx)(yb) = ee$.
\end{proof}


\chapter{Associative classes}
\begin{definition}
Let $A$ be a class and $f: A\times A\to A$ a binary function. Then $f$ is called \udef{associative} if $\sSet{A, (A, f), (A, f^d)}$ is an operator bistructure.

We call $\sSet{A, f}$ an \udef{associative class}.
\end{definition}

Any property related to the operator structure $\sSet{A, (A,f)}$ is prefixed by ``left'' and any property related to the operator structure $\sSet{A, (A, f^d)}$ is rpefixed by ``right''.

\section{Undefined operations}
We can simulate a partial function by adding an absorbing element $u$ and mapping undefined operations to $u$. The problem with this is that elements can then only be cancellative if they can be composed with anything. Indeed, if $a,b\in A$ such that $f(a,b) = u$. Then $f(a,b) = u = f(a,u)$ and the only way $a$ can be left-cancellative is by having $b=u$.

The problem is that we have made any two undefined operations the same, while we really want any two undefined operations to be different.

TODO: modify set theory to allow $u\neq u$??

\subsection{Connections}
\begin{definition}
Let $A$ be an associative class and $x,y\in A$. We call
\begin{itemize}
\item $\leftconnections{x} \defeq \setbuilder{y\in A}{\text{$yx$ is defined}}$ the class of \udef{left connections};
\item $\rightconnections{x} \defeq \setbuilder{y\in A}{\text{$xy$ is defined}}$ the class of \udef{right connections}.
\end{itemize}
We write
\begin{itemize}
\item $x \preceq_L y$ if $\leftconnections{x} \subseteq \leftconnections{y}$;
\item $x \preceq_R y$ if $\rightconnections{x} \subseteq \rightconnections{y}$.
\end{itemize}
\end{definition}

\begin{lemma}
Let $A$ be an associative class and $x,y,z\in A$. Then
\begin{enumerate}
\item $\leftconnections{xy} = \leftconnections{x}$;
\item $\rightconnections{xy} = \rightconnections{y}$;
\item $x \preceq_L y \iff xz \preceq_L y \iff x \preceq_L yz$ if the relevant terms are defined;
\item $x \preceq_R y \iff zx \preceq_R y \iff x \preceq_R zy$ if the relevant terms are defined.
\end{enumerate}
\end{lemma}


\section{Types and properties of elements}
\subsection{Distinguishability}
\begin{definition}
Let $A$ be an associative class and $x,y\in A$. 
We say
\begin{itemize}
\item $x$ and $y$ are \udef{left-distinguishable} if
\[ x\neq y \quad\implies\quad \exists a\in A: \;\text{$ax$ or $ay$ is defined and $ax \neq ay$}; \]
\item $x$ and $y$ are \udef{right-distinguishable} if
\[ x\neq y \quad\implies\quad \exists a\in A: \; \text{$xa$ or $ya$ is defined and $xa \neq ya$}; \]
\item $x$ and $y$ are \udef{distinguishable} if they are left- \emph{or} right-distinguishable;
\item $x$ and $y$ are \udef{bidistinguishable} if they are left- \emph{and} right-distinguishable.
\end{itemize}
We say
\begin{itemize}
\item $A$ is left-distinguishable if every two elements in $A$ are left-distinguishable;
\item $A$ is right-distinguishable if every two elements in $A$ are right-distinguishable;
\item $A$ is distinguishable if every two elements in $A$ are distinguishable;
\item $A$ is bidistinguishable if every two elements in $A$ are bidistinguishable.
\end{itemize}
\end{definition}

\begin{lemma}
Let $A$ be an associative class and $x,y\in A$. 
Then
\begin{enumerate}
\item $x$ and $y$ are left-distinguishable \textup{if and only if}
\[ x = y \quad\iff\quad \forall a\in A: \;\Big(\text{$ax$ or $ay$ is defined} \implies ax = ay\Big); \]
\item $x$ and $y$ are right-distinguishable \textup{if and only if}
\[ x = y \quad\iff\quad \forall a\in A: \; \Big(\text{$xa$ or $ya$ is defined} \implies xa = ya\Big). \]
\end{enumerate}
In both cases the direction $\Rightarrow$ is automatic.
\end{lemma}

\subsubsection{Cancellation}
\begin{definition}
Let $A$ be an associative class and $x\in A$. 
We call $x$
\begin{itemize}
\item \udef{left-cancellative} or \udef{monic} if $x\cdot -: y\mapsto xy$ is injective;
\item \udef{right-cancellative} or \udef{epic} if $-\cdot x: y\mapsto yx$ is injective.
\end{itemize}
\end{definition}

Left and right cancellative are dual properties.

\begin{lemma}
Let $A$ be an associative class and $x,y$ in $A$.
\begin{enumerate}
\item If $x$ and $y$ are left-(resp. right-)cancellative, then $xy$ is left-(resp. right-)cancellative.
\item If $xy$ is left-cancellative, then $y$ is left-cancellative.
\item If $xy$ is right-cancellative, then $x$ is right-cancellative.
\end{enumerate}
\end{lemma}
\begin{proof}
Let $z_1, z_2\in A$.

(1) Assume that $x$ and $y$ are left-cancellative and $(xy)z_1 = (xy)z_2$. By associativity, we have $x(yz_1) = x(yz_2)$. Thus by injectivity we get first $yz_1 = yz_2$ and then $z_1 = z_2$.

Right-cancellation is similar.

(2) Assume $yz_1 = yz_2$. Then $xyz_1 = xyz_2$, so $z_1 = z_2$ because $xy$ is left-cancellative.

(3) Assume $z_1x = z_2x$. Then $z_1xy = z_2xy$ so $z_1 = z_2$ because $xy$ is right-cancellative.
\end{proof}

\begin{lemma}
Let $A$ be an associative class.
\begin{enumerate}
\item If $\forall x\in A$ there exists a left-cancellative element $a_x$ such that $a_xx$ exists, then $A$ is left-distinguishable.
\item If $\forall x\in A$ there exists a right-cancellative element $a$ such that $ax$ exists, then $A$ is right-distinguishable.
\end{enumerate}
\end{lemma}
\begin{proof}
(1) Take $x,y\in A$ and assume $\forall b\in A: \Big(\text{$bx$ or $by$ is defined} \implies bx = by\Big)$. In particular, this means that $a_xx = a_xy$. By left-cancellation, we have $x=y$.

(2) Dual to (1).
\end{proof}

\subsubsection{Identity and objects}
\begin{definition}
Let $A$ be a class, $f: A\times A \not\to A$ a binary partial function and $e\in A$ an idempotent (i.e.\ $e^2 = e$).

We call $e$
\begin{itemize}
\item a \udef{centre identity} if $xy = xey$ for all $x,y\in A$ such that both sides are defined;
\item a \udef{left identity} if $x = ex$ for all $x\in A$ such that $ex$ is defined;
\item a \udef{right identity} if $x = xe$ for all $x\in A$ such that $xe$ is defined;
\item an \udef{identity} if $e$ is both a left and a right identity.
\end{itemize}

We call $e$
\begin{itemize}
\item a \udef{weak left identity} if $x = ex$ for all $x\in A$ such that $ex$ is defined and $\leftconnections{e} = \leftconnections{x}$;
\item a \udef{weak right identity} if $x = xe$ for all $x\in A$ such that $xe$ is defined and $\rightconnections{e} = \rightconnections{x}$;
\item a \udef{weak identity} if $e$ is both a weak left and a weak right identity.
\end{itemize}
\end{definition}

\begin{lemma}
Let $A$ be an associative class and $e\in A$ an idempotent.
Then
\begin{enumerate}
\item $e$ is a left identity \textup{if and only if} $e$ is left-cancellative;
\item $e$ is a right identity \textup{if and only if} $e$ is right-cancellative.
\end{enumerate}
\end{lemma}
\begin{proof}
(1) First assume $e$ is a left identity. Let $x,y\in A$ be such that $ex = ey$. Then $x = ex = ey = y$.

Now assume that $e$ is left-cancellative and that $ex$ is defined. Then $ex = eex$ by idempotency and thus $x = ex$ by left-cancellation.

(2) Dual.
\end{proof}

\begin{lemma} \label{uniquenessIdentity}
Let $A$ be an associative class and $e,e'\in A$. Then
\begin{enumerate}
\item if $e,e'$ are weak left identities such that $ee' = e'e$, then $e=e'$;
\item if $e,e'$ are weak right identities such that $ee' = e'e$, then $e=e'$.
\end{enumerate}
Also
\begin{enumerate} \setcounter{enumi}{2}
\item if $e$ is a left identity and $e'$ a right identity such that $ee'$ is defined, then $e=e'$;
\end{enumerate}
and
\begin{enumerate} \setcounter{enumi}{3}
\item if $e,e'$ are identities and there exists $x\in A$ such that $ex$ and $e'x$ are both defined, then $e = e'$;
\item if $e,e'$ are identities and there exists $x\in A$ such that $xe$ and $xe'$ are both defined, then $e = e'$.
\end{enumerate}
\end{lemma}
\begin{proof}
(1) We have $\leftconnections{e} = \leftconnections{ee'} = \leftconnections{e'e} = \leftconnections{e'}$, so $e = e'e = ee' = e'$.

(2) Dual.

(3) We have $e = ee' = e'$.

(4) We have $x = ex = ee'x$, so $ee'$ is defined. We conclude with (3).

(5) Dual to (4).
\end{proof}

\begin{lemma} \label{distinguishableIdentity}
Let $A$ be an associative class and $e\in A$.
\begin{enumerate}
\item If $e$ is a left identity and $A$ is right-distinguishable, then $e$ is a weak right identity.
\item If $e$ is a right identity and $A$ is left-distinguishable, then $e$ is a weak left identity.
\end{enumerate}
\end{lemma}
\begin{proof}
(1) Take $x\in A$ such that $xe$ is defined and $\rightconnections{e} = \rightconnections{x}$. Take arbitrary $a\in A$. If $xa$ is defined, then $xea$ is also defined and $xea = xa$. By right-distinguishability, we have $xe = x$. Thus $e$ is a right identity.

(2) Dual.
\end{proof}

\begin{lemma} \label{identityConnection}
Let $A$ be an associative class and $e,x\in A$. Then
\begin{enumerate}
\item if $e$ is a left identity such that $ex$ is defined, then $\leftconnections{e} = \leftconnections{x}$;
\item if $e$ is a right identity such that $xe$ is defined, then $\rightconnections{e} = \rightconnections{x}$;
\end{enumerate}
and
\begin{enumerate} \setcounter{enumi}{2}
\item if $e$ is a left identity such that $xe$ is defined, then $e \preceq_R x$;
\item if $e$ is a right identity such that $ex$ is defined, then $e \preceq_L x$.
\end{enumerate}
\end{lemma}
\begin{proof}
(1) We have $\leftconnections{e} = \leftconnections{ex} = \leftconnections{x}$.

(2) Dual.

(3) Take $a\in \rightconnections{e} = \rightconnections{xe}$. Then $xea$ is defined and $xea = xa$, so $a\in \rightconnections{x}$.

(4) Dual.
\end{proof}


\subsubsection{Inverses}
\begin{definition}
Let $A$ be an associative class and $x\in A$. 
We say an element $y\in A$ is
\begin{itemize}
\item a \udef{left-inverse} of $x$ if $yx$ is a left-identity;
\item a \udef{right-inverse} of $x$ if $xy$ is a right-identity;
\item a \udef{(two-sided) inverse} of $x$ it is both a left- and a right-inverse.
\end{itemize}
We say $(x,y)\in A^2$ is a pair of
\begin{itemize}
\item \udef{mutual left-inverses} if $x$ is a left-inverse of $y$ and $y$ a left-inverse of $x$;
\item \udef{mutual right-inverses} if $x$ is a right-inverse of $y$ and $y$ a right-inverse of $x$.
\end{itemize}
We call $x$ \udef{invertible} if it has a (left/right) mutual inverse.
\end{definition}

\begin{lemma}
Let $A$ be an associative class and $x\in A$.
\begin{enumerate}
\item If $x$ has a left-inverse, then it is left-cancellative.
\item If $x$ has a right-inverse, then it is right-cancellative.
\end{enumerate}
\end{lemma}
\begin{proof}
(1) Let $y\in A$ be a left-inverse of $x$. Take arbitrary $a,b\in A$. Assume $xa = xb$. Then $yxa = yxb$ and so $a = b$.

(2) Dual.
\end{proof}

\begin{proposition}
Let $A$ be an associative class and $x,y_1,y_2,l,r \in A$.
\begin{enumerate}
\item If $A$ is right-distinguishable and both $(x, y_1)$ and $(x,y_2)$ are pairs of mutual left-inverses, then $y_1 = y_2$.
\item If $A$ is left-distinguishable and both $(x, y_1)$ and $(x,y_2)$ are pairs of mutual right-inverses, then $y_1 = y_2$.
\end{enumerate}
Also
\begin{enumerate} \setcounter{enumi}{2}
\item If $l$ is a left-inverse of $x$ and $r$ a right-inverse of $x$, then $l = r$.
\item If $x$ has an inverse, then this inverse is unique.
\end{enumerate}
\end{proposition}
\begin{proof}
(1) We have $y_1 = (y_2x)y_1 = y_2(xy_1)$, so the latter is defined, which implies $xy_1 \preceq_R y_2$ by \ref{identityConnection}. Thus $y_1 \preceq y_2$. Similarly $y_1xy_2$ is defined and this implies $y_2 \preceq y_1$. Thus $\rightconnections{xy_1} = \rightconnections{y_1} = \rightconnections{y_2}$. By \ref{distinguishableIdentity} $xy_1$ is a weak right identity, so $\rightconnections{xy_1} = \rightconnections{y_2}$ implies $y_1 = y_2xy_1 = y_2$.

(2) Dual.

(3) We have $l = l(xr) = (lx)r = r$.

(4) Any inverse of $x$ is both a left- and a right-inverse.
\end{proof}

If $x$ is invertible, we denote the unique inverse by $x^{-1}$.




\begin{lemma}
Let $A$ be an associative class and $x,y \in A$ such that $xy$ is defined.
\begin{enumerate}
\item If $x$ has left-inverse $x^{-L}$ and $y$ has a left-inverse $y^{-L}$, then $xy$ has a left-inverse $y^{-L}x^{-L}$.
\item If $x$ has right-inverse $x^{-R}$ and $y$ has a right-inverse $y^{-R}$, then $xy$ has a right-inverse $y^{-R}x^{-R}$.
\item If $x$ has inverse $x^{-1}$ and $y$ has inverse $y^{-1}$, then $xy$ has inverse $y^{-1}x^{-1}$.
\item If $xy$ has ya left-inverse $z$, then $y$ has a left-inverse $zx$.
\item If $xy$ has a right-inverse $z$, then $x$ has a right-inverse $yz$.
\end{enumerate}
\end{lemma}
\begin{proof}
(1) Set $e = x^{-L}x$. We calculate
\[ y^{-L}x^{-L}xy = y^{-L}ey = y^{-L}y, \]
which is a left identity. Then we just need to show that $xyy^{-L}x^{-L}xy = xy$. Indeed
\[ xyy^{-L}x^{-L}xy = xyy^{-L}ey = x(yy^{-L}y) = xy. \]

(2) Dual to (1).

(3) Follows from (1) and (2).

(4) Set $e = z(xy)$. We calculate
\[ e = z(xy) = (zx)y, \]
so $zx$ is a left-inverse of $y$.

\end{proof}



\begin{lemma}
Let $\sSet{A, f}$ be an associative class and $x,y$ in $A$ with identity $e$. Then
\begin{enumerate}
\item if $x$ has a left inverse, it is left-cancellative;
\item if $x$ has a right inverse, it is right-cancellative;
\item if $x$ has a left inverse and is right-cancellative, it is invertible;
\item if $x$ has a right inverse and is left-cancellative, it is invertible.
\end{enumerate}
\end{lemma}
\begin{proof}
(1) Let $l$ be a left inverse of $x$ and assume $f(x, z_1) = f(x,z_2)$, then $f(l, f(x,z_1)) = f(l, f(x,z_2))$ and thus
\[ z_1 = f(e,z_1) = f(f(l,x), z_1) = f(l, f(x,z_1)) = f(l, f(x,z_2)) = f(f(l,x), z_2) = f(e,z_2) = z_2. \]

(2) Similar.

(3) Let $l$ be a left inverse of $x$. It is enough to show that $l$ is also a right inverse of $x$. We calculate
\[ f(f(x,l), x) = f(x, f(l,x)) = f(x, e) = x = f(e,x). \]
Beacuse $x$ is right-cancellative, this means $f(x,l) = e$ and thus that $l$ is a right inverse.

(4) Similar.
\end{proof}


\section{Left and right relation}
\begin{definition}
Let $\sSet{A, f}$ be an associative class and $x, y\in A$. Then
\begin{itemize}
\item let $L$ be the relation defined by $xLy \defequiv \exists a: f(a, x) = y$;
\item let $R$ be the relation defined by $xRy \defequiv \exists a: f(x, a) = y$;
\item let $L'$ be the relation defined by $xL^!y \defequiv  \exists! a: f(a, x) = y$;
\item let $R'$ be the relation defined by $xR^!y \defequiv \exists! a: f(x, a) = y$.
\end{itemize}
\end{definition}

The relations $L$ and $R$ are dual. The relations $L^!$ and $R^!$ are dual.

\begin{lemma}
The relations $L$ and $R$ are transitive.
\end{lemma}
\begin{proof}
Let $\sSet{A, f}$ be an associative class and $x, y, z\in A$ such that $xLy$ and $yLz$. Then there exist $a,b\in A$ such that $f(a,x) = y$ and $f(b,y) = z$. Then
\[ z = f(b,y) = f\big(b,f(a,x)\big) = f\big(f(b,a), x\big), \]
so $xLz$. The statement for $R$ is dual.
\end{proof}

\begin{lemma}
Let $\sSet{A, f}$ be an associative class and $x\in A$. Then the following are equivalent:
\begin{enumerate}
\item $x$ is right-cancellative
\item $xL \subseteq xL^!$;
\item $xL = xL^!$.
\end{enumerate}
As are the following:
\begin{enumerate}
\item $x$ is left-cancellative
\item $xR \subseteq xR^!$;
\item $xR = xR^!$.
\end{enumerate}
\end{lemma}
\begin{proof}
$(1) \Rightarrow (2)$ Take $y\in xL$. Then there exists $a\in A$ such that $f(a,x) = y$. Assume there exists another $b\in A$ such that $f(b,x) = y$. Then $f(a,x) = f(b,x)$, so $a=b$. Thus the $a\in A$ is unique and $y\in xL^!$.

$(2) \Rightarrow (3)$ The inclusion $xL \supseteq xL^!$ is immediate.

$(3) \Rightarrow (1)$ Take $a,b\in A$ and assume $f(a,x) = f(b,x) = y$. Then $xLy$, so $xL^!y$, so $a=b$.

The proof of the second part is dual.
\end{proof}

\begin{lemma}
Let $\sSet{A, f}$ be an associative class. Then
\begin{enumerate}
\item $L = \bigcup_{a\in A}f(a, -)$;
\item $R = \bigcup_{a\in A}f(-, a)$.
\end{enumerate}
\end{lemma}

For given and fixed $f$, we will often write
\begin{itemize}
\item $\lambda_a: A\to A$ for $f(a, -)$;
\item $\rho_a: A\to A$ for $f(-, a)$.
\end{itemize}

\begin{lemma} \label{lambdaRhoCommute}
For all $a,b\in A$, we have $\lambda_a\circ \rho_b = \rho_b \circ \lambda_a$.
\end{lemma}
\begin{proof}
We calculate, for arbitrary $x\in A$,
\[ \lambda_a\big(\rho_b(x)\big) = f\big(a, f(x,b)\big) = f\big(f(a,x), b\big)  = \rho_b\big(\lambda_a(x)\big). \]
\end{proof}
\begin{corollary} \label{LRcommute}
Let $A$ be a class and $f: A\times A \to A$ an associative binary function. Then $L;R = R;L$.
\end{corollary}
\begin{proof}
Let $x,y\in A$. Then $x(L;R)y$ iff there exist $a,b\in A$ such that the top path in
\[ \begin{tikzcd}
x \ar[r, maps to, "\lambda_a"] \ar[d, maps to, swap, "\rho_b"] & f(a, x) \ar[d, maps to, "\rho_b"] \\
f(x, b) \ar[r, maps to, swap, "\lambda_a"] & y
\end{tikzcd} \]
holds. By \ref{lambdaRhoCommute} this is equivalent to the bottom path holding. The bottom path implies $x(R;L)y$.
\end{proof}

\begin{proposition} \label{functionsLeftRightRelations}
Let $g,h$ be functions in the assocative class of functions with composition $\circ$. Then
\begin{enumerate}
\item $gLh$ \textup{if and only if} $\ker g \subseteq \ker h$;
\item $gRh$ \textup{if and only if} $\im g \supseteq \im h$. 
\end{enumerate}
\end{proposition}

\subsection{Principal ideals}
\begin{definition}
Let $\sSet{A, f}$ be an associative class and $x, y\in A$. Then
\begin{itemize}
\item the \udef{left principal ideal} generated by $x$ is $f(\widetilde{A}, x) = f(A, x)\cup \{x\}$;
\item the \udef{right principal ideal} generated by $x$ is $f(x, \widetilde{A}) = f(x,A)\cup \{x\}$.
\end{itemize}
\end{definition}

\begin{lemma} \label{idealAbsorption}
Let $\sSet{A, f}$ be an associative class and $x, y\in A$. Then
\begin{enumerate}
\item $f(A, f(x,y)) \subseteq f(A, y)$;
\item $f(f(x,y), A) \subseteq f(x, A)$.
\end{enumerate}
\end{lemma}
\begin{proof}
(1) Take $z\in f(A, f(x,y))$. Then there exists $z'\in A$ such that
\[ z = f(z', f(x,y)) = f(f(z', x), y) \in f(A, y). \]

(2) By duality.
\end{proof}

\begin{lemma}
Let $\sSet{A, f}$ be an associative class and $x, y\in A$. Then
\begin{enumerate}
\item $f(A, x) \subseteq f(A,y)$ \textup{if and only if} $\forall a\in A: \exists b\in A: \; f(a, x) = f(b, y)$;
\item $f(x, A) \subseteq f(y, A)$ \textup{if and only if} $\forall a\in A: \exists b\in A: \; f(x, a) = f(y, b)$.
\end{enumerate}
If $A$ contains an identity $e$, then
\begin{enumerate} \setcounter{enumi}{2}
\item $f(A, x) \subseteq f(A,y)$ \textup{if and only if} $\exists b\in A: \; x = f(b, y)$;
\item $f(x, A) \subseteq f(y, A)$ \textup{if and only if} $\exists b\in A: \; x = f(y, b)$.
\end{enumerate}
\end{lemma}

\begin{proposition} \label{invertibilityFromPrincipalIdeals}
Let $\sSet{A, f}$ be an associative class. Then $f$ has an identity and every $x\in A$ is invertible \textup{if and only if}
\[ \forall x\in A: \quad f(x, A) = A = f(A,x). \]
\end{proposition}
\begin{proof}
$\Rightarrow$ We clearly have $f(x, A) \subseteq A$. The other inclusion follows from \ref{idealAbsorption}: $A = f(A, e) = f(A, f(x^{-1}, x)) \subseteq f(A, x)$.

$\Leftarrow$ Pick some $x\in A$, so $x\in f(x,a)$, meaning there exists an $a\in A$ such that $x = xa$. We claim $a$ is a right-identity for $f$. Indeed, take arbitrary $y\in A$. Then $y = f(b,x)$ for some $b\in A$ and so
\[ f(y, a) = f(f(b,x), a) = f(b,f(x,a)) = f(b,x) = y. \]
In the same way we can also find a left-identity. So $A$ contains an identity $e \defeq a$ by \ref{leftRightIdentity}.

Now for all $x\in A$ we have $e\in A = f(x,A)$, so we can find a right-inverse of $x$. Similarly, we can find a left-inverse of $x$. This means $x$ is invertible by \ref{leftRightInverse}.
\end{proof}


\subsection{Green's relations}
\begin{lemma}
Let $\sSet{A, f}$ be an associative class. Then $\greensL$ is the reflexive closure of the symmetric part of $L$ and $\greensR$ is the reflexive closure of the symmetric part of $R$.
\end{lemma}

\begin{lemma}
Let $\sSet{A, f}$ be an associative class and $x, y\in A$. Then the following are equivalent:
\begin{enumerate}
\item $x \greensL y$;
\item $x\big((L\cap L^\transp) \cup \id_A\big)y$
\item $\exists a,b\in \widetilde{A}: (f(a, x) = y) \land (f(b, y) = x)$;
\item $\begin{tikzcd}[sep=large]
x \ar[r, maps to, shift left, "\exists a: \lambda_a"] & y \ar[l, maps to, shift left, "\exists b: \lambda_b"]
\end{tikzcd}$
\item $f(\widetilde{A},x) = f(\widetilde{A}, y)$;
\item $f(A,x) = f(A, y)$;
\end{enumerate}
as are
\begin{enumerate}
\item $x \greensR y$;
\item $x\big((R\cap R^\transp) \cup \id_A\big)y$
\item $\exists a,b\in \widetilde{A}: (f(x, a) = y) \land (f(y, b) = x)$;
\item $\begin{tikzcd}[sep=large]
x \ar[r, maps to, shift left, "\exists a: \rho_a"] & y \ar[l, maps to, shift left, "\exists b: \rho_b"]
\end{tikzcd}$
\item $f(x, \widetilde{A}) = f(y, \widetilde{A})$;
\item $f(x, A) = f(y, A)$.
\end{enumerate}
\end{lemma}


\subsubsection{Egg-box diagrams}


From \ref{commutingEquivalenceRelations} we also know that the $\greensL,\greensR$-egg-box diagram decomposes into blocks, which are $\greensD$-equivalence classes. The columns are $\greensL$-equivalence classes, the rows are $\greensR$-equivalence classes, and the cells are $\greensH$-equivalence classes.

\begin{example}
Let $A = (\{1,2,3\} \to \{1,2,3\})$ with the binary function $A\times A \to A: (f,g)\mapsto f;g$. We can represent an element $f$ of $A$ as $(f(1) f(2) f(3))$. We have
\begin{itemize}
\item $f\greensL g$ if $f$ and $g$ have the same image;
\item $f\greensR g$ if $f$ and $g$ have the same kernel.
\end{itemize}
An egg-box diagram can be drawn as follows:
\[ \begin{array}{|c|c|c|c|c|c|c|}
\hline
\mathbf{(1 1 1)} & \mathbf{(2 2 2)} & \mathbf{(3 3 3)} &&& &  \\ \hline
&& & \mathbf{(1 2 2)}, & \mathbf{(1 3 3)}, & (2 3 3), &  \\
&& & (2 1 1)  & (3 1 1)  & (3 2 2)  &  \\ \hline
&& & (2 1 2), & (3 1 3), & \mathbf{(3 2 3)},  &  \\
&& & \mathbf{(1 2 1)}  & (1 3 1)  & (2 3 2)  &  \\ \hline
&& & (2 2 1), & (3 3 1), & (3 3 2),  &  \\
&& & (1 1 2)  & \mathbf{(1 1 3)}  & \mathbf{(2 2 3)}  &  \\ \hline
&& &&& & \mathbf{(1 2 3)}, (2 3 1), (3 1 2)  \\
&& &&& & (1 3 2), (2 1 3), (3 2 1)  \\ \hline
\end{array} \]
The bold elements are idempotents.
\end{example}

\subsubsection{Greens theorem}

\begin{lemma}
Let $\sSet{A, f}$ be an associative class and $x\in A$. If $x$ is an idempotent, then
\[ \lambda_x|_{[x]_\greensR} = \id_{[x]_\greensR} \qquad \text{and} \qquad \rho_x|_{[x]_\greensL} = \id_{[x]_\greensL}. \]
Thus $x$ is a left identity for $[x]_\greensR$ and a right identity for $[x]_\greensL$.
\end{lemma}
\begin{proof}
Take $y\in [x]_\greensR$. Then there exist $a,b\in \widetilde{A}$ such that $\begin{tikzcd}
x \ar[r, maps to, shift left, "\rho_a"] & y \ar[l, maps to, shift left, "\rho_b"]
\end{tikzcd}$. Then we have
\[ xy = xxb = xb = y. \]
The other claim is dual.
\end{proof}

\begin{theorem}[Green's theorem]
Let $\sSet{A, f}$ be an associative class and $H$ an $\mathcal{H}$-class in $A$. Then either
\begin{enumerate}
\item $H^2\perp H$; or
\item $H^2 = H$, $f|_{H\times H}$ has an identity and each $x\in H$ is invertible.
\end{enumerate}
\end{theorem}
\begin{proof}
Suppose $H^2\cap H \neq \emptyset$, then there exist $a,b\in H$ such that $ab = c\in H$. By the Green's lemma \ref{GreensLemma} we have that $\rho_b:H\to H$ and $\lambda_a: H\to H$ are bijections.

Then for all $h\in H$, $\rho_b(h) = hb \in H$. Again by the Green's lemma, this means that $\lambda_h: H\to H$ is a bijection. Similarly $\rho_h: H\to H$ is a bijection for all $h$. So for all $h\in H$ we have $hH = H = Hh$. The result follows by \ref{invertibilityFromPrincipalIdeals}.
\end{proof}
\begin{corollary} \label{GreensTheoremCorollary}
Let $\sSet{A, f}$ be an associative class and $H$ an $\mathcal{H}$-class in $A$. Then
\begin{enumerate}
\item if $x$ is an idempotent in $H$, then we have the second case;
\item no $\mathcal{H}$-class can contain more than one idempotent. 
\end{enumerate}
\end{corollary}

\subsection{Regular elements and generalised inverses}
\begin{definition}
Let $\sSet{A, f}$ be an associative class and $x, y\in A$. We call
\begin{itemize}
\item $x$ \udef{regular} if $\exists a\in A: \; x = xax$
\item $x$ and $y$ \udef{generalised inverses} if $x = xyx$ and $y = yxy$.
\end{itemize}
\end{definition}

\begin{proposition} 
Let $\sSet{A, f}$ be an associative class. If $x\in A$ is regular, then every element in $[x]_{\greensD}$ is regular.
\end{proposition}
So it makes sense to call a $\greensD$-class \udef{regular} if it consists of regular elements and \udef{irregular} otherwise.
\begin{proof}
Let $x$ be regular with $x = xx'x$ and $x\greensD y$. Then we have $a,b,c,d \in \widetilde{A}$ such that
$\lambda_a\circ\rho_b|_{[x]_\greensH}: [x]_\greensH \to [y]_\greensH$ is a bijection with inverse $\lambda_c\circ \rho_d|_{[y]_\greensH}: [y]_\greensH \to [x]_\greensH$, as in \ref{greensDisomorphism}. Then we have
\[ y = (\lambda_a\circ\rho_b)(x) = (\lambda_a\circ\rho_b)(xx'x) = axx'xb = a(\lambda_c\circ\rho_d)(y)x'(\lambda_c\circ\rho_d)(y)b = ydx'cy. \]
So $y$ is regular.
\end{proof}
\begin{corollary}
If there is an idempotent $x\in [a]_{\greensD}$, then $[a]_{\greensD}$ is regular.
\end{corollary}
\begin{proof}
An idempotent is regular: $x = f(x,x) = f(f(x,x), x)$.
\end{proof}

\begin{proposition}
Let $\sSet{A, f}$ be an associative class. Then $x$ is regular \textup{if and only if} it has a generalised inverse.
\end{proposition}
\begin{proof}
Clearly every element with a generalised inverse is regular. Conversely, assume $x$ regular with $x = xax$. Then $y = axa$ is a generalised inverse of $x$: $x(axa)x = xax = x$ and $(axa)x(axa) = a(xax)axa = axaxa = axa$.
\end{proof}
Note that we do not have that $x = xyx$ implies $y = yxy$.

\begin{proposition} \label{greensRelationsRegularElements}
Let $\sSet{A, f}$ be an associative class and $x\in A$ a regular element with $x = f(f(x, y), x)$. Then
\begin{enumerate}
\item $f(x,y)$ and $f(y,x)$ are idempotent;
\item $f(y,x) \greensL x$ and $x \greensR f(x, y)$.
\end{enumerate}
\end{proposition}
\begin{proof}
(1) We calculate
\[ f(f(x,y), f(x,y)) = f(f(f(x,y), x), y) = f(x, y)\] and \[f(f(y,x), f(y,x)) = f(y, f(x, f(y,x))) = f(y,x). \]

(2) Using \ref{idealAbsorption}, we have
\[ f(A, x) = f(A, f(x,f(y,x))) \subseteq f(A, f(y,x)) \subseteq f(A,x). \]
Thus $f(A, x) = f(A, f(y,x))$. The second part is dual.
\end{proof}
\begin{corollary}
In a regular $\mathcal{D}$-class each $\mathcal{L}$-class and each $\mathcal{R}$-class contains at least one idempotent.
\end{corollary}
\begin{proof}
Let $[x]_\mathcal{L}$ be an $\mathcal{L}$-class in a regular $\mathcal{D}$-class. By regularity there exists a $y\in A$ such that $xyx = x$. From the proposition, we have that $[x]_\mathcal{L}$ contains the idempotent $yx$ and $[x]_\mathcal{R}$ the idempotent $xy$.
\end{proof}
\begin{corollary} \label{idempotentsHclass}
If $x,x'\in A$ are generalised inverses, then $[xx']_\mathcal{H} = [x]_\mathcal{R}\cap [x']_\mathcal{L}$ and $[x'x]_\mathcal{H} = [x']_\mathcal{R}\cap [x]_\mathcal{L}$.
\end{corollary}
\begin{proof}
From $x\greensR (xx')$ and $(xx')\greensL x'$, we get the first equality. The second is dual.
\end{proof}
We can depict the situation in the corollary as follows:
\[ \hspace{-8.4em} \exists a,b,c,d \in \widetilde{A}: \qquad \begin{tikzcd}[sep=large]
x \ar[r, maps to, shift left, "\rho_a"] \ar[d, maps to, shift left, "\lambda_c"] & xx' \ar[l, maps to, shift left, "\rho_b"] \ar[d, maps to, shift left, "\lambda_c"] \\
x'x \ar[u, maps to, shift left, "\lambda_d"] \ar[r, maps to, shift left, "\rho_a"] & x' \ar[u, maps to, shift left, "\lambda_d"] \ar[l, maps to, shift left, "\rho_b"]
\end{tikzcd} \]

So generalised inverses along one diagonal imply idempotents along the other. In fact, the other direction also holds:
\begin{proposition}
Let $\sSet{A, f}$ be an associative class and $e,f$ idempotents in $A$ such that $e\greensD f$. Then there exist $x\in [e]_\greensR\cap [f]_\greensL$ and $x'\in [e]_\greensL\cap [f]_\greensR$ such that
\begin{itemize}
\item $x,x'$ are generalised inverses;
\item $e = xx'$ and $f = x'x$.
\end{itemize}
\end{proposition}
\begin{proof}
Because $e\greensD f$, we can find $x,x'\in A$ such that
\[ \hspace{-8.4em} \exists a,b,c,d \in \widetilde{A}: \qquad \begin{tikzcd}[sep=large]
e \ar[r, maps to, shift left, "\rho_a"] \ar[d, maps to, shift left, "\lambda_c"] & x \ar[l, maps to, shift left, "\rho_b"] \ar[d, maps to, shift left, "\lambda_c"] \\
x' \ar[u, maps to, shift left, "\lambda_d"] \ar[r, maps to, shift left, "\rho_a"] & f \ar[u, maps to, shift left, "\lambda_d"] \ar[l, maps to, shift left, "\rho_b"]
\end{tikzcd} \]
Then
\begin{align*}
xx' &= (df)(fb) = dfb = e \\
x'x &= (ce)(ea) = cea = f \\
xx'x &= ex = eea = ea = x \\
x'xx' &= fx' = ffb = fb = x',
\end{align*}
which completes the proof.
\end{proof}
\begin{corollary}
Let $\sSet{A, f}$ be an associative class, $y\in A$ and $e,f$ idempotents in $A$. Then $e\greensD f$ \textup{if and only if} there exist generalised inverses $x,x'$ such that $e = xx'$ and $f = x'x$.
\end{corollary}
\begin{proof}
The direction $\Rightarrow$ follows from the proposition. The converse from \ref{greensRelationsRegularElements}.
\end{proof}

\begin{lemma}
Let $\sSet{A, f}$ be an associative class, $x\in A$. Then no $\greensH$-class contains more than one generalised inverse of $x$.
\end{lemma}
\begin{proof}
Assume $x$ has two generalised inverses, $x_1'$ and $x_2'$. From \ref{idempotentsHclass} and \ref{GreensTheoremCorollary} we get that $xx_1' = xx_2'$ and $x_1'x = x_2'x$. Thus
\[ x_1' = x_1'(xx_1') = x_1'xx_2' = (x_1'x)x_2' =  x_2'xx_2' = x_2'. \]
\end{proof}

\begin{proposition}
Let $\sSet{A, f}$ be an associative class and $x,y\in A$. Then $xy\in [x]_\greensR \cap [y]_\greensL$ \textup{if and only if} $[x]_\greensL \cap [y]_\greensR$ contains an idempotent.
\end{proposition}
\begin{proof}
First assume $[x]_\greensL \cap [y]_\greensR$ contains an idempotent $e$. We can depict the situation as
\[ \hspace{-8.4em} \exists a,b,c,d \in \widetilde{A}: \qquad \begin{tikzcd}[sep=large]
x \ar[r, maps to, shift left, "\rho_a"] \ar[d, maps to, shift left, "\lambda_c"] &  \ar[l, maps to, shift left, "\rho_b"] \ar[d, maps to, shift left, "\lambda_c"] \\
e \ar[u, maps to, shift left, "\lambda_d"] \ar[r, maps to, shift left, "\rho_a"] & y \ar[u, maps to, shift left, "\lambda_d"] \ar[l, maps to, shift left, "\rho_b"]
\end{tikzcd} \]
Then we can calculate
\[ xy = deea = dea = xa \in [x]_\greensR \cap [y]_\greensL. \]
Now assume $xy\in [x]_\greensR \cap [y]_\greensL$.
We can depict the situation as
\[ \hspace{-8.4em} \exists a,b,c,d \in \widetilde{A}: \qquad \begin{tikzcd}[sep=large]
x \ar[r, maps to, shift left, "\rho_a"] \ar[d, maps to, shift left, "\lambda_c"] & xy \ar[l, maps to, shift left, "\rho_b"] \ar[d, maps to, shift left, "\lambda_c"] \\
e \ar[u, maps to, shift left, "\lambda_d"] \ar[r, maps to, shift left, "\rho_a"] & y \ar[u, maps to, shift left, "\lambda_d"] \ar[l, maps to, shift left, "\rho_b"]
\end{tikzcd} \]
Now we need to show that $e$ is idempotent. Indeed, starting from the three other corners, we see that $e = cx$ and $e = yb$ and $e = c(xy)b = (cx)(yb) = ee$.
\end{proof}

\subsection{Commutation}
\begin{definition}
Let $\sSet{A,f}$ be an associative class and $x,y \in A$. We say $x$ and $y$ \udef{commute} if $f(x,y) = f(y,x)$. We write $x\commute y$.
\end{definition}


\subsubsection{Centraliser or commutant}
\begin{definition}
Let $\sSet{A,f}$ be an associative class and $B\subseteq A$ a subclass. The \udef{centraliser} or \udef{commutant} of $B$ is defined as $Z_A(B) \defeq B^\commute$.

In particular we define the \udef{centre} of $A$ as the centraliser of all of $A$: $Z_A \defeq Z_A(A)$.
\end{definition}
Thus
\[ Z_A(B) = \setbuilder{x\in A}{\forall b\in B:\;f(x,b) = f(b,x)}. \].

Note that taking the commuting forms a Galois connection. In particular $B \subseteq B^{\commute\commute}$.

\begin{lemma}
Let $\sSet{A,f}$ be an associative class and $B\subseteq A$. If $f$ is commutative, then $Z_A(B) = A$.
\end{lemma}

\begin{proposition}
Let $\sSet{A,f}$ be an associative class and $B\subseteq A$. Then $Z_A(B)$ is closed under $f$.
\end{proposition}
\begin{proof}
Take arbitrary $x,y\in Z_A(B)$. Take arbitrary $b\in B$. Then
\[ f(f(x,y),b) = f(x,f(y,b)) = f(x,f(b,y)) = f(f(x,b),y) = f(f(b,x),y) = f(b,f(x,y)), \]
which means that $f(x,y)\in Z_A(B)$.
\end{proof}

\subsection{Normaliser}
\begin{definition}
Let $\sSet{A,f}$ be an associative class and $B\subseteq A$ a subclass. An element $x\in A$ is said to \udef{normalise} $B$ if $f(x,B) = f(B,x)$.

The \udef{normaliser} of $B$ in $A$ is the set of all elements in $A$ that normalise $B$:
\[ N_A(B) \defeq \setbuilder{x\in A}{f(x,B) = f(B,x)}. \]
\end{definition}

\begin{proposition}
Let $\sSet{A,f}$ be an associative class and $B\subseteq A$. Then
\begin{enumerate}
\item $N_A(B)$ is closed under $f$;
\item $Z_A(B) \subseteq N_A(B)$;
\item $Z_A(\{a\}) = N_A(\{a\})$ for all $a\in A$.
\end{enumerate}
\end{proposition}
\begin{proof}
(1) Take arbitrary $x,y\in N_A(B)$. Then
\[ f(f(x,y),B) = f(x,f(y,B)) = f(x,f(B,y)) = f(f(x,B),y) = f(f(B,x),y) = f(B,f(x,y)), \]
which means that $f(x,y)\in N_A(B)$.

(2) If $f(x,b) = f(b,x)$ for all $b\in B$, then $f(x,B) = f(B,x)$.

(3) $f(x,\{a\}) = f(x,a)$ and $f(\{a\}, x) = f(a,x)$.
\end{proof}

\subsection{Composition of relations}
\begin{definition}
Let $\mathfrak{R}$ be the class of all relations, with absorbing element $\mathrm{NULL}$ adjoined.
\end{definition}

\begin{proposition}
Let $R,S \in \mathfrak{R}$ be relations. Then
\begin{enumerate}
\item $R \greensL S$ \textup{if and only if} $R$ and $S$ have the same image and cokernel;
\item $R \greensR S$ \textup{if and only if} $R$ and $S$ have the same preimage and kernel.
\end{enumerate}
\end{proposition}
\begin{proof}
TODO
\end{proof}

\part{Category Theory}
\setcounter{chapter}{0} % Reset chapter counter
\url{https://arxiv.org/pdf/1912.10642.pdf}
\url{https://arxiv.org/pdf/0810.1279.pdf}
\url{http://katmat.math.uni-bremen.de/acc/acc.pdf}

\chapter{Basic concepts}
\section{Categories}
\subsection{Definitions and examples}
\begin{definition}
A \udef{category} $\cat{C}$ consists of
\begin{enumerate}
\item a collection $\ob(\cat{C})$ of \udef{objects} $X,Y,Z,\ldots$
\item a collection $\mor(\cat{C})$ of \udef{morphisms} or \udef{arrows} $f,g,h,\ldots$
\end{enumerate}
such that
\begin{itemize}
\item each morphism $f$ has an associated \udef{domain} object $\dom(f)$ and \udef{codomain} object $\codom(f)$; we write
\[ f: X \to Y \qquad\text{or}\qquad \begin{tikzcd}
X \rar{f} & Y
\end{tikzcd}  \]
to mean $f$ is a morphism with $\dom(f) = X$ and $\codom(f) = Y$;
\item each object has a designated \udef{identity morphism} $\id_X: X\to X$;
\item for any pair $f,g$ such that $\codom(f) = \dom(g)$, there exists a \udef{composite morphism}
\[ g\circ f: \dom(f) \to \codom(g); \]
we call such pairs \udef{composable}; any sequence of morphisms may be called composable if each morphism is composable with its neighbours;
\end{itemize}
subject to the axioms
\begin{itemize}[leftmargin=3cm]
\item[\textbf{Identity}] for any $f: X\to Y$, the composites $\id_Y\circ f = f\circ \id_X = f$;
\item[\textbf{Associativity}] for any composable triple $f,g,h$ we have
\[ h\circ (g\circ f) = (h\circ g)\circ f \eqdef h\circ g \circ f. \]
\end{itemize}
We also write $gf$ or $g\cdot f$ instead of $gf$.
\end{definition}

Two arrows are called \udef{parallel} if they have the same domain and codomain.

The notation heavily suggests that we can think of the objects as sets and the morphisms as functions. Such categories do indeed constitute the main examples of categories. They are called \udef{concrete categories}. The other categories are \udef{abstract categories}.

TODO: concrete categories are categories with a faithful functor to $\cat{Set}$?

\begin{example}
Some examples of concrete categories are
\begin{itemize}
\item All sets are objects in the category $\cat{Set}$ and all functions are morphisms.
\item The category $\cat{Poset}$ has partially ordered sets as objects and order-preserving functions as morphisms.
\end{itemize}
Some examples of abstract categories are
\begin{itemize}
\item A set may be regarded as a category in which the elements of the set define the objects and the only morphisms are the required identities. A category is \udef{discrete} if every morphism is an identity.
\item A preorder $(P, \precsim)$ can be regarded as a category. The elements of $P$ are the objects of the category and for $x,y\in P$ there exists a unique morphism $x\to y$ if and only if $x\precsim y$. Transitivity implies the existence of composites and reflexivity the existence of identity morphisms.
\item In particular (TODO: Von Neumann??) ordinals are preorders and thus define categories. We use the notation $\mathbb{0}, \mathbb{1}, \mathbb{2}, \ldots$ for the categories defined by $0,1,2,\ldots$. Also $\bbomega$ is the category defined by $\omega$.
\end{itemize}
\end{example}

If the category is small enough, we can depict it using a diagram. Points are objects and arrows are morphisms.

\begin{example}
\begin{align*}
\mathbb{1}&: \begin{tikzcd}
0 \arrow[loop left]
\end{tikzcd} \\
\mathbb{2}&: \begin{tikzcd}[ampersand replacement=\&]
0 \arrow[loop left] \ar[r] \& 1 \arrow[loop right]
\end{tikzcd} & &\text{\udef{walking arrow} or \udef{free arrow}} \\
\mathbb{I}&: \begin{tikzcd}[ampersand replacement=\&]
0 \arrow[loop left] \ar[r, shift left] \& 1 \ar[l, shift left] \arrow[loop right]
\end{tikzcd} & &\text{\udef{walking isomorphism} or \udef{free isomorphism}} \\
\mathbb{3}&: \begin{tikzcd}[ampersand replacement=\&]
0 \arrow[loop left] \ar[r] \ar[rr, bend left] \& 1 \ar[r] \arrow[loop below] \& 2 \arrow[loop right]
\end{tikzcd}
\end{align*}
\end{example}

\begin{lemma} \label{arrowsOnlyDefinition}
We can define the notion of category without referencing objects, as each object can be identified with its identity morphism. The definition then becomes:

\upshape
A category $\cat{C}$ consists of
\begin{enumerate}
\item a collection $\mor(\cat{C})$ of arrows $f,g,h,\ldots$
\item a collection of pairs of arrows, called \udef{composable pairs}
\item an operation assigning each composable pair to an arrow
\[ (g,f) \mapsto gf \]
if $(g,f)$ is a composable pair, we say $gf$ is defined.
\end{enumerate}
We call an arrow $I$ and \udef{identity arrow} if for all arrows $f$ in $\cat{C}$,
\begin{align*}
If &= f & &\text{whenever $If$ is defined} \\
fI &= f & &\text{whenever $fI$ is defined.}
\end{align*}
The category must then satisfy the axioms
\begin{itemize}[leftmargin=3cm]
\item[\textbf{Associativity}] for any composable triple $f,g,h$ we have
\begin{align*}
\text{$hg$ and $gf$ are defined} &\iff \text{$hg$ and $(hg)f$ are defined} \\ &\iff \text{$gf$ and $h(gf)$ are defined}
\end{align*}
and if this holds, then
\[ h(gf) = (hg)f \eqdef hgf; \]
\item[\textbf{Identity}] for each arrow $f$ there exist identity arrows $I,I'$ in $\cat{C}$ such that $If$ and $fI'$ are defined.
\end{itemize}
\end{lemma}


\subsection{Questions of size: are categories graphs?}
Based on all this we may think a category is a special type of graph, and this would be true if it were not for the question of size. A graph has a \textit{set} of points and a \textit{set} of edges. The collections of objects and morphisms in a category do \emph{not} need to be sets. Indeed the collection of objects in the category $\cat{Set}$ is the collection of all sets, which is not a set by \ref{setOfSets}.

\begin{definition}
A category $\cat{C}$ is \udef{small} if its collection of morphisms is a set.

A category $\cat{C}$ is \udef{locally small} for any objects $X,Y$ if the collection of morphisms from $X$ to $Y$ is a set. This set is called a \udef{hom-set} and is denoted $\Hom(X,Y)$ or $\cat{C}(X,Y)$.
\end{definition}
In view of \ref{arrowsOnlyDefinition}, a small category has only a set of objects.

\begin{lemma}
If a category $\cat{C}$ is small, then
\begin{align*}
\dom&: \mor(\cat{C}) \to \ob(\cat{C}) \\
\codom&: \mor(\cat{C}) \to \ob(\cat{C}) \\
\id&: \ob(\cat{C}) \to \mor(\cat{C}): X\mapsto \id_X
\end{align*}
are functions.
\end{lemma}

\begin{lemma}
A small category
\begin{itemize}
\item with at most one element in every hom-set defines a preorder;
\item such that for all objects $X,Y$, the set $\Hom(X,Y)\cup \Hom(Y,X)$ is at most a singleton defines a partial order.
\end{itemize}
\end{lemma}
We call such categories \udef{preorder categories} and \udef{poset categories}.

\begin{definition}
A category with at most one morphism in each category is called \udef{thin}.
\end{definition}

\subsection{Duality}
Intuitively category-theoretic duality is obtained by ``flipping all the arrows''.
\begin{definition}
Let $\cat{C}$ be a category. The \udef{opposite category} $\cat{C^{op}}$ has
\begin{enumerate}
\item the objects of $\cat{C}$ as objects;
\item a morphism $f^\text{o}: Y\to X$ for each morphism $f:X\to Y$ in $\cat{C}$.
\end{enumerate}
And
\begin{itemize}
\item for each object $X$, $\id_X^\text{o}$ serves as the identity $\id_X$ in $\cat{C^{op}}$;
\item for all composable $f,g$ in $\mor(\cat{C})$, $(fg)^\text{o} = g^\text{o}f^\text{o}$.
\end{itemize}
\end{definition}
It is easy to verify that the opposite category is indeed a category.

\begin{lemma}
The category $(\cat{C^{op})^{op}}$ is again the category $\cat{C}$.
\end{lemma}

\begin{lemma}
Let $f,g$ be morphisms, then $f = g \iff f^o = g^o$.
\end{lemma}

A statement about the category $\cat{C^{op}}$ can often be interpreted as a statement about $\cat{C}$. This is referred to as the ``dual statement''. In other words: a statement in $\cat{C}$is true if and only if the dual statement is true in $\cat{C^{op}}$.

If we prove that something is true in a collection of categories including $\cat{C}$ and $\cat{C^{op}}$, we have in effect proven both the statement and the dual statement. We then say the second follows from the first by duality. For an example see \ref{singleSplitImpliesDoubleSplit} and \ref{monicEpicCompositions}.

We also say properties of morphisms and categories are dual if having one in $\cat{C}$ means having the other in $\cat{C^{op}}$.

\begin{example}
Using the fact that monic and epic are dual properties of morphisms (see later), we have, for example, the following implications:
\begin{quote}
For any composable morphisms $g,f$ in any category $\cat{C}$: If $gf$ is monic, then $f$ is monic.
\end{quote}
Implies
\begin{quote}
For any composable morphisms $g^\text{o},f^\text{o}$ in any category $\cat{C^{op}}$: If $g^\text{o}f^\text{o}$ is monic, then $f^\text{o}$ is monic.
\end{quote}
Implies
\begin{quote}
For any composable morphisms $g^\text{o},f^\text{o}$ in any category $\cat{C^{op}}$: If $(fg)^\text{o}$ is monic, then $f^\text{o}$ is monic.
\end{quote}
Implies
\begin{quote}
For any composable morphisms $g,f$ in any category $\cat{C}$: If $fg$ is epic, then $f$ is epic.
\end{quote}
So proving the first also proves the last, by duality.
\end{example}



\subsection{Constructions of categories}
\subsubsection{Subcategories}
\begin{definition}
Let $\cat{C}$ be a category. A category $\cat{D}$ is a \udef{subcategory} of $\cat{C}$ if all its objects are objects of $\cat{C}$ and all its morphisms are morphisms of $\cat{C}$.
\end{definition}

\subsubsection{Core and skeleton}
\begin{definition}
Let $\cat{C}$ be a category.
\begin{itemize}
\item The \udef{core} of $\cat{C}$ is the groupoid whose objects are the objects of $\cat{C}$ and whose morphisms are the isomorphisms of $\cat{C}$.
\end{itemize}
\end{definition}

\subsubsection{Product categories}
\begin{definition}
The \udef{product} $\cat{C}\times \cat{D}$ of categories $\cat{C}$ and $\cat{D}$  is a category consisting of
\begin{enumerate}
\item ordered pairs $(c,d)$ where $c\in \cat{C}$ and $d\in \cat{D}$ as objects;
\item ordered pairs $(f,g): (c,d) \to (c',d')$ where $f:c\to c'$ and $g:d\to d'$ as morphisms.
\end{enumerate}
Composition and identities are defined component-wise.
\end{definition}



\section{Functors}
\begin{definition}
A \udef{(covariant) functor} $F:\cat{C}\to\cat{D}$ consists of
\begin{enumerate}
\item an object $Fc$ in $\cat{D}$ for each $c\in C$;
\item a morphism $Ff: Fc\to Fc'$ for each morphism $f:c\to c'$ in $\cat{C}$
\end{enumerate}
satisfying the functorial properties:
\begin{itemize}
\item for any composable pair $g,f$ in $\cat{C}$, $Fg\circ Ff = F(g\circ f)$;
\item for each object $c$ in $\cat{C}$, $F(\id_c) = \id_{Fc}$.
\end{itemize}
A \udef{contravariant functor} $F:\cat{C}\to\cat{D}$ is similar, except
\begin{enumerate}
\item $f:c\to c'$ maps to $Ff: Fc'\to Fc$
\item if $g,f$ are composable, then $Ff \circ Fg = F(g\circ f)$.
\end{enumerate}
An \udef{endofunctor} is a functor between a category and itself.
\end{definition}
Functors between small categories are functions.

\begin{example}
Examples of covariant functors:
\begin{itemize}
\item For any category $\cat{C}$, there is an \udef{identity functor} $I_\cat{C}$ that maps objects and morphisms to themselves.
\item There is an endofunctor $\mathcal{P}:\cat{Set} \to \cat{Set}$ that maps sets to their powerset and functions $f:A\to B$ to their image function $f[\cdot]:\mathcal{P}(A)\to \mathcal{P}(B)$.
\item \udef{Forgetful functors} are functors from the category of some type of structured set to a category of structured set with less (or) no structure. For example, we have a forgetful functor $U: \cat{Poset} \to \cat{Set}$ that maps the cateory $\cat{Poset}$ into the category $\cat{Set}$ by forgetting the order. 
\end{itemize}
Examples of contravariant functors:
\begin{itemize}
\item The contravariant endofunctor $\mathcal{P}:\cat{Set} \to \cat{Set}$ that maps sets to their powerset and functions $f:A\to B$ to their inverse image function $f^{-1}[\cdot]:\mathcal{P}(B)\to \mathcal{P}(A)$.
\end{itemize}
\end{example}

\begin{lemma} \label{functorMorphismPreservation}
Functors preserve split monomorphisms, split epimorphisms and isomorphisms.
\end{lemma}
Functors do not necessarily preserve monomorphisms and epimorphisms.
\begin{corollary}
Let $F:\cat{C}\to \cat{D}$ be a functor. Let $f:X \to Y$ be monic (resp. epic). If $F(f)$ is not monic (resp. epic), then $f$ is not split monic (resp. split epic).
\end{corollary}

\subsection{Functors as morphisms}
\begin{definition}
We define $\cat{Cat}$ as the category whose objects are small categories and whose morphisms are functors between them.

We define $\cat{CAT}$ as the category whose objects are locally small categories and whose morphisms are functors between them.
\end{definition}
\begin{lemma}
$\cat{Cat}$ is not small, but it is locally small.

$\cat{CAT}$ is not locally small.
\end{lemma}
There is an inclusion functor $\cat{Cat} \hookrightarrow \cat{CAT}$.

We can obviously consider isomorphisms in the category $\cat{Cat}$, but this is not a very natural notion to work with. For example, a category is not typically isomorphic to its opposite category.

The more natural concept is equivalence of categories.

\begin{lemma} \label{varianceOfComposite}
Let $F,G$ be composable morphisms in $\cat{CAT}$, i.e.\ functors. Then $FG$ is contravariant if exactly one of $F,G$ is contravariant and covariant otherwise.
\end{lemma}

\subsection{Opposites and functors}
\begin{lemma}
For any category $\cat{C}$, taking the opposite is a contravariant functor $o: \cat{C}\to \cat{C^{op}}$ which maps objects to themselves and morphisms $f:X\to Y$ to $f^\text{o}: Y\to X$.

Dually, taking the opposite is also a contravariant functor $o: \cat{C^{op}}\to \cat{C}$.
\end{lemma}

We will often just insert $o$ wherever necessary to make the variances match. For example we can view a functor $F:\cat{C}\to \cat{D}$ as a functor $F:\cat{C^{op}}\to \cat{D^{op}}$ where the latter expression would be more correctly written as
\[ F^o \defeq oFo: \cat{C^{op}}\to \cat{D^{op}} \]

In the same vein, given a contravariant functor $F:\cat{C}\to \cat{D}$ we can construct a functor $Fo: \cat{C^{op}}\to \cat{D}$, which is covariant by \ref{varianceOfComposite}. In identifying the two, we have an alternative definition of contravariant functor:

\begin{lemma}
A contravariant functor $F:\cat{C}\to\cat{D}$ is a (covariant) functor $F:\cat{C^{op}}\to \cat{D}$.
\end{lemma}


\begin{lemma}
We can view taking the opposite category as a (covariant) functor $\operatorname{op}: \cat{CAT} \to \cat{CAT}$ which sends categories $\cat{C}$ to $\cat{C^{op}}$ and functors $F$ to $F^o$.
\end{lemma}


For any locally small category $\cat{C}$, functors restricted to hom-sets become functions on those hom-sets. In particular:
\begin{lemma}
Let $\cat{C}$ be a locally small category and $A$ a set of morphisms in $\cat{C}$. Then
\begin{enumerate}
\item for any objects $X,Y$ in $\cat{C}$
\[ o[\cat{C}(Y,X)] = \setbuilder{f^o}{f \in \cat{C}(Y,X)} = \cat{C^{op}}(X,Y).  \]
\item $o|_A: A \to o[A]$ is bijective with inverse $o|_{o[A]}: o[A] \to A$.
\end{enumerate}
\end{lemma}

\subsection{(Commuting) diagrams}
Informally a diagram in a category is a drawing with points and arrows.

We always assume that all identity arrows and composite arrows are present in the diagram, even if they are rarely drawn.

We say that such a diagram commutes if, for any two points $x,y$ in the diagram, every path from $x$ to $y$ defines the same morphism.

\begin{example}
Let $(N_1, 0_1, S_1)$ and $(N_2, 0_2, S_2)$ be Peano systems and $\pi$ the unique isomorphism defined in \ref{existenceUniquenessPeano}. The condition
\[ \forall n\in N_1: \pi(S_1n) = S_2\pi(n) \]
is equivalent to saying
\[ \begin{tikzcd}
N_1 \ar[r, "\pi"] \ar[d, "S_1"] & N_2 \ar[d, "S_2"] \\
N_1 \ar[r, "\pi"] & N_2
\end{tikzcd} \qquad \text{commutes.} \]
The two paths in the diagram correspond to the two sides of the equation.
\end{example}

More formally we have the following definition:
\begin{definition}
Let $J$ be a category.
A \udef{diagram} of type (or shape) $\cat{J}$ in a category $\cat{C}$ is a (covariant) functor
\[ D: \cat{J} \to \cat{C}. \]
The category $\cat{J}$ is called the \udef{index category} or \udef{scheme} of the diagram $D$.

Usually we are interested in diagrams where $\cat{J}$ is small or even finite. In these cases we call the diagram \udef{small} or \udef{finite}.

If the index category $\cat{J}$ is a preordered set, then we call the diagram a \udef{commutative diagram}.

Let $\cat{I}$ be a subcategory of $\cat{J}$. Then the functor $D$ restricted to $\cat{I}$ is a \udef{subdiagram} of $D$.
\end{definition}
To actually draw such a diagram, you draw each object and morphism in the index category and name it using its image under $D$.

There is no requirement of injectivity, so the same object or morphism in $\cat{C}$ may appear several times in the diagram.

Every path in the index category with the same start $X$ and end $Y$ defines a morphism in the hom-set $\Hom(X,Y)$. In a preorder category there is at most one morphism in each hom-set, so each such path must correspond to the same morphism. Thus the informal and formal definitions of commutative diagram agree.

\begin{example}
The commutative square shown in the last example has the shape
\[ \mathbb{S}: \begin{tikzcd}
\boldsymbol{1} \ar[loop left] \ar[r] \ar[dr] \ar[d] & \boldsymbol{2} \ar[d] \ar[loop right] \\
\boldsymbol{3} \ar[loop left] \ar[r] & \boldsymbol{4} \ar[loop right]
\end{tikzcd}. \]
All morphisms in the category have been drawn. We also call this preorder category the \udef{commutative square}.
\end{example}

Any two paths in a commutative diagram with the same start and end yield an equation of morphism. Constructing proofs with these equations is known as \udef{diagram chasing} or \udef{abstract nonsense}.

\begin{example}
Warning! When drawing the diagram you draw the index category and label with the object category. You do not draw the image in the object category! This is not necessarily a preorder category when the index category is a preorder category. In fact it is not necessarily even a category. So, for example, the functor $F$ that maps
\[ \begin{tikzcd}
X_1 \dar{f} & X_2 \dar{g} \\ Y_1 & Y_2
\end{tikzcd} \qquad\text{to}\qquad \begin{tikzcd}
X = F(X_1) = F(X_2) \ar[d, shift right, swap, "f' = F(f)"] \ar[d, shift left, "g' = F(g)"] \\ Y = F(Y_1) = F(Y_2)
\end{tikzcd}  \]
should be drawn as
\[ \begin{tikzcd}
X \dar{f'} & X \dar{g'} \\ Y & Y
\end{tikzcd}, \qquad\text{not}\qquad \begin{tikzcd}
X \ar[d, swap, shift right, "f'"] \ar[d, shift left, "g'"] \\ Y
\end{tikzcd} \]
\end{example}

\begin{lemma}
Functors preserve commutative diagrams: Let $D:\cat{J}\to \cat{C}$ be a commutative diagram and $F:\cat{C}\to \cat{B}$ a functor. Then $F\circ D$ is a commutative diagram.
\end{lemma}

\subsubsection{Diagram chasing results}
\begin{lemma}
Consider the diagram
\[ \begin{tikzcd}
X \rar{f}\dar{k} & Y \rar{g}\dar{h} & Z\dar{l} \\
A \rar{p} & B \rar{q} & C
\end{tikzcd}. \]
Suppose the left and right squares commute (as subdiagrams), then the whole diagram commutes.
\end{lemma}
\begin{proof}
There are two pairs of nodes that may have more than one morphism between them: $(X,C)$ and $(A,Z)$. We show that there is only one morphism in $\Hom(X,C)$. The other case is similar.

The question boils down to whether $qhf = qpk$. Now we know that $hf = pk$ because the left square commutes, so the identity is automatic.
\end{proof}


\section{Morphisms}
\subsection{Mono- and epimorphisms}
\begin{definition}
A morphism $f: X \to Y$ is called
\begin{itemize}
\item a \udef{monomorphism} or \udef{monic} or \udef{left-cancellative} if, for any parallel morphisms $g,h: Z\to X$
\[ fg = fh \qquad \text{implies} \qquad g = h \]
\item an \udef{epimorphism} or \udef{epic} or \udef{right-cancellative} if, for any parallel morphisms $g,h: Y\to Z$
\[ gf = hf \qquad \text{implies} \qquad g = h. \]
\end{itemize}
\end{definition}

\begin{lemma}
Let $\cat{C}$ be a category.
\begin{enumerate}
\item A morphism $m:X\to Y$ is a monomorphism if for every object $A$ in $\cat{C}$ and every pair of morphisms $f,g: A\to X$, the diagram
\[ \begin{tikzcd}
A \ar[r, shift left, "f"] \ar[r, shift right, swap, "g"] & X \ar[r, "m"] & Y
\end{tikzcd} \qquad \text{commutes.} \]
\item A morphism $e:X\to Y$ is an epimorphism if for every object $A$ in $\cat{C}$ and every pair of morphisms $f,g: Y\to A$, the diagram
\[ \begin{tikzcd}
X \ar[r, "e"] & Y \ar[r, shift left, "f"] \ar[r, shift right, swap, "g"] & A
\end{tikzcd} \qquad \text{commutes.} \]
\end{enumerate}
\end{lemma}

\begin{lemma}
Let $\cat{C}$ be a category.

A monomorphism in $\cat{C}$ is an epimorphism in $\cat{C^{op}}$ and an epimorphism in $\cat{C}$ is a monomorphism in $\cat{C^{op}}$. Thus monic and epic are dual properties.
\end{lemma}
\begin{proof}
Assume $f$ is monic in $\cat{C}$. Assume $g^of^o = h^of^o$ for all $g^o, h^o$ in $\cat{C^{op}}$. Then
\[ (fg)^o = (fh)^o \implies fg = fh \implies g=h \implies g^o = h^o. \]
The other statement follows by duality.
\end{proof}

\begin{proposition} \label{injectiveMonoSurjectiveEpi}
In any concrete category
\begin{enumerate}
\item $f$ is an injective function implies $f$ is monomorphic;
\item $f$ is a surjective function implies $f$ is epimorphic.
\end{enumerate}
In the category $\cat{Set}$, the opposite implications also hold.
\end{proposition}
\begin{proof}
Set $f:X\to Y$.
\begin{enumerate}
\item Assume $f$ injective. Let $g,h: Z\to X$ such that $g\neq h$. Because we are in a concrete category, this means there exists a $z\in Z$ such that $g(z) \neq h(z)$. By injectivity of $f$ this means $f(g(z)) \neq f(h(z))$ and so $fg \neq fh$. We conclude $f$ is monic by contraposition.
\item Assume $f$ surjective. Let $g,h: Y\to Z$ such that $g\neq h$. Because we are in a concrete category, this means there exists a $y\in Y$ such that $g(y) \neq h(y)$. By surjectivity of $f$ we can find an $x\in X$ such that $f(x) = y$. So $g(f(x)) \neq h(f(x))$ and we conclude $gf \neq hf$, meaning $f$ is epic by contraposition.
\end{enumerate}
Suppose we are now in the category $\cat{Set}$.
\begin{enumerate}
\item Assume $f$ monic. To prove injectivity, take arbitrary $x_1,x_2\in X$ and assume $f(x_1) = f(x_2)$. Now define the constant functions $x_1: Z\to X: z\mapsto x_1$ and $x_2: Z\to X: z\mapsto x_2$. Then
\[ fx_1 = f(x_1) = f(x_2) = f(x_2) \qquad \implies \qquad x_1 = x_2 \]
because $f$ monic.
\item Assume $f$ epic. Left compose $f$ with both the characteristic function $\chi_{f[X]}$ and the constant function $1: Y\to \{1\}: y\mapsto 1$. It is clear $\chi_{f[X]}f = 1f$, so $\chi_{f[X]} = 1$ meaning $f$ is surjective.
\end{enumerate}
\end{proof}

\begin{lemma} \label{monicEpicCompositions}
Let the morphisms $g,f$ be composable such that $gf$ exists.
\begin{enumerate}
\item If $f$ and $g$ are monic, then $gf$ is monic.
\item If $gf$ is monic, then $f$ is monic.
\end{enumerate}
Dually:
\begin{enumerate}
\setcounter{enumi}{2}
\item If $f$ and $g$ are epic, then $gf$ is epic.
\item If $gf$ is epic, then $g$ is epic.
\end{enumerate}
\end{lemma}
\begin{proof}
The first claim is obvious. Then second is true because for morphisms $h,k$ composible with $f$,
\[ fh = fk \implies gfh = gfk \implies h = k. \]
The rest follows by duality.
\end{proof}

\subsection{Left and right inverses}
\begin{definition}
Let $f:X\to Y$ be a morphism.
\begin{itemize}
\item A \udef{left inverse} (or \udef{retraction}) of $f$ is a morphism $g: Y\to X$ such that
\[ g\circ f = \id_X. \]
\item A \udef{right inverse} (or \udef{section}) of $f$ is a morphism $h: Y\to X$ such that
\[ f\circ h = \id_Y. \]
\item A \udef{(two-sided) inverse} of $f$ is a morphism that is both a left and a right inverse.
\end{itemize}
An \udef{isomorphism} is a morphism $f: X\to Y$ for which a two-sided inverse exists.

If an isomorphism exists from $X$ to $Y$ (or equivalently from $Y$ to $X$), we say $X$ and $Y$ are \udef{isomorphic}, and write $X\cong Y$.
\end{definition}

An \udef{endomorphism} is a morphism $f$ such that $\dom(f) = \codom(f)$. An \udef{automorphism} is an endomorphism that is an isomorphism.

\begin{lemma}
Let $f:X\to Y$ and $g:Y\to X$ be morphisms in a category $\cat{C}$. Then
\begin{enumerate}
\item $g$ is a left inverse of $f$ if and only if the diagram
\[ \begin{tikzcd}
X \rar{f} & Y \rar{g} & X
\end{tikzcd} \qquad \text{commutes;} \]
\item $g$ is a right inverse of $f$ if and only if the diagram
\[ \begin{tikzcd}
Y \rar{g} & X \rar{f} & Y
\end{tikzcd} \qquad \text{commutes.} \]
\end{enumerate}
\end{lemma}

\begin{definition}
A \udef{groupoid} is a category in which every morphism is an isomorphism.

A \udef{group} is a groupoid with one object.
\end{definition}

\begin{lemma}
For any category $\cat{C}$ there exists a subcategory containing all the objects of $\cat{C}$ and only the morphisms of $\cat{C}$ that are isomorphisms. This subcategory is called the \udef{maximal groupoid} inside $\cat{C}$.
\end{lemma}

\begin{lemma}
Let $f:X\to Y$ be a morphism. If $g: Y\to X$ is a left inverse and $h: Y\to X$ a right inverse of $f$, then $g=h$ and $f$ is an isomorphism. In particular an isomorphism can have at most one inverse.
\end{lemma}
\begin{proof}
The proof is identical to \ref{leftRightInverse}:
\[ g = g\id_Y = g(fh) = (gf)h = \id_Xh = h. \]
\end{proof}

\begin{lemma}
Let $f$ be a morphism.
\begin{enumerate}
\item If $f$ has a left inverse, it is clearly a monomorphism.
\item If $f$ has a right inverse, it is clearly an epimorphism.
\end{enumerate}
\end{lemma} 
The converses are \emph{not true}. We do however call a morphism with a left inverse a \udef{split monomorphism} and a morphism with a right inverse a \udef{split epimorphism}. Clearly split monic and split epic are dual properties.

\begin{lemma}
In the category $\cat{Set}$
\begin{enumerate}
\item every monomorphism is split;
\item the assertion that every epimorphism is split is equivalent to the axiom of choice.
\end{enumerate}
\end{lemma}
\begin{proof}
This is a restatement of \ref{injectiveInverse} and \ref{surjectiveInverse}.
\end{proof}

By definition an isomorphism is a morphism that is split monic and split epic, but we can relax this condition slightly:
\begin{lemma} \label{singleSplitImpliesDoubleSplit}
If a monomorphism is split epic, it is also split monic, and thus an isomorphism.

If an epimorphism is split monic, it is also split epic, and thus an isomorphism.
\end{lemma}
\begin{proof}
Assume a morphism $f:X\to Y$ is monic and has a right inverse $r$. We claim $r$ is also a left inverse. Indeed
\[ fr = \id_Y \implies frf = \id_Yf = f\id_X \implies rf = \id_x  \]
where the second implication is because $f$ is monic. The second claim of the lemma follows by duality.
\end{proof}


\begin{lemma}
In a preorder category, every morphism that is monic or epic, is an isomorphism.

In a poset category, every morphism that is monic or epic, is an identity.
\end{lemma}






\section{Naturality}
\subsection{Natural transformations}
\begin{definition}
Let $\cat{C}, \cat{D}$ be categories and $F,G: \cat{C}\to \cat{D}$ functors. A \udef{natural transformation} $\alpha: F \Rightarrow G$ consists of
\begin{enumerate}
\item an arrow $\alpha_c: Fc\to Gc$ in $\cat{D}$, called a \udef{component} of the natural transformation, for each object $c\in \cat{C}$
\end{enumerate}
such that, for any morphism $f:c\to c'$ in $\cat{C}$, 
\[ \begin{tikzcd}
Fc \ar[r, "\alpha_c"] \ar[d, swap, "Ff"] & Gc \ar[d, "Gf"] \\
Fc' \ar[r, "\alpha_{c'}"] & Gc'
\end{tikzcd} \qquad \text{commutes}.\]
A \udef{natural isomorphism} is a natural transformation in which every component is an isomorphism. We write $F\cong G$ in this case.
\end{definition}

We represent the natural transformation $\alpha: F \Rightarrow G$ between $F,G: \cat{C}\to \cat{D}$ diagrammatically as
\[ \begin{tikzcd}[column sep=huge]
\cat{C}
  \arrow[bend left=50]{r}[name=U,label=above:$F$]{}
  \arrow[bend right=50]{r}[name=D,label=below:$G$]{} &
\cat{D}
  \arrow[shorten <=10pt,shorten >=10pt,Rightarrow,to path={(U) -- node[label=right:$\alpha$] {} (D)}]{}
\end{tikzcd} \]

The origin of this commutative square can be found in the following picture:

\begin{figure}[h!]
\centering
\begin{tikzpicture}[commutative diagrams/every diagram]
\matrix[matrix of math nodes, commutative diagrams/every cell] (mC) at (0,1.5) {c \\ \\ c' \\};
\path[commutative diagrams/.cd,every arrow,every label](mC-1-1)edge[blue] node {$f$}(mC-3-1);
\matrix[matrix of math nodes, commutative diagrams/every cell] (mD) at (6,0) {Fc &&  \\ && Gc \\ Fc'  && \\ && Gc' \\};
\path[commutative diagrams/.cd,every arrow,every label](mD-1-1)edge[red] node {$\alpha_c$}(mD-2-3);
\path[commutative diagrams/.cd,every arrow,every label] (mD-2-3) edge[blue] node {$Gf$} (mD-4-3);
\path[commutative diagrams/.cd,every arrow,every label](mD-3-1)edge[red] node {$\alpha_{c'}$}(mD-4-3);
\path[commutative diagrams/.cd,every arrow,every label](mD-1-1)edge[blue] node {$Ff$}(mD-3-1);

\path[commutative diagrams/.cd,every arrow,every label](mC-1-1)edge node {$F$}(mD-1-1);
\path[commutative diagrams/.cd,every arrow,every label](mC-1-1)edge[pos=0.34, swap] node {$G$}(mD-2-3);

\path[commutative diagrams/.cd,every arrow,every label](mC-3-1)edge node {$F$}(mD-3-1);
\path[commutative diagrams/.cd,every arrow,every label](mC-3-1)edge node[pos=0.36, swap] {$G$}(mD-4-3);

\node[draw=black,fit=(mC),circle, label={above left:$\cat{C}$}] (catC) {};
\node[draw=black,fit=(mD),circle, label={above right:$\cat{D}$}] (catD) {};
\end{tikzpicture}
\end{figure}
Thus a natural transformation is one that shifts the image of a functor along the morphisms in the target category.

This leads us to the following proposition, which is sometimes used to define natural transformations:
\begin{proposition}
Let $F,G:\cat{C}\to\cat{D}$ be functors. The natural transformations $\alpha: F\Rightarrow G$ correspond bijectively to functors $H: \cat{C}\times\mathbb{2}\to \cat{D}$ such that
\[ \begin{tikzcd}
\cat{C} \ar[r,"i_0"] \ar[dr, swap, "F"] & \cat{C}\times\mathbb{2} \ar[d,"H"] & \cat{C} \ar[l, "i_1"] \ar[dl, "G"] \\
& \cat{D} &
\end{tikzcd} \qquad \text{commutes.} \]
Here $i_0, i_1$ are the functors that map objects $c$ to $(c,0)$ and $(c,1)$ respectively.

The same holds for natural isomorphisms, if $\mathbb{2}$ is replaced by $\mathbb{I}$.
\end{proposition}

\begin{example}
Let $M$ be a monoid. A \udef{dynamical system} is a functor $F: \cat{B}M\to\cat{C}$ from the delooping space to some category. Often this category is $\cat{Top}$ or $\cat{Vect}$. Then the single object of $\cat{B}M$ is mapped to an object $c$ in $\cat{C}$ and the morphisms of $\cat{B}M$ become actions on the object: each element $m\in M$ specifies a morphism in $\cat{B}M$, which is mapped to a morphism in $\cat{C}$
\[ m: c\to c: x\mapsto m\cdot x. \]

A morphism of dynamical systems, or \udef{intertwiner}, is a map between natural transformation $\alpha:F\to G$. This can be seen as a map between objects in $\cat{C}$ that is covariant, in the sense that
\[ \alpha(m\cdot x) = m\cdot \alpha(x). \]
\end{example}

\subsubsection{Natural transformations as canonical maps}
A canonical map $\alpha$ is a map between objects that arises naturally from the definition or the construction of the objects.

TODO: make rigourous! explicate for universal algebra.

In some sense the canonical map works at the level of structure and thus must commute with morphisms.

A \udef{canonical map} is a functor such that there exists a natural transformation from the identit functor to it.

TODO: what about between different categories?






\subsection{Equivalence of categories}
\begin{definition}
An \udef{equivalence of categories} consists of functors $F: \cat{C} \to \cat{D}$ and $G: \cat{D} \to \cat{C}$ such that we have natural isomorphisms $\id_\cat{C} \cong GF$ and $\id_\cat{D} \cong FG$.

The categories $\cat{C}, \cat{D}$ are \udef{equivalent} if there exists an equivalence between them. We write $\cat{C} \simeq \cat{D}$.
\end{definition}

Except for considerations of size, this equivalence is an equivalence relation.
\begin{lemma}
The notion of equivalence is reflexive, symmetric and transitive.
\end{lemma}

\begin{definition}
A functor $F:\cat{C}\to\cat{D}$ is called
\begin{itemize}
\item \udef{full} if for all $x,y\in \cat{C}$, the map $\cat{C}(x,y) \to \cat{D}(Fx,Fy)$ is surjective;
\item \udef{faithful} if for all $x,y\in \cat{C}$, the map $\cat{C}(x,y) \to \cat{D}(Fx,Fy)$ is injective;
\item and \udef{essentially surjective on objects} if for every object $d\in\cat{D}$ there is some $c\in\cat{C}$ such that $d$ is isomorphic to $Fc$.
\end{itemize}
These are local definitions. We also call $F$
\begin{itemize}
\item an \udef{embedding} if it is injective on objects;
\item a \udef{full embedding} if it is full, faithful and an embedding.
\end{itemize}
The domain of a full embedding defines a \udef{full subcategory}.
\end{definition}

\begin{proposition}
A functor defining an equivalence of categories is full, faithful, and essentially surjective on objects.

Assuming the axiom of choice, any functor with these properties defines an equivalence of categories.
\end{proposition}

\begin{lemma} \label{isomorphismCreationReflection}
If $F:\cat{C}\to\cat{D}$ is a full and faithful functor, the $F$  reflects and creates isomorphisms:
\begin{enumerate}
\item If $f$ is a morphism in $\cat{C}$ so that $Ff$ is an isomorphism in $\cat{D}$, then $f$ is an isomorphism.
\item If $x$ and $y$ are objects in $\cat{C}$ such that $Fx$ and $Fy$ are isomorphic in $\cat{D}$, then $x$ and $y$ are isomorphic in $\cat{C}$.
\end{enumerate}
\end{lemma}
By \ref{functorMorphismPreservation} the converses hold for any functor.

\begin{definition}
The \udef{isomorphism class} of an object in a category is the collection of objects isomorphic to the object.

A category $\cat{C}$ is \udef{skeletal} if it contains one object in each isomorphism class.
\end{definition}
\begin{lemma}
Let $\cat{C}$ be a category.
There is a unique - up to isomorphism - skeletal category equivalent to $\cat{C}$.
\end{lemma}
This category is called the \udef{skeleton} of $\cat{C}$, denoted $\cat{skC}$.
\begin{proof}
Choose an object in each isomorphism class and define $\cat{skC}$ to be the full subcategory on this collection of objects. The full embedding $\cat{skC} \hookrightarrow \cat{C}$ is an equivalence of categories.
\end{proof}

\begin{lemma}
An equivalence between skeletal categories is an isomorphism of categories
\end{lemma}
\begin{corollary}
Two categories are equivalent \textup{if and only if} their skeletons are isomorphic.
\end{corollary}

\begin{definition}
A category $\cat{C}$ is called
\begin{itemize}
\item \udef{essentially small} if it is equivalent to a small category (i.e.\ its skeleton is small);
\item \udef{essentially discrete} if it is equivalent to a discrete category (i.e.\ its skeleton is discrete).
\end{itemize}
\end{definition}

\begin{lemma}
Every preorder category is equivalent to a poset category.
\end{lemma}

\section{Monoidal categories}
\begin{definition}
A \udef{monoidal category} is a category $\cat{C}$ equipped with
\begin{enumerate}
\item a functor $\otimes: \cat{C}\times\cat{C} \to \cat{C}$ called the \udef{tensor product};
\item an object $1\in \cat{C}$ called the \udef{unit object} or the \udef{tensor unit}
\end{enumerate}
We write $(\cat{C}, \otimes, 1, \alpha, \lambda, \rho)$.
\end{definition}

\subsection{Enrichment}
\begin{definition}
Let $(\cat{V}, \otimes, 1, \alpha, \lambda, \rho)$ be a monoidal category. A \udef{$\cat{V}$-category} $\cat{C}$ (or \udef{$\cat{V}$-enriched category} or \udef{category enriched over $\cat{V}$}) contains
\begin{enumerate}
\item 
\end{enumerate}
such that
\end{definition}


\chapter{Higher category theory}
\section{2-categories}
\subsection{Functor categories}
\begin{proposition} \label{functorCategory}
Let $\cat{C}, \cat{D}$ be categories. There is a category of functors $\cat{C}\to \cat{D}$ with as morphisms natural transformations called the \udef{functor category} $[\cat{C},\cat{D}]$.
\end{proposition}
\begin{proof}
There are clearly identity natural transformations $I:F\Rightarrow F$ for each functor $F: \cat{C}\to \cat{D}$ with components given by $I_{Fc}$ for all $c\in\cat{C}$.

To verify composition, let $\alpha: F \Rightarrow G$ and $\beta: G\Rightarrow H$ be natural transformations between parallel functors $F,G,H: \cat{C}\to \cat{D}$. Then the composition	$\beta\alpha: F\Rightarrow H$ has components
\[ (\beta\alpha)_c = \beta_c\alpha_c. \]
We just need to show $\beta\alpha$ is a natural transformation. Consider the diagram
\[ \begin{tikzcd}
Fc \ar[r, "\alpha_c"] \ar[d, "Ff"] & Gc \ar[d, "Gf"] \ar[r, "\beta_c"] & Hc \ar[d,"Hf"] \\
Fc' \ar[r, "{\alpha_{c'}}"] & Gc' \ar[r, "{\beta_{c'}}"] & Hc'
\end{tikzcd} \]
Both squares commute by naturality of $\alpha,\beta$. The rectangle then commutes: the only non-trivial part is the equality of paths $Fc\to Hc'$, but any such path can be deformed into another by flipping over squares (TODO better reference).
\end{proof}

\subsection{The 2-category of categories}


\chapter{Representability and universal properties}
\section{The Yoneda lemma}
\url{https://math.stackexchange.com/questions/149376/excessive-use-of-the-yoneda-lemma}
\url{https://qchu.wordpress.com/2012/04/02/the-yoneda-lemma-i/}
\begin{lemma}
Let $\cat{C},\cat{D}$ be categories. For every $c\in\cat{C}$, there is a functor defined by
\[ \evalMap_c: [\cat{C},\cat{D}] \to \cat{D}: F\mapsto Fc. \]
\end{lemma}
\begin{proof}
A natural transformation $\alpha: F \Rightarrow G$ is mapped to its component $\alpha_c: Fc \to Gc$. By \ref{functorCategory}, for all $F$, $I_F\mapsto I_{Fc}$ and the composition of composable natural transformations $\beta,\alpha$ has the component $\beta_c\alpha_c$ at $c$.
\end{proof}
So this gives us a functor
\[ \evalMap: (\cat{Set^C}, \cat{C}) \to \cat{Set}: \quad (F,c)\mapsto Fc.\]

Let $\cat{C}$ be a small category. Then for a given $c\in\cat{C}$, $\cat{C}(c,-)$ is the covariant functor represented by $c$ and thus an object in $[\cat{C},\cat{\Set}]$, which is locally small. Then we can consider the covariant functor represented by $\cat{C}(c,-)$,
\[ [\cat{C},\cat{\Set}]\to \cat{Set}:\quad F \mapsto \Hom(\cat{C}(c,-), F). \]

By currying, we can consider the two-sided represented functor $\cat{C}(-,-): \cat{C^{op}}\times\cat{C}\to \cat{Set}$ as a contravariant functor $\cat{C} \to \cat{Set^C}: c\mapsto \cat{C}(c,-)$. Composing this with the contravariant functor represented by $F$ gives the covariant functor
\[ \cat{C} \to \cat{Set}:\quad c \mapsto \Hom(\cat{C}(c,-), F). \]

Combining these two functors we have a bifunctor
\[ (\cat{Set^C}, \cat{C}) \to \cat{Set}: \quad(F,c) \mapsto \Hom(\cat{C}(c,-), F). \]

If we now let $\cat{C}$ be only locally small, not necessarily small, $\cat{Set^C}$ is no longer necessarily locally small. It turns out however that $\Hom(\cat{C}(c,-), F)$ still always yields a set. This assertion is part of the Yoneda lemma. The other part is the assertion that there is a natural isomorphism $\Phi$:
\[ \begin{tikzcd}
\cat{C}\times\cat{Set^C} \ar[rr, bend left=50, "{\Hom(\cat{C}(c,-), F)}"] \ar[rr, bend right=50, "\evalMap_c(F)"] & \Phi\Downarrow\cong & \cat{Set}
\end{tikzcd}. \]

\begin{theorem}[Yoneda lemma] \label{YonedaLemma}
Let $\cat{C}$ be a locally small category.

Given a functor $F: \cat{C} \to \cat{Set}$ and an object $c$ in $\cat{C}$, the function
\[ \Phi_{c,F}: \Hom(\cat{C}(c,-), F) \to Fc: \alpha \mapsto \alpha_c(I_c) \]
is a bijection.

Also $\Phi$ is a natural transformation with components $\Phi_{c,F}$.
\end{theorem}
\begin{corollary}[Yoneda embedding] \label{YonedaEmbedding}
Let $\cat{C}$ be a locally small category. The functors
\[ \begin{tikzcd}
\cat{C} \ar[r, hook] & \cat{Set^{C^{op}}} \\
c \ar[d, swap, "f"]\ar[r, mapsto] \ar[r,mapsto, shift right=1.7em] & \cat{C}(o(-),c) \ar[d, "f_*"] \\
d \ar[r,mapsto] & \cat{C}(o(-),d)
\end{tikzcd} \qquad\text{and}\qquad \begin{tikzcd}
\cat{C^{op}} \ar[r, hook] & \cat{Set^C} \\
c \ar[r, mapsto] \ar[r,mapsto, shift right=1.7em] & \cat{C}(c,-) \\
d \ar[u, "f^o"] \ar[r,mapsto] & \cat{C}(d,-) \ar[u, swap, "f^*"]
\end{tikzcd} \]
define full and faithful embeddings.
\end{corollary}
\begin{proof}
The functors are clearly injective on objects as morphisms with different domain or codomain are different.

Consider first the left embedding. We want the transformation $f_*$ to be natural (this is the transformation with all components equal to $f_*$). Indeed the functors $\cat{C}(o(-),c)$ and $\cat{C}(o(-),c)$ map morphisms $g^o$ in $\cat{C^{op}}$ to $g^*$. Now $f_*$ and $g^*$ clearly commute, proving the naturality of $f_*$.

Then the function $\cat{C}(c,d) \to \Hom(\cat{C}(-,c),\cat{C}(-,d)): f\mapsto f_*$ is the inverse of the Yoneda map $\Phi_{c,\cat{C}(-,d)}$:
\[ \Phi_{c,\cat{C}(-,d)}(f_*) = f_*(I_c) = f\circ I_c = f. \]
Thus by the Yoneda lemma it is bijection, meaning
\[ \cat{C}(c,d) \cong \Hom(\cat{C}(-,c),\cat{C}(-,d)) \]
and the left functor is full and faithful by definition.

The statement for the right functor can be proven analogously. Alternatively, we can see that it is a dual statement in the following way: taking the category to be $\cat{C^{op}}$, we get
\[ \cat{C^{op}}(c,d) \cong \Hom(\cat{C^{op}}(-,c),\cat{C^{op}}(-,d)) \]
or
\[ o[\cat{C}(d,c)] \cong \Hom(o\cdot\cat{C}(c,-)\cdot o,o\cdot\cat{C}(d,-)\cdot o) \]
now for every $\alpha \in \Hom(\cat{C}(c,-),\cat{C}(d,-))$, there is an $\alpha^o$ in $\Hom(o\cdot\cat{C}(c,-)\cdot o,o\cdot\cat{C}(d,-)\cdot o)$, so we have an isomorphism
\[ \cat{C}(d,c) \cong \Hom(\cat{C}(c,-),\cat{C}(d,-)). \]
\end{proof}

It is common to refer both to the Yoneda lemma \ref{YonedaLemma} and its corollary on the Yoneda embedding, \ref{YonedaEmbedding}, as the Yoneda lemma.

\begin{proposition} \label{representablyIsomorphic}
Let $\cat{C}$ be a locally small category. The following are equivalent:
\begin{enumerate}
\item $f:x\to y$ is an isomorphism in $\cat{C}$;
\item $f_*: \cat{C}(-,x) \Rightarrow \cat{C}(-,y)$ is a natural isomorphism;
\item $f^*: \cat{C}(y,-) \Rightarrow \cat{C}(x,-)$ is a natural isomorphism.
\end{enumerate}
\end{proposition}
\begin{proof}
We have already shown $f_*$ and $f^*$ are natural transformations in \ref{YonedaEmbedding}.
The proof follows from the Yoneda embedding \ref{YonedaEmbedding} and \ref{isomorphismCreationReflection}.

Alternatively, a partial proof is as follows: If $f$ has an inverse $f^{-1}$, then $(f^{-1})^*$ is an inverse of $f^*$ and $(f^{-1})_*$ an inverse of $f_*$.
\end{proof}

\section{Representable functors}
\subsection{Functors represented by objects}
\begin{definition}
Let $\cat{C}$ be a locally small category and $c$ an object in $\cat{C}$.

Given a morphism $f: X\to Y$ in $\cat{C}$, we can define the functions
\begin{enumerate}
\item $f_*: \cat{C}(c,X) \to \cat{C}(c,Y): g\mapsto fg$ defined by post-composition;
\item $f^*: \cat{C}(Y,c) \to \cat{C}(X,c): g\mapsto gf$ defined by pre-composition.
\end{enumerate}
These are morphisms in the category $\cat{Set}$.

Based on this we can define the \udef{covariant functor represented by $c$} $\cat{C}(c,-)$ and the \udef{contravariant functor represented by $c$} $\cat{C}(-,c)$:
\[ \begin{tikzcd}
\cat{C} \ar[r, "{\cat{C}(c,-)}"] & \cat{Set} \\
x \ar[d, swap, "f"]\ar[r, mapsto] \ar[r,mapsto, shift right=1.7em] & \cat{C}(c,x) \ar[d, "f_*"] \\
y \ar[r,mapsto] & \cat{C}(c,y)
\end{tikzcd} \qquad\qquad \begin{tikzcd}
\cat{C} \ar[r, "{\cat{C}(-,c)}"] & \cat{Set} \\
x \ar[r, mapsto] \ar[r,mapsto, shift right=1.7em] & \cat{C}(x,c) \ar[d, "f^*"] \\
y \ar[u, "f"] \ar[r,mapsto] & \cat{C}(y,c) 
\end{tikzcd} \]
We also have the \udef{two-sided represented functor} $\cat{C}(-,-):\cat{C}\times \cat{C} \to \cat{Set}$, which is contravariant in the first argument and covariant in the second:
\[ \begin{tikzcd}
\cat{C}\times\cat{C} \ar[r, "{\cat{C}(-,-)}"] & \cat{Set} \\
(x,y) \ar[d, shift left]\ar[r, mapsto] \ar[r,mapsto, shift right=1.7em] & \cat{C}(x,y) \ar[d, "{f^*h_*: g\mapsto hgf}"] \\
(w,z) \ar[u, shift left, "{(f,h)}"] \ar[r,mapsto] & \cat{C}(w,z)
\end{tikzcd} \]
\end{definition}
The placement of the asterisk indicates the placement of the function being composed with $f$: below means to the right and above to the left. This is consistent with the notation for indicating bases in the part on matrix representations of linear functions (TODO ref).

\begin{proposition}
Let $\cat{C}$ be a locally small category and $f: X\to Y$ a morphism. Then $f_*$ and $f^*$ are dual as follows:
\begin{align*}
(f^\text{o})^* &= of_*o = (f^*)^o; \\
(f^\text{o})_* &= of^*o = (f^*)^o.
\end{align*}
\end{proposition}
\begin{proof}
We start with the locally small category $\cat{C^{op}}$, the morphism $f^\text{o}: Y\to X$ and an object $c$. Then 
\[ (f^\text{o})^*: \cat{C^{op}}(X,c) \to \cat{C^{op}}(Y,c): g^\text{o} \mapsto g^\text{o}f^\text{o}. \]
We can rewrite this as
\[ (f^\text{o})^*: o[\cat{C}(c,X)] \to o[\cat{C}(c,Y)]: g^o \mapsto (fg)^o \]
or
\[ o|_{o[\cat{C}(Y,c)]}(f^\text{o})^*o|_{\cat{C}(c, X)}: \cat{C}(c,X) \to \cat{C}(c,Y): g \mapsto fg. \]
This last function is exactly $f_*$. The other equality follows by duality: replacing $\cat{C}$ with $\cat{C^{op}}$ and $f: X\to Y$ with $f^o: Y\to X$ gives
\[ (f^o)_* = o|_{o[\cat{C^{op}}(X,c)]}(f^\text{oo})^*o|_{\cat{C^{op}}(c, Y)} = o|_{\cat{C}(c,X)}f^*o|_{o[\cat{C}(Y,c)]} \]
so
\[ f^* =  o|_{o[\cat{C}(c,X)]}(f^o)_* o|_{\cat{C}(Y,c)}. \]
\end{proof}
\begin{corollary} \label{dualityRepresentedFunctors}
The co- and contravariant represented functors are dual in the following sense:
\begin{align*}
o\cdot \cat{C^{op}}(c,-) \cdot o &= \cat{C}(-,c) \\
o\cdot \cat{C^{op}}(-,c) \cdot o &= \cat{C}(c,-) \\
\end{align*}
\end{corollary}
Note that $o$ still works as a function on sets of morphisms in $\cat{C}$, even though they are now said to be in the category $\cat{Set}$. We also use that $o$ acts on transformations as $o(\alpha) = o\alpha o$ (TODO!).
\begin{proof}
We verify the equality for all objects and morphisms:
\[ (o\cdot \cat{C^{op}}(c,-) \cdot o)(x) = (o\cdot \cat{C^{op}}(c,-))(x) = o(\cat{C^{op}}(c,x)) = o(o(\cat{C}(x,c))) = \cat{C}(x,c) \]
and
\[ (o\cdot \cat{C^{op}}(c,-) \cdot o)(f) = (o\cdot \cat{C^{op}}(c,-))(f^o) = o((f^o)_*) =o(of^*o) = oof^*oo = f^*. \]
\end{proof}

\begin{proposition} \label{monicEpicInPrePostComposition}
Let $\cat{C}$ be a locally small category and $f: X\to Y$ a morphism. Then
\begin{enumerate}
\item $f: X\to Y$ is monic \textup{if and only if} for all $c$ in $\cat{C}$, $f_*: \cat{C}(c,X) \to \cat{C}(c,Y)$ is injective;
\end{enumerate}
and, dually,
\begin{enumerate}
\setcounter{enumi}{1}
\item $f:X\to Y$ is epic \textup{if and only if} for all $c$ in $\cat{C}$, $f^*: \cat{C}(Y,c) \to \cat{C}(X,c)$ is injective.
\end{enumerate}
\end{proposition}
\begin{proof}
Assume $f$ monic. To show injectivity, assume $f_*(g) = f_*(h)$, which means $fg = fh$. By monicity $g = h$, showing $f^*$ is injective.

The converse is equally direct, as is the second statement.

The second statement also follows by duality, recognising that because $o$ is bijective when restricted to the relevant sets, $of^*o$ is injective if and only if $f^*$ is, by \ref{injectiveMonoSurjectiveEpi} and \ref{monicEpicCompositions}.
\end{proof}

\subsection{Representable functors}
\begin{definition}
Let $\cat{C}$ be a locally small category.

A covariant / contravariant functor $F: \cat{C} \to \cat{Set}$ is \udef{representable} if there is an object $c\in\cat{C}$ such that $F$ is naturally isomorphic to the covariant / contravariant functor represented by $c$, i.e.\
\[ F \cong \begin{cases}
\cat{C}(c,-) & \text{$F$ covariant} \\
\cat{C}(-,c) & \text{$F$ contravariant.}
\end{cases} \]
We say $F$ is \udef{represented} by the object $c\in\cat{C}$.

A \udef{representation} of a functor $F: \cat{C} \to \cat{Set}$ is given by an object $c\in\cat{C}$ and a natural isomorphism $\alpha$ from $F$ to the functor represented by $c$.
\end{definition}
A functor is represented by at most one object, up to isomorphism, by \ref{representablyIsomorphic}.

By the Yoneda lemma, every natural isomorphism $\alpha$ corresponds to en element $\alpha_c(I_c) \in Fc$.
\begin{definition}
A \udef{univeral property} of an object $c\in \cat{C}$ is expressed by a representable functor $F$ together with a \udef{universal element} $x\in Fc$ that corresponds to a natural isomorphism between $F$ and the functor represented by $c$.
\end{definition}
\section{The category of elements}
\chapter{Limits and colimits}
\chapter{Adjunctions}
\chapter{Monads}

\chapter{Abelian categories}
Five lemma. Short exact sequences. Split exact. Exact functors.


\part{Model Theory}
\setcounter{chapter}{0} % Reset chapter counter
\chapter{Universal algebra}
TODO: forgetful functors.

\section{Algebras and terms}
\begin{definition}
A \udef{signature} or \udef{operational type} or \udef{operator domain} is a pair $(\Omega, \alpha)$ where $\Omega$ is a set whose elements are called \udef{operator symbols} or just operators and $\alpha: \Omega \to \N$ is a function. We call $\alpha(\omega)$ the \udef{arity} of the operator $\omega\in\Omega$. If the arity of $\omega\in\Omega$ is $n$, then we say $\omega$ is an \udef{$n$-ary} operator.
\end{definition}
We also say \udef{unary} instead of $1$-ary, \udef{binary} instead of $2$-ary and \udef{ternary} instead of $3$-ary.

\begin{definition}
A \udef{structure} of type $(\Omega,\alpha)$, also called an \udef{$\Omega$-structure} or \udef{$\Omega$-algebra}, is a set $A$, called the \udef{carrier}, equipped with a function
\[ \omega_A: A^{\alpha(\omega)}\to A \]
for each $\omega\in\Omega$. We call $\omega_A$ the \udef{interpretation} of $\omega$ in $A$.

If $\alpha(\omega) = 0$ we take $\omega_A$ to be a constant.
\end{definition}

\begin{definition}
Let $A$ be an $\Omega$-algebra. An \udef{$\Omega$-subalgebra} of $A$ is a subset that is closed under the operations of $\Omega$.
\end{definition}

\begin{lemma} \label{intersectionSubalgebra}
Let $A$ be an $\Omega$-algebra. Let $\mathcal{E}$ be a family of subalgebras. Then $\bigcap \mathcal{E}$ is also a subalgebra. 
\end{lemma}

\begin{definition}
Let $A$ be an $\Omega$-algebra and $X$ a subset of $A$. The subalgebra of $A$ \udef{generated} by $X$ is the intersection of all subalgebras containing $X$. We call $X$ the \udef{generating set} of this subalgebra.
\end{definition}
Every algebra has a (non-unique) generating set. For example the algebra itself.

\begin{definition}
The \udef{trivial} $\Omega$-algebra is the algebra generated by $\emptyset$.
\end{definition}

\subsection{Homomorphisms}
\begin{definition}
Let $A,B$ be $\Omega$-algebras. A \udef{homomorphism} of $\Omega$-algebras is a function $f:A\to B$ such that
\[ \forall \omega\in\Omega, \forall a_1,\ldots,a_{\alpha(\omega)}\in A: \quad f(\omega_A(a_1,\ldots, a_{\alpha(\omega)})) = \omega_B(f(a_1), \ldots, f(a_{\alpha(\omega)})). \]
\end{definition}

\begin{proposition}
Let $f,g:A\to B$ be two homomorphisms between $\Omega$-algebras $A,B$. If $f,g$ agree on a generating set of $A$, then they are equal.
\end{proposition}
\begin{proof}
The set $\setbuilder{x\in A}{f(x) = g(x)}$ is a subalgebra of $A$. By hypothesis it contains a generating set of $A$ and thus it is all of $A$.
\end{proof}

\begin{proposition}
Let $f:A\to B$ be a homomorphism. Then $\im f$ is a subalgebra of $B$.
\end{proposition}

\begin{proposition}
Let $(\Omega,\alpha)$ be a signature. Then the $\Omega$-algebras form a category with homomorphisms as arrows.
\end{proposition}
Consequently we have the concepts of isomorphism, endomorphism and automorphism.

\begin{proposition} \label{bijectiveHomomorphism}
Let $f:A\to B$ be a bijective homomorphism. Then $f^{-1}$ is also a homomorphism and thus $f$ is an isomorphism.
\end{proposition}
\begin{proof}
Take arbitrary $\omega\in\Omega$ and $b_1,\ldots, b_{\alpha(\omega)}\in B$. We calculate
\begin{align*}
f^{-1}(\omega_B(b_1,\ldots, b_{\alpha(\omega)})) &= f^{-1}(\omega_B(ff^{-1}b_1,\ldots, ff^{-1}b_{\alpha(\omega)})) \\
&= f^{-1}f(\omega_A(f^{-1}b_1,\ldots, f^{-1}b_{\alpha(\omega)})) = \omega_A(f^{-1}b_1,\ldots, f^{-1}b_{\alpha(\omega)}).
\end{align*}
\end{proof}

\subsection{Relations on algebras}
\subsubsection{Direct product}
\begin{definition}
Let $\{A_i\}_{i\in I}$ be a family of $\Omega$-algebras. The \udef{direct product} $\prod_{i\in I} A_i$ is the $\Omega$-algebra whose carrier is the Cartesian product of $\{A_i\}_{i\in I}$ and where operations are carried out componentwise and relations are verified pointwise.

Similarly a \udef{direct power} of $A$ is a Cartesian power of $A$ with operations and relations defined componentwise.
\end{definition}

The direct product of $\{A,B\}$ is simply written $A\times B$.

\subsubsection{Relations as algebras}
\begin{definition}
Let $A,B$ be $\Omega$-algebras and $R$ a relation on $(A,B)$ is called \udef{$\Omega$-compatible} (or just \udef{compatible}) if $R$ is a subalgebra of $A\times B$.
\end{definition}

\begin{lemma}
Let $A,B,C$ be $\Omega$-algebras and $\Gamma, \Delta$ $\Omega$-compatible relations on $(A, B)$ and $(B, C)$, respectively. Then
\begin{enumerate}
\item $\Gamma^{\transp} \subset B\times A$ is an $\Omega$-algebra;
\item $\Gamma;\Delta \subset A\times C$ is an $\Omega$-algebra;
\item for any subalgebra $A'$ of $A$, $A'\Gamma \subset B$ is an $\Omega$-algebra;
\item for any subalgebra $B'$ of $B$, $\Gamma B' \subset A$ is an $\Omega$-algebra.
\end{enumerate}
\end{lemma}

\subsubsection{Congruences}
\begin{definition}
Let $A$ be an $\Omega$-algebra. A \udef{congruence} $\mathfrak{q}$ on $A$ is an $\Omega$-compatible equivalence relation.
\end{definition}

\begin{example}
Any algebra has the \udef{trivial congruences} $I_A$ and $A^2$.
\end{example}

An algebra is \udef{simple} if there are no congruences on it other than the trivial ones. We assume a simple algebra is non-trivial.

\begin{lemma} \label{basicCongruenceLemma}
Let $\mathfrak{q}$ be a congruence on an $\Omega$-algebra $A$ and $B$ an $\Omega$-subalgebra of $A$. Then
\begin{enumerate}
\item $\mathfrak{q}^n$ is a congruence on $A^n$ for all $n\in \N$,
\item $\mathfrak{q}|_B^B$ is a congruence on $B$.
\end{enumerate}
\end{lemma}
\begin{proof}
(1) The extension of $\mathfrak{q}$ is an equivalence relation by \ref{relationPropertiesDirectProduct}.

TODO subalgebra of $(A^n)^2$ with canonical isomorphism.

(2) The restriction is clearly still reflexive, symmetric and transitive. It is an $\Omega$-algebra by \ref{intersectionSubalgebra}.
\end{proof}
For simplicity we may write $B/\mathfrak{q}$ instead of $B/(\mathfrak{q}|_B^B)$.

\begin{proposition} \label{kernelCongruence}
Let $f:A\to B$ be a homomorphism. Then $\ker f$ is a congruence on $A$.
\end{proposition}

\subsubsection{Quotient algebras}
\begin{proposition} \label{quotientAlgebra}
Let $A$ be an $\Omega$-algebra and $\mathfrak{q}$ an equivalence relation. Then there exists an interpretation of $A/\mathfrak{q}$ such that the function
\[ A \to A/\mathfrak{q}: a\mapsto [a]_\mathfrak{q} \]
is a homomorphism if and only if $\mathfrak{q}$ is a congruence.

Explicitly, this interpretation is unique and given by
\[ \omega_{A/\mathfrak{q}}([a_1]_{\mathfrak{q}},\ldots,[a_{\alpha(\omega)}]_{\mathfrak{q}}) = [\omega_A(a_1,\ldots, a_{\alpha(\omega)})]_{\mathfrak{q}} \qquad \forall \omega\in\Omega. \]\end{proposition}
\begin{proof}
The requirement that $[\cdot]_\mathfrak{q}$ be a homomorphism forces the interpretation $\omega_{A/\mathfrak{q}}$ of $\omega$ to be the one given.

We just need to show that $\omega_{A/\mathfrak{q}}$ is well-defined if and only if $\mathfrak{q}$ is a congruence. To that end, choose arbitrary $a_1, \ldots, a_{\alpha(\omega)}$ and $a'_1, \ldots, a'_{\alpha(\omega)}$ such that $a_1'\in[a_1]_\mathfrak{q}, \ldots, a'_{\alpha(\omega)}\in [a_{\alpha(\omega)}]_\mathfrak{q}$. This is equivalent to choosing $(a_1,a'_1),\ldots, (a_{\alpha(\omega)},a'_{\alpha(\omega)}) \in \mathfrak{q}$. Then
\begin{align*}
[\omega_A(a_1,\ldots, a_{\alpha(\omega)})]_{\mathfrak{q}} = [\omega_A(a'_1,\ldots, a'_{\alpha(\omega)})]_{\mathfrak{q}} &\iff (\omega_A(a_1,\ldots, a_{\alpha(\omega)}),\omega_A(a'_1,\ldots, a'_{\alpha(\omega)})) \in \mathfrak{q} \\
&\iff \omega_{A^2}((a_1,a'_1),\ldots, (a_{\alpha(\omega)},a'_{\alpha(\omega)})) \in \mathfrak{q}
\end{align*}
where the first statement is the requirement of being well-defined and the last is the requirement for being a subalgebra of $A^2$.
\end{proof}
\begin{definition}
The $\Omega$-algebra $A/\mathfrak{q}$ is called the \udef{quotient algebra} of $A$ by $\mathfrak{q}$. The function $A \to A/\mathfrak{q}: a\mapsto [a]_\mathfrak{q}$ is known as the quotient map.
\end{definition}

\begin{proposition}[Factor theorem] \label{factorTheorem}
Let $f:A\to B$ be a homomorphism of $\Omega$-algebras and $\mathfrak{q}$ a congruence on $A$ such that $\mathfrak{q}\subseteq \ker f$. Then
\[ f': A/\mathfrak{q} \to B: [a]_\mathfrak{q} \mapsto f'([a]_\mathfrak{q}) = f(a) \]
is a well-defined homomorphism with $\im f' = \im f$. Further, $f'$ is injective \textup{if and only if} $\mathfrak{q} = \ker f$.
\end{proposition}
Note that $\ker f$ is a congruence by \ref{kernelCongruence}. TODO: universal property.
\begin{proof}
To show the function is well defined, take $a,a'\in A$ such that $[a]_\mathfrak{q} = [a']_\mathfrak{q}$, i.e. $(a,a')\in \mathfrak{q}$. This implies $(a,a')\in\ker f$, so $f(a) = f(a')$ and $f'$ is well-defined.

We see that $f'$ is a homomorphism by the calculation
\begin{align*}
f'(\omega_{A/\mathfrak{q}}([a_1], \ldots, [a_{\alpha(\omega)}])) &= f'([\omega_{A}(a_1, \ldots, a_{\alpha(\omega)})]) = f(\omega_{A}(a_1, \ldots, a_{\alpha(\omega)})) \\
&= \omega_B(f(a_1), \ldots f(a_{\alpha(\omega)})) = \omega_B(f'([a_1]), \ldots, f'([a_{\alpha(\omega)}])).
\end{align*}
Finally $f'$ is injective iff no two distinct $\mathfrak{q}$-classes are identified by $f'$, which is exactly the condition $\mathfrak{q} = \ker f$.
\end{proof}

\begin{lemma}
Let $A$ be an $\Omega$-algebra and $\mathfrak{q}$ a congruence on $A$. Then
\[ [(x,y)]_{\mathfrak{q}^2} = [x]_\mathfrak{q}\times [y]_\mathfrak{q}. \]
In particular $A^2/\mathfrak{q}^2 = \setbuilder{[x]_\mathfrak{q}\times [y]_\mathfrak{q}}{x,y\in A}$.
\end{lemma}
\begin{proof}
We calculate
\[ (a,b) \in [(x,y)]_{\mathfrak{q}^2} \iff a\mathfrak{q}x \land b\mathfrak{q}y \iff a\in [x]_\mathfrak{q} \land b\in [y]_\mathfrak{q} \iff (a,b)\in [x]_\mathfrak{q}\times [y]_\mathfrak{q}. \]
\end{proof}

\subsubsection{Isomorphism theorems}
\begin{theorem}[First isomorphism theorem] \label{firstIsomorphism}
Let $f:A\to B$ be a homomorphism of $\Omega$-algebras. Then we have the isomorphism
\[ A/\ker f \cong \im f. \]
\end{theorem}
\begin{proof}
From the factor theorem \ref{factorTheorem} we get an injective homomorphism $f': A/\ker f \to B$ which is made surjective by restricting the codomain to $\im f$. By \ref{bijectiveHomomorphism} this is an isomorphism.
\end{proof}

\begin{theorem}[Second isomorphism theorem]
Let $A$ be an $\Omega$-algebra, $B$ an $\Omega$-subalgebra of $A$ and $\mathfrak{q}$ a congruence on $A$. Then we have the isomorphism
\[ (\mathfrak{q}B)/\mathfrak{q} \cong B/(\mathfrak{q}\cap B^2). \]
\end{theorem}
Note that $\mathfrak{q}B = B\mathfrak{q}$ because $\mathfrak{q}$ is a congruence. Also $\mathfrak{q}\cap B^2 = \mathfrak{q}|_B^B$, so the quotient is well-defined by \ref{basicCongruenceLemma}. Further, $\mathfrak{q}$ should really be restricted in $(\mathfrak{q}B)/\mathfrak{q}$, as in \ref{basicCongruenceLemma}.
\begin{proof}
Take the homomorphism $[\cdot]_\mathfrak{q}:A \to A/\mathfrak{q}$ as defined in \ref{quotientAlgebra} and restrict it to $B$. Applying the first isomorphism theorem \ref{firstIsomorphism} yields the required result.
\end{proof}

\begin{theorem}[Third isomorphism theorem]
Let $A$ be an $\Omega$-algebra and $\mathfrak{q},\mathfrak{r}$ congruences on $A$ such that $\mathfrak{q} \subseteq \mathfrak{r}$. Then $\mathfrak{r}/\mathfrak{q}$ is a congruence on $A/\mathfrak{q}$ and we have the isomorphism
\[ (A/\mathfrak{q})/(\mathfrak{r}/\mathfrak{q}) \cong A/\mathfrak{r}. \]
\end{theorem}
This is a slight abuse of notation: clearly $\mathfrak{q}$ is a congruence on $A$, but $\mathfrak{r}\subseteq A^2$. What we mean is that we take the quotient ``pointwise'':
\[ \mathfrak{r}/\mathfrak{q} = \setbuilder{([x]_\mathfrak{q}, [y]_\mathfrak{q})}{(x,y)\in \mathfrak{r}}. \]
\begin{proof}
Applying the factor theorem \ref{factorTheorem} to the homomorphism $A\to A/\mathfrak{r}$ from \ref{quotientAlgebra}. We get a surjective homomorphism
\[ f: A/\mathfrak{q} \to A/\mathfrak{r}: [a]_\mathfrak{q} \mapsto [a]_{\mathfrak{r}}. \]
We apply the first isomorphism theorem \ref{firstIsomorphism} to this homomorphism to get $(A/\mathfrak{q})/\ker f \cong A/\mathfrak{r}$. We just need to show that $\ker f = \mathfrak{r}/\mathfrak{q}$.

Indeed
\[ ([x]_\mathfrak{q},[y]_\mathfrak{q}) \in \ker f \iff [x]_\mathfrak{r} = [y]_\mathfrak{r} \iff (x, y)\in \mathfrak{r} \iff ([x]_\mathfrak{q},[y]_\mathfrak{q}) \in \mathfrak{r}/\mathfrak{q}. \]
\end{proof}
In particular, we see that $A/\mathfrak{q}$ is simple if and only if $\mathfrak{q}$ is a maximal proper congruence on $A$.

\section{Free algebras and varieties}

\section{Algebraic theories}
\subsection{Properties of a single binary operator}
\begin{definition}
Let $(\Omega, \alpha)$ be a signature, $A$ an $\Omega$-structure and $\omega$ a binary operator. We call an interpretation $\omega_A$
\begin{itemize}
\item \udef{associative} if $\forall x,y,z\in A: \omega_A(\omega_A(x,y),z) = \omega_A(x,\omega_A(y,z))$;
\item \udef{commutative} if $\forall x,y\in A: \omega_A(x,y) = \omega_A(y,x)$;
\item \udef{idempotent} if $\forall x\in A: \omega_A(x,x) = x$.
\end{itemize}
We say
\begin{itemize}
\item $A$ has a \udef{left-identity} $e_L$ for $\omega_A$ if $\forall x\in A: \omega_A(e_L, x) = x$;
\item $A$ has a \udef{right-identity} $e_R$ for $\omega_A$ if $\forall x\in A: \omega_A(x, e_R) = x$;
\item $A$ has an \udef{identity} $e$ if $e$ is both a left- and a right-identity.
\end{itemize}
We say
\begin{itemize}
\item $A$ has a \udef{left-absorbing element} $u_L$ for $\omega_A$ if $\forall x\in A: \omega_A(u_L, x) = u_L$;
\item $A$ has a \udef{right-absorbing element} $u_R$ for $\omega_A$ if $\forall x\in A: \omega_A(x, u_R) = u_R$;
\item $A$ has an \udef{absorbing element} $u$ if $u$ is both a left- and a right-absorbing element.
\end{itemize}
Let $A$ have an identity $e$ for $\omega_A$, then we say an element $x\in A$
\begin{itemize}
\item has a \udef{left-inverse} $y$ if $\omega_A(y,x) = e$;
\item has a \udef{right-inverse} $y$ if $\omega_A(x,y) = e$;
\item has an \udef{(two-sided) inverse} $y$ if $\omega_A(x,y) = e = \omega_A(y,x)$.
\end{itemize}
\end{definition}

TODO require that absorbing element is no identity?

TODO expand signature to include: identity / absorbing element / inverse

\begin{lemma} \label{leftRightIdentity}
Let $(\Omega, \alpha)$ be a signature, $A$ an $\Omega$-structure and $\omega$ a binary operator.
\begin{enumerate}
\item If $A$ has both a left-identity $e_L$ and a right-identity $e_R$, then $A$ has an identity $e$ and
\[ e= e_L = e_R. \]
\item If $A$ has both a left-absorbing element $u_L$ and a right-absorbing element $u_R$, then $A$ has an absorbing element $u$ and
\[ u = u_L = u_R. \]
\end{enumerate}
\end{lemma}
\begin{proof}
(1) Assume $A$ has a left- and a right-identity. Then $e_L = \omega_A(e_L, e_R) = e_R$.

(2) Assume $A$ has a left- and a right-absorbing element. Then $u_L = \omega_A(u_L, u_R) = u_R$.
\end{proof}
\begin{corollary}
A structure may have multiple left-identities or multiple right-identities, but if it has both, then the identity is unique.

An absorbing element is similarly unique.
\end{corollary}

\subsection{Properties of two binary operators}
\begin{definition}
Let $(\Omega, \alpha)$ be a signature, $A$ an $\Omega$-structure and $\omega, \chi$ binary operators. We say
\begin{itemize}
\item $\omega_A$ is \udef{left-distributive} over $\chi_A$ if
\[ \forall x,y,z\in A: \; \omega_A(x,\chi_A(y,z)) = \chi_A(\omega_A(x,y),\omega_A(x,z)); \]
\item $\omega_A$ is \udef{right-distributive} over $\chi_A$ if
\[ \forall x,y,z\in A: \; \omega_A(\chi_A(x,y), z) = \chi_A(\omega_A(x,z),\omega_A(y,z)); \]
\item $\omega_A$ is \udef{distributive} over $\chi_A$ if it is left- and right-distributive;
\item $\omega_A$ is \udef{self-distributive} if it is distributive over itself.
\end{itemize}
We say
\begin{itemize}
\item $\omega_A, \chi_A$ are linked by the \udef{absorption law} if
\[ \forall x,y\in A:\; \omega_A(x,\chi_A(x,y)) = x = \chi_A(x,\omega_A(x,y)) \]
\end{itemize}
\end{definition}

\begin{lemma} \label{absorptionIdempotency}
Let $(\Omega, \alpha)$ be a signature, $A$ an $\Omega$-structure and $\omega, \chi$ binary operators. If $\omega_A, \chi_A$ are linked by the absorption law, then they are both idempotent.
\end{lemma}
\begin{proof}
For all $x\in A$ we have $\omega_A(x,x) = \omega_A(x,\chi_A(x,\omega_A(x,x))) = x$.
\end{proof}

\subsection{Notation for binary operators}
Prefix, infix, postfix, Polish, necessity of brackets.

\part{Algebra}
\setcounter{chapter}{0} % Reset chapter counter
\chapter{Magmas}
\section{Semigroups}

\subsection{Bands}

\section{Monoids}

\begin{lemma}
A locally small category with a single object is a monoid.
\end{lemma}

\begin{proposition} \label{prop:leftRightInverseMonoid}
Let $(M,\cdot,1)$ be a monoid and $a\in M$. If $a$ has both a left inverse $l$ and a right inverse $r$, then $l=r$.
\end{proposition}
\begin{proof}
We calculate
\[ l = l\cdot 1 = l\cdot(a\cdot r) = (l\cdot a)\cdot r= 1\cdot r = r. \]
\end{proof}

TODO: \udef{delooping} $\cat{B}M$.

\subsection{Ordered monoids}
\begin{definition}
An \udef{ordered monoid} is a monoid $(M, \cdot, 0)$ on which a partial order $\preceq$ is defined that is compatible, i.e. $\forall x,y,z\in M$
\[ x\preceq y \implies x\cdot z \preceq y \cdot z \land z\cdot x \preceq z \cdot y. \]

Positive: $x > 0$.
\end{definition}

\subsection{The Archimedean property}
\begin{definition}
Let $(M,+,\leq)$ be a totally ordered monoid and $x,y\in M$ positive. Then
\begin{itemize}
\item $x$ is \udef{infinitesimal w.r.t.} $y$ or $y$ is \udef{infinite w.r.t.} $x$ if $nx<y$ for all $n\in\N$;
\item $M$ is \udef{Archimedean} if there is no pair $(x,y)$ such that $x$ is infinitesimal w.r.t. $y$.
\end{itemize}
\end{definition}
every submonoid is Archimedean.

abelian??

\section{Divisibility}
$m|n$ order relation.

$\sup\{n,m\} = kgv(n,m)$ and $\inf\{n,m\} = ggd(n,m)$







\chapter{Groups}
\url{https://www.maths.ed.ac.uk/~tl/gt/gt.pdf}

\section{Basic definitions}

\begin{definition}
A \udef{group} is a structured set $(G,\boldsymbol{\cdot})$ where $\boldsymbol{\cdot}$ is a binary operation on $G$
\[\boldsymbol{\cdot}: G\times G \to G: (g,h)\mapsto g\cdot h\]
such that
\begin{enumerate}
\item $\boldsymbol{\cdot}$ is associative:
\[ \forall g_1,g_2,g_3 \in G: \quad g_1\cdot (g_2\cdot g_3) = (g_1\cdot g_2)\cdot g_3 \]
\item there exists an \udef{identity} $e$:
\[ \forall g\in G: \quad g\cdot e = e\cdot g = g \]
\item every element has an \udef{inverse}:
\[ \forall g\in G: \exists h \in G: \quad  gh = hg = e \]
We write the inverse $h$ as $g^{-1}$. 
\end{enumerate}
If $\boldsymbol{\cdot}$ is satisfies
\[ \forall g_1, g_2 \in G: \quad g_1 \cdot g_2 = g_2\cdot g_1 \]
then the group is called \udef{commutative} or \udef{abelian}.

The cardinality of $G$ is the \udef{order} of the group, denoted $|G|$.
\end{definition}

\begin{example}
The \udef{Klein 4-group} has carrier $A = \{e,a,b,c\}$ and is defined by the Cayley table
\[ \begin{array}{l|llll}
\boldsymbol{\cdot} & e & a & b & c \\ \hline
e & e & a & b & c \\
a & a & e & c & b \\
b & b & c & e & a \\
c & c & b & a & e
\end{array}. \]
It is commutative (which can be seen by noting that the Cayley table equals its transpose).
It is also
\begin{itemize}
\item the unique commutative $4$-element group with $a^2 = b^2 = c^2 = e$;
\item the unique commutative $4$-element semigroup with identity $e = a^2 = b^2 = c^2$, and $ab = c$.
\end{itemize}
\end{example}

\begin{lemma}
Let $S$ be a semigroup. Then $S$ is a group \textup{if and only if}
\[ \forall x\in S: \; xS = S = Sx. \]
\end{lemma}

\begin{proposition}
A group is a structure of type $(e, (\cdot)^{-1}, \boldsymbol{\cdot})$ with arity defined by
\[ \alpha(e) = 0, \qquad \alpha((\cdot)^{-1}) = 1, \qquad \alpha(\boldsymbol{\cdot}) = 2. \]
Conversely, a $(e, (\cdot)^{-1}, \boldsymbol{\cdot})$-algebra is a group if
\begin{itemize}
\item $\boldsymbol{\cdot}$ is associative;
\item $\forall g\in G: g\cdot e = e\cdot g = g$;
\item $\forall g\in G: g\cdot g^{-1} = g^{-1}\cdot g = e$.
\end{itemize}
\end{proposition}
In particular the concepts of homomorphism and isomorphism apply.

\subsection{Notations}
We can use whatever symbols we want to denote the group operation, but there are two main conventions:
\begin{enumerate}
\item In \udef{multiplicative notation} the group operation is denoted by $\boldsymbol{\cdot}$, $*$ or just by concatenation (i.e.\ we write $gh$ instead of $g\cdot h$). In this case the inverse of $g$ is written $g^{-1}$, the neutral element $e$ is denoted $1$ and we can define
\[ g^n \defeq \underbrace{gg\ldots g}_{\text{$n$ factors}}\]
which is unambiguous due to associativity. Also
\[ g^{-n} \defeq (g^{-1})^n = (g^{n})^{-1}. \]
\item \udef{Additive notation} is mainly used for abelian groups. Conversion between multiplicative and additive notation is as follows:
\[ \begin{tikzcd}
g\cdot h \arrow[r, leftrightarrow]& g+h \\
1 \arrow[r, leftrightarrow]& 0 \\
g^{-1} \arrow[r, leftrightarrow]& -g \\
g^n \arrow[r, leftrightarrow] & ng.
\end{tikzcd} \]
\end{enumerate}

\begin{lemma} \label{calculusRepeatedGroupOperation}
Let $G$ be a group, $g\in G$ and $m,n\in \Z$. Then
\begin{enumerate}
\item in multiplicative notation we have
\[ g^mg^n = g^{m+n} \qquad (g^m)^n = g^{mn}; \]
\item in additive notation we have
\[ mg+ng = (m+n)g \qquad n(mg) = (mn)g. \]
\end{enumerate}
These statements are equivalent.
\end{lemma}

\subsection{Translation invariance}
\begin{definition}
Let $G$ be a group, $X$ a set and $f:G\times G \to X$ a binary function. Then $f$ is called
\begin{itemize}
\item \udef{left translation invariant} if $\forall x,y,z\in G:\; f(x, y) = f(zx,zy)$;
\item \udef{right translation invariant} if $\forall x,y,z\in G:\; f(x, y) = f(xz,yz)$;
\item \udef{translation invariant} if $f$ is left and right translation invariant.
\end{itemize}
\end{definition}

TODO: just for relations??
\begin{proposition}[Universal property translation invariance]
Let $G$ be a group. Define
\[ \Delta_r:G\times G\to G: (x,y) \to xy^{-1} \qquad \Delta_l:G\times G\to G: (x,y) \to x^{-1}y. \]
\begin{enumerate}
\item For any set $X$ and right translation invariant function $f:G\times G\to X$, there exists a unique $\widetilde{f}: G\to X$, such that
\[ \begin{tikzcd}
G\times G \rar{\Delta_r} \ar[dr, swap, "{f}"] & G \dar[dashed]{\widetilde{f}} \\
 & X
\end{tikzcd} \qquad\text{commutes.} \]
\item For any set $X$ and left translation invariant function $f:G\times G\to X$, there exists a unique $\widetilde{f}: G\to X$, such that
\[ \begin{tikzcd}
G\times G \rar{\Delta_l} \ar[dr, swap, "{f}"] & G \dar[dashed]{\widetilde{f}} \\
 & X
\end{tikzcd} \qquad\text{commutes.} \]
\end{enumerate}
\end{proposition}
\begin{proof}
TODO

(1) $\widetilde{f} = f(-, e)$;

(2) $\widetilde{f} = f(e, -)$;
\end{proof}

If a function $G\times G\to X$ is both left and right invariant, then we can choose either factorisation. We typically choose the right invariant factorisation $f = \widetilde{f}\circ \Delta_r$. We often write just $\Delta$ to denote $\Delta_r$.

\begin{lemma}
A binary relation is
\begin{enumerate}
\item left translation invariant \textup{if and only if} it is left compatible;
\item right translation invariant \textup{if and only if} it is right compatible.
\end{enumerate}
\end{lemma}
\begin{proof}
TODO
\end{proof}

For a right compatible binary relation $R$, we have
\[ xRy \qquad\iff\qquad xy^{-1} \in \widetilde{R}. \]

\begin{proposition}
Let $G$ be a group.
\begin{enumerate}
\item Let $f:G \to H$ be a group homomorphism. The kernel $\ker f$ is translation invariant as a Boolean-valued function. We have
\[ \widetilde{\ker}f = \setbuilder{g\in G}{f(g) = e_H}. \]
\item A $\mathfrak{q}$ on $G$ is translation invariant, when viewed as a Boolean-valued function. 
\end{enumerate}
\end{proposition}
\begin{proof}
(1) We have
\[ (x,y)\in \ker f \iff f(x)=f(y) \iff f(x)f(z) = f(y)f(z) \iff f(xz) = f(yz) \iff (xz,yz)\in\ker. \]
Left translation invariance is dual.

(2) TODO
\end{proof}
When dealing with groups, we will redefine $\ker$ to mean $\widetilde{\ker}$.

\subsection{Subgroups}
\begin{definition}
Let $(G,\boldsymbol{\cdot})$ be a group. We call $(H,*)$ a \udef{subgroup} if it is a group and $H\subseteq G$ and $* = \boldsymbol{\cdot}|_H$.
\end{definition}

\begin{lemma}[Subgroup criterion]
Let $(G,\boldsymbol{\cdot})$ be a group and $H$ a non-empty subset of $G$. The following are equivalent:
\begin{enumerate}
\item $(H,\boldsymbol{\cdot}|_H)$ is a subgroup;
\item for all $a,b\in H$:
\begin{itemize}
\item $a\cdot b \in H$,
\item $a^{-1}\in H$;
\end{itemize}
\item for all $a,b\in H: a\cdot b^{-1} \in H$.
\end{enumerate}
\end{lemma}

\begin{lemma}
Let $G$ be a group and $H_1, H_2$ be subgroups. Then $H_1\cap H_2$ is again a subgroup of $G$.
\end{lemma}
\begin{lemma}
Let $f:G\to H$ be a group homomorphism. Then $\ker(f)$ is a subgroup.
\end{lemma}

\subsubsection{Cosets}
\begin{definition}
Let $G$ be a group and $H\subseteq G$ a subgroup. We call a subset of the form
\begin{itemize}
\item $g\cdot H$ for some $g\in G$ a \udef{left coset};
\item $H\cdot g$ for some $g\in G$ a \udef{right coset}.
\end{itemize}
A \udef{coset} is a subset that is either a left coset or a right coset.
\end{definition}

\begin{lemma} \label{differentCosetsDisjoint}
Let $G$ be a group and $H\subseteq G$ a subgroup. Any two left (resp. right) cosets are either identical or disjoint.
\end{lemma}
\begin{proof}
Take $g,h\in G$. Assume $x\in gH\cap hH$. Then there exist $x_1,x_2\in H$ such that $gx_1 = x = hx_2$. Thus $g = hx_2x_1^{-1}$ and $h = gx_1x_2^{-1}$, meaning $gH = hx_2x_1^{-1}H = hH$ by \ref{groupCriterion}.
\end{proof}

\subsubsection{Lagrange's theorem}
\begin{theorem}
Let $G$ be a group and $H$ a subgroup of $G$. Then
\[ |G| = [G:H]\cdot |H|. \]
\end{theorem}
If $G$ is finite, $|G|$ and $|H|$ are natural numbers. If $G$ is infinite, the theorem still holds, but the orders and index are cardinals.

\subsubsection{Normal subgroups}
\begin{definition}
Let $G$ be a group. A subgroup $N\subseteq G$ is called \udef{normal} or \udef{self-conjugate} if $gNg^{-1} \subseteq N$ for all $g\in G$.

We write $N \lhd G$.
\end{definition}

\begin{proposition} \label{congruenceNormalSubgroup}
Let $G$ be a group.
A translation invariant binary relation $\mathfrak{q}$ on $G$ is a congruence \textup{if and only if} $\widetilde{\mathfrak{q}}$ is a normal subgroup.
\end{proposition}
\begin{proof}
First assume $\mathfrak{q}$ is a congruence and take $z\in\widetilde{\mathfrak{q}}$. We can find $(x,y)\in \mathfrak{q}$ such that $z = xy^{-1}$. Take arbitrary $g\in G$. We need to show that $g(xy^{-1})g^{-1} \in \widetilde{\mathfrak{q}}$. Because $\mathfrak{q}$ is reflexive, we have $(g,g)\in\mathfrak{q}$. Because is it is a subalgebra of $G^2$, we have $(gx,gy) = (g,g)\cdot (x,y)\in \mathfrak{q}$. So $g(xy^{-1})g^{-1} = gx(gy)^{-1} \in \widetilde{\mathfrak{q}}$.

Now assume $\widetilde{\mathfrak{q}}$ is a normal subgroup. We first check that $\mathfrak{q}$ is an equivalence relation:
\begin{itemize}
\item \emph{reflexivity}: $e = gg^{-1}\in \widetilde{\mathfrak{q}}$, so $(g,g)\in\mathfrak{q}$ for all $g\in G$;
\item \emph{symmetry}: if $xy^{-1}\in \widetilde{\mathfrak{q}}$, then $yx^{-1} = (xy^{-1})^{-1}\in \widetilde{\mathfrak{q}}$;
\item \emph{transitivity}: if $xy^{-1}, yz^{-1}\in \widetilde{\mathfrak{q}}$, then $xy^{-1}yz^{-1} = xz^{-1}\in \widetilde{\mathfrak{q}}$.
\end{itemize}
Now we need to show that $\mathfrak{q}$ is a subalgebra. Take $(x,y) \in \mathfrak{q}$. Then $x^{-1}(y^{-1})^{-1} = x^{-1}y = y^{-1}(xy^{-1})^{-1}y$, so $\widetilde{\mathfrak{q}}$ is closed inder taking the inverse.

Take $(x,y),(u,v)\in\mathfrak{q}$. Then $uv^{-1}, xy^{-1}$ and $y^{-1}x = y^{-1}(xy^{-1})y$ are elements of $\widetilde{\mathfrak{q}}$. Then
\[ xu(yv)^{-1} = xuv^{-1}y^{-1} = x(uv^{-1})(y^{-1}x)x^{-1} \in \widetilde{\mathfrak{q}}, \]
so $\mathfrak{q}$ is closed under the group operation.
\end{proof}
\begin{corollary} \label{kernelNormalSubgroup}
Let $f: G\to H$ be a group homomorphism. Then $\ker f$ is a normal subgroup.
\end{corollary}

If $N\subseteq G$ is a normal subgroup, we define the quotient algebra
\[ G/N \defeq G/(\setbuilder{(x,y)\in G^2}{xy^{-1}\in N}). \]
This is a group because homomorphisms preserve associativity, inverses and identity (TODO ref). We call such a group a \udef{quotient group}.

We denote the equivalence classes by $[x]_N$.

\subsection{Conjugation}
\begin{definition}
Let $G$ be a group and $g\in G$ an element. Then the mapping
\[ \Ad_g: G\to G: x\mapsto g^{-1}xg \]
is called \udef{conjugation by $g$}.
We also write $h^g \defeq \Ad_g(h) = g^{-1}hg$.
\end{definition}
Thus a subgroup is normal if and only if $\Ad_g[N] \subseteq N$ for all $g\in G$.


\subsubsection{Conjugacy}
\begin{definition}
Let $G$ be a group. Elements $g,h\in G$ are called \udef{conjugate} if $\exists x: \; \Ad_x(g) = h$.
\end{definition}

\begin{proposition}
Conjugacy is an equivalence relation.
\end{proposition}
The equivalence classes under conjugation are called \udef{conjugacy classes}.


\subsubsection{Centraliser and normaliser}

\begin{lemma}
Let $G$ be a group and $A\subseteq G$ a subset. Then
\begin{enumerate}
\item $Z_G(A) = \setbuilder{g\in G}{\forall a\in A:\;\Ad_g(a) = a}$;
\item $N_G(A) = \setbuilder{g\in G}{\Ad_g[A] = A}$.
\end{enumerate}
\end{lemma}

\begin{proposition}
Let $G$ be a group and $A\subseteq G$ a subset. Then
\begin{enumerate}
\item $Z_G(A) \lhd N_G(A) \lhd G$;
\item $N_G(A)$ is the largest subgroup of $G$ in which $A$ is normal.
\end{enumerate}
\end{proposition}
\begin{corollary}
$Z_G\lhd G$.
\end{corollary}


\subsubsection{Inner and outer automorphisms}
\begin{lemma}
Let $G$ be a group and $g\in G$ an element. Then $\Ad_g$ is an automorphism.
\end{lemma}
\begin{definition}
Let $G$ be a group.
Automorphisms of the form $\Ad_g$ for some $g\in G$ are called \udef{inner automorphisms}. Automorphisms that are not of this form are called \udef{outer automorphisms}.

The set of inner automorphisms forms a group, denoted $\Inn(G)$.
\end{definition}

\begin{theorem}[N/C theorem]
Let $G$ be a group and $H\subseteq G$ a subgroup. Then
\[ N_H(H) / Z_G(H) \cong \Inn(H). \]
\end{theorem}
\begin{proof}
TODO \url{https://proofwiki.org/wiki/Centralizer_is_Normal_Subgroup_of_Normalizer}
\end{proof}
\begin{corollary}
Let $G$ be a group. Then $G / Z_G \cong \Inn(G)$.
\end{corollary}


\subsection{Direct product}
\begin{definition}
The \udef{direct product} $G \equiv H\otimes F$ of two groups $H$ and $F$ is defined with the following operation:
\[ (H\otimes F) \times (H\otimes F) \rightarrow (H\otimes F): ((h_1,f_1),(h_2,f_2)) \mapsto (h_1\cdot h_2, f_1\cdot f_2)\]
\end{definition}
The direct product is a group with
\[ \begin{cases}
e_G = (e_H,e_F) \\
g^{-1} = (h^{-1}, f^{-1})\qquad \forall g = (h,f) \in G.
\end{cases} \]
The groups $F$ and $H$ are subgroups of $G$ and can be recovered by considering, respectively the elements of $G$ of the form $(e_H, g)$ and $(g ,e_F)$.

\subsection{Semidirect product}
TODO

\section{Types of groups}
\begin{example}
Examples of Groups:
\begin{enumerate}
\item The \udef{trivial group} $\{e\}$.
\item $\Z_n = 0:(n-1)$ with addition modulo $n$, is a group of order $n$.
\item $\Z_n$, the group of all $n^{\text{th}}$ roots of $1$ with the ordinary product, is of order $n$.
\begin{itemize}
\item $Z_2 = \{1,-1\}$
\item $Z_3 = \{1, e^{i2/3\pi}, e^{i1/3\pi}\}$
\end{itemize}
\item $S_n$, the group of all permutations of $n$ elements, is of order $n!$.
\item Integers with addition.
\item $\mathbb{R}\setminus\{0\}$ with multiplication.
\item The square (i.e.\ $n\times n$) invertible matrices with matrix multiplication form a group.
\end{enumerate}
\end{example}

\begin{proposition}
The groups $Z_n$ and $\Z_n$ are isomorphic.
\end{proposition}
We use $Z_n$ to denote the group if we are using multiplicative notation and $\Z_n$ if we are using additive notation.

In particular we have $\Z_2 = \sSet{\{0,1\},+}$ and $Z_2 = \sSet{\{1,-1\},\cdot}$.

\begin{lemma}
All groups of order $2$ are isomorphic to $\Z_2$.
\end{lemma}
\begin{proof}
Let $G = \sSet{\{e,g\}, \cdot}$ be a groups of order $2$, with $e$ the identity.
We must have $g\cdot g = e$. Indeed, from $g \neq e$, we get $g\cdot g \neq g\cdot e = g$ and $g\cdot g = e$ is the only other option.

Consider the function
\[ f: G \to \Z_2: \begin{cases}
e\mapsto 0 \\ g \mapsto 1
\end{cases}. \]
This functions is clearly bijective. We just need to see that it is a homomorphism. Indeed we have
\[ \begin{cases}
f(e\cdot e) = f(e) = 0 = 0+0 = f(e) + f(e) & f(e\cdot g) = f(g) = 1 = 0+1 = f(e) + f(g) \\
f(g\cdot e) = f(g) = 1 = 1+0 = f(g) + f(e) & f(g\cdot g) = f(e) = 0 = 1+1 = f(g) + f(g). 
\end{cases} \] 
\end{proof}

\subsection{Words, relations and presentations}
\begin{example}
Quaternion group
\[ \mathbb{H} \defeq \group\setbuilder{a,b}{a^4=e, a^2=b^2, b^{-1}ab = a^{-1}} \]
\end{example}

\subsection{Cyclic groups}
\begin{definition}
A group is called \udef{cyclic} if it is generated by a single element.
\end{definition}
\begin{lemma}
\begin{enumerate}
\item The group $(\Z, +)$ is cyclic.
\item Every cyclic group is a an image of $\Z$ by a homomorphism.
\item Every cyclic group is isomorphic to $\Z$ or $\Z/m\Z$.
\end{enumerate}
\end{lemma}
We write $\Z_m$ or $C_m$ for $\Z/m\Z$.
 
\subsection{Torsion groups and orders of elements}
\begin{definition}
Let $G$ be a group. An element $a$ satisfying $a^n = 1$ for some $n$ is said to be of \udef{finite order}. In this case the \udef{order} of the element $a$ is $n$.

A group in which every element is of finite order is called a \udef{torsion group} or a \udef{periodic group}.
\end{definition}
\begin{lemma}
Every finite group is a torsion group. The converse is not true.
\end{lemma}
\begin{proof}
Let $G$ be a finite group. Assume $G$ is not a torsion group. Then we can find an element $g\in G$ that is not of finite order. Consider the mapping $\N \to G: n\mapsto g^n$. The image of this mapping is a subset of $G$ and thus finite, so the mapping is not injective, so we can find $n<m\in\N$ such that $g^n = g^m$. Then $g^{m-n}=1$ with $m-n\in \N$, so $g$ is of finite order, which is a contradiction. This falsity of the converse is shown by the following examples.
\end{proof}

\begin{example}
The set
\[ \setbuilder{z\in \C}{z^n = 1\;\text{for some}\; n\in \Z} \]
together with complex multiplication forms an infinite torsion group.
\end{example}

\begin{lemma}
Let $G$ be an abelian group. Then the set of all elements of finite order forms a subgroup, called the \udef{torsion subgroup}.
\end{lemma}


\subsection{Permutation groups}
\begin{proposition} \mbox{}
\begin{enumerate}
\item Let $X$ be a set. The set of bijections $X\to X$ forms a group;
\item Let $X,Y$ be sets. The groups of bijections on $X$ and $Y$ are isomorphic \textup{if and only if} $X$ and $Y$ are equinumerous.
\end{enumerate}
\end{proposition}

\begin{definition}
We call the group of bijections on a set $X$ the \udef{symmetric group} of $X$, denoted $S(X)$.
\begin{itemize}
\item The \udef{degree} of $S(X)$ is the cardinality of $X$.
\item For any cardinal $n$, we denote the unique permutation group of degree $n$ by $S_n$.
\item Elements of $S_n$ are called \udef{permutations} and subgroups of $S_n$ are called \udef{permutation groups}.
\end{itemize}
\end{definition}

TODO: cfr. Clifford algebra with $V$ containing the transpositions.

\begin{proposition}
For all sets $X$, we have $S(X) = S_{|X|}$.
\end{proposition}
In general we state and prove results for $S_n$, without loss of generality.

\begin{lemma}
For all cardinals $\kappa >_c 2$, the permutation group $S_\kappa$ is non-abelian.
\end{lemma}

\subsubsection{Cycles}
\subsubsection{Transpositions, parity and the alternating group}
\begin{definition}
Let $S_n$ be a permutation group and $x\in S_n$. Consider the set $\Fixedpoints(x)$. If $|\Fixedpoints(x)| = 2$, then $x$ is called a \udef{transposition}.
\end{definition}

\begin{definition}
Let $S_n$ be a finite permutation group. We call the function
\[ p_n: S_n \to Z_2: x\mapsto \begin{cases}
1 & \text{$|\Fixedpoints(x)|$ is even} \\
-1 & \text{$|\Fixedpoints(x)|$ is odd}
\end{cases} \]
the \udef{parity homomorphism}.

The group $A_n \defeq \ker p_n$ is called the \udef{alternating group} of \udef{degree} $n$.
\end{definition}

\begin{proposition}
The parity homomorphism is a homomorphism.
\end{proposition}

\begin{lemma}
The alternating group $A_n$ has order $n!/2$.
\end{lemma}

\subsection{Dihedral groups}
\begin{definition}
Dihedral group of order $2n$.
\[ D_n \defeq \group\setbuilder{ a,b }{a^n=b^2=e, b^{-1}ab = a^{-1} }. \]
\end{definition}

\begin{proposition}
Let $n\in \N$. Then
\[ Z_G = \begin{cases}
\{e, a^{n/2}\} & \text{$n$ is even} \\
\{e\} & \text{$n$ is odd}.
\end{cases} \]
\end{proposition}
\begin{corollary} \label{dihedralDoubleCover}
For all $n\in \N$ there is a short exact sequence
\[ \begin{tikzcd}
1 \rar & \Z_2 \rar & D_{2n} \rar & D_n \rar & 1.
\end{tikzcd} \]
\end{corollary}

\subsubsection{Full dihedral group}
TODO: full dihedral group $D$ of isometries of $\C$ that fix the origin.
\[ \begin{tikzcd}
1 \rar & \T \rar & D\rar & Z_2 \rar & 1
\end{tikzcd} \qquad\text{is short exact.} \]

\section{Short exact sequences}
\subsection{Quotient sequences}
\begin{proposition}
Let $G$ be a group and $N\lhd G$ a normal subgroup. Then
\[ \begin{tikzcd}
1 \rar & N \rar[hookrightarrow]{\subseteq} & G \rar{[\cdot]_N} & G/N \rar & 1
\end{tikzcd} \]
is a short exact sequence.
\end{proposition}
\begin{proof}
Clearly the inclusion $N\hookrightarrow G$ is injective and $[\cdot]_N$ is surjective. Finally note that $x\in \ker[\cdot]_N \iff [x]_N = [e]_N \iff xe^{-1} = x\in N$.
\end{proof}

\begin{proposition}
For any short exact sequence of groups
\[ \begin{tikzcd}
1 \rar & H_1 \rar{\alpha} & G \rar{\beta} & H_2 \rar & 1
\end{tikzcd}, \]
there exist isomorphisms $f,g$ and subgroup $N\lhd G$ such that
\[ \begin{tikzcd}
1 \rar & H_1 \dar{f} \rar{\alpha} & G \dar{\id_G} \rar{\beta} & H_2 \dar{g} \rar & 1 \\
1 \rar & N \rar[hookrightarrow] & G \rar{[\cdot]_N} & G/N \rar & 1
\end{tikzcd} \]
commutes.
\end{proposition}
\begin{proof}
We can take $N = \im(\alpha) = \ker(\beta)$, which is a normal subgroup by \ref{kernelNormalSubgroup}. Because $\alpha$ is injective, $\alpha|^{\im(\alpha)}: H_1 \to N$ is bijective. We take $f$ to be this.

The function $\beta': G/N \to H_2$ defined in the factor theorem \ref{factorTheorem} is bijective because $\beta$ is surjective.

Both constituent squares clearly commute. So the rectangle commutes by \ref{commutingRectangle}.
\end{proof}

So in some sense all short exact sequences of groups are of the form
\[ \begin{tikzcd}
1 \rar & N \rar[hookrightarrow] & G \rar & G/N \rar & 1
\end{tikzcd}. \]
Given $G$ and either $N$ or $G/N$, we can easily find the third group in the sequence. Given $N$ and $G/N$, there may be several inequivalent ways to complete the short exact sequence. Groups that fit in the middle of a short exact sequence are called group extensions.

\subsection{Group extensions}
\begin{definition}
Let $N,Q$ be groups. An \udef{extension} of $Q$ by $N$ is a group $G$ such that
\[
\begin{tikzcd}
1 \ar[r] & N \ar[r, "\iota"] & G \ar[r, "\pi"] & Q \ar[r] & 1
\end{tikzcd}.
\]
is a short exact sequence.
\end{definition}
\begin{lemma}
If $G$ is an extension of $Q$ by $N$, then $G$ is a group (TODO: closure), $\iota(N)$ is a normal subgroup of $G$ and $Q$ is isomorphic to $Q$.
\end{lemma}

\begin{example}
The real numbers $\R$ are an extension of the unit complex numbers by the integers $\Z$:
\[ \begin{tikzcd}
0 \rar & \Z \rar{\subseteq} & \R \ar[rr, "\theta\mapsto e^{2\pi i \theta}"] && \T \rar & 1
\end{tikzcd} \]
\end{example}

\subsubsection{Equivalent group extensions}
\begin{definition}
Two extensions $G,G'$ of $Q$ by $N$ are \udef{equivalent} if there is a homomorphism $T:G\to G'$ making the following diagram commutative:
\[
\begin{tikzcd}
1 \ar[r] & N \ar[r, "\iota"] \ar[equal]{d} & G \ar[r, "\pi"] \ar[d,"T"] & Q \ar[r] \ar[d, equal] & 1 \\
1 \ar[r] & N \ar[r, "\iota"] & G' \ar[r, "\pi"] & Q \ar[r] & 1.
\end{tikzcd}
\]
\end{definition}
\begin{lemma}
If $G,G'$ are equivalent extensions, then they are isomorphic. So equivalence of extension is an equivalence relation.
\end{lemma}
\begin{proof}
The short five lemma (TODO).
\end{proof}
The converse is \emph{not} true! TODO: For instance, there are $8$ inequivalent extensions of the Klein four-group by $\mathbb{Z}/2\mathbb{Z}$, but there are, up to group isomorphism, only four groups of order $8$ containing a normal subgroup of order $2$ with quotient group isomorphic to the Klein four-group.

\subsubsection{Split exact sequences}
\url{https://kconrad.math.uconn.edu/blurbs/grouptheory/splittinggp.pdf}

\subsubsection{Double covers}
\begin{definition}
Let $G_1, G_2$ be groups. If $G_1$ is an extension of $G_2$ by $\Z_2$, i.e.\
\[\begin{tikzcd}
1 \rar & \Z_2 \rar & G_1 \rar & G_2 \rar & 1
\end{tikzcd} \]
is a short exact sequence, then we call $G_1$ a \udef{double cover} of $G_2$.
\end{definition}

\begin{example}
For all $n\in \N$, the dihedral group $D_{2n}$ is a double cover of $D_n$:
\[\begin{tikzcd}
1 \rar & \Z_2 \rar & D_{2n} \rar & D_n \rar & 1
\end{tikzcd} \]
See \ref{dihedralDoubleCover}.
\end{example}
Quaternionic group gives inequivalent extension.


\section{Group action}
\begin{definition}
An \udef{action} of a group $G$ on a set $X$ is a mapping
\[ \cdot: G\times X \to X: (g,x) \mapsto g\cdot x \]
satisfying
\begin{enumerate}
\item $e\cdot x = x$ for all $x\in X$ where $e$ is the neutral element of $G$;
\item $g(h\cdot x) = (gh)\cdot x$ for all $x\in X$ and $g,h\in G$.
\end{enumerate}
We will often just right $gx$ instead of $g\cdot x$.

A set $X$ with a given action of $G$ on it is called a \udef{$G$-set}.
\end{definition}
This definition can be reformulated using the curried form of $\pi$, namely
\[ \rho \defeq \operatorname{curry}(\pi): G \to (X\to X). \]
Then the rest of the definition of group action amounts to the statement that 
\[ \rho: G \to S(X) \quad \text{is a group homomorphism.} \]

We may then specify $G$-sets by the data $(X,\rho)$, where $X$ is a set and $\rho: G \to S(X)$ a group homomorphism.

TODO: opposite action.

\begin{definition}
Given two $G$-sets $X,Y$, a \udef{$G$-equivariant mapping} or \udef{intertwiner} is a map $f:X \to Y$ such that
\[ f(gx) = gf(x) \]
for all $g\in G$ and $x\in X$.
\end{definition}
We can express this by saying $f: (X,\rho_1) \to (Y,\rho_2)$ is a map between $G$-sets such that
\[ f\circ \rho_1(g) = \rho_2(g)\circ f \]
for all $g\in G$.

\begin{lemma}
The $G$-sets form a locally small category with $G$-equivariant maps as morphisms.
\end{lemma}


\subsection{Orbits and stabilisers}
\begin{definition}
Let $G$ be a group acting on a set $X$. We define the \udef{orbit} of $x\in X$ as the set
\[ Gx = G\cdot x \defeq \setbuilder{g\cdot x \in X}{g\in G}  \]
and the \udef{stabiliser} of $x\in X$ as the set
\[ G_x \defeq \setbuilder{g\in G}{g\cdot x = x}. \]
\end{definition}
\begin{proposition}[Orbit-stabiliser theorem]
Let $X$ be a $G$-set and $x\in X$. Then
\begin{enumerate}
\item $|Gx| = [G:G_x]$;
\item $|G| = |Gx|\cdot |G_x|$.
\end{enumerate}
\end{proposition}

\subsection{Actions of groups on themselves}
\subsubsection{Regular actions}
\begin{definition}
A group $G$ has a natural left action on the set $G$:
\[ G\times \operatorname{Field}(G) \to \operatorname{Field}(G): (g,h) \mapsto gh. \]
This action of $G$ is called the \udef{left-regular} group action.

Similarly, the natural right action on the set $G$ is called the \udef{right-regular} group action:
\[ \operatorname{Field}(G) \times G \to \operatorname{Field}(G): (h,g) \mapsto hg. \]
\end{definition}

\subsubsection{Conjugation}
Let $G$ be a group. The conjugation mapping
\[ \Ad: G\times \operatorname{Field}(G) \to \operatorname{Field}(G): (g,h) \mapsto \Ad_g(h) = g^{-1}hg \]
is a group action by $G$ on itself.

The orbits under this action are the conjugacy classes.

The stabiliser of $a\in G$ when $G$ is acting on itself by conjugation, is the \udef{centraliser} of $\{a\}$.





\section{Group action}
We have seen that symmetry transformations naturally form a group. Based on the concrete set of transformations that are symmetries we saw they form this abstract structure which we called a group. The advantage of working with this abstract entity is that it contains exactly the relevant details about the symmetry. We need not worry ourselves about the peculiarities of the particular system and we can easily make use of results others have obtained solving other problems.

Once we have thoroughly studied the symmetries of our system, we will want a way to move back from studying abstract groups to studying transformations of the system we are actually interested in.

Sometimes there is a natural correspondence between the set of group elements and the set of transformations. If this is the case the group can be interpreted as acting on the system in a \udef{canonical} (or natural) way.

\begin{example}
\begin{itemize}
\item Dihedral group $D_4$ acts quite naturally on a blanc, square piece of paper.
\item The symmetric group $\mathcal{S}_n$ of all permutations of a set of $n$ elements acts naturally on a set of $n$ elements.
\item The group of $n\times n$ matrices acts naturally on $n$-dimensional vectors through matrix multiplication.
\end{itemize}
\end{example}

In general the transition back may not be so clear, simple or natural. For instance there may be a subset of the $n$-dimensional vectors with a symmetry group isomorphic to $D_4$. To what transformations do these group elements correspond? We cannot just rotate and flip these vectors. It is for understanding these cases that the concept of a \udef{group action} is useful.

\subsection{Definition}
We start with a group $G$ and a set $X$. The set $X$ is frequently the set of configurations of the system and thus transformations of the system are functions of the type $f:\,X\to X$; to keep things general, we only assume we have set and we are agnostic as to its origins.
A group action quite simply associates a transformation of the set to every element of the group.

We do however require that this association has some fairly natural features, so that the nature and essence of the group is not lost in transition: the group action must respect the identity element and and group operation. This leads us to the following definition:
\begin{definition}
Let $G$ be a group and $X$ a set, then a \udef{(left) group action $\varphi$} of $G$ on $X$ is a function
\[ \varphi: \, G\times X \to X: \, (g,x)\mapsto \varphi(g,x) = g\cdot x \]
with the properties:
\begin{enumerate}
\item For the identity element $e$ and all $x\in X$: 
\[e\cdot x = x \]
\item For all $g,h \in G$ and $x\in X$:
\[ (gh)\cdot x = g\cdot (h\cdot x) \]
\end{enumerate}
Notice that we have introduced the notation $g\cdot x$ meaning apply the transformation attributed to $g$ through the group action to the element $x$.
\end{definition}

The above definition is for a \textit{left} group action. We can analogously define a right group action. The only difference between the two is that in the right group action in the transformation
\[ x \cdot (gh) = (x\cdot g)\cdot h \]
the transformation associated with $g$ gets applied first. Using the formula $(gh)^{-1} = h^{-1}g^{-1}$ we can always construct a left group action from a right one and vice versa, so typically we only consider left group actions.

An important property is immediately apparent from the definition:
\begin{eigenschap}
The transformation associated with $g$ (i.e.\ $x\mapsto g\cdot x$) is always a bijection because the inverse is given by $x \mapsto g^{-1}\cdot x$.
\end{eigenschap}

\subsection{Types of action}
What follows is simply an enumeration of some properties group actions may have. The action of $G$ on $X$ is called
\begin{enumerate}
\item \udef{transitive} if $X$ is non-empty and for each $x,y$ in $X$ there exists a $g \in G$ such that $g\cdot x = y$.
\item \udef{faithful} if for every distinct $g,h$ in $G$ there exists an $x \in X$ such that $g\cdot x \neq h\cdot x$. In other words the mapping of elements of $G$ to transformations of $X$ is 1-to-1 or injective.
\end{enumerate}

\subsection{Orbits and stabilizers}
\begin{definition}
Consider a group $G$ acting on a set $X$. The \udef{orbit} of an element $x$ of $X$ is denoted $G\cdot x$.
\[ G\cdot x = \{ g\cdot x | g\in G \}. \]
\end{definition}

The \udef{stabilizer subgroup} of $G$ with respect to an element $x$ of $X$ is the set of all elements in $G$ that fix $x$ and is denoted $G_x$.
\[G_x = \{ g\in G | g\cdot x = x \} \]

\subsection{Continuous group action}
A continuous group action on a topological space $X$ is a group action of a topological group $G$ that is continuous: i.e.\,
\[G \times X \to X : \;(g, x) \mapsto g \cdot x \]
is a continuous map.

This is the proper type of group action to use with topological groups, if their topologicalness is relevant and to be preserved.

\subsection{Representations}
If the group action is the action of a group on a vector space such that the transformations the group elements are mapped to are linear transformations, we call this group action a \udef{representation}.

\begin{definition}
A \udef{representation} of a group $G$ on an $n$-dim vector space $V$ is a mapping of the elements of $G$ to the set of invertible linear operations acting on $V$:
\[D: G \rightarrow GL(V): g \mapsto D(g)\]
Such that
\begin{itemize}
\item $D(e) = \mathbb{1}_V$
\item $D(g_1\cdot g_2) = D(g_1)D(g_2) = D(g_3)$
\end{itemize}
\end{definition}

\begin{example}
\begin{itemize}
\item Representations of $Z_3 = \{e,\omega, \omega^2\} \qquad (\omega = e^{i2/3\pi})$
\begin{itemize}
\item Trivial representations
\[D(e) = D(\omega) = D(\omega^2) = \mathbb{1}_V\]
\item Representation $\GL(1, \mathbb{C})$
\[ D(e) = 1, \quad D(\omega) = e^{i\frac{2}{3}\pi} , \qquad D(\omega^2) = e^{i\frac{1}{3}\pi} \]
\item Regular representation:
\[D(e) = 
\begin{pmatrix}
1&0&0\\0&1&0\\0&0&1
\end{pmatrix}, \qquad D(\omega) = \begin{pmatrix}
0&0&1\\1&0&0\\0&1&0
\end{pmatrix}, \qquad D(\omega^2) = \begin{pmatrix}
0&1&0\\0&0&1\\1&0&0
\end{pmatrix}
\]
In general a we can define a regular representation for any finite group $G$ as follows: Let $V$ be a vector space with basis $e_t$ indexed by the elements of $G$, $t \in G$. The mapping $D: e_t \mapsto e_{ts}$ defines the \udef{(left) regular representation} of $G$. This notion can be extended to groups of infinite order.
\end{itemize}
\item The standard representation of a subgroups $H$ of $\GL(n,\C)$ on the vector space $\C^n$ is given by the inclusion:
\[ D: H \to \GL(\C^n) = \GL(n,\C): h \mapsto h \]
\end{itemize}
\end{example}


\begin{definition}
Two representations are \udef{equivalent} if there exists a linear operator $S$ such that
\[D(g) \mapsto D'(g) = S^{-1}D(g)S\]
In other words there exists a similarity transformation $S$
\end{definition}

\begin{definition}
A representation is \udef{unitary} if $\forall g \in G$
\[D(g)D^\dagger(g) = D^\dagger(g)D(g) = \mathbb{1}_V\]
\end{definition}

\begin{definition}
Consider a \undline{representation $D$} of a \undline{group $G$} on a \undline{vector space $V$}
\begin{enumerate}
\item A subspace $W$ of $V$ is called \udef{invariant} if $D(g)w$ is in $W$ for all $w \in W$ and all $g \in G$. An invariant subspace $W$ is called nontrivial if $W\neq\{0\}$ and $W \neq V$.
\item We call $D$ \udef{reducible} if there exists a nontrivial subspace $U$ of $V$ that is invariant under $D$.
\item $D$ is \udef{irreducible} if the only subspaces invariant under all elements of the image of $D$ are $\emptyset$ and $V$
\item $D$ is \udef{completely reducible} if we can decompose $V$ into invariant subspaces:
\[V = U_1\oplus U_2 \oplus \ldots \oplus U_n\]
There then exists a similarity transformation such that
\[\forall g: D(g) = \begin{pmatrix}
D_1(g) & 0 & \dots & 0\\
0 & D_2(g)  & \dots & 0\\
\vdots & & \ddots & \vdots\\
0&0&\dots&D_n(g)
\end{pmatrix}\qquad \text{with}\quad D \equiv D_1\oplus D_2 \oplus \ldots \oplus D_n\]
\end{enumerate}
\end{definition}

\begin{example}
The regular representation of $Z_3$ is completely reducible. The linear operators $D(e), D(\omega)$ and $D(\omega^2)$ have eigenvalues $1,\omega, \omega^2$ with eigenvectors 
\[ \begin{pmatrix}
1\\1\\1
\end{pmatrix}\, \qquad \begin{pmatrix}
1\\ \omega^2 \\ \omega
\end{pmatrix} \qquad \text{and} \qquad \begin{pmatrix}
1 \\ \omega \\ \omega^2
\end{pmatrix}. \]
Each eigenvector generates an invariant subspace. We can then apply the following coordinate transformation
\[ S = \frac{1}{\sqrt{3}}\begin{pmatrix}
1&1&1\\
1&\omega^2 & \omega \\
1&\omega& \omega^2
\end{pmatrix} \]
in order to get the following matrices
\[D'(e) = \begin{pmatrix}
1&0&0\\
0&1&0\\
0&0&1
\end{pmatrix}, \qquad D'(\omega) = \begin{pmatrix}
1 & 0 & 0\\
0& \omega & 0 \\
0&0&\omega^2
\end{pmatrix}, \qquad D'(\omega^2) = \begin{pmatrix}
1 & 0 & 0\\
0& \omega^2 & 0 \\
0&0&\omega
\end{pmatrix}\]
\[ D' = D_1\oplus D_2 \oplus D_3 = \diag\{1,1,2\}\oplus \diag\{1,\omega, \omega^2\} \oplus \diag\{1, \omega^2, \omega\} \]
\end{example}

\subsubsection{Projective representations}
Bargmann theorem



\section{Topological groups}
A group is a set with an extra structure layered on top: the group operation that satisfies the group axioms. A topological space is also a set with an extra structure layered on top: the topology, as discussed in a previous part. Now here's a novel idea: let's layer both of these structures on a set at once. This gives no new mathematics because the two structures do not interact in any way; in order for interesting things to occur, we must pose some additional requirements.

\begin{definition}
A \udef{topological group} $G$ is a topological space that is also a a group such that the group operations of
\begin{enumerate}
\item product
\[ G\times G \to G: \, (x,y)\mapsto xy \]
\item and taking inverses
\[ G\to G: \, x\mapsto x^{-1} \]
\end{enumerate}
are \textbf{continuous}.
\end{definition}
TODO also need that points are closed?

\begin{lemma}
The continuity of the product and inverse is equivalent to the continuity of $G\times G \to G: (s,r)\mapsto sr^{-1}$.
\end{lemma}
TODO; reframe as criterion?

\begin{lemma}
Let $G$ be a topological group. The following are homeomorphisms:
\begin{enumerate}
\item $G\to G: s\mapsto s^{-1}$;
\item $G\to G: s\mapsto rs$ for any $r\in G$.
\end{enumerate}
\end{lemma}
An important consequence of this is that the topology of $G$ is determined by the topology near the identity $e$.

Topological groups are also sometimes called continuous groups.



\section{Grothendieck group}
Given a commutative monoid $M$, the Grothendieck group $G(M)$ is the ``most general'' Abelian group that arises from $M$. Intuitively it is formed by adding additive inverses for all elements of $M$.



 
TODO Grothendieck construction for Abelian monoids: $G(M)$.
Universality, functoriality

Cancellation property: simplified construction.

Grothendieck map $M\to G(M)$ is injective \textup{if and only if} $M$ has cancellation.

\subsection{The integers}
\begin{definition}
$\Z$
\end{definition}


\section{Ordered groups}

\begin{lemma}
Let $(G,+,\leq)$ be an ordered group and $x,y\in G$. Then
\[ (\forall \varepsilon > 0: x< y+\varepsilon) \implies x\leq y. \]
\end{lemma}
\begin{proof}
The proof is by contraposition. Assume $x>y$, then we can take $\varepsilon = x-y>0$. This implies $x = y+\varepsilon$ and so $x \geq y+\varepsilon$. 
\end{proof}


\chapter{Rings}
TODO: addition, multiplication and scalar multiplication of functions: pointwise.

TODO: unital homomorphisms; unital subalgebra.

\begin{lemma}
Let $R$ be a ring and $a\in R$.
\begin{enumerate}
\item If $a$ has a left and a right inverse, they are equal. Thus $a$ has an inverse.
\item The inverse of $a$ is unique, if it exists.
\end{enumerate}
\end{lemma}
\begin{proof}
Let $l$ be a left inverse of $a$ and $r$ a right inverse. Then
\[ l = l(ar) = (la)r = r. \]
The unicity of the inverse is an easy consequence.
\end{proof}

\begin{lemma} \label{lemma:productInvertibility}
Let $R$ be a ring and $a,b\in R$. Then $a$ and $b$ are invertible \textup{if and only if} $ab$ and $ba$ are invertible.
\end{lemma}
\begin{proof}
Assume $a,b$ invertible. Then $b^{-1}a^{-1}$ is an inverse for $ab$ and $a^{-1}b^{-1}$ is an inverse for $ba$.

Assume both $ab$ and $ba$ have inverses. Then from
\begin{align*}
a[b(ab)^{-1}] &= \vec{1} & [(ba)^{-1}b]a &= \vec{1} \\
[(ab)^{-1}a]b &= \vec{1} & b[a(ba)^{-1}] &= \vec{1}
\end{align*} 
we see that both $a$ and $b$ have left and right inverses.
\end{proof}

\begin{proposition} \label{prop:everyProperIdealInMaximalIdeal}
Every proper ideal is contained in a maximal ideal.
\end{proposition}

\begin{lemma} \label{lemma:nonInvertibleGeneratedIdeals}
Let $R$ be a unital ring. If $a\in R$ is non-invertible, then the generated ideal $(a)$ is not the whole ring.
\end{lemma}
\begin{proof}
If $(a) = R$, then $1\in (a)$, implying $ab=1$ for some $b\in R$. A contradiction.
\end{proof}

\begin{proposition}
Let $f$ be a ring homomorphism. If $f$ is invertible as a function (i.e. bijective), its inverse $f^{-1}$ is also a ring homomorphism.
\end{proposition}

\begin{proposition} \label{prop:kernelIsIdeal}
Kernel of Ring Homomorphism is Ideal
\end{proposition}

\begin{definition}
A $*$-rng is a structured set $(R,+,\cdot, *)$, where $R$ is a rng and $*:R\to R$ is an involutive anti-automorphism. That is, $\forall x,y\in R$:
\begin{itemize}
\item $(xy)^* = y^*x^*$;
\item $(x+y)^* = x^* + y^*$;
\item $(x^*)^* = x$.
\end{itemize}
This is also known as an \udef{involutive rng} or \udef{rng with involution}.
\end{definition}
\begin{lemma}
If $R$ is a unital ring with involution, then $1^* = 1$.
\end{lemma}
\begin{proof}
From $1^*x = (x^*1)^* = (x^*)^* = x$, we see that $1^*$ is a multiplicative identity, which is unique.
\end{proof}
\begin{definition}
An element of a $*$-rng is \udef{self-adjoint} if $x^* = x$.
\end{definition}

\section{Group rings}
\begin{definition}
Let $G$ be a finite group and $R$ a r(i)ng. The \udef{group ring} $RG$ is the set of functions $(G\to R)$ with pointwise addition and the convolution product
\[ (x\star y)(g) = \sum_h x(h)y(h^{-1}g) = \sum_{g=hk}x(h)y(k) \]
for all $x,y\in RG$ and $g\in G$. 
\end{definition}
The a group ring can be seen as a free module generated by $G$. (TODO: this as definition?)



\chapter{Fields}
\section{Totally ordered fields}
\begin{definition}
Let $K$ be a set with binary operations $+,\cdot$ such that $(K,+,\cdot)$ is a field and a binary relation $\leq$ such that $(F,\leq)$ is a total order. Then the structured set $(K,+,\cdot,\leq)$ is a \udef{totally ordered field} if $\forall a,x,y\in K$:
\begin{enumerate}
\item $x\leq y \implies x+a \leq y+a$;
\item $x\leq y \land a\geq 0 \implies ax \leq ay$;
\item $x\leq y \land a\leq 0 \implies ax \geq ay$.
\end{enumerate}
\end{definition}

\begin{lemma}
Let $(K,+,\cdot,\leq)$ be a totally ordered field and $a,b,c,d\in K$. Then
\begin{enumerate}
\item $a\leq b \land c\leq d \implies (a+c) \leq (b+d)$;
\item $a\leq b \implies -b\leq -a$;
\item $a\geq 0 \iff -a\leq 0$;
\item $a\geq 0 \land b\geq 0 \implies ab \geq 0$;
\item $a\geq 0 \implies a^n \geq 0$ for all $n\in\N$;
\item $a\leq 0 \implies (a^{2n} \geq 0 \land a^{2n+1}\leq 0)$ for all $n\in\N$;
\item $a > 0 \iff a^{-1} > 0$ and $a < 0 \iff a^{-1} < 0$;
\item if $b \geq a>0$, then $b^{-1} \leq a^{-1}$;
\item $0<1$.
\end{enumerate}
\end{lemma}
\begin{proof}
(1) By applying point 1. of the definition twice, we get $(a+c)\leq(b+c)\leq(b+d)$.

(2) This is the result of multiplying by $-1$.

(3) Idem, using $-0=0$.

(4) Special case of point 2. of the definition.

(5) By induction on $n$ and point 2. of the definition.

(6) By induction on $n$ and point 3. of the definition.

(7) By multiplying $a \geq 0$ with $(a^{-1})^{2}$, which is positive, we get $a^{-1}\geq 0$. Also $a\neq 0 \iff a^{-1}\neq 0$.

(8) By point 7. and point 4. of the lemma $a^{-1}b^{-1}$ is positive, so multiplying $b\geq a$ by $a^{-1}b^{-1}$ yields $b^{-1} \leq a^{-1}$.

(9) Assume, towards a contradiction, that this is false, so $1\leq 0$. Then $1$ is negative and multiplying the inequality with $1$ yields $1\cdot 1 \geq 1\cdot 0$, or $1\geq 0$. Taking both inequalities gives $1=0$ by anti-symmetry, which is prohibited for fields.
\end{proof}

\begin{definition}
Let $F\subset K$ be totally ordered fields. Then we call $F$ \udef{dense} in $K$ if
\[ \forall a,b\in K: \exists x\in F: a<x<b. \]
\end{definition}
TODO: topology definition?

\chapter{Valuation theory}
Absolute values on integral domains.

\part{Discrete Mathematics}
\setcounter{chapter}{0} % Reset chapter counter
\chapter{Summation}
\[ a\sum_{i=1}^n x_i = \sum_{i=1}^n ax_i \]
\[ \sum_{i=1}^n\sum_{j=1}^m x_{ij} = \sum_{j=1}^m\sum_{i=1}^n x_{ij} \]

$\delta_{ij}$

Introduce $a:b$ for sequence. (bounds inclusive)

$-$ means everything.

\begin{proposition} \label{productOfSum}
Let $R$ be a ring, $n\in \N$ and $\seq{a_i},\seq{b_i}\in R^n$. Then
\[ \prod_{i=0}^{n-1}(a_i + b_i) = \sum_{\seq{s_i}\in \{0,1\}^n}\prod_{i=0}^{n-1}\bigg(k\mapsto \begin{cases}
a_i & (k=0) \\ b_i & (k=1)
\end{cases}\bigg)(s_i). \]
\end{proposition}

\begin{proposition} \label{sumAsDecimation}
Let $G$ be an Abelian group, $f:\N\to G$ a function and $M,N \geq 1 \in \N$. Then
\[ \sum_{k=0}^{M\cdot N-1}f(k) = \sum_{k_0 = 0}^{M-1}\sum_{k_1 = 0}^{N-1} f(k_1\cdot M + k_0). \]
\end{proposition}
\begin{proof}
We prove this by induction on $N$. For the base case $N=1$, we have
\begin{align*}
\sum_{k=0}^{M\cdot N-1}f(k) &= \sum_{k=0}^{M-1}f(k) \\
&= \sum_{k_0=0}^{M-1}f(k_0) \\
&= \sum_{k_0=0}^{M-1}\sum_{k_1 = 0}^{0}f(k_1 \cdot M + k_0) \\
&= \sum_{k_0=0}^{M-1}\sum_{k_1 = 0}^{N-1}f(k_1 \cdot M + k_0).
\end{align*}
For the induction step, assume the induction hypothesis, i.e.\ that the statement holds for $N-1$. Then
\begin{align*}
\sum_{k=0}^{M\cdot N-1}f(k) &= \sum_{k=0}^{M\cdot (N-1)-1 + M}f(k) \\
&= \sum_{k=0}^{M\cdot (N-1)-1}f(k) + \sum_{k=M\cdot (N-1)}^{M\cdot (N-1)-1 + M}f(k) \\
&= \sum_{k_0 = 0}^{M-1}\sum_{k_1 = 0}^{N-2} f(k_1\cdot M + k_0) + \sum_{k_0=0}^{M-1}f(M\cdot (N-1) + k_0) \\
&= \sum_{k_0 = 0}^{M-1}\Big(\sum_{k_1 = 0}^{N-2} f(k_1\cdot M + k_0) + f(M\cdot (N-1) + k_0)\Big) \\
&= \sum_{k_0 = 0}^{M-1}\sum_{k_1 = 0}^{N-1} f(k_1\cdot M + k_0).
\end{align*}
By induction we conclude that the statement holds for all $M,N \geq 1$.
\end{proof}

\chapter{Combinatorics}
todo: inclusion-exclusion principle + move after analysis
\section{Permutations}
Permutation group $S_n$.

Factorials

\section{Combinations}

\[ \begin{pmatrix}
n \\ k
\end{pmatrix} = \frac{n!}{k!(n-k!)}\]


$(k,m)$ shuffle

Binomial theorem. Fill in $x=1, y=-1$.

\section{Finite sets}
\begin{definition}
We call a set $X$ an $n$-set if its cardinality is $n$.
\end{definition}

\subsection{Covering finite sets}
\begin{definition}
Let $X$ be a finite set.
\begin{itemize}
\item A \udef{cover} is a collection $\mathcal{F}$ of sets such that $\bigcup \mathcal{F} = X$;
\item A \udef{$k$-cover} is a cover containing $k$ sets.
\end{itemize}
\end{definition}

\url{https://core.ac.uk/download/pdf/82680601.pdf}

\part{Elements of Mathematics}
\setcounter{chapter}{0} % Reset chapter counter
\chapter{Elements of Euclidean geometry}
TODO: Tusi couple

In this section we will give a practical rundown of some of the classic results of Euclidean geometry, especially those results that have elementary proofs not needing more involved machinery. We will focus on practical things like calculating angles, surface areas and volumes. A more general discussion will follow in the section about spaces.

\section{Flat shapes}
Pick theorem

\subsection{Circles} are shapes consisting of all the points at a certain fixed distance from a central point. We call this distance the radius, denoted $r$. 
We can then calculate the circumference
\[ C_\circ = 2\pi r \]
and the surface area
\[ A_\circ = \pi r^2. \]
\begin{definition}
Chord. Diameter.
\end{definition}

\subsection{Rectangles}
Rectangles are made up of four lines which intersect at right angles. Rectangles have a length $l$ and a width $w$. The surface area is
\[A_\square = lw.\]
Squares are rectangles with the same length and width.

\subsection{Triangles}
Triangles are shapes with three sides. To calculate the surface area we choose one side to be our base $b$. The shortest possible distance from the line that extends this side to the point not on this line is called the height $h$. The surface area is then
\[ A_\triangle = \frac{1}{2}bh. \]
TODO picture of proof: box around triangle. Draw line from point to base that splits box in two. In each half the triangle takes up half the area.

\begin{proposition}
A triangle, inscribed in a circle, consisting of two chords and a diameter always has a right angle.
\end{proposition}
\begin{proof}
Rotate the triangle 180\textdegree.
\end{proof}

\section{Solids}
\subsection{Spheres} are shapes consisting of all the points at a certain fixed distance from a central point. We call this distance the radius, denoted $r$. The surface area is
\[ S_\text{sphere} = 4\pi r^2 \]
and the volume is
\[V_\text{sphere} = \frac{4}{3}\pi r^3.\]

\subsection{Prisms} are extrusions of a polygonal base, not necessarily in a direction orthogonal to the plane of the base. If $A$ is the surface area of the base, the volume of the prism is
\[V_\text{prism} = Ah. \]

\subsection{Cylinders} are like prisms, but with plane curves instead of polygons as their base. Again, if $A$ is the surface area of the base, the volume is
\[ V_\text{cylinder} = Ah. \]

\subsection{Triangular pyramids} have a volume
\[ V_\text{pyramid} = \frac{1}{3}Ah \]

\section{Angles}
We often use greek letters like $\alpha, \beta$ or $\gamma$ to denote angles. For quantifying angles we can use degrees. Or we can identify each direction with a point on a circle of radius $1$ and use the distance along the edge of the edge of the circle to represent the angle. We call that distance the angle in \udef{radians}. In that case $360$ degrees (or $360^\circ$) is the full circumference, or $2\pi$ radians (also written as $2\pi$ rad or just $2\pi$); $180^\circ$ is half that, or $\pi$.

So we can convert any angle in degrees to radians by dividing by $360^\circ$ and multiplying by $2\pi$. To convert the other way we divide by $2\pi$ and multiply by $^\circ$.

Radians are often useful to work with because the arc length $L$ of a circular arc with radius $r$ and subtending an angle $\theta$ (measured in radians), is
\[ L = \theta r.\] 

We will usually use radians, and it should be assumed that all angles are expressed in radians, unless degrees or other units are explicitly stated.

\section{Some classic results and theorems}
TODO similar triangles, inner angles

Viviani's theorem

\section{Trigonometry}
TODO
\subsection{Defining sine and cosine}
With triangle, but not all numbers, so circle.

Domain and image

The squaring notation $\sin^2\theta$

Table of angles

\subsubsection{Some useful identities}
\begin{itemize}
\item \textbf{Pythagorean identity}
\[ \cos^2 \theta + \sin^2\theta = 1 \]
\item \textbf{Periodicity}
\[ \begin{cases}
\cos(\theta + 2\pi) = \cos\theta \\
\sin(\theta + 2\pi) = \sin\theta
\end{cases} \]
\item Cosine is \textbf{even}, sine is \textbf{odd}
\[ \begin{cases}
\cos(-\theta) = \cos\theta \\
\sin(-\theta) = -\sin\theta
\end{cases} \]
\item \textbf{Complementary angles}. Two angles are \udef{complementary} if their sum is $\pi/2$.
\[ \begin{cases}
\cos \left(\frac{\pi}{2} - \theta\right) = \sin\theta \\
\sin \left(\frac{\pi}{2} - \theta\right) = \cos\theta
\end{cases} \]
\item \textbf{Supplementary angles}. Two angles are \udef{supplementary} if their sum is $\pi$.
\[ \begin{cases}
\cos \left(\pi - \theta\right) = -\cos\theta \\
\sin \left(\pi - \theta\right) = \sin\theta
\end{cases} \]
\end{itemize}
TODO fig.
\subsubsection{Addition formulae}
\begin{align*}
\cos(\theta+\phi) &= \cos\theta\cos\phi - \sin\theta\sin\phi \\
\sin(\theta+\phi) &= \sin\theta\cos\phi + \cos\theta\sin\phi \\
\cos(\theta-\phi) &= \cos\theta\cos\phi + \sin\theta\sin\phi \\
\sin(\theta-\phi) &= \sin\theta\cos\phi - \cos\theta\sin\phi
\end{align*}
\subsubsection{Double- and half-angle formulae}
Double-angle
\begin{align*}
\sin 2\theta &= 2\sin\theta\cos\theta \\
\cos 2\theta &= \cos^2\theta - \sin^2\theta \\
&= 2\cos^2\theta - 1 \\
&= 1- 2\sin^2\theta
\end{align*}
Half-angle
\[ \cos^2\theta = \frac{1+\cos 2\theta}{2} \qquad \text{and}\qquad \sin^2 \theta = \frac{1 - \cos 2\theta}{2}. \]
\subsection{Other trigonometric functions}
tangent, cotangent, secant, cosecant (primary / secondary)

\subsection{Angles and sides in triangles.}
TODO figure vertices $A,B,C$ and sides $a,b,c$ opposite.
\subsubsection{Law of sines}
\[ \frac{\sin A}{a} = \frac{\sin B}{b} = \frac{\sin C}{c} \]
\subsubsection{Law of cosines}
\begin{align*}
a^2 &= b^2 + c^2 - 2bc\cos A \\
b^2 &= a^2 + c^2 - 2ac\cos B \\
c^2 &= a^2 + b^2 - 2ab\cos C
\end{align*}

\subsection{Waves}
\subsection{Plane waves}
frequency , wavelength, angular frequency, period

\subsection{The wave equation}

\subsection{Group and phase velocity}

\section{Cyclometric functions}
Name places me geographically.

Restrict domain to make bijective.
Both notations $\sin^{-1}$ and $\arcsin$

TODO + continuity of $\cos^{-1}$

\section{Hyperbolic functions}
Definition using $e$.
\[ \cosh x = \frac{e^x + e^{-x}}{2}, \qquad \sinh x = \frac{e^x - e^{-x}}{2} \]
Origin of name:
\[ \cosh^2 t - \sinh^2 t = 1 \]
(Proof:)
\begin{align*}
\cosh^2 t - \sinh^2 t &= \left(\frac{e^x + e^{-x}}{2}\right)^2 - \left(\frac{e^x - e^{-x}}{2}\right)^2 \\
&= \frac{1}{4}\left(e^{2t} + 2 + e^{-2t} - (e^{2t} - 2 + e^{-2t})\right) \\
&= \frac{2+2}{4} = 1
\end{align*}

Cosh is catenary curve.

Many properties similar to regular sine and cosine:
\begin{itemize}
\item $\cosh 0 =1$ and $\sinh 0= 0$;
\item The hyperbolic cosine is \ueig{even} ($\cosh(-x) = \cosh x$) and the hyperbolic sine is \ueig{odd} ($\sinh(-x) = \sinh x$).
\item Addition formulae (notice sign difference with $\cosh$):
\begin{align*}
\cosh(x+y) &= \cosh x\cosh y + \sinh x\sinh y \\
\sinh(x+y) &= \sinh x\cosh y + \cosh x\sinh y
\end{align*}
\item Double angle formulae (again sign difference for $\cosh$):
\begin{align*}
\sinh 2x &= 2\sinh x\cosh x \\
\cosh 2x &= \cosh^2 x + \sinh^2 x \\
&= 2\cosh^2 x - 1 \\
&= 1 + 2\sinh^2 x
\end{align*}
\end{itemize}
Other hyperbolic functions:
\begin{align*}
\tanh x &= \frac{\sinh x}{\cosh x} = \frac{e^x - e^{-x}}{e^x + e^{-x}} & \sech x &= \frac{1}{\cosh x} = \frac{2}{e^x + e^{-x}} \\
\coth x &= \frac{\cosh x}{\sinh x} = \frac{e^x + e^{-x}}{e^x - e^{-x}} & \csch x &= \frac{1}{\sinh x} = \frac{2}{e^x - e^{-x}}
\end{align*}
Inverse hyperbolic functions:
\begin{align*}
\sinh^{-1} x &= \log_e \left(x + \sqrt{x^2 + 1}\right) \\
\tanh^{-1} x &= \frac{1}{2}\log_{e} \left(\frac{1+x}{1-x}\right) \qquad (-1< x <1) \\
\cosh^{-1} x &= \log_e \left(x + \sqrt{x^2 - 1}\right) \qquad (x \geq 1)
\end{align*}
with restriction because $\cosh$ is not automatically bijective. ($\tanh$?)


\part{Convergence and Topology}
\setcounter{chapter}{0} % Reset chapter counter
\url{https://en.wikipedia.org/wiki/Cauchy_space}

\url{https://en.wikipedia.org/wiki/Cauchy-continuous_function}

\url{https://en.wikipedia.org/wiki/Proximity_space}

\url{https://www.bioinf.uni-leipzig.de/~studla/Publications/PREPRINTS/01-pfs-007-subl1.pdf}

\chapter{Convergence}
\section{Convergence spaces}
Intuition: directed sets and refinement.

\begin{definition}
Let $X$ be a set and let $\xi$ be a relation between downward directed sets in $\powerset(X)$ and elements of $X$. Then
\begin{itemize}
\item we denote the image function of $\xi$ as $\lim_\xi: \directed(\powerset(X)) \to \powerset(X): F\mapsto F\xi$; we call $\lim_\xi F$ the \udef{$\xi$-limit} of $F$;
\item $\xi$ is called a \udef{preconvergence} on $X$ if $\lim_\xi$ is order-preserving when $\directed(\powerset(X))$ is ordered by refinement:
\[ F \preceq G \implies \lim_\xi F \subseteq \lim_\xi G; \]
\item $\xi$ is called a \udef{convergence} if it is a preconvergence and it is \udef{centered}:
\[ \forall x\in X: \quad x\in \lim_\xi \big\{\{x\}\big\}. \]
\end{itemize}
If $\xi$ is a convergence, then we call $\sSet{X, \xi}$ a \udef{convergence space}.

If $\lim_\xi F \neq \emptyset$, we say the directed set $F$ \udef{converges}.
\end{definition}
We write ${\lim_\xi}^{-1}$ for the preimage function of $\xi$ restricted to directed sets $\nsim \powerset(X)$. Thus ${\lim_\xi}^{-1}(x)$ is the set of all directed sets $D$ such that
\begin{itemize}
\item $D\overset{\xi}{\longrightarrow} x$;
\item $D \nsim \powerset(X)$.
\end{itemize}

\begin{lemma}
Let $X$ be a set and $x\in X$. A preconvergence $\xi$ on $X$ is a convergence \textup{if and only if} $\forall x\in X: \{x\} \in {\lim_\xi}^{-1}(x)$.
\end{lemma}

\begin{lemma}
Let $X$ be a set, $\xi$ a convergence on $X$ and $x\in X$. Then $\lim_\xi^{-1}(x)$ is upwards closed.
\end{lemma}

\subsection{Filters and convergence}
\begin{lemma}
Let $X$ be a set and $\xi$ a convergence on $X$. Let $A,B\in \directed(X)$ be downward directed sets. If $A \approx B$, then $\lim_\xi A = \lim_\xi B$.
\end{lemma}
\begin{proof}
We have $A \preceq B$ and $B \preceq A$, so $\lim_\xi A \subseteq \lim_\xi B$ and $\lim_\xi B \subseteq \lim_\xi A$.
\end{proof}
This lemma means we can view $\lim_\xi$ as a function on the quotient $\directed(X) /\approx$, which is order isomorphic to $\filters(X)$ (indeed there is one filter in each equivalence class in $\directed(X) /\approx$ and for filters the refinement relation simplifies to inclusion).

Consequently we will usually think of a convergence on a set $X$ as a relation between filters on $\powerset(X)$ and elements of $X$. The axioms then become
\begin{itemize}
\item $\xi$ is a preconvergence on $X$ if $\lim_\xi$ is order-preserving when $\powerfilters(X)$ is ordered by inclusion:
\[ F \subseteq G \implies \lim_\xi F \subseteq \lim_\xi G; \]
\item $\xi$ is called a convergence if it is a preconvergence and it is centered:
\[ \forall x\in X: \quad x\in \lim_\xi \pfilter{x}. \]
\end{itemize}

Any convergence defined for filters uniquely extends to a convergence defined for all downward directed sets. 

\begin{lemma}
Let $X$ be a set and $\xi$ a preconvergence on $X$. Then
\begin{enumerate}
\item $\forall F,G \in \powerfilters(X): \quad \lim_\xi(F\cap G) \subseteq \lim_\xi F \cap \lim_\xi G$;
\item $\forall F,G \in \powerfilters(X): \quad \lim_\xi(F\cup G) \supseteq \lim_\xi F \cup \lim_\xi G$.
\end{enumerate}
\end{lemma}
\begin{proof}
Reformulation of \ref{orderPreservingFunctionLatticeOperations}, because a preconvergence is order-preserving.
\end{proof}
\begin{corollary} \label{limitDegenerateFilter}
For any convergence $\xi$ on $X$ we have $\lim_\xi \powerset(X) = X$.
\end{corollary}
\begin{proof}
Clearly $\pfilter{x} \subseteq \powerset(X)$ for all $x\in X$, so
\[ \lim_\xi(\powerset(X)) \supseteq \lim_\xi \left(\bigcup_{x\in X}\pfilter{x}\right) \supseteq X. \]
\end{proof}

In the sequel we will usually consider convergence as a property of filters, however sometimes it will be easier to consider downward directed sets (e.g.\ for continuity).

\subsubsection{Limit points}
\begin{definition}
Let $\sSet{X,\xi}$ be a convergence space and $x\in X$. Then
\begin{itemize}
\item $x$ is called a \udef{limit point} if there exists a filter $F\in\powerfilters(X)$ other than $\pfilter{x}$ that converges to $x$;
\item $x$ is called an \udef{isolated point} if it is not a limit point.
\end{itemize}
Let $A\subseteq X$ be a subset. Then $x$ is called a \udef{limit point of $A$} if there exists a filter $F\in\powerfilters(X)$ other than $\pfilter{x}$ that converges to $x$ and has $A\in F$.
\end{definition}

\begin{proposition}
Let $\sSet{X,\xi}$ be a convergence space and $x\in X$. The following are equivalent:
\begin{enumerate}
\item $x$ is a limit point;
\item $\vicinity_\xi(x) \neq \pfilter{x}$;
\item $x\in \adh_\xi\big(X\setminus\{x\}\big)$;
\end{enumerate}
\end{proposition}
\begin{proof}
TODO (??)
\end{proof}

\subsubsection{Approaches}
\begin{definition}
Let $\sSet{X,\xi}$ be a convergence space. An \udef{approach} on $X$ is a function $F: X\to \powerfilters(X)$ such that $F(x) \overset{\xi}{\longrightarrow} x$ for all $x\in X$.
\end{definition}

\begin{lemma}
Let $\sSet{X,\xi}$ be a convergence space. The function $X\to \powerfilters(X): x\mapsto \pfilter{x}$ is an approach.
\end{lemma}

\subsection{Depth requirements}
\begin{definition}
Let $\sSet{X,\xi}$ be a convergence space. Take arbitrary $F,G \in \powerfilters(X)$, $\mathcal{F}\subseteq \powerfilters(X)$ and $x\in X$. The convergence space is called
\begin{itemize}
\item a \udef{Kent space} if $F \to x \implies F\cap \pfilter{x} \to x$;
\item \udef{finitely deep} or a \udef{limit space} if $\lim_\xi(F\cap G) = \lim_\xi F \cap \lim_\xi G$;
\item a \udef{Choquet space} or a \udef{pseudotopological space} if
\[ F\overset{\xi}{\longrightarrow} x \qquad\iff\qquad \text{$U\overset{\xi}{\longrightarrow} x$ for all ultrafilters $U$ such that $F\subseteq U$.} \]
\item a \udef{pretopological space} if 
\[ \lim_\xi\Big(\bigcap \mathcal{F}\Big) = \bigcap \setbuilder{\lim_\xi(F')}{F'\in \mathcal{F}}. \]
\end{itemize}
\end{definition}

The point of a Kent space is that if $F\to x$, then ``essentially'' every $A\in F$ contains $x$.

\begin{lemma} \label{finiteDepthLemma}
Let $X$ be a set and $\xi$ a preconvergence on $X$. Then the following are equivalent:
\begin{enumerate}
\item $\xi$ is finitely deep;
\item $\forall F,G \in \powerfilters(X): \quad \lim_\xi(F\cap G) \supseteq \lim_\xi F \cap \lim_\xi G$;
\item $\lim_\xi^{-1}(x)\cup \powerset(X)$ is a filter for all $x\in X$.
\end{enumerate}
\end{lemma}
In other words, a preconvergence is finitely deep \textup{if and only if} $\lim_\xi^{-1}(x)$ is \emph{directed} (and thus a filter) for all $x\in X$.

\begin{proposition}
Let $X$ be a set and $\xi$ a preconvergence on $X$. Then each of the following statements implies the next:
\begin{enumerate}
\item $\xi$ is pretopological;
\item $\xi$ is Choquet;
\item $\xi$ is finitely deep.
\end{enumerate}
If $\xi$ is a convergence space, then these also imply
\begin{enumerate} \setcounter{enumi}{3}
\item $\xi$ is Kent.
\end{enumerate}
\end{proposition}
\begin{proof}
$(1) \Rightarrow (2)$ By \ref{filtersCoatomistic}, we have
\[ F = \bigcap_{\substack{U\in \powerultrafilters(X) \\ U\geq F}} U. \]
By the pretopological property this means
\[ \lim_\xi(F) = \lim_\xi\left(\bigcap_{\substack{U\in \powerultrafilters(X) \\ U\geq F}} U\right) = \bigcap_{\substack{U\in \powerultrafilters(X) \\ U\geq F}} \lim_\xi(U). \]
Thus $x\in \lim_\xi(F)$ iff $x\in \lim_\xi(U)$ for all ultrafilters $U$ such that $F\subseteq U$.

$(2) \Rightarrow (3)$ Take $F,G\in\powerfilters(X)$. By \ref{finiteDepthLemma} it is enough to prove $\lim_\xi(F\cap G) \supseteq \lim_\xi F \cap \lim_\xi G$. Take $x\in \lim_\xi F \cap \lim_\xi G$. If $U$ is an ultrafilter such that either $F\subseteq U$ or $G\subseteq U$, then $U\to x$.

Now take an ultrafilter $U' \supseteq F\cap G$. Then either $U'\supseteq F$ or $U'\supseteq G$ by \ref{finiteUltrafilterFactorisation}. By the previous remark $U'\to x$. By the Choquet property we conclude that $x\in \lim_\xi(F\cap G)$.

$(3) \Rightarrow (4)$. This is immediate if $\pfilter(x) \to x$ for all $x\in X$, i.e.\ if $\xi$ is a convergence.
\end{proof}

\begin{example}
There exist Choquet spaces that are not pretopological. Take any infinite set $X$. Then there exists a nonprincipal ultrafilter $U$ on $\powerset(X)$ (take the cofinite filter and extend to ultrafilter by ultrafilter lemma, see \ref{filterFreeIFFfinerThanCofinite}).

Take some $x_0$ and let each principal ultrafilter converge to $x_0$. Let no other ultrafilter converge. Define convergence of all other filters by the Choquet property. This convergence is not pretopological: We have $\bigcap \lim^{-1}(x_0) = \big\{\{X\}\big\}$, but $\{X\} \not\to x_0$ because $\{X\}\subseteq U$ and $U \not\to x_0$.
\end{example}

TODO finitely deep = quotient of (pre)topologies; inf of topologies; prime topologies dense in it (?)

\subsection{The lattices of preconvergences and convergences}
\begin{definition}
Let $X$ be a set and $\xi,\zeta$ preconvergences on $X$. We say $\xi$ is \udef{finer} (or \udef{stronger}) than $\zeta$, denoted $\xi \leq \zeta$, if $\lim_\xi F \subseteq \lim_\zeta F$ for all $F\in\powerfilters(X)$. We also say $\zeta$ is \udef{coarser} (or \udef{weaker}) than $\xi$.
\end{definition}
We can think of strength as the ability to stop a filter converging. So $\xi$ is strictly stronger that $\zeta$ if there are filters that converge in $\zeta$, but not in $\xi$. 

\begin{lemma}
Let $X$ be a set and $\xi$ a preconvergence on $X$. Then
\begin{enumerate}
\item $E_{\powerfilters(X), X} \leq \xi \leq U_{\powerfilters(X), X}$;
\item $\xi$ is a convergence \textup{if and only if} $\iota_X \leq \xi$.
\end{enumerate}
\end{lemma}

\begin{definition}
Let $X$ be a set.
\begin{itemize}
\item The \udef{empty preconvergence} on $X$ is $E_{\powerfilters(X), X}$, i.e.\ the limit of all filters is the empty set.
\item The \udef{discrete convergence} $\iota_X$ on $X$ is defined by $x \in \lim_\iota F \iff F = \pfilter{x}$ if $F$ is a proper filter. If $F = \powerset(X)$, then $\lim_\iota F = X$.
\item The \udef{chaotic convergence} on $X$ is $U_{\powerfilters(X), X}$, i.e.\ the limit of all filters is $X$.
\end{itemize}
\end{definition}

\begin{proposition} \label{latticeConvergences}
Let $X$ be a set and and let $\Xi$ be a set of (pre)convergences on $X$. The sets of preconvergences and of convergences are complete bounded lattices and for all $F \in \powerfilters(X)$:
\[ \lim_{\bigwedge \Xi} F = \bigcap_{\xi\in\Xi}\lim_\xi F \quad\text{and}\quad \lim_{\bigvee \Xi} F = \bigcup_{\xi\in\Xi}\lim_\xi F. \]
\begin{itemize}
\item The top of both is the chaotic convergence.
\item The bottom of the lattice of preconvergences is the empty preconvergence.
\item The bottom of the lattice of convergences is the discrete convergence.
\end{itemize}
\end{proposition}

\begin{lemma}
A preconvergence is a convergence \textup{if and only if} it is coarser than the discrete convergence.
\end{lemma}

\subsection{Directional convergence}
\begin{definition}
Let $\sSet{X,\xi}$ be a convergence space, $D\subseteq X$ and $x\in X$. Then a filter $F\in \powerfilters(X)$ is said to \udef{converge to $x$ in the direction $D$} if $F\overset{\xi}{\longrightarrow} x$ and $D\in F$.

We write $F\overset{\xi, D}{\longrightarrow} x$.
\end{definition}

\begin{lemma}
Let $\sSet{X,\xi}$ be a convergence space, $D\subseteq X$ and $x\in X$. Then $F$ converges to $x$ in the direction $D$ \textup{if and only if} there exists a net $\seq{x_i}_{i\in I}\subseteq D$ that converges to $x$ such that $F = \TailsFilter(\seq{x_i}_{i\in I})$.
\end{lemma}
\begin{proof}
If there exists such a net, then $F$ converges to $x$ by definition of net convergence. Every tail of $\seq{x_i}$ is a subset of $D$. So $D\in F$ by upward closure.

Now assume $F$ converges to $x$ in the direction $D$. Consider the index set $I_F = \setbuilder{(A,x)\in F\times X}{x\in A}$ as in \ref{filterIndex}. Now define $I_F^D = I_F\setminus\setbuilder{(A,x)\in F\times X}{x\notin D}$. We claim this set if directed and the associated filter of $I_F^D \to X: (A,x) \mapsto x$ is $F$.

Directedness follows because for all $(A,x), (B,y)\in I_F^D$, the set $A\cap B\cap D$ is not empty as all three sets are elements of the filter $F$ and the filter is not trivial.

Finally we show that $F$ is generated by the tails of $I_F^D \to X: (A,x) \mapsto x$. Take $A\in F$. Then $A\cap D\in F$ and by \ref{tailsFilterIndex}
\[ A\cap D = \setbuilder{y\in X}{(B,y) \geq (A\cap D,x)}. \]
Then $y\in B\subseteq A\cap D \subseteq D$, so this is also a tail of $I_F^D \to X: (A,x) \mapsto x$.
\end{proof}


\section{The vicinity filter}
\begin{definition}
Let $X$ be a set, $\xi$ a preconvergence on $X$ and $x\in X$. The \udef{vicinity filter} of $\xi$ at $x$ is
\[ \vicinity_\xi(x) \defeq \bigcap \lim_\xi^{-1}(x) = \bigcap \setbuilder{F\in \powerfilters(X)}{x\in \lim_\xi F}. \]
A subset $V\subseteq X$ is called a \udef{vicinity} of $x$ for $\xi$ if $V\in \vicinity_\xi(z)$.

We extend the vicinity filters to be defined for elements of $\powerset(X)$ and $\powerset^2(X)$ by contours.
\end{definition}

\begin{lemma}
Let $X$ be a set, $\xi$ a preconvergence on $X$ and $A\subseteq X$. Then
\begin{align*}
\vicinity_\xi(A) &= \bigcap \setbuilder{F\in \powerfilters(X)}{\exists x\in A: \; x\in \lim_\xi F} \\
&= \bigcap \setbuilder{F\in \powerfilters(X)}{A \mesh \lim_\xi F}.
\end{align*}
\end{lemma}

\begin{lemma}
Let $\sSet{X,\xi}$ be a convergence space. Then $\vicinity_\xi$ is an approach \textup{if and only if} $\xi$ is pretopological.
\end{lemma}

\begin{lemma} \label{vicinityOfSetLemma}
Let $\sSet{X,\xi}$ be a convergence space and $A\subseteq X$. Then $\vicinity_\xi(A) \subseteq \upset\{A\}$.
\end{lemma}
\begin{proof}
We have $\vicinity_\xi(A) \subseteq \bigcap_{x\in A}\pfilter{x} = \upset \{A\}$. 
\end{proof}
\begin{corollary} \label{vicinityOfSetCorollary}
For $A\subseteq X$ and $x\in X$, we have
\begin{enumerate}
\item $A \subseteq \bigcap\vicinity_\xi(A)$;
\item $\vicinity_\xi(x) \subseteq \pfilter{x}$;
\item $\{x\}\in\vicinity_\xi(x)^{\mesh}$.
\end{enumerate}
\end{corollary}

\begin{lemma}
Let $X$ be a set, $\xi$ a preconvergence on $X$ and $x\in X$. Then
\[ \vicinity_\xi(x) = \bigcap_{F\in {\lim_\xi}^{-1}(x)\;\cap\; \ultrafilters(\powerset(X))}F. \]
\end{lemma}
\begin{proof}
TODO
\end{proof}

\begin{lemma}
Let $\sSet{X,\xi}$ be a convergence space and $x\in X$. Then
\begin{enumerate}
\item if $F\to x$, then $\vicinity_\xi(x)\lhd F$
\end{enumerate}
\end{lemma}
\begin{proof}
(1) Simple application of \ref{containmentInclusionProperFilter}.
\end{proof}

\begin{lemma} \label{vicinityMapAntitone}
Let $X$ be a set and $\zeta, \xi$ convergences on $X$. If $\zeta \leq \xi$, then $\vicinity_\zeta(x) \supseteq \vicinity_\xi(x)$ for all $x\in X$.
\end{lemma}
\begin{proof}
Assume $\zeta \leq \xi$. For all $F\in \powerfilters(X)$ we have $\lim_\zeta^{-1}(x) \subseteq \lim_\xi^{-1}(x)$, so
\[ \vicinity_\zeta(x) = \bigcap \lim_\zeta^{-1}(x) \supseteq \bigcap \lim_\xi^{-1}(x) = \vicinity_\xi(x). \]
\end{proof}

\subsection{Pretopological convergence}


\section{Base and pavement}
\subsection{Pavement}
\begin{definition}
Let $\sSet{X,\xi}$ be a convergence space.
\begin{itemize}
\item A \udef{pavement} of $\xi$ at a point $x\in X$ is a family of filters $\mathcal{H}$ such that
\[ {\lim_\xi}^{-1}(x) = \upset\mathcal{H}. \]
\item The \udef{paving number} of $\xi$ at $x$ is the least cardinality $\kappa$ such that there is a pavement of $\xi$ of cardinality $\kappa$ at $x$.
\item The \udef{paving number} $\lambda$ of $\xi$ is
\[ \lambda = \sup\setbuilder{\kappa}{\text{$\kappa$ is the paving number of $\xi$ at $x$ for some $x\in X$}}. \]
We say $\xi$ is \udef{$\kappa$-paved} if $\lambda \leq \kappa$.
\item We call the convergence $\xi$ \udef{pretopological} if it is $1$-paved.
\end{itemize}
\end{definition}
Being $\kappa$-paved is a pointwise property.

\begin{proposition}
A finitely deep convergence that is finitely paved is pretopological.
\end{proposition}

\begin{lemma}
Let $\sSet{X,\xi}$ be a pretopological convergence space. Then $\mathcal{H}\in\powerset^2(X)$ is a pavement of $\xi$ \textup{if and only if} $\setbuilder{\vicinity_\xi(x)}{x\in X} \subseteq \mathcal{H}$.
\end{lemma}

\subsection{A base of a convergence}
\begin{definition}
Let $\sSet{X,\xi}$ be a convergence space. A set $\mathcal{Z}\subseteq \powerset(X)$ is called a \udef{base} of the converence $\xi$ if for each point $x\in X$ and each pavement $\mathcal{H}$ of $\xi$ at $x$, each $F\in\mathcal{H}$ is based in $\mathcal{Z}$.

We also say $\xi$ is \udef{based in} $\mathcal{Z}$.
\end{definition}
TODO filter based in set.



\subsubsection{Bases of pretopological convergences}
\begin{definition}
Let $\sSet{X,\xi}$ be a convergence space and $\mathcal{Z}$ a base of $\xi$. The \udef{effective portion} of $\mathcal{Z}$ is
\[ \setbuilder{Z\in \mathcal{Z}}{\exists x\in X: \; Z\in \vicinity_\xi(x)} = \mathcal{Z}\cap \bigcup\setbuilder{\vicinity_\xi(x)}{x\in X}. \]
\end{definition}

\begin{lemma}
Let $\sSet{X,\xi}$ be a pretopological convergence space and $\mathcal{Y}, \mathcal{Z}$ bases of $\xi$. Then
\begin{enumerate}
\item the effective portion of $\mathcal{Z}$ is a base of $\xi$;
\item the effective portions of $\mathcal{Y}$ and $\mathcal{Z}$ are equally fine.
\end{enumerate}
\end{lemma}
\begin{proof}
(1) Immediate because any pavement of $\xi$ contains $\setbuilder{\vicinity_\xi(x)}{x\in X}$.

(2) Let $\mathcal{Y}'$ and $\mathcal{Z}'$ be the respective effective portions. Take $Y\in \mathcal{Y}'$; we need to show that there exists $Z\in \mathcal{Z}'$ such that $Z\subseteq Y$. Take $x\in X$ such that $Y\in \vicinity(x)$. Then there exist $Z \in \mathcal{Z}'$ such that $Z\subseteq Y$ because $\vicinity(x)$ is based in $\mathcal{Z}'$. 
\end{proof}
\begin{corollary} \label{cardinalityPretopologicalBase}
Let $\sSet{X,\xi}$ be a pretopological convergence space with bases $\mathcal{X},\mathcal{Y}$. Then there exists a base $\mathcal{Z} \subseteq \mathcal{X}$ with cardinality smaller than or equal to $\mathcal{Y}$.
\end{corollary}
\begin{proof}
Take the effective portion $\mathcal{X}'$ of $\mathcal{X}$. Because the effective portion $\mathcal{Y}'$ of $\mathcal{Y}$ refines $\mathcal{X}'$, we have
\[ \forall Y\in \mathcal{Y}': \exists X_Y \in \mathcal{X}': \; X_Y\subseteq Y. \]
The set $\mathcal{Z} \defeq \setbuilder{X_Y}{Y\in \mathcal{Y}'} \subseteq \mathcal{X}$ has cardinality smaller than $\mathcal{Y}'$, which has cardinality smaller than $\mathcal{Y}$.
\end{proof}

\section{Adherence and inherence}
\begin{definition}
Let $\sSet{X,\xi}$ be a convergence space and $\mathcal{A}\subseteq \powerset(X)$ a family of subsets. We define
\begin{itemize}
\item the \udef{adherence} of $\mathcal{A}$ as
\[ \adh_\xi(\mathcal{A}) = \bigcup_{\mathcal{A}\lhd F}\lim_\xi F; \]
\item the \udef{inherence} of $\mathcal{A}$ as
\[ \inh_\xi(\mathcal{A}) = \setbuilder{x\in X}{x\in \lim_\xi F \implies F\amesh\mathcal{A}}. \]
\end{itemize}
\end{definition}

\begin{lemma}
\begin{enumerate}
\item If $F\to x$, then $x\in \adh_\xi(F)$;
\item If $F = \mathfrak{F}(\mathcal{A}), then \adh_\xi(F) = \adh_xi(\mathcal{A})$.
\end{enumerate}
\end{lemma}

\begin{lemma} \label{singletonAdherence}
Let $\sSet{X,\xi}$ be a convergence space and $x\in X$. Then $\adh_\xi(\{x\}) = \lim_\xi \pfilter{x}$.
\end{lemma}
\begin{proof}
If $F$ is a filter such that $\{x\} \lhd F$, then $F = \pfilter{x}$ by \ref{ultrafilterContainment}. So $\adh_\xi \{x\} = \lim_\xi \pfilter{x}$.
\end{proof}

\begin{proposition}
Let $\sSet{X,\xi}$ be a convergence space and $F\subseteq \powerfilters(X)$ a filter. Then $\ker(F) \subseteq \adh_\xi(F)$
\end{proposition}
\begin{proof}
Take $x\in \ker(F)$. Then $\pfilter{x}\amesh F$, so $F\vee \pfilter{x}$ is a proper filter that converges to $x$. Now $F\lhd F\vee \pfilter{x}$ by \ref{containmentInclusionProperFilter}, so $x\in \lim_\xi F\vee \pfilter{x} \subseteq \adh_\xi(F)$.
\end{proof}

\begin{definition}
Let $X$ be a set, $\xi$ a convergence on $X$ and $A\subseteq X$ a subset. We define
\begin{itemize}
\item the \udef{adherence} $\adh_\xi(A)$ of $A$ by
\[ x\in \adh_\xi(A) \defequiv A\in \vicinity_\xi(x)^{\mesh}; \]
\item the \udef{inherence} $\inh_\xi(A)$ of $A$ by
\[ x\in \inh_\xi(A) \defequiv A\in \vicinity_\xi(x). \]
\end{itemize}
\end{definition}

\begin{proposition} \label{principalAdherenceInherence}
Let $\sSet{X,\xi}$ be a convergence space and $A\subseteq X$ a subset. Then
\begin{enumerate}
\item $x\in \adh_\xi(A)$ \textup{if and only if} $A\in \vicinity_\xi(x)^{\mesh}$;
\item $x\in \inh_\xi(A)$ \textup{if and only if} $A\in \vicinity_\xi(x)$.
\end{enumerate}
\end{proposition}
\begin{proof}
TODO
\end{proof}
\begin{corollary} \label{setAdherenceInherence}
Let $\sSet{X,\xi}$ be a convergence space and $A,B\subseteq X$ subsets. Then
\[ \adh_\xi(A) \mesh B \qquad\iff\qquad \{A\} \amesh \vicinity_\xi(B). \]
\end{corollary}
\begin{proof}
We have
\begin{align*}
\adh_\xi(A) \mesh B &\iff \exists b\in B: b\in \adh_\xi(A) \\
&\iff \exists b\in B: A\in \vicinity_\xi(b)^{\mesh} \\
&\iff A \in \bigcup_{b\in B}\vicinity_\xi(b)^{\mesh} = \left(\bigcap_{b\in B}\vicinity_\xi(b)\right)^{\mesh} = \vicinity_\xi(B)^{\mesh}.
\end{align*}
\end{proof}

\begin{proposition} \label{adherenceInherenceCharacterisation}
Let $X$ be a set, $\xi$ a convergence on $X$ and $A \subseteq X$ a set. Then
\begin{enumerate}
\item $\displaystyle x\in \adh_\xi(A) \iff A \in \bigcup_{F\in {\lim_\xi}^{-1}(x)} F^{\mesh} \iff A \in \bigcup_{F\in {\lim_\xi}^{-1}(x)} F$;
\item $x\in \adh_\xi(A) \iff \exists U\in \vicinity_\xi(x): \; U\setminus A \notin \vicinity_\xi(x)$;
\item $\displaystyle \adh_\xi(A) = \bigcup_{A \in F^{\mesh}}\lim_\xi F = \bigcup_{\substack{\text{$G$ proper filter} \\ A\in G}} \lim_\xi G$;
\item $\displaystyle x\in \inh_\xi(A) \iff A \in \bigcap_{F\in {\lim_\xi}^{-1}(x)} F$;
\item $\displaystyle \inh_\xi(A) = \bigcap_{A \in F}\lim_\xi F$.
\end{enumerate}
\end{proposition}
\begin{proof}
TODO
\end{proof}
\begin{corollary} \label{inherenceComplementAdherence}
Let $X$ be a set, $\xi$ a convergence on $X$ and $A \subseteq X$. Then
\[ (\inh_\xi A)^c = (\adh_\xi A^c). \]
\end{corollary}
\begin{corollary} \label{inherenceAdherenceProperties}
Let $X$ be a set, $\xi$ a convergence on $X$ and $A,B \subseteq X$. Then
\begin{enumerate}
\item $\adh_\xi \emptyset = \emptyset$;
\item if $A \subseteq B$, then $\adh_\xi A \subseteq \adh_\xi B$;
\item $A\subseteq \adh_\xi A$;
\item $\adh_\xi(A\cup B) = \adh_\xi A \cup \adh_\xi B$.
\end{enumerate}
and
\begin{enumerate} \setcounter{enumi}{4}
\item $\inh_\xi X = X$;
\item if $A \subseteq B$, then $\inh_\xi A \subseteq \inh_\xi B$;
\item $\inh_\xi A \subseteq A$;
\item $\inh_\xi(A\cap B) = \inh_\xi A \cap \inh_\xi B$.
\end{enumerate}
\end{corollary}

\begin{lemma}
Let $X$ be a set, $\xi$ a convergence on $X$ and $A \subseteq X$ a set. Then
\[ \adh_\xi A = \bigcup_{A \in F^{\mesh}}\lim_\xi F = \bigcup_{\substack{\text{$G$ proper filter} \\ A\in G}}\lim_\xi G. \]
\end{lemma}
\begin{proof}
The first equality is obvious. For the second equality: Let $F$ be a filter such that $A\in F^{\mesh}$ then $F \vee \upset A$ contains $A$ and is a proper filter by \ref{joinProperFilter}, so $\lim_\xi F \subseteq \lim_x F \vee \upset A \subseteq \bigcup_{A\in G}\lim_\xi G$.

For the other inclusion: $A\in G$ iff $\upset\{A\} \subseteq G$, so $G = G \vee \upset A$. But $G$ is a proper filter, so it must mesh with $\upset A$. In particular it must mesh with $A$.
\end{proof}



\begin{proposition}
Let $X$ be a set, $\xi$ a convergence on $X$ and $A \subseteq X$. Then the following are equivalent:
\begin{enumerate}
\item $x\in \adh_\xi A$;
\item $A\in \bigcup_{F\in {\lim_\xi}^{-1}(x)} F^{\mesh}$;
\item $A \mesh \vicinity_\xi(x)$.
\end{enumerate}
\end{proposition}
\begin{proof}
$(1) \Leftrightarrow (2)$ We have
\[ x\in \bigcup_{A\in F^{\mesh}}\lim_\xi F \iff \exists F: A \in F^{\mesh} \land x\in \lim_\xi F \iff A\in  \bigcup_{F\in {\lim_\xi}^{-1}(x)} F^{\mesh}. \]

$(2) \Leftrightarrow (3)$ We have, using \ref{orderReversingSimilarityLatticeOperations},
\[ A \mesh \vicinity_\xi(x) \iff A\in \left(\vicinity_\xi(x)\right)^{\mesh} = \left(\bigcap_{F\in {\lim_\xi}^{-1}(x)}F\right)^{\mesh} = \bigcup_{F\in {\lim_\xi}^{-1}(x)} F^{\mesh}. \]
\end{proof}

\begin{lemma} \label{inherenceAdherenceInclusion}
Let $X$ be a set, $\zeta, \xi$ convergences on $X$ such that $\zeta \leq \xi$ and $A\subseteq X$ a subset. Then
\[ \inh_\xi(A) \subseteq \inh_\zeta(A) \subseteq A \subseteq \adh_\zeta(A) \subseteq \adh_\xi(A). \]
\end{lemma}
\begin{proof}
We only need to show the first and last inclusion.

From \ref{vicinityMapAntitone} we have $\vicinity_\xi(x) \subseteq \vicinity_\zeta(x)$ and $\vicinity_\zeta(x)^{\mesh} \subseteq \vicinity_\xi(x)^{\mesh}$ for all $x\in X$.

So we have the implications
\[ x\in \inh_\xi(A) \implies A\in \vicinity_\xi(x) \implies A\in \vicinity_\zeta(x) \implies x\in \inh_\zeta(A) \]
and
\[ x\in \adh_\zeta(A) \implies A\in \vicinity_\zeta(x)^{\mesh} \implies A\in \vicinity_\xi(x)^{\mesh} \implies x\in \adh_\xi(A). \]
\end{proof}

\subsection{General adherence (TODO)}
\begin{definition}
Let $X$ be a set, $\xi$ a convergence on $X$ and $\mathcal{A} \subseteq \powerset(X)$. The \udef{adherence} of $\mathcal{A}$ is defined as
\[ \adh_\xi\mathcal{A} \defeq \bigcup_{\substack{F \in \powerfilters(X) \\ F \mesh \mathcal{A}}} \lim_\xi F. \]
The \udef{(principal) adherence} of a set $A\subseteq X$ is the adherence of $\{A\}$.
\end{definition}
Clearly if $\xi,\zeta$ are convergences on $X$, then $\xi \leq \zeta$ implies $\adh_\xi\mathcal{A} \subseteq \adh_\zeta\mathcal{A}$.

\begin{lemma}
Let $X$ be a set, $\xi$ a convergence on $X$ and $\{A_i\}_{i\in I} \subseteq \powerset(X)$. Then
\[ \adh_\xi \{A_i\}_{i\in I} \subseteq \bigcap_{i\in I}\adh_\xi A_i. \]
\end{lemma}
\begin{proof}
We have $x\in \adh_\xi \{A_i\}_{i\in I}$ iff $\exists F: x\in \lim F \land F\amesh \{A_i\}_{i\in I}$ iff $\exists F: \forall i\in I: x\in \lim F \land F\amesh A_i$. This implies $\forall i\in I: \exists F: x\in \lim F \land F\amesh A_i$ iff $\forall i\in I: x\in \adh_\xi A_i$.
\end{proof}



\begin{lemma}
Let $X$ be a set and $F$ be a filter in $\powerfilters(X)$. Then $\ker F = \adh_{\iota_X} F$.
\end{lemma}
\begin{proof}
Let $x\in X$. Then $\pfilter{x} \amesh F$ iff $\Big\{\{x\}\Big\} \amesh F$ iff $x\in \ker F$.
\end{proof}
\begin{corollary}
Let $\xi$ be a convergence on a set $X$ and $F$ be a filter in $\powerfilters(X)$. Then $\ker F \subseteq \adh_\xi F$.
\end{corollary}
\begin{proof}
This follows from $\iota_X \leq \xi$.
\end{proof}

\subsection{Inherence}
\begin{definition}
Let $X$ be a set, $\xi$ a convergence on $X$ and $A \subseteq X$. The \udef{(principal) inherence} of $A$, denoted $\inh_\xi A$, is defined by
\[ x\in \inh_\xi A \defequiv A\in \vicinity_\xi(x). \]
\end{definition}

\begin{lemma}
Let $X$ be a set, $\xi$ a convergence on $X$ and $A \subseteq X$. Then
\begin{enumerate}
\item $\inh_\xi A = (\adh_\xi A^c)^c$;
\item $x\in \inh_\xi A \iff (x\in \lim_\xi F \implies A\in F)$.
\end{enumerate}
\end{lemma}

\begin{proposition}
Let $X$ be a set, $\xi$ a convergence on $X$ and $A,B \subseteq X$. Then
\begin{enumerate}
\item $\inh_\xi X = X$;
\item if $A \subseteq B$, then $\inh_\xi A \subseteq \inh_\xi B$;
\item $\inh_\xi A\subseteq A$;
\item $\inh_\xi(A\cap B) = \inh_\xi A \cap \inh_\xi B$.
\end{enumerate}
\end{proposition}

\begin{lemma} \label{subsetWithVicinitiesInInherence}
Let $\sSet{X,\xi}$ be a convergence space and $A,B\subseteq X$ subsets. If for every $a\in A$ there exists a vicinity of $a$ that is a subset of $B$, then $A\subseteq \inh_\xi(B)$.
\end{lemma}
\begin{proof}
Assume that for every $a\in A$ there exists a $U_a \in \vicinity_\xi(a)$ such that $U_a \subseteq B$. Then $B\in \vicinity_\xi(a)$ for all $a$ in $A$ and thus $a\in \inh_\xi(B)$ for all $a\in A$.
\end{proof}

\subsection{Topology}
\subsubsection{Open and closed sets}
\begin{definition}
Let $X$ be a set and $\xi$ a convergence on $X$.
\begin{itemize}
    \item A subset $O \subseteq X$ is called \udef{open} if $\inh_\xi(O) = O$.
    \item A subset $C \subseteq X$ is called \udef{closed} if $\adh_\xi(C) = C$.
\end{itemize}
The set of all open sets in $\sSet{X,\xi}$ is called the \udef{topology} of $\sSet{X,\xi}$ and is denoted $\topology_\xi$.
\end{definition}

\begin{lemma}
Let $X$ be a set, $\xi$ a convergence on $X$ and $A\subseteq X$ a subset. Then $A$ is open \textup{if and only if} $A^c$ is closed.
\end{lemma}
\begin{proof}
Assume $\inh_\xi(A) = A$. Then $A^c = \inh_\xi(A)^c = \adh_\xi(A^c)$.
\end{proof}

\begin{lemma} \label{completeClosureTopology}
Let $X$ be a set and $\xi$ a convergence on $X$. Then
\begin{enumerate}
\item the topology $\topology_\xi$ is closed under arbitrary unions;
\item the set of closed sets in $\sSet{X,\xi}$ is closed under arbitrary intersections.
\end{enumerate}
\end{lemma}
\begin{proof}
(1) Let $\{A_i\}_{i\in I}$ be a set of open sets. From \ref{inherenceAdherenceProperties} and \ref{orderPreservingFunctionLatticeOperations} we get $\inh\left(\bigcup_{i\in I}A_i\right) \subseteq \bigcup_{i\in I}A_i$ and $\bigcup_{i\in I}\inh(A_i) \subseteq \inh\left(\bigcup_{i\in I} A_i\right)$. Thus
\[ \bigcup_{i\in I}A_i  = \bigcup_{i\in I}\inh_(A_i) \subseteq \inh\left(\bigcup_{i\in I} A_i\right) \subseteq \bigcup_{i\in I}A_i. \]

(2) Let $\{A_i\}_{i\in I}$ be a set of closed sets. From \ref{inherenceAdherenceProperties} and \ref{orderPreservingFunctionLatticeOperations} we get $\bigcap_{i\in I}A_i \subseteq \adh\left(\bigcap_{i\in I}A_i\right)$ and $\adh\left(\bigcap_{i\in I} A_i\right)\subseteq \bigcap_{i\in I}\adh(A_i)$. Thus
\[ \bigcap_{i\in I}A_i \subseteq \adh\left(\bigcap_{i\in I}A_i\right) \subseteq \bigcap_{i\in I}\adh(A_i) = \bigcap_{i\in I}A_i. \]
\end{proof}
\begin{corollary}
The topology $\topology_\xi$ is a complete sublattice of $\powerset(X)$.
\end{corollary}

\begin{lemma} \label{openClosedCriteria}
Let $\sSet{X,\xi}$ be a convergence space and $O,C\subseteq X$ subsets. The following are equivalent:
\begin{enumerate}
\item $O$ is open;
\item $O\subseteq \inh(O)$;
\item $O\subseteq \interior(O)$;
\item $O\in \vicinity(O)$;
\item $O\in \neighbourhood(O)$;
\item for all $x\in O$ there exists $U_x\in \vicinity(x)$ such that $U_x\subseteq O$;
\end{enumerate}
as are the following:
\begin{enumerate}
\item $C$ is closed;
\item $\adh(C)\subseteq C$;
\item $\closure(C)\subseteq C$.
\end{enumerate}
\end{lemma}
\begin{proof}
TODO \ref{subsetWithVicinitiesInInherence}
\end{proof}

\begin{lemma}
Let $X$ be a set, $\zeta, \xi$ convergences on $X$ such that $\zeta\leq\xi$ and $A\subseteq X$ a subset.
\begin{enumerate}
\item if $A$ is open in $\xi$, then it is also open in $\zeta$;
\item if $A$ is closed in $\xi$, then it is also closed in $\zeta$.
\end{enumerate}
\end{lemma}
\begin{proof}
Assume $A$ is open in $\xi$. Then $\inh_\xi(A) = A$. But $\inh_\xi(A) \subseteq \inh_\zeta(A) \subseteq A$ by \ref{inherenceAdherenceInclusion}, so $\inh_\xi(A) = \inh_\zeta(A) = A$. The argument for closedness is similar.
\end{proof}

\subsubsection{Interior, closure and boundary}
\begin{definition}
Let $\sSet{X,\xi}$ be a convergence space.
\begin{itemize}
\item The dual closure mapping of $A\in \powerset(X)$ into $\topology_\xi$ is called the \udef{interior} of $A$, denoted $\interior(A)$ or $A^\circ$. 
\item The closure mapping of $A\in \powerset(X)$ into the set of closed sets in $X$ is called the \udef{closure} of $A$, denoted $\closure(A)$ or $\overline{A}$. 
\end{itemize}
The \udef{boundary} of $A\in\powerset(X)$ is $\partial A \defeq \overline{A}\setminus A^{\circ}$
\end{definition}

\begin{lemma}
$\interior^2 = \interior$ and $\closure^2 = \closure$.
\end{lemma}

\begin{proposition}
???

$\interior(\closure(A))$ and $\closure(\interior(A))$?
\end{proposition}

\subsubsection{Neighbourhoods}
\begin{definition}
Let $\sSet{X,\xi}$ be a convergence space and $x\in X$. We call a subset $A\subseteq X$ a \udef{neighbourhood} of $x$ is there exists an open set $O$ such that $x\in O \subseteq A$.

The set of all neighbourhoods of $x$ is denoted $\neighbourhood_\xi(x)$.
\end{definition}

\begin{proposition} \label{interiorClosureMembership}
Let $\sSet{X,\xi}$ be a convergence space, $A\subseteq X$ and $x\in X$. Then
\begin{enumerate}
\item $x\in \interior_\xi(A) \iff A \in \neighbourhood_\xi(x)$;
\item $x\in \closure_\xi(A) \iff A \in \neighbourhood_\xi(x)^{\mesh}$.
\end{enumerate}
\end{proposition}

\begin{lemma} \label{interiorModificationNeighbourhoods}
Let $\sSet{X,\xi}$ be a convergence space, $A\subseteq X$ and $x\in X$. Then
\begin{enumerate}
\item $A\in\neighbourhood(x)$ \textup{if and only if} $\interior(A)\in\neighbourhood(x)$;
\item if $\mathcal{A}\in\powerset^2(X)$ is a base for $\neighbourhood(x)$, then $\interior^{\imf}(\mathcal{A})$ is also a base for $\neighbourhood(x)$.
\end{enumerate}
\end{lemma}
\begin{proof}
(1) We have $A\in\neighbourhood(x) \iff x\in \interior(A) = \interior^2(A) \iff \interior(A) \in \neighbourhood(x)$.

(2) First $\interior^{\imf}(\mathcal{A})$ is a filter base, because it is closed under finite intersections: for all $\interior(A),\interior(B)\in \interior^{\imf}(\mathcal{A})$ we have $\interior(A)\cap\interior(B) = \interior(A\cap B) \in \interior^{\imf}(\mathcal{A})$ because $A\cap B \in \mathcal{A}$.

We clearly have $\mathcal{A} \preceq \interior^{\imf}(\mathcal{A})$ because $\interior(A) \subseteq A$. For the opposite inequality, we need to show that for all $A\in \mathcal{A}$, $\interior(A)$ is a neighbourhood of $x$. This is point (1).
\end{proof}

\subsubsection{Topological convergence}
\begin{definition}
A pretopological convergence space $\sSet{X,\xi}$ is called \udef{topological} if the topology $\topology_\xi$ is a base of $\xi$.
\end{definition}
\begin{definition}
A pretopological convergence space $\sSet{X,\xi}$ is called \udef{topological} if $\adh_\xi^2 = \adh_\xi$.
\end{definition}
TODO pretopological assumption necessary?

\begin{proposition} \label{pretopologicalSpaceTopological}
Let $\sSet{X,\xi}$ be a pretopological convergence space. The following are equivalent:
\begin{enumerate}
\item $\xi$ is topological;
\item $\inh_\xi^2 = \inh_\xi$;
\item $\adh_\xi = \closure_\xi$;
\item $\inh_\xi = \interior_\xi$;
\item $\vicinity_\xi(x) = \neighbourhood_\xi(x)$ for all $x\in X$;
\item $\forall U\in \vicinity_\xi(x): \exists V\in \vicinity_\xi(x): \forall y\in V: U\in\vicinity_\xi(y)$.
\end{enumerate}
\end{proposition}
\begin{proof}
TODO
\end{proof}

\begin{proposition}
Let $\sSet{X,\xi}$ be a topological convergence space. Then $\xi$ is topological \textup{if and only if} $\neighbourhood_\xi(x) \overset{\xi}{\longrightarrow} x$ for all $x\in X$.
\end{proposition}

\subsection{Dense sets}
\begin{definition}
Let $\sSet{X, \xi}$ be a convergence space and $A$ a subset. The subset $A$ is called \udef{dense} in $X$ if $\adh_\xi(A) = X$.
\end{definition}

\subsubsection{Strict density}
TODO strict density (Colebunders)

\begin{lemma} \label{openDensityLemma}
Let $\sSet{X, \xi}$ be a convergence space and $A$ a subset.
If $A^c$ contains a non-empty open set, then $A$ is not dense,
\end{lemma}
\begin{proof}
If $A$ is dense, then $\adh(A) = \inh(A^c)^c = X$, so $\inh(A^c) = \emptyset$. If $A^c$ contains an open set $B$, then $\inh(A^c) \supseteq \inh(B) = B$. This means that $\inh(A^c) \neq \emptyset$ and thus that $A$ is not dense.
\end{proof}

\subsection{Accumulation points}
\begin{definition}
Let $\sSet{X,\xi}$ be a convergence space and $\mathcal{A}\subseteq \powerset(X)$ a family of subsets. A point $x\in X$ is called an \udef{accumulation point} of $\mathcal{A}$ if $\mathcal{A}\amesh \vicinity_\xi(x)$.
\end{definition}

TODO $x\in \adh_\xi(A\setminus\{x\})$.

\begin{proposition} \label{subfilterToAccumulationPoint}
Let $\sSet{X,\xi}$ be a pretopological convergence space, $F\in\powerfilters(X)$ and $x\in X$. If $x$ is an accumulation point of $F$, then there exists a proper filter $G \geq F$ such that $G\overset{\xi}{\longrightarrow} x$. 
\end{proposition}
\begin{proof}
By \ref{joinProperFilter}, we have that $F\cup \vicinity_\xi(x)$ is a proper filter. We can take this filter to be $G$.
\end{proof}

\subsection{Cover}
\begin{definition}
Let $X$ be a set, $\xi$ a convergence on $X$, $A\subseteq X$ and $\mathcal{A}\subseteq \powerset(X)$.
We say $\mathcal{A}$ is an \udef{$\xi$-cover} (or simply \udef{cover}) of $A$ if every filter converging to a point in $A$ contains an element of $\mathcal{A}$. We write $\mathcal{A} \succ_\xi A$.
\end{definition}
So we have
\[ \mathcal{A} \succ_\xi A \;\iff\; \forall F\in \powerfilters(X): \Big( \lim_\xi F \amesh A \implies F \mesh \mathcal{A} \Big). \]

\begin{proposition}
Let $X$ be a set, $\xi$ a convergence on $X$, $A\subseteq X$ and $\mathcal{A}\subseteq \powerset(X)$. Then
\[ \mathcal{A} \succ_\xi A \quad\iff\quad \adh_\xi [\mathcal{A}]^c \perp A \]
\end{proposition}


\section{Pointwise properties}
\begin{definition}
A property $\mathbf{P}$ of preconvergences is called \udef{pointwise} if there exists a property $\mathbf{Q}$ of classes of filters such that for all preconvergences $\xi$ on $X$,
\[ \xi \in \mathbf{P} \iff \forall x\in X: {\lim_\xi}^{-1}(x) \in \mathbf{Q}. \]
\end{definition}

\subsection{Isolation and primeness}
\begin{definition}
Let $\sSet{X,\xi}$ be a convergence space.
\begin{itemize}
\item A point $x\in X$ is called \udef{isolated} if ${\lim_\xi}^{-1}(x) = \Big\{ \pfilter{x} \Big\}$.
\item The convergence space $\sSet{X,\xi}$ is called \udef{prime} if it contains at most one non-isolated point. Such a point is called \udef{distinguished}.
\end{itemize}
\end{definition}

A convergence space is discete if and only if each point is isolated. A discrete convergence space is prime.


\section{Examples of (pre)convergences}
\subsection{Preconvergences on two-point sets}
\subsection{Preconvergences on thee-point sets}

\subsection{Convergences on ordered sets}
\subsubsection{Ordering of filters}
\begin{definition}
Let $\sSet{X,\leq}$ be a poset and $F,G\in\powerfilters(X)$. We define
\begin{align*}
F \leq_w G \qquad&\defequiv\qquad \liminf F \leq \limsup G; \\
F \leq_s G \qquad&\defequiv\qquad \limsup F \leq \liminf G.
\end{align*}
We call $\leq_w$ the \udef{weak ordering} of filters and $\leq_s$ the \udef{strong ordering} of filters.
\end{definition}

\begin{lemma} \label{filterInequalityCriterion}
Let $\sSet{X,\leq}$ be a poset and $F,G\in\powerfilters(X)$. Then
\begin{align*}
F \leq_w G \qquad &\iff\qquad \forall A\in F, B\in G: \exists x\in A, y\in B: \; x \leq y; \\
F \leq_s G \qquad &\iff\qquad \exists A\in F, B\in G: \forall x\in A, y\in B: \; x \leq y.
\end{align*}
\end{lemma}
\begin{proof}
We have the equivalences
\begin{align*}
F \leq_w G &\iff \forall A\in F, B\in G: \exists x\in A, y\in B: \; x \leq y \\
&\iff \forall A\in F, B\in G: \bigwedge A \leq \bigvee B \\
&\iff \bigvee_{A\in F}\bigwedge A \leq \bigwedge_{B\in G}\bigvee B \\
&\iff \liminf F \leq \limsup G.
\end{align*}
The other equivalence is proved similarly.
\end{proof}

\begin{lemma}
Let $\sSet{X,\leq}$ be a poset and $F,G\in\powerfilters(X)$. Then
\begin{align*}
F \leq_w G \qquad &\iff\qquad \forall C\in F\otimes G: \exists (x,y) \in A: \; x \leq y; \\
F \leq_s G \qquad &\iff\qquad \exists C\in F\otimes G: \forall (x,y) \in A: \; x \leq y.
\end{align*}
\end{lemma}
\begin{proof}
First assume $F\leq_w G$. Take $C\in F\otimes G$. Then there exist $A\in F, B\in G$ such that $A\times B \subseteq C$. By assumption, we can find $x\in A$ and $y\in B$ such that $x\leq y$. In this case $(x,y)\in C$.

For the converse, take $A\in F$ and $B\in G$. Then $A\times B\in F\otimes G$ and by assumption there exists $(x,y)\in A\times B$ such that $x\leq y$.

TODO $\leq_s$
\end{proof}
TODO contours for propositions.


\subsubsection{Convergence compatible with the order}
\begin{definition}
Let $\sSet{X,\leq}$ be a poset. A convergence $\xi$ in $X$ is called
\begin{itemize}
\item \udef{compatible with the weak order} if $F \leq_w G$, $x\in\lim_\xi F$ and $y\in \lim_\xi G$ imply $x\leq y$;
\item \udef{compatible with the strong order} if $F \leq_s G$, $x\in\lim_\xi F$ and $y\in \lim_\xi G$ imply $x\leq y$.
\end{itemize}
\end{definition}


\begin{lemma}
Let $\sSet{X,\leq}$ be a poset and $\xi$ a convergence on $X$. Then
\begin{enumerate}
\item $\xi$ is compatible with the weak order \textup{if and only if} $\lim_\xi F\subseteq \{\liminf F\}\cap\{\limsup F\}$ for all $F\in\powerfilters(X)$;
\item $\xi$ is compatible with the strong order \textup{if and only if} $\lim_\xi F\subseteq \interval{\liminf F, \limsup F}$ for all $F\in\powerfilters(X)$.
\end{enumerate}
\end{lemma}
\begin{proof}
We have $F \leq \pfilter{x} \iff \liminf F \leq x$ by \ref{filterInequalityCriterion}. This implies 
$F \leq \pfilter{\liminf F}$ and thus for all $y\in\lim_\xi F$ we must have $y\leq \liminf F$ by compatability of the order. By a similar argument, we have $\limsup F \leq y$.
\end{proof}
\begin{corollary} \mbox{}
The weakest convergence compatible with the weak order is given by $\lim F = \liminf F =\limsup F$, if they coincide, for all $F\in \powerfilters(X)$.
\end{corollary}
TODO: there is no weakest convergence compatible with the strong order??

\subsubsection{Order closure and order regularity}
\begin{definition}
Let $\sSet{X, \leq}$ be a poset and $A\subseteq X$ a subset. The \udef{order closure} of $A$ is the set
\[ \closure_o(A) \defeq \setbuilder{x\in X}{\exists a,b\in A: a\leq x\leq b}. \]
We call a convergence $\xi$ on $X$ \udef{order-regular} if for all $F\in \powerfilters(X)$, the filter
\[ \setbuilder{\closure_o(A)}{A\in F} \]
converges to the same points as $F$.
\end{definition}
Order closure is closure into the lattice of (closed) intervals TODO. $\interval{a,b}\vee \interval{c,d} = \interval{(a\wedge c), (b\vee d)}$ and $\interval{a,b}\wedge \interval{c,d} = \interval{(a\vee c), (b\wedge d)}$.


\begin{lemma}
A convergence is order-regular \textup{if and only if} it is based in the closed intervals.
\end{lemma}

\begin{proposition}
Let $\sSet{X,\leq}$ be a poset with order-regular convergence $\xi$ and $F,G,H\in \powerfilters(X)$. If $F,G\overset{\xi}{\longrightarrow} x\in X$ and
\[ \forall A\in F,B\in G: \exists C\in H: \;\exists a\in A, b\in B:\; C\subseteq \interval{a,b},  \]
then $H\overset{\xi}{\longrightarrow} x$.
\end{proposition}
\begin{proof}
TODO
\end{proof}
\begin{corollary}[Squeeze theorem]
Let $\seq{x_n}, \seq{y_n}$ and $\seq{z_n}$ be sequences in $X$ such that $\seq{x_n}, \seq{y_n}\overset{\xi}{\longrightarrow} x\in X$ and
\[ \forall n\in \N: \quad x_n \leq z_n \leq y_n, \]
then $\seq{y_n} \overset{\xi}{\longrightarrow} x$.
\end{corollary}

\subsubsection{Order convergence}
\url{https://core.ac.uk/download/pdf/82382859.pdf}
TODO move down!

\begin{definition}
Let $\sSet{P,\leq}$ be a poset and $F\in\powerfilters(P)$. Then $F$ converges to $x$ in the \udef{order convergence} on $P$ if there exist nets $\seq{l_i}_{i\in I}$ and $\seq{u_i}_{i\in I}$ in $P$ such that
\begin{itemize}
    \item $\seq{l_i}_{i\in I}$ is increasing and $\sup_{i\in I} l_i = x$;
    \item $\seq{u_i}_{i\in I}$ is decreasing and $\inf_{i\in I} u_i = x$;
    \item $\setbuilder{[l_i, u_i]}{i\in I} \subseteq F$.
\end{itemize}
We say $F$ is \udef{bounded by} $\seq{l_i}_{i\in I}$ and $\seq{u_i}_{i\in I}$.
\end{definition}

TODO: initial or final??

\begin{example}
Order convergence not necessarily topological.
\end{example}


\begin{definition}
Let $\sSet{P,\leq}$ be a poset equipped with order convergence, $\sSet{X,\xi}$ a convergence space, $x_0\in P$ and $f: P\to X$ a function. Then
\begin{itemize}
\item $f$ is called \udef{left continuous} at $x_0$ if $f|_{\downset x_0}$ is continous at $x_0$;
\item $f$ is called \udef{right continuous} at $x_0$ if $f|_{\upset x_0}$ is continous at $x_0$.
\end{itemize}
\end{definition}

\begin{proposition} \label{leftRightConvergence}
Let $\sSet{P,\leq}$ be a poset equipped with order convergence, $\sSet{X,\xi}$ a convergence space of finite depth and $f: P\to X$ a function. Then $f$ is continuous at $x_0\in P$ \textup{if and only if} it is left and right continuous at $x_0$.
\end{proposition}
\begin{proof}
If $f$ is continuous at $x_0\in P$, then it is left and right continuous at $x_0$ by \ref{continuityRestrictionExpansion}.

Conversely, assume $f$ is left and right continuous and take $F$ bounded by $\seq{l_i}_{i\in I}$ and $\seq{u_i}_{i\in I}$ such that $F\to x_0$. Define $F_0 = \mathfrak{F}\setbuilder{[l_i, u_i]}{i\in I} = \upset \setbuilder{[l_i, u_i]}{i\in I}$ and note that for any $A\in F_0$, there exists an $i\in I$ such that $[l_i,u_i]\subseteq A$.


Then $F_0|_{\downset x_0}$ is bounded by $\seq{l_i}_{i\in I}$ and $\seq{x_0}$ and $F_0|_{\upset x_0}$ is bounded by $\seq{x_0}$ and $\seq{u_i}_{i\in I}$. So $f[F_0|_{\downset x_0}] \to f(x_0)$ and $f[F_0|_{\upset x_0}] \to f(x_0)$, meaning $f[F_0|_{\downset x_0}]\cap f[F_0|_{\upset x_0}] \to f(x_0)$ by finite depth.

We conclude by showing that $f[F_0|_{\downset x_0}]\cap f[F_0|_{\upset x_0}] \preceq f[F]$. To that end, take $A\in f[F_0|_{\downset x_0}]\cap f[F_0|_{\upset x_0}]$. We need to show that there is a subset of $A$ in $f[F]$.

Indeed, let $A_1\in F_0|_{\downset x_0}$ be such that $f[A_1] = A$ and $A_2\in F_0|_{\upset x_0}$ be such that $f[A_2] = A$.
Then there exist $i,j\in I$ such that $[l_i,x_0]\subseteq A_1$ and $[x_0, u_j]\subseteq A_2$. Set $k = \max\{i,j\}$.

Now $f\Big[[l_k,u_k]\Big] = f\Big[[l_k,x_0]\Big] \cup f\Big[[x_0,u_k]\Big]\subseteq f[A_1]\cup f[A_2] = A$ and also $f\Big[[l_k,u_k]\Big]\in F$ by order convergence.
\end{proof}

\subsubsection{Scott convergence}

\begin{definition}
Let $L$ be a complete lattice.
\begin{itemize}
\item The \udef{Scott convergence}, or \udef{lower convergence}, $S_*$ on $L$ is defined by
\[ x \in \lim_{S_*} F \iff x \leq \bigvee_{A\in F}\bigwedge_{a\in A}a = \liminf F. \]

\item The \udef{upper convergence}, $S^*$ on $L$ is the Scott convergence on the dual $L^o$. This means it is defined by
\[ x \in \lim_{S^*} F \iff x \geq \bigwedge_{A\in F}\bigvee_{a\in A}a = \limsup F. \]
\item The convergence $S_* \wedge S^*$ is called \udef{convergence in order}.
\end{itemize}
\end{definition}
Note that $F$ is a filter in $\powerfilters(L)$, not a filter in $\filters(L)$.

\begin{proposition}
Convergence in order is Hausdorff.
\end{proposition}
\begin{proof}
This is due to the distributive inequality TODO ref.
\end{proof}

\chapter{Continuity}

\section{Mapping filters}

\begin{proposition} \label{preimageFilter}
Let $X,Y$ be sets, $F\in\powerfilters(X)$, $G\in\powerfilters(Y)$ and $f: X\to Y$ a function.
\begin{enumerate}
\item if $\im(f)\in G$, then there exists a $H\in\powerfilters(X)$ such that $G = \upset f^{\imf\imf}[H]$;
\item if $f^{\imf\imf}[F]\subseteq G$ , then there exists a $H\in\powerfilters(X)$ such that $G = \upset f^{\imf\imf}[H]$;
\item if $G$ is an ultrafilter, then the $H$ in the previous points may be taken to be an ultrafilter.
\end{enumerate}
\end{proposition}
\begin{proof}
TODO \ref{baseTraceFilter}
\end{proof}

\section{Continuous functions}
\begin{definition}
Let $\sSet{X,\xi}$ and $\sSet{Y,\zeta}$ be (pre)convergence spaces. A function $f: X\to Y$ is called \udef{continuous} if it preserves limits: for all $D \in \powerdirected(X)$ and $x\in X$:
\[ D \overset{\xi}{\longrightarrow} x \implies f^{\imf\imf}(D) \overset{\zeta}{\longrightarrow} f(x). \]

The set of all continuous functions $\sSet{X,\xi} \to \sSet{Y,\zeta}$ is denoted $\cont(\xi, \zeta)$ or $\cont(X,Y)$, if the convergence is clear. If $X=Y$, we also write $\cont(X)$.

If $\xi,\zeta$ are preconvergences we write $\cont_\text{pre}(\xi, \zeta)$ for the set of continuous functions.
\end{definition}
In other words, a function is continuous if it is relation-preserving as a function
\[ \sSet{X \cup \powerset^2(X), \overset{\xi}{\longrightarrow}} \to \sSet{Y \cup \powerset^2(Y), \overset{\zeta}{\longrightarrow}}. \]

Note that for filters $F\in\powerfilters(X)$, $f[F]$ is in general a directed set, but not necessarily a filter.

\begin{lemma}
Let $\sSet{X,\xi}$ and $\sSet{Y,\zeta}$ be (pre)convergence spaces. A function $f: X\to Y$ is continuous \textup{if and only if} for all $D \in \powerdirected(X)$
\[ f^{\imf}\left[\lim_\xi D \right] \subseteq \lim_\zeta f^{\imf\imf}[D]. \]
\end{lemma}
\begin{proof}
Immediate from \ref{relationPreserving}.
\end{proof}

\begin{lemma} \label{continuityComposition}
Let $\sSet{X,\xi}$, $\sSet{Y,\sigma}$ and $\sSet{Z,\zeta}$ be (pre)convergence spaces. If $f: X\to Y$ and $g: Y\to Z$ are continuous, then $g\circ f$ is continuous.
\end{lemma}
\begin{proof}
Let $F\to x\in X$. Then $f[F] \to f(x)$ by continuity of $f$ and $g[f[F]] \to g(f(x))$ by continuity of $g$. So $g\circ f$ is continuous.
\end{proof}

\begin{lemma} \label{finerCoarserContinuity}
Let $\sSet{X,\xi}, \sSet{Y,\zeta}$ be (pre)convergence spaces and $f\in \cont_\text{(pre)}(\xi, \zeta)$.
\begin{enumerate}
\item Let $\sigma$ be a (pre)convergence on $X$ such that $\sigma \leq \xi$, then $f\in \cont_\text{(pre)}(\sigma, \zeta)$.
\item Let $\tau$ be a (pre)convergence on $Y$ such that $\tau \geq \zeta$, then $f\in \cont_\text{(pre)}(\xi, \tau)$.
\end{enumerate}
\end{lemma}
\begin{proof}
(1) Let $F\overset{\sigma}{\longrightarrow} x \in X$, then $F\overset{\xi}{\longrightarrow} x$ (because $\sigma \leq \xi$), so $f[F]\overset{\zeta}{\longrightarrow} f(x)$, meaning $f\in \cont_\text{(pre)}(\sigma, \zeta)$.

(2) Let $F\overset{\xi}{\longrightarrow} x \in X$, then $f[F]\overset{\zeta}{\longrightarrow} f(x)$, so $f[F]\overset{\tau}{\longrightarrow} f(x)$  (because $\zeta \leq \tau$), meaning $f\in \cont_\text{(pre)}(\xi, \tau)$.
\end{proof}

\begin{proposition} \label{continuityVicinityFilter} \label{adherenceInherenceContinuity}
Let $\sSet{X,\xi}$ and $\sSet{Y,\zeta}$ be convergence spaces, $f: X\to Y$ a function, $A\subseteq X$ a subset and $x\in X$. Then the following are equivalent:
\begin{enumerate}
\item $\vicinity_\zeta(f(x)) \subseteq \upset f^{\imf\imf}[\vicinity_\xi(x)]$;
\item for all $U\in \vicinity_\zeta(f(x))$, there exists $V\in \vicinity_\xi(x)$ such that $f^\imf[V] \subseteq U$;
\item for all $U\in \vicinity_\zeta(f(x))$, $f^{\preimf}[U]\in \vicinity_\xi(x)$;
\item $f^{\preimf\imf}\vicinity_\zeta(f(x)) \subseteq \vicinity_\xi(x)$;
\item $f^\preimf[\adh_\zeta(A)] \supseteq \adh_\xi(f^\preimf[A])$;
\item $f^\imf[\adh_\xi(A)] \subseteq \adh_\zeta(f^\imf[A])$;
\item $f^\preimf[\inh_\zeta(A)] \subseteq \inh_\xi(f^\preimf[A])$.
\end{enumerate}
All these points hold if $f$ is continuous.
\end{proposition}
\begin{proof}
We first show that continuity of $f$ implies (1): We have
\begin{align*}
\vicinity_\zeta(f(x)) &= \bigcap\setbuilder{G\in\powerfilters(Y)}{G\to f(x)} \\
&\subseteq \bigcap\setbuilder{\upset f^{\imf\imf}(F)\in\powerfilters(Y)}{F\to x} \\
&= \bigcap (\upset\circ f^{\imf\imf})^{\imf}\Big[\setbuilder{F\in\powerfilters(X)}{F\to x}\Big] \\
&= \upset f^{\imf\imf}\Big[\bigcap \setbuilder{F\in\powerfilters(X)}{F\to x}\Big] = \upset f^{\imf\imf}[\vicinity_\xi(x)],
\end{align*}
where we have used \ref{imageFiltersPreservesIntersection}.

Now take $B\in \bigcap\setbuilder{\upset f^{\imf\imf}(F)\in\powerfilters(Y)}{F\to x}$. This means that for all $F\in\lim^{-1}_\xi(x)$, there exists an $A_F\in F$ such that $f^\imf[A_F]\subseteq B$. By the upward closure of the filters $F$, the set $A = \bigcup_{F\in \lim^{-1}_\xi(x)}A_F$ is an element of each $F$ and
\[ f^\imf[A] = \bigcup_{F\in \lim^{-1}_\xi(x)}f^{\imf}[A_F] \subseteq B.  \]
Now $A\in \vicinity_\xi(x)$ and thus $B\in \upset f^{\imf\imf}[\vicinity_\xi(x)]$.


$(1) \Leftrightarrow (2)$ Reformulation.

$(2) \Leftrightarrow (3)$ We get (3) from (2) by noting that $V\subseteq f^{\preimf}f^\imf(V) \subseteq f^\preimf(U)$ and that $\vicinity_\xi(x)$ is a filter, so this implies $f^\preimf(U)\in \vicinity_\xi(x)$.

Conversely, we may take $V = f^{\preimf}[U]$.

$(3) \Leftrightarrow (4)$ Reformulation.

$(4) \Rightarrow (5)$ We calculate
\begin{align*}
x\in \adh_\zeta(f^\preimf[A]) \iff& f^\preimf[A]\in \vicinity_\xi(x)^{\mesh} \subseteq \left(f^{\preimf\imf}\vicinity_\zeta(f(x))\right)^{\mesh} \\
\implies& f^\imf f^\preimf[A] \in f^{\imf\imf}\left(\left(f^{\preimf\imf}\vicinity_\zeta(f(x))\right)^{\mesh}\right) \subseteq \left(f^{\imf\imf} f^{\preimf\imf}\vicinity_\zeta(f(x))\right)^{\mesh} \subseteq \left(\vicinity_\zeta(f(x))\right)^{\mesh} \\
\implies& A \in \left(\vicinity_\zeta(f(x))\right)^{\mesh} \\
\iff& f(x)\in \adh_\zeta(A) \\
\iff& x\in f^{\preimf}(\adh_\zeta(A)).
\end{align*}
We have used that (TODO ref)
\begin{itemize}
\item $F\subseteq G$ implies $G^{\mesh} \subseteq F^{\mesh}$;
\item $f^{\imf\imf}(F^{\mesh}) \subseteq \left(f^{\imf\imf}(F)\right)^{\mesh}$;
\item $F^{\mesh}$ is a filter if $F$ is a directed set;
\item $f^\imf f^\preimf[A] \subseteq A$.
\end{itemize}

$(1) \Rightarrow (6)$ We calculate
\begin{align*}
y\in f^\imf(\adh(A)) \iff& \exists x\in \adh(A):\;f(x) = y\\
\iff& \exists x\in X: \; x\in \adh(A) \land f(x) = y\\
\iff& \exists x\in X: \; A\in\vicinity(x)^{\mesh}  \land f(x) = y \\
\implies& \exists x\in X: \; f(x) = y \land f^\imf(A)\in f^{\imf\imf}\left(\vicinity(x)^{\mesh}\right) \subseteq f^{\imf\imf}\left(\vicinity(x)\right)^{\mesh} \subseteq \vicinity(f(x))^{\mesh} \\
\iff& \exists x\in X: \; f(x) = y \land f(x)\in \adh\Big(f^\imf(A)\Big) \\
\implies& y\in \adh\Big(f^\imf(A)\Big).
\end{align*}

$(5) \Rightarrow (7)$ We calculate, using \ref{inherenceComplementAdherence},
\begin{align*}
f^\preimf[\inh_\xi(A)] &= f^\preimf[\adh_\xi(A^c)^c] \\
&= \im(f)\setminus f^{\preimf}[\adh_\xi(A^c)] \\
&\subseteq f^{\preimf}[\adh_\xi(A^c)]^c \\
&\subseteq \left(\adh_\zeta(f^{\preimf}[A^c])\right)^c \\
&\subseteq \left(\adh_\zeta(f^{\preimf}[A]^c)\right)^c \\
&= \inh_\zeta(f^{\preimf}[A]).
\end{align*}

$(6) \Rightarrow (7)$ We calculate
\begin{align*}
x\in f^\preimf[\inh_\xi(A)] \iff& f(x) \in \inh_\zeta(A) = \adh_\zeta(A^c)^c \subseteq \adh_\zeta\Big(f^\imf\circ f^{\preimf}(A^c)\Big)^c \\
\iff& f(x) \in \adh_\zeta\Big(f^\imf\big(f^{\preimf}(A)^c\big)\Big)^c \subseteq f^\imf\Big(\adh_\xi\Big(f^{\preimf}(A)^c\big)\Big)^c \\
\implies& x\in \adh_\xi\Big(f^{\preimf}(A)^c\big)^c = \inh_\xi\big(f^{\preimf}(A)\big).
\end{align*}

$(7) \Rightarrow (1)$ We calculate
\begin{align*}
B\in \vicinity_\zeta(f(x)) \iff& f(x) \in \inh_\zeta(B) \\
\implies& x\in f^{\preimf}\Big(\inh_\zeta(B)\Big) \subseteq \inh_\xi\Big(f^\preimf(B)\Big) \\
\implies& f^\preimf(B) \in \vicinity_\xi(x) \\
\implies& f^\imf\big(f^\preimf(B)\big) \in f^{\imf\imf}[\vicinity_\xi(x)] \\
\implies& B \in \upset f^{\imf\imf}[\vicinity_\xi(x)].
\end{align*}
\end{proof}
\begin{proof}[Alternative proof of (2), given continuity of $f$]
We calculate
\begin{align*}
f[\adh_\xi(A)] &= f\left[\bigcup \setbuilder{\lim_\xi F}{A \lhd F} \right] = \bigcup f\left[ \setbuilder{\lim_\xi F}{A \lhd F} \right] = \bigcup \setbuilder{f\left[\lim_\xi F\right]}{A \lhd F} \\
&\subseteq \bigcup \setbuilder{\lim_\zeta f[F]}{A \lhd F} \subseteq \bigcup \setbuilder{\lim_\zeta f[F]}{f[A] \lhd f[F]} \subseteq \bigcup \setbuilder{\lim_\zeta G}{f[A] \lhd G} \\
&= \adh_\zeta(f[A]),
\end{align*}
where we have used that $A \lhd B$ implies $f[A] \lhd f[B]$ (TODO ref).
\end{proof}
\begin{corollary} \label{pretopologicalContinuityVicinities}
If $\zeta$ is pretopological, then any of these points implies that $f$ is continuous.
\end{corollary}
\begin{proof}
We prove that in this case (1) implies the continuity of $f$. Take $F\to x$. Then $F\supseteq \vicinity_\xi(x)$ and thus $f[F]\supseteq f[\vicinity_\xi(x)]$. By assumption this means $\upset f[F]\supseteq \upset f[\vicinity_\xi(x)]\supseteq \vicinity_\zeta(f(x))$. So $f[F]\to f(x)$.
\end{proof}

If $\xi$ is pretopological, we can also give a simplified proof that the continuity of $f$ implies (1): We have that $\vicinity_\xi(x)$ converges to $x$, so by continuity $f[\vicinity_\xi(x)]$ converges to $f(x)$ and thus $\upset f[\vicinity_\xi(x)] \supseteq \vicinity_\zeta(f(x))$ by the pretopological property.
\begin{corollary} \label{preimageOpenClosed}
Let $\sSet{X,\xi}$ and $\sSet{Y,\zeta}$ be convergence spaces and $f: X\to Y$ a continuous function. Then
\begin{enumerate}
\item if $O\subseteq Y$ is open, then $f^\preimf(O)$ is open;
\item if $C\subseteq Y$ is closed, then $f^\preimf(C)$ is closed.
\end{enumerate}
\end{corollary}
\begin{proof}
(1) We use \ref{openClosedCriteria}, so take $x\in f^\preimf(O)$. Then $f(x)\in O$ and by \ref{openClosedCriteria} there exists a $U_{f(x)}\in \vicinity_\zeta\big(f(x)\big)$ such that $U_{f(x)} \subseteq O$. Then $f^\preimf(U_{f(x)}) \subseteq f^\preimf(O)$ and $f^\preimf(U_{f(x)})\in \vicinity_\xi(x)$.

(2) We use the fact that $C^c$ is open and \ref{imagePreimageUniqueness} to calculate
\[ f^\preimf(C) = f^\preimf(Y\setminus C^c) = f^\preimf(Y)\setminus f^\preimf(C^c) = X\setminus f^\preimf(C^c) = f^\preimf(C^c)^c. \]
This is closed because $f^\preimf(C^c)$ is open by (1).
\end{proof}

\begin{lemma} \label{identityContinuity}
Let $\xi$ and $\zeta$ be two convergences on the same set $X$. Then $\id_X: \sSet{X,\xi} \to \sSet{X,\zeta}$ is continuous \textup{if and only if} $\xi \leq \zeta$. I.e.\ $\xi$ is finer than $\zeta$.
\end{lemma}
\begin{proof}
This is essentially a restatement of definitions:
\begin{align*}
\text{$\id_X: \sSet{X,\xi} \to \sSet{X,\zeta}$ is continuous} &\iff \forall F\in\powerfilters(X): \id_X\left[\lim_\xi F \right] \subseteq \lim_\zeta \id_X[F] \\
&\iff \forall F\in\powerfilters(X): \lim_\xi F \subseteq \lim_\zeta F \\
&\iff \xi \leq \zeta.
\end{align*}
\end{proof}

\begin{lemma} \label{continuityConstructions}
Let $\sSet{X,\xi}$ and $\sSet{Y,\zeta}$ be a convergence space.
\begin{enumerate}
\item \textup{(Identity function)} The identity function $\id_X:X\to X$ is continuous.
\item \textup{(Constant function)} For all $y$ in $Y$, the constant function $\underline{y}: X \to Y$ is continuous.
\end{enumerate}
\end{lemma}
\begin{proof}
(1) Let $F\to x \in X$. Then $x\in \lim_\xi(\id_X[F]) = \lim_\xi(F)$.

(2) Let $F\to x \in X$. Then $\underline{y}[F] = \pfilter{y} \to y = \underline{y}(x)$.
\end{proof}

\begin{lemma} \label{inverseImageContinuity}
Let $\sSet{X,\xi}$ and $\sSet{Y,\zeta}$ be preconvergence spaces and $f: X\to Y$ a function. Then $f$ is continuous \textup{if and only if} $\forall F\in \powerfilters(Y)$ and $y\in Y$
\[ \exists x\in f^{\preimf}[y]: \; f^{\preimf\imf}[F] \overset{\xi}{\longrightarrow} x \quad\implies\quad F \overset{\zeta}{\longrightarrow} y. \]
\end{lemma}
\begin{proof}
For $\Rightarrow$, assume $f$ continuous and that there exists $x\in f^{\preimf}[y]$ such that $f^{\preimf\imf}[F] \to x$. By continuity we have $f^{\imf\imf}[f^{\preimf\imf}[F]] \to y$. Now $f^{\imf\imf}[f^{\preimf\imf}[F]] \leq F$ by \ref{relationTimesTransposeSubsetIdentity}, so $F \overset{\zeta}{\longrightarrow} y$.

For $\Leftarrow$, take arbitrary $G \overset{\xi}{\longrightarrow} x \in X$. Then $f^{\preimf\imf}[f^{\imf\imf}[G]] \geq G$ by \ref{totalityEquivalences}, so $f^{\preimf\imf}[f^{\imf\imf}[G]] \overset{\xi}{\longrightarrow} x$ and $x \in f^{\preimf}[\{f(x)\}]$. This means $f^{\imf\imf}[G] \overset{\zeta}{\longrightarrow} f(x)$ and thus $f$ is continuous.
\end{proof}

\begin{proposition} \label{continuityUnderConvergenceLatticeOperations}
Let $\sSet{X,\xi}$ and $\sSet{Y,\zeta}$ be convergence spaces. Let $\Xi$ be a set of convergences on $X$ and $Z$ a set of convergences on $Y$. Then
\begin{enumerate}
\item $\cont\left(\xi, \bigwedge Z\right) = \bigcap_{\sigma\in Z} \cont(\xi, \sigma)$;
\item $\cont\left(\bigvee \Xi, \zeta\right) = \bigcap_{\sigma\in \Xi} \cont(\sigma, \zeta)$;
\item $\cont\left(\bigwedge \Xi, \zeta\right) \supseteq \bigcup_{\sigma\in Z} \cont(\xi, \sigma)$;
\item $\cont\left(\xi, \bigvee Z\right) \supseteq \bigcup_{\sigma\in \Xi} \cont(\sigma, \zeta)$;
\end{enumerate}
\end{proposition}
\begin{proof}
(1) We calculate, using \ref{latticeConvergences}:
\begin{align*}
f \in \cont\left(\xi, \bigwedge Z\right) &\iff f\left[\lim_\xi F \right] \subseteq \lim_{\bigwedge Z} \upset f[F] = \bigcap_{\sigma\in Z}\lim_\sigma \upset f[F] \\
&\iff \forall \sigma\in Z: \; f\left[\lim_\xi F \right] \subseteq \lim_\sigma \upset f[F] \\
&\iff \forall \sigma\in Z: \; f \in \cont(\xi,\sigma) \\
&\iff f\in \bigcap_{\sigma\in Z}\cont(\xi,\sigma).
\end{align*}

(2) Similarly, we have
\begin{align*}
f \in \cont\left(\bigvee \Xi, \zeta\right) &\iff f\left[\lim_{\bigvee \Xi} F \right] = \bigcup_{\sigma\in\Xi}f\left[\lim_\sigma F \right] \subseteq \lim_{\zeta} \upset f[F] \\
&\iff \forall \sigma\in \Xi: \; f\left[\lim_\sigma F \right] \subseteq \lim_\zeta \upset f[F] \\
&\iff \forall \sigma\in \Xi: \; f \in \cont(\sigma, \zeta) \\
&\iff f\in \bigcap_{\sigma\in \Xi}\cont(\sigma, \zeta).
\end{align*}

(3) Now we have
\begin{align*}
f \in \cont\left(\bigwedge \Xi, \zeta\right) &\iff f\left[\lim_{\bigwedge \Xi} F \right] = \bigcap_{\sigma\in\Xi}f\left[\lim_\sigma F \right] \subseteq \lim_{\zeta} \upset f[F] \\
&\impliedby \exists \sigma\in \Xi: \; f\left[\lim_\sigma F \right] \subseteq \lim_\zeta \upset f[F] \\
&\iff \exists \sigma\in \Xi: \; f \in \cont(\sigma, \zeta) \\
&\iff f\in \bigcup_{\sigma\in \Xi}\cont(\sigma, \zeta).
\end{align*}

(4) Finally we have
\begin{align*}
f \in \cont\left(\xi, \bigvee Z\right) &\iff f\left[\lim_\xi F \right] \subseteq \lim_{\bigvee Z} \upset f[F] = \bigcup_{\sigma\in Z}\lim_\sigma \upset f[F] \\
&\impliedby \exists \sigma\in Z: \; f\left[\lim_\xi F \right] \subseteq \lim_\sigma \upset f[F] \\
&\iff \exists \sigma\in Z: \; f \in \cont(\xi,\sigma) \\
&\iff f\in \bigcup_{\sigma\in Z}\cont(\xi,\sigma).
\end{align*}
\end{proof}

\subsubsection{Homeomorphisms}
\begin{definition}
Let $\sSet{X,\xi}$ and $\sSet{Y,\zeta}$ be convergence spaces. A function $f: X\to Y$ is called a \udef{homeomorphism} if
\begin{itemize}
\item $f$ is bijective;
\item both $f$ and $f^{-1}$ are continuous.
\end{itemize}
\end{definition}

\begin{proposition} \label{homeomorphismPreservation}
If $f$ homeomorphism, then
\begin{enumerate}
\item $f[\lim F] = \lim f[F]$;
\item $f[\vicinity(x)] = \vicinity(f[x])$;
\item $f[\adh(A)] = \adh(f[A])$.
\end{enumerate}
\end{proposition}


\subsection{Directional continuity}
\begin{definition}
Let $f: \sSet{X,\xi} \to \sSet{Z,\zeta}$ be a function between convergence spaces, $x\in X$ and $D\subseteq X$. We called $f$ \udef{directionally continuous} at $x$ in the direction $D$ if for all $F\in\powerfilters(X)$
\[ F \overset{\xi,D}{\longrightarrow} x \qquad \implies \qquad f[F] \overset{\zeta}{\longrightarrow} f(x). \]
\end{definition}

\begin{lemma}
Let $f: \sSet{X,\xi} \to \sSet{Z,\zeta}$ be a function between convergence spaces, $x_0\in X$ and $D$ a vicinity of $x_0$. Then $f$ is directionally continuous at $x_0$ in $D$ \textup{if and only if} $f$ is continuous at $x_0$.
\end{lemma}
\begin{proof}
TODO inherence
\end{proof}

\section{Initial and final convergences}
\begin{definition}
Let $Y$ be a set.
\begin{itemize}
\item Given a set of (pre)convergence spaces $\{\sSet{Z_i,\zeta_i}\}_{i\in I}$ and a set of functions $\{f_i: Y\to Z_i\}_{i\in I}$, we define the \udef{initial (pre)convergence} $\mu$ on $Y$ w.r.t. $\{f_i: Y\to Z_i\}$ as the coarsest (pre)convergence on $Y$ that makes all functions in $\{f_i: Y\to Z_i\}$ continuous:
\[ \mu = \bigvee \setbuilder{\sigma}{\forall i\in I: f_i\in  \cont_\text{(pre)}(\sigma, \zeta_i)}. \]
\item Given a set of convergence spaces $\{\sSet{X_i,\xi_i}\}_{i\in I}$ and a set of functions $\{g_i: X_i \to Y\}_{i\in I}$, we define the \udef{final convergence} $\nu$ on $Y$ w.r.t. $\{g_i: X_i\to Y\}$ as the finest convergence on $Y$ that makes all functions in $\{g_i: X_i \to Y\}$ continuous:
\[ \nu = \bigwedge \setbuilder{\sigma}{\forall i\in I: g_i\in  \cont_\text{(pre)}(\xi_i, \sigma)}. \]
\end{itemize}
\end{definition}

\begin{proposition} \label{initialFinalConvergenceModification}
Let $Y$ be a set.
\begin{enumerate}
\item Let $\{f_i: Y\to \sSet{Z_i, \zeta_i}\}_{i\in I}$ be set of functions to convergence spaces and $\mu$ the initial \emph{pre}convergence on $Y$ w.r.t. this set. Then $\mu$ is also the initial convergence w.r.t. $\{f_i: Y\to \sSet{Z_i, \zeta_i}\}$.
\item Let $\{g_i: \sSet{X_i, \xi_i} \to Y\}_{i\in I}$ be set of functions from convergence spaces and $\nu$ the final \emph{pre}convergence on $Y$ w.r.t. this set. Then $\nu \vee \iota_Y$ is the final convergence w.r.t. $\{g_i: \sSet{X_i, \xi_i} \to Y\}$.
\end{enumerate}
\end{proposition}
\begin{proof}
TODO convergence modification
\end{proof}

\begin{proposition} \label{initialFinalConvergence}
Let $Y$ be a set, $F\in \powerfilters(Y)$ and $y\in Y$.
\begin{enumerate}
\item Let $\{f_i: Y\to \sSet{Z_i, \zeta_i}\}_{i\in I}$ be set of functions to \emph{pre}convergence spaces and $\mu$ the initial \emph{pre}convergence on $Y$ w.r.t. this set. Then
\[ F \overset{\mu}{\longrightarrow} y \quad\iff\quad \forall i\in I: \; f_i[F] \overset{\zeta_i}{\longrightarrow} f_i(y). \]
\item Let $\{g_i: \sSet{X_i, \xi_i} \to Y\}_{i\in I}$ be set of functions from \emph{pre}convergence spaces and $\nu$ the final \emph{pre}convergence on $Y$ w.r.t. this set. Then
\[ F \overset{\nu}{\longrightarrow} y \quad\iff\quad \exists i\in I: \exists x\in g_i^{\preimf}\{y\}: \; g_i^{\preimf\imf}[F] \overset{\xi_i}{\longrightarrow} x. \]
\end{enumerate}
\end{proposition}
Note that point (1) still holds true for convergences, by \ref{initialFinalConvergenceModification}. In the convergence case, point (2) needs to be modified to 
\[ F \overset{\nu}{\longrightarrow} y \quad\iff\quad \exists i\in I: \exists x\in g_i^{\preimf}\{y\}: \; g_i^{\preimf\imf}[F] \overset{\xi_i}{\longrightarrow} x \;\;\lor\;\; F = \pfilter{y}. \]
\begin{proof}
(1) The direction $\Rightarrow$ is clear: $\mu$ makes all $f_i$ continuous by \ref{finerCoarserContinuity}.

For $\Leftarrow$, assume $F$ such that $\forall i\in I: \; f_i[F] \overset{\zeta_i}{\longrightarrow} f_i(y)$, but $F \overset{\mu}{\not\to} y$. Then define the preconvergence $\mu'$ with the same limits as $\mu$, but with the addition of $G\overset{\mu'}{\longrightarrow} y$ for all $G\geq F$. Now $\mu'$ makes all $f_i$ continuous (because $G\geq F$ implies $f_i[G] \geq f_i[F]$), so $\mu'\leq \mu$. Thus $F \overset{\mu}{\longrightarrow} y$, which is a contradiction.

(2) By \ref{finerCoarserContinuity}, $g_i$ is continuous for all $i\in I$. Then the direction $\Leftarrow$ follows from \ref{inverseImageContinuity}.

For $\Rightarrow$, assume $F \overset{\nu}{\longrightarrow} y$ and $\forall i\in I: \forall x\in g_i^{-1}[y]: \; g_i^{-1}[F] \not\to x$. Define the $\nu'$ from $\nu$ by removing all limits of the form $G \to y$ for $G \leq F$. Now we have $g_i^{-1}[G] \leq g_i^{-1}[F]$, so $\forall x\in g_i^{-1}[y]: \; g_i^{-1}[G] \not\to x$. By \ref{inverseImageContinuity}, $\nu'$ still makes all $g_i$ continuous, meaning $\nu'\geq \nu$. Thus $F \overset{\nu'}{\longrightarrow} y$, which is a contradiction.
\end{proof}
\begin{corollary}[Characteristic property of initial and final convergence] \label{characteristicPropertyInitialFinalConvergence}
Let $Y$ be a set, $\sSet{X,\xi}$ and $\sSet{Z, \zeta}$ (pre)convergence spaces.
\begin{enumerate}
\item Let $\{f_i: Y\to \sSet{Z_i, \zeta_i}\}_{i\in I}$ be set of functions to (pre)convergence spaces and $\mu$ the initial (pre)convergence on $Y$ w.r.t. this set. A function $g: \sSet{X, \xi}\to Y$ is continuous \textup{if and only if} $f_i \circ g$ is continuous for all $i\in I$.
\[ \begin{tikzcd}
Y \ar[r, "f_i"] & Z_i \\ X \ar[u, "g"] \ar[ur, swap, "f_i\circ g"]
\end{tikzcd} \]
\item Let $\{g_i: \sSet{X_i, \xi_i} \to Y\}_{i\in I}$ be set of functions from (pre)convergence spaces and $\nu$ the final (pre)convergence on $Y$ w.r.t. this set. A function $f: Y\to \sSet{Z,\zeta}$ is continuous \textup{if and only if} $f\circ g_i$ is continuous for all $i\in I$.
\[ \begin{tikzcd}
X_i \ar[r, "g_i"] \ar[dr, swap, "f\circ g_i"] & Y \ar[d, "f"] \\ & Z
\end{tikzcd} \]
\end{enumerate}
\end{corollary}
\begin{proof}
(1) Take arbitrary $F\overset{\xi}{\longrightarrow} x\in X$. Then the continuity of $g$ is equivalent to the convergence $g[F] \overset{\mu}{\longrightarrow} g(x)$. By the proposition this is equivalent to
\[ \forall i \in I: \; f_i[g[F]] \overset{\zeta_i}{\longrightarrow} f_i(g(x)),  \]
which is equivalent to the continuity of $f_i\circ g$ for all $i\in I$.

(2) Take arbitrary $F \in \powerfilters(Z)$ and $z\in Z$ such that $F \overset{\zeta}{\not\to} z$. Then, by \ref{inverseImageContinuity}, the continuity of $f$ is equivalent to
\[ \forall y\in f^{-1}[z]: \; f^{-1}[F] \overset{\nu}{\not\to} y \]
By the proposition this is equivalent to
\[ \forall y\in f^{-1}[z]: \forall i\in I: \forall x\in g_{i}^{-1}[y]: \; g_i^{-1}[f^{-1}[F]] \overset{\xi_i}{\not\to} x. \]
Using the equality $g^{-1}_i[f^{-1}[F]] = (f\circ g_i)^{-1}[F]$, we can rewrite this as
\[ \forall i\in I: \forall x\in (f\circ g_{i})^{-1}[z]: \; (f\circ g_i)^{-1}[F] \overset{\xi_i}{\not\to} x. \]
By \ref{inverseImageContinuity}, this is equivalent to the  continuity of all $f\circ g_i$.
\end{proof}

TODO: final convergence does not preserve finite depth! (what about Kent space??)

TODO: initial/final not universal property, but product is.

TODO $\prod, \coprod$

\subsection{Initial pretopological convergence}
TODO

\begin{proposition} \label{pretopologicalInitialFinalConvergence}
Let $X$ be a set and $\xi$ the initial convergence on $X$ w.r.t. a set of functions $\{f_i: X\to \sSet{Y_i, \zeta_i}\}_{i\in I}$ from $X$ to pretopological spaces $Y_i$. Then $\xi$ is pretopological and
\[ \vicinity_\xi(x) = \mathfrak{F}\setbuilder{f_i^\preimf(U)}{i\in I, U\in \vicinity_{\zeta_i}(f(x))}. \]
\end{proposition}
\begin{proof}
We have, by \ref{initialFinalConvergence}
\begin{align*}
F \overset{\xi}{\longrightarrow} x &\iff \forall i\in I: f_i^{\imf\imf}[F] \overset{\zeta_i}{\longrightarrow} f_i(x) \\
&\iff \forall i\in I: \vicinity_{\zeta_i}\!\big(f_i(x)\big) \subseteq \upset f_i^{\imf\imf}[F] \\
&\iff \forall i\in I: \forall U\in \vicinity_{\zeta_i}\!\big(f_i(x)\big): \exists M\in f_i^{\imf\imf}[F]: \; M \subseteq U \\
&\iff \forall i\in I: \forall U\in \vicinity_{\zeta_i}\!\big(f_i(x)\big): \exists N\in F: \; f_i^{\imf}[N] \subseteq U \\
&\iff \forall i\in I: \forall U\in \vicinity_{\zeta_i}\!\big(f_i(x)\big): \exists N\in F: \; N \subseteq f_i^{\preimf}[U] \\
&\iff \forall i\in I: \forall U\in \vicinity_{\zeta_i}\!\big(f_i(x)\big): f_i^{\preimf}[U] \in F \\
&\iff \forall i\in I: f^{\preimf\imf}\Big[\vicinity_{\zeta_i}\!\big(f_i(x)\big)\Big] \subseteq F \\
&\iff \bigcup_{i\in I} f^{\preimf\imf}\Big[\vicinity_{\zeta_i}\!\big(f_i(x)\big)\Big] \subseteq F
\end{align*}
Thus the initial convergence is pretopological and
\begin{align*}
\vicinity_\xi(x) &= \mathfrak{F}\bigcup_{i\in I} f_i^{\preimf\imf}\Big[\vicinity_{\zeta_i}\!\big(f_i(x)\big)\Big] \\
&= \mathfrak{F}\bigcup_{i\in I}\setbuilder{f_i^\preimf(U)}{U\in \vicinity_{\zeta_i}(f(x))} \\
&= \mathfrak{F}\setbuilder{f_i^\preimf(U)}{i\in I, U\in \vicinity_{\zeta_i}(f(x))}.
\end{align*}
\end{proof}

\begin{proposition} \label{topologicalInitialFinalConvergence}
Let $X$ be a set and $\xi$ the initial convergence on $X$ w.r.t. a set of functions $\{f_i: X\to \sSet{Y_i, \zeta_i}\}_{i\in I}$ from $X$ to topological spaces $Y_i$. Then $\xi$ is topological and $\topology_\xi$ is generated by $\setbuilder{f_i^\preimf(U)}{i\in I, U\in \topology_{\zeta_i}}$.
\end{proposition}
\begin{proof}
By \ref{pretopologicalInitialFinalConvergence}, $\xi$ is pretopological. Each $U\in \vicinity_{\zeta_i}(f(x)) = \neighbourhood_{\zeta_i}(f(x))$ contains an open set, so $f_i^\preimf(U)$ also contains an open set by \ref{preimageOpenClosed}. Thus $\vicinity_\xi(x)$ is based in open sets. This means that the initial convergence is topological.

Finally $\setbuilder{f_i^\preimf(U)}{i\in I, U\in \topology_{\zeta_i}}$ is a set of open sets. TODO: a set of open sets that generates the convergence also generates the topology. 
\end{proof}

\subsection{Constructions}
\subsubsection{Product convergence}
\begin{definition}
Let $\sSet{X_i, \xi_i}$ be a convergence space for all $i\in I$. The \udef{product convergence space} $\prod_{i\in I}X_i$ is the initial convergence on $\bigtimes_{i\in I}X_i$ w.r.t. the set of projections $p_i: \bigtimes_{i\in I}X_i \to X_i$.
\end{definition}

\begin{proposition} \label{convergenceProductFilter}
Let $\sSet{X_i, \xi_i}$ be a (pre)convergence space for all $i\in I$, $F \in \powerfilters(\prod_{i\in I}X_i)$ and $x\in \prod_{i\in I}X_i$. Then $F\to x$ \textup{if and only if} $\forall i\in I: \exists F_i\in \powerfilters(X_i): 
\; F_i \overset{\xi_i}{\longrightarrow} p_i(x)$ and
\[ F \geq \bigotimes_{i\in I}F_i \defeq \setbuilder{\bigtimes_{i\in I}A_i}{\forall i\in I: A_i \in F_i \;\land\; A_i = X_i, \,\text{except for finitely many $A_i$}}. \]
\end{proposition}
\begin{proof}
The direction $\Leftarrow$ follows straight form \ref{initialFinalConvergence} because $p_i(F) = F_i \to p_i(x)$.

TODO adjoint of $\coprod_{i\in I}p_i$.
\end{proof}
\begin{corollary} \label{productVicinity}
Let $\sSet{X_i, \xi_i}$ be a (pre)convergence space for all $i\in I$ and $x\in \prod_{i\in I}X_i$. Then
\[\vicinity_{\prod \xi_i}(x) = \upset \setbuilder{\bigtimes_{i\in I}A_i}{\forall i\in I: A_i \in \vicinity_{\xi_i}(p_i(x)) \;\land\; A_i = X_i, \,\text{except for finitely many $A_i$}}. \]
In particular $\vicinity_{\xi\otimes \zeta}((x,y)) = \upset \vicinity_\xi(x)\otimes \vicinity_\zeta(y)$.
\end{corollary}

\begin{corollary} \label{productAdherence}
Let $\sSet{X_i, \xi_i}$ be a (pre)convergence space and $A_i\subseteq X_i$ for all $i\in I$. Then
\[ \adh_{\prod_{i\in I}\xi_i}\left(\bigtimes_{i\in I}A_i\right) = \bigtimes_{i\in I}\adh_{\xi_i}(A_i). \]
In particular $\adh_{\xi\otimes \zeta}(A\times B) = \adh_\xi(A) \times \adh_\zeta(B)$.
\end{corollary}
\begin{proof}
We have
\begin{align*}
\seq{x_i}_{i\in I} \in \adh_{\prod_{i\in I}\xi_i}\left(\bigtimes_{i\in I}A_i\right) &\iff \bigtimes_{i\in I}A_i \in \vicinity_{\prod \xi_i}(\seq{x_i}_{i\in I})^{\mesh} \\
&\iff \forall i\in I: \forall B \in \vicinity_{\xi_i}(x_i): A_i\mesh B \\
&\iff \forall i\in I: A_i \in \vicinity_{\xi_i}(x_i)^{\mesh} \\
&\iff \forall i\in I: x_i \in \adh_{\xi_i}(A_i) \\
&\iff \seq{x_i}_{i\in I} \in \bigtimes_{i\in I}\adh_{\xi_i}(A_i).
\end{align*}
\end{proof}
\begin{corollary} \label{productInherence}
Let $\sSet{X_i, \xi_i}$ be a (pre)convergence space and $A_i\subseteq X_i$ for all $i\in I$. Then
\[ \inh_{\prod_{i\in I}\xi_i}\left(\bigtimes_{i\in I}A_i\right) = \bigtimes_{i\in I}\inh_{\xi_i}(A_i). \]
In particular $\inh_{\xi\otimes \zeta}(A\times B) = \inh_\xi(A) \times \inh_\zeta(B)$.
\end{corollary}
\begin{proof}
We have
\begin{align*}
\seq{x_i}_{i\in I} \in \inh_{\prod_{i\in I}\xi_i}\left(\bigtimes_{i\in I}A_i\right) &\iff \bigtimes_{i\in I}A_i \in \vicinity_{\prod \xi_i}(\seq{x_i}_{i\in I}) \\
&\iff \forall i\in I: \exists B_i \in \vicinity_{\xi_i}(x_i): B_i\subseteq A_i \\
&\iff \forall i\in I: A_i \in \vicinity_{\xi_i}(x_i) \\
&\iff \forall i\in I: x_i \in \inh_{\xi_i}(A_i) \\
&\iff \seq{x_i}_{i\in I} \in \bigtimes_{i\in I}\inh_{\xi_i}(A_i).
\end{align*}
\end{proof}

\begin{definition}
Let $\{X_i\}_{i\in I}$ be a set of sets and $\setbuilder{F_i \in \powerfilters(X_i)}{i\in I}$ a set of filters. Then the \udef{product filter} is defined by
\[ \bigotimes_{i\in I}F_i \defeq \setbuilder{\bigtimes_{i\in I}A_i}{\forall i\in I: A_i \in F_i \;\land\; A_i = X_i, \,\text{except for finitely many $A_i$}}.  \]
\end{definition}

\begin{lemma}
The product filter of proper filters is proper.
\end{lemma}
\begin{proof}
Let $\{X_i\}_{i\in I}$ be a set of sets and $\setbuilder{F_i \in \powerfilters(X_i)}{i\in I}$ a set of proper filters. Assume, towards a contradiction, that $\bigotimes_{i\in I}F_i$ is not proper. Take $\bigtimes_{i\in I}A_i\in \bigotimes_{i\in I}F_i$, then $\left(\bigtimes_{i\in I}A_i\right)^c \in \bigotimes_{i\in I}F_i$ and thus we can find $\bigtimes_{i\in I}B_i\in \bigotimes_{i\in I}F_i$ such that $\bigtimes_{i\in I}B_i \subseteq \left(\bigtimes_{i\in I}A_i\right)^c$. We have
\[ \emptyset = \left(\bigtimes_{i\in I}A_i\right) \cap \left(\bigtimes_{i\in I}B_i\right) = \bigtimes_{i\in I}(A_i\cap B_i). \]
Now the right-hand side is only empty if there exists $i\in I$ such that $A_i\cap B_i = \emptyset$. In this case $F_i$ is not a proper filter.
\end{proof}

\begin{lemma} \label{projectionsOfProductFilter}
Let $X,Y$ be sets, $F\in \powerfilters(X)$ and $G\in \powerfilters(Y)$. Then
\begin{enumerate}
\item $p_1^{\imf\imf}[F\otimes G] = F$;
\item $p_2^{\imf\imf}[F\otimes G] = G$.
\end{enumerate}
\end{lemma}
\begin{proof}
The elements of $F$ form a basis of $p_1^{\imf\imf}[F\otimes G]$. Similarly the elements of $G$ form a basis of $p_2^{\imf\imf}[F\otimes G]$.
\end{proof}

\begin{lemma} \label{filterFactorisationInequality}
Let $X,Y$ be sets and $H \in \powerfilters(X\times Y)$. Then $p_1^{\imf\imf}(H)\otimes p_2^{\imf\imf}(H) \subseteq H$.
\end{lemma}
\begin{proof}
Take some $A\in p_1^{\imf\imf}(H)\otimes p_2^{\imf\imf}(H)$. Then there exist $B,C\in H$ such that $A = p_1^\imf(B)\times p_2^\imf(C)$, which means that $B\cap C\subseteq A$. Now $B\cap C \in H$, so $A\in H$.
\end{proof}

\begin{lemma} \label{projectionsOfUltrafilterAreUltra}
Let $X,Y$ be sets and $H \in \powerfilters(X\times Y)$. If $H$ is an ultrafilter, then $p_1^{\imf\imf}[H]$ and $p_2^{\imf\imf}[H]$ are also ultrafilters.
\end{lemma}
\begin{proof}
Assume, towards a contradiction, that $p_1^{\imf\imf}[H]$ is not an ultrafilter. In this case we can find a proper filter $F\supsetneq p_1^{\imf\imf}[H]$.

Now we show that $F\otimes p_2^{\imf\imf}[H] \amesh H$.
Take $A\in F$ and $B, C\in H$. We need to show that $A\times p_2^{\imf}(B) \mesh C$. Because $B\cap C\in H$ and $p_1^{\imf\imf}[H]\subseteq F$, we have $p_1^\imf(B\cap C)\mesh A$. Take $a\in p_1^\imf(B\cap C)\cap A$. There exists a $b$ such that $(a,b)\in B\cap C$. Also $(a,b) \in A\times p_2^\imf(B\cap C)$. Thus $(B\cap C) \mesh \big(A\times p_2^\imf(B\cap C)\big)$. This implies $C \mesh A\times p_2^{\imf}(B)$.

Then we have that $\big(F\otimes p_2^{\imf\imf}[H]\big)\vee H$ is proper. Because $H$ is an ultrafilter, we must have $\big(F\otimes p_2^{\imf\imf}[H]\big)\vee H = H$, which means that $F\otimes p_2^{\imf\imf}[H] \subseteq H$.

Finally we apply $p_1^{\imf\imf}$ to both sides of the inclusion to get $F \subseteq p_1^{\imf\imf}[H]$. This is a contradiction.
\end{proof}

\begin{lemma} \label{productPrincipalUltrafilter}
Let $X,Y$ be sets, $x\in X$ and $y\in Y$. Then $\pfilter{(x,y)} = \pfilter{x} \otimes \pfilter{y}$.
\end{lemma}
\begin{proof}
It is enough to check that $(x,y)\in\pfilter{x}\otimes\pfilter{y}$ and that $\pfilter{x}\otimes\pfilter{y}$ is not trivial. Both these statements are immediately clear.
\end{proof}

\begin{lemma} \label{intersectionProductFilters}
Let $X,Y$ be sets, $F,G\in \powerfilters(X)$ and $H\in\powerfilters(Y)$. Then
\[ (F\cap G)\otimes H = (F\otimes H)\cap (G\otimes H). \]
\end{lemma}
\begin{proof}
$\boxed{\subseteq}$ Take $A\in (F\cap G)$ and $B\in H$. Then $A\times B\in F\otimes H$ and $A\times B\in G\otimes H$.

$\boxed{\supseteq}$ Take $A \in (F\otimes H)\cap (G\otimes H)$. Then we can find $B_1\in F$, $C_1,C_2\in H$ and $B_2\in G$ such that $B_1\times C_1 \subseteq A$ and $B_2\times C_2 \subseteq A$. Now $(B_1\cup B_2)\times (C_1\cap C_2) \subseteq A$. Also $B_1\cup B_2 \in F\cap G$ and $C_1\cap C_2 \in H$, so $(B_1\cup B_2)\times (C_1\cap C_2) \in (F\cap G)\otimes H$. By upwards closure $A\in (F\cap G)\otimes H$.
\end{proof}

\begin{lemma} \label{convergenceFiniteProductFilter}
Let $\sSet{X,\xi}$ and $\sSet{Y,\zeta}$ be convergence spaces, $F\in \powerfilters(X\times Y)$, $G\in \powerfilters(X)$ and $H\in \powerfilters(Y)$. Then
\begin{enumerate}
\item $F \overset{\xi \otimes \zeta}{\longrightarrow} (x,y)$ \textup{if and only if} $p_1^{\imf\imf}(F) \overset{\xi}{\longrightarrow} x$ and $p_2^{\imf\imf}(F) \overset{\zeta}{\longrightarrow} y$;
\item $G\otimes H \overset{\xi \otimes \zeta}{\longrightarrow} (x,y)$ \textup{if and only if} $G \overset{\xi}{\longrightarrow} x$ and $H \overset{\zeta}{\longrightarrow} y$.
\end{enumerate}
\end{lemma}
\begin{proof}
Point (1) is a restatement of \ref{initialFinalConvergence}. Point (2) follows from point (1) because $p_1^{\imf\imf}(F\otimes G) = F$ and $p_2^{\imf\imf}(F\otimes G) = G$ by \ref{projectionsOfProductFilter}.
\end{proof}


\begin{lemma}
Let $\sSet{X,\xi}, \sSet{Y,\sigma}$ and $\sSet{Z,\zeta}$ be convergence spaces and $f: X\to Y\times Z$ be a function of the form
\[ f: X\to Y\times Z: x\mapsto f(x) = (f_Y(x), f_Z(x)). \]
The $f$ is continuous \textup{if and only if} $f_Y$ and $f_Z$ are continuous.
\end{lemma}
\begin{proof}
Follows immediately from \ref{characteristicPropertyInitialFinalConvergence},
because $f_Y = p_1\circ f$ and $f_Z = p_2\circ f$.
\end{proof}
\begin{corollary} \label{continuousEmbeddingProduct}
Let $\sSet{X,\xi}$ and $\sSet{Y,\sigma}$ be convergence spaces. Then for all $y\in Y$, the function $X\to X\times Y: x\mapsto (x,y)$ is continuous.
\end{corollary}
\begin{proof}
The functions $\id_X$ and $\underline{y}$ are continuous.
\end{proof}
\begin{corollary} \label{productContinuousFunctions}
Let $\sSet{X,\xi}$ and $\sSet{Y,\sigma}$ be convergence spaces and $f_X: X\to X, f_Y: Y\to Y$ be continuous functions. Then
\[ f_X\times f_Y: X\times Y \to X\times Y: (x,y)\mapsto (f_X(x), f_Y(y))  \]
is continuous.
\end{corollary}
\begin{proof}
The functions $p_1\circ f_X: (x,y) \mapsto f_X(x)$ and $p_2\circ f_Y: (x,y) \mapsto f_Y(y)$ are continuous.
\end{proof}

\subsubsection{Subspace convergence}
\begin{definition}
Let $\sSet{X,\xi}$ be a convergence space and $A\subseteq X$ a subset. The \udef{subspace convergence} $\xi|_A$ on $A$ is the initial convergence w.r.t. $\{\iota: A \hookrightarrow X: a\mapsto a\}$. The convergence space $\sSet{A,\xi|_A}$ is called a \udef{convergence subspace} of $X$.
\end{definition}

\begin{lemma}
Let $\sSet{X,\xi}$ be a convergence space and $\sSet{A,\xi|_A}$ a subspace.
\begin{enumerate}
\item If $F\in \powerfilters(A)$ and $F\overset{\xi|_A}{\longrightarrow} a$, then $F\overset{\xi}{\longrightarrow} a$.
\item If $F\in \powerfilters(X)$, $a\in A$ and $F\overset{\xi}{\longrightarrow} a$, then $F|_A\overset{\xi|_A}{\longrightarrow} a$.
\end{enumerate}
\end{lemma}
\begin{proof}
(1) Assume $F\overset{\xi|_A}{\longrightarrow} a$, then $\iota^{\imf\imf}[F] = F$ and $\iota(a) = a$. Then $F\overset{\xi}{\longrightarrow} a$ by continuity.

(2) Assume $F\overset{\xi}{\longrightarrow} a\in A$. Then $\iota^{\preimf\imf}[F] = F|_A$ and $\iota^\preimf\{a\} = \{a\}$, so $F|_A\overset{\xi|_A}{\longrightarrow} a$ by \ref{initialFinalConvergence}.
\end{proof}

\begin{lemma}
Restrictions on continuous functions are continuous (domain + codomain).
\end{lemma}

\begin{proposition} \label{continuityRestrictionExpansion}
Let $\sSet{X,\xi}$, $\sSet{Y,\sigma}$ and $\sSet{Z,\zeta}$ be convergence spaces.
\begin{enumerate}
\item \textup{(Restricting the domain)} If $f:X\to Y$ is continuous and $A$ is a subspace of $X$, then the restricted function $f|_{A}:A\to Y$ is continuous.
\item \textup{(Restricting the range)} Let $f:X\to Y$ be continuous. If $Z$ is a subspace of $Y$ containing the image set $f[X]$, then $f:X\to Z$ is continuous.
\item \textup{(Expanding the range)} Let $f:X\to Y$ be continuous. If $Y$ is a subspace of $Z$, then $f:X\to Z$ is continuous.
\end{enumerate}
\end{proposition}
\begin{proof}
(1) Composition of continuous maps: $f|_{A} = f\circ\iota$.

(2) Composition of continuous maps: $f:X\to Z = \iota \circ (f: X\to Y)$.

(3) Characteristic property \ref{characteristicPropertyInitialFinalConvergence}: if $f:X\to Y = \iota \circ (f:X\to Z)$ is continuous, then $f:X\to Z$ is too.
\end{proof}



\subsubsection{Quotient convergence}
\begin{proposition}
Each convergence space is the quotient convergence space of a topological space.
\end{proposition}

\subsection{Projective and injective limits}
\begin{definition}
Let $\sSet{I, \prec}$ be an upwards directed set.
\begin{itemize}
\item Let $\{sSet{X_i, \xi_i}\}_{i\in I}$ be a set of convergence spaces and for each $j \succ i$, let $p_{j,i}: X_j \to X_i$ be a continuous mapping. Then the structure $\sSet{I, \{\sSet{X_i, \xi_i}\}_{i\in I}, \{p_{j,i}\}_{j\succ i}}$ is called a \udef{projective system} (or \udef{inverse system}) if for all $k \succ j \succ i \in I$ the diagram
\[ \begin{tikzcd}
X_k \ar[rr, "p_{k,i}"] \ar[rd, "p_{k,j}"] & & X_i \\
& X_j \ar[ur, "p_{j,i}"]
\end{tikzcd} \qquad \text{commutes.} \]
Let $\sSet{X, \xi}$ be a convergence space and $p_i: X\to X_i$ a continuous mapping for all $i\in I$. 
\end{itemize}
\end{definition}

\chapter{Separation axioms and other properties of convergences spaces}

\section{Distinguishability, separation and regularity}
\subsection{Distinguishable points}
\begin{definition}
Let $\sSet{X,\xi}$ be a convergence space and $x,y\in X$. We call $x$ and $y$ \udef{distinguishable} if ${\lim_\xi}^{-1}(x) \neq {\lim_\xi}^{-1}(y)$.
\end{definition}

In orther words, $x,y \in X$ are distinguishable if there exists a filter $F \in \powerfilters(X)$ such that
\[ \Big(x\in \lim_\xi F \land y\notin \lim_\xi F\Big)\;\lor\; \Big(x\notin \lim_\xi F \land y\in \lim_\xi F\Big). \]

We say $F$ \udef{distinguishes} $x$ and $y$.

\begin{proposition} \label{distinguishabilityPrincipalUltrafilters}
Let $\sSet{X,\xi}$ be a Kent convergence space and $x,y\in X$. Then $x$ and $y$ are indistinguishable \textup{if and only if}
\[ \pfilter{x} \to y \qquad \text{and}\qquad \pfilter{y}\to x. \]
\end{proposition}
\begin{proof}
The direction $\Rightarrow$ is clear: from $\pfilter{x} \to x$, we get $\pfilter{x}\to y$ by indistinguishability.

The direction $\Leftarrow$ is proved by contradiction. Assume $x$ and $y$ are distinguishable, so there exists a filter $F$ such that $F\to x$ but $F\not\to y$. Then $F\cap \pfilter{x} \to y$ by the definining property of Kent spaces. Now $F\cap \pfilter{y} \subseteq F$, so $F\to y$. This is a contradiction.
\end{proof}


\subsection{Separation}
\begin{definition}
Let $\sSet{X,\xi}$ be a convergence space and $A,B\subseteq X$. We say $A$ and $B$ are \udef{separated} if $\adh_\xi(A) \perp B$ and $A\perp \adh_\xi(B)$.

We call two points $x,y\in X$ \udef{separated} if $\{x\}$ and $\{y\}$ are separated.
\end{definition}

\begin{proposition} \label{separatednessPrincipalUltrafilters}
Let $X$ be a set, $\xi$ a convergence on $X$ and $x,y\in X$. Then $x$ and $y$ are separated \textup{if and only if}
\[ \pfilter{x} \not\to y \qquad \text{and}\qquad \pfilter{y}\not\to x. \]
\end{proposition}
\begin{proof}
By \ref{singletonAdherence} we have $\adh_\xi(\{x\}) = \lim_\xi\pfilter{x}$ and $\adh_\xi(\{y\}) = \lim_\xi\pfilter{y}$.
\end{proof}
In other words, $x$ and $y$ are not separated iff $\pfilter{x} \to y$ or $\pfilter{y} \to x$.

\begin{lemma} \label{separatedDistinguishable}
Let $\sSet{X,\xi}$ be a convergence space and $x,y\in X$. If $x,y$ are separated, then they are distinguishable.
\end{lemma}
\begin{proof}
Assume $x,y$ separated. WLOG we can assume $\pfilter{x} \not\to y$. So $\pfilter{x}$ distinguishes $x$ and $y$.
\end{proof}

\subsubsection{Separation by convergent filters}
\begin{definition}
Let $\sSet{X,\xi}$ be a convergence space and $A,B\subseteq X$. We say $A$ and $B$ are \udef{separated by convergent filters} if for all approaches $F:X\to \powerfilters(X)$, the contours satisfy $\neg(F(A)\amesh F(B))$.

We call two points $x,y\in X$ separated by convergent filters if $\{x\}$ and $\{y\}$ are separated by convergent filters.
\end{definition}

\begin{lemma} \label{pointsSeparatedConvergentFilters}
Let $\sSet{X,\xi}$ be a convergence space. For $x,y\in X$ the following are equivalent
\begin{enumerate}
\item $x,y$ are separated by convergent filters;
\item $\lim^{-1}(x)\perp \lim^{-1}(y)$.
\end{enumerate}
\end{lemma}
\begin{proof}
$(1) \Rightarrow (2)$ Assume $\lim^{-1}(x)\amesh \lim^{-1}(y)$, i.e.\ there exists some filter $F\in \lim^{-1}(x)\cap \lim^{-1}(y)$. Then the constant approach $x,y\mapsto F$ has $F(x) = F(y)$, meaning $F(x)\amesh F(y)$ and thus $x,y$ are not separated by convergent filters.

$(2) \Rightarrow (1)$ Assume $\lim^{-1}(x)\perp \lim^{-1}(y)$. Pick an arbitrary approach $F$. If $F(x)\amesh F(y)$, then $F(x)\vee F(y)$ is proper by \ref{joinProperFilter} and in $\lim^{-1}(x)\cap \lim^{-1}(y)$ by monotonicity, which is a contradiction.
\end{proof}


\subsubsection{Separation by vicinities}
\begin{definition}
Let $\sSet{X,\xi}$ be a convergence space and $A,B\subseteq X$. We say $A$ and $B$ are \udef{separated by vicinities} if there exist $U\in \vicinity_\xi(A)$ and $V\in\vicinity_\xi(B)$ such that $U\perp V$.

We call two points $x,y\in X$ separated by vicinities if $\{x\}$ and $\{y\}$ are separated by vicinities, i.e.\ there exist vicinities $U,V$ of $x,y$, resp., such that $U\perp V$.
\end{definition}

\begin{lemma}
Let $\sSet{X,\xi}$ be a convergence space and $A,B\subseteq X$. The following are equivalent:
\begin{enumerate}
\item $A$ and $B$ are separated by vicinities;
\item $\neg(\vicinity_\xi(A)\amesh \vicinity_\xi(B))$.
\end{enumerate}
\end{lemma}

\begin{proposition} \label{disjointVicinitiesConvergentFilterSeparation}
Let $\sSet{X,\xi}$ be a convergence space.
\begin{enumerate}
\item Separation by vicinities implies separation by convergent filters.
\item If $\xi$ is pretopological, the converse also holds.
\end{enumerate}
\end{proposition}
\begin{proof}
$(1)$ Assume $A,B\subseteq X$ are separated by vicinities $U$ and $V$. Assume, towards a contradiction, that $F(A)\amesh F(B)$ for some approach $F$. Then for all $a\in A$, $U\in F(a)$, so $U\in F(A)$. Similarly $V\in F(B)$. This means $U\mesh V$, which is a contradiction.

$(2)$ If $\xi$ is pretopological, then $\vicinity_\xi$ is an approach. Thus $\neg(\vicinity_\xi(A)\amesh \vicinity_\xi(B))$, meaning there exist disjoint vicinities of $A,B$.
\end{proof}
Thus in the pretopological case all approaches $F$ satisfy $\neg(F(A)\mesh F(B))$ iff the vicinity filter satisfies $\neg(\vicinity(A)\mesh \vicinity(B))$.


\begin{proposition} \label{separationByVicinitiesEquivalences}
Let $\sSet{X,\xi}$ be a convergence space.
\begin{enumerate}
\item If $A,B \subseteq X$ are separated by vicinities, then .
\item If $\xi$ is pretopological and $\lim^{-1}(x)\perp \lim^{-1}(y)$ for some $x,y\in X$, then $x,y$ are separated by vicinities.
\end{enumerate}
\end{proposition}
\begin{proof}
$(1)$ Assume $A,B$ are separated by vicinities $U$ and $V$. Assume, towards a contradiction, that $\lim^{-1}(A)\amesh \lim^{-1}(B)$, i.e.\ there exists a filter $F$ that converges to $x\in A$ and $y\in B$. Then $U,V\in F$ and thus $U\cap V = \emptyset \in F$, meaning $F$ is not a proper filter.

$(2)$ Assume $x,y$ are not separated by vicinities. Then $\vicinity_\xi(x)\amesh \vicinity_\xi(y)$, meaning $x$ is an accumulation point of $\vicinity_\xi(y)$. By \ref{subfilterToAccumulationPoint}, there exists a filter $G\geq \vicinity_\xi(y)$ such that $G\to x$. Then $G\in \lim^{-1}(x)\cap \lim^{-1}(y)$.
\end{proof}

\subsubsection{Separation by neighbourhoods}
\begin{definition}
Let $\sSet{X,\xi}$ be a convergence space and $A,B\subseteq X$. We say $A$ and $B$ are \udef{separated by neighbourhoods} if there exist disjoint neighbourhoods of $A$ and $B$.

We call two points $x,y\in X$ separated by neighbourhoods if $\{x\}$ and $\{y\}$ are separated by neighbourhoods.
\end{definition}

\begin{lemma} \label{neighbourhoodSeparationLemma}
Let $\sSet{X,\xi}$ be a convergence space and $A,B\subseteq X$. Then $A,B$ are separated by neighbourhoods \textup{if and only if} there exists an open set $U$ such that $A \subseteq U \subseteq \overline{U} \subseteq B^c$.
\end{lemma}
\begin{proof}
WLOG we may take $\overline{U}^c$ to be the neighbourhood of $B$.
\end{proof}

\subsubsection{Separation by closed vicinities}
\subsubsection{Separation by functions}
\url{https://en.wikipedia.org/wiki/Separated_sets}

\subsection{Regularity}
\begin{definition}
Let $\sSet{X,\xi}$ be a convergence space and $\mathcal{Z} \subseteq \powerset(X)$. The convergence $\xi$ is called \udef{$\mathcal{Z}$-regular} if for all $x\in X, F\in\lim_\xi^{-1}(x)$ there exists a filter base $G\subseteq \mathcal{Z}$ such that
\begin{itemize}
\item $G \preceq F$;
\item $G \overset{\xi}{\longrightarrow} x$.
\end{itemize}
\end{definition}

\section{Separation properties}
\subsection{$T_0$ or Kolmogorov}
\begin{definition}
We call a convergence space $\sSet{X, \xi}$ \udef{Kolmogorov} or \udef{$T_0$} if every pair of distinct points in $X$ is distinguishable.
\end{definition}

\subsection{$R_0$ or symmetric}
\begin{definition}
Let $X$ be a set and $\xi$ a convergence on $X$. Then $\xi$ is called \udef{symmetric} or \udef{$R_0$} if all distinguishable pairs of points are separated.
\end{definition}

\begin{proposition} \label{R0conditions}
Let $X$ be a set and $\xi$ a convergence on $X$. Then the following are equivalent:
\begin{enumerate}
\item $\xi$ is an $R_0$ convergence;
\item $x$ and $y$ are indistinguishable \textup{if and only if} they are separated;
\item for all $x,y\in X$, $x$ and $y$ are indistinguishable \textup{if and only if} $\pfilter{x} \to y$;
\item $\adh_\xi(\{x\})$ is the set of points that are indistinguishable from $x$ for all $x\in X$;
\end{enumerate}
The following are consequences of the above. If $\xi$ is a Kent convergence, then they are also equivalent:
\begin{enumerate} \setcounter{enumi}{4}
\item the set $\setbuilder{\adh_\xi(\{x\})}{x\in X}$ is a partition of $X$;
\item for all $x,y\in X$:  $\pfilter{x} \to y$ \textup{if and only if} $\pfilter{y} \to x$;
\end{enumerate}
\end{proposition}
\begin{proof}
$(1) \Rightarrow (2)$ The converse to the $R_0$ condition, that separated points are distinguishable, is automatic (see \ref{separatedDistinguishable}).

$(2) \Rightarrow (3)$ All points that are indistinguishable from $x$ are in $\adh_\xi(\{x\}) = \lim_\xi\{x\}$ by construction. Assume $y$ is distinguishable from $x$. Then $y$ is separated from $x$, so $\pfilter{x}\not\to x$ by \ref{separatednessPrincipalUltrafilters}.

$(3) \Rightarrow (4)$ $\adh_\xi(\{x\}) = \lim_\xi\pfilter{x}$ by \ref{singletonAdherence}.

$(4) \Rightarrow (1)$ Assume $x,y\in X$ are distinguishable. Then $y\notin\adh_\xi(\{x\})$ and $x\notin\adh_\xi(\{y\})$. This means $\pfilter{x} \not\to y$ and $\pfilter{y} \not\to x$ and we conclude by \ref{separatednessPrincipalUltrafilters}.


$(4) \Rightarrow (5)$ Indistinguishability is an equivalence relation.

$(5) \Rightarrow (6)$ Equivalence relations are symmetric.

$(6) \Rightarrow (1)$ Assume $\xi$ of a Kent convergence and $x,y$ distinguishable. Then $\pfilter{x}\not\to y$ or $\pfilter{y}\not\to x$ by \ref{distinguishabilityPrincipalUltrafilters}. Because of (6), the ``or'' becomes an ``and'' and we conclude with \ref{separatednessPrincipalUltrafilters}.
\end{proof}

TODO: characterisation  ``every open set is a union of closed sets'' or
``every closed set is an intersection of open sets''.

?? $\adh_\xi(\{x\})$ is closed for all $x\in X$ ??

\subsection{$T_1$ or Fréchet}
\begin{definition}
Let $X$ be a set and $\xi$ a preconvergence on $X$. Then $\xi$ is called \udef{Fréchet} or \udef{$T_1$} if all distinct points in $X$ are separated.
\end{definition}
\url{https://en.wikipedia.org/wiki/T1_space}

\begin{proposition} \label{FrechetCharacterisation}
Let $X$ be a set and $\xi$ a convergence on $X$. Then the following are equivalent:
\begin{enumerate}
\item $\xi$ is a $T_1$ convergence;
\item $\xi$ is both $T_0$ and $R_0$;
\item all singletons are closed, i.e.\ $\forall x\in X: \; \adh_\xi(\{x\}) = \{x\}$;
\item $\forall x\in X: \; \lim_\xi\pfilter{x} = \{x\}$;
\item $\forall x\in X: \; \lim_\xi \pfilter{x} \subseteq \{x\}$;
\end{enumerate}
\end{proposition}
\begin{proof}
$(1) \Leftrightarrow (2)$ Definitions together with point (2) of \ref{R0conditions}.

$(1) \Leftrightarrow (3)$ From point (4) of \ref{R0conditions}.

$(3) \Leftrightarrow (4)$ From \ref{singletonAdherence}.

$(4) \Leftrightarrow (5)$ Convergences are centered.
\end{proof}
TODO: every singleton in $X$ is closed;  every finite subset of $X$ is closed.
\begin{corollary} \label{finiteConvergenceDiscrete}
Let $X$ be a finite set, then the only $T_1$ convergence on $X$ is the discrete convergence $\iota_X$.
\end{corollary}
\begin{proof}
Let $\xi$ be a $T_1$ convergence on $X$ and $F$ a proper filter in $\powerset(X)$. Now $F$ is principal (TODO ref), so $F = \upset \ker F$. If $\ker F$ is a singleton, then $\lim F = \ker F$ by point (2). Otherwise $\lim F = \emptyset$ by point (4). This is discrete convergence.
\end{proof}

\begin{lemma}
If $\sSet{X,\xi}$ is a $T_1$ convergence on $X$, then $\xi$ is also $T_0$.
\end{lemma}
\begin{proof}
The filters $\pfilter{x}$ and $\pfilter{y}$ distinguish $x,y\in X$.
\end{proof}

\begin{proposition}
Let $\xi$ be a $T_1$ preconvergence on a set $X$ and $F$ a filter in $\powerfilters(X)$. If $x\in \lim F$, then $\ker F \subseteq \{x\}$.
\end{proposition}
\begin{proof}
Suppose $\ker F$ in non-empty and take $y \in \ker F$. Then $F \subseteq \pfilter{y}$ and so
\[ x \in \lim F \subseteq \lim \pfilter{y} \subseteq \{y\}. \]
This means that $y = x$.
\end{proof}
\begin{corollary}
The kernel of a convergent proper filter in a $T_1$ space is empty or a singleton.
\end{corollary}
By constrast, in $T_2$ spaces the \emph{limit} (not kernel) of a convergent proper filter is empty or a singleton.

\begin{proposition} \label{setKernelVicinityFilter}
Let $\sSet{X,\xi}$ be a $T_1$ pretopological convergence space and $A \subseteq X$ a subset. Then $\bigcap\vicinity(A) = A$.
\end{proposition}
\begin{proof}
We have $A\subseteq \bigcap\vicinity(A)$ from \ref{vicinityOfSetCorollary}. For the converse, take $x\notin A$. By $T_1$, we have $\adh\{x\} \perp A$, which implies $\forall y\in A:\; y\notin\adh\{x\}$. By \ref{adherenceInherenceCharacterisation}, this is equivalent to
\[ \forall y\in A: \forall U\in \vicinity(y):\; U\setminus \{x\} \in \vicinity(y). \]
This implies
\[ \forall V\in \vicinity(A): \; V\setminus\{x\} \in \vicinity(A), \]
which means that $x\notin \bigcap\vicinity(A)$.
\end{proof}
\begin{corollary}
In a topological $T_1$ convergence space, every set is an intersection of open sets.
\end{corollary}
Note that this corollary can also simply be proved by writing
\[ A = \bigcap_{x\in A^c}\{x\}^c. \]

TODO: big question marks:
\begin{proposition}
Let $X$ be a set and $\xi$ a convergence on $X$. If $\xi$ is $T_1$, then $\forall x\neq y\in X: \exists F,G\in \powerfilters(X)$ such that
\[ x\in \lim_\xi F \;\land\; y\in \lim_\xi G \;\land\; x\notin \lim_\xi G \;\land\; y\notin \lim_\xi F. \]
If $\xi$ is topological, the converse also holds.
\end{proposition}
\begin{proof}
$\Rightarrow$ We may take $F = \pfilter{x}$ and $G = \pfilter{y}$.

$\Leftarrow$ If $\xi$ is topological, we may take $F = \vicinity_\xi(x)$ and $G = \vicinity_\xi(y)$. It is enough to show that $y\notin \adh_\xi(\{x\})$.

Assume, towards a contradiction, that $y\in \adh_\xi(\{x\})$. Then $\{x\} \in \vicinity_\xi(y)^{\mesh}$
\end{proof}

\subsection{$R_1$ or reciprocal}
\begin{definition}
Let $\sSet{X,\xi}$ be a convergence space. Then $\xi$ is called \udef{$R_1$}, \udef{reciprocal} or \udef{preregular} if all distinguishable points are separated by convergent filters.
\end{definition}
TODO review definition.

\url{https://gdz.sub.uni-goettingen.de/id/PPN235181684_0187?tify={%22pages%22:[191],%22panX%22:0.893,%22panY%22:0.579,%22view%22:%22info%22,%22zoom%22:0.894}}


\begin{proposition} \label{R1Conditions}
Let $X$ be a set and $\xi$ a convergence on $X$. Then the following are equivalent:
\begin{enumerate}
\item $\xi$ is an $R_1$ convergence;
\item if $x$ and $y$ are distinguishable, then $\lim_\xi^{-1}(x)\perp \lim_\xi^{-1}(y)$;
\item $x$ and $y$ are distinguishable \textup{if and only if} $\lim_\xi^{-1}(x)\perp \lim_\xi^{-1}(y)$;
\item for all $x,y\in X$, either $\lim_\xi^{-1}(x) = \lim_\xi^{-1}(y)$ or $\lim_\xi^{-1}(x)\perp \lim_\xi^{-1}(y)$;
\item for all $x,y\in X$: $\lim_\xi^{-1}(x)\mesh \lim_\xi^{-1}(y)$ implies $\lim_\xi^{-1}(x) = \lim_\xi^{-1}(y)$;
\item the set $\setbuilder{\lim^{-1}_\xi(x)}{x\in X}$ is a partition of the set on convergent filters on $X$;
\item if there exists a filter $F$ such that $F \to x$ and $F \to y$, then $x$ and $y$ are indistinguishable.
\end{enumerate}
\end{proposition}
\begin{proof}
$(1) \Leftrightarrow (2)$ By \ref{pointsSeparatedConvergentFilters}.

$(2) \Rightarrow (3)$ If $\lim_\xi^{-1}(x)\perp \lim_\xi^{-1}(y)$, then $x$ and $y$ are definitely distinguishable, e.g.\ by $\pfilter{x}$.

$(3) \Rightarrow (4) \Rightarrow (5) \Rightarrow (6)$ Clear.

$(6) \Rightarrow (7)$ From $F\to x$ and $F\to y$, we get $F\in \lim_\xi^{-1}(x)$ and $F\in \lim_\xi^{-1}(y)$, so $\lim_\xi^{-1}(x)\amesh \lim_\xi^{-1}(y)$. Using (3) this implies $\lim_\xi^{-1}(x) = \lim_\xi^{-1}(y)$.

$(7) \Rightarrow (2)$ By contraposition.
\end{proof}
\begin{corollary}
Any $R_1$ convergence space is also $R_0$.
\end{corollary}
\begin{proof}
Compare point (2) with point (3) of \ref{R0conditions} and note that $\pfilter{x}\to y$ implies $\lim_\xi^{-1}(x)\amesh \lim_\xi^{-1}(y)$. 
\end{proof}


\subsection{$T_2$ or Hausdorff}
\begin{definition}
Let $X$ be a set and $\xi$ a preconvergence on $X$. Then $\xi$ is called \udef{Hausdorff} or \udef{$T_2$} if every proper $\xi$-limit contains at most one point.
\end{definition}
By ``proper $\xi$-limit'' we mean we exclude from this condition the degenerate filter $\powerset(X)$. Otherwise there would be no $T_2$ convergences by \ref{limitDegenerateFilter}.

\begin{proposition}
Let $X$ be a set and $\xi$ a convergence on $X$. Then the following are equivalent:
\begin{enumerate}
\item $\xi$ is a $T_2$ convergence;
\item if $x \neq y$, then ${\lim_\xi}^{-1}(x)\perp {\lim_\xi}^{-1}(y)$;
\item $\xi$ is $T_0$ and $R_1$;
\item $\xi$ is $T_1$ and $R_1$;
\end{enumerate}
\end{proposition}
\begin{proof}
$(1) \Leftrightarrow (2)$ If $F \in \lim_\xi^{-1}(x)\cap \lim_\xi^{-1}(y)$, then $F \to x$ and $F\to y$, so $\xi$ would not be $T_2$.

$(2) \Leftrightarrow (3)$ Clear.

$(3) \Leftrightarrow (4)$ $R_1$ implies $R_0$ and $R_0+T_0$ is equivalent with $T_1$.
\end{proof}

\begin{proposition}
Every $T_2$ convergence is also $T_1$. If the space is finite, the converse also holds.
\end{proposition}
\begin{proof}
Let $\sSet{X, \xi}$ be a $T_2$ convergence space and $x\in X$. By $T_2$, $\lim_\xi\pfilter{x}$ is a singleton. Definition of convergence this singleton is $\{x\}$.

Now let $X$ be a finite set and let $F$ be a proper filter in $\powerfilters(X)$. Then $F$ is principal by \ref{finiteFiltersPrincipal} and not free because it is proper. So we can take $x\in \ker F$ and $F \subseteq \pfilter{x}$. Thus $\lim F \subseteq \lim \pfilter{x} \subseteq \{x\}$, meaning the convergence is $T_2$.
\end{proof}
\begin{corollary}
Let $X$ be a finite set, then the only $T_2$ convergence on $X$ is the discrete convergence $\iota_X$.
\end{corollary}
\begin{proof}
By \ref{finiteConvergenceDiscrete}.
\end{proof}

\begin{proposition}
TODO: move: topological Hausdorff implies regular. This does not hold for non-topological convergence in general.
\end{proposition}

\subsection{$R_2$ or regular}
\begin{definition}
Let $\sSet{X,\xi}$ be a convergence space. Then $\xi$ is called \udef{regular} or \udef{$R_2$} if it is based in $\adh_\xi^{\imf}(\powerset^2(X))$.
\end{definition}

\url{https://en.wikipedia.org/wiki/Regular_space}

\begin{proposition}
Let $\sSet{X,\xi}$ be a convergence space. Then the following are equivalent:
\begin{enumerate}
\item $\xi$ is an $R_2$ convergence;
\item for all $F\in\powerfilters(X)$, $F\overset{\xi}{\longrightarrow} x$ implies $\adh_\xi[F] = \setbuilder{\adh_\xi(A)}{A\in F} \overset{\xi}{\longrightarrow} x$.
\item for all $F\in\powerfilters(X)$: $F\overset{\xi}{\longrightarrow} x$ \textup{if and only if} $\adh_\xi[F] = \setbuilder{\adh_\xi(A)}{A\in F} \overset{\xi}{\longrightarrow} x$.
\end{enumerate}
\end{proposition}
\begin{proof}
$(1) \Leftrightarrow (2)$ Assume (1), then there exists a filter $G$ based in $\setbuilder{\adh_\xi(A)}{A\in\powerset(X)}$ such that $G\to x$ and $G\subseteq F$. We just need to show that $G\subseteq \adh[F]$. Indeed take $A\in G$, then $A = \adh(B)$ for some $B\subseteq X$. Now $B\in F$ TODO: is this wrong?

Because $\adh_\xi[F]\preceq F$ Then $\adh_\xi(A) \supseteq A$, so $\adh$ 

$(2) \Leftrightarrow (3)$ One direction is given by definition. The other follows by monotonicity because $\adh_\xi[F] \preceq F$.
\end{proof}

\begin{lemma}
Any $R_2$ convergence is also $R_1$.
\end{lemma}

\begin{proposition} \label{regularityBySeparation}
Let $\xi$ be a convergence.
\begin{itemize}
\item If $\xi$ is regular, then $\xi$ separates points from the complements of their vicinities by convergent filters.
\item If $\xi$ is pretopological, then the converse also holds.
\end{itemize}
\end{proposition}
\begin{proof}
(1) Fix an arbitrary approach $F: X\to \powerfilters(X)$ and $V_x\in \vicinity(x)$ for all $x\in X$.

By regularity we have $\vicinity(x)\subseteq \upset\adh[F(x)]$ for all $x\in X$.
This means
\begin{align*}
& \forall x\in X: \forall V\in \vicinity(x): \exists U \in F(x): \quad \adh_\xi(U) \subseteq V \\
\iff& \forall x\in X: \forall V\in \vicinity(x): \exists U \in F(x): \quad (\adh_\xi(U) \perp V^c) \\
\iff& \forall x\in X: \forall V\in \vicinity(x): \exists U \in F(x): \quad \neg(\adh_\xi(U) \mesh V^c) \\
\iff& \forall x\in X: \forall V\in \vicinity(x): \exists U \in F(x): \quad \neg(\{U\} \amesh \vicinity(V^c)) \\
\iff& \forall x\in X: \forall V\in \vicinity(x): \quad \neg(\forall U \in F(x):\{U\} \amesh \vicinity(V^c)) \\
\iff& \forall x\in X: \forall V\in \vicinity(x): \quad \neg(F(x) \amesh \vicinity(V^c)) \\
\implies& \forall x\in X: \forall V\in \vicinity(x): \quad \neg(F(x) \amesh F(V^c)).
\end{align*}
We have used \ref{setAdherenceInherence}.

(2) In the pretopological case, we can take $F = \vicinity$ and we can run the argument in reverse, because regularity is implied by $\vicinity(x)\subseteq \upset\adh[\vicinity(x)]$ for all $x\in X$.
\end{proof}

\begin{proposition} \label{topologicalRegularity}
Let $\sSet{X,\xi}$ be a topological space. Then the following are equivalent:
\begin{enumerate}
\item $\xi$ is regular;
\item for any $x\in X$ and any base $\mathcal{B}$ of $\neighbourhood_\xi(x)$, $\closure^\imf(\mathcal{B})$ is also a base of $\neighbourhood_\xi(x)$;
\item for any closed set $C$ there exist disjoint open sets $U,V$ such that $x\in U$ and $C\subseteq V$;
\item for any open set $O\subseteq X$ and $x\in O$ there exists an open set $U$ such that $x\subseteq U\subseteq \overline{U} \subseteq O$.
\end{enumerate}
\end{proposition}
\begin{proof}
$(1) \Leftrightarrow (2)$ TODO ref.

$(1) \Leftrightarrow (3)$ By \ref{regularityBySeparation}.

$(2) \Leftrightarrow (4)$ By \ref{neighbourhoodSeparationLemma}.
\end{proof}

\subsection{$T_3$ or regular Hausdorff}
\begin{definition}
Let $\sSet{X,\xi}$ be a convergence space. Then $\xi$ is called \udef{$T_3$} if it is regular and Hausdorff.
\end{definition}

\begin{proposition}
Let $X$ be a set and $\xi$ a convergence on $X$. Then the following are equivalent:
\begin{enumerate}
\item $\xi$ is a $T_3$ convergence, i.e.\ $R_2$ and $T_2$;
\item $\xi$ is $R_2$ and $T_0$.
\end{enumerate}
\end{proposition}

\subsection{$R_3$ or normal}
\begin{definition}
Let $\sSet{X,\xi}$ be a convergence space. Then $\xi$ is called \udef{normal} or \udef{$R_3$} if all disjoint closed sets are separated by convergent filters.
\end{definition}

\begin{proposition}
Let $\sSet{X,\xi}$ be a pretopological convergence space and $A,B\subseteq X$. Then $\lim^{-1}(\adh_\xi(A)) \perp \lim^{-1}(\adh_\xi(B))$ \textup{if and only if} there exist disjoint vicinities of $\adh_\xi(A)$ and $\adh_\xi(B)$.
\end{proposition}

\begin{proposition}
Any $R_3$ convergence is also $R_2$.
\end{proposition}


\subsection{$T_4$ or normal Hausdorff}
\begin{definition}
Let $\sSet{X,\xi}$ be a convergence space. Then $\xi$ is called \udef{$T_4$} if it is normal and Hausdorff.
\end{definition}

\begin{proposition}
Let $X$ be a set and $\xi$ a convergence on $X$. Then the following are equivalent:
\begin{enumerate}
\item $\xi$ is a $T_4$ convergence, i.e.\ $R_3$ and $T_2$;
\item $\xi$ is $R_3$ and $T_4$;
\item $\xi$ is $R_3$ and $T_0$.
\end{enumerate}
\end{proposition}


\section{Countability properties}
\subsection{$C1$ or first countable}
\begin{definition}
A convergence space $\sSet{X, \xi}$ is called \udef{first countable} or \udef{$C_1$} if for all $x\in X$ the vicinity filter $\vicinity_\xi(x)$ has a countable base.
\end{definition}

\subsubsection{Strongly first countable}

\subsection{$C2$ or second countable}
\begin{definition}
A convergence space $\sSet{X, \xi}$ is called \udef{second countable} or \udef{$C_2$} if $\xi$ has a countable base.
\end{definition}

\begin{lemma}
Second countable implies first countable.
\end{lemma}

\begin{lemma} \label{C2openBase}
Let $\sSet{X,\xi}$ be a $C_2$ topological convergence space. Then $\xi$ has a countable base of open sets.
\end{lemma}
\begin{proof}
See \ref{cardinalityPretopologicalBase}.
\end{proof}

\begin{lemma} \label{AnySetCountableIntersectionOfOpenSets}
Let $\sSet{X,\xi}$ be a $C_2$ and $T_1$ topological convergence space. Then every set $A\subseteq X$ can be written as both
\begin{enumerate}
\item a countable intersection of open sets; and
\item a countable union of closed sets.
\end{enumerate}
\end{lemma}
\begin{proof}
(1) By \ref{C2openBase}, $\xi$ has a countable base of open sets. By \ref{setKernelVicinityFilter} we have $A = \bigcap \neighbourhood(A) = \bigcap \upset \mathcal{F} = \bigcap \mathcal{F}$, for some subset $\mathcal{F}$ of the countable base of open sets.

(2) We can write $A^c$ as a countable intersection of open sets by (1). Taking the complement yields the result.
\end{proof}


\begin{proposition} \label{countableRegularityImpliesNormality}
Let $\sSet{X,\xi}$ be a topological space. If $\xi$ is regular and second countable, then $\xi$ is normal.
\end{proposition}
\begin{proof}
TOOD eg \url{https://www.math.auckland.ac.nz/~gauld/750-05/section3.pdf}
\end{proof}

\section{Comparison with reals and metrisability}
\subsection{Functional convergence properties}
\subsubsection{Functional closure}
\begin{definition}
Let $\sSet{X,\xi}$ be a convergence space and $A\subseteq X$. We call $A$ \udef{functionally closed} if there exists a continuous function $f: X\to \R$ and a closed set $C\subseteq \R$ such that $A = f^{-1}[C]$.
\end{definition}

\begin{proposition}
Every functionally closed set if closed. In a metric space the converse holds.
\end{proposition}
\begin{proof}
TODO + see \ref{distanceToSetContinuous}
\end{proof}

\begin{lemma} \label{functionallyClosedZeroSet}
Let $\sSet{X,\xi}$ be a convergence space and $A\subseteq X$. Then $A$ is functionally closed \textup{if and only if} there exists a continuous function $g: X\to \R$ such that $A = g^{-1}[\{0\}]$.
\end{lemma}
This means we may take $C = \{0\}$ in the definition of functionally closed.
\begin{proof}
If there exists such a function $g$, then $A$ is clearly functionally closed. For the converse, fix a continuous function $f: X\to \R$ and a closed set $C\subseteq \R$ such that $A = f^{-1}[C]$. Then we use the continuity of $d_C: \R\to\R: x\mapsto \inf_{c\in C}d(x,c)$ (see \ref{distanceToSetContinuous}) and set $g = d_C\circ f$.
\end{proof}

\subsubsection{Functional separation}
\begin{definition}
Let $\sSet{X,\xi}$ be a convergence space and $A,B\subseteq X$. Then $A$ and $B$ are \udef{functionally separated} if there exists a continuous function $f: X\to [0,1]\subseteq \R$ with $f[A] = \{0\}$ and $f[B] = \{1\}$.
\end{definition}

\begin{proposition}
Two sets are functionally separated \textup{if and only if} they are included in disjoint functionally closed sets.
\end{proposition}
\begin{proof}
First assume $A,B$ are functionally separated by $f$. Then $f^{-1}[\;\{0\}\;]$ and $f^{-1}[\;\{1\}\;]$ are disjoint functionally closed sets containing $A$, resp. $B$.

Conversely, assume $A\subseteq f^{-1}[\{0\}]$ and $B\subseteq g^{-1}[\{0\}]$ (we can take $C = \{0\}$ by \ref{functionallyClosedZeroSet}) with $f^{-1}[\{0\}]\perp g^{-1}[\{0\}]$. Then
\[ h: X\to \R: x\mapsto \frac{|f(x)|}{|f(x)|+|g(x)|} \]
is well-defined everywhere because $f^{-1}[\{0\}]\perp g^{-1}[\{0\}]$ and functionally separates $A$ and $B$.
\end{proof}

\begin{lemma} \label{UrysohnsLemmaLemma}
Let $\sSet{X,\xi}$ be a topological space and $D\subseteq \R$ a dense subset of the reals. Suppose $\sSet{U_d}{d\in D}$ is a set of open subsets of $X$ satisfying
\begin{enumerate}
\item if $d < t$, then $\overline{U_d}\subseteq U_t$;
\item $\bigcup_{d\in D}U_d = X$;
\item $\bigcap_{d\in D}U_d = \emptyset$.
\end{enumerate}
Then $f: X\to \R: x\mapsto \inf\setbuilder{d\in D}{x\in U_d}$ is continuous.
\end{lemma}
\begin{proof}
The function $f$ is well-defined, because $\setbuilder{d\in D}{x\in U_d}$ is never empty.

We will prove continuity at $x\in X$ using \ref{pretopologicalContinuityVicinities}. To that end, take some $U\in \vicinity_\R(f(x))$. By density, we can find some $d,t\in D$ such that $[d,t]\subseteq U$ and $d < f(x) < t$.

Now set $V = U_t\setminus\overline{U_d}$. It is then enough to prove that $V\in \vicinity_\xi(x)$ and $f[V]\subseteq [d,t]$. For the first point, note that $V = U_t\cap\overline{U_d}^c$, so $V$ is open.

Next we show $x\in V$. Indeed, from $f(x) < t$, we get that there exists $f(x) \leq s < t$ such that $x\in U_s$. Thus $x\in U_t$. From $d < f(x)$, we get that there exists $s>d$ such that $x\notin U_s$. By assumption ($d < t \implies \overline{U_d}\subseteq U_t$) this means that $x\notin \overline{U_d}$.

Finally note that if $y\in V \subseteq U_t$, then $f(y) \leq t$ and if $y\notin \overline{U_d}$, then $f(y) > s$ for all $s< d$. Thus $d \leq f(y)$ and $f[V]\subseteq [d,t]$.
\end{proof}

\begin{theorem}[Urysohn's lemma]
Let $\sSet{X,\xi}$ be a normal topological space. If $C_1,C_2$ are disjoint closed sets, then they are functionally separated.
\end{theorem}
\begin{proof}
We will construct a family $\setbuilder{U_r}{r\in Q}$ of open sets satisfying the conditions of \ref{UrysohnsLemmaLemma}.

First set $U_r = \emptyset$ for all $r<0$ and $U_r = X$ for all $r>1$.

Now set $U_1 = C_2^c$. By normality (TODO ref) there exists an open set $U_0$ such that $C_1 \subseteq U_0 \subseteq \overline{U_0} \subseteq U_1$.

We can enumerate the rationals in a sequence $\seq{r_n}_{n\in \N}$ with $r_0 = 0$ and $r_1 = 1$. We now recursively define $U_{r_n}$ with recursion invariant $d < t \implies \overline{U_d}\subseteq U_t$ as follows: Set
\[ V =  U_i \qquad \text{where $i = \max_{\substack{k < n \\ r_k < r_n}}k$} \qquad\text{and}\qquad W = U_j \qquad \text{where $j = \min_{\substack{k < n \\ r_k > r_n}}k$}. \]
By the recursion invariant, we have $\overline{U_i}\subseteq U_j$, so by normality we can define a $U_{r_k}$ such that $\overline{U_i} \subseteq U_{r_k} \subseteq \overline{U_{r_k}} \subseteq U_j$. This assignment satisfies the recursion invariant.

Then by \ref{UrysohnsLemmaLemma} the function $f: X\to \R: x\mapsto \inf\setbuilder{r\in \Q}{x\in U_r}$ is continuous. It functionally separates $C_1$ and $C_2$.
\end{proof}

\begin{corollary}[Urysohn's metrisation theorem]
Every regular and second countable topological space is pseudometrisable.
\end{corollary}
\begin{proof}
Let $\sSet{X,\xi}$ be such a topological space. By \ref{countableRegularityImpliesNormality}, $\xi$ is automatically normal. TODO embedding into $[0,1]^F$.
\end{proof}
TODO converse??


\subsubsection{Functional regularity}
\begin{definition}
Let $\sSet{X, \xi}$ and $\sSet{Y,\zeta}$ be convergence spaces. We call $\xi$ \udef{$\zeta$-functionally regular} if $\xi$ is the initial convergence on $X$ w.r.t. $\cont(\xi,\zeta)$.

In particular we call $\xi$ \udef{completely regular} if it is $\R$-functionally regular.
\end{definition}

\chapter{Compactness}
\begin{definition}
Let $\sSet{X,\xi}$ be a convergence space.
\begin{itemize}
\item The space $\sSet{X,\xi}$ is called \udef{compact} if every ultrafilter converges.
\item A subset $A\subseteq X$ is called \udef{compact} if the subspace $\sSet{A, \xi|_A}$ is compact.
\end{itemize}
\end{definition}

\begin{proposition}
Let $\sSet{X,\xi}$ be a convergence space and $A\subseteq X$ a subset.
\begin{enumerate}
\item If $X$ is compact and $A$ is closed, then $A$ is compact.
\item If $X$ is a Hausdorff space and $A$ is compact, then $A$ is closed.
\end{enumerate}
\end{proposition}
\begin{proof}
(1) Take some ultrafilter $F$ in $A$. Then $\upset F$ is an ultrafilter in $X$ by \ref{traceUltrafilters}, so it converges to some $x\in X$ by compactness. Now $A \in (\upset F)^{\mesh} \subseteq \vicinity(x)^{\mesh}$, so $x\in \adh(A) = A$. Thus $F \to x$ in $A$.

(2) We need to show that $\adh(A)\subseteq A$. Take $x\in \adh(A)$, meaning there exists a filter $F\to x$ such that $A\in F$ by \ref{adherenceInherenceCharacterisation}. By the ultrafilter lemma, \ref{ultrafilterLemma}, $F$ is contained in an ultrafilter $G$ and $G\to x$. Now $G\vee \upset\{A\}$ is proper (and thus an ultrafilter), so $G|_A$ is an ultrafilter in $A$ by \ref{traceUltrafilters}. Then $G|_A$ converges to a point in $A$, and $G$ converges to the same point in $X$. By Hausdorff this point must be $x$, so $x\in A$.
\end{proof}
\begin{corollary}
If $A$ is compact and $B\subseteq X$ is closed, then $A\cap B$ is compact.
\end{corollary}

\begin{proposition} \label{compactConstructions}
Let $f: \sSet{X,\xi}\to \sSet{Y,\zeta}$ be a continuous function.
\begin{enumerate}
\item If $X$ is compact, then $\im(f)$ is compact.
\item 
\end{enumerate}
\end{proposition}
\begin{proof}
It is enough to show that for any ultrafilter $G$ on $\im(f)$, there exists an ultrafilter $F$ on $X$ such that $f^{\imf\imf}[F] = G$.
\end{proof}

\begin{theorem}[Tychonoff]
Let $\{\sSet{X_i, \xi_i}\}_{i\in I}$ be a family of compact convergence spaces. Then $\prod_{i\in I}X_i$ is compact.
\end{theorem}

\section{Covers}
\begin{definition}
Let $\sSet{X,\xi}$ be a convergence space and $\mathcal{C}\subseteq \powerset(X)$ a family of sets. We call $\mathcal{C}$ 
\begin{itemize}
\item a \udef{cover of convergence filters} if $\mathcal{C}\mesh F$ for each convergent filter $F\in\powerfilters(X)$;
\item a \udef{cover of vicinities} if $\mathcal{C}\mesh \vicinity_\xi(x)$ for all $x\in X$;
\item a \udef{cover of neighbourhoods} if $\mathcal{C}\mesh \neighbourhood_\xi(x)$ for all $x\in X$;
\item an \udef{open cover} if $\mathcal{C} \subseteq \topology_\xi$ and $\bigcup \mathcal{C} = X$;
\item a \udef{cover} of $X$ if $\bigcup \mathcal{C} = X$.
\end{itemize}
\end{definition}

\begin{lemma}
Let $\sSet{X,\xi}$ be a convergence space.
\begin{enumerate}
\item Every cover of vicinities is a cover of convergent filters.
\item If $\xi$ is pretopological, then every cover of convergent filters is a cover of vicinities.
\end{enumerate}
\end{lemma}

\begin{lemma}
Let $\sSet{X,\xi}$ be a convergence space and $\mathcal{C}\subseteq \powerset(X)$ a family of sets. Then $\mathcal{C}$ is a cover of neighbourhoods \textup{if and only if} $\interior^\imf[\mathcal{C}]$ is an open cover.
\end{lemma}
\begin{proof}
First assume $\mathcal{C}$ is a cover of neighbourhoods. Clearly $\mathcal{C} \subseteq \topology_\xi$. For all $x\in X$ there exists $A\in \mathcal{C}$ such that $A\in \neighbourhood(x)$ and thus $\interior(A) \in \neighbourhood(x)$ by \ref{interiorModificationNeighbourhoods}. This means that $x\in \interior(A)$ and thus $X\subseteq \bigcup\interior^\imf[\mathcal{C}]$.

Next assume $\interior^\imf[\mathcal{C}]$ is an open cover. Then for all $x\in X$ there exists $A\in \mathcal{C}$ such that $x\in \interior(A)$, which implies $A\in \neighbourhood(x)$ by \ref{interiorClosureMembership}.
\end{proof}


\begin{proposition}
Let $\sSet{X,\xi}$ be a convergence space. Then the following are equivalent:
\begin{enumerate}
\item $X$ is compact;
\item every cover of convergent filters has a finite subset that covers $X$;
\item every cover of vicinities has a finite subset that covers $X$;
\end{enumerate}
\end{proposition}
\begin{proof}
$(1) \Rightarrow (2)$ Let $\mathcal{C}$ be a cover of convergent filters and assume, towards a contradiction, that $\mathcal{C}$ does not have a finite subset that covers $X$. Then
\[ \setbuilder{X\setminus(C_1\cup \ldots \cup C_n)}{n\in \N; C_1,\ldots, C_n \in \mathcal{C}} \]
is a filter base and does not contain $\emptyset$, so the filter is generates is proper and we can find an ultrafilter $G$ that contains it by the ultrafilter lemma \ref{ultrafilterLemma}.

Now $G$ is convergent, so $\mathcal{C}\mesh G$, i.e.\ there exists a $C\in \mathcal{C}\cap G$. By construction $X\setminus C\in G$, so $C\cap (X\setminus C)\in G$, which means $G$ cannot be an ultrafilter.

$(2) \Rightarrow (3)$ Every cover of vicinities is a cover of convergent filters and thus has a finite subset that covers $X$.

$(3) \Rightarrow (1)$ Assume, towards a contradiction, that $X$ is not compact. Then there exists an ultrafilter $G$ that does not converge. This means that $\vicinity(x) \not\subseteq G$ for all $x\in X$. Let $\mathcal{C}$ consist of one vicinity from $\vicinity(x)\setminus G$ for all $x\in X$. There exists a finite subset $\{C_1, \ldots, C_n\}\subseteq \mathcal{C}$ that covers $X$, so $C_1\cap \ldots \cap C_n = X \in G$. Now $G$ is prime by \ref{booleanMaximalFiltersIdeals} and thus at least one $C_1, \ldots, C_n$ is an element of $G$, which is a contradiction.
\end{proof}
\begin{corollary} \label{topologyCompactnessOpenCover}
Let $\sSet{X,\xi}$ be a topological convergence space. Then $X$ is compact \textup{if and only if} every open cover has a finite subset that covers $X$.
\end{corollary}

\section{Relative compactness}
\begin{definition}
Let $\sSet{X,\xi}$ be a convergence space and $A\subseteq X$ a subset. We call $A$ \udef{relatively compact} if $\adh_\xi$ is compact.
\end{definition}

\section{Local compactness}
\begin{definition}
A convergence space $\sSet{X, \xi}$ is called \udef{locally compact}
if every point has a compact vicinity.
\end{definition}


\chapter{Pretopological and Choquet convergence}
\section{Pretopological convergence}
A convergence space $\sSet{X,\xi}$ is pretopological if it $1$-paved, i.e.\ at each point $x$ there is a filter $G$ that converges to $x$ such that
\[ F\in {\lim}_\xi^{-1}(x) \implies G \leq F. \]

\begin{lemma}
Let $\xi$ be a convergence on a set $X$. Then the following are equivalent:
\begin{enumerate}
\item $\xi$ is a pretopology;
\item $\lim_\xi^{-1}(x)$ has a least element for all $x\in X$;
\item $\vicinity_\xi(x) \in \lim_\xi^{-1}(x)$ for all $x\in X$;
\item $\vicinity_\xi(x) \leq F \implies F\in\lim^{-1}_\xi x$;
\item $x\in\lim_\xi F \iff \vicinity_\xi(x) \leq F$.
\end{enumerate}
\end{lemma}
\begin{proof}
$(1) \Leftrightarrow (2)$. Because $G$ converging to $x$ is equivalent to $G\in \lim_\xi^{-1}(x)$, we see that $G$ is a least element of $\lim_\xi^{-1}(x)$.

$(2) \Leftrightarrow (3)$. The vicinity filter $\vicinity_\xi(x)$ is the infimum of $\lim_\xi^{-1}(x)$, and a set contains a least element iff it contains an infimum.

$(3) \Leftrightarrow (4)$. The set $\lim^{-1}_\xi x$ is upwards closed.

$(4) \Leftrightarrow (5)$. The opposite implication to (4) is immediate because $\vicinity_\xi(x)$ is the infimum of $\lim^{-1}_\xi$.
\end{proof}
Thus in a pretopological convergence space we can determine whether $x$ is in the limit of a filter $F$ by simply comparing $F$ to the vicinity filter of $x$.

\begin{proposition}
Let $\xi$ be a convergence on a set $X$. Then $\xi$ is a pretopological convergence \textup{if and only if} $\lim_\xi: \powerfilters(X)\to \powerset(X)$ is a complete meet-semilattice homomorphism, i.e.\
for all families of filters $\mathcal{F} \subseteq \powerset(\powerfilters(X))$,
\[ \lim_\xi \bigwedge_{F\in \mathcal{F}}F = \bigcap_{F\in\mathcal{F}}\lim_\xi F. \] 
\end{proposition}
\begin{proof}
We have
\begin{align*}
x\in \lim_\xi \bigwedge_{F\in \mathcal{F}}F &\iff \bigwedge_{F\in \mathcal{F}}F \geq \vicinity_\xi(x) \iff \forall F\in\mathcal{F}: F\geq \vicinity_\xi(x) \\
&\iff \forall F\in \mathcal{F}: x\in\lim_\xi F \iff x\in \bigcap_{F\in\mathcal{F}}\lim_\xi F.
\end{align*}
\end{proof}
\begin{corollary}
The set of pretopological convergences on a set $X$ forms a complete meet-subsemilattice of the lattice of convergences.
\end{corollary}
\begin{proof}
Complete meet-semilattice homomorphism form a complete meet-subsemilattice, \ref{semilatticeOfSemilatticeHomomorphisms}.
\end{proof}
\begin{corollary}
There exists a closure operator $S_0$ that maps a convergence $\xi$ to the finest pretopological convergence coarser than $\xi$.
\end{corollary}

\subsection{Pretopological modification}
\begin{definition}
Let $X$ be a set. The closure operator $S_0$ on the lattice of convergences on $X$ is called the \udef{pretopologiser}. For any convergence $\xi$ on $X$, the pretopological convergence $S_0\xi$ is called the \udef{pretopological modification} of $\xi$.
\end{definition}

\begin{proposition}
Let $X$ be a set and $\xi$ a convergence on $X$. Then $S_0\xi$ is defined by
\[ x\in \lim_{S_0\xi} F \iff F \geq \vicinity_\xi(x). \]
\end{proposition}

\begin{proposition}
Let $\sSet{X,\xi}$ and $\sSet{Y,\zeta}$ be two pretopological convergence spaces. Then the following are equivalent for a function $f: X\to Y$:
\begin{enumerate}
\item $f\in \cont(\xi, \zeta)$;
\item for all $x\in X$: $f[\vicinity_\xi(x)] \geq \vicinity_\zeta(f(x))$;
\item for all $A\subseteq X$: $f(\adh_\xi) \subseteq \adh_\zeta f[A]$;
\item for all $B\subseteq Y$: $f^{-1}[\inh_\zeta B] \subseteq \inh_\xi(f^{-1}[B])$.
\end{enumerate}
\end{proposition}

\section{Choquet convergence spaces}
\begin{definition}
A convergence space $\sSet{X, \xi}$ is called a \udef{Choquet space} if
\[ x\in \lim_\xi F \quad\iff\quad \forall U\in \ultrafilters(\powerset(X)): U \geq F \implies x\in\lim_\xi U. \]
\end{definition}

\[ \lim_\xi F = \bigcap_{U\in \ultrafilters(\powerset(X))\land F \subseteq U}\lim_\xi U. \]

A pretopological convergence is completely determined by the vicinity filters. A Choquet convergence is completely determined by the convergence of ultrafilters.

\begin{lemma}
Every pretopological convergence space is a Choquet space.
\end{lemma}

\subsection{Compact Choquet spaces}

\chapter{Related types of spaces}
\section{Cauchy spaces}
\begin{definition}
Let $\sSet{X,\xi}$ be a convergence space. The set
\[ \mathcal{F} \defeq \setbuilder{F\in\powerfilters(X)}{\text{$F$ converges}}. \]
is called the \udef{Cauchy structure} of $\xi$.
\end{definition}
A Cauchy space is a set $X$ together with a set of directed sets that could be the set of convergent directed sets in some convergence space.
\begin{definition}
Let $X$ be a set and $\mathcal{F}$ a family of filters in $\directed(\powerset(X))$ such that
\begin{itemize}
\item $\pfilter{x} \in \mathcal{F}$ for all $x\in X$;
\item $\mathcal{F}$ is upwards closed.
\end{itemize}
We call $\sSet{X, \mathcal{F}}$ a \udef{Cauchy space} and $\mathcal{F}$ a \udef{Cauchy structure}.
\end{definition}
As with convergence spaces we may impose additional axioms.

\subsection{Equivalence and convergence}
TODO finite depth?
\begin{definition}
Let $\sSet{X, \mathcal{F}}$ be a Cauchy space. Define the relation $\sim$ on $(\mathcal{F},\mathcal{F})$ by
\[ F \sim G \qquad \defequiv \qquad F\cap G \in \mathcal{F}. \]
We call the filters $F,G$ equivalent.

We define the \udef{Cauchy convergence} on $X$ by
\[ F \to x \qquad \defequiv \qquad F \sim \pfilter{x}. \]
\end{definition}

\begin{lemma}
The Cauchy convergence is a convergence. In particular it is a Kent space.
\end{lemma}
\begin{proof}
Clearly $\pfilter{x}\cap\pfilter{x} = \pfilter{x} \in \mathcal{F}$, so $\pfilter{x} \to x$.

Let $F\to x$ and $F\subseteq G$. Then $G\in \mathcal{F}$ and $G\cap \pfilter{x} \supseteq F \cap\pfilter{x} \in \mathcal{F}$, so $G \to x$.

The Kent property is immediate.
\end{proof}

\begin{lemma}
Let $\sSet{X,\xi}$ be a convergence space. The Cauchy convergence on the Cauchy space $\sSet{X,\mathcal{F}}$ is the dual Kent closure of $\xi$.
\end{lemma}

\begin{proposition}
Let $\sSet{X, \mathcal{F}}$ be a Cauchy space of finite depth. Then
\begin{enumerate}
\item the relation $\sim$ is an equivalence relation;
\item the Cauchy convergence is of finite depth;
\item the Cauchy convergence is $R_1$.
\end{enumerate}
\end{proposition}

\subsubsection{Cauchy continuity}
TODO: equivalent to continuity of Cauchy convergence?

\subsection{Completeness}
\begin{definition}
Let $\sSet{X, \mathcal{F}}$ be a Cauchy space. We call $X$ \udef{Cauchy complete} (or just \udef{complete}) if every $F\in \mathcal{F}$ converges in the Cauchy convergence.
\end{definition}

\begin{proposition}
\begin{enumerate}
\item Each closed subspace of a complete Cauchy space is complete.
\item A subspace of a complete Hausdorff Cauchy space is complete \textup{if and only if} it is closed.
\item The product of complete Cauchy spaces is complete (TODO products!).
\item Each compact uniform convergence is complete.
\end{enumerate}
\end{proposition}
\begin{proof}
TODO
\end{proof}

\subsubsection{Completion}
\begin{definition}
Let $\sSet{X, \mathcal{F}}$ be a Cauchy space. A \udef{completion} of $\sSet{X, \mathcal{F}}$ is another Cauchy space $\sSet{Y, \mathcal{G}}$ and a function $k: X\to Y$ such that
\begin{itemize}
\item $k: X\to Y$ is a Cauchy embedding;
\item $k[X]$ is dense in $Y$.
\end{itemize}
\end{definition}

\section{Closure spaces}
\url{https://www.researchgate.net/profile/Peter-Stadler-2/publication/239066337_Higher_Separation_Axioms_in_Generalized_Closure_Spaces/links/53d1cf440cf2a7fbb2e95303/Higher-Separation-Axioms-in-Generalized-Closure-Spaces.pdf?origin=publication_detail}

\section{Merotopic and nearness spaces}
\url{https://en.wikipedia.org/wiki/Proximity_space}

\chapter{Uniform convergence}
\section{Uniformities}
\subsection{Operations on filters}
\begin{definition}
Let $X$ be a set and $F,G\in\powerfilters(X^2)$. We define
\[ F^{\transp} \defeq \setbuilder{A^\transp}{A\in F}  \]
and
\[ F;G \defeq \upset \setbuilder{A;B}{A\in F, B\in G}. \]
\end{definition}

We have $F^{\transp} = t^{\imf\imf}[F]$ where $t: X^2 \to X^2: (x,y)\mapsto (y,x)$.

TODO Galois connection $\powerfilters(X^2) \leftrightarrow \powerfilters(X)^2$.

\begin{lemma} \label{meshingFilterComposition}
Let $X$ be a set and $F,G\in\powerfilters(X^2)$. If $F;G \neq \powerset(X^2)$, then $p_2^{\imf\imf}[F]\amesh p_1^{\imf\imf}[G]$.
\end{lemma}
\begin{proof}
Assume $F;G \neq \powerset(X^2)$ and, towards a contradiction, that there exist $A\in F$ and $B\in G$ such that $p_2^{\imf}[A]\perp p_1^{\imf}[B]$. Then $\emptyset = A;B$, so $\emptyset \in F;G$. This means that $F;G = \powerset(X^2)$, which is a contradiction.
\end{proof}
TODO with Galois connection??

\begin{lemma} \label{compositionProductFilters}
Let $X$ be a set and $F,G, G', H\in\powerfilters(X)$. Then
\[ (F\otimes G);(G'\otimes H) = \begin{cases}
F\otimes H & (G\amesh G') \\
\powerset(X ^2) & (\text{otherwise}).
\end{cases}  \]
\end{lemma}
\begin{proof}
TODO
\end{proof}

Note that for all proper filters $G$ we have $G\amesh G$. However $\powerset(X)\cancel\amesh\powerset(X)$, so $\big(F\otimes \powerset(X)\big);\big(\powerset(X)\otimes G\big) = \powerset(X^2)$, independent of $F$ and $G$.

\begin{lemma} \label{componentInclusionsFilterComposition}
Let $X$ be a set and $F,G\in\powerfilters(X^2)$. Then
\begin{enumerate}
\item $p_1^{\imf\imf}[F]\subseteq p_1^{\imf\imf}[F;G]$;
\item $p_2^{\imf\imf}[G]\subseteq p_2^{\imf\imf}[F;G]$.
\end{enumerate}
\end{lemma}
\begin{proof}
(1) This follows immediately from $p_1^{\imf}[A;B] = \setbuilder{p_1((x,y))}{(x,y)\in A \land y\in p_1^{\imf}(B)} \subseteq p_1^{\imf}[A]$.

(2) Similar.
\end{proof}

\begin{lemma} \label{filterCompositionFactorisationLemma}
Let $X$ be a set, $H\in \powerfilters(X^2)\setminus \{\powerset(X^2)\}$ and $F,G\in\powerfilters(X)$. Then
\[ F\otimes G = F\otimes p_1^{\imf\imf}(H); H; p_2^{\imf\imf}(H)\otimes G. \]
\end{lemma}
\begin{proof}
This follows from the fact that for all $A\in F$, $B\in G$ and $C,D,E\in H$,
\[ A\times B = \big(A \times p_1^{\imf}(C)\big);D;\big(p_2^\imf(E)\times B\big). \]
Indeed we have
\begin{align*}
(a,b)\in \big(A \times p_1^{\imf}(C)\big);D;\big(p_2^\imf(E)\times B\big) &\iff \exists c,d: \begin{cases}
(a,c)\in A\times p_1^\imf(C),\\ (c,d)\in D,\\ (d,b)\in p_2^\imf(E)\times B
\end{cases} \\
&\iff \begin{cases}
a\in A, b\in B, \\
\exists c,d: \; (c,d)\in D, c\in p_1^\imf(C), d\in p_2^\imf(E)
\end{cases} \\
&\iff a\in A, b\in B \\
&\iff (a,b)\in A\times B.
\end{align*}
The statement $\exists c,d: \; (c,d)\in D, c\in p_1^\imf(C), d\in p_2^\imf(E)$ is true because we may take $(c,d)\in C\cap D\cap E$, which is not empty because $H$ is proper.
\end{proof}
Set $I\defeq F\otimes p_1^{\imf\imf}(H); H; p_2^{\imf\imf}(H)\otimes G$. The inclusion $\subseteq$ can also be calculated using \ref{componentInclusionsFilterComposition} and \ref{filterFactorisationInequality}:
\[ F\otimes G = p_1^{\imf\imf}\big[F\otimes p_1^{\imf\imf}(H)\big] \otimes p_2^{\imf\imf}\big[p_2^{\imf\imf}(H)\otimes G\big] \subseteq p_1^{\imf\imf}[I]\otimes p_2^{\imf\imf}[I] \subseteq I. \]

\subsection{Uniformities}
\begin{definition}
Let $X$ be a set. Let $\mathcal{U}$ be a set of filters in $\powerfilters(X^2)$. Suppose
\begin{itemize}
\item $\pfilter{x}\otimes \pfilter{x} \in \mathcal{U}$ for all $x\in X$;
\item $\mathcal{U}$ is upwards closed;
\item if $F\in \mathcal{U}$, then $F^\transp\in\mathcal{U}$;
\item if $F, G\in \mathcal{U}$, then $F;G\in\mathcal{U}$.
\end{itemize}
Then we call $\mathcal{U}$ a \udef{uniformity} on $X$ and $\sSet{X,\mathcal{U}}$ a uniform space.

If we drop the first condition, we get a \udef{preuniformity} and a \udef{preuniform space}.

A uniformity is called \udef{factorisable} if for all $H\in\mathcal{U}$, there exist $F,G\in \powerfilters(X^2)$ such that $F\otimes G \in\mathcal{U}$ and $F\otimes G\subseteq H$.
\end{definition}

\begin{lemma}
Let $\sSet{X,\mathcal{U}}$ be a uniform space. Then $\mathcal{U}$ is factorisable \textup{if and only if} $p_1^{\imf\imf}(H)\otimes p_2^{\imf\imf}(H) \in \mathcal{U}$ for all $H\in\mathcal{U}$.
\end{lemma}
\begin{proof}
Assume $\mathcal{U} \ni F\otimes G \subseteq H$. Then
\[ F\otimes G = p_1^{\imf\imf}(F\otimes G)\otimes p_2^{\imf\imf}(F\otimes G) \subseteq p_1^{\imf\imf}(H)\otimes p_2^{\imf\imf}(H) \in \mathcal{U}, \]
where we have used \ref{projectionsOfProductFilter}.
\end{proof}

\subsubsection{Diagonality}
\begin{lemma}
Let $X$ be a set, $\mathcal{F}$ an upwards closed set of filters in $\powerfilters(X)$ and $\mathcal{C}$ a cover of $X$. If
\[ \forall C\in\mathcal{C}: \; \upset\{\Delta_C\} \in \mathcal{F}, \]
where $\Delta_C = \setbuilder{(c,c)}{c\in C}$, then $\pfilter{x}\otimes\pfilter{x}\in\mathcal{F}$ for all $x\in X$.
\end{lemma}
\begin{proof}
Because $\{(x,x)\}\subseteq \Delta_C$ for some $C\in\mathcal{C}$ and $\pfilter{x}\otimes\pfilter{x} = \upset \big\{\{(x,x)\}\big\} \supseteq \upset\{\Delta_C\} \in\mathcal{F}$.
\end{proof}

\begin{definition}
Let $X$ be a set and $\mathcal{S}$ be a set of subsets of $X$. A uniformity $\mathcal{U}$ on $X$ is called \udef{$\mathcal{S}$-diagonal} if $\forall S\in\mathcal{S}: \; \upset\{\Delta_S\} \in \mathcal{U}$.

The uniformity is simply called \udef{diagonal} if $\mathcal{S} = \{X\}$.
\end{definition}
We may call the first requirement in the definition of uniform space ``pointwise diagonality''.

\begin{lemma}
Let $\sSet{X,\mathcal{U}}$ be a uniform space and $\mathcal{S}_1, \mathcal{S}_2$ sets of subsets of $X$. If $\mathcal{S}_1 \subseteq \downset \mathcal{S}_2$, then $\mathcal{S}_2$-diagonality implies $\mathcal{S}_1$-diagonality.
\end{lemma}
\begin{proof}
Assume $\mathcal{S}_1 \subseteq \downset \mathcal{S}_2$ and that $\mathcal{U}$ is $\mathcal{S}_2$-diagonal.

Take $S\in \mathcal{S}_1$. Then there exists an $S'\in \mathcal{S}_2$ such that $S\subseteq S'$. Thus $\upset\{\Delta_{S'}\} \subseteq \upset\{\Delta_{S}\}$ and $\upset\{\Delta_{S'}\}\in \mathcal{U}$. By upwards closure, $\upset\{\Delta_{S}\} \in \mathcal{U}$.
\end{proof}

\subsubsection{Entourages}
\begin{definition}
Let $\sSet{X, \mathcal{U}}$ be a uniform space. Then $\entourage_\mathcal{U} \defeq \bigcap \mathcal{U}$ is called the \udef{entourage filter} of $\mathcal{U}$ and the elements of $\entourage_\mathcal{U}$ are called \udef{entourages}.

We call the uniform space \udef{topological} if $\mathcal{U} = \upset\{\entourage_\mathcal{U}\}$.
\end{definition}

\begin{lemma}
Let $\sSet{X,\mathcal{U}}$ be a uniform space with entourage filter $\entourage$. Then
\begin{enumerate}
\item $\entourage\subseteq \upset\{\Delta\}$, where $\Delta = \setbuilder{(x,x)}{x\in X}$;
\item $\entourage^\transp = \entourage$;
\item $\entourage;\entourage \subseteq \entourage$.
\end{enumerate}
If $\sSet{X,\mathcal{U}}$ is a topological uniform space, then
\begin{enumerate}[1'.] \setcounter{enumi}{2}
\item $\entourage;\entourage = \entourage$.
\end{enumerate}
Any filter in $\powerfilters(X\times X)$ satisfying properties 1., 2. and 3'. is the entourage filter of a topological uniformity.
\end{lemma}
TODO: can we improve 3??
\begin{proof}
(1) We have
\[ \entourage = \bigcap \mathcal{U} \subseteq \bigcap \setbuilder{\pfilter{x}\otimes \pfilter{x}}{x\in X} = \upset\big\{\setbuilder{(x,x)}{x\in X}\big\}. \]

(2) We have
\[ \entourage^\transp = \bigcap \setbuilder{H^\transp}{H\in\mathcal{U}} = \bigcap \setbuilder{H}{H\in\mathcal{U}} = \entourage, \]
because the transpose is bijective and thus its image function is preserved under intersection.

(3) First take $A;B \in\entourage;\entourage$. We claim $A\subseteq A;B$. Indeed take $(a,b)\in A$. By (1), we have $(b,b)\in B$ and so $(a,b)\in A;B$. Thus $\entourage;\entourage \subseteq \entourage$.

(3') In the topological case, we have that $\entourage\in\mathcal{U}$ and thus $\entourage;\entourage\in \mathcal{U}$, so $\entourage \subseteq \entourage;\entourage$.

To show any such filter is an entourage filter, we check the four requirements
\begin{itemize}
\item From (1), we have for all $x\in X$
\[ \entourage \subseteq \upset\{\Delta\} \subseteq \upset\{(x,x)\} = \pfilter{x}\otimes\pfilter{x}. \]
\item Upwards closure is by construction.
\item If $\entourage \subseteq H$, then $\entourage = \entourage^\transp \subseteq H^\transp$.
\item If $\entourage\subseteq G,H$, then $\entourage = \entourage;\entourage \subseteq G;H$.
\end{itemize}
\end{proof}
\begin{corollary}
A topological uniform space is diagonal.
\end{corollary}

\begin{proposition}
If a uniform space is topological and factorisable, then it is trivial.
\end{proposition}
\begin{proof}
Let $\sSet{X,\mathcal{U}}$ be a topological and factorisable uniform space with entourage filter $\entourage_\mathcal{U}$. Then $\entourage_\mathcal{U} \subseteq p_1^{\imf\imf}[\entourage_\mathcal{U}]\otimes p_2^{\imf\imf}[\entourage_\mathcal{U}]$. Now each $A\in \entourage_\mathcal{U}$ contains $\Delta$, so $p_1^\imf[A] = X = p_2^\imf[A]$. Thus $p_1^{\imf\imf}[\entourage_\mathcal{U}]\otimes p_2^{\imf\imf}[\entourage_\mathcal{U}] = \{X^2\}$.

Now $\entourage_\mathcal{U} = \upset\big\{\{X^2\}\big\}$, which is trivial.
\end{proof}

\subsection{Uniform relations}
\begin{definition}
Let $X$ be a sets. A \udef{uniform relation} on $X$ is a relation $R_\mathcal{U}$ on $\powerfilters(X)^2$ such that
\begin{itemize}
\item $\pfilter{x}\mathrel{R_\mathcal{U}}\pfilter{x}$ for all $x\in X$;
\item $F\,R_\mathcal{U}$ is upwards closed for all $F\in \powerfilters(X)$;
\item $R_\mathcal{U}$ is symmetric;
\item $R_\mathcal{U}$ is transitive when restricted to $\powerfilters(X)\setminus\{\powerset(X)\}$.
\end{itemize}
We call the structured set $\sSet{X,R_\mathcal{U}}$ a \udef{uniform relation space}.
\end{definition}

\begin{lemma} \label{uniformRelationRelatedElementLemma}
Let $R_\mathcal{U}$ be a uniform relation on $X$ and $F,G\in\powerfilters(X)$. Assume $F\mathrel{R_\mathcal{U}}\setminus\{\powerset(X)\} \neq \emptyset \neq G\mathrel{R_\mathcal{U}}\setminus\{\powerset(X)\}$. Then
\begin{enumerate}
\item $F\mathrel{R_\mathcal{U}} F$;
\item if $F\amesh G$, then $F\mathrel{R_\mathcal{U}} G$.
\end{enumerate}
\end{lemma}
\begin{proof}
(1) There exists $H\in\powerfilters(X)\setminus\{\powerset(X)\}$ such that $F\mathrel{R_\mathcal{U}} H$. By symmetry we have $H\mathrel{R_\mathcal{U}} F$ and by transitivity $F\mathrel{R_\mathcal{U}} F$.

(2) If $F\amesh G$, then $F\vee G \neq \powerset(X)$ by \ref{joinProperFilter}. From (1) we have $F\mathrel{R_\mathcal{U}} F$ and $G\mathrel{R_\mathcal{U}} G$. By upwards closure, $F\mathrel{R_\mathcal{U}} (F\vee G)$ and $G\mathrel{R_\mathcal{U}} (F\vee G)$. By transitivity and symmetry, $F\mathrel{R_\mathcal{U}} G$.
\end{proof}

\begin{lemma} \label{uniformRelationUpwardsClosure}
Let $R_\mathcal{U}$ be a uniform relation on $X$ and $F,G, F', G'\in\powerfilters(X)$. If $F \subseteq F'$, $G\subseteq G'$ and $F\mathrel{R_\mathcal{U}} G$, then $F'\mathrel{R_\mathcal{U}} G'$.
\end{lemma}
\begin{proof}
By upwards closure, we have $F\mathrel{R_\mathcal{U}} G'$ and  (using symmetry) $F'\mathrel{R_\mathcal{U}} G$. Thus, again using symmetry,
\[ F'\mathrel{R_\mathcal{U}} G \;\text{and}\; G \mathrel{R_\mathcal{U}} F \;\text{and}\; F \mathrel{R_\mathcal{U}} G', \]
so $F'\mathrel{R_\mathcal{U}} G'$ by transitivity.
\end{proof}

\begin{proposition} \label{uniformRelationGaloisConnection}
Let $X$ be a set; $\mathcal{U}$ a uniformity on $X$ and $R$ a uniform relation on $X$. We define a uniformity $\Theta(R)$ and a uniform relation $\Xi(\mathcal{U})$ by
\begin{align*}
\forall H\in \powerfilters(X^2):\qquad H\in\Theta(R) \quad&\defequiv\quad p_1^{\imf\imf}[H]\mathrel{R}p_2^{\imf\imf}[H]; \\
\forall F,G\in\powerfilters(X)\setminus\{\powerset(X)\}:\qquad F\mathrel{\Xi(\mathcal{U})}G \quad&\defequiv\quad F\otimes G\in \mathcal{U}. 
\end{align*}
Then the functions
\begin{align*}
&\Theta: \{\text{uniform relations on $X$}\} \to \{\text{uniformities on $X$}\} \\
&\Xi: \{\text{uniformities on $X$}\} \to \{\text{uniform relations on $X$}\}
\end{align*}
form a Galois connection $(\Theta, \Xi)$. Additionally,
\begin{enumerate}
\item $\im(\Theta)$ is the set of factorisable uniformities on $X$;
\item $\im(\Xi)$ is the set of uniform relations on $X$, i.e.\ $\Xi$ is surjective.
\end{enumerate}
\end{proposition}
\begin{proof}
The prove $\Theta(R)$ is a uniformity, we verify the conditions:
\begin{itemize}
\item From $\pfilter{x}\mathrel{R}\pfilter{x}$, we get $\pfilter{x}\otimes \pfilter{x}\in\Theta(R)$.
\item If $H\in\Theta(R)$ and $H\subseteq H'$, then $p_1^{\imf\imf}[H]\mathrel{R} p_2^{\imf\imf}[H]$, $p_1^{\imf\imf}[H]\subseteq p_1^{\imf\imf}[H']$ and $p_2^{\imf\imf}[H]\subseteq p_2^{\imf\imf}[H']$. Thus, by \ref{uniformRelationUpwardsClosure}, we have $p_1^{\imf\imf}[H']\mathrel{R} p_2^{\imf\imf}[H']$ and so $H'\in\Theta(R)$.
\item Take $H\in \Theta(R)$. Then
\[ p_1^{\imf\imf}[H]\mathrel{R} p_2^{\imf\imf}[H] \iff p_2^{\imf\imf}[H^\transp]\mathrel{R} p_1^{\imf\imf}[H^\transp] \iff p_1^{\imf\imf}[H^\transp]\mathrel{R} p_2^{\imf\imf}[H^\transp] \iff H^\transp\in \Theta(R). \]
\item Take $H_1, H_2\in \Theta(R)$. If $H_1;H_2 = \powerset(X^2)$, then $H_1;H_2\in\Theta(R)$ by upwards closure. If $H_1;H_2 \neq \powerset(X^2)$, then $p_2^{\imf\imf}[H_1]\amesh p_1^{\imf\imf}[H_2]$ by \ref{meshingFilterComposition} and thus $p_2^{\imf\imf}[H_1]\mathrel{R} p_1^{\imf\imf}[H_2]$ by \ref{uniformRelationRelatedElementLemma} (we have that $p_1^{\imf\imf}[H_1]\mathrel{R}p_2^{\imf\imf}[H_1]$, so $p_1^{\imf\imf}[H_1]\mathrel{R}\neq \emptyset$. Similarly $p_1^{\imf\imf}[H_2]\mathrel{R}p_2^{\imf\imf}[H_2]$ and $p_1^{\imf\imf}[H_2]\mathrel{R}\neq \emptyset$). So we have
\[ p_1^{\imf\imf}[H_1]\;\mathrel{R} \;p_2^{\imf\imf}[H_1] \;\mathrel{R} \;p_1^{\imf\imf}[H_2]\; \mathrel{R} \;p_2^{\imf\imf}[H_2]. \]
By \ref{componentInclusionsFilterComposition} and upward closure, we get $p_1^{\imf\imf}[H_1;H_2]\mathrel{R}p_2^{\imf\imf}[H_1;H_2]$, which means $H_1;H_2\in \Theta(R)$.
\end{itemize}

The prove $\Xi(\mathcal{U})$ is a uniform relation, we verify the conditions:
\begin{itemize}
\item From $\pfilter{x}\otimes \pfilter{x}\in\mathcal{U}$, we get $\pfilter{x}\mathrel{\Xi(\mathcal{U})}\pfilter{x}$.
\item Assume $F\mathrel{\Xi(\mathcal{U})}G$ and $G\subseteq G'$. Then $F\otimes G\in \mathcal{U}$ and $F\otimes G\subseteq F\otimes G'$. By upwards closure, $F\otimes G'\in\mathcal{U}$.
\item Symmetry is immediate from $(F\otimes G)^\transp = G\otimes F$.
\item For transitivity, assume $F\mathrel{\Xi(\mathcal{U})}G$, $G\mathrel{\Xi(\mathcal{U})}H$ and $G \neq \powerset(X)$. Then $F\otimes G, G\otimes H\in\mathcal{U}$ and $G \amesh G$ (this would not hold if $G = \powerset(X)$), so $(F\otimes G);(G\otimes H) = F\otimes H$ by \ref{compositionProductFilters}. Thus $F\mathrel{\Xi(\mathcal{U})}H$.
\end{itemize}

To show $(\Theta,\Xi)$ is a Galois connection, we need to prove that $\Theta(R) \subseteq \mathcal{U}$ \textup{if and only if} $R \subseteq \Xi(\mathcal{U})$.

First assume $\Theta(R) \subseteq \mathcal{U}$ and take $F,G\in\powerfilters(X)$ such that $F\mathrel{R} G$. Then $F\otimes G\in \Theta(R)\subseteq\mathcal{U}$ which means that $F\mathrel{\Xi(\mathcal{U})}G$.

Now assume $R \subseteq \Xi(\mathcal{U})$ and take $H\in \Theta(R)$. Then $p_1^{\imf\imf}[H]\mathrel{R}p_2^{\imf\imf}[H]$, which implies $p_1^{\imf\imf}[H]\mathrel{\Xi(\mathcal{U})}p_2^{\imf\imf}[H]$. Thus $p_1^{\imf\imf}[H]\otimes p_2^{\imf\imf}[H]\in \mathcal{U}$. By upwards closure and \ref{filterFactorisationInequality} we have $H\in \mathcal{U}$.

(1) It is clear that $\im(\Theta)$ consists of factorisable uniformities. For the other inclusion, let $\mathcal{U}$ be a factorisable uniformity. It is enough to show that $\mathcal{U} \subseteq \Theta(\Xi(\mathcal{U}))$. Take $H\in \mathcal{U}$. By factorisability $p_1^{\imf\imf}[H]\otimes p_2^{\imf\imf}[H]\in \mathcal{U}$. Then $p_1^{\imf\imf}[H]\mathrel{\Xi(\mathcal{U})} p_2^{\imf\imf}[H]$ and thus $H\in \Theta(\Xi(\mathcal{F}))$.

(2) It is enough to prove that for all uniform relations $R$ we have $\Xi(\Theta(R)) \subseteq R$. Take $F,G\in\powerfilters(X)$. We have
\[ F\mathrel{\Xi(\Theta(R))} G \implies F\otimes G \in\Theta(R) \implies F\mathrel{R}G. \]
\end{proof}


\subsubsection{Uniform convergence}
\begin{definition}
Let $\sSet{X,R}$ be a uniform relation space. Then the \udef{uniform convergence} $\Gamma(R)$ on $X$ is defined by
\[ F \overset{\Gamma(R)}{\longrightarrow} x \qquad\defequiv\qquad F\mathrel{R}\pfilter{x}. \]
We also denote the uniform convergence by $F\overset{u}{\longrightarrow} x$.
\end{definition}
If $\mathcal{U}$ is a uniformity, we write $\Gamma(\mathcal{U})$ to mean $\Gamma(\Xi(\mathcal{U}))$. We have
\[ F \overset{\Gamma(\mathcal{U})}{\longrightarrow} x \qquad\iff\qquad F\otimes \pfilter{x}\in\mathcal{U}. \]

\begin{lemma} \label{associatedUniformConvergence}
A uniform convergence is a convergence. It is also reciprocal ($R_1$).
\end{lemma}
\begin{proof}
Let $\sSet{X,R}$ be a uniform relation space. We have that $\Gamma(R)$ is centered, i.e.\ $\pfilter{x} \overset{\Gamma(R)}{\longrightarrow} x$, because $\pfilter{x}\mathrel{R}\pfilter{x}$.

We have that $\Gamma(R)$ is monotonic by upwards closure of $R$.

We prove reciprocity of $\Gamma(R)$ using point (5). of \ref{R1Conditions}. Assume $F \overset{\Gamma(R)}{\longrightarrow} x$ and $F \overset{\Gamma(R)}{\longrightarrow} y$. Then $F\mathrel{R}\pfilter{x}$ and $F\mathrel{R}\pfilter{y}$, so $\pfilter{x}\mathrel{R}\pfilter{y}$ by symmetry and transitivity. This implies
\[ G \in {\lim}_{\Gamma(R)}^{-1}(x) \iff G\mathrel{R}\pfilter{x}\iff G\mathrel{R}\pfilter{y} \iff G\in {\lim}_{\Gamma(R)}^{-1}(y), \]
and so $\lim_{\Gamma(R)}^{-1}(x) = \lim_{\Gamma(R)}^{-1}(y)$.
\end{proof}

\begin{proposition} \label{topologicalInducedUniformConvergence}
Let $\sSet{X,\mathcal{U}}$ be a topological uniform space. Then the associated convergence $\Gamma(\mathcal{U})$ is topological and
\[ \neighbourhood_{\Gamma(\mathcal{U})}(x) = \upset \setbuilder{V x}{V\in \entourage_\mathcal{U}}. \]
\end{proposition}
\begin{proof}
We first show that $\Gamma(\mathcal{U})$ is pretopological with vicinity filter $\upset \setbuilder{V x}{V\in \entourage_\mathcal{U}}$:

Because $\entourage_\mathcal{U}$ is a topological entourage filter, we have
\begin{align*}
\entourage_\mathcal{U} &= \entourage_\mathcal{U};\entourage_\mathcal{U} \\
&\subseteq \entourage_\mathcal{U}; \pfilter{x}\otimes \pfilter{x} \\
&= \upset \setbuilder{V; \{(x,x)\}}{V\in \entourage_\mathcal{U}} \\
&= \upset \setbuilder{V x\times\{x\}}{V\in \entourage_\mathcal{U}} \\
&= \upset \setbuilder{V x}{V\in \entourage_\mathcal{U}}\otimes \pfilter{x}.
\end{align*}
Thus $\upset \setbuilder{V x}{V\in \entourage_\mathcal{U}} \overset{\Gamma(\mathcal{U})}{\longrightarrow} x$.

Now we show that if $F\overset{\Gamma(\mathcal{U})}{\longrightarrow} x$, then $\upset \setbuilder{V x}{V\in \entourage_\mathcal{U}}\subseteq F$. Indeed we have
\begin{align*}
F\otimes \pfilter{x}\in \mathcal{U} \implies& \entourage_\mathcal{U}\subseteq F\otimes \pfilter{x} \\
\implies& \forall V\in \entourage_\mathcal{U}: \exists A\in F: \; A\times \{x\} \subseteq V \\
\implies& \forall  V\in \entourage_\mathcal{U}: \exists A\in F: \; A \subseteq V x \\
\implies& \upset \setbuilder{V x}{V\in \entourage_\mathcal{U}}\subseteq F.
\end{align*}

Finally to show that $\Gamma(\mathcal{U})$ is topological, we use \ref{pretopologicalSpaceTopological}: Take $Vx\in \upset \setbuilder{V x}{V\in \entourage_\mathcal{U}}$. Then because $\entourage_\mathcal{U} = \entourage_\mathcal{U};\entourage_\mathcal{U}$, we can find $U,U'\in \entourage_\mathcal{U}$ such that $V = U;U'$. Consider $U'x$. For all $y\in U'x$, we have that $zUy \implies zU;U'x \iff zVx$, so $Uy \subseteq Vx$. Thus $Vx \in \upset \setbuilder{V y}{V\in \entourage_\mathcal{U}}$.
\end{proof}


\subsubsection{Uniform Cauchy structure}
\begin{definition}
Let $\sSet{X,R}$ be a uniform relation space and let $\mathcal{F}\subseteq \powerfilters(X)$ be defined by
\[ F\in \mathcal{F} \qquad\defequiv\qquad F\mathrel{R} F. \]
Then $\mathcal{F}$ is called the \udef{induced Cauchy structure} and $\sSet{X,\mathcal{F}}$ is called the \udef{induced Cauchy space}.
\end{definition}
\begin{lemma}
The induced Cauchy space $\sSet{X, \mathcal{F}}$ of a uniform relation space $\sSet{X,R}$ is a Cauchy space.
\end{lemma}
\begin{proof}
We immediately have $\pfilter{x}\in\mathcal{F}$ for all $x\in X$ because $\pfilter{x}\mathrel{R} \pfilter{x}$.

We need to show upwards closure. Let $F\in \mathcal{F}$, meaning $F\mathrel{R} F$, and $F\subseteq G$. By upwards closure of $F\mathrel{R}$, we have $F\mathrel{R} G$. By symmetry we have $G\mathrel{R} F$ and by transitivity $G\mathrel{R} G$, so $G\in\mathcal{F}$.
\end{proof}

\begin{lemma}
Let $\sSet{X, R}$ be a uniform relation space, $F$ a Cauchy filter and $F\mathrel{R} G$. Then $G$ is a Cauchy filter.
\end{lemma}
\begin{proof}
The relationships $F\mathrel{R} F$ and $F\mathrel{R} G$ imply $G\mathrel{R} G$ by transitivity and symmetry.
\end{proof}


\begin{lemma} \label{uniformlyConvergentImpliesCauchy}
Let $\sSet{X,R}$ be a uniform relation space and $F\in\powerfilters(X)$. If $F$ is uniformly convergent (i.e.\ $F\mathrel{R}\pfilter{x}$ for some $x\in X$), then $F$ is a Cauchy filter.
\end{lemma}
\begin{proof}
If $F$ converges uniformly to $x$, then $F\mathrel{R}\pfilter{x}$ and by symmetry also $\pfilter{x}\mathrel{R}F$. By transitivity $F\mathrel{R}F$. 
\end{proof}
If the converse to this lemma holds, then the uniform space is called complete.

\paragraph{Completeness}

\begin{definition}
A uniform relation space $\sSet{X,R}$ is called \udef{complete} if all Cauchy filters converge.
\end{definition}

A uniform space is complete iff for all $F\in\powerfilters(X)$,
\[ F\mathrel{R}F \quad\iff\quad \exists x\in X:\; F\mathrel{R}\pfilter{x}. \]


\begin{lemma}
Let $\sSet{X,R}$ be a uniform relation space. If $\sSet{X,\Gamma(R)}$ is compact, then $\sSet{X,R}$ is complete.
\end{lemma}
\begin{proof}
Assume $\sSet{X,\Gamma(R)}$ is compact and take $F\in\powerfilters(X)$ such that $F\mathrel{R}F$. Then we can find an ultrafilter $F'\supseteq F$ by the ultrafilter lemma \ref{ultrafilterLemma}. By compactness $F'$ converges uniformly, so $F'\mathrel{R}\pfilter{x}$ for some $x\in X$. By upwards closure of $R$, we have $F\mathrel{R} F'$ and thus $F \mathrel{R}\pfilter{x}$. So $F$ is uniformly convergent. Because $F$ was chosen arbitrarily this makes $\sSet{X,R}$ complete.
\end{proof}

\begin{proposition}
Let $\sSet{X,R}$ be a complete uniform space and $A\subseteq X$ a closed subset. Then $A$ is complete.
\end{proposition}
\begin{proof}
Let $F$ be a Cauchy filter and let $F$ be in $A$, i.e.\ such that $A \in F$. Then $F\to x$ for some $x\in X$ by completeness. Thus $A \in F^\mesh \subseteq \vicinity(x)^\mesh$, so $x\in \adh(A) = A$ and $F$ converges in $A$.
\end{proof}

\begin{proposition}
Hausdorff: complete iff closed??
\end{proposition}
\begin{proof}
TODO
\end{proof}


\subsubsection{Induced uniform relation}
\begin{definition}
Let $\sSet{X,\xi}$ be a reciprocal ($R_1$) convergence space. Let $\Phi(\xi)$ be a relation on $\powerfilters(X)$ defined by
\[ F\mathrel{\Phi(\xi)}G \qquad\defequiv\qquad \exists: x\in X: \; \big(F\overset{\xi}{\longrightarrow} x\big) \land \big(G\overset{\xi}{\longrightarrow} x\big) \]
for $F,G\in\powerfilters(X)$.
Then $\Phi(\xi)$ is the \udef{uniform relation associated to} $\xi$.
\end{definition}

\begin{lemma} \label{uniformRelationAssociatedToR1Convergence}
The uniform relation associated to a reciprocal convergence is a uniform relation.
\end{lemma}
\begin{proof}
\begin{itemize}
\item We have $\pfilter{x}\overset{\xi}{\longrightarrow} x$, so $\pfilter{x}\mathrel{\Phi(\xi)}\pfilter{x}$.
\item The set $F\mathrel{\Phi(\xi)}$ is upwards closed by monotonicity of the convergence $\xi$.
\item Symmetry is clear by construction.
\item For transitivity, take proper filters $F,G,H$ such that $F\mathrel{\Phi(\xi)}G$ and $G\mathrel{\Phi(\xi)}H$. Then there exist $x,y\in X$ such that
\[ \big(F\overset{\xi}{\longrightarrow} x\big) \land \big(G\overset{\xi}{\longrightarrow} x\big) \land \big(G\overset{\xi}{\longrightarrow} y\big) \land \big(H\overset{\xi}{\longrightarrow} y\big). \]
Thus $G\in \lim^{-1}_\xi(x) \cap\lim^{-1}_\xi(y)$, so $\lim^{-1}_\xi(x) \mesh\lim^{-1}_\xi(y)$. Using reciprocity, we apply \ref{R1Conditions} to get $\lim^{-1}_\xi(x) = \lim^{-1}_\xi(y)$. Thus $H \overset{\xi}{\longrightarrow} x$, which means that $F\mathrel{\Phi(\xi)}H$.
\end{itemize}
\end{proof}

\begin{proposition} \label{uniformConvergenceGaloisConnection}
Let $X$ be a set. The functions
\begin{align*}
&\Phi: \{\text{$R_1$ convergences on $X$}\} \to \{\text{uniform relations on $X$}\} \\
&\Gamma: \{\text{uniform relations on $X$}\} \to \{\text{$R_1$ convergences on $X$}\}
\end{align*}
form a Galois connection $(\Phi, \Gamma)$. Additionally,
\begin{enumerate}
\item $\im(\Phi)$ is the set of complete uniform relations on $X$;
\item $\im(\Gamma)$ is the set of all reciprocal convergences on $X$; i.e\ $\Gamma$ is surjective.
\end{enumerate}
\end{proposition}
\begin{proof}
The functions $\Phi$ and $\Gamma$ are well-defined by \ref{associatedUniformConvergence} and \ref{uniformRelationAssociatedToR1Convergence}.

To prove the Galois connection, let $R$ be a uniform relation on $X$ and $\xi$ a convergence on $X$. Then we need to prove $\Phi(\xi) \subseteq R$ \textup{if and only if} $\xi \subseteq \Gamma(R)$.

First assume $\Phi(\xi) \subseteq R$. Take $F\overset{\xi}{\longrightarrow} x$. Because also $\pfilter{x}\overset{\xi}{\longrightarrow} x$, we have $F\mathrel{\Phi(\xi)}\pfilter{x}$. By assumption $F\mathrel{R}\pfilter{x}$ and by definition $F\overset{\Gamma(R)}{\longrightarrow} x$.

Now assume $\xi \subseteq \Gamma(R)$. Take $F,G\in \powerfilters(X)$ such that $F\mathrel{\Phi(\xi)}G$. Then $\exists x\in X$ such that $F\overset{\xi}{\longrightarrow} x$ and $G\overset{\xi}{\longrightarrow} x$. By assumption $F\overset{\Gamma(R)}{\longrightarrow} x$ and $G\overset{\Gamma(R)}{\longrightarrow} x$, so by definition $F\mathrel{R}\pfilter{x}$ and $G\mathrel{R}\pfilter{x}$. By symmetry and transitivity $F\mathrel{R}G$.

\begin{enumerate}
\item For the inclusion $\subseteq$: assume $F\mathrel{\Phi(\xi)} F$. Then $F\to x$ for some $x\in X$, so $F\mathrel{\Phi(\xi)} \pfilter{x}$. Thus $F$ converges uniformly to some $x$.

For the other inclusion, $\supseteq$, take a complete uniform relation $R$. It is enough to show that $R \subseteq \Phi(\Gamma(R))$. Take $F,G\in \powerfilters(X)$ such that $F\mathrel{R} G$. By completeness, $F\mathrel{R} \pfilter{x}$ for some $x\in X$. By symmetry and transitivity, $G\mathrel{R} \pfilter{x}$ as well. Thus $F\overset{\Gamma(R)}{\longrightarrow} x$ and $G\overset{\Gamma(R)}{\longrightarrow} x$, so $F\mathrel{\Phi(\Gamma(R))} G$.

\item It is enough to prove $\Gamma(\Phi(\xi)) \subseteq \xi$ for any reciprocal convergence $\xi$. Take $F\overset{\Gamma(\Phi(\xi))}{\longrightarrow} x$. Then $F\mathrel{\Phi(\xi)} \pfilter{x}$ and so $\exists y\in X$ such that $F\overset{\xi}{\longrightarrow} y$ and $\pfilter{x}\overset{\xi}{\longrightarrow} y$. By reciprocity (and because $\pfilter{x}\subseteq \lim_{\xi}^{-1}(x) \cap \lim_{\xi}^{-1}(y)$), we have
\[ F \in {\lim}_{\xi}^{-1}(y) = {\lim}_{\xi}^{-1}(x). \]
So $F\overset{\xi}{\longrightarrow} x$.
\end{enumerate}
\end{proof}

\begin{corollary}
Let $X$ be a set. The functions
\begin{align*}
&\Theta \circ \Phi: \{\text{$R_1$ convergences on $X$}\} \to \{\text{uniformities on $X$}\} \\
&\Gamma\circ \Xi: \{\text{uniformities on $X$}\} \to \{\text{$R_1$ convergences on $X$}\}
\end{align*}
form a Galois connection $(\Theta \circ \Phi, \Gamma\circ \Xi)$. Additionally,
\begin{enumerate}
\item $\im(\Theta \circ \Phi)$ is the set of complete, factorisable uniformities on $X$;
\item $\im(\Gamma\circ \Xi)$ is the set of all reciprocal convergences on $X$.
\end{enumerate}
\end{corollary}
\begin{proof}
The Galois connection follows from \ref{uniformRelationGaloisConnection} and \ref{uniformConvergenceGaloisConnection}.

(1) The inclusion $\subseteq$ is immediate, because a uniformity is called complete if and only if its associated uniform relation is complete.

For the inclusion $\supseteq$, take some complete, factorisable uniformity $\mathcal{U}$. Then $\mathcal{U} = \Theta(R)$ for some uniform relation $R$, by \ref{uniformRelationGaloisConnection}. Now it is enough to note that $R$ is also complete, so there exists a reciprocal convergence $\xi$ such that $\mathcal{U} = \Theta(\Phi(\xi))$ by \ref{uniformConvergenceGaloisConnection}.

(2) Immediate because both $\Gamma$ and $\Xi$ are surjective.
\end{proof}

We can summarise the mappings between uniformlities $\mathcal{U}$, uniform relations $R$ and uniform convergences $\xi$ as follows
\[ \begin{tikzcd}[labels = {font=\large}]
\mathcal{U} \arrow[bend left, rrr, "\substack{F\mathrel{\Xi(\mathcal{U})}G \Leftrightarrow F\otimes G\in \mathcal{U} \vspace{0.1em} \\ \vspace{0.1em} \Xi}"] &&& \arrow[lll, bend left, "\substack{\Theta \vspace{0.8em} \\ H\in\Theta(R) \Leftrightarrow p_1^{\imf\imf}[H]\mathrel{R}p_2^{\imf\imf}[H]}"] R \arrow[rrr, bend left, "\substack{F\overset{\Gamma(R)}{\longrightarrow} x \Leftrightarrow F\mathrel{R}\pfilter{x} \vspace{0.1em} \\ \vspace{0.1em} \Gamma}"] &&& \arrow[lll, bend left, "\substack{\Phi \vspace{0.1em} \\ F\mathrel{\Phi(\xi)}G \Leftrightarrow \exists x: (F\to x)\land (G\to x)}"] \xi
\end{tikzcd} \]

\subsection{Properties of uniform spaces}
\begin{definition}
For any property $\mathbf{P}$ that a convergence space may have, we say a uniform space $\sSet{X,\mathcal{U}}$ has property $\mathbf{P}$ if $\Gamma(\Xi(\mathcal{U}))$ has property $\mathbf{P}$.
\end{definition}

\subsubsection{Compactness}
\begin{proposition} \label{compactUltrafilterFactorisation}
Let $\sSet{X,\mathcal{U}}$ be a compact uniform space. If $H$ is an ultrafilter in $\mathcal{U}$, then 
\begin{enumerate}
\item $p_1^{\imf\imf}[H]\otimes p_2^{\imf\imf}[H] \in\mathcal{U}$;
\item $p_1^{\imf\imf}[H]$ and $p_2^{\imf\imf}[H]$ converge in $\Gamma(\Xi(\mathcal{U}))$;
\item $\lim\big(p_1^{\imf\imf}[H]\big) = \lim\big(p_2^{\imf\imf}[H]\big)$.
\end{enumerate}
\end{proposition}
\begin{proof}
Let $H$ be an ultrafilter in $\mathcal{U}$. Then $p_1^{\imf\imf}[H]$ and $p_2^{\imf\imf}[H]$ are ultrafilters by \ref{projectionsOfUltrafilterAreUltra}. By compactness there exist $x,y\in X$ such that $p_1^{\imf\imf}[H] \overset{\Gamma(\Xi(\mathcal{U}))}{\longrightarrow} x$ and $p_2^{\imf\imf}[H] \overset{\Gamma(\Xi(\mathcal{U}))}{\longrightarrow} y$. Thus $p_1^{\imf\imf}[H]\otimes \pfilter{x}\in\mathcal{U}$ and $p_2^{\imf\imf}[H]\otimes \pfilter{y}\in\mathcal{U}$. 

By \ref{filterCompositionFactorisationLemma} we have
\[ \pfilter{x}\otimes \pfilter{y} = \pfilter{x}\otimes p_1^{\imf\imf}[H]; H; p_2^{\imf\imf}[H] \otimes \pfilter{y} \in \mathcal{U}. \]

Finally
\[ p_1^{\imf\imf}[H] \mathrel{\Xi(\mathcal{U})} \pfilter{x}, \quad \pfilter{x} \mathrel{\Xi(\mathcal{U})} \pfilter{y} \quad\text{and}\quad \pfilter{y}\mathrel{\Xi(\mathcal{U})} p_2^{\imf\imf}[H], \]
so $p_1^{\imf\imf}[H] \mathrel{\Xi(\mathcal{U})} p_2^{\imf\imf}[H]$ by transitivity. Thus $p_1^{\imf\imf}[H]\otimes p_2^{\imf\imf}[H] \in \mathcal{U}$.
\end{proof}

\subsubsection{Depth properties}
\begin{definition}
Let $\sSet{X,\mathcal{U}}$ be a uniform space. We say $\mathcal{U}$ is \udef{uniformly Choquet} if for all $H\in \powerfilters(X^2)$,
\[ H\in\mathcal{U} \qquad\iff\qquad \text{$I \in \mathcal{U}$ for all ultrafilters $I$ such that $H\subseteq I$.} \]
\end{definition}

\begin{lemma}
Let $\sSet{X,\mathcal{U}}$ be a uniform space. If $\mathcal{U}$ is uniformly Choquet, then $\Gamma(\Xi(\mathcal{U}))$ is Choquet.
\end{lemma}
\begin{proof}
????????????????
\end{proof}

\section{Uniform continuity}
\begin{definition}
Let $\sSet{X,\mathcal{U}}$ and $\sSet{Y,\mathcal{V}}$ be uniform spaces. A function $f: X\to Y$ is called \udef{uniformly continuous} if
\[ H\in \mathcal{U} \quad\implies\quad \upset (f, f)^{\imf\imf}[H]\in \mathcal{V}. \]
\end{definition}


\begin{proposition} \label{uniformContinuityEntourages}
Let $\sSet{X,\mathcal{U}}$, $\sSet{Y,\mathcal{V}}$ be uniform spaces and $f: X\to Y$ a function.
\begin{enumerate}
\item If $f$ is uniformly continuous, then $\entourage_\mathcal{V} \subseteq \upset (f, f)^{\imf\imf}[\entourage_\mathcal{U}]$;
\item If $\mathcal{V}$ is topological, then opposite implication also holds.
\end{enumerate}
\end{proposition}
\begin{proof}
(1) By uniform continuity we have
\[ \entourage_\mathcal{V} \subseteq \bigcap\setbuilder{\upset (f, f)^{\imf\imf}[H]}{H\in \mathcal{U}} = \upset (f, f)^{\imf\imf}\left[\bigcap \mathcal{U}\right] = \upset (f, f)^{\imf\imf}\left[\entourage_\mathcal{U}\right]. \]
The first equality follows from \ref{imageFiltersPreservesIntersection}.

(2) Assume $\entourage_\mathcal{V} \subseteq \upset (f, f)^{\imf\imf}[\entourage_\mathcal{U}]$ and take $H\in \mathcal{U}$. Then $\entourage_\mathcal{U}\subseteq H$, so
\[ \entourage_\mathcal{V} \subseteq \upset (f, f)^{\imf\imf}[\entourage_\mathcal{U}] \subseteq \upset (f, f)^{\imf\imf}[H]. \]
Thus $\upset (f, f)^{\imf\imf}[H] \in\mathcal{V}$.
\end{proof}

\begin{proposition}
Let $X,Y$ be sets, $\sSet{X,\mathcal{U}}, \sSet{Y,\mathcal{V}}$ uniform spaces, $\sSet{X,R}, \sSet{Y,S}$ uniform relation spaces and $\sSet{X,\xi}, \sSet{Y,\zeta}$ reciprocal convergence spaces. Let $f: X\to Y$ be a function.
\begin{enumerate}
\item if $f: \sSet{X,\mathcal{U}} \to \sSet{Y,\mathcal{V}}$ is uniformly continuous, then $\big(f: \sSet{X,\Xi(\mathcal{U})} \to \sSet{Y,\Xi(\mathcal{V})}\big)^{\imf\imf}$ is relation preserving;
\item if $\big(f: \sSet{X,R} \to \sSet{Y,S}\big)^{\imf\imf}$ is relation preserving, then $f: \sSet{X,\Gamma(R)} \to \sSet{Y,\Gamma(S)}$ is continuous;
\item if $f: \sSet{X,\xi} \to \sSet{Y,\zeta}$ is continuous, then $\big(f: \sSet{X,\Phi(\xi)} \to \sSet{Y,\Phi(\zeta)}\big)^{\imf\imf}$ is relation preserving;
\item if $\big(f: \sSet{X,R} \to \sSet{Y,S}\big)^{\imf\imf}$ is relation preserving, then $f: \sSet{X,\Theta(R)} \to \sSet{Y,\Theta(S)}$ is uniformly continuous.
\end{enumerate}
\end{proposition}
\begin{proof}
(1) Assume $f$ uniformly continuous and $F\mathrel{\Xi(\mathcal{U})} G$. Then $F\otimes G\in \mathcal{U}$. By uniform continuity $\upset (f, f)^{\imf\imf}[F\otimes G] = f^{\imf\imf}[F] \otimes f^{\imf\imf}[G] \in \mathcal{V}$, so $f^{\imf\imf}[F] \mathrel{\Xi(\mathcal{V})} f^{\imf\imf}[G]$.

(2) Assume $f^{\imf\imf}$ relation preserving and take $F \overset{\Gamma(R)}{\longrightarrow} x$. Then $F\mathrel{R}\pfilter{x}$ and, by relation preservation, $f^{\imf\imf}[F]\mathrel{S}f^{\imf\imf}[\pfilter{x}]$. Now $f^{\imf\imf}[\pfilter{x}] = \pfilter{f}(x)$, so $f^{\imf\imf}[F] \overset{\Gamma(S)}{\longrightarrow} f(x)$.

(3) Assume $f$ is continuous and $F\mathrel{\Phi(\xi)} G$. Then there exists $x\in X$ such that $F\overset{\xi}{\longrightarrow} x$ and $G\overset{\xi}{\longrightarrow} x$. By continuity $f^{\imf\imf}[F]\overset{\zeta}{\longrightarrow} f(x)$ and $f^{\imf\imf}[G]\overset{\zeta}{\longrightarrow} f(x)$, so $f^{\imf\imf}[F] \mathrel{\Phi(\zeta)} f^{\imf\imf}[G]$.

(4) Assume $f^{\imf\imf}$ is relation preserving and take $H\in \Theta(R)$. Then $p_1^{\imf\imf}[H]\mathrel{R}p_2^{\imf\imf}[H]$, so $f^{\imf\imf}\big[p_1^{\imf\imf}[H]\big]\mathrel{S}f^{\imf\imf}\big[p_2^{\imf\imf}[H]\big]$. Now
\[ f^{\imf\imf}\big[p_1^{\imf\imf}[H]\big] = (f\circ p_1)^{\imf\imf}[H] = \big(p_1\circ(f,f) \big)^{\imf\imf}[H] = p_1^{\imf\imf}\big[(f,f)^{\imf\imf}[H]\big]. \]
Similarly $f^{\imf\imf}\big[p_2^{\imf\imf}[H]\big] = p_2^{\imf\imf}\big[(f,f)^{\imf\imf}[H]\big]$. Thus $p_1^{\imf\imf}\big[(f,f)^{\imf\imf}[H]\big]\mathrel{S}p_2^{\imf\imf}\big[(f,f)^{\imf\imf}[H]\big]$, which means that $(f,f)^{\imf\imf}[H]\in \Theta(S)$.
\end{proof}
\begin{corollary}
Let $X,Y$ be sets, $\sSet{X,\mathcal{U}}, \sSet{Y,\mathcal{V}}$ uniform spaces, $\sSet{X,R}, \sSet{Y,S}$ uniform relation spaces and $\sSet{X,\xi}, \sSet{Y,\zeta}$ reciprocal convergence spaces. Let $f: X\to Y$ be a function.
\begin{enumerate}
\item if $\mathcal{U}$ is factorisable and $\big(f: \sSet{X,\Xi(\mathcal{U})} \to \sSet{Y,\Xi(\mathcal{V})}\big)^{\imf\imf}$ is relation preserving, then $f: \sSet{X,\mathcal{U}} \to \sSet{Y,\mathcal{V}}$ is uniformly continuous;
\item if $R$ is complete and $f: \sSet{X,\Gamma(R)} \to \sSet{Y,\Gamma(S)}$ is continuous, then $\big(f: \sSet{X,R} \to \sSet{Y,S}\big)^{\imf\imf}$ is relation preserving.
\end{enumerate}
\end{corollary}
\begin{proof}
(1) From the proposition, we have that $f: \sSet{X,\Theta(\Xi(\mathcal{U}))} \to \sSet{Y,\Theta(\Xi(\mathcal{V}))}$ is uniformly continuous. Because $\mathcal{U}$ is factorisable, $\Theta(\Xi(\mathcal{U})) = \mathcal{U}$. Also $\Theta(\Xi(\mathcal{V})) \subseteq \mathcal{V}$.

(2) From the proposition, we have that $\big(f: \sSet{X,\Phi(\Gamma(R))} \to \sSet{Y,\Phi(\Gamma(S))}\big)^{\imf\imf}$ is relation preserving. Because $R$ is complete, we have $R = \Phi(\Gamma(R))$. Also $\Phi(\Gamma(S)) \subseteq S$.
\end{proof}

\begin{proposition}
Let $X,Y$ be sets, $\sSet{X,\mathcal{U}}, \sSet{Y,\mathcal{V}}$ uniform spaces and $\sSet{X,\xi}, \sSet{Y,\zeta}$ reciprocal convergence spaces. Let $f: X\to Y$ be a function.
\begin{enumerate}
\item if $\mathcal{U}$ is compact, $\mathcal{V}$ is uniformly Choquet and $f: \sSet{X,\Gamma(\Xi(\mathcal{U}))} \to \sSet{Y,\Gamma(\Xi(\mathcal{V}))}$ is continuous, then $f: \sSet{X,\mathcal{U}} \to \sSet{Y,\mathcal{V}}$ is uniformly continuous.
\end{enumerate}
\end{proposition}
\begin{proof}
Take $H\in\mathcal{U}$. Take arbitrary ultrafilter $I\in\powerfilters(Y^2)$ such that $(f,f)^{\imf\imf}[H]\subseteq I$. Then there exists an ultrafilter $J\in \powerfilters(X^2)$ such that $\upset (f,f)^{\imf\imf}[J] = I$ by \ref{preimageFilter}.

By \ref{compactUltrafilterFactorisation}, there exists $x\in X$ such that $p_1^{\imf\imf}[J]\overset{\Gamma(\Xi(\mathcal{U}))}{\longrightarrow} x$ and $p_2^{\imf\imf}[J]\overset{\Gamma(\Xi(\mathcal{U}))}{\longrightarrow} x$. By continuity, symmetry and transitivity,
\[ \mathcal{V} \ni (f\circ p_1)^{\imf\imf}[J]\otimes (f\circ p_2)^{\imf\imf}[J] = p_1^{\imf\imf}\big[(f,f)^{\imf\imf}[J]\big]\otimes p_2^{\imf\imf}\big[(f,f)^{\imf\imf}[J]\big] \subseteq (f,f)^{\imf\imf}[J] = I. \]
As this is true for arbitrary ultrafilter $I$, we have $(f,f)^{\imf\imf}[H]\in\mathcal{V}$ because $\mathcal{V}$ is uniformly Choquet.
\end{proof}

\section{Star refinement}
\begin{definition}
Let $X$ be a set and $U,V\subseteq \powerset(X)$ covers of $X$.
\begin{itemize}
\item The \udef{star} of $A\subseteq X$ w.r.t. $U$ is the set defined by
\[ \operatorname{star}_U(A) \defeq \bigcup\setbuilder{B\in U}{A\mesh B}. \]
\item The \udef{star} of $U$ is
\[ U^* \defeq \setbuilder{\operatorname{star}_U(B)}{B\in U}. \]
\item The cover $U$ \udef{star refines} the cover $V$ if $U^*\subseteq \downset V$. This is denoted $U <^* V$.
\end{itemize}
\end{definition}
Note the different direction compared with the definition of refinement.

\begin{lemma}
Star refinement is a transitive relation on $\powerset^2(X)$.
\end{lemma}
\begin{proof}
Assume $U <^* V$ and $V<^* W$, i.e.\ $U^* \subseteq \downset V$ and $V^* \subseteq \downset W$. The last inclusion implies $\downset V^* \subseteq \downset W$. Now $V\subseteq \downset V^*$ because $A\subseteq \operatorname{star}_V(A)$ for all $A\in V$, which implies $\downset V\subseteq \downset V^*$. Then we have
\[ U^* \subseteq \downset V \subseteq \downset V^* \subseteq \downset W, \]
and, by transitivity of inclusion, $U <^* W$.
\end{proof}

\begin{example}
Star refinement is not reflexive in general. 

Consider the set $X = \{0,1,2\}$ and the cover $U = \big\{\{0,1\}, \{1,2\}\big\}$. Then $U^* = \big\{\{0,1,2\}\big\}$, but $\{0,1,2\}\notin \downset U$.
\end{example}

\begin{lemma}
Let $X$ be a set and $C,D\subseteq \powerset(X)$. If $C\subseteq D$, them $C^*\subseteq \downset D^*$.
\end{lemma}
\begin{proof}
Take $A \in C^*$. Then $A = \operatorname{star}_C(B) = \bigcup \setbuilder{B'\in C}{B'\mesh B}$
\end{proof}

\subsubsection{Tolerance cover}
\begin{definition}
Let $V$ be a tolerance relation on a set $X$. The \udef{(tolerance) cover} of $X$ associated to $V$ is defined as
\[ C_V \defeq \setbuilder{xV}{x\in X}. \]
\end{definition}

\begin{lemma}
Let $V, W$ be tolerance relations on $X$ and $C_V, C_W$ their associated covers.
\end{lemma}

\begin{lemma} \label{toleranceCoverStarRefinement}
Let $V$ be a tolerance relation on $X$ and $C_V$ its associated cover. Then
\begin{enumerate}
\item $\operatorname{star}_{C_V}(xV) = x(V;V;V)$;
\item $C_V <^* C_{V;V;V}$.
\end{enumerate}
\end{lemma}
\begin{proof}
(1) We calculate
\begin{align*}
\operatorname{star}_{C_V}(xV) &= \bigcup \setbuilder{yV}{xV\mesh yV} \\
&= \bigcup \setbuilder{yV}{x(V;V)y} \\
&= x(V;V;V).
\end{align*}

(2) By (1), each $\operatorname{star}_{C_V}(xV)\in C_V^*$ is an element of $C_{V;V;V}$.
\end{proof}

\subsubsection{Uniform covers}
\begin{definition}
Let $X$ be a set. A \udef{uniform cover set} of $X$ is a filter of covers in $\sSet{\powerset^2(X), <^*}$. Members of a uniform cover set are called \udef{uniform covers}.
\end{definition}

\begin{proposition}
Let $X$ be a set.
\begin{enumerate}
\item Let $\entourage$ be a topological entourage filter. Then
\[ \setbuilder{C}{\exists V\in\entourage: C_V\subseteq \downset C} = \setbuilder{C_V}{V\in\entourage}_{<^*} \]
is a uniform cover set.
\item Let $\mathcal{C}$ be a uniform cover set. Then
\[ \upset \setbuilder{\bigcup\setbuilder{A\times A}{A\in C}}{C\in\mathcal{C}} \]
is a unform filter set.
\end{enumerate}
\end{proposition}
\begin{proof}
(1) We first verify the equality. Take $V\in \entourage$. Then there exists $W\in \entourage$ such that $W;W;W\subseteq V$.


upwards closure: Let $C$ be a uniform cover and $C <^* D$. Then 
\end{proof}



\begin{definition}
Let $X$ be a set. A \udef{uniform cover set} of $X$ is a filter of covers in $\sSet{\powerset^2(X), <^*}$.
\end{definition}

\section{Function spaces}
\subsection{Uniform convergences and equicontinuity}
\subsubsection{Diagonal maps and filters}
\begin{definition}
Let $X$ be a set and $A\subseteq X$ be a subset. The \udef{diagonal map} on $X$ is defined as
\[ \Delta: \powerset(X)\to \powerset(X^2): A\mapsto \setbuilder{(a,a)}{a\in A}. \]
We usually write $\Delta_A$ instead of $\Delta(A)$. Similarly if $\mathcal{S}\in\powerset^2(X^2)$, then we write $\Delta^\imf_\mathcal{S}$ instead of $\Delta^\imf(\mathcal{S})$.
\end{definition}

TODO: covers upwards directed.

\subsubsection{Uniform convergences and equicontinuity}
\begin{definition}
Let $X$ be a set and $\sSet{Y, \mathcal{U}}$ a diagonal uniform space. For any set $S$, let $\Delta_S = \setbuilder{(s,s)\in S^2}{s\in S}$ be the diagonal on $S$.
\begin{itemize}
\item If $\mathcal{S}$ is a set of subsets of $X$. Then
\[ \mathcal{U}_\mathcal{S} \defeq \setbuilder{H\in \powerfilters\big((X\to Y)^2\big)}{\forall S\in \mathcal{S}:\; (\evalMap, \evalMap)^{\imf\imf}(H\otimes \{\Delta_S\})^\ttransp \in \mathcal{U}} \]
is called the \udef{$\mathcal{S}$-uniformity}.
\item If $\mathcal{S}$ is a set of subsets of $X$ and $K\subseteq (X\to Y)$. Then $K$ is called \udef{$\mathcal{S}$-equicontinuous} if
\[ \forall S\in \mathcal{S}:\; \upset\big\{(\evalMap, \evalMap)^\imf(\Delta_K\times \Delta_S)\big\}\in \mathcal{U}. \]
\end{itemize}
In particular:
\begin{itemize}
\item $\setbuilder{\{x\}}{x\in X}$-convergence is called \udef{pointwise convergence};
\item $\{X\}$-uniform convergence is just called \udef{uniform convergence};
\item $\setbuilder{\{x\}}{x\in X}$-equicontinuity is called \udef{(pointwise) equicontinuity};
\item $\{X\}$-equicontinuity is called \udef{uniform equicontinuity}.
\end{itemize}
\end{definition}

\[ \mathcal{U}_\mathcal{S} = \left((\evalMap, \evalMap)^{\imf\imf\imf}\big(\{-\}\otimes \{-\}^\imf(\Delta^\imf_\mathcal{S})\big)^\ttransp\right)^\preimf[\powerset(\mathcal{U})]. \]

\begin{lemma}
Let $X$ be a set, $\sSet{Y, \mathcal{U}}$ a preuniform space and $\mathcal{S}$ a set of subsets of $X$. The preuniformity of $\mathcal{S}$-uniform convergence is a preuniformity on $(X\to Y)$.

If $\sSet{Y, \mathcal{U}}$ is diagonal, then the preuniformity of $\mathcal{S}$-uniform convergence is a diagonal uniformity.
\end{lemma}
\begin{proof}
We verify the conditions:
\begin{itemize}
\item Upwards closure: If $G\subseteq H$ for some $G,H\in\powerfilters(X\to Y)$, then
\[ (\evalMap, \evalMap)^{\imf\imf}(G\otimes \{\Delta_S\})^\ttransp \subseteq (\evalMap, \evalMap)^{\imf\imf}(H\otimes \{\Delta_S\})^\ttransp \]
and so we conclude by the upwards closure of $\mathcal{U}$.
\item Closure under transposition: we have
\[ (\evalMap, \evalMap)^{\imf\imf}(H^\transp\otimes \{\Delta_S\})^\ttransp = \Big((\evalMap, \evalMap)^{\imf\imf}(H\otimes \{\Delta_S\})^\ttransp\Big)^\transp, \]
and so we conclude by the closure under transposition of $\mathcal{U}$.
\item Closure under composition: take $G,H\in\mathcal{U}_\mathcal{S}$ and $S\in\mathcal{S}$. It is enough to show, because they form bases, that for all $A\in G, B\in H$,
\[ (\evalMap, \evalMap)^\imf\big(A;B \times \Delta_S\big)^\ttransp \subseteq (\evalMap, \evalMap)^\imf\big(A \times \Delta_S\big)^\ttransp ; (\evalMap, \evalMap)^\imf\big(B \times \Delta_S\big)^\ttransp. \]
We first pick an arbitrary element of the left-hand set. Take $(x,x)\in\Delta_S$ and $(f, g)\in A;B$, so there exists $h:X\to Y$ such that $(f,h)\in A$ and $(h,g)\in B$. Then
\[ (\evalMap, \evalMap)\big((f,g), (x,x)\big)^\ttransp = (f(x), g(x)) \]
and also
\begin{align*}
(f(x), g(x)) &\in \big\{\big(f(x), h(x)\big)\big\};\big\{\big(h(x), g(x)\big)\big\} \\
&= \big\{(\evalMap, \evalMap)\big((f,h), (x,x)\big)^\ttransp\big\};\big\{(\evalMap, \evalMap)\big((h,g), (x,x)\big)^\ttransp\big\} \\
&\subseteq (\evalMap, \evalMap)^\imf\big(A \times \Delta_S\big)^\ttransp ; (\evalMap, \evalMap)^\imf\big(B \times \Delta_S\big)^\ttransp.
\end{align*}
\end{itemize}
Now assume $\sSet{Y, \mathcal{U}}$ diagonal. We verify the diagonality of $\mathcal{U}_\mathcal{S}$.

For any $S\in \mathcal{S}$, we have
\begin{align*}
(\evalMap, \evalMap)^{\imf\imf}\big(\{\Delta_{(X\to Y)}\}\otimes \{\Delta_S\}\big)^\ttransp &= \upset \Big\{ (\evalMap, \evalMap)^{\imf}\big(\Delta_{(X\to Y)}\times \Delta_S\big)^\ttransp \Big\} \\
&= \upset \Big\{ \bigcup_{f\in (X\to Y)}\bigcup_{x\in S}(\evalMap, \evalMap)^{\imf}\big(\{(f,f)\}\times \{(x,x)\}\big)^\ttransp \Big\} \\
&= \upset \Big\{ \bigcup_{f\in (X\to Y)}\bigcup_{x\in S}\big\{(\evalMap, \evalMap)\big((f,x),(f,x)\big)\big\} \Big\} \\
&= \upset \Big\{ \bigcup_{f\in (X\to Y)}\bigcup_{x\in S}\big\{\big(f(x), f(x)\big)\big\} \Big\} \\
&= \upset \Big\{ \Delta_{Y} \Big\} \in \mathcal{U}.
\end{align*}
This means that $\upset \{\Delta_{(X\to Y)}\} \in \mathcal{U}_\mathcal{S}$.
\end{proof}

\begin{lemma}
Take $\mathcal{S}_1, \mathcal{S}_2 \in\powerset^2(X)$. If $\mathcal{S}_1 \preceq \mathcal{S}_2$ (i.e.\ $\mathcal{S}_2$ refines $\mathcal{S}_1$), then $\mathcal{U}_{\mathcal{S}_2} \subseteq \mathcal{U}_{\mathcal{S}_1}$.
\end{lemma}

\begin{proposition}
Let $\sSet{X, \xi}$ be a convergence set, $\sSet{Y, \mathcal{U}}$ a uniform space and $\mathcal{S}$ a set of compact subsets of $X$. Then the $\mathcal{S}$-uniformity is equal to the pointwise uniformity.
\end{proposition}
\begin{proof}

\end{proof}

\begin{lemma}
Let $X$ be a set and $\sSet{Y, \mathcal{U}}$ a uniform space. Then the uniformity of pointwise convergence induces pointwise convergence on $(X\to Y)$.
\end{lemma}
\begin{proof}
Let $H$ be a filter in $\powerfilters(X\to Y)$ and $f\in (X\to Y)$. Set $\mathcal{S} = \setbuilder{\{x\}}{x\in X}$. Then
\begin{align*}
H\overset{\text{pt-wise}}{\longrightarrow} f &\iff \forall x\in X: \; \evalMap^{\imf\imf}(H, \pfilter{x}) \overset{\mathcal{U}}{\longrightarrow} f(x) \\
&\iff \forall x\in X: \; \evalMap^{\imf\imf}(H, \pfilter{x}) \otimes \pfilter{f}(x) \in\mathcal{U} \\
&\iff \forall x\in X: \; \evalMap^{\imf\imf}(H, \pfilter{x}) \otimes \evalMap^{\imf\imf}(\pfilter{f}, \pfilter{x}) \in\mathcal{U} \\
&\iff \forall x\in X: \; (\evalMap, \evalMap)^{\imf\imf}\big((H\otimes \pfilter{x}) \otimes (\pfilter{f}, \pfilter{x})\big) \in\mathcal{U} \\
&\iff \forall x\in X: \; (\evalMap, \evalMap)^{\imf\imf}\big((H\otimes \pfilter{f}) \otimes (\pfilter{x}, \pfilter{x})\big)^\ttransp \in\mathcal{U} \\
&\iff \forall x\in X: \; (\evalMap, \evalMap)^{\imf\imf}\big((H\otimes \pfilter{f}) \otimes \{\Delta_{\{x\}}\}\big)^\ttransp \in\mathcal{U} \\
&\iff H\otimes \pfilter{f} \in \mathcal{U}_\mathcal{S} \\
&\iff H \overset{\mathcal{U}_\mathcal{S}}{\longrightarrow} f.
\end{align*}
\end{proof}

\subsection{Continuous convergence}
\begin{definition}
Let $\sSet{X,\xi}$ and $\sSet{Y,\zeta}$ be convergence spaces. A filter $H\in(X\to Y)$ converges in \udef{continuous preconvergence} to $f\in(X\to Y)$ if
\[ \evalMap^{\imf\imf}(H\otimes F) \overset{\zeta}{\longrightarrow} f(x) \]
for all $F\overset{\xi}{\longrightarrow} x$.

The space $(X\to Y)$ equipped with the continuous convergence is denoted $(X \to Y)_c$.
\end{definition}

The space $\cont(X,Y)$, when viewed as a convergence subspace of $(X\to Y)_c$, is denoted $\cont_c(X,Y)$

\begin{lemma}
Let $\sSet{X,\xi}$ and $\sSet{Y,\zeta}$ be convergence spaces. Then
\begin{enumerate}
\item $(X\to Y)_c$ is a preconvergence space;
\item $\cont_c(X,Y)$ is a convergence space.
\end{enumerate}
\end{lemma}
\begin{proof}
It is clear that the continuous preconvergence is monotonic. We just need to show that it is centered, when restricted to $\cont(X,Y)$. Take $f\in \cont(X,Y)$. The for all $F\to x\in X$ we have $\evalMap^{\imf\imf}(\pfilter{f}\otimes F) = f^{\imf\imf}[F] \to f(x)$ by continuity. Thus $\pfilter{f}\to f$ in $\cont_c(X,Y)$.
\end{proof}

\begin{lemma}
Let $\sSet{X,\xi}$ and $\sSet{Y,\zeta}$ be convergence spaces and $H\in \powerfilters(X\to Y)$. If $\xi$ is pretopological, then $H\overset{(X\to Y)_c}{\longrightarrow} f$ \textup{if and only if}
\[ H\overset{(X\to Y)_c}{\longrightarrow} f\qquad \text{if and only if} \qquad \evalMap^{\imf\imf}\big(H\otimes \vicinity(x)\big) \overset{\zeta}{\longrightarrow} f(x). \]
\end{lemma}
\begin{proof}
TODO
\end{proof}

\begin{example}
There exists a sequence of continuous functions that converges pointwise to a continuous function, but does not converge in continuous convergence.

Consider $\seq{f_n} \subseteq (\R\to\R)$ defined by
\[ f_n: \R\to\R: x\mapsto \begin{cases}
n\cdot x & (x\in \interval{0,n^{-1}}) \\
2 - nx & (x\in \interval{n^{-1}, 2n^{-1}}) \\
0 & (\text{otherwise}).
\end{cases} \]
TODO image.
\end{example}

\begin{lemma}
Let $\sSet{X,\xi}$ and $\sSet{Y,\zeta}$ be convergence spaces. The continuous preconvergence on $(X\to Y)$ is the coarsest preconvergence such that the evaluation map $\evalMap: (X\to Y)\times X \to Y$ is continuous.
\end{lemma}
\begin{proof}
We first show that the continuous convergence makes the evaluation map continuous. A filter $G\in\powerfilters\big((X\to Y)_c\times X\big)$ converges iff $p_1[G]\to f$ and $p_2[G]\to x$ converge. By definition of continuous convergence $\evalMap^{\imf\imf}(p_1[G] \otimes p_2[G]) \to f(x)$ and thus $\evalMap^{\imf\imf}(p_1[G] \otimes p_2[G]) \subseteq \evalMap^{\imf\imf}(G) \to f(x)$.

Now assume there is some other preconvergence $\sigma$ on $(X\to Y)$ that makes the evaluation map continuous. Take $H \overset{\sigma}{\longrightarrow} f$. Then for all $F\overset{\xi}{\longrightarrow}x$ we have $H\otimes F\overset{\sigma\otimes \xi}{\longrightarrow} (f,x)$, so $\evalMap^{\imf\imf}(H\otimes F) \overset{\zeta}{\longrightarrow} f(x)$ by continuity of the evaluation map. Thus $H\overset{(X,Y)_c}{\longrightarrow} f$.
\end{proof}

\begin{lemma}
Let $\sSet{X,\xi}$ and $\sSet{Y,\zeta}$ be convergence spaces, then the continuous convergence is finer than pointwise convergence on $(X\to Y)$.
\end{lemma}
\begin{proof}
Assume $H\to f$ in continuous convergence. Then
\[ \forall x\in X: \; \evalMap^{\imf\imf}(H\otimes \pfilter{x}) \to f(x), \]
because $\pfilter{x}\to x$. This is exactly the requirement for $H\to f$ in pointwise convergence.
\end{proof}


\begin{lemma}
Let $\sSet{X,\xi}$ and $\sSet{Y,\zeta}$ be convergence spaces. If $Y$ is of finite depth, then so is $(X\to Y)_c$.
\end{lemma}
\begin{proof}
Let $G, H\overset{(X\to Y)_c}{\longrightarrow} f$. Then for all $F\to x\in X$, we have
\begin{align*}
\evalMap^{\imf\imf}\big((G\cap H)\otimes F\big) &= \evalMap^{\imf\imf}\big((G \otimes F) \cap (H \otimes F)\big) \\
&= \evalMap^{\imf\imf}(G \otimes F) \cap \evalMap^{\imf\imf}(H \otimes F) \; \overset{\zeta}{\longrightarrow} f(x),
\end{align*}
using \ref{intersectionProductFilters}, \ref{imageFiltersPreservesIntersection} and the finite depth of $Y$.
\end{proof}
Note that this does not imply the Kent property, because $(X\to Y)_c$ is only a preconvergence space. The space $\cont_C(X,Y)$ does have the Kent property if $Y$ is of finite depth.

\begin{lemma}
Let $\sSet{X,\xi}$, $\sSet{Y,\zeta}$ be convergence spaces such that $(X\to Y)_c$ is a Kent space. Then all limit functions are continuous.
\end{lemma}
\begin{proof}
Take $H\in\powerfilters\big((X\to Y)\big)$ such that $H\overset{(X\to Y)_c}{\longrightarrow} f$. We have, for all $F\to x\in X$,
\begin{align*}
\evalMap^{\imf\imf}\big((H\cap \pfilter{f})\otimes F\big) &= \evalMap^{\imf\imf}\big((H \otimes F) \cap (\pfilter{f} \otimes F)\big) \\
&= \evalMap^{\imf\imf}(H \otimes F) \cap \evalMap^{\imf\imf}(\pfilter{f} \otimes F),
\end{align*} 
using \ref{intersectionProductFilters} and \ref{imageFiltersPreservesIntersection}. By the Kent property this converges to $f(x)$. And thus by monotonicity of the convergence,
\[ \evalMap^{\imf\imf}(\pfilter{f} \otimes F) = f^{\imf\imf}[F] \overset{\zeta}{\longrightarrow} f(x). \]
Thus $f$ is continuous.
\end{proof}
The hypothesis of $(X\to Y)_c$ being a Kent space is necessary.
\begin{example}
There exist non-continuous limit points in spaces of continuous convergence.

Set $X = \{a,b\}$ and $Y = \{1,2\}$.

Let $X$ have the convergences $\begin{cases}
\pfilter{a} \to a \\
\pfilter{b} \to b \\
\pfilter{b} \to a
\end{cases}$ and let $Y$ have the convergences $\begin{cases}
\pfilter{1} \to 1 \\
\pfilter{1} \to 2 \\
\pfilter{2} \to 2
\end{cases}$.
Consider the functions $f: X\to Y: \begin{cases}
a\mapsto 1 \\ b\mapsto 2
\end{cases}$ and $g: X\to Y: \begin{cases}
a\mapsto 1 \\ b\mapsto 1
\end{cases}$.

Then it is straightforward to verify that $\pfilter{g} \overset{(X\to Y)_c}{\longrightarrow} f$, but $f$ is not continuous. Indeed $\pfilter{b} \to a$, but $f^{\imf\imf}[\pfilter{b}] = \pfilter{2} \not\to f(a) = 1$.
\end{example}

\begin{proposition}[Universal property of the continuous convergence structure] \label{universalPropertyContinuousConvergence}
Let $\sSet{X, \xi}, \sSet{Y,\sigma}$ and $\sSet{Z,\zeta}$ be convergences spaces. A function $h: X\to (Y \to Z)_c$ if continuous \textup{if and only if} $\curry_1^{-1}(h): (X\times Y)\to Z$ is continuous.
\end{proposition}
\begin{proof}
First assume $h$ is continuous. Then $\curry_1^{-1}(h) = \evalMap \circ (h,\id_Y)$ and thus $\curry_1^{-1}(h)$ is continuous. (TODO ref).

Now assume $\curry_1^{-1}(h): (X\times Y)\to Z$ is continuous and take $F\to x\in X$. Now for all $G\to y\in Y$ we have that
\begin{align*}
\evalMap^{\imf\imf}\Big(h^{\imf\imf}[F]\otimes G\Big) &= \big(\evalMap \circ (h,\id_Y)\big)^{\imf\imf}(F\otimes G) \\
&= \curry_1^{-1}(h)^{\imf\imf}(F\otimes G) \\
&\to \curry_1^{-1}(h)(x,y) = h(x)(y)
\end{align*}
by continuity of $\curry_1^{-1}(h)$. Thus $h^{\imf\imf}[F]$ converges to $h(x)$ by definition of continuous convergence.
\end{proof}

\begin{proposition}
Let $\sSet{X, \xi}, \sSet{Y,\sigma}$ and $\sSet{Z,\zeta}$ be convergences spaces. Then
\[ \curry_1: (X\times Y \to Z)_c \to (X\to (Y\to Z)_c)_c \]
is a homeomorphism.
\end{proposition}
\begin{proof}
TODO
\end{proof}
\begin{corollary}
The category of convergence spaces is cartesian closed.
\end{corollary}

\begin{proposition}
$\cont_c(X,Y)$ is closed subset of $(X\to Y)_c$??
\end{proposition}

\begin{proposition}
Let $\sSet{X, \xi}, \sSet{Y,\sigma}$ and $\sSet{Z,\zeta}$ be convergences spaces. The composition operation
\[ \circ: (Y\to Z)_c \times (X\to Y)_c \to (X\to Z)_c \]
is continuous.
\end{proposition}
\begin{proof}
By \ref{universalPropertyContinuousConvergence}, this is equivalent to the continuity of $\curry_1^{-1}(\circ): (Y\to Z)_c \times (X\to Y)_c \times X \to Z$, which follows from the commutativity of the following diagram:
\[\begin{tikzcd}
(Y\to Z)_c \times (X\to Y)_c \times X \ar[d, swap, "\id_{(Y\to Z)}\times \evalMap"] \ar[dr, "\curry_1^{-1}(\circ)"] & \\
(Y\to Z)_c\times Y \ar[r, "\evalMap"] & Z.
\end{tikzcd} \]
\end{proof}


\subsubsection{Continuous uniform convergence}
\begin{definition}
Let $\sSet{X, \xi}$ be a convergence space and $\sSet{Y, \mathcal{U}}$ a uniform space. The \udef{uniformity of continuous convergence} is defined by
\[ \mathcal{U}_c \defeq \setbuilder{H\in \powerfilters\big((X\to Y)^2\big)}{\forall F\overset{\xi}{\longrightarrow} x:\; (\evalMap, \evalMap)^{\imf\imf}\Big(H\otimes (F\otimes F)\Big)^\ttransp \in \mathcal{U}}. \]
\end{definition}

\begin{lemma}
Let $\sSet{X, \xi}$ be a convergence space and $\sSet{Y, \mathcal{U}}$ a uniform space. Then the uniformity of continuous convergence
\begin{enumerate}
\item is a preuniformity on $(X\to Y)$;
\item a uniformity on $\cont(X,Y)$.
\end{enumerate}
\end{lemma}
\begin{proof}
Let $\sSet{X,\xi}$ be a convergence space and $\sSet{Y,\mathcal{U}}$ a uniform space. We verify the three conditions
\begin{itemize}
\item Take $f\in \cont(X,Y)$. We need to show that $f^{\imf\imf}[F]\mathrel{\mathcal{U}}f^{\imf\imf}[F]$ for all $F\overset{\xi}{\longrightarrow}x$. Because $f$ is continuous, we have that $f^{\imf\imf}[F]$ converges and the statement follows from \ref{uniformlyConvergentImpliesCauchy}.
\item Take $H,K_1,K_2$ in $\powerfilters(\cont(X,Y))$. Assume $H\mathrel{\mathcal{U}_c}K_1$ and $K_1\subseteq K_2$. Now $\evalMap^{\imf\imf}(K_1\otimes F) \subseteq \evalMap^{\imf\imf}(K_2\otimes F)$ for all $F\overset{\xi}{\longrightarrow}x$. Then upwards closure of $\mathcal{U}_c$ follows from the upwards closure of $\mathcal{U}$.
\item The symmetry and transitivity of $\mathcal{U}_c$ follow from the symmetry and transitivity of $\mathcal{U}$.
\end{itemize}
\end{proof}

\begin{lemma}
Let $\sSet{X, \xi}$ be a convergence space and $\sSet{Y, \mathcal{U}}$ a uniform space. If $\mathcal{U}$ is factorisable, then
\[ \mathcal{U}_c = (\Theta\circ\Phi)\Big(\big(\sSet{X,\xi} \to \sSet{Y, (\Xi\circ\Gamma)(\mathcal{U})}\big)_c\Big). \]
\end{lemma}

\begin{lemma}
Let $\sSet{X, \xi}$ be a convergence space and $\sSet{Y, \mathcal{U}}$ a uniform space. Then the uniformity of continuous convergence induces continous convergence on $\cont(X,Y)$.
\end{lemma}
\begin{proof}
Let $H$ be a filter in $\powerfilters\big(\cont(X,Y)\big)$ and $f\in \cont(X, Y)$. Then
\begin{align*}
H\overset{(X\to Y)_c}{\longrightarrow} f &\iff \forall F\overset{\xi}{\longrightarrow}x: \; \evalMap^{\imf\imf}(H\otimes F) \overset{\mathcal{U}}{\longrightarrow} f(x) \\
&\iff \forall F\overset{\xi}{\longrightarrow}x: \; \evalMap^{\imf\imf}(H\otimes F)\otimes \pfilter{f}(x) \in\mathcal{U} \\
&\iff \forall F\overset{\xi}{\longrightarrow}x: \; \evalMap^{\imf\imf}(H\otimes F)\otimes f^{\imf\imf}(F) \in\mathcal{U} \\
&\iff \forall F\overset{\xi}{\longrightarrow}x: \; \evalMap^{\imf\imf}(H\otimes F)\otimes \evalMap^{\imf\imf}(\pfilter{f}\otimes F) \in\mathcal{U} \\
&\iff \forall F\overset{\xi}{\longrightarrow}x: \; (\evalMap, \evalMap)^{\imf\imf}\big((H\otimes F)\otimes (\pfilter{f}\otimes F)\big) \in\mathcal{U} \\
&\iff \forall F\overset{\xi}{\longrightarrow}x: \; (\evalMap, \evalMap)^{\imf\imf}\big((H\otimes \pfilter{f})\otimes (F \otimes F)\big)^\ttransp \in\mathcal{U} \\
&\iff H\otimes \pfilter{f}\in \mathcal{U}_c \\
&\iff H\overset{\mathcal{U}_c}{\longrightarrow} f.
\end{align*}
\end{proof}

\begin{proposition}
Let $\sSet{X, \xi}$ be a convergence space, $\sSet{Y, \mathcal{U}}$ a uniform space and $\mathcal{S}$ a set of compact subsets of $X$. The continuous convergence uniformity is stronger than the $\mathcal{S}$-uniformity on $\cont(X,Y)$.
\end{proposition}
\begin{proof}

\end{proof}

\subsection{The Arzelà-Ascoli theorem}


\section{Metric spaces}
\begin{definition}
Let $X$ be a set. A \udef{metric} on $X$ is a function $d: X\times X\to \R$ satisfying, for all $x,y,z\in X$,
\begin{itemize}
\item \emph{positivity}: $d(x,y) \geq 0$;
\item \emph{definiteness}: $d(x,y) = 0$ \textup{if and only if} $x=y$;
\item \emph{symmetry}: $d(x,y) = d(y,x)$;
\item the \emph{triangle inequality}: $d(x,z) \leq d(x,y) + d(y,z)$.
\end{itemize}
The structured set $\sSet{X,d}$ is called a \udef{metric space}.
\end{definition}

\subsection{Uniform structures on a metric space}
\begin{proposition} \label{metricUniformities}
Let $\sSet{X,d}$ be a metric space and $\sigma$ an order-regular convergence on $\R^{\geq 0}$ such that addition is continuous. Then the relation $\mathcal{U}^\sigma$ on $\powerfilters(X)$ defined by
\[ F\mathrel{\mathcal{U}^\sigma}G \qquad\defequiv\qquad d^{\imf\imf}\big[F\otimes G\big] \overset{\sigma}{\longrightarrow} 0 \]
is a uniformity on $X$.
\end{proposition}
\begin{proof}
We have that $F\mathrel{\mathcal{U}^N}G$ iff $F\otimes G$ converges to a point in $d^\preimf\{0\}$ in the initial convergence $\psi$ of $d$. Now $d^\preimf\{0\} = \setbuilder{(x,x)}{x\in X}$. In order to apply \ref{uniformityFromConvergenceOnSquare} TODO, we just need to verify the two points. 
\begin{itemize}
\item Let $H\in \powerfilters{X\times X}$. Then, by symmetry of the metric,
\[ H\overset{\psi}{\longrightarrow} (x,x) \iff d^{\imf\imf}[H] \to 0 \iff d^{\imf\imf}[H^\transp] \to 0 \iff H^\transp\overset{\psi}{\longrightarrow} (x,x). \]
\item Assume $H_1\overset{\psi}{\longrightarrow} (x,x)$ and $H_2\overset{\psi}{\longrightarrow} (x,x)$. Then $d^{\imf\imf}[H_1] \overset{\sigma}{\longrightarrow} 0$ and $d^{\imf\imf}[H_2] \overset{\sigma}{\longrightarrow} 0$. By continuity of addition, $d^{\imf\imf}[H_1]+d^{\imf\imf}[H_2] \overset{\sigma}{\longrightarrow} 0$. Now $\pfilter{0} \leq d^{\imf\imf}[H_1;H_2] \leq d^{\imf\imf}[H_1]+d^{\imf\imf}[H_2]$, then $H_1;H_2 \overset{\psi}{\longrightarrow} (x, x)$. TODO squeeze theorem, order closure.
\end{itemize}
\end{proof}

\subsubsection{Standard metric uniformity and convergence}
\begin{definition}
Let $\sSet{X,d}$ be a metric space. The uniformity given by \ref{metricUniformities} when $\R$ is equipped with the standard convergence is called the \udef{(standard) metric uniformity} $\mathcal{U}_d$.

We also define, for $\epsilon >0$ and $x\in X$,
\begin{itemize}
\item The \udef{$\epsilon$-entourage} as the set
\[ V_\epsilon \defeq \setbuilder{(y,z)\in X^2}{d(y,z)\leq \epsilon}. \]
\item The \udef{$\epsilon$-ball centered at $x$} as the set
\[ \ball_d(x,\epsilon) \defeq \setbuilder{y\in X}{d(x,y)< \epsilon} \]
of all points $y$ whose distance to $x$ is less than $\epsilon$.
\item The \udef{closed $\epsilon$-ball centered at $x$} as the set
\[ \cball_d(x,\epsilon) \defeq \setbuilder{y\in X}{d(x,y)\leq \epsilon} \]
of all points $y$ whose distance to $x$ is less than or equal to $\epsilon$.
\item The \udef{$\epsilon$-sphere centered at $x$} as the set
\[ \sphere_d(x,\epsilon) \defeq \setbuilder{y\in X}{d(x,y) = \epsilon}. \]
\end{itemize}
\end{definition}

\begin{lemma}
Let $\sSet{X,d}$ be a metric space. Then
\begin{enumerate}
\item $\cball_d(x,\epsilon) = V_\epsilon x = xV_\epsilon$;
\item $\sphere_d(x,\epsilon) = \cball_d(x,\epsilon)\setminus \ball_d(x,\epsilon)$.
\end{enumerate}
\end{lemma}

\begin{lemma}
Let $\sSet{X,d}$ be a metric space. The metric uniformity is topological and a base of the entourage filter is given by $\{V_\epsilon\}_{\epsilon>0}$.
\end{lemma}
\begin{corollary}
Let $\sSet{X, d_X}$ and $\sSet{Y, d_Y}$ be metric spaces and $f:X\to Y$ a function. Then $f$ is uniformly continuous \textup{if and only if}
\[ \forall \epsilon >0: \exists \delta >0: \forall x,y\in X: \quad d_X(x,y) \leq \delta \implies d_Y(f(x), f(y)) \leq \epsilon. \]
\end{corollary}
\begin{proof}
By \ref{uniformContinuityEntourages} we have that $f$ is uniformly continuous \textup{if and only if}
\begin{align*}
\entourage_Y \subseteq \upset(f\times f)^{\imf\imf}[\entourage_X] &\iff \forall V_\epsilon \in \entourage_Y: \exists V_\delta \in \entourage_X: \; (f\times f)^{\imf\imf}[V_\delta] \subseteq V_\epsilon \\
&\iff \forall \epsilon>0: \exists \delta>0: \forall (y,y')\in (f\times f)^{\imf\imf}[V_\delta]: \; (y,y')\in V_\epsilon \\
&\iff \forall \epsilon>0: \exists \delta>0: \forall (x,x')\in V_\delta:\; (f(x),f(x'))\in V_\epsilon \\
&\iff \forall \epsilon>0: \exists \delta>0: \forall x,x'\in X: \; d_X(x,x')\leq \delta \implies d_Y(f(x),f(x'))\leq \epsilon.
\end{align*}
\end{proof}

\begin{lemma}
Let $\sSet{X,d}$ be a metric space and $\seq{x_n}$ a sequence in $X$. Then $\seq{x_n}$ is a Cauchy sequence \textup{if and only if}
\[ \forall \epsilon >0: \exists N\in\N: \forall m,n \geq N: \quad d(x_m, x_n)\leq \epsilon. \]
\end{lemma}
\begin{proof}
We have that $\TailsFilter\seq{x_n}$ is a Cauchy filter \textup{if and only if} $\entourage_d \subseteq \TailsFilter\seq{x_n}\otimes \TailsFilter\seq{x_n}$. This is true iff $\forall \epsilon>0$:
\begin{align*}
V_\epsilon \in \TailsFilter\seq{x_n}\otimes \TailsFilter\seq{x_n} &\iff \exists M,N\in\N: \setbuilder{x_n}{n\geq N}\times\setbuilder{x_n}{n\geq M} \subseteq V_\epsilon \\
&\iff \exists N\in\N: \setbuilder{x_n}{n\geq N}\times\setbuilder{x_n}{n\geq N} \subseteq V_\epsilon \\
&\iff \exists N\in\N: \forall m,n \geq N: d(x_m, x_n)\leq \epsilon.
\end{align*}
\end{proof}

\begin{definition}
Let $\sSet{X,d}$ be a metric space. The convergence $\Gamma(\mathcal{U}_d)$ induced by the metric uniformity $\mathcal{U}_d$ is called the \udef{metric convergence}.

If $F\in\powerfilters(X)$ converges to $x\in X$ in the metric convergence, we write $F\overset{d}{\longrightarrow} x$.
\end{definition}

\begin{lemma}
Let $\sSet{X,d}$ be a metric space. The metric convergence is topological and
\[ \neighbourhood_d(x) = \upset\setbuilder{\cball_d(x,\epsilon)}{\epsilon >0}. \]
\end{lemma}
\begin{proof}
The metric convergence is topological by \ref{topologicalInducedUniformConvergence}, which also gives the form of the neighbourhood filter. 
\end{proof}
\begin{corollary}
Let $\sSet{X,d}$ be a metric space and $\seq{x_n}$ a sequence in $X$. Then $\seq{x_n}$ converges to $x\in X$ \textup{if and only if}
\[ \forall \epsilon>0: \exists N\in\N: \forall n\geq N: \quad d(x_n,x)\leq \epsilon. \]
\end{corollary}

\subsubsection{Uniform convergence}
\begin{proposition}
Let $X$ be a set, $\sSet{Y,d}$ a metric space and $\seq{f_n}$ a sequence in $(X\to Y)$. The following are equivalent:
\begin{enumerate}
\item  $\seq{f_n}$ converges uniformly to $f: X\to Y$;
\item $\forall \epsilon > 0: \exists N\in \N: \forall n \geq N: \forall x\in X:  d(f_n(x), f(x)) \leq \epsilon$;
\item $\forall \epsilon > 0: \exists N\in \N: \forall n \geq N: \sup_{x\in X} d(f_n(x), f(x)) \leq \epsilon$.
\end{enumerate}
\end{proposition}
\begin{proof}
The equivalence of (2) and (3) is clear. We prove $(1) \Leftrightarrow (2)$.
\begin{align*}
\seq{f_n} \overset{\text{unif.}}{\longrightarrow} f &\iff \TailsFilter\seq{f_n} \otimes \pfilter{f} \in \mathcal{U}_{(X\to Y)} \\
&\iff (\evalMap, \evalMap)^{\imf\imf}\big((\TailsFilter\seq{f_n} \otimes \pfilter{f})\otimes \{\Delta_X\}\big)^\ttransp \in \mathcal{U}_d \\
&\iff \{V_\epsilon\}_\epsilon \subseteq (\evalMap, \evalMap)^{\imf\imf}\big((\TailsFilter\seq{f_n} \otimes \pfilter{f})\otimes \{\Delta_X\}\big)^\ttransp \\
&\iff \forall \epsilon>0: \exists N\in \N: \; (\evalMap, \evalMap)^{\imf}\big((\setbuilder{f_n}{n\geq N} \times \{f\})\times \Delta_X\big)^\ttransp \subseteq V_\epsilon \\
&\iff \forall \epsilon>0: \exists N\in \N: \; \bigcup_{n\geq N}\bigcup_{x\in X}(\evalMap, \evalMap)^{\imf}\big((\{f_n\} \times \{f\})\times \{(x,x)\}\big)^\ttransp \subseteq V_\epsilon \\
&\iff \forall \epsilon>0: \exists N\in \N: \forall n\geq N: \forall x\in X: \;(\evalMap, \evalMap)^{\imf}\big((\{f_n\} \times \{f\})\times \{(x,x)\}\big)^\ttransp \in V_\epsilon \\
&\iff \forall \epsilon>0: \exists N\in \N: \forall n\geq N: \forall x\in X: \; (\evalMap, \evalMap)^{\imf}\big(\{((f_n, f), (x,x))^\ttransp\}\big) \in V_\epsilon \\
&\iff \forall \epsilon>0: \exists N\in \N: \forall n\geq N: \forall x\in X: \; (\evalMap, \evalMap)\big((f_n,x),(f,x)\big) \in V_\epsilon \\
&\iff \forall \epsilon>0: \exists N\in \N: \forall n\geq N: \forall x\in X: \; (f_n(x), f(x)) \in V_\epsilon \\
&\iff \forall \epsilon>0: \exists N\in \N: \forall n\geq N: \forall x\in X: \; d(f_n(x), f(x)) \leq \epsilon.
\end{align*}
\end{proof}
\begin{corollary}

\end{corollary}

\subsubsection{Lipschitz and Hölder diagonal spaces}
\begin{definition}
Let $\sSet{X,d}$ be a metric space. The \udef{$M$-Lipschitz diagonal}
\end{definition}
$\vicinity_\sigma(0) = \upset\{\interval{0,1}\}$


\subsection{Continuous convergence structure}
\begin{proposition}
Let $\sSet{X,\xi}$ be a compact convergence space and $\sSet{Y,d}$ a metric space. The continuous convergence on $\cont(X,Y)$ is given by
\[ \forall H\in\powerset(\cont(X,Y)_c): \qquad H\to f \iff \sup_{x\in X}d(H(x), f(x)) \to 0. \]
\end{proposition}
\begin{proof}
First assume $\sup_{x\in X}d(H(x), f(x)) \to 0$. 

Now assume $\sup_{x\in X}d(H(x), f(x)) \not\to 0$. Then there exists $A\in \neighbourhood(0)$ such that $A \notin \sup_{x\in X}d(H(x), f(x))$ we can construct the set
\[ \setbuilder{\setbuilder{x\in X}{d(h(x), f(x)) \notin A \forall h\in S}}{S\in H}. \]
We claim this is a proper filter in $X$. It is contained in an ultrafilter by the ultrafilter lemma \ref{ultrafilterLemma} and this ultrafilter $F$ converges by compactness. Thus $d(H[F], f[F]) \not\to 0$ and so $H\not\to f$.
\end{proof}

\chapter{Topological convergence and topological spaces}
A topology specifies which points are close to each other and which are not. This is useful for determining continuity and the existence of holes for example.

Obviously one way to get an idea of which points are close to other points is by explicitly supplying a notion of distance. In fact this a particular type of topological space called a metric space. This way of describing the topology will turn out to be too restrictive, however.

Another way of describing topology is saying that it is concerned with the properties of a geometric object that are preserved under continuous deformations, such as stretching, twisting, crumpling and bending, but not tearing or gluing. Those continuous deformations do not change \textit{which} points are close to each other, just exactly how close they are. Tearing separates points that were close and gluing makes points that were not in each others neighbourhoods suddenly neigbours.

A famous example of such a continuous deformation is the deformation between a doughnut and a coffee cup. Because a topologist is only interested in properties that are preserved under such a transformation, the joke goes that for him as doughnut and a mug is the same thing.

All this can be achieved by defining which subsets of the space are \textit{neighbourhoods} of each point. A neighbourhood of a point is an open set around that set and can be thought of as a sort of generalisation of the open intervals on the real line. TODO motivate definition.

\url{https://en.wikipedia.org/wiki/List_of_topologies}

\url{http://www.dynamics-approx.jku.at/lena/Cooper/riesz.pdf} For order convergence!!!

\section{Axiomatisations and basic concepts}
TODO Motivation.
\subsection{Building blocks}
The basic building blocks of topology are neighbourhoods, open sets, closed sets, interior, closure, boundaries, limit points, convergence and nearness. Each of these concepts can be axiomatised and given any one, the others are uniquely fixed.

We first describe how these concepts are related, and then give each axiomatisation and show they are equivalent.

\url{https://en.wikipedia.org/wiki/Characterizations_of_the_category_of_topological_spaces} 
\url{https://mathoverflow.net/questions/19152/why-is-a-topology-made-up-of-open-sets/19173#19173}

\subsection{Neighbourhoods, open sets, closed sets}
TODO: function spaces: closed sets point-wise topology >< uniform topology (bounded by functions v bounded by horizontals)

Let $X$ be a set.

Neighbourhoods: sets that ``completely surround'' $x$.

For every $x\in X$ we specify which subsets of $X$ are neighbourhoods of $x$. This family of sets is denoted $\neighbourhood(x)$. We would like the following to hold:
\begin{enumerate}
\item Every neighbourhood of $x$ must contain $x$.
\item Every set that contains a neighbourhood of $x$ is a neighbourhood of $x$.
\item The intersection of two neighbourhoods is again a neighbourhood.
\item The point $x$ must in some sense be in the interior of each of its neighbourhoods. TODO: \url{https://math.stackexchange.com/questions/2692678/why-does-the-definition-of-a-topology-via-neighborhoods-include-this-axiom}
\end{enumerate}

\begin{proposition}
Let $(X,\mathcal{T})$ be a topological space and $x\in X$. Then
\begin{enumerate}
\item if $N\in\neighbourhood(x)$, then $x\in N$;
\item if $M\subset X$ and there exists $N\in\neighbourhood(x)$ such that $N\subset M$, then $M\in\neighbourhood(x)$;
\item if $M,N\in\neighbourhood(x)$, then $M\cap N\in\neighbourhood(x)$;
\item $\forall N\in\neighbourhood(x): \exists M\in\neighbourhood(x): \forall y\in M: N\in\neighbourhood(y)$.
\end{enumerate}
Conversely, every function $X\to \mathcal{P}(X)$ that satisfies these properties is the neighbourhood topology for some topology on $X$.
\end{proposition}
The fourth point is the least obvious. It essentially says that if $y$ is sufficiently close to $x$ (i.e.\ $y\in M(x)$), then $x$ is also close to $y$.
\begin{proof}
TODO
\end{proof}

\begin{definition}
A \udef{topology} on a set $X$ is a collection $\mathcal{T}\in \mathcal{P}(X)$ of subsets of $X$ having the following properties:
\begin{enumerate}
\item Both $\emptyset$ and $X$ are in $\mathcal{T}$.
\item The union of the elements of any subcollection of $\mathcal{T}$ is in $\mathcal{T}$.
\item The intersection of the elements on any finite subcollection of $\mathcal{T}$ is in $\mathcal{T}$.
\end{enumerate}
A set $X$ for which a topology $\mathcal{T}$ has been specified is called a \udef{topological space}.
\end{definition}
These axioms formalise an idea of open subset.
\begin{definition}
We call a subset $U$ of $T$ an \udef{open set} if $U$ is in $\mathcal{T}$. A \udef{closed set} is any set that can be constructed as $X \setminus U$ for some open set $U$. In a given topology a set may be open, closed, both or neither. If a set is both open and closed, it is called \udef{clopen}.

We say $U$ is an \udef{open neighbourhood} of $x$ if $U$ is an open set containing $x$. This is sometimes denoted $U(x)$.

A \udef{neighbourhood} of $x$ is a set containing an open neighbourhood of $x$. We denote by $\neighbourhood(x)$ the set of neighbourhoods of $x$. The function $\neighbourhood$ is called the \udef{neighbourhood topology}.
\end{definition}

\begin{lemma}
Let $X$ be a topological space. Every open set $O$ can be written as $X\setminus K$ for some closed $K$.
\end{lemma}
\begin{proof}
Lemma \ref{BooleanConsequences}.
\end{proof}

\begin{example}
\begin{itemize}
\item Let $X$ be a three-element set, $X = \{a,b,c\}$. There are many possible topologies on $X$, to name a few:
\begin{itemize}
\item $\left\{\emptyset, X\right\}$
\item $\left\{\emptyset, \{a\}, \{a,b\}, X\right\}$
\item $\left\{\emptyset, \{a\}, X\right\}$
\item $\left\{\emptyset, \{a,b\}, X\right\}$
\item $\left\{\emptyset, \{a,b\}, \{a,c\}, \{b\}, X\right\}$
\item $\left\{\emptyset, \{a,b\}, \{c\}, X\right\}$
\item $\left\{\emptyset, \{a\}, \{b\}, \{a,b\}, X\right\}$
\item $\ldots$
\end{itemize}
\item For any set $X$, the collection of all subsets of $X$ is a topology, called the \udef{discrete topology}.
\item For any set $X$, the topology $\mathcal{T} = \left\{\emptyset, X\right\}$ is called the \udef{trivial} topology.
\item In any topology, both $X$ and $\emptyset$ are both open and closed.
\item Let $X$ be a set. Let $\mathcal{T}_f$ be a collection of all subsets $U$ of $X$ such that $X\setminus U$ is finite or $U=\emptyset$. Then $\mathcal{T}_f$ is a topology on $X$ called the \udef{finite complement topology}.
\end{itemize}
\end{example}

We can also characterise the topology with closed sets or with neighbourhoods:
\begin{proposition}
Let $(X,\mathcal{T})$ be a topological space. Let $\mathcal{T}_c$ be the family of closed subsets of $X$. Then
\begin{enumerate}
\item Both $\emptyset$ and $X$ are in $\mathcal{T}_c$.
\item Let $\mathcal{E}$ be a subset of $\mathcal{T}_c$. Then $\bigcap\mathcal{E}\in\mathcal{T}_c$.
\item If $A,B\in \mathcal{T}_c$, then $A\cup B\in \mathcal{T}_c$.
\end{enumerate}
Conversely, given any set $X$ and any family $\mathcal{T}_c\subset\mathcal{P}(X)$ that satisfies these propserties, the family
\[ \mathcal{T} = \setbuilder{O\subset X}{X\setminus O\in\mathcal{T}_c} \]
is a topology on $X$. 
\end{proposition}


Obviously a set can have different topologies.
\begin{definition}
Sometimes we can compare them. Two topologies $\mathcal{T}$ and $\mathcal{T}'$ are \udef{comparable} if either $\mathcal{T} \subseteq \mathcal{T}'$ or $\mathcal{T} \supseteq \mathcal{T}'$.
\begin{itemize}[leftmargin=2cm]
\item[\boxed{\mathcal{T} \subseteq \mathcal{T}'}] In topology $\mathcal{T}'$ there are more open sets. This allows more granular specification of neighbourhoods. We say $\mathcal{T}'$ is \udef{finer} than $\mathcal{T}$. If $\mathcal{T}'$ is a proper superset, we say it is \udef{strictly finer} than $\mathcal{T}$.
\item[\boxed{\mathcal{T} \supseteq \mathcal{T}'}] In this case $\mathcal{T}'$ is \udef{(strictly) coarser} than $\mathcal{T}$.
\end{itemize}
\end{definition}

\subsection{Closure and interior of a set}
\url{https://en.wikipedia.org/wiki/Kuratowski_closure_axioms}

\begin{definition}
Given any subset $A$ of a topological space $X$,
\begin{itemize}
\item The \udef{interior} of $A$, denoted $A^\circ$, is the union of all open sets contained in $A$;
\item The \udef{closure} of $A$, denoted $\bar{A}$, is the intersection of all closed sets containing $A$. 
\end{itemize}
The \udef{boundary} of $A$ is $\partial A \defeq \bar{A}\setminus A^{\circ}$.
\end{definition}
We immediately have the inclusions
\[ A^\circ \subset A \subset \bar{A} \]

\begin{lemma}
The interior and closure are dual in the sense that
\[ A^\circ = X\setminus\overline{(X\setminus A)} = \overline{(A^c)}^c \qquad \bar{A} = X\setminus(X\setminus A)^\circ = ((A^c)^\circ)^c \]
where $X$ is a topological space and $A$ is a subset.
\end{lemma}
\begin{proposition}\label{closure}
Let $A$ be a subset of the topological space $X$, then
\[ x\in \bar{A} \qquad \text{\textup{if and only if}}\qquad \text{every open set $U$ containing $x$ intersects $A$}.\]
\end{proposition}
\begin{proof}
We prove the contrapositive.
\[ x\notin \bar{A} \iff \text{there exists an open set $U$ containing $x$ that does not intersect $A$.} \]
\begin{itemize}
\item[$\boxed{\Rightarrow}$] The set $U = X\setminus \bar{A}$ is an open set containing $x$ that does not intersect $A$.
\item[$\boxed{\Leftarrow}$] If there exists such a $U$, then $X\setminus U$ is a closed set containing $A$, so $X\setminus U \supset \bar{A}$. Therefore $x$ cannot be in $\bar{A}$.
\end{itemize}
\end{proof}
\begin{proposition}\label{interior}
Let $A$ be a subset of the topological space $X$, then
\[ x\in A^\circ \qquad \text{\textup{if and only if}}\qquad \text{there exists an open set $U$ such that $x\in U \subset A$}.\]
\end{proposition}
\begin{proof}
The interior is the union of all open sets $U\subset A$. Thus if $x\in A^\circ$, then $x$ is in such a $U$.
\end{proof}
\begin{lemma}
Given any subset $A$ of $X$,
\begin{itemize}
\item $\bar{A}$ is the smallest closed set containing $A$;
\item $A^\circ$ is the largest open set contained in $A$.
\end{itemize}
Consequently the closure and interior are idempotent:
\[ \overline{\bar{A}} = \bar{A} \qquad \text{and} \qquad (A^\circ)^\circ = A^\circ. \]
\end{lemma}

\begin{lemma}
Let $A,B$ be subsets of a topological space $X$. Then
\begin{enumerate}
\item $\overline{A\cup B} = \overline{A}\cup \overline{B}$;
\item $(A\cap B)^\circ = A^\circ \cap B^\circ$.
\end{enumerate}
These properties do not hold for arbitrary unions and intersections.
\end{lemma}
\begin{proof}
TODO
\end{proof}
\begin{lemma}
TODO: Intersection of Interiors contains Interior of Intersection and Closure of Union contains Union of Closure and Closure of Intersection is Subset of Intersection of Closures and Union of Interiors is Subset of Interior of Union
\end{lemma}

\begin{lemma} \label{closureInteriorSubsets}
Let $A\subseteq B$ be sets in a topological space $X$. Then
\begin{enumerate}
\item $\overline{A} \subseteq \overline{B}$;
\item $A^\circ \subseteq B^\circ$.
\end{enumerate}
\end{lemma}

\subsection{Boundaries}
\url{https://math.stackexchange.com/questions/2254363/definitions-of-a-topological-space-reference}
\url{https://math.stackexchange.com/questions/4398247/axiomatizations-of-the-boundary-operator}
\url{https://mathoverflow.net/questions/175800/which-sets-occur-as-boundaries-of-other-sets-in-topological-spaces}

\subsection{Metrics}
Quantales and continuity spaces: \url{https://link.springer.com/content/pdf/10.1007/s000120050018.pdf}
\url{https://arxiv.org/abs/1311.4940}
All Topologies Come From Generalized Metrics - Kopperman

\subsection{Limit points}
\begin{definition}
If $A$ is a subset of the topological space $X$ and if $x$ is a point of $X$ (not necessarily of $A$), we say $x$ is a \udef{limit point} (also sometimes called \udef{cluster point} or \udef{point of accumulation}) of $A$ if every (open) neighbourhood of $x$ intersects $A$ in some point other than $x$ itself.

The set $A'$ of all limit points of $A$ is called the \udef{derived set} of $A$.

An \udef{isolated point} of $A$ is a point $x\in A$ that is not an accumulation point for $A$.
\end{definition}
So $x$ is a limit point of $A$ if it belongs to the closure of $A\setminus \{x\}$.
\begin{example}
Consider $\R$. If $A= ]0,1]$, then the point $0$ is a limit point of $A$. In fact every point in $[0,1]$ is a limit point and no other points of $\R$ are limit points.
\end{example}
This motivates the following assertion:
\begin{proposition}
Let $A$ be a subset of a topological space $X$. Then
\[ \bar{A} = A \cup A' = A^\circ \cup A' \]
where $A'$ is the derived set of $A$.
\end{proposition}
\begin{corollary}
A topological space is closed if and only if it contains all its limit points.
\end{corollary}

\begin{definition}
Let $(X,\mathcal{T})$ be a topological space. A subset $A\subset X$ is \udef{perfect} in $X$ if it is closed and every point of $A$ is an accumulation point of $A$.
\end{definition}
\begin{lemma}
If $A$ has no isolated points, then $\overline{A}$ is perfect in $X$.
\end{lemma}

\subsection{Special subsets}
\begin{definition}
Let $(X,\mathcal{T})$ be a topological space. A set $A\subset X$ is called
\begin{itemize}
\item a \udef{$\mathcal{G}_\delta$-set} if it is a countable intersection of open sets;
\item an \udef{$\mathcal{F}_\sigma$-set} if it is a countable union of closed sets.
\end{itemize}
\end{definition}

\section{Topologies}
\subsection{The basis of a topology}
For many topologies specifying \textit{all} the open sets can be challenging. In this section we give a way to specify a smaller collection of subsets of $X$, called a basis, and generate the topology in terms of that.

\begin{definition}
If $X$ is a set, a \udef{basis} is a subset $\mathcal{B}$ of the powerset of $X$ such that
\begin{enumerate}
\item For each $x\in X$, there is at least one basis element $B\in\mathcal{B}$ containing $x$.
\item If $x$ belongs to the intersection of two basis elements $B_1$ and $B_2$, then there is a basis element $B_3\in\mathcal{B}$ containing $x$ such that $B_3\subset B_1 \cap B_2$.
\end{enumerate}
The \udef{topology $\mathcal{T}$ generated by $\mathcal{B}$} is defined as follows: A subset $U$ of $X$ is said to be open in $X$ if, for each $x\in U$, there is a basis element $B\in\mathcal{B}$ such that $x\in B$ and $B\subset U$. 
\end{definition}
Each basis element is itself an open set. It is not too difficult to check that $\mathcal{T}$ is indeed a topology.

\begin{example}
\begin{itemize}
\item For any set $X$, the collection of all one-point subsets of $X$ is a basis for the discrete topology.
\item The collection of all open intervals on the real line is a basis for the \udef{standard topology} on the real line $\R$.
\end{itemize}
\end{example}

There is an easier way to obtain the topology $\mathcal{T}$ from a basis $\mathcal{B}$:
\begin{lemma}
The topology $\mathcal{T}$ equals the collection of all unions of elements in $\mathcal{B}$.
\end{lemma}

Here is a way to obtain a basis from a topology on $X$.
\begin{lemma}
Suppose that $\mathcal{C}$ is a collection of open sets of $X$ such that for each open set $U$ of $X$ and each $x \in U$, there is an element $C$ of $\mathcal{C}$ such that $x\in C \subset U$. Then $\mathcal{C}$ is a basis for the topology of $X$.
\end{lemma}

We can link the basis to the coarseness of the topology.
\begin{lemma} \label{basisCoarseness}
Let $\mathcal{B}$ and $\mathcal{B}'$ be bases for the topologies $\mathcal{T}$ and $\mathcal{T}'$, respectively, on $X$. The following are equivalent:
\begin{enumerate}
\item $\mathcal{T}'$ is finer than $\mathcal{T}$.
\item For each $x\in X$ and each basis element $B\in\mathcal{B}$ containing $x$, there is a basis element $B'\in\mathcal{B'}$ such that $x\in B'\subset B$.
\end{enumerate}
\end{lemma}

Closures of sets can also be described using a basis.
\begin{lemma}
Let $A$ be a subset of $X$ which has a topology generated by a basis $\mathcal{B}$, then $x\in\bar{A}$ if and only if every basis element $B\in\mathcal{B}$ containing $x$ intersects $A$.
\end{lemma}

\subsubsection{Subbasis}
\begin{definition}
If $X$ is a set, a \udef{subbasis} is a subset $\mathcal{S}$ of the powerset of $X$ such that $X = \bigcup \mathcal{S}$.

The \udef{topology $\mathcal{T}$ generated by $\mathcal{S}$} is the collection of all unions of finite intersections of elements of $\mathcal{S}$. 
\end{definition}
The topology $\mathcal{T}$ is exactly the coarsest topology that makes all sets in the subbasis open.

\subsection{The subspace topology}
\begin{definition}
Let $X$ be a topological space with topology $\mathcal{T}$. Let $Y$ be a subspace of $X$. The collection
\[ \mathcal{T}_Y = \{ Y\cap U\;|\; U\in \mathcal{T} \} \]
is a topology on $Y$ called the \udef{subspace topology}. With this topology, $Y$ is called a \udef{subspace} of $X$.
\end{definition}
\begin{lemma}
Let $Y$ be a subspace of $X$. A set $A$ is closed in $Y$ \textup{if and only if} it equals the intersection of a closed set of $X$ with $Y$.
\end{lemma}

\begin{lemma}
If $\mathcal{B}$ is a basis for the topology of $X$, then
\[\mathcal{B}_Y = \{ B\cap Y \;|\; B\in \mathcal{B} \}\]
is a basis for the subspace topology on $Y$.
\end{lemma}

\begin{lemma}
Let $Y$ be a subspace of $X$.
\begin{enumerate}
\item If $A$ is open in $Y$ and $Y$ is open in $X$, then $A$ is open in $X$.
\item If $A$ is closed in $Y$ and $Y$ is closed in $X$, then $A$ is closed in $X$.
\end{enumerate}
\end{lemma}
 
We reserve the notation $\overline{A}$ to stand for the closure of $A$ in $X$, not $Y$.
\begin{lemma} \label{subspaceClosure}
Let $X$ be a topological space and $Y\subset X$ a subspace. Let $A$ be a subset of $Y$, then the closure of $A$ in $Y$ is
\[ \Closure_Y(A) = \overline{A}\cap Y.  \]
\end{lemma}

\begin{lemma} \label{notLimitPointSingletonOpen}
Let $A\subseteq X$ be a subspace of $X$. Then $a\in A\setminus A'$, then $\{a\}$ is open in $A$.
\end{lemma}
\begin{proof}
Assume such an $a$. Then there exists an open neighbourhood $U$ of $a$ in $X$ that does not intersect $A$ in any other point. By definition of the subspace topology $\{a\}$ is open.
\end{proof}

\subsection{Topology and order}


\url{https://planetmath.org/orderedspace}
\url{https://www.jstor.org/stable/2032122?seq=2#metadata_info_tab_contents}
\url{http://www.math.wm.edu/~lutzer/drafts/PragueSurveyFinal.pdf}
\url{https://ncatlab.org/nlab/show/pospace}
\url{file:///C:/Users/user/Downloads/order-topological-lattices.pdf}

\subsubsection{Specialisation preorder}
\begin{definition}
Let $(X,\mathcal{T})$ be a topological space and $x,y\in X$. We say $x$
\end{definition}

\paragraph{Alexandrov topology}
\url{https://planetmath.org/inducedalexandrofftopologyonaposet}
\url{https://arxiv.org/pdf/0708.2136.pdf}
\url{https://ncatlab.org/nlab/show/specialization+topology}
\url{http://math.uchicago.edu/~may/REU2018/REUPapers/Asness.pdf}

\subsubsection{Order topology on totally ordered sets}
\begin{definition}
Let $(X,\leq)$ be a linearly ordered set. Let $\mathcal{B}$ be the collection of all sets of the following type:
\begin{enumerate}
\item All open intervals $]a,b[$ in $X$;
\item All intervals of the form $[a_0, b[$, where $a_0$ is the smallest element (if any) of $X$;
\item All intervals of the form $]a, b_0]$, where $b_0$ is the largest element (if any) of $X$;
\end{enumerate}
The collection $\mathcal{B}$ is a basis for a topology, called the \udef{order topology}.
\end{definition}

\paragraph{Product of linearly ordered topology}


\section{Separation axioms}
\subsection{$T_0$}
\subsubsection{Kolmogorov quotient}

\subsection{$T_1$}
\begin{proposition}
Let $X$ be a topological space satisfying $T_1$; let $A$ be a subset of $X$.
Then the point $x$ is a limit point of $A$ \textup{if and only if} every neighbourhood of $x$ contains infinitely many points of $A$.
\end{proposition}
\subsection{Hausdorff spaces}
\begin{definition}
A topological space $X$ is called a \udef{Hausdorff space} if for each pair $x_1, x_2$ of distinct points in $X$, their exist neighbourhoods $U(x_1)$, $U(x_2)$ that are disjoint.
\end{definition}
In Hausdorff spaces distinct points can be told apart topologically, hence Hausdorff spaces are also called \udef{separated spaces}. In particular the Hausdorff condition implies the uniqueness of limits, which is not otherwise guaranteed.

\begin{proposition}
Every finite point set in a Hausdorff space is closed. TODO: T1
\end{proposition}
\begin{proof}
It suffices to show that every one-point set $\{x_0\}$ is closed. Indeed if $\{x_0\}$ was not closed, the closure of $\{x_0\}$ would contain another point. This other point has a disjoint neighbourhood by Hausdorff, so this fails by proposition \ref{closure}.
\end{proof}
\begin{proposition}
Limits are unique (sequences, filters, nets)
\end{proposition}
\begin{lemma}
\begin{enumerate}
\item A subspace of a Hausdorff space is Hausdorff.
\item Every totally ordered set is Hausdorff in the order topology.
\item Every metric topology is Hausdorff.
\end{enumerate}
\end{lemma}



\section{Functions on topological spaces}
\subsection{Open and closed maps}
TODO
\subsection{Continuity and continuous functions}
Intuitively, a continuous map is a map between topological spaces that does not make jumps. In particular let $f: X\to Y$ be a potentially continuous function. Say we want to stay in a neighbourhood $V(f(x_0))$, then we want there to be a neighbourhood $U(x_0)$ such that points inside $U(x_0)$ map to points in $V$, i.e.\
\[ x\in U(x_0) \implies f(x) \in V(f(x_0)). \]
That is $f(U(x_0)) \subset V$, or $U(x_0)\subset f^{-1}(V)$. So we conclude that for any point $x_0$ and neighbourhood $V(f(x_0))$ in $Y$, $f^{-1}(V)$ must contain a neighbourhood of $x_0$. Thus $f^{-1}(V)$ can be written as a union of open sets, $\bigcup_{x\in f^{-1(V)}}U(x)$, and therefore must be open. This motivates the definition:
\begin{definition}
Let $X,Y$ be topological spaces.
\begin{itemize}
\item A function $f:X\to Y$ is \udef{continuous} if for each open set $V$ of $Y$, $f^{-1}[V]$ is an open subset of $X$.
\item The function $f$ is \udef{continuous at $x_0$} if for each open neighbourhood $V$ of $f(x_0)$, their is an open neighbourhood $U$ of $x_0$ such that $f[U]\subset V$.
\end{itemize}
If a function is not continuous, it is \udef{discontinuous}.

The set of all continuous functions $X\to Y$ is denoted $\cont(X,Y)$. If $X=Y$, we also write $\cont(X)$.
\end{definition}
\begin{lemma} \label{globalContinuityFromAllPoints}
A function $f:X\to Y$ is continuous \textup{if and only if} it is continuous at every point.
\end{lemma}

\begin{lemma} \label{continuityAtIsolatedPoint}
Let $f:X\to Y$ be a function between topological spaces. If $\{x_0\}\subset X$ is open, then $f$ is continuous at $x_0$.
\end{lemma}

\begin{lemma}
\begin{enumerate}
\item If the topology of $Y$ is given by a basis $\mathcal{B}$, then to prove continuity of $f$ it suffices to show that the inverse image of every basis element is open.
\item If the topology of $Y$ is given by a subbasis $\mathcal{S}$, then to prove continuity of $f$ it suffices to show that the inverse image of every subbasis element is open.
\end{enumerate}
\end{lemma}
\begin{proposition}\label{continuity}
Let $X, Y$ be topological spaces; $f:X\to Y$. The following are equivalent:
\begin{enumerate}
\item $f$ is continuous;
\item $f[\bar{A}]\subset \overline{f[A]}$;
\item for every closed set $B$ of $Y$, the set $f^{-1}[B]$ is closed in $X$. TODO $f$ closed.
\end{enumerate}
\end{proposition}
\begin{proof}
We proceed round-robin-style.
\begin{itemize}[leftmargin=2cm]
\item[$\boxed{(1) \Rightarrow (2)}$] Let $x\in \bar{A}$ and $V$ a neighbourhood of $f(x)$. Then $f^{-1}[V]$ is an open set containing $x$, so it must intersect $A$ in some point $y$ by proposition \ref{closure}. Then $V$ intersects $f[A]$ in $f(y)$, so $f(x) \in \overline{f[A]}$ as desired.
\item[$\boxed{(2) \Rightarrow (3)}$] Let $B$ be closed in $Y$. We observe that $f[f^{-1}[B]]\subset B$. Choose some $x\in \overline{f^{-1}[B]}$, then
\[ f(x) \in f\left[\overline{f^{-1}[B]}\right] \subset \overline{f[f^{-1}[B]]} \subset \bar{B} = B, \]
so that $x\in f^{-1}[B]$. Thus $\overline{f^{-1}[B]}\subset f^{-1}[B]$, meaning $f^{-1}[B]$ is closed.
\item[$\boxed{(3) \Rightarrow (1)}$] Let $V$ be an open set in $Y$. Set $B = Y\setminus V$. Then $V = Y\setminus B$ and
\[ f^{-1}[V] = f^{-1}[Y\setminus B] = f^{-1}[Y]\setminus f^{-1}[B] = X \setminus f^{-1}[B]\]
using lemma \ref{imagePreimageUniqueness}. Thus $f^{-1}[V]$ is open.
\end{itemize}
\end{proof}

\subsubsection{Homeomorphisms \textit{or} topological isomorphisms}
\begin{definition}
Let $X,Y$ be topological spaces and $f:X\to Y$ a bijection. Then $f$ is a \udef{homeomorphism} if both $f$ and $f^{-1}$ are continuous.
\end{definition}
\begin{lemma}
A homeomorphism is a bijection $f$ such that $f(U)$ is open \textup{if and only if} $U$ is open.
\end{lemma}
\subsubsection{Constructing continuous functions}
\begin{proposition} \label{continuousConstructions}
Let $X,Y$ and $Z$ be topological spaces.
\begin{enumerate}
\item \textup{(Identity function)} The identity function $I:X\to X$ is continuous.
\item \textup{(Constant function)} If $f:X\to Y$ maps all of $X$ into a single $y_0$ of $Y$, then $f$ is continuous.
\item \textup{(Inclusion)} Let $A$ be a subspace of $X$, then the inclusion $A\hookrightarrow X$ is continuous.
\item \textup{(Composites)} If $f:X\to Y$ and $g:Y\to Z$ are continuous, then $g\circ f: X\to Z$ is continuous.
\item \textup{(Restricting the domain)} If $f:X\to Y$ is continuous and $A$ is a subspace of $X$, then the restricted function $f|_{A}:A\to Y$ is continuous.
\item \textup{(Restricting the range)} Let $f:X\to Y$ be continuous. If $Z$ is a subspace of $Y$ containing the image set $f[X]$, then $f:X\to Z$ is continuous.
\item \textup{(Expanding the range)} Let $f:X\to Y$ be continuous. If $Y$ is a subspace of $Z$, then $f:X\to Z$ is continuous.
\item \textup{(Local formulation of continuity)} The map $f:X\to Y$ is continuous is $X$ can be written as the union of open sets $U_\alpha$ such that $f|_{U_\alpha}$ is continuous for each $\alpha$.
\end{enumerate}
\end{proposition}
\begin{proposition}[The pasting lemma]
Let $X=A\cup B$ where $A,B$ are closed in $X$. Let $f:A\to Y$ and $g:B\to Y$ be continuous such that $f(x)=g(x)$ for all $x\in A\cap B$. Then the function defined by
\[ h: X\to Y: x\mapsto h(x) = \begin{cases}
f(x) & (x\in A) \\ g(x) & (x\in B)
\end{cases} \]
is continuous.
\end{proposition}
\begin{proof}
Let $C$ be a closed subset of $Y$, then $f^{-1}[C]$ and $g^{-1}[C]$ are both closed. So
\[ h^{-1}[C] = f^{-1}[C]\cup g^{-1}[C] \]
is closed, meaning $h$ is continuous, all by proposition \ref{continuity}.
\end{proof}

TODO: complex conjugation continuous.

\subsection{Limits of functions}
\begin{definition}
Let $X,Y$ be topological spaces. Let $p$ be a limit point of $A\subseteq X$ and $f: A\to Y$. We say $L\in Y$ is a \udef{limit} of $f(x)$ as $x$ approaches $p$ if
\[ \forall\;\text{open neighbourhood}\; V(L):\exists \;\text{open neighbourhood}\; U(p):\; f[(U\cap A)\setminus \{p\}] \subseteq V. \]
We write $f(x)\to L$ as $x\to p$ or
\[ \lim_{x\to p}f(x) = L. \]
\end{definition}
Note that the value of $f$ at $p$ is irrelevant to the definition of the limit. The domain of $f$ does not even need to contain $p$.

\begin{proposition}
Let $f:X\to Y$ be a functions between topological spaces. Then $f$ is continuous at $p\in X$ \textup{if and only if} $\lim_{x\to p}f(x) = f(p)$.
\end{proposition}

Limits may or may not exist and may or may not be unique, but uniqueness is guaranteed if $Y$ is Hausdorff.
\begin{proposition} \label{HausdorffUniqueLimit}
Let $f: A\subseteq X\to Y$ be a function and $X,Y$ be topological spaces. If $Y$ is Hausdorff, then there is a most one limit of $f$ at any point $p\in X$.
\end{proposition}
\begin{proof}
Assume $L_1$ and $L_2$ are two distinct limits of $f(x)$ as $x\to p$. Because $Y$ is Hausdorff there are two disjoint open neighbourhoods $V_1, V_2$ of $L_1,L_2$. Let $U_1,U_2$ be the corresponding open neighbourhoods of $p$. Then $U_1\cap U_2$ must be an open neighbourhood of $p$, so that $U_1\cap U_2\cap A$ contains a point other than $p$, by virtue of $p$ being a limit point. This however means that $f[(U_1\cap A)\setminus \{p\}]$ and $f[(U_2\cap A)\setminus \{p\}]$ are not disjoint, so neither are $V_1,V_2$: a contradiction.
\end{proof}

\subsection{Initial and final topologies}

\subsection{Sets of functions}
\begin{definition}
Let $X, Y$ be topological spaces.
\begin{itemize}
\item The set of continuous functions in $(X\to Y)$ is denoted $\cont(X, Y)$.
\item The set of continuous functions in $(X\to Y)$ which vanish at infinity is denoted $\cont_0(X, Y)$.
\item The set of continuous functions in $(X\to Y)$ with compact support is denoted $\cont_c(X, Y)$.
\item The set of bounded continuous functions in $(X\to Y)$ is denoted $\cont_b(X, Y)$.
\end{itemize}
If we omit $Y$, we generally mean $Y = \R$.
\end{definition}

TODO: define these notions in general!
TODO: ideals and multiplier algebras.


\section{The product topology}
\subsection{Finite Cartesian products}
\begin{definition}
The \udef{product topology} on $X\times Y$ is the topology having as basis the collection $\mathcal{B}$ of all sets of the form $U\times V$, where $U$ is an open subset of $X$ and $V$ is an open subset of $Y$.
\end{definition}
\begin{lemma} \label{basisFiniteProductTopology}
If $\mathcal{B}$ is a basis for the topology of $X$ and $\mathcal{C}$ a basis for the topology of $Y$, then
\[ \mathcal{D} = \{ B\times C\;|\; B\in \mathcal{B}\;\text{and}\; C\in \mathcal{C} \} \]
is a basis for the topology of $X\times Y$.
\end{lemma}
\begin{proposition}
Let $A$ be a subspace of $X$ and $B$ a subspace of $Y$. The product topology on $A\times B$ is the same as the subspace topology on $A\times B$, when viewed as a subset of $X\times Y$.
\end{proposition}

\begin{definition}
Let $X,Y$ be topological spaces. The maps
\begin{align*}
&\pi_1: X\times Y\to X: (x,y)\mapsto x
&\pi_2: X\times Y\to Y: (x,y)\mapsto y
\end{align*}
are called the \udef{projections} of $X\times Y$ onto its first and second factors, respectively.
\end{definition}
\begin{proposition}
The collection
\[ \mathcal{S} = \{ \pi_1^{-1}(U)\;|\; U\;\text{open in}\; X \}\cup \{ \pi_2^{-1}(V)\;|\; V\;\text{open in}\;Y  \} \]
is a subbasis for the product topology on $X\times Y$.
\end{proposition}
\begin{proof}
Let $\mathcal{T}$ denote the product topology on $X\times Y$; let $\mathcal{T'}$ be the topology generated by $\mathcal{S}$.
\begin{itemize}[leftmargin=2cm]
\item[$\boxed{\mathcal{T}'\subset\mathcal{T}}$] We need to prove all elements of $\mathcal{S}$ are open. Indeed $\pi_1^{-1}(U) = U\times Y$ is open and $\pi_2^{-1}(V) = X\times V$ is also open.
\item[$\boxed{\mathcal{T}\subset\mathcal{T}'}$] Let $B\times C$ be an element of the basis, in other words $B\subset X$ and $C\subset Y$ are open. Then $B\times C = \pi_1^{-1}(B)\cap \pi_2^{-1}(C)$.
\end{itemize}
\end{proof}
In particular $\pi_1$ and $\pi_2$ are continuous.
\begin{proposition}\label{continuityCompositeFunctions}
Let $A,X,Y$ be topological spaces and let
\[ f:A\to X\times Y: a\mapsto f(a) = (f_1(a),f_2(a)). \]
Then $f$ is continuous \textup{if and only if} the functions $f_1$ and $f_2$ are continuous.
\end{proposition}
There is no useful criterion for the continuity of a map $f:A\times B \to X$.

\begin{proposition}
Let $X,Y$ be metrisable topological spaces with metrics $d_X$ and $d_Y$. Then $X\times Y$ is metrisable. Possible, equivalent, metrics include
\[ d_\text{max} = \max\circ \{d_X, d_Y\} \]
and
\[ d_\text{graph} = d_X \circ \pi_1 + d_Y \circ \pi_2. \]
\end{proposition}
\begin{proof}
We first prove that the product topology and the metric topology generated by $d_\text{max}$ are the same using \ref{basisCoarseness}.

First take an element of a basis for the product topology, which by \ref{basisFiniteProductTopology} can be taken of the form
\[ B = B_{d_X}(x, \epsilon_1)\times B_{d_Y}(y, \epsilon_2) \qquad \text{for some $x\in X, y\in Y, \epsilon_1,\epsilon_2 >0$.} \]
Then we can find a basiselement $B_{d_\text{max}}((x,y), \min\{\epsilon_1,\epsilon_2\})$ of the metric topology generated by $d_\text{max}$ that is a subset.

Conversely, take $B_{d_\text{max}}((x,y), \epsilon)$. Then $B_{d_X}(x, \epsilon)\times B_{d_Y}(y, \epsilon)$ is a subset.

The equivalence of the two metrics can then be seen by applying \ref{ballsCoarseness} twice:
\[ B_{d_\text{max}}((x,y), \epsilon) \subset B_{d_\text{graph}}((x,y), \epsilon) \qquad B_{d_\text{graph}}((x,y), \epsilon/2) \subset B_{d_\text{max}}((x,y), \epsilon). \]
\end{proof}
\begin{corollary} \label{convergenceFiniteProductTopology}
A sequence $(x_n,y_n)_n$ converges to $(x,y)$ in the product topology \textup{if and only if} $(x_n)_n$ converges to $x$ and $(y_n)_n$ converges to $y$.
\end{corollary}
TODO:also nets?

\subsection{Arbitrary Cartesian products}
\begin{definition}
Let $X = \prod_{\alpha\in J}X_\alpha$ and define
\[ \mathcal{S}_\beta = \{\pi_\beta^{-1}(U_\beta)\;|\; U_\beta\;\text{open in}\;X_\beta\} \qquad \text{and}\qquad \mathcal{S} = \bigcup_{\beta\in J}.\]
Then the topology on $X$ generated by the subbasis $\mathcal{S}$ is the \udef{product topology} and then $X$ is called a \udef{product space}.
\end{definition}
\begin{lemma}
\begin{itemize}
\item The product topology on $\prod X_\alpha$ has as a basis all sets of the form $\prod_\alpha U_\alpha$, where $U_\alpha$ is open in $X_\alpha$ for all $\alpha$ and $U_\alpha = X_\alpha$ except for finitely many values of $\alpha$.
\item If each $X_\alpha$ has a basis $\mathcal{B}_\alpha$, a basis for the product topology is given by all the sets of the form $\prod_{\alpha\in J}B_\alpha$ where $B_\alpha\in\mathcal{B}_\alpha$ for finitely many values of $\alpha$ and $B_\alpha = X_\alpha$ for the rest.
\end{itemize}
\end{lemma}
If we remove the condition that $U_\alpha = X_\alpha$ except for finitely many values of $\alpha$, we get the box topology.

Some results that held for finite Cartesian product also hold for arbitrary products:
\begin{lemma}
Let the topology on $\prod X_\alpha$ be the product topology.
\begin{itemize}
\item If each space $X_\alpha$ is Hausdorff, then $\prod X_\alpha$ is Hausdorff.
\item Let $A_\alpha$ be subsets of $X_\alpha$, then
\[ \prod \bar{A}_\alpha = \overline{\prod A_\alpha}. \]
\item Let $A_\alpha$ be subspaces of $X_\alpha$, for each $\alpha\in J$. Then $\prod A_\alpha$ is a subspace of $\prod X_\alpha$ if both products are given the product topology.
\end{itemize}
\end{lemma}
\begin{proposition}
Let $\prod X_\alpha$ have the product topology. Let $f:A\to \prod X_\alpha$ be given by
\[ f(a)=(f_\alpha(a))_{\alpha\in J} \qquad \text{where $f_\alpha = \pi_\alpha\circ f:A\to X_\alpha$ for each $\alpha \in J$}.\]
Then $f$ is continuous \textup{if and only if} each function $f_\alpha$ is continuous.
\end{proposition}
This does not hold for the box topology.
TODO: Universal mapping property; generalise to initial topologies.
\begin{corollary} \label{productInclusionsContinuous}
Assume we have some points $c_i \in X_i$ for all $i\in J$. Then the functions
\[ i_\alpha: X_\alpha \to X: p \mapsto \left(\begin{cases}
c_i & (i\neq \alpha) \\ p & (i = \alpha)
\end{cases}\right)_{i\in J} \]
are continuous.
\end{corollary}
\begin{proof}
Consider $i_\alpha$. Then for $i = \alpha$, the function $\pi_i \circ i_\alpha: X_\alpha \to X_i$ is the identity in $X_\alpha$ and thus continuous, \ref{continuousConstructions}. For $i \neq \alpha$, the function $\pi_i \circ i_\alpha: X_\alpha \to X_i$ is a constant function $p \mapsto c_i$ and thus continuous, \ref{continuousConstructions}.
\end{proof}

\begin{lemma}
Given points $\vec{x}=(x_i)_{i\in \N}$ and $\vec{y}=(y_i)_{i\in \N}$ of $\R^\N$, define the metric
\[ D(\vec{x}, \vec{y}) = \sup\left\{\frac{\bar{d}(x_i,y_i)}{i}\;|\; i\in \N\right\}, \]
where $\bar{d}$ is the standard bounded metric on $\R$. Then $D$ induces the product topology on $\R^\N$.
\end{lemma}
\begin{proof}
Let $\mathcal{T}$ denote the product topology on $\R^\N$ and $\mathcal{T}_D$ the topology induced by $D$. We prove two inclusions using lemma \ref{basisCoarseness}.
\begin{itemize}[leftmargin=2cm]
\item[$\boxed{\mathcal{T}_D\subset\mathcal{T}}$] Choose arbitrary basis element $B_D(\vec{x},\epsilon)$. Then choose an $N\in\N$ such that $1/N<\epsilon$. Take the basis element
\[ V = ]x_1-\epsilon,x_1+\epsilon[\;\times\;]x_1-\epsilon,x_1+\epsilon[\;\times \ldots\times\; ]x_N-\epsilon, x_N+\epsilon[\;\times \R\times\R\times \ldots \]
for the product topology. We assert that $V\subset B_D(\vec{x},\epsilon)$. Indeed, for all $\vec{y}\in\R^\N$,
\[ \frac{\bar{d}(x_i,y_i)}{i} \leq \frac{1}{i} \leq \frac{1}{N} \qquad \text{if $i\geq N$}. \]
Therefore,
\[ D(\vec{x},\vec{y}) \leq \max\left\{ \frac{\bar{d}(x_1,y_1)}{1},\frac{\bar{d}(x_2,y_2)}{2},\ldots, \frac{\bar{d}(x_N,y_N)}{N}, \frac{1}{N} \right\}. \]
So if $\vec{y}\in V$, then $D(\vec{x},\vec{y})< \epsilon$ and $V\subset B_D(\vec{x},\epsilon)$.
\item[$\boxed{\mathcal{T}\subset\mathcal{T}_D}$] Choose an arbitrary basis element $U = \prod U_i$. Let $U_i=\R$ if $i\notin \{\alpha_1,\ldots, \alpha_n\}$. For each $i\in \{\alpha_1,\ldots, \alpha_n\}$ choose an interval $]x_i-\epsilon_i,x_i+\epsilon_i[\subset U_i$ and define
\[ \epsilon = \min\{\epsilon_i/i\;|\;i=\alpha_1,\ldots, \alpha_n\}. \]
We can easily see that $B_D(\vec{x},\epsilon) \subset U$.
\end{itemize}
\begin{corollary}
Countable products of metrisable spaces are metrisable.
\end{corollary}
\begin{corollary}
Countable products of Hausdorff spaces are Hausdorff.
\end{corollary}
\end{proof}
\begin{lemma}
The product $\R^J$, with $J$ an uncountable index set, is not metrisable.
\end{lemma}
\begin{proof}
In a metrisable space, by TODO ref, we have that if $x\in \bar{A}$, then there exists a sequence of points in $A$ converging to $x$. We construct a counterexample. Let $A$ be the subset of $\R^J$ containing all points $(x_i)_{i\in J}$ such that $x_i=1$ for all but finitely many $i$. Now the point $(0)_{i\in J}$ is in the closure of $A$, but has no sequence in $A$ converging to it. To see that it is in the closure, let $\prod U_\alpha$ be a basis element containing $(0)_{i\in J}$. The intersection $A\cap \prod U_\alpha$ is never empty. Indeed for only finitely many $\alpha$, $U_\alpha\neq \R$. Set $x_\alpha = 0$ for these $\alpha$ and $x_i = 1$ for the rest.
\end{proof}
\subsection{Box topology}
\begin{definition}
Let $X = \prod_{\alpha\in J}X_\alpha$ and take as a basis for a topology the collection of all sets of the form
\[ \prod_{\alpha\in J}U_\alpha \qquad \text{($U_\alpha$ open in $X_\alpha$)}. \]
The topology generated by this basis is the \udef{box topology}.
\end{definition}
The following properties hold, like in the product topology:
\begin{lemma}
Let the topology on $\prod X_\alpha$ be the box topology.
\begin{itemize}
\item If each space $X_\alpha$ is Hausdorff, then $\prod X_\alpha$ is Hausdorff.
\item Let $A_\alpha$ be subsets of $X_\alpha$, then
\[ \prod \bar{A}_\alpha = \overline{\prod A_\alpha}. \]
\item Let each $X_\alpha$ have a basis $\mathcal{B}_\alpha$. The collection of all the sets of the form 
\[ \prod_{\alpha\in J}B_\alpha \qquad B_\alpha\in\mathcal{B}_\alpha \]
serves as a basis for the box topology.
\item Let $A_\alpha$ be subspaces of $X_\alpha$, for each $\alpha\in J$. Then $\prod A_\alpha$ is a subspace of $\prod X_\alpha$ if both products are given the box topology.
\end{itemize}
\end{lemma}
\subsubsection{Failure of metrisability}
\begin{lemma}
$\R^\omega$ is not metrisable in the box topology.
\end{lemma}
\subsubsection{Failure of continuity}
\subsubsection{Failure of compactness}

\section{The quotient topology}
Let $X$ be a topological space. A quotient set can always be defined by a surjective function $f:X\to A$ to a set $A$. Then $A$ can be identified with a partition $X^*$ of $X$. Now we would like to define a topology on the partition. We can think of the quotient as shrinking each partition to a single point. Thus it is natural to call a subset of $A$ open if the union of the corresponding partitions is open:
\[ \text{$V$ is open in $A$}\quad \Leftrightarrow_{\text{def}}\quad \text{$p^{-1}(V)$ is open in $X$}. \]
This gives us the following definition:
\begin{definition}
Let $X,Y$ be topological spaces and $p:X\to Y$ a surjective map. The map $p$ is a \udef{quotient map} if
\[ \text{$V$ is open in $Y$} \iff \text{$p^{-1}(V)$ is open in $X$} \]
\end{definition}
This condition is stronger than continuity.
\begin{definition}
Let $X$ be a topological space.
\begin{itemize}
\item Let a be $A$ a subset, and $p:X\to A$ a surjective map. There exists exactly one topology on $A$ relative to which $p$ is a quotient map; it is called the \udef{quotient topology} induced by $p$.
\item Let $X^*$ be a partition of $X$ and $p:X\to X^*$ the surjective map that carries each point of $X$ to its partition. In the quotient topology induced by $p$, the space $X^*$ is called a \udef{quotient space} of $X$.
\end{itemize}
\end{definition}
We can also characterise the notion of quotient map in another way, starting from the following definition:
\begin{definition}
A subset $C$ of a topological space $X$ is \udef{saturated} with respect to a surjective map $p:X\to Y$ if $C$ is the complete inverse image of a subset of $Y$, i.e.\ it contains every set $p^{-1}(\{y\})$ that it intersects.
\end{definition}
\begin{lemma}
A surjective map $p$ is a quotient map \textup{if and only if} $p$ is continuous and maps saturated open sets of $X$ to open sets of $Y$.
\end{lemma}
\begin{corollary}
\begin{itemize}
\item Surjective continuous open maps are quotient maps.
\item Surjective continuous closed maps are quotient maps.
\end{itemize}
\end{corollary}
There are quotient maps that are neither open or closed.
\begin{proposition}
Let $p:X\to Y$ be a quotient map; let $A$ be a subset of $X$ that is saturated w.r.t. $p$; let $q:A\to p(A) = p|_{A}$.
\begin{enumerate}
\item If $A$ is either open or closed in $X$, then $q$ is a quotient map.
\item If $p$ is either an open map or a closed map, then $q$ is a quotient map.
\end{enumerate}
\end{proposition}
\begin{lemma}
Let $p,q$ be quotient maps.
\begin{enumerate}
\item A composite $q\circ p$ of quotient maps is a quotient map.
\item The product $p\times q$ is not necessarily a quotient map.
\item A quotient space of a Hausdorff space is not necessarily Hausdorff.
\end{enumerate}
\end{lemma}
\begin{proof}
Point (1) follows from
\[ p^{-1}(q^{-1}(U)) = (q\circ p)^{-1}(U). \]
\end{proof}
TODO: theorem 22.2 + corollary



\section{Density}
\begin{definition}
Let $(X,\mathcal{T})$ be a topological space and let $A$ be a subset of $X$. Then $A$ is called
\begin{enumerate}
\item \udef{dense} in $X$ if the closure of $A$ is the whole of $X$: $\overline{A} = X$;
\item \udef{rare} or \udef{nowhere dense} if its closure has empty interior: $(\overline{A})^\circ = \emptyset$;
\item \udef{meagre} (or a \udef{set of first category}) if it is a countable union of rare subsets of $X$;
\item \udef{nonmeagre} (or a \udef{set of second category}) if it is not meagre;
\item \udef{comeagre} if its complement $X\setminus A$ is meagre in $X$.
\end{enumerate}
\end{definition}
\begin{lemma} \label{densityEquivalences}
Let $A$ be a subset of a topological space $X$. $A$ being is dense in $X$ is equivalent to any of the following:
\begin{enumerate}
\item every element of $X$ either lies in $A$ or is a limit point of $A$;
\item $A^c$ has empty interior.
\end{enumerate}
\end{lemma}
\begin{proof}
For the first point: $\overline{A} = A\cup A' = X$.

For the second point:
\[ X = \overline{A} \iff X = ((A^c)^\circ)^c \iff (A^c)^\circ = X^c = \emptyset. \]
\end{proof}

\begin{lemma} \label{nowhereDensityEquivalence}
Let $X$ be a topological space and $A\subset X$ a subset. Then $A$ is nowhere dense \textup{if and only if} $\overline{A}^c$ is dense.
\end{lemma}
\begin{proof}
We calculate:
\[ (\overline{A})^\circ = \emptyset \iff \overline{(\overline{A}^c)}^c = \emptyset \iff \overline{(\overline{A}^c)} = X. \]
\end{proof}

\begin{lemma} \label{meagreSubset}
Any subset of a meagre set is meagre.
\end{lemma}
\begin{proof}
Let $A = \bigcup_k R_k$ be meagre and $B \subseteq A$. Then
\[ B = B\cap A = B\cap \left( \bigcup_k R_k \right) = \bigcup_k B\cap R_k. \]
Now for each $k$, $B\cap R_k \subset R_k$. So $\overline{B \cap R_k}^\circ \subseteq \overline{R_k}^\circ = \emptyset$, using lemma \ref{closureInteriorSubsets}, and thus $B\cap R_k$ is nowhere dense. 
\end{proof}

\begin{lemma} \label{denseSubsetOfDenseSubspaceIsDense}
Let $Y$ be a dense subspace of a topological space $X$. Let $S$ be a dense subset of $Y$. Then $S$ is dense in $X$.
\end{lemma}
\begin{proof}
Let $\overline{S}$ be the closure of $S$ in $X$. Then by \ref{subspaceClosure} we have $Y = Y\cap \overline{S}$, so $\overline{S} \supseteq Y \supseteq S$, which, taking the closure, implies $\overline{S} \supseteq \overline{Y} \supseteq \overline{S}$. Thus $\overline{S} = \overline{Y} = X$.
\end{proof}

\begin{definition}
A topological space $Y$ has the \udef{unique extension property} if for any topological space $X$, any continuous functions $f,g:X\to Y$ and any dense subset $E\subset X$ we have
\[ \forall x\in E: f(x)=g(x) \quad\implies\quad f = g. \]
\end{definition}

TODO:
\begin{proposition}
\begin{enumerate}
\item Assume $X$ Hausdorff and quotient map open, then $X/\sim$ is Hausdorff iff $\sim$ is closed in $X\times X$.
\item $X$ Hausdorff iff diagonal is closed
\item Let $f,g: A\to B$ be continuous functions. If $B$ is Hausdorff, then $\setbuilder{x\in A}{f(x) = g(x)}$ is closed. (Pre-image of diagonal set)
\end{enumerate}
\end{proposition}

\begin{proposition} \label{uniqueExtensionHausdorff}
A topological space $Y$ has the unique extension property \textup{if and only if} $Y$ is Hausdorff.
\end{proposition}
\begin{proof}
First assume $Y$ Hausdorff. Take functions $f,g: E\subset X \to Y$ that agree on $E$. They must agree on a closed set (TODO ref), thus at least on $\overline{E} = X$.

Now suppose $Y$ is not Hausdorff. TODO \url{https://www.jstor.org/stable/2315068?seq=1#metadata_info_tab_contents}
\end{proof}



\begin{lemma}
Let $X$ be a topological space and $E\subset X$ a .
\begin{enumerate}
\item If for any dense subspace $E\subset X$ the only continuous extension of $\id_E$ to $X$ is $\id_X$, then $X$ is $T_0$.
\item If $X$ is $T_2$, then for any dense subspace $E\subset X$ the only continuous extension of $\id_E$ to $X$ is $\id_X$. 
\end{enumerate}
$T_1$ is neither necessary nor sufficient.
\end{lemma}
\begin{proof}
TODO \url{https://math.stackexchange.com/questions/1592144/does-the-identity-map-on-a-dense-subset-of-a-space-extend-uniquely/1592169}
\end{proof}


\subsection{The Baire property}
\begin{definition}
A topological space $X$ has the \udef{Baire property} if it satisfies either of the following equivalent conditions:
\begin{enumerate}
\item every countable union of closed nowhere dense sets has empty interior;
\item every countable intersection of open dense sets is dense.
\end{enumerate}
These properties are equivalent because a subset has empty interior if and only if its complement is dense, see lemma \ref{densityEquivalences}.
\end{definition}


\begin{lemma} \label{BaireEquivalents}
A topological space $X$ is Baire \textup{if and only if} either of the following equivalent conditions:
\begin{enumerate}
\item every meagre subset of $X$ is either empty or not open;
\item every non-empty open subset of $X$ is a nonmeagre subset of $X$;
\item every comeagre subset of $X$ is dense in $X$.
\end{enumerate}
\end{lemma}
\begin{proof}
We prove the characterisation of spaces with the Baire property using countable unions implies the first point, the last point implies the countable intersection Baire condition.
\begin{itemize}[leftmargin=3cm]
\item[$\boxed{\text{Baire}\Rightarrow (1)}$] Every meagre set $A = \bigcup_k R_k$ (where all $R_k$ are nowhere dense) is a subset of $\bigcup_k \overline{R_k}$ where $\overline{R_k}$ are closed nowhere dense sets. Thus if the Baire property holds, $\bigcup_k \overline{R_k}$ has empty interior, meaning $A$ has empty interior. So either $A$ is empty or not open.
\item[$\boxed{(1) \Leftrightarrow (2)}$] By contraposition.
\item[$\boxed{(1) \Rightarrow (3)}$] Suppose $A$ is a meagre set. Then $A^\circ$ must also be meagre, by \ref{meagreSubset}. Now $A^\circ$ is certainly open, so by $(1)$ it must be empty. Thus $A^c$ is dense, by lemma \ref{densityEquivalences}.
\item[$\boxed{(3) \Rightarrow \text{Baire}}$] Let $A = \bigcap_k O_k$ where all $O_k$ are open dense sets. Then $A^c = \bigcup_k O_k^c$. Now for each $k$, $O_k^c$ is nowhere dense by lemma \ref{nowhereDensityEquivalence}, because $\overline{O_k^c}^c = O_k^\circ$ is still dense. Thus $A^c$ is meagre and $A$ is comeagre, so $A$ is dense in $X$.
\end{itemize}
\end{proof}

A topological space has the Baire property if and only if it has the property locally, in the following sense:
\begin{lemma}
A topological space $X$ has the Baire property \textup{if and only if} every point in $X$ has a neighbourhood with the Baire property.
\end{lemma}
\begin{proof}
If $X$ is Baire, the neighbourhood can simply be taken to be $X$.

Assume every point in $X$ has a neighbourhood with the Baire property.
We will prove point (2) in lemma \ref{BaireEquivalents} holds.
Take a non-empty open subset $A$ of $X$.
As $A$ is non-empty, we can take a point $x\in A$ and find a neighbourhood $U$ of $x$ with the Baire property.
Then $A\cap U$ is a non-empty open subset of $U$ and thus must not be meagre in $U$.
By contraposition of lemma \ref{meagreSubset}, we see that $A$ must be non-meagre in $X$, proving the Baireness of $X$.
\end{proof}

\begin{theorem}[Baire category theorem] \label{BaireCategory} \hspace{1em}
\begin{enumerate}
\item Every complete pseudometric space has the Baire property.
\item Every locally compact Hausdorff space has the Baire property.
\end{enumerate}
\end{theorem}
\begin{proof}
TODO + relocate
\end{proof}

\section{Connectedness}
\begin{definition}
Let $X$ be a topological space. A \udef{separation} of $X$ is a pair $U,V$ of disjoint nonempty open subsets of $X$ whose union is $X$.

The space $X$ is said to be \udef{connected} if there does not exist a separation of $X$.
\end{definition}
\begin{lemma}
A space $X$ is connected \textup{if and only if} the only subsets of $X$ that are both open and closed in $X$ are $\emptyset$ and $X$.
\end{lemma}
\begin{proof}
We prove the contrapositive of both implications.
\begin{itemize}
\item[$\boxed{\Rightarrow}$] Let $A$ be a nonempty proper subset of $X$ that is both open and closed in $X$. The sets $A$ and $X\setminus A$ form a separation.
\item[$\boxed{\Leftarrow}$] Let $U,V$ be a separation. Then $U$ is open. It is also closed, because its complement in $X$ is $V$, which is open.
\end{itemize}
\end{proof}
The following lemma characterises separations in the subspace topology.
\begin{lemma}
Let $Y$ be a subspace of a topological space $X$. A pair of disjoint nonempty sets $A,B$ constitute a separation of the subspace $Y$ \textup{if and only if} neither set contains a limit point of the other in $X$.
\end{lemma}
\begin{proof}

\end{proof}

\subsection{Path connectedness}

\section{Compactness}
TODO relatively compact.

TODO every compact set in (pseudo??)metric space is closed and bounded. Converse is not automatic (Heine-Borel property).

\begin{proposition} \label{imageCompactIsCompact}
The continuous image of a compact set is compact.
\end{proposition}
\begin{corollary} \label{imageCompactIsClosedBounded}
If $f:X\to Y$ is a continuous function from a topological space to a metric space and $K\subset X$ a compact set, then $f[K]$ is closed and bounded.
\end{corollary}
\begin{corollary}
Let $f:X\to V$ be a function from a compact topological space to a TVS with Heine Borel property. Then $\im(f)$ has a maximum and minimum.
\end{corollary}

\begin{proposition}
A compact set in a Hausdorff space is closed.
\end{proposition}

\begin{proposition} \label{compactToHausdorffHomeomorphism}
Let $f:X\to Y$ be a continuous bijection between topological spaces. If $X$ is compact and $Y$ is Hausdorff, then $f$ is a homeomorphism.
\end{proposition}
\begin{proof}
We just need to prove $f^{-1}$ is continuous. This is equivalent to $f$ being closed by \ref{continuity}. TODO
\end{proof}

\begin{proposition}
Let $X$ be a topological space. Then $X$ is compact \textup{if and only if} every ultrafilter converges.
\end{proposition}
\begin{proof}
We prove the contrapositive. Assume $U$ is an ultrafilter that does not converge. Then for all $x\in X$ we have $\neighbourhood(x) \nleq U$, so we can find some $V_x \in \neighbourhood(x) \setminus U$ and thus also an open set $U_x \setminus V_x \in \neighbourhood(x) \setminus U$. (If $U_x$ were in $U$, then also $V_x\in U$, which is not possible.) The set $\{U_x\}_{x\in X}$ is an open cover. By compactness, we can find a finite subcover $\{U_{x_i}\}_{i=0}^n$. Then $U_{x_0}\cup \ldots \cup U_{x_n} = X \in U$. Because every ultrafilter on $\powerset(X)$ is a prime filter \ref, we can find a $U_{x_i}$ such that $U_{x_i}\in U$, which is a contradiction.

Now assume $X$ is not compact. Then there exists an open cover $\{U_i\}_{i\in I}$ that does not have a finite subcover. Consider the family $\{U_i^c\}_{i\in I}$. Every finite intersection of sets in $\{U_i^c\}_{i\in I}$ is nonempty. (If there was such an empty intersection, then taking the complement would give a finite union equaling $X$.) This means the $\cap$-closure of $\{U_i^c\}_{i\in I}$ does not contain $\emptyset$, meaning $F = \mathfrak{F}\{U_i^c\}_{i\in I}$ is a proper filter. Assume, towards a contradiction, that $F$ converges to $a\in X$. We can find a $U_a$ in the cover $\{U_i\}_{i\in I}$ such that $a\in U_a$. Because $F \geq \neighbourhood(x)$, we have $U_a \in F$. But $U_a^c$ is an element of the subbasis of $F$, so $U_a^c\in F$. Thus $U_a\cap U_a^c = \emptyset \in F$, meaning $F$ is not proper.
\end{proof}

\url{https://fa.ewi.tudelft.nl/~hart/37/publications/the_papers/betaR.pdf}

\subsection{Limit point compactness}
\subsection{Local compactness}






\chapter{Sequences, nets and filters}
TODO: move before metric spaces

Sequences, nets and filters can be seen as probes of the topology.
\section{Sequences}
\subsection{Sequential filters}

\begin{lemma}
Let $X$ be a set and $\seq{x_n}$ a sequence in $X$. Then
\[ \Tails(\seq{x_n}) \defeq \setbuilder{\setbuilder{x_n}{n \geq k}}{k\in \N} = \setbuilder{\seq{x_n}[k:\infty]}{k\in \N} \]
is a downward directed subset of $\powerset(X)$.
\end{lemma}
\begin{proof}
Take $A,B\in \Tails(\seq{x_n})$. If $A = \setbuilder{x_n}{n \geq a}$ and $B = \setbuilder{x_n}{n \geq b}$, then $A$ is a lower bound of $\{A,B\}$ if $a\geq b$. Otherwise $B$ is a lower bound.
\end{proof}

\begin{definition}
Let $X$ be a set and $\seq{x_n}$ a sequence in $X$. We call
\begin{itemize}
\item $\Tails(\seq{x_n})$ the \udef{sequential filter base} of $\seq{x_n}$;
\item the filter generated by $\Tails(\seq{x_n})$ in $\powerset(X)$ the \udef{(sequential) filter} of $\seq{x_n}$, which we denote $\TailsFilter(\seq{x_n})$;
\item any filter that is generated by a sequence a \udef{sequential filter};
\item the \udef{kernel} of the sequence the kernel of the associated sequential filter.
\end{itemize}
We have 
\[ \TailsFilter(\seq{x_n}) = \upset \Tails(\seq{x_n}). \]
\end{definition}

\begin{proposition}
Let $X$ be a set. A filter $F\in \powerfilters(X)$ is sequential \textup{if and only if} $F$ contains a countable set and admits a countable base consisting of almost equal sets.
\end{proposition}
\begin{proof}
$\Rightarrow$ Assume $F = \TailsFilter(\seq{x_n})$ for some sequence $\seq{x_n}$. Then every set in $\Tails(\seq{x_n})$ is countable and an element of $F$.

$\Leftarrow$ Assume $F$ contains a countable set. Consider the intersection of all countable sets in $F$.
\end{proof}

\subsection{Limits and convergence}
\begin{definition}
Let $(X,\mathcal{T})$ be a topological space and $(a_n)_{n\in\N}$ a sequence in $X$. We can view this sequence as a function $\N\subset (\N\cup\{\infty\})\to X$.

We define \udef{limit} of the sequence as a limit of this function at $\infty$ if $\N\cup\{\infty\}$ is equipped with the order topology.
\end{definition}
By \ref{HausdorffUniqueLimit} a sequence has at most one limit $L\in X$ if $X$ is Hausdorff. In this case we call it the limit of the sequence and write
\[ \lim_{n\to \infty}a_n = L. \]
We call a sequence \udef{divergent} if it does not have a limit and
 \udef{convergent} if it does have a limit $L$. In this last case we say the sequence \udef{converges} to $L$.

\begin{proposition} \label{sequenceConvergence}
Let $(X,\mathcal{T})$ be a topological space and $(a_n)_{n\in\N}$ a sequence in $X$. Then the sequence converges to $L\in X$ \textup{if and only if}
\[ \forall \;\text{open neighbourhood}\; V(L): \exists n_0\in \N: \forall n\geq n_0: a_n\in V(L). \]
\end{proposition}
\begin{proof}
Assume the sequence converges to $L$ and take an arbitrary open neighbourhood $V(L)$. Then there exists an open neighbourhood $U(\infty)$ such that $a[U\setminus\{\infty\}] \subseteq V$. Now by definition of the order topology, there exists an interval $]m, \infty]\subseteq U(\infty)$. Then $a[\;m, \infty[\;] \subseteq V$ and by setting $n_0=m+1$ we get the criterion of the proposition.

Conversely assume the criterion and fix $V(L)$. Then we can take $U(\infty)=]n_0, \infty]$.
\end{proof}

\begin{lemma} \label{subsequencesConverge}
Let $(X,\mathcal{T})$ be a topological space and $(a_n)_{n\in\N}$ a sequence in $X$ that converges to $L$. Then all subsequences converge to $L$.
\end{lemma}

TODO: Bolzano-Weierstrass (sequence version + accumulation point version)

\subsection{Sequential spaces}
\subsubsection{The sequential topology}
\begin{definition}
Let $(X,\mathcal{T})$ be a topological space and $S\subseteq X$ a subset.
\begin{itemize}
\item The \udef{sequential closure} of $S$ in $X$ is the set
\[ \operatorname{SeqCl}(S) \defeq \setbuilder{x\in X}{\text{$\exists$ a sequence in $S$ that converges to $x$ in $X$}}. \]
\item The \udef{sequential interior} of $S$ in $X$ is the set
\[ \operatorname{SeqInt}(S) \defeq \setbuilder{s\in S}{\text{every sequence in $X$ that converges to $s$ has a tail in $S$}}. \]
\end{itemize}
We call $S$
\begin{itemize}
\item \udef{sequentially open} if $S = \operatorname{SeqInt}(S)$;
\item \udef{sequentially closed} if $S = \operatorname{SeqCl}(S)$;
\item a \udef{sequential neighbourhood} of a point $x\in X$ if $x\in \operatorname{SeqInt}(S)$.
\end{itemize}
\end{definition}
A sequentially closed set is a set $S$ such that all limits of sequences in $S$ are also in $S$.

\begin{lemma} \label{sequentialInteriorClosure}
Let $(X,\mathcal{T})$ be a topological space and $R,S\subseteq X$  subsets. Then
\begin{enumerate}
\item $\operatorname{SeqInt}(S) = (\operatorname{SeqCl}(S^c))^c$;
\item $\operatorname{SeqCl}(\emptyset) = \emptyset$ and $\operatorname{SeqCl}(X) = X$;
\item $S\subseteq \operatorname{SeqCl}(S)$;
\item $\operatorname{SeqCl}(R\cup S) = \operatorname{SeqCl}(R)\cup \operatorname{SeqCl}(S)$;
\item $\operatorname{SeqCl}(S) \subseteq \bar{S}$ and $\operatorname{SeqInt}(S) \supseteq S^\circ$.
\end{enumerate}
\end{lemma}
\begin{proof}
(1) Both sides of the equation are equivalent to
\[ \setbuilder{s\in S}{\text{$\nexists$ a sequence in $X\setminus S$ that converges to $s$ in $X$}}. \]

(2) There are no sequences that converge to a point in $\emptyset$ and all points $x\in X$ are the limit of a constant sequence $n\mapsto x$.

(3) The constant sequence $n\mapsto x$ converges to $x$.

(4) We can find a subsequence in $R$ or in $S$. All subsequences converge by \ref{subsequencesConverge}.

(5) Let $x\in \operatorname{SeqCl}(S)$, so there is a sequence $(a_n)$ in $S$ that converges to $x$. Take an arbitrary open neighbourhood $V$ of $x$. Then by convergence there is a subsequence of $(a_n)$ that is a sequence in $V$. In particular $V$ intersects $S$. So $x\in \bar{S}$ by \ref{closure}.
\end{proof}

\begin{proposition} \label{sequentialTopology}
Let $(X,\mathcal{T})$ be a topological space. The set of all sequentially open sets forms a topology $\mathcal{T}_\text{seq}$ on $X$. This topology is finer than the original topology.
\end{proposition}
\begin{proof}
First note that sequentially closed sets are complements of sequentially open sets by point 1. of \ref{sequentialInteriorClosure}.

By point 2. of \ref{sequentialInteriorClosure}, $\emptyset$ and $X$ are both clopen.

We will prove the rest using closed sets. By point 4. of \ref{sequentialInteriorClosure} finite unions of sequentially closed sets are sequentially closed.

Let $\bigcap_{i\in I}K_i$ be an arbitrary intersection of sequentially closed sets $K_i$. We only need to prove
\[ \operatorname{SeqCl}\left(\bigcap_{i\in I}K_i\right) \subseteq \bigcap_{i\in I}K_i \]
because the other inclusion is immediate. Take an $x\in \operatorname{SeqCl}\left(\bigcap_{i\in I}K_i\right)$. Then there is a sequence in $\bigcap_{i\in I}K_i$ that converges to $x$. Because of the intersection this sequence is in each $K_i$ and thus so is $x$.

The fineness of the topology follows from point 5. of \ref{sequentialInteriorClosure}.
\end{proof}
\begin{corollary}
Let $(X,\mathcal{T})$ be a topological space and $S\subseteq X$  a subset. Then
\[ \text{open/closed} \quad\implies\quad \text{sequentially open/closed.} \]
\end{corollary}

\begin{proposition} \label{sequentialTopologySameConvergentSequences}
Let $(X,\mathcal{T})$ be a topological space and $(x_n)$ a sequence in $X$. Then $x_n\to x$ in $(X,\mathcal{T})$ \textup{if and only if} $x_n\to x$ in $(X,\mathcal{T}_\text{seq})$.
\end{proposition}
\begin{proof}
Now $\mathcal{T}_\text{seq}$ is finer than $\mathcal{T}$, so the $\Leftarrow$ direction is evident. For the $\Rightarrow$ direction, assume $x_n\to x$ in the original topology. Let $V(x)$ be an open neighbourhood in the sequential topology. By definition of the sequential topology $(x_n)$ has a tail in $V$. This means $x_n\to x$ in the sequential topology by \ref{sequenceConvergence}.
\end{proof}

\subsubsection{Transfinite sequential closure}
It is possible that the sequential closure is not idempotent (unlike the normal topological closure), i.e.\
\[ \operatorname{SeqCl}(\operatorname{SeqCl}(S)) \neq \operatorname{SeqCl}(S). \]

\subsubsection{Sequential continuity}
\begin{definition}
A function $f:(X,\mathcal{T})\to(Y,\mathcal{T}')$ is called \udef{sequentially continuous} if
\[ f:(X,\mathcal{T}_\text{seq})\to(Y,\mathcal{T}'_\text{seq}) \]
is continuous. i.e.\ $f$ is continuous when $X,Y$ are equipped with their sequential topologies.
\end{definition}

\begin{proposition} \label{sequentialContinuity}
A function $f:(X,\mathcal{T})\to(Y,\mathcal{T}')$ is sequentially continuous \textup{if and only if} for every sequence $(x_n)_{n\in\N}$ in $X$ and $x\in X$
\[ x_n \to x \;\;\text{in}\; (X,\mathcal{T}) \quad\implies\quad f(x_n)\to f(x) \;\;\text{in}\; (Y,\mathcal{T}'). \]
\end{proposition}
\begin{proof}
First assume this property holds and we want to prove sequential continuity. Let $S\subset Y$ be sequentially closed. Then we need to prove $f^{-1}[S]$ is also sequentially closed. Indeed take a converging sequence $(x_n)$ in $f^{-1}[S]$ with limit $x$. Then $(f(x_n))$ converges to $f(x)$ and $f(x)\in S$. This implies $x\in f^{-1}[S]$, meaning it is sequentially closed. 

Conversely, assume $f$ is sequentially continuous. Let $(x_n)$ be a sequence in $X$ that converges to $x$. Let $V(f(x))\in \mathcal{T}'$ be an open neighbourhood of $f(x)$; $V$ is also sequentially open. Then by continuity we have a $U(x)\in\mathcal{T}_\text{seq}$ such that $f[U]\subseteq V$. Because $U$ is sequentially open, there is an $n_0\in\N$ such that $\forall n\geq n_0: x_n\in U$.
This implies $\forall n\geq n_0: f(x_n)\in f[U]\subseteq V$ and so $(f(x_n))$ converges to $f(x)$.
\end{proof}
\begin{proposition}
Every continuous function is sequentially continuous.
\end{proposition}
\begin{proof}
We use the characterisation of sequential continuity in \ref{sequentialContinuity}. Let $x_n\to x$. Let $V$ be an open neighbourhood of $f(x)$. Then there exists an open neighbourhood $U(x)$ such that $f[U]\subset V$. By \ref{sequenceConvergence} $U$ contains all but finitely many elements of the sequence $(x_n)$. Thus $V$ contains all but finitely many of the elements of the sequence $(f(x_n))$. Take $n_0$ larger than the indices of all elements of $(f(x_n))$ omitted from $V$. By \ref{sequenceConvergence} $f(x_n)\to f(x)$.
\end{proof}

\subsubsection{Sequential spaces}
\begin{definition}
A topological space $(X,\mathcal{T})$ is called a \udef{sequential space} if $\mathcal{T} = \mathcal{T}_\text{seq}$.
\end{definition}

\begin{lemma}
A topological space $(X,\mathcal{T})$ is a sequential space if every sequentially open set is open.
\end{lemma}
\begin{proof}
We already know $\mathcal{T} \subseteq \mathcal{T}_\text{seq}$ from \ref{sequentialTopology}. The hypothesis of the lemma is that $\mathcal{T} \supseteq \mathcal{T}_\text{seq}$. Together this gives $\mathcal{T} = \mathcal{T}_\text{seq}$.
\end{proof}

\begin{lemma}
Let $(X,\mathcal{T})$ be a topological space. Then $X$ equipped with its sequential topology is a sequential space.
\end{lemma}
\begin{proof}
It is enough to show that a sequentially closed set in $(X,\mathcal{T}_\text{seq})$ is also sequentially closed in $(X,\mathcal{T})$. (i.e.\ that passing to the finer topology does not introduce even more sequentially open sets). By \ref{sequentialTopologySameConvergentSequences} the definition of $\operatorname{SeqCl}$ is the same in both topologies, yielding the proof.
\end{proof}

\begin{proposition}
Let $(X,\mathcal{T})$ be a topological space. Then the following are equivalent:
\begin{enumerate}
\item $(X,\mathcal{T})$ is a sequential space;
\item for every subset $S\subset X$ that is not closed in $X$, there exists some $x\in \bar{S}\setminus S$ for which there exists a sequence in $S$ that converges to $x$;
\item $(X,\mathcal{T})$ is the quotient of a first countable space;
\item $(X,\mathcal{T})$ is the quotient of a metric space.
\end{enumerate}
\end{proposition}
TODO: relocate observation about metric spaces.

\begin{proposition}[Universal property of sequential spaces]
Let $(X,\mathcal{T})$ be a topological space. Then $X$ is sequential \textup{if and only if} for every topological space $Y$, a function $f:X\to Y$ is continuous $\Leftrightarrow$ $f$ is sequentially continuous.
\end{proposition}


\subsubsection{$T$-sequential and $N$-sequential spaces}

\subsubsection{Fréchet–Urysohn spaces}
\begin{definition}
A topological space $(X,\mathcal{T})$ is called a \udef{Fréchet–Urysohn space} if for every subset $S\subseteq X$
\[ \operatorname{SeqCl}(S) = \bar{S}. \]
\end{definition}
Clearly every Fréchet–Urysohn space is a sequential space.

\begin{proposition} \label{FrechetUrysohn}
Let $(X,\mathcal{T})$ be a topological space. Then the following are equivalent:
\begin{enumerate}
\item $(X,\mathcal{T})$ is a Fréchet–Urysohn space;
\item every subspace of $X$ is a sequential space;
\item for every subset $S\subset X$ that is not closed in $X$ and for all $x\in \bar{S}\setminus S$ there exists a sequence in $S$ that converges to $x$.
\end{enumerate}
\end{proposition}

\subsection{Sequences in ordered space}
In this section we will be considering sequences in a totally ordered set $(X,\leq)$ equipped with the order topology.

\begin{lemma} \label{convergentSequenceIsBounded}
A convergent sequence in a totally ordered space has an upper and a lower bound.
\end{lemma}
\begin{proof}
Let $x_n\to x$. Choose a basis element containing $x$. If it is of the form $]a,b_0]$ for some greates element $b_0$, then $b_0$ is the upper bound. If not, it is of the form $[a_0,b[$ or $]a,b[$. Find an $n_0$ corresponding to this basis element. Then an upper bound is given by
\[ \max(x[\;[0,n_0]\;]\cup\{b\}). \]
The lower bound is analogous.
\end{proof}

\begin{proposition} \label{limitPreservesInequality}
Let $(a_n)$ and $(b_n)$ be convergent sequences in a totally ordered space such that $a_n\leq b_n$ for all $n\in\N$. Then
\[ \lim_{n\to \infty}a_n \leq \lim_{n\to \infty}b_n. \]
\end{proposition}
\begin{proof}
Let $a_n\to a$ and $b_n\to b$. If $a=b$ then the proposition is valid. Now assume $a\neq b$. If $a$ or $b$ are either the greatest or the least element, the proposition is valid. Now assume this is not the case.

Assume towards a contradiction that $a>b$. Then we can find open neighbourhoods of $a$ and $b$ of the form $]b,d[$ and $]c,a[$, respectively. Now find $n_0, n_1$ such that $\forall n\geq n_0: a_n \in ]b,d[$ and $\forall n\geq n_1: b_n \in ]c,a[$. 
Then for all $n \geq \max\{n_0,n_1\}$ we have $a_n\in ]b,d[$ and $b_n\in ]c,a[$, implying $a_n > b_n$ which is a contradiction.
\end{proof}
It is easy to show that this does not in general hold for the strict inequality $<$.

\begin{proposition}[Squeeze theorem for sequences]
Let $(a_n)$, $(b_n)$ and $(c_n)$ be sequences in a totally ordered space such that
\[ \forall n\in \N: a_n\leq b_n \leq c_n. \]
If $(a_n)$ and $(c_n)$ are convergent with the same limit $L$, then
\[ \lim_{n\to \infty}b_n = L. \]
\end{proposition}
\begin{proof}
Let $V(L)$ be an open neighbourhood of $L$. By definition of the order topology there is an interval $I = ]x,y[ \subset V$ such that $L\in I$. Then find $n_0$ and $n_1$ such that $\forall n\geq n_0: a_n\in I$ and $\forall n\geq n_1: c_n\in I$. Then set $n_2 = \max\{n_0,n_2\}$ and we have $\forall n\geq n_2:$
\[ x \leq a_n \leq b_n \quad \text{and} \quad  b_n \leq c_n \leq y. \]
By transitivity we have $b_n\in I \subset V$.
\end{proof}

\begin{proposition} \label{sequenceToSupInf}
Let $X$ be an ordered space and $A$ a subspace.  Assume the axiom of dependent choice.
\begin{enumerate}
\item If $A$ has a supremum $a$, then there exists a sequence in $A$ that converges to $a$ in $X$.
\item If $A$ has an infimum $b$, then there exists a sequence in $A$ that converges to $b$ in $X$.
\end{enumerate}
\end{proposition}
\begin{proof}
Assume the supremum $a$ of $A$ exists. If $a\in A$ we can take the constant sequence $(a)_{n\in \N}$.

If $a\notin A$, we can find for each $x_i\in A$ an $x_{i+1}$ satisfying $x_i < x_{i+1} < a$. The sequence thus defined converges by monotone convergence.
\end{proof}
In many cases the axiom of dependent choice is superfluous, if the details of the spaces $X,A$ allow for the construction of $x_{i+1}$ from $x_i$.

\subsubsection{Divergence to $\pm\infty$}
\begin{definition}
Let $(x_n)$ be a sequence in a totally ordered space $X$. Then
\begin{itemize}
\item $(x_n)$ \udef{diverges to $+\infty$} if $\forall M\in X: \exists n_0\in\N: \forall n\geq n_0: x_0 > M$; and
\item $(x_n)$ \udef{diverges to $-\infty$} if $\forall M\in X: \exists n_0\in\N: \forall n\geq n_0: x_0 < M$.
\end{itemize}
We write $\lim_{n\to\infty}x_n = +\infty$ and $\lim_{n\to\infty}x_n = -\infty$, respectively.
\end{definition}

\begin{lemma}
Let $(x_n)$ be a sequence in a totally ordered space $X$. Then
\begin{enumerate}
\item if $(x_n)$ is increasing, but not bounded above, it diverges to $+\infty$;
\item if $(x_n)$ is decreasing, but not bounded below, it diverges to $-\infty$.
\end{enumerate}
\end{lemma}

\subsection{Sequences in complete ordered space}
\subsubsection{Monotone convergence}
\begin{proposition}[Monotone convergence] \label{sequenceMonotoneConvergence}
Let $(X,\leq)$ be a complete totally ordered space and let $(x_n)$ be a sequence in $X$.
\begin{enumerate}
\item If $(x_n)$ is increasing and bounded above, then it is convergent with limit $\sup_n x_n$.
\item If $(x_n)$ is decreasing and bounded below, then it is convergent with limit $\inf_n x_n$.
\end{enumerate}
\end{proposition}
\begin{proof}
We prove the first point. The second is analogous.

Let $V(\sup_n x_n)$ be an open and $]x,y[\subset V$ such that $\sup_n x_n \in ]x,y[$. Now because $x<\sup_n x_n$ it is not an upper bound of the sequence and there exists an $x_{n_0}> x$. Because the sequence is increasing (and $y$ is a a strict upper bound), all $x_n$ where $n\geq n_0$ are in $]x,y[\subset V$.
\end{proof}

\subsubsection{Limes superior and inferior}
\begin{definition}
Let $(X,\leq)$ be a complete totally ordered space and let $(x_n)$ be a sequence in $X$. We define
\begin{itemize}
\item the \udef{limes superior} or \udef{limit superior} or \udef{limsup} of $(x_n)$ as
\[ \limsup_{n\to\infty} x_n = \lim_{n\to \infty} \sup\setbuilder{x_m}{m\geq n}; \]
\item the \udef{limes inferior} or \udef{limit inferior} or \udef{liminf} of $(x_n)$ as
\[ \liminf_{n\to\infty} x_n = \lim_{n\to \infty} \inf\setbuilder{x_m}{m\geq n}. \]
\end{itemize}
\end{definition}
The liminf and limsup may not exist.
\begin{lemma}
Let $(X,\leq)$ be a complete totally ordered space and let $(x_n)$ be a sequence in $X$. The limsup and liminf exist \textup{if and only if}  $(x_n)$ is bounded above and below.
\end{lemma}
\begin{proof}
The sequences $\sup\setbuilder{x_m}{m\geq n}$ and $\inf\setbuilder{x_m}{m\geq n}$ are bounded if the limsup and liminf exist and bound $(x_n)$.

The converse follows because the sequences $\sup\setbuilder{x_m}{m\geq n}$ and $\inf\setbuilder{x_m}{m\geq n}$ are monotone.
\end{proof}

\begin{proposition} \label{characterisationLimsupLiminf}
Let $(X,\leq)$ be a complete totally ordered space and let $(x_n)$ be a bounded sequence in $X$. Then 
\begin{enumerate}
\item $L_s = \limsup_{n\to\infty} x_n$ \textup{if and only if}
\begin{align*}
&\forall b > L_s: \exists n_0\in \N:\forall n\geq n_0: x_n < b \qquad \text{and} \\
&\forall a < L_s: \forall n_0\in \N:\exists n\geq n_0: a < x_n
\end{align*}
\item $L_i = \liminf_{n\to\infty} x_n$ \textup{if and only if}
\begin{align*}
&\forall a < L_i: \exists n_0\in \N:\forall n\geq n_0: a < x_n \qquad \text{and} \\
&\forall b > L_i: \forall n_0\in \N:\exists n\geq n_0: x_n < b.
\end{align*}
\end{enumerate}
\end{proposition}

\begin{proposition}
Let $(X,\leq)$ be a complete totally ordered space and let $(x_n)$ be a sequence in $X$. Then $(x_n)$ is convergent \textup{if and only if}
\[ \liminf_{n\to \infty} x_n = \limsup_{n\to \infty} x_n. \]
In this case
\[ \lim_{n\to \infty} x_n = \liminf_{n\to \infty} x_n = \limsup_{n\to \infty} x_n. \]
\end{proposition}
\begin{proof}
Assume $(x_n)$ is a sequence with identical liminf and limsup. Now
\[ \inf\setbuilder{x_m}{m\geq n} \leq x_n \leq \sup\setbuilder{x_m}{m\geq n} \]
so we can apply the squeeze theorem for sequences.

For the converse we use \ref{characterisationLimsupLiminf}.
\end{proof}

\begin{lemma} \label{monotonicityLimsupLiminf}
Let $(a_n)$ and $(b_n)$ be bounded sequences in a totally ordered space such that $a_n\leq b_n$ for all $n\in\N$. Then
\[ \limsup_{n\to \infty}a_n \leq \limsup_{n\to \infty}b_n \quad\text{and}\quad \liminf_{n\to \infty}a_n \leq \liminf_{n\to \infty}b_n. \]
\end{lemma}








\begin{lemma} \label{sequencesSupInf}
There exist sequences converging to supremum and infimum.
\end{lemma}

\subsection{Completeness}


\section{Nets}
\begin{definition}
Let $(I,\leq)$ be a directed set and $X$ a set. Then a \udef{net} in $X$ is a function $I\to X$. The directed set $I$ is called the \udef{index set}.
\end{definition}
In particular sequences are nets because $(\N,\leq)$ is a directed set.

Note we do \emph{not} require $I$ to be a partial order.

\subsection{Relating filters and nets}
\subsubsection{From nets to filters}
\begin{lemma}
Let $X$ be a set, $I$ a directed index set and $\seq{x_i}_{i\in I}$ a net in $X$. Then
\[ \Tails(\seq{x_i}_{i\in I}) \defeq \setbuilder{\setbuilder{x_i}{i \geq j}}{j\in I} \]
is a downward directed subset of $\powerset(X)$.
\end{lemma}
\begin{proof}
Take $A,B\in \Tails(\seq{x_i}_{i\in I})$. If $A = \setbuilder{x_i}{i \geq a}$ and $B = \setbuilder{x_i}{i \geq b}$, then we can find some $k\in I$ such that $k \geq a$ and $k\geq b$. Then $\setbuilder{x_i}{i \geq k}$ is a subset of both $A$ and $B$.
\end{proof}

\begin{definition}
Let $X$ be a set, $I$ a directed index set and $\seq{x_i}_{i\in I}$ a net in $X$. We call
\begin{itemize}
\item $\Tails(\seq{x_i}_{i\in I})$ the \udef{filter base} of $\seq{x_i}_{i\in I}$;
\item the filter generated by $\Tails(\seq{x_i}_{i\in I})$ in $\powerset(X)$ the \udef{associated filter} of $\seq{x_i}_{i\in I}$, which we denote $\TailsFilter(\seq{x_i}_{i\in I})$;
\item the \udef{kernel} of the net the kernel of the associated filter.
\end{itemize}
We have 
\[ \TailsFilter(\seq{x_i}_{i\in I}) = \upset \Tails(\seq{x_i}_{i\in I}). \]
\end{definition}

\subsubsection{From filters to nets}
\begin{lemma} \label{filterIndex}
Let $X$ be a set and $F\in\powerfilters(X)$ a proper filter. Then
\[ I_F \defeq \setbuilder{(A,x)\in F\times X}{x\in A} \]
is a directed proset when ordered by $(A,x)\leq (B,y) \iff B\subseteq A$.
\end{lemma}
\begin{proof}
Transitivity and reflexivity follow from the properties of $\subseteq$.

Take $(A,x)$ and $(B,y)$ in $I_F$. Then $A\cap B\in F$ and $A\cap B \neq \emptyset$ because $F$ is proper. So we can find $z\in A\cap B$. Then $(A,x) \geq (A\cap B,z)$ and $(B,y) \geq (A\cap B, z)$.
\end{proof}

\begin{definition}
Let $X$ be a set, $F\in\powerfilters(X)\setminus\powerset(X)$ and $I_F$ the directed set of $F$ as in \ref{filterIndex}. Then
\[ I_F \to X: (A,x) \mapsto x \]
is the \udef{associated net} of $F$.
\end{definition}

\begin{lemma} \label{tailsFilterIndex}
Let $X$ be a set, $F\in\powerfilters(X)\setminus\powerset(X)$ and $I_F$ the directed set of $F$ as in \ref{filterIndex}. Then for any $(A,x)\in I_F$,
\[ A = \setbuilder{y}{(B,y) \geq (A,x)}. \]
\end{lemma}
\begin{proof}
$\boxed{\subseteq}$ For all $a\in A$ we have $(A,a)\geq (A,x)$.

$\boxed{\supseteq}$ For any $(B,y) \geq (A,x)$, we have $y\in B\subseteq A$, so $y\in A$.
\end{proof}
\begin{corollary}
Let $X$ be a set and $F\in\powerfilters(X)$. Then $F$ equals the filter associated to its associated net.
\end{corollary}
\begin{proof}
By the lemma we have that $A\subseteq X$ is a tail of the net iff it is an element of the filter.
\end{proof}

\subsection{Convergence}
\begin{definition}
Let $\sSet{X,\xi}$ be a convergence space $\seq{x_i}_{i\in I}$ a net on $X$ and $x\in X$. The net $\seq{x_i}_{i\in I}$ \udef{converges} to $x$ in $\xi$, denoted $\seq{x_i}_{i\in I} \overset{\xi}{\longrightarrow} x$, if the associated filter converges to $x$.
\end{definition}

\begin{proposition}
A topological space is Hausdorff \textup{if and only if} every net converges to at most one point.
\end{proposition}

\subsection{Subnets}

TODO: generalise:
\begin{proposition}
Let $X$ be a topological space and $A\subset X$. If there is a sequence of points of $A$ converging to $x$, then $x\in\bar{A}$. The converse holds if $X$ is metrisable.
\end{proposition}



\chapter{Some topologies}
These are the topologies that these object usually posses. If nothing else is said, these topologies will be assumed.

\section{The metric topology}
\begin{definition}
A \udef{metric} on a set $X$ is a function
\[ d:X\times X \to \R \]
with the properties:
\begin{enumerate}
\item $\forall x,y\in X: d(x,y)\geq 0$ and equality holds \textup{if and only if} $x=y$;
\item $\forall x,y\in X: d(x,y)= d(y,x)$;
\item $\forall x,y,z\in X: d(x,y)+d(y,z)\geq d(x,z)$.
\end{enumerate}
\end{definition}
The number $d(x,y)$ is often called the \udef{distance} between $x$ and $y$ in the metric $d$.
\begin{definition}
\begin{itemize}
\item The \udef{$\epsilon$-ball centered at $x$} is the set
\[ B_d(x,\epsilon) = \{y\;|\; d(x,y)< \epsilon\} \]
of all points $y$ whose distance to $x$ is less than $\epsilon$.
\item The \udef{closed $\epsilon$-ball centered at $x$} is the set
\[ \overline{B}_d(x,\epsilon) = \{y\;|\; d(x,y)\leq \epsilon\} \]
of all points $y$ whose distance to $x$ is less than or equal to $\epsilon$.
\item The \udef{$\epsilon$-sphere centered at $x$} is the set
\[ S_d(x,\epsilon) = \{y\;|\; d(x,y) = \epsilon\}. \]
\end{itemize}
\end{definition}

TODO: tolerance space for give $\epsilon$ !!

\begin{lemma}
Let $X$ be a set and $d$ a metric on $X$. Then the collection of all $\epsilon$-balls forms a basis for a topology on $X$.
\end{lemma}

\begin{definition}
\begin{itemize}
\item A \udef{metric space} $(X,d)$ is a set $X$ together with a metric $d$.
\item The topology generated by the $\epsilon$-balls in $X$ is called the \udef{metric topology} generated by $d$.
\item If $(X,\mathcal{T})$ is a topological space such that there exists a metric $d$ on $X$ such that $\mathcal{T}$ is the metric topology, then $X$ is called \udef{metrisable}.
\end{itemize}
\end{definition}

TODO: metric is continuous + use this to define topology????


Only the local behaviour of the metric is important:
\begin{proposition}
Let $(X,d)$ be a metric space. Define
\[ \bar{d}:X\times X\to \R: (x,y)\mapsto \min\{d(x,y),1\}. \]
Then $\bar{d}$ is a metric that induces the same topology as $d$.
\end{proposition}
The metric $\bar{d}$ is called the \udef{standard bounded metric} corresponding to $d$.

\begin{proposition} \label{ballsCoarseness}
Let $d,d'$ be metrics on the set $X$, inducing $\mathcal{T}$ and $\mathcal{T}'$, respectively. Then $\mathcal{T}'$ is finer than $\mathcal{T}$ \textup{if and only if}
\[ \forall x\in X:\forall \epsilon>0:\exists \delta>0:\quad B_{d'}(x,\delta)\subset B_d(x,\epsilon). \]
\end{proposition}
\begin{proof}
Application of lemma \ref{basisCoarseness}.
\end{proof}

\begin{definition}
Let $(X,d)$ be a metric space and $S\subset X$ a subset. We call $S$ \udef{bounded} if there exists a ball $B_d(x,\epsilon)$ such that $S\subseteq B_d$.
\end{definition}

\subsection{Continuous functions in metric spaces}
For maps between metric spaces, the continuity requirement is equivalent to the $\epsilon-\delta$ formulation:
\begin{proposition}
Let $(X,d_X), (Y,d_Y)$ be metric spaces. The continuity of $f:X\to Y$ is equivalent to the condition that, for all $x\in X$:
\[ \forall \epsilon>0:\exists \delta>0:\; d_X(x,y)<\delta \implies d_Y(f(x),f(y))<\epsilon. \]
\end{proposition}
\begin{definition}
Let $f_n:X\to Y$ be a sequence of functions from the set $X$ to the metric space $(Y,d)$. The sequence \udef{converges uniformly} to the function $f:X\to Y$ if
\[ \forall \epsilon>0:\exists N\in \N:\forall n>N:\forall x\in X: \quad d(f_n(x),f(x))<\epsilon.  \]
\end{definition}
\begin{theorem}[Uniform limit theorem]
Let $f_n:X\to Y$ be a sequence of continuous functions from the topological space $X$ to the metric space $Y$. If $(f_n)$ converges uniformly to $f$, then $f$ is continuous.
\end{theorem}
\begin{proof}
We show $f$ is continuous at every point, so choose a point $x_0$ and a neighbourhood $V$ of $f(x_0)$. We then need to show that we can find a neighbourhood $U$ of $x_0$ such that $f(U)\subset V$ Now choose an $\epsilon>0$ such that $B(f(x_0), \epsilon)\subset V$. By uniform convergence, we can find an $N\in\N$ such that
\[\forall n>N:\forall x \in X:\quad d(f_n(x),f(x))<\epsilon/3.\]
By continuity of $f_N$, choose a neighbourhood $U$ of $x_0$ such that $f(U)\subset B(f_N(x_0),\epsilon/3)$. We claim this is the $U$ we need: take an arbitrary $x\in U$, then
\begin{align*}
d(f(x),f(x_0))&\leq d(f(x), f_N(x)) + d(f_N(x), f_N(x_0)) + d(f_N(x_0),f(x_0)) \\
&< \epsilon/3 + \epsilon/3 + \epsilon/3 = \epsilon.
\end{align*}
So $f(U)\subset B(f(x_0), \epsilon)\subset V$.
\end{proof}

TODO: convex sets; distance point to set (as special case of distance set to set).

TODO: A subspace Y of a Banach space X is complete if and only if the set Y is closed in X.

\begin{proposition} \label{distanceToSetContinuous}
Let $\sSet{X,d}$ be a metric space and $A\subseteq X$. Then the function
\[ d_A: X\to \R: x\mapsto d(x,A) = \inf\setbuilder{d(x,a)}{a\in A} \]
is continuous.
\end{proposition}
\begin{proof}
TODO
\end{proof}
TODO: We have that $x\in \overline{A}$ iff $d_A(x) = 0$.

\subsection{Maps between metric spaces}
\begin{definition}
Let $(X,d_X)$ and $(Y,d_Y)$ be metric spaces. A map $f:X\to Y$ is an \udef{isometry} or \udef{distance preserving} if
\[ \forall a,b \in X: \quad d_Y(f(a),f(b)) = d_X(a,b). \]
\end{definition}
\begin{lemma} \label{isometryInjective}
An isometry is automatically injective.
\end{lemma}
\begin{proof}
Let $f: X\to Y$. Assume $f(a) = f(b) = y$, then
\[ 0 = d_Y(y,y) = d_Y(f(a),f(b)) = d_X(a,b). \]
By non-degeneracy of the metric we have $a=b$, meaning $f$ is injective.
\end{proof}
\begin{lemma} \label{isometryContinuous}
An isometry is automatically continuous.
\end{lemma}
\begin{proof}
For an isometry $f:X\to Y$, we have
\[ f[B(x,\epsilon)] = B(f(x),\epsilon), \]
so $f$ is continuous at each point $x$. It is then globally continuous by \ref{globalContinuityFromAllPoints}.
\end{proof}
TODO: merge lemmas?

\begin{lemma} \label{isometryClosed}
Let $f:X\to Y$ be an isometry and $X$ a complete metric space. Then $f$ is a closed map.
\end{lemma}
\begin{proof}
Take $K\subset X$ closed and $y\in\overline{f[K]}$. Then there exists a sequence $(f(x_n))$ in $f[K]$ converging to $y$. The sequence $(f(x_n))$ is convergent and thus Cauchy. Because $d(f(x_n), f(x_m)) = d(x_n,x_m)$, the sequence $(x_n)$ must also be Cauchy. It is convergent because $X$ is complete and the limit $x$ lies in $K$ because it is complete (TODO ref). By continuity of $f$, \ref{isometryContinuous}, we have
\[ y = \lim_{n\to\infty} f(x_n) = f(\lim_{n\to\infty}x_n) = f(x) \in f[K]. \]
So $\overline{f[K]} = f[K]$ and $f$ is closed.
\end{proof}

\subsection{Cauchy sequences and completeness}
\begin{definition}
Let $(X,d)$ be a metric space and $(x_n)$ a sequence in $X$. Then $(x_n)$ is called a \udef{Cauchy sequence} if
\[ \forall 0 < \epsilon \in \R: \exists n_0\in \N: \forall m,n \geq n_0: d(x_m,x_n) < \epsilon.  \]
\end{definition}

\begin{lemma}
Let $(X,d)$ be a metric space. Cauchy sequences in $X$ are bounded.
\end{lemma}

\begin{proposition}
Let $(X,d)$ be a metric space. Convergent sequences in $X$ are Cauchy.
\end{proposition}
Spaces in which the converse holds are special.
\begin{definition}
Let $(X,d)$ be a metric space, then $X$ is called \udef{(Cauchy) complete} if all Cauchy sequences converge.
\end{definition}

\begin{proposition} \label{CauchyCriterion}
Let $(X,d)$ be a metric space and $(a_n), (b_n)$ sequences in $X$. If $(b_n)$ is Cauchy and there exists some $A\in\R$ such that
\[ \forall m,n\in\N: d(a_n,a_m) \leq A d(b_n,b_m), \]
then $(a_n)$ is also Cauchy.
\end{proposition}

\begin{proposition}[Completeness criterion] \label{completenessCriterion}
Let $(X,d)$ be a metric space and $S\subset X$ a dense subset. If every Cauchy sequence in $S$ converges in $X$, then $X$ is complete.
\end{proposition}
This proposition depends on the axiom of countable choice.
\begin{proof}
TODO
\end{proof}

\begin{lemma}
The real numbers with the standard topology are complete.
\end{lemma}

\subsubsection{Completion}
\begin{proposition}
Let $(X,d_X)$ be a metric space. There exists a complete metric space $(Y,d_Y)$ and an isometry $\pi:X\hookrightarrow Y$ such that $\pi[X]$ is a dense subspace of $Y$.
\end{proposition}
We view $X$ as a subspace of $Y$ through $\pi$ and call $Y$ the \udef{completion} of $X$.
\begin{proof}
Let $Y'$ be the space of Cauchy sequences in $X$. Introduce the equivalence relation on $\seq{x_i},\seq{y_j}\in Y'$:
\[ \seq{x_i} \sim \seq{y_i} \qquad \iff\qquad \lim_{i\to\infty} d_X(x_i,y_i) = 0. \]
Let $Y$ be the set of equivalence classes in $Y'$ under this equivalence relation. Define
\[ d_Y: Y\times Y\to \R: ([\seq{x_i}],[\seq{y_i}]) \mapsto \lim_{i\to\infty}d_X(x_i,y_i) \qquad\text{and}\qquad \pi: X\to Y: x\mapsto \seq{x}_i. \]
We need to show that $d_Y$ is well-defined, that it is a metric on $Y$, that $\pi[X]$ is dense in $Y$ and that $(Y,d_Y)$ is complete:
\begin{itemize}
\item Let $[\seq{x_i'}] = [\seq{x_i}]$. Then
\begin{align*}
d_Y([\seq{x_i'}],[\seq{y_i}]) &= \lim_{i\to\infty}d_X(x'_i,y_i) = \lim_{i\to\infty}d_X(x'_i,y_i) + \lim_{i\to\infty}d_X(x_i,x'_i) \\
&= \lim_{i\to\infty}d_X(x_i,x_i')+d_X(x'_i,y_i) \\
&\geq \lim_{i\to\infty}d_X(x_i,y_i) = d_Y([\seq{x_i}],[\seq{y_i}]).
\end{align*}
Similarly we can show $d_Y([\seq{x_i'}],[\seq{y_i}])\leq d_Y([\seq{x_i}],[\seq{y_i}])$, so $d_Y([\seq{x'_i}],[\seq{y_i}]) = d_Y([\seq{x_i}],[\seq{y_i}])$.

We must also show that the domain and codomain of $d_Y$ make sense, i.e.\ the limit exists and does not diverge. It is enough to show that $(d_X(x_i,y_i))$ is a Cauchy sequence, due to the completeness of $\R$. To this end, let $\epsilon>0$. As $\seq{x_i}$ and $\seq{y_i}$ are Cauchy, we can find $N_x,N_y\in\N$ such that $d_X(x_m,x_n)< \epsilon/2$ and $d_X(y_m,y_n) < \epsilon/2$ for all $m,n \geq N_x,N_y$. Then $\forall m,n \geq \max\{N_x,N_y\}$:
\begin{align*}
|d_X(x_m,y_m) - d_X(x_n,y_n)| &\leq |d_X(x_m,x_n)+d_X(x_n,y_m) - d_X(x_n,y_n)| \\
&\leq |d_X(x_m,x_n)+d_X(y_m,y_n)+ d_X(y_n,x_n) - d_X(x_n,y_n)| \\
&= |d_X(x_m,x_n)+d_X(y_m,y_n)| \\
&< \epsilon/2 + \epsilon/2= \epsilon.
\end{align*}
So $(d_X(x_i,y_i))$ is Cauchy and thus converges in $\R$.
\item That $d_Y$ is a metric is easy to check.
\item To prove $\pi[X]$ is dense in $Y$, we just need to show that every element $y = [\seq{x_i}]\in Y$ is the limit of a sequence in $\pi[X]$, because all metric spaces are sequential. We claim $\seq{\pi(x_j)}_j$ converges to $y$.

Let $\epsilon>0$. Because $\seq{x_i}$ is Cauchy, we can find an $N\in\N$ such that $\forall m,n>N: d_X(x_m,x_n) < \epsilon/2$. Take $j\geq N$ arbitrary. Then
\[ \forall i\geq N: d_X(x_i,x_j) < \epsilon/2 \quad \implies\quad \lim_{i\to\infty} d_X(x_i,x_j) = d_Y([\seq{x_i}],\seq{\pi(x_j)}_j) \leq \epsilon/2  < \epsilon. \]
\item For completeness, it is enough, by \ref{completenessCriterion}, to show that Cauchy sequences in $\pi[X]$ converge in $Y$.

Let $\seq{\pi(x_j)}_j$ be a Cauchy sequence in $\pi[X]$, then $\seq{x_i}$ is Cauchy in $X$ because $\pi$ is isometric. So $\seq{\pi(x_j)}_j$ converges to $\seq{x_i}$ by the previous point.
Let $\seq{\pi(x_j)}_j$ be a Cauchy sequence in $\pi[X]$, then $\seq{x_i}$ is Cauchy in $X$ because $\pi$ is isometric. So $\seq{\pi(x_j)}_j$ converges to $\seq{x_i}$ by the previous point.
\end{itemize}
\end{proof}

\begin{proposition}
Let $(X,d)$ be a metric space. The completion $(Y,\pi)$ of $X$ is unique in the following sense: for any other such completion $(Y',\pi')$, there exists a unique isometric isomorphism $\theta:Y\to Y'$ satisfying $\theta\circ \pi = \pi'$.
\end{proposition}
\begin{proof}
Since $\pi$ is an isometry, it is injective, so $\pi^{-1}:\pi[X]\to X$ is a surjective isometry and so $\pi'\circ\pi^{-1}:\pi[X]\to\pi'[X]$ is too. Now we must have $\theta|_{\pi[X]} = \pi'\circ\pi^{-1}:\pi[X]\to\pi'[X]$ 

TODO universal property!!
\end{proof}



\subsection{Equicontinuity}
Cfr uniform limit theorem
\begin{definition}
Let $(X,d_X)$ and $(Y,d_Y)$ be metric spaces and let $\mathcal{F}$ be a family of functions in $(X\to Y)$.
We say
\begin{itemize}
\item $\mathcal{F}$ is an \udef{equicontinuous family} at $x'\in X$ if
\[ \forall \epsilon> 0: \forall x\in X: \exists \delta>0: \forall f\in\mathcal{F}:\; d_X(x,x') < \delta \implies d_Y(f(x),f(x')) < \epsilon; \]
\item $\mathcal{F}$ is a \udef{uniform equicontinuous family} at $x'\in X$ if
\[ \forall \epsilon> 0: \exists \delta>0: \forall x\in X: \forall f\in\mathcal{F}:\; d_X(x,x') < \delta \implies d_Y(f(x),f(x')) < \epsilon. \]
\end{itemize}
We say $\mathcal{F}$ is (uniformly) equicontinuous if it is (uniformly) equicontinuous at all $x'\in X$.
\end{definition}

\begin{itemize}
\item For continuity, $\delta$ may depend on $\epsilon,f,x_0$;
\item For uniform continuity, $\delta$ may depend on $\epsilon$ and $f$;
\item For pointwise equicontinuity, $\delta$ may depend on $\epsilon$ and $x_0$;
\item For uniform equicontinuity, $\delta$ may depend only on $\epsilon$.
\end{itemize}

TODO: generalise to $X$ general topological space (esp for TVSs).

\begin{proposition}
Let $(f_n)_{n\in\N}$ be an equicontinuous family between metric spaces. If $f_n\to f$ pointwise, then $f$ is continuous.
\end{proposition}
\begin{proof}
TODO
\end{proof}

TODO Reed / Simon

\subsection{Hölder and Lipschitz continuity}
\begin{definition}
Let $f: X\to Y$ be a map between metric spaces, then $f$ is called \udef{$\alpha$-Hölder continuous}, where $0 < \alpha \leq 1$, if there exists a constant $M$ such that
\[ d_Y(f(x), f(y)) \leq M d_X(x,y)^\alpha \qquad \forall x,y\in X. \]

If $\alpha = 1$, then $f$ is called \udef{Lipschitz continuous}.

The set of $\alpha$-Hölder continuous functions $X\to Y$ is denoted $\cont^{0,\alpha}(X,Y)$.
\end{definition}
The $0$ in $\cont^{0,\alpha}(X,Y)$ appears because we are not considering any derivatives.

\begin{lemma} \label{HolderLipschitzContinuity}
Let $f: X\to Y$ be a map between metric spaces and $0<\alpha \leq \beta \leq 1$.

If $f$ is $\beta$-Hölder continuous, then it is $\alpha$-Hölder continuous.
\end{lemma}
TODO consolidate lemmas + uniform continuity.
\begin{lemma} \label{LipschitzcontinuousContinuous}
A Lipschitz continuous function between metric spaces is continuous.
\end{lemma}
\begin{proof}
    Let $f$ be a Lipschitz continuous function and $\seq{x_n}$ a sequence such that $x_n \to x$. Then
    \begin{align*}
        d\left(\lim_{n\to\infty}f(x_n), f(x)\right) &= \lim_{n\to\infty}d(f(x_n), f(x)) \\
        &\leq \lim_{n\to\infty}M d(x_n, x) = M d\left(\lim_{n\to\infty}x_n, x\right) = M d(x,x) = 0. 
    \end{align*} 
\end{proof}

The converse in not necessarily true. TODO example.

\subsubsection{Contractions}
\begin{definition}
Let $f: X\to Y$ be a map between metric spaces, then $f$ is called a \udef{contraction} if it is Lipschitz continuous with Lipschitz constant $M < 1$.
\end{definition}

\begin{proposition} \label{contractionFixedPoint}
Let $f: X\to X$ be a contraction. If $X$ is a complete metric space, then $f$ has a unique fixed point.
\end{proposition}
\begin{proof}
Uniqueness is easy: assume that $f$ has two fixed points $x_1, x_2$. Then $d(x_1,x_2) = d(fx_1, fx_2) \leq M d(x_1,x_2)$. Since $M < 1$ this is only possible if $d(x_1,x_2) = 0$, meaning $x_1  x_2$.

For existence: take some $x_0\in X$. Then define the sequence $\seq{T^n(x_0)}_n$. This is a Cauchy sequence because, for $m>n>1$
\[ d(x_m,x_n) \leq \sum_{i=n}^{m-1}d(x_{i+1}, x_i) \leq \sum^{m-1}_{i=n}M^{i-1}d(x_2,x_1) = \frac{M^{n-1}(1-M^{m-n+1})}{1-M}d(x_2,x_1) \leq \frac{M^{n-1}}{1-M}. \]
By completeness it has a limit. Now $T$ is continuous by \ref{LipschitzcontinuousContinuous}. Then
\[ T\left(\lim_{n\to \infty} T^n(x_0)\right) = \lim_{n\to \infty}T(T^n(x_0)) = \lim_{n\to \infty}T^{n+1}(x_0) = \lim_{n\to \infty} T^n(x_0),  \]
so the limit is a fixed point.
\end{proof}

The construction of the sequence $\seq{T^n(x_0)}_n$ is called \udef{fixed point iteration}. Following the proof of the proposition it is clear we can obtain the fixed point starting the iteration from any point in $X$.

\begin{corollary}
Let $f: X\to X$ be a function on a complete metric space such that $f^n$ is a contration for some $n\in \N$. Then $f$ has a unique fixed point.
\end{corollary}
\begin{proof}
By the proposition we know that $f^n$ has a unique fixed point: $f^n(x) = x$. Applying $f$ to both sides gives
\[ f(f^n(x)) = f^n(f(x)) = f(x), \]
so $f(x)$ is also a fixed point of $f^n$. By uniqueness $f(x) = x$. This shows $f$ has a fixed point.

For uniqueness it is enough to note that any fixed point of $f$ is a fixed point of $f^n$, \ref{fixedPointsMultipleComposition}.
\end{proof}

\subsection{Pseudometric spaces}
\begin{definition}
Let $X$ be a set.
\begin{itemize}
\item A map $p: X\times X\to \R_{\geq 0}$ is called a \udef{pseudometric} or \udef{semimetric} if
\begin{itemize}
\item $\forall x\in X: p(x,x) = 0$;
\item $\forall x,y\in X: p(x,y) = p(y,x)$;
\item $\forall x,y,z\in X: p(x,z)\leq p(x,y)+p(y,z)$.
\end{itemize}
Unlike a metric space, points in a pseudometric space need not be distinguishable; that is, one may have $d(x,y)=0$ for distinct values $x\neq y$.
\item The pair $(X,p)$ is called a \udef{pseudometric space}.
\item The \udef{pseudometric topology} is the topology generated by the basis of open balls
\[ B_p(x_0, \epsilon) = \setbuilder{x\in X}{p(x_0,x)<\epsilon}. \]
A topological space is said to be a \udef{pseudometrizable space} if it can be given a pseudometric such that the pseudometric topology coincides with the given topology on the space.
\end{itemize}
\end{definition}

\begin{proposition}
The notions of compactness, limit point compactness
and sequential compactness are equivalent in a pseudometric space.
\end{proposition}

\url{https://link.springer.com/article/10.1007/BF01351999}

\section{Topologies of $\R$}
\begin{definition}
The \udef{lower limit topology} on $\R$ is the topology generated by the basis
\[ \{ [a,b[ \;|\; a< b \}. \]
\end{definition}
\begin{definition}
\begin{itemize}
\item Let $K$ denote the set
\[ K = \{1/n \;|\; n\in \mathbb{N}_0\}. \]
\item The \udef{$K$-topology} on $\R$ is the topology generated by the basis
\[ \{ ]a,b[ \;|\; a< b \}\;\cup\;\{ ]a,b[\setminus K \;|\; a< b \}. \]
\end{itemize}
\end{definition}
\begin{lemma}
The lower limit and $K$-topologies are strictly finer than then standard topology on $\R$, but are not comparable with one another.
\end{lemma}
\begin{proposition}
Let $X$ be a topological space and $f,g:X\to \R$ continuous functions, then
\begin{enumerate}
\item $f+g, f-g$ and $f\cdot g$ are continuous.
\item If $g(x)\neq 0$ for all $x\in X$, then $f/g$ is continuous.
\end{enumerate}
\end{proposition}
\begin{proof}
First define the map
\[ h:X\to \R\times\R: x\mapsto (f(x),g(x)) \]
which is continuous by proposition \ref{continuityCompositeFunctions}. The functions $f+g,f-g,f\cdot g, f/g$ are the composition of $h$ and the continuous functions $+,-,\cdot,/$.
\end{proof}

\section{Uniform topology}
\begin{definition}
Given an index set $J$ and points $\vec{x}=(x_i)_{i\in J}$ and $\vec{y}=(y_i)_{i\in J}$ of $\R^J$. Define the metric $\bar{\rho}$ on $\R^J$ by
\[ \bar{\rho}(\vec{x}, \vec{y}) = \sup\{\bar{d}(x_i,y_i)\;|\; i\in J\}, \]
where $\bar{d}$ is the standard bounded metric on $\R$. The metric space $(\R^J, \bar{\rho})$ is the \udef{uniform topology} on $\R^J$ and $\bar{\rho}$ is the \udef{uniform metric}.
\end{definition}
\begin{proposition}
On the set $\R^J$, the box topology is finer than the uniform topology is finer than the product topology. These topologies are all different \textup{if and only if} $J$ is infinite.
\end{proposition}

\section{Set-theoretic topology}
\url{https://en.wikipedia.org/wiki/Set-theoretic_limit}
\url{https://math.stackexchange.com/questions/3384916/topology-of-set-theoretic-limits}
\url{https://math.stackexchange.com/questions/2799181/is-there-a-way-to-express-the-set-theoretic-limit-in-terms-of-topology-filters}
\url{https://math.stackexchange.com/questions/97440/do-limits-of-sequences-of-sets-come-from-a-topology}
\url{https://math.stackexchange.com/questions/1947171/the-topology-of-sets}

\section{Initial and final topologies}

\chapter{More topological constructions}
\section{Fibre bundles}
\section{Cones and suspensions}
\section{Wedge sum and smash product}

\chapter{Convergence and topology on algebraic structures}
\section{Convergence relational structures}
\section{Convergence order structures}
\subsection{Difference operators}
TODO

\chapter{Convergence groups}
\section{Convergence}
\begin{definition}
Let $G$ be a set and
\begin{itemize}
\item $\boldsymbol{\cdot}: G\times G \to G$ a binary operation such that $\sSet{G, \boldsymbol{\cdot}}$ is a group;
\item $\xi$ a relation on $(\powerset(\powerset(G)), G)$ such that $\sSet{G, \xi}$ is a convergence space;
\end{itemize}
such that
\begin{itemize}
\item $\boldsymbol{\cdot}: G\times G \to G$ is continuous; and
\item $^{-1}: G\to G: x\mapsto x^{-1}$ is continuous.
\end{itemize}
Then we call $\sSet{G, \boldsymbol{\cdot}, \xi}$ a \udef{convergence group}.
\end{definition}

\begin{lemma} \label{convergenceGroupCriterion}
A group $G$ with a convergence structure is a convergence group \textup{if and only if}
\[ G\times G \to G: (x,y) \mapsto xy^{-1} \]
is continuous.
\end{lemma}
\begin{proof}
If $G$ is a convergence group, the function $(x,y) \mapsto xy^{-1}$ is the composition of two continuous functions and thus continuous.

Conversely, assume $(x,y) \mapsto xy^{-1}$ continuous. Then $y \mapsto 1y^{-1} = y^{-1}$ is continuous by \ref{continuousEmbeddingProduct}. Then $\boldsymbol{\cdot}: (x,y) \mapsto xy = x(y^{-1})^{-1}$ is a composition of continuous functions.
\end{proof}

\begin{lemma} \label{closureGroupOperation}
Let $\sSet{G, \boldsymbol{\cdot}, \xi}$ be a convergence group and $A,B$ subsets of $G$. Then
\[ \adh_\xi(A)\cdot \adh_\xi(B) \subseteq \adh_\xi{A\cdot B}. \]
\end{lemma}
\begin{proof}
From the continuity of $\cdot$ and \ref{adherenceInherenceContinuity} together with $\adh_{\xi\otimes\xi}(A\times B) = \adh_\xi(A)\times \adh_\xi(B)$ (TODO ref).
\end{proof}

\begin{lemma} \label{shiftHomeomorphism}
Let $\sSet{G,\cdot, 1, \xi}$ be a convergence group.
\begin{enumerate}
\item For all $a\in G$, both
\[ \lambda_a: G\to G: x\mapsto ax \qquad\text{and}\qquad \rho_a: G\to G: x\mapsto xa \]
are homeomorphisms.
\item $F \to x$ \textup{if and only if} $F\cdot x^{-1} \to 1$ \textup{if and only if} $x^{-1}\cdot F\to 1$.
\item $\vicinity_\xi(x) = \vicinity_\xi(1) \cdot x = x\cdot \vicinity_\xi(1)$.
\item Let $f: \sSet{G,\cdot, 1, \xi} \to \sSet{H,\cdot, 1, \zeta}$ be a group homomorphism. Then $f$ is continuous \textup{if and only if} it is continuous at $1$.
\end{enumerate}
\end{lemma}
\begin{proof}
(1) The functions are clearly bijective. They are also continuous because the constant function $\underline{a}$ is continuous as well as the multiplication $(x,y)\mapsto xy$. Thus the composition is continuous.

(2) Assume $F\to x$, then by continuity of $\rho_{x^{-1}}$, we get $F\cdot x^{-1} \to 1$. The converse follows from continuity of $\rho_x$. The second equivalence follows from the continuity of $\lambda^{x}$ and $\lambda_{x^{-1}}$.

(3) We calculate
\begin{align*}
\vicinity_\xi(x) &= \bigcap \setbuilder{F\in \powerfilters(X)}{x\in \lim_\xi F} \\
&= \bigcap \setbuilder{F\in \powerfilters(X)}{1\in \lim_\xi F\cdot x^{-1}} \\
&= \bigcap \setbuilder{G\cdot x\in \powerfilters(X)}{1\in \lim_\xi G} \\
&= \bigcap \setbuilder{G\in \powerfilters(X)}{1\in \lim_\xi G}\cdot x = \vicinity_\xi(1)\cdot x.
\end{align*}
TODO $\rho_x$ homeomorphism.

(4) If $f$ is continuous, it is automatically continuous at $1$. Now assume $f$ continuous at $1$ and let $x\in G$. Then
\[ F\to x \iff F\cdot x^{-1} \to 1 \implies f[F\cdot x^{-1}] = f[F]\cdot f(x)^{-1} \to f(1) = 1 \iff f[F] \to f(x). \]
\end{proof}

The convergence structure of a convergence group is completely determined by $\lim^{-1}(1)$. Thus the following lemma gives a way to generate convergence groups.

\begin{proposition} \label{groupConvergenceConstruction}
Let $\sSet{G, +, 0}$ be a commutative group. And $\mathcal{G} \subseteq \powerfilters(G)$ a family of filters. There exists a convergence $\xi$ on $G$ such that $\mathcal{G} = \lim^{-1}_\xi(0)$ \textup{if and only if}
\begin{enumerate}
\item $\pfilter{0} \in \mathcal{G}$;
\item if $F \in \mathcal{G}$ and $G\supseteq F$, then $G\in \mathcal{G}$;
\item if $F,G\in \mathcal{G}$, then $F - G\in \mathcal{G}$.
\end{enumerate}
\end{proposition}
The group convergence is completely determined by $\lim^{-1}_\xi(0)$ due to the translation homeomorphism \ref{shiftHomeomorphism}.
\begin{proof}
It is clear that $F \to x$ iff $F - x \in \mathcal{G}$ determines a convergence. We just need to show that $u: (x,y) \mapsto x - y$ is continuous.

Let $F \to (x,y) \in G\times G$, so by \ref{convergenceProductFilter} there exist $F_1\to x$ and $F_2 \to y$ such that $F_1\otimes F_2 \leq F$. Then $F_1 - x \in \mathcal{G}$ and $F_2 - y \in \mathcal{G}$. From point (3) (and commutativity) we get $F_1 - F_2 - (x - y) \in \mathcal{G}$, so $F_1 - F_2 \to x-y$. Now $F_1 - F_2 = u[F_1\otimes F_2] \leq u[F]$ by \ref{filterFactorisationInequality}, so $u[F] \to x-y$ and thus $u$ is continuous.
\end{proof}

\begin{proposition} \label{HausdorffCriterionConvergenceGroup}
Let $\sSet{G,\cdot, 1, \xi}$ be a convergence group. Then $G$ is Hausdorff \textup{if and only if} $\{1\}$ is closed.
\end{proposition}
\begin{proof}
The direction $\Rightarrow$ is clear, since every Hausdorff convergence is $T_1$ and in a $T_1$ convergence all singletons are closed.

Conversely, assume $F\to x,y$ in $G$. Then $FF^{-1} \to xy^{-1}$. Now $FF^{-1} \subseteq \pfilter{1}$, so $\pfilter{1} \to xy^{-1}$ and thus $xy^{-1}\in \adh_\xi(\pfilter{1}) = \adh_\xi(\{1\}) = \{1\}$, meaning $x = y$.
\end{proof}

\begin{lemma} \label{vicinityFactorisation}
Let $\sSet{G,\cdot, 1, \xi}$ be a pretopological convergence group and $x,y\in G$. If $U\in \vicinity_\xi(xy)$, then there exist $V\in \vicinity_\xi(x)$ and $W\in \vicinity_\xi(y)$ such that $V\cdot W\subseteq U$.

If $x=y$, then we can take $V = W$.
\end{lemma}
\begin{proof}
Consider the function $f: G\times G \to G: (x,y)\mapsto xy$. Then by \ref{pretopologicalContinuityVicinities} and \ref{productVicinity} we have
\[ \vicinity_\xi(xy) \subseteq \upset f[\vicinity_{\xi\otimes\xi}((x,y))] = \upset f[\upset \vicinity_\xi(x)\otimes \vicinity_\xi(y)] = \upset f[\vicinity_\xi(x)\otimes \vicinity_\xi(y)] = \upset (\vicinity_\xi(x)\cdot \vicinity_\xi(y)) \]
This implies the first result.

If $x=1=y$, then consider $V\cap W$. This is still a neighbourhood of $x=y$ and $(V\cap W)\cdot(V\cap W) \subseteq V\cdot W \subseteq U$.
\end{proof}

\begin{lemma} \label{symmetricBase}
Let $\sSet{G,\cdot, 1, \xi}$ be a pretopological convergence group, then $\vicinity_\xi(1)$ is based in the symmetric subsets.
\end{lemma}
Symmetric subsets are subsets $V$ such that $V^{-1} = V$.
\begin{proof}
Because $^{-1}$ is a group homeomorphism, we have $\vicinity_\xi(1) = \vicinity_\xi(1^{-1}) = (\vicinity_\xi(1))^{-1}$. So $V$ is a vicinity of $1$ iff $V^{-1}$ is a vicinity of $1$. Thus $V\cap V^{-1}\subseteq V$ is a vicinity of $1$ and $V\cap V^{-1}$ is also a symmetric set.
\end{proof}

\begin{proposition} \label{pretopologicalGroupConvergence}
Each pretopological convergence group is topological.
\end{proposition}
\begin{proof}
Let $\sSet{G,\cdot, 1, \xi}$ be a pretopological convergence group. To prove the convergence it topological, it is enough to prove that $\inh_\xi(A) \subseteq \inh^2_\xi(A)$ for all $A\subseteq X$. Fix such an $A$ and take an arbitrary $x\in \inh_\xi(A)$. Then $A\in \vicinity_\xi(x)$ by \ref{principalAdherenceInherence}.

By \ref{vicinityFactorisation} there exist $V\in \vicinity_\xi(x)$ and $W\in\vicinity_\xi(1)$ such that $V\cdot W \subseteq A$.

Now for all $y\in V$, $y\cdot W$ is a vicinity of $y$ by \ref{shiftHomeomorphism}, so $V \subseteq \inh_\xi(A)$ by \ref{subsetWithVicinitiesInInherence}. By the upward closure of the vicinity filter, $V\in \vicinity_\xi(x)$ implies $\inh_\xi(A) \in \vicinity_\xi(x)$. Thus $x\in \inh_\xi^2(A)$ by \ref{principalAdherenceInherence}.
\end{proof}

\begin{proposition}
Every topological group is regular.
\end{proposition}
\begin{proof}
By \ref{topologicalRegularity} we check that for all open $U$ and $x\in U$ there exists an open set $V$ such that $x\in V\subseteq \overline{V}\subseteq U$. In fact it is enough to check this for $e = 1$.\

Because $1\cdot 1 = 1$, we can find $W\in\neighbourhood(1)$ such that $W\cdot W \subseteq U$ by \ref{vicinityFactorisation}. We claim $V= W\cap W^{-1}$ works. Indeed it is an open neighbourhood of $1$ and clearly $V\subseteq U$. We just need to show that $\overline{V}\subseteq U$. Take $y\in \overline{V}$. Then $yV\in \vicinity_\xi(y)$ by \ref{shiftHomeomorphism} and $V\in \vicinity_\xi(y)^{\mesh}$ by \ref{principalAdherenceInherence}. So $yV\mesh V$ and we can find $v_1,v_2\in V$ such that $v_1 = yv_2$. Thus
\[ y = v_1v_2^{-1} \in V\cdot V^{-1} = V\cdot V \subseteq U. \]
\end{proof}
TODO: in fact completely regular.

\begin{proposition}
Let $G$ be a convergence group. Then the pseudotopological modification $\chi(G)$ is a convergence
group.
\end{proposition}
(This is in general not true for the pretopological modification).
\begin{proposition} \label{initialConvergenceGroup}
Let $G$ be a group, $\{G_i\}_{i\in I}$ a set of convergence groups and $\{f_i: G \to G_i\}_{i\in I}$ a set of group homomorphisms. Then the initial convergence on $G$ w.r.t. $\{f_i: G \to G_i\}_{i\in I}$ makes $G$ a convergence group.
\end{proposition}
\begin{proof}
By \ref{convergenceGroupCriterion} we just need to verify that $u: G\times G \to G: (x,y)\mapsto x\cdot y^{-1}$ is continuous. Using \ref{characteristicPropertyInitialFinalConvergence}, we need to verify that $f_i\circ u$ is continuous for all $i\in I$. Because the $f_i$ are group homomorphisms, we have
\[ f_i(x\cdot y^{-1}) = f_i(x)\cdot f_i(y)^{-1} \]
for all $x, y \in G$. This means that $f_i\circ u = u_i \circ (f_i\times f_i)$, where $u_i: G_i\times G_i \to G_i: (x,y)\mapsto x\cdot y^{-1}$.

By \ref{productContinuousFunctions} this is a composition of continuous functions and thus continuous.
\end{proof}

\begin{proposition}
Let $G$ be a convergence group. Any subgroup $H\in\vicinity(1)$ is closed.
\end{proposition}
\begin{proof}
Take $g\in \adh(H)$, so $H\in \vicinity(g)^{\mesh}$ by \ref{principalAdherenceInherence}. Now, by \ref{homeomorphismPreservation}, $gH\in \vicinity(g)$ because $\lambda_g$ is a homeomorphism by \ref{shiftHomeomorphism}. Thus $gH \mesh H = 1\cdot H$. As cosets are either the same or disjoint (by \ref{differentCosetsDisjoint}), we have $gH = H$ and in particular $g = g\cdot 1 \in gH = H$. So $\adh(H) = H$.
\end{proof}

\begin{proposition}
Let $G$ be a convergence group and $A,B\subseteq G$ subsets.
\begin{enumerate}
\item If $A$,$B$ are compact, then $A\cdot B$ is compact.
\item If $A$ is open, then $A\cdot B$ is open.
\item If $A$ is closed and $B$ is compact, then $A\cdot B$ is closed.
\end{enumerate}
\end{proposition}
\begin{proof}
TODO

(1) Continuous image of compact is compact.

(2) $A\cdot B = \bigcup_{b\in B} A\cdot b$ is a union of open sets and thus open.

(3)
\end{proof}

\subsection{Locally compact groups}
\begin{lemma}
A topological convergence group is locally compact \textup{if and only if} $\vicinity(1)$ has a base of compact sets.
\end{lemma}
\begin{proof}
TODO 

It is regular, so is based in closed sets. There is a compact vicinity of $1$. All closed subsets of this vicinity are compact.
\end{proof}

\section{Cauchy structure}
\begin{proposition} \label{groupCauchyStructure}
Let $\sSet{G, \cdot, 1, \xi}$ be a convergence group. Consider the family
\[ \mathcal{F} \defeq \setbuilder{F\in \powerfilters(G)}{F\cdot F^{-1} \overset{\xi}{\longrightarrow} 1}. \]
Then
\begin{enumerate}
\item $\sSet{G, \mathcal{F}}$ is a Cauchy space;
\item if $F\in \powerfilters(G)$ Cauchy converges to $x$, then it converges to $x$;
\item if $F\in \powerfilters(G)$ converges to $x$, then $F\in \mathcal{F}$; if $G$ is a Kent space, then $F$ Cauchy converges to $x$.
\end{enumerate}
\end{proposition}
\begin{proof}
(1) We have $\pfilter{x}\cdot \pfilter{x}^{-1} \supseteq \pfilter{1}$, so $\pfilter{x}\cdot \pfilter{x}^{-1} \to 1$ and thus $\pfilter{x} \in \mathcal{F}$.

Take $F\in \mathcal{F}$ and consider $H\supseteq F$. Clearly $F\cdot F^{-1} \subseteq H\cdot H^{-1}$, so $H\cdot H^{-1} \to 1$ and thus $H\in \mathcal{F}$.

(2) Notice that
\[ (F\cap \pfilter{x})\cdot(F\cap \pfilter{x})^{-1} = F\cdot F^{-1} \cap F\cdot \pfilter{x}^{-1} \cap \pfilter{x}\cdot F^{-1} \cap \pfilter{x}\cdot\pfilter{x}^{-1}. \]
So if $F$ Cauchy converges to $x$, then $F\cap\pfilter{x} \in \mathcal{F}$ and this filter converges to $1$. In particular this means that $F\cdot\pfilter{x}^{-1} \to 1$. Then we use $F\cdot\pfilter{x}^{-1} = F\cdot x^{-1}$ (TODO ref?) to conclude that $F \to x$.

(3) Assume $F\to x$. Then by \ref{convergenceFiniteProductFilter}, $F\otimes F \to (x,x)$. By continuity of $(x,y)\mapsto x\cdot y^{-1}$, we have $F\cdot F^{-1}\to x\cdot x^{-1} = 1$, so $F$ is a Cauchy filter.

Now assume $G$ is a Kent space. Then $F\cap \pfilter{x} \to x$, meaning $F\cap \pfilter{x} \in \mathcal{F}$ by the previous argument and so $F$ Cauchy converges to $x$.
\end{proof}


\begin{definition}
Let $\sSet{G, \cdot, 1, \xi}$ be a convergence group. The family $\mathcal{F}$ of \ref{groupCauchyStructure} is the \udef{associated Cauchy structure} of the convergence group.
\end{definition}

\begin{proposition}
Every locally compact convergence group of finite depth is complete.
\end{proposition}
TODO: finite depth necessary?
\begin{proof}
Let $\sSet{G, \cdot, 1, \xi}$ be a locally compact convergence group and let $F$ be a proper Cauchy filter. Then $F\cdot F^{-1} \to 1$, so there exists a compact set $K\in F\cdot F^{-1}$. This means there exist $A, B \in F$ such that $A\cdot B^{-1} \subseteq K$. Take $x_0 \in B$, then $F_0 \subseteq K\cdot x_0$ and $K\cdot x_0$ is compact by TODO ref. Now by the ultrafilter lemma \ref{ultrafilterLemma} we can take an ultrafilter $G \supseteq F$. Then $K\cdot x_0 \in G$, so $G$ converges to some $y$.

Then $G \cap \pfilter{y}$ is a Cauchy filter. By finite depth, $F\cap G \cap \pfilter{y} = F\cap \pfilter{y}$ is also a Cauchy filter, so $F$ Cauchy converges to $y$.
\end{proof}

\section{Metrisability and norm}

\begin{theorem}[Birkhoff-Kakutani]
Let $\sSet{G,\cdot, 1, \xi}$ be a convergence group. Then the following are equivalent:
\begin{enumerate}
\item $G$ is pseudometrisable;
\item the topology on $G$ is induced by a left translation invariant pseudometric;
\item the topology on $G$ is induced by a right translation invariant pseudometric;
\item $G$ is first countable and topological.
\end{enumerate}
\end{theorem}

\begin{definition}
Let $\sSet{G, \cdot, 1}$ be a group. We call a function $\norm{\cdot}: G\to \R^+$ a \udef{group norm} on $G$ if $\forall x,y\in G$ the following hold
\begin{enumerate}
\item triangle inequality / subadditivity: $\norm{xy} \leq \norm{x} + \norm{y}$;
\item positivity $\norm{x} > 0$ if $x \neq 1$;
\item inversion $\norm{x^{-1}} = \norm{x}$.
\end{enumerate}
We call the norm \udef{cyclically permutable} if $\forall x,y\in G$: $\norm{xy} = \norm{yx}$.

If $\norm{\cdot}$ is a group norm for $G$, then we call $\sSet{G, \cdot, 1, \norm{\cdot}}$ a \udef{normed group}.
\end{definition}


\part{Number Systems}
\setcounter{chapter}{0} % Reset chapter counter
\chapter{The integers  $\Z$}



\chapter{The rational numbers $\Q$}

\begin{proposition}
The ordered set $(\Q,\leq)$ is not complete.
\end{proposition}
\begin{proof}
Consider the set
\[ A = \setbuilder{x\in\Q}{x<0\land x^2<2}. \]
\end{proof}

\begin{proposition}
The rational numbers are embedded in any ordered field $K$ by
\[ \Q \cong \Q\cdot 1 \subset K. \]
Any totally ordered field has characteristic zero.
\end{proposition}

\section{The Archimedean property}
\begin{definition}
A totally ordered field $(K,+,\cdot, \leq)$ is said to have the Archimedean property if $(K,+,\leq)$ is an Archimedean monoid. We say
\begin{itemize}
\item $x$ is \udef{infinitesimal} if $x$ is infinitesimal w.r.t. $1$;
\item $x$ is \udef{infinite} if $x$ is infinite w.r.t. $1$;
\end{itemize}
\end{definition}

\begin{lemma}
Let $(K,+,\cdot, \leq)$ be a totally ordered field. Then
\begin{enumerate}
\item if $x\in K$ is infinitesimal, then $1/x$ is infinite, and vice versa;
\item if $x\in K$ is infinitesimal and $r\in \Q\cdot 1$, then $rx$ is also infinitesimal.
\end{enumerate}
\end{lemma}

\begin{proposition}[Axiom of Archimedes]
Let $(K,+,\cdot, \leq)$ be a totally ordered field. Then the following are equivalent:
\begin{enumerate}
\item $K$ is Archimedean;
\item for all $x\in K$ there exists $n\in\N$ such that $x < n\cdot 1$;
\item for all positive $\varepsilon\in K$ there exists $n\in\N$ such that $(n\cdot 1)^{-1}< \varepsilon$.
\end{enumerate}
\end{proposition}

\begin{lemma}
Let $(K,+,\cdot, \leq)$ be an Archimedean totally ordered field. Then for all $x\in K$ there exists an $n\in\N\cdot 1$ such that
\[ n\cdot 1 \leq x < n\cdot 1 + 1. \]
\end{lemma}



\chapter{The real numbers $\R$}

\begin{proposition}
The field of real numbers is Archimedean.
\end{proposition}
\begin{proof}
Assume, towards a contradiction, that the field of real numbers is Archimedean. Then $\N$ has an upper bound, and by completeness a least upper bound $s = \sup\N$. Then $s-1$ is not an upper bound and thus there exists and $n\in\N$ such that $s-1<n$. But then $s<n+1 \in \N$ so that $s$ is not an upper bound of $\N$, yielding a contradiction.
\end{proof}

\begin{proposition}
There exists a totally ordered field and it is unique up to isomorphism.
\end{proposition}

\begin{proposition}
The rational numbers $\Q$ are dense in $\R$.
\end{proposition}
\begin{proof}
Let $x < y$ be real numbers. By the Archimedean property there exists a natural number $n>(y-x)^{-1}$. TODO: complete.
\end{proof}

\section{Affinely extended real number system}
\begin{definition}
Dedekind–MacNeille completion of the real numbers
\[ \overline{\R} = \R\cup\{-\infty, +\infty\}. \]
\end{definition}


\section{Functions on the real numbers}

\subsection{Functions from reals to integers}
\subsubsection{Rounding}
\subsubsection{Floor and ceiling}
$\floor{x}$ and $\ceil{x}$
\subsubsection{Fractional and integer part}

\section{Irrational numbers}
\subsection{Beatty sequences}
TODO \url{https://en.wikipedia.org/wiki/Lambek%E2%80%93Moser_theorem}
\begin{definition}
Let $r$ be a positive irrational number. The \udef{Beatty sequence} generated by $r$ is the sequence
\[ \mathcal{B}_r: \N \to \N: n\mapsto \floor{nr}  \]
\end{definition}

\begin{lemma}
Let $r$ be a positive irrational number. For a positive integer $m$ the following are equivalent:
\begin{enumerate}
\item $m$ is a term in the Beatty sequence $\mathcal{B}_r$;
\item $1 - \frac{1}{r} < \fractional\left(\frac{m}{r}\right)$;
\item $m = \floor{\left(\floor{\frac{m}{r}}+1\right)}$.
\end{enumerate}
\end{lemma}

\begin{theorem}[Rayleigh]
TODO
\end{theorem}

\begin{theorem}[Uspensky]
If $r_1, \ldots, r_n$ are positive numbers such that the Beatty sequences $\mathcal{B}_{r_1}, \ldots, \mathcal{B}_{r_n}$ partition the positive integers, then $n\leq 2$.
\end{theorem}
This means there is no equivalent to Rayleigh's theorem for more than two sequences.
\begin{proof}
TODO
\end{proof}

\subsubsection{Beatty series}
\url{https://math.stackexchange.com/questions/2052179/how-to-find-sum-i-1n-left-lfloor-i-sqrt2-right-rfloor-a001951-a-beatty-s}


\subsection{Games}

\chapter{Complex numbers}
TODO: prove $\C$ cannot be a totally ordered field.
TODO
\[ -1 = i^2 = \sqrt{-1}\sqrt{-1} = \sqrt{(-1)^2} = \sqrt{1} = 1 \]


The set of the complex numbers is denoted $\C$.

\begin{lemma} \label{lemma:boundedThenReal}
Let $z\in\C$. Suppose there is a $C\geq 0$ such that
\[ \forall t\in\R: \quad |z+it|^2\leq C+t^2, \]
then $z\in \R$.
\end{lemma}
\begin{proof}
Write $z = a+bi$ for some $a,b\in \R$. Then
\[ |z+it|^2-t^2 = a^2 + (b+t)^2 - t^2 = a^2+b^2+2bt. \]
The left side is bounded by $C$ for all $t\in\R$. If $b>0$, the right side is unbounded for $t\to +\infty$. If $b<0$, the right side is unbounded for $t\to -\infty$. So we need $b=0$ and thus $z=a\in\R$.
\end{proof}

\section{Solutions to quadratic equations}
 We define\footnote{In engineering $i$ is often called $j$, because $i$ is used to denote electric current}
\[ i \equiv \sqrt{-1} \]
We call $i$ the \udef{imaginary unit}. The choice of terminology is not great here; it creates an artificial divide between the ``real'' numbers and this new ``imaginary'' thing. Many a maths teacher has taken great pains to explain that imaginary and complex numbers are no more imaginary than real numbers. I would tend to make the opposite case: all numbers are imaginary mathematical constructs that happen to be quite useful. We are just more familiar with real numbers.

So let us, arguendo, accept this new mathematical object. We are now faced with two questions: is it useful and what are the consequences? We will start by exploring the consequences.

We have already remarked upon the fact that the real numbers form a field. It makes some sense to try to find a field that contains both the real numbers and this new imaginary unit. There may be many such fields. We will try to find the smallest, which we will call the \udef{complex numbers}, notated $\C$. Obviously this new field contains $\R \cup \{i\}$ as a subset, but this in itself is \emph{not} a field. As stated above, in a field both multiplication and addition work well and in particular give results that are still part of the field.  For instance, imagine multiplying $i$ with a real number, like $2$ (something that we can do in a field by definition). We can show that $2i$ is not an element of $\R \cup \{i\}$. Now $2i$ cannot be a real number, because
\begin{align*}
(2i)^2 &= 2i2i \\
&= 4i^2 \qquad \text{by commutativity} \\
&= -4
\end{align*}
and we know that the square of a real number cannot be negative. It might be the case that $2i = i$, but subtracting $i$ from both sides of the equation would give us $i=0$, which can quite easily be seen to be contradictory. This means $2i$ is an entirely new number not yet considered. We could give it a new name (like we did for $\sqrt{-1}$) if we wanted to, but given we can multiply $i$ by any real number and get a new object, we will not start naming them just yet. So we can sum up our current knowledge as follows
\[ \C \; \supset \; \R \cup \{a\cdot i \; | \; a \in \R_0\} \]
We call the set $\{a\cdot i \; | \; a \in \R_0\}$ the \udef{imaginary numbers}. We are not finished yet, because there are still more numbers that must be included in the set of complex numbers. Take for instance $i + 1$. Again this must be a complex number if we wish the complex numbers to be a field. Performing simple calculations as before we can see that $i + 1$ cannot be either a real number (as that would mean that $i$ was a real number minus $1$) or a purely imaginary one (assuming $i + 1 = a\cdot i$ imaginary, we would see that $1 = (a-1)i$ which can easily be seen to be absurd). Consequently we can see that if we add a real number to an imaginary number, the result is a new complex number that we have not considered yet. The complex numbers must therefore include all numbers of the form $a + bi$ and for each $a,b \in \R$ we have a unique complex number.

Finally we can remark that the set of complex numbers that we have created so far is in fact a field. We can verify for example that the result of addition or multiplication of two numbers of the form $a+bi$ can also be written in that form (making use of the fact that the addition and multiplication operations fulfill the requirements to be part of a field):
\begin{align*}
(a_1 + b_1i) + (a_2 + b_2i) &= (a_1 + a_2) + (b_1 + b_2)i \\
(a_1 + b_1i)(a_2 + b_2i) &= (a_1a_2 - b_1b_2) + (a_1b_2 + a_2b_1)i
\end{align*}
Because we were looking for the smallest field that contains both the real numbers and the imaginary unit, we can stop here.

The treatment so far has included the essential arguments necessary to prove that all complex numbers can be written as
\[a + bi \qquad \text{with} \qquad a,b \in \R.\]
In other words we can say
\[ \C = \{a + bi \; | \; a,b \in \R \}. \]

\section{How to represent complex numbers}
In the previous section we saw that every complex number can be written as $a + bi (a,b \in \R)$. Conversely for every $a,b $ in $\R$ there is a unique complex number $a + bi$. Thus we can see that every complex number can be constructed using two real numbers. We give those real numbers special names: For a complex number $z = a + bi$, we call $a$ the real part (denoted $\Re(z)$) and $b$ the complex part (denoted $\Im(z)$).

TODO complex plane

modulus argument
Euler formula??
Conversions

\section{Practical calculations}
The following methods give a practical way to perform calculations with complex numbers. Assume we have two complex numbers $z_1$ and $z_2$. 
\subsection{Addition} is usually easiest if the complex numbers are in the form $a+bi$. Then we have
\begin{align*}
z_1 + z_2 &= (a_1 + b_1i) + (a_2 + b_2i) \\
&= (a_1+a_2) + (b_1+b_2)i
\end{align*} 
\subsection{Multiplication} is usually easiest if the complex numbers are in the form $re^{i\phi}$. Then we have
\begin{align*}
z_1 \cdot z_2 &= r_1e^{i\phi_1}\cdot r_2e^{i\phi_2} \\
&= (r_1\cdot r_2)e^{i(\phi_1+\phi_2)}
\end{align*}
So we multiply the moduli and add the arguments.
\subsection{Exponentiation} with an integer (or real) exponent is again usually easiest if the complex number is in the form $re^{i\phi}$.
\begin{align*}
z^n &= (re^{i\phi})^n \\
&= r^n e^{in\phi}
\end{align*}
\section{Trigonometry revisited}
\subsection{Waves and complex numbers}

Cayley-Dickinson

\part{Linear Algebra}
\setcounter{chapter}{0} % Reset chapter counter
\chapter{Vector spaces}

Gauss-Jordan reduction

TODO projective transformations

orientation
\url{https://en.wikipedia.org/wiki/Orientation_(vector_space)}
also for fixed set of $n$ vectors

\url{http://www.physics.rutgers.edu/~gmoore/618Spring2018/GTLect2-LinearAlgebra-2018.pdf}

\section{Formal definition}
A vector space is a collection of vectors, which are objects that have a natural addition and scalar multiplication.
\begin{definition}
A \udef{vector space} over a field $\mathbb{F}$ is a set $V$ together with an \udef{addition}
\[ +: V\times V \to V \]
and a \udef{scalar multiplication}
\[ \cdot: \mathbb{F}\times V \to V \]
such that $(V,+)$ is a commutative group and the following properties hold:
\begin{itemize}[leftmargin=4cm]
\item[\textbf{Distributivity 1}] $\lambda\cdot(v+w) = \lambda v + \lambda w$ for all $\lambda \in \mathbb{F}$ and all $v,w \in V$.
\item[\textbf{Distributivity 2}] $(\lambda_1+\lambda_2)\cdot v = \lambda_1 v + \lambda_2 v$ for all $\lambda_1, \lambda_2 \in \mathbb{F}$ and all $v \in V$.
\item[\textbf{Mixed associativity}] $\lambda_1\cdot(\lambda_2\cdot v) = (\lambda_1 \lambda_2) \cdot v$ for all $\lambda_1, \lambda_2 \in \mathbb{F}$ and all $v \in V$.
\item[\textbf{Multiplicative identity}] $1\cdot v = v$ for all $v \in V$.
\end{itemize}
This vector space can be denoted $\sSet{\mathbb{F}, V, +}$.
\end{definition}
In the definition we have used the following convention: for all $v,w\in V$ and $\lambda\in \mathbb{F}$, we denote $+(v,w)$ as $v+w$ and $\cdot(\lambda, v)$ as $\lambda \cdot v$ or $\lambda v$.

We call the elements of the field \udef{scalars} and the elements of the set $V$ \udef{vectors}. The zero of the group is known as the \udef{zero vector}.

Almost always we will actually be interested in $\mathbb{F} = \R$ or $\mathbb{F} = \C$.
\subsection{Examples}
\begin{enumerate}
\item The $n$-tuples in $\mathbb{F}^n$ with pointwise addition and multiplication. If the entries of the $n$-tuples are written one above the other in a column, it is called a \udef{column vector}.
\item The polynomials in $\mathbb{F}[X]$.
\item The polynomials in $\mathbb{F}[X]_{\leq n}$ of maximally degree $n$.
\item For any set $S$, the functions $(S\to \mathbb{F})$, denoted $\mathbb{F}^S$, with pointwise addition and multiplication.
\item For any topological space $X$, the continuous functions in $(X\to \C)$, denoted $\cont(X)$.
\item The trivial vector space $\{ 0\}$. A vector space can never be empty, because a commutative group always has a neutral element.
\item The set of all possible \textit{displacements} in (Euclidean) space forms a vector space. Once we have chosen an origin, we can view space as a vector space.
\end{enumerate}
\subsection{Some elementary lemmas}
\begin{lemma}
Given the vector space $(\mathbb{F}, V, +)$  and arbitrary $u,v,w\in V$ and $\lambda \in \mathbb{F}$, we have
\begin{enumerate}
\item $0v = 0 = \lambda \cdot 0$;
\item $(-1)v = -v = 1(-v)$;
\item $(-\lambda)v = -(\lambda v) = \lambda(-v)$;
\item $u+v = w+v \implies u = w$.
\end{enumerate}
By $-v$ we mean the additive inverse of $v$.
\end{lemma}
\begin{proof}
\begin{enumerate}
\item First, use distributivity to get
\[ 0v = (0+0)v = 0v + 0v. \]
The apply the previous lemma to $0+0v = 0v = 0v+0v$ to get $0=0v$. The equality $\lambda\cdot 0 = 0$ is proved analogously.
\item To show that $(-1)\cdot v$ is the additive inverse of $v$, i.e.\ $-v$, we simply add $(-1)\cdot v + v$ and observe the result is $0$.
\[ (-1)\cdot v + v = (-1)\cdot v + 1\cdot v = (1+(-1))\cdot v = 0\cdot v = 0. \]
\item Similar to the previous point.
\item The additive inverse $-v$ exists, so we can just add it left and right.
\end{enumerate}
\end{proof}
\subsection{Subspaces}
\begin{definition}
A \textit{subset} $U$ of a vector space $V$ is called a \udef{subspace} of $V$ if $U$ is also a vector space.
\end{definition}
The subset $U$ automatically inherits a lot of the structure of $V$. We only need to verify a couple of conditions.
\begin{proposition}[Subspace criterion] \label{subspaceCriterion}
A subset $U$ of a vector space $V$ is a subspace of $V$ \textup{if and only if} $U$ satisfies the following conditions:
\begin{enumerate}
\item \textbf{Additive identity}: $0 \in U$. Alternatively it is enough to show that $U$ is not empty.
\item \textbf{Closed under addition}: $v,w \in U$ implies $v+w\in U$;
\item \textbf{Closed under scalar multiplication}: $\lambda \in \mathbb{F}$ and $u\in U$ implies $\lambda u \in U$.
\end{enumerate}
\end{proposition}
Alternatively the last two criteria are equivalent to:
\[ v,w\in U; \lambda \in \mathbb{F} \qquad \text{implies} \qquad v+\lambda w \in U. \]

If the question is whether a set is a subspace, this criterion is almost always the answer. An elementary application:
\begin{proposition}
Any arbitrary intersection of subspaces is a subspace.
\end{proposition}
\begin{corollary}
Let $V$ be a vector space. Then the subspaces of $V$ form a complete sublattice of $\sSet{\powerset(V),\subseteq}$.
\end{corollary}

\begin{definition}
The closure operator into the complete lattice of subspaces of $V$ is called the \udef{span}.

If $D$ is a subset of $V$ such that $V = \Span(D)$, then $D$ \udef{spans} $V$.

\begin{itemize}
\item A vector space is called \udef{finite-dimensional} if it is spanned by a finite set of vectors.
\item A vector space is \udef{infinite-dimensional} if it is not finite-dimensional.
\end{itemize}
\end{definition}

\begin{definition}
Let $V$ be a vector space. A \udef{hyperplane} in $V$ is a coatom in the lattice of subspaces of $V$.
\end{definition}

\section{Basis and dimension}
\subsection{Linear combinations and span}
\begin{definition}
A \udef{(finite) linear combination} of vectors $v_1, \ldots, v_n$ is a vector of the form
\[ a_1v_1 + \ldots + a_nv_n \]
where $a_1, \ldots, a_n \in \mathbb{F}$.
\end{definition}

\begin{proposition}
Let $V$ be a vector space over a field $\F$ and $D\subseteq V$ a subset. Then $\Span(D)$ is the set of all finite linear combinations of vectors in $D$ if $D \neq \emptyset$. If $D = \emptyset$, then $\Span(D) = \{0\}$.
\end{proposition}

\subsection{Linear independence}
\begin{definition}
A set of vectors $D$ is \udef{linearly independent} if the only linear combinations in $D$ that equal $0$ are the trivial ones with all scalars zero. i.e.\,
\[ \sum_{i=1}^n a_iv_i = 0 \qquad\implies\qquad a_1=\ldots=a_n = 0 , \]
assuming the $v_i$ are vectors in $D$ and the $a_i$ are scalars.

\udef{Linear dependence} is the opposite of linear independence.
\end{definition}
The empty set $D=\emptyset$ is taken as linearly independent. No non-trivial combinations of vectors in $\emptyset$ are equal to zero, because there are no non-trivial combinations of vectors in $\emptyset$.

\begin{lemma}
Let $D$ be a linearly dependent set of vectors. Then there exists a vector $v\in D$ such that
\begin{enumerate}
\item $v$ is a linear combination of other vectors in $D$;
\item $v\in \Span(D\setminus\{v\})$;
\item $\Span(D) = \Span(D\setminus\{v\})$.
\end{enumerate}
\label{linearDependence}
\end{lemma}
\begin{proof}
Take a linear combination of vectors in $D$ equalling zero,
\[ \sum_i a_iv_i = 0. \]
By linear dependence such a combination can be found such that not all $a_i$ are zero. In particular at least two must be non-zero. Take $a_j\neq 0$. Then
\[ v_j = \sum_{i\neq j}\frac{a_iv_i}{a_j}. \]

To prove the last point, take a $u\in \Span(D)$. Then
\[ u = \sum_i b_iv_i = b_j v_j + \sum_{i\neq j} b_iv_i = b_j\sum_{i\neq j}\frac{a_iv_i}{a_j} + \sum_{i\neq j} b_iv_i = \sum_{i\neq j}\left(\frac{b_ja_i}{a_j}+b_i\right)v_i.  \]
So $u\in \Span(D\setminus\{v\})$. The opposite inclusion is obvious. 
\end{proof}

\subsection{Bases}
\begin{definition}
A \udef{basis} of a vector space $V$ is a set of vectors in $V$ that spans $V$ and is linearly independent.
\end{definition}
\begin{example}
The \udef{standard basis} or \udef{natural basis} of $\mathbb{F}^n$ is given by
\begin{align*}
(1,0,0,&\ldots,0), \\
(0,1,0,&\ldots,0), \\
(0,0,1,&\ldots,0), \\
&\ldots \\
(0,0,0,&\ldots,1).
\end{align*}
We will denote it $\mathcal{E}$ or $\mathcal{E}_n$.
\end{example}
\subsubsection{In finite-dimensional spaces}
\begin{proposition}
A finite set $\{v_1, \ldots, v_n\}$ of vectors in $V$ is a basis of $V$ \textup{if and only if} every $v\in V$ can be written uniquely in the form
\[ v = a_1v_1 + \ldots + a_nv_n, \]
where $a_1, \ldots, a_n \in \mathbb{F}$.
\end{proposition}
\begin{proof}
We prove both directions.
\begin{itemize}
\item[$\boxed{\Rightarrow}$] Suppose $\{v_1, \ldots, v_n\}$ is a basis of $V$. Then any vector $v$ can be written as $a_1v_1 + \ldots + a_nv_n$, because the basis spans the space. We just need to show the decomposition is unique. To that end, assume there was another decomposition $v = b_1v_1 + \ldots + b_nv_n$. Subtracting both decompositions gives
\[ 0 = (a_1-b_1)v_1 + \ldots + (a_n-b_n)v_n. \]
Because $\{v_1, \ldots, v_n\}$ is linearly independent, $a_i = b_i$ for all $i$.
\item[$\boxed{\Leftarrow}$] Now suppose every vector has such a decomposition. Clearly $\{v_1, \ldots, v_n\}$ spans $V$. The unique decomposition of $0$ gives linear independence.
\end{itemize}
\end{proof}

\begin{theorem}[Steinitz exchange lemma] \label{SteinitzExchange}
Let $V$ be a vector space.
If $U = \{u_1, \ldots, u_m\}$ is a linearly independent set of $m$ vectors in $V$, and $W = \{ w_1, \ldots, w_n \}$ spans $V$, then:
\begin{enumerate}
\item $m\leq n$;
\item There is a set $\{u_1, \ldots, u_m, w'_{m+1}, \ldots, w'_n\} \supset U$ that spans $V$ where $w'_{m+1},\ldots, w'_n \in W$.
\end{enumerate}
\end{theorem}
\begin{proof}
We obtain the set $\{u_1, \ldots, u_m, w'_{m+1}, \ldots, w'_n\}$ by starting with the list $B_0 = (w_1, \ldots, w_n)$ and applying the following steps for each element $u_i \in U$, in the process defining sets $B_1, \ldots, B_m$. Each of these sets spans $V$.
\begin{enumerate}
\item Add $u_i$ to $B_{i-1}$. The set is now linearly dependent, because $B_{i-1}$ spans $V$.
\item By lemma \ref{linearDependence}, we can find a vector $v$ that is a linear combination of $B_{i-1}\setminus \{v\}$. Because $u_1,\ldots, u_i$ are linearly independent, we can choose this vector to be an element of $W$. Define $B_i = B_{i-1}\setminus\{v\}$. By lemma \ref{linearDependence}, $B_i$ still spans $V$, as required.
\end{enumerate}
This process only stops when we have had all elements of $U$.
\end{proof}
\begin{corollary}
If a vector space $V$ has a basis with $n$ vectors, then any basis of $V$ has $n$ vectors. \label{nBasis}
\end{corollary}

\begin{theorem}
Suppose $V$ is a finite-dimensional vector space spanned by $D = \{v_1, \ldots, v_n\}$.
\begin{enumerate}
\item We can find a subset of $D$ that is a basis of $V$, i.e.\ $D$ can be reduced to a basis;
\item Each linearly independent set of vectors can be expanded to a basis.
\end{enumerate}
\label{basis}
\end{theorem}
\begin{proof}
\begin{enumerate}
\item Remove $0$ from $D$, if it is an element. If $D$ is not linearly independent, find a vector in $D$ that is a linear combination of other vectors in $D$. Repeat until the set is linearly independent. This process stops due to the finite number of vectors. The set spans $V$ at every step.
\item Follows easily from the Steinitz exchange lemma, taking $W$ to be a basis.
\end{enumerate}
\end{proof}
\begin{corollary}
Every finite-dimensional vector space has a basis. \label{existenceBasis}
\end{corollary}

Thanks to corollaries \ref{nBasis} and \ref{existenceBasis}, the following definition makes sense:
\begin{definition}
The \udef{dimension} of a finite-dimensional vector space is the length of any basis of the vector space.
The dimension of $V$ (if $V$ is finite-dimensional) is denoted by $\dim V$ or $\dim_\mathbb{F}V$.\footnote{The latter notation is particularly useful if when distinguishing between real and complex vector spaces, because every complex vector space can be seen as a real vector space. In this case $\dim_\R V = 2\dim_\C V$, because $v$ and $iv$ are linearly independent over $\R$.}

If $V = \{0\}$, we take $\dim V = 0$.
\end{definition}

\begin{corollary}
Every linearly independent set of vectors in $V$ with length $\dim V$ is a basis of $V$. \label{maxLinearlyIndependent}
\end{corollary}
\begin{corollary}
Every spanning set of vectors in $V$ with length $\dim V$ is a basis of $V$.
\end{corollary}

\begin{proposition}
Let $V$ be a finite-dimensional vector space and $U$ a subspace of $V$. Then
\begin{enumerate}
\item $U$ is finite-dimensional and $\dim U \leq \dim V$;
\item $\dim U = \dim V \iff U=V$.
\end{enumerate}
\label{vectorSpaceEquality}
\end{proposition}
\begin{proof}
We construct a basis for $U$ using the following process:
\begin{enumerate}
\item If U=\{0\}, then we can take the basis $\emptyset$ and we are done. If $U\neq \{0\}$, we choose a nonzero vector $v_1 \in U$.
\item If $U$ is the span of all the vectors we have chosen, we are done. If not choose a vector in $U$, not in the span of the other vectors.
\item Repeat step (2).
\end{enumerate}
By construction, the chosen set of vectors is linearly independent. By the Steinitz exchange lemma this process must stop. In particular it must stop before reaching $\dim V$ vectors.

If the process reaches this upper bound, then by corollary \ref{maxLinearlyIndependent}, the set of vectors in $U$ is also a basis for $V$.
\end{proof}
We now have two tools for proving equalities of finite-dimensional vector spaces: either by proving both inclusions, or by leveraging point (2) of the previous proposition.

\subsubsection{In infinite-dimensional spaces}
Our definition of a basis of a vector spaces still makes sense for infinite-dimensional vector spaces, and many results of the previous section still make sense for infinite-dimensional vector spaces.

For infinite-dimensional vector spaces, there are, however, other notions of basis we might be interested in. In particular, our definition of basis requires all vectors to be constructible as \emph{finite} linear combinations of basis elements. In some contexts we might want to relax this to allow infinite combinations as well. For that, of course, we need some notion of infinite sum. Often we construct infinite sums as the limit of a sequence of finite sums, in which case we need a topology on our vector space that allows us to take limits.\footnote{Although other options exist, such as taking sums over hyperintegers.}  

In order to distinguish our purely algebraic definition of basis from these other notions of basis, a basis in the sense defined above is sometimes known as an \udef{algebraic basis} of \udef{Hamel basis}.

We will be discussing Hamel bases in this section.

\begin{theorem} \label{extensionReductionBasis}
Let $V$ be a vector space.
\begin{enumerate}
\item Any spanning set contains a basis.
\item Any linearly independent subset can be expanded to a basis.
\end{enumerate}
\label{infBasis}
\end{theorem}
\begin{proof}
Requires the axiom of choice. We will use Zorn's lemma twice.
\begin{enumerate}
\item Let $S$ be a spanning subset of $V$. Define
\[ \mathcal{A} = \{ D\subset S \;|\; \text{$D$ is linearly independent}\} \]
ordered by inclusion. It is easy to see that any chain on $\mathcal{A}$ has an upper bound on $\mathcal{A}$, by just taking the union which is still linearly independent. It follows from Zorn's lemma that $\mathcal{A}$ has a maximal element $R$. 
We show that $\Span(R) \supset S$ by contradiction. If $\Span(R) \not\supset S$, we can consider $R\cup \{v\}$ for some $v \in S$ that is not in $\Span(R)$ and we obtain an element of $\mathcal{A}$ which is greater than a maximal element. This is a contradiction. Then from $\Span(R) \supset S$ we conclude
\[ \Span(R) = \Span(\Span(R)) \supset \Span(S) = V \]
from which it follows that $\Span(R) = V$.
\item Let $S$ be a linearly independent subset of $V$. Define
\[ \mathcal{A} = \{ D\subset V \;|\; S \subset D \; \text{and $D$ is linearly independent}\} \]
ordered by inclusion. 
It is easy to see that any chain on $\mathcal{A}$ has an upper bound on $\mathcal{A}$, by just taking the union. It follows from Zorn's lemma that $\mathcal{A}$ has a maximal element $R$. We show that $\Span(R) = V$ by contradiction. If $\Span(R) \neq V$, we can consider $R\cup \{v\}$ for some $v\notin \Span(R)$ and we obtain an element of $\mathcal{A}$ which is greater than a maximal element. This is a contradiction.
\end{enumerate}
\end{proof}
\begin{corollary} \label{existenceHamelBasis}
Every vector space has a Hamel basis
\end{corollary}

\begin{theorem}[Dimension theorem for vector spaces]
Given a vector space $V$, any two bases have the same cardinality.
\end{theorem}
\begin{proof}
The finite-dimensional case has already been proved. Suppose $A$ is a basis of $V$ with $|A| \geq \aleph_0$. Let $B$ be another basis of $V$. Each element $a\in A$ can be written as a finite combination of elements in $B$. Collect all the elements that go into the finite linear combination in a finite set $B_a \subset B$. We claim
\[ B = \bigcup_{a\in A} B_a. \]
Indeed, assume $b\in B \setminus (\cup_{a\in A} B_a)$. Since $A$ spans $V$, so does $\cup_{a\in A} B_a$. Thus $b$ can be written as a non-trivial combination of vectors in $\cup_{a\in A} B_a\subset B$, contradicting the linear independence of $B$. Then we have
\[ |B| = \left| \bigcup_{a\in A}B_a \right| \leq \aleph_0 \cdot |A| = |A| \]
A similar argument gives
\[ |A| \leq \aleph_0 \cdot |B| = |B|. \]
By the Schröder–Bernstein theorem \ref{SchroederBernstein}, we conclude $|A| = |B|$.
\end{proof}
TODO: does this proof work with only the ultrafilter lemma?

Thus the notion of dimension (also known as \udef{Hamel-dimension}) also makes sense for infinite-dimensional vector spaces, except it is a cardinality, not a number.

TODO: do we need a strong cardinality assignment? (Assumed for now)

Many textbooks state results using dimensions only for the finite-dimensional case. As we will see, these results almost always generalise directly to the infinite-dimensional case as well, if we assume the axiom of choice.

\begin{note}
The inverse of this theorem (i.e.\ the infinite-dimensional analogue of proposition \ref{vectorSpaceEquality}) does not hold: infinite-dimensional vector spaces always have proper subspaces with a basis of the same cardinality. This is obvious because dropping one vector in the Hamel basis of an infinite-dimensional vector space will not change the cardinality, but will make it a proper subspace.
\end{note}

 \begin{corollary}
 Let $V$ and $W$ be vector spaces.
 \begin{enumerate}
 \item If $\dim V > \dim W$, then no linear map from $V$ to $W$ is injective.
 \item If $\dim V < \dim W$, then no linear map from $V$ to $W$ is surjective.
 \end{enumerate}
 \end{corollary}

\begin{lemma}
Let $V$ be an infinite-dimensional vector space over a field $\mathbb{F}$. Assume $|\mathbb{F}|\leq \dim_{\mathbb{F}} V$, then $\dim_{\mathbb{F}} V = |V|$. \label{vsCardinality}
\end{lemma}
\begin{proof}
Let $B$ be a basis of $V$. It is supposed infinite. There is a surjection
\[\bigcup_{n\in\N}(\mathbb{F}\times B)^{n} \to V: (a_i,v_i)^{i<n} \mapsto \sum_{i<n}a_iv_i. \]
So we have
\[ |V| \leq \left|\bigcup_{n\in\N}(F\times B)^{n}\right| = \sum_{n\in \N}|F\times B|^n \leq \aleph_0\cdot |\mathbb{F}| \cdot |B| = \max\{\aleph_0, |\mathbb{F}|, |B|\} = |B|. \]
Thus $|V|\leq \dim_{\mathbb{F}} V$. The other inequality is obvious. By the Schröder–Bernstein theorem \ref{SchroederBernstein}, we conclude $\dim_{\mathbb{F}} V = |V|$.
\end{proof}

\section{Constructing vector spaces}
\subsection{Sums of subspaces}
\begin{definition}
Suppose $\{U_i\}_{i\in I}$ a set of subspaces of a vector space $V$. The \udef{sum} of these subspaces, denoted $\sum_{i\in I}U_i$, is the set of all finite linear combinations of elements in $\bigcup_{i\in I}U_i$:
\[ \sum_{i\in I}U_i = \Span\left(\bigcup_{i\in I} U_i\right) = \setbuilder{\sum_{i\in J} u_i}{\text{$J\subset I$ finite}, u_i\in \bigcup_{i\in I}U_i}. \]
\end{definition}
For finite sums this reduces to
\[ U_1+\ldots + U_m = \setbuilder{\sum_{i=1}^m u_i}{u_1\in U_1, \ldots, u_m\in U_m}. \]

\begin{proposition} \label{basisSum}
Let $\{U_i\}_{i\in I}$ be a set of subspaces of a vector space $V$ and $\beta_i$ a basis of $U_i$ for all $i\in I$. Then
\[ \sum_{i\in I}U_i = \Span\left(\bigcup_{i\in I}\beta_i\right). \]
\end{proposition}
\begin{proof}
From $\bigcup_{i\in I}\beta_i \subseteq \bigcup_{i\in I} U_i$, we get $\Span\left(\bigcup_{i\in I}\beta_i\right) \subseteq \Span\left(\bigcup_{i\in I} U_i\right) = \sum_{i\in I}U_i$.

Conversely, take $u\in \sum_{i\in I}U_i$. Then $u = \sum_{j\in J}u_j$ where $J$ is finite subset of $I$ and $u_i\in U_i$. Now each $u_j$ can be written as $\sum_k a_{j,k}v_{j,k}$, where $a_{j,k}$ are scalars and $v_{j,k}$ are vectors in $\beta_j$. So
\[ u = \sum_{j,k}a_{j,k}v_{j,k}, \]
which is a finite linear combination of vectors in $\bigcup_{i\in I}\beta_i$. So $u\in \Span\left(\bigcup_{i\in I}\beta_i\right)$.
\end{proof}

\begin{proposition}
Let $V$ be a vector space and $A,B,C$ subspaces. Then
\begin{enumerate}
\item $A+(B\cap C) \subseteq (A+B)\cap (A+C)$;
\item $(A+B)\cap C \supseteq (A\cap C) + (B\cap C)$. 
\end{enumerate}
\end{proposition}
\begin{proof}
(1) Take $v = v_1+v_2 \in A+(B\cap C)$ where $v_2 \in B$ and $v_2 \in C$, so $v_1+v_2\in A+B$ and $v_1+v_2\in A+C$.

(2) Take $v = v_1+v_2\in (A\cap C) + (B\cap C)$. Then $v_1,v_2\in C$ and thus $v\in (A+B)\cap C$.
\end{proof}

\begin{theorem}[Dimension of a sum]
Let $U_1$ and $U_2$ be subspaces of a finite-dimensional vector space, then
\[ \dim(U_1 + U_2) = \dim U_1 + \dim U_2 - \dim(U_1\cap U_2). \]
\label{dimOfASum}
\end{theorem}
\begin{proof}
Let $\dim U_1 = r, \dim U_2 = s$ and $\dim(U_1\cap U_2) = t$. Then $t\leq r$ and $t\leq s$.  Take a basis $\{v_1,\ldots, v_t\}$ of $U_1\cap U_2$. This can be expanded to a basis $\beta_{U_1} = \{ v_1, \ldots, v_t, u_{t+1}, \ldots u_{r} \}$ of $U_1$ and also to a basis $\beta_{U_2} = \{ v_1, \ldots, v_t, u'_{t+1}, \ldots u'_{s} \}$ of $U_2$. We will show that $\{ v_1, \ldots, v_t, u_{t+1}, \ldots u_{r}, u'_{t+1}, \ldots, u'_{s} \}$ is a basis of $U_1\cap U_2$. This completes the proof because
\begin{align*}
\dim(U_1 + U_2) &= t + (s-t) + (r-t) = s + r -t\\
&= \dim U_1 + \dim U_2 - \dim(U_1\cap U_2).
\end{align*}
The spanning property is easy. Linear independence is slightly more difficult: Take a linear combination
\[ \sum_{i=1}^t\alpha_i v_i + \sum^r_{j=t+1}\beta_ju_j + \sum^s_{k=t+1}\beta'_ku_k' =0. \]
We must show this combination is trivial. Indeed observe that
\[ \sum_{i=1}^t\alpha_i v_i + \sum^r_{j=t+1}\beta_ju_j  =-\sum^s_{k=t+1}\beta'_ku_k'. \]
The left-hand side is a vector in $U_1$, the right-hand side is a vector in $U_2$, so it must lie in $U_1\cap U_2$, so we rewrite the left-hand side as
\[ \sum_{i=1}^t\lambda_i v_i =  -\sum^s_{k=t+1}\beta'_ku_k'.\]
Due to $\beta_{U_2}$ being a basis, this linear combination must be trivial and all $\beta'_k$ are zero. This leaves us 
\[ \sum_{i=1}^t\alpha_i v_i + \sum^r_{j=t+1}\beta_ju_j =0 \]
from our original linear combination. Due to $\beta_{U_2}$ being a basis this combination must also be trivial. 
\end{proof}
\begin{note}
If $\dim(U_1\cap U_2)<\dim U_1$ and $\dim(U_1\cap U_2)< \dim U_2$, this proof generalises to infinite-dimensional vector spaces.
\end{note}

\subsection{(Internal) direct sum}
\begin{definition}
Suppose $\{U_i\}_{i\in I}$ is a set of subspaces of $V$. The sum $\sum_{i\in I}U_i$ is called a \udef{direct sum} if each element $u$ of the sum can be \emph{uniquely} written as
\[ u = \sum_{i\in I}u_i \qquad (u_i\in U_i) \]
where only finitely many of the $u_i$ are nonzero.

In this case we write $\bigoplus_{i\in I} U_i$, or $U_1 \oplus \ldots \oplus U_m$ if $I = \{1,\ldots, m\}$. 
\end{definition}

\begin{proposition}[Conditions for a direct sum] \label{directSumCriterion}
Let $\{U_i\}_{i\in I}$ be a set of subspaces of a vector space $V$ and $\beta_i$ a basis of $U_i$ for all $i\in I$. Let $U,W\subseteq V$ also be subspaces of $V$.
\begin{enumerate}
\item The sum $\sum_{i\in I}U_i$ is direct \textup{if and only if} $0$ has the unique decomposition as in the definition.
\item The sum $\sum_{i\in I}U_i$ is direct \textup{if and only if} the union $\bigcup_{i\in I}\beta_i$ is disjoint and linearly independent.
\item The sum $U+W$ is direct \textup{if and only if} $U\cap W = \{0\}$.
\end{enumerate}
\end{proposition}
\begin{proof}
TODO
\end{proof}
\begin{corollary}
Let $\{U_i\}_{i\in I}$ be a set of subspaces of a vector space $V$ and $\beta_i$ a basis of $U_i$ for all $i\in I$. Then
\[ \dim\left(\bigoplus_{i \in I}U_i\right) = \sum_{i\in I} \dim U_i \]
\end{corollary}

\begin{definition}
In a vector space $V$, a subspace $W$ is a \udef{complementary subspace} (or a \udef{complement}) of the subspace $U$ if $V = U \oplus W$.
\end{definition}

\begin{proposition}
Let $V$ be a vector space, then each subspace of $V$ has a complement.
\end{proposition}
\begin{proof}
Let $U$ be a subspace of $V$. Then, by \ref{existenceHamelBasis}, we can find a basis $B$ of $U$ and, by \ref{extensionReductionBasis}, we can extend it to a basis $D$ of $V$. Now $V = U \oplus \Span(D\setminus B)$ by \ref{basisSum} and \ref{directSumCriterion}.
\end{proof}
Note this requires the axiom of choice, and is in fact equivalent with it.
\begin{corollary}
Suppose $V$ is finite-dimensional and $U_1,\ldots, U_m$ are subspaces of $V$. Then $U_1+\ldots+ U_m$ is a direct sum \textup{if and only if}
\[ \dim(U_1+\ldots+U_m) = \dim U_1 + \ldots \dim U_m. \]
\label{directSumDimensionCriterion}
\end{corollary}

\subsection{External direct sum}
\begin{definition}
Let $U,W$ be vector spaces over the same field $\mathbb{F}$. We define the vector space  $U\oplus W$, called the \udef{(external) direct sum}, as the set $U\times W$ with the operations
\[ \begin{cases}
(u_1,w_1) + (u_2, w_2) = (u_1 +_U u_2, w_1 +_W w_2) & (u_1,u_2 \in U; w_1, w_2 \in W) \\
r\cdot (u,w) = (ru, rw) & (r\in \mathbb{F}; u\in U; w \in W)
\end{cases} \]
In general we can define a direct sum of an arbitrary collection of vector spaces $\{U_i\}_{i\in I}$, denoted
\[ \bigoplus_{i\in I}U_i \]
as the vector space with as field the subset of the Cartesian product $\prod_{i\in I}U_i$ where all but finitely many of the terms are zero. The operations are defined point-wise.
\end{definition}

\begin{proposition}
Suppose $V_1, \ldots V_m$ are vector spaces over $\mathbb{F}$. Then
\[ \dim(V_1\oplus\ldots \oplus V_m) = \dim V_1 + \ldots + \dim V_m \]
\label{dimDirectSum}
\end{proposition}
\begin{proof}
We construct a basis $\beta$ of $V_1\oplus\ldots \oplus V_m$ from bases $\beta_{V_i}$ of $V_i$:
\[ \beta = (\beta_{V_1} \times \{0 \} \times \ldots \times \{0 \}) \cup (\{0 \} \times \beta_{V_2}\times \{0 \} \times \ldots \times \{0 \}) \cup \ldots \cup (\{0 \}\times \ldots \times \{0 \} \times \beta_{V_m}). \]
All these unions are disjunct, so
\begin{align*}
|\beta| &= |(\beta_{V_1} \times \{0 \} \times \ldots \times \{0 \}) \cup  \ldots \cup (\{0 \}\times \ldots \times \{0 \} \times \beta_{V_m})| \\
&= |(\beta_{V_1} \times \{0 \} \times \ldots \times \{0 \})| + \ldots + |(\{0 \}\times \ldots \times \{0 \} \times \beta_{V_m})| \\
&= | \beta_{V_1}| + \ldots + |\beta_{V_m}| \\
&= \dim V_1 + \ldots + \dim V_m.
\end{align*}
\end{proof}

\begin{proposition}
Let $U,W$ be subspaces of $V$. Then the external direct sum of $U$ and $W$ is isomorphic to the internal direct sum of $U$ and $W$.
\end{proposition}
\begin{proof}
The map $f: U\times W\to V: (u,w) \mapsto u+w$ is an isomorphism.
\end{proof}
For this reason we use the same symbol for both.

\begin{definition}
Let $V,W, X,Y$ be vector spaces over $\mathbb{F}$. Let $S: V\to X$ and $T: W\to Y$ be linear maps. Then the \udef{direct sum} of $S$ and $T$ is a linear map
\[ S\oplus T: V \oplus W \to X\oplus Y: (v,w) \mapsto (S(v), T(v)). \]
\end{definition}

\begin{lemma}
Let $V,W$ be vector spaces over a field $\F$ and $A,C\in\Lin(V)$ and $B,D\in\Lin(W)$. Then
\begin{enumerate}
\item $a(A\oplus B) + b(C\oplus D) = (aA+bC)\oplus (aB + bD)$;
\item $(A\oplus B)(C\oplus D) = AC\oplus BD$;
\item $(A\oplus B)^k = A^k\oplus B^k$.
\end{enumerate}
\end{lemma}

\subsubsection{Matrix representation}
TODO: move
Assume $V$ and $W$ are finite-dimensional vector spaces with resp. bases $\{\vec{e}_i\}_{i=1}^m$ and $\{\vec{f}_j\}_{j=1}^n$.
As in the proof of proposition \ref{dimDirectSum}, we can take the basis $\{\vec{e}_i\}_i\times\{0\} \cup \{0\}\times\{\vec{f}_j\}_j$ of $V\oplus W$.

We can naturally fit the basis into a list of $m+n$ elements:
\[ (\vec{e}_1,0),\ldots (\vec{e}_m, 0), (0, \vec{f}_1), \ldots, (0,\vec{f}_n)  \]
\subsubsection{Linear maps}
TODO: also move
Let $S: V\to X$ and $T:W\to Y$ be linear maps, with matrix representations $A$ and $B$, respectively. The matrix representation of $S\oplus T$ is given by
\[ A\oplus B = \begin{bmatrix}
A & 0 \\
0 & B
\end{bmatrix} \]
with respect to the basis $\{\vec{e}_i\}_i\times\{0\} \cup \{0\}\times\{\vec{f}_j\}_j$.

\section{Linear maps}
\begin{definition}
Let $(\mathbb{R}, V, +)$ and $(\mathbb{R}, W, +)$ be vector spaces over the same field. A \udef{linear map} or \udef{linear transformation} is a function $L:V\to W$ with the following properties:
\begin{itemize}[leftmargin=3cm]
\item[\textbf{Additivity}] $L(u+v) = L(u)+L(v)$ for all $u,v \in V$;
\item[\textbf{Homogeneity}] $L(\lambda v) = \lambda L(v)$ for all $\lambda \in \mathbb{R}$ and all $v\in V$.
\end{itemize}
These conditions are equivalent to the condition that
\[ L(\lambda_1 v_1 + \lambda_2v_2) = \lambda_1L(v_1) + \lambda_2 L(v_2) \qquad \text{for all $\lambda_1,\lambda_2\in \mathbb{F}$ and all $v_1,v_2\in V$.} \]
We denote the set of all linear maps from $V$ to $W$ as $\Lin_\mathbb{F}(V,W)$, or $\Lin(V,W)$. The set of endomorphisms on $V$ is denoted $\Lin(V) \defeq \End(V) = \Lin(V,V)$.
\end{definition}

\begin{lemma} \label{linearMaps}
Let $L\in \Lin(V,W)$.
\begin{enumerate}
\item $L(0) = 0$ and $L(-v) = -L(v)$
\item $L\left(\sum^n_{i=1}\lambda_i v_i\right) = \sum_{i=1}^n\lambda_i L(v_i)$.
\item A linear map is completely determined by the images of a basis of $V$.
\item Let $D$ be a set of vectors. Then $L[D]$ is linearly independent \textup{if and only if} $D$ is linearly independent.
\end{enumerate}
\end{lemma}

\subsection{Examples}
\begin{enumerate}
\item The zero map that maps everything to zero.
\item Identity maps.
\item Differentiation of polynomials.
\item Integration of polynomials.
\item Shifting elements in a list.
\item Projections.
\end{enumerate}

\begin{definition}
A (linear) \udef{operator} between two vector spaces $V$ and $W$ is a linear partial function $T: V \not\to W$ such that the domain $\dom(T)$ is a vector space.

We also say an operator is a function $T: \dom(T)\subseteq V\to W$.
\end{definition}
The requirement that $\dom(T)$ be a subspace of $V$ is necessary for linearity to make sense!

Some authors (e.g\ Axler) use the word ``operator'' to mean a linear endomorphism.

\subsection{Image and kernel}
\begin{definition}
Let $L \in \Lin(V,W)$. The \udef{kernel} or \udef{null space} of $L$ is the set of vectors that $L$ maps to zero:
\[ \ker(L) = \{ v\in V \;|\; L(v) = 0 \}. \]
\end{definition}
\begin{proposition} \label{kernelSubspace}
The kernel of $L\in \Lin(V,W)$ is a subspace of $V$.
\end{proposition}
\begin{definition}
The dimension of the kernel of a linear map is its \udef{nullity}.
\end{definition}
\begin{proposition} \label{injectivityKernelTriviality}
Let $L\in\Lin(V,W)$. Then $L$ is injective if and only if $\ker(L) = 0$.
\end{proposition}
TODO: generalise to groups
\begin{proof}
We show both implications.
\begin{itemize}
\item[\boxed{\Rightarrow}] We know $\{0\}\subset \ker(L)$ by lemma \ref{linearMaps}. Suppose $v\in \ker(L)$, then $L(v) = 0 = L(0)$. So $v=0$ by injectivity and $\{0\}\supset \ker(L)$.
\item[\boxed{\Leftarrow}] Suppose $u,v \in V$ such that $L(u)=L(v)$. Then
\[ 0 = L(u) - L(v) = L(u-v). \]
Thus $u-v\in \ker(L)$, meaning $u-v = 0$ and $u=v$.
\end{itemize}
\end{proof}

\begin{definition}
Let $L \in \Lin(V,W)$. The \udef{image} or \udef{range} of $L$ is the set of vectors that are of the form $L(v)$ for some $v\in V$:
\[ \im(L) = \{ L(v) \;|\; v\in V \}. \]
\end{definition}
\begin{proposition}
The range of $L\in \Lin(V,W)$ is a subspace of $W$.
\end{proposition}
\begin{definition}
The dimension of the image of a linear map is its \udef{rank}.
\end{definition}

\begin{theorem}
Every short exact sequence of vector spaces splits.
\end{theorem}
\begin{proof}
Let
\[ \begin{tikzcd}
0 \rar & U \rar{S} & V \rar{T} & W \rar & 0
\end{tikzcd} \]
be a short exact sequence of vector spaces.
By the splitting lemma TODO ref, it is enough to find a left inverse of $S$. Pick a basis $\beta$ of $U$. Because $S$ is injective, $S[\beta]$ is linearly independent and we can extend it to a basis $\beta'$. We can now define the left inverse by specifying how the basis elements are mapped, by \ref{linearMaps}. To wit: $\beta'\setminus S[\beta]$ is mapped to $0$ and each element $S[\beta]$ has exactly one origin be injectivity and it is to this origin that it is now mapped.
\end{proof}
\begin{corollary} \label{directSumKernelImage}
Let $L \in \Lin(V,W)$. Then
\[ V \cong \ker L \oplus \im L. \]
\end{corollary}
\begin{proof}
Given $L$ we have the short exact sequence
\[ \begin{tikzcd}
0 \rar & \ker L \ar[r, hook] & V \rar{L} & \im L \rar & 0.
\end{tikzcd} \]
The isomorphism then follows from the splitting lemma TODO ref.
\end{proof}
\begin{corollary}[Dimension theorem for linear maps] \label{dimensionLinearMaps}
Let $L \in \Lin(V,W)$. Then
\[ \dim(V) = \dim(\ker L) + \dim(\im L). \]
\end{corollary}
This corollary is also known as the rank-nullity theorem or the fundamental theorem of linear maps.
\begin{proof}
By $\dim(V) = \dim(\ker L \oplus \im L) = \dim(\ker L) + \dim(\im L)$.

Alternatively this can be proven directly as follows:

Take a basis $\beta_0$ of $\ker(L)$. We can expand this to a basis $\beta$ of $V$, by theorem \ref{infBasis}. It is easy to show that $L[\beta\setminus \beta_0]$ is a basis of $\im(L)$. Now $L[\beta\setminus \beta_0] =_c \beta\setminus \beta_0$ and $(\beta\setminus \beta_0) \cap \beta_0 = \emptyset$. Thus $|\beta| = |(\beta\setminus \beta_0) \cup \beta_0| = |\beta\setminus \beta_0| + |\beta_0|$. This proves the assertion.
\end{proof}
\begin{corollary}
Let
\[ \begin{tikzcd}
0 \rar & V_1 \rar & V_2 \rar & \ldots \rar & V_n \rar & 0
\end{tikzcd} \]
be an exact sequence of vector spaces, then
\[ \sum_{i=1}^n (-1)^i\dim(V_i) = 0. \]
\end{corollary}
\begin{proof}
Let $f_i$ be the map $V_i\to V_{i+1}$. By exactness $\im f_i=\ker f_{i+1}$ and $\dim(\im f_i)=\dim(\ker f_{i+1})$. By the previous corollary $\dim(V_i) = \dim(\ker f_i) + \dim(\im f_i)$. Then
\[ \sum_{i=1}^n (-1)^i\dim(V_i) = \sum_{i=1}^n (-1)^i\dim(\ker f_i) + \sum_{i=1}^n (-1)^i\dim(\ker f_{i+1}) = \sum_{i=2}^{n} (-1)^i\dim(\ker f_i) - \sum_{i=2}^{n} (-1)^i\dim(\ker f_{i}) = 0. \]
\end{proof}
\begin{corollary} \label{dimensionImageSmaller}
Let $L \in \Lin(V,W)$. Then
\[ \dim(\im L) \leq \dim(V). \]
\end{corollary}
\begin{proof}
TODO ref cardinal arithmetic.
\end{proof}

\begin{lemma} \label{rankMapComposition}
Let $S,T$ be compatible linear maps. Then
\[ \text{rank of $ST$}\;\leq\;\min\{\text{rank of $S$, rank of $T$}\}. \]
If $T$ is invertible, then the rank of $ST$ equals the rank of $S$. Similarly if $S$ is invertible, then the rank of $ST$ equals the rank of $T$.
\end{lemma}
\begin{proof}
Clearly $\im(ST) \subset \im(S)$, so $\dim\im(ST)\leq \dim\im(S)$.
We also have $ST = S|_{\im T}T$, where $S|_{\im T}$ is $S$ restricted to $\im T$.  Then corollary \ref{dimensionImageSmaller} applied to $S|_{\im T}$ gives $\dim\im(ST)\leq \dim\im T$. Together these inequalities give the result.

To show equality in the invertible case, first assume $T$ invertible:
\[ \dim\im ST \leq \dim\im STT^{-1} = \dim\im S. \]
Together with the first inequality this gives an equality. The case for $S$ invertible is similar.
\end{proof}

\begin{proposition} \label{kernelCompositionLinearMaps}
Let $S,T$ be compatible linear maps. Then
\begin{enumerate}
\item $\ker(ST)\supseteq \ker(T)$;
\item $\dim\ker(ST) = \dim\ker(T) + \dim(\im(T)\cap\ker(S))$.
\end{enumerate}
\end{proposition}
\begin{proof}
(1) $x\in\ker(T) \implies (ST)x = S(Tx) = S(0) = 0 \implies x\in\ker(ST)$.
(2) Consider the restriction $T|_{\ker(ST)}$. Applying the dimension theorem gives
\[ \dim\ker(ST) = \dim\ker(T|_{\ker(ST)}) + \dim\im(T|_{\ker(ST)}) = \dim\ker(ST) = \dim\ker(T) + \dim\im(T|_{\ker(ST)}) , \]
so it is enough to show $\im(T|_{\ker(ST)}) = \im(T)\cap\ker(S)$. First take $v\in\im(T|_{\ker(ST)}$, then there exists some $w\in\ker(ST)$ such that $v=Tw$, meaning $v\in\im(T)$. Also $Sv = STw = 0$, meaning $v\in\ker(S)$.

Then take $v\in\im(T)\cap\ker(S)$, so we can find a $w$ such that $v = Tw$. Also $Sv = STw = 0$, so $w\in\ker(ST)$ and $v\in\im(T|_{\ker{ST}})$.
\end{proof}

\subsection{Algebraic operations on linear maps}
\begin{definition}
Suppose $K,L \in \Lin_{\mathbb{F}}(V,W)$ and $\lambda \in \mathbb{F}$.
\begin{itemize}
\item The \udef{sum} $K+L$ is defined by $(K+L)(v) = Kv+Lv$ for all $v\in V$;
\item The \udef{scalar product} is defined by $(\lambda K)(v) = \lambda K(v)$ for all $v\in V$.
\end{itemize}
\end{definition}
\begin{proposition}
\begin{itemize}
\item The sum of linear maps is again a linear maps. Scalar multiples of linear maps are linear maps.
\item With addition and scalar multiplication defined as above, $\Lin_\mathbb{F}(V,W)$ is a vector space.
\end{itemize}
\end{proposition}

\begin{definition}
Let $K\in \Lin_\mathbb{F}(U,V)$ and $L\in \Lin_\mathbb{F}(V,W)$. The \udef{product} $LK$ is defined as the composition
\[ (LK)(u) = L(K(u)) \qquad \text{for all $u\in U$.} \]
If the product of two linear maps $K,L$ makes sense, we call the linear maps \udef{compatible}.
\end{definition}
\begin{proposition}
The product of two (compatible) linear maps is a linear map.
\end{proposition}
\begin{proposition}[Algebraic properties of linear maps]
The product of linear maps has the following properties. 
\begin{itemize}[leftmargin=4.2cm]
\item[\textbf{Associativity}] Let $L_1, L_2, L_3$ be compatible linear maps, then
\[ (L_1L_2)L_3 = L_1(L_2L_3) \]
\item[\textbf{Identity}] Let $L\in \Lin(V,W)$. The identity maps $I_V:V\to V$ and $I_W:W\to W$ are linear and have the property that
\[ LI_V = I_W L = L. \]
\item[\textbf{Distributive properties}]
$ (S_1+S_2)T = S_1T + S_2T \qquad \text{and} \qquad S(T_1 + T_2) = ST_1 + ST_2 $
whenever $T,T_1, T_2 \in \Lin(U,V)$ and $S,S_1, S_2\in \Lin(V,W)$.
\end{itemize}
These properties mean that for any vector space $V$, $\Lin(V)$ forms a unital algebra.
\end{proposition}
Note that multiplication of linear maps is not commutative, not even for maps that are compatible both ways.

\subsection{Invertibility and isomorphisms}
\begin{proposition} \label{inverseLinear}
Let $L$ be a linear map. If $L$ is invertible as a function (i.e.\ bijective), its inverse $L^{-1}$ is linear.
\end{proposition}
\begin{proof}
We calculate for $x,y$ vectors and $a\in\mathbb{F}$
\[ L^{-1}(ax + y) = L^{-1}(aLL^{-1}x + LL^{-1}y) = L^{-1}L(aL^{-1}x + L^{-1}y) = aL^{-1}x + L^{-1}y. \]
\end{proof}

\begin{definition}
\begin{itemize}
\item An invertible linear map is called an \udef{isomorphism}.
\item Two vector spaces $V,W$  are \udef{isomorphic} if there is an isomorphism between them. This is denoted $V\cong W$.
\end{itemize}
\end{definition}

\begin{proposition} \label{isomorphicDimension}
Let $V,W$ be vector spaces over the same field $\mathbb{F}$ and $n\in \N$. Then
\begin{enumerate}
\item $V\cong W \iff \dim V = \dim W$;
\item $V \cong \mathbb{F}^n \iff \dim V = n$;
\item $\F^n \cong \F^m \iff n=m$.
\end{enumerate}
\label{isomorphicCondition}
\end{proposition}
\begin{proof}
We prove the first statement. The second and third follow easily, using $\dim_\mathbb{F} \mathbb{F}^n = n$.
\begin{itemize}
\item[$\boxed{\Rightarrow}$] Let $T:V\to W$ be an isomorphism. Then $\ker T = \{0\}$ and $\im T = W$. Thus
\[ \dim V = \dim \ker T + \dim \im T = 0 + \dim W = \dim W. \]
\item[$\boxed{\Leftarrow}$] Assume $\dim V = \dim W$. Thus there exists an invertible function from a basis of $V$ to a basis of $W$. This can be extended by linearity to a function on $V$, because it is defined on a Hamel basis. It is easy to see this function is linear and bijective.
\end{itemize}
\end{proof}

\begin{proposition} \label{mappingOfBasisByIsomorphism}
Let $L\in\Lin(V,W)$ be an isomorphism. Let $\beta$ be a basis of $V$, then $L[\beta]$ is a basis of $W$.
\end{proposition}

\begin{proposition} \label{invertibleFiniteDim}
Suppose $V$ is a finite-dimensional vector space and $L\in \Lin(V)$ is a linear map on $V$, then
\[ L \;\text{is invertible} \iff L \;\text{is injective} \iff L \;\text{is surjective} \]
\end{proposition}
\begin{proof}
All we need to prove is
\[ L \;\text{is injective} \iff L \;\text{is surjective} \]
\begin{itemize}
\item[$\boxed{\Rightarrow}$] Assume $L$ injective. Then $\ker L = \{0\}$. By the dimension theorem for linear maps, theorem \ref{dimensionLinearMaps}
\[ \dim \im L = \dim V - \dim \ker L = \dim V. \]
Because $\im L \subset V$ and using proposition \ref{vectorSpaceEquality}, we conclude that $\im L = V$ and thus $L$ is surjective.
\item[$\boxed{\Leftarrow}$] Assume $L$ surjective. Then, by the dimension theorem for linear maps,
\[ \dim \ker L = \dim V - \dim \im L = 0, \]
which means $L$ is injective.
\end{itemize}
\end{proof}
Remark that the proof of the first implication uses proposition \ref{vectorSpaceEquality}, and thus cannot be generalised to infinite-dimensional vector spaces. In the proof of the second implication the subtraction of infinite cardinals is only uniquely defined if  $\dim V > \dim \im L$, which is clearly not the case.

\begin{example}
Counterexamples to the previous theorem in the infinite-dimensional case are given by the left shift map on $\mathbb{F}^\N$ (which is injective, but not surjective) and the right shift map on $\mathbb{F}^\N$ (which is surjective, but not injective).
\end{example}


\subsection{Types of linear maps}
\subsubsection{Finite-rank operators}
\begin{definition}
A linear map $T: V\to V$ is said to be a \udef{finite-rank operator} if it has finite rank.
\end{definition}
\subsubsection{Idempotents}

\begin{lemma}
Let $V$ be a vector space and $U\subseteq V$ a subspace. Then for any complement $W$ of $U$ in $V$,
\[ P: U\oplus W = V \to V: u+w \mapsto u \]
is an idempotent such that $\im P = U$.
\end{lemma}

\begin{proposition} \label{directSumKernelImageIdempotent}
Let $V$ be a vector space and $P$ an idempotent linear map. Then
\[ V = \im P \oplus \ker P. \]
\end{proposition}
\begin{proof}
For any $v\in V$, we can write $v= (v-Pv)+Pv$ where $Pv\in \im P$ and $(v-Pv)\in \ker P$ because
\[ P(v-Pv) = Pv- P^2v = Pv - Pv = 0. \]
So we have $V = \im P + \ker P$. To show that the sum is direct, we take $u\in \im P \cap \ker P$. Then $u = Pw$ for some $w\in V$ and applying $P$ gives $0 = Pu = P^2w = Pw = 0$. So the sum is direct by \ref{directSumCriterion}.
\end{proof}

\begin{lemma} \label{idempotentImageEquivalence}
Let $V$ be a vector space, $P$ an idempotent linear map and $v\in V$. Then $v\in \im P$ \textup{if and only if} $v = Pv$.
\end{lemma}
\begin{proof}
We have that $v\in \im P$ iff $\exists u\in V: Pu = v$. Then we have $v = Pu = P^2u = Pv$.
\end{proof}

\begin{lemma}
Let $V$ be a vector space and $P$ an idempotent linear map. Then $P' \defeq \id_V - P$ is an idempotent linear map such that
\[ \im P' = \ker P \qquad\text{and}\qquad \ker P' = \im P. \]
Consequently, $V = \im P\oplus \im P'$.
\end{lemma}
\begin{proof}
Clearly $\id_V - P$ is idempotent: $(\id_V - P)^2 = \id_V - P - P + P^2 = \id_V - 2P + P = \id_V - P$. It is enough to show that $\im P' = \ker P$, because $P = \id_V - P'$.

Assume $v\in \ker P$. Then $P'v = v - Pv = v-0 = v$, so $v\in \im P'$.

Assume $v\in \im P'$. Then, by \ref{idempotentImageEquivalence}, $v = P'v = v - Pv$, so $Pv = 0$.

The last remark follows from \ref{directSumKernelImageIdempotent}
\end{proof}



TODO trace:
\begin{lemma}
Let $V$ be a vector space and $P$ an idempotent linear map. Then
\[ \Tr(P) = \dim(\im(P)). \]
\end{lemma}


\subsubsection{Invariant, reducing and irreducible subspaces}
\begin{definition}
Let $V$ be a vector space, $T$ a linear operator on $V$ and $U\subseteq V$ a subspace. Then $U$ is called
\begin{itemize}
    \item \udef{invariant} under $T$ is $T[U]\subseteq U$;
    \item \udef{reducing} for $T$ if $V = U\oplus W$ and both $U$ and $W$ are invariant under $T$;
    \item \udef{irreducible} w.r.t. $T$ if for all $W\subseteq U$ such that $W$ is reducing for $T$, we have $W = U$ or $W = \emptyset$.
\end{itemize}
\end{definition}

\begin{lemma}
Let $V$ be a vector space, $T$ a linear operator on $V$ and $P$ an idempotent operator on $V$ with image $U = P[V]$. Then 
\begin{enumerate}
\item $U$ is invariant under $T$ \textup{if and only if} $PTP = TP$;
\item the following are equivalent:
\begin{enumerate}
\item $U$ is reducing for $T$;
\item $P[V]$ and $(\id_V - P)[V]$ are invariant under $T$;
\item $PT = PTP = TP$;
\item $PT = TP$.
\end{enumerate} \textup{if and only if} .
\end{enumerate}
\end{lemma}
\begin{proof}
(1) The invariance of $U$ under $T$ can be stated as $TP[V] \subseteq \im P$. By \ref{idempotentImageEquivalence} this can be restated as $TPv = PTPv$ for all $v\in V$.

(2) Points (a) and (b) are equivalent by \ref{idempotentImageEquivalence}.

Points (b) and (c) are equivalent by point (1) and $(\id_V - P)T(\id_V - P) = T - PT - TP + PTP = T(\id_V - P) + PTP - PT$.

Point (d) follows immediately from (c). The converse follows from $P(TP) = P(PT) = PT$ and $(PT)P = (TP)P = TP$.
\end{proof}

\subsubsection{Irreducible operators}
\begin{definition}
Let $V$ be a vector space and $T$ an operator on $V$. Then $T$ is called \udef{irreducible} if $V$ is irreducible w.r.t. $T$.
\end{definition}

\section{Sets of vectors}
\begin{proposition}
Let $V$ be vector space and consider a function $f: \powerset{V} \to \powerset{V}$ and define
\[ \mathcal{X} = \setbuilder{X\subseteq V}{A \subseteq X \implies f(A)\subseteq X}. \]
Then $\mathcal{X}$ is closed under arbitrary intersections and thus a complete sublattice of $\powerset(V)$.

The closure operator into $\mathcal{X}$ is given by the intersection of all supersets in $\mathcal{X}$.
\end{proposition}

Most of the types of sets of vectors in this section are of this form.

\subsection{Star-shaped sets}
\begin{definition}
A subset $S$ of a real or complex vector space $V$ is called
\begin{itemize}
\item \udef{star-shaped} at $a\in V$ if for all $x\in S$ and $0\leq r \leq 1$, $rx + (1-r)a\in C$;
\item \udef{absolutely star-shaped} at $a\in V$ if for all $x\in S$ and $|r| \leq 1$, $rx + (1-r)a\in C$.
\end{itemize}
\end{definition}

\subsection{Affine sets}
\begin{definition}
A subset $A$ of a real or complex vector space $V$ is called \udef{affine} if for all $x,y\in A$ and $\lambda\in\F$, $\lambda x + (1-\lambda)y\in A$.

The closure of a set $X$ into the lattice of affine sets is called the \udef{affine hull} of $X$ and is denoted $\affine(X)$.
\end{definition}

\begin{proposition}
Let $V$ be a vector space and $A\subseteq V$ a subset. Then following are equivalent:
\begin{enumerate}
\item $A$ is affine;
\item for all $x\in A$: the set $A-x$ is a vector subspace;
\item for some $x\in A$: the set $A-x$ is a vector subspace.
\end{enumerate}
\end{proposition}
\begin{proof}
$(1) \Rightarrow (2)$ Assume $A$ affine and take arbitrary $x\in A$. We verify the subspace criterion \ref{subspaceCriterion}: clearly $0\in A-x$ because $x\in A$.

Take $y-x, z-x \in A-x$. Then $y-x + z-x = 2\Big(\frac{1}{2}y + \frac{1}{2}z\Big) - x -x \in A-x$.

Take $y-x\in A-x$ and $\lambda\in \R$. Then $\lambda(y-x) = \lambda y + (1-\lambda)x - x \in A-x$.

$(2) \Rightarrow (3)$ Immediate.

$(3) \Rightarrow (1)$ Assume $A-x$ is a vector subspace for some $x\in A$. Take $y,z\in A$ and $\lambda\in \F$. Then $A-x \ni \lambda(y-x)+(1-\lambda)(z-x) = \lambda y + (1-\lambda)z - x$, so $\lambda y + (1-\lambda)z \in A$. 
\end{proof}


\subsection{Balanced set}
\begin{definition}
A subset $B$ of a vector space $V$ over a field $\F$ with valuation $|\cdot|$ is called \udef{balanced} if for all $|r|\leq 1$, $rC \subseteq C$.

\begin{itemize}
\item The closure of a set $X\subseteq V$ into the lattice of balanced sets is called the \udef{balanced hull} of $X$ and is denoted $\balanced(X)$.
\item The dual closure of a set $X\subseteq V$ into the lattice of balanced sets is called the \udef{balanced core} of $X$ and is denoted $\balancedCore(X)$.
\end{itemize}
\end{definition}
Note that $0$ is an element of any balanced set. The lattice of balanced sets is closed under unions and thus a complete sublattice of $\powerset(X)$.

\begin{lemma}
Let $V$ be a vector space and $B\subseteq V$ a subset. Then
\begin{enumerate}
\item $\balanced(B) = \bigcup_{|r|\leq 1}rB = \cball(0,1)\cdot B$;
\item $\balancedCore(B) = \begin{cases}
\bigcap_{|r|\geq 1}rB & 0\in B\\
\emptyset & 0\notin B
\end{cases}$.
\end{enumerate}
\end{lemma}

\begin{lemma}
Let $V$ be a vector space and $B\subseteq V$ a subset. Then the following are equivalent:
\begin{enumerate}
\item $B$ is balanced;
\item $\balanced(B) = \cball(0,1)\cdot B \subseteq B$;
\item $\balanced(B) = \cball(0,1)\cdot B = B$;
\item $B\subseteq \balancedCore(B)$;
\item for all $|r|\geq 1$, $C\subseteq rC$;
\item $B$ is symmetric and star-shaped at $0$.
\end{enumerate}
\end{lemma}

\begin{lemma} \label{balancedLemma}
Let $V$ be a vector space and $B\subseteq V$ a balanced subset. Then
\begin{enumerate}
\item for all $\lambda\in \F$: $\lambda B = |\lambda| B$.
\end{enumerate}
\end{lemma}

\begin{lemma} \label{balancedCoreConvexSet}
The balanced core of a convex set is convex.
\end{lemma}
\begin{proof}
Let $B\subseteq V$ be a convex subset of a vector space $V$. Then
$\balancedCore(B) = \begin{cases}
\bigcap_{|r|\geq 1}rB & 0\in B\\
\emptyset & 0\notin B
\end{cases}$. The empty set is convex. For all $r\in \F$, $rB$ is convex by \ref{translationScalingConvexSet} and arbitrary intersections of convex sets are convex.
\end{proof}

\subsection{Convex sets}
\begin{definition}
A subset $C$ of a real or complex vector space $V$ is called \udef{convex} if for all $x,y\in C$ and $0\leq r \leq 1$, $rx + (1-r)y\in C$.

The closure of a set $X\subseteq V$ into the lattice of convex sets is called the \udef{convex hull} of $X$ and is denoted $\convex(X)$.
\end{definition}
Note that this is a stronger property than metric convexity!
\begin{example}
Let $C$ be the set of all vectors with norm in $\Q\cap [0,1]$. The is metrically convex, but not a convex set of vectors.
\end{example}

\begin{lemma}
Let $V$ be a vector space and $X\subseteq V$ a subset. Then $\convex(X) = \setbuilder{rx + (1-r)y}{0\leq r \leq 1, x,y\in B}$.
\end{lemma}

\begin{lemma}
Let $V$ be a vector space an $X\subseteq V$ a subset. Then the following are equivalent:
\begin{enumerate}
\item $X$ is convex;
\item for all $0\leq r \leq 1$, $rX + (1-r)X \subseteq X$.
\end{enumerate}
\end{lemma}

\begin{lemma} \label{translationScalingConvexSet}
Let $V$ be a vector space over $\F$, $v\in V$, $\lambda\in \F$ and $X\subseteq V$ a convex subset subset. Then $v+\lambda X$ is convex.
\end{lemma}
\begin{proof}
Take $v+\lambda x_1, v+\lambda x_2 \in v+\lambda X$ and $r\in [0,1]$. Then
\[ r(v+\lambda x_1) + (1-r)(v+\lambda x_2) = v + \lambda(rx_1 + (1-r)x_2) \in v+\lambda X. \]
\end{proof}

\subsubsection{Absolutely convex sets}
\begin{definition}
A subset $B$ of a vector space $V$ over a field $\F$ with valuation $|\cdot|$ is called \udef{absolutely convex} or \udef{disked} if for all $x,y\in C$ and $|r| \leq 1$, $rx + (1-r)y\in C$.

The closure of a set $X\subseteq V$ into the lattice of absolutely convex sets is called the \udef{absolute convex hull} or \udef{disked hull} of $X$ and is denoted $\disked(X)$.
\end{definition}

\begin{lemma}
Let $V$ be a vector space and $X\subseteq V$ a subset. Then the following are equivalent:
\begin{enumerate}
\item $X$ is absolutely convex;
\item $X$ is convex and balanced;
\item for all $x,y\in X$ and $|r| \leq 1$:  $rx + (1-r)y\in X$;
\item for all $|a|+|b| \leq 1$, $aX +bX \subseteq X$;
\item for all $|a|+|b| \leq |c|$, $aX +bX \subseteq cX$.
\end{enumerate}
\end{lemma}

\begin{lemma}
Let $V$ be a vector space and $X\subseteq V$ a subset. Then $\disked(X) = \convex(\balanced(X))$.
\end{lemma}
In general $\disked(X) \neq \balanced(\convex(X))$.



\subsection{Cones}
\begin{definition}
A subset $C$ of a real or complex vector space $V$ is called a \udef{cone} if for all real $r>0$, $rC \subseteq C$. A cone is called
\begin{itemize}
\item \udef{pointed} if it contains the origin and \udef{blunt} if not;
\item \udef{flat} if $\exists x\neq 0: x\in C \land -x\in C$, and \udef{salient} if not.
\end{itemize}
The closure of a set $X$ into the lattice of cones is called the \udef{conic hull} of $X$ and is denoted $\conic(X)$.
\end{definition}

\begin{lemma}
Let $V$ be a vector space and $X\subseteq V$ a subset. Then $\conic(X) = \R^{> 0}\cdot X$.
\end{lemma}

The closure of a set $X\subseteq V$ into the lattice of cones is given by $\R\cdot X$.

\begin{lemma} \label{coneEqualityLemma}
A subset $C$ of a vector space $V$ is a cone \textup{if and only if} $rC = C$ for all $r> 0$.
\end{lemma}

\begin{lemma} \label{convexityAdditiveClosure}
A cone $C$ is convex if and only if $C + C \subseteq C$. 
\end{lemma}
\begin{proof}
Assume $C$ convex. Take $v,w\in C$, then $v/2 + w/2\in C$ by convexity and so $v+w = 2(v/2+w/2)\in C$.

Assume $C$ closed under addition. Take $v,w\in C$ and $\lambda\in[0,1]$. Then $(1-\lambda)v$ and $\lambda w$ are elements of $C$ and so the convex combination $(1-\lambda)v + \lambda w$ is too.
\end{proof}


\subsection{Absorbing sets}
\begin{definition}
Let $V$ be a vector space and $A,B\subseteq V$. The $A$ \udef{absorbs} $B$ if there exists a real $r>0$ such that for all $|c| \geq r$: $B\subseteq cA$.

The set $A$ is called \udef{absorbing} if it absorbs $\{v\}$ for all $v\in V$.
\end{definition}

\begin{lemma}
Let $V$ be a vector space and $A\subseteq V$ a subset. Then the following are equivalent:
\begin{enumerate}
\item $A$ is absorbing;
\item for all $v\in V$ there exists an $\epsilon\in \F$ such that $\epsilon v\in A$.
\end{enumerate}
\end{lemma}

\subsection{Translation invariance}
TODO Unique factorisation through $(x,y)\mapsto y-x$. (Universal property)

eg kernel, commutator, metric

\subsubsection{Quotient spaces}
TODO: need closed $U$? For quotient map to be continuous? TODO show quotient topology.

\begin{proposition}
Let $V$ be a vector space. Then $\mathfrak{q}\subset V\times V$ is a congruence \textup{if and only if} the set
\[ U_\mathfrak{q} = \setbuilder{w-v}{(v,w)\in\mathfrak{q}} \]
is a vector space.
\end{proposition}
\begin{proof} Then

- $\mathfrak{q}$ is reflexive iff $0\in U_\mathfrak{q}$;

- $\mathfrak{q}$ is symmetric iff $U_\mathfrak{q}$ is closed under multiplication with $-1$;

- $\mathfrak{q}$ is transitive iff $U_\mathfrak{q}$ is closed under addition;

- $\mathfrak{q}$ is a subalgebra of $V\oplus V$ iff $U_\mathfrak{q}$ is closed under addition and scalar multiplication.

As $U_\mathfrak{q}$ is a subset of $V$, we use the subspace criterion.
\end{proof}
Then the equivalences
\[ [v]_\mathfrak{q}=[w]_\mathfrak{q} \iff (v,w)\in\mathfrak{q} \iff w-v\in U_\mathfrak{q} \iff w+U_\mathfrak{q} = v+U_\mathfrak{q} \]
motivate the following definition:
\begin{definition}
Let $V$ be a vector space.
\begin{itemize}
\item An \udef{affine subset} of $V$ is a subset of $V$ of the form $v+U$ for some $v\in V$ and some subspace $U$ of $V$.
\item An affine subset $v+U$ is \udef{parallel} to $U$.
\end{itemize}
Suppose $U$ subspace of $V$. The \udef{quotient vector space} $V/U$ is the vector space of all affine subsets of $V$ parallel to $U$:
\[ V/U = \{ v+U \;|\; v\in V \}, \]
which is a vector space by virtue of being a quotient algebra.

We call the dimension of $V/U$ the \udef{codimension} of $U$ in $V$:
\[ \codim(U) = \dim(V/U). \]
\end{definition}

\begin{proposition}
Let $U$ be a subspace of a vector space $V$. Then
\[ \dim V = \dim U + \dim V/U = \dim U + \codim U.  \]
\end{proposition}
\begin{proof}
Apply the dimension theorem for linear maps to the quotient map.
\end{proof}

\begin{definition}
Let $f:V\to W$ be a linear map of vector spaces. The \udef{cokernel} of $f$ is the quotient space
\[ \coker(f) = W/\im(f). \]
The dimension of the cokernel is called the \udef{corank}.
\end{definition}
\begin{lemma}
Let $U$ be a subspace of a vector space $V$. The codimension of $U$ is the corank of the inclusion $U\hookrightarrow V$:
\[ \codim(U) = \dim\coker(U\hookrightarrow V). \]
\end{lemma}

\begin{proposition} \label{splittingMap}
Let $T\in \Lin(V,W)$. Then $T$ induces a linear map
\[ \tilde{T}: V/\ker(T) \to W: v +\ker(T) \mapsto Tv \]
with the following properties:
\begin{enumerate}
\item $\tilde{T}$ is injective;
\item $\im\tilde{T} = \im T$;
\item $\tilde{T}$ is an isomorphism from $V/\ker(T)$ to $\im T$.
\end{enumerate}
\end{proposition}

 TODO each short exact sequence of vector spaces splits \url{https://en.wikipedia.org/wiki/Rank%E2%80%93nullity_theorem}


\section{Functionals}
\begin{definition}
Let $V$ be a vector space over a field $\mathbb{F}$.
\begin{enumerate}
\item A \udef{functional} on $V$ is a map $V\to \F$;
\item A \udef{linear functional} on $V$ is a linear map from $V$ to $\mathbb{F}$;
\item A \udef{real functional} on $V$ is a map $V\to \R$.
\end{enumerate}
\end{definition}

\begin{lemma} \label{kernelHyperplane}
Let $V$ be a vector space and $U\subseteq V$ a subspace. Then $U$ is a hyperplane \textup{if and only if} it is the kernel of a functional.
\end{lemma}

\begin{lemma} \label{functionalBoundedNeighbourhood}
Let $f: V\to \F$ be a linear functional and $x\notin \ker(f)$. Let $A\subseteq V$ be a balanced set. Then $(x+A)\perp \ker(f)$ \textup{if and only if} $A \subseteq f^{\preimf}(\ball(0,|f(x)|))$.
\end{lemma}
\begin{proof}
Suppose $A \subseteq f^{\preimf}(\ball(0,|f(x)|))$. Then for all $a\in A$: $f(x+a) = f(x) + f(a) \neq 0$.

Conversely, suppose $A \not\subseteq f^{\preimf}(\ball(0,|f(x)|))$, i.e.\ there exists $a\in A$ such that $|f(a)| \geq |f(x)|$. Then $v= -\frac{f(x)}{f(a)}a\in A$, because $A$ is balanced and so $f(x+ v) = f(x)-\frac{f(x)}{f(a)}f(a) = 0$ and so $(x+A) \mesh \ker(f)$.
\end{proof}

\subsection{Real functionals}
\begin{definition}
Let $V$ be a real or complex vector space. Let $f: V\to \R$ be a real functional. We say
\begin{itemize}
\item $f$ is \udef{subadditive} or satisfies the \udef{triangle inequality} if $\forall x,y\in V: f(x+y) \leq f(x) + f(y)$;
\item $f$ is \udef{convex} if $\forall x,y\in V, \lambda\in[0,1]: f(\lambda x + (1-\lambda)y) \leq \lambda f(x) + (1-\lambda)f(y)$;
\item $f$ is \udef{positively homogeneous} if $\forall x\in V,\lambda\geq 0: f(\lambda x) = \lambda f(x)$;
\item $f$ is \udef{absolutely homogeneous} if $\forall x\in V,\lambda\in\F: f(\lambda x) = |\lambda| f(x)$;
\item $f$ \udef{separates points} if $\forall v\in V: f(v) = 0 \implies v = 0$.
\end{itemize}
We call $f$
\begin{itemize}
\item \udef{sublinear} if it is subadditive and positively homogeneous;
\item a \udef{seminorm} if it is subadditive and absolutely homogeneous;
\item a \udef{norm} if it is a point-separating seminorm.
\end{itemize}
\end{definition}
TODO general valued fields.

\begin{lemma}
Let $V$ be a real or complex vector space and $f: V\to \R$ be a real functional. Then
\begin{enumerate}
\item absolute homogeneity $\implies$ positive homogeneity;
\item subadditivity+positive homogeneity $\implies$ convexity $\implies$ subadditivity.
\end{enumerate}
\end{lemma}
Thus norms and seminorms are sublinear.

\begin{lemma}
A subadditive, absolutely homogenous function $f:V\to \R$ is non-negative:
\[ f: V\to \R_{\geq 0}. \]
Thus norms and seminorms are functions $V\to \R_{\geq 0}$.
\end{lemma}
\begin{proof}
For all $v\in V$ we have $0 = f(v-v) \leq f(v)+f(-v) = 2f(v)$, so $f(v) \geq 0$.
\end{proof}

\begin{proposition}[Reverse triangle inequality] \label{reverseTriangleInequality}
Let $V$ be a vector space and $\norm{\cdot}: V\to \R$ a function that satisfies the triangle inequality and has $\norm{-v} = \norm{v}$ for all $v\in V$. Then $\forall v,w\in V$:
\begin{enumerate}
\item $|\norm{v}-\norm{w}|\leq \norm{v-w}$;
\item $|\norm{v}-\norm{w}|\leq \norm{v+w}$.
\end{enumerate}
In particular this holds if $\norm{\cdot}$ is a norm or seminorm.
\end{proposition}
\begin{proof}
We calculate $\norm{v} = \norm{v-w+w} \leq \norm{v-w} + \norm{w}$, so $\norm{v}-\norm{w}\leq \norm{v-w}$. By swapping $v\leftrightarrow w$ we also get $-\norm{v}+\norm{w}\leq \norm{w-v} = \norm{v-w}$ and thus the first inequality is established.

For the second inequality, set $w\to -w$ and use $\norm{-w} = \norm{w}$.
\end{proof}

\subsubsection{Epigraphs}
\begin{definition}
Let $V$ be a vector space and $f: V\to \R$ a real functional on $V$. Then \udef{epigraph} of $f$ is defined as
\[ \epigraph(f) \defeq \setbuilder{(v,r)\in V\times \R}{f(v)\leq r}. \]
\end{definition}

\begin{lemma} \label{epigraphLemma}
Let $V$ be a vector space and $f: V\to \R$ a real functional on $V$. Then for all $v\in V$:
\[ f(v) = \inf\setbuilder{r}{(v,r)\in \epigraph(f)}. \]
\end{lemma}

\begin{proposition}
Let $V$ be a real vector space and $f: V\to \R$ a functional. Then
\begin{enumerate}
\item $f$ is convex \textup{if and only if} $\epigraph(f)$ is a convex subset of $V\oplus \R$;
\item $f$ is positively homogeneous \textup{if and only if} $\epigraph(f)$ is a cone in $V\oplus \R$.
\end{enumerate}
\end{proposition}
\begin{proof}
(1) First assume $f$ convex and pick $(v, s), (w,t)\in \epigraph(f)$ and $\lambda\in [0,1]$. Then we need to show that $(\lambda v + (1-\lambda)w, \lambda s + (1-\lambda)t) \in \epigraph(f)$. This is equivalent to saying $f(\lambda v + (1-\lambda)w) \leq \lambda s + (1-\lambda)t$. Indeed we have $f(\lambda v + (1-\lambda)w) \leq \lambda f(v) + (1-\lambda)f(w) \leq \lambda s + (1-\lambda)t$ by the convexity of $f$.

Conversely, assume $\epigraph(f)$ convex. Then $(v, f(v)), (w,f(w))\in \epigraph(f)$, $(\lambda v + (1-\lambda)w, \lambda f(v) + (1-\lambda)f(w)) \in \epigraph(f)$ for all $\lambda\in [0,1]$. This implies $f(\lambda v + (1-\lambda)w) \leq \lambda f(v) + (1-\lambda)f(w)$.

(2) First assume $f$ is positively homogeneous, take $(v,s)\in \epigraph(f)$ and $r>0$. Then we need to show that $r(v,s) = (rv,rs)\in \epigraph(f)$. This follows because of the implications $f(v)\leq s \implies rf(v) \leq rs \implies f(rv) \leq rs$.

Conversely, assume that $\epigraph(f)$ is a cone. Then $\lambda\cdot \epigraph(f) = \epigraph(f)$ for all $\lambda>0$ by \ref{coneEqualityLemma}. We then calculate using \ref{epigraphLemma}:
\begin{align*}
f(\lambda v) &= \inf\setbuilder{r}{(\lambda v,r)\in \epigraph(f)} \\
&= \inf\setbuilder{r}{(\lambda v,r)\in \lambda\cdot\epigraph(f)} \\
&= \inf\setbuilder{r}{\lambda(v,\lambda^{-1}r)\in \lambda\cdot\epigraph(f)} \\
&= \inf\setbuilder{r}{(v,\lambda^{-1}r)\in \epigraph(f)} \\
&= \inf\setbuilder{\lambda r}{(v,r)\in \epigraph(f)} = \lambda f(v).
\end{align*} 
\end{proof}
\begin{corollary}
A functional on a real vector space is sublinear \textup{if and only if} its epigraph is a convex cone.
\end{corollary}

\subsubsection{Convex functionals}

\begin{proposition}
Let $p: V\to\R$ be convex functional. Then
\[ P: V\to\R: x\mapsto \inf_{t>0} t^{-1}p(tx) \]
is sublinear and $P(x)\leq p(x)$.

Also, if $f:V\to \R$ is a linear functional, then $f\leq p \iff f\leq P$.
\end{proposition}
\begin{proof}
For sublinearity: let $x,y\in V$, then for all $s,t>0$
\[ P(x+y) \leq \frac{s+t}{st}p\left(\frac{st}{s+t}(x+y)\right) = \frac{s+t}{st}p\left(\frac{s}{s+t}(tx)+\frac{t}{s+t}(sy)\right) \leq t^{-1}p(tx) + s^{-1}p(sy). \]
This implies that $P(x+y)\leq P(x)+P(y)$.

For positive homogeneity: let $x\in V,\lambda\geq 0$
\[ P(\lambda x) = \inf_{t>0} t^{-1}p(t\lambda x) = \inf_{t\lambda>0} \lambda (t\lambda)^{-1}p(t\lambda x) = \inf_{t>0} \lambda (t)^{-1}p(tx) = \lambda P(x). \]

Finally we prove that $f\leq p \implies f\leq P$ for linear functionals $f$. For all $t>0$ we have $f(tx) \leq p(tx)$, which implies $f(x) = t^{-1}f(tx) \leq t^{-1}p(tx) \leq P(x)$. So $f\leq P$.
\end{proof}

\subsubsection{Seminorms}
\begin{lemma}
The kernel of a seminorm is a vector space.
\end{lemma}
Note this does not follow from \ref{kernelSubspace} because seminorms are not linear.
\begin{proof}
Let $p:V\to \R$ be a seminorm. We verify the subspace criterion \ref{subspaceCriterion}. First $0\in\ker(p)$ because $p(0) = p(0\cdot 0) = |0|p(0) = 0$.

Now take $v,w\in \ker(p)$ and $\lambda\in \F$. Then $0\leq p(v+\lambda w) \leq p(v)+|\lambda|p(w) = 0$, so $v+\lambda w\in\ker(p)$.
\end{proof}

\begin{proposition} \label{gaugeSeminorms}
Let $V$ be a vector space, $p: V\to \R$ a seminorm and $\lambda\in \F$. Then
\[ \setbuilder{v\in V}{p(v) < \lambda} \qquad\text{and}\qquad \setbuilder{v\in V}{p(v) \leq \lambda} \]
are absolutely convex and absorbent.
\end{proposition}
\begin{proof}
Take $|\mu| + |\nu| \leq 1$ and $v,w\in\setbuilder{v\in V}{p(v) \leq \lambda}$. Then $p(|\mu|v + |\nu|w)\leq |\mu|p(v) + |\nu|p(w) \leq (|\mu|+|\nu|)\lambda \leq \lambda$.

For absorbence, take $v\in V$. Then $\frac{\lambda}{2p(v)} v\in \setbuilder{v\in V}{p(v) < \lambda}$.
\end{proof}

\subsubsection{Gauges}
\begin{definition}
Let $V$ be a vector space and $A\subseteq V$ an absorbent subset. The function
\[ p_A: V\to \R^{\geq 0}: v\mapsto \inf\setbuilder{\lambda\in \R^{\geq 0}}{v\in \lambda A} \]
is called the \udef{gauge} or \udef{Minkowski functional} of $A$.
\end{definition}
The function $p_A$ is well-defined (i.e.\ $p_A(v)$ is finite for all $v\in V$) because $A$ is absorbent.

\begin{lemma} \label{gaugeLemma}
Let $V$ be a vector space and $A\subseteq V$ an absorbent subset and $\lambda\in \R^{> 0}$. Then
\begin{enumerate}
\item $\lambda > p_A(v) \implies \lambda^{-1}v\in A$;
\item $\lambda^{-1}v\in A \implies \lambda \geq p_A(v)$.
\end{enumerate}
In particular, we have
\[ p_A^{\preimf}[\ball(0,1)] = \setbuilder{v\in V}{p_A(v) < 1} \subseteq A \subseteq \setbuilder{v\in V}{p_A(v) \leq 1} = p_A^{\preimf}[\cball(0,1)]. \]
\end{lemma}

\begin{proposition} \label{gaugeClassification}
Let $V$ be a vector space and $f: V\to \R^{\geq 0}$ a function.
Then the following are equivalent:
\begin{enumerate}
\item $f$ is positively homogenous;
\item for all $A\subseteq V$ such that $f^{\preimf}(\ball(0,1)) \subseteq A \subseteq f^{\preimf}(\cball(0,1))$, $A$ is absorbent and $f = p_A$;
\item $f = p_A$ for some absorbent subset $A$.
\end{enumerate}
\end{proposition}
Note that positive homogeneity is equivalent to strictly positive homogeneity.
\begin{proof}
$(1) \Rightarrow (2)$ To show absorbence, take some $v\in V$. Then for any $\epsilon>0$, $f\big((f(v)+\epsilon)^{-1}v\big) = \frac{f(v)}{f(v)+\epsilon} < 1$, so $(f(v)+\epsilon)^{-1}v\in A$.

Now take some $A$ and fix some arbitrary $v\in V$. We have $p_A(v) = \inf\setbuilder{\lambda\in \R^{\geq 0}}{v\in \lambda A}$, so
\[ \begin{aligned}
p_A(v) &\leq \inf\setbuilder{\lambda\in \R^{\geq 0}}{v\in \lambda f^{\preimf}(\ball(0,1))} \\
&= \inf\setbuilder{\lambda\in \R^{\geq 0}}{v\in f^{\preimf}(\ball(0,\lambda))} \\
&= \inf\setbuilder{\lambda\in \R^{\geq 0}}{f(v) < \lambda} = f(v)
\end{aligned} \quad\text{and}\quad \begin{aligned}
p_A(v) &\geq \inf\setbuilder{\lambda\in \R^{\geq 0}}{v\in \lambda f^{\preimf}(\cball(0,1))} \\
&= \inf\setbuilder{\lambda\in \R^{\geq 0}}{v\in f^{\preimf}(\cball(0,\lambda))} \\
&= \inf\setbuilder{\lambda\in \R^{\geq 0}}{f(v) \leq \lambda} = f(v).
\end{aligned} \]
We conclude that $f(v) = p_A(v)$.

$(2) \Rightarrow (3)$ Immediate.

$(3) \Rightarrow (1)$ We calculate for $t \geq 0$
\begin{align*}
f(tv) &= \inf\setbuilder{\lambda\in \R^{\geq 0}}{tv\in \lambda A} = \inf\setbuilder{\lambda\in \R^{\geq 0}}{v\in t^{-1}\lambda A} \\
&= \inf\setbuilder{t\lambda\in \R^{\geq 0}}{v\in \lambda A} = t\inf\setbuilder{\lambda\in \R^{\geq 0}}{v\in \lambda A} = tf(v).
\end{align*}
\end{proof}

\begin{lemma} \label{gaugeZeroLemma}
Let $V$ be a vector space, $A\subseteq V$ an absorbent subset and $a\in A$. If there exists a subspace $U\subseteq A$ such that $a\in U$, then $p_A(a) = 0$.
\end{lemma}
\begin{proof}
For all $\epsilon > 0$, $\epsilon^{-1}a\in A$, so $a\in \epsilon A$.
\end{proof}

\begin{proposition}
Let $V$ be a vector space and $A\subseteq V$ an absorbent subset. Then
\begin{enumerate}
\item $p_A$ is absolutely homogenous if $A$ is balanced;
\item $p_A$ is subadditive if $A$ is convex;
\item $p_A$ is point-separating if $A$ is balanced and contains only the trivial subspace.
\end{enumerate}
\end{proposition}
\begin{proof}
(1) By \ref{balancedLemma} we have $\mu A = |\mu| A$ and thus
\begin{align*}
p_A(\mu\cdot v) &= \inf\setbuilder{\lambda\in \R^{\geq 0}}{\mu\cdot v\in \lambda A} = \inf\setbuilder{\lambda\in \R^{\geq 0}}{v\in \frac{\lambda}{\mu} A} \\
&= \inf\setbuilder{\lambda\in \R^{\geq 0}}{v\in \frac{\lambda}{|\mu|} A} = \inf\setbuilder{|\mu|\lambda\in \R^{\geq 0}}{v\in \lambda A} = |\mu|\cdot p_A(v).
\end{align*}

(2) Take $v,w\in V$. Now take arbitrary $\epsilon > 0$, so $(p_A(v)+\epsilon)^{-1}v \in A$ and $(p_A(w)+\epsilon)^{-1}w \in A$ by \ref{gaugeLemma}. By convexity of $A$, we have
\[ \frac{v+w}{p_A(v)+p_A(w)+2\epsilon} = \frac{p_A(v)+\epsilon}{p_A(v)+p_A(w)+2\epsilon}(p_A(v)+\epsilon)^{-1}v + \frac{p_A(w)+\epsilon}{p_A(v)+p_A(w)+2\epsilon}(p_A(w)+\epsilon)^{-1}w \in A. \]
By \ref{gaugeLemma} this means $p_A(v)+p_A(w)+2\epsilon \geq p_A(v+w)$ and because $\epsilon$ was arbitrary, we conclude that $p_A(v+w) \leq p_A(v)+p_A(w)$.

(3) Assume $A$ contains only the trivial subspace. Then for all $v\in V$ there exists some $\lambda\in \F$ such that $\lambda\cdot v\notin A$. Now for all $|c|\geq |\lambda|$, $c\cdot v\notin A$ because $A$ is balanced. Then $p_A(2\lambda\cdot v) \neq 0$ and because $p_A$ is absolutely homogeneous we have $p_A(v) = (2\lambda)^{-1}p_A(2\lambda\cdot v) \neq 0$.
\end{proof}
\begin{corollary}
The gauge of an absolutely convex and absorbent subset is a seminorm. If the subset contains only the trivial subspace, then the gauge is a norm.
\end{corollary}

\begin{proposition}
Let $V$ be a vector space, $A,B\subseteq V$ absolutely convex and absorbent subsets and $\mathcal{E}$ a set of absolutely convex and absorbent subsets.
\begin{enumerate}
\item For all $\lambda\in \F\setminus\{0\}$, the gauge of $\lambda A$ is $|\lambda|^{-1}p_A$.
\item The gauge of $\bigcap \mathcal{E}$ is $\sup\setbuilder{p_K}{K\in \mathcal{E}}$.
\item If $A\subseteq B$, then $p_B \leq p_A$.
\end{enumerate}
\end{proposition}

\subsection{Hahn-Banach extension theorems}
\begin{theorem}[Hahn-Banach majorised by convex functionals] \label{convexHahnBanach}
Let $V$ be a real vector space, $U\subset V$ a subspace and $p$ a convex functional on $V$. Let $f:U\to\R$ be a linear functional that is bounded by $p$:
\[ \forall u\in U: \quad f(u) \leq p(u). \]
Then $f$ has an extension $\tilde{f}: V\to \R$ such that $\tilde{f}$ is a linear functional on $V$ bounded by $p$:
\[ \forall v\in V: \tilde{f}(v) \leq p(v) \qquad \text{and} \qquad \forall u\in U: \tilde{f}(u) = f(u). \]
\end{theorem}
\begin{proof}
As a first step, we want to extend $f$ to a functional $g$ on a space that is one dimension larger than $U$. This means $g$ is of the form
\[ g: U\oplus\Span\{v_1\}\to\R: v + \alpha v_1 \mapsto f(v) + \alpha c \]
for some $v_1\in V\setminus U$.

If we want $g$ to be majorised by $p$, then we need to find a $c$ such that
\[ \forall v\in U: \forall \alpha\in\R: \; g(\alpha v_1 + v) = \alpha c + f(v) \leq p(\alpha v_1 + v) \]
this means that we need
\[ \forall v\in U: \forall \alpha\in\R:\; \frac{-p(v - |\alpha|v_1) + f(v)}{|\alpha|} \leq c \leq \frac{p(v + |\alpha|v_1) - f(v)}{|\alpha|} \]
and we can find such a $c$ if and only if
\[ \forall v\in U: \forall \alpha\in\R:\; -p(v - |\alpha|v_1) + f(v) \leq p(v + |\alpha|v_1) - f(v), \]
which is equivalent to $2f(v) \leq p(v+|\alpha|v_1)+p(v-|\alpha|v_1)$. This follows from
\begin{align*}
f(v) \leq p(v) &= p(\tfrac{1}{2}(v+|\alpha|v_1) + \tfrac{1}{2}(v-|\alpha|v_1)) \\
&\leq \tfrac{1}{2}p(v+|\alpha|v_1) + \tfrac{1}{2}p(v-|\alpha|v_1).
\end{align*}
So we can extend the domain of $f$ by one dimension such that it is still majorised by $p$.

An extension by multiple dimensions is determined by a subset of $V\times \R$. Consider the family of all such subsets that determine a majorised extension of $f$. This is a family of finite character. We apply the Teichmüller-Tukey lemma, \ref{ZornEquivalents}, to obtain a maximal element.

This maximal element has domain $V$, because if it did not, it could be extended and was not a maximal element.
\end{proof}
Clearly if $V$ has a well-ordered Hamel basis, we do not need choice as we can just take successive $v$s in the basis and find $c$s constructively.
\begin{corollary}[Hahn-Banach majorised by sublinear functionals] \label{sublinearHahnBanach}
Any majorant $p$ that is sublinear is also convex and can be used in the Hahn-Banach theorem.
\end{corollary}
\begin{corollary}[Hahn-Banach majorised by seminorms] \label{seminormHahnBanach}
Let $(\mathbb{F},V,+)$ be a real or complex vector space, $U\subset V$ a subspace and $p$ a seminorm on $V$. Let $f:U\to\mathbb{F}$ be a linear functional that is bounded by $p$:
\[ \forall u\in U: \quad |f(u)| \leq p(u). \]
Then $f$ has an extension $\tilde{f}: V\to \R$ such that $\tilde{f}$ is a linear functional on $V$ bounded by $p$:
\[ \forall v\in V: |\tilde{f}(v)| \leq p(v) \qquad \text{and} \qquad \forall u\in U: \tilde{f}(u) = f(u). \]
\end{corollary}
\begin{proof}
For \emph{real} vector fields, we notice that every seminorm is a sublinear function, so we can use \ref{sublinearHahnBanach} to find an extension $\tilde{f}$. We then just need to check it satisfies $\forall v\in V: |\tilde{f}(v)| \leq p(v)$.
From \ref{sublinearHahnBanach} we know $\forall v\in V: \tilde{f}(v) \leq p(v)$.
To prove $-\tilde{f}(v) \leq p(v)$, we calculate
\[ -\tilde{f}(v) = \tilde{f}(-v) \leq p(-v) = |-1|p(v) = p(v). \]

For \emph{complex} vector fields, we can write $f= f_1 + if_2$ with $f_1,f_2$ real functionals on $U$, which can also be seen as a real vector space. First take $f_1$. Now $\forall u\in U f_1(u) \leq |f(x)| \leq p(x)$, so we can extend $f_1$ to $\tilde{f}_1$ by \ref{sublinearHahnBanach}.

Now by complex linearity, $if(u) = f(iu)$ so
\[ i[f_1(u) + if_2(u)] = -f_2(u) + if_1(u) = f_1(iu) + if_2(iu) \implies f_2(u) = -if_1(iu). \]
So we set $\tilde{f}(v) = \tilde{f}_1(v)-i\tilde{f}_1(iv)$. It is easy to show $\tilde{f}$ is $\C$-linear. For boundedness, write $\tilde{f}(v) = |\tilde{f}(v)|e^{i\theta}$ then
\[ |\tilde{f}(v)| = e^{-i\theta}\tilde{f}(v) = \tilde{f}(e^{-i\theta}v) = \tilde{f}_1(e^{-i\theta}v) \leq p(e^{-i\theta}v) = |e^{-i\theta}|p(v) = p(v). \]
\end{proof}

\begin{corollary}
Let $X$ be a normed space and $Z\subset X$ a subspace. Any bounded linear functional in $\tdual{Z}$ can be extended to a bounded linear functional in $\tdual{X}$ with the same norm.
\end{corollary}
\begin{proof}
Let $f:Z\to \mathbb{F}$ be such a functional. Extend $f$ by the previous theorem, \ref{seminormHahnBanach}, using $p(x) = \norm{f}_Z\norm{x}$.
\end{proof}
\begin{corollary} \label{existenceBoundedFunctionalOfSameNorm}
Let $X$ be a normed space and $x_0\neq 0$ an element of $X$. Then there exists a bounded linear functional $\omega_{x_0}$ such that
\[ \norm{\omega_{x_0}} = 1 \qquad \text{and} \qquad \omega_{x_0}(x_0)=\norm{x_0}. \]
\end{corollary}
\begin{proof}
Extend the functional $f: \Span\{x_0\}\to \mathbb{F}$ defined by
\[ f(x) = f(ax_0) = a\norm{x_0}. \]
\end{proof}
\begin{corollary}
Let $X$ be a normed space. Then $\forall x\in X:$
\[ \norm{x} = \sup_{\substack{f\in X' \\ f\neq 0}}\frac{|f(x)|}{\norm{f}}. \]
\end{corollary}
\begin{proof}
We calculate
\[ \norm{x} \geq \sup_{\substack{f\in X' \\ f\neq 0}}\frac{|f(x)|}{\norm{f}} \geq \frac{|\omega_{x}(x)|}{\norm{\omega_{x}}} = \frac{\norm{x}}{1} = \norm{x} \]
where the first inequality follows from $|f(x)|\leq \norm{f}\norm{x}$ for all $f\in X', x\in X$.
\end{proof}

\subsubsection{Hahn-Banach separation}

\begin{lemma} \label{gaugeSeparationLemma}
Let $V$ be a real vector space, $A$ an absorbent set and $x_0 \notin A$. Consider the functional $f_{x_0}: \Span\{x_0\}\to \F: tx_0 \mapsto t$. Then $f_{x_0}(x)\leq p_A(x)$ for all $x\in \Span\{x_0\}$.
\end{lemma}
\begin{proof}
Let $x = tx_0$. If $t\leq 0$, then the inequality is immediate. Suppose $t>0$. Because $p_A(x_0) \geq 1$ (by the converse of \ref{gaugeLemma}), we have
\[ f_{x_0}(x) = f_{x_0}(tx_0) = t \leq tp_A(x_0) = p_A(tx_0) = p_A(x)  \]
using positive homogeneity (\ref{gaugeClassification}).
\end{proof}

\begin{proposition}
Let $V$ be a real or complex vector space, $a\in V$ and $A$ a
subset such that $A-a$ is absolutely convex and absorbent. If $U$ is a subspace such that $A\perp U$, then there exists a hyperplane $H \supseteq U$ such that $A\perp H$.
\end{proposition}
\begin{proof}
Consider the set $U+A-a$. This is absolutely convex and absorbent.

Then we have
\[ U\perp A \implies 0\notin U+A \implies -a \notin U+A-a \implies a \notin U+A-a. \]
Consider the functional $f_{a}$ of \ref{gaugeSeparationLemma}, which is majorised by the gauge $p_{U+A-a}$. Then $f_a$ can be extended to all $V$ by the Hahn-Banach extension theorem \ref{seminormHahnBanach}.

We note that $U\subseteq \ker(f_a)$, because $p_{U+A-a}(u) = 0$ by \ref{gaugeZeroLemma}.

In order to conclude with \ref{functionalBoundedNeighbourhood}, we need to show that $A-a \subseteq f_a^{\preimf}(\ball(0,|f_a(a)|)) = f_a^{\preimf}(\ball(0,1))$.
Indeed $A-a \subseteq U+A-a \subseteq p_{U+A-a}^\preimf[\cball(0,1)] \subseteq f_{a}^\preimf[\cball(0,1)]$.
\end{proof}

\begin{proposition}
Let $A$ be a open convex subset of a locally convex TVS and $M$ a vector subspace such that $A\perp M$. Then there exists a closed hyperplane $H\supseteq M$ such that $A\perp H$.
\end{proposition}



\subsubsection{Banach limits}
\begin{proposition}
There exists a linear map $L:l^\infty(\N) \to \C$ satisfying
\begin{enumerate}
\item $\displaystyle L(x) = \lim_{n\to \infty}x_n$ if the limit exists;
\item $L((x_{n+1})_{n\in\N}) = L((x_n)_{n\in\N})$;
\item if $\forall n\in\N:x_n\geq 0$, then $L(x) \geq 0$;
\item $\norm{L} = 1$.
\end{enumerate}
Such a linear map is called a \udef{Banach limit}.
\end{proposition}
\begin{proof}
TODO, after Cesàro means.
\end{proof}





\chapter{Algebras}
TODO $\GL(A)$ which forms group under multiplication.

Multiplicative map: preserves multiplication.

Anti-commute

representation = algebra homomorphism with linear operators on a vector space.

Envelope of a representation: module to algebra.

\section{Definition}
\begin{definition}
Vector space with associative bilinear function.
\end{definition}


\section{Semisimple algebra}
\begin{proposition}
Let $V$ be a Hilbert space with a subspace $U$. Let $A$ be a bounded linear operator on $V$. If $U$ is stable under $A$, then $U^\perp$ is stable under $A^*$.
\end{proposition}
\begin{proof}
Let $u\in U$ and $v\in U^\perp$. Then from
\[ \inner{u, A^*v} = \inner{Au,v} = 0 \]
we see that $A^*v \in U^\perp$. Thus $U^\perp$ is stable under $A^*$.
\end{proof}
\begin{corollary}
Let $D$ be a ring of bounded linear operators on the Hilbert space $V$. If $D=D^*$, then $V$ is a semisimple $D$-module.
\end{corollary}
\begin{proof}
Let $S$ be the set of direct sums of simple subrepresentations of $V$:
\[ S = \setbuilder{\bigoplus_{i\in I}V_i }{\text{$V_i$ simple subrepresentations of $V$ and all $V_i$ are orthogonal}}. \]
Then $S$ is a poset ordered by inclusion. Now any chain $C$ in $S$ has an upper bound $\bigcup C$ and $\bigcup C$ is in $S$ because every $v\in\bigcup C$ can be written uniquely as a finite linear sum
\[ v = \sum_{\substack{i\in J\\ \text{$J$ finite}}} v_i \qquad v_i\in V_i. \]
By Zorn's lemma $S$ has a maximal element $U$, which is closed by \ref{directSumOrthogonalClosed}. We now claim that $U=V$. Assume, towards a contradiction, that $U\neq V$. Then $U^\perp$ is stable under $D$ by the proposition, closed by \ref{orthogonalComplementClosed} and thus contains a simple subrepresentation $W$ by \ref{existenceIrreps}. Then $U\subset U\oplus W \in S$, meaning $U$ is not a maximal element. This is a contradiction.
\end{proof}
Note that for the corollary it is important that $V$ be a Hilbert space, not only for the condition $D=D^*$ which could also be fulfilled by a set of symmetric operators or a group of unitary operators.


\section{Graded and filtered algebras}
TODO; move to rings. TODO move filtration:
\begin{definition}
Let $X$ be a set. A \udef{filtration} on $X$ is a family of subsets $\seq{X_i}_{i=0}^\infty$ such that $X_i \subseteq X_{i+1}$ and $X = \bigcup_{i=0}^\infty X_i$.
\end{definition}

\subsection{Graded algebras}
\begin{definition}
Let $A$ be an algebra and let $S$ be a semigroup. An \udef{$S$-grading} on $A$ is a set $\{A_s\}_{s\in S}$ of vector subspaces of $A$ indexed by $S$ such that $A_sA_t \subseteq A_{st}$ and $A = \bigoplus_{s\in S}A_s$.
\end{definition}


\begin{proposition}
Let $A$ be an algebra over a field $\F$, $f:A\to A$ a diagonalisable algebra homomorphism. Then
\begin{enumerate}
\item $\spec(f)$ is a multiplicative subsemigroup $S$ of $\F$;
\item $A_s \defeq \setbuilder{a\in A}{f(a) = sa}$ defines an $S$-grading on $\F$.
\end{enumerate}
\end{proposition}
TODO!!!

\begin{corollary}
Let $A$ be an algebra and $f$ and involutive algebra homomorphism. Then this involution defines a $Z_2$-grading.
\end{corollary}

\subsubsection{Grade operator}
\begin{definition}
Let $A = \bigoplus_{i=0}^\infty A_k$ be a graded algebra. Then, for $r\in \N$, we call the projection $A\to A_r: a\mapsto \grade{a}_r$ respecting this decomposition the \udef{grade operator}.
\end{definition}

\subsection{$\Z_2$-graded or superalgebras}

\subsection{Filtered algebra}
\begin{definition}
A \udef{filtered algebra} is an algebra $F$ together with a filteration $\seq{F_i}_{i=0}^\infty$ of subspaces such that $F_i\cdot F_j \subseteq F_{i+j}$.
\end{definition}

\subsubsection{Associated graded algebra}
\begin{definition}
Let $\sSet{F,\seq{F_i}_{i=0}^\infty}$ be a filtered algebra. The \udef{associated graded algebra} is defined as
\[ G = \bigoplus_{i=0}^\infty G_i \qquad\text{where}\qquad G_i = \begin{cases}
F_0 & (i=0) \\
F_{i}/F_{i-1} & (i \geq 1).
\end{cases} \]
\end{definition}
TODO define the multiplication!

TODO: $G$ is isomorphic to $F$ as vector space, but \emph{not} as an algebra!!


\section{Tensor algebra}
\[ \mathcal{T}(V) \defeq \R \bigoplus_{n=1}^\infty V^n = \R \bigoplus_{n=1}^\infty \underbrace{V\otimes \ldots \otimes V}_{\text{$n$ times}}. \]

Transpose: $v\otimes w \to w\otimes v$.

\subsection{Tensor product}
+ Graded tensor product

\section{Matrix algebras}
\begin{definition}
TODO
\end{definition}

\subsection{Natural isomorphism}
Remove parentheses block matrix.

Also $A^{n\times n} \cong \C^{n\times }\otimes A$.

\section{Algebra modules}
\begin{definition}
Let $A$ be an algebra over a field $\F$. A \udef{left $A$-module} is a vector space $V$ over $\F$ together with an operation $\cdot: A\times V \to V$ such that
\begin{itemize}
\item $\cdot: A\times V \to V$ determines a module when $A$ is thought of as a ring and $V$ as an Abelian group;
\item $\lambda(a\cdot v) = (\lambda a)\cdot v = a\cdot (\lambda)$ for all $\lambda\in \F, a\in A$ and $v\in V$.
\end{itemize}
\end{definition}

\chapter{Lie groups and Lie algebras}
\section{Definitions}
\[ g(x) = \exp(ix^aX_a) \]
Lie algebra has the operation $[X_a,X_b]= X_aX_b - X_bX_a$. 
\section{Matrix groups}
We now consider some extremely important examples of topological groups: the matrix groups.
If we take the set of real, $N\times N$ matrices with a non-zero determinant, it turns out that they form a group with the matrix multiplication:
\begin{enumerate}
\item The matrix multiplication is associative;
\item The identity is the identity matrix $\mathbb{1}$;
\item Because their determinant is not zero, every matrix in this set has an inverse.
\item Because the matrices are square, the multiplication of two matrices gives a matrix of the same dimensions. In other words the matrix multiplication is a closed operation.
\end{enumerate}
We call this group the \udef{real general linear group} $\GL(N, \R)$. It also has a complex counterpart, the complex general linear group $\GL(N, \C)$.

TODO: topological
We can also immediately see that the operations of matrix multiplication and inversion are smooth. (For inversion this is obviously only true after restriction to the open subset of invertible matrices, which luckily all matrix Lie groups are in turn a subset of). This follows quite readily because both operations are in effect comprised of addition and multiplication operations, which are infinitely differentiable. (e.g\ $A^{-1} = \frac{1}{\det(A)}\mathrm{adj}(A)$)

\begin{example}
TODO: $A^2 = \mathbb{1}$
\end{example}

These groups, along with all their subgroups, are known as the matrix groups and are very important in physics.

\subsubsection{Continuous parameters}
It is sometimes interesting to know how many degrees of freedom a particular set of transformations has. For example, rotations in the 2D plane are characterized with one parameter: the angle of rotation. In 3D we need three parameters. This notion of continuous parameter is formalised below.

\begin{definition}
A function $A : \R \to \GL(n, \C)$ is called a \udef{one-parameter subgroup} of $\GL(n, \C)$ if
\begin{enumerate}
\item $A$ is continuous,
\item $A(0) = \mathbb{1}_n$,
\item $A(t+s) = A(t)A(s)$ for all $t,s \in \R$.
\end{enumerate}
We also call the image of $A$ a one-parameter subgroup.
\end{definition}

A one-parameter subgroup has one continuous parameter. A subgroup of $\GL(n, \C)$ with $m$ continuous parameters, is a function $A : \R^m \to \GL(n, \C)$ such that each function of the form
\[ x \mapsto A(a_1, a_2, \ldots , a_{i-1}, x, a_{i+1}, \ldots, a_m) \]
gives a one-parameter subgroup for fixed $a_1,\ldots, a_m$.

We can speak of an $m$-parameter subgroup because, while different parametrisations may be found, any subgroup of $\GL(n,\C)$ constructed in this way must always be constructed with the same number of parameters. To see that this must be the case, consider two parametrised subgroups $A : \R^m \to \GL(n, \C)$ and $B : \R^{m'} \to \GL(n, \C)$ with the same image.

TODO !! + dimension of manifold

\subsubsection{Examples}
We now give names to the most important matrix groups, and list the number of continuous parameters.
\begin{enumerate}
\item General linear group
\[ \GL(N,\R) = \{N\times N \;\text{real matrices}, \; \det M \neq 0\} \]
\begin{itemize}
\item We have $N^2$ independent parameters (= the entries of the matrix), so $\dim \GL(N,\R) = N^2$
\item Each complex number can be described with two real ones, so $\dim \GL(N, \C) = 2N^2$
\end{itemize}
\item Special linear group
\[ \SL(N,\R) = \{M\in\GL(N,\R), \;\det M = 1\} \]
\begin{itemize}
\item $\dim \SL(N,\R) = N^2-1$: 1 dimension is used to fix determinant.
\item $\dim \SL(N,\C) = 2(N^2-1)$: 1 dimension is used to fix the real part of the determinant, and 1 to fix the imaginary part.
\end{itemize}
\item Unitary matrices
\[ \U(N) = \{U\in\GL(N,\C), U^\dagger\mathbb{1}_NU = \mathbb{1}_N\} \]
\begin{itemize}
\item $U^\dagger U$ is Hermitian, meaning that the complex transpose of $U$ is $U$.
\item $U^\dagger U = \mathbb{1}_N$ yields only $N^2$ independent equations, not $2N^2$ because of the Hermiticity of the equation.
\item $\dim \U(N) = 2N^2-N^2$ = $N^2$
\end{itemize}
\item Special unitary groups
\[ \SU(N) = \{U\in\U(N),\; \det U = 1\} \]
\begin{itemize}
\item For unitary matrices we have that $|\det U| = 1$. This fixes one continuous parameter and thus one dimension.
\item $\dim \SU(N) = N^2 - 1$
\end{itemize}
\item Orthogonal groups
\begin{itemize}
\item $\Ogroup(N) = \{O\in\GL(N,\R),\; O^\intercal\mathbb{1}_N O = \mathbb{1}_N\}$
\begin{itemize}
\item $O^\intercal O$ is symmetric, so $\frac{N(N+1)}{2}$ independent equations (half the matrix already fixed by the other half)
\item $\dim \Ogroup = N^2 - \frac{N(N+1)}{2} = \frac{N(N-1)}{2}$
\end{itemize}
\item $\SO(N) = \{O\in\Ogroup(N),\; \det O = 1\}$
\begin{itemize}
\item For orthogonal matrices $\det O = \pm 1$. This does not fix any continuous parameters.
\item $\dim \SO = \dim \Ogroup = \frac{N(N-1)}{2}$
\end{itemize}
\end{itemize}
\item Using a non definite metric $\eta = \diag(\mathbb{1}_p, -\mathbb{1}_q)$
\begin{itemize}
\item $\U(p,q) = \{ U\in \GL(N,\C), U^\dagger\eta U = \eta \}$
\item $\Ogroup(p,q) = \{ O\in \GL(N,\R), O^\intercal\eta O = \eta \}$ \\
In particular $\SO(1,3)$ is the \udef{Lorentz group} (with mostly minus convention).
\end{itemize}
\end{enumerate}

Here are some of the most important examples written more explicitly in terms of their continuous parameters:
\begin{itemize}
\item $\U(1) \equiv \{z\in\mathbb{C}|\; |z|=1\}, \boldsymbol{\cdot}$ has one real parameter. Every element $z$ of this group can be written $z=e^{i\alpha}$ for a real $\alpha$.
\item $\SO(2)$ has one real parameter.
\[ R(\theta) = \begin{pmatrix}\cos(\theta) & -\sin(\theta)\\ \sin(\theta) & \cos(\theta)\end{pmatrix} \]
\item $\SO(3)$ has three real parameters.
\[ R(\theta_{12},\theta_{13},\theta_{23}) = R_1(\theta_{12})R_2(\theta_{13})R_3(\theta_{23}) \]
where
\[R_1(\theta_{12}) = \begin{pmatrix}\cos(\theta_{12}) & -\sin(\theta_{12})&0\\ \sin(\theta_{12}) & \cos(\theta_{12})&0\\0&0&1\end{pmatrix}\]
\[R_2(\theta_{13}) = \begin{pmatrix}\cos(\theta_{13}) &0& -\sin(\theta_{13})\\0&1&0\\ \sin(\theta_{13}) &0& \cos(\theta_{13})\end{pmatrix}\]
\[R_3(\theta_{23}) = \begin{pmatrix}1&0&0\\ 0&\cos(\theta_{23}) & -\sin(\theta_{23})\\0& \sin(\theta_{23}) & \cos(\theta_{23})\end{pmatrix}\]
\item $\SU(2)$ has three real parameters and its elements can be seen as complex $2\times 2$ rotations.
\[ \U(\alpha, \beta, \gamma) = \begin{pmatrix}\cos\theta e^{i\alpha} & -\sin\theta e^{i\beta}\\ \sin\theta e^{-i\beta} & \cos\theta e^{-i\alpha}\end{pmatrix} \]
\end{itemize}


\chapter{Modules}
\section{Modules}
\begin{definition}
Let $\sSet{R, +, \cdot, 0,1}$ be a (unital) ring. A \udef{left $R$-module} is an Abelian group $\sSet{M, \boldsymbol{+}, \vec{0}}$ together with an operation $\boldsymbol{\cdot}: R\times M \to M$ such that, for all $r,s\in R$ and $x,y \in M$, we have
\begin{itemize}
\item $r \boldsymbol{\cdot} (x\boldsymbol{+}y) = r\boldsymbol{\cdot}x \boldsymbol{+} r\boldsymbol{\cdot} y$;
\item $(r+s)\boldsymbol{\cdot} x = r\boldsymbol{\cdot}x \boldsymbol{+} s\boldsymbol{\cdot} x$;
\item $(r\cdot s)\boldsymbol{\cdot}x = r\boldsymbol{\cdot}(s\boldsymbol{\cdot}x)$;
\item $1\boldsymbol{\cdot}x = x$.
\end{itemize}
A \udef{right $R$-module} is defined analogously in terms of an operation $\boldsymbol{\cdot}: M\times R \to R$.
\end{definition}
TODO: drop last requirement in the case of $R$ non-unital.

\subsection{Module morphisms}
The category of left $R$-modules is an Abelian category. If $f: M\to N$ is a module morphism, then $\ker(f)$ is a submodule of $M$ and $\im(f)$ is a submodule of $N$. (TODO)

\begin{lemma}
Injective iff $\ker = \{0\}$. 
\end{lemma}

\subsection{Simple modules}

\begin{theorem}[Schur's lemma] \label{SchursLemma}
Let $M,N$ be simple modules over a ring $R$ and $f:M\to N$ a homomorphism of $R$-modules. Then $f$ is either invertible or zero.
\end{theorem}
\begin{proof}
Since $\ker(f)$ is a submodule of $M$ and $M$ is simple, either $\ker(f) = M$ or $\ker(f) = \{0\}$. In the first case $f = 0$. Assume $f\neq 0$. We need to show that $f$ is invertible.

The injectivity of $f$ follows immediately from $\ker(f) = \{0\}$. For surjectivity, observe that $\im(f)$ is a submodule of $N$ and thus either $\{0\}$ or $\{N\}$. In the first case $f = 0$, which was excluded, so $\im(f) = N$ and $f$ is surjective.
\end{proof}
\begin{corollary}
Then endomorphism ring of a simple module is a division ring.
\end{corollary}

\section{Representation theory}
TODO: unitarity \url{https://sites.ualberta.ca/~vbouchar/MAPH464/section-unitary.html}.
\subsection{Group algebras}
Let $G$ be a group. Then the group operation $\cdot: G\times G\to G$ extends uniquely to a bilinear function $F(G)\times F(G) \to F(G)$ on the free vector space $F(G)$. (This is by \ref{universalPropertyFreeVectorSpace}, where we can apply the universal property to each argument separately and note that the order doesn't matter, TODO).

With this operation as multiplication, $F(G)$ becomes an algebra.

\begin{definition}
Let $G$ be a group and $\F$ a field. The algebra $F_\F(G)$ described above is called the \udef{group algebra} and is denoted $\F[G]$.
\end{definition}

\subsection{Representations}

\begin{definition}
Let $G$ be a group and $V$ a vector space. 

A \udef{representation} of $G$ on $V$ is a group homomorphism $R: G\to \GL(V)$.
\end{definition}
In other words, we require $R(g_1)R(g_2) = R(g_1g_2)$ for all $g_1,g_2\in G$.

\begin{example}
Let $G$ be a group and $V$ a vector space. The \udef{trivial representation} of $G$ on $V$ is given by $R(g) = \id_V$ for all $g\in G$.
\end{example}

\begin{lemma} \label{representationEquivalents}
Let $V$ be a vector space over a field $\F$ and $G$ a group.
\begin{enumerate}
\item Every representation $R$ of $G$ on $V$ can be uniquely extended to an algebra homomorphism $\overline{R}: \F[G]\to \GL(V)$. Conversely, every algebra homomorphism $\overline{R}: \F[G]\to \GL(V)$ restricts to a representation.
\item Every representation $R$ of $G$ on $V$ makes $V$ a $\F G$-module with module operation
\[ \boldsymbol{\cdot}: \F[G]\times V\to V: \Big(\sum_{i=0}^{n-1}\lambda_ig_i, v\Big) \mapsto \sum_{i=0}^{n-1}\lambda_iR(g_i)(v).  \]
Conversely, every $\F[G]$-module determines a representation.
\end{enumerate}
\end{lemma}
The points of this lemma give two more equivalent ways to see representations.
\begin{proof}
The extensions of the operations to the group algebra go by the universal property \ref{universalPropertyFreeVectorSpace}.
\end{proof}

Viewing representations as $\F[G]$-modules allows the use of the definitions and results of universal algebra. We will often leave the representation $R$ implicit and call the vector space $V$ the representation, since it carries an operation for each element of the group algebra $\F[G]$.

\subsection{Morphisms between representations}

\begin{lemma}
Let $G$ be a group and $V,W$ vector spaces that carry representations of $G$. Let $\varphi: V\to W$ be a linear map such that
\begin{center}
\begin{tikzcd}
V\arrow[r, "\varphi"]\arrow[d, "g\cdot -"]& W\arrow[d, "g\cdot -"] \\V\arrow[r, "\varphi"]& W
\end{tikzcd}
commutes for every $g\in G$.
\end{center}
Then $\varphi$ is a $\F[G]$-module homomorphism.
\end{lemma}
In other symbols $\forall g\in G: \forall v\in V: g\cdot \varphi(v) = \varphi\big(g\cdot v\big)$.
\begin{proof}
For all $\seq{\lambda_i}_{i=0}^{n-1}\in \F^n$ and $\seq{g_i}_{i=0}^{n-1}\in G^n$, we have
\[ \varphi\bigg(\Big(\sum_{i=0}^{n-1}\lambda_i g_i\Big)\cdot v\bigg)  = \varphi\Big(\sum_{i=0}^{n-1}\lambda_i g_i\cdot v\Big) = \sum_{i=0}^{n-1}\lambda_i\varphi(g_i\cdot v) = \sum_{i=0}^{n-1}\lambda_i g_i\cdot \varphi(v) = \Big(\sum_{i=0}^{n-1}\lambda_i g_i\Big)\cdot \varphi(v). \]
\end{proof}

\begin{definition}
A map $\varphi$ is called a \udef{morphism of representations} if it is a $\F[G]$-module homomorphism. We also call such a map \udef{$G$-linear}.
\end{definition}

\subsection{Subrepresentations}
\begin{definition}
A \udef{subrepresentation} of a representation $R$ of $G$ on $V$ is a vector subspace $W$ of $V$ that is invariant under $G$: $\forall g\in G: \big(R(g)\big)^\imf(W) \subseteq W$.


\end{definition}
The notion of subrepresentation is exactly the notion inherited from universal algebra.



$\ker \varphi$, $\coker \varphi$, $\Im \varphi$ subrepresentations.

Irrep has no proper, non-trivial subrepresentation.

\begin{lemma} \label{existenceIrreps}
Every representation has an irreduciple subrepresentation.
\end{lemma}
\begin{proof}
If $V$ has no non-trivial subrepresentations, then $V$ is simple and we are done. 
Otherwise take the set of non-trivial subrepresentations. This forms a poset ordered by inclusion and by the maximal chain principle \ref{ZornEquivalents} this poset has a maximal chain $C$ and it is clear that $\bigcap C$ is a simple subrepresentation. In particular $\bigcap C$ is closed because it is an intersection of closed subspaces.
\end{proof}

Representation gives representation on dual.

Direct sums and tensor products of representations. Also symmetric and exterior powers.

$\Hom(V,W)$ has representation via $V^*\otimes W$.

A space is irreducible if and only if it is completely reducible and indecomposable.

\begin{proposition}
Let $\varphi$ be a $G$-linear map. If $\varphi$ is invertible as a function, its inverse is also $G$-linear.
\end{proposition}
\begin{proof}
Assume $\varphi$ invertible and take an arbitrary $g\in G$. Then
\[ g\varphi = \varphi g \implies \varphi^{-1} g = g \varphi^{-1} \]
by multiplying left and right by $\varphi^{-1}$. So
\[ \forall g\in G: g \varphi^{-1} = \varphi^{-1} g \]
meaning $\varphi^{-1}$ is $G$-linear.
\end{proof}

\begin{proposition}
Let $(V_1,\rho_1)$ and $(V_2,\rho_2)$ be isomorphic, then $V_1$ is irreducible \textup{if and only if} $V_2$ is irreducible.
\end{proposition}

\begin{proposition}
Let $(V,\rho)$ be a representation of $G$. Every element of the centre $Z(G)$ of $G$ defines an isomorphism $V\to V$.
\end{proposition}
\begin{proof}
Every $g_0\in Z(G)$ defines a $G$-linear map $\rho(g_0)$:
\[ \forall g\in G: \rho(g_0)\rho(g) = \rho(g_0g) = \rho(gg_0) = \rho(g) \rho(g_0). \]
The map $\rho(g_0)$ has an inverse $\rho(g_0^{-1})$.
\end{proof}

\begin{proposition}
Let $G$ be an Abelian group. The irreps of $G$ are $1$-dimensional and thus homomorphisms
\[ \rho: G\to \GL(\C). \]
\end{proposition}
\begin{proof}
Let $V$ be an irrep. The action of each $g\in G$ is an isomorphism and thus a scalar multiple by Schur's lemma. Thus every subspace of $V$ must be invariant, so also a subrepresentation. This means $V$ may not have any proper, non-trivial subspaces, meaning it is $1$-dimensional.
\end{proof}

\section{Hilbert modules}
Let $B$ be a $C^*$-algebra. A \udef{Hilbert $B$-module} $E$ is essentially a $B$-module with a $B$-valued inner product $\inner{\cdot, \cdot}_B: E\times E \to B$.

To be more precise: a (right) Hilbert $B$-module is a complex Banach space $E$ equipped with a right $B$-module structure and a positive\footnote{i.e.\ $\inner{\xi, \xi}_B$ is an element of the positive cone $B^+$.} definite $B$-valued inner product which is linear in the second and anti-linear in the first and satisfies, for all $\xi,\eta \in E$ and $b\in B$
\[ (\inner{\xi,\eta}_B)^* = \inner{\eta,\xi}_B, \qquad \inner{\xi, \eta}_B b = \inner{\xi,\eta\cdot b}_B, \qquad \text{and} \qquad \norm{\xi}^2 = \norm{\inner{\xi,\xi}_B}. \]
We can also define left Hilbert $B$-modules analogously. The Hilbert $\C$-modules are precisely the complex Hilbert spaces.

Any $C^*$-algebra $B$ can be seen as a Hilbert $B$-module by equipping it with the following inner product:
\[ \inner{\cdot, \cdot}_B: B\times B\to B: (a,b) \mapsto a^*b. \]

If $E$ and $F$ are Hilbert $B$-modules, then a map $T: E \to F$ is called \udef{adjointable} if there exists a map $T^*: F \to E$ such that for all $\xi\in E, \eta\in F: \inner{T\xi,\eta}_B = \inner{\xi, T^*\eta}_B$. Adjointable operators are bounded and $B$-linear.





\chapter{Representation theory}
\section{Finite groups}
\subsection{Character tables}
\subsubsection{For $\mathbb{Z}_n$}
Denoting $\mathbb{Z}_n = \{\bar{0}, \bar{1}, \bar{2},\ldots, \overline{n-1}\}$
\[ \begin{array}{l|lllll}
g_i & \bar{0} & \bar{1} & \bar{2} & \hdots & \overline{n-1} \\
|\text{Cl}| & 1 & 1 & 1 & \hdots & 1 \\ \hline
\chi_0 & 1 & 1 & 1 & \hdots & 1 \\
\chi_1 & 1 & \omega_n & \omega^2_n & \hdots & \omega_n^{n-1} \\
\chi_2 & 1 & \omega_n^2 & \omega^4_n & \hdots & \omega_n^{2(n-1)} \\
\vdots & \vdots & \vdots & \vdots &  & \vdots \\
\chi_{n-1} & 1 & \omega_n^{n-1} & \omega_n^{2(n-1)} & \hdots & \omega_n^{(n-1)(n-1)}
\end{array} \]

\subsection{Complete reducibility of complex representations}

\subsection{Schur's lemma, isotypic decomposition and duals}
\subsection{Orthogonality in the character tables}
\subsection{The sum of squares formula}
\subsection{The number of irreps is the number of conjugacy classes}
\subsection{Dimensions of irreps divide the order of the group}

\chapter{Normed spaces and inner product spaces}
In this chapter we will always use either $\mathbb{F} = \R$ or $\mathbb{F} = \C$.

TODO: \url{https://math.stackexchange.com/questions/2151779/normed-vector-spaces-over-finite-fields/2568231}
\section{Normed spaces}
\begin{definition}
A \udef{norm} on a vector space $V$ is a function
\[ \norm{\cdot}: V \to \mathbb{R} \]
that has the following properties:
\begin{itemize}[leftmargin=6cm]
\item[\textbf{Triangle inequality}\footnote{Also known as the property of being \udef{subadditive}.}] $\norm{u+v} \leq \norm{u}+\norm{v}$;
\item[\textbf{Absolute homogeneity}] $\norm{\lambda v} = |\lambda|\cdot\norm{v}$;
\item[\textbf{Point-separating}] If $\norm{v} = 0$, then $v = 0$.
\end{itemize}
A \udef{seminorm} is a function $V\to \mathbb{R}$ that is subadditive and absolutely homogeneous.

A \udef{normed space} $(\mathbb{F},V,+,\norm{\cdot})$ is a vector space $(\mathbb{F},V,+)$ equipped with a norm $\norm{\cdot}$.
\end{definition}
\begin{lemma}
A subadditive, absolutely homogenous function $f:V\to \R$ is non-negative:
\[ f: V\to \R_{\geq 0}. \]
Thus norms and seminorms are functions $V\to \R_{\geq 0}$.
\end{lemma}
\begin{proof}
TODO
\end{proof}

\begin{lemma}[Reverse triangle inequality]
Let $(V,\norm{\cdot})$ be a normed space. Then $\forall v,w\in V: |\norm{v}-\norm{w}|\leq \norm{v-w}$.
\end{lemma}
\begin{proof}
$\norm{v} = \norm{v-w+w} \leq \norm{v-w} + \norm{w}$.
\end{proof}

\begin{definition}
A vector with norm 1 is called a \udef{unit vector}. Unit vectors are often written with a hat:
\[ \norm{\vhat{v}} = 1. \]
\end{definition}

\begin{lemma}
A subspace of a normed vector space is a normed space, with the norm given by the restriction of the norm in the larger space.
\end{lemma}

\begin{proposition}
Every normed space can be viewed as a metric space with the metric $d:V\times V \to [0,\infty[$ given by
\[ d(x,y) = \norm{x-y}. \]
This metric has the properties of
\begin{itemize}[leftmargin=6cm]
\item[\textbf{Translation invariance}] $d(x+a, y+a) = d(x,y)$;
\item[\textbf{Scaling}] $d(\lambda x, \lambda y) = |\lambda|d(x,y)$.
\end{itemize}
Conversely, any metric with translation invariance and scaling determines a norm:
\[ \norm{x} = d(x,\vec{0}). \]
Passing from norm to metric back to norm, we recover the original norm.
\end{proposition}
\begin{lemma}
A linear map $L:V\to W$ between normed spaces is an isometry for the metric \textup{if and only if} it preserves the norm, i.e.
\[ \forall v\in V: \quad \norm{v}_V = \norm{L(v)}_W. \]
\end{lemma}
\begin{proof}
Assume $L$ is an isometry, then
\[ \norm{v} = d(v,\vec{0}) = d(L(v),L(\vec{0})) = \norm{L(v) - L(\vec{0})} = \norm{L(v) - \vec{0}} = \norm{L(v)}. \]
Assume $L$ preserves the norm, then
\[ d(L(v_1), d(v_2)) = \norm{L(v_1)-L(v_2)} = \norm{L(v_1-v_2)} = \norm{v_1-v_2} = d(v_1,v_2). \]
\end{proof}

\begin{proposition}
Let $V$ be a normed vector space, then the norm $\norm{\cdot}:V\to \R$ is a continuous map.
\end{proposition}
\begin{proof}
The reverse triangle inequality, $|\norm{v}-\norm{w}| \leq norm{v-w}$, implies that the norm is Lipschitz continuous with Lipschitz constant $1$, so we can use \ref{lemma:LipschitzcontinuousContinuous}.
\end{proof}

\begin{definition}
Let $V$ be a vector space. Two norms $\norm{\cdot}_1$ and $\norm{\cdot}_2$ on $V$ are \udef{equivalent} if there exist $a,b\in \R$ such that
\begin{align*}
\forall v\in V: a\norm{v}_1&\leq \norm{v}_2 \\
\forall v\in V: b\norm{v}_2&\leq \norm{v}_1
\end{align*}
\end{definition}

\begin{proposition}
Equivalent norms induce the same topology.
\end{proposition}
\begin{proof}
TODO
\end{proof}

Because normed product spaces are metric spaces, we have a notion of convergence and can define infinite sums:
\begin{definition}
In a normed space $V$, we can define a \udef{infinite linear combination} as an infinite sum
\[ \sum_{i\in I} c_i v_i  \]
where $\{v_i\}_{i\in I}$ is a set of vectors and $\{c_i\}_{i I}$ a set of scalars, if that sum converges in the norm topology.
\end{definition}
\begin{note}
This finite sum is defined using nets:
Ordered by inclusion, the set $J = \{I'\subset I \;|\; I' \; \text{is finite}\}$ is a directed set. This means
\[ \left(\sum_{i\in A}c_iv_i \right)_{A\in J} \]
is a net. The infinite sum is defined if this net converges.
\end{note}

\begin{lemma}
Every proper subspace $U$ of a normed vector space $V$ has empty interior.
\end{lemma}
\begin{proof}
Suppose $U$ has a non-empty interior. Then it contains some ball $B(u,\epsilon)$. Now every vector in $V$ can be translated and rescaled to fit inside the ball $B(u,\epsilon)$. Indeed let $v\in V$ and set $u' = u+ \frac{\epsilon}{2\norm{v}}v \in B(u,\epsilon)$. Then, since $U$ is a subspace $v = \frac{2\norm{v}}{\epsilon}(u'-u)\in U$. So $U=V$.
\end{proof}

\begin{lemma}[Riesz's lemma] \label{lemma:RieszsLemma}
Let $V$ be a normed vector space. Given a non-dense subspace $X$ and a number $\theta<1$, there exists a unit vector $v\in V$ such that $\inf_{x\in X}\norm{x-v}\geq \theta$.
\end{lemma}
\begin{proof}
Take a vector $v_1$ not in the closure of $X$ and put $a = \inf_{x\in X}\norm{x-v_1}$. Then $a>0$ by lemma \ref{lemma:sequencesSupInf}. For $\epsilon > 0$, let $x_1\in X$ be such that $\norm{x_1+v_1}<a+\epsilon$. Then take
\[ v = \frac{v_1 - x_1}{\norm{v_1-x_1}} \qquad \text{so} \qquad \norm{v}=1. \]
And
\[ \inf_{x\in X}\norm{x-v} = \inf_{x\in X}\norm{x-\frac{v_1 - x_1}{\norm{v_1-x_1}}} = \inf_{x\in X}\norm{\frac{x-v_1 + x_1}{\norm{v_1-x_1}}} = \frac{\inf_{x\in X}\norm{x-v_1}}{\norm{v_1-x_1}} \geq \frac{a}{a+\epsilon}. \]
By choosing $\epsilon >0$ small, $a/(a+\epsilon)$ can be made arbitrarily close to $1$.
\end{proof}
For finite-dimensional spaces we can even take $\theta=1$.

\subsection{Linear independence and bases in normed spaces}
\url{https://math.stackexchange.com/questions/1518029/are-uncountable-schauder-like-bases-studied-used}

\subsection{Finite-dimensional normed (sub)spaces}

\begin{lemma} \label{lemma:coordinateContinuity}
Let $V$ be a normed vector space and $\{x_1, \ldots, x_n\}$ a linearly independent set of vectors. There exists a $c>0$ such that $\forall \alpha_1,\ldots, \alpha_n \in \mathbb{F}$:
\[ \norm{\alpha_1x_1 + \ldots + \alpha_nx_n} \geq c(|\alpha_1|+\ldots+|\alpha_n|) . \]
\end{lemma}
\begin{proof}
TODO ref locally convex spaces?
\end{proof}
TODO This is equivalent with continuity of coordinate functions.

\begin{proposition} \label{prop:finiteDimComplete}
Every finite-dimensional subspace of a normed vector space is complete.
\end{proposition}
\begin{proof}
Take a basis $\{e_i\}_{i=1}^n$ and let $c$ be as in lemma \ref{lemma:coordinateContinuity}. Consider an arbitrary Cauchy sequence $(v_k)_{k\in\N}$. We can write
\[ v_k = \alpha_{k,1}e_1 + \ldots + \alpha_{k,n}e_n. \]
We claim that $(\alpha_{k,i})_{k\in\N}$ is Cauchy in $\mathbb{F}$ for all $1\leq i\leq n$. Indeed, take an $\epsilon>0$. By the Cauchy nature of $(v_k)_{k\in\N}$ we can find a $k_0$ such that $\forall k', k''>k_0:$
\[ c\epsilon > \norm{v_{k'} - v_{k''}} \geq \norm{\sum_{i=1}^n (\alpha_{k',i}-\alpha_{k'',i})e_i}\geq c\sum_{i=1}^n |\alpha_{k',i}-\alpha_{k'',i}| \geq c |\alpha_{k',i}-\alpha_{k'',i}|. \]
Dividing left and right by $c$ gives exactly the Cauchy condition for each $1\leq i\leq n$. By the completeness of $\R$ or $\C$, each of these sequences has a limit $\alpha_i$.
Then $v= \sum_{i=1}^n\alpha_ie_i$ is an element of the subspace. The sequence $(v_k)$ converges to $v$ because
\[ \norm{v_k-v} = \norm{\sum_{i=1}^n (\alpha_{k,i}-\alpha_i)e_i} \leq \sum_{i=1}^n |\alpha_{k,i}-\alpha_i|\norm{e_i} \]
and the right-hand side goes to zero as $k\to \infty$.
\end{proof}
\begin{corollary} \label{corollary:finiteDimClosed}
Every finite-dimensional subspace of a normed vector space is closed.
\end{corollary}
TODO ref for proof.

\begin{proposition}
On a finite-dimensional vector space all norms are equivalent.
\end{proposition}
\begin{proof}
Let $\{e_i\}_{i=1}^n$ be a basis and take an arbitrary vector $v = \sum_{i=1}^nv_ie_i$. Let $\norm{\cdot}_1$ and $\norm{\cdot}_2$ be two norms.
We calculate
\[ \norm{v}_1 \leq \sum_{i=1}^n|v_i|\norm{e_i}_1 \leq k\sum_{i=1}^n|v_i| \leq \frac{k}{c_2}\norm{v}_2 \]
where the first inequality is the triangle inequality, the second comes from $k=\max\norm{e_i}_1$ and the third is lemma \ref{lemma:coordinateContinuity}. A similar calculation gives the other necessary inequality.
\end{proof}

\begin{proposition}
In a finite-dimensional normed space $V$, any subset $M \subseteq V$ is compact if and only if $M$ is closed and bounded.
\end{proposition}
\begin{proof}
TODO
\end{proof}

\begin{proposition}
The closed unit ball of a vector space is compact \textup{if and only if} the vector space is finite-dimensional.
\end{proposition}
\begin{proof}
One direction is given by the previous proposition. For the other direction, we show the contrapositive: let the vector space be infinite-dimensional.
We define a sequence of unit vectors $(e_i)_{i\in\N}$ recursively as follows:
\begin{itemize}
\item $e_1$ is just a unit vector;
\item for $e_{n+1}$ apply Riesz's lemma \ref{lemma:RieszsLemma} to the subspace $\Span\{e_i\}_{i=1}^n$ and $\theta = 1/2$. This subspace cannot be dense, because it is a closed (by corollary \ref{corollary:finiteDimClosed}) finite-dimensional subspace of an infinite-dimensional vector space.
\end{itemize}
This yields a sequence such that for all $m,n$
\[ \norm{e_m - e_n}\geq \frac{1}{2}. \]
This sequence is not Cauchy and thus not convergent.
\end{proof}







\subsection{Norms on constructed vector spaces}
\subsubsection{Direct sum}
\[ \norm{x\oplus y}_{X\oplus Y} = \norm{x}_X + \norm{y}_Y \]
TODO + arbitrary direct sums.
\subsubsection{The graph norm}
Let $L:V\to W$ be a linear map between normed spaces. The graph of $L$
\[ \setbuilder{(v,w)\in V\oplus W}{w = Lv} \]
has a natural norm inherited from the direct sum:
\[ \norm{(v,Lv)} = \norm{v}_V + \norm{Lv}_W. \]
This norm can also be seen as a norm on $V$: the \udef{graph norm} induced by $L$ is defined as
\[ \norm{v}_L := \norm{v}_V + \norm{Lv}_W. \]


\section{Operators on normed spaces}


\subsection{Bounded operators}
\begin{definition}
An operator $L$ between normed vector spaces is called \udef{bounded} if it is Lipschitz continuous.

In other words, there exists an $M>0$ such that $\forall v\in \dom(L)$
\[ \norm{L(v)} \leq M \norm{v}. \]

The set of bounded operators from $V$ to $W$ is denoted $\Bounded(V,W)$. If $V=W$, we write $\Bounded(V)$.
\end{definition}

\begin{theorem} \label{theorem:boundedLinearMaps}
Let $L$ be a linear operator between normed spaces $V,W$. The following are equivalent:
\begin{enumerate}
\item $L$ is continuous everywhere in $\dom(L)$;
\item $L$ is continuous at $x_0 \in \dom(L)$;
\item $L$ is continuous at $0$;
\item $L$ is bounded.
\end{enumerate}
\end{theorem}
\begin{proof}
We proceed round-robin-style:
\begin{itemize}[leftmargin=2cm]
\item[$\boxed{(1) \Rightarrow (2)}$] Trivial.
\item[$\boxed{(2) \Rightarrow (3)}$] Let $\seq{x_n}$ converge to $0$, then
\[ \lim_{n\to\infty}L(x_n) = \lim_{n\to\infty}L(x_n+x_0) - L(x_0) = L(\lim_{n\to\infty}x_n+x_0) - L(x_0) = L(x_0) - L(x_0) = 0. \]
Continuity follows because normed vector spaces are sequential spaces.
\item[$\boxed{(3) \Rightarrow (4)}$] From continuity at zero, there exists a $\delta>0$ such that $\norm{L(h)} = \norm{L(h)-L(0)} \leq 1$ for all $h\in \dom(L)$ with $\norm{h}\leq \delta$. Thus for all nonzero $v\in \dom(L)$
\[ \norm{L(v)} = \norm{\frac{\norm{v}}{\delta}L(\delta \frac{v}{\norm{v}})} = \frac{\norm{v}}{\delta}\norm{L(\delta \frac{v}{\norm{v}})}\leq \frac{\norm{v}}{\delta}. \]
\item[$\boxed{(4) \Rightarrow (1)}$] Lipschitz continuity implies continuity \ref{lemma:LipschitzcontinuousContinuous}.
\end{itemize}
\end{proof}
\begin{corollary} \label{corollary:boundedAntiLinearMaps}
An anti-linear map between complex vector spaces is also continuous \textup{if and only if} it is bounded.
\end{corollary}
\begin{proof}
An anti-linear map $A:V\to W$ is an $\R$-linear map $A:V_\R\to W_\R$. Now $V_\R, W_\R$ have the same norms as $V,W$ and thus the same topology. So $A:V\to W$ is continuous if and only if $A:V_\R\to W_\R$ is continuous.
\end{proof}
\begin{corollary}
All norm-decreasing homomorphisms are continuous.
\end{corollary}

\begin{proposition}
Let $V,W$ be normed spaces. Then $T:V\to W$ is bounded \textup{if and only if}
$T^{-1}[B(\vec{0},1)]$ has nonempty interior.
\end{proposition}
\begin{proof}
TODO!
\end{proof}

\begin{lemma} \label{lemma:kerClosed}
Let $T$ be a bounded linear operator. Then $\ker(T)$ is closed.
\end{lemma}
\begin{proof}
Suppose $T$ bounded and thus continuous. Then $\ker L = L^{-1}[\{0\}]$ and thus closed, by proposition \ref{prop:continuity}.
\end{proof}
\begin{proof}
Let $v\in \overline{\ker(T)}$. Then find a sequence $(v_n)$ in $\ker(T)$ that converges to $v$. Then by continuity $(Tv_n)$ converges to $Tv$, but for all $n\in\N: Tx_n = 0$, so the limit is $Tv=0$. Thus $v\in\ker(T)$, making it closed.
\end{proof}

\begin{proposition}\label{prop:continuousMapCriterion}
Let $L:V\to W$ be a linear map between normed spaces.
\begin{enumerate}
\item If $V$ is finite-dimensional, then $L$ is continuous.
\item If $W$ is finite-dimensional, then $L$ is continuous \textup{if and only if} $\ker L$ is closed.
\end{enumerate}
\end{proposition}
\begin{proof}
\begin{enumerate}
\item This follows from a consideration of the graph norm $\norm{v}_L = \norm{v}+\norm{Lv}$ and the fact that on a finite-dimensional space any two norms are equivalent: for all $v$ we can choose an $M$ such that
\[ \norm{Lv}\leq \norm{v}_L \leq M\norm{v}. \]
\item Assume $W$ finite-dimensional. Consider the map $\bar{L}:V/\ker L\to W: v+\ker{L}\mapsto L(v)$, defined in proposition \ref{prop:splittingMap}. Then $V/\ker L$ is isomorphic to a subspace of $W$ and thus is finite-dimensional. By the first point, $\bar{L}$ must be continuous. Let $\pi: V\to V/\ker L$ denote the quotient map, which is continuous (TODO is this where closure of $\ker L$ is used?). Then $L = \bar{L}\circ \pi$ is a composition of continuous maps and thus continuous.

Conversely, we have the lemma \ref{lemma:kerClosed}.
\end{enumerate}
\end{proof}

\subsubsection{The normed space of bounded operators}
\begin{lemma} \label{lemma:operatorNorm}
Let $(V,\norm{\cdot}_V)$ and $(W,\norm{\cdot}_W)$ be normed spaces and $L\in\Lin(V, W)$. Then $L$ is bounded \textup{if and only if}
\[ \sup\setbuilder{\frac{\norm{Lx}_W}{\norm{x}_V}}{x\in V\setminus\{0\}} \] 
exists.
\end{lemma}
\begin{definition}
Let $(V,\norm{\cdot}_V)$ and $(W,\norm{\cdot}_W)$ be normed spaces and $L\in\Lin(V, W)$ bounded. Then
\[ \norm{L} \defeq \sup\setbuilder{\frac{\norm{Lx}_W}{\norm{x}_V}}{x\in V\setminus\{0\}} \]
is called the \udef{operator norm} of $L$.
\end{definition}

\begin{proposition} \label{prop:BoundedSpace}
Let $(V,\norm{\cdot}_V)$ and $(W,\norm{\cdot}_W)$ be normed spaces. Then the set $\Bounded(V,W)$ of bounded linear maps is a normed subspace of $\Lin(V,W)$ equipped with the operator norm.
\end{proposition}

\begin{proposition} \label{prop:operatorNorm}
Let $L\in\Bounded(V,W)$ be a bounded operator and let $B(\vec{0},\epsilon)$ be an open ball centered at $\vec{0}$. Then
\begin{align*}
\norm{L} &= \frac{\sup L[B(\vec{0},\epsilon)]}{\epsilon} \\
&= \frac{\sup L[\overline{B}(\vec{0},\epsilon)]}{\epsilon} \\
&= \sup\setbuilder{\norm{Lx}}{\norm{x} = 1}.
\end{align*}
\end{proposition}
\begin{proof}
TODO
\end{proof}

\begin{lemma}
Let $S,T$ be compatible bounded operators. Then
\[ \norm{ST} \leq \norm{S}\norm{T}. \]
\end{lemma}
\begin{proof}
$\norm{ST} = \sup\setbuilder{\frac{\norm{STx}}{\norm{x}}}{\norm{x}=1} \leq \sup\setbuilder{\frac{\norm{S}\norm{Tx}}{\norm{x}}}{\norm{x}=1}\leq \norm{S}\;\norm{T}$.
\end{proof}

\subsubsection{Operators bounded below}
\begin{definition}
Let $T$ be a bounded linear operator. We say $T$ is \udef{bounded below} if
\[ \exists b>0:\forall v\in \dom(T): \quad \norm{Tv}\geq b\norm{v} \]
\end{definition}


\begin{proposition} \label{prop:boundedBelow}
Let $T:V\to W$ be a bounded operator that is bounded below. Then
\begin{enumerate}
\item $T$ is injective;
\item if $T$ is surjective, the inverse $T^{-1}: W\to V$ exists and is bounded.
\end{enumerate}
\end{proposition}
\begin{proof}
To show $T$ is injective, take $x_1,x_2\in \dom T$ such that $Tx_1 = Tx_2$. Then
\[ 0 = \norm{Tx_1 - Tx_2} = \norm{T(x_1 - x_2)} \geq b\norm{x_1 - x_2} \geq 0. \]
So $\norm{x_1 - x_2} = 0$ and thus $x_1=x_2$.

The existence of $T^{-1}$ is then clear. For boundedness notice that $T^{-1}y \in \dom(T)$, so because $T$ is bounded below,
\[ \norm{T^{-1}y} \leq \frac{1}{b}\norm{TT^{-1}y} = \frac{1}{b}\norm{y}. \]
\end{proof}

\begin{lemma} \label{lemma:boundedBelowBounded}
Let $T:\dom(V)\to W$ be an injective operator. Then $T$ is bounded \textup{if and only if} $T^{-1}:\im(T)\to \dom(T)$ is bounded below.
\end{lemma}
\begin{proof}
Assume $T$ bounded. Then for all $x\in\im T$: $\norm{x} = \norm{TT^{-1}x} \leq \norm{T}\norm{T^{-1}x}$, so $T^{-1}$ is bounded below by $1/\norm{T}$.

Assume $T^{-1}$ bounded below. Then for all $x\in\dom(T)$: $\norm{x} = \norm{T^{-1}Tx} \geq b\norm{Tx}$, so $T$ is bounded by $1/b$.
\end{proof}

\subsection{Closed operators}
\begin{definition}
A \udef{closed operator} is an operator with closed graph.
\end{definition}
This is not the same as a closed map in the topological sense!

\begin{proposition} \label{prop:closedGraphEquivalence}
Let $X,Y$ be normed spaces and $T: \dom(T)\subset X \to Y$ be a linear operator. Then
the following are equivalent:
\begin{enumerate}
\item the graph of $T$ is closed in $X\oplus Y$;
\item if $(x_n)_{n\in\N}\subset \dom(T)$ converges to $x\in X$ and $(Tx_n)_{n\in\N}$ converges to $y$, then $x\in\dom(T)$ and $Tx = y$.
\end{enumerate}
\end{proposition}
TODO: remove domain from proposition?
\begin{corollary}
All bounded operators have closed graph. (? If domain is closed?)
\end{corollary}
The converse is not true in general.

\url{https://en.wikipedia.org/wiki/Unbounded_operator#Closed_linear_operators}
\url{https://en.wikipedia.org/wiki/Closed_graph_theorem_(functional_analysis)}

\begin{proposition} \label{prop:algebraClosedOperators}
Let $T$ be a closed and $S$ a bounded operator, then
\begin{enumerate}
\item $S+T$ is closed;
\item $TS$ is closed;
\item if $T$ is injective, then $T^{-1}: \im(T) \to \dom(T)$ is closed.
\end{enumerate}
\end{proposition}
\begin{proof}
(1) TODO

(2) TODO

(3) We use \ref{prop:closedGraphEquivalence}. Take $\seq{y_n}\subset \dom(T^{-1})$ such that $y_n\to y$ and $T^{-1}y_n\to x$. Set $x_n = T^{-1}y_n$, so then $Tx_n = y_n\to y$. Because $T$ is closed it follows that $Tx = y$, so $T^{-1}y = x$, meaning $T^{-1}$ is closed.
\end{proof}
TODO example $ST$ need not be closed.

\subsubsection{Closable operators}
\begin{definition}
A linear operator is called \udef{closable} if it has closed extension.
\end{definition}

\begin{proposition}
A linear operator $T$ is closable \textup{if and only if} for all sequences $\seq{x_n}\subset\dom(T)$
\[ \left(x_n\to 0 \land T(x_n)\to v\right) \quad\implies\quad v = 0. \]
\end{proposition}
\begin{proof}
TODO
\end{proof}

\begin{lemma}
A closable operator $T$ has a minimal closed extension $\overline{T}$, which is given by the closure of the graph of $T$.
\end{lemma}
\begin{proof}
TODO
\end{proof}

\subsection{Compact operators}
\begin{definition}
A linear map $L:V\to W$ between normed spaces is called \udef{compact} if $L[\overline{B}(\vec{0}, 1)]$ is relatively compact.

I.e. the image of the closed unit ball has compact closure.

The space of compact maps from $V$ to $W$ is denoted $\mathcal{K}(V,W)$.
\end{definition}

These operators were introduced to study equations of the form
\[ (T-\lambda I)x(t) = p(t). \]

\begin{proposition}
Let $L\in\Hom(V,W)$. The following are equivalent:
\begin{enumerate}
\item $L$ is compact;
\item the image of any bounded subset of $V$ is relatively compact in $W$;
\item there exists a neighbourhood $U$ of $0$ in $V$ such that the image of $U$ is a subset of a compact set in $W$;
\item for any bounded sequence $(x_n)_{n\in\N} \subseteq V$, then sequence $(Lx_n)_{n\in\N}$ contains a converging subsequence.
\end{enumerate}
\end{proposition}
\begin{proof}
TODO
\end{proof}
\begin{corollary}
All maps of finite rank are compact.
\end{corollary}
\begin{proof}
Closed balls in $\C^n$ are compact.
\end{proof}

\begin{proposition}
Let $V$ be a normed space. Then $\mathcal{K}(V)$ is a closed two-sided ideal in $\Bounded(V)$.
\end{proposition}

\section{Inner product spaces}
\begin{definition}
An \udef{inner product} on a vector space $V$ is a function
\[ \inner{\cdot,\cdot}: V\times V \to \mathbb{F}  \]
that has the following properties:
\begin{itemize}[leftmargin=4.5cm]
\item[\textbf{Linearity}] in the \emph{second}\footnote{Some authors take linearity in the first component.} component
\[\inner{v,\lambda_1 w_1 + \lambda_2 w_2} = \lambda_1\inner{v,w_1} + \lambda_2\inner{v,w_2},\]
where $\lambda_1,\lambda_2 \in \mathbb{F}$ and $v,w_1,w_2\in V$.
\item[\textbf{Conjugate symmetry}\footnote{This is for $\mathbb{F} = \C$. For $\mathbb{F} = \R$ this reduces to normal symmetry $\inner{v,w} = \inner{w,v}$.}] $\inner{v,w} = \overline{\inner{w,v}}$ for all $v,w\in V$.
\item[\textbf{Positivity}\footnote{By conjugate symmetry we know that $\inner{v,v}$ is a real number, so this condition makes sense.}] $\inner{v,v} \geq 0$ for all $v\in V$.
\item[\textbf{Definiteness}]$\inner{v,v} = 0$ if and only if $v= 0$.
\end{itemize}
An \udef{inner product space} or \udef{pre-Hilbert space} $(\mathbb{F}, V,+,\inner{\cdot,\cdot})$ is a vector space $(\mathbb{F}, V,+)$ together with an inner product $\inner{\cdot,\cdot}$ on $V$.

A real finite-dimensional inner product space is called a \udef{Euclidean space}.
\end{definition}
\begin{lemma}
An inner product over a complex vector space $V$ is anti-linear in the first component.
\end{lemma}

\begin{lemma} \label{lemma:nonDegeneracyInnerProduct}
Definiteness implies the inner product on $V$ is non-degenerate:
\[ [\forall u\in V:\inner{u,v} = 0] \implies v = 0. \]
\end{lemma}
The converse is not true.

There are some generalised notions of inner product:
\begin{definition}
Let $V$ be a complex vector space.
\begin{enumerate}
\item A \udef{sesquilinear form} is a function $V\times V\to \C$ that is linear in the second component and anti-linear in the first.
\item A \udef{Hermitian form} is a conjugate symmetric sesquilinear form.
\item A \udef{pre-inner product} is a positive Hermitian form, i.e. an inner product without the requirement of definiteness.
\end{enumerate}
\end{definition}

\begin{example}
\begin{enumerate}
\item The \udef{standard inner product} on $\R^n$ is given by
\[ \inner{a,b} = \inner{\begin{bmatrix}
a_1 \\ \vdots \\ a_n
\end{bmatrix},\begin{bmatrix}
b_1 \\ \vdots \\ b_n
\end{bmatrix}} = \begin{bmatrix}
a_1 & \hdots & a_n
\end{bmatrix}\begin{bmatrix}
b_1 \\ \vdots \\ b_n
\end{bmatrix} = a^\transp b \]
This is also known as the \udef{dot product} $a\cdot b$.
\item The \udef{standard inner product} on $\C^n$ is given by
\[ \inner{a,b} = \inner{\begin{bmatrix}
a_1 \\ \vdots \\ a_n
\end{bmatrix},\begin{bmatrix}
b_1 \\ \vdots \\ b_n
\end{bmatrix}} = \begin{bmatrix}
\bar{a}_1 & \hdots & \bar{a}_n
\end{bmatrix}\begin{bmatrix}
b_1 \\ \vdots \\ b_n
\end{bmatrix} = \bar{a}^\transp b \]
\item The \udef{Frobenius inner product} on $\C^{m\times n}$ is given by
\[ \inner{A,B}_F =  \Tr(\overline{A}^\transp B) = \overline{\vectorisation_C(A)}^\transp \vectorisation_C(B)\]
\item On the vector space $\mathcal{C}[a,b]$ of continuous real functions on $[a,b]$, we can take the inner product
\[ \inner{f,g} = \int_a^b f(x)\cdot g(x) \diff{x}. \]
\end{enumerate}
\end{example}

\begin{definition}
Two vectors $u,v \in V$ are \udef{orthogonal} if $\inner{u,v} =0$. This is denoted $u\perp v$.
\end{definition}
\begin{lemma} \label{lemma:elementaryOrthogonality}
Let $V$ be an inner product space.
\begin{enumerate}
\item $0$ is the only vector orthogonal to itself.
\item $0$ is orthogonal to all $v\in V$;
\item Let $x,y\in V$. If, for all $v\in V$, $\inner{v,x} = \inner{v,y}$, then $x=y$.
\end{enumerate}\end{lemma}
\begin{proof}
The first is a consequence of definiteness, the second a consequence of linearity: $\inner{v,0} = \inner{v,0\cdot0} = 0\inner{v,0} = 0$.

The third is also a consequence of linearity: assume $\forall v\in V: \inner{v,x} = \inner{v,y}$, then $\inner{v,x-y}=0$ and $x-y$ is orthogonal to all $v\in V$ and in particular to $0$. Thus $x-y$ must be zero.
\end{proof}

\begin{proposition}
Every inner product gives rise to a norm, defined by
\[ \norm{\cdot} = \sqrt{\inner{\cdot,\cdot}}. \]
\end{proposition}
\begin{proof}
The only non-trivial part is the triangle inequality. This will be proved later using the Cauchy-Schwarz inequality.
\end{proof}


\begin{lemma}
Let $V$ be an inner product space. Then
\[ \norm{v+w}^2 = \norm{v}^2+\norm{w}^2+2\Re\inner{v,w} \]
\end{lemma}
\begin{lemma} \label{lemma:orthogonalDecomposition}
Let $v,w\in V$, with $w\neq 0$. We can decompose $v$ as a multiple of $w$ and a vector $u$ orthogonal to $w$:
\[ v = cw+u = \left(\frac{\inner{v,w}}{\norm{w}^2}\right)w + \left( v- \frac{\inner{v,w}w}{\norm{w}^2} \right). \]
\end{lemma}
\begin{proof}
The only thing to check is $\inner{w, v- \frac{\inner{v,w}w}{\norm{w}^2}} = 0$, which is a simple calculation.
\end{proof}

\subsection{Pythagoras and Cauchy-Schwarz}
\begin{theorem}[Pythagorean theorem]
Suppose $u\perp v$. Then $\norm{u+v}^2 = \norm{u}^2 + \norm{v}^2$.
\end{theorem}
\begin{proof}
\[ \norm{u+v}^2 = \inner{u+v,u+v} = \inner{u,u}+ \inner{u,v} + \inner{v,u} + \inner{v,v} = \norm{u}^2 + \norm{v}^2. \]
\end{proof}

\begin{theorem}[Cauchy-Schwarz-Bunyakovsky inequality.] \label{theorem:CauchySchwarz}
Let $V$ be a vector space with a pre-inner product $\inner{\cdot,\cdot}$. Let $v,w\in V$. Then
\[ |\inner{v,w}|^2\leq \inner{v,v}\cdot\inner{w,w}. \]
Suppose $\inner{\cdot,\cdot}$ is definite (i.e. an inner product), then
this is an equality \textup{if and only if} $v$ and $w$ are scalar multiples.
\end{theorem}
This result is also known as the Cauchy-Schwarz inequality, or the CSB inequality.
\begin{proof}
Consider 
\[ \inner{v-\lambda w, v-\lambda w} = \inner{v,v}-\lambda\inner{v,w}-\overline{\lambda}\inner{w,v} + |\lambda|^2 \inner{w,w} \geq 0. \]
Suppose $\inner{v,w}=re^{i\theta}$ (if $\mathbb{F} = \R$, then $\theta=0$ or $\theta = \pi$). The inequality must still hold for all $\lambda$ of the form $te^{-i\theta}$ for some $t\in \R$. The inequality thus becomes
\[ 0\leq \inner{v,v}-te^{-i\theta}re^{i\theta}-te^{i\theta}re^{-i\theta} + t^2 \inner{w,w} = \inner{v,v}-2rt + t^2 \inner{w,w}. \]
On the right we have a quadratic formula in $t$. This may never be negative and the discriminant may therefore not be positive. Calculating the discriminant gives $(2r)^2 - 4\inner{v,v}\inner{w,w}$. Thus
\[ 0\geq r^2 - \inner{v,v}\inner{w,w} = |\inner{v,w}|^2 - \inner{v,v}\inner{w,w}. \]
\end{proof}
In the case of an inner product, there is a simpler proof:
\begin{proof}
Take the decomposition from lemma \ref{lemma:orthogonalDecomposition} and apply the Pythagorean theorem to obtain
\[ \norm{v}^2 = \frac{|\inner{v,w}|^2}{\norm{w}^2} + \norm{u}^2 \geq \frac{|\inner{v,w}|^2}{\norm{w}^2}. \]
This also shows the claim about scalar multiples.
\end{proof}
\begin{corollary} \label{lemma:innerBoundedFunctionals}
Let $V$ be an inner product space. The functions
\[\inner{v,\cdot}: V\to \mathbb{F}: x\mapsto \inner{v,x} \]
are bounded linear functionals for all $v\in V$.
\end{corollary}
\begin{corollary} \label{corollary:preInnerProductCSBZero}
Let $V$ be a vector space with a pre-inner product $\inner{\cdot,\cdot}$. Then
\[ \inner{x,x}=0\lor\inner{y,y}=0 \quad\implies\quad \inner{x,y} = 0. \]
\end{corollary}
\begin{definition}
The Cauchy-Schwarz inequality allows us to define the \udef{angle} $\theta$ between two vectors $v,w$ by
\[ \cos\theta = \frac{\inner{v,w}}{\norm{v}\norm{w}}.\]
\end{definition}
\begin{lemma}
If $v\perp w$, then the angle between them is $\pi/2 + k\pi$.
\end{lemma}

\begin{theorem}[Triangle inequality]
Let $v,w\in V$. Then
\[ \norm{v+w} \leq \norm{v}+\norm{w} \qquad\text{or, equivalently}\qquad \norm{v}-\norm{w}\leq \norm{v-w}. \]
This inequality is an equality if and only if one of $u,v$ is a nonnegative multiple of the other.
\end{theorem}
\begin{proof}
We calculate
\begin{align*}
\norm{v+w}^2 &= \norm{v}^2+\norm{w}^2+2\Re\inner{v,w} \\
&\leq \norm{v}^2+\norm{w}^2+2|\inner{v,w}| \\
&\leq \norm{v}^2+\norm{w}^2+2\norm{v}\norm{w} \\
&= (\norm{v}+\norm{w})^2.
\end{align*}
The substitution $v\to v-w$ gives the second form.
\end{proof}

\subsection{Parallelogram law and polarisation}
\begin{theorem}[Parallelogram law] \label{theorem:parallelogramLaw}
Let $V$ be an inner product space and $v,w\in V$. Then
\[ \norm{v+w}^2 + \norm{v-w}^2 = 2 (\norm{v}^2+\norm{w}^2). \]
\end{theorem}
\begin{proof}
We calculate
\begin{align*}
\norm{v+w}^2 + \norm{v-w}^2 = \inner{v+w, v+w}+\inner{v-w,v-w} = 2(\norm{v}^2 + \norm{w}^2).
\end{align*}
\end{proof}
\begin{corollary}[Appolonius' identity]
Let $V$ be an inner product space and $x,y,z\in V$. Then
\[ \norm{z-x}^2 + \norm{z-y}^2 = \frac{1}{2}\norm{x-y}^2 + 2\norm*{z-\frac{1}{2}(x+y)}^2. \]
\end{corollary}
\begin{proof}
Apply the parallelogram law to $u = \frac{1}{2}(z-x)$ and $v = \frac{1}{2}(z-y)$.
\end{proof}

\begin{theorem}[Polarisation identities] \label{theorem:polarisationIdentities}
Polarisation identities allow us to recover the inner product from the norm.
\begin{enumerate}
\item For real inner product spaces, $\mathbb{F} = \R$:
\begin{align*}
\inner{v,w} &= \frac{1}{2}(\norm{v+w}^2 - \norm{v}^2-\norm{w}^2) \\
&= \frac{1}{2}(\norm{v}^2 + \norm{w}^2-\norm{v-w}^2) \\
&= \frac{1}{4}(\norm{v+w}^2 - \norm{v-w}^2) = \frac{1}{4}\sum_{k=0}^1 (-1)^k\norm{v+(-1)^k w}^2.
\end{align*}
\item For complex inner product spaces, $\mathbb{F} = \C$:
\[ \inner{x,y} = \frac{1}{4}\sum_{k=0}^3 i^k\norm{i^k x+y}^2. \]
\item For general sesquilinear forms:
\[ S(x,y) = \frac{1}{4}\sum_{k=0}^3 i^k S(i^k x+y, i^k x+y). \]
\end{enumerate}
\end{theorem}
\begin{corollary}
A sesquilinear form is Hermitian \textup{if and only if} $\inner{v,v}$ is real for all $v\in V$.
\end{corollary}
\begin{proof}
The direction $\Rightarrow$ is obvious. For the other direction, assume $\inner{v,v}$ is real for all $v\in V$ and in particular $\inner{u+i^kv,u+i^kv}$ is real. We calculate
\begin{align*}
\overline{\inner{u,v}} &= \frac{1}{4}\sum^3_{k=0}(-i)^k\inner{u+i^kv,u+i^kv} \\
&= \frac{1}{4}\sum^3_{k=0}(-i)^k\inner{v+(-i)^ku,v+(-i)^ku} & &\text{Using (conjugate) linearity and $i^k(-i)^k=1$}\\
&= \frac{1}{4}\sum^3_{k=0}i^k\inner{v+i^ku,v+i^ku} & &\text{Substituting $k\to k+2$}\\
&= \inner{v,u}.
\end{align*}
\end{proof}
Not all norms on vector spaces can be obtained from an inner product. If a norm can be obtained from an inner product, we can use polarisation to recover the inner product. If a norm cannot be obtained from an inner product, the putative inner product suggested by polarisation will turn out not to be an inner product.
\begin{proposition}
A norm can be obtained from an inner product \textup{if and only if} it satisfies the parallelogram law.
\end{proposition}
\begin{corollary}
The space $l^p$ is an inner product space \textup{if and only if} $p=2$.
\end{corollary}
\begin{proof}
The inner product on $l^2$ is defined by $\inner{x_n, y_n} = \sum_{n=1}^\infty \overline{x_n}y_n$.

If $p\neq 2$ we can find a counterexample to the parallelogram law: let $x=(1,1,0,0,\ldots)\in l^p$ and $y = (1,-1,0,0,\ldots)\in l^p$. Then
\[ \norm{x}_p = \norm{y}_p = 2^{1/p} \qquad \text{and} \qquad \norm{x+y} = \norm{x-y} = 2 \]
and the parallelogram law is then not valid if $p\neq 2$.
\end{proof}

\section{Orthogonal and orthonormal sets of vectors}
\begin{definition}
\begin{itemize}
\item A set of vectors $D$ is called \udef{orthogonal} if for any two vectors $v,w\in D$, $v\perp w$ \textup{if and only if} $v\neq w$.
\item A set of vectors $D$ is called \udef{orthonormal} if for any two vectors $v,w\in D$,
\[ \inner{v,w} = \begin{cases}
0 & (v\neq w) \\ 1 & (v=w)
\end{cases}. \]
\end{itemize}
In particular an orthonormal set is an orthogonal set of unit vectors.
\end{definition}

\subsection{Orthogonal sets and sequences}
\begin{lemma} \label{lemma:orthogonalLinearlyIndependent}
Every orthogonal set of vectors is linearly independent.
\end{lemma}
\begin{lemma}
Every subset of an orthogonal (resp. orthonormal) set is orthogonal (resp. orthonormal).
\end{lemma}

\begin{theorem}[Gram-Schmidt procedure]
Every finite set of linearly independent vectors $D = \{v_1,\ldots, v_n\}$ can be transformed into an orthonormal set $D' = \{e_1,\ldots,e_n\}$ with the same number of vectors such that the spans are the same: $\Span(D') = \Span(D)$.
\end{theorem}
\begin{proof}
The procedure goes as follows:
\begin{align*}
e_1 &= \frac{v_1}{\norm{v_1}} \\
e_2 &= \frac{v_2 - \inner{e_1,v_2}e_1}{\norm{v_2 - \inner{e_1,v_2}e_1}} \\
&\hdots \\
e_j &= \frac{v_j - \inner{e_1,v_j}e_1- \ldots - \inner{e_{j-1},v_j}e_{j-1}}{\norm{v_2 - \inner{e_1,v_2}e_1- \ldots - \inner{e_{j-1},v_j}e_{j-1}}} \\
&\hdots
\end{align*}
\end{proof}

If we only need an orthogonal set $\{y_1,\ldots,y_n\}$, not an orthonormal one, we can use the procedure
\[ y_{k+1} = v_{k+1} - \sum_{i=1}^k \frac{\inner{v_{k+1}, y_i}}{\inner{y_i,y_i}}y_i. \]

\begin{lemma} \label{lemma:orthogonality}
Let $(\mathbb{F}, V,+,\inner{\cdot,\cdot})$ be an inner product space. Then
\[ \inner{v,w}=0 \qquad \iff \qquad \forall a\in\mathbb{F}:\;\norm{v}\leq\norm{v+aw}.  \]
\end{lemma}
\begin{proof}
The implication $\Rightarrow$ is a consequence of the Pythagorean theorem. For the other implication, assume $\forall a\in\mathbb{F}:\;\norm{v}\leq\norm{v+aw}$. Then
\[ \norm{v}^2 \leq \norm{v-aw}^2 = \norm{v}^2 - 2\Re\inner{v,aw} + \norm{aw}^2 \]
which implies $2\Re\inner{v,aw} \leq a^2\norm{w}^2$. Let $\inner{v,w} = re^{i\theta}$. (If $\mathbb{F} = \R$, then $\theta=0$.) Then in particular the inequality holds for all $a=te^{i\theta}$ with $t\in\R$. This yields
\[ 2\Re(te^{-i\theta}re^{i\theta}) \leq t^2\norm{w}^2 \qquad \text{or}\qquad 2rt\leq t^2\norm{w}^2. \]
Letting $t\geq 0$, we can divide out a $t$: $2r\leq t\norm{w}^2$. Then letting $t\to 0$ gives $r=0$ and thus $\inner{v,w}=0$.
\end{proof}

\begin{proposition}
Let $V$ be an inner product space and $D = \{e_1,\ldots, e_n\}$ a finite orthonormal set of vectors. Then $\forall v\in V$
\[ \inf_{c_i\in\mathbb{F}}\norm{v-\sum_{i=1}^nc_ie_i} = \norm{v-\sum_{i=1}^n\inner{e_i,v}e_i} \]
\end{proposition}
\begin{proof}
We calculate
\begin{align*}
\norm{v-\sum_{i=1}^nc_ie_i}^2 &= \inner{v-\sum_{i=1}^nc_ie_i,v-\sum_{j=1}^nc_je_j} \\
&= \norm{v} - \sum_{j=1}^n c_j\inner{v,e_j} - \sum_{i=1}^n\bar{c}_i\inner{e_i,v} + \sum_{i,j=1}^n\bar{c}_ic_j\inner{e_i,e_j} \\
&= \norm{v} - 2\Re\left(\sum_{i=1}^nc_i\overline{\inner{e_i,v}}\right) + \sum_{i=1}^n|c_i|^2 \\
&= \sum_{i=1}^n\left(|c_i|^2 - 2\Re\left(\sum_{i=1}^nc_i\overline{\inner{e_i,v}}\right) + |\inner{e_i,v}|^2\right) +\norm{v} - \sum_{i=1}^n|\inner{e_i;v}|^2 \\
&= \sum_{i=1}^n|c_i - \inner{e_i,v}|^2 +\norm{v} - \sum_{i=1}^n|\inner{e_i,v}|^2.
\end{align*}
This is clearly minimised when $c_i = \inner{e_i,v}$.
\end{proof}
\begin{corollary}
Let $v\in\Span(D)$, then $v = \sum_{i=1}^n \inner{e_i,v}e_i$.
\end{corollary}
We call the numbers $\inner{e_i,v}$ the \udef{Fourier coefficients} of $v$ w.r.t. $D$.
\begin{proof}
In this case $\inf_{c_i\in\mathbb{F}}\norm{v-\sum_{i=1}^nc_ie_i} = 0$.
\end{proof}
\begin{corollary}[Bessel inequality]
Let $\{e_i\}_{i\in I}$ be an orthonormal family and $v\in V$, then
\[ \sum_{i\in I}|\inner{e_i,v}|^2 = \sup \left\{\sum_{\substack{i\in I' \subset I\\ I' \;\text{finite}}} |\inner{e_i,v}|^2 \right\} \leq \norm{v}^2. \]
\end{corollary}
\begin{proof}
In the previous proof,
\[ 0 \leq \norm{v-\sum_{i=1}^nc_ie_i}^2 = \sum_{i=1}^n|c_i - \inner{e_i,v}|^2 +\norm{v} - \sum_{i=1}^n|\inner{e_i,v}|^2 = \norm{v} - \sum_{i=1}^n|\inner{e_i,v}|^2. \]
Where we have set $c_i = \inner{e_i,v}$. Thus the supremum must also be $\leq \norm{v}$.
\end{proof}
\begin{corollary}
For any $v\in V$, $\inner{e_i,v} = 0$ except for countably many $i\in I$. \label{corollary:countableComponents}
\end{corollary}
\begin{proof}
Ref TODO. \url{https://proofwiki.org/wiki/Uncountable_Sum_as_Series}.
\end{proof}
TODO: link with metric topology being sequential?

\begin{corollary}[Riemann-Lebesgue lemma]
For any sequence $\seq{e_i}_{i\in J \subset I}$, we have
\[ \lim_{i\in J} \inner{e_i,v} = 0. \]
\end{corollary}

\begin{corollary}
We can also obtain the Cauchy-Schwarz inequality from the Bessel inequality.
\end{corollary}
\begin{proof}
Let $x,y\in V$. Then $\{x/\norm{x}\}$ is an orthonormal set. Applying the Bessel inequality for $y$ gives $\norm{y}^2 \geq |\inner{x/\norm{x}, y}|^2 \implies |\inner{x,y}|^2\leq \norm{x}^2\norm{y}^2 \implies |\inner{x,y}| \leq \norm{x}\;\norm{y}$.
\end{proof}

\subsection{Orthonormal bases}
\begin{definition}
Let $D$ be an orthonormal set of vectors in an inner product space $V$, then $D$ is said to be
\begin{enumerate}
\item \udef{maximal}, if it is a maximal element in the set of orthonormal sets ordered by inclusion;
\item \udef{total}, if the smallest closed subspace that includes $D$ is $V$ (i.e. $\Span(D)$ is dense in $V$);
\item an \udef{orthonormal basis} (o.n. basis) or a \udef{Hilbert basis} if any vector in $V$ can be written as a (possibly infinite) linear combination of elements of $D$.
\end{enumerate}
\end{definition}
\begin{note}
Hilbert bases are in general not Hamel bases.  E.g., take $\R^\mathbb{N}$. Then 
\begin{align*}
(1,0,0,&\ldots), \\
(0,1,0,&\ldots), \\
(0,0,1,&\ldots), \\
&\ldots
\end{align*}
is an orthonormal basis, but not a Hamel basis (consider $(1,1,1,\ldots)$).
\end{note}

\begin{proposition}
If $V$ is finite-dimensional, then the notions of maximal orthonormal set, total orthonormal set and orthonormal set coincide. Such an orthonormal set is also a (Hamel) basis of $V$.
\end{proposition}
\begin{proof}
Corollaries of Gram-Schmidt.
\end{proof}

\begin{proposition}
Let $V$ be an inner product space and $D = \{e_i\}_{i\in I}$ an orthonormal set. The $D$ is an o.n. basis \textup{if and only if} $D$ is total.
\end{proposition}
\begin{proof}
$\boxed{\Rightarrow}$ Assume $D$ an o.n. basis. Then there exists a sequence of partial sums converging to any element $v\in V$. Each of these partial sums is a finite linear combination of elements in $D$ and thus this sequence is a sequence in $\Span(D)$. This means $v\in\overline{\Span(D)}$.

$\boxed{\Leftarrow}$ Assume $\overline{\Span(D)} = V$. Because the topology on $V$ is a metric topology, we can find a sequence $(v_n)$ in $\Span(D)$ that converges to any $v\in V$.
\end{proof}


\begin{proposition} \label{prop:totalONBParsevalEquivalence} \label{prop:plancherel}
Let $V$ be an inner product space and $D = \{e_i\}_{i\in I}$ an orthonormal set. The following are equivalent:
\begin{enumerate}
\item $D$ is an orthonormal basis of $V$;
\item $D$ is total in $V$;
\item \textup{(Parseval's identity)} for all $v,w\in V$,
\[ \inner{v,w} = \sum_{i\in I}\inner{v,e_i}\inner{e_i,w}; \]
\item \textup{(Bessel equality)} for all $v\in V$,
\[ \norm{v}^2 = \sum_{i\in I}|\inner{e_i,v}|^2; \]
\item for all $v\in V$: if $v\perp D$, then $v=0$;
\item \textup{(Plancherel formula)} for all $v\in V$,
\[ v = \sum_{i\in I}\inner{e_{i},v}e_{i}. \]
\end{enumerate}
\end{proposition}
\begin{proof}
We proceed round-robin-style.
\begin{itemize}[leftmargin=2cm]
\item[$\boxed{(1) \Rightarrow (2)}$] Assume $D$ an o.n. basis. Then there exists a net of partial sums converging to any element $v\in V$. Each of these partial sums is a finite linear combination of elements in $D$ and thus this net is a net in $\Span(D)$. This means $v\in\overline{\Span(D)}$.
\item[$\boxed{(2) \Rightarrow (3)}$] Fix $v,w\in V$. Because $V$ is a metric spaces and thus sequential, we can find sequences $(v_j)_{j\in J}$ and $(w_k)_{k\in K}$ in $\Span(D)$ converging to $v$ and $w$. Now the linear maps $u\mapsto \overline{\inner{u, e_i}}$ and $u\mapsto \inner{e_i, u}$ are bounded by Cauchy-Schwarz and thus continuous by theorem \ref{theorem:boundedLinearMaps} (TODO corollary CSB). Then we can calculate, using the fact that each $v_j$ and $w_k$ is a finite linear combination of $e_i$,
\begin{align*}
\inner{v,w} &= \inner{\lim_{j}v_j, \lim_k w_k} = \lim_{j}\lim_{k}\inner{v_j,w_k} \\
&= \lim_{j}\lim_{k}\inner{\sum_{i=1}^{N_{j}}\inner{e_i,v_j}e_i,\sum_{i'=1}^{N_k}\inner{e_{i'},w_k}e_{i'}} \\
&= \lim_{j}\lim_{k}\sum_{i=1}^{N_{j}}\sum_{i'=1}^{N_k}\inner{v_j,e_i}\inner{e_{i'},w_k}\inner{e_i,e_{i'}} = \lim_{j}\lim_{k}\sum_{i=1}^{N_{j}}\sum_{i'=1}^{N_k}\inner{v_j,e_i}\inner{e_{i'},w_k}\delta_{i,i'} \\
&= \lim_{j}\lim_{k}\sum_{i=1}^{\min\{N_{j},N_{k}\}}\inner{v_j,e_i}\inner{e_i,w_k} \\
&= \lim_{j}\lim_{k}\sum_{i\in I}\inner{v_j,e_i}\inner{e_i,w_k} \\
&= \sum_{i\in I}\lim_{j}\lim_{k}\inner{v_j,e_i}\inner{e_i,w_k} \\
&= \sum_{i\in I}\inner{v,e_i}\inner{e_i,w}.
\end{align*}
For the interchange of the limits and the summation in the penultimate equality we can use Tannery's theorem, \ref{theorem:tannery}. Indeed $|\inner{e_i,w_k}|$ is bounded by $\norm{w_k}$ by the Bessel inequality. By the continuity of the norm we have $\lim_k \norm{w_k} = \norm{w}$, so the sequence $\norm{w_k}$ is bounded.
\item[$\boxed{(3) \Rightarrow (4)}$] Set $v=w$.
\item[$\boxed{(4) \Rightarrow (5)}$] If $v\perp D$, then
\[ \norm{v}^2 = \sum_{i\in I}|\inner{e_i,v}|^2 = 0 \qquad\text{which implies $v=0$.} \]
\item[$\boxed{(5) \Rightarrow (6)}$] The vector $v-\sum_{i\in I}\inner{e_i,v}e_i$ is perpendicular to $D$:
\[ \forall e_j\in D: \quad \inner{e_j, v-\sum_{i\in I}\inner{e_i,v}e_i} = \inner{e_j, v}-\sum_{i\in I}\inner{e_i,v}\inner{e_j,e_i} = \inner{e_j, v} - \inner{e_j, v} = 0. \]
So $v-\sum_{i\in I}\inner{e_i,v}e_i = 0$ and the Plancherel formula holds.
\item[$\boxed{(6) \Rightarrow (1)}$] By definition of o.n. basis.
\end{itemize}
\end{proof}

\begin{lemma}
Every orthonormal basis is a maximal orthonormal family.
\end{lemma}
\begin{proof}
Let $D = \{e_i\}_{i\in I}$ be an o.n. basis. Assume, towards a contradiction, that there exists an o.n. family $D' \supsetneq D$. Let $x\in D'\setminus D$. Then, using the Plancherel formula and the fact that $D'$ is orthogonal,
\[ x = \sum_{i\in I}\inner{e_{i},x}e_{i} = \sum_{i\in I} 0 = 0. \]
As $0$ can never be an element of an o.n. family, this is a contradiction.  
\end{proof}
There are maximal orthonormal families that are not bases.
\begin{example}
Consider the space $l^2(\N)$ and take the subspace $X$ generated by the family of elements
\[ \left( \sum_{n=1}^\infty n^{-1}e_n, e_2,e_3,e_4,\ldots \right) \]
with the inner product induced by the inner product of $l^2$. In this space $F=\{e_2,e_3,\ldots\}$ is orthonormal and maximal, but not a basis.
\end{example}

Maximal orthonormal families feel like kinds bases, especially given the next couple of results. We would really like the concepts of orthonormal basis and maximal orthonormal family to coincide. Spaces in which they do not are missing something; they are not complete. This is one good reason we are most often interested in Hilbert spaces.

\begin{theorem}
\begin{itemize}
\item Every vector space has a maximal orthonormal set.
\item Every orthonormal set can be extended to a maximal orthonormal set.
\end{itemize}
\end{theorem}
\begin{proof}
The first statement follows easily from the second. The second statement is proved using Zorn's lemma. Let $S$ be an orthonormal set. Define
\[ \mathcal{A} = \{ D\subset V \;|\; S\subset D \; \text{and $D$ is orthonormal} \} \]
ordered by inclusion. It is easy to see that any chain on $\mathcal{A}$ has an upper bound on $\mathcal{A}$, by just taking the union which is still orthonormal. It follows from Zorn's lemma that $\mathcal{A}$ has a maximal element $R$. This is by definition an orthonormal basis.

In the finite-dimensional case this can also be proved using Gram-Schmidt.
\end{proof}

\begin{proposition}
Given a vector space $V$, any two maximal orthonormal sets have the same cardinality.
\end{proposition}
\begin{proof}
Take $D = \{e_i\}_{i\in I}$ and $D' = \{f_j\}_{j\in J}$ maximal orthonormal sets.
\end{proof}
\begin{definition}
An inner product space is \udef{separable} if it is separable as a metric space, i.e. it admits a countable dense subset.
\end{definition}
\begin{proposition}
An inner product space is separable \textup{if and only if} it admits an orthonormal basis with at most countably many vectors.
\end{proposition}
\begin{proof}
TODO
\end{proof}

\subsection{Orthogonal complements}
\begin{definition}
Let $U$ be a subset of an inner product space $V$. The \udef{orthogonal complement} $U^\perp$ of $U$ is the set of vectors in $V$ that are orthogonal to every vector in $U$:
\[ U^\perp = \{ v\in V\;|\; \inner{v,u}=0\; \forall u\in U \}. \]
\end{definition}
We can also consider the orthogonal complement of a subspace with respect to another subspace, not the full space.
\begin{definition}
Let $U\subseteq W$ be subsets of an inner product space $V$. The \udef{orthogonal complement} of $U$ with respect to $W$ is the set of vectors in $W$ that are orthogonal to every vector in $U$:
\[ W\ominus U = \{ w\in W\;|\; \inner{w,u}=0\; \forall u\in U \}. \]
\end{definition}

\begin{proposition} \label{prop:OrthogonalComplementProperties}
Let $U,W$ be \emph{subsets} of an inner product space $V$.
\begin{enumerate}
\item $U^\perp$ is a subspace of $V$;
\item $U^\perp = \Span(U)^\perp$;
\item $\{0\}^\perp = V$;
\item $V^\perp = \{0\}$;
\item $U\cap U^\perp \subset \{0\}$;
\item If $U\subset W$, then $W^\perp \subset U^\perp$.
\end{enumerate}
\end{proposition}

\begin{proposition} \label{prop:ominusUnderIsometry}
Let $V$ be an inner product space and $T:V\to V$ an isometry. Let $A\supseteq B$ be subspaces. Then
\[ T[A\ominus B] = T[A]\ominus T[B]. \]
\end{proposition}
\begin{proof}
Take $v\in T[A\ominus B]$. Then there exists an $x\in A$ such that $T(x) = v$ and $\inner{x,b}=0$ for all $b\in B$. Then by isometry $\inner{T(x), T(b)}=0$ for all $b\in B$. So $v\in T[A]\ominus T[B]$. This reasoning can be inverted to give the other inclusion. 
\end{proof}

\begin{proposition} \label{prop:linearDeMorgan}
Let $W_1,W_2$ be subspaces of an inner product space $V$. Then
\[ (W_1+W_2)^\perp = W_1^\perp \cap W_2^\perp. \]
\end{proposition}
\begin{proposition} \label{prop:orthogonalComplementClosed}
Let $U$ be a \emph{subset} of an inner product space $V$. Then $U^\perp$ is closed and $\overline{U}^\perp = U^\perp$. This can be rephrased as
\[ \overline{U}^\perp = \overline{U^\perp} = U^\perp. \]
Also
\[\overline{U} \subset (U^\perp)^\perp. \]
\end{proposition}
\begin{proof}
Let $x\in \overline{U^\perp}$. Then there exists a sequence $(x_i)$ in $U^\perp$ that converges to $x$. For all $u\in U$, the functional $\inner{u,\cdot}:y\mapsto \inner{u,y}$ is bounded (by Cauchy-Schwarz). Thus all these functionals are continuous. Applying any one to the sequence $x_i$ gives a sequence of zeros. Thus $\inner{u,x} = 0$ for all $u\in U$. Thus $x\in U^\perp$ and hence $U^\perp \supset \overline{U^\perp}$ meaning $U^\perp$ is closed.

Now $\overline{U}\supset U$, so $\overline{U}^\perp \subset U^\perp$. For the other inclusion, take an $x\in U^\perp$. Take an arbitrary $y\in \overline{U}$. Then there exists a sequence $(y_i)$ in $U$ that converges to $y$. Apply the bounded functional $\inner{x,\cdot}$ to the sequence $(y_i)$, yielding a sequence of zeros. Thus $\inner{x,y}=0$. Thus $x\in \overline{U}^\perp$.

Finally let $u\in \overline{U}$. Take a sequence $u_i\to u$. Take an arbitrary element $x\in U^\perp$. As before $\inner{x,u} = \lim_i\inner{x,u_i} = 0$. So $u\in (U^\perp)^\perp$.
\end{proof}
\begin{corollary} \label{corollary:orthogonalComplementClosed}
If the subset $U$ is dense in $V$, then $U^\perp = \{0\}$. 
\end{corollary}
\begin{proof}
\[ U^\perp = \overline{U}^\perp = V^\perp = \{0\}. \]
\end{proof}
\begin{proposition}
Let $U$ be a finite-dimensional subspace of an inner product space $V$.
\begin{enumerate}
\item $V=U\oplus U^\perp$;
\item $U = (U^\perp)^\perp$.
\end{enumerate}
\end{proposition}
Notice that $V$ may be infinite dimensional!
\begin{proof}
We start with the first point. The sum $U + U^\perp$ is definitely direct, $U\oplus U^\perp$, by proposition \ref{prop:OrthogonalComplementProperties} and the criterion for a direct sum, proposition \ref{prop:directSumCriterion}. Clearly $U\oplus U^\perp\subseteq V$, so we just need to show that $V \subseteq U\oplus U^\perp$.

To that end, take a vector $v\in V$. Let $\{e_i\}_{i=1}^n$ be an orthonormal basis of $U$. We can write
\[ v = \left(v - \sum_{i=1}^n\inner{v,e_i}e_i\right) + \left(\sum_{i=1}^n\inner{v,e_i}e_i\right). \]
The first part is an element of $U^\perp$, the second of $U$, so $v\in U\oplus U^\perp$.

For the second point: any finite-dimensional subspace $U$ is automatically closed, so $U = \overline{U} \subset (U^\perp)^\perp$, by proposition \ref{prop:orthogonalComplementClosed}. For the other inclusion, take $v\in (U^\perp)^\perp$. By the first point, we can write $v = v_1 + v_2$ where $v_1\in U$ and $v_2\in U^\perp$. Because $v\in (U^\perp)^\perp$ and $v_2\in U^\perp$, we must have
\[ 0 = \inner{v_2, v} = \inner{v_2, v_1+v_2} = \inner{v_2, v_1} + \inner{v_2,v_2} = \norm{v_2}. \]
So $v=v_1\in U$.
\end{proof}

TODO all projection results for projection onto finite dim? See proposition before Bessel inequality. In fact better: projection onto summand of direct sum! Put under decompositions.


A result dual to proposition \ref{prop:linearDeMorgan} also holds in finite-dimensional spaces:
\begin{proposition}
Let $W_1,W_2$ be subspaces a finite-dimensional space $V$. Then
\[ (W_1\cap W_2)^\perp = W_1^\perp + W_2^\perp. \]
\end{proposition}
\begin{proof}
We start by applying proposition \ref{prop:linearDeMorgan} to $W_1^\perp$ and $W_2^\perp$:
\[ (W_1^\perp+W_2^\perp)^\perp = (W_1^\perp)^\perp \cap (W_2^\perp)^\perp = W_1 \cap W_2. \]
Taking the orthogonal complement of both sides gives the result. In infinite dimensions $(W_1^\perp+W_2^\perp)$ is not necessarily closed. 
\end{proof}



\section{Maps on inner product spaces}

\begin{lemma}[Continuity of inner product]
Let $V$ be an inner product space. Then the inner product is a continuous function $V\times V \to \mathbb{F}$.
\end{lemma}
\begin{proof}
We show that if $x_n \to x$ and $y_n \to y$, then $\inner{x_n,y_n}\to \inner{x,y}$. By the triangle and Cauchy-Schwarz inequalities
\begin{align*}
|\inner{x_n,y_n}-\inner{x,y}| &= |\inner{x_n,y_n}-\inner{x_n,y}+\inner{x_n,y} - \inner{x,y}| \\
&\leq |\inner{x_n, y_n-y}| + |\inner{x_n-x, y}| \\
&\leq \norm{x_n}\norm{y_n-y} + \norm{x_n-x}\norm{y}.
\end{align*}
Because the right-hand side converges to $0$, the left-hand side must too.
\end{proof}

\subsection{Bounded operators}
\begin{lemma} \label{lemma:operatorNormInnerProduct}
Let $T\in\Bounded(V,W)$, then
\begin{align*}
\norm{T} &= \sup_{w\in \im(T),v \in \dom(T)} \frac{|\inner{w,Tv}|}{\norm{w}\,\norm{v}} \\
&= \sup\setbuilder{|\inner{w,Tv}|}{w\in \im(T)\;\land\; v\in\dom{T}\;\land\; \norm{w} = 1 = \norm{v}} \\
&= \sup_{w\in W,v \in \dom(T)} \frac{|\inner{w,Tv}|}{\norm{w}\,\norm{v}} \\
&= \sup\setbuilder{|\inner{w,Tv}|}{w\in W\;\land\; v\in\dom{T}\;\land\; \norm{w} = 1 = \norm{v}}.
\end{align*}
\end{lemma}
\begin{proof}
We prove
\[ \norm{T} \leq \sup_{w\in \im(T),v \in \dom(T)} \frac{|\inner{w,Tv}|}{\norm{w}\,\norm{v}} \leq \sup_{w\in W,v \in \dom(T)} \frac{|\inner{w,Tv}|}{\norm{w}\,\norm{v}} \leq \norm{T}. \]
The first two inequalities follow from the characterisation \ref{prop:operatorNorm}
\[ \norm{T} = \sup_{v \in \dom(T)} \frac{\norm{Tv}}{\norm{v}} = \sup_{v \in \dom(T)} \frac{\inner{Tv,Tv}}{\norm{Tv}\,\norm{v}} \]
and the inclusions
\begin{align*}
\setbuilder{\frac{|\inner{w,Tv}|}{\norm{w}\,\norm{v}}}{v\in\dom(T), w = Tv} &\subseteq \setbuilder{\frac{|\inner{w,Tv}|}{\norm{w}\,\norm{v}}}{v\in\dom(T), w\in\im(T)}\\
&\qquad\quad\subseteq \setbuilder{\frac{|\inner{w,Tv}|}{\norm{w}\,\norm{v}}}{v\in\dom(T), w\in V}.
\end{align*}
The last equality follows from the Cauchy-Schwarz inequality \ref{theorem:CauchySchwarz}:
\[ \frac{|\inner{w,Tv}|}{\norm{w}\,\norm{v}} \leq \frac{\norm{w}\,\norm{Tv}}{\norm{w}\,\norm{v}} = \frac{\norm{Tv}}{\norm{v}} \leq \frac{\norm{T}\,\norm{v}}{\norm{v}} = \norm{T} \]
for all $v\in\dom(T), w\in V$. 
\end{proof}

\subsection{Isometries}
\begin{lemma} \label{lemma:equalityOfMapsInnerProductSpaces}
Let $V$ be an inner product space and $S,T\in\Hom(V)$. Then $S=T$ \textup{if and only if}
\[ \forall v,w\in V: \inner{Tv,w} = \inner{Sv,w}. \]
\end{lemma}
\begin{proof}
The direction $\boxed{\Rightarrow}$ is obvious. For the other direction, use
\[ 0 = \inner{Tv,w} - \inner{Sv,w} = \inner{(T-S)v,w} \]
for all $v,w$. In particular $w=(T-S)v$. The result follows from definiteness of the inner product.
\end{proof}

\begin{lemma}
Let $V,W$ be inner product spaces. Let $f:V\to W$ be a function. Then $f$ preserves the metric (i.e. is an isometry) \textup{if and only if} $f$ also preserves the inner product:
\[ \forall x,y \in V: \quad \inner{f(x),f(y)}_W = \inner{x,y}_V. \]
\end{lemma}
The proof is a simple application of the polarisation identities.

\begin{definition}
Let $V,W$ be an inner product spaces. A linear map $U\in\Hom(V,W)$ is called \udef{unitary} if it is an isometry and invertible.

Unitary operators on real vector spaces are also called \udef{orthogonal operators}.
\end{definition}
Because every isometry is injective (see lemma \ref{lemma:isometryInjective}), it is enough for a linear map to be isometric and surjective to be unitary.

\begin{lemma}
Every unitary map is bounded and has norm $1$.
\end{lemma}
\begin{proof}
Let $U: V\to W$ be a unitary map between inner product spaces. Then $\forall v\in V: \norm{U(v)} = \norm{v}$.
\end{proof}

Unitary operators transform orthonormal bases to orthonormal bases:
\begin{proposition}
Let $T\in \Hom(V,W)$ with $V,W$ inner product spaces and let $V$ have an orthonormal basis $\{e_i\}_{i\in I}$. Then $T$ is unitary \textup{if and only if} $\{Te_i\}_{i\in I}$ is an orthonormal basis of $W$.
\end{proposition}
\begin{proof}
Assume $T$ unitary. The family $\{Te_i\}_{i\in I}$ is certainly orthonormal, by preservation of the inner product. Now let $w\in W$ and so $T^{-1}w\in V$. By the Plancherel formula, proposition \ref{prop:plancherel}, we can write
\[ T^{-1}w = \sum_{n=1}^\infty \inner{e_{i_n},T^{-1}w}e_{i_n} = \lim_{N\to\infty}\sum_{n=1}^N \inner{e_{i_n},T^{-1}w}e_{i_n} \]
and so
\[ w = TT^{-1}w = T\lim_{N\to\infty}\sum_{n=1}^N \inner{e_{i_n},T^{-1}w}e_{i_n} = \lim_{N\to\infty}\sum_{n=1}^N \inner{e_{i_n}T^{-1}w}Te_{i_n} \]
because $T$ is bounded and thus continuous, by theorem \ref{theorem:boundedLinearMaps}.
Thus $\{Te_i\}_{i\in I}$ is an orthonormal basis of $W$.

Conversely, assume $\{Te_i\}_{i\in I}$ is an orthonormal basis of $W$. We first prove $T$ is bounded, which is a simple application of Parseval's identity, proposition \ref{prop:totalONBParsevalEquivalence}:
\[ \norm{Tv}^2 = \sum_{i\in I}|\inner{Te_i,Tv}|^2 = \sum_{i\in I}|\inner{e_i,v}|^2 = \norm{v}^2. \]
The rest of the proof is again an application of the Plancherel formula.
\end{proof}

\begin{lemma}
Let $U$ be a unitary map. If $\lambda$ is an eigenvalue of $U$, then $|\lambda| = 1$.
\end{lemma}
\begin{proof}
Let $v$ be an eigenvector associated to the eigenvalue $\lambda$. Then
\[ \inner{v,v} = \inner{L(v),L(v)} = \inner{\lambda v, \lambda v} = \lambda^2\inner{v,v},  \]
so $\lambda^2 = 1$.
\end{proof}

\subsection{Symmetric operators}
\begin{definition}
Let $(\mathbb{F},V,+,\inner{\cdot,\cdot})$ be an inner product space. A linear operator $L$ is called \udef{symmetric} if, $\forall v,w\in \dom(L)$
\[ \inner{L(v),w} = \inner{v,L(w)}. \]
\end{definition}

\begin{proposition}
Let $V$ be an inner product space and $L$ a symmetric operator on $V$. Then eigenvectors of $L$ associated to different eigenvalues are orthogonal.
\end{proposition}
\begin{proof}
Let $v,w$ be eigenvectors of $L$ with eigenvalues $\lambda, \mu$ such that $\lambda \neq \mu$. Then
\[ \lambda\inner{v,w} = \inner{\lambda v,w}=\inner{L(v),w} = \inner{v,L(w)} = \inner{v,\mu w} = \mu \inner{v,w} \]
and consequently $\inner{v,w} =0$.
\end{proof}

\subsection{Impact on subspaces}
\subsubsection{Invariant and reducing subspaces}
\begin{definition}
Let $V$ be an inner product space and $T$ a linear operator on $V$.
\begin{itemize}
\item A subspace $U\subseteq V$ is said to be \udef{invariant} under $T$ if $T[U] \subset U$.
\item A subspace $U\subseteq V$ is said to be \udef{reducing} for $T$ if both $U$ and $U^\perp$ are invariant under $T$.
\end{itemize}
\end{definition}

\subsection{Quadratic form associated with an operator}
\begin{definition}
Let $T$ be a linear operator on an inner product space $V$. The \udef{quadratic form associated with $T$} is
\[ Q_T: \dom(T)\to \F: u\mapsto \inner{u,Tu}. \]
\end{definition}
\begin{lemma} \label{lemma:symmetricRealQuadraticForm}
If $T$ is a symmetric operator, then its associated quadratic form is real-valued.
\end{lemma}
\begin{proof}
Assume $T$ symmetric, then for all $u\in\dom(T)$
\[ Q_T(u) = \inner{u,Tu} = \inner{Tu,u} = \overline{\inner{u,Tu}} = \overline{Q(u)}. \]
\end{proof}

\subsubsection{Rayleigh quotient}
\begin{definition}
Let $T$ be a linear operator on an inner product space $V$. The \udef{Rayleigh quotient} for $T$ is 
\[ J_T: \dom(T)\setminus\{0\}\to \F: u\mapsto \frac{Q(u)}{\norm{u}^2} = \frac{\inner{u,Tu}}{\norm{u}^2}. \]
We may also write just $J$ if the intended operator $T$ is clear.
\end{definition}

\subsubsection{Numerical range}
\url{https://users.math.msu.edu/users/shapiro/pubvit/downloads/numrangenotes/numrange_notes.pdf}

\url{https://pskoufra.info.yorku.ca/files/2016/07/Numerical-Range.pdf}

\url{http://www.math.wm.edu/~ckli/nrnote}

\url{https://link-springer-com.ezproxy.ulb.ac.be/content/pdf/10.1007%2F978-3-319-01448-7.pdf}

\begin{definition}
Let $T$ be a linear operator on an inner product space $V$ and $J_T$ the Rayleigh quotient of $T$. The range $\NumRange(T) \defeq \im(J_T)$ is known as the \udef{numerical range}.
\end{definition}

The numerical range of $T$ can equivalently be defined as the image of the unit sphere under the quadratic form associated to $T$.

\begin{lemma}
Let $V$ be an inner product space and $T$ a bounded symmetric operator on $V$. Then
\begin{enumerate}
\item the directional derivative $\partial_v(J_T(u))$ exists if $u\neq 0$ and is equal to (TODO remove and place in proof?)
\[ \partial_v(J_T)|_u = \frac{\inner{u,u}\Big( \inner{v,Tu} + \inner{u,Tv} \Big) - \inner{u,Tu}\Big(\inner{u,v}+\inner{v,u}\Big)}{\inner{u,u}^2}; \]
\item $u\in V\setminus \{ 0 \}$ is a critical point of $J_T$ \textup{if and only if} $u$ is an eigenvector of $T$ with corresponding eigenvalue $\lambda = J_T(u)$.
\end{enumerate}
\end{lemma}
\begin{proof}
TODO: critical point in $\C$ v $\R$?? (For symmetric operators $J$ is real valued)
\ref{prop:derivativeBilinearForm}
\end{proof}

\subsubsection{Numerical radius}
\begin{definition}
Let $T$ be a linear operator on an inner product space $V$. Then
\[ \nr(T) \defeq \sup_{u\in \dom(T)\setminus\{0\}} |J_T(u)| \]
is the \udef{numerical radius}.
\end{definition}
If $Q_T$ is the quadratic form associated to an operator $T$, we have
\[ |Q_T(u)| \leq \norm{u}^2\nr(T). \]

\begin{proposition}
Let $T$ be a bounded operator on an inner product space $V$, then $\forall u\in \dom(T)\setminus\{0\}$
\[ |J_T(u)| \leq \nr(T) \leq \norm{T}. \]
If $T$ is also symmetric and has $\dom(T)=V$, then $\norm{T} = \nr(T)$.
\end{proposition}
\begin{proof}
The first claim follows simply from the Cauchy-Schwarz inequality \ref{theorem:CauchySchwarz}
\[ |J(u)| \leq \frac{\norm{u}\,\norm{Tu}}{\norm{u}^2} = \frac{\norm{Tu}}{\norm{u}} \leq \frac{\norm{T}\norm{u}}{\norm{u}} = \norm{T}. \]
For the second claim we need to also show the inverse inequality. By \ref{lemma:operatorNormInnerProduct} it is enough to show that $|\inner{w,Tv}| \leq \nr(T)$ for all $w,v\in V$ with $\norm{v} = 1 = \norm{w}$.

Take arbitrary unit vectors $v,w\in V$ and let $\theta$ be such that $|\inner{w,Tv}| = e^{i\theta}\inner{w,Tv}$. Then $\inner{e^{-i\theta}w,Tv}$ is real, so, viewing it as a sesquilinear form, the imaginary parts of the polarisation identity \ref{theorem:polarisationIdentities} cancel:
\begin{align*}
\inner{e^{-i\theta}w,Tv} &= \frac{1}{4}\sum_{k=0}^3i^k \inner{(i^ke^{-i\theta}w + v), T((i^ke^{-i\theta}w + Tv))} \\
&= \frac{1}{4}\Big( \inner{v+e^{-i\theta}w, T(v+e^{-i\theta}w)} - \inner{v-e^{-i\theta}w, T(v-e^{-i\theta}w)} \Big),
\end{align*}
where we have used that the quadratic form is real by \ref{lemma:symmetricRealQuadraticForm}.

Thus
\begin{align*}
|\inner{w,Tv}| &= |\inner{e^{-i\theta}w,Tv}| \\
&\leq \frac{1}{4}\Big( |\inner{v+e^{-i\theta}w, T(v+e^{-i\theta}w)}| + |\inner{v-e^{-i\theta}w, T(v-e^{-i\theta}w)}| \Big) \\
&\leq \frac{1}{4}\nr(T)\Big( \norm{v+w}^2 + \norm{v-w}^2 \Big) \\
&= \frac{1}{4}\nr(T)\Big( 2\norm{v}^2 + 2\norm{w}^2 \Big) = \nr(T),
\end{align*}
where we have used the fact that $v,w$ are unit vectors and the parallelogram law \ref{theorem:parallelogramLaw}.
\end{proof}

\chapter{Bilinear and multilinear maps}
TODO: bilinear maps, bilinear forms = bilinear functionals
orthogonality

\section{Bilinear form}
\begin{definition}
Let $V$ be a vector space over a field $\F$. A function $B:V\times V\to \F$ is called a \udef{bilinear form} on $V$ if for all $v\in V$, both $B(v,-)$ and $B(-,v)$ are linear.
\end{definition}

\subsection{Quadratic forms}
\begin{definition}
Let $V$ be a vector space over a field $\F$ and $B$ a bilinear form on $V$. The function
\[ q_B: V\to \F: v\mapsto B(v,v) \]
is called the associated \udef{quadratic form}.
\end{definition}

\begin{proposition}
Let $V$ be a vector space over a field $\F$ and $B$ a bilinear form on $V$. Then
\[ B(v,w) + B(w,v) = q_B(v+w) - q_B(v) - q_B(w). \]
\end{proposition}
\begin{corollary}
If $B$ is symmetric and $\F$ is not of characteristic $2$, then we can recover the bilinear form from the associated quadratic form:
\[ B(v,w) = \frac{1}{2}\Big(q_B(v+w) - q_B(v) - q_B(w)\Big). \]
\end{corollary}
So there is a bijection between symmetric and bilinear forms over fields not of characteristic $2$.

\begin{proposition}
Let $V$ be a vector space over a field $\F$ and $q: V\to \F$ a function. Then $q$ is the quadratic form associated to some bilinear form \textup{if and only if}
\begin{itemize}
    \item $\forall \lambda \in \F\forall v\in V: \; q(\lambda v) = \lambda^2 q(v)$;
    \item the parallelogram law holds: $\forall v,w\in V$
    \[ q(v+w) + q(v-w) = 2(q(v)+q(w)). \]
\end{itemize}
\end{proposition}
\begin{proof}
First assume $q$ is the quadratic form associated with the bilinear form $B$. Then $q(\lambda v) = B(\lambda v, \lambda v) = \lambda^2 B(v,v) = \lambda^2 q(v)$. The proof of the parallelogram law is the same as in an inner product space.

For the converse, we need to show that $q(v+w) - q(v) - q(w)$ is bilinear. TODO! (only real / complex??)
\end{proof}

Sylvester's law of inertia


\section{Tensor product}
\url{https://kconrad.math.uconn.edu/blurbs/linmultialg/tensorprod.pdf}
\subsection{Free vector space}
Given any set, we can construct a vector space by viewing each element in the set as a (linearly) independent (basis) vector. The vector space then consists of formal linear combinations of these vectors.

To be more precise:
\begin{definition}
Let $S$ be a set and $K$ a field. Then define
\[ F_K(S) \defeq \setbuilder{f\in(S\to K)}{f^{-1}[K\setminus\{0\}]\;\text{is finite}}. \]
Define the following operations on $F_K(S)$:
\begin{align*}
+ &: F(S)\times F(S) \to F(S): (f,g)\mapsto (f+g: x\mapsto f(x)+g(x)) \\
\cdot &: K\times F(S) \to F(S): (\lambda,f)\mapsto (\lambda f: x\mapsto \lambda f(x)) \\
\end{align*}
The operations $+,\cdot$ are well-defined and make $F(S)$ into a vector space, called the \udef{free vector space} over $S$.
\end{definition}

\begin{proposition}
Let $S$ be a set. Then we can identify $S$ with a subset of $F(S)$ by
\[ \iota: S\hookrightarrow F(S): x\mapsto \chi_{\{x\}}. \]
With this identification $S$ forms a basis for $F(S)$.
\end{proposition}

\begin{lemma}
Let $V$ be a vector space over $K$ and $\beta$ a basis for $V$. Then $V\cong  F_K(V)$.
\end{lemma}

\subsubsection{The free functor}
\begin{proposition}
The operation $F$ of finding the free vector space over a set can be extended to an contravariant functor
\[ F: \cat{Set} \to \cat{Vect}. \]
\end{proposition}
\begin{proof}
Let $f:X\to Y$ be a function between sets.
\end{proof}
TODO: just specific instance up to isomorphism?? covariant?? isomorphism class of all vector spaces with basis $S$?? Is this why tensor product only up to isomorphism??

\subsubsection{Universal property}
\begin{proposition}
Let $\phi$ be an arbitrary function from $S$ to a vector space $W$ over a field $K$, then there exists a unique linear map $\overline{\phi}: F(S)\to W$ such that the diagram
\[ \begin{tikzcd}
S \ar[r,"\iota"] \ar[dr,"\phi"] & F(S) \ar[d,dashed,"\overline{\phi}"] \\
& W
\end{tikzcd} \qquad \text{commutes.} \]
Furthermore, $F(S)$ is the unique $K$-vector space with this property.
\end{proposition}

\subsection{Abstract definition}
The idea behind the tensor product of two vector spaces $V,W$ over a field $K$ is to create the most general set of pairings that is a vector space and such that the pairings are bilinear: $\forall \lambda\in K: \forall v_1,v_2,v\in V:\forall w_1,w_2,w\in W$:
\[ (\lambda v_1+v_2, w) = \lambda (v_1,w)+(v_2,w) \qquad\text{and}\qquad (v,\lambda w_1+w_2) = \lambda (v,w_1) + (v,w_2). \]
This will be realised as a quotient of a free vector space.

To be more precise:
\begin{definition}
Let $V,W$ be vector spaces over a field $K$.
Consider the set $\operatorname{Field}(V)\times \operatorname{Field}(W)$, which we will refer to as $V\times W$. Construct the sets
\begin{align*}
R_1 &= \setbuilder{(\lambda(v_1,w)+(v_2,w),(\lambda v_1+v_2, w))\in F_K(V\times W)}{\lambda\in K; v_1,v_2\in V; w\in W} \\
R_2 &= \setbuilder{(\lambda(v,w_1)+(v,w_2),(v, \lambda w_1 + w_2))\in F_K(V\times W)}{\lambda\in K; v_1,v_2\in V; w\in W} \\
R &= R_1\cup R_2
\end{align*}
The \udef{tensor product} $V\otimes W$ of the vector spaces $V$ and $W$ is the quotient vector space
\[ V\otimes W := F(V\times W)/R^\equiv \]
where $R^\equiv$ is the reflexive symmetric transitive closure of $R$.

The equivalence class $[(v,w)]$ is denoted $v\otimes w$. An element of $V\otimes W$ that can be written as $v\otimes w$ is called a \udef{pure tensor} or \udef{simple tensor}.
\end{definition}
In order for the definition to be well-defined, we need for $R^\equiv$ to be a congruence on $F$.

The \udef{tensor product} $V\otimes W$ of two vector spaces $V$ and $W$ over a common field $K$ is the quotient vector space
\[ V\otimes W := F(V\times W)/\sim \]
where $\sim$ is the equivalence relation over the the free vector space $F(V\times W)$ with the properties of
\begin{itemize}
\item \textit{Distributivity}: $(v+v', w) \sim (v,w) + (v',w)$ and $(v, w+w') \sim (v,w) + (v,w')$.
\item \textit{Scalar multiples}: $c(v,w) \sim (cv,w) \sim (v,cw)$.
\end{itemize}



TODO definition via bases: $V\otimes W = F(\beta_V\times \beta_W)$

\subsection{Universal property}
See also proposition \ref{dimHomset}.


\subsection{Tensor product of linear maps}
The tensor product also operates on linear maps between vector spaces.
\begin{definition}
Given two linear maps $S: V\to X$ and $T:W\to Y$, then the \udef{tensor product} of the linear maps $S$ and $T$ is the linear map
\[ S\otimes T: V\otimes W \to X\otimes Y \]
defined by
\[ (S\otimes T)(v\otimes w) = S(v)\otimes T(w). \]
For vectors that are not pure tensors, this definition is extended by linearity.
\end{definition}
\begin{lemma}
The tensor product of linear maps is well-defined.
\end{lemma}

With this definition the tensor product becomes a bifunctor from the category of vector spaces to itself, covariant in both arguments.

TODO: functional calculus on tensor product.
TODO: tensor product of operator algebras


\subsection{Operator-valued matrices}
\begin{proposition}
Let $A$ be an algebra over a field $\mathbb{F}$ and $\beta$ any set. Consider the direct sum $A^\beta = \bigoplus_{i\in\beta}A$. Then $A^\beta \cong F_\F(\beta)\otimes A$.
\end{proposition}
\begin{proof}
TODO
\end{proof}


\subsection{Matrix representation}
\subsubsection{Finding a basis}
Assume $V$ and $W$ are finite-dimensional vector spaces with resp. bases $\{\vec{e}_i\}_i$ and $\{\vec{f}_j\}_j$. Then the set $\{ \vec{e}_i\otimes \vec{f}_j \}_{i,j}$ forms a basis for $V\otimes W$. Indeed,
\begin{itemize}
\item Take two arbitrary vectors $\vec{v} = \sum_i a_i \vec{e}_i \in V$ and $\vec{w} = \sum_j b_j \vec{f}_j \in W$.
Using the distributivity and scalar multiples properties of $\sim$, we can write the tensor product $\vec{v}\otimes \vec{w}$ as
\begin{equation} \vec{v}\otimes \vec{w} = (\sum_i a_i \vec{e}_i)\otimes(\sum_j b_j \vec{f}_j) = \sum_{i,j}a_ib_j (\vec{e_i}\otimes \vec{f}_j). \label{eq:vtensorw} \end{equation}
So any pure tensor can be written as the sum of vectors of the form $\vec{e}_i\otimes \vec{f}_j$. In general a vector in $V\otimes W$ can be written as a finite sum of pure tensors, meaning the set of vectors $\{ \vec{e}_i\otimes \vec{f}_j \}_{i,j}$ spans $V\otimes W$.
\item For linear independence we, observe that for any linearly independent $v_1, v_2, w_1, w_2$, the vector $v_1\otimes w_1 + v_2\otimes w_2$ cannot be written as a pure tensor.
\end{itemize}

Clearly it follows that
\[ \dim(V\otimes W) = \dim(V)\cdot\dim(W) \]

\subsubsection{Coordinates and the outer product}
The coordinates of a vector with respect to the basis $\{ \vec{e}_i\otimes \vec{f}_j \}_{i,j}$ can naturally be put into a matrix. Taking the tensor product of two vectors corresponds to taking the outer product of their coordinate vectors. That is, setting $\co(v) = \vec{v}$ and $\co(w) = \vec{w}$, we get
\[ \co(v\otimes w)_{i,j} = a_ib_j = (\vec{v}\vec{w}^\transp)_{i,j} \]
which follows from \eqref{eq:vtensorw} above.

For this reason $\otimes$ is also used to denote the outer product.

If we want a proper column vector as our coordinate vector, we can apply row-by-row vectorisation to this matrix.
\[ \co(v\otimes w) = \vectorisation_R(\vec{v}\vec{w}^\transp) = \vec{v}\otimes\vec{w} = \co(v)\otimes\co(w). \]
where $\otimes$ is also used to denote the Kronecker product.

Coordinates for vectors that are not pure tensors can easily be found by the linearity of the coordinate map.
\subsubsection{Linear maps and the Kronecker product}
Letting the coordinates be columns, we can hope to find a matrix for the linear map $S\otimes T$. Fix bases for the spaces $V,W,X,Y$. Let $A$ and $B$ be the matrices of $S$ and $T$ with respect to these bases. Use these bases to fix the bases for $V\otimes W$ and $X\otimes Y$.

\begin{eigenschap}
The matrix of the map $S\otimes T$ with respect to these bases is the matrix $A\otimes B$, where $\otimes$ is the Kronecker product.
\end{eigenschap}

This follows from a simple calculation:
\begin{align*}
\co\left(S\otimes T(v\otimes w)\right) &= \co\left(S(v)\otimes T(w)\right) \\
&= \co\left(S(v)\right)\otimes \co\left(T(w)\right) \\
&= A\co(v)\otimes B\co(v) \\
&= (A\otimes B)\co(v)\otimes\co(w) \qquad (\text{using the mixed product}) \\
&= (A\otimes B)\co(v\otimes w).
\end{align*}
Again this calculation can be extended to non-pure tensors by linearity.

\subsection{Properties}
TODO currying.
\url{https://math.stackexchange.com/questions/679584/why-is-texthomv-w-the-same-thing-as-v-otimes-w}
And reference later!
\subsection{Multilinear maps}
\begin{definition}
Let $V^k = V\times \ldots \times V$. A function $f: V^k\to \R$ is \udef{$k$-linear} if it is linear in each of its arguments.
\begin{itemize}
\item A $k$-linear function $f:V^k\to \R$ is \udef{symmetric} if for all permutations $\sigma\in S_k$
\[ f(v_{\sigma(1)},\ldots, v_{\sigma(k)}) = f(v_1,\ldots, v_k). \]
\item A $k$-linear function $f:V^k\to \R$ is \udef{alternating} if for all permutations $\sigma\in S_k$
\[ f(v_{\sigma(1)},\ldots, v_{\sigma(k)}) = (\sgn\sigma)f(v_1,\ldots, v_k). \]
\end{itemize}
We call the space of all alternating $k$-linear maps $A_k(V)$.
\end{definition}
In particular $A_1(V) = V^*$.
\begin{note}
Given a $k$-linear function $f$ and a permutation $\sigma\in S_k$, we define the $k$-linear function $\sigma f$ by
\[ (\sigma f)(v_1,\ldots, v_k) = f(v_{\sigma(1)},\ldots, v_{\sigma(k)}). \]
Then a symmetric map is one such that $\sigma f = f$ for all $\sigma\in S_k$ and an alternating map is one such that $\sigma f = (\sgn \sigma)f$ for all $\sigma\in S_k$.
\end{note} 
\begin{lemma}
Let $\sigma,\tau \in S_k$ and $f$ a $k$-linear map on $V$. Then $\tau(\sigma f) = (\tau \sigma)f$.
\end{lemma}
\subsubsection{The symmetrising and alternating maps}
\begin{definition}
Let $f$ be a $k$-linear map on a vector space $V$.
\begin{itemize}
\item The \udef{symmetrisation} of $f$, $Sf$, is the map
\[ Sf = \frac{1}{k!}\sum_{\sigma\in S_k}\sigma f. \]
\item The \udef{anti-symmetrisation} or \udef{skew-symmetrisation} of $f$, $Af$, is the map
\[ Af = \frac{1}{k!}\sum_{\sigma\in S_k}(\sgn \sigma)\sigma f. \]
\end{itemize}
\end{definition}
\begin{lemma}
\begin{enumerate}
\item The $k$-linear map $Sf$ is symmetric. If $f$ is symmetric, then $Sf = f$.
\item The $k$-linear map $Af$ is alternating. If $f$ is alternating, then $Af = f$.
\end{enumerate}
\end{lemma}
\begin{lemma} \label{idempotenceA}
Let $f$ be a $k$-linear functional and $g$ an $l$-linear functional on $V$. Then
\[ A(A(f)\otimes g) = A(f\otimes g) = A(f\otimes A(g)). \]
\end{lemma}
\subsubsection{The wedge product}
\begin{definition}
Let $f\in A_k(V)$ and $g\in A_l(V)$. The \udef{wedge product} of $f$ and $g$ is given by
\[ f\wedge g = \frac{(k+l)!}{k!l!}A(f\otimes g). \]
\end{definition}
We can also write
\[ (f\wedge g)(v_1,\ldots,v_{k+l}) = \frac{1}{k!l!}\sum_{\sigma\in S_{k+l}}(\sgn \sigma)f(v_{\sigma(1)},\ldots,v_{\sigma(k)})g(v_{\sigma(k+1)},\ldots, v_{\sigma(k+l)}). \]
We can reduce redundancies in this definition in the following way:
We call $\sigma\in S_{k+l}$ a \udef{$(k,l)$-shuffle} if
\[ \sigma(1)<\ldots<\sigma(k) \qquad \text{and}\qquad \sigma(k+1)<\ldots<\sigma(k+l). \]
The we write
\[ (f\wedge g)(v_1,\ldots,v_{k+l}) = \sum_{\text{$(k,l)$-shuffles $\sigma$}}(\sgn \sigma)f(v_{\sigma(1)},\ldots,v_{\sigma(k)})g(v_{\sigma(k+1)},\ldots, v_{\sigma(k+l)}). \]
\begin{proposition}
Let $f\in A_k(V)$ and $g\in A_l(V)$. Then
\[ f\wedge g = (-1)^{kl}g\wedge f. \]
\end{proposition}
\begin{lemma}
The wedge product is associative:
\[ (f\wedge g)\wedge h = f\wedge (g\wedge h). \]
\end{lemma}
Proof using \ref{idempotenceA}.

\begin{lemma}
Let $\alpha^1,\ldots, \alpha^k$ be linear functionals on $V$ and $v_1,\ldots,v_k\in V$, then
\[ (\alpha^1\wedge\alpha^k)(v_1,\ldots, v_k) = \det[\alpha^i(v_j)]. \]
\end{lemma}

\subsection{Tensors}
A $(p,k)$-tensor is a multilinear function $V^k\to V^p$.

\section{Real, complex and quaternionic vector spaces}
\begin{definition}
A function $f$ between complex vector spaces is \udef{anti-linear} (or \udef{conjugate-linear}) in the first component:
\[f(\lambda_1 v_1 + \lambda_2 v_2) = \overline{\lambda_1}f(v_1) + \overline{\lambda_2}f(v_2),\]
where $\lambda_1,\lambda_2 \in \C$ and $v_1,v_2\in \dom(f)$.
\end{definition}
\subsection{Complex structure on a real vector space}
\begin{definition}
Let $V$ be a real vector space. A \udef{complex structure} on $V$ is a linear map $J: V\to V$ such that $J^2 = -I_V$.
\end{definition}

\subsection{The real vector spaces associated to a complex vector space}
Let $V = (\C, V, +)$. Then define $V_\R \defeq (\R,V,+)$.

every anti-linear map $A:V\to W$ is an $\R$-linear map $A:V_\R\to W_\R$. (They are equal as sets).

\section{Clifford algebras}
\begin{definition}
Let $V$ be a vector space over a field $\mathbb{F}$ and $q$ a quadratic form defined on $V$.
Let $\mathcal{T}(V)$ be the tensor algebra
\[ \mathcal{T}(V) \defeq \mathbb{F}\oplus \bigoplus_{n=1}^\infty V^n = \mathbb{F}\oplus \bigoplus_{n=1}^\infty \underbrace{V\otimes \ldots \otimes V}_{\text{$n$ times}}. \]
Let $\mathcal{I}(V,q)$ be the (two-sided) ideal in $\mathcal{T}(V)$ generated by
\[ \setbuilder{\vec{v}\otimes \vec{v} - q(v) \vec{1}}{\vec{v}\in V}. \]
Then the \udef{Clifford algebra} $\Cl(V,q)$ associated with $V$ and $q$ is the quotient
\[ \Cl(V,q) \defeq \mathcal{T}(V)/\mathcal{I}(V,q). \]
We call
\begin{itemize}
    \item elements of the Clifford algebra \udef{multivectors};
    \item elements of $\Span(V^k)$ \udef{$k$-vectors};
    \item elements of $V$ \udef{vectors}; we use bold face to denote these elements (e.g.\ $\vec{v}$);
    \item elements of $\F\vec{1}$ \udef{scalars}.
\end{itemize}
Elements of the Clifford algebra are called \udef{multivectors}.
\end{definition}
Let $\pi_q$ be the canonical projection
\[ \pi_q: \mathcal{T}(V) \to \Cl(V,q). \]


TODO: make all vectors bold!

\begin{lemma}
The embedding $V\hookrightarrow \Cl(V,q)$ is faithful, i.e.\ $\pi_q|_V$ is injective.
\end{lemma}
\begin{proof}
TODO
\end{proof}
Clearly $\pi_q|_V(\vec{v})^2 = q(\vec{v}) \vec{1}$ for all $\vec{v}\in V$.

\begin{lemma} \label{vectorInverseCliffordAlgebra}
Let $\Cl(V,q)$ be a Clifford algebra. Then
\[ \Cl^\times(V,q) \cap V = \setbuilder{\vec{v}\in V}{q(v) \neq 0}. \]
The inverse of $\vec{v}\in \Cl^\times(V,q) \cap V$ is $\vec{v}^{-1} = \vec{v}/q(\vec{v})$.
\end{lemma}
\begin{proof}
First take $\vec{v}\in \setbuilder{\vec{v}\in V}{q(\vec{v}) \neq 0}$. This definition of $\vec{v}^{-1}$ is indeed a multiplicative inverse: $\vec{v}\vec{v}^{-1} = \vec{v}^2/q(\vec{v}) = q(\vec{v})/q(\vec{v})\vec{1} = \vec{1}$.

Conversely, take $\vec{v}\in \Cl^\times(V,q) \cap V$. If $q(\vec{v}) = 0$, then $\vec{v}$ would be a zero divisor (as $\vec{v}\vec{v} = q(\vec{v})\vec{1} = 0$). No zero divisor can have an inverse (\ref{inverseZeroDivisor}), 
\end{proof}

\begin{lemma} \label{CliffordRelation}
The algebra $Cl(V,q)$ is generated by the vector space $V$ and $\vec{1}$, subject to the relations
\[ \vec{v}\vec{v} = q(\vec{v})\vec{1} \qquad \forall \vec{v}\in V. \]
\end{lemma}
TODO generators of an algebra!


Clifford algebras can also be defined by their universal property:
\begin{proposition}[Universal property of Clifford algebras] \label{CliffordUniversalProperty}
Let $V$ be a vector space over a field $\mathbb{F}$ and $q$ a quadratic form on $V$. 

Then for any unital associative algebra $A$ over $\mathbb{F}$ and linear map $j: V \to A$ such that
\[ j(\vec{v})^2 = q(\vec{v}) \vec{1} \qquad \forall \vec{v}\in V \]
there exists a unique algebra homomorphism $\widetilde{j}: \Cl(V,q)\to A$ such that the following diagram commutes:
\[ \begin{tikzcd}
V \rar{\pi_q|_V} \ar[dr, swap, "{j}"] & \Cl(V,q) \dar[dashed]{\widetilde{j}} \\
 & A
\end{tikzcd} \]
Furthermore, $\Cl(V,q)$ is the unique associative $\mathbb{F}$-algebra with this property.
\end{proposition}
\begin{corollary}
Let $(V,q)$ and $(V',q')$ be vector spaces with quadratic forms. If a linear map $f:V\to V'$ preserves to quadratic form, $q'\circ f = q$, then $f$ extends to a unique algebra homomorphism
\[ \widetilde{f}: \Cl(V,q) \to \Cl(V',q'). \]
Now let $(V^{\prime\prime},q^{\prime\prime})$ be another vector space equipped with a quadratic form and let $g: V'\to V^{\prime\prime}$ be a linear map preserving the quadratic form. Then
\[ \widetilde{g\circ f} = \widetilde{g}\circ\widetilde{f}. \]
Also isomorphisms of vector spaces extend to isomorphisms of Clifford algebras.
\end{corollary}
\begin{proof}
Let the algebra $A$ of the proposition be $\Cl(V',q')$. Then $\pi_q|_V\circ f$ satisfies the requirement for $j$:
\[ [(\pi_{q'}|_V\circ f)(\vec{v})]^2 = q'(f(\vec{v}))^2 \vec{1} = q(\vec{v})^2\vec{1}. \]
Thus by the proposition, there is a unique extension of $f:V\to V'$ to a map $\Cl(V,q) \to \Cl(V',q')$.

The composition relation follows from uniqueness.
\end{proof}
\begin{corollary} \label{qOrthogonalMaps}
The orthogonal group
\[ \Ogroup(V,q) = \setbuilder{g\in\GL(V)}{q\circ g = q} \]
extends canonically to a group of automorphisms of $\Cl(V,q)$:
\[ \Ogroup(V,q) \subset \Aut(\Cl(V,q)). \]
\end{corollary}


\subsection{Involutions}
\subsubsection{Grade involution}
Given a Clifford algebra $\Cl(V,q)$, consider the map $\alpha: V \to V: \vec{v}\mapsto -\vec{v}$ on the vector space $V$.
Now $\alpha$ is always an element of $\Ogroup(V,q)$, so by \ref{qOrthogonalMaps} it extends to a map on the Clifford algebra.
\[ \widetilde{\alpha}: \Cl(V,q) \to \Cl(V,q). \]
Since $\alpha^2 = I_V$, we have that
\[ \widetilde{\alpha}^2 = \widetilde{\alpha^2} = \widetilde{I_V} = I_{\Cl(V,q)} \]
meaning $\widetilde{\alpha}$ is an involution on the Clifford algebra. From now on we drop the tilde and just write $\alpha: \Cl(V,q) \to \Cl(V,q)$ for the \udef{grade involution}.

\begin{lemma}
The grade involution $\alpha: \Cl(V,q) \to \Cl(V,q)$
\begin{enumerate}
\item is an algebra homomorphism and thus multiplicative:
\[ \alpha(xy) = \alpha(x)\alpha(y) \qquad\forall x,y\in \Cl(V,q); \]
\item is unital, $\alpha(\vec{1}) = \vec{1}$, and thus preserves inverses:
\[ \alpha(x^{-1}) = \alpha(x)^{-1} \qquad\forall x\in \Cl(V,q); \]
\item generates a $\Z_2$-grading 
\[ \Cl(V,q) = \Cl^0(V,q)\oplus \Cl^1(V,q). \]
\end{enumerate}
\end{lemma}

\subsubsection{Transpose}
The transpose map defined on the tensor algebra $\mathcal{T}(V)$, i.e.\ the linear map that reverses to order of homogeneous elements:
\[ v_1\otimes \ldots \otimes v_r \mapsto v_r \otimes \ldots \otimes v_1, \]
preserves the ideal $\mathcal{I}(V,q)$, and so determines a well-defined map on the Clifford algebra $\Cl(V,q)$:
\[ (-)^t: \Cl(V,q)\to \Cl(V,q). \]
\begin{lemma}
The transpose $(-)^t: \Cl(V,q)\to \Cl(V,q)$ is
\begin{enumerate}
\item an involution;
\item an anti-automorphism:
\[ \forall x,y\in \Cl(V,q): \quad (xy)^t = y^tx^t; \]
\item unital.
\end{enumerate}
\end{lemma}

\subsubsection{Clifford conjugation}
\begin{definition}
The composition of the grade involution and the transpose is called \udef{Clifford conjugation}:
\[ x\mapsto \overline{x} \defeq \alpha(x^t). \]
\end{definition}
\begin{lemma}
The grade involution and transpose commute
\[ \alpha \circ(-)^t = (-)^t\circ \alpha \]
and Clifford conjugation is thus equal to both.
\end{lemma}

\begin{lemma}
Clifford conjugation is
\begin{enumerate}
\item an involution;
\item an anti-automorphism:
\[ \forall x,y\in \Cl(V,q): \quad (xy)^t = y^tx^t; \]
\item unital.
\end{enumerate}
\end{lemma}

\subsubsection{Quaternion types of Clifford algebra types}

\subsection{The norm mapping}
\begin{definition}
We define the \udef{norm mapping} $N$ by
\[ N: \Cl(V,q)\to \Cl(V,q): x\mapsto x \overline{x}. \]
\end{definition}
\begin{lemma} \label{normIsQuadraticForm}
\begin{enumerate}
\item If $v\in V$, then $N(v) = q(v)$.
\item $\alpha\circ N = N\circ \alpha$.
\end{enumerate}
\end{lemma}

\subsection{Scalar and outer products}
\begin{lemma}
Let $\Cl(V,q)$ be a Clifford algebra and $\vec{v},\vec{w}\in V$. Then $\vec{v}\vec{w} + \vec{w}\vec{v}$ is a scalar multiple of the identity.
\end{lemma}
\begin{proof}
We can calculate
\[ q(\vec{v}+\vec{w})\vec{1} = (\vec{v}+\vec{w})^2 = \vec{v}^2 + \vec{v}\vec{w} + \vec{w}\vec{v} + \vec{w}^2 = \vec{v}\vec{w} + \vec{w}\vec{v} + q(\vec{v})\vec{1} + q(\vec{w})\vec{1}, \]
so $vw+wv = [q(\vec{v}+\vec{w})-q(\vec{v})-q(\vec{w})]\vec{1}$.
\end{proof}

\begin{definition}
Let $\F$ be a field whose characteristic is not $2$ and $\Cl(V,q)$ a Clifford algebra over $\F$. We can then write, for $\vec{v},\vec{w}\in V$
\begin{align*}
\vec{v}\vec{w} &= \frac{\vec{v}\vec{w}+\vec{w}\vec{v}}{2} + \frac{\vec{v}\vec{w}-\vec{w}\vec{v}}{2} \\
&\defeq \vec{v}\cdot \vec{w} + \vec{v}\wedge \vec{w}.
\end{align*}
We call the symmetric part $\vec{v}\cdot \vec{w}$ the \udef{scalar product} (sometimes also called the \udef{inner product}),
and the antisymmetric part $\vec{v}\wedge \vec{w}$ the \udef{outer product}.
\end{definition}
In the sequel, whenever we talk about the scalar and outer product, we will always assume the characteristic of the field is not $2$.

\begin{lemma}
Let $\Cl(V,q)$ be a Clifford algebra and $\vec{v}\in V$. We have $\vec{v}\cdot \vec{v} = \vec{v}\vec{v} = \vec{v}^2 = q(\vec{v})\vec{1}$.
\end{lemma}
So $\vec{v}^2$ can cause no confusion.

\begin{definition}
Let $\Cl(V,q)$ be a Clifford algebra. We call $\vec{v},\vec{w}\in V$
\begin{itemize}
\item \udef{orthogonal} or \udef{perpendicular} if $\vec{v}\cdot \vec{w} = 0$;
\item \udef{parallel} if $\vec{v}\wedge \vec{w} = 0$.
\end{itemize}
\end{definition}
\begin{lemma}
Let $\Cl(V,q)$ be a Clifford algebra and $\vec{v},\vec{w}\in V$. The following are equivalent:
\begin{enumerate}
\item $\vec{v}$ and $\vec{w}$ are orthogonal;
\item $\vec{v}\vec{w} = \vec{v}\wedge \vec{w}$;
\item $\vec{v}\vec{w} = -\vec{w}\vec{v}$.
\end{enumerate}
\end{lemma}

\subsection{Clifford algebras as filtered algebras}
\begin{proposition}
A Clifford algebra $\Cl(V,q)$ has a filtration $F_k \defeq \Span(V^{k})$.
\end{proposition}

We then have an associated graded algebra and a grade operator.

\begin{proposition}
Let $V$ be a vector space and $q$ a quadratic form on $V$.
\begin{enumerate}
\item As graded algebras, $\Cl(V,q)$ is naturally isomorphic to the exterior algebra ${\textstyle\bigwedge}^* V$.
\item As algebras ${\textstyle\bigwedge}^* V \cong \Cl(V,0)$.
\item As vector spaces, there is an isomorphism
\[ {\textstyle\bigwedge}^* V \to \Cl(V,q): v_1\wedge \ldots \wedge v_n \mapsto \frac{1}{r!}\sum_{\sigma\in S_n}\sgn(\sigma)v_{\sigma(1)}\hdots v_{\sigma(r)} \]
compatible with the fibrations.
\end{enumerate}
\end{proposition}

\begin{lemma}
Let $\Cl(V,q)$ be a Clifford algebra and $\vec{v},\vec{w}\in V$. Then $\grade{\vec{v}\vec{w}}_1 = 0$.
\end{lemma}
\begin{proof}
We have $\alpha(\vec{v}\vec{w}) = (-\vec{v})(-\vec{w}) = \vec{v}\vec{w}$, so $\vec{v}\vec{w}\in\Cl^0(V,q)$, while $V\subseteq \Cl^1(V,q)$.
\end{proof}

\begin{proposition}
Let $\Cl(V,q)$ be a Clifford algebra and $\vec{v}_1, \ldots, \vec{v}_k\in V$. Then $\vec{v}_1\ldots\vec{v}_k \in V^{k-1}$ \textup{if and only if} $\vec{v}_1,\ldots,\vec{v}_k$ are linearly dependent.
\end{proposition}
\begin{proof}
If $\vec{v}_1, \ldots, \vec{v}_k\in V$ are linearly dependent, then we can eliminate one of the $\vec{v}_j$ and then $\vec{v}_1\ldots\vec{v}_k \in V^{k-1}$.

Now assume $\vec{v}_1, \ldots, \vec{v}_k\in V$ are linearly independent.
\end{proof}

\subsection{Orthogonal decomposition}
As $q(u,v)$ is a bilinear form, we can consider orthogonal subspaces with respect to it. Then $V=V_1\oplus V_2$ is a $q$-orthogonal decomposition if and only if $\forall v_1\in V_1, v_2\in V_2$:
\[ q(v_1,v_2) = 0 \qquad \iff \qquad q(v_1+v_2) = q(v_1) + q(v_2). \]
\begin{proposition}
Let $V=V_1\oplus V_2$ be a $q$-orthogonal decomposition. Then there is a natural isomorphism of Clifford algebras
\[ \Cl(V,q) \;\to\; \Cl(V_1,q|_{V_1}) \hat{\otimes} \Cl(V_2, q|_{V_2}):\quad v_1+v_2\mapsto v_1\otimes \vec{1} + \vec{1}\otimes v_2. \]
\end{proposition}

\section{The Pin and Spin groups}

\begin{proposition}
Let $V$ be a finite-dimensional real or complex vector space of dimension $\dim V = n$. Then the group $\Cl^\times(V,q)$ of multiplicative units in the Clifford algebra is a Lie group of dimension $2^n$ and the corresponding Lie algebra $\mathfrak{cl}^\times(V,q)$ is the full Clifford algebra $\Cl(V,q)$ with the Lie bracket
\[ [x,y] = xy - yx.  \]
\end{proposition}

\subsection{Inner automorphisms of $\Cl(V,q)$}
Characteristic for field not 2!!


\begin{proposition} \label{AdOrthogonalDecomposition}
Let $\vec{v}\in V\cap \Cl^\times(V,q)$. Then for all $\vec{w}\in V$:
\[ \Ad_{\vec{v}}(\vec{w}) = \frac{\vec{v}\vec{w}\vec{v}}{q(\vec{v})} = \frac{\vec{v}\vec{w}+\vec{w}\vec{v}}{q(\vec{v})}\vec{v} - \vec{w}. \]
In particular $\Ad_{\vec{v}}[V] = V$.
\end{proposition}
\begin{proof}
From \ref{vectorInverseCliffordAlgebra} we have $q(\vec{v})\neq 0$ and $\vec{v}^{-1} = \vec{v}/q(\vec{v})$.
We then calculate
\[ q(\vec{v})\Ad_{\vec{v}}(\vec{w}) = q(\vec{v})\vec{v}\vec{w}\vec{v}^{-1} = \vec{v}\vec{w}\vec{v} = (\vec{v}\vec{w} + \vec{w}\vec{v} - \vec{w}\vec{v})\vec{v} = (\vec{v}\vec{w}+\vec{w}\vec{v})\vec{v} - q(\vec{v})\vec{w}. \]
\end{proof}

\begin{lemma} \label{AdOrthogonalMap}
Let $\vec{v}\in V$ such that $q(\vec{v})\neq 0$. Then $\Ad_{\vec{v}}\in \Ogroup(V,q)$.
\end{lemma}
TODO renew notation
\begin{proof}
Clearly $\Ad_{\vec{v}}$ is invertible. We then calculate using \ref{AdOrthogonalDecomposition}
\begin{align*}
q(\Ad_v(w)) &= q\left(\frac{q(v,w)}{q(v)}v - w\right) = q\left(\frac{q(v,w)}{q(v)}v,-w\right) + q\left(\frac{q(v,w)}{q(v)}v\right) + q(w) \\
&= -\frac{q(v,w)}{q(v)}q(v,w)+\left(\frac{q(v,w)}{q(v)}\right)^2q(v) + q(w) = q(w).
\end{align*}
\end{proof}


\subsection{Pin and Spin groups}
\begin{definition}
Let $P(V,q)$ be the subgroup of $\Cl^\times(V,q)$ generated by elements $v\in V$ with $q(v)\neq 0$.

We also define the group
\[ \Gamma(V,q) \defeq \setbuilder{x\in\Cl^\times(V,q)}{\Ad_x[V] = V}. \]
which is called the \udef{Clifford group}, \udef{Lipschitz group} or \udef{Clifford-Lipschitz group}.
\end{definition}
That $\Gamma(V,q)$ is a group follows from the following observation: If $\Ad_x[V]=V$ and $\Ad_y[V]=V$, then
\[ \Ad_{xy}[V] = \Ad_x[\Ad_y[V]] = \Ad_x[V] = V. \]


\begin{lemma} \label{PsubgroupVpreserving}
There is an inclusion
\[ P(V,q) \subset \Gamma(V,q). \]
\end{lemma}
\begin{proof}
The generators of $P(V,q)$ are in $\Gamma(V,q)$ by \ref{AdOrthogonalDecomposition} and $\Gamma(V,q)$ is a group.
\end{proof}

\begin{lemma}
The $\Ad$ function defines a representation
\[ \Ad: P(V,q) \to \Ogroup(V,q). \]
\end{lemma}
\begin{proof}
This mapping is well-defined by \ref{AdOrthogonalMap} and the identity
\[ \Ad_{xy} = \Ad_x\circ \Ad_y. \]
This identity also shows that the mapping is a group homomorphism, and thus that it is a representation.
\end{proof}

\begin{lemma}
Let $x\in \Gamma(V,q)$. Then
\begin{enumerate}
\item $\alpha(x)\in \Gamma(V,q)$;
\item $x^t\in \Gamma(V,q)$.
\end{enumerate}
\end{lemma}
\begin{proof}
We calculate
\[ V = \alpha[V] = \alpha[\Gamma(V,q)] = \alpha(\alpha(x)) V \alpha(x)^{-1} = \Ad_{\alpha(x)}[V] \]
and
\[ V = (\alpha[V])^t = \alpha(x^t) V (x^t)^{-1} = \Ad_{x^t}[V] \]
and the third follows by multiplicative closure.
\end{proof}
A consequence of this lemma is that $N(x)\in \Gamma(V,q)$ for all $x\in \Gamma(V,q)$. But we will show that something stronger holds, namely $N(x)\in\F^\times$.

\begin{definition}
The \udef{Pin group} of $(V,q)$ is the subgroup $\Pin(V,q)$ of $P(V,q)$ generated by the elements $v\in V$ with $q(v) = \pm 1$.

The \udef{Spin group} of $(V,q)$ is defined by
\[ \Spin(V,q) = \Pin(V,q) \cap \Cl^0(V,q) \]
where $\Cl^0(V,q)$ is the even subalgebra of $\Cl(V,q)$.
\end{definition}

\subsection{The twisted adjoint representation}
Consider the map $\Ad_v$ acting on the $q$-orthogonal decomposition (TODO ref)
\[ V = \Span\{v\}\oplus\Span\{v\}^\perp \qquad \text{for some $v\in P(V,q)$,} \]
where $\Span\{v\}^\perp = \setbuilder{w\in V}{q(v,w)=0}$. Then, by the formula
\[ \Ad_v(w) = \frac{q(v,w)}{q(v)}v - w ,\]
we see that elements of $\Span\{v\}$ are mapped to themselves:
\begin{align*}
\Ad_v(\lambda v) &= \frac{q(v,\lambda v)}{q(v)}v - \lambda v = \lambda\frac{q(2v)-2q(v)}{q(v)}v - \lambda v \\
&= 2\lambda v -\lambda v = \lambda v.
\end{align*}
and that elements $w\in\Span\{v\}^\perp$ are mapped to $-w$.

This means that $\Ad_v$ is orientation-preserving if $\dim(V)$ is odd and orientation-reversing otherwise.

We would prefer the action of $\Ad_v$ to do the opposite: fix the hyperplane $\Span\{v\}^\perp$ and invert $\Span\{v\}$. To that end we introduce the twisted adjoint representation.
\begin{definition}
The \udef{twisted adjoint representation} $\widetilde{\Ad}: \Cl^\times(V,q) \to \GL(\Cl(V,q))$ is defined by
\[ \widetilde{\Ad}_x(y) = \alpha(x)yx^{-1} \qquad \forall x\in \Cl^\times(V,q), \forall y\in \Cl(V,q) \]
where $\alpha$ is the grade involution.
\end{definition}
\begin{lemma}
Let $x,y\in \Cl^\times(V,q)$ and $v,w\in V$. Then
\begin{enumerate}
\item $\widetilde{\Ad}_{xy} = \widetilde{\Ad}_x\circ \widetilde{\Ad}_y$;
\item $\widetilde{\Ad}_x = \Ad_x$ if $x\in \Cl^0(V,q)$;
\item $\widetilde{\Ad}_v(w) = w-\frac{q(v,w)}{q(v)}v$.
\end{enumerate}
\end{lemma}

We have
\[ \Gamma(V,q) = \setbuilder{x\in\Cl^\times(V,q)}{\widetilde{\Ad}_x[V] = V}. \]

\begin{proposition}
Let $V$ be a finite-dimensional vector space over a field $\mathbb{F}$ and $q$ non-degenerate. Then the kernel of the homomorphism
\[ \widetilde{\Ad}: \Gamma(V,q) \to \GL(V) \]
is exactly the group $\mathbb{F}^\times$.
\end{proposition}
\begin{proof}
Choose an orthogonal basis $v_1,\ldots, v_n$ for $V$ w.r.t. the bilinear form $q(-,-)$ (TODO ref; also proof here only finite-dim: can it generalise?). Suppose $x\in \Cl^\times(V,q)$ is in the kernel of $\widetilde{\Ad}$, then
\[ \alpha(x)v = vx \qquad \text{for all $v\in V$.} \]
Now we can write $x = x_0 + x_1$ where $x_0$ is even and $x_1$ is odd and both are polynomial expressions in $v_1,\ldots, v_n$. Making use of $v_iv_j = \pm v_jv_i$, we can write $x_0 = a_0 + v_1a_1$ where $a_0,a_1$ are polynomial expressions in $v_2,\ldots, v_n$. Then $a_0$ is even and $a_1$ is odd, so
\[ v_1a_0 + v_1^2a_1 = v_1(a_0+v_1a_1) = (a_0+v_1a_1)v_1 = a_0v_1 + v_1 a_1 v_1 = v_1a_0-v_1^2 a_1. \]
Thus $v^2_1a_1 = -q(v_1)a_1 = 0$, so $a_1=0$ and $x_0$ does not involve $v_1$. By induction $x_0$ does not involve any of $v_1,\ldots, v_n$ and thus $x_0 = \lambda \vec{1}$ for $\lambda\in\mathbb{F}$.

A similar argument shows that $x_1$ is independent of $v_1,\ldots, v_n$ and thus $x_1=0$. Here it is important that $v_1x_1 = -x_1v_1$, because in $x_1 = a_0 + v_1a_1$, $a_0$ is now odd and $a_1$ even.

Thus $x = x_0+x_1 = \lambda \vec{1}$ and $x\neq 0$, so $x\in \mathbb{F}^\times$.
\end{proof}
This proof only works for the \textit{twisted} adjoint representation, not the adjoint representation.

It is also clearly important that $q$ be non-degenerate.

\begin{corollary} \label{normHomomorphism}
The restriction of the norm $N$ to $\Gamma(V,q)$ gives a homomorphism
\[ N: \Gamma(V,q) \to \mathbb{F}^\times. \]
\end{corollary}
\begin{proof}
If we can show that $N[\Gamma(V,q)]\subset \mathbb{F}^\times$, then the multiplicativity of $N$ follows from
\[ N(xy) = xy\alpha((xy)^t) = xy\alpha(y^t)\alpha(x^t) = xN(y)\alpha(x^t) = x\alpha(x^t)N(y) = N(x)N(y). \]

Thus by the proposition it is enough to show that for all $x\in \Gamma(V,q)$, $N(x)\in \ker(\widetilde{\Ad})$. Because $\alpha(x)vx^{-1}\in V$, the transpose leaves it unchanged:
\[ \alpha(x)vx^{-1} = (\alpha(x)vx^{-1})^t = (x^t)^{-1}v\alpha(x^t). \]
This can be rewritten as
\[ v = x^t\alpha(x)vx^{-1}(\alpha(x^t))^{-1} = \alpha(\alpha(x^t)x)v(\alpha(x^t)x)^{-1} = \widetilde{\Ad}_{N(x)}(v). \]
Thus $N(x)\in \ker(\widetilde{\Ad})$.
\end{proof}
\begin{corollary}
Let $x\in \Gamma(V,q)$, then $\widetilde{\Ad}_x \in \Ogroup(V,q)$. Thus there is a group homomorphism
\[ \widetilde{\Ad}: \Gamma(V,q) \to \Ogroup(V,q). \]
\end{corollary}
\begin{proof}
First assume $v\in V^\times = \setbuilder{v\in V}{q(v)\neq 0}\subset \Gamma(V,q)$. Then by \ref{normIsQuadraticForm} and the previous corollary
\[ q(\widetilde{\Ad}_x(v)) = N(\widetilde{\Ad}_x(v)) = N(\alpha(x)vx^{-1}) = N(\alpha(x))N(v)N(x^{-1}) = N(v)N(x)N(x^{-1}) = N(v) = q(v). \]

Now assume $q(v) = 0$. If $q(\widetilde{\Ad}_x(v))$ were not zero, then $\widetilde{\Ad}_x(v)\in V^\times$ and thus
\[ q(v) = q(\widetilde{\Ad}_{x^{-1}}\circ\widetilde{\Ad}_x(v)) = q(\widetilde{\Ad}_x(v)) \neq 0 \]
which is a contradiction.
\end{proof}

\subsection{Double coverings}
We define $SP(V,q) \defeq P(V,q)\cap \Cl^0(V,q)$.
\begin{theorem}
The homomorphisms
\[ \widetilde{\Ad}: P(V,q)\to \Ogroup(V,q) \qquad \text{and} \qquad \widetilde{\Ad}: SP(V,q)\to \SO(V,q) \]
are surjective. So we have the short exact sequence
\[ \begin{tikzcd}
1 \rar & \F^\times \rar & \Gamma(V,q) \rar{\widetilde{\Ad}} & \Ogroup(V,q) \rar & 1.
\end{tikzcd} \]
\end{theorem}

\begin{proposition}
The images $\widetilde{\Ad}(\Pin(V,q))$ and $\widetilde{\Ad}(\Spin(V,q))$ are both normal subgroups of $\Ogroup(V,q)$.
\end{proposition}
\begin{proof}
By a simple calculation we have $\widetilde{\Ad}_{f(v)} = f\circ\widetilde{\Ad}_v\circ f^{-1}$ for all $v \in V$ and $f\in\Ogroup(V,q)$.
\end{proof}

\begin{definition}
A field $\mathbb{F}$ of characteristic $\neq 2$ is called \udef{spin} if for all $a\in \mathbb{F}^\times$ at least one of the equations $t^2 = a$ and $t^2 = -a$ has a solution $t$ in $\mathbb{F}$.
\end{definition}
Thus $\mathbb{F}$ is spin if
\[ \mathbb{F}^\times = (\mathbb{F}^\times)^2 \cup (-(\mathbb{F}^\times)^2). \]

\begin{lemma}
The following fields are spin:
\begin{enumerate}
\item $\R$;
\item $\C$;
\item $\mathbb{F}_p$ with $p$ prime and $p \equiv 3\mod 4$.
\end{enumerate}
\end{lemma}

\begin{theorem}
Let $V$ be a finite-dimensional vector space over a spin field
$\mathbb{F}$, and suppose $q$ is a non-degenerate quadratic form on $V$. Then there are
short exact sequences
\[ \begin{tikzcd}
1 \rar & F \rar & \Spin(V,q) \rar{\widetilde{\Ad}} & \SO(V,q) \rar & 1
\end{tikzcd} \]
\[ \begin{tikzcd}
1 \rar & F \rar & \Pin(V,q) \rar{\widetilde{\Ad}} & \Ogroup(V,q) \rar & 1
\end{tikzcd} \]
where
\[ F = \begin{cases}
\Z_2 = \{1,-1\} & \sqrt{-1}\notin\mathbb{F} \\
\Z_4 = \{\pm 1, \pm\sqrt{-1}\}	& \text{otherwise.}
\end{cases} \]
This result holds for general fields if $\SO(V,q)$ and $\Ogroup(V,q)$ are replaced by appropriate normal subgroups of $\Ogroup(V,q)$.
\end{theorem}
\begin{proof}
Suppose $x = v_1\ldots v_r\in\Pin(V,q)$ is in the kernel of $\widetilde{\Ad}$. Then $x\in \mathbb{F}^\times$ and so
\[ x^2 = N(x) = N(v_1)\ldots N(v_r) = \pm 1 \]
by \ref{normHomomorphism}.
\end{proof}

\begin{proposition}
Let $\mathbb{F}$ be a spin field. Then either
\[ \Gamma(V,q)=P(v,q) \qquad \text{or} \qquad \Gamma(V,q)/P(V,q) \cong \Z_2. \]
\end{proposition}
\begin{proof}
TODO

We have a group homomorphism $\widetilde{\Ad}: \Gamma(V,q)\to \Ogroup(V,q)$ with $\ker(\widetilde{\Ad}) = \F^\times$. 
\end{proof}

\section{Real and complex Clifford algebras}

\begin{definition}
$q$-orthonormal basis.
\end{definition}

\begin{proposition}
There is an algebra isomorphism
\[ \Cl_{r,s} \cong \Cl^0_{r,s+1} \qquad \forall r,s\in\N. \]
In particular $\Cl_n \cong \Cl^0_{n+1}$.
\end{proposition}
\begin{proof}
Take a $q$-orthonormal basis $\{e_i\}_{i=1}^{r+s+1}$ of $\R^{r+s+1}$ and let $\R^{r+s}$ be spanned by the basis $\{e_i\}_{i=1}^{r+s}$. Then define a linear map $f:\R^{r+s}\to \Cl^0_{r,s+1}$ by
\[ f(e_i) = e_{r+s+1}e_i \qquad (i=1,\ldots,r+s). \]
We hope to apply the universal property \ref{CliffordUniversalProperty} to extend it to a map $\widetilde{f}: \Cl_{r,s}\to \Cl^0_{r,s+1}$. So we check $f(v)^2 = q(v) \vec{1}$. Indeed, let $v = \sum_{i=1}^{r+s}v_ie_i$, then
\[ f(v)^2 = \sum_{i,j=1}^{r+s}v_iv_je_{r+1}e_ie_{r+1}e_j = -\sum_{i,j=1}^{r+s}v_iv_je_{r+1}e_{r+1}e_ie_j = \sum_{i,j=1}^{r+s}v_iv_je_ie_j = q(v) \vec{1}. \]

It is easy to see $\widetilde{f}$ is bijective.
\end{proof}


\subsection{Geometric algebra}
\begin{definition}
A \udef{geometric algebra} is a Clifford algebra $\Cl(\R^n, \norm{\cdot}^2)$, where $\norm{\cdot}$ is the standard norm. We denote the $n$-dimensional geometric algebra by $\G^n$.
\end{definition}

In particular we have $u\cdot v = \inner{u,v}$.

\subsubsection{The pseudoscalar}
TODO orientations and stuff

\subsubsection{Rotors and rotations}
\begin{definition}
Let $I$ be a unit pseudoscalar, $\vec{n}$ some unit vector and $\theta \in \R$. 
\end{definition}

\section{Representations}

\begin{definition}
Let $K \subseteq k$ be fields, $V$ a vector space over $k$ and $q$ a quadratic form on $V$. Then a \udef{$K$-representation} of the Clifford algebra $\Cl(V,q)$ is a $k$-algebra homomorphism
\[ \rho: \Cl(V,q) \to \Hom_K(W,W) \]
where $W$ is a finite dimensional vector space over $K$. The space $W$ is then a \udef{$\Cl(V,q)$-module} over $K$.
\end{definition}

Usually we will take the field $K$ to be $\R,\C,\mathbb{H}$.

\begin{lemma}
\begin{enumerate}
\item A complex representation of $\Cl_{r,s}$ automatically extends to a representation of
\[ \Cl_{r,s}\otimes_\R \C \cong \cCl_{r+s}. \]
\item A quaternionic representation of $\Cl_{r,s}$ is automatically complex.
\end{enumerate}
\end{lemma}

\section{Lie algebra structures}

\begin{proposition}
The Lie subalgebra of $(\Cl_n, [\cdot,\cdot])$ corresponding to the subgroup $\Spin_n\subset \Cl_n^\times$ is
\[ \mathfrak{spin}_n = {\textstyle \bigwedge^2}\R^n. \]
In particular, $\bigwedge^2\R^n$ is closed under the bracket operation.
\end{proposition}
\begin{proof}
We are looking for tangent vectors to the submanifold $\Spin_n$ at $\vec{1}$. Fix an orthonormal basis $e_1,\ldots, e_n$ of $\R^n$ and consider the curve
\[ \gamma(t) = (e_i\cos t+ e_j\sin t)(-e_i\cos t+ e_j\sin t) = (\cos^2 t - \sin^2 t)+2e_ie_j\sin t\cos t = \cos(2t)\sin(2t)e_ie_j. \]
This curve lies in $\Spin_n$, satisfies $\gamma(0) = \vec{1}$ and its tangent vector at $\gamma(0)$ is $2e_ie_j$. Hence $\mathfrak{spin}_n$ contains $\Span_\R\{e_ie_j\} = \bigwedge^2\R^n$. Since $\dim_\R(\mathfrak{spin}_n)$
\end{proof}






\section{Geometry and geometric algebra}
\url{http://www.faculty.luther.edu/~macdonal/GAConstruct.pdf}
\subsection{Definitions}

\begin{definition}
A \udef{geometric algebra} $\mathfrak{G}$ is a real unital associative algebra of the form
\[ \mathfrak{G} = \bigoplus_{r\in \N}\mathfrak{G}_r \]
such that
\begin{align*}
\mathfrak{G}_0 &= \Span\{\vec{1}\} \\
\mathfrak{G}_r &= \Span\setbuilder{\vec{a}_1\vec{a}_2\ldots\vec{a}_r}{\vec{a}_1,\ldots,\vec{a}_r \in \mathfrak{G}_1, \; \forall i,j\leq r: \vec{a}_i \vec{a}_j = -\vec{a}_j \vec{a}_i} & \text{for all $r>1$}.
\end{align*}
We also assume that the multiplication satisfies
\[ \forall \vec{a} \in \mathfrak{G}_1: \quad \vec{a}^2 = \vec{a}\vec{a} = \lambda \vec{1} \in \mathfrak{G}_0 \qquad \text{for some $\lambda \in \R^{> 0}$} \]
and that for each element $a$ of $\mathfrak{G}_r\setminus\{0\}$ there exists a vector $\vec{a}\in\mathfrak{G}_1$ such that $\vec{a}a \in \mathfrak{G}_{r+1}\setminus\{0\}$.

We then call
\begin{itemize}
\item $\sqrt{\lambda}$ the \udef{magnitude of $\vec{a}$}, denoted $|\vec{a}|$;
\item the projection $\mathfrak{G} \to \mathfrak{G}_r$ the \udef{grade operator}, denoted $\grade{\cdot}_r$;
\item the multiplication of $\mathfrak{G}$ the \udef{geometric product} on $\mathfrak{G}$;
\item elements of $\mathfrak{G}$ \udef{multivectors};
\item elements of $\mathfrak{G}_r$ \udef{$r$-vectors} or \udef{homogenous multivectors}; in particular $0$-vectors are called \udef{scalars}, $1$-vectors \udef{vectors}, $2$-vectors \udef{bivectors} \ldots
\item $r$-vectors of the form $\vec{a}_1\vec{a}_2\ldots\vec{a}_r$ where $\vec{a}_1,\ldots,\vec{a}_r \in \mathfrak{G}_1$ anti-commute are called \udef{simple $r$-vectors} or \udef{$r$-blades}.
\end{itemize}
We use lowercase letters $a,b,c \ldots$ to denote multivectors, Greek letters $\mu, \nu, \lambda \ldots$ for scalars and bold letters $\vec{u}, \vec{v}, \vec{w} \ldots$ for vectors. Often we will use subscripts to denote the grade of a multivector, e.g\ $a_r$ is an $r$-vector. Capital letters with subscript, e.g\ $A_r$, will be used to denote $r$-blades.
\end{definition}

So for any $a\in \mathfrak{G}$, we can write
\[ a = \grade{a}_0 + \grade{a}_1 + \ldots + \grade{a}_n  = \sum_{r=0}^n \grade{a}_i  \] for some $n\in\N$.

We make the convention that negative grades are always zero.

Because of the assumption that no $\mathfrak{G}_r$ is trivial, we need $\mathfrak{G}_1$ to be infinite-dimensional.


\begin{definition}
Let $\mathfrak{G}$ be a geometric algebra. We define the \udef{reverse} operation $\dagger$ on $\mathfrak{G}$ as the unique linear operation such that
\begin{align*}
\vec{1}^\dagger &= \vec{1} \\
\vec{u}^\dagger &= \vec{u} \qquad \vec{u}\in\mathfrak{G}_1 \\
(\vec{v}_1 \vec{v}_2 \ldots \vec{v}_r)^\dagger &= \vec{v}_r \ldots \vec{v}_2 \vec{v}_1 \qquad \vec{v}_1, \ldots, \vec{v}_r \in \mathfrak{G}_1.
\end{align*}
\end{definition}
Note that we have specified $\dagger$ on all basis elements of $\mathfrak{G}$, so it is well-defined and uniquely determined, cfr. \ref{linearMaps}.

\begin{lemma}
Let $a,b \in \mathfrak{G}$. Then
\begin{enumerate}
\item $(a^\dagger)^\dagger = a$;
\item $(ab)^\dagger = b^\dagger a^\dagger$;
\item $\grade{a^\dagger}_r = \grade{a}_r^\dagger = (-1)^{r(r-1)/2}\grade{a}_r$;
\item $\grade{a}_r = (-1)^{r(r-1)/2}\grade{a}_r^\dagger$;
\item $\grade{a_rb_s}_t = (-1)^{\frac{1}{2}(r(r-1) + s(s-1) + t(t-1))}\grade{b_sa_r}_t$.
\end{enumerate}
\end{lemma}
Notice we are only interested in the exponent of $(-1)$ modulo $2$.

\begin{definition}
Let $\mathfrak{G}$ be a geometric algebra. We define the \udef{inner product} $\cdot$ on homogeneous multivectors by
\[ a_r\cdot b_s = \begin{cases}
0 & \text{$r=0$ or $s=0$} \\
\grade{a_rb_s}_{|r-s|} & \text{else}.
\end{cases}  \]
The inner product is bilinear on homogeneous multivectors and can thus be extended linearly to arbitrary multivectors.
\end{definition}
So for arbitrary multivectors we have
\[ a\cdot b = \sum_r\sum_s \grade{a}_r\cdot \grade{b}_s. \]

\begin{definition}
Let $\mathfrak{G}$ be a geometric algebra. We define the \udef{outer product} $\wedge$ on homogeneous multivectors by
\[ a_r\wedge b_s = \grade{a_rb_s}_{r+s} \]
The outer product is bilinear on homogeneous multivectors and can thus be extended linearly to arbitrary multivectors.
\end{definition}
So for arbitrary multivectors we have
\[ a\wedge b = \sum_r\sum_s \grade{a}_r\wedge \grade{b}_s. \]
For scalars $\lambda \in \mathfrak{G}^0$, we have
\[ a \wedge \lambda = \lambda\wedge a = \lambda a \qquad a \in \mathfrak{G}. \]
We have explicitly excluded this in the inner product.

\begin{lemma}
If $a = \vec{v}_1 \vec{v}_2\ldots \vec{v}_r$, then $a\in \bigoplus_{i\leq r}\mathfrak{G}_i$.

If $a$ is an $r$-blade and $\beta$ an orthogonal basis for $\mathfrak{G}_1$, then there exist $\vec{e}_1,\ldots, \vec{e}_r \in \beta$ such that
\[ a = \lambda \vec{e}_1 \ldots \vec{e}_r \]


 be an $r$-blade, then
\[\vec{v}_1 \vec{v}_2\ldots \vec{v}_r = \vec{v}_1 \wedge \vec{v}_2 \wedge\ldots\wedge \vec{v}_r.\]
\end{lemma}

\begin{note}
We introduce an order of operations (from highest priority to lowest):
\begin{enumerate}
\item outer product;
\item inner product;
\item geometric product.
\end{enumerate}
\end{note}

\begin{lemma}
Let $a_r,b_s$ be homogeneous vectors in $\mathfrak{G}$. Then
\begin{enumerate}
\item $a_r\cdot b_s = (-1)^{s(r-1)}b_s\cdot a_r$ for $r\geq s$;
\item $a_r \wedge b_s = (-1)^{rs}b_s \wedge a_r$.
\end{enumerate}
\end{lemma}
\begin{proof}
(1) We calculate
\[ a_r\cdot b_s = \grade{a_rb_s}_{|r-s|} = (-1)^{\frac{1}{2}(r(r-1) + s(s-1) + |r-s|(|r-s|-1))}\grade{a_rb_s}_{|r-s|}. \]
We can simplify the exponent, assuming $r\geq s$, to
\[ r^2 + s^2 -r -sr \equiv r+s+r+sr \equiv s+sr \equiv sr-s \mod 2.\]
(2) We calculate
\[ a_r \wedge b_s = \grade{a_rb_s}_{r+s} = (-1)^{\frac{1}{2}(r(r-1) + s(s-1) + (r+s)((r+s)-1))}\grade{a_rb_s}_{r+s}. \]
We can simplify the exponent to
\[ r^2 -r+s^2 -s + rs \equiv r-r+s-s+rs \equiv rs \mod 2. \]
\end{proof}

By the fact that the definitions of inner and outer product make sense it is obvious that the geometric product does not preserve grade, or even homogeneity. It does, however, preserve a $\Z_2$-grading: we can split
\[ \mathfrak{G} = \mathfrak{G}_\text{even} \oplus \mathfrak{G}_\text{odd} \qquad \text{where}\quad \begin{cases}
\mathfrak{G}_\text{even} \defeq \bigoplus_{r\in \N}\mathfrak{G}_{2r} \\
\mathfrak{G}_\text{odd}\; \defeq \bigoplus_{r\in \N}\mathfrak{G}_{2r+1}.
\end{cases} \]
We have the linear projection operators $\grade{\cdot}_+:\mathfrak{G}\to \mathfrak{G}_\text{even}$ and $\grade{\cdot}_-:\mathfrak{G}\to \mathfrak{G}_\text{odd}$. We also write $\mathfrak{G}_+$ instead of $\mathfrak{G}_\text{even}$ and $\mathfrak{G}_-$ instead of $\mathfrak{G}_\text{odd}$.
\begin{proposition}
Let $\mathfrak{G} = \mathfrak{G}_\text{+} \oplus \mathfrak{G}_\text{-}$ be a geometric algebra and let $p,q\in\{+,-\}\cong \Z_2$. Then
\[ \mathfrak{G}_p\mathfrak{G}_q \subset \mathfrak{G}_{pq}. \]
\end{proposition}
\begin{proof}
The grading operators $\grade{\cdot}_\pm$ are linear maps and thus determined by their action on basis elements. Thus it is enough to show that
\[ \grade{A_rB_s}_+ =  \begin{cases}
A_rB_s & (r+s \equiv 0 \mod 2) \\
0 & (r+s \equiv 1 \mod 2)
\end{cases} \qquad \grade{A_rB_s}_- =  \begin{cases}
0 & (r+s \equiv 0 \mod 2) \\
A_rB_s & (r+s \equiv 1 \mod 2)
\end{cases} \]
for any $r$-blade $A_r$ and $s$-blade $B_s$. In fact by associativity of the geometric product, it is enough to show
\[ \vec{v}A_r \]
\end{proof}

\begin{lemma}
Let $\vec{u}, \vec{v} \in \mathfrak{G}_1$. Then
\[ \vec{u}\cdot \vec{v} = \grade{\vec{u}\vec{v}}_0 = \frac{1}{2}(\vec{u}\vec{v} + \vec{v}\vec{u}). \]
\end{lemma}
\begin{proof}
We start from $(\vec{u}+\vec{v})^2 = \vec{u}^2 + \vec{u}\vec{v} + \vec{v}\vec{u} + \vec{v}^2$ and rearrange to get
\[ \vec{u}\vec{v} + \vec{v}\vec{u} = (\vec{u}+\vec{v})^2 - \vec{u}^2 - \vec{v}^2 = |\vec{u}+\vec{v}|^2 - |\vec{u}|^2 - |\vec{v}|^2  \]
which is scalar. Since $\grade{\vec{u}\vec{v}}_0 = \grade{\vec{v}\vec{u}}_0$, we have
\[ \frac{1}{2}(\vec{u}\vec{v} + \vec{v}\vec{u}) = \frac{1}{2}\grade{\vec{u}\vec{v} + \vec{v}\vec{u}}_0 = \grade{\vec{u}\vec{v}}_0 = \vec{u}\cdot \vec{v}. \]
\end{proof}
\begin{corollary}
The geometric inner product restricted to $\mathfrak{G}_1$ is bilinear, symmetric and positive definite. It is thus an inner product as previously defined.
\end{corollary}
The associated definitions are thus also applicable here. In particular two vectors $\vec{u},\vec{v}$ are called \udef{orthogonal} if $\vec{u}\cdot \vec{v} = 0$. By the lemma this is the case when $\vec{u}\vec{v} = - \vec{v}\vec{u}$. This means that the $r$ vectors making up $r$-blades are linearly independent, \ref{orthogonalLinearlyIndependent}.




\begin{lemma}
For any algebra satisfying the other axioms, the direct sum $\mathfrak{G} = \bigoplus_{r\in\N}\mathfrak{G}_r$ is well-defined.
\end{lemma}
\begin{proof}
We need to show that for all $r>s\in \N$, we have $\mathfrak{G}_r\cap\mathfrak{G}_s = \{0\}$. Assume, towards a contradiction, that here exist $a_r,b_s$ such that $a_r = b_s$. Then both $a_r$ and $b_s$ can be written as sums of blades. Now let $D$ be the set of all vectors featured in a blade in this sum. By Gram-Schmidt, we can find an orthogonal basis for $D$ and rewrite $a_r$ and $b_s$ in this basis. As $r>s$, we can find elements of this orthogonal basis to multiply $a_r$ with such that it becomes zero, but $b_s$ remains non-zero (unless it already was zero). TODO: improve proof.
\end{proof}


\begin{lemma}
Let $\vec{u}, \vec{v}_1,\ldots, \vec{v}_n$ be vectors in $\mathfrak{G}_1$. Then
\[ \vec{u}\cdot (\vec{v}_1 \vec{v}_2 \ldots \vec{v}_n) = \sum_{i=1}^n (-1)^{k+1}(\vec{u}\cdot \vec{v}_i)\vec{v}_1\ldots\breve{\vec{v}}_i\ldots \vec{v}_n, \]
where the breve indicates the vector under it is omitted from the product.
\end{lemma}
\begin{proof}

\end{proof}

\begin{lemma}
\[ \vec{u}a_r = \vec{u}\cdot a_r + \vec{u}\wedge a_r = \grade{\vec{u}a_r}_{r-1} + \grade{\vec{u}a_r}_{r+1}. \]
\end{lemma}

\begin{proposition}
Let $\vec{v}\in\mathfrak{G}_1$ and $a_r\in\mathfrak{G}_r$. Then

\end{proposition}





\begin{lemma}
The outer product is associative:
\[ a\wedge(b\wedge c) = (a\wedge b)\wedge c \]
The inner product is not associative, but homogeneous multivectors obey
\begin{align*}
a_r\cdot(b_s \cdot c_t) &= (a_r\wedge b_s)\cdot c_t & &\text{for $r+s\leq t$ and $r,s>0$} \\
a_r\cdot(b_s \cdot c_t) &= (a_r\cdot b_s)\cdot c_t & &\text{for $r+t\leq s$}
\end{align*}
\end{lemma}

\begin{lemma}
\[ \vec{u}\wedge a\wedge \vec{v}\wedge b = -\vec{v}\wedge a\wedge \vec{u}\wedge b  \]
\end{lemma}
$\vec{v}\wedge a \wedge \vec{v} \wedge b = 0$

\subsection{Affine spaces}
\subsection{Projections on 1D spaces}
\[ \sin(\theta) = \norm{a_\perp}/ \norm{a} \qquad \cos(\theta) = \norm{a_\parallel}/\norm{a}. \]
\subsection{The geometric product}

\subsection{Hodge duality}
\subsection{Cross product and triple product}
Cross product not associative

Triple product nice way to find normal vectors with specific orientation.

\chapter{Coordinates and matrices}

\url{file:///C:/Users/user/Downloads/2013%20Matrix%20Computations%204th(1).pdf}
\url{file:///C:/Users/user/Downloads/(Cambridge%20mathematical%20textbooks)%20Garcia,%20Stephan%20Ramon_%20Horn,%20Roger%20A.%20-%20A%20Second%20Course%20in%20Linear%20Algebra-Cambridge%20University%20Press%20(2017)(1).pdf}

TODO Haynsworth inertia additivity formula

\section{Coordinates}
In this chapter we will purely be interested in finite-dimensional spaces. From proposition \ref{isomorphicDimension} we know that for any $n$-dimensional vector space $V$ over a field $\mathbb{F}$, $V\cong \mathbb{F}^n$. If we choose a basis $\beta$, we can explicitly give an isomorphism.
\begin{definition}
Let $V$ be an $n$-dimensional vector space with basis $\beta = \{e_1,\ldots, e_n\}$. Because $\beta$ is a basis, we can uniquely write every $v\in V$ as $a_1e_1+\ldots +a_ne_n$. Then we define the \udef{coordinate map w.r.t. $\beta$} as
\[ \co_\beta: V \to \mathbb{F}^n: v \mapsto \co_\beta(v) = \co_\beta(a_1e_1+\ldots +a_ne_n) = \begin{pmatrix}
a_1 \\ \vdots \\ a_n
\end{pmatrix}. \]
The vector $\begin{pmatrix}
a_1 \\ \vdots \\ a_n
\end{pmatrix}$ is called a \udef{coordinate vector}.

The vector $\co_\beta(v)$ is also denoted $[v]_\beta$.
\end{definition}
This coordinate map is indeed an isomorphism.

We conventionally write coordinate vectors as column vectors in $\F^n$. We will represent such vectors in bold type: $\vec{v}\in\F^n$.

\begin{lemma}
Let $\mathcal{E}$ be the standard basis of $\F^n$. Then
\[ \co_\mathcal{E} = \id = \co_\mathcal{E}^{-1}. \]
\end{lemma}

\section{Matrices}
\begin{definition}
A matrix is a rectangular grid of numbers. If it has $m$ rows and $n$ columns, we call it a \udef{$(m\times n)$-matrix}.
If the numbers are elements of the field $\mathbb{F}$, we denote the set of $(m\times n)$-matrices as $\mathbb{F}^{m\times n}$.

If $m=n$,we call the matrix a \udef{square matrix}.
\end{definition}
\begin{example}
An example of a ($2\times 4$)-matrix:
\[ \begin{bmatrix}
1 &2 &3\\4 &5&6
\end{bmatrix} = \begin{pmatrix}
1 &2 &3\\4 &5&6
\end{pmatrix} \]
Sometimes square brackets are used, sometimes parentheses. It's just a matter of style.
\end{example}
Matrices are usually denoted using capital letters.

\begin{definition}
Let $n,m\in\N_0$.
\begin{itemize}
\item The \udef{zero matrix} of dimension $n\times m$ is the $(n\times m)-$matrix
\[ \mathbb{0}^{n\times m} = \begin{pmatrix}
0 & \hdots & 0 \\
\vdots & \ddots & \\
0 & \hdots & 0
\end{pmatrix}. \]
\item The \udef{matrix of ones} or \udef{all-ones matrix} of dimension $n\times m$ is the $(n\times m)-$matrix
\[ \mathbb{J}^{n\times m} = \begin{pmatrix}
1 & \hdots & 1 \\
\vdots & \ddots & \\
1 & \hdots & 1
\end{pmatrix}. \]
\item The \udef{identity matrix} of dimension $n$ is the $(n\times n)$-matrix
\[ \mathbb{1}_n = \begin{pmatrix}
1 & 0 & 0 & \hdots & 0\\
0 & 1 & 0 & \hdots & 0\\
0 & 0 & 1 & \hdots & 0\\
\vdots & \vdots & \vdots & \ddots & \vdots \\
0 & 0 & 0 & \hdots & 1
\end{pmatrix} \]
The components of $\mathbb{1}_n$ are given by
\[ [\mathbb{1}_n]_{ij} = \delta_{ij}. \]
\end{itemize}
If $m=n$, we abbreviate these matrices as $\mathbb{0}_n$ and $\mathbb{J}_n$.
\end{definition}

\begin{lemma}
The $(m\times n)$-matrices in $\F^{m\times n}$ naturally form a vector space with point-wise addition and scalar multiplication. 
\end{lemma}

\begin{definition}
We call matrices $A,B$ \udef{conformal} for a certain operation if the operation is defined on these matrices. So $A,B$ are conformal for addition if they have the same dimensions. 
\end{definition}

\subsection{Components of matrices}
The element on the $i^\text{th}$ row and $j^\text{th}$ column of a matrix $A$ is denoted $[A]_{i,j}$ or $a_{i,j}$. These numbers are known as the \udef{components} of the matrix.

Vectors in $\F^n$ can be seen as matrices by writing them as column vectors. In this way we identify $\F^n$ with $\F^{n\times 1}$.

Let $\vec{v}\in \F^n$. The components of $\vec{v}$ are of the form $[\vec{v}]_{i,1}$. We abbreviate this to $[\vec{v}]_i$.

\begin{lemma}
The functions
\[ [-]_{ij}: A\mapsto [A]_{ij} \qquad \text{and} \qquad [-]_i: \vec{v}\mapsto [\vec{v}]_i  \]
are linear.
\end{lemma}

Conversely, consider a set of numbers $a_{i,j}$ where $i\in (1:m), j\in (1:n)$. Then by $[a_{i,j}]$ we mean the matrix consisting of those numbers.

We can also consider components after applying a function. For some linear map $f$, we often write
\[ f_i \defeq [-]_i\circ f. \]

\begin{definition}
Let $A$ be an $(m\times n)$-matrix.
\begin{itemize}
\item If $1\leq j\leq m$, then $[A]_{j,-}$ denotes the $(1\times n)$-matrix consisting of row $j$ of $A$.
\item If $1\leq k\leq n$, then $[A]_{-,k}$ denotes the $(m\times 1)$-matrix consisting of column $k$ of $A$.
\end{itemize}
\end{definition}

\begin{lemma}
Let $A,B\in \F^{m\times n}$ and $\lambda\in \F$, then
\begin{itemize}
\item $[A+B]_{ij} = [A]_{ij} + [B]_{ij}$;
\item $[\lambda A]_{ij} = \lambda[A]_{ij}$.
\end{itemize}
\end{lemma}

\begin{definition}
Let $A\in\F^{m\times n}$ be a matrix. A component $[A]_{i,j}$ is
\begin{itemize}
\item on the \udef{diagonal} if $i=j$;
\item \udef{off-diagonal} $i\neq j$;
\item on the $k^\text{th}$ \udef{superdiagonal} if $j = i+k$;
\item on the $k^\text{th}$ \udef{subdiagonal} if $j = i-k$.
\end{itemize}
\end{definition}

\subsubsection{Submatrices}
\begin{definition}
Let $A\in\F^{m\times n}$ be a matrix and $I\subseteq 0:m$ and $J\subseteq 0:n$ sets, then $[A]_{I,J}$ is the matrix consisting only of those entries whose row number is in $I$ and whose column number is in $J$. A matrix of this form is called a \udef{submatrix}.

In this context, we abbreviate
\begin{itemize}
\item $I = 0:m$ and $J = 0:n$ by $-$;
\item $I=\{i\}$ by $i$.
\end{itemize}
\end{definition}

\begin{example}
Let
\[ A = \begin{pmatrix}
1 & 2 & 3 & 4 \\ 5 & 6 & 7 & 8 \\ 9 & 10 & 11 & 12
\end{pmatrix}, \]
then
\[ [A]_{0:2,0:3} = \begin{pmatrix}
1 & 2 & 3 \\ 5 & 6 & 7
\end{pmatrix}, \qquad [A]_{1,-} = \begin{pmatrix}
5 & 6 & 7 & 8
\end{pmatrix} \qquad\text{and}\qquad [A]_{-, 2} =  \begin{pmatrix}
3 \\ 7 \\ 11
\end{pmatrix}. \]
\end{example}

\begin{lemma} \label{standardBasisFromIdentityMatrix}
Let $\{\vec{e}_i\}_{i=0}^n$ be the standard basis of $\F^n$. Then
\[ \vec{e}_i = [\mathbb{1}_n]_{-, i}. \]
\end{lemma}

\begin{lemma}
Let $A\in \F^{m\times n}$. Then $A = [A]_{-,-}$.
\end{lemma}

\begin{lemma} \label{componentsSubmatrixFromEnumeration}
Let $A\in\F^{m\times n}$ be a matrix and $I\subseteq 0:m$ and $J\subseteq 0:n$ sets.
Then $\big[[A]_{I,J}\big]_{r,s} = [A]_{\pi_I(r), \pi_J(s)}$.
\end{lemma}

TODO $\pi_I$ is the unique monotonic enumeration of $I$.

\subsubsection{Types of matrices}
\begin{definition}
Let $A\in\F^{n\times n}$. We say
\begin{itemize}
\item $A$ is \udef{upper triangular} if $i>j \implies [A]_{i,j} = 0$;
\item $A$ is \udef{strictly upper triangular} if $i\geq j \implies [A]_{i,j} = 0$;
\item $A$ is \udef{lower triangular} if $i<j \implies [A]_{i,j} = 0$;
\item $A$ is \udef{strictly lower triangular} if $i\leq j \implies [A]_{i,j} = 0$;
\item $A$ is \udef{triangular} if it is upper or lower triangular.
\end{itemize}
We say
\begin{itemize}
\item $A$ is \udef{diagonal} if $i\neq j \implies [A]_{i,j} = 0$; in this case we write $A = \diag([A]_{11},\ldots, [A]_{nn})$;
\item $A$ is \udef{tridiagonal} if $|i\neq j| \geq 2 \implies [A]_{i,j} = 0$;
\item $A$ is \udef{bidiagonal} if it is tridiagonal and triangular.
\end{itemize}
We say
\begin{itemize}
\item $A$ is a \udef{permutation matrix} if exactly one entry in each row and in each column is 1; all other entries are 0.
\end{itemize}
\end{definition}

\subsection{Matrix multiplication}
Assume we have $n$ vectors $v_1,\ldots, v_n$ in $\F^m$. We may be interested in linear combinations of these vectors, say $a_1v_1+\ldots + a_nv_n$. We can collect the coefficients $a_i$ in a column vector in $\F^n$. The vectors $v_i$ can be written as columns and placed in a matrix.

Consider the action that pairs such a matrix of column vectors with the element in its column space determined by a column matrix. This action is called \udef{matrix multiplication} and is denoted by juxtaposing the matrix and the vector (sometimes separated by a dot).

\begin{example}
Let $v_1 = (1,3,4)$ and $v_2 = (2,5,6)$ be vectors in $\R^3$. These can be placed as columns in a matrix:
\[ \begin{pmatrix}
1 & 2 \\ 3 & 5 \\ 4 & 6
\end{pmatrix} \]
Consider the linear combination $2v_1 + v_2$, we can write this as the matrix multiplication
\[ 2v_1 + v_2 = \begin{pmatrix}
1 & 2 \\ 3 & 5 \\ 4 & 6
\end{pmatrix}\begin{pmatrix}
2 \\ 1
\end{pmatrix} = \begin{pmatrix}
2\cdot 1 + 1\cdot 2 \\
2\cdot 3 + 1\cdot 5 \\
2\cdot 4 + 1\cdot 6
\end{pmatrix} = \begin{pmatrix}
4 \\ 11 \\ 14
\end{pmatrix} \]
\end{example}
\begin{example}
A very important case (and one we will explore in more detail later) is given by systems of linear equations. We might have the following equations:
\[ \begin{cases}
2x + y -z = 3 \\
-x + y +3z = 2 \\
x+y = -2
\end{cases} \]
This can be rewritten as
\[ x\begin{pmatrix}
2 \\-1 \\ 1
\end{pmatrix} + y \begin{pmatrix}
1 \\1 \\ 1
\end{pmatrix} + z\begin{pmatrix}
-1 \\ 3 \\ 0
\end{pmatrix} = \begin{pmatrix}
3 \\2 \\ -2
\end{pmatrix} \]
where each row is an equation. In this case the coefficients are the unknowns $x,y,z$. So using the notation of matrix multiplication, the equations become
\[ \begin{pmatrix}
2 & 1 & -1 \\
-1 & 1 & 3 \\
1 & 1 & 0
\end{pmatrix}\begin{pmatrix}
x \\ y \\ z
\end{pmatrix} = \begin{pmatrix}
3 \\ 2 \\ -2
\end{pmatrix}. \]
\end{example}

We can view matrix multiplication as a function
\[ \F^{m\times n}\times \F^n \to \F^m: (A,\vec{v}) \mapsto \begin{pmatrix}
\sum_{i=1}^n [A]_{1,i}[\vec{v}]_i \\ \vdots \\ \sum_{i=1}^n [A]_{m,i}[\vec{v}]_i
\end{pmatrix}. \]


Let $B$ be an $(m\times n)$-matrix and $\vec{v}$ a vector in $\F^n$. Then the matrix multiplication $B\cdot \vec{v}$ gives a vector in $\F^m$. This can be used as the input for another matrix multiplication, if multiplied by a $(k\times m)$ matrix $A$. So the expression $A(B\vec{v})$makes sense.

Now we would like to define matrix multiplication between the matrices $B,A$ by the condition that
\[ (A\cdot B)\vec{v} = A(B\vec{v}). \]
In other words we are asserting the associativity of the matrix multiplication.

Consider the component equations: $[A(B\vec{v})]_i = [(B\cdot A)\vec{v}]_i$. We can then calculate:

\begin{align*}
[A(B\vec{v})]_i &= \sum_{j=1}^m [A]_{i,j}[B\vec{v}]_j = \sum_{j=1}^m [A]_{i,j}(\sum_{k=1}^n [B]_{j,k}[\vec{v}]_k) = \sum_{j=1}^m\sum_{k=1}^n [A]_{i,j}[B]_{j,k}[\vec{v}]_k \\ &= \sum_{k=1}^n\left(\sum_{j=1}^m[A]_{i,j}[B]_{j,k}\right)[\vec{v}]_k \eqdef  [(A\cdot B)\vec{v}]_i
\end{align*}
The last equation can only be satisfied for all $[v]_i$ if the matrix multiplication is defined such that
\[ [A\cdot B]_{i,k} \defeq \sum_{j=1}^m[A]_{i,j}[B]_{j,k}. \]

Of course our construction only works if the dimensions of $A,B$ are such that $A(B\vec{v})$ is well-defined.

\begin{definition}
Let $A,B$ be matrices.
\begin{itemize}
\item The matrices $A,B$ are conformal for multiplication if $A\in \F^{k\times m}$ and $B\in\F^{m\times n}$ for some $k,m,n\in\N_0$.
\item If $A$ and $B$ are conformal, we define the product $AB$ by
\[ [AB]_{i,k} = \sum_{j=1}^m[A]_{i,j}[B]_{j,k}. \]
\end{itemize}
Thus matrix multiplication can be thought of as a map $\F^{k\times m}\times \F^{m\times n} \to \F^{k\times n}$
\end{definition}
The compatibility requirement can be abbreviated by
\[ [k\times m] \cdot [m\times n] = [k\times n]. \]

Notice that the matrix multiplication $\F^{m\times n}\times \F^n \to \F^m$ we originally defined is a special case of this more general matrix multiplication if we identify $\F^n$ with $\F^{n\times 1}$ and $\F^m$ with $\F^{m\times 1}$ (that is, we view $\F^n,\F^m$ as column vectors). In this case we have the multiplication
\[ [m\times n] \cdot [n\times 1] = [m\times 1]. \]

\begin{lemma} \label{linearityMatrixMultiplication}
The matrix multiplication map $\F^{k\times m}\times \F^{m\times n} \to \F^{k\times n}$ is linear in both arguments.
\end{lemma}
\begin{proof}
For linearity in the first argument, assume $A_1,A_2\in\F^{k\times m}$ and $\lambda\in \F$. Then
\[ [(\lambda A_1+ A_2)B]_{ij} = \sum_{j=1}^m[\lambda A_1+ A_2]_{i,j}[B]_{j,k} = \lambda \left(\sum_{j=1}^m[A_1]_{i,j}[B]_{j,k}\right) + \left(\sum_{j=1}^m[A_2]_{i,j}[B]_{j,k}\right). \]
The proof of linearity in second argument is similar.
\end{proof}

TODO: associative class! (In fact category!)

\begin{lemma} \label{submatrixMultiplication}
Let $A\in \F^{k\times m}$, $B\in \F^{m\times n}$, $I\subseteq 0:k$ and $J\subseteq 0:n$. Then
\[ [A]_{I,-}\cdot [B]_{-,J} = [AB]_{I,J}. \]
\end{lemma}
\begin{proof}
Using \ref{componentsSubmatrixFromEnumeration}, we have
\begin{align*}
\big[[A]_{I,-}\cdot [B]_{-,J}\big]_{r,s} &= \sum_{p\in 0:m}\big[[A]_{I,-}\big]_{r, p} \big[[B]_{-,J}\big]_{p,s} \\
&= \sum_{p\in 0:m}[A]_{\pi_I(r), p}[B]_{p, \pi_J(s)} \\
&= [A\cdot B]_{\pi_I(r), \pi_J(s)} \\
&= \big[[A\cdot B]_{I,J}\big]_{r,s}.
\end{align*}
\end{proof}

\subsubsection{Identity}
TODO: the identity matrix is an identity for matrix multiplication!


\begin{lemma}
Let $A\in\F^{m\times n}$. Then
\[ A = \mathbb{1}_m\cdot A = A \cdot \mathbb{1}_n. \]
\end{lemma}
\begin{proof}
By a simple calculation in components:
\[ [\mathbb{1}_m\cdot A]_{ij} = \sum_{k=1}^m\delta_{ik}[A]_{kj} = [A]_{ij}. \]
The other equation is similar.
\end{proof}

\begin{lemma} \label{matrixOfOnesMultiplication}
Let $l,m,n\in\N_0$, then
\[ \mathbb{J}^{l\times m}\mathbb{J}^{m\times n} = m\mathbb{J}^{l\times n}. \]
\end{lemma}
\begin{proof}
\[ [\mathbb{J}^{l\times m}\mathbb{J}^{m\times n}]_{i,j} = \sum_{k=1^m}[\mathbb{J}^{l\times m}]_{i,k}[\mathbb{J}^{m\times n}]_{k,j} = \sum_{k=1^m}1 = m. \]
\end{proof}
\begin{corollary}
Let $a,b,c,d\in\R$, then
\[ (a\mathbb{1}_n + b\mathbb{J}_n)(c\mathbb{1}_n + d\mathbb{J}_n) = ac\mathbb{1}_n + (ad+bc +bdn)\mathbb{J}_n \]
\end{corollary}

\subsubsection{Left and right inverses}
\begin{definition}
Let $A\in \mathbb{F}^{m\times n}$. A matrix $B$ is a \udef{left inverse} of $A$ if $BA = \mathbb{1}_n$. A matrix $B$ is a \udef{right inverse} of $A$ if $AB = \mathbb{1}_m$.
\end{definition}
Not all matrices have a left and/or right inverses.

\subsubsection{Square matrices}
\begin{lemma}
For any $n\in\N_0$, the vector space $\F^{n\times n}$ of square matrices is a monoid with as operation matrix multiplication and as neutral element the identity matrix $\mathbb{1}_n$.
\end{lemma}

In particular we can define integer powers of matrices. We set $A^0 = \mathbb{1}_n$ by convention.

We also have that if square matrices have both a left inverse and a right inverse, they are the same. See \ref{leftRightInverseMonoid}.

In fact, we have something stronger:
\begin{lemma}
Let $A\in \mathbb{F}^{n\times n}$ be a square matrix. Then $A$ has a left inverse \textup{if and only if} $A$ has a right inverse. Both inverses are the same.
\end{lemma}
\begin{proof}
Consider the map $f:\F^{n\times n}\to \F^{n\times n}: B\mapsto AB$. (This will later be called the left regular representation of $A$). 
 
We follow a chain of implications.
\begin{itemize}
\item \textit{$A$ has left inverse $\Rightarrow$ $f$ is injective}. Assume there exist matrices $B_1,B_2$ such that $AB_1 = AB_2$, then
\[ 0 = A^{-1}0 = A^{-1}(AB_1-AB_2) = A^{-1}A(B_1-B_2) = B_1-B_2. \]
\item \textit{$f$ is injective $\Rightarrow$ $f$ is surjective}. By \ref{invertibleFiniteDim}.
\item \textit{$f$ is surjective $\Rightarrow$ $A$ has right inverse}. By surjectivity there exists a matrix $B$ such that $f(B) = AB=\mathbb{1}_n$. Then $B$ is a right inverse by definition.
\end{itemize}
We have proven that the existence of a left inverse implies the existence of a right inverse. The opposite implication is obtained by considering the right regular representation $f:\F^{n\times n}\to \F^{n\times n}: B\mapsto BA$.
\end{proof}
\begin{definition}
A square matrix is called \udef{invertible} or \udef{nonsingular} or \udef{nondegenerate} if it has a left and right inverse.

If it is not invertible, it is called \udef{singular} or \udef{degenerate}.

We denote the inverse of $A$ as $A^{-1}$.
\end{definition}
\begin{lemma}
Let $A,B\in \mathbb{F}^{n\times n}$ be invertible matrices. Then $AB$ is invertible with inverse
\[ (AB)^{-1} = B^{-1}A^{-1}. \]
\end{lemma}

\begin{lemma}
Let $a,b\in\R$. Then
\[ (a\mathbb{1}_n +b\mathbb{J}_n)^{-1} = \frac{1}{a(a+nb)}[(a+nb)\mathbb{1}_n -b\mathbb{J}_n] = \frac{1}{a}\mathbb{1}_n - \frac{b}{a(a+nb)}\mathbb{J}_n. \]
\end{lemma}
\begin{proof}
By multiplication, using \ref{matrixOfOnesMultiplication}.
\end{proof}

\begin{definition}
Let $A\in\F^{n\times n}$. We say $A$ is \udef{strictly diagonally dominant} if
\[ \sum_{\substack{j \in 1:n \\ j\neq i}}|[A]_{ij}| < |[A]_{ii}|. \]
\end{definition}
\begin{proposition} \label{invertibleDiagonallyDominant}
If $A\in\F^{n\times n}$ is strictly diagonally dominant, then it is invertible.
\end{proposition}
\begin{proof}
We prove the contraposition. Assume, then, that $A$ is not invertible, so there exists an $x\neq 0$ such that $Ax = 0$ (TODO ref). So then for all $i\in 1:n$ we have
\[ \sum_{j = 1}^n [A]_{ij}x_j. \]
Setting $|x_m| = \max_{i\in 1:n}|x_i|$, we have
\[ [A]_{mm}x_m = -\sum_{\substack{j\in 1:n \\ j\neq m}}[A]_{mj}x_j \]
and so
\[ |[A]_{mm}|\cdot |x_m| = |\sum_{\substack{j\in 1:n \\ j\neq m}}[A]_{mj}x_j| \leq \sum_{\substack{j\in 1:n \\ j\neq m}}|[A]_{mj}|\cdot |x_j| \leq \sum_{\substack{j\in 1:n \\ j\neq m}}|[A]_{mj}|\cdot |x_m|.\]
In particular this means $A$ cannot be strictly diagonally dominant.
\end{proof}

\subsection{Matrices and linear maps}
\subsubsection{Matrices as maps $\F^n\to \F^m$}
The matrix multiplication $\F^{m\times n}\times \F^n \to \F^m$ can be curried to produce a map $\ell$. The input of $\ell$, i.e.\ a matrix in $\F^{m\times n}$ is typically written as a subscript. So, for a matrix $A\in\F^{m\times n}$ we have
\[ \ell_A: \F^n \to \F^m: \vec{v}\mapsto \ell_A(\vec{v}) = A\vec{v}. \]
Because the matrix multiplication is linear in the second argument (see \ref{linearityMatrixMultiplication}), the map $\ell_A: \F^n \to \F^m$ is linear for all matrices $A\in \F^{m\times n}$. So we have
\[ \ell: \F^{m\times n}\to \Hom(\F^n, \F^m). \]

Additionally this map $\ell$ is linear due to the matrix multiplication being linear in the first argument (see again \ref{linearityMatrixMultiplication}).

\begin{proposition} \label{ellIsomorphism}
For all $n,m\in\N_0$, the map
\[ \ell: \F^{m\times n}\to \Hom(\F^n, \F^m) \]
is an isomorphism.
\end{proposition}
\begin{proof}
We will explicitly construct an inverse. Take some $L\in\Hom(\F^n, \F^m)$. Now $L$ is completely determined by the images of the elements in the standard basis $\mathcal{E}= \{\vec{e}_i\}_{i=1}^n$, so we just need to find a matrix $A$ such that $\ell_A$ maps the basis elements to the same elements as $L$.

Now $A\vec{e}_1$ is just the first column of $A$. So we set the first column of $A$ to be $L(\vec{e}_1)$. Similarly $A\vec{e}_i$ is the $i^\text{th}$ column of $A$ and we set it equal to $L(\vec{e}_i)$. This gives the required matrix.
\end{proof}
The matrix $A$ that satisfies $\ell_A = L$ is often denoted $A_L$.

The construction in the previous proof is important for practically finding matrices associated with linear maps. The construction can be recapped as follows:
\[ A_L = \begin{pmatrix}
L(\vec{e}_1) & L(\vec{e}_2) & \hdots & L(\vec{e}_n)
\end{pmatrix} = \begin{pmatrix}
L_1(\vec{e}_1) & L_1(\vec{e}_2) & \hdots & L_1(\vec{e}_n)  \\
L_2(\vec{e}_1) & L_2(\vec{e}_2) & & \\
\vdots & & \ddots & \\
L_m(\vec{e}_1) & & & L_m(\vec{e}_n)
\end{pmatrix} \qquad [A_L]_{ij} = L_i(\vec{e}_j) \]
where $\mathcal{E}= \{\vec{e}_i\}_{i=1}^n$ is the standard basis of $\F^n$ and $L_i(\vec{e}_j) = [L(\vec{e}_j)]_i$ is the $i^\text{th}$ component of $L(\vec{e}_j)$.

If $\F^n$ is equipped with the standard inner product, then this can also be written as $L_i(\vec{e}_j) = \inner{\vec{e}_i, L(\vec{e}_j)}$, so
\[ A_L = \begin{pmatrix}
\inner{\vec{e}_1, L(\vec{e}_1)} & \inner{\vec{e}_1, L(\vec{e}_2)} & \hdots & \inner{\vec{e}_1, L(\vec{e}_n)}  \\
\inner{\vec{e}_2, L(\vec{e}_1)} & \inner{\vec{e}_2, L(\vec{e}_2)} & & \\
\vdots & & \ddots & \\
\inner{\vec{e}_m, L(\vec{e}_1)} & & & \inner{\vec{e}_m, L(\vec{e}_n)}
\end{pmatrix}. \]

\begin{proposition}
The map $\ell$ translates matrix multiplication into function composition:
\[ \ell_{AB} = \ell_A \circ \ell_B \qquad\text{and}\qquad \ell^{-1}(f\circ g) = \ell^{-1}(f)\ell^{-1}(g). \]
\end{proposition}
\begin{proof}
Let $A\in \F^{k\times m}$ and $B\in \F^{m\times n}$. Let $\vec{v}\in\F^n$, then
\[ \ell_{AB}(\vec{v}) = (AB)\vec{v} = A(B\vec{v}) = A(\ell_B(\vec{v})) = \ell_A(\ell_B(\vec{v})) = (\ell_A\circ\ell_B)(\vec{v}). \]
\end{proof}

\begin{lemma}
For all $n\in\N$ we have $\id_{\F^{n\times n}} = \ell_{\mathbb{1}_n}$.
\end{lemma}

\begin{lemma} \label{invertibleMapInvertibleMatrix}
Let $A$ be a matrix over $\F$. Then $\ell_A$ is invertible \textup{if and only if} $A$ is square and invertible.
\end{lemma}
\begin{proof}
Assume $\ell_A: \F^n\to\F^m$ invertible. Then $\F^n\cong\F^m$, so $m=n$ by \ref{isomorphicDimension}. Also $\ell^{-1}((\ell_A)^{-1})$ is an inverse of A because
\[ \ell^{-1}((\ell_A)^{-1})\cdot A = \ell^{-1}((\ell_A)^{-1})\cdot\ell^{-1}(\ell_A) = \ell^{-1}[(\ell_A)^{-1}\ell_A] = \ell^{-1}(\id_{\F^n}) = \mathbb{1}_n. \]

Conversely, assume $A$ invertible with inverse $A^{-1}$. Then $\ell_{A^{-1}}$ is the inverse of $\ell_A$:
\[ \ell_{A^{-1}}\ell_A = \ell_{\mathbb{1}_n} = \id_{\F^n} \qquad \text{and}\qquad \ell_A\ell_{A^{-1}} = \ell_{\mathbb{1}_n} = \id_{\F^n}. \]
\end{proof}

\subsubsection{Linear maps as matrices}
We can associate a matrix to any linear map by passing to coordinates. Let $L: V\to W$ be a linear map from an $n$-dimensional vector space $V$ to an $m$-dimensional vector space $W$. If we fix bases $\mathcal{V}$ of $V$ and $\mathcal{W}$ of $W$, then $\ell^{-1}$ associates a unique matrix with the linear map
\[ \co_{\mathcal{W}}\circ L \circ \co^{-1}_{\mathcal{V}}: \F^n\to\F^m. \]
In other words, $A$ is the unique matrix such that
\[ \begin{tikzcd}
V\rar{L}\dar[swap]{\co_{\mathcal{V}}} & W \dar{\co_{\mathcal{W}}} \\
\F^n \rar{\ell_A} & \F^m
\end{tikzcd} \qquad \text{commutes.} \]
We call this matrix $A \defeq \ell^{-1}(\co_{\mathcal{W}}\circ L \circ \co^{-1}_{\mathcal{V}})$ the \udef{matrix of the linear map $L$} w.r.t. the bases $\mathcal{V}$ and $\mathcal{W}$. This matrix is denoted
\[ (L)_{\mathcal{V}}^{\mathcal{W}} \qquad \text{or}\qquad \prescript{\mathcal{W}}{}{(L)}^{\mathcal{V}} \qquad \text{or}\qquad (L)_{\mathcal{W}\leftarrow \mathcal{V}}. \]

The commutativity of the diagram translates to the following lemma:
\begin{lemma} \label{commutativityMatrixLinearMap}
Let $L:V\to W$ be a linear map and $\mathcal{V},\mathcal{W}$ bases of $V,W$ respectively. Then
\[ (L)^\mathcal{W}_\mathcal{V} \circ \co_\mathcal{V} = \co_{\mathcal{W}}\circ L. \]
\end{lemma}
A practical way to calculate matrices associated with linear maps is given by the following lemma.
\begin{lemma}
Let $L:V\to W$ be a linear map and $\mathcal{V}=\{\vec{v}_i\}_{i=1}^n,\mathcal{W} = \{\vec{w}_i\}_{i=1}^m$ bases of $V,W$ respectively. Then
\[ (L)^\mathcal{W}_\mathcal{V} = \begin{pmatrix}
\co_\mathcal{W}(L(\vec{v}_1)) & \co_\mathcal{W}(L(\vec{v}_2)) & \hdots & \co_\mathcal{W}(L(\vec{v}_n)).
\end{pmatrix} \]
\end{lemma}
\begin{proof}
The $i^\text{th}$ column of $(L)^\mathcal{W}_\mathcal{V}$ is equal to $(L)^\mathcal{W}_\mathcal{V}\vec{e}_i = (L)^\mathcal{W}_\mathcal{V}(\co_\mathcal{V}(\vec{v}_i))$, where $\vec{e}_i$ is the $i^\text{th}$ element of the standard basis $\mathcal{E}$. This is equal to $\co_{\mathcal{W}}(L(\vec{v}))$ by \ref{commutativityMatrixLinearMap}.
\end{proof}

\begin{proposition} \label{algebraMatricesLinearMaps}
Let $U,V,W$ be vector spaces with bases $\mathcal{U},\mathcal{V},\mathcal{W}$, resp., and $S:V\to W, T:U\to V$ linear maps. Then
\[ (S)_\mathcal{V}^\mathcal{W}(T)_\mathcal{U}^\mathcal{V} = (S\circ T)_{\mathcal{U}}^{\mathcal{W}}. \]
\end{proposition}
\begin{proof}
We calculate
\begin{align*}
\ell^{-1}(\co_{\mathcal{W}}\circ S \circ \co^{-1}_{\mathcal{V}})\ell^{-1}(\co_{\mathcal{V}}\circ T \circ \co^{-1}_{\mathcal{U}}) &= \ell^{-1}(\co_{\mathcal{W}}\circ S \circ \co^{-1}_{\mathcal{V}}\circ\co_{\mathcal{V}}\circ T \circ \co^{-1}_{\mathcal{U}}) \\
&= \ell^{-1}(\co_{\mathcal{W}}\circ (S \circ T) \circ \co^{-1}_{\mathcal{U}}) = (S\circ T)_{\mathcal{U}}^{\mathcal{W}}.
\end{align*}
\end{proof}

\begin{proposition}
Let $V,W$ be finite-dimensional vector spaces over a field $\mathbb{F}$ with bases $\mathcal{V}$ and $\mathcal{W}$, respectively. The mapping
\[ (-)^{\mathcal{W}}_{\mathcal{V}}:\Hom_\mathbb{F}(V,W) \to \mathbb{F}^{m\times n}: L\mapsto (L)^{\mathcal{W}}_{\mathcal{V}} \]
is an isomorphism.
\end{proposition}
\begin{proof}
It is equal to
\[ \ell^{-1}\circ(\co_\mathcal{W})_*\circ(\co^{-1}_\mathcal{V})^*. \]
Now $\ell$ is an isomorphism by \ref{ellIsomorphism} and thus $\ell^{-1}$ is one by \ref{inverseLinear}. The maps $(\co_\mathcal{W})_*$ and $(\co^{-1}_\mathcal{V})^*$ are injective maps between finite-dimensional spaces by \ref{monicEpicInPrePostComposition} and thus isomorphisms by \ref{invertibleFiniteDim}.
So we have a composition of isomorphisms, which is an isomorphism.
\end{proof}
\begin{corollary}
Let $V,W$ be finite-dimensional vector spaces over a field $\mathbb{F}$, then
\[ \dim_{\mathbb{F}}\Hom_\mathbb{F}(V,W) = (\dim_{\mathbb{F}} V)\cdot (\dim_{\mathbb{F}} W). \] \label{dimHomset}
\end{corollary}

\begin{lemma}
Let $A\in\F^{m\times n}$ be a matrix and let $\mathcal{E}_m$ and $\mathcal{E}_n$ be the standard bases of $\F^m$ and $\F^n$. Then
\[ (\ell_{A})_{\mathcal{E}_n}^{\mathcal{E}_m} = A. \]
\end{lemma}

\subsubsection{Changing basis with matrices}
In particular we can apply all the theory of the previous section to the identity map $\id: V\to V$. Let $\beta, \beta'$ be two bases of $V$. Then \ref{commutativityMatrixLinearMap} gives
\[ \co_{\beta'}(v) = (\id)_{\beta}^{\beta'}\co_\beta(v) \]
which captures the effect of transforming from one basis to another.
\begin{definition}
Matrices of the form $(\id)_{\beta}^{\beta'}$ are called \udef{transition matrices} or \udef{change-of-basis matrices}.
\end{definition}
\begin{lemma}
Let $\beta, \beta'$ be two bases of a vector space $V$. Then
\[ \left((\id)_{\beta}^{\beta'}\right)^{-1} = (\id)_{\beta'}^{\beta}. \]
\end{lemma}
\begin{proof}
This follows from \ref{algebraMatricesLinearMaps} which gives
\[ (\id)_{\beta'}^{\beta}(\id)_{\beta}^{\beta'} = (\id)_\beta^\beta = \mathbb{1}_n. \]
\end{proof}

Let $L\in \Hom(V)$. If we know $(L)_\beta^\beta$, we can calculate $(L)_{\beta'}^{\beta'}$ using
\begin{align*}
(L)_{\beta'}^{\beta'} &= (\id)_{\beta}^{\beta'}(L)_\beta^\beta(\id)_{\beta'}^{\beta} \\
&= \left((\id)_{\beta'}^{\beta}\right)^{-1}(L)_\beta^\beta(\id)_{\beta'}^{\beta}
\end{align*}
\begin{definition}
Let $A,B\in \mathbb{F}^{n\times n}$, then $A$ and $B$ are called \udef{similar} if there exists an invertible matrix $P\in\mathbb{F}^{n\times n}$ such that
\[ B = P^{-1}A P\] 
\end{definition}

Any similar matrices may be seen as matrices of the same linear transformation w.r.t. different bases.



\section{The transpose and standard inner product}
\subsection{The transpose}
\begin{definition}
Let $A\in \mathbb{F}^{m\times n}$. The \udef{transpose} of $A$, denoted $A^\transp$, is defined by
\[ [A^\transp]_{ij} = [A]_{ji}. \]
\end{definition}
\begin{lemma}
The transpose is a linear operation.
\end{lemma}

\begin{lemma}
Let $A,B$ be matrices such that $AB$ is defined, then
\[ (AB)^\transp = B^\transp A^\transp. \]
\end{lemma}
\begin{proof}
We simply calculate
\[ [(AB)^\transp]_{ij} = [AB]_{ji} = \sum_{k}[A]_{jk}[B]_{ki} = \sum_{k}[B]_{ki}[A]_{jk} = \sum_{k}[B^\transp]_{ik}[A^\transp]_{kj} = [B^\transp A^\transp]_{ij}. \]
\end{proof}

\begin{definition}
\begin{itemize}
\item A square matrix $A$ such that $A=A^\transp$ is called \udef{symmetric}.
\item A square matrix $A$ such that $A=-A^\transp$ is called \udef{skew symmetric}.
\end{itemize}
\end{definition}

\subsection{The standard inner product}
Let $\vec{v},\vec{w}\in\F^n$. Then the standard inner product is given by
\[ \inner{\vec{v},\vec{w}} \defeq \overline{\vec{v}}^\transp \vec{w}. \]
This reduces to $\vec{v}^\transp\vec{w}$ if $\F = \R$.

In the sequel we will always assume $\F^n$ is equipped with the standard inner product, unless otherwise specified.

The inner product also induces a norm. There are multiple norms that may be of interest. To avoid confusion we may denote the norm that arises from the inner product with a subscript 2:
\[ \norm{\vec{v}} = \norm{\vec{v}}_2 = \sqrt{\inner{\vec{v},\vec{v}}}. \]

\begin{lemma} \label{componentsFromStandardInnerProduct}
Let $\{\vec{e}_i\}_{i=1}^n$ be the standard basis of $\F^n$ and $\{\vec{f}_i\}_{i=1}^m$ the standard basis of $\F^m$. Let $A\in\F^{n\times m}$. Then
\[ [A]_{i,j} = \vec{e}^\transp_i A \vec{f}_j = \inner{\vec{e}_i, A\vec{f}_j}. \]
\end{lemma}
\begin{proof}
Using \ref{standardBasisFromIdentityMatrix} and \ref{submatrixMultiplication}, we calculate
\begin{align*}
\vec{e}^\transp_i A \vec{f}_j &= [\mathbb{1}_n]_{-,i}^\transp \cdot A \cdot [\mathbb{1}_m]_{-,j} \\
&= [\mathbb{1}_n]_{i, -} \cdot [A]_{-,-} \cdot [\mathbb{1}_m]_{-,j} \\
&= [\mathbb{1}_n\cdot A]_{i,-}\cdot [\mathbb{1}_m]_{-,j} \\
&= [\mathbb{1}_n\cdot A\cdot\mathbb{1}_m]_{i,j} = [A]_{i,j}.
\end{align*}
\end{proof}

\subsubsection{The adjoint}
\begin{definition}
Let $A\in \F^{m\times n}$. We call $A^* \defeq \ell^{-1}\big(\ell(A)^*\big)$ the \udef{adjoint} of $A$, where we have equiped $\F^m$ and $\F^n$ with their standard inner products.
\end{definition}

\begin{lemma}
Let $A\in \F^{m\times n}$. Then $A^* = \overline{A}^\transp$.
\end{lemma}
In other words, the adjoint is the conjugate transpose.
\begin{proof}
Take arbitrary $\vec{v}\in\F^m$ and $\vec{w}\in\F^n$. Then
\[ \inner{\vec{v}, A\vec{w}} = \overline{\vec{v}}^\transp A\vec{w} = \overline{\overline{A}^\transp\vec{v}}^\transp\vec{w} = \inner{\overline{A}^\transp\vec{v}, \vec{w}}. \]
TODO!
\end{proof}

\begin{definition}
Let $A\in\F^{m\times n}$ be a matrix. For properties related to the adjoint, we say $A$ has the property if $\ell_A$ has the property.

In addition, we say
\begin{itemize}
\item $A$ is \udef{Hermitian} if $A$ is self-adjoint;
\item $A$ is \udef{skew Hermitian} if $A$ is swew-adjoint.
\end{itemize}
\end{definition}
Thus, e.g.\
\begin{itemize}
\item if $A^*A = AA^*$, then $A$ is called normal;
\item if $A^2 = A = A^*$, then $A$ is called an orthogonal projection;
\item if $A^*A = \mathbb{1}_n$, then $A$ is called isometric;
\item if $A^*A = \mathbb{1}_n$ and $AA^* = \mathbb{1}_m$, then $A$ is called unitary.
\end{itemize}

\begin{lemma}
If $A\in \F^{m\times n}$ is unitary, then $m=n$.
\end{lemma}
\begin{proof}
TODO
\end{proof}

\begin{lemma}
Let $U\in\F^{n\times n}$ be a matrix. Then the following are equivalent
\begin{enumerate}
\item $U$ is unitary, i.e.\ $U^*U=\mathbb{1}_n$;
\item the columns of $U$ are orthonormal;
\item $UU^*=\mathbb{1}_n$;
\item $U^*$ is unitary;
\item the row of $U$ are orthonormal;
\item $U$ is invertible and $U^{-1} = U^*$.
\end{enumerate}
\end{lemma}
Thus every square isometric matrix is unitary.
\begin{proof}
TODO
\end{proof}

\begin{corollary}
If $\alpha,\beta$ are orthonormal bases of $\F^{n}$, then the transition matrix $(\id)_\alpha^\beta$ is unitary.
\end{corollary}


\subsection{Orthogonal matrices}
\begin{proposition}
Let $A\in\R^{n\times n}$ and $\R^n$ have the standard inner product. The following are equivalent:
\begin{enumerate}
\item The columns of $A$ form an orthonormal basis of $\R^n$;
\item The rows of $A$ form an orthonormal basis of $\R^n$;
\item $A^\transp A = \mathbb{1}_n$;
\item $A^{-1} = A^\transp$;
\end{enumerate}
\end{proposition}
\begin{definition}
A matrix $A\in\R^{n\times n}$ satisfying any of the above is called an \udef{orthogonal matrix}.
\end{definition}
We can formulate the spectral theorem as follows:
\begin{proposition}
Let $A\in\R^{n\times n}$ be a symmetric matrix. The there exists an orthogonal matrix $P\in \R^{n\times n}$ such that $P^{-1}AP = P^\transp A P$ is a diagonal matrix.
\end{proposition}

$\det(A) = \pm 1$. Orthogonal matrix means orthogonal transformation.
For orthogonal matrix $P$: $\norm{A}_F = \norm{PA}_F$
s

\subsection{Inequalities}
\subsubsection{Cauchy-Schwarz corollaries}
\begin{lemma}
Let $A\in \C^{n\times n}$ be positive ($A\geq 0$). Then
\[ |[A]_{i,j}| \leq \sqrt{[A]_{i,i}[A]_{j,j}}. \]
\end{lemma}
\begin{proof}
Consider the energy form $\inner{-, -}_A = \inner{-, A-} = \overline{(-)}^\transp A(-)$. This is a pre-inner product. Let $\{\vec{e}_i\}_{i=0}^n$ be the standard basis of $\F^n$. Then $[A]_{i,j} = \inner{\vec{e}_i, \vec{e}_j}_A$ by \ref{componentsFromStandardInnerProduct}. Thus, in particular, $\sqrt{[A]_{i,i}} = \sqrt{\inner{\vec{e}_i, \vec{e}_i}_A} = \norm{\vec{e}_i}_A$.

The Cauchy-Schwarz inequality \ref{CauchySchwarz} then yields the result.
\end{proof}

\section{Block matrices}
Any given matrix can be \textit{interpreted} as consisting of submatrices. Eg,
\[ \begin{bmatrix}
1&2&3\\4&5&6\\7&8&9
\end{bmatrix} \quad \text{can be viewed as} \quad 
\begin{bmatrix}
\begin{pmatrix}
1 & 2
\end{pmatrix} & 3 \\
\begin{pmatrix}
4 & 5 \\ 7 & 8
\end{pmatrix} & \begin{pmatrix}
6 \\ 9
\end{pmatrix}
\end{bmatrix}
\quad \text{with partitioning} \quad
\left[\begin{array}{cc|c}
1&2&3\\ \hline
4&5&6\\7&8&9
\end{array}\right]
 \]
\begin{definition}
A matrix consisting of submatrices (also known as blocks) is called a \udef{block matrix} or \udef{partitioned matrix}.

If $A\in\F^{n\times n}, \vec{x},\vec{y}\in\F^n$ and $c\in \F$, we call the block matrices
\[ \begin{bmatrix}
c & \vec{x}^\transp \\ \vec{y} & A
\end{bmatrix}, \begin{bmatrix}
\vec{x}^\transp & c \\ A & \vec{y}
\end{bmatrix}, \begin{bmatrix}
A & \vec{x} \\
\vec{y}^\transp & c
\end{bmatrix}, \quad\text{and}\quad \begin{bmatrix}
\vec{x} & A \\ c & \vec{y}^\transp
\end{bmatrix} \]
\udef{bordered matrices}. They have been obtained from $A$ by \udef{bordering}.
\end{definition}

The partition can be specified at the level of the dimensions: let $A$ be an $(m\times n)$-matrix, let $m_1,\ldots, m_k\leq m$ be the number of rows in each horizontal partition and let $n_1,\ldots, n_l\leq n$ be the number of columns in each vertical partition. Clearly
\[ m = \sum_{i=1}^k m_i \qquad \text{and} \qquad n = \sum_{i=1}^l n_i. \]

We then say the block matrix has dimensions $(m_1|\ldots|m_k) \times (n_1|\ldots|n_l)$, or $A$ is a $(m_1|\ldots|m_k) \times (n_1|\ldots|n_l)$-matrix.

\begin{example}
The block matrix considered above is a $(1|2)\times (2|1)$-matrix.
\end{example}

We write $A_{i,j}$ or $(A)_{i,j}$ for the block in the $i^\text{th}$ horizontal partition and the $j^\text{th}$ vertical partition. So if $A$ is a $(m_1|\ldots|m_k) \times (n_1|\ldots|n_l)$-matrix, we can write
\[ A = \begin{bmatrix}
A_{1,1} & \hdots & A_{1,l} \\
\vdots & \ddots & \vdots \\
A_{k,1} & \hdots & A_{k,l}
\end{bmatrix} \qquad \text{where} \qquad A_{i,j}\in\F^{m_i\times n_j}. \]

Any $m\times n$-matrix can be partitioned into columns, which means identifying it with an $m\times (1|1|\ldots|1)$-matrix.

Similarly, any $m\times n$-matrix can be partitioned into rows, which means identifying it with a $(1|1|\ldots|1)\times n$-matrix.

\begin{lemma}
Let $A$ be a partitioned matrix of dimensions $(m_1|\ldots|m_k) \times (n_1|\ldots|n_l)$, then $A^\transp$ is a partitioned matrix of dimensions $(n_1|\ldots|n_l) \times (m_1|\ldots|m_k)$ and
\[ (A)_{i,j} = (A^\transp)_{j,i}. \]
\end{lemma}

\begin{definition}
Let $A,B$ be matrices. We can then form the \udef{direct sum} matrix $A\oplus B$ as follows:
\[ A\oplus B = \begin{pmatrix}
A & \mathbb{0} \\ \mathbb{0} & B
\end{pmatrix}. \]
\end{definition}

\begin{proposition}
Let $A$ be a partitioned matrix of dimensions $(m_1|\ldots|m_k) \times (n_1|\ldots|n_l)$ and $B$ a partitioned matrix of dimensions $(n_1|\ldots|n_l)\times (p_1|\ldots|p_q)$.

The matrix product of $A$ and $B$ is an $(m_1|\ldots|m_k) \times (p_1|\ldots|p_q)$-matrix with blocks
\[ (AB)_{i,j} = \sum_{t}A_{i,t}B_{t,j}. \]
That is, multiplication of two block matrices can be carried out as if their blocks were scalars.
\end{proposition}
We can abbreviate the dimension requirements as
\[ [(m_1|\ldots|m_k) \times (n_1|\ldots|n_l)]\cdot[(n_1|\ldots|n_l)\times (p_1|\ldots|p_q)] = [(m_1|\ldots|m_k) \times (p_1|\ldots|p_q)]. \]
\begin{corollary} \label{multiplicationBlockMatrices}
Let $A,B$ be matrices of dimensions $k\times l$ and $l \times m$. Then
\[ AB = [AB]_{-,-} = \sum_k[A]_{-,k}[B]_{k, -}; \]
and
\begin{itemize}
\item we can partition $A$ into rows and $B$ into columns to get
\[ [AB]_{i,j} = [A]_{i,-}[B]_{-, j}; \]
\item we can partition $B$ into columns to get
This can also be written as
\[ [AB]_{-, j} = [A]_{-,-}[B]_{-,j} = A[B]_{-, j}; \]
\item we can partition $A$ into rows to get
This can also be written as
\[ [AB]_{i, -} = [A]_{i,-}[B]_{-,-} = [A]_{i,-}B. \]
\end{itemize}
\end{corollary}
In particular this means we can write
\[ AB = \begin{pmatrix}
A
\end{pmatrix} \begin{pmatrix}
[B]_{-,1} & \hdots & [B]_{-, m}
\end{pmatrix} = \begin{pmatrix}
A[B]_{-,1} & \hdots & A[B]_{-,m}
\end{pmatrix}. \]

\subsection{Identities and inverses}
\begin{lemma}
Let $X,Y,Z$ be conformal matrices and $Y,Z$ invertible, then
\[ \begin{bmatrix}
Y & X \\ \mathbb{0} & Z
\end{bmatrix}^{-1} = \begin{bmatrix}
Y^{-1} & -Y^{-1}XZ^{-1} \\ \mathbb{0} & Z^{-1}.
\end{bmatrix}. \]
In particular
\[ \begin{bmatrix}
\mathbb{1} & X \\ \mathbb{0} & \mathbb{1}
\end{bmatrix}^{-1} = \begin{bmatrix}
\mathbb{1} & -X \\ \mathbb{0} & \mathbb{1}
\end{bmatrix}. \]
\end{lemma}
\begin{corollary}
If $A$ is an upper triangular matrix with nonzero diagonal entries, the $A$ is invertible and $[A^{-1}]_{i,i} = [A]_{i,i}^{-1}$
\end{corollary}
\begin{proof}
Induction on dimension.
\end{proof}

\begin{lemma}
Let $X,Y,Z$ be conformal matrices. Then
\[ \begin{bmatrix}
Y & X \\ \mathbb{0} & Z
\end{bmatrix} = \begin{bmatrix}
\mathbb{1} & A \\ \mathbb{0} & \mathbb{1}
\end{bmatrix}\begin{bmatrix}
Y & X-AZ + YA \\ \mathbb{0} & Z
\end{bmatrix}\begin{bmatrix}
\mathbb{1} & -A \\ \mathbb{0} & \mathbb{1}
\end{bmatrix} \]
for all conformal $A$.
\end{lemma}

\begin{lemma} \label{schurComplementLemma}
Let $A,B,C,D$ be conformal matrices. Then, if $D$ is invertible 
\[ M = \begin{bmatrix}
A & B \\ C & D
\end{bmatrix} = \begin{bmatrix}
\mathbb{1} & BD^{-1} \\ \mathbb{0} & \mathbb{1}
\end{bmatrix}\begin{bmatrix}
A-BD^{-1}C & \mathbb{0} \\ \mathbb{0} & D
\end{bmatrix}\begin{bmatrix}
\mathbb{1} & \mathbb{0} \\ D^{-1}C & \mathbb{1}
\end{bmatrix}  \]
and if $A$ is invertible,
\[ M = \begin{bmatrix}
A & B \\ C & D
\end{bmatrix} = \begin{bmatrix}
\mathbb{1} & \mathbb{0} \\ CA^{-1} & \mathbb{1}
\end{bmatrix}\begin{bmatrix}
A & \mathbb{0} \\ \mathbb{0} & D - CA^{-1}B
\end{bmatrix}\begin{bmatrix}
\mathbb{1} & A^{-1}B \\ \mathbb{0} & \mathbb{1}
\end{bmatrix}.  \]
\end{lemma}
\begin{definition}
The matrix $M/D \defeq A-BD^{-1}C$ is the \udef{Schur complement} of $D$ in $M$.

Similarly $M/A \defeq D-CA^{-1}B$ is the Schur complement of $A$ in $M$.
\end{definition}
\begin{lemma}
Let $A,B,C,D$ be conformal matrices such that all requisite matrices are invertible. Then
\begin{align*}
M^{-1} &= \begin{bmatrix}
A & B \\ C & D
\end{bmatrix}^{-1} = \begin{bmatrix}
\mathbb{1} & \mathbb{0} \\ -D^{-1}C & \mathbb{1}
\end{bmatrix}\begin{bmatrix}
(M/D)^{-1} & \mathbb{0} \\ \mathbb{0} & D^{-1}
\end{bmatrix}\begin{bmatrix}
\mathbb{1} & -BD^{-1} \\ \mathbb{0} & \mathbb{1}
\end{bmatrix} \\
&= \begin{bmatrix}
(M/D)^{-1} & -(M/D)^{-1}BD^{-1} \\ -D^{-1}C(M/D)^{-1} & D^{-1}+D^{-1}C(M/D)^{-1}BD^{-1}\end{bmatrix} \\
M^{-1} &= \begin{bmatrix}
A & B \\ C & D
\end{bmatrix}^{-1} = \begin{bmatrix}
\mathbb{1} & -A^{-1}B \\ \mathbb{0} & \mathbb{1}
\end{bmatrix}\begin{bmatrix}
A^{-1} & \mathbb{0} \\ \mathbb{0} & (M/A)^{-1}
\end{bmatrix}\begin{bmatrix}
\mathbb{1} & \mathbb{0} \\ -CA^{-1} & \mathbb{1}
\end{bmatrix} \\
&= \begin{bmatrix}
A^{-1}+A^{-1}B(M/A)^{-1}CA^{-1} & -A^{-1}B(M/A)^{-1} \\-(M/A)^{-1}CA^{-1} & (M/A)^{-1}
\end{bmatrix}.
\end{align*}
\end{lemma}

\begin{lemma}
Let $A,B$ be conformal matrices. Then
\[ \begin{bmatrix}
AB & A \\ \mathbb{0} & \mathbb{0}
\end{bmatrix} \qquad\text{and}\qquad \begin{bmatrix}
\mathbb{0} & A \\ \mathbb{0} & BA
\end{bmatrix} \]
are similar
\end{lemma}
\begin{proof}
$\begin{bmatrix}
\mathbb{1} & \mathbb{0} \\ B & \mathbb{1}
\end{bmatrix}\begin{bmatrix}
AB & A \\ \mathbb{0} & \mathbb{0}
\end{bmatrix} = \begin{bmatrix}
\mathbb{0} & A \\ \mathbb{0} & BA
\end{bmatrix}\begin{bmatrix}
\mathbb{1} & \mathbb{0} \\ B & \mathbb{1}
\end{bmatrix}.$
\end{proof}

\begin{proposition}
Let $A,B,C,U,V$ be conformal matrices. Then
\begin{enumerate}
\item \textup{(Push-through identity)} $(\mathbb{1}+UV)^{-1}U = U(\mathbb{1}+VU)^{-1}$;
\item $(\mathbb{1}+A)^{-1} \begin{aligned}[t]
&= \mathbb{1}-(\mathbb{1}+A)^{-1} A\\
&= \mathbb{1}-A(\mathbb{1}+A)^{-1}
\end{aligned}$
\item $(\mathbb{1} + UV)^{-1} = \mathbb{1} - U(\mathbb{1}+VU)^{-1}V$;
\item \textup{(Woodbury identity)} $(B + UCV)^{-1} = B^{-1} - B^{-1}U(C^{-1}+VB^{-1}U)VB^{-1}$.
\end{enumerate}
\end{proposition}
\begin{proof}
(1) From $U(\mathbb{1} + VU) = (\mathbb{1} + UV)U$.

(2) From $\mathbb{1} = (\mathbb{1}+A)(\mathbb{1}+A)^{-1} = (\mathbb{1}+A)^{-1} + A(\mathbb{1}+A)^{-1}$.

(3) $\begin{aligned}[t]
(\mathbb{1} + UV)^{-1} &= \mathbb{1}-(\mathbb{1} + UV)^{-1}UV & &\text{using (2)} \\
&= \mathbb{1}-U(\mathbb{1} + VU)^{-1}V & &\text{using (1)}.
\end{aligned}$

(4) $\begin{aligned}[t]
(B + UCV)^{-1} &= (B(\mathbb{1} + B^{-1}UCV))^{-1} \\
&= (\mathbb{1} + (B^{-1}U)(CV))^{-1}B^{-1} \\
&= (\mathbb{1}-(B^{-1}U)(\mathbb{1} + (CV)(B^{-1}U))^{-1}(CV))B^{-1} & &\text{using (3)}\\
&= B^{-1}-B^{-1}U(\mathbb{1} + CVB^{-1}U)^{-1}CVB^{-1} \\
&= B^{-1}-B^{-1}U(C^{-1}(\mathbb{1} + CVB^{-1}U))^{-1}VB^{-1} \\
&= B^{-1}-B^{-1}U(C^{-1} + VB^{-1}U)^{-1}VB^{-1}.
\end{aligned}$
\end{proof}
\begin{corollary}[Sherman-Morrison formula]
Let $A\in\F^{n\times n}$ and $\vec{u},\vec{v}\in\F^n$. 
Then $A+\vec{u}\vec{v}^\transp$ is invertible iff $1 + \vec{v}^\transp A^{-1}\vec{u} \neq 0$. In this case
\[ (A+\vec{u}\vec{v}^\transp)^{-1} = A^{-1} - \frac{A^{-1}\vec{u}\vec{v}^\transp A^{-1}}{1 + \vec{v}^\transp A^{-1}\vec{u}}. \]
\end{corollary}
\begin{corollary}[Hua's identity]
Let $A,B$ be conformal matrices. Then
\begin{align*}
(A+B)^{-1} &= A^{-1} - A^{-1}(B^{-1}+A^{-1})^{-1}A^{-1} \\
&= A^{-1} - A^{-1}(AB^{-1}+\mathbb{1})^{-1} \\
(A-B)^{-1} &= A^{-1} + A^{-1}B(A-B)^{-1} \\
&= \sum_{k=0}^\infty (A^{-1}B)^kA^{-1}.
\end{align*}
\end{corollary}

\section{Vector spaces associated with a matrix}
\subsection{Row and column space}
\begin{definition}
Let $A$ be an $(n\times m)$-matrix. It can be partitioned into both rows and columns. Let $R_1,\ldots, R_n$ be the rows of $A$ and $C_1,\ldots, C_m$ the columns of $A$:
\[ A = \begin{pmatrix}
R_1 \\ \vdots \\ R_n
\end{pmatrix} = \begin{pmatrix}
C_1 & \hdots & C_m
\end{pmatrix}. \]
Then
\begin{itemize}
\item $\Span\{R_1,\ldots, R_n\}$ is the \udef{row space} $\Row(A)$ of $A$; and
\item $\Span\{C_1,\ldots, C_m\}$ is the \udef{column space} $\Col(A)$ of $A$.
\end{itemize}
We call $\dim\Row(A)$ the \udef{row rank} and $\dim\Col(A)$ the \udef{column rank}.
\end{definition}
Clearly $\Col(A) = \Row(A^\transp)$.

\begin{lemma} \label{columnSpace}
Let $A\in \F^{m\times n}$ and $\vec{b}\in \F^m$. Then
\[ \exists \vec{x}\in\F^n: A\vec{x}=\vec{b} \quad\iff\quad \vec{b}\in\Col(A). \]
Moreover, if the columns of $A$ are linearly independent, then the $\vec{x}\in\F^n$ is unique. 
\end{lemma}
\begin{proof}
By \ref{multiplicationBlockMatrices}
\begin{align*}
A\vec{x} &= [A\vec{x}]_{-,-} = \sum_{j=1}^n[A]_{-,j}[\vec{x}]_{j,-} = \sum_{j=1}^n[A]_{-,j}[\vec{x}]_{j} \\
&= [A]_{-,1}[\vec{x}]_1 + \ldots [A]_{-,n}[\vec{x}]_n  \\
&= C_1[\vec{x}]_1 + \ldots C_n[\vec{x}]_n.
\end{align*}
\end{proof}

\begin{proposition} \label{rowColSubspaces}
Let $A$ and $B$ be matrices. Then
\begin{enumerate}
\item $\Col(B)\subseteq \Col(A) \iff B = AX$ for some matrix $X$;
\item $\Row(B)\subseteq \Row(A) \iff B = YA$ for some matrix $Y$.
\end{enumerate}
\end{proposition}
Note that for point (1) to hold, $A$ and $B$ must have the same number of rows. For point (2) they must have the same number of columns.
\begin{proof}
(1) $\boxed{\Rightarrow}$ Assume $\Col(B)\subseteq \Col(A)$. Then   by \ref{columnSpace}, for each column $\vec{b}_j = [B]_{-,j}$ of $B$ we can find an $\vec{x}_j\in \F^n$ such that $A\vec{x}_j = \vec{b}_j$. Then
\[ B = \begin{pmatrix}
\vec{b}_1 & \hdots & \vec{b}_k
\end{pmatrix} = \begin{pmatrix}
A\vec{x}_1 & \hdots & A\vec{x}_k
\end{pmatrix} = A \begin{pmatrix}
\vec{x}_1 & \hdots & \vec{x}_k
\end{pmatrix} = AX.\]

$\boxed{\Leftarrow}$ By \ref{multiplicationBlockMatrices} we can write
\[ AB = \begin{pmatrix}
A[B]_{-,1} & \hdots & A[B]_{-,m}
\end{pmatrix}, \]
so every column in $AB$ is of the form $A[B]_{-,i}$, which is a linear combination of the columns in $A$.

(2) We simply calculate using point (1):
\[ \Row(B)\subseteq \Row(A) \iff \Col(B^\transp)\subseteq \Col(A^\transp) \iff B^\transp = A^\transp X \iff B = X^\transp A. \]
\end{proof}
\begin{corollary} \label{RowColSpaceProduct}
Let $A,B$ be conformal matrices. Then
\begin{enumerate}
\item $\Col(AB)\subseteq \Col(A)$;
\item $\Row(AB)\subseteq \Row(B)$.
\end{enumerate}
\end{corollary}
\begin{corollary} \label{RowColSpaceInverse}
A matrix $A$ is invertible \textup{if and only if} $\Col(A) = \Row(A) = \F^n$.
\end{corollary}
\begin{proof}
$\boxed{\Rightarrow}$ From $A = \mathbb{1}_nA$ we see that $\Col(A)\subseteq\Col(\mathbb{1}_n) = \F^n$.

From $\mathbb{1}_n = AA^{-1}$ we see that $\Col(\mathbb{1}_n)\subseteq\Col(A)$, so $\Col(\mathbb{1}_n) = \Col(A)$. The calculation of the row space is similar.

$\boxed{\Leftarrow}$ From $\Col(A) = \Col(\mathbb{1}_n)$, there exists an $X$ such that $\mathbb{1}_n = AX$, so $X$ is the inverse of $A$.
\end{proof}
\begin{corollary}
Let $A\in\F^{m\times n}$. Then
\[ \dim\Row(A) = \dim\Col(A) \]
i.e.\ the row rank equals the column rank.
\end{corollary}
\begin{proof}
Take a basis for $\Col(A)$ and let $X$ have these vectors as columns. Then $\Col(X) = \Col(A)$, so $A = XY$ for some $Y\in\F^{k\times n}$.

By point 2. of the proposition, we have $\Row(A)\subseteq \Row(Y)$ and due to the dimensions, we have $\dim\Row(Y)\leq k$. So
\[ \dim\Row(A) \leq \dim\Row(Y) \leq k = \dim\Col(A). \]

Consider $A^\transp$, which can be factorised as before by taking a basis of its column space and putting the vectors in the columns of $X'$. Then $A^\transp = X'Y'$. As before, we have
\[ \dim\Col(A) = \dim\Row(A^\transp) \leq \dim\Col(A^\transp) = \dim\Row(A). \]

Combining the inequalities gives $\dim\Col(A) = \dim\Row(A)$.
\end{proof}
\begin{definition}
We can unambiguously call $\dim\Col(A)=\dim\Row(A)$ the \udef{rank} of the matrix $A$. We write $\Rank(A)$.
\end{definition}

\begin{lemma} \label{extendToInvertible}
Let $A\in\F^{m\times n}$.
\begin{enumerate}
\item If $\Rank(A) = m$, then $n\geq m$ and there exists $X\in\F^{(n-m)\times n}$ such that
\[ \begin{bmatrix}
A \\ X
\end{bmatrix} \in \F^{n\times n} \quad \text{is invertible.} \]
\item If $\Rank(A) = n$, then $m\geq n$ and there exists $Y\in\F^{m\times (m-n)}$ such that
\[ \begin{bmatrix}
A & Y
\end{bmatrix} \in \F^{m\times m} \quad \text{is invertible.} \]
\end{enumerate}
\end{lemma}
\begin{proof}
(1) In this case the rows are linearly independent and elements of $\F^n$, so they can be extended to a basis of $\F^n$. This extension is the matrix $X$.

(2) Transpose and apply (1).
\end{proof}

\begin{lemma}[Full-rank factorisation]
Let $A\in\F^{m\times n}$ and $k=\dim\Col(A)$. Then $A$ can be factorised as $A=XY$ where $X\in\F^{m\times k}$, $Y\in\F^{k\times n}$ and
\[ k = \Rank(A) = \Rank(X) = \Rank(Y). \]
Moreover, for any matrix $X$, the following are equivalent:
\begin{enumerate}
\item the columns of $X$ form a basis of $\Col(A)$;
\item there is a unique $Y\in\F^{k\times n}$ such that $A=XY$.
\end{enumerate}
Clearly considering $A^\transp$ yields dual equivalences.
\end{lemma}
\begin{proof}
By \ref{columnSpace} $Xv = [A]_{-,j}$ has a unique solution $v=y_j$ for all $j$ if and only if the columns of $X$ are linearly independent and $[A]_{-,j}$ is in their span, i.e.\ they from a basis for $\Col(A)$.
\end{proof}

\begin{proposition} \label{imageColumnSpace}
Let $L$ be a linear map. Then
\[ \im(L) = \Col(A_L). \]
\end{proposition}
This implies that the rank of $L$ is the rank of $A$.

\begin{proposition}
Let $A,B\in\F^{m\times n}$. Then
\[ \Col(A)=\Col(B) \iff \text{$\exists$ invertible $X$ such that $A = BX$}. \]
\end{proposition}
\begin{proof}
$\boxed{\Leftarrow}$ follows from \ref{rowColSubspaces}.

$\boxed{\Rightarrow}$ Let $C$ be a matrix whose columns form a basis of $\Col(A)=\Col(B)$. Then we can find matrices $S,T$ such that $CS = A$ and $CT = B$ are full-rank factorisations and these can be extended to invertible matrices by \ref{extendToInvertible}
\[ X_1 = \begin{bmatrix}
S \\ U
\end{bmatrix} \in \F^{n\times n} \qquad X_2 = \begin{bmatrix}
T \\ V
\end{bmatrix} \in \F^{n\times n}. \]
Now we claim $X = X_2^{-1}X_1$ fulfils the requirements:
\[ BX = (CT + 0V)X_2^{-1}X_1 = \begin{bmatrix}
C & \mathbb{0}
\end{bmatrix}X_2(X_2^{-1}X_1) = \begin{bmatrix}
C & \mathbb{0}
\end{bmatrix}X_1 = CS = A. \]
\end{proof}

\subsection{Null space}
\begin{definition}
Let $A\in \F^{m\times n}$ be a matrix. The \udef{null space} $\Null(A)$ of $A$ is the kernel of $\ell_A$. The dimension of $\Null(A)$ is called the \udef{nullity} of $A$.
\end{definition}
In other words:
\[ \Null(A) = \setbuilder{\vec{v}\in \F^n}{A\vec{v} = 0}. \]

\begin{proposition}
Let $A\in\F^{m\times n}$ be a matrix, then
\[ \Null(A) = \Col(A^*)^\perp. \]
\end{proposition}
\begin{proof}
$\vec{v}\in\Null(A) \iff A\vec{v} = 0 \iff \forall \vec{w}\in\F^n:\inner{A\vec{v},\vec{w}}=0 \iff \forall \vec{w}\in\F^n: \inner{\vec{v},A^*\vec{w}}=0 \iff \vec{v}\in\Col(A^*)^\perp$.
\end{proof}

\begin{lemma} \label{dimensionTheoremMatrices}
Let $A\in \F^{m\times n}$ be a matrix. Then
\[ \Rank(A) + \dim\Null(A) = n. \]
\end{lemma}
\begin{proof}
This is the dimension theorem applied to $\ell_A$, using $\im(\ell_A) = \Col(A)$.
\end{proof}

We can also formulate \ref{kernelCompositionLinearMaps} for matrices:
\begin{proposition} \label{nullSpaceProduct}
Let $A,B$ be conformal matrices. Then
\begin{enumerate}
\item $\Null(AB)\supseteq \Null(B)$;
\item $\dim\Null(AB) = \dim\Null(B) + \dim(\Col(B)\cap\Null(A))$.
\end{enumerate}
\end{proposition}
Note that (1) is the opposite inclusion to $\Col(AB)\subseteq \Col(A)$ and $\Row(AB)\subseteq \Row(B)$.
\begin{corollary}[Sylvester's law of nullity]
Let $A,B$ be square matrices. Then
\[ \max\{\dim\Null(A),\dim\Null(B)\} \leq \dim\Null(AB) \leq \dim\Null(A) + \dim\Null(B). \]
\end{corollary}

\subsection{The rank of a matrix}
\subsection{Rank equalities and inequalities}
\begin{proposition} \label{rankMultiplication}
Let $A,B,C,D$ be conformal matrices, then
\begin{enumerate}
\item $\Rank(AB) \leq \min\{\Rank(A),\Rank(B)\}$;
\item $\max\{\Rank(A),\Rank(C)\} \leq \Rank \begin{bmatrix}
A & C
\end{bmatrix}$;
\item $\max\{\Rank(A),\Rank(D)\} \leq \Rank \begin{bmatrix}
A \\ D
\end{bmatrix}$.
\end{enumerate}
\end{proposition}
\begin{proof}
(1) This is the matrix form of \ref{rankMapComposition}. It is also an immediate consequence of \ref{RowColSpaceProduct}.
\end{proof}

\begin{lemma}
Let $A\in\F^{m\times n}$, then
\[ \Rank(A) \leq \min\{m,n\}. \]
\end{lemma}
\begin{definition}
Let $A\in\F^{m\times n}$.
\begin{itemize}
\item If $\Rank(A) = \min\{m,n\}$, we say $A$ has \udef{full rank}.
\item If $\Rank(A) = m$, we say $A$ has \udef{full row rank}.
\item If $\Rank(A) = n$, we say $A$ has \udef{full column rank}.
\end{itemize}
\end{definition}

\begin{lemma}
Let $X,A,Y$ be conformal matrices. Then
\begin{enumerate}
\item if $X$ has full column rank, then $\Rank(A) = \Rank(XA)$;
\item if $Y$ has full row rank, then $\Rank(A) = \Rank(AY)$;
\item $A$ is invertible \textup{if and only if} it has full row rank and full column rank.
\end{enumerate}
\end{lemma}
\begin{proof}
(1) By \ref{dimensionTheoremMatrices}, $\Null(X) = \{0\}$. So $\vec{v}\in\Null(XA) \iff \vec{v}\in\Null(A)$, so $\Null(XA) = \Null(A)$. Then \ref{dimensionTheoremMatrices} implies $\Rank(XA) = \Rank(A)$.

(2) $\Rank(AY) = \Rank(Y^\transp A^\transp) = \Rank(A^\transp) = \Rank(A)$.

(3) Consequence of \ref{RowColSpaceInverse}.
\end{proof}

\begin{proposition}[Sylvester's rank inequality] \label{SylvesterRankInequality}
Let $A\in\F^{m\times k}$ and $B\in\F^{k\times n}$, then
\[ \Rank(AB) \geq \Rank(A)+\Rank(B) - k. \]
In addition, the following are equivalent:
\begin{enumerate}
\item $\Null(A)\subseteq \Col(B)$;
\item $\Rank(AB) = \Rank(A)+\Rank(B) - k$.
\end{enumerate}
In particular if $AB$ is a full-rank factorisation, i.e.\ $\Rank(AB) = k$, then (1) and (2) hold.
\end{proposition}
\begin{proof}
Let $AB = XY$ be a full-rank factorisation of $AB$ and set $r=\Rank(AB)$. Then define
\[ C = \begin{bmatrix}
A & X
\end{bmatrix}\in\F^{m\times(k+r)} \quad\text{and}\quad D=\begin{bmatrix}
B \\ -Y
\end{bmatrix}\in\F^{(k+r)\times n}\]
so $CD = \begin{bmatrix}
A & X
\end{bmatrix}\begin{bmatrix}
B \\ -Y
\end{bmatrix} = AB-XY = 0$.
This means that $\Col(D)\subseteq \Null(C)$, so $\Rank(D)\leq \dim\Null(C)$. Then
\[ \Rank(A) + \Rank(B) \leq \Rank(C) + \Rank(D) \leq \Rank(C) + \dim\Null(C) = k+\Rank(AB). \]
using \ref{dimensionTheoremMatrices} for $\Rank(C) + \dim\Null(C) = k+r$.

Now for the equivalent statements:

$\boxed{(1)\Rightarrow (2)}$ In this case \ref{nullSpaceProduct} becomes
\[ \dim\Null(AB) = \dim\Null(A) + \dim\Null(B). \]
Using \ref{dimensionTheoremMatrices}, this becomes
\[ n - \Rank(AB) = k-\Rank(A) + n-\Rank(B), \]
which can be arranged to give (2).

$\boxed{(2)\Rightarrow (1)}$ Assume, towards contraposition, $\Null(A) \nsubseteq \Col(B)$. Then we can find $\vec{v}\in\Null(A)$ such that $\vec{v}\notin \Col(B)$. Then
\[ \Rank \begin{bmatrix}
B & \vec{v}
\end{bmatrix} = \Rank(B) + 1 \qquad\text{and}\qquad \Rank(A \begin{bmatrix}
B & \vec{v}
\end{bmatrix}) = \Rank\begin{bmatrix}
AB & A\vec{v}
\end{bmatrix} = \Rank(AB). \]
And by the inequality
\[ \Rank(AB) \geq \Rank(A)+\Rank(B)+1-k, \]
so in particular $\Rank(AB) > \Rank(A)+\Rank(B)-k$ and thus $\Rank(AB) \neq \Rank(A)+\Rank(B)-k$.

Finally:

$\boxed{(\Rank(AB) = k)\Rightarrow (1)}$ From $\Rank(AB)\leq\Rank(A)\leq k$ and $\Rank(AB)\leq\Rank(B)\leq k$ we get
\[ \Rank(A) = \Rank(B) = \Rank(AB) = k \]
and so
\[ \Rank(A) + \Rank(B) - k = k =\Rank(AB). \]
\end{proof}
\begin{corollary}
Let $A\in\F^{m\times k}$ and $B\in\F^{k\times n}$ such that $\Null(A)\subseteq \Col(B)$, then
\[ AB = \mathbb{0}_{m\times n} \iff \Col(B)\subseteq \Null(A) \iff \Rank(A) + \Rank(B) = k \]
\end{corollary}

\begin{proposition}[Frobenius's rank inequality]
Let $A,B,C$ be conformal matrices, then
\[ \Rank(ABC) \geq \Rank(AB)+\Rank(BC) - \Rank(B). \]
\end{proposition}
\begin{proof}
Let $B = XY$ be a full-rank factorisation. Then
\begin{align*}
\Rank(ABC) &= \Rank(AXYC) \geq \Rank(AX) + \Rank(YC) - \Rank(B) \\
&\geq \Rank(AXY) + \Rank(XYC) - \Rank(B) = \Rank(AB)+\Rank(BC) - \Rank(B)
\end{align*}
using Sylvester's rank inequality \ref{SylvesterRankInequality}.
\end{proof}

\begin{proposition}
Let $A,B$ be conformal matrices. Then
\[ |\Rank(A)-\Rank(B)|\leq\Rank(A+B)\leq \Rank(A) + \Rank(B). \]
\end{proposition}
\begin{proof}
Let $A=X_1Y_1$ and $B=X_2Y_2$ be full-rank factorisations with $r = \Rank(A)$ and $s=\Rank(B)s$. Define
\[ C = \begin{bmatrix}
X_1 & X_2
\end{bmatrix}\qquad\text{and}\qquad \begin{bmatrix}
Y_1 \\ Y_2
\end{bmatrix}. \]
Then $CD = X_1Y_1 + X_2Y_2 = A+B$.

The second inequality follows from \ref{rankMultiplication}:
\[ \Rank(A+B) = \Rank(CD) \leq \min\{\Rank(C),\Rank(D)\} \leq r+s. \]

The first inequality follows from \ref{rankMultiplication}, which gives $\Rank(C)\geq\max\{r,s\}$ and $\Rank(D)\geq\max\{r,s\}$, Sylvester's rank inequality \ref{SylvesterRankInequality}, which gives
\[ \Rank(A+B) \geq \Rank(C) + \Rank(D) - (r+s) \geq r+r-(r+s) = r-s \]
and
\[ \Rank(A+B) \geq \Rank(C) + \Rank(D) - (r+s) \geq s+s-(r+s) = s-r. \]
These combine to give the first inequality.
\end{proof}

\begin{proposition}[Guttman rank additivity formula]
Let $M=\begin{pmatrix}
A & B \\ C & D
\end{pmatrix}$ and $A$ or $D$ invertible, then
\begin{align*}
\Rank(M) &= \Rank(D) + \Rank(A-BD^{-1}C) \\
&= \Rank(A) + \Rank(D-CA^{-1}B).
\end{align*}
\end{proposition}
\begin{proof}
This uses the Schur complement decomposition \ref{schurComplementLemma} and the fact that the transformation matrices are invertible.
\end{proof}

\subsubsection{The index of matrices}
\begin{proposition}
Let $A\in\F^{n\times n}$ be a square matrix. Then
\begin{enumerate}
\item $\Rank(A^k)\geq \Rank(A^{k+1})$ for all $k\in\N$;
\item if $\Rank(A^k) = \Rank(A^{k+1})$, then for all $p\in\N$
\[ \Rank(A^k) = \Rank(A^{k+p}) \qquad\text{and}\qquad \Col(A^k) = \Col(A^{k+p}); \]
\item there is a least integer $q\in [0,n]$ such that $\Rank(A^q) = \Rank(A^{q+1})$.
\end{enumerate}
\end{proposition}
These assertions remain true for exponent zero so long as we define $A^0 = \mathbb{1}_n$.

\begin{definition}
The \udef{index} of a square matrix is the least positive integer $q$ such that $\Rank(A^q) = \Rank(A^{q+1})$.
\end{definition}
In particular, invertible matrices have index zero.

\section{Matrix operations}
All functions on $\F$ can of course be extended component-wise to functions on $\F^{m\times n}$. In particular, let $\overline{A}$ be the component-wise complex conjugate of $A\in\F^{m\times n}$.
\subsection{Elementary row and column operations}
\begin{definition}
Let $A$ be an $(m\times n)$-matrix. The following are \udef{elementary row operations} (EROs):
\begin{itemize}[leftmargin=3cm]
\item[$\boxed{R_i \to \lambda R_i}$] Replacing a row by a non-zero multiple (i.e.\ $\lambda \neq 0$).
\item[$\boxed{R_i \leftrightarrow R_j}$] Swapping two rows.
\item[$\boxed{R_i \to R_i+ \lambda R_j}$] Adding a multiple of a row to a different row.
\end{itemize}
The \udef{elementary column operations} (ECO) are these operations applied to the columns. The are the operations given by transposing the matrix, applying an ERO and transposing again.

Two matrices $A,B$ are called \udef{row equivalent} if it is possible to obtain $B$ from $A$ by applying EROs. We write $A\sim_R B$ or just $A\sim B$.

Two matrices $A,B$ are called \udef{column equivalent} if it is possible to obtain $B$ from $A$ by applying ECOs. We write $A\sim_C B$.

Two matrices $A,B$ are called \udef{row+column equivalent} if it is possible to obtain $B$ from $A$ by applying EROs and ECOs. We write $A\sim_{R+C} B$.
\end{definition}
In general the theory will be developed for EROs. The relevant results for ECOs are obtained by transposition.

\begin{lemma}
The relations $\sim_R, \sim_C$ and $\sim_{R+C}$ are equivalence relations.
\end{lemma}

\begin{lemma} \label{matricesEROs}
Applying the elementary row operations to an $(n\times m)$-matrix is the same as multiplying from the left by the matrices obtained by applying the ERO to the $(n\times n)$-unit matrix:
\begin{itemize}[leftmargin=3cm]
\item[$\boxed{R_i \to \lambda R_i}$]
\[ \begin{pmatrix}
1 &&&&&& \\
 & \ddots &&&&& \\ 
  & & 1 &&&& \\
 & &  & \lambda &&& \\
 & &  & & 1 && \\
 &&&&& \ddots & \\
 &&&&&& 1
\end{pmatrix} \]
\item[$\boxed{R_i \leftrightarrow R_j}$]
\[ \begin{pmatrix}1&&&&&&\\&\ddots &&&&&\\&&0&&1&&\\&&&\ddots &&&\\&&1&&0&&\\&&&&&\ddots &\\&&&&&&1\end{pmatrix} \]
\item[$\boxed{R_i \to R_i+ \lambda R_j}$]
\[ \begin{pmatrix}1&&&&&&\\&\ddots &&&&&\\&&1&&&&\\&&&\ddots &&&\\&&m&&1&&\\&&&&&\ddots &\\&&&&&&1\end{pmatrix} \]
\end{itemize}
\end{lemma}
\begin{corollary}
ECOs can be performed by multiplying by an invertible matrix from the right.
\end{corollary}
\begin{corollary}
Let $A,B\in\F^{m\times n}$ be matrices.
\begin{enumerate}
\item If $A\sim_R B$, then $\Row(A) = \Row(B)$ and $\Null(A) = \Null(B)$.
\item If $A\sim_C B$, then $\Col(A) = \Col(B)$.
\end{enumerate}
\end{corollary}
\begin{corollary}
Let $A,B\in\F^{m\times n}$ be matrices such that $A\sim_{R+C} B$. Then $\Rank(A) = \Rank(B)$ and $\dim\Null(A) = \dim\Null(B)$.
\end{corollary}
\begin{proof}
The dimension of the null space is preserved because $\dim\Null(A) = n - \Rank(A)$, by \ref{dimensionTheoremMatrices}.
\end{proof}

\subsubsection{Gauss-Jordan elimination}
\begin{definition}
Let $A$ be an $(n\times m)$-matrix.
\begin{itemize}
\item The first non-zero element from the left on each row is called the \udef{leading coefficient} of \udef{pivot} of that row.
\item The matrix $A$ is in \udef{row echelon form} (ref) if
\begin{itemize}
\item rows of all zeros are at the bottom;
\item the leading coefficients of all other rows are one;
\item the leading coefficients of each non-zero row is strictly to the right of the leading coefficient of the row above it.
\end{itemize}
\item The matrix $A$ is in \udef{reduced row echelon form} (rref) if
\begin{itemize}
\item it is in echelon form;
\item all the coefficients in the column above a leading coefficient are zero.
\end{itemize}
\end{itemize}
\end{definition}

\begin{proposition}[Gauss-Jordan elimination]
Any matrix is row equivalent to a matrix in reduced echelon form.
\end{proposition}
The proof gives an algorithm for computing the reduced echelon form. Sometimes the part of the algorithm that brings the matrix to an (unreduced) echelon form is called Gaussian elimination. We first give an algorithm for Gaussian elimination and then use this for full Gauss-Jordan elimination.
\begin{proof}[Proof (Gaussian elimination)]
Let $A\in\F^{m\times n}$ be the input-matrix. The algorithm is recursive, we call it $\operatorname{ref}$.
\begin{enumerate}
\item If $A$ is empty (i.e.\ $m=0$ or $n=0$), then return $A$.
\item Find the entry in the first column of largest non-zero absolute value. Swap the corresponding row with the first row.
\begin{itemize}
\item If all the entries in the first column are zero, return
\[ \begin{pmatrix}
0 & \operatorname{ref}\left([A]_{1:m,2:n}\right)
\end{pmatrix}. \]
\item The algorithm also works if we choose any non-zero entry, not necessarily the largest.
\end{itemize}
\item Divide the first row by $[A]_{1,1}$.
\item Perform the ERO $R_i\to R_i - [A]_{i,1}R_1$ for each row $i\in 2:m$.
\item Return
\[ \begin{pmatrix}
1 & [A]_{1,2:n} \\
0 & \operatorname{ref}\left([A]_{2:m,2:n}\right)
\end{pmatrix}. \]
\end{enumerate}
The algorithm terminates because each subsequent call involves a matrix of strictly smaller dimensions.
\end{proof}
\begin{proof}[Proof (Obtaining the reduced echelon form)]
For each leading coefficient $[A]_{ij}$, perform the EROs $R_k\to R_k-[A]_{k,j}R_i$ for all rows $k\in 1:i-1$.
\end{proof}

\begin{corollary}
Let $A$ be any matrix. Then
\[ A \;\sim_{R+C}\; \begin{pmatrix}
\mathbb{1}_k & 0^{k\times p} \\
0^{q\times k} & 0^{q\times p}
\end{pmatrix}. \]
The row and column ranks are both equal to $k$ and the nullity is equal to $p$.

This also means we can write
\[ A = P\begin{pmatrix}
\mathbb{1}_k & 0^{k\times p} \\
0^{q\times k} & 0^{q\times p}
\end{pmatrix}Q \]
for some invertible matrices $P,Q$.
\end{corollary}
This factorisation of $A$ is called the \udef{rank normal form}.


We can use Gauss-Jordan elimination to find bases for $\Row(A), \Col(A)$ and $\Null(A)$:
\begin{itemize}
\item[$\boxed{\Row(A)}$] The row space is preserved by row equivalence. So we can perform Gauss-Jordan elimination, discard the null rows and the remaining rows will form a basis of $\Row(A)$.
\item[$\boxed{\Col(A)}$] Applying an ERO to a matrix $A$ is the same as multiplying from the left by some invertible matrix $E$. Due to \ref{multiplicationBlockMatrices},
\[ EA = \begin{pmatrix}
E[A]_{-,1} & \hdots & E[A]_{-,m}
\end{pmatrix}, \]
so $\Col(EA) = \ell_E[\Col(A)]$ and because $\ell_E$ is an isomorphism, we have $\Col(A) \cong \Col(EA)$.

In the echelon form it is easy to see that the columns containing leading elements form a basis. By applying the inverse map we see that the corresponding columns in the original matrix form a basis of the original column space $\Col(A)$, see \ref{mappingOfBasisByIsomorphism}.
\item[$\boxed{\Null(A)}$] The row space is preserved by row equivalence, so we first perform Gauss-Jordan elimination $A\sim_R U$. Then solve $Ux=0$ for $x$ (see later).
\end{itemize}

\begin{proposition}
Let $L:\F^n\to\F^m$ be a linear map. Then for any bases $\beta_n,\beta_m$ of $\F^n$ and $\F^m$, we have
\[ (L)_{\beta_n}^{\beta_m} \;\sim_{R+C}\; \begin{pmatrix}
\mathbb{1}_k & 0^{k\times q} \\
0^{p\times k} & 0^{p\times q}
\end{pmatrix} \]
and
\begin{enumerate}
\item $L$ is injective \textup{if and only if} $p=0$;
\item $L$ is surjective \textup{if and only if} $q=0$.
\end{enumerate}
\end{proposition}


\subsubsection{Calculating inverse matrices}
Any invertible square matrix $A$ is row equivalent to $\mathbb{1}_n$. We want to find the matrix associated to this row reduction. One way to keep track of this matrix is to simultaneously apply the EROs to $A$ and $\mathbb{1}_n$:
\begin{lemma}
Let $A\in\F^{n\times n}$ be a square matrix. Then
\[ \left(\begin{array}{c|c} A & \mathbb{1}_n \end{array}\right) \sim_R \left(\begin{array}{c|c} \mathbb{1}_n & A^{-1} \end{array}\right). \]
\end{lemma}
\begin{proof}
$A^{-1}\begin{pmatrix}
A & \mathbb{1_n}
\end{pmatrix} = \begin{pmatrix}
\mathbb{1}_n & A^{-1}
\end{pmatrix}$.
\end{proof}


\subsection{The trace}
\begin{definition}
Let $A\in \mathbb{F}^{n\times n}$ be a square matrix. The \udef{trace} of $A$, denoted $\Tr(A)$, is the sum of the diagonal entries of $A$:
\[ \Tr(A) = \sum_{i=1}^n (A)_{i,i}. \]
\end{definition}
\begin{proposition}
Let $A\in \mathbb{F}^{n\times n}$ be a square matrix, then
\begin{enumerate}
\item $\Tr(cA+B) = c\Tr(A) + \Tr(B)$ for all $c\in\F$;
\item $\Tr(\overline{A}) = \overline{\Tr(A)}$;
\item $\Tr(A^\transp) = \Tr(A)$;
\item $\Tr(A^*) = \overline{\Tr(A)}$.
\end{enumerate}
\end{proposition}
\begin{proposition}
Let $A,B,C \in\F^{n\times n}$ be square matrices. Then
\begin{enumerate}
\item $\Tr(AB) = \sum_{i,j=1}^n [A]_{i,j}[B]_{j,i}$;
\item $\Tr(AB) = \Tr(BA)$;
\item $\Tr(ABC) = \Tr(CAB) = \Tr(BCA)$.
\end{enumerate}
\end{proposition}
\begin{proof}
Point (1) follows by direct computation
\[ \Tr(AB) = \sum_{i=1}^n (AB)_{i,i} = \sum_{i=1}^n\sum_{j=1}^n A_{i,j}B_{j,i}. \]

(2) We use (1) and the fact that $[A]_{i,j}[B]_{j,i} = [B]_{i,j}[A]_{j,i}$.

(3) Follows straight from (2): $\Tr\big((AB)C\big) = \Tr\big(C(AB)\big)$ etc.
\end{proof}
The trace of a product of matrices is invariant under any cyclic permutation of the matrices.

The trace of a product of matrices is not invariant under noncyclic permutation of the matrices:
\[ \Tr(ABC) \neq \Tr(BAC). \]



\begin{lemma} \label{averageTraceOverDiagonal}
Let $A\in\F^{n\times n}$. Then $A$ is similar to a matrix $B$ with
\[ [B]_{i,i} = \frac{1}{n}\Tr(A) \qquad \forall i\leq n. \]
\end{lemma}
\begin{proof}
It is enough to show that this holds for matrices with zero trace: if $A$ is any matrix, then $A- \frac{\Tr(A)}{n}\mathbb{1}$ is a matrix with zero trace. If this is similar to a matrix $S(A- \frac{\Tr(A)}{n}\mathbb{1})S^{-1}$ with zeros on the diagonal, then $SAS^{-1}$ has $\frac{\Tr(A)}{n}$ on the diagonal.

So assume WLOG that $\Tr(A) = 0$ and $A\neq 0$. We can find $\vec{v}\in\F^n$ such that $\{\vec{v}, A\vec{v}\}$ is linearly independent: assume, towards contraposition, that $A\vec{v}$ is a scalar multiple of $\vec{v}$, for all $\vec{v}$. This must be the same multiple $\lambda$ for all $\vec{v}$. If not, i.e.\ there exist $\vec{v},\vec{w}$ such that $A\vec{v} = \lambda_1 \vec{v}$ and $A\vec{w} = \lambda_2 \vec{w}$ with $\lambda_1\neq \lambda_2$, then $\vec{v}+\vec{w}$ is not mapped to a multiple. In this case $A$ is $\lambda \id$, but $\Tr(A) = 0$ fixes $\lambda=0$, which we excluded.

Now we can extend $\{\vec{v}, A\vec{v}\}$ to a basis of $\F^n$ and put these as columns in a matrix $S = \begin{bmatrix}
\vec{v} & A\vec{v} & S_1
\end{bmatrix}$. We can partition $S^{-1} = \begin{bmatrix}
\vec{x}^\transp \\ S_2
\end{bmatrix}$. Then we have
\[ S^{-1}S = \begin{bmatrix}
\vec{x}^\transp \vec{v} & \vec{x}^\transp A\vec{v} & \vec{x}^\transp S_1 \\ S_2\vec{v} & S_2A\vec{v} & S_2S_1
\end{bmatrix} = \mathbb{1}. \]
In particular $\vec{x}^\transp A\vec{v} = 0$. Then
\[ S^{-1}AS = \begin{bmatrix}
\vec{x}^\transp A\vec{v} & \vec{x}^\transp A^2\vec{v} & \vec{x}^\transp A S_1 \\
S_2A\vec{v} & S_2A^2\vec{v} & S_2AS_1
\end{bmatrix}  = \begin{bmatrix}
0 & \star \\ \star & S_2AS_1
\end{bmatrix}. \]
Now $S_2AS_1$ has trace zero and we can repeat the argument. By induction we can make all elements on the diagonal zero.
\end{proof}



\begin{proposition}
The trace of a matrix of a linear map is independent of the choice of basis:
\[ \Tr(L)_{\beta}^{\beta} = \Tr(L)_{\beta'}^{\beta'}. \]
\end{proposition}
\begin{proof}
\[ \Tr(L)_{\beta'}^{\beta'} = \Tr\left[((I)_{\beta'}^{\beta})^{-1}(L)_\beta^\beta(I)_{\beta'}^{\beta}\right] = \Tr\left[(L)_\beta^\beta(I)_{\beta'}^{\beta}((I)_{\beta'}^{\beta})^{-1}\right] = \Tr(L)_\beta^\beta \]
\end{proof}
This allows us to make the following definition:
\begin{definition}
The \udef{trace} of a linear map on a finite-dimensional vector space is the trace of any matrix representation of that map.
\end{definition}
\subsection{The determinant}
\begin{definition}
A map
\[ f: \mathbb{F}^{n\times n}\to \mathbb{F} \]
with the following properties
\begin{itemize}
\item[\textbf{D-1}] $f(\mathbb{1}_n) = 1$;
\item[\textbf{D-2}] $f(A)$ changes sign if two rows in $A$ are swapped;
\item[\textbf{D-3}] $f$ is linear in the first row:
\[ f(\begin{pmatrix}
\lambda A_{1,-} + \mu A'_{1,-} \\ A_{2,-} \\ \vdots \\ A_{n,-} 
\end{pmatrix}) = \lambda f(\begin{pmatrix}
A_{1,-} \\ A_{2,-} \\ \vdots \\ A_{n,-} 
\end{pmatrix}) + \mu f(\begin{pmatrix}
A'_{1,-} \\ A_{2,-} \\ \vdots \\ A_{n,-} 
\end{pmatrix}); \]
\end{itemize}
is called a \udef{determinant map}.
\end{definition}
We will show that there exists one and only one determinant map. We are therefore justified in calling it the determinant.

The determinant of $A$ is often denoted $\det(A)$ or $|A|$.

\begin{lemma}
Let $f: \mathbb{F}^{n\times n}\to \mathbb{F}$ be a determinant map.
\begin{enumerate}
\item $f$ is linear in each row;
\item if a matrix $A$ has a row of zeros, or two identical rows, then $f(A) = 0$;
\item the ERO $R_i \to R_i+ \lambda R_j$ does not change the determinant;
\item the ERO $R_i \leftrightarrow R_j$ changes the sign of the determinant;
\item the ERO $R_i \to \lambda R_i$ multiplies the determinant by $\lambda$;
\item $f(A)$ is non-zero \textup{if and only if} $A\sim_R \mathbb{1}_n$;
\item $f(A)$ is non-zero \textup{if and only if} $A$ is invertible.
\end{enumerate}
\end{lemma}

\subsubsection{Leibniz formula}
\begin{proposition}[Liebniz formula]
There is one and only one determinant map. It is given by
\[ \det: \F^{n\times n}\to \F: A\mapsto \sum_{\sigma\in S^n}\sgn(\sigma)\prod_{i=1}^n[A]_{i,\sigma(i)}. \]
\end{proposition}
\begin{proof}
Let $f$ be a determinant map and $\mathcal{E} = \{\vec{e}_i\}_{i=1}^n$ the standard basis of $\F^n$, considered as rows and $A\in\F^{n\times n}$. Then
\[ f(A) = f(\begin{pmatrix}
\sum_{j_1=1}^n [A]_{1j_1}\vec{e}_{j_1} \\ \vdots \\ \sum_{j_n=1}^n [A]_{nj_n}\vec{e}_{j_n}
\end{pmatrix}) = \sum_{j_1,\ldots, j_n=1}[A]_{1j_1}\ldots[A]_{nj_n}f(\begin{pmatrix}
\vec{e}_{j_1} \\ \vdots \\ \vec{e}_{j_n}
\end{pmatrix}). \]
Now $f(\begin{pmatrix}
\vec{e}_{j_1} \\ \vdots \\ \vec{e}_{j_n}
\end{pmatrix})$ is only non-zero if all $j_i$ are different, i.e.\ $j_i = \sigma(i)$ for some permutation $\sigma\in S^n$. Also
\[ f(\begin{pmatrix}
\vec{e}_{\sigma(1)} \\ \vdots \\ \vec{e}_{\sigma(n)}
\end{pmatrix}) = \sgn(\sigma)f(\mathbb{1}_n) = \sgn(\sigma). \]
So the only possible candidate for a determinant map is
\[ f(A) = \sum_{\sigma\in S^n}\sgn(\sigma)\prod_{i=1}^n[A]_{i,\sigma(i)}. \]
This satisfies \textbf{D-1}, \textbf{D-2} and \textbf{D-3}.
\end{proof}
In the case of $3\times 3$ the Leibniz formula reduces to
\[ \begin{vmatrix}
a_{11} & a_{12} & a_{13} \\
a_{21} & a_{22} & a_{23} \\
a_{31} & a_{32} & a_{33}
\end{vmatrix} = a_{11}a_{22}a_{33} + a_{12}a_{23}a_{31} + a_{13}a_{21}a_{32} - a_{31}a_{22}a_{13} - a_{32}a_{23}a_{11} - a_{33}a_{23}a_{12}. \]
The rule of Sarrus is the following mnemonic:
\begin{corollary}[Rule of Sarrus]
The determinant of a $3\times 3$ matrix can be seen as the sum of the following diagonals (with the correct sign):
\[ \begin{tikzpicture}
\node (A) {$\begin{vmatrix}
a_{11} & a_{12} & a_{13} \\
a_{21} & a_{22} & a_{23} \\
a_{31} & a_{32} & a_{33}
\end{vmatrix}$};
\node[right of=A, node distance=5.8em] (B) {$\begin{matrix}
a_{11} & a_{12} \\
a_{21} & a_{22} \\
a_{31} & a_{32}
\end{matrix}$};
\draw (A.south west) -- ++(7.5em,4em) ++(.2,.1) node {--};
\draw (A.south west) ++(2em,0) --  ++(7.5em,4em) ++(.2,.1) node {--};
\draw (A.south west) ++(4em,0) --  ++(7.5em,4em) ++(.2,.1) node {--};
\draw (A.north west) -- ++(7.5em,-4em) ++(.2,-.1) node {+};
\draw (A.north west) ++(2em,0) --  ++(7.5em,-4em) ++(.2,-.1) node {+};
\draw (A.north west) ++(4em,0) --  ++(7.5em,-4em) ++(.2,-.1) node {+};
\end{tikzpicture} \]
\end{corollary}

\subsubsection{Laplace expansion}
\begin{definition}
Let $A\in\F^{n\times n}$ be a square matrix. A \udef{minor determinant} or \udef{minor} is the determinant of a submatrix. In particular the \udef{$(i,j)$-minor} $M_{i,j}$ is the minor
\[ M_{i,j} \defeq \det([A]_{(1:n)\setminus\{i\},(1:n)\setminus\{j\}}). \]

The \udef{$(i,j)$-cofactor} $C_{i,j}$ is defined as
\[ C_{i,j} \defeq (-1)^{i+j}M_{i,j}. \]
\end{definition}
\begin{proposition}[Laplace expansion] \label{LaplaceExpansion}
Let $A\in\F^{n\times n}$. Then
\[ \det(A) = \sum_{i=1}^n [A]_{i,j}C_{i,j} = \sum_{i=1}^n [A]_{j,i}C_{j,i} \]
for all $j\in 1:n$.
\end{proposition}
This is also known as ``expansion along the $j^\text{th}$ column'', resp., row.
\begin{proof}
Fix some $j\in 1:n$. Then we can calculate
\begin{align*}
\det(A) &=  \sum_{\sigma\in S^n}\sgn(\sigma)\prod_{k=1}^n[A]_{k,\sigma(k)} = \sum_{i=1}^n\sum_{\substack{\sigma\in S^n \\ \sigma(i)=j}}\sgn(\sigma)\prod_{k=1}^n[A]_{k,\sigma(k)} \\
&= \sum_{i=1}^n [A]_{i,j}\sum_{\substack{\sigma\in S^n \\ \sigma(i)=j}}\sgn(\sigma)\prod_{k\in (1:n)\setminus\{j\}}^n[A]_{k,\sigma(k)} = \sum_{i=1}^n [A]_{i,j}C_{i,j}.
\end{align*}
The expression for column expansion can be obtained by transposition.
\end{proof}

\subsubsection{Volume}

\subsubsection{Properties}
\begin{lemma}
Let $A\in\F^{n\times n}$. Then
\[ \det(A^\transp) = \det(A). \]
\end{lemma}
\begin{proof}
We calculate using the Leibniz formula:
\begin{align*}
\det(A^\transp) &= \sum_{\sigma\in S^n}\sgn(\sigma)\prod_{i=1}^n[A^\transp]_{i,\sigma(i)} = \sum_{\sigma\in S^n}\sgn(\sigma)\prod_{i=1}^n[A]_{\sigma(i),i} \\
&= \sum_{\sigma\in S^n}\sgn(\sigma)\prod_{i=1}^n[A]_{i,\sigma^{-1}(i)} = \sum_{\sigma\in S^n}\sgn(\sigma^{-1})\prod_{i=1}^n[A]_{i,\sigma^{-1}(i)} = \det(A).
\end{align*}
\end{proof}

\begin{lemma}
Let $A\in\F^{n\times n}$. Then
\begin{enumerate}
\item $\det(\lambda A) = \lambda^n\det(A)$;
\item $\det(\overline{A}) = \overline{\det(A)}$;
\item $\det(A^*) = \overline{\det(A)}$;
\item if $A$ is triangular, then
\[ \det(A) = \prod_{i=1}^n [A]_{i,i}. \]
\end{enumerate}
\end{lemma}

\begin{proposition}
Let $A,B\in\F^{n\times n}$. Then
\[ \det(AB) = \det(A)\det(B). \]
\end{proposition}
\begin{proof}
First assume $\det(B) = 0$. Then $AB$ is not invertible, for if $AB$ were invertible, $B$ would have inverse $(AB)^{-1}A$. So $\det(AB) = 0$ and the formula holds.

Now assume $\det(B) \neq 0$, then $A\mapsto \det(AB)/\det(B)$ satisfies the definition of a determinant map and thus is equal to $\det$.
\end{proof}
\begin{corollary}
If $U$ is unitary, then $|\det(U)| = 1$.
\end{corollary}
\begin{proof}
$1=\det{\mathbb{1}} = \det(U^*U) = \det(U^*)\det(U) = \overline{\det(U)}\det(U) = |\det(U)|$.
\end{proof}
\begin{corollary}
Let $A,B\in\F^{n\times n}$. Then
\[ \det(AB) = \det(BA). \]
In fact we can arbitrarily commute any matrix product inside the determinant.
\end{corollary}
\begin{corollary}
The determinant of a matrix of a linear map is independent of the choice of basis:
\[ \det(L)_{\beta}^{\beta} = \det(L)_{\beta'}^{\beta'}. \]
\end{corollary}
This allows us to make the following definition:
\begin{definition}
The \udef{determinant} of a linear map on a finite-dimensional vector space is the determinant of any matrix representation of that map.
\end{definition}

\begin{lemma}
Let $A\in\F^{m\times m}$ and $n\in\N$. Then
\[ \det\begin{bmatrix}
\mathbb{1}_n & \mathbb{0} \\ \mathbb{0} & A
\end{bmatrix} = \det(A) = \det\begin{bmatrix}
A & \mathbb{0} \\ \mathbb{0} & \mathbb{1}_n
\end{bmatrix}. \]
\end{lemma}
\begin{proof}
By induction on $n$.
\end{proof}

\begin{lemma}
Let $A,B,C$ be conformal matrices and $A,D$ square. Then
\[ \det\begin{bmatrix}A & B \\ \mathbb{0} & D \end{bmatrix} = \det(A)\det(D). \]
\end{lemma}
\begin{proof}
This follows from
\[ \begin{bmatrix}A & B \\ \mathbb{0} & D \end{bmatrix} = \begin{bmatrix}\mathbb{1} & \mathbb{0} \\ \mathbb{0} & D \end{bmatrix} \begin{bmatrix}\mathbb{1} & B \\ \mathbb{0} & \mathbb{1} \end{bmatrix} \begin{bmatrix}A & \mathbb{0} \\ \mathbb{0} & \mathbb{1} \end{bmatrix}, \]
the product rule, the previous lemma and the fact that the central matrix is triangular with only ones on the diagonal.
\end{proof}

\begin{lemma} \label{blockDeterminant}
Let $M = \begin{pmatrix}
A & B \\ C & D
\end{pmatrix}$ be a partitioned matrix. Then
\begin{align*}
\det(M) &= \det(A)\det(D-CA^{-1}B) \\
&= \det(D)\det(A-BD^{-1}C)
\end{align*}
if $A$, resp. $D$, is invertible.
\end{lemma}
\begin{proof}
This follows straight from the Schur complement.
\end{proof}
\begin{corollary}[Cauchy expansion of the determinant]
Let $A\in\F^{n\times n}, \vec{x},\vec{y}\in\F^n$ and $c\in \F$. Then
\[ \det \begin{bmatrix}
c & \vec{x}^\transp \\ \vec{y} & A
\end{bmatrix} = (c-\vec{x}^\transp A^{-1}\vec{y})\det(A) = c\det(A) - \vec{x}^\transp \adj(A) \vec{y}. \]
\end{corollary}
\begin{corollary}
Let $A,B,C,D\in\F^{n\times n}$.
\begin{enumerate}
\item If $A$ or $D$ is invertible and commutes with $B$, then
\[ \det\begin{bmatrix}
A & B \\ C & D
\end{bmatrix} = \det(DA-CB). \]
\item If $A$ or $D$ is invertible and commutes with $C$, then
\[ \det\begin{bmatrix}
A & B \\ C & D
\end{bmatrix} = \det(AD-BC). \]
\end{enumerate}
\end{corollary}

\begin{lemma}
Let $a,b\in\R$. Then
\[ \det(a\mathbb{1}_n+b\mathbb{J}_n) = a^{n-1}(a+nb). \]
\end{lemma}
\begin{proof}
We can partition $a\mathbb{1}_n+b\mathbb{J}_n$ and use the ERO $R_i\to R_i-R_1$ for $1<i\leq n$ to obtain:
\[ a\mathbb{1}_n+b\mathbb{J}_n = \begin{pmatrix}
a+b & b\mathbb{J}^{1\times (n-1)} \\
b\mathbb{J}^{(n-1)\times 1} & a\mathbb{1}_{n-1}+b\mathbb{J}_{n-1}
\end{pmatrix} = \begin{pmatrix}
a+b & b\mathbb{J}^{1\times (n-1)} \\
-a\mathbb{J}^{(n-1)\times 1} & a\mathbb{1}_{n-1}
\end{pmatrix}. \]
Then by \ref{blockDeterminant}, we have
\[ \det(a\mathbb{1}_n+b\mathbb{J}_n) = a^{n-1}(a+b +b\mathbb{J}^{1\times (n-1)}\mathbb{J}^{(n-1)\times 1}) = a^{n-1}(a+nb). \]
\end{proof}

\begin{lemma}[Weinstein-Aronszajn identity]
Let $A\in\F^{m\times n}$ and $B\in\F^{n\times m}$. Then
\[ \det(\mathbb{1}_{m} + AB) = \det(\mathbb{1}_{n} + BA). \]
Also, for any $\lambda\in\R_0$,
\[ \det(AB - \lambda\mathbb{1}_{m}) = (-\lambda)^{m-n}\det(BA - \lambda\mathbb{1}_{n}). \]
\end{lemma}
This is also sometimes referred to as the Sylvester determinant identity.
\begin{proof}
Applying the two equalities in \ref{blockDeterminant} to the matrix
\[ M = \begin{bmatrix}
\mathbb{1}_m & -A \\
B & \mathbb{1}_n
\end{bmatrix} \]
give
\begin{align*}
M &= \det(\mathbb{1}_m)\det(\mathbb{1}_n - B\mathbb{1}_m^{-1}(-A)) = \det(\mathbb{1}_n+BA) \\
&= \det(\mathbb{1}_n)\det(\mathbb{1}_m - (-A)\mathbb{1}_n^{-1}B) = \det(\mathbb{1}_m+AB).
\end{align*}
\end{proof}
\begin{corollary}[Matrix determinant lemma]
Let $A\in\F^{n\times n}$ and $\vec{u},\vec{v}\in\F^n$. Then
\begin{align*}
\det(A+\vec{u}\vec{v}^\transp) &= \det(A)\det(\mathbb{1}_n + A^{-1}\vec{u}\vec{v}^\transp) \\
&= \det(A)(\mathbb{1}_n + \vec{v}^\transp A^{-1}\vec{u}) \\
&= \det(A) + \vec{v}^\transp \adj(A)\vec{u},
\end{align*}
which is interesting because $\vec{v}^\transp A^{-1}\vec{u} \in\F$  and $\vec{v}^\transp \adj(A)\vec{u}\in\F$ are scalars.
\end{corollary}

TODO
\[ \log\det M=\Tr\log M \]


\subsection{Adjugate}
\begin{definition}
Let $A\in\F^{n\times n}$ be a square matrix. The \udef{adjugate matrix} or \udef{classical adjoint} $\adj(A)$ is the transposed cofactor matrix:
\[ [\adj(A)]_{ij} = C_{ji} \]
where $C_{ij}$ is the $(i,j)-$cofactor.
\end{definition}

\begin{lemma}
Let $A\in\F^{n\times n}$, $\vec{b}\in \F^n$ and $k\in(1:n)$. Then
\[ \det(\left[\begin{cases}
[A]_{ij} & (j\neq k) \\
[\vec{b}]_i & (j=k)
\end{cases}\right]) = [\adj(A)b]_k \]
\end{lemma}
\begin{proof}
We calculate
\[ [\adj(A)\vec{b}]_k = \sum_l [\adj(A)]_{kl}[\vec{b}]_l = \sum_l C_{lk}[\vec{b}]_l. \]
Now in the definition of $C_{lk}$, the $k^\text{th}$ column is excluded. So the $(l,k)-$cofactor of $A$ is the same as the $(l,k)-$cofactor of 
\[ \left[\begin{cases}
[A]_{ij} & (j\neq k) \\
[\vec{b}]_i & (j=k)
\end{cases}\right] \]
which is the matrix where the $k^\text{th}$ column of $A$ is replaced by $\vec{b}$. Then $\sum_l C_{lk}[\vec{b}]_l$ is the determinant of this matrix by \ref{LaplaceExpansion}.
\end{proof}

\begin{proposition} \label{adjunctDeterminant}
Let $A\in\F^{n\times n}$. Then
\[ A\cdot\adj(A) = \adj(A)\cdot A = \det(A) \mathbb{1}_n. \]
\end{proposition}
\begin{proof}
\[ [\adj(A)\cdot A]_{ij} = \sum_k [A]_{ik}C_{jk} \]
Clearly if $i=j$, we have the Laplace expansion \ref{LaplaceExpansion} and the expression equals $\det(A)$. If $i\neq j$, then the $j^\text{th}$ row does not enter into the expression (it is left out of the $(i,j)-$minor) and thus may just as well be replaced by a copy of the $i^\text{th}$ row. In this case we get the expression for the determinant of a matrix with two identical rows. This must be $0$.
\end{proof}
\begin{corollary}
Let $A\in\F^{n\times n}$ be invertible. Then
\[ A^{-1} = \frac{1}{\det(A)}\adj(A). \]
\end{corollary}

\begin{proposition}
Let $A,B\in\F^{n\times n}$ be square matrices. Then
\begin{enumerate}
\item $\adj(\mathbb{0}_n) = \mathbb{0}_n$;
\item $\adj(\mathbb{1}_n) = \mathbb{1}_n$;
\item $\adj(\lambda A) = \lambda^{n-1}\adj(A)$ for any $\lambda\in \F$;
\item $\adj(A^\transp) = \adj(A)^\transp$;
\item $\det(\adj(A)) = \det(A)^{n-1}$;
\item $\adj(AB) = \adj(A)\adj(B)$.
\end{enumerate}
\end{proposition}
\begin{proof}
Points 2. and 5. are direct consequences of \ref{adjunctDeterminant}. TODO rest.
\end{proof}
Cauchy-Binet

\subsection{Generalised inverses or pseudoinverses}
\subsubsection{Moore-Penrose pseudoinverse}

\subsection{Pfaffian}

\subsection{Vectorisation}
For calculations it is often useful to put the matrix of coordinates into a column vector. The process of fitting a matrix into a column vector is known as the \udef{vectorisation} of a matrix. Two obvious ways to do this are by going row-by-row or column by column.
\begin{itemize}
\item Column-by-column we get
\[ \vectorisation_C: \R^{m\times n}\to \R^{mn\times 1}: A \mapsto \vectorisation_C(A) = [a_{1,1},\ldots,a_{m,1}, a_{1,2},\ldots, a_{m,2}, \;\; \ldots \;\; , a_{1,n}, \ldots, a_{m,n}]^\transp \]
\begin{example}
\[ \text{If} \qquad A = \begin{pmatrix}
a & b \\ c & d
\end{pmatrix}, \qquad \text{then} \qquad \vectorisation_C(A) = \begin{pmatrix}
a \\ c \\ b \\ d
\end{pmatrix}. \]
\end{example}
\item Row-by-row we get
\[ \vectorisation_R: \R^{m\times n}\to \R^{mn\times 1}: A \mapsto \vectorisation_R(A) = [a_{1,1},\ldots,a_{1,n}, a_{2,1},\ldots, a_{2,n}, \;\; \ldots \;\; , a_{m,1}, \ldots, a_{m,n}]^\transp \]
\begin{example}
\[ \text{If} \qquad A = \begin{pmatrix}
a & b \\ c & d
\end{pmatrix}, \qquad \text{then} \qquad \vectorisation_R(A) = \begin{pmatrix}
a \\ b \\ c \\ d
\end{pmatrix}. \]
\end{example}
\end{itemize}
Obviously these are related by
\[ \vectorisation_C(A) = \vectorisation_R(A^\transp) \]

Vectorisation is a self-adjunction in the monoidal closed structure of any category of matrices. (TODO)
\subsection{The Hadamard product}
\[ \vectorisation(A\circ B) = \vectorisation(A) \circ \vectorisation(B) \]
This works for both $\vectorisation_C$ and $\vectorisation_R$.
\subsection{The outer product}
Given two column vectors $\vec{u} = [u_1 \hdots u_m]^\transp$ and $\vec{v} = [v_1 \hdots v_n]^\transp$, the \udef{outer product} is defined by
\[ \vec{u}\otimes \vec{v} = \vec{u}\vec{v}^\transp = \begin{bmatrix}
u_1 \\ \vdots \\ u_m
\end{bmatrix} \begin{bmatrix}
v_1 & \hdots & v_n
\end{bmatrix} = \begin{bmatrix}
u_1v_1 & \hdots & u_1v_n \\
\vdots & \ddots & \vdots \\
u_mv_1 & \hdots & u_m v_n
\end{bmatrix} \]
\subsection{The Kronecker product}
Given two matrices $A,B$ of dimensions $m\times n$ and $p\times q$, the \udef{Kronecker product} yields a $(pm\times qn)$-matrix:
\[ A\otimes B = \begin{bmatrix}
a_{1,1}B & \hdots & a_{1,n}B \\
\vdots & \ddots & \vdots \\
a_{m,1}B & \hdots & a_{m,n}B
\end{bmatrix} \]

The same symbol is used as for the outer product, because the Kronecker product can be seen as a generalisation of the outer product. Indeed for column vectors $\vec{u}, \vec{v}$ we have the identities
\begin{align}
\vec{u}\otimes_{\text{Kron}} \vec{v} &= \vectorisation_R(\vec{u}\otimes_\text{outer} \vec{v}) \\
&= \vectorisation_C(\vec{v}\otimes_\text{outer} \vec{u})
\end{align}
and \[ \vec{u}\otimes_\text{outer}\vec{v} = \vec{u}\otimes_{\text{Kron}} \vec{v}^\transp \]

In terms of matrix elements we can write, assuming the indices start at zero
\[ (A\otimes B)_{i,j} = a_{\lfloor{i/p}\rfloor, \lfloor{j/q}\rfloor}b_{i\%p,j\%q}. \]
where $\%$ denotes the remainder. For indices starting from 1, we have
\[ (A\otimes B)_{i,j} = a_{\lceil{i/p}\rceil, \lceil{j/q}\rceil}b_{(i-1)\%p+1,(j-1)\%q+1}. \]

\subsubsection{Properties}
The Kronecker product is bilinear and associative.
\begin{itemize}
\item[\textbf{Transpose}]
\[ (A\otimes B)^\transp = A^\transp \otimes B^\transp \]
\item[\textbf{Determinant}]
\[ \det(A\otimes B) = \det(A)^m\det(B)^n \]
if $A$ is an $n\times n$ matrix and $B$ an $m\times m$ matrix.
\item[\textbf{Trace}]
\[ \Tr(A\otimes B) = \Tr(A)\Tr(B) \]
\item[\textbf{Mixed product}]
Let $A,B,C,D$ be conformal matrices, then
\[ (A\otimes B)(C \otimes D) = (AC)\otimes (BD). \]
The proof is as follows:
\begin{align}
(A\otimes B)(C \otimes D) &= \begin{bmatrix}
a_{1,1}B & \hdots & a_{1,n}B \\
\vdots & \ddots & \vdots \\
a_{m,1}B & \hdots & a_{m,n}B
\end{bmatrix}\begin{bmatrix}
c_{1,1}D & \hdots & c_{1,n}D \\
\vdots & \ddots & \vdots \\
c_{m,1}D & \hdots & c_{m,n}D
\end{bmatrix} \\
&= \begin{bmatrix}
\sum_k a_{1,k}c_{k,1} BD & \hdots & \sum_k a_{1,k}c_{k,p} BD \\
\vdots & \ddots & \vdots \\
\sum_k a_{m,k}c_{k,1} BD & \hdots & \sum_k a_{m,k}c_{k,p} BD
\end{bmatrix}  =  \begin{bmatrix}
(AC)_{1,1} BD & \hdots & (AC)_{1,p} BD \\
\vdots & \ddots & \vdots \\
(AC)_{m,1} BD & \hdots & (AC)_{m,p} BD
\end{bmatrix}\\
&= (AC)\otimes (BD)
\end{align}
Using the fact that multiplication of two block matrices can be carried out as if their blocks were scalars.

As an immediate consequence:
\[ A\otimes B = (I_n\otimes B)(A\otimes I_k) = (A\otimes I_k)(I_n \otimes B). \]

\item[\textbf{Inverse}]
The product $A\otimes B$ is invertible iff $A$ and $B$ are invertible. In that case the inverse is given by
\[ (A\otimes B)^{-1} = A^{-1}\otimes B^{-1}. \]
This follows easily from the mixed product.
\item[\textbf{Moore-Penrose pseudoinverse}]
\[ (A\otimes B)^+ = A^+\otimes B^+. \]

\item[\textbf{A vectorisation trick}]
Let $A,B,C$ be matrices of dimensions $k\times l, l\times m$ and $m\times n$. Then
\[ \vectorisation(ABC) = (C^\transp\otimes A)\vectorisation(B). \]
From this we obtain some other formulations:
\begin{align}
\vectorisation(ABC) &= (I_n\otimes AB)\vectorisation(C) \\
&= (C^\transp B^\transp \otimes I_k)\vectorisation(A) \\
\vectorisation(AB) &= (I_m\otimes A)\vectorisation(B) = (B^\transp \otimes I_k)\vectorisation(A)
\end{align}
\end{itemize}

\subsection{The commutator}
\begin{theorem}[Shoda's theorem]
Let $A\in\F^{n\times n}$. Then there exists $X,Y$ such that $A=[X,Y] = XY-YX$ \textup{if and only if} $\Tr(A) = 0$.
\end{theorem}
\begin{proof}
Assume $A = [X,Y]$. Then $\Tr(A) = \Tr(XY) -\Tr(YX) = \Tr(XY) - \Tr(XY) = 0$.

Conversely, assume $\Tr(A) = 0$. By \ref{averageTraceOverDiagonal} $A$ is similar to a matrix $B$ that has zeros on the diagonal. Let $X = \diag(1,2,\ldots, n)$ and
\[ [Y]_{i,j} = \begin{cases}
(i-j)^{-1}[B]_{i,j} & (i\neq j) \\
1 & (i=j).
\end{cases} \]
Then
\[ [X,Y]_{i,j} = [XY-YX]_{i,j} = i[Y]_{i,j}-j[Y]_{i,j} = (i-j)[Y]_{i,j} = [B]_{i,j}. \]
So $B$ is the commutator $[X,Y]$. Then $A$ is the commutator $[SXS^{-1}, SYS^{-1}]$.
\end{proof}

\subsection{Kruskal rank and spark}
\begin{definition}
Let $A\in \F^{m\times n}$. The \udef{Kruskal rank} is the largest number $\KruskalRank(A)$ such that all $\KruskalRank(A)$-sets of columns are linearly independent.
\end{definition}
\begin{lemma}
Let $A\in \F^{m\times n}$. Then $\KruskalRank(A) \leq \Rank(A)$.
\end{lemma}
\begin{proof}
If $\Rank(A) = k$, then there exists a $k$-set of columns that spans $\Col(A)$. Then every linearly independent set is smaller than $k$ by the Steinitz exchange lemma \ref{SteinitzExchange} and $\KruskalRank(A) \leq k$.
\end{proof}

\begin{definition}
Let $A\in \F^{m\times n}$. Then the \udef{spark} of $A$ is defined as
\[ \operatorname{spark}(A) \defeq \min\setbuilder{\norm{x}_0}{x\in\Null(A)} \]
where $\norm{x}_0$ is the number of non-zero elements of $x$.
\end{definition}
TODO $\norm{\cdot}_0$ norm for finite fields.

\begin{lemma}
Let $A\in \F^{m\times n}$. Then
\[ \operatorname{spark}(A) = \KruskalRank(A) + 1. \]
\end{lemma}
\begin{proof}
TODO
\end{proof}

\begin{lemma}
Let $A\in \F^{m\times n}$. If $\Rank(A) = n$, then $\KruskalRank(A) = n$.
\end{lemma}

\begin{proposition}
Let $A\in \F^{m\times n}$. Then
\[ \KruskalRank(A) \geq \frac{1}{\mu(A)} \]
where $\mu(A) = \max_{i\neq j} \frac{|\inner{[A]_{_,i},[A]_{-,j}}|}{\norm{[A]_{_,i}}\norm{[A]_{_,j}}}$.
\end{proposition}
\begin{proof}
TODO
\end{proof}


\section{Eigenvalues and eigenvectors}
\subsection{The spectrum}
In this section we study vectors that are mapped to multiples of themselves by a given matrix $A$, i.e.\ vectors $\vec{v}$ such that
\[ A\vec{v} = \lambda \vec{v} \qquad\text{for some $\lambda\in\F$.} \]
Clearly for this to be possible, $A$ needs to be square.
\begin{definition}
Suppose $A\in \F^{n\times n}$.
\begin{itemize}
\item  A scalar $\lambda\in \mathbb{F}$ is called an \udef{eigenvalue} of $A$ if there exists a $\vec{v}\in \F^n$ such that $\vec{v}\neq 0$ and $A\vec{v} = \lambda v$.
\item Such a vector $\vec{v}$ is called an \udef{eigenvector}.
\item The set of all eigenvectors associated with an eigenvalue $\lambda$ is called the \udef{eigenspace} $E_\lambda(A)$. Because
\[ E_\lambda(A) = \ker(\ell_{\lambda \mathbb{1}_{n} - A}), \]
it is indeed a vector space.

The dimension of $E_\lambda(A)$ is the \udef{geometric multiplicity} of $\lambda$.
\end{itemize}
The set of all eigenvalues is called the \udef{spectrum} of $A$.
\end{definition}
\begin{proposition}
Let $A\in \F^{n\times n}$ and $\lambda\in \mathbb{F}$, then
\[ \text{$\lambda$ is an eigenvalue of $A$} \qquad \iff \qquad \text{$\lambda \mathbb{1}_{n} - A$ is invertible.} \]
\end{proposition}
\begin{proof}
The equation $A\vec{v} = \lambda \vec{v}$ is equivalent to $(A-\lambda \mathbb{1}_n)\vec{v} = 0$. So there exist eigenvectors associated to $\lambda$ iff the kernel of $\ell_{A-\lambda \mathbb{1}_n}$ is not trivial iff $\ell_{A-\lambda \mathbb{1}_n}$ is injective (\ref{injectivityKernelTriviality}) iff $\ell_{A-\lambda \mathbb{1}_n}$ is invertible (\ref{invertibleFiniteDim}) iff $A-\lambda \mathbb{1}_n$ is invertible (\ref{invertibleMapInvertibleMatrix}).
\end{proof}

\begin{proposition}[Gerschgorin circle theorem]
Let $A\in \F^{n\times n}$. If $\lambda$ is an eigenvalue of $A$, then there is an $i\in 1:n$ such that
\[ |\lambda - [A]_{ii}| \leq \sum_{\substack{j\in 1:n \\ j \neq i}}|[A]_{ij}|. \]
\end{proposition}
\begin{proof}
If $|\lambda - [A]_{ii}| > \sum_{\substack{j\in 1:n \\ j \neq i}}|[A]_{ij}|$ for all $i$, then $(A-\lambda\mathbb{1})$ is strictly diagonally dominant and thus invertible by \ref{invertibleDiagonallyDominant}.
\end{proof}

\begin{proposition}
Let $A\in\F^{n\times n}$ be a matrix. Suppose $\lambda_1, \ldots, \lambda_m$ are distinct eigenvalues of $A$ and $\vec{v}_1,\ldots, \vec{v}_m$ are corresponding eigenvectors. Then $\{\vec{v}_1,\ldots, \vec{v}_m\}$ is linearly independent.
\end{proposition}
\begin{proof}
The proof goes by contradiction. Assume $\{\vec{v}_1,\ldots, \vec{v}_m\}$ is linearly dependent. Let $k$ be the smallest positive integer such that
\[ \vec{v}_k \in \Span\{\vec{v}_1,\ldots, \vec{v}_{k-1}\}. \]
So there exists a nontrivial linear combination
\[ \vec{v}_k = a_1\vec{v}_1+\ldots +a_{k-1}\vec{v}_{k-1}. \]
Multiplying by $A$ gives
\[ \lambda_k\vec{v}_k = a_1\lambda_k\vec{v}_1+\ldots +a_{k-1}\lambda_k\vec{v}_{k-1}. \]
Multiplying the previous combination by $\lambda_k$ and subtracting both equations gives
\[ 0= a_1(\lambda_k-\lambda_1)\vec{v}_1 +\ldots + a_{k-1}(\lambda_k - \lambda_{k-1})\vec{v}_{k-1}. \]
By assumption of linear independence of $\{\vec{v}_1,\ldots, \vec{v}_{k-1}\}$ this combination must be trivial, however none of the $(\lambda_k-\lambda_i)$ can be zero, so all the $a_i$ must be zero. This is a contradiction with the assumption of linear dependence.
\end{proof}
\begin{corollary}
The matrix $A\in\F^{n\times n}$ has at most $n$ linearly independent eigenvalues.
\end{corollary}
\begin{corollary}
Suppose $\lambda_1, \ldots, \lambda_m$ are distinct eigenvalues of $A$. Then
\[ E_{\lambda_1}(A) \oplus \ldots \oplus E_{\lambda_m}(A) \]
is a direct sum. Furthermore, the sum of geometric multiplicities is less than or equal to the dimension of $V$:
\[ \dim E_{\lambda_1}(A) + \ldots + \dim E_{\lambda_m}(A) \leq \dim V. \]
\end{corollary}

\subsubsection{The characteristic equation}
\begin{definition}
Let $A\in\F^{n\times n}$. The \udef{characteristic polynomial} $p_A(x)$ of $A$. Is the polynomial
\[ p_A(x) \defeq \det(x\mathbb{1}_n - A). \]
\end{definition}
The characteristic polynomial is also sometimes defined as $\det(A - x\mathbb{1}_n)$. This differs by a sign $(-1)^{n}$.
\begin{lemma}
The characteristic polynomial of any square matrix is a monic polynomial.
\end{lemma}
\begin{lemma}
The characteristic polynomials of similar matrices are identical.
\end{lemma}

\begin{proposition}
Let $A\in\F^{n\times n}$. Then $p_A(x)$ can be factorised as
\[ p_A(x) = \prod_{i=1}^m (x - \lambda_i)^{m_i}  \]
where $\lambda_i$ are the eigenvalues of $A$ and the multiplicities $m_i$ are positive integers such that $\sum_{i=1}^m m_i = n$.
\end{proposition}
\begin{corollary}
The eigenvalues of $A$ are the solutions of the equation
\[ p_A(x) = 0. \]
\end{corollary}
\begin{corollary}
The determinant of a matrix is the product of its eigenvalues, counting algebraic multiplicity: for $A\in\F^{n\times n}$
\[ \det(A) = \prod_{i=1}^m\lambda_i^{m_i}. \]
\end{corollary}
\begin{proof}
We have
\[ \det(A) = (-1)^n\det(0-A) = (-1)^np_A(0) = (-1)^n\prod_{i=1}^m(0 - \lambda_i)^{m_i} = \prod_{i=1}^m\lambda_i^{m_i}. \]
\end{proof}

\begin{definition}
Let $A\in\F^{n\times n}$. The equation
\[ p_A(x) = \det(x\mathbb{1}_n - A) = 0 \]
is called the \udef{characteristic equation} of $A$.

Let $\lambda$ be a solution of the characteristic equation. The multiplicity of $\lambda$ as a root of $p_A(x)$ is the \udef{algebraic multiplicity} of $\lambda$.
\end{definition}
\begin{lemma}
Let $A\in\F^{n\times n}$ and $\lambda$ be an eigenvalue of $A$.

The geometric multiplicity of $\lambda$ is less than or equal to the algebraic multiplicity of $\lambda$.
\end{lemma}
\begin{proof}
Set $k=\dim E_\lambda$. Take a basis of $E_\lambda(A)$ and extend it to a basis $\beta$ of $\F^{n}$. With respect to this basis the matrix of $\ell_A$ is of the form
\[ (\ell_A)_\beta^\beta =  \begin{pmatrix}
\lambda \mathbb{1}_k & B \\ 0 & C
\end{pmatrix} = P^{-1}(\ell_A)_\mathcal{E}^\mathcal{E}P = P^{-1}AP \]
for some matrices $B,C$ and some invertible matrix $P$ where $\mathcal{E}$ is the standard basis of $\F^n$. Then 
\[ p_A(x)= p_{P^{-1}AP} = p_{\lambda \mathbb{1}_{k}}(x)p_C(x) = (\lambda - x)^kp_C(x), \]
so the algebraic multiplicity of $\lambda$ is at least the geometric multiplicity $k$. It may be greater if $\lambda$ is also an eigenvector of $C$, but in this case the eigenvector is a linear combination of the eigenvectors already chosen for $\beta$.
\end{proof}

\subsubsection{Diagonalisable matrices}
\begin{definition}
A matrix $A\in\F^{n\times n}$ is called \udef{diagonalisable} if $\F^n$ has a basis of eigenvectors of $A$.
\end{definition}
\begin{proposition}
Let $A\in\F^{n\times n}$ and let $\lambda_1,\ldots, \lambda_m$ denote the distinct eigenvalues of $A$. The following are equivalent:
\begin{enumerate}
\item $A$ is diagonalisable;
\item there exist $1$-dimensional subspaces $U_1,\ldots, U_n$ of $A$, each invariant under $\ell_A$, such that
\[ \F^n = U_1\oplus \ldots \oplus U_n; \]
\item $\F^n = E_{\lambda_1}(A) \oplus \ldots \oplus E_{\lambda_m}(A);$
\item $n = \dim E_{\lambda_1}(A) + \ldots + \dim E_{\lambda_m}(A);$
\item for each $\lambda_i$ the geometric multiplicity is equal to the algebraic multiplicity and the sum of algebraic multiplicities is $n$.
\end{enumerate}
\end{proposition}
In the case of complex vector spaces, the sum of algebraic multiplicities is always $n$ by the fundamental theorem of algebra.
\begin{corollary}
If $A\in\F^{n\times n}$ has $n$ distinct eigenvalues, then $A$ is diagonalisable.
\end{corollary}
So a matrix may fail to be diagonalisible for two reasons: not enough geometric multiplicity or not enough geometric and algebraic multiplicity

\subsection{Spectral theorem}
\begin{theorem}
Let $V$ be a complex finite-dimensional inner product space. Let $L$ be an operator on $V$. Then
\begin{enumerate}
\item there exists an orthonormal basis of $V$ consisting of eigenvectors of $L$ \textup{if and only if} $L$ is normal;
\item if $L$ is self-adjoint, then the eigenvalues of $L$ are real.
\end{enumerate}
\end{theorem}

TODO: replace following: + in real case we need self-adjoint!
\begin{theorem}[Spectral theorem for matrices]
Let $V$ be a finite-dimensional inner product space over $\R$ or $\C$. Let $L=L^*$ be self-adjoint. Then
\begin{enumerate}
\item there exists an orthonormal basis of $V$ consisting of eigenvectors of $L$;
\item the eigenvalues of $L$ are real.
\end{enumerate}
\end{theorem}
\begin{proof}
We first prove the theorem for complex vector spaces. The proof is by finite induction:

By the fundamental theorem of algebra the characteristic polynomial has at least 1 root $\lambda_1$. Choose a corresponding eigenvector $\vec{v}_1$. Then by
\[ \lambda_1\inner{\vec{v}_1,\vec{v}_1} = \inner{\vec{v}_1, L\vec{v}_1} = \inner{L\vec{v}_1, \vec{v}_1} = \overline{\lambda_1}\inner{\vec{v}_1, \vec{v}_1}, \]
$\lambda_1$ is real.

Now $\Span\{\vec{v}_1\}^\perp$ is invariant under $L$:
\[ \vec{x}\in \Span\{\vec{v}_1\}^\perp \quad\iff\quad \inner{\vec{x},\vec{v}_1} = 0 \quad\implies\quad \inner{L\vec{x},\vec{v}_1} = \inner{\vec{x},L\vec{v}_1} = \lambda_1\inner{\vec{x},\vec{v}_1} = 0. \]

We can now apply the same argument to $L|_{\Span\{\vec{v}_1\}^\perp}:\Span\{\vec{v}_1\}^\perp\to \Span\{\vec{v}_1\}^\perp$, whose eigenvector are orthogonal to $v_1$. Finite induction then finishes the proof in the complex case. 

In the real case: we can linearly extend $L$ to be an operator $L_\C$ on the complexification $V_\C$. Then $L_\C$ has formally the same characteristic polynomial as $L$, except now interpreted as a function $\C\to\C$, not $\R\to\R$. Now we know the roots of $p_{L_\C}(x)$ are real, so they are also roots of $p_L(x)$. The rest of the proof can be completed in the same way. 
\end{proof}

The spectral decomposition is a special case of both the Schur decomposition and the singular value decomposition.

TODO: Kronecker product: multiply eigenvalues: all there (by multiplicity)

\subsection{Computing eigenvalues and vectors}
\subsubsection{Power method}
+ inverse

\subsubsection{Deflation}
\url{https://quickfem.com/wp-content/uploads/IFEM.AppE_.pdf}

\subsubsection{QR}

\section{Matrix classes and decompositions}
\subsection{Matrix classes}
\subsubsection{Rank-1 projections}
\begin{proposition}
Let $P\in\F^{n\times n}$. Then $P$ is a rank-1 (orthogonal) projection \textup{if and only if} there is a unit vector $\vec{u}\in\F^n$ such that $P= \vec{u}\vec{u}^*$.
\end{proposition}
\begin{proof}
Due to $P$ being rank-1, we can find a unit vector $\vec{u}$ such that $\Col(P) = \Span\{\vec{u}\}$. So the columns of $P$ are all multiples of $\vec{u}$, meaning we can write $P$ as $\vec{u}\vec{v}^*$ for some $\vec{v}\in\F^n$.

Now $P = P^*= (\vec{u}\vec{v}^*)^* = \vec{v}\vec{u}^*$, so $\vec{u}\vec{v}^* = \vec{v}\vec{u}^*$ and thus $\Col(P) = \Span\{\vec{v}\}$, meaning $\vec{v} = \lambda \vec{u}$.

Also $P^2 = \vec{u}\vec{v}^*\vec{u}\vec{v}^* = \vec{u}\inner{v,u}\vec{v}^* = \inner{v,u}\vec{u}\vec{v}^*$, so $1 = \inner{\vec{v},\vec{u}} = \overline{\lambda}\inner{\vec{u},\vec{u}} = \overline{\lambda}$.

So $\vec{v}=\vec{u}$ and $P= \vec{u}\vec{u}^*$.
\end{proof}
We write $P_{\vec{u}}$ to denote $\vec{u}\vec{u}^*$. Then in particular $P_{\vec{u}}\vec{v} = \inner{\vec{u},\vec{v}}\vec{u}$.

\subsubsection{Householder matrices}
\begin{definition}
Let $\vec{w}\in\F^n$ be a non-zero vector. Set $\vec{u} = \vec{w}/\norm{\vec{w}}$. Then the corresponding \udef{Householder matrix} is
\[ U_{\vec{w}} \defeq \mathbb{1}_n - 2P_{\vec{u}} = \mathbb{1}_n - 2 \frac{\vec{w}\vec{w}^*}{\inner{\vec{w},\vec{w}}} = \mathbb{1}_n - 2 \frac{\vec{w}\vec{w}^*}{\vec{w}^*\vec{w}}. \]
The corresponding transformation $\ell_{U_{\vec{w}}}$ is called a \udef{Householder transformation}.
\end{definition}
The Householder transformation reflects vectors across a hyperplane orthogonal to $\vec{w}$.

\begin{lemma}
Householder matrices are unitary, Hermitian and involutive.
\end{lemma}

For any two vectors of the same length, we can construct a unitary matrix that maps one to the other, using Householder matrices.

\begin{proposition}
Let $\vec{x},\vec{y}\in\F^n$ such that $\norm{\vec{x}} = \norm{\vec{y}} \neq 0$. Let
\[ \sigma = \begin{cases}
1& (\inner{\vec{x}\vec{y}} = 0) \\ -\overline{\inner{\vec{x},\vec{y}}}/|\inner{\vec{x},\vec{y}}| & (\inner{\vec{x}\vec{y}} \neq 0),
\end{cases} \]
and let $\vec{w} = \vec{y}-\sigma \vec{x}$. Then $\sigma U_{\vec{w}}$ is unitary and $\sigma U_{\vec{w}}\vec{x} = \vec{y}$.
\end{proposition}
The use of $\sigma$ is purely to improve numerical stability.

\subsubsection{Upper Hessenberg matrices}
\begin{definition}
A matrix $A$ is called an \udef{upper Hessenberg matrix} if $i>j+1\implies [A]_{i,j}=0$.
\end{definition}
This is a matrix of the form
\[ \begin{bmatrix}
\star & \star & \star & \star & \star \\
\star & \star & \star & \star & \star \\
0 & \star & \star & \star & \star \\
0 & 0 & \star & \star & \star \\
0 & 0 & 0 & \star & \star \\
\end{bmatrix} \]
Every square matrix is unitarily similar to an upper Hessenberg matrix, and the unitary
similarity can be constructed from a sequence of Householder matrices and complex rotations.

\subsection{Matrix decompositions}
\subsubsection{LU and LDU factorisation}
\subsubsection{QR factorisation}
The QR factorization of an $m \times n$ matrix $A$ is a factorisation
$A=QR$
where $Q\in\F^{m\times n}$ has orthonormal columns and $R\in\F^{n\times n}$ is square upper triangular.
This requires that $m\geq n$.

\begin{proposition}[QR factorisation]
Let $A\in\F^{m\times n}$ and $m\geq n$. Then
\begin{enumerate}
\item there exists a unitary $U\in\F^{m\times m}$ and an upper triangular matrix $R\in\F^{n\times n}$ such that
\[ A = U \begin{bmatrix}
R \\ \mathbb{0}
\end{bmatrix} \]
we can take $R$ to have real, non-negative values on the diagonal;
\item writing $U = \begin{bmatrix}
Q & Q'
\end{bmatrix}$, we get the decomposition
\[ A = QR \]
where $Q$ has orthonormal columns;
\item if $\Rank(A)= n$, then fixing the values on the diagonal of $R$ to be positive makes the factorisation unique; all values on the diagonal are non-zero.
\end{enumerate}
\end{proposition}
\begin{proof}
We can find a (unitary) Householder transformation $U_1$ that maps the first columns $[A]_{-,1}$ to $\norm{[A]_{-,1}}\vec{e}_1$. So
\[ U_1 A = \begin{bmatrix}
\norm{[A]_{-,1}} & \star \\ \mathbb{0} & A_1
\end{bmatrix} \]
Then we can do the same for $A_1$, meaning $U_2 = \mathbb{1}\oplus U'$ transforms $A$ as
\[ U_2U_1A = \begin{bmatrix}
\norm{[A]_{-,1}} & \star & \star \\
0 & \norm{[A_1]_{-,1}} & \star \\
\mathbb{0} & \mathbb{0} & A_2
\end{bmatrix}. \]
Repreating this gives the required factorisation.
\end{proof}
The factorisation $A = U \begin{bmatrix}
R \\ \mathbb{0}
\end{bmatrix}$ is called the wide QR factorisation and $A=QR$ the (narrow) QR factorisation.

\subsection{Polar decomposition}
\subsubsection{Singular value decomposition}
spectral decomposition is a special case
\subsubsection{Schur decomposition}
spectral decomposition is a special case




\section{Systems of linear equations}
\url{https://encyclopediaofmath.org/wiki/Motzkin_transposition_theorem}
TODO
A homogeneous system of linear equations with more variables than equations has non-zero solutions.

An inhomogeneous system of linear equations with more equations thanvariables has no solution for some choice of the constant terms.

Calculation of inverse via row reduction.

free and bounded variables.

\begin{lemma}
If $Ax=b$ is consistent for all $b\in \F^n$, then $A$ has a right inverse $B$, i.e.\ $AB = \mathbb{1}$.
\end{lemma}
\begin{proof}
For each $e_i$ in the standard basis we can find a $c_i$ such that $Ac_i = e_i$. Then
\[ A \begin{pmatrix}
c_1 & c_2 & \hdots & c_n
\end{pmatrix} = \begin{pmatrix}
Ac_1 & Ac_2 & \hdots & Ac_n
\end{pmatrix} = \begin{pmatrix}
e_1 & e_2 & \hdots & e_n
\end{pmatrix} = \mathbb{1}_n. \]
\end{proof}

\subsection{Cramer's rule}
\begin{proposition}
\[ x_i = \frac{\det(A_i)}{\det(A)} \]
\end{proposition}
\begin{proof}
$x = \begin{pmatrix}
x_1 \hdots x_n
\end{pmatrix}^\transp$
\begin{align*}
x_i &= \det \begin{pmatrix}
e_1 & \hdots & e_{i-1} & x & e_{i+1} & \hdots & e_n
\end{pmatrix} \\
&= \det \begin{pmatrix}
A^{-1}a_1 & \hdots & A^{-1}a_{i-1} & A^{-1}b & A^{-1}a_{i+1} & \hdots & A^{-1}a_n
\end{pmatrix} \\
&= \det (A^{-1}\begin{pmatrix}
a_1 & \hdots & a_{i-1} & b & a_{i+1} & \hdots & a_n
\end{pmatrix}) = \det(A^{-1}A_i) \\
&= \frac{\det(A_i)}{\det(A)}.
\end{align*}
\end{proof}


\section{Polynomials applied to endomorphisms}
\section{The spectra of matrices}
What are eigenvectors of rotation? -> complex eigenvalues.

Real matrix: complex conjugate eigenvalues have complex conjugate eigenvectors

Finite order endomorphisms are diagonalisable over $\C$ (or any algebraically closed field where the characteristic of the field does not divide the order of the endomorphism) with roots of unity on the diagonal. This follows since the minimal polynomial is separable, because the roots of unity are distinct.

See \url{https://en.wikipedia.org/wiki/Minimal_polynomial_(linear_algebra)}


\section{Euclidean geometry}
\begin{definition}
The \udef{$n$-dimensional Euclidean space} is $\R^n$ equipped with the inner product
\[ \inner{\begin{pmatrix}
x_1 \\ x_2 \\ \vdots \\ x_n
\end{pmatrix}, \begin{pmatrix}
y_1 \\ y_2 \\ \vdots \\ y_n
\end{pmatrix}} = x_1y_1 + x_2y_2 \ldots x_ny_n. \]
\end{definition}

TODO: $z$-vector points into page due to orientation (point up out of the page would give it a left-handed orientation).

\subsection{Affine subspaces}

Line as intersection of planes with normals $n_1,n_2$. Then direction of line is $n_1\times n_2$.

\subsection{Distances}

\subsection{Distance point to plane}
\begin{lemma}
\[ d(P,\pi) = \frac{|p_x\alpha + p_y\beta +p_z\gamma -d|}{\sqrt{\alpha^2 + \beta^2 + \gamma^2}} \]
\end{lemma}

Thus in the Cartesian expression for a plane, $\alpha x + \beta y + \gamma z = d$, the $d$ is the distance to the origin. (Also $(\alpha, \beta, \gamma)$ is the normal vector).

\subsection{Angles}

\subsection{Rotations}

\subsection{Spheres}











\chapter{Indices and symbols}
\section{Contravariant and covariant vectors and tensors}
When working in finite-dimensional spaces with specified bases, we often get expressions of the form
\[ v = \sum_{i=1}^n a_i \vec{e}_i. \]
We may replace this expression with
\[ v = a^i \vec{e}_i \]
if we take the convention that if an index is repeated once up and once down, then there is a sum over all values of that index. This is the \udef{Einstein summation convention}.

Note that coordinates have their indices up, and basis vectors have their indices down.

Now in the dual space, we have the dual basis $\{\varphi^j\}_j$. In the dual space we take the opposite convention: coordinates have their indices down, and dual basis vectors have their indices up, so
\[ \varphi = b_j \varphi^j. \]
This allows us to write
\[ \varphi(v) = \varphi(a_i \vec{e}_i) = a_i b^j \varphi^j(\vec{e}_i) = a_i b^i \]
where for the last equality we have used that $\varphi^j(\vec{e}_i)$ only does not vanish if $i=j$ and is $1$ in this case.

We call vectors in $V$ \udef{contravariant} vectors and vectors in $V^*$ \udef{covariant} vectors, or covectors.

Per convention we put the coordinates of contravariant vectors in column vectors. This means, by proposition \ref{transpDual}, we must put the coordinates of covariant vectors in row vectors. Indeed, let $v=a^i \vec{e}_i\in V$ and $\varphi = b_j\varphi^j \in V^*$, then
\[ \varphi(v) = a^ib_i = \begin{bmatrix}
a^1 & \hdots & a^n
\end{bmatrix}\begin{bmatrix}
b_1 \\ \vdots  \\ b_n
\end{bmatrix}. \]

We can view covariant vectors as functions that take a contravariant vector and produce a number, and we can view contravariant vectors as functions that take a covariant vector and produce a number. In general we may have linear functions that accept several co- and contravariant vectors and produce a number. By (TODO), such functions are tensor products of various co- and contravariant vectors. They would have multiple up- and down-indices. e.g\
\[ \vec{T} = \tensor{T}{^i_j_k^l^m}(\vec{e}_i\otimes \vec{e}^j\otimes \vec{e}^k\otimes \vec{e}_l \otimes \vec{e}_m) \]

Where $\tensor{T}{^i_j_k^l^m}$ are the coordinates w.r.t. the basis vectors $\vec{e}_i\otimes \vec{e}^j\otimes \vec{e}^k \otimes\vec{e}_l \otimes \vec{e}_m$.

For example, once the basis has been chosen, matrices map contravariant vectors to contravariant vectors. And contravariant vectors map covariant vectors to numbers, so by reverse currying a matrix is maps a contravariant and a covariant vector to a number.

For the indices of matrices we have taken the convention that the first index is for rows and the second for the columns. For a constant row index, the column index spells out a covariant vector, so the column index is down. Conversely, the row index is up. A matrix $A$ with components $(A)_{i,j}$ becomes
\[ \tensor{A}{^i_j}(\vec{e}_i\otimes \vec{e}^j). \]
This is consistent with the observation that the matrix sends a vector to a function on covectors, in other words is a function which accepts vectors in first place and covectors in second place.

In the expressions so far only repeated indices were present. Such repeated indices are called  \udef{bound indices} or \udef{dummy indices}. They may be replaced in the expression by other letters, so long as there is no clash. If an index is not repeated, it is a \udef{free index} and may not just be changed.

\subsection{``Tensors are objects that transform as tensors''}
Variants:
\begin{itemize}
\item ``$N$ arbitrary numbers are not the components of a vector'' (Peres p.65)
\end{itemize}


\section{Covectors}

\subsection{Multi-index notation}
Let $e_1,\ldots, e_n$ be a basis for a real vector space $V$. Let $\alpha^1,\ldots, \alpha^n$ be the dual basis for $V^*$. A \udef{multi-index}
\[ I = (i_1,\ldots,i_k)\]
is a $k$-tuple of numbers $\in (1,\ldots,n)$. We write
\[ \begin{cases}
e_I \defeq e_{i_1}\otimes\ldots\otimes e_{i_k}\\
\alpha^I \defeq \alpha^{i_1}\wedge\ldots \wedge\alpha^{i_k}
\end{cases}. \]
The covector $\alpha^I$ is completely determined by the values in $I$, the order only changes the sign. A multi-index $I = (i_1,\ldots,i_k)$ is \udef{ascending} if
\[ 1\leq i_1<\ldots<i_k\leq n. \]
\begin{proposition}
Let $I,J$ be ascending multi-indices of length $k$, then
\[ \alpha^I(e_J) = \begin{cases}
1 & I=J \\ 0& I\neq J
\end{cases}. \]
\end{proposition}
\begin{proposition}
The covectors $\alpha^I$, with $I$ an ascending multi-index of length $k$, form a basis of $A_k(V)$.
\end{proposition}
\begin{corollary}
If $\dim V=n$, then
\[ \dim A_k(V) = \begin{pmatrix}
n\\k
\end{pmatrix}. \]
\end{corollary}

\section{Symmetrisation and anti-symmetrisation of indices}

\[ T_{\{a_1\dots a_n\}} = \frac{1}{n!} \sum_{\sigma\in S_n} T_{a_{\sigma(1)} \dots a_{\sigma(n)}} \]

\[ T_{[a_1\dots a_n]} = \frac{1}{n!} \sum_{\sigma\in S_n} (\sgn \sigma)T_{a_{\sigma(1)} \dots a_{\sigma(n)}} \]
\section{Symbols}
\subsection{Kronecker delta}
\begin{definition}
The \udef{Kronecker delta} is defined by
\[ \delta_{ij} = \delta^i_j = \begin{cases}
1 & (i=j) \\
0 & (i \neq j)
\end{cases}.\]
\end{definition}
\subsection{Levi-Civita symbol}
\begin{definition}
The \udef{Levi-Civita symbol} is defined by
\[ \varepsilon_{a_{1}\ldots a_{n}} = \begin{cases}
+1 & \text{$(a_{1},\ldots, a_{n})$ is an even permutation of $(1,\ldots, n)$} \\
-1 & \text{$(a_{1},\ldots, a_{n})$ is an odd permutation of $(1,\ldots, n)$} \\
0 & \text{otherwise}
\end{cases}.\]
The indices may be placed up or down.
\end{definition}

\begin{lemma} \label{LeviCivitaProduct}
The Levi-Civita symbol is given by the explicit expression
\[ \varepsilon_{a_{1}\ldots a_{n}} = \prod_{1\leq i<j\leq n}\sgn(a_j-a_i).\]
\end{lemma}

\begin{proposition}
Working in $n$ dimensions, when all $i_1,\ldots i_n;j_1,\ldots, j_n$ take values in $\{ 1,\ldots, n \}$:
\begin{enumerate}
\item $\displaystyle \varepsilon_{i_1\ldots i_n}\varepsilon^{j_1\ldots j_n} = n!\delta^{j_1}_{[i_1}\ldots \delta^{j_n}_{i_n]} = \sum_{\sigma\in S_n} (-1)^{{\sgn}(\sigma)} \delta^{j_1}_{i_{\sigma(1)}} \dots \delta^{j_n}_{i_{\sigma(n)}}$
\item $\displaystyle \varepsilon _{i_{1}\dots i_{n}}\varepsilon ^{i_{1}\dots i_{n}}=n!$
\item $\displaystyle \varepsilon _{i_{1}\dots i_{k}~i_{k+1}\dots i_{n}}\varepsilon ^{i_{1}\dots i_{k}~j_{k+1}\dots j_{n}}=k!(n-k)!~\delta _{[i_{k+1}}^{j_{k+1}}\dots \delta _{i_{n}]}^{j_{n}}$.
\end{enumerate}
\end{proposition}
\begin{proof}
\begin{enumerate}
\item Both sides of the equation are a sum over the same indices. We consider each term in the sum separately and show that the sums are equal term-by-term. We split the terms into two categories.
\begin{enumerate}
\item First consider the case that $j_1\ldots j_n$ is not a permutation of $(1,\ldots, n)$, i.e.\ a number is repeated. Then $\varepsilon_{i_1\ldots i_n}\varepsilon_{j_1\ldots j_n}$ is automatically zero. The right-hand side is definitely zero if the $i$s do not take the same values as the $j$s. If they do take the same values, there is a number that is repeated at least twice. For every term in the sum over permutations, there is another term with the repeated $i$s swapped, which also adds a minus due to the change of sign of the permutation. Hence the sum over permutations is zero.
\item Now assume that $j_1\ldots j_n$ is a permutation of $(1,\ldots, n)$. Then either the $i$s are also a permutation, or $\delta^{j_1}_{i_{\sigma(1)}} \dots \delta^{j_n}_{i_{\sigma(n)}}$ is always zero. The only possible non-zero term is with a $\sigma\in S_n$ such that $j_k = i_{\sigma(k)}$ for all $k$. If $\sgn(i_1,\ldots, i_n) = \sgn(j_1,\ldots, j_n)$, then $\sgn(\sigma)=1$ and both sides match. If $\sgn(i_1,\ldots, i_n) = -\sgn(j_1,\ldots, j_n)$, then $\sgn(\sigma)=-1$ and both sides again match. 
\end{enumerate}
So, in fact, we have shown something slightly stronger, namely 
\[ \varepsilon_{i_1\ldots i_n}\varepsilon_{j_1\ldots j_n} = n!\delta^{j_1}_{[i_1}\ldots \delta^{j_n}_{i_n]} \]
where there is no sum over indices.
\item The number of permutations of any $n$-element set number is exactly $n!$. Every permutation is either even or odd and $(+1)^2 = (-1)^2 = 1$. Non-permutations do not contribute to the sum.
\item The sum on the left only has terms where the $i$s and $j$s are permutations of $(1,\ldots, n)$. In each such term we can bring the indices with values $1-k$ to the first $k$ spots, each by a transposition. Because both Levi-Civita symbols have the same first $k$ indices, each will need the same number of transpositions and thus the sign does not change. Then by considering lemma \ref{LeviCivitaProduct} we see that we have obtained a product of cases 1. and 2. This yields the answer.

\end{enumerate}
\end{proof}
\begin{corollary}
In two dimensions, where all $i,j,m,n$ each take values in $\{1,2\}$,
\begin{enumerate}
\item $\varepsilon _{ij}\varepsilon ^{mn}={\delta _{i}}^{m}{\delta _{j}}^{n}-{\delta _{i}}^{n}{\delta _{j}}^{m}$
\item $\varepsilon _{ij}\varepsilon ^{in}={\delta _{j}}^{n}$
\item $\varepsilon _{ij}\varepsilon ^{ij}=2.$
\end{enumerate}
\end{corollary}
\begin{corollary}
In three dimensions, where all $i,j,k,m,n$ each take values in $\{1,2,3\}$,
\begin{enumerate}
\item $\varepsilon _{ijk}\varepsilon ^{imn}={\delta _{j}}^{m}{\delta _{k}}^{n}-{\delta _{j}}^{n}{\delta _{k}}^{m}$
\item $\varepsilon _{jmn}\varepsilon ^{imn}={\delta _{j}}^{i}$
\item $\varepsilon _{ijk}\varepsilon ^{ijk}=6.$
\end{enumerate}
\end{corollary}
\begin{proposition}
Working in 3 dimensions,
\begin{align*}
\varepsilon _{ijk}\varepsilon _{lmn}&={\begin{vmatrix}\delta _{il}&\delta _{im}&\delta _{in}\\\delta _{jl}&\delta _{jm}&\delta _{jn}\\\delta _{kl}&\delta _{km}&\delta _{kn}\\\end{vmatrix}}\\[6pt]&=\delta _{il}\left(\delta _{jm}\delta _{kn}-\delta _{jn}\delta _{km}\right)-\delta _{im}\left(\delta _{jl}\delta _{kn}-\delta _{jn}\delta _{kl}\right)+\delta _{in}\left(\delta _{jl}\delta _{km}-\delta _{jm}\delta _{kl}\right).
\end{align*}
This can directly be generalised to $n$ dimensions.
\end{proposition}

\section{Writing matrix operations using using tensor notation}
A matrix $A$ with components $(A)_{i,j}$ becomes
\[ \tensor{A}{^i_j}(\vec{e}_i\otimes \vec{e}^j). \]
\subsection{Trace}
The trace of $\tensor{A}{^i_j}$ is $\tensor{A}{^i_i}$.
\subsection{Matrix multiplication}
\[ \tensor{(AB)}{^{i}_{k}}=\tensor{A}{^{i}_{j}}\tensor{B}{^{j}_{k}} \]
which in particular for matrix-vector multiplication becomes
\[ (Av)^i = \tensor{A}{^i_j} v^j. \]
\subsection{Transpose}
The transpose of $\tensor{A}{^i_j}$ is $\tensor{(A^\transp)}{^j_i}$.

Or: $(A^\transp)_{ab} = A_{ba}$ and $\tensor{(A^\transp)}{_i^j} = \tensor{A}{^j_i}$?
\subsection{Determinant}
\begin{align*}
\det(A) &= \varepsilon^{j_1\ldots j_n}\tensor{A}{^{1}_{j_1}}\ldots \tensor{A}{^{n}_{j_n}} \\
&= \frac{1}{n!}\varepsilon_{i_1\ldots i_n}\varepsilon^{j_1\ldots j_n}\tensor{A}{^{i_1}_{j_1}}\ldots \tensor{A}{^{i_n}_{j_n}}
\end{align*}

\chapter{Ordered vector spaces}
TODO link ordered groups.
\begin{definition}
Let $\sSet{\R, V, +}$ be a real vector space and $\precsim$ a preorder on the set $V$. Then $\precsim$ is a \udef{vector preorder} if it is compatible with the vector space structure as follows: $\forall x,y,z\in V, \lambda\in\R$
\begin{enumerate}
\item $x\precsim y$ implies $x+z \precsim y+z$;
\item if $\lambda\geq 0$, then $x \precsim y$ implies $\lambda x \precsim \lambda y$.
\end{enumerate}
We call $(\R, V, +, \precsim)$ a \udef{preordered vector space}.

\begin{itemize}
\item If $\precsim$ is a partial order, we call $(\R, V, +, \precsim)$ a \udef{partially ordered vector space} or simply a \udef{ordered vector space}.
\item If $\sSet{V, \precsim}$ is a lattice, we call $(\R, V, +, \precsim)$ a \udef{vector lattice} or a \udef{Riesz space}.
\end{itemize}
\end{definition}

\begin{lemma} \label{positiveConeOrderCharacterisation}
Let $\sSet{\R, V, +}$ be a real vector space and $\precsim$ a preorder on the set $V$. The compatibility of the order can equivalently be expressed by:
$\forall x,y,z\in V, \lambda\in\R$
\[ \begin{cases}
\text{$x \precsim y$ implies $x+z \precsim y+z$;} \\
\text{if $\lambda\geq 0$ and $0 \precsim x$, then $0 \precsim \lambda x$.}
\end{cases} \]
\end{lemma}

\begin{lemma} \label{elementaryVectorPreorderManipulations}
Let $V$ be a preordered vector space. For all $v,w \in V$ we have
\[ v \precsim w \;\iff\; 0 \precsim  w - v \;\iff\; -w \precsim -v.  \]
\end{lemma}
\begin{proof}
We get the implications
\[ v \precsim w \implies 0 \precsim  w - v \implies -w \precsim -v \implies v-w \precsim 0 \implies v\precsim w \]
by subsequently adding $-v, -w, v,w$ to both sides by compatibility of the order.
\end{proof}
\begin{corollary}
Let $V$ be a preordered vector space and $\alpha \in \R\setminus\{0\}$. Then for all $v,w\in V$
\[ v \precsim w \quad \iff \quad \begin{cases}
\alpha v \precsim \alpha w & (0 < \alpha) \\
\alpha v \succsim \alpha w & (\alpha > 0)
\end{cases}. \] 
\end{corollary}

\begin{lemma} \label{additionVectorInequalities}
Let $V$ be a preordered vector space. For all $v,w, x, y \in V$ we have
\[ \begin{cases}
v \precsim w \\ x \precsim y
\end{cases} \implies v+ x \precsim w+y. \]
\end{lemma}
\begin{proof}
We calculate $v + x \leq w + x \leq w+y$.
\end{proof}

\begin{example}
\begin{itemize}
\item The finite-dimensional vector spaces $\R^n$ with coordinate-wise addition, scalar multiplication and order are Riesz spaces.
\item The finite-dimensional vector spaces $\R^n$ with coordinate-wise addition, scalar multiplication and lexicographical order are Riesz spaces.
\item Let $X$ be a set. The set $(X\to \R)$ is a real vector space with point-wise addition and scalar multiplication. If the order is also defined point-wise, i.e.\ $f \leq g$ iff $\forall x\in X: f(x) \leq g(x)$, then $(X\to \R)$ is a Riesz space.
\end{itemize}
\end{example}

\begin{lemma}
Let $X$ be a topological space. The spaces
\begin{enumerate}
\item $\cont(X,\R)$;
\item $\cont_0(X,\R)$;
\item $\cont_c(X,\R)$; and
\item $\cont_b(X,\R)$
\end{enumerate}
with point-wise operations are Riesz spaces.
\end{lemma}
\begin{proof}
In all these cases the join and meet of $f,g$ are given by
\begin{align*}
f \vee g &= \frac{1}{2}(f+g)+ \frac{1}{2}|f-g| \\
f \wedge g &= \frac{1}{2}(f+g) - \frac{1}{2}|f-g|.
\end{align*}
So the join and meet are still continuous and have the same properties as $f,g$.
\end{proof}

\section{Upsets and downsets}
\begin{lemma} \label{sumMultipleUpDownsets}
Let $V$ be a preordered vector space, $S\subseteq V$ a subset, $v\in V$ and $\alpha\in \R$. Then
\begin{enumerate}
\item if $\alpha > 0$, then $(\alpha S)^u = \alpha S^u$ and $(\alpha S)^l = \alpha S^l$;
\item if $\alpha < 0$, then $(\alpha S)^u = \alpha S^l$ and $(\alpha S)^l = \alpha S^u$;
\item $(S+v)^u = S^u + v$ and $(S+v)^l = S^l + v$.
\end{enumerate}
\end{lemma}
\begin{corollary}
Let $V$ be a preordered vector space, $S\subseteq V$ a subset, $v\in V$ and $\alpha\in \R$. Then
\begin{enumerate}
\item $\sup(S+v) = \sup(S)+v$ and $\inf(S+v) = \inf(S)+v$;
\item if $\alpha > 0$, then $\sup(\alpha S) = \alpha \sup(S)$ and $\inf(\alpha S) = \alpha \inf(S)$;
\item if $\alpha < 0$, then $\sup(\alpha S) = \alpha \inf(S)$ and $\inf(\alpha S) = \alpha \sup(S)$.
\end{enumerate}
\end{corollary}

\section{The positive cone}
\begin{definition}
Let $V$ be a preordered vector space. The subset
\[ V^+ \defeq \setbuilder{v\in V}{0 \precsim v} \]
is called the \udef{positive cone} of $V$. The elements of the positive cone $V^+$ are called the \udef{positive elements} of $V$.
\end{definition}
That the positive cone is in fact a cone follows from \ref{positiveConeOrderCharacterisation}
\begin{proposition} \label{positiveCone}
Let $V$ be a vector space.
\begin{enumerate}
\item A vector preorder on $V$ is uniquely determined by its positive cone:
\[ x \precsim y \quad\iff\quad y-x \in V^+. \]
\item The positive cone of a vector preorder is pointed and convex.
\item Any pointed convex cone in $V$ determines a (unique) vector preorder.
\item A vector preorder is a partial order \textup{if and only if} the positive cone is salient.
\end{enumerate}
\end{proposition}
Convexity is equivalent to closure under addition (see \ref{convexityAdditiveClosure})
\begin{proof}
(1) This is just \ref{elementaryVectorPreorderManipulations}.

(2) $V^+$  is pointed by reflexivity: $0\precsim 0$. It is closed under addition by \ref{additionVectorInequalities}.

(3) Compatibility with addition is immediate from the definition of the order. Compatibility with scalar multiplication is due to it being a cone (see \ref{positiveConeOrderCharacterisation}). Reflexivity is equivalent with pointedness. Finally transitivity follows from closure under addition:
\begin{align*}
 \begin{cases}
x\precsim y \\ y\precsim x
\end{cases} &\iff \quad \begin{cases}
0 \precsim y -x \\ 0 \precsim z-y
\end{cases} \iff\quad \begin{cases}
y-x \in V^+ \\ z-y \in V^+
\end{cases} \\
&\implies (z-y)+(y-x) = z-x \in V^+ \iff x \precsim z. 
\end{align*}

(4) We have $x\precsim y$ and $y\precsim x$ iff $(y-x) \in V^+$ and $-(y-x) \in V^+$. Thus both salience and anti-symmetry are equivalent to this situation implying $x-y = 0$.
\end{proof}


\begin{lemma} \label{scalarMultiplicationInequalities}
Let $V$ be a preordered vector space, $v\in V^+$ and $\alpha\in \R$.
\begin{enumerate}
\item If $\alpha \geq 1$, then $\alpha v \succsim v$.
\item If $\alpha \leq 1$, then $\alpha v \precsim v$.
\end{enumerate}
\end{lemma}
\begin{proof}
(1) We have $v\succsim 0$ and $(\alpha-1) \geq 0$, so $(\alpha-1)v \succsim 0$ and $\alpha v \succsim v$.

(2) We have $v\succsim 0$ and $(\alpha-1) \leq 0$, so $(\alpha-1)v \precsim 0$ and $\alpha v \precsim v$.
\end{proof}

\section{Riesz spaces}

\begin{lemma} \label{lemmaRieszSpaces}
Let $V$ be a Riesz space, $u,v,w\in V$ and $\alpha\in \R$, then
\begin{enumerate}
\item $-(v \wedge w) = (-v)\vee (-w)$ and $-(v \vee w) = (-v)\wedge (-w)$;
\item if $\alpha \geq 0$, then $\alpha(v \wedge w) = (\alpha v)\wedge (\alpha w)$ and $\alpha(v \vee w) = (\alpha v)\vee (\alpha w)$;
\item if $\alpha \leq 0$, then $\alpha(v \wedge w) = (\alpha v)\vee (\alpha w)$ and $\alpha(v \vee w) = (\alpha v)\wedge (\alpha w)$;
\item $u+(v \wedge w) = (u+v)\wedge (u+w)$ and $u+(v \vee w) = (u+v)\vee (u+w)$.
\end{enumerate}
\end{lemma}
\begin{proof}
We apply \ref{imagePolars} to

(1) the reverse order-embedding $v\mapsto -v$;

(2) the order-embedding $v\mapsto \alpha v$ for $\alpha > 0$; (if $\alpha = 0$ the result is trivial);

(3) the reverse order-embedding $v\mapsto \alpha v$ for $\alpha > 0$; (if $\alpha = 0$ the result is trivial);

(4) the order-embedding $v\mapsto u+v$.
\end{proof}

\begin{proposition}[Riesz decomposition]
Let $V$ be a Riesz space and $v,w_1,w_2\in V^+$ such that $v \leq w_1 + w_2$. Then $\exists v_1, v_2\in V^+$ such that $v = v_1 + v_2$ and $v_1 \leq w_2, v_2 \leq w_2$.
\end{proposition}
\begin{proof}
Set $v_1 = v\wedge w_1$ and $v_2 = v - v_1$. These satisfy all the properties. We verify the inequality $v_2 \leq w_2$: from $w_2 \geq v - w_1$ we get
\[ w_2 = 0\vee w_2 \geq 0\vee (v - w_1) = v + (-v)\vee(-w_1) = v - v\wedge w_1 = v-v_1 = v_2. \]
\end{proof}

\begin{proposition} \label{sumAsMeetJoin}
Let $V$ be a Riesz space and $v,w\in V$, then
\[ (v \vee w) + (v \wedge w) = v+w. \]
\end{proposition}
\begin{proof}
We calculate
\[ (v \vee w) + (v \wedge w) = v + 0 \vee (w-v) + w + (v-w)\wedge 0 = (v + w) + 0 \vee (w-v) - 0 \vee (w-v) = v + w. \]
\end{proof}

\begin{proposition}
Let $V$ be a Riesz space, $u,v,w\in V$ and $x,y,z\in V^+$. Then
\begin{enumerate}
\item $(u+v)\vee (2w) \leq u\vee w + v\vee w$;
\item $(u+v)\vee z \leq u\vee z + v\vee z$;
\item $(x+y)\wedge z \leq x\wedge z + y\wedge z$.
\end{enumerate}
\end{proposition}
\begin{proof}
(1) From $u\leq u\vee w$ and $v\leq v\vee w$, we get $u+v \leq u\vee w + v\vee w$. Similarly from $w\leq u\vee w$ and $w\leq v\vee w$, we get $2w \leq u\vee w + v\vee w$. Together this gives (1).

(2) We have $2z \geq z$ by \ref{scalarMultiplicationInequalities}.

(3) TODO (use Birkhoff inequality??)
\end{proof}

\begin{proposition}[Infinite distributivity in Riesz spaces]
Let $V$ be a Riesz space, $v\in V$ and $S\subseteq V$ a subset. Then
\begin{enumerate}
\item if $\bigvee S$ exists, then $\left(\bigvee S\right) \wedge v = \bigvee (S\wedge v)$;
\item if $\bigwedge S$ exists, then $\left(\bigwedge S\right) \vee v = \bigwedge (S\vee v)$;
\end{enumerate}
\end{proposition}
\begin{proof}
We already have the inequality $\left(\bigvee S\right) \wedge v \geq \bigvee (S\wedge v)$ from \ref{infiniteDistributiveInequalities}. To show the other inequality, it is enough to show that for 
\end{proof}
\begin{corollary}
Riesz spaces are distributive lattices.
\end{corollary}

\subsection{Positive elements}
\subsubsection{Positive and negative parts}
\begin{definition}
Let $V$ be a Riesz space and $v\in V$. Then we define
\begin{align*}
v^+ &\defeq v \vee 0 \\
v^- &\defeq (-v) \vee 0 = - (v \wedge 0).
\end{align*}
We call $v^+$ the \udef{positive part} of $v$ and $v^-$ the \udef{negative part} of $v$.
\end{definition}

\begin{lemma} \label{MeetJoinAsPositiveNegative}
Let $V$ be a Riesz space and $v,w\in V$. Then
\begin{enumerate}
\item $v\vee w = (v-w)^+ + w = (v-w)^- + v$;
\item $v\wedge w = v - (v-w)^+ = w - (v-w)^-$.
\end{enumerate}
\end{lemma}
\begin{proof}
We calculate $v\vee w = (v-w)\vee 0 + w = (v-w)^+ + w$. The other equalities are similar.
\end{proof}

\begin{proposition} \label{PositiveNegativeElements} \label{minimalPositiveDecomposition} 
Let $V$ be a Riesz space and $v,w\in V$. Then
\begin{enumerate}
\item $v^+, v^- \in V^+$;
\item $v= v^+ - v^-$;
\item $v^+\perp v^-$ (i.e.\ $v^+ \wedge v^- = 0$).
\end{enumerate}
Furthermore,
\begin{enumerate} \setcounter{enumi}{3}
\item if $p,q\in V$ satisfy 1 and 2, i.e.\ $p,q\in V^+$ and $v = p-q$, then $p \geq v^+$ and $q \geq v^-$; we may say $v=v^+-v^-$ is the minimal such decomposition; 
\item the elements $v^+, v^-$ are uniquely determined by properties 2 and 3. 
\end{enumerate}
Also
\begin{enumerate} \setcounter{enumi}{5}
\item $(-v)^- = v^+$ and $(-v)^+ = v^-$;
\item if $\alpha \geq 0$, then $(\alpha v)^+ = \alpha v^+$ and $(\alpha v)^- = \alpha v^-$;
\item $-v^- \leq v \leq v^+$;
\item $v\leq w$ \textup{if and only if} $v^+ \leq w^+$ and $v^- \geq w^-$.
\end{enumerate}
\end{proposition}
\begin{proof}
(1) Evident from definitions.

(2) We calculate $v^+ - v = (v \vee 0) - v = (v-v) \vee (0-v) = 0\vee (-v) = v^-$.

(3) We calculate $0 = v^- - v^- = v^-  + (v\wedge 0) = (v^- + v)\wedge (0 + v^-) = v^+ \wedge v^-$.

(4) From $v\leq p$ and $0\leq p$, we get $v^+ = v \vee 0 \leq p$. Then we also have $v^- = v^+ - v \leq p - v = q$.

(5) Assume $p,q\in V$ satisfy (2) and (3), then (1) automatically follows from (3). Using \ref{MeetJoinAsPositiveNegative}, we calculate
\[ 0 = p\wedge q = p - (p-q)^+ = p - v^+. \]
So $p = v^+$ and $q = p - v = v^+ - v = v^-$.

(6) It is evident that $(-v)^- = (--v)\vee 0 = v\vee 0$.

(7) We calculate $\alpha v^+ = \alpha (v \vee 0) = (\alpha v) \vee 0 = (\alpha v)^+$; the calculation for $\alpha v^-$ is similar.

(8) This is clear from $-v^- = v\wedge 0 \;\leq\; v \;\leq\; v \vee 0 = v^+$.

(9) $v\leq w$ implies $v^+ = v\vee 0 \leq w\vee 0 = w^+$ and $-v^- = v\wedge 0 \leq w\wedge 0 = - w^-$.

Conversely, we have $v = v^+ - v^- \leq w^+ - w^- = w$.
\end{proof}


\begin{proposition} \label{triangleInequalityPositiveNegativeElements}
Let $V$ be a Riesz space and $v,w\in V$. Then
\begin{enumerate}
\item $(v+w)^+ \leq v^+ + w^+$;
\item $(v+w)^- \leq v^- + w^-$.
\end{enumerate}
\end{proposition}
\begin{proof}
From \ref{PositiveNegativeElements} we get $v \leq v^+$ and $w\leq w^+$, so $v+w \leq v^+ + w^+$. Also $0 \leq v^+ + w^+$. So
\[ (v+w)^+ = (v+w)\vee 0 \leq v^+ + w^+. \]
Then we also have
\[ (v+w)^- = (-v-w)^+ \leq (-v)^+ + (-w)^+ = v^- + w^-. \]
\end{proof}

\subsubsection{Absolute value}
\begin{definition}
Let $V$ be a Riesz space and $v\in V$. Then the \udef{absolute value} of $v$ is
\[ |v| \defeq v\vee (-v) = -(v\wedge (-v)). \]
\end{definition}

If the Riesz space is a real function space with pointwise order, then $|f| = |\cdot|\circ f$ as usual, where $|\cdot|: \R\to \R$ is the usual absolute value function.

\begin{lemma} \label{absoluteValue}
Let $V$ be a Riesz space, $v,w\in V$ and $\alpha\in \R$. Then
\begin{enumerate}
\item $|v| = v^+ + v^-$;
\item $|v| \in V^+$;
\item if $v\in V^+$, then $|v| = v$;
\item $|v| = |-v|$;
\item $|\alpha v| = |\alpha|\cdot |v|$;
\item $\big||v|\big| = |v|$;
\item $0 \leq v^+ \leq |v|$ and $0 \leq v^- \leq |v|$;
\item $|v| = 0$ \textup{if and only if} $v = 0$.
\end{enumerate}
\end{lemma}
\begin{proof}
We prove (1):
\[ |v| = v\vee (-v) = (2v)\vee 0 - v = 2v^+ - v = 2v^+ - (v^+ - v^-) = v^+ - v^- \]
and (6), using the absorption law:
\[ \big||v|\big| = |v|\vee (-|v|) = v \vee (-v) \vee \big( v\wedge (-v) \big) = v \vee (-v) = |v|. \]
The rest are immediate consequences, using the results of \ref{PositiveNegativeElements}.
\end{proof}

\begin{lemma}
Let $V$ be a Riesz space and $v,w\in V$, then
\begin{enumerate}
\item $(v+w)\vee (v-w) = v + |w|$;
\item $(v+w)\wedge (v-w) = v - |w|$;
\end{enumerate}
or, equivalently,
\begin{enumerate} \setcounter{enumi}{2}
\item $v \vee w = \frac{1}{2}\big(v+w + |v - w|\big)$;
\item $v \wedge w = \frac{1}{2}\big(v+w - |v - w|\big)$.
\end{enumerate}
\end{lemma}
\begin{proof}
We calculate, using \ref{lemmaRieszSpaces}
\[ (v+w)\vee (v-w) = v+ w\vee(-w) = v+ |w| \quad\text{and}\quad (v+w)\wedge (v-w) = v + w\wedge(-w) = v-|w|. \]
The next two equalities follow by the substitutions $v+w \leftrightarrow v$ and $v-w \leftrightarrow w$.
\end{proof}
\begin{corollary}
Let $V$ be a Riesz space and $v,w\in V$, then
\[ |v - w| = (v \vee w) - (v \wedge w). \]
\end{corollary}
\begin{corollary} \label{meetJoinAbsoluteValues}
Let $V$ be a Riesz space and $v,w\in V$, then
\begin{enumerate}
\item $|v| \vee |w| = \frac{1}{2}\Big(|v|+|w| + \big||v| - |w|\big|\Big)$;
\item $|v| \wedge |w| = \frac{1}{2}\Big(|v|+|w| - \big||v| - |w|\big|\Big)$.
\end{enumerate}
\end{corollary}
\begin{proof}
Substitute $v\to |v|$ and $w\to |w|$.
\end{proof}
\begin{corollary}
Let $V$ be a Riesz space and $u,v,w\in V$, then
\begin{enumerate}
\item $|u\vee v - u\vee w| + |u\wedge v - u\wedge w| = |v-w|$;
\item $|v^+-w^+|\leq |v-w|$ and $|v^- - w^-|\leq |v-w|$.
\end{enumerate}
\end{corollary}
\begin{proof}
(1) Using the proposition, we get
\[ |u\vee v - u\vee w| + |u\wedge v - u\wedge w| = (u\vee v)\vee(u\vee w) - (u\vee v)\wedge (u\vee w) + (u\wedge v)\vee(u\wedge w) - (u\wedge v)\wedge(u\wedge w). \]
Using distributivity this simplifies to $(v\vee w) - (v\wedge w) = |v-w|$.

(2) Using (1) we have
\[ |v-w| = |0\vee v - 0\vee w| + |0\wedge v - 0\wedge w| = |v^+ -w^+| + |v^- - w^-| \geq \begin{cases}
|v^+ -w^+| \\
|v^- - w^-|.
\end{cases}  \]
\end{proof}

\begin{proposition}
Let $V$ be a Riesz space and $v,w\in V$, then
\begin{enumerate}
\item $|v|\vee |w| = \frac{1}{2}\Big( |v+w| + |v - w| \Big)$;
\item $|v|\wedge |w| = \frac{1}{2}\Big| |v+w| - |v - w| \Big|$;
\end{enumerate}
or, equivalently,
\begin{enumerate} \setcounter{enumi}{2}
\item $|v+w|\vee |v-w| = |v| + |w|$;
\item $|v+w|\wedge |v-w| = \big| |v| - |w| \big|$.
\end{enumerate}
Also
\begin{enumerate} \setcounter{enumi}{4}
\item $|v|+|w| = |v+w| + |v-w| - \big||v|-|w|\big|$;
\item $|v+w|+|v-w| = 2|v| + 2|w| - \big||v+w|-|v-w|\big|$;
\end{enumerate}
and
\begin{enumerate} \setcounter{enumi}{6}
\item $|v|+|w| = \big||v|-|w|\big| + \big||v+w|-|v-w|\big|$.
\end{enumerate}
\end{proposition}
The only real trick is in the proof of (1). All the other results follow from elementary substitutions.
\begin{proof}
(3,4,6) Are equivalent to (1,2,5) by the replacements $v \leftrightarrow v+w$ and $w \leftrightarrow v-w$.

(1) We calculate
\begin{align*}
|v|\vee |w| &= v\vee (-v)\vee w \vee (-w) = \big(v\vee(-w)\big)\vee \big((-v)\vee w\big) \\
&= \frac{1}{2}\Big((v-w) + |v + w|\Big)\vee \frac{1}{2}\Big( (-v+w) + |- v - w| \Big) \\
&= \frac{1}{2}|v+ w| + \frac{1}{2}\big((v-w)\vee (-v+w)\big) = \frac{1}{2}\Big( |v+w| + |v - w| \Big).
\end{align*}

(5) Using \ref{sumAsMeetJoin}, (1) and \ref{meetJoinAbsoluteValues} we get
\begin{align*}
|v|+|w| &= |v|\vee|w| + |v|\wedge |w| \\
&= \frac{1}{2}\Big( |v+w| + |v - w| \Big) + \frac{1}{2}\Big(|v|+|w| - \big||v| - |w|\big|\Big).
\end{align*}
This simplifies to the required equation.

(7) Follows from substituting (6) into (5).

(2) Follows from (7) and \ref{meetJoinAbsoluteValues}.
\end{proof}
\begin{corollary}
Let $V$ be a Riesz space and $v,w\in V$, then the following are equivalent:
\begin{enumerate}
\item $|v|\wedge |w| = 0$;
\item $|v+w| = |v-w|$;
\item $|v|\vee |w| = |v+w|$.
\end{enumerate}
\end{corollary}

The absolute value also satisfies the triangle inequality.
\begin{proposition}[Triangle and reverse triangle inequality in Riesz spaces]
Let $V$ be a Riesz space and $v,w\in V$, then
\[ |v| + |w| \geq \big|v+w\big| \geq \big||v|-|w|\big|. \]
\end{proposition}
\begin{proof}
The first inequality is the triangle inequality. It follows straight from \ref{triangleInequalityPositiveNegativeElements}.

The second inequality is the reverse triangle inequality and follows from the triangle inequality as in \ref{reverseTriangleInequality}.
\end{proof}

\subsection{Subsets}
\begin{definition}
Let $V$ be a Riesz space. A subset $E$ is called
\begin{itemize}
\item a \udef{Riesz subspace} if it is both a subspace and a sublattice;
\item \udef{solid} if for all $v\in E$ the interval $[-|v|,|v|]$ is a subset of $E$;
\item a \udef{band} if for all subsets $S\subseteq E$ we have $\sup(S) \subset E$. 
\end{itemize}
\end{definition}

\begin{lemma}
Let $V$ be a Riesz space and $v,w\in V$, then
\[ |w|\leq |v| \iff -|v| \leq w \leq |v|. \]
\end{lemma}
\begin{proof}
We have $w \leq |w|$ and $|w| \leq |v|$, so $w\leq |v|$. Also $-w \leq |-w| = |w|$, so $-|v| \leq -|w| \leq w$.

Conversely, $-|v| \leq w$ implies $-w\leq |v|$. So $|w| = w\vee (-w) \leq |v|$.
\end{proof}
\begin{corollary}
Let $V$ be a Riesz space and $E\subseteq V$ a subset. Then $E$ is solid \textup{if and only if}
\[ \forall v\in E: \forall w\in V: \; |w|\leq |v| \implies w\in E. \]
\end{corollary}

\begin{lemma}
Let $V$ be a Riesz space and $E\subseteq V$ a subset. Then $E$ is an (order) ideal \textup{if and only if} it is a solid Riesz subspace.
\end{lemma}
\begin{proof}
TODO
\end{proof}


\subsection{Disjointness}

\subsection{Archimedean}

\chapter{Some results and applications}
\section{Rotations}
Rodrigues' rotation formula

eigenvectors and eigenvalues of rotation.
\section{Pauli matrices}

\[ \sigma_x = \begin{pmatrix}
0 & 1 \\ 1 & 0
\end{pmatrix} \qquad \sigma_y = \begin{pmatrix}
0 & -i \\ i & 0
\end{pmatrix} \qquad \sigma_z = \begin{pmatrix}
1 & 0 \\ 0 & -1
\end{pmatrix} \]
All have eigenvalues $\pm 1$. The eigenspaces are spanned by
\[ v_{x+} = \frac{1}{\sqrt{2}}\begin{pmatrix}
1 \\ 1
\end{pmatrix}, \quad v_{x-} = \frac{1}{\sqrt{2}}\begin{pmatrix}
1 \\ -1
\end{pmatrix}, \quad v_{y+} = \frac{1}{\sqrt{2}}\begin{pmatrix}
1 \\ i
\end{pmatrix}, \quad v_{y-} = \frac{1}{\sqrt{2}}\begin{pmatrix}
1 \\ -i
\end{pmatrix}, \quad v_{z+} = \begin{pmatrix}
1 \\ 0
\end{pmatrix}, \quad v_{z-} = \begin{pmatrix}
0 \\ 1
\end{pmatrix}, \quad  \]

\[ \Tr[\sigma_i \sigma_j] = \delta_{ij} \]



\part{Convergence Vector Spaces}
\setcounter{chapter}{0} % Reset chapter counter
\chapter{Vector space convergence}
TODO: \url{https://math.stackexchange.com/questions/2001771/existence-of-at-least-one-continuous-coordinate-functional}

\url{https://math.stackexchange.com/questions/60057/does-there-exist-a-linearly-independent-and-dense-subset}

\begin{definition}
Let $\sSet{\F,V,+}$ be a vector space and $\xi$ a convergence on $V$. Then $\sSet{\F,V,+, \xi}$ is a \udef{convergence vector space} (or CVS) if
\begin{itemize}
\item vector addition $+: V\times V \to V$ is continuous;
\item scalar multiplication $\cdot: \F\times V \to V$ is continuous.
\end{itemize}
Let $W$ be another vector convergence space. The set of continuous lineat maps from $V$ to $W$ is denoted $\contLin(V,W)$.
\end{definition}

\begin{lemma}
If $\sSet{\F,V,+, \xi}$ is a convergence vector space, then $\sSet{V,+, 0, \xi}$ is a convergence group.
\end{lemma}
\begin{proof}
We just need to show that $v\mapsto -v$ is continuous, but this scalar multiplication and thus continuous by assumption.
\end{proof}

\begin{lemma} \label{continuityLemmaVectorConvergence}
If $\sSet{\F,V,+, \xi}$ is a convergence vector space, then
\begin{enumerate}
\item the function $V \to V: v \mapsto \lambda\cdot v$ is a homeomorphism for all $\lambda\in \F\setminus\{0\}$;
\item the function $\F \to \Span\{v\}: \lambda \mapsto \lambda\cdot v$ is continuous and invertible for all $v\in V\setminus\{0\}$;
\item the function $\F \to \Span\{v\}: \lambda \mapsto \lambda\cdot v$ is a homeomorphism \textup{if and only if} $\Span\{v\}$ is Hausdorff.
\end{enumerate}
\end{lemma}
TODO picture!
\begin{proof}
The functions $\lambda \mapsto (\lambda, v)$ and $v \mapsto (\lambda, v)$ are continuous by \ref{continuousEmbeddingProduct}. Composition with the continuous scalar product gives the result by continuity of composition (\ref{continuityComposition}).

They are both clearly invertible. (For the second, note that the kernel is $\{0\}$. We can also argue using \ref{scalarMultiplicationBijection}). The inverse of the first is of the same form and thus immediately continuous.

(3) If $u: \F \to \Span\{v\}: \lambda \mapsto \lambda\cdot v$ is a homeomorphism, then $\Span\{v\}$ is Hausdorff, because $\F$ is.

Now assume $\Span\{v\}$ Hausdorff. We need to show that $u^{-1}$ is continuous. It is enough to show continuity at $0$. We use \ref{pretopologicalContinuityVicinities}, so take $\Gamma \in \vicinity_\F(0)$ and we need to show that $\Gamma \in (u^{-1})^{\imf\imf}\big[\vicinity_{\Span\{v\}}(0)\big]$. WLOG we may take $\Gamma = \ball(0, \epsilon)$.

Now consider $\sphere(0,\epsilon)$, which is compact. Then $u^{\imf}\big(\sphere(0,\epsilon)\big) = \sphere(0,\epsilon)\cdot v$ is compact by \ref{compactConstructions} and thus closed by \ref{compactClosedSets} (as $\Span\{v\}$ we assumed Hausdorff). Now $0 \in \big(\sphere(0,\epsilon)\cdot v\big)^c$, so there exists a vicinity $U$ of $0$ disjoint from $\sphere(0,\epsilon)\cdot v$.

For each $F$ that converges to $0$ in $\Span\{v\}$, $\neighbourhood_\F(0)\cdot F$ also converges to $0$, so there exists $\delta_F>0$ and $C_F\in F$ such that $\ball(0, \delta_F)\cdot C_F \subseteq U$. If $\delta_F > 1$, then $C_F \subseteq \ball(0, \delta_F)\cdot C_F \subseteq \ball(0,\epsilon)\cdot v$, so $\ball(0,\epsilon)\cdot v\in F$.

Now assume $\delta_F \leq 1$. Then $\cball(0, \epsilon\cdot\delta_F^{-1})\setminus \ball(0,\epsilon)$ is compact, which, as before, means $\big(\cball(0, \epsilon\cdot\delta_F^{-1})\setminus \ball(0,\epsilon)\big)\cdot v$ is closed and we take a vicinity $U_F$ of $0$ disjoint from it. Now $C_F\cap U_F\in F$. We also claim that $C_F\cap U_F\subseteq \ball(0,\epsilon)\cdot v$: take $x\in C_F\cap U_F$. Then $x = \lambda v$ and $\delta_F |\lambda| \leq \epsilon$ (as $\ball(0, \delta_F)\cdot C_F \subseteq U$), which is equivalent to $|\lambda| \leq \epsilon\cdot\delta_F^{-1}$. Now because $x\in U_F$, this means $|\lambda| < \epsilon$ and thus $x\in \ball(0,\epsilon)\cdot v$.

As $\ball(0,\epsilon)\cdot v \in F$ for all $F$ that converge to $0$, we have $\ball(0,\epsilon)\cdot v\in \vicinity_{\Span\{v\}}$, so $\ball(0,\epsilon) = \Gamma \in (u^{-1})^{\imf\imf}\big[\vicinity_{\Span\{v\}}(0)\big]$, which is what we had to prove.
\end{proof}

TODO: rewrite proof of (3) using balanced neighbourhoods. \url{https://proofwiki.org/wiki/Isomorphism_from_Cartesian_Space_to_Finite-Dimensional_Subspace_of_Hausdorff_Topological_Vector_Space_is_Homeomorphism}.

Alternate proof in Beattie / Butzmann 3.3.19.

\begin{lemma} \label{continuityLinearCombination}
Let $V$ be a convergence vector space over a field $\F$. Let $x_0, \ldots, x_n$ be $n$ vectors in $V$. Then the function
\[ \F^n \to V: (\lambda_0, \ldots, \lambda_{n-1}) \mapsto \sum_{k=0}^{n-1} \lambda_kx_k \]
is continuous.
\end{lemma}
\begin{proof}
Call this function $g$.
For all $k\in \interval{1,n}$, the function $g_k: \F\to V: \lambda \mapsto \lambda x_k$ is continuous by \ref{continuityLemmaVectorConvergence}, so $g = (+)\circ (g_1|\ldots|g_n)$ is continuous by \ref{continuityParallelComposition} and the continuity of the sum.
\end{proof}

\begin{proposition} \label{vectorSpaceConvergenceConstruction}
Let $V$ be a vector space over a field $\F$. And $\mathcal{F} \subseteq \powerfilters(V)$ a family of filters. There exists a vector space convergence $\xi$ on $V$ such that $\mathcal{F} = \lim^{-1}_\xi(0)$ \textup{if and only if}
\begin{enumerate}
\item if $F \in \mathcal{F}$ and $G\supseteq F$, then $G\in \mathcal{F}$;
\item if $F,G \in \mathcal{F}$, then $F + G\in \mathcal{F}$;
\item if $F\in \mathcal{F}$, then $\neighbourhood_\F(0)\cdot F \in \mathcal{F}$;
\item if $v\in V$, then $\neighbourhood_\F(0)\cdot v \in \mathcal{F}$;
\item if $F\in \mathcal{F}$ and $\lambda\in \F$, then $\lambda\cdot F \in \mathcal{F}$.
\end{enumerate}
\end{proposition}
Note the similarity with \ref{groupConvergenceConstruction} for convergence groups. A group convergence is completely determined by $\lim^{-1}_\xi(0)$ due to the translation homeomorphisms \ref{shiftHomeomorphism}.
\begin{proof}
Assume first that $\mathcal{F} = \lim^{-1}_\xi(0)$ for some vector space convergence $\xi$.
\begin{enumerate}
\item This is just the monotonicity of the convergence.
\item If $F,G\to 0$, then $F\otimes G \to (0,0)$ by \ref{convergenceFiniteProductFilter}. By continuity of addition we have $F+G\to 0$.
\item The convergence on the scalar field is pretopological, so $\neighbourhood_\F(0)\to 0$. By \ref{convergenceFiniteProductFilter}, $\neighbourhood_\F(0)\otimes F \to (0,0)$ and by continuity of the scalar multiplication $\neighbourhood_\F(0)\cdot F \to 0$.
\item By \ref{continuityLemmaVectorConvergence}.
\item By \ref{continuityLemmaVectorConvergence}.
\end{enumerate}

Now assume the five points hold. Define the convergence $\xi$ by $F\to v$ iff $F-v \in \mathcal{F}$. We need to show that this is a convergence and that it makes both the vector addition and scalar multiplication continuous.

Monotonicity is guaranteed by (1). To show the convergence is centered, note that $\mathcal{F} \neq \emptyset$ by (4), so long as $V\neq \emptyset$. Then for any $F\in \mathcal{F}$, $\big\{\{0\}\big\} = 0\cdot F \in \mathcal{F}$ by (5).

To show that the vector addition is continuous, take $F\to (v_1, v_2)$. Then $p_1^{\imf\imf}(F) = F_1\to v_1$ and $p_2^{\imf\imf}(F) = F_2 \to v_2$, i.e.\ $F_1-v_1 \in \mathcal{F}$ and $F_2-v_2 \in \mathcal{F}$. By (1), $(F_1-v_1) + (F_2-v_2) = (F_1+F_2) - (v_1 + v_2) \in \mathcal{F}$, so $F_1+F_2 \to v_1 + v_2$. Thus by \ref{filterFactorisationInequality}, $F_1+F_2 = +^{\imf\imf}[F_1\otimes F_2] \subseteq +^{\imf\imf}[F] \to v_1+v_2$ and the addition is continuous.

Let $G \to (\lambda, v)$. Then $G_1 = p_1^{\imf\imf}(G) \to \lambda$ and $G_2 = p_2^{\imf\imf}(G) \to v$, so $G_1 \supseteq \neighbourhood_\F(\lambda)$. We have
\begin{align*}
\cdot^{\imf\imf}[G] - \lambda\cdot v &\supseteq \cdot^{\imf\imf}[G_1\otimes G_2] - \lambda\cdot v = G_1\cdot G_2 - \lambda\cdot v \\
&\supseteq \neighbourhood_\F(\lambda) \cdot G_2 - \lambda\cdot v \\
&= (\neighbourhood_\F(0) + \lambda)\cdot((G_2 - v) + v) - \lambda\cdot v \\
&\supseteq \lambda\cdot (G_2 - v) + \neighbourhood_\F(0)\cdot(G_2-v) + \lambda\cdot v + \neighbourhood_\F(0)\cdot v - \lambda\cdot v \\
&= \lambda\cdot (G_2 - v) + \neighbourhood_\F(0)\cdot(G_2-v) + \neighbourhood_\F(0)\cdot v \in \mathcal{F}.
\end{align*}
So $\cdot^{\imf\imf}[G] \to \lambda\cdot v$, making the scalar multiplication continuous. Note the last inclusion is not an equality because we go from one instance of $G_2$ and $\neighbourhood_\F(0)$ to two!
\end{proof}
\begin{corollary} \label{vicinityFilterAtOrigin}
Let $\sSet{V, \xi}$ be a convergence vector space. Then
\begin{enumerate}
\item for all $A\in \vicinity_\xi(0)$ and $\lambda\in \F\setminus\{0\}$: $\lambda A\in \vicinity_\xi(0)$;
\item each $A \in \vicinity_\xi(0)$ is absorbent;
\item if $\xi$ is topological, then $\neighbourhood_\xi(0)$ has a balanced base;
\item if $\xi$ is topological, then $\neighbourhood_\xi(0)$ has a closed, balanced base.
\end{enumerate}
\end{corollary}
If $\sSet{V,\xi}$ is equable, then $\vicinity_\xi(0)$ has a balanced base, see \ref{equableConvergenceBalancedBase}.

This is not true for all convergence vector spaces. What is true for all convergence vector spaces is that each $A \in \vicinity_\xi(0)$ contains a balanced subset. This is immediate since $\{0\}$ is balanced and $\{0\} \subseteq A$ because $A$ is absorbent.
\begin{proof}
(1) Take arbitrary $\lambda\in\F\setminus\{0\}$. Then $v\mapsto \lambda v$ is a homeomorphism by \ref{continuityLemmaVectorConvergence}, so $\lambda A \in \vicinity_\xi(\lambda 0) = \vicinity_\xi(0)$, by \ref{homeomorphismPreservation}.

(2) For absorbence, take $A\in \vicinity_\xi(0)$ and $v\in V$. As $\neighbourhood_\F(0)\cdot v \to 0$, we must have $A\in \upset\neighbourhood_\F(0)\cdot v$, so there exists $\Gamma \in \neighbourhood_\F(0)$ such that $\Gamma\cdot v \subseteq A$. Now we can find a $r>0$ such that $\ball(0,r)\subseteq \Gamma$, so for all $|c|\geq r^{-1}$ we have $v\in cA$.

(3) By point (3) of of the proposition, $\neighbourhood_\F(0)\cdot \vicinity_\xi(0)$ converges to $0$ and thus $\vicinity_\xi(0) \subseteq \neighbourhood_\F(0)\cdot\vicinity_\xi(0)$. Take $A\in \vicinity_\xi(0)$. Then there exists a $\Gamma\in\neighbourhood_\F(0)$ and $B\in \vicinity_\xi(0)$ such that $\Gamma\cdot B \subseteq \vicinity_\xi(0)$. We can find some ball $\ball(0,\epsilon) \subseteq \Gamma$, so $\ball(0,1)\cdot \epsilon B\subseteq \epsilon B \subseteq A$. Thus $\epsilon B$ is balanced and a neighbourhood by point(1). So every $A\in \vicinity_\xi(0)$ contains a balanced set in $\vicinity_\xi(0)$.

(4) Take arbitrary $A\in \neighbourhood_\xi(0)$. By regularity, \ref{topologicalGroupsRegular}, $A$ contains a closed neighbourhood $B$. By (3), $B$ contains a balanced neighbourhood $C$. Now consider $\closure_\xi(C)$, which is closed and a subset of $B$ (as $\closure(C)\subseteq \closure(B) = B$). It is now enough to note that the closure of a balanced set is balanced: if $x$ is the limit of a filter $F$ in $B$, then $r\cdot x$ is the limit of $r\cdot F$ for all $|r|\leq 1$. Now $r\cdot F$ is a filter in $B$ by balance.
\end{proof}

\begin{proposition} \label{vectorSumInherenceAdherence}
Let $\sSet{V,\xi}$ be a vector space convergence and $A,B\subseteq V$. Then
\begin{enumerate}
\item $\adh(A)+\adh(B) \subseteq \adh(A+B)$;
\item $\inh(A)+\inh(B) \subseteq A+\inh(B) \subseteq \inh(A+B)$;
\item $\interior(A)+\interior(B) \subseteq A+\interior(B) \subseteq \interior(A+B)$.
\end{enumerate}
\end{proposition}
TODO: same for closure?
\begin{proof}
(1) We use \ref{productAdherence} and \ref{adherenceInherenceContinuity} to compute
\[ \adh(A)+\adh(B) = +^\imf[\adh(A)\times\adh(B)] = +^\imf[\adh(A\times B)] \subseteq \adh(+^\imf[A\times B]) = \adh(A+B). \]

(2) The inclusion $\inh(A)+\inh(B) \subseteq A+\inh(B)$ is immediate. Now for all $v\in V$ we have $v+\inh(B) = \inh(v+B)$, so
\[ A+\inh(B) = \bigcup_{v\in A}v+\inh(B) = \bigcup_{v\in A}\inh(v+B) \subseteq \inh\left(\bigcup_{v\in A} v+B\right) = \inh(A+B), \]
where we have used the monotonicity of $\inh$ and \ref{orderPreservingFunctionLatticeOperations}.

(3) Similar to (2).
\end{proof}
Notice that the argument used for (2) and (3) does not work for the adherence because $\adh(A)+\adh(B) \nsubseteq A+\adh(B)$ in general.

\begin{lemma}
Let $V$ be a convergence vector space over a field $K$ and $F\subseteq K$ a subfield. Then the $F$-vector space $V_F$ with the same convergence structure is also a convergence vector space.
\end{lemma}
\begin{proof}
It is enough that the restriction of the scalar multiplication to $F$ remains continuous. Alternatively, we can use \ref{vectorSpaceConvergenceConstruction} and note that passing to the field $F$ simply represents a weakening of condition $(5)$.
\end{proof}

\section{Equable filters and spaces}
\begin{definition}
Let $\sSet{V,\xi}$ be a convergence vector space and $F\in \powerfilters(V)$ a filter.
\begin{itemize}
\item The filter $F$ is called \udef{equable} if $\neighbourhood_\F(0)\cdot F = F$.
\item The space $\sSet{V, \xi}$ is called \udef{equable} if each filter that converges to $0$ contains an equable filter that still converges to $0$
\end{itemize}
\end{definition}

\begin{lemma} \label{TVSEquable}
Let $\sSet{V,\xi}$ be a topological convergence vector space. Then $\neighbourhood_\xi(0)$ is equable and thus $\sSet{V,\xi}$ is equable.
\end{lemma}
\begin{proof}
By \ref{vicinityFilterAtOrigin}, $\neighbourhood_\xi(0)$ has a balanced base and $U\in \neighbourhood_\xi(0)$ iff $\lambda U\in \neighbourhood_\xi(0)$ for all $\lambda\in \F\setminus\{0\}$. Thus we have
\[ \neighbourhood_\xi(0) = \upset\{\cball(0,1)\}\cdot \neighbourhood_\xi(0) = \upset\{\cball(0,\epsilon)\}\cdot \neighbourhood_\xi(0) = \neighbourhood_\F(0)\cdot \neighbourhood_\xi(0). \]
This shows that $\neighbourhood_\xi(0)$ is equable. Now each filter that converges to $0$ contains $\neighbourhood_\xi(0)$, which is equable, so $\sSet{V, \xi}$ is equable.
\end{proof}

\begin{proposition} \label{equableConvergenceBalancedBase}
Let $\sSet{V,\xi}$ be an equable convergence vector space. Then $\vicinity_\xi(0)$ has a balanced base.
\end{proposition}
\begin{proof}
We have
\[ \vicinity_\xi(0) = \bigcap_{F\in \lim^{-1}(0)} F = \bigcap_{F\in \lim^{-1}(0)} \neighbourhood_\F(0)\cdot F. \]
Now take some $A\in \vicinity_\xi(0)$. As each $\neighbourhood_\F(0)\cdot F$ has a balanced base and $A\in\neighbourhood_\F(0)\cdot F$, we can find a balanced $B_F \in \neighbourhood_\F(0)\cdot F$ that is contained in $A$. Now $B = \bigcup_{F\in\lim^{-1}(0)}B_F$ is balanced by \ref{balancedLemma} and a subset of $A$. Since $B\in \neighbourhood_\F(0)\cdot F$ for all $F$ by upwards closure, we have $B\in \vicinity_\xi(0)$.
\end{proof}

\section{Spaces of functions}

\begin{lemma} \label{linearFunctionsClosedSubset}
Let $\sSet{V, \xi}$ be a convergence vector space and $\sSet{W, \zeta}$ a Hausdorff convergence vector space over the same field. Then $\Lin(V,W)$ is a closed subset of $(V\to W)_p$ and $(V\to W)_c$.
\end{lemma}
\begin{proof}
By \ref{openClosedConvergenceInclusions} it is enough to show that $\Lin(V,W)$ is closed in the pointwise convergence.

It is enough to show $\adh_p\big(\Lin(V,W)\big) \subseteq \Lin(V,W)$. Suppose $f\in \adh_p\big(\Lin(V,W)\big)$.  By \ref{principalAdherenceInherence}, we can take a convergent filter $H\in (V\to W)_p$ with limit $f$. 

Now we have, for all $v,w\in V, \lambda \in \F$, by \ref{filterPairingLemma},
\[ \upset\evalMap_{\lambda v+w}^{\imf\imf}(H) = \upset\evalMap_{\lambda v+w}^{\imf\imf}(H|_{\contLin(V,W)}) = \upset\lambda\evalMap_{v}^{\imf\imf}(H) + \upset\evalMap_{w}^{\imf\imf}(H). \] 

To show that $f$ is linear, we calculate, for all $v,w\in V, \lambda \in \F$
\[ f(\lambda v + w) = \evalMap_{\lambda v+w}(f) \in \lim\upset\evalMap_{\lambda v+w}^{\imf\imf}(H) = \lim \upset\lambda\evalMap_{v}^{\imf\imf}(H) + \upset\evalMap_{w}^{\imf\imf}(H) = \{\lambda f(v) + f(w)\},  \]
where we have used that the convergence on $W$ is Hausdorff.
\end{proof}

\begin{proposition} \label{continuousConvergenceVectorSpace}
Let $\sSet{X,\xi}$ be a convergence space and $\sSet{V, \zeta}$ a convergence vector space. Then
\begin{enumerate}
\item $(X\to V)_c$ is a preconvergence vector space;
\item $\cont_c(X,V)$ is a convergence vector space.
\end{enumerate}
\end{proposition}
\begin{proof}
Comparing with \ref{continuousConvergenceGroup}, we just need to show the continuity of pointwise scalar multiplication $\cdot_\text{pt}: \F\times (X\to V)_c \to (X\to V)_c$. By \ref{universalPropertyContinuousConvergence}, this is equivalent to the continuity of $\curry_1^{-1}(\cdot_\text{pt}): \F\times (X\to V)_c\times X \to V$, which follows from the commutativity of the following diagram:

\[ \begin{tikzcd}[column sep=large]
\big(\F\times (X\to V)_c\times X\big) \to V \arrow[r, "{\curry_1^{-1}(\cdot_\text{pt})}"] \arrow[d, "a"] & V \\
\F\times \big((X\to V)_c\times X\big) \arrow[r, "(\id_\F|\evalMap)"] & \F\times V \arrow[u, "{\boldsymbol{\cdot}}"]
\end{tikzcd} \]
where the map $a$ is just the associator and thus continuous by  \ref{associatorTernaryProducts}.
\end{proof}

\begin{proposition} \label{continuousDualComplete}
Let $\sSet{X,\xi}$ be a convergence space. Then $\contLin_c(X,\F)$ is a complete convergence vector space.
\end{proposition}
\begin{proof}
TODO Beattie Butzmann p.84
\end{proof}


\section{Initial and final vector space convergences}
\subsection{Initial vector space convergence}
\begin{proposition} \label{initialVectorSpaceConvergence}
Let $V$ be a vector space, $\{V_i\}_{i\in I}$ a set of convergence vector spaces and $\{L_i: V \to V_i\}_{i\in I}$ a set of linear maps. Then the initial convergence on $V$ w.r.t. $\{L_i: V \to V_i\}_{i\in I}$ makes $V$ a convergence vector space.
\end{proposition}
\begin{proof}
Continuity of vector addition follows from \ref{initialConvergenceGroup}.

We verify continuity of scalar multiplication $m: \F\times V \to V: (\lambda, v) \mapsto \lambda v$. Using \ref{characteristicPropertyInitialFinalConvergence}, we need to verify that $L_i\circ m$ is continuous for all $i\in I$. Because the $L_i$ are linear, we have
\[ L_i(\lambda v) = \lambda L_i(v) \]
for all $\lambda \in \F, v \in V$. This means that $L_i\circ m = m_i \circ (\id_{V_i}, L_i)$, where $m_i$ is scalar multiplication in $V_i$. Now $(\id_{\F}, L_i)$ is continuous by \ref{continuityFunctionTuple}, so $L_i \circ m$ is continuous.
\end{proof}
\begin{corollary}
Let $\{V_i\}_{i\in I}$ be a set of convergence vector spaces. Then the direct product $\prod_{i\in I} V_i$ with the product convergence is a convergence vector space.
\end{corollary}
\begin{corollary} 
Let $\sSet{I, \{\sSet{V_i, \xi_i}\}_{i\in I}, \{p_{j,i}\}_{i\preceq j}}$ be an projective system of convergence vector spaces with linear linking morphisms $p_{j,i}$. Then the projective limit $\varprojlim_{i\in I} V_i$ in the category set, equipped with the projective limit preconvergence structure and the projective limit vector space structure is a convergence vector space.
\end{corollary}

\subsection{Quotient spaces}
\begin{proposition}
Let $\sSet{V, \xi}$ be a convergence vector space, $W$ a vector space and $q: \sSet{V, \xi} \to W$ a surjective linear function. Then the quotient convergence on $W$ w.r.t. $q$ is a vector space convergence.
\end{proposition}
\begin{proof}
Continuity of vector addition follows from
\ref{quotientConvergenceGroup}.

We verify continuity of scalar multiplication $m: \F\times W \to W: (\lambda, w) \mapsto \lambda w$. Take $F\overset{\F}{\longrightarrow}\lambda$ and $G \overset{W}{\longrightarrow} w$. Then, by \ref{initialFinalConvergence}, there exist $w'\in q^{\preimf}\{w\}$ and $G' \overset{\xi}{\longrightarrow} w'$ such that $q^{\imf\imf}[G'] \subseteq G$. Then
\[ F\cdot G \supseteq F\cdot q^{\imf\imf}[G'] = q^{\imf\imf}[F\cdot G'] \to q(\lambda \cdot w') = \lambda q(w') = \lambda w, \]
which shows that $m$ is continuous.
\end{proof}
\begin{corollary}
Let $\sSet{V,\xi}$ be a convergence vector space and $U\subseteq V$ a subspace. Then $G/U$ is a convergence vector space.
\end{corollary}
\begin{proof}
The function $[\cdot]_U: V\to V/U$ is linear by (TODO ref universal algebra aspects of vector spaces) and \ref{quotientAlgebra}. It is clearly surjective.
\end{proof}
\begin{corollary}
Let $\sSet{V, \xi}$ and $\sSet{W, \zeta}$ be CVSs. Let $f: V\to W$ be a continuous linear function and $U\subseteq G$ a subspace such that $N\subseteq \ker f$. Then there exists a unique continuous linear function $f': A/N \to B$ such that
\[ \begin{tikzcd}
A \arrow[r, "{[\cdot]_N}"] \arrow[dr, swap, "f"] & A/N \arrow[d, dashed, "{f'}"] \\
& B
\end{tikzcd} \qquad\text{commutes.} \]
Further, $f'$ is injective \textup{if and only if} $N = \ker f$.
\end{corollary}
\begin{proof}
The linear function $f'$ is the one from \ref{factorTheorem}. It is continuous by \ref{characteristicPropertyInitialFinalConvergence}.
\end{proof}

\subsection{Direct sum}
\begin{definition}
Let $\{\sSet{V_i, \xi_i}\}_{i\in I}$ be a set of convergence vector spaces. Then the \udef{direct sum convergence} is the final \emph{vector space} convergence on $\bigoplus_{i\in I}V_i$ w.r.t. the set $\{e_j: V_j \to \bigoplus_{i\in I}V_i\}$ of natural injections.
\end{definition}
The direct sum convergence is the final vector space convergence. It is equal to the final convergence if and only if the direct sum is trivial (TODO why?).

\begin{proposition}
Let $\{\sSet{V_i, \xi_i}\}_{i\in I}$ be a set of convergence vector spaces. Then the direct sum convergence is the final convergence on $\bigoplus_{i\in I}V_i$ w.r.t. the set of natural injections $\setbuilder{e_J: \prod_{j\in J} V_j \to \bigoplus_{i\in I}V_i}{\text{$J\subseteq I$ finite}}$.
\end{proposition}
\begin{proof}

\end{proof}
\begin{corollary} \label{finiteDirectSumIsProduct}
If $I$ is finite, then $\bigoplus_{i\in I}V_i = \prod_{i\in I}V_i$.
\end{corollary}

\begin{proposition}
Let $\{\sSet{V_i, \xi_i}\}_{i\in I}$ be a set of convergence vector spaces. The inclusion map
\[ \bigoplus_{i\in I}V_i \hookrightarrow \prod_{i\in I}V_i \]
is a continuous map into a dense subspace.
\end{proposition}
\begin{proof}

\end{proof}

\begin{proposition}
Let $\{\sSet{V_i, \xi_i}\}_{i\in I}$ be a set of convergence vector spaces. The direct sum $\bigoplus_{i\in I}V_i$ is topological \textup{if and only if} all $V_i$ are topological and only finitely many are non-trivial (i.e. not $\{0\}$).
\end{proposition}

\subsubsection{Finite direct sums}
\begin{lemma} \label{directSumsOpenClosedSets}
Let $\sSet{V,\xi}, \sSet{W,\zeta}$ be convergence vector spaces and $A\subseteq V, B\subseteq W$ subsets. Then
\begin{enumerate}
\item if $A,B$ are open, then $A\oplus B$ is open;
\item if $A,B$ are closed, then $A\oplus B$ is closed.
\end{enumerate}
\end{lemma}
\begin{proof}
(1) We have that $A\oplus W = \proj_1^{\preimf}(A)$ and $V\oplus B = \proj_2^{\preimf}(B)$ are open by \ref{preimageOpenClosed}. Thus $A\oplus B = A\oplus W \cap V\oplus B$ is open by \ref{propertiesTopology}.

(2) We have that $A\oplus W = \proj_1^{\preimf}(A)$ and $V\oplus B = \proj_2^{\preimf}(B)$ are closed by \ref{preimageOpenClosed}. Thus $A\oplus B = A\oplus W \cap V\oplus B$ is closed by \ref{propertiesTopology}.
\end{proof}

\subsubsection{Internal direct sums}
\begin{definition}
Let $\sSet{V,\xi}$ be a convergence vector space and $A,B\subseteq V$ subspaces such that $A\cap B = \{0\}$. We call the internal algebraic direct sum $A\oplus^i B$ a \udef{convergence direct sum} if $+: A\oplus B \to A\oplus^i B$ is a homeomorphism.
\end{definition}

\begin{lemma} \label{sumOnDirectSumsContinuous}
Let $\sSet{V,\xi}$ be a convergence vector space and $A,B\subseteq V$ subspaces such that $A\cap B = \{0\}$. Then
\[ +: A\oplus B \to A\oplus^i B \]
is continuous.
\end{lemma}
The function $+: A\oplus B \to A\oplus^i B$ is bijective, but not necessarily homeomorphic.
\begin{proof}
Since the convergence on $A\oplus B$ is the subspace convergence of $A\oplus B$ in $V\oplus V$, by \ref{productAndSubspaceConvergencesCommute},
we have $+: A\oplus B \to A\oplus^i B = +: V\oplus V|_{A\oplus B} \to A\oplus^i B$.
Then the result follows from the definition of a convergence vector space and \ref{continuityRestrictionExpansion}.
\end{proof}

\begin{proposition} \label{internalConvergenceDirectSumEquivalents}
Let $\sSet{V,\xi}$ be a convergence vector space and $A,B\subseteq V$ subspaces such that $A\cap B = \{0\}$. The following are equivalent:
\begin{enumerate}
\item internal algebraic direct sum $A\oplus^i B$ is a convergence direct sum;
\item both $P_A: A\oplus^i B \to A: x+y \mapsto x$ and $P_B: A\oplus^i B \to B: x+y \mapsto y$ are continuous;
\item one of $P_A$ and $P_B$ is continuous;
\item the function $(P_A, P_B): A\oplus^i B \to A\oplus B$ is continuous.
\end{enumerate}
\end{proposition}
Note that $(P_A, P_B): A\oplus^i B \to A\oplus B$ is the inverse of $+: A\oplus B \to A\oplus^i B$.
\begin{proof}
$(1) \Rightarrow (2)$ We have
\[ (\proj_A: A\oplus B \to A) = (P_A: A\oplus^i B \to A)\circ \big((+): A\oplus B \to A\oplus^i B\big). \]
Since $(+)$ is a homeomorphism, the continuity of $P_A$ is equivalent to the continuity of $\proj_A$. The function $\proj_A$ is continuous by \ref{finiteDirectSumIsProduct}.

The argument for the continuity of $P_B$ is similar.

$(2) \Leftrightarrow (3)$ The function $\id_{A\oplus^i B}$ is continuous and $P_B = \id_{A\oplus^i B} - P_A$.

$(2) \Rightarrow (4)$ This is the defining property of the product and thus follows from \ref{finiteDirectSumIsProduct}.

$(4) \Rightarrow (1)$ From point (4) and \ref{sumOnDirectSumsContinuous} it follows that $+: A\oplus B \to A\oplus^i B$ is a homeomorphism.
\end{proof}

\begin{example}
Consider the space $\ell^\infty(\N)$. Let $e_n$ be the sequence defined by $e_n(k) \defeq \delta_{n,k}$. Let
\begin{itemize}
\item $V$ be the subspace of sequences $s\in \ell^\infty(\N)$ such that $s(2k+1) = 0$ for all $k\in \N$;
\item $W$ be the subspace of sequences $s\in \ell^\infty(\N)$ such that $s(2k+1) = \frac{s(2k)}{k+1}$ for all $k\in \N$.
\end{itemize}
We have $V\cap W = \{0\}$ and $V\oplus W = \ell^\infty(\N)$.
Let $P_V$ be the projector on $V$ along $W$. This projector is not continuous.

Consider $v_n \defeq e_{2n} \in V$ and $w_n \defeq e_{2n}+ \frac{1}{1+n}e_{2n+1}\in W$.
Then $v_n - w_n = -\frac{1}{1+n}e_{2n+1} \to 0$, but $P_V(v_n-w_n) = v_n = e_{2n} \not\to 0$.
\end{example}

\begin{proposition} \label{convergenceDirectSumsClosedSubspaces}
Let $\sSet{V,\xi}$ be a Hausdorff convergence vector space and $A,B\subseteq V$ subspaces such that $A\cap B = \{0\}$.
If $A\oplus^i B$ is a convergence direct sum, then $A$ and $B$ are closed subspaces of $A\oplus^i B$.
\end{proposition}
The converse holds for Banach spaces, see (TODO ref.)
\begin{proof}
The set $\{0\}\subseteq V$ is closed by \ref{HausdorffCriterionConvergenceGroup}, so $A = P_B^{\preimf}\big(\{0\}\big)$ is closed as a subset of $A\oplus^i B$ by \ref{preimageOpenClosed} since $P_A$ is continuous by \ref{internalConvergenceDirectSumEquivalents}.

The argument for $B$ is similar. 
\end{proof}

\begin{proposition}
Let $\sSet{V,\xi}$ be a Hausdorff convergence vector space and $A,B\subseteq V$ subspaces such that $A\cap B = \{0\}$. If $A$ is closed and $B$ is finite-dimensional, then $A\oplus^i B$ is a convergence direct sum.
\end{proposition}
\begin{proof}
By \ref{internalConvergenceDirectSumEquivalents}, it is enough the show that $P_B$ is continuous. We have $\ker(P_B) = A$. By the factor theorem \ref{factorTheorem},
we can decompose
\[ \begin{tikzcd}
A\oplus^i B \ar[rr, "{P_B}"] \ar[dr, "{[\cdot]_A}"] && B \\
{}& (A\oplus^i B)/A \ar[ur, "{P_B'}"] &{}
\end{tikzcd}. \]
Equipping $(A\oplus^i B)/A$ with the quotient convergence makes $[\cdot]_A$ continuous. Since $A$ is closed, $(A\oplus^i B)/A$ is Hausdorff, by \ref{quotientConvergenceGroupProperties}. Then $P_B'$ is continuous by \ref{finiteDimensionalLinearMapContinuous}. Thus $P_B = P_B'\circ [\cdot]_A$ is continuous.
\end{proof}

\subsubsection{Complemented subspaces}
\begin{definition}
Let $\sSet{V,\xi}$ be a convergence vector space and $A\subseteq V$ a subspace. Then $A$ is called \udef{complemented} if there exists another subspace $B\subseteq V$ such that
\begin{itemize}
\item $A\oplus^i B = V$; and
\item $A\oplus^i B$ is a convergence direct sum.
\end{itemize}
\end{definition}

\subsection{Inductive systems}
\begin{proposition} 
Let $\sSet{I, \{\sSet{V_i, \xi_i}\}_{i\in I}, \{e_{i,j}\}_{i\preceq j}}$ be an inductive system of convergence vector spaces with linear linking morphisms $e_{i,j}$. Then the inductive colimit $\varinjlim_{i\in I} V_i$ in the category set, equipped with the inductive colimit preconvergence structure and the inductive colimit vector space structure is a convergence vector space.
\end{proposition}
\begin{proof}
By \ref{elementsOfInductiveLimitImages} and \ref{finalConvergenceConvergence} the inductive preconvergence structure is in fact a convergence structure.

Let $\nu$ be the inductive limit convergence. We first show the continuity of the addition.
Let $F,G\in \powerfilters(\varinjlim_{i\in I} V_i)$ be filters that converge to $v$, resp., $w$. By \ref{initialFinalConvergence} there exist $F'\overset{\xi_i}{\longrightarrow} v'$ and $G'\overset{\xi_j}{\longrightarrow} w'$ such that $\upset e_i^{\imf\imf}(F')\subseteq F$ and $e_j^{\imf\imf}(G')\subseteq G$.

Now take $k\succeq i,j$. Then $e_{i,k}^{\imf\imf}(F') + e_{j,k}^{\imf\imf}(G')$ converges to $e_{i,k}(v')+e_{i,k}(w')$ by continuity of the linking morphisms and addition in $V_k$. By the continuity of $e_k$ and the construction of $v+w$ in \ref{vectorSpaceInductiveLimit}, we have
\begin{align*}
F + G &\supseteq \upset e_i^{\imf\imf}(F') + \upset e_j^{\imf\imf}(G') \\
&= \upset e_k^{\imf\imf}\big(e_{i,k}^{\imf\imf}(F')\big) + \upset e_k^{\imf\imf}\big(e_{j,k}^{\imf\imf}(G')\big) \\
&= \upset e_k^{\imf\imf}\big(e_{i,k}^{\imf\imf}(F') + e_{j,k}^{\imf\imf}(G')\big) \to e_k\big(e_{i,k}(v')+e_{i,k}(w')\big) = v + w.
\end{align*}
To show the continuity of scalar multiplication, take a filter $F\in \powerfilters(\varinjlim_{i\in I} V_i)$ that converges to $v$ and a filter $H\in \powerfilters(\F)$ that converges to $\lambda$. By \ref{initialFinalConvergence} there exist $F'\overset{\xi_i}{\longrightarrow} v'$ such that $\upset e_i^{\imf\imf}(F')\subseteq F$. By linearity, we have
\[ H \cdot F \supseteq H\cdot e_i^{\imf\imf}(F') = e_i^{\imf\imf}(H\cdot F') \to e_i(\lambda v') = \lambda e_i(v') = \lambda v. \]
\end{proof}
TODO generalise to groups and general algebraic structures.

\begin{proposition}
Let $\sSet{I, \{\sSet{V_i, \xi_i}\}_{i\in I}, \{e_{i,j}\}_{i\preceq j}}$ be an inductive system of convergence vector spaces. Let $\sSet{F, \{f_i\}_{i\in I}}$ be a cone under the inductive system such that
\begin{enumerate}
\item all $f_i$ are injective;
\item $F = \bigcup_{i\in I}f_i^\imf(V_i)$;
\item $F$ carries the final convergence structure w.r.t. $\{f_i\}_{i\in I}$.
\end{enumerate}
Then $\sSet{F, \{f_i\}_{i\in I}}$ is the inductive limit.
\end{proposition}
\begin{proof}
Let $\sSet{F', \{f_i'\}_{i\in I}}$ be another cone under the inductive system. We define a function $u: F \to F'$ by mapping $x\in F$ to the unique element of
\[ \bigcup_{i\in I}f_i^{\prime\imf}\Big(f_i^{\preimf}\big(\{x\}\big)\Big). \]
We need to show that this set indeed has exactly one element. For each $i\in I$, the set $f_i^{\prime\imf}\Big(f_i^{\preimf}\big(\{x\}\big)\Big)$ has at most one element by injectivity. There is also at least one $i\in I$ such that this set has an element by the second point. Finally let $y_i$ be the unique element in $f_i^{\prime\imf}\Big(f_i^{\preimf}\big(\{x\}\big)\Big)$ and $y_j$ the unique element in $f_j^{\prime\imf}\Big(f_j^{\preimf}\big(\{x\}\big)\Big)$. Take $k\succeq i,j$. Then, by compatibility, 
\begin{align*}
f_i^{\prime\imf}\circ f_i^{\preimf}\big(\{x\}\big) \cup f_j^{\prime\imf}\circ f_j^{\preimf}\big(\{x\}\big) &= (f_k'\circ e_{i,k})^{\imf}\circ (f_k\circ e_{i,k})^{\preimf}\big(\{x\}\big) \cup (f_k'\circ e_{j,k})^{\imf}\circ (f_k\circ e_{j,k})^{\preimf}\big(\{x\}\big) \\
&= f_k^{\prime\imf}\circ e_{i,k}^{\imf}\circ e_{i,k}^{\preimf} \circ f_k^\preimf \big(\{x\}\big) \cup f_k^{\prime\imf}\circ e_{j,k}^{\imf}\circ e_{j,k}^{\preimf} \circ f_k^\preimf \big(\{x\}\big) \\
&= f_k^{\prime\imf} \circ f_k^\preimf \big(\{x\}\big) \cup f_k^{\prime\imf}\circ f_k^\preimf \big(\{x\}\big) \\
&= f_k^{\prime\imf} \circ f_k^\preimf \big(\{x\}\big),
\end{align*}
where we have used that $e_{i,k}^{\imf}\circ e_{i,k}^{\preimf} \circ f_k^\preimf \big(\{x\}\big) = f_k^\preimf \big(\{x\}\big)$, because $f_k^\preimf \big(\{x\}\big) \subseteq \im(e_{i,k})$.

By construction, $u$ is linear and also the unique function such that $f_i' = u\circ f_i$ for all $i\in I$. Finally $u$ is continuous by \ref{characteristicPropertyInitialFinalConvergence}.
\end{proof}

\begin{proposition}
The colimit of a reduced inductive family of complete convergence vector spaces is complete.
\end{proposition}
\begin{proof}
TODO Beattie / Butzmann p. 101.
\end{proof}
\begin{corollary}
Let $\{\sSet{V_i, \xi_i}\}_{i\in I}$ be a set of complete convergence vector spaces. Then $\bigoplus_{i\in I}V_i$ is complete.
\end{corollary}

\section{Dimension, coordinates and bases}
\subsection{Finite dimensional spaces}

\begin{proposition} \label{finiteDimensionalHausdorffEuclidean}
Let $\sSet{V,\xi}$ be a finite-dimensional Hausdorff convergence vector space over a complete field $\F$. Then
\begin{enumerate}
\item every linear functional $f\in \Lin(V,\F)$ is continuous;
\item for every basis $\{x_1, \ldots, x_n\}$, the function $\F^n \to V: (\lambda_1, \ldots, \lambda_n)\mapsto \sum_{k=1}^n \lambda_k x_k$ is a homeomorphism.
\end{enumerate}
\end{proposition}
In particular, $V$ is homeomorphic to $\F^{\dim(V)}$.
\begin{proof}
We prove (1) and (2) simultaneously by induction on the dimension $n$ of $V$. First we show the base step $n=1$.

Take some linear functional $f\in \Lin(V,\F)$. If $f = \constant{0}$, then it is continuous by \ref{continuityConstructions}. If not, then there exists $v\in V$ such that $f(v) \neq 0$. Now set $w \defeq \frac{v}{f(v)}$, so $f(w) = 1$. For all $\lambda\cdot w\in \Span\{w\} = V$, we have $f(\lambda w) = \lambda f(w) = 1$.
Since $V$ is Hausdorff, $f$ is a homeomorphism by \ref{continuityLemmaVectorConvergence}. This implies that $V$ is homeomorphic to $\F$ and, in particular, that $f$ is continuous.

Now suppose (1) and (2) hold for $n-1$. First take some linear functional $f\in \Lin(V,\F)$. If $f = \constant{0}$, then it is continuous by \ref{continuityConstructions}. If not, then $\dim\big(\ker(f)\big) = n-1$, by the dimension theorem \ref{dimensionLinearMaps}. Since $\ker(f)$ is Hausdorff by \ref{HausdorffSubspace}, it is homeomorphic to $\F^{n-1}$ by the induction hypothesis. Now $\F^{n-1}$ is complete by \ref{productCompleteSpacesComplete}.

Since $V$ is Hausdorff, $\ker(f)$ is a closed subset of $V$ by \ref{closedComplete}. This implies that $f$ is continuous by \ref{continuityLinearFunctionals}.

To show $V \cong \F^n$, take a basis $\{x_1, \ldots, x_n\}$. Then the function $g: \F^n \to V: (\lambda_1, \ldots, \lambda_n)\mapsto \sum_{k=1}^n \lambda_k x_k$ is a bijection by \ref{finiteBasisUniqueDecomposition}. It is continuous by \ref{continuityLinearCombination}.  

For the continuity of $g^{-1}$, note that $\proj_k\circ g^{-1}$ is a functional, and thus continuous, for all $k\in \interval{1,n}$. Thus $g^{-1}$ is continuous by \ref{characteristicPropertyInitialFinalConvergence}.
\end{proof}
\begin{corollary} \label{finiteDimensionalLinearMapContinuous}
Let $\sSet{V,\xi}$ and $\sSet{W,\zeta}$ be convergence vector spaces over a complete field and $f: V\to W$ a linear map. If $V$ is finite-dimensional and Hausdorff, then $f$ is continuous.
\end{corollary}
\begin{proof}
Let $\{x_0,. \ldots, x_{n-1}\}$ be a basis of $V$. The function $f$ can be decomposed as
\[ \begin{tikzcd}
\sum_{k=0}^{n-1}\lambda_kx_k \ar[r, mapsto] & (\lambda_0, \ldots, \lambda_{k-1}) \ar[r, mapsto] & \sum_{k=0}^{n-1}\lambda_kf(x_k).
\end{tikzcd} \]
The first part is continuous by \ref{finiteDimensionalHausdorffEuclidean} and the second part by \ref{continuityLinearCombination}.
\end{proof}
\begin{corollary} \label{finiteDimensionalHausdorffLinearBijectionIsHomeomorphism}
Every bijective linear map between finite-dimensional Hausdorff convergence vector spaces is a homeomorphism.
\end{corollary}


\begin{example}
\url{https://terrytao.wordpress.com/2011/05/24/locally-compact-topological-vector-spaces/}

discusses some almost counterexamples to \ref{finiteDimensionalHausdorffEuclidean}
\end{example}


\begin{lemma} \label{finiteDimSubspaceClosed}
Let $\sSet{V,\xi}$ be a Hausdorff CVS over a complete field. Then any finite dimensional subspace $U$ of $V$ is closed.
\end{lemma}
\begin{proof}
Let $W$ be a finite dimensional subspace of $V$. Then $U$ equipped with the subspace convergence is a convergence vector space by \ref{initialVectorSpaceConvergence}. It is complete by \ref{finiteDimensionalHausdorffEuclidean} and \ref{productCompleteSpacesComplete}.

Since $V$ is complete, $U$ is closed by \ref{closedComplete}.
\end{proof}

\begin{lemma} \label{compactHCVSZeroDim}
Let $V$ be a compact Hausdorff vector space over $\F$. Then $V = \{0\}$.
\end{lemma}
\begin{proof}
Suppose $v\in V$ and $v\neq 0$. Then $\Span\{v\}$ is a subspace that is not compact, by \ref{continuityLemmaVectorConvergence} and the fact that $\F$ is not compact. 

Since $\Span\{v\}$ is finite-dimensional, it is closed, by \ref{finiteDimSubspaceClosed}. Thus it should be compact by \ref{compactClosedSets}. This is a contradiction.
\end{proof}

\begin{proposition}
Let $\sSet{V,\xi}$ be a Hausdorff convergence vector space over a complete and locally compact field $\F$. Then
\begin{enumerate}
\item if $V$ is finite-dimensional, then $V$ is locally compact;
\item if $V$ is locally compact and topological, then $V$ is finite-dimensional.
\end{enumerate}
\end{proposition}
In particular, this proposition applies if $\F = \R$ or $\F = \C$, by \ref{realsLocallyCompact} and \ref{propertiesRealNumbers}.

TODO generalise to convergence space?

TODO: Aliprantis / Border p.180

TODO: There exist infinite-dimensional locally compact CVS! See Beattie / Butzmann p. 133.
\begin{proof}
(1) If $V$ is finite-dimensional, then $V\cong \F^{\dim(V)}$ by \ref{finiteDimensionalHausdorffEuclidean}. Since $\F$ is locally compact, we have that $\F^{\dim(V)}$ is locally compact by \ref{productLocallyCompact}.

(2) If $V$ is locally compact and topological, then there exists a compact neighbourhood $K$ of $0$. The set $\frac{1}{2}K$ is also a neighbourhood of $0$ by \ref{vicinityFilterAtOrigin}, so $\frac{1}{2}K$ contains an open set $O$ that contains the origin. Now $k+O$ is open for all $k\in K$ by \ref{shiftHomeomorphism} and $K\subseteq \bigcup_{k\in K}k+O$. Thus the sets $k+O$ form an open cover of $K$. By compactness and \ref{topologyCompactnessOpenCover}, there exists a finite set $S\subseteq K$ such that $K\subseteq \bigcup_{k\in S}k+O$.

Set $M \defeq \Span(S)$, which is a finite-dimensional subspace and thus closed by \ref{finiteDimSubspaceClosed}. The quotient space $V/M$ is then Hausdorff by \ref{quotientConvergenceGroupProperties}. We have $K\subseteq \bigcup_{k\in S}k+O \subseteq M+O \subseteq M+\frac{1}{2}K$. Then $[K]_{M}^\imf \subseteq [\frac{1}{2}K]_{M}^\imf$, so $2[K]_M^\imf \subseteq [K]_M^\imf$.

For all $n\in \N$, we have
\[ 2^n[K]_M^\imf = 2^{n-1}\big(2[K]_M^\imf\big) \subseteq 2^{n-1}[K]_M^\imf \subseteq \ldots \subseteq [K]_M^\imf. \]
Since $K$ is absorbent (by \ref{vicinityFilterAtOrigin}), we have $V = \bigcup_{n\in \N} 2^nK$. By \ref{functionImagePreimageGaloisConnection},
\[ [K]_M \subseteq V/M = [V]_M^\imf = \Big[\bigcup_{n\in \N} 2^nK\Big]^\imf_M = \bigcup_{n\in \N}[2^nK]_{M}^\imf = \bigcup_{n\in \N}2^n[K]_{M}^\imf \subseteq [K]_M, \]
so $V/M = [K]_M$, which is compact by \ref{compactConstructions} (and the fact that $[\cdot]_M$ is continuous, i.e. the definition of the quotient convergence).

Since $V/M$ is compact, it is zero-dimensional by \ref{compactHCVSZeroDim}. By the dimension theorem \ref{dimensionTheoremQuotientSpace}, we have $\dim V = \dim M$. So $V$ is finite-dimensional.
\end{proof}

\subsection{Hamel coordinate functions}
\begin{lemma} \label{finiteNonZeroHamelCoordinateFunctions}
Let $V$ be a vector space and $\seq{e_i}_{i\in I}$ a Hamel basis of $V$. Let $\seq{\varphi_i}_{i\in I}$ be the associated coordinate functions. Then for all $v\in V$, at most finitely many $\varphi_i(v)$ are non-zero.
\end{lemma}
\begin{proof}
By definition of a Hamel basis, we have $v = \sum_{i\in I}\lambda_i e_i = \sum_{i\in I}\varphi_i(v) e_i$ and the sum must be finite.
\end{proof}
TODO: compare existence dual basis.

\begin{proposition}
Let $\sSet{V,\norm{\cdot}}$ be a Banach space and $\seq{e_i}_{i\in I}$ a Hamel basis of $V$. Let $\seq{\varphi_i}_{i\in I}$ be the associated coordinate functions. At most finitely many $\varphi_i$ are continuous.
\end{proposition}
\begin{proof}
Suppose there existed a countable sequence $\seq{e_i}_{i\in I}$ of basis elements, each with a continuous coordinate function $\varphi_n$. Consider the vector $v = \sum_{n=0}^\infty 2^{-n}e_i/\norm{e_i}$. Then, by continuity, each $\varphi_n(v)$ is non-zero. This is impossible by \ref{finiteNonZeroHamelCoordinateFunctions}.
\end{proof}



\section{Topological vector spaces}
\begin{definition}
A \udef{topological vector space} (or TVS) is a convergence vector space that is topological.
\end{definition}
As with convergence groups, any pretopological vector space convergence is topological, see \ref{pretopologicalGroupConvergence}.

\subsection{Neighbourhoods and base}
\begin{proposition} \label{TVSconstruction}
Let $V$ be a vector space and $N\in\powerfilters(V)$. Then $N = \neighbourhood_\xi(0)$ for some topological convergence on $V$ \textup{if and only if}
\begin{enumerate}
\item for all $A\in N$ and $\lambda\in \F\setminus\{0\}$: $\lambda A\in N$;
\item each $A \in N$ is absorbent;
\item $N$ has a balanced base;
\item for all $A\in N$, there exists some $B\in N$ such that $B+B\subseteq A$.
\end{enumerate}
\end{proposition}
\begin{proof}
We adapt \ref{vectorSpaceConvergenceConstruction} to the present situation.

First assume $N$ has a balanced and absorbent base. We check the five conditions for $\mathcal{F} = \pfilter{N}$. In this case the convergence will necessarily be topological by \ref{pretopologicalGroupConvergence}.

\begin{enumerate}
\item Immediate because $\mathcal{F} = \pfilter{N}$.
\item Take $F,G\in \pfilter{N}$. We need to show that $\upset (F + G) \supseteq N$, which means that for all $A \in N$ there exist $B\in F$ and $C\in G$ such that $B+C\subseteq A$. We can take $B = C$ equal to the $B$ of point (2).
\item Take $F\in \pfilter{N}$. We need to show that $\upset(\neighbourhood_\F(0)\cdot F) \supseteq N$, which means that for all $A\in N$ there exists a $\Gamma\in \neighbourhood_\F(0)$ and $B\in F$ such that $\Gamma \cdot B\subseteq A$. We can take $B = \balancedCore(A) \in N \subseteq F$ and $\Gamma = B(0,1)$.
\item Take $v\in V$. We need to show that all $A\in N$ contain $\Gamma\cdot v$ for some $\Gamma \in \neighbourhood_\F(0)$. Because $A$ is absorbent, there exists an $r>0$ such that $v\in cA$ for all $|c|\geq r$. Conversely $c^{-1}v \in A$ for all $|c^{-1}| \leq r^{-1}$. So $\ball(0,r^{-1})\cdot v \subseteq A$ and $B(0,r^{-1}) \in \neighbourhood_\F(0)$.
\item Take $F\in \pfilter{N}$ and $\lambda\in \F$. We need to show that for all $A\in N$ there exists a $B\in F$ such that $\lambda\cdot B\subseteq A$. We can take $B = \lambda^{-1}A \in N\subseteq F$.
\end{enumerate}

Now assume $\xi$ is a topological vector space convergence and $N = \neighbourhood_\xi(0)$. The first 3 points follow from \ref{vicinityFilterAtOrigin}. The fourth from \ref{vicinityFactorisation}.
\end{proof}
\begin{corollary} \label{TVSbase}
Let $V$ be a vector space and $\mathcal{B}\subseteq\powerset(V)$. If $\mathcal{B}$ is such that
\begin{enumerate}
\item all $A\in \mathcal{B}$ are balanced and absorbent;
\item for each $A\in\mathcal{B}$, there exists some $B\in \mathcal{B}$ such that $B+B\subseteq A$;
\end{enumerate}
then $\mathfrak{F}(\mathcal{B}) = \neighbourhood_\xi(0)$ for some topological convergence on $V$.
\end{corollary}
\begin{proof}
We verify the 4 points of the proposition:
\begin{enumerate}
\item From point (2) of the hypothesis, we can prove by induction that for all $A\in\mathcal{B}$ and $n\in \N$, there exists $B\in \mathcal{B}$ such that $2^nB \subseteq A$.

Now take arbitrary $A\in\mathfrak{F}(\mathcal{B})$ and $\lambda\in\F\setminus\{0\}$. Then there exists a finite set $\{B_i\}_{i=0}^k\subseteq \mathcal{B}$ such that $B_0\cap \ldots \cap B_k \subseteq A$. Pick $n\in\N$ such that $2^{-n}\leq |\lambda|$ (and thus $|\lambda^{-1}2^{-n}| \leq 1$). Then we can find a finite set $\{C_i\}_{i=0}^k\subseteq \mathcal{B}$ such that $C_i \subseteq 2^{-n}B_i$. As each $B_i$ is absorbent, we have
\[ C_i \subseteq 2^{-n}B_i = \lambda (\lambda^{-1}2^{-n}) \subseteq \lambda B_i. \]
Thus $(C_0\cap \ldots \cap C_k) \subseteq \lambda (C_0\cap \ldots \cap C_k) \subseteq \lambda A$.
\item By \ref{absorbingSetProperties}, finite intersections of absorbent sets are absorbent. Thus each element of $\mathfrak{F}(\mathcal{B})$ contains an absorbent set and, by \ref{absorbingSetProperties}, is also absorbent.
\item Each element of $\mathfrak{F}(\mathcal{B})$ contains a finite intersection of balanced sets, which is also balanced by \ref{balancedLemma}.
\item Take arbitrary $A\in\mathfrak{F}(\mathcal{B})$. Then there exists a finite set $\{B_i\}_{i=0}^k\subseteq \mathcal{B}$ such that $B_0\cap \ldots \cap B_k \subseteq A$. By assumption, we can find a finite set $\{C_i\}_{i=0}^k\subseteq \mathcal{B}$ such that $C_i + C_i \subseteq B_i$. Then, using \ref{orderPreservingFunctionLatticeOperations}, we have
\[ \Big(\cap_{i\leq k}C_i\Big) + \Big(\cap_{j\leq k}C_j\Big) \subseteq \bigcap_{i,j\leq k} C_i + C_j \subseteq \bigcap_{i\leq k}C_i + C_i \subseteq \bigcap_{i\leq k} B_i \subseteq A. \]
\end{enumerate}
\end{proof}



\section{Properties of subsets}
\begin{lemma} \label{sumOpenSetsOpen}
Let $\sSet{V,\xi}$ be a vector space convergence and $A,B\subseteq V$ subsets. If $A$ is open, then $A+B$ is open.
\end{lemma}
\begin{proof}
We have $A + B = \bigcup_{b\in B} A+b$. Now each $A+b$ is open because adding $b$ is a homeomorphism and thus the union is open by \ref{propertiesTopology}.
\end{proof}

\begin{proposition} \label{inherenceAdherenceBalanced}
Let $\sSet{V,\xi}$ be a vector space convergence and $A\subseteq V$. Then
\begin{enumerate}
\item if $A$ is balanced, then $\adh(A)$ is balanced;
\item if $A$ is balanced and $0\in\inh(A)$, then $\inh(A)$ is balanced;
\item if $A$ is open and contains the origin, then $\balanced(A)$ is open.
\end{enumerate}
\end{proposition}
\begin{proof}
(1) We use \ref{productAdherence} and \ref{adherenceInherenceContinuity} to compute
\begin{align*}
\cball(0,1)\cdot \adh_\xi(A) &= \cdot^\imf[\adh_\F(\cball(0,1))\times \adh_\xi(A)] = \cdot^\imf[\adh_{\F\otimes \xi}(\cball(0,1)\times A)] \\
&\subseteq \adh_{\xi}\big(\cdot^\imf[\cball(0,1)\times A]\big) = \adh_{\xi}(\cball(0,1)\cdot A) = \adh_\xi(A).
\end{align*}

(2) Take $0<r<1$. Then multiplying by $r$ is a homeomorphism and thus $r\cdot\inh(A) = \inh(r\cdot A) \subseteq \inh(A)$. Now take $r=0$. If $0\in\inh(A)$, then
\[ r\cdot\inh(A) = \{0\} \subseteq \inh(A). \]

(3) For all $r\neq 0$, $r\cdot A$ is open. Thus
\[ \balanced(A) = \bigcup_{|r|\leq 1}r\cdot A = \{0\}\cup \bigcup_{0< r\leq 1}r\cdot A = \bigcup_{0< r\leq 1}r\cdot A \]
is a union of open sets and thus open by \ref{propertiesTopology}.
\end{proof}
\begin{proof}[Alternative proof of (1)]
Take $|\lambda|\leq 1$ and $v\in \adh_\xi(A)$, then we need to show that $\lambda v \in \adh_\xi(A)$. We have $A\in \vicinity_\xi(v)^{\mesh}$ and $A\subseteq \lambda^{-1}A$ (TODO: why, is this correct??). So for all $B\in \vicinity_\xi(v)$:
\[ A\mesh B \quad\implies\quad \lambda^{-1}A\mesh B \quad\implies\quad A\mesh \lambda B. \]
Thus $A\in \vicinity_\xi(\lambda v)^{\mesh}$, which is what we needed to show by \ref{principalAdherenceInherence}.
\end{proof}

\begin{proposition} \label{inherenceAdherenceCone}
Let $\sSet{V,\xi}$ be a vector space convergence and $K\subseteq V$ a cone. Then
\begin{enumerate}
\item $\adh(K)$ is a cone;
\item $\inh(K)$ is a cone;
\item if $K\neq \emptyset$, then $0\in \adh(K)$;
\item if $K$ is salient, then $0\notin \inh(K)$.
\end{enumerate}
\end{proposition}
\begin{proof}
(1) For all $r\in \interval[o]{0,+\infty}$, multiplying by $r$ is a homeomorphism. Then
\[ \interval[o]{0,+\infty}\cdot \adh(K) = \bigcup_{r> 0}r\cdot \adh(K) = \bigcup_{r>0}\adh(r\cdot K) \subseteq \bigcup_{r>0}\adh(K) = \adh(K). \]

(2) For all $r\in \interval[o]{0,+\infty}$, multiplying by $r$ is a homeomorphism. Then
\[ \interval[o]{0,+\infty}\cdot \inh(K) = \bigcup_{r> 0}r\cdot \inh(K) = \bigcup_{r>0}\inh(r\cdot K) \subseteq \bigcup_{r>0}\inh(K) = \inh(K). \]

(3) Assume $K\neq \emptyset$. Then we can take $v\in K$. Now $\seq{n^{-1}v}$ is a sequence in $K$ that converges to $0$ by \ref{continuityLemmaVectorConvergence}. So $0\in \adh(K)$.

(4) Assume $K$ is salient. Suppose, towards a contradiction, that $0\in \inh(K)$. Then $K\in \vicinity(0)$ and so $K$ is absorbent by \ref{vicinityFilterAtOrigin}.

Take $v\in V$. By absorption, we have $\ball(0,\epsilon)\cdot\{v\} \subseteq K$ for some $\epsilon > 0$. In particular both $\epsilon v\in K$ and $-\epsilon v\in K$. Thus $K$ is flat, not salient.
\end{proof}

\begin{proposition} \label{inherenceAdherenceConvex}
Let $\sSet{V,\xi}$ be a vector space convergence and $A\subseteq V$. Then
\begin{enumerate}
\item if $A$ is convex, then $\adh(A)$, $\inh(A)$ and $\interior(A)$ is convex;
\item if $A$ is open, then $\convex(A)$ is open.
\end{enumerate}
\end{proposition}
TODO same for closure?
\begin{proof}
(1) For all $0<r<1$, we have
\[ r\adh(A) + (1-r)\adh(A) = \adh(rA) + \adh\big((1-r)A\big) \subseteq \adh\big(rA + (1-rA)\big) \subseteq \adh(A) \]
by \ref{vectorSumInherenceAdherence}. 

The argument for $\inh(A)$ and $\interior(A)$ is similar.

(2) We have $\convex(A) = \bigcup_{0\leq r\leq 1} rA + (1-r)A$ by \ref{convexHullLemma}. For all $0\leq r\leq 1$, we have that $rA + (1-r)A$ is open by \ref{sumOpenSetsOpen}, so the union is open by \ref{propertiesTopology}.
\end{proof}

\begin{lemma}
Let $\sSet{V,\xi}$ be a convergence vector space, $C\subseteq V$ a convex subset and $0<r\leq 1$. Then
\[ r\inh_\xi(C) + (1-r)\adh_\xi(C) \subseteq \inh_\xi(C). \]
\end{lemma}
TODO also for algebraic convergence!
\begin{proof}
Take $v\in \inh_\xi(C)$ and $w\in \adh_\xi(C)$. Then, by \ref{principalAdherenceInherence}, $C\in \vicinity_\xi(v)$. Then $(C- v)\in \vicinity_\xi(0)$, so $-\frac{r}{1-r}(C-v)\in \vicinity_\xi(0)$ and $w - \frac{r}{1-r}(C-v) \in \vicinity_\xi(w)$. Again by \ref{principalAdherenceInherence}, we have $C \mesh \big(w - \frac{r}{1-r}(C-v)\big)$, so we can take $u \in C \cap \big(w - \frac{r}{1-r}(C-v)\big)$. Since $u\in w - \frac{r}{1-r}(C-v)$, we can write $u = w - \frac{r}{1-r}(u'-v)$ for some $u'\in C$. Rearranging gives $rv + (1-r)w = ru'+ (1-r)u \in C$.
\end{proof}
\begin{corollary} \label{adherenceInherenceClosureConvexSets}
Let $\sSet{V,\xi}$ be a convergence vector space and $C\subseteq V$ a convex subset with nonempty interior. Then
\begin{enumerate}
\item $\adh_\xi\big(\inh_\xi(C)\big) = \adh_\xi(C)$;
\item $\inh_\xi\big(\adh_\xi(C)\big) = \inh_\xi(C)$.
\end{enumerate}
\end{corollary}
The first point says that $\inh_\xi(C)$ is dense in $\adh_\xi(C)$.
\begin{proof}
(1) We have $\adh_\xi\big(\inh_\xi(C)\big) \subseteq \adh_\xi(C)$ by \ref{principalInherenceAdherenceProperties}.

Now take $v\in \adh_\xi(C)$. To show $v\in \adh_\xi\big(\inh_\xi(C)\big)$, we look for a proper filter that converges to $v$ and contains $\inh_\xi(C)$. As $\inh_\xi(C)$ was assumed nonempty, we can find $w\in \inh_\xi(C)$. Consider the tail filter of $\seq{n^{-1}w + (1-n^{-1})v}_{n\in \N}$. This filter clearly converges to $v$ and each term of the sequence is contained in $\inh_\xi(C)$ by the lemma.

(2) We have $\inh_\xi(C) \subseteq \inh_\xi\big(\adh_\xi(C)\big)$ by \ref{principalInherenceAdherenceProperties}.

Now take $v\in \inh_\xi\big(\adh_\xi(C)\big)\subseteq \adh_\xi(C)$, which implies that $\adh_\xi(C)\in \vicinity_\xi(x)$ and thus $\adh_\xi(C)-x\in \vicinity_\xi(0)$, which is absorbent by \ref{vicinityFilterAtOrigin}. Now take $w\in \inh_\xi(C)$. By arbsorbence, there exists $0<\epsilon \leq 1$ such that $\epsilon(v-w)\in \adh_\xi(C)-v$. Thus $v+ \epsilon(v-w)\in \adh_\xi(C)$. Also
\[ v - \epsilon(v-w) = (1-\epsilon)v - \epsilon w \in \inh_\xi(C). \]
Thus we have
\[ v = \frac{1}{2}\big(v - \epsilon(v-w)\big) + \frac{1}{2}\big(v + \epsilon(v-w)\big)\in \inh_\xi(C). \]
\end{proof}

\begin{proposition}
Let $\sSet{V,\xi}$ be a vector space convergence and $A\subseteq V$ a subspace. Then $\adh(A)$ is a subspace.
\end{proposition}
\begin{proof}
Clearly $\adh_\xi(A)$ is not empty. It is then enough to note that $\adh_\xi(A)+\adh_\xi(A)\subseteq \adh_\xi(A)$, by \ref{vectorSumInherenceAdherence}, and $\F\cdot \adh_\xi(A) \subseteq \adh_\xi(A)$, for which we use \ref{productAdherence} and \ref{adherenceInherenceContinuity} to compute
\begin{align*}
\F\cdot \adh_\xi(A) &= \cdot^\imf[\adh_\F(\F)\times \adh_\xi(A)] = \cdot^\imf[\adh_{\F\otimes \xi}(\F\times A)] \\
&\subseteq \adh_{\xi}\big(\cdot^\imf[\F\times A]\big) = \adh_{\xi}(\F\cdot A) = \adh_\xi(A).
\end{align*}
\end{proof}
\begin{corollary} \label{hyperplaneClosedDense}
A hyperplane in a convergence vector space is either closed or dense, but not both.
\end{corollary}
\begin{proof}
Let $H$ be a hyperplane in a vector space $V$. Then $H \subseteq \adh(H)$ and $\adh(H)$ is a subspace. Because $H$ is a coatom, we have either $\adh(H) = H$ or $\adh(H) = V$. In the first case $H$ is closed, in the second dense.

If the hyperplane were both closed and dense, then $H = \adh(H) = V$, which is a contradiction.
\end{proof}



\chapter{Algebraic convergence}
\begin{definition}
Let $V$ be a vector space over a field $\F$. The \udef{algebraic convergence} $\mathfrak{a}$ on $V$ is defined by
\[ F\overset{\mathfrak{a}}{\longrightarrow} v \quad\defequiv\quad \exists w\in V: \; \neighbourhood_\F(0)\cdot w \subseteq F - v \]
for all $F\in \powerfilters(V)$ and $v\in V$.
\end{definition}

\begin{lemma} \label{algebraicConvergenceStrongerThanVectorConvergence}
Let $V$ be a vector space over a field $\F$ and $\xi$ a vector space convergence on $V$. Then $\mathfrak{a} \subseteq \xi$.
\end{lemma}
\begin{proof}
This is immediate because $\neighbourhood_\F(0)\cdot w \overset{\xi}{\longrightarrow} 0$ and $F-v\overset{\xi}{\longrightarrow}0$ implies $F\overset{\xi}{\longrightarrow} v$.
\end{proof}

\begin{lemma}
Let $V$ be a vector space over a field $\F$. Then the following are equivalent:
\begin{enumerate}
\item $V$ is 1D;
\item $\mathfrak{a}$ is topological;
\item $\mathfrak{a}$ is pretopological;
\item $\mathfrak{a}$ is pseudotopological;
\item $\mathfrak{a}$ is of finite depth;
\item $\mathfrak{a}$ is a vector space convergence.
\end{enumerate}
\end{lemma}
\begin{proof}
If $V$ is 1D, then $V= \Span\{v\}$ for some/any $v\in V$ and $\neighbourhood_F(0)\cdot w = \neighbourhood_F(0)\cdot v$ for all $w\in v$. Thus $\sSet{V, \mathfrak{a}}$ is homeomorphic with $\F$ and $(2),(3),(4),(5),(6)$ follow.

Now $(2)\Rightarrow (3) \Rightarrow (4) \Rightarrow (5)$ is immediate by \ref{depthImplications}, so we prove $(5)\Rightarrow (1)$. Take arbitrary $v,w\in V$. Then $\neighbourhood_\F(0)\cdot v \cap \neighbourhood_\F(0)\cdot w$ converges to $0$ by finite depth, so there exists $u\in V$ such that $\neighbourhood_\F(0)\cdot u \subseteq \neighbourhood_\F(0)\cdot v\cap \neighbourhood_\F(0)\cdot w$. Thus for any $\epsilon > 0$, there exist $\epsilon_1, \epsilon_2 >0$ such that
\[ \cball(0,\epsilon_1)v \cup \cball(0,\epsilon_2)w \subseteq \cball(0,\epsilon)u. \]
In particular, $\epsilon_1v = \lambda_1 u$ and $\epsilon_2w = \lambda_2 u$ for some $\lambda_1,\lambda_2 \in\F$, which implies that $v,w$ are linearly dependent. If any two vectors are linearly dependent, then the vector space is 1D.

Finally we prove $(6)\Rightarrow (1)$. Assume that $V$ is not 1D, so we can find linearly independent $v,w\in V$. Construct the sequence $\seq{\frac{v+nw}{n^2}}_{n\in \N}$, which converges in any convergence vector space. Any set in the tails filter contains a pair of linearly independent vectors, so the tails filter does not contain a filter of the form $\neighbourhood_\F(0)\cdot u$ and thus does not converge in $\mathfrak{a}$, which is a contradiction.
\end{proof}

\begin{lemma} \label{constructionsInAlgebraicConvergence}
Let $V$ be a vector space over a field $\F$, $v\in V$ and $A\subseteq V$ a subset. Then
\begin{enumerate}
\item $w\mapsto w+v$ is a homeomorphism;
\item $\vicinity_\mathfrak{a}(v) = v+\vicinity_\mathfrak{a}(0)$;
\item $\begin{aligned}[t]
\vicinity_\mathfrak{a}(0) &= \bigcap_{v\in V} \upset \neighbourhood_\F(0)\cdot v \\
&= \setbuilder{B\in \powerset(V)}{\forall v\in V: \exists \Gamma_v\in \neighbourhood_\F(0):\; \Gamma_v\cdot v\subseteq B} \\
&= \setbuilder{\bigcup_{v\in V} \Gamma_v\cdot v}{\forall v\in V:\; \Gamma_v \in \neighbourhood_\F(0)};
\end{aligned}$
\item $\inh_\mathfrak{a}(A) = \setbuilder{x\in V}{\forall v\in V:\exists \Gamma_v \in \neighbourhood_\F(0):\; x + \Gamma_v\cdot v \subseteq A}$;
\item $\adh_\mathfrak{a}(A) = \setbuilder{x\in V}{\exists v\in V: \forall \Gamma\in\neighbourhood_\F(0):\; (x+\Gamma\cdot v)\mesh A}$.
\end{enumerate}
\end{lemma}
\begin{proof}
(1, 2, 3) are immediate from the definition and \ref{homeomorphismPreservation}.

(4) We have $\inh_\mathfrak{a}(A) = \setbuilder{x}{A\in \vicinity_\mathfrak{a}(x)} = \setbuilder{x}{A-x\in \vicinity_\mathfrak{a}(0)}$. From (1) we get 
\begin{align*}
\inh_\mathfrak{a}(A) &= \setbuilder{x}{\forall v\in V: \exists \Gamma_v\in \neighbourhood_\F(0): \Gamma_v\cdot v \subseteq A-x} \\
&= \setbuilder{x}{\forall v\in V: \exists \Gamma_v\in \neighbourhood_\F(0): x + \Gamma_v\cdot v \subseteq A}.
\end{align*}

(5) We calculate
\begin{align*}
\adh_\mathfrak{a}(A) &= \big(\inh_\mathfrak{a}(A^c)\big)^c \\
&= \setbuilder{x\in V}{\exists v\in V:\forall \Gamma \in \neighbourhood_\F(0):\; \neg(x + \Gamma\cdot v \subseteq A^c)} \\
&= \setbuilder{x\in V}{\exists v\in V:\forall \Gamma \in \neighbourhood_\F(0):\; (x + \Gamma\cdot v) \mesh A}.
\end{align*}
\end{proof}

\begin{lemma}
Let $V$ be a vector space. Then every subspace $U\subseteq V$ is algebraically closed.
\end{lemma}
\begin{proof}
We need to show that $\adh_\mathfrak{a}(U)\subseteq U$. Take $x\in \adh_\mathfrak{a}(U)$. Then take a $v\in V$ such that $\forall \Gamma\in\neighbourhood_\F(0):\; (x+\Gamma\cdot v)\mesh U$.

Pick some $\Gamma\in\neighbourhood_\F(0)$. Then $x+\lambda v\in U$ for some $\lambda\in \Gamma$. Then take $\ball(0,|\lambda|/2)\in \neighbourhood_\F(0)$, so $x+\mu v\in U$ for some $\mu\in \ball(0,|\lambda|/2)$. In particular $\lambda \neq \mu$. If either $\lambda =0$ or $\mu = 0$, then $x\in U$ and we are done. Suppose $\lambda\neq 0 \neq \mu$. Then
\[ \lambda^{-1}(x+\lambda v) - \mu^{-1}(x+\mu v) = (\lambda^{-1} - \mu^{-1})x \in U. \]
So $x\in U$.
\end{proof}

\begin{lemma}
Let $V$ be a vector space  over a field $\F$, $\sSet{W,\zeta}$ a convergence vector space over $\F$ and $f: V\to W$ a function. If $f$ is homogeneous, then $f: \sSet{V, \mathfrak{a}}\to \sSet{W,\zeta}$ is continuous at $0$.
\end{lemma}
\begin{proof}
Suppose $F\overset{\mathfrak{a}}{\longrightarrow} 0$, so there exists $v\in V$ such that $F\supseteq \neighbourhood_\F(0)\cdot v$. Then
\[ \upset f^{\imf\imf}(F) \supseteq \upset f^{\imf\imf}\big(\neighbourhood_\F(0)\cdot v\big) = \neighbourhood_\F(0)\cdot f(v) \overset{\zeta}{\longrightarrow} 0. \]
\end{proof}

\section{The core}
\begin{definition}
Let $V$ be a vector space and $A\subseteq V$ a subset. Then algebraic inherence $\inh_\mathfrak{a}(A)$ is also called the \udef{core} of $A$.
\end{definition}

\begin{proposition} \label{coreProperties}
Let $V$ be a vector space and $A \subseteq V$ a subset. Then
\begin{enumerate}
\item $A$ is absorbing \textup{if and only if} $0\in \inh_\mathfrak{a}(A)$;
\item if $A$ is convex, then $\inh_\mathfrak{a}(A)$ is convex;
\item if $A$ is balanced, then $\inh_\mathfrak{a}(A)$ is balanced.
\end{enumerate}
\end{proposition}
\begin{proof}
(1) We have that
\begin{align*}
\text{$A$ is absorbing} &\iff \forall v\in V: \exists \epsilon >0: \; \ball(0,\epsilon)\cdot v\subseteq A \\
&\iff \forall v\in V: \exists \Gamma \in \neighbourhood_\F(0): \; \Gamma\cdot v\subseteq A \\
&\iff 0\in \inh_\mathfrak{a}(A).
\end{align*}

(2) Take $x,y \in \inh_\mathfrak{a}(A)$. Take an arbitrary $v\in V$. Then there exist $\Gamma_v,\Gamma_v'\in \neighbourhood_\F(0)$ such that $x+ \Gamma_v\cdot v \subseteq A$ and $y+ \Gamma_v'\cdot v \subseteq A$. Then
\begin{align*}
\lambda x+(1-\lambda)y + (\Gamma_v\cap\Gamma_v')\cdot v &= \lambda x+(1-\lambda)y + (\Gamma_v\cap\Gamma_v')\big(\lambda v+(1-\lambda)v\big) \\
&= \lambda \big(x+ (\Gamma_v\cap\Gamma_v')\cdot v\big) + (1-\lambda)\big(y+ (\Gamma_v\cap\Gamma_v')\cdot v\big) \\
&\subseteq \lambda \big(x+ \Gamma_v\cdot v\big) + (1-\lambda)\big(y+ \Gamma_v'\cdot v\big) \\
&\subseteq A,
\end{align*}
where the last inclusion follows from \ref{convexCriteria}. Since $\Gamma_v\cap\Gamma_v'\in\neighbourhood_\F(0)$, we have that $\lambda x+(1-\lambda)y\in \inh_\mathfrak{a}(A)$.

(3) Take $x\in \inh_\mathfrak{a}(A)$ and $|r|\leq 1$. We need to show that $rx\in \inh_\mathfrak{a}(A)$. To that end, take arbitrary $v\in V$. Then there exists $\Gamma_v\in\neighbourhood_\F(0)$ such that $x+\Gamma_v\cdot v\subseteq A$. Now
\[ rx+ r\Gamma_v\cdot v = r\big(x+\Gamma_v\cdot v\big) \subseteq rA \subseteq A. \]
Since $r\Gamma_v\in \neighbourhood_\F(0)$, this implies $rx\in \inh_\mathfrak{a}(A)$.
\end{proof}

\begin{proposition} \label{algebraicallyOpen}
Let $V$ be a vector space and $A \subseteq V$ an algebraically open subset. Then
\begin{enumerate}
\item $A+U$ is algebraically open for any subspace $U\subseteq V$;
\end{enumerate}
\end{proposition}
\begin{proof}
TODO
\end{proof}

\begin{lemma} \label{rectangleSubsetLemma}
Let $V$ be a real vector space and $U\subseteq V$ an algebraically open convex subset. For all $a\in U$ and $u,v\in V$, there exists $\epsilon > 0$ such that
\[ a + \interval{0,\epsilon}\cdot u + \interval{0,\epsilon}\cdot v \subseteq U. \]
\end{lemma}
\begin{proof}
We have $0\in a-U = \inh_{\mathfrak{a}}(a-U)$, by \ref{constructionsInAlgebraicConvergence}, so $a-U$ is absorbing by \ref{coreProperties}.

Thus there exist $\epsilon_1,\epsilon_2 > 0$ such that
\[ \interval{0,\epsilon_1}\cdot \{u\} \subseteq \cball(0,\epsilon_1)\cdot \{u\} \subseteq U-a \]
and
\[ \interval{0,\epsilon_2}\cdot \{v\} \subseteq \cball(0,\epsilon_2)\cdot \{v\} \subseteq U-a. \]
Now set $\epsilon \defeq \frac{1}{2}\min\{\epsilon_1, \epsilon_2\}$, so $a + 2\interval{0,\epsilon}\cdot u \subseteq U$ and $a + 2\interval{0,\epsilon}\cdot v \subseteq U$. By convexity,
\[ a + \interval{0,\epsilon}\cdot u + \interval{0,\epsilon}\cdot v = \frac{1}{2}\big(a + 2\interval{0,\epsilon}\cdot u\big) + \frac{1}{2}\big(a + 2\interval{0,\epsilon}\cdot v\big) \subseteq \frac{1}{2}U + \frac{1}{2}U \subseteq U. \]
\end{proof}

\subsection{Strict convexity}
\begin{definition}
Let $V$ be a vector space and $X\subseteq V$ a subset. Then $X$ is called \udef{strictly convex} if
\[ \forall x\neq y \in X: t\in\interval[o]{0,1}: \quad tx + (1-t)y\in \inh_\mathfrak{a}(X). \]
\end{definition}

\begin{proposition} \label{strictConvexityEquivalentsConvexSubset}
Let $V$ be a vector space and $X\subseteq V$ a convex subset.
Then the following are equivalent:
\begin{enumerate}
\item $X$ is strictly convex;
\item $X\setminus \ext(X) \subseteq \inh_\mathfrak{a}(X)$;
\item $X\setminus \inh_\mathfrak{a}(X) \subseteq \ext(X)$.
\end{enumerate}
\end{proposition}
\begin{proof}
$(1) \Rightarrow (2)$ Suppose $X$ is strictly convex and take $a\in X\setminus \ext(X)$. By \ref{notExtremePointLemma} there exists $x\neq y\in X$ and $t\in\interval[o]{0,1}$ such that $a  = tx+(1-t)y$. Then, by assumption, $a\in \inh_\mathfrak{a}(X)$.

$(2) \Rightarrow (1)$ Take arbitrary $x\neq y\in X$ and $t\in \interval[o]{0,1}$. Then $tx+(1-t)y\in X$ by convexity, but $tx+(1-t)y\notin \ext(X)$ by \ref{notExtremePointLemma}. So $tx+(1-t)y \in X\setminus \ext(X)$ and thus $tx+(1-t)y\in \inh_\mathfrak{a}(X)$ by assumption.

$(2) \Leftrightarrow (3)$ Immediate.
\end{proof}
\begin{corollary}
Let $\sSet{V,\xi}$ be a convergence vector space and $X\subseteq V$ a convex subset. Then
\begin{enumerate}
\item if $X\setminus \inh_\xi \subseteq \ext(X)$, then $X$ is strictly convex;
\item if $X$ is open, then $X$ is strictly convex.
\end{enumerate}
\end{corollary}
\begin{proof}
(1) Since $\mathfrak{a}\leq \xi$, by \ref{algebraicConvergenceStrongerThanVectorConvergence}, we have $\inh_\xi(X)\subseteq \inh_\mathfrak{a}(X)$, by \ref{principalInherenceAdherenceProperties}. Thus $X\setminus \inh_\xi \subseteq \ext(X)$ implies $X\setminus \inh_\mathfrak{a}\subseteq X\setminus \inh_\xi \subseteq \ext(X)$ and thus that $X$ is strictly convex by the proposition.

(2) If $X$ is $\xi$-open, then it is $\mathfrak{a}$-open by \ref{openClosedConvergenceInclusions} and \ref{algebraicConvergenceStrongerThanVectorConvergence}. Thus $X = \inh_\mathfrak{a}(X)$, so $X\setminus \inh_\mathfrak{a}(X) = \emptyset \subseteq \ext(X)$.
\end{proof}

\section{The algebraic dual}
\begin{definition}
Let $V$ be a vector space. The \udef{algebraic dual} of $V$ is the set $\contLin(\sSet{V,\mathfrak{a}}, \F)$, where $\mathfrak{a}$ is the algebraic convergence. It is denoted $\dual{V}$.
\end{definition}

\begin{proposition} \label{algebraicDual}
Let $V$ be a vector space. Then the algebraic dual of $V$ is the set of all linear functionals: $V^* = \Lin(V,\F)$.

Thus $V^* \supseteq \sSet{V,\xi}^*$ for all vector space convergences $\xi$ on $V$.
\end{proposition}
\begin{proof}
We need to show that all linear functionals are continuous when $V$ is equipped with the algebraic convergence. Assume $F\overset{\mathfrak{a}}{\longrightarrow} x$. Then there exists a $v\in V$ such that $\neighbourhood_\F(0)\cdot v+x \subseteq F$ and so $\neighbourhood_\F(0)\cdot f(v)+f(x) \subseteq f^\imf[F]$, meaning $f^\imf[F] \overset{\F}{\longrightarrow} f(x)$. Thus $f$ is continuous.
\end{proof}


\begin{proposition} \label{dualBasisDimension}
Let $V$ be a vector space. Then $\dim V^* \geq \dim V$ and
\[ \dim V^* = \dim V \iff \text{$V$ is finite-dimensional}. \]
If $V$ is finite-dimensional with a basis $v_1, \ldots, v_n$, then the \udef{dual basis} $\varphi_1, \ldots, \varphi_n$ is the set of linear functionals on $V$ such that
\[ \varphi_j(v_k) = \begin{cases}
1 & (k=j), \\ 0 & (k\neq j)
\end{cases}. \]
This dual basis is indeed a basis of $V^*$.
\end{proposition}
\begin{proof}
We first assume $V$ is finite-dimensional and prove the dual basis is a basis, which proves $\dim V^* = \dim V$. We then assume $V$ is infinite-dimensional and prove $\dim V^* \neq \dim V$.\footnote{Reference: \url{https://mathoverflow.net/questions/13322/slick-proof-a-vector-space-has-the-same-dimension-as-its-dual-if-and-only-if-i}}
\begin{enumerate}
\item Assume $V$ is finite-dimensional. To show the dual basis spans $V^*$, take a linear functional $\varphi$. Now define $a_i = \varphi(v_i)$. It is clear that $\varphi = \sum_{i=1}^n a_i\varphi_i$. To show linear independence, take a combination
\[ b_1\varphi_1 + \ldots +b_n\varphi_n =0. \]
Filling in all basis vectors $v_i$ in turn, gives $b_i=0$ for all $i$.
\item Assume $V$ is infinite-dimensional. At first let us assume $\dim_{\mathbb{F}}V \geq |\mathbb{F}|$. Then we can apply lemma \ref{vsCardinality} to obtain $\dim_{\mathbb{F}}V = |V|$. Let $\beta$ be a basis for $V$. The elements of $V^*$ correspond bijectively to functions from $\beta$ to $\mathbb{F}$. Thus
\[ |V^*| = |\mathbb{F}^\beta| = |\mathbb{F}|^{|\beta|} > |\beta| = |V|. \]
Now we relax the condition $\dim_{\mathbb{F}}V \geq |\mathbb{F}|$. We first note that every field contains a subfield that is at most denumerable. Take such a field $K\subset \mathbb{F}$. We introduce the new vector space $W = \Span_K(\beta)$. Every functional from $W$ to $K$ extends to a functional from $V$ to $\mathbb{F}$. Hence
\[ \dim_\mathbb{F} V = \dim_K W < \dim_K W^* \leq \dim_{\mathbb{F}} V^* \]
using $\dim_{K}W \geq |K| \geq \aleph_0$.
\end{enumerate}
\end{proof}
\begin{corollary}
Let $V$ be a finite-dimensional vector space. Then the algebraic convergence is the unique Hausdorff vector space convergence on $V$.
\end{corollary}
\begin{proof}
Consider a basis $v_1, \ldots, v_n$ of $V$ with dual basis $\varphi_1, \ldots, \varphi_n$. Let $\xi$ be some vector space convergence. By definition we have $\mathfrak{a} \subseteq \xi$. Now take $F \overset{\xi}{\longrightarrow} v$. We have $F = v_1\cdot \varphi_1^{\imf\imf}[F] + \ldots + v_n\cdot \varphi_n^{\imf\imf}[F]$. Now each $\varphi_1^{\imf\imf}[F]$ converges in both $\mathfrak{a}$ and $\xi$ by \ref{algebraicDual} and the proposition, so by continuity of addition and scalar multiplication, $F$ also converges in $\mathfrak{a}$. 
\end{proof}
\begin{corollary}
Let $\sSet{V,\xi}$ be a convergence vector space. If $V$ is finite-dimensional, then $\sSet{V,\xi}^* = \sSet{V,\mathfrak{a}}^*$.
\end{corollary}
\begin{proof}
We have $\sSet{V,\xi}^* \subseteq \sSet{V,\mathfrak{a}}^*$ by \ref{algebraicDual}. Because $V$ is finite-dimensional, we obtain equality equality of space from equality of dimension by \ref{vectorSpaceEquality}.
\end{proof}

\begin{proposition} \label{algebraicDualComplete}
Let $V$ be a vector space over the field $\F$. Then $\Lin(V,\F) = \dual{\sSet{V, \mathfrak{a}}}$ is pointwise-complete.
\end{proposition}
\begin{proof}
Let $F\in\powerfilters\big(\Lin(V,\F)\big)$ be a Cauchy filter. Take a $v\in V$ then $\evalMap_v: \Lin(V,\F)\to \F$ is continuous and linear, which means it is uniformly continuous by \ref{uniformContinuityGroupHomomorphism}. Thus $\upset\evalMap_v^{\imf\imf}(F)$ is Cauchy by \ref{continuousImageOfCauchy}. Now $\upset\evalMap_v^{\imf\imf}(F)$ converges to a unique $y$ in $\F$ by Hausdorffity and completeness of $\F$. Consider the function $f: V\to \F$ that maps each such $v$ to the corresponding $y$.

It is clear that $f$ is linear and thus in $\Lin(V,\F)$. By \ref{initialFinalConvergence}, $F$ converges to this $f$ in the pointwise convergence.
\end{proof}
TODO $\big(\Lin(V,\F),V\big)$ \udef{semi-Montel space}.

\subsection{The bidual space}
TODO!
\begin{definition}
Let $V$ be a convergence vector space. The \udef{bidual space} is the dual of the dual $\abidual{V} = \adual{(\adual{V})}$.
\end{definition}
TODO continuous convergence!!

\begin{definition}
Let $V$ be a vector space over $\mathbb{F}$ and $v\in V$. The \udef{evaluation map} $\evalMap: V\to \abidual{V}: v\mapsto \evalMap_v$ is given by
\[ \evalMap_v: \adual{V} \to \mathbb{F}: l\mapsto l(v). \]
\end{definition}

\begin{lemma}
Let $V$ be a vector space. The evaluation map $\evalMap: V\to \abidual{V}: v\mapsto \evalMap_v$ is linear:
\[ \forall v,w\in V, a\in\mathbb{F}: \quad \evalMap_{av + w} = a\evalMap_v + \evalMap_w. \]
\end{lemma}
\begin{lemma}
Let $V$ be vector space over $\mathbb{F}$. The evaluation map is injective.
\end{lemma}
\begin{proof}
Assume $\evalMap_v = \evalMap_w$ for some $v,w\in V$. Then
\[ 0 = \evalMap_v - \evalMap_w  = \evalMap_{v-w}. \]
So $\forall l\in \adual{V}: \evalMap_{v-w}(l) = l(v-w) = 0$. Now define the sublinear functional by
\[ p(x) = \begin{cases}
\alpha & x = \alpha(v-w) \\
0 & \text{else}.
\end{cases} \]
Then the functional $f$ defined on $\Span\{v-w\}$ by $f(\alpha(v-w)) = \alpha$ is bounded by $p$ and can be extended to a functional on all $V$ by the Hahn-Banach theorem \ref{sublinearHahnBanach} if $v-w\neq 0$. Then $f(v-w) \neq 0$, which contradicts our assumptions. Thus $v=w$.
\end{proof}

\begin{proposition}
The mapping $\evalMap: V\to \abidual{V}: v\mapsto \evalMap_v$ is an isomorphism \textup{if and only if} $V$ is finite-dimensional.
\end{proposition}
\begin{proof}
Assume $V$ finite dimensional. As the evaluation map is injective, it is an isomorphism by \ref{invertibleFiniteDim}.
The other direction is a dimensional argument by proposition \ref{dualBasisDimension}.
\end{proof}




\chapter{Functionals}
\begin{definition}
Let $X$ be a set and $\F$ a field.
\begin{itemize}
\item A \udef{functional} on $X$ is a map $V\to \F$;
\item A \udef{real functional} on $X$ is a map $V\to \R$.
\end{itemize}
If $X$ is a vector space on $\F$, then a \udef{linear functional} is a functional that is a linear function.

If $X$ is a convergence space, then a \udef{continuous functional} is a functional that is a continuous function.
If $\F = \C$, then the set of continuous functionals is denoted $\cont(X)$.
\end{definition}

\begin{proposition}
Let $\sSet{X, \xi}$ be a convergence space and $\F$ a field. Then $\cont_c(X, \F)$ with pointwise operations is a complete convergence vector space.
\end{proposition}
\begin{proof}
TODO Beattie / Butzmann p84.
\end{proof}
\begin{corollary}
Let $\sSet{X, \xi}$ be a convergence space and $\F$ a field. Then $\contLin_c(X, \F)$ with pointwise operations is a complete convergence vector space.
\end{corollary}
\begin{proof}
As $\F$ is Hausdroff, $\contLin_c(X, \F)$ is closed by \ref{linearFunctionsClosedSubset} and thus complete by \ref{closedComplete}.
\end{proof}

\begin{lemma} \label{continuityDominatedFunctional}
Let $V$ be a TVS and $f:V\to \F$ a continuous functional. If $g:V\to \F$ is a functional such that $|g(v)|\leq |f(v)|$ for all $v\in V$, then $g$ is continuous.
\end{lemma}
\begin{proof}
We use \ref{pretopologicalContinuityVicinities} to show continuity. To that end take $K\in \neighbourhood_\F(0)$. Then there exists $\epsilon >0$ such that $\ball(0,\epsilon)\subseteq K$ and so
\[ g^{\preimf}(K) \supseteq g^\preimf[\ball(0,\epsilon)] \supseteq f^\preimf[\ball(0,\epsilon)] \in \neighbourhood_V(0). \]
\end{proof}


\section{Real functionals}
\begin{definition}
Let $V$ be a real or complex vector space. Let $f: V\to \R$ be a real functional. We say
\begin{itemize}
\item $f$ is \udef{subadditive} or satisfies the \udef{triangle inequality} if $\forall x,y\in V: f(x+y) \leq f(x) + f(y)$;
\item $f$ is \udef{quasi-subadditive} if $\exists K>0: \forall x,y\in V: f(x+y) \leq K\big(f(x) + f(y)\big)$;
\item $f$ is \udef{point-separating} if $\forall x\in V: f(x) = 0 \implies x = 0$.
\end{itemize}
We call $f$
\begin{itemize}
\item \udef{sublinear} if it is subadditive and positively homogeneous;
\item a \udef{seminorm} if it is subadditive and absolutely homogeneous;
\item a \udef{quasi-seminorm} if it is quasi-subadditive and absolutely homogeneous;
\item a \udef{norm} is a point-separating seminorm;
\item a \udef{quasi-norm} is a point-separating quasi-seminorm.
\end{itemize}
Let $C\subseteq V$ be a convex subset. Then we call a real functional $g: C \to \R$
\begin{itemize}
\item \udef{convex} if $\forall x,y\in V, \lambda\in[0,1]: g(\lambda x + (1-\lambda)y) \leq \lambda g(x) + (1-\lambda)g(y)$.
\end{itemize}
\end{definition}

TODO general valued fields.

\begin{lemma}
Let $V$ be a real or complex vector space and $f: V\to \R$ be a real functional. Then
\begin{enumerate}
\item absolute homogeneity $\implies$ positive homogeneity;
\item subadditivity $\implies$ quasi-subadditivity;
\item subadditivity+positive homogeneity $\implies$ convexity $\implies$ subadditivity.
\end{enumerate}
\end{lemma}
Thus norms and seminorms are sublinear.

\begin{lemma} \label{seminormPositivity}
Let $f:V\to \R$ be a quasi-seminorm. Then $\im(f)\subseteq \R^+$.
\end{lemma}
Thus (quasi)-seminorms are often considered as function in $V\to \R^+$.
\begin{proof}
For all $v\in V$ we have $0 = f(v-v) \leq K\big(f(v)+f(-v)\big) = 2Kf(v)$, so $f(v) \geq 0$.
\end{proof}

\begin{proposition}[Reverse triangle inequality] \label{reverseTriangleInequality}
Let $V$ be a vector space and $\norm{\cdot}: V\to \R$ a function that satisfies the triangle inequality and has $\norm{-v} = \norm{v}$ for all $v\in V$. Then $\forall v,w\in V$:
\begin{enumerate}
\item $|\norm{v}-\norm{w}|\leq \norm{v-w}$;
\item $|\norm{v}-\norm{w}|\leq \norm{v+w}$.
\end{enumerate}
In particular this holds if $\norm{\cdot}$ is a norm or seminorm.
\end{proposition}
\begin{proof}
We calculate $\norm{v} = \norm{v-w+w} \leq \norm{v-w} + \norm{w}$, so $\norm{v}-\norm{w}\leq \norm{v-w}$. By swapping $v\leftrightarrow w$ we also get $-\norm{v}+\norm{w}\leq \norm{w-v} = \norm{v-w}$ and thus the first inequality is established.

For the second inequality, set $w\to -w$ and use $\norm{-w} = \norm{w}$.
\end{proof}

\subsection{Extended real functionals}
\begin{definition}
Let $V$ be a real or complex vector space. An \udef{extended real functional} is a function $V \to \overline{\R}$.
\end{definition}

\begin{lemma} \label{realPartExtendedRealFunctional}
Let $V$ be a real or complex vector space and $f: V\to \overline{\R}$ an extended real functional. If $f$ is a quasi-seminorm and there exists $v\in V$ such that $f(v)\in\R$, then $f^{\preimf}(\R)$ is a subspace of $V$. 
\end{lemma}
\begin{proof}
We verify the subspace criteria (\ref{subspaceCriterion}). 

By assumption, $f^{\preimf}(\R)$ is not empty.

Take $v,w\in f^{\preimf}(\R)$. As $\im(f) \subseteq \overline{\R}^+$ by \ref{seminormPositivity}, we have $0\leq f(v+w)$. Also $f(v+w)\leq K\big(f(v)+f(w)\big)\in \R$. Thus $f(v+w)\in\R$.

Take $\lambda\in\F$. Then $f(\lambda v) = |\lambda|f(v)\in \R$.
\end{proof}

\subsection{Epigraphs}
\begin{definition}
Let $V$ be a vector space and $f: V\to \R$ a real functional on $V$. Then \udef{epigraph} of $f$ is defined as
\[ \epigraph(f) \defeq \setbuilder{(v,r)\in V\times \R}{f(v)\leq r}. \]
\end{definition}

\begin{lemma} \label{epigraphLemma}
Let $V$ be a vector space and $f: V\to \R$ a real functional on $V$. Then for all $v\in V$:
\[ f(v) = \inf\setbuilder{r}{(v,r)\in \epigraph(f)}. \]
\end{lemma}

\begin{proposition} \label{epigraphProperties}
Let $V$ be a real vector space and $f: V\to \R$ a functional. Then
\begin{enumerate}
\item $f$ is convex \textup{if and only if} $\epigraph(f)$ is a convex subset of $V\oplus \R$;
\item $f$ is positively homogeneous \textup{if and only if} $\epigraph(f)$ is a cone in $V\oplus \R$.
\end{enumerate}
\end{proposition}
\begin{proof}
(1) First assume $f$ convex and pick $(v, s), (w,t)\in \epigraph(f)$ and $\lambda\in [0,1]$. Then we need to show that $(\lambda v + (1-\lambda)w, \lambda s + (1-\lambda)t) \in \epigraph(f)$. This is equivalent to saying $f(\lambda v + (1-\lambda)w) \leq \lambda s + (1-\lambda)t$. Indeed we have $f(\lambda v + (1-\lambda)w) \leq \lambda f(v) + (1-\lambda)f(w) \leq \lambda s + (1-\lambda)t$ by the convexity of $f$.

Conversely, assume $\epigraph(f)$ convex. Then $(v, f(v)), (w,f(w))\in \epigraph(f)$, $(\lambda v + (1-\lambda)w, \lambda f(v) + (1-\lambda)f(w)) \in \epigraph(f)$ for all $\lambda\in [0,1]$. This implies $f(\lambda v + (1-\lambda)w) \leq \lambda f(v) + (1-\lambda)f(w)$.

(2) First assume $f$ is positively homogeneous, take $(v,s)\in \epigraph(f)$ and $r>0$. Then we need to show that $r(v,s) = (rv,rs)\in \epigraph(f)$. This follows because of the implications $f(v)\leq s \implies rf(v) \leq rs \implies f(rv) \leq rs$.

Conversely, assume that $\epigraph(f)$ is a cone. Then $\lambda\cdot \epigraph(f) = \epigraph(f)$ for all $\lambda>0$ by \ref{coneEqualityLemma}. We then calculate using \ref{epigraphLemma}:
\begin{align*}
f(\lambda v) &= \inf\setbuilder{r}{(\lambda v,r)\in \epigraph(f)} \\
&= \inf\setbuilder{r}{(\lambda v,r)\in \lambda\cdot\epigraph(f)} \\
&= \inf\setbuilder{r}{\lambda(v,\lambda^{-1}r)\in \lambda\cdot\epigraph(f)} \\
&= \inf\setbuilder{r}{(v,\lambda^{-1}r)\in \epigraph(f)} \\
&= \inf\setbuilder{\lambda r}{(v,r)\in \epigraph(f)} = \lambda f(v).
\end{align*} 
\end{proof}
\begin{corollary}
A functional on a real vector space is sublinear \textup{if and only if} its epigraph is a convex cone.
\end{corollary}

\subsection{Types of real functionals}
\subsubsection{Convex functionals}
TODO can a convex functional on a convex set be extended to a vector space??

\begin{lemma} \label{preimageConvexSetConvexFunctionalIsConvex}
Let $V$ be a vector space, $C\subseteq V$ a convex subset, $f:C\to \R$ a convex functional and $\epsilon >0$. Then $f^{\preimf}(\interval{0,\epsilon})$ is a convex set.
\end{lemma}
\begin{proof}
Take $x,y\in f^{\preimf}(\interval{0,\epsilon})$ and $0\leq r\leq 1$. Then
\[ f\big(rx + (1-r)y\big) \leq rf(x) + (1-r)f(y) \leq r\epsilon + (1-r)\epsilon = \epsilon, \]
as $f(x), f(y)\leq \epsilon$. So $rx + (1-r)y \in f^{\preimf}(\interval{0,\epsilon})$ and $f^{\preimf}(\interval{0,\epsilon})$ is convex.
\end{proof}

\begin{lemma} \label{convexContinuityLemma}
Let $V$ be a vector space over a field $\F$, $C\subseteq V$ a convex subset and $f:C\to \R$ a convex functional. Let $v,w\in V$ be such that $v-w, v, v+w\in C$ and $\delta\in \interval{0,1}$. Then
\[ |f(v+\delta w) - f(v)| \leq \delta \Big(\max\big\{ f(v+w), f(v-w) \big\} - f(v)\Big). \]
\end{lemma}
\begin{proof}
From $v+\delta w = v - \delta v + \delta v + \delta w = (1-\delta)v+\delta(v+w)$, we get $f(v+\delta w)\leq (1-\delta)f(v) + \delta f(v+w)$ and thus
\[ f(v+\delta w) - f(v)\leq \delta\big(f(v+w)- f(v)\big). \]
Replacing $w$ by $-w$ similarly gives $f(v-\delta w)-f(v)\leq \delta\big(f(v-w)- f(v)\big)$. Now $v = \frac{1}{2}(v+\delta w) + \frac{1}{2}(v-\delta w)$, so $f(v)\leq \frac{1}{2}f(v+\delta w) + \frac{1}{2}f(v-\delta w)$, which we can rewrite as
\[ -f(v+\delta w) \leq f(v-\delta w) - 2f(v). \]
Then we calculate
\begin{align*}
|f(v+\delta w) - f(v)| &= \max\big\{f(v+\delta w) - f(v), f(v) - f(v+\delta w)\big\} \\
&\leq \max\big\{\delta\big(f(v) + f(v+w)\big), f(v) + f(v-\delta w) - 2f(v)\big\} \\
&= \max\big\{\delta\big(f(v) + f(v+w)\big), f(v-\delta w) - f(v)\big\} \\
&\leq \max\big\{\delta\big(f(v) + f(v+w)\big), \delta\big(f(v-w)- f(v)\big)\big\} \\
&= \delta \Big(\max\big\{ f(v+w), f(v-w) \big\} - f(v)\Big).
\end{align*}
\end{proof}

\begin{proposition} \label{convexContinuity}
Let $\sSet{V,\xi}$ be an real convergence vector space, $C\subseteq V$ an open convex set, $f: C\to \R$ a convex functional and $v\in C$. Then each of the following points implies the next:
\begin{enumerate}
\item $f$ is continuous;
\item $f$ is continuous at $v$;
\item $f$ is bounded on some vicinity $U\in\vicinity_\xi(v)$ for some point $v\in C$;
\item for all $w\in C$ there exists a vicinity $U'\in\vicinity_\xi(w)$ such that $f^\imf(U')$ has an upper bound.
\end{enumerate}
If $\sSet{V,\xi}$ is equable, then they are all equivalent.
\end{proposition}
\begin{proof}
The implication $(1) \Rightarrow (2)$ is immediate. 

$(2) \Rightarrow (3)$ 
We have $\vicinity_\R\big(f(v)\big) \subseteq \upset f^{\imf\imf}\big(\vicinity_{\xi|_C}(v)\big)$ by \ref{continuityVicinityFilter}, so there exists $U\in \vicinity_{\xi|_C}(v)$ such that $f^\imf(U) \subseteq \interval{f(v)-1, f(v)+1}$. By \ref{subspaceVicinityFilter}, we may take $U\in \vicinity_\xi(v)$.

$(3) \Rightarrow (4)$ Pick arbitrary $w\in C$. The function $g: \lambda \mapsto w + \lambda(w-v)$ is continuous. By \ref{openClosedCriteria}, we can find an $A\in \vicinity_\xi(w)$ such that $A\subseteq C$. Then $g^\preimf(A)$ is a vicinity of $0$ by \ref{continuityVicinityFilter}. In particular, there exists $\epsilon \in g^\preimf(A)$ for some $\epsilon >0$, so $u = g(\epsilon) = w + \epsilon(w-v) \in A\subseteq C$. Rearranging gives $w = \frac{1}{1+\epsilon}u + \frac{\epsilon}{1+\epsilon}v$.

We may take $U\in\vicinity_\xi(v)$ such that $f^\imf(U)$ is bounded. Let $M$ be an upper bound of $f^\imf(U)$. We claim that $f$ is upper bounded on
\[ U' \defeq w + \frac{\epsilon}{1+\epsilon}(U-v) = \frac{1}{1+\epsilon}u + \frac{\epsilon}{1+\epsilon}U. \]
Indeed, for all $u'\in U$ we have
\[ f\Big(\frac{1}{1+\epsilon}u + \frac{\epsilon}{1+\epsilon}u'\Big) \leq \frac{1}{1+\epsilon}f(u) + \frac{\epsilon}{1+\epsilon}f(u') \leq \frac{1}{1+\epsilon}f(u) + \frac{\epsilon}{1+\epsilon}M. \]

$(4) \Rightarrow (1)$ Take arbitrary $w$. By \ref{pretopologicalContinuityVicinities}, it is enough to show $\vicinity_\R\big(f(w)\big) \subseteq \upset f^{\imf\imf}\big(\vicinity_\xi(w)\big)$.

Any vicinity of $f(w)$ contains a vicinity of the form $\interval{f(w)-\epsilon, f(w)+\epsilon}$ for some $\epsilon >0$. Now, by assumption and \ref{equableConvergenceBalancedBase}, we can find a vicinity $w+A$ of $w$ such that $A$ is balanced and $f^\imf(w+A)$ is bounded above, say by $f(w)+M$ for some $M \geq 0$.

Now take $0 < \delta \leq 1$ such that $\delta M \leq \epsilon$. Then we claim that $f^\imf(w + \delta A) \subseteq \interval{f(w)-\epsilon, f(w)+\epsilon}$. Indeed, take $a\in A$. Then $-a$ is also an element of $A$, so $f(w+a), f(w-a)\leq f(w)+M$. By \ref{convexContinuityLemma}, we have
\begin{align*}
|f(w+\delta A) - f(w)| &\leq \delta \Big(\max\big\{ f(w+a), f(w-a) \big\} - f(w)\Big) \\
&\leq \delta(f(w) + M - f(w)) = \delta M \leq \epsilon.
\end{align*}
\end{proof}

TODO picture Aliprantis / Border p. 189

TODO: also works for algebraic convergence?

\begin{example}
A convex functional does not need to be continuous at the boundary points of its domain of definition. For example, the function
\[ f: \interval{0,1}\to \R: x\mapsto \Iverson{x=0} \]
is convex, but not continuous at $0$.
\end{example}

\begin{proposition}
Let $p: V\to\R$ be convex functional. Then
\[ P: V\to\R: x\mapsto \inf_{t>0} t^{-1}p(tx) \]
is sublinear and $P(x)\leq p(x)$.

Also, if $f:V\to \R$ is a linear functional, then $f\leq p \iff f\leq P$.
\end{proposition}
\begin{proof}
For sublinearity: let $x,y\in V$, then for all $s,t>0$
\[ P(x+y) \leq \frac{s+t}{st}p\left(\frac{st}{s+t}(x+y)\right) = \frac{s+t}{st}p\left(\frac{s}{s+t}(tx)+\frac{t}{s+t}(sy)\right) \leq t^{-1}p(tx) + s^{-1}p(sy). \]
This implies that $P(x+y)\leq P(x)+P(y)$.

For positive homogeneity: let $x\in V,\lambda\geq 0$
\[ P(\lambda x) = \inf_{t>0} t^{-1}p(t\lambda x) = \inf_{t\lambda>0} \lambda (t\lambda)^{-1}p(t\lambda x) = \inf_{t>0} \lambda (t)^{-1}p(tx) = \lambda P(x). \]

Finally we prove that $f\leq p \implies f\leq P$ for linear functionals $f$. For all $t>0$ we have $f(tx) \leq p(tx)$, which implies $f(x) = t^{-1}f(tx) \leq t^{-1}p(tx) \leq P(x)$. So $f\leq P$.
\end{proof}

\subsubsection{Sublinear functionals}

\begin{lemma}
Let $V$ be a \emph{real} vector space and $f: V \to \R$ a sublinear functional. Then $f': V\to \R: x\mapsto \max\{f(x), f(-x)\}$ is a seminorm.
\end{lemma}
We call the seminorm $f'$ the \udef{associated seminorm} of the sublinear functional.
\begin{proof}
Take arbitrary $x,y\in V$ and $a\in \R$

For subadditivity, we have
\begin{align*}
f'(x+y) &= \max\{f(x+y), f(-x-y)\} \\
&\leq \max\{f(x)+f(y), f(-x)+ f(-y)\} \\
&\leq \max\{f(x)+f(y), f(-x)+ f(-y), f(x) + f(-y), f(-x) + f(y)\} \\
&= \max\{f(x), f(-x)\} + \max\{f(y), f(-y)\} \\
&= f'(x) + f'(y).
\end{align*}

For absolute homogeneity, we have,
\begin{align*}
f'(ax) &= \begin{cases}
f'(|a|x) & (a \geq 0) \\ f'(-|a|x) & (a < 0)
\end{cases} \\
&= \begin{cases}
\max\{f(|a|x), f(-|a|x)\} & (a \geq 0) \\ \max\{f(-|a|x), f(|a|x)\} & (a < 0)
\end{cases} \\
&= \max\{|a|f(x), |a|f(-x)\} = |a|f'(x).
\end{align*}
\end{proof}

\begin{lemma} \label{superSubtractiveContinuityEverywhere}
Let $\sSet{V, \xi}$ be a convergence vector space and $f: V\to \R$ a function such $f(0) = 0$ and $|f(v) - f(w)| \leq \max\{|f(v-w)|, |f(w-v)|\}$ for all $v,w\in V$.

Then continuity of $f$ at $0$ implies continuity everywhere.
\end{lemma}
\begin{proof}
Take arbitrary $F\overset{\xi}{\longrightarrow} u$. Then $F-u \to 0$, so $|f^{\imf\imf}(F-u)|\to 0$ by continuity at $0$. Similarly $|f^{\imf\imf}(u-F)|\to 0$. Then, by assymption, $|f^{\imf\imf}(F) - f(u)| \leq \max\{|f^{\imf\imf}(F-u)|, |f^{\imf\imf}(u-F)|\} \to 0$, so $|f^{\imf\imf}(F) - f(u)| \to 0$, by TODO ref squeeze theorem. We conclude that $f^{\imf\imf}(F)\to f(u)$.
\end{proof}

\begin{proposition} \label{sublinearContinuity}
Let $\sSet{V,\xi}$ be a convergence vector space and $f: V\to \R$ a sublinear functional. Then the following are equivalent:
\begin{enumerate}
\item $f$ is continuous;
\item $f$ is continuous at $0$;
\item $f$ is bounded on some $U\in\vicinity_\xi(0)$.
\end{enumerate}
\end{proposition}
\begin{proof}
The implication $(1) \Rightarrow (2)$ is immediate. 

The implication $(2) \Rightarrow (1)$ follows from \ref{superSubtractiveContinuityEverywhere}: 
for arbitrary $v,w\in V$, we have $f(v) = f(v-w+w) \leq f(v-w) + f(w)$, so $f(v) - f(w) \leq f(v-w)$. Similarly $f(w) - f(v) \leq f(w-v)$, so
\begin{align*}
|f(v) - f(w)| &= \max\{f(v) - f(w), f(w) - f(v)\} \\
&\leq \max\{f(v-w), f(w-v)\} \\
&\leq \max\{|f(v-w)|, |f(w-v)|\}.
\end{align*}

The equivalence $(2) \Leftrightarrow (3)$ is given by \ref{continuityToNormedSpace}.
\end{proof}
\begin{corollary} \label{continuityAbsFunctional}
Let $\sSet{V,\xi}$ be a convergence vector space and $f: V\to \R$ a \emph{linear} functional. Then $|f|$ is continuous \textup{if and only if} $f$ is continuous.
\end{corollary}
\begin{proof}
The functional $|f|$ is sublinear and bounded on the same sets as $f$. We can then compare the proposition to \ref{continuityToNormedSpace}.
\end{proof}

\subsubsection{Seminorms}
\begin{lemma} \label{kernelSeminormVectorSpace}
The kernel of a seminorm is a vector space.
\end{lemma}
Note this does not follow from \ref{kernelSubspace} because seminorms are not linear.
\begin{proof}
Let $p:V\to \R$ be a seminorm. We verify the subspace criterion \ref{subspaceCriterion}. First $0\in\ker(p)$ because $p(0) = p(0\cdot 0) = |0|p(0) = 0$.

Now take $v,w\in \ker(p)$ and $\lambda\in \F$. Then $0\leq p(v+\lambda w) \leq p(v)+|\lambda|p(w) = 0$, so $v+\lambda w\in\ker(p)$.
\end{proof}

\begin{proposition} \label{uniformContinuitySeminorms}
Let $\sSet{V,\xi}$ be a convergence vector space and $p: V\to \R^+$ a seminorm. Then the following are equivalent:
\begin{enumerate}
\item $p$ is continuous;
\item $p$ is continuous at $0$;
\item $p$ is uniformly continuous.
\end{enumerate}
\end{proposition}
Cfr. \ref{uniformContinuityGroupHomomorphism}.
\begin{proof}
$(1) \Rightarrow (2)$ Immediate.

$(2) \Rightarrow (3)$ We have that $p\circ \Delta$ is a pseudometric. We first show that $p\circ \Delta$ is a continuous pseudometric by taking $H\in \uniformity_V$. By definition of the uniformity, we have $\Delta^{\imf\imf}(H)\overset{\xi}{\longrightarrow} 0$, so $(p\circ \Delta)^{\imf\imf}(H) \overset{\R}{\longrightarrow} 0$ by continuity of $p$ at $0$. Thus $H\in \uniformity_{p\circ \Delta}$ and $p\circ \Delta$ is continuous.

We conclude that $p = (p\circ \Delta)(0,-)$ is uniformly continuous by \ref{partialApplicationMetricUniformlyContinuous}.

$(3) \Rightarrow (1)$ This follows from \ref{preservationUniformStructure}.
\end{proof}

\begin{proposition} \label{seminormFactorisation}
Let $V$ be a vector space and $p: V\to \R$ a function. Then $p$ is a seminorm \textup{if and only if} there exists a normed space $\sSet{W,\norm{\cdot}}$ and linear function $T: V\to W$ such that $p = \norm{\cdot}\circ T$.
\end{proposition}
\begin{proof}
If $p$ is of the form $\norm{\cdot}\circ T$, then it is clearly a seminorm.

Now assume $p$ is a seminorm, then define $W \defeq V/\ker(p)$, which works because $\ker(p)$ is a vector space by \ref{kernelSeminormVectorSpace}. Then define
\[ \norm{[x]_{\ker(p)}} \defeq p(x). \]
We show that this is well-defined: if $[x]_{\ker(p)} = [y]_{\ker(p)}$, then $x-y\in \ker(p)$, so 
\[ |p(x) - p(y)| \leq p(x-y) = 0, \]
by the reverse triangle inequality \ref{reverseTriangleInequality}. It is easy to verify that $\norm{\cdot}$ is a norm.

Finally set $T \defeq [\cdot]_{\ker(p)}$, which is linear. We have $p = \norm{\cdot}\circ [\cdot]_{\ker(p)} = \norm{\cdot}\circ T$.
\end{proof}

\begin{example}
The kernel of a seminorm is a vector space, but, unlike linear functionals, it is not generally a hyperplane. Equivalently, the space $W$ in \ref{seminormFactorisation} is not generally $1$ dimensional.

For example take any norm on a space that is not $1$ dimensional.
\end{example}


\subsection{Gauges}
\begin{definition}
Let $V$ be a vector space and $A\subseteq V$ an absorbent subset. The function
\[ p_A: V\to \overline{\R^+}: v\mapsto \inf\setbuilder{\lambda\in \R^{\geq 0}}{v\in \lambda A} \]
is called the \udef{gauge} or \udef{Minkowski functional} of $A$.
\end{definition}

\begin{lemma}
Let $V$ be a vector space and $A,B\subseteq V$. Then
\begin{enumerate}
\item if $A$ absorbs $B$, then $p_A^\imf(B)$ is bounded;
\item if $A$ is balanced and $p_A^\imf(B)$ is bounded, then $A$ absorbs $B$.
\end{enumerate}
\end{lemma}
\begin{proof}
(1) Assume $A$ absorbs $B$. Then there exists $r >0$ such that $B\subseteq cA$ for all $|c|\geq r$. In particular $v\in rA$, so $p_A(v) \leq r$, for all $v\in B$. Thus $p_A^\imf(B)$ is bounded by $r$.

(2) Assume $p_A^\imf(B)$ is bounded. Then we can find an upper bound $s>0$. Take $|c|\geq s$ arbitrarily. We need to show that $B\subseteq cA$.

For all $v\in B$, we can find some $\lambda \leq s$ such that $v\in\lambda A = c\left(\frac{\lambda}{c}\right)A \subseteq cA$. The last inclusion follows because $A$ is balanced and $\left|\frac{\lambda}{c}\right|\leq 1$ (as $\lambda \leq s \leq |c|$).
\end{proof}
\begin{corollary} \label{gaugeWellDefined}
Let $V$ be a vector space and $A\subseteq V$. Then
\begin{enumerate}
\item if $A$ is absorbent, then $p_A(v)$ is finite for all $v\in V$;
\item if $A$ is balanced and $p_A(v)$ is finite for all $v\in V$, then $A$ is absorbent.
\end{enumerate}
\end{corollary}

\begin{lemma} \label{gaugeScaling}
Let $V$ be a vector space and $A\subseteq V$ an absorbent subset. For all $v\in V$ and $t\geq 0$:
\[ p_A(tv) = p_{t^{-1}A}(v) = t p_A(v). \]
Thus $p_A$ is positively homogeneous.
\end{lemma}
\begin{proof}
We calculate
\begin{align*}
p_A(tv) &= \inf\setbuilder{\lambda\in \R^{\geq 0}}{tv\in \lambda A} \\
&= \inf\setbuilder{\lambda\in \R^{\geq 0}}{v\in t^{-1}\lambda A} = p_{t^{-1}A}(v) \\
&= \inf\setbuilder{t\lambda\in \R^{\geq 0}}{v\in \lambda A} \\
&= t\inf\setbuilder{\lambda\in \R^{\geq 0}}{v\in \lambda A} = tf(v).
\end{align*}
\end{proof}

\begin{lemma} \label{semibalancedClosureGauge}
Let $V$ be a vector space and $A\subseteq V$ an absorbent subset. Then $p_A = p_{\semibalanced(A)}$.
\end{lemma}
\begin{proof}
We calculate
\begin{align*}
\setbuilder{\lambda\in\R^{\geq 0}}{v\in \lambda \semibalanced(A)} &= \setbuilder{\lambda\in\R^{\geq 0}}{v\in \lambda\cdot\interval{0,1}\cdot A} \\
&= \setbuilder{\lambda\in\R^{\geq 0}}{\exists k\in \interval{0,1}:\; v\in \lambda k\cdot A} \\
&= \setbuilder{\lambda\in\R^{\geq 0}}{\exists \lambda'\leq \lambda:\; v\in \lambda' \cdot A} \\
&= \upset \setbuilder{\lambda'\in\R^{\geq 0}}{v\in \lambda' \cdot A}.
\end{align*}
Thus the infima of both $\setbuilder{\lambda\in\R^{\geq 0}}{v\in \lambda \semibalanced(A)}$ and $\setbuilder{\lambda'\in\R^{\geq 0}}{v\in \lambda' \cdot A}$ are the same.
\end{proof}

\begin{lemma} \label{gaugeLemma}
Let $V$ be a vector space, $A\subseteq V$ an absorbent subset and $\lambda\in \R^{> 0}$. Then
\begin{enumerate}
\item if $A$ is semibalanced, then $p_A(v) < \lambda \implies \lambda^{-1}v\in A$;
\item $\lambda^{-1}v\in A \implies p_A(v) \leq \lambda$;
\item $p_A^{\preimf}[\ball(0,1)] \;\subseteq\; \semibalanced(A) \;\subseteq\; p_A^{\preimf}[\cball(0,1)]$;
\item $\frac{v}{p_A(v)+\epsilon} \in \semibalanced(A)$ for all $v\in V$ and $\epsilon >0$.
\end{enumerate}
\end{lemma}
\begin{proof}
(1) If $p_A(v) < \lambda$, then there exists $\lambda' \leq \lambda$ such that $v\in \lambda' A$. As we can write $\lambda' = k\lambda$ for some $k\in\interval{0,1}$, we have
\[ v\in \lambda k A \subseteq \lambda \cdot \interval{0,1}\cdot A = \lambda A, \]
where we have used that $A = \semibalanced(A) = \interval{0,1}\cdot A$. Thus $\lambda^{-1}v \in A$.

(2) Assume $\lambda^{-1}v \in A$. Then $v\in \lambda A$, so we immediately have $p_A \leq \lambda$.

(3) This follows from (1), the fact that $p_A(v) < p_A(v) + \epsilon$ and the fact that $p_A = p_{\semibalanced(A)}$, \ref{semibalancedClosureGauge}.
\end{proof}

\begin{lemma} \label{gaugeClassificationLemma}
Let $V$ be a vector space, $f: V\to \R^{\geq 0}$ a positively homogeneous function and $A \subseteq V$ a semibalanced subset.
Then the following are equivalent:
\begin{enumerate}
\item $f = p_A$;
\item $f^{\preimf}[\ball(0,1)] \subseteq A \subseteq f^{\preimf}[\cball(0,1)]$.
\end{enumerate}
\end{lemma}
\begin{proof}
$(1) \Rightarrow (2)$ We calculate, using \ref{gaugeLemma},
\begin{align*}
x\in p_{A}^\preimf[\ball(0,1)] \iff& p_{A}(x) < 1 \\
\implies& x\in A \\
\implies& p_{A}(x) \leq 1 \\
\iff& x\in p_{A}^\preimf[\cball(0,1)].
\end{align*}
Reformulating in terms of sets gives the result.

$(2) \Rightarrow (1)$ We calculate, for arbitrary $v\in V$,
\[ \begin{aligned}
p_A(v) &= \inf\setbuilder{\lambda\in \R^{\geq 0}}{v\in \lambda A} \\
&\leq \inf\setbuilder{\lambda\in \R^{\geq 0}}{v\in \lambda f^{\preimf}[\ball(0,1)]} \\
&= \inf\setbuilder{\lambda\in \R^{\geq 0}}{v\in f^{\preimf}[\ball(0,\lambda)]} \\
&= \inf\setbuilder{\lambda\in \R^{\geq 0}}{f(v) < \lambda} = f(v)
\end{aligned} \quad\text{and}\quad \begin{aligned}
p_A(v) &= \inf\setbuilder{\lambda\in \R^{\geq 0}}{v\in \lambda A} \\
&\geq \inf\setbuilder{\lambda\in \R^{\geq 0}}{v\in \lambda f^{\preimf}[\cball(0,1)]} \\
&= \inf\setbuilder{\lambda\in \R^{\geq 0}}{v\in f^{\preimf}[\cball(0,\lambda)]} \\
&= \inf\setbuilder{\lambda\in \R^{\geq 0}}{f(v) \leq \lambda} = f(v).
\end{aligned} \]
We conclude that $f(v) = p_A(v)$.
\end{proof}

\begin{proposition} \label{gaugeClassification}
Let $V$ be a vector space and $f: V\to \R^{\geq 0}$ a function.
Then the following are equivalent:
\begin{enumerate}
\item $f$ is positively homogenous;
\item $f = p_A$ for some semibalanced, absorbent subset $A$.
\end{enumerate}
\end{proposition}
\begin{proof}
Assume $f$ positively homogeneous. Then we can set $A = f^{\preimf}[\ball(0,1)]$ in \ref{gaugeClassificationLemma} because $f^{\preimf}[\ball(0,1)]$ is semibalanced. Indeed
\[ \interval{0,1}\cdot f^{\preimf}[\ball(0,1)] = f^{\preimf}[\interval{0,1}\cdot\ball(0,1)] = f^{\preimf}[\ball(0,1)]. \]

The converse is given by \ref{gaugeScaling}.
\end{proof}

\begin{lemma} \label{gaugeZeroLemma}
Let $V$ be a vector space, $A\subseteq V$ an absorbent subset and $a\in A$. If there exists a subspace $U\subseteq A$ such that $a\in U$, then $p_A(a) = 0$.
\end{lemma}
\begin{proof}
For all $\epsilon > 0$, $\epsilon^{-1}a\in A$, so $a\in \epsilon A$.
\end{proof}

\begin{proposition} \label{gaugeProperties}
Let $V$ be a vector space and $A\subseteq V$ an absorbent subset. Then
\begin{enumerate}
\item $p_A$ is absolutely homogenous if $A$ is balanced;
\item $p_A$ is sublinear if $A$ is convex;
\item $p_A$ is point-separating if $A$ is balanced and contains only the trivial subspace.
\end{enumerate}
\end{proposition}
\begin{proof}
(1) By \ref{balancedLemma} we have $\mu A = |\mu| A$ and thus
\begin{align*}
p_A(\mu\cdot v) &= \inf\setbuilder{\lambda\in \R^{\geq 0}}{\mu\cdot v\in \lambda A} = \inf\setbuilder{\lambda\in \R^{\geq 0}}{v\in \frac{\lambda}{\mu} A} \\
&= \inf\setbuilder{\lambda\in \R^{\geq 0}}{v\in \frac{\lambda}{|\mu|} A} = \inf\setbuilder{|\mu|\lambda\in \R^{\geq 0}}{v\in \lambda A} = |\mu|\cdot p_A(v).
\end{align*}

(2) We just need to show subadditivity. Positive homogeneity is automatic by \ref{gaugeScaling}. Take $v,w\in V$. Now take arbitrary $\epsilon > 0$, so $(p_A(v)+\epsilon)^{-1}v \in A$ and $(p_A(w)+\epsilon)^{-1}w \in A$ by \ref{gaugeLemma} (as $A$ is semibalanced by \ref{convexAbsorbentImpliesSemibalanced}). By convexity of $A$, we have
\[ \frac{v+w}{p_A(v)+p_A(w)+2\epsilon} = \frac{p_A(v)+\epsilon}{p_A(v)+p_A(w)+2\epsilon}(p_A(v)+\epsilon)^{-1}v + \frac{p_A(w)+\epsilon}{p_A(v)+p_A(w)+2\epsilon}(p_A(w)+\epsilon)^{-1}w \in A. \]
By \ref{gaugeLemma} this means $p_A(v)+p_A(w)+2\epsilon \geq p_A(v+w)$ and because $\epsilon$ was arbitrary, we conclude that $p_A(v+w) \leq p_A(v)+p_A(w)$.

(3) Assume $A$ contains only the trivial subspace. Then for all $v\in V$ there exists some $\lambda\in \F$ such that $\lambda\cdot v\notin A$. Now for all $|c|\geq |\lambda|$, $c\cdot v\notin A$ because $A$ is balanced. Then $p_A(2\lambda\cdot v) \neq 0$ and because $p_A$ is absolutely homogeneous we have $p_A(v) = (2\lambda)^{-1}p_A(2\lambda\cdot v) \neq 0$.
\end{proof}
\begin{corollary}
The gauge of an absolutely convex and absorbent subset is a seminorm. If the subset contains only the trivial subspace, then the gauge is a norm.
\end{corollary}

\begin{proposition} \label{continuityConvexGauge}
Let $\sSet{V, \xi}$ be a convergence vector space and $A\subseteq V$ a convex, absorbent subset. Then $p_A: \sSet{V,\xi} \to \F$ is continuous \textup{if and only if} $A\in\vicinity_\xi(0)$.
\end{proposition}
\begin{proof}
First assume $A\in\vicinity_\xi(0)$. By \ref{gaugeProperties}, we have that $p_A$ is sublinear. By \ref{gaugeClassificationLemma}, we have that $A \subseteq p_A^\preimf[\cball(0,1)]$ and thus that $p_A$ is bounded on $A$. Then $p_A$ is continuous by \ref{sublinearContinuity}.

Now assume $p_A$ continuous. Then $\ball(0,1) \in \neighbourhood_\F(0) = \neighbourhood_\F(p_A(0))$, so $p_A^\preimf[\ball(0,1)] \in \vicinity_\xi(0)$ by \ref{continuityVicinityFilter}. By \ref{gaugeClassificationLemma} (using the fact that $A$ is semibalanced, \ref{convexSemibalanced}), $p_A^\preimf[\ball(0,1)] \subseteq A$, so $A\in\vicinity_\xi(0)$.
\end{proof}
\begin{corollary} \label{gaugeContinuousAlgebraicConvergence}
Let $V$ be a vector space. Then $p_A: \sSet{V,\mathfrak{a}} \to \F$ is continuous for any convex, absorbent subset $A \subseteq V$.
\end{corollary}
\begin{proof}
By \ref{coreProperties}, we have $0\in\inh_\mathfrak{a}(A)$. Then $A\in \vicinity_\mathfrak{a}(0)$ by \ref{principalAdherenceInherence}.
\end{proof}

\begin{lemma} \label{absoluteFunctionalGauge}
Let $\sSet{V, \xi}$ be a convergence vector space and $f: V\to \F$ a linear functional. Then $f$ is continuous \textup{if and only if} $|f| = p_K$ for some $K\in\vicinity_\xi(0)$.
\end{lemma}
\begin{proof}
We have that $|f|$ is positively homogeneous. Set $K \defeq |f|^{\preimf}[\ball(0,1)] = |f|^{\preimf}(\interval[o]{0,1})$. Then $|f| = p_{K}$, by \ref{gaugeClassificationLemma}. 

First assume $f$ is continuous. Then $p_K^\preimf\big[\ball(0,1)\big] = K$ is in $\vicinity_\xi(0)$ by \ref{continuityVicinityFilter}.

Now assume $K\in\vicinity_\xi(0)$. As $K \subseteq p_K^\preimf\big[\cball(0,1)\big]$ (\ref{gaugeClassificationLemma}), we have that $|f|$ is bounded on $K$. It is also sublinear and thus continuous by \ref{sublinearContinuity}.
\end{proof}


\begin{proposition} \label{gaugeInherenceAdherence}
Let $V$ be a vector space and $A\subseteq V$ an absorbent semibalanced subset. Then
\begin{enumerate}
\item $\inh_\mathfrak{a}(A) \subseteq p_A^{\preimf}[\ball(0,1)]$;
\item $p_A^{\preimf}[\cball(0,1)] \subseteq \adh_\mathfrak{a}(A)$.
\end{enumerate}
The inclusions are equalities if $A$ is convex.
\end{proposition}
\begin{proof}
(1) Take $x\in \inh_\mathfrak{a}(A)$. Then by \ref{constructionsInAlgebraicConvergence}, there exists $\Gamma\in\neighbourhood_\F(0)$ such that $v+\Gamma\cdot v = (1 + \Gamma)\cdot v \subseteq A$. There exists $\lambda \in 1 + \Gamma$ that is real and strictly less than $1$. Thus $p_A(x) <1$, meaning $x\in p^\preimf_A\big(\ball(0,1)\big)$.

(Convex) For the other inclusion we use that $p_A$ is continuous by \ref{gaugeContinuousAlgebraicConvergence}. By \ref{adherenceInherenceContinuity} and \ref{gaugeClassificationLemma} we have
\[ p_A^{\preimf}[\ball(0,1)] = p_A^{\preimf}\Big[\interior_\F\big(\ball(0,1)\big)] \subseteq \inh_\mathfrak{a}\big(p_A^{\preimf}[\ball(0,1)]\big) \subseteq \inh_\mathfrak{a}(A). \]

(2) Take $x\in p^\preimf_A\big(\cball(0,1)\big)$, which means that $p_A(x) \leq 1$, so $p_A(x) < 1+\epsilon$ for all $\epsilon > 0$. By \ref{gaugeLemma}, $(1+\epsilon)^{-1}x = x - \epsilon(1+\epsilon)^{-1}x \in A$. Because $\frac{\epsilon}{1+\epsilon} < \epsilon$, we have $x - \epsilon(1+\epsilon)^{-1}x \in x + \ball(0,\epsilon)\cdot x$ and so $x + \ball(0,\epsilon)\cdot x \mesh A$.

Now for all $\Gamma\in\neighbourhood_\F(0)$, we have $\ball(0,\epsilon) \subseteq \Gamma$ for some $\epsilon >0$. Thus $x+ \Gamma\cdot x \mesh A$ and $x\in\adh_\mathfrak{a}(A)$ by \ref{constructionsInAlgebraicConvergence}.

(Convex) For the other inclusion we again use that $p_A$ is continuous by \ref{gaugeContinuousAlgebraicConvergence}. From \ref{gaugeClassificationLemma} we have $A \subseteq p^\preimf_A\big(\cball(0,1)\big)$. Thus, by \ref{adherenceInherenceContinuity},
\[ \adh_\mathfrak{a}(A) \subseteq \adh_\mathfrak{a}\Big(p^\preimf_A\big(\cball(0,1)\big)\Big) \subseteq p_A^\preimf\big(\overline{\cball(0,1)}\big) = p^\preimf_A\big(\cball(0,1)\big). \]
\end{proof}

\begin{example}
TODO square with corner missing.
\end{example}



\begin{proposition} \label{gaugeConstructions}
Let $V$ be a vector space and $A,B\subseteq V$ semibalanced subsets. Then
\begin{enumerate}
\item the gauge of $A\cap B$ is $\max\{p_A, p_B\}$;
\item if $A\subseteq B$, then $p_B \leq p_A$;
\item if $p_B \leq p_A$ and $A$ is convex, then $\adh_\mathfrak{a}(A) \subseteq \adh_\mathfrak{a}(B)$.
\end{enumerate}
\end{proposition}
\begin{proof}
(1) Clearly $p_A \leq p_{A\cap B}$:
\begin{align*}
p_A(v) &= \inf\setbuilder{\lambda\in\R^{\geq 0}}{v\in \lambda A} \\
&\leq \setbuilder{\lambda\in\R^{\geq 0}}{v\in \lambda (A\cap B)} \\
&= p_{A\cap B}(v).
\end{align*}
Similarly $p_B \leq p_{A\cap B}$. Thus $\max\{p_A, p_B\} \leq p_{A\cap B}$.

For the other inequality we use \ref{gaugeLemma}: take $v\in V$ and arbitrary $\epsilon > 0$. Then $\max\{p_A(v)+\epsilon, p_B(v)+\epsilon\} > p_A(v)$, so $\max\{p_A(v)+\epsilon, p_B(v)+\epsilon\}^{-1}v\in A$. Similarly $\max\{p_A(v)+\epsilon, p_B(v)+\epsilon\}^{-1}v\in B$. Thus $\max\{p_A(v)+\epsilon, p_B(v)+\epsilon\}^{-1}v \in A\cap B$ and finally $p_{A\cap B}(v) \leq \max\{p_A(v)+\epsilon, p_B(v)+\epsilon\}$. Taking the limit $\epsilon \to 0$, we have $p_{A\cap B} \leq \max\{p_A, p_B\}$.

(2) First assume $A\subseteq B$. Then $\frac{v}{p_A(v)+\epsilon}\in A$ by \ref{gaugeLemma}, for all $\epsilon >0$. Then $\frac{v}{p_A(v)+\epsilon}\in B$, so $p_B(v) \leq p_A(v)+\epsilon$ (again by \ref{gaugeLemma}). Taking the limit $\epsilon\to 0$ yields the result.

(3) Assume $p_B \leq p_A$ and take $v\in \adh_\mathfrak{a}(A)$. Then $p_B(v) \leq p_A(v) \leq 1$ by \ref{gaugeInherenceAdherence}, which also gives $v\in\adh_\mathfrak{a}(B)$.
\end{proof}


\section{Linear functionals}
\begin{lemma} \label{kernelHyperplane}
Let $V$ be a vector space and $U\subseteq V$ a subspace. Then $U$ is a hyperplane \textup{if and only if} it is the kernel of a linear functional.
\end{lemma}

\begin{lemma} \label{functionalBoundedNeighbourhood}
Let $f: V\to \F$ be a linear functional and $x\notin \ker(f)$. Let $A\subseteq V$ be a balanced set. Then $(x+A)\perp \ker(f)$ \textup{if and only if} $A \subseteq f^{\preimf}(\ball(0,|f(x)|))$.
\end{lemma}
\begin{proof}
Suppose $A \subseteq f^{\preimf}(\ball(0,|f(x)|))$. Then for all $a\in A$: $f(x+a) = f(x) + f(a) \neq 0$.

Conversely, suppose $A \not\subseteq f^{\preimf}(\ball(0,|f(x)|))$, i.e.\ there exists $a\in A$ such that $|f(a)| \geq |f(x)|$. Then $v= -\frac{f(x)}{f(a)}a\in A$, because $A$ is balanced and so $f(x+ v) = f(x)-\frac{f(x)}{f(a)}f(a) = 0$ and so $(x+A) \mesh \ker(f)$.
\end{proof}

\begin{proposition} \label{linearFunctionalOpen}
Let $V$ be a convergence vector space and $f:V\to \F$ a non-zero linear functional. Then $f$ is an open map.
\end{proposition}
\begin{proof}
It is enough to show $f$ is open when $V$ is equipped with the algebraic convergence.

Let $A$ be an algebraically open map. We use \ref{openClosedCriteria} to show $f^\imf[A]$ is also open. Because $f$ is non-zero, there exists a $v\in V$ such that $f(v) \neq 0$. Take some $y\in f^\imf[A]$. Then there exists an $x\in A$ such that $f(x) = y$. Because $A$ is open, $x\in \inh_\mathfrak{x}(A)$ and there exists $x+ \Gamma_v\in \neighbourhood_\F(0)$ such that $x+\Gamma_v\cdot v \subseteq A$ by \ref{constructionsInAlgebraicConvergence}.

Now $f^\imf[x+ \Gamma_v\in \neighbourhood_\F(0)] = y+\Gamma_v \cdot f(v) \subseteq f^\imf[A]$ and $y+\Gamma_v \cdot f(v)$ is a vicinity of $y$, so we are done.
\end{proof}

\begin{lemma} \label{complexRangeExtensionRealFunctional}
Let $V$ be a complex vector space and $g: V_\R\to \R$ a linear functional. Then there exists a unique linear functional $f: V\to \C$ such that $g = \Re(f)$.
\end{lemma}
\begin{proof}
We can write $f = g + ih$ for some function $h: V\to \R$. Then for all $x\in V$
\[ g(ix)+ih(ix) = f(ix) = if(x) = ig(x) - h(x). \]
Comparing real parts gives $h(x) = - g(ix)$. So $f$ must be given by $f(x) = g(x) - ig(ix)$. Clearly $f$ is real-linear. We just need to verify that this makes $f$ complex-linear. Indeed, take $\lambda = a +ib \in \C = \R+i\R$ arbitrarily. Then for all $v\in V$
\begin{align*}
f(\lambda v) &= f\big((a+ib)v\big) \\
&= af(v) + bf(iv) \\
&= af(v) + b\big(g(iv) - ig(i^2v)\big) \\
&= af(v) + b\big(g(iv) + ig(v)\big) \\
&= af(v) + ib\big(-ig(iv) + g(v)\big) \\
&= af(v) + ibf(v) = (a+ib)f(v) = \lambda f(v).
\end{align*}
\end{proof}

\begin{lemma} \label{linearDependenceLinearFunctionals}
Let $V$ be a vector space and $f_0,\ldots, f_n, f$ linear functionals in $(V\to \F)$. Then the following are equivalent:
\begin{enumerate}
\item $f$ is a linear combination of $f_0,\ldots, f_n$;
\item there exists a $C>0$ such that $f(v) \leq C \max_{0\leq i\leq n}|f_i(v)|$;
\item $\ker(f) \supseteq \bigcap_{0\leq i \leq n}\ker(f_i)$;
\end{enumerate}
\end{lemma}
\begin{proof}
The implications $(1) \Rightarrow (2) \Rightarrow (3)$ are clear.

Now assume $(3)$ holds. Consider the function
\[ \begin{pmatrix}
f_0 \\ \vdots \\ f_n
\end{pmatrix}: V\to \F^{n+1}: v\mapsto \begin{pmatrix}
f_0(v) \\ \vdots \\ f_n(v)
\end{pmatrix}. \]
Due to the assumption, we can find a linear function $F: \F^{n+1}\to \F$ such that $f = F\circ \begin{pmatrix}
f_0 \\ \vdots \\ f_n
\end{pmatrix}$.

This function $F$ can be represented as a matrix by \ref{ellIsomorphism}. Thus $f$ is a linear combination of $f_0,\ldots, f_n$.
\end{proof}
\begin{corollary}
Let $V$ be a vector space and $f_0,\ldots, f_n$ linearly independent linear functionals in $(V\to \F)$. Then there exist $v_0, \ldots, v_n$ such that $f_i(v_j) = \delta_{i,j}$.
\end{corollary}
\begin{proof}
The proof is by induction. The case $n=1$ is clear: if there was no such $a_1$, then $f_1$ would be zero and thus not linearly independent.

Suppose the statement holds for $n-1$ and take $f_0,\ldots, f_n$ linearly independent linear functionals with corresponding $v_0,\ldots, v_{n-1}$. By point (3) of the proposition we can find $v_n \in \bigcap_{0\leq i \leq n}\ker(f_i)\setminus \ker(f_n)$, which after rescaling can be taken to be such that $f_n(v_n) = 1$. By construction $f_i(v_n) = 0$ for $i<n$.

Now replace $v_i$ with $v_i-f_n(v_i)v_n$ and rescale.
\end{proof}

\subsection{The dual space}
\begin{definition}
Let $\sSet{V,\xi}$ be a convergence vector space over a field $\mathbb{F}$.

The \udef{dual} of $V$ is the vector space of all continuous linear functionals on $V$.

The dual is denoted $\sSet{V,\xi}^{*}$ (or just $V^*$ is the convergence is clear from the context).
\end{definition}

\begin{proposition} \label{continuityLinearFunctionals}
Let $\sSet{V, \xi}$ be a CVS and $f:V\to \F$ a linear functional on $V$. Then the following are equivalent:
\begin{enumerate}
\item $f\in V^{*}$, i.e.\ $f$ is continuous;
\item there exists a vicinity $U\in \vicinity_\xi(0)$ such that $f$ is bounded on $U$;
\item $\ker(f)$ is closed;
\item $\ker(f)$ is not dense.
\end{enumerate}
\end{proposition}
\begin{proof}
$(1) \Leftrightarrow (2)$ By \ref{continuityToNormedSpace}.

$(1) \Rightarrow (3)$ Because $\ker(f) = f^{\preimf}(\{0\})$ and $\{0\}$ is closed in $\F$, $\ker(f)$ is closed by \ref{preimageOpenClosed}.

$(3) \Rightarrow (1)$ If $f = \underline{0}$, then continuity is immediate. Now assume $f \neq \underline{0}$, which means that $f$ is surjective. By the factor theorem \ref{factorTheorem}, we can find an injective linear function $f': V/\ker(f)\to \F$ such that
\[ \begin{tikzcd}
V \arrow[r, "{[\cdot]_{\ker(f)}}"] \arrow[dr, swap, "f"] & V/\ker(f) \arrow[d, dashed, "{f'}"] \\
& \F
\end{tikzcd} \qquad\text{commutes.} \]
Thus $f'$ is bijective linear. We give $V/\ker(f)$ the quotient convergence. As $\ker(f)$ was assumed closed, $V/\ker(f)$ is Hausdorff (by \ref{quotientConvergenceGroupProperties}) and thus $f'$ is a homeomorphism by \ref{finiteDimensionalHausdorffLinearBijectionIsHomeomorphism}. We conclude that $f= f'\circ [\cdot]_{\ker(f)}$ is continuous as a composition of continuous functions.

$(3) \Leftrightarrow (4)$ By \ref{kernelHyperplane} $\ker(f)$ is a hyperplane and by \ref{hyperplaneClosedDense} this hyperplane is either closed or dense, but not both.
\end{proof}

\begin{note}
In the topological case, we have the following argument for the implication $\ker(f)$ closed $\Rightarrow$ $f$ bounded on a neighbourhood of $0$.

Assume $\ker(f)$ closed. If $\ker(f) = V$, then $f$ is constant and thus continuous by \ref{continuityConstructions}. If $\ker(f) \neq V$, we can find some some $x\in \ker(f)^c$, which is open. Thus $\ker(f)^c - x$ is a neighbourhood of the origin, meaning we can take a balanced subset $A$ by \ref{vicinityFilterAtOrigin}. Now $(x+A)\perp \ker(f)$ by construction, so $f$ is bounded on $A$ by \ref{functionalBoundedNeighbourhood}.
\end{note}

\section{Locally convex spaces}
\begin{definition}
A \udef{locally convex} convergence vector space is a convergence vector space that is locally convex.
\end{definition}

\begin{lemma} \label{locallyConvexTVSLocallyConvexOpen}
If $\sSet{V,\xi}$ is a locally convex topological vector space, then $V$ is locally convex open.
\end{lemma}
\begin{proof}
Take $v\in V$ and $A\in \neighbourhood(v)$. Since $\xi$ is locally convex, $A$ contains a convex neighbourhood $B$ of $v$. Then $\interior(B)$ is convex by \ref{inherenceAdherenceConvex} and it is a neighbourhood of $v$ since $\xi$ is locally open.
\end{proof}

\begin{lemma} \label{locallyConvexNeighbourhoodsLemma}
Let $\sSet{V,\xi}$ be a TVS. Then the following are equivalent:
\begin{enumerate}
\item $\xi$ is locally convex;
\item $\neighbourhood_\xi(0)$ is based in the convex sets;
\item $\neighbourhood_\xi(0)$ is based in the absolutely convex sets;
\item $\neighbourhood_\xi(0) = \upset\disked^{\imf}\big(\neighbourhood_\xi(0)\big)$.
\end{enumerate}
\end{lemma}
\begin{proof}
$(1) \Leftrightarrow (2)$ One direction is immediate, for the other it is enough to note that if $U$ is convex, then so is the translated set $x+U$ for all $x\in V$.

$(2) \Leftrightarrow (3)$ One direction is immediate, the other follows because the balanced core of a convex set is convex by \ref{balancedCoreConvexSet}.

$(3) \Leftrightarrow (4)$ Because $\disked$ preserves intersections (as it is a closure operator), we have that $\upset\disked^{\imf}\big(\neighbourhood_\xi(0)\big)$ is a filter.

We automatically have $\neighbourhood_\xi(0) \subseteq \upset\disked^{\imf}\big(\neighbourhood_\xi(0)\big)$. If $\neighbourhood_\xi(0)$ is based in the absolutely convex sets, then $\upset\disked^{\imf}\big(\neighbourhood_\xi(0)\big)$ contains a base of $\upset\disked^{\imf}\big(\neighbourhood_\xi(0)\big)$ and so the other inclusion holds.

If $\neighbourhood_\xi(0) = \upset\disked^{\imf}\big(\neighbourhood_\xi(0)\big)$, then $\neighbourhood_\xi(0)$ is based in the absolutely convex sets, because clearly $\upset\disked^{\imf}\big(\neighbourhood_\xi(0)\big)$ is.
\end{proof}

\begin{proposition} \label{LCTVSconstruction}
Let $V$ be a vector space and $N\in\powerfilters(V)$. Then $N = \neighbourhood_\xi(0)$ for some locally convex topological convergence on $V$ \textup{if and only if}
\begin{enumerate}
\item for all $A\in N$ and $\lambda\in \F$: $\lambda A\in N$;
\item each $A \in N$ is absorbent;
\item $N$ has an absolutely convex base.
\end{enumerate}
\end{proposition}
\begin{proof}
This almost completely follows from \ref{TVSconstruction} and \ref{locallyConvexNeighbourhoodsLemma}. We just need to show that for all $A\in N$, there exists some $B\in N$ such that $B+B\subseteq A$. We may take $B = \frac{1}{2}A'$, where $A'$ is a convex subset of $A$, because for all $v,w\in A'$ we have $\frac{1}{2}v + \frac{1}{2}w \in A'$ by convexity.
\end{proof}

\begin{proposition}
Every locally convex TVS is locally path connected.
\end{proposition}
\begin{proof}
TODO
\end{proof}

\subsection{Seminormed spaces}
\begin{definition}
Let $V$ be a vector space and $S$ a set of seminorms on $V$. 
Then the initial vector space convergence w.r.t.\ $S$ is called the \udef{seminorm convergence} on $V$ and $V$ equipped with the seminorm convergence is called a \udef{seminormed space}.
\end{definition}
Note that the seminorm convergence is not in general an initial convergence.
\begin{example}
Let $\sSet{V, \norm{\cdot}}$ be a normed space. Suppose $u,v\in V$ are unit vectors (i.e.\ $\norm{u} = 1 = \norm{v}$). Then $\norm{\pfilter{u}} \to v$ in the initial convergence.
\end{example}

In fact the seminorm convergence is the initial convergence iff $S = \{\underline{0}\}$.

\begin{lemma} \label{seminormPseudometricLemma}
Let $V$ be a vector space and $p:V\to \R$ a seminorm. Then
\begin{enumerate}
\item $p\circ \Delta$ is a pseudometric;
\item the pseudometric convergence is a vector space convergence;
\item the pseudometric convergence is the initial vector space convergence w.r.t. $\{p\}$;
\item the pseudometric uniformity is equal to the vector space (i.e.\ group) uniformity.
\end{enumerate}
\end{lemma}
\begin{proof}
Every seminorm is a cyclically permutable group seminorm, so most of the results follow from \ref{groupSeminormConvergence}. We just need to show that the resulting convergence makes the scalar multiplication continuous and that any other vector space convergence that makes $p$ continuous is contained in this convergence.

We first show continuity of the scalar multiplication. As the convergence in $\F$ is topological, it is enough to consider the convergence of $\neighbourhood_\F(\lambda)\cdot F$ for some $\lambda \in \F$ and $F\to v\in V$. 

Now we have
\begin{align*}
\upset p^{\imf\imf}\big(\neighbourhood_\F(\lambda)\cdot F - \lambda \pfilter{v}\big) &= \upset p^{\imf\imf}\big((\neighbourhood_\F(\lambda)\cdot F - \neighbourhood_\F(\lambda)\pfilter{v}) + (\neighbourhood_\F(\lambda)\pfilter{v} -\lambda \pfilter{v})\big) \\
&\leq_w \upset p^{\imf\imf}\big(\neighbourhood_\F(\lambda)\cdot F - \neighbourhood_\F(\lambda)\pfilter{v}\big) + \upset p^{\imf\imf}\big(\neighbourhood_\F(\lambda)\pfilter{v} -\lambda \pfilter{v}\big) \\
&= \big|\neighbourhood_\F(\lambda)\big|\cdot p^{\imf\imf}(F-\pfilter{v}) + \big(\big|\neighbourhood_\F(\lambda)\big| - \lambda\big)\pfilter{p}(v) \\
&\to 0 + 0 = 0.
\end{align*}
The weak inequality follows from \ref{pointwisefunctionToFilterInequality}. Then $p^{\imf\imf}(F-\pfilter{v})$ converges to $0$ by \ref{metricConvergence}.
Since $\big|\neighbourhood_\F(\lambda)\big|$, $\big(\big|\neighbourhood_\F(\lambda)\big| - \lambda\big)$ and $\pfilter{p}(v)$ converge in $\R$, we obtain the convergence by continuity of the multiplication.

Thus $\upset p^{\imf\imf}\big(\neighbourhood_\F(\lambda)\cdot F - \lambda \pfilter{v}\big)$ converges by the squeeze theorem (TODO ref).

Finally suppose $F$ converges to $v$ in some other vector space convergence that makes $p$ continuous. Then $\upset p^{\imf\imf}(F-v)$ converges to $0$ in this convergence, but this implies that $F$ converges to $v$ in the pseudometric convergence.
\end{proof}

\begin{proposition} \label{initialSeminormConvergence}
Let $V$ be a vector space, $S$ a set of seminorms on $V$, $F\in\powerfilters{V}$ and $v\in V$.
\begin{enumerate}
\item the seminormed convergence w.r.t.\ $S$ is topological and given by the join of the seminorm convergences determined by each $p\in S$;
\item $F\to v$ in the seminormed convergence w.r.t.\ $S$ \textup{if and only if} $p^{\imf}\big(F-v\big)\overset{\F}{\longrightarrow} 0$ for all $p\in S$;
\item the neighbourhood filter of the origin in the seminormed convergence is
\begin{align*}
\neighbourhood_\xi(0) &= \mathfrak{F}\setbuilder{p^\preimf[\ball(0,\epsilon)]}{p\in S, \epsilon > 0} \\
&= \mathfrak{F}\setbuilder{p^\preimf[\cball(0,\epsilon)]}{p\in S, \epsilon > 0}.
\end{align*}
\end{enumerate}
\end{proposition}
Note that $\neighbourhood_\xi(0)$ is the same as the neighbourhood filter at $0$ in the initial (not necessarily vector space) convergence w.r.t. $S$.
\begin{proof}
The seminormed convergence w.r.t.\ $S$ is clearly smaller than the initial convergenc w.r.t.\ $\{\id_V: V\to \sSet{V,p}\}_{p\in S}$. This is a vector space convergence by \ref{initialVectorSpaceConvergence} and thus exactly the seminorm convergence. 

Point (2) then follows from \ref{seminormPseudometricLemma} and \ref{metricConvergence}.

The convergence is topological by \ref{pretopologicalInitialConvergence}, which also gives the form of the neighbourhood filter of the origin.
\end{proof}

\begin{proposition} \label{dualSeminormedConvergence}
Let $V$ be a vector space and $S$ a set of seminorms on $V$. Let $\xi$ be the initial convergence on $V$ w.r.t. $S$. Then $f\in \sSet{V, \xi}^*$ \textup{if and only if}
\begin{itemize}
\item $f\in \sSet{V,\mathfrak{a}}^*$;
\item there exists a finite subset $A\subseteq S$ and $C>0$ such that $|f(v)| \leq C\max_{g\in A} g(v)$ for all $v\in V$.
\end{itemize}
\end{proposition}
\begin{proof}
We have by \ref{continuityLinearFunctionals}
\[ f\in \sSet{V, \xi}^* \iff \exists D>0: \exists U\in \neighbourhood_\xi(0):\; f^\imf[U] \subseteq \cball(0,D). \]
Now $U\in \neighbourhood_\xi(0)$ iff there exists a finite $A = \{p_n^\preimf[\ball(0,\epsilon_n)]\}_{n=0}^N \subseteq \setbuilder{p^\preimf[\ball(0,\epsilon)]}{p\in S, \epsilon > 0}$ such that $\bigcap A\subseteq U$. So WLOG we may take $U$ of this form.
Now 
\begin{align*}
f^\imf\Big[\bigcap A\Big] \subseteq \cball(0,D) &\iff \forall v\in V: \; \Big(\forall n\leq N: p_n(v)\leq \epsilon_n \Big) \implies |f(v)|\leq D \\
&\iff \forall v\in V: \; \max_{n\leq N}\epsilon_n^{-1}p_n(v)\leq 1 \implies |f(v)|\leq D \\
&\iff \forall v\in V: \; \max_{n\leq N}\epsilon_n^{-1}p_n\left(\frac{\max_{n\leq N}\epsilon_n^{-1}p_n(v)}{\max_{n\leq N}\epsilon_n^{-1}p_n(v)} v\right)\leq 1 \implies |f(v)|\leq D \\
&\iff \forall v\in V: \; \left(\max_{n\leq N}\epsilon_n^{-1}p_n(v)\right)^{-1}\max_{n\leq N}\epsilon_n^{-1}p_n(v)\leq 1 \implies \left(\max_{n\leq N}\epsilon_n^{-1}p_n(v)\right)^{-1}|f(v)|\leq D \\
&\iff \forall v\in V: \; 1\leq 1 \implies |f(v)|\leq D\max_{n\leq N}\epsilon_n^{-1}p_n(v) \\
&\iff \forall v\in V: \; |f(v)|\leq D\max_{n\leq N}\epsilon_n^{-1}p_n(v).
\end{align*}
WLOG we may take all $\epsilon_n = \epsilon = \min_{n\leq N}\epsilon_n$. We may then take $C = D/\epsilon$.
\end{proof}


\begin{proposition} \label{locallyConvexSeminormTopology}
Let $V$ be a vector space. A topological convergence on $V$ is locally convex \textup{if and only if} it is the seminorm convergence w.r.t. some set $S$ of seminorms on $V$.
\end{proposition}
\begin{proof}
Since a shifted convex set is still convex (\ref{translationScalingConvexSet}), localy convexity is equivalent to saying that the neighbourhood filter at the origin is based in the convex sets.

If $V$ has the seminorm convergence w.r.t. $S$, then $V$ is locally convex by \ref{initialSeminormConvergence} because $p^\preimf[\ball(0,\epsilon)]$ is convex for all $p\in S$ by \ref{preimageConvexSetConvexFunctionalIsConvex}.

Now let $V$ be a locally convex TVS, so there exists an absolutely convex base $\mathcal{B}$ of $\neighbourhood(0)$ by \ref{locallyConvexNeighbourhoodsLemma}.

Since $\{\inh(B)\}_{B\in\mathcal{B}}$ is a basis of $\neighbourhood(0)$ and $\inh(B) \subseteq \inh_\mathfrak{a}(B)$ by \ref{principalInherenceAdherenceProperties}, we may replace $\mathcal{B}$ by $\{\inh_\mathfrak{a}(B)\}_{B\in\mathcal{B}}$.
This basis consists of absolutely convex, algebraically open sets by \ref{coreProperties}.

Then $S = \setbuilder{p_B}{B\in \mathcal{B}}$ is a set of continuous seminorms, by \ref{gaugeProperties} and \ref{continuityConvexGauge}.

In order to show that the convergence on $V$ is the seminorm convergence w.r.t. $S$, we verify the form of $\neighbourhood(0)$ given in \ref{initialSeminormConvergence}.

Since $\epsilon B\in \neighbourhood(0)$ for each $B\in\mathcal{B}$ and all $\epsilon > 0$, we have that $\setbuilder{\epsilon B}{\epsilon > 0, B\in \mathcal{B}}$ is a basis of $\neighbourhood(0)$.
Then
\begin{align*}
\neighbourhood(0) &= \mathfrak{F}\setbuilder{\epsilon B}{\epsilon > 0, B\in \mathcal{B}} \\
&= \mathfrak{F}\setbuilder{\epsilon p_B^\preimf\big(\ball(0,1)\big)}{\epsilon > 0, B\in \mathcal{B}} \\
&= \mathfrak{F}\setbuilder{p_B^\preimf\big(\ball(0,\epsilon)\big)}{\epsilon > 0, B\in \mathcal{B}},
\end{align*}
where we have used \ref{gaugeInherenceAdherence}. Thus the convergence is indeed the seminorm convergence w.r.t.\ $S = \setbuilder{p_B}{B\in \mathcal{B}}$.
\end{proof}

\begin{proposition}
Let $V$ be a vector space. The functions
\begin{align*}
\powerset\setbuilder{A\subseteq V}{\text{$A$ is convex}} &\to \setbuilder{p: V\to \R}{\text{$p$ is a seminorm}}: &&\mathcal{B}\mapsto \setbuilder{p_K}{K\in \mathcal{B}} \\
\setbuilder{p: V\to \R}{\text{$p$ is a seminorm}} &\to \powerset\setbuilder{A\subseteq V}{\text{$A$ is convex}}: &&S\mapsto \setbuilder{p^\preimf[U]}{p\in S, U\in \neighbourhood_\R(0)}
\end{align*}
form an antitone Galois connection, where we order the seminorms pointwise.
\end{proposition}
\begin{proof}
TODO + previous as corollary
\end{proof}

Note: metrisable is not equivalent to normable!


\subsection{Locally convex modification}
\begin{definition}
Let $\sSet{V, \xi}$ be a convergence vector space. The \udef{locally convex modification} of $V$ is the initial convergence on $V$ w.r.t. the set of continuous seminorms on $V$.

We denote the locally convex modification of $\xi$ by $\lconvMod(\xi)$.
\end{definition}

\begin{lemma} \label{locallyConvexModLemma}
Let $\sSet{V, \xi}$ be a convergence vector space. The locally convex modification $\lconvMod(\xi)$ is the least locally convex topological vector space convergence that contains $\xi$.
\end{lemma}
In particular $\xi \leq \lconvMod(\xi)$.
\begin{proof}
Let $S$ be the set of continuous seminorms on $\sSet{V, \xi}$.

As $\xi$ makes all the functions in $S$ continuous, we must have $\xi \leq \lconvMod(\xi)$.

The convergence $\lconvMod(\xi)$ is locally convex topological by \ref{locallyConvexSeminormTopology}.

Finally, let $\zeta$ be another locally convex topological vector convergence on $V$ such that $\xi \leq \zeta$. Then, by \ref{locallyConvexSeminormTopology}, $\zeta$ is the initial convergence w.r.t. some set $S'$ of seminorms. We must have $S'\subseteq S$, so $\lconvMod(\xi)\leq \zeta$.
\end{proof}

\begin{proposition} \label{neighbourhoodLconvMod}
Let $\sSet{V,\xi}$ be a vector convergence space and $S$ the set of continuous seminorms on $V$. Then
\begin{enumerate}
\item $\neighbourhood_{\lconvMod(\xi)}(0) = \upset \bigcup_{p\in S} p^{\preimf\imf}\big(\neighbourhood_\R(0)\big) = \upset \setbuilder{p^{\preimf}\big(\ball_\R(0, \epsilon)\big)}{p\in S, \epsilon > 0}$;
\item $\neighbourhood_{\lconvMod(\xi)}(0) = \upset\disked^{\imf}\big(\vicinity_\xi(0)\big)$.
\end{enumerate}
\end{proposition}
\begin{proof}
(1) Comparing with \ref{pretopologicalInitialConvergence}, we need to show that for any two continuous seminorms $p_1, p_2$ on $V$ and $\epsilon_1,\epsilon_2 > 0$, there exists a continuous seminorm $q$ on $V$ and $\epsilon >0$ such that
\[ q^\preimf\big(\ball_\R(0,\epsilon)\big) \subseteq p_1^\preimf\big(\ball_\R(0,\epsilon_1)\big) \cap p_2^\preimf\big(\ball_\R(0,\epsilon_2)\big). \]
We may simply take $\epsilon = \min\{\epsilon_1, \epsilon_2\}$ and $q = p_1+p_2$, which is continuous. Indeed, $x\in (p_1+p_2)^\preimf\big(\ball_\R(0,\epsilon)\big)$ is equivalent to $p_1(x) + p_2(x) < \epsilon$, which, by positivity of the seminorm \ref{seminormPositivity}, implies $p_1(x)\leq \epsilon \leq \epsilon_1$ and $p_2(x)\leq \epsilon \leq \epsilon_2$.

(2) By \ref{locallyConvexNeighbourhoodsLemma}, $\neighbourhood_{\lconvMod(\xi)}(0)$ has a base of absolutely convex sets. As $\vicinity_\xi(0)\subseteq\neighbourhood_{\lconvMod(\xi)}(0)$, we have $\upset\disked^{\imf}\big(\vicinity_\xi(0)\big) \subseteq \neighbourhood_{\lconvMod(\xi)}(0)$. The show the converse, we just need to prove that the pretopological convergence with vicinity filter $\upset\disked^{\imf}\big(\vicinity_\xi(0)\big)$ at $0$ is a locally convex topological vector space convergence. We prove this by verifying the three points of \ref{LCTVSconstruction}.

First, take $A\in \vicinity_\xi(0)$. Then for all $\lambda\in\F\setminus\{0\}$, $\lambda A \in \vicinity_\xi(0)$ by \ref{vicinityFilterAtOrigin}. As $\disked(\lambda A) = \lambda\disked(A)$, by \ref{balancedHullHomogeneous}, \ref{convexHullHomogeneous} and \ref{diskedIsCoBal}, we have $\lambda \disked(A)\in \upset\disked^{\imf}\big(\vicinity_\xi(0)\big)$.

Secondly, each $A\in \vicinity_\xi(0)$ is absorbent by \ref{vicinityFilterAtOrigin}. As $A\subseteq \disked(A)$, $\disked(A)$ is absorbent by \ref{absorbingSetProperties} and so each element of $\upset\disked^{\imf}\big(\vicinity_\xi(0)\big)$ is absorbent.

The final point is immediate by construction.
\end{proof}

\begin{proposition} \label{lconvContinuityLinearFunctions}
Let $\sSet{V,\xi}$ and $\sSet{W,\zeta}$ be vector convergence spaces and $T: V\to W$ a linear function. Then
\begin{enumerate}
\item $T: \sSet{V,\xi} \to \sSet{W, \lconvMod(\zeta)}$ is continuous \textup{if and only if} $T: \sSet{V,\lconvMod(\xi)} \to \sSet{W, \lconvMod(\zeta)}$ is continuous;
\item if $T: \sSet{V,\xi} \to \sSet{W, \zeta}$ is continuous, then $T: \sSet{V,\lconvMod(\xi)} \to \sSet{W, \lconvMod(\zeta)}$ is continuous.
\end{enumerate}
\end{proposition}
\begin{proof}
(1) The direction $\Leftarrow$ is immediate by \ref{locallyConvexModLemma}.

Now suppose $T: \sSet{V,\xi} \to \sSet{W, \lconvMod(\zeta)}$ is continuous. Take arbitrary $F\in \powerfilters(V)$ that converges to $0$ in $\lconvMod(\xi)$. We just need to show that $\upset T^{\imf\imf}(F)$ converges to $0$ in $\lconvMod(\zeta)$, which, by \ref{initialFinalConvergence}, is equivalent to $\upset (p\circ T)^{\imf\imf}(F)$ converging to $0$ for all continuous seminorms on $\zeta$. And indeed all these filter do converge because $p\circ T$ are continuous seminorms on $\xi$ and thus $\upset (p\circ T)^{\imf\imf}(F)$ converges to $0$ by \ref{initialFinalConvergence}.

(2) If $T: \sSet{V,\xi} \to \sSet{W, \zeta}$ is continuous, then $T: \sSet{V,\xi} \to \sSet{W, \lconvMod(\zeta)}$ is continuous by \ref{locallyConvexModLemma} and $T: \sSet{V,\lconvMod(\xi)} \to \sSet{W, \lconvMod(\zeta)}$ is continuous by point (1).
\end{proof}

\begin{proposition} \label{equicontinuousSetsLconvMod}
Let $\sSet{V,\xi}$ be a convergence vector space and $\sSet{W,\zeta}$ a locally convex TVS. Then $\contLin(V,W)$ and $\contLin\big(\lconvMod(V), W\big)$ have the same equicontinuous subsets.
\end{proposition}
\begin{proof}
As sets, we have $\contLin(V,W) = \contLin\big(\lconvMod(V), W\big)$ by \ref{lconvContinuityLinearFunctions}.

Since $\xi \leq \lconvMod(\xi)$, we have that each equicontinuous subset of $\contLin\big(\lconvMod(V), W\big)$ is an equicontinuous subset of $\contLin(V,W)$.

Now suppose $H$ is an equicontinuous subset of $\contLin(V,W)$. 
In order to prove equicontinuity of $H$ as a subset of $\contLin\big(\lconvMod(V), W\big)$, it is enough to prove
\[ \upset \evalMap^{\imf\imf}\big(\{H\}\otimes \neighbourhood_{\lconvMod(\xi)}(0)\big) \overset{\zeta}{\longrightarrow} 0 \]
by \ref{equicontinuityGroupHomomorphisms}.

By equicontinuity of $H$ as a subset of $\contLin(V,W)$ and \ref{equicontinuityGroupHomomorphisms}, we have
\[ \upset \evalMap^{\imf\imf}(\{H\}\otimes F) \overset{\zeta}{\longrightarrow} 0 \]
for all $F\overset{\xi}{\longrightarrow} 0$. Because $\zeta$ is pretopological, this implies that
\begin{align*}
\bigcap_{F\overset{\xi}{\longrightarrow} 0}\upset \evalMap^{\imf\imf}(\{H\}\otimes F) &= \upset \evalMap^{\imf\imf}\Big(\bigcap_{F\overset{\xi}{\longrightarrow} 0}\big(\{H\}\otimes F\big)\Big) \\
&= \upset\evalMap^{\imf\imf}\Big(\{H\}\otimes \bigcap_{F\overset{\xi}{\longrightarrow} 0} F\Big) = \upset\evalMap^{\imf\imf}\big(\{H\}\otimes \vicinity_\xi(0)\big)
\end{align*}
converges to $0$ in $\zeta$. The equalities follow from \ref{imageUpsetsPreservesIntersection}, \ref{productGaloisConnections} and \ref{upsetResiduatedImageGaloisConnection}.
Now, we have for all $A\subseteq V$, by \ref{orderPreservingFunctionLatticeOperations} and \ref{linearFunctionsPreserveDiskedHull}, that
\begin{align*}
\disked\big(\evalMap^{\imf}(H\times A)\big) &= \disked\Big(\bigcup_{f\in H}\evalMap^{\imf}\big(\{f\}\times A\big)\Big) \\
&= \disked\Big(\bigcup_{f\in H}f^\imf(A)\Big) \\
&\supseteq \bigcup_{f\in H}\disked\big(f^\imf(A)\big) \\
&= \bigcup_{f\in H}f^\imf\big(\disked(A)\big) \\
&= \bigcup_{f\in H}\evalMap^\imf\big(\{f\}\times \disked(A)\big) \\
&= \evalMap^{\imf}\big(H\times \disked(A)\big),
\end{align*}
so, by \ref{neighbourhoodLconvMod},
\[ \upset\disked^\imf\Big(\evalMap^{\imf\imf}\big(\{H\}\otimes \vicinity_\xi(0)\big)\Big) \subseteq \upset \evalMap^{\imf\imf}\Big(\{H\}\otimes \disked^{\imf}\big(\vicinity_\xi(0)\big)\Big) = \upset \evalMap^{\imf\imf}\big(\{H\}\otimes \neighbourhood_{\lconvMod(\xi)}(0)\big), \]
which converges to $0$ by local convexity of $\zeta$ (\ref{locallyConvexNeighbourhoodsLemma}).
\end{proof}

\subsection{Initial and final locally convex spaces}
\begin{proposition}
The initial convergence w.r.t. a set of linear functions to LCTVSs is a locally convex topological vector convergence.
\end{proposition}
\begin{proof}
Let $V$ be a vector space and $\{u_i: V\to V_i\}$ a set of linear functions to LCTVSs.

The initial convergence on $V$ w.r.t. $\{u_i: V\to V_i\}$ is a topological vector convergence by \ref{pretopologicalInitialConvergence} and \ref{initialVectorSpaceConvergence}.

In order to show that the initial convergence is locally convex, we prove that $\id: \lconvMod(V) \to V$ is continuous. By \ref{characteristicPropertyInitialFinalConvergence}, this follows if each $u_i: \lconvMod(V) \to V_i$ is continuous, which in turn follows from \ref{lconvContinuityLinearFunctions}.
\end{proof}

\begin{proposition}
Let $V$ be a vector space and $\{u_i: \sSet{V_i, \xi_i} \to V\}$ a set of linear functions. Let $\nu$ be the final vector space convergence on $V$ w.r.t. this set and $\nu'$ be the final vector space convergence on $V$ w.r.t. $\{u_i: \sSet{V_i, \lconvMod(\xi_i)} \to V\}$. Then $\lconvMod(\nu) = \lconvMod(\nu')$.
\end{proposition}
\begin{proof}
We clearly have $\nu \leq \nu'$, so $\lconvMod(\nu) \leq \lconvMod(\nu')$. To show the converse, we need to show that $\id: \sSet{V, \lconvMod(\nu')} \to \sSet{V, \lconvMod(\nu)}$ is continuous. By \ref{lconvContinuityLinearFunctions}, this is equivalent to the continuity of $\id: \sSet{V, \nu'} \to \sSet{V, \lconvMod(\nu)}$, which by \ref{characteristicPropertyInitialFinalConvergence} is equivalent to the continuity of each $u_i: \sSet{V_i, \lconvMod(\xi_i)} \to \sSet{V, \lconvMod(\nu)}$, which in turn follows from the continuity of each $u_i: \sSet{V_i, \xi_i}\to \sSet{V, \nu}$ by \ref{lconvContinuityLinearFunctions}.
\end{proof}

\begin{proposition}
Let $\sSet{I, \{\sSet{V_i, \xi_i}\}_{i\in I}, \{e_{i,j}\}_{i\preceq j}}$ be an inductive system of LCTVSs with linear linking morphisms $e_{i,j}$.

Then $\topMod\big(\varinjlim_{i\in I}V_i\big) = \lconvMod\big(\varinjlim_{i\in I}V_i\big)$.
\end{proposition}
\begin{proof}
TODO?? Beattie / Butzmann p. 105
\end{proof}

\section{Hahn-Banach extension theorems}
\begin{theorem}[Hahn-Banach majorised by convex functionals] \label{convexHahnBanach}
Let $V$ be a real vector space, $U\subset V$ a subspace and $p$ a convex functional on $V$. Let $f:U\to\R$ be a linear functional that is bounded by $p$:
\[ \forall u\in U: \quad f(u) \leq p(u). \]
Then $f$ has an extension $\tilde{f}: V\to \R$ such that $\tilde{f}$ is a linear functional on $V$ bounded by $p$:
\[ \forall v\in V: \tilde{f}(v) \leq p(v) \qquad \text{and} \qquad \forall u\in U: \tilde{f}(u) = f(u). \]
\end{theorem}
\begin{proof}
As a first step, we want to extend $f$ to a functional $g$ on a space that is one dimension larger than $U$. This means $g$ is of the form
\[ g: U\oplus\Span\{v_1\}\to\R: v + \alpha v_1 \mapsto f(v) + \alpha c \]
for some $v_1\in V\setminus U$.

If we want $g$ to be majorised by $p$, then we need to find a $c$ such that
\[ \forall v\in U: \forall \alpha\in\R: \; g(\alpha v_1 + v) = \alpha c + f(v) \leq p(\alpha v_1 + v) \]
this means that we need
\[ \forall v\in U: \forall \alpha\in\R:\; \frac{-p(v - |\alpha|v_1) + f(v)}{|\alpha|} \leq c \leq \frac{p(v + |\alpha|v_1) - f(v)}{|\alpha|} \]
and we can find such a $c$ if and only if
\[ \forall v\in U: \forall \alpha\in\R:\; -p(v - |\alpha|v_1) + f(v) \leq p(v + |\alpha|v_1) - f(v), \]
which is equivalent to $2f(v) \leq p(v+|\alpha|v_1)+p(v-|\alpha|v_1)$. This follows from
\begin{align*}
f(v) \leq p(v) &= p(\tfrac{1}{2}(v+|\alpha|v_1) + \tfrac{1}{2}(v-|\alpha|v_1)) \\
&\leq \tfrac{1}{2}p(v+|\alpha|v_1) + \tfrac{1}{2}p(v-|\alpha|v_1).
\end{align*}
So we can extend the domain of $f$ by one dimension such that it is still majorised by $p$.

We can iterate the construction to extend $f$ by multiple dimensions. Each extension can be viewed as a subset of $V\times \R$, by identifying it with its graph.
Consider the family of all such subsets that determine a majorised extension of $f$ (not just those obtained by iteration of the previous construction!). This is a family of finite character. We apply the Teichmüller-Tukey lemma, \ref{ZornEquivalents}, to obtain a maximal element.

This maximal element has domain $V$, because if it did not, it could be extended and was not a maximal element.
\end{proof}
Clearly if $V$ has a well-ordered Hamel basis, we do not need choice as we can just take successive $v$s in the basis and find $c$s constructively.
\begin{corollary}[Hahn-Banach majorised by sublinear functionals] \label{sublinearHahnBanach}
Any majorant $p$ that is sublinear is also convex and can be used in the Hahn-Banach theorem.
\end{corollary}
\begin{corollary}[Hahn-Banach majorised by seminorms] \label{seminormHahnBanach}
Let $(\mathbb{F},V,+)$ be a real or complex vector space, $U\subset V$ a subspace and $p$ a seminorm on $V$. Let $f:U\to\mathbb{F}$ be a linear functional that is bounded by $p$:
\[ \forall u\in U: \quad |f(u)| \leq p(u). \]
Then $f$ has an extension $\tilde{f}: V\to \R$ such that $\tilde{f}$ is a linear functional on $V$ bounded by $p$:
\[ \forall v\in V: |\tilde{f}(v)| \leq p(v) \qquad \text{and} \qquad \forall u\in U: \tilde{f}(u) = f(u). \]
\end{corollary}
\begin{proof}
First assume $V$ is a \emph{real} vector space. Because every seminorm is a sublinear function, we can use \ref{sublinearHahnBanach} to find an extension $\tilde{f}$. We then just need to check it satisfies $\forall v\in V: |\tilde{f}(v)| \leq p(v)$.
From \ref{sublinearHahnBanach} we know $\forall v\in V: \tilde{f}(v) \leq p(v)$.
To prove $-\tilde{f}(v) \leq p(v)$, we calculate
\[ -\tilde{f}(v) = \tilde{f}(-v) \leq p(-v) = |-1|p(v) = p(v). \]

If $V$ is a \emph{complex} vector space, consider the realification $V_\R$ and apply the preceding proof to obtain a linear functional $g: V_\R \to \R$ that extends $f$ and is majorised by $p$. Then by \ref{complexRangeExtensionRealFunctional} we can find a complex-linear functional $\tilde{f}:V \to \C$ such that $\Re(\tilde{f}) = g$.

We just need to show that $f$ is bounded by $p$. Take arbitrary $v\in V$ and write $\tilde{f}(v) = |\tilde{f}(v)|e^{i\theta}$ then
\[ |\tilde{f}(v)| = \Re|\tilde{f}(v)| = \Re\Big(e^{-i\theta}\tilde{f}(v)\Big) = \Re\Big(\tilde{f}(e^{-i\theta}v)\Big) = g(e^{-i\theta}v) \leq p(e^{-i\theta}v) = |e^{-i\theta}|p(v) = p(v). \]
\end{proof}
\begin{corollary}
Let $V$ be a locally convex convergence vector space, $U\subseteq V$ a subspace and $f:U\to \F$ a continuous functional. Then $f$ has a continuous extension to $V$.
\end{corollary}
\begin{proof}
We have that $|f| = p_K$ for some $K\in \vicinity_U(0)$ by \ref{absoluteFunctionalGauge}. 
By continuity of the inclusion map, we can find an $M \in \vicinity_V(0)$ such that $M\cap U = K$. Then $|f|\leq p_M$ and $f$ can be extended by the Hahn-Banach extension theorem to a functional $f'$ defined in the whole of $V$. Then $f'$ is bounded on $M$ and thus continuous by \ref{boundedOnVicinityImpliesContinuous}.
\end{proof}


\subsection{Hahn-Banach separation}

\begin{lemma} \label{gaugeSeparationLemma}
Let $V$ be a real vector space, $A$ an absorbent, semibalanced set and $x_0 \notin A$. Consider the functional $f_{x_0}: \Span\{x_0\}\to \F: tx_0 \mapsto t$. Then $f_{x_0}(x)\leq p_A(x)$ for all $x\in \Span\{x_0\}$.
\end{lemma}
\begin{proof}
Let $x = tx_0$. If $t\leq 0$, then the inequality is immediate. Suppose $t>0$. Because $p_A(x_0) \geq 1$ (by the converse of \ref{gaugeLemma}), we have
\[ f_{x_0}(x) = f_{x_0}(tx_0) = t \leq tp_A(x_0) = p_A(tx_0) = p_A(x)  \]
using positive homogeneity (\ref{gaugeScaling}).
\end{proof}

\begin{theorem}[Mazur] \label{MazurTheorem}
Let $V$ be a real or complex convergence vector space and $A$ an open and convex set. If $U$ is a subspace such that $A\perp U$, then there exists a closed hyperplane $H \supseteq U$ such that $A\perp H$.
\end{theorem}
\begin{proof}
First suppose $V$ is a \emph{real} vector space. Because $A$ is open, it is algebraically open. Take $a\in A$. Then $0\in a-A = \inh_{\mathfrak{a}}(a-A)$, by \ref{constructionsInAlgebraicConvergence}, so $a-A$ is absorbing by \ref{coreProperties}. It is also semibalanced (by \ref{convexAbsorbentImpliesSemibalanced} as it is convex by \ref{translationScalingConvexSet}).

Then we have
\[ U\perp A \iff 0\notin U-A \iff a \notin a-A+U. \]
Consider the functional $f_{a}$ of \ref{gaugeSeparationLemma}, which is majorised by the gauge $p_{a-A+U}$, which is sublinear by \ref{gaugeProperties}. Then $f_a$ can be extended as an $\R$-linear function to all $V$ by the Hahn-Banach extension theorem \ref{sublinearHahnBanach}.

We note that $U\subseteq \ker(f_a)$, because $p_{a-A+U}(u) = 0$ by \ref{gaugeZeroLemma}.

In order to conclude with \ref{functionalBoundedNeighbourhood}, we need to show that $A-a \subseteq f_a^{\preimf}(\ball(0,|f_a(a)|)) = f_a^{\preimf}(\ball(0,1))$.
Indeed $A-a \subseteq U+A-a = \inh_\mathfrak{a}(U+A-a) \subseteq p_{U+A-a}^\preimf[\ball(0,1)] \subseteq f_{a}^\preimf[\ball(0,1)]$ by \ref{algebraicallyOpen} and \ref{gaugeInherenceAdherence} ($U+A-a$ is semibalanced because $A-a$ is).

Note that $\ker(f_a)^c$ contains the open set $A$ and thus $\ker(f_a)$ is not dense by \ref{openDensityLemma}. By \ref{hyperplaneClosedDense} this means that $\ker(f_a)$ is closed.

Now suppose $V$ is a \emph{complex} vector space. We can consider the realification $V_\R$ with the same convergence, which is a real convergence vector space by TODO \ref{}. So we can use the preceding proof to find a real hyperplane $K$ in $V$. Then \ref{realComplexHyperplane} gives that $H = K\cap iK$ is a complex hyperplane in $V$. Now $H$ and $A$ must be disjoint because $K$ and $A$ are disjoint and $H \subseteq K$.

Also $H$ is closed because $K$ and $iK$ are closed (the first by the preceding proof, the second because multiplication by $i$ is a homeomorphism \ref{continuityLemmaVectorConvergence}) and the intersection of two closed sets is closed.
\end{proof}
\begin{corollary} \label{functionalZeroOnClosedSubSpace}
Let $V$ be a locally convex vector space and $M$ a closed subspace. There exists a non-zero bounded linear functional $f$ on $V$ such that $M\subseteq \ker(f)$.
\end{corollary}
\begin{proof}
The set $M^c$ is open and by local convexity it contains a convex set $C$. We may take $C$ open by replacing it with its interior, which is convex by \ref{inherenceAdherenceConvex}. Now $C\perp M$ and we can apply the theorem. Now $\ker(f)$ is closed, so $f$ is bounded by \ref{continuityLinearFunctionals}.
\end{proof}

\begin{theorem}[Hahn-Banach separation theorem] \label{HahnBanachSeparation}
Let $V$ be a convergence vector space. Suppose $A,B$ are disjoint, non-empty, convex sets and that $A$ is open. Then there exists a continuous linear functional $f:V\to \F$ such that $f^\imf[A]$ and $f^\imf[B]$ are disjoint.
\end{theorem}
\begin{proof}
The set $A-B = \bigcup_{b\in B}A-b$ is convex and open by \ref{sumOpenSetsOpen}.
The set $A-B$ and the vector space $\{0\}$ are disjoint, so by \ref{MazurTheorem} we can find a closed hyperplane that is disjoint with $A-B$.

By \ref{kernelHyperplane} and \ref{continuityLinearFunctionals} this is the kernel of a continuous linear functional $f$.
\end{proof}
\begin{corollary} \label{separatingFunctionalOrderedImage}
Let $V$ be a real or complex convergence vector space and $A,B$ as in the proposition. Then there exists a continuous linear functional $f:V\to \F$ and $t\in \R$ such that
\[ \Re f(a) < t \leq \Re f(b) \]
for all $a\in A$ and $b\in B$.
\end{corollary}
This means $A$ and $B$ are separated by a closed affine hyperplane $\ker(f)+v$, where $v \in f^\preimf[\{t\}]$.

We can reverse the inequalities by replacing $f$ by $-f$.
\begin{proof}
Apply the proposition to the realification $V_\R$. This gives us an $\R$-linear functional $g: V\to \R$ such that $g^\imf[A]$ and $g^\imf[B]$ are disjoint convex sets. Additionally $g^{\imf}[A]$ is open in $\R$ by \ref{linearFunctionalOpen}.

Because $g^\imf[A]$ and $g^\imf[B]$ are convex, we either have $g^\imf[A]\leq g^\imf[B]$ or $g^\imf[A]\geq g^\imf[B]$. In the second case we simply replace $g$ by $-g$ to obtain the first case. We may take $t= \sup g^\imf[A]$. This is not in $g^\imf[A]$ because it is open.

If $V$ is a real vector space we take $f=g$ are done. If $V$ is complex, we can find a suitable $f$ by \ref{complexRangeExtensionRealFunctional}.
\end{proof}
\begin{corollary} \label{locallyConvexHahnBanachSeparationClosedSet}
Let $\sSet{V, \xi}$ be a locally convex vector convergence space. Let $B$ be a closed convex set and $v\notin B$, then there exists a continuous linear functional $f:V\to \F$ such that $f(v) \notin \overline{f^\imf[B]}$.
\end{corollary}
\begin{proof}
As $B^c$ is open, it is a neighbourhood of $v$ and we can find an convex neighbourhood $U$ of $v$ in $B^c$. Now replace $U$ by its interior. This is open and still convex by \ref{inherenceAdherenceConvex}, so we can apply the theorem.

Then $f^\imf[U]$ is open by \ref{linearFunctionalOpen} and thus in $\neighbourhood_\F\big(f(v)\big)$. It is however disjoint from $f^\imf[B]$, which means that $f(v) \notin \overline{f^\imf[B]}$.
\end{proof}
\begin{corollary}
Let $V$ be a locally convex TVS. Suppose $A,B$ are disjoint, non-empty, convex sets and that $A$ is compact, $B$ is closed. Then there exists a continuous linear functional $f:V\to \F$ and $s,t\in \R$ such that
\[ \Re f(a) < t < s < \Re f(b) \]
for all $a\in A$ and $b\in B$.
\end{corollary}
\begin{proof}
TODO
\end{proof}
\begin{corollary} \label{locallyConvexDualPair}
Let $V$ be a Hausdorff locally convex convergence vector space and $v\in V$. If $f(v) = 0$ for all $f\in V^*$, then $v = 0$.
\end{corollary}
\begin{proof}
We prove the contrapositive. Assume $v\neq 0$. By \ref{FrechetCharacterisation}, we have that $\{0\}$ is closed, so $\{0\}^c$ is a neighbourhood of $v$. By local convexity, $\{0\}^c$ contains a convex vicinity of $x$. We may take this vicinity to be open by taking the interior (which is still convex by \ref{inherenceAdherenceConvex}). Let this open vicinity be $A$ and set $B = \{0\}$. By Hahn-Banach separation, there exists $f\in \dual{V}$ such that $f^\imf[A]$ and $f^\imf[B] = \{0\}$ are disjoint. Thus $f(v) \neq 0$.
\end{proof}

\subsection{Banach limits}
\begin{proposition}
There exists a linear map $L:l^\infty(\N) \to \C$ satisfying
\begin{enumerate}
\item $\displaystyle L(x) = \lim_{n\to \infty}x_n$ if the limit exists;
\item $L((x_{n+1})_{n\in\N}) = L((x_n)_{n\in\N})$;
\item if $\forall n\in\N:x_n\geq 0$, then $L(x) \geq 0$;
\item $\norm{L} = 1$.
\end{enumerate}
Such a linear map is called a \udef{Banach limit}.
\end{proposition}
\begin{proof}
TODO, after Cesàro means.
\end{proof}

\section{Continuous functionals}

TODO???
\begin{proposition}
Let $V$ and $W$ be TVSs and $f: V\to W$ a linear function.
\begin{enumerate}
\item If $f$ is continuous and $W$ is Hausdorff, then $\ker(f)$ is closed.
\item If $f$ has closed kernel and finite-dimensional image, then $f$ is continuous.
\end{enumerate}
\end{proposition}
\begin{proof}
(1) Because $W$ is Hausdorff, it is also $T_1$ and thus $\{0\}$ is closed by \ref{FrechetCharacterisation}. Then $\ker(f) = f^{\preimf}(\{0\})$ is closed by \ref{continuity}.

(2) 
\end{proof}
??

\chapter{Boundedness and bornology}

\section{Bornologies}
\begin{definition}
Let $V$ be a vector space. A bornology $\mathcal{B}\subseteq \powerset(V)$ is called a \udef{vector bornology} if
\begin{itemize}
\item $A+B\in \mathcal{B}$ for all $A,B\in \mathcal{B}$;
\item $\lambda A \in \mathcal{B}$ for all $\lambda\in \F$ and $A\in \mathcal{B}$;
\item $\balanced(A) \in \mathcal{B}$ for all $A\in \mathcal{B}$.
\end{itemize}
It is called a \udef{convex vector bornology} if, in addition,
\begin{itemize}
\item $\convex(A) \in \mathcal{B}$ for all $A\in \mathcal{B}$.
\end{itemize}
A pair $\sSet{V,\mathcal{B}}$, where $V$ is a vector space and $\mathcal{B}$ is a vector bornology on $V$ is called a \udef{bornological vector space}.
\end{definition}

\subsection{Convergence derived from bornology}
\begin{definition}
Let $\sSet{V, \mathcal{B}}$ be a bornological vector space. The \udef{associated convergence} $\bornConv{\mathcal{B}}$ is defined by
\[ F\overset{\bornConv{\mathcal{B}}}{\longrightarrow} v \qquad\defequiv\qquad \exists B\in \mathcal{B}: \; \neighbourhood_\F(0)\cdot B \subseteq F-v, \]
for all $F\in \powerfilters(V)$ and $v\in V$.
\end{definition}

\begin{lemma}
Let $\sSet{V, \mathcal{B}}$ be a bornological vector space. The associated convergence $\bornConv{\mathcal{B}}$ is a vector space convergence.
\end{lemma}
\begin{proof}
It is straightforward to verify the points in \ref{vectorSpaceConvergenceConstruction}.
\end{proof}

\section{Von Neumann boundedness}
\begin{definition}
Let $\sSet{V,\xi}$ be a convergence vector space on a field $\F$. A subset $A\subseteq V$ is called \udef{von Neumann bounded} or \udef{vN bounded} if $\neighbourhood_\F(0) \cdot^{\imf\imf} \upset\{A\} \overset{\xi}{\longrightarrow} 0$.
\end{definition}
It may be more correct to trace this definition of bounded to Mazur and Orlicz or Kolmogorov.

Clearly subsets of bounded sets are bounded.

\begin{lemma} \label{vonNeumannBoundednessAbsorption}
Let $\sSet{V,\xi}$ be a convergence vector space on a field $\F$ and $A\subseteq V$ a subset. 
\begin{enumerate}
\item If $A$ is von Neumann bounded, then $A$ is absorbed by all vicinities of the origin (i.e.\ by all elements of $\vicinity_{\xi}(0)$).
\item If $\xi$ is topological, the following are equivalent:
\begin{enumerate}
\item $A$ is von Neumann bounded;
\item $A$ is absorbed by all vicinities of the origin;
\item $A$ is absorbed be all elements of a base of $\neighbourhood_\xi(0)$;
\item $\forall U\in \neighbourhood_{\xi}(0): \exists r > 0: \forall c \geq r: \; A \subseteq cU$;
\item $\{\interval{0,\epsilon}\}_{\epsilon >0}\cdot \{A\} \overset{\xi}{\longrightarrow} 0$.
\end{enumerate}
\end{enumerate}
\end{lemma}
\begin{proof}
We just need to show that ``$A$ is absorbed by all vicinities of the origin'' is equivalent to $\vicinity_\xi(0) \subseteq \neighbourhood_\F(0) \cdot \{A\}$. Indeed $A$ is absorbed by $V\in \vicinity_\xi(0)$ iff there exists $\epsilon >0$ such that $\ball(0,\epsilon)\cdot A \subseteq V$ (see \ref{absorbingSetLemma}). Then (1) follows from the fact that $\vicinity_\xi(0)$ is smaller than all convergent filters.

(2) The implication $(a) \Rightarrow (b)$ is given by point (1).

$(b)\Rightarrow (a)$ As above, $(b)$ implies $\neighbourhood_\xi(0) \subseteq \neighbourhood_\F(0) \cdot \{A\}$. By topologicity this implies that $\neighbourhood_\F(0) \cdot \{A\}$ converges and thus that $A$ is von Neumann bounded.

The implication $(b)\Rightarrow (c)$ is immediate and the converse follows by \ref{absorbingSetProperties}.

$(a) \Rightarrow (d)$ is immediate.

$(d) \Rightarrow (c)$ Pick a balanced base of $\neighbourhood_{\xi}(0)$ (which is possible due to \ref{vicinityFilterAtOrigin}). Then the result follows immediately from the observation that $cU = |c|U$ (\ref{balancedLemma}) for all balanced sets.

$(d) \Leftrightarrow (e)$ We have that $\forall c \geq r: A\subseteq cU$ is equivalent to $\interval{0,r^{-1}}\cdot A \subseteq U$. Thus $(d)$ is equivalent to $\neighbourhood_\xi(0)\subseteq \{\interval{0,\epsilon}\}_{\epsilon >0}\cdot \{A\}$, which is equivalent to $(e)$ because $\xi$ is topological.
\end{proof}


\begin{lemma} \label{vonNeumannBoundedSetLemma}
Let $\sSet{V,\xi}$ be a convergence vector space and $A,B\subseteq V$ von Neumann bounded subsets. Then
\begin{enumerate}
\item every $C\subseteq A$ is von Neumann bounded;
\item the von Neumann bounded sets cover $V$;
\item $\lambda A$ is von Neumann bounded for all $\lambda\in\F$;
\item $A+B$ is von Neumann bounded;
\item $\balanced(A)$ is von Neumann bounded.
\end{enumerate}
If $\xi$ is of finite depth, then
\begin{enumerate} \setcounter{enumi}{2}
\item $A\cup B$ is von Neumann bounded.
\end{enumerate}
If $\xi$ is topological, then
\begin{enumerate} \setcounter{enumi}{3}
\item $\closure_\xi(A)$ is von Neumann bounded.
\end{enumerate}
If $\xi$ is topological and locally convex, then
\begin{enumerate}\setcounter{enumi}{6}
\item $\disked(A)$ is von Neumann bounded.
\end{enumerate}
\end{lemma}
\begin{proof}
(1) We have $\neighbourhood_\F(0) \cdot^{\imf\imf} \upset\{A\} \subseteq \neighbourhood_\F(0) \cdot^{\imf\imf} \upset\{C\}$.

(2) It is enough to show that $\{v\}$ is von Neumann bounded for all $v\in V$. We have that $\neighbourhood_\F(0) \cdot^{\imf\imf} \pfilter{v}$ converges for all $v\in V$ by continuity of scalar multiplication and the fact that $\pfilter{v}$ converges. 

(3) We have, by \ref{balancedClosures},
\[ \neighbourhood_\F(0) \cdot \upset\{A\} = \upset \{\cball(0,\epsilon) \cdot A\}_{\epsilon >0} = \upset \{\cball(0,\epsilon)\cdot \cball(0,1) \cdot A\}_{\epsilon >0} = \upset \{\cball(0,\epsilon)\cdot \balanced(A)\}_{\epsilon >0} = \neighbourhood_\F(0) \cdot \upset\{\balanced(A)\} \]

(4,5) Immediate by continuity of scalar multiplication and addition.

(6) We have $\neighbourhood_\F(0)\cdot^{\imf\imf}\upset\{A\cup B\} = \neighbourhood_\F(0)\cdot^{\imf\imf}\upset\{A\} \cap \neighbourhood_\F(0)\cdot^{\imf\imf}\upset\{B\}$.

(7) By \ref{vicinityFilterAtOrigin}, we may take a closed and balanced base of $\vicinity_{\xi}(0)$.  
By \ref{vonNeumannBoundednessAbsorption} it is enough to show absorption by these basis elements. This is immediate, because for all closed and bounded $U$, we have $A\subseteq cU$ iff $\closure_\xi(A) \subseteq cU$ iff $\disked(A) \subseteq cU$.

(8) Similar, now taking convex balanced base, \ref{locallyConvexNeighbourhoodsLemma}.
\end{proof}
\begin{corollary}
Let $\sSet{V, \xi}$ be a convergence vector space. The set of von Neumann bounded subsets of $V$ forms a vector space bornology.
\end{corollary}

\begin{definition}
Let $\sSet{V, \xi}$ be a convergence vector space. The set of von Neumann bounded subsets of $V$ is called the \udef{von Neumann bornology} of $V$ and is denoted $\vNborn{\xi}$.
\end{definition}

\subsection{Von Neumann boundedness and continuity}
\begin{proposition} \label{continuousMappingBoundedSets}
Let $\sSet{V,\xi}, \sSet{W,\zeta}$ be vector convergence spaces, $f:V\to W$ a function and $A\subseteq V$ a von Neumann bounded subset. Then
\begin{enumerate}
\item if $f$ is linear and continuous, then $f^\imf(A)$ is von Neumann bounded;
\item if $\zeta$ is topological and $f$ is positively homogeneous and continuous at $0$, then $f^\imf(A)$ is bounded.
\end{enumerate} 
\end{proposition}
\begin{proof}
(1) As $A$ is von Neumann bounded set, we have $\neighbourhood_\F(0) \cdot \upset\{A\} \overset{\xi}{\longrightarrow} 0$. By continuity and linearity of $f$, we have
\[ \neighbourhood_\F(0) \cdot \upset\big\{f^{\imf}(A)\big\} = \upset f^{\imf\imf}\big(\neighbourhood_\F(0)\cdot \{A\}\big) \overset{\zeta}{\longrightarrow} 0, \]
so $f^\imf(A)$ is von Neumann bounded.

(2) We have $\neighbourhood_\F(0) \cdot \upset\{A\} \subseteq \{\interval{0,\epsilon}\}_{\epsilon >0} \cdot \upset\{A\} \overset{\xi}{\longrightarrow} 0$. By positive homogeneity and continuity at $0$, we have
\[ \{\interval{0,\epsilon}\}_{\epsilon >0} \cdot \upset\big\{f^{\imf}(A)\big\} = \upset f^{\imf\imf}\big(\{\interval{0,\epsilon}\}_{\epsilon >0} \cdot \upset\{A\}\big) \overset{\zeta}{\longrightarrow} f(0) = 0. \]
By \ref{vonNeumannBoundednessAbsorption} this implies that $f^{\imf}(A)$ is von Neumann bounded.
\end{proof}
\begin{corollary}
Let $\sSet{V,\xi}, \sSet{W,\zeta}$ be vector convergence spaces and $f:V\to W$ a linear function. If $f: \sSet{V,\xi}\to \sSet{W,\zeta}$ is continuous, then $f: \sSet{V,\vNborn{\xi}}\to \sSet{W,\vNborn{\zeta}}$ is bounded.
\end{corollary}

\begin{lemma} \label{boundedSetVicinityBase}
Let $\sSet{V,\xi}$ be a topological vector convergence space. If $U\subseteq V$ is a von Neumann bounded neighbourhood of the origin, then
\[ \neighbourhood_\xi(0) = \neighbourhood_\F(0) \cdot\upset\{U\} = \upset\{\epsilon U\}_{\epsilon >0}. \]
\end{lemma}
TODO does this imply topological for equable CVSs?
\begin{proof}
As $U$ is von Neumann bounded, we have $\neighbourhood_\xi(0) \subseteq \neighbourhood_\F(0) \cdot \upset\{U\}$. The converse inclusion is given by the fact that $U\in \neighbourhood_\xi(0)$, so $\upset\{U\}\subseteq \neighbourhood_\xi(0)$ and $\neighbourhood_\F(0) \cdot \upset\{U\} \subseteq \neighbourhood_\F(0) \cdot\neighbourhood_\xi(0) = \neighbourhood_\xi(0)$ by \ref{TVSEquable}.

For the second equality, we have that $U$ contains a balanced neighbourhood $U'$ of the origin by \ref{vicinityFilterAtOrigin}, which is von Neumann bounded by \ref{vonNeumannBoundedSetLemma}. By \ref{balancedClosures}, and the previous argument, we have
\[ \neighbourhood_\xi(0) = \neighbourhood_\F(0) \cdot\upset\{U'\} = \upset \{\cball(0,\epsilon)\cdot U'\}_{\epsilon > 0} = \upset \{\epsilon\cball(0,1)\cdot U'\}_{\epsilon > 0} = \upset \{\epsilon U'\}_{\epsilon > 0}. \]
Then we have
\[ \neighbourhood_\xi(0) = \neighbourhood_\F(0) \cdot\upset\{U\} \subseteq \upset \{\epsilon U\}_{\epsilon > 0} \subseteq \upset \{\epsilon U'\}_{\epsilon > 0} = \neighbourhood_\xi(0). \]
\end{proof}
\begin{corollary} \label{metrisableBoundedNeighbourhood}
Let $\sSet{V,\xi}$ be a topological vector convergence space. If $U\subseteq V$ is a von Neumann bounded neighbourhood of the origin, then
\begin{enumerate}
\item $\sSet{V,\xi}$ is pseudometrisable;
\item if $U$ is convex, then $\sSet{V,\xi}$ is seminormable.
\end{enumerate}
\end{corollary}
\begin{proof}
(1) Clearly $\neighbourhood_\xi(0)$ has a countable base, so the neighbourhood filter at every point has a countabel base by \ref{shiftHomeomorphism}, which means that $\xi$ is second countable. It is also regular by \ref{topologicalGroupsRegular} and this pseudometrisable by Urysohn's metrisation theorem \ref{UrysohnMetrisationTheorem}.

(2) TODO Narici / Beckenstein p. 160.
\end{proof}

\begin{proposition} \label{boundedOnVicinityImpliesContinuous}
Let $\sSet{V,\xi}$ vector convergence space and $\sSet{W,\zeta}$ a topological vector convergence space and $f: V\to W$ a positively homogeneous function. Then
\begin{enumerate}
\item if there exists $U\in \vicinity_\xi(0)$ such that $f^\imf(U)$ is bounded, then $f$ is continuous at the origin;
\item if $f$ is linear, then this implies the continuity of $f$ everywhere.
\end{enumerate}
\end{proposition}
\begin{proof}
(1) By \ref{pretopologicalContinuityVicinities}, it is enough to verify $\neighbourhood_\zeta(0) \subseteq \upset f^{\imf\imf}\big(\vicinity_\xi(0)\big)$. Indeed, by \ref{boundedSetVicinityBase}, we have
\[ \neighbourhood_\zeta(0) = \upset \{\epsilon f^\imf(U)\}_{\epsilon > 0} = \upset f^{\imf\imf}\big(\{\epsilon U\}_{\epsilon > 0}\big) \subseteq \upset f^{\imf\imf}\big(\vicinity_\xi(0)\big), \]
where the last inclusion follows from \ref{vicinityFilterAtOrigin}.

(2) The continuity of $f$ is equivalent to the continuity of $f$ at $0$, by \ref{shiftHomeomorphism}.
\end{proof}
\begin{corollary} \label{continuityToNormedSpace}
Let $\sSet{V, \xi}$ be a convergence vector space, $\sSet{W, \norm{\cdot}}$ a normed space and $f: V\to W$ a positively homogeneous function. Then
\begin{enumerate}
\item $f$ is continuous at $0$ \textup{if and only if} $f$ is bounded on some $U\in \vicinity_\xi(0)$;
\item if $f$ is linear, then this is equivalent to the continuity of $f$ everywhere.
\end{enumerate}
\end{corollary}
\begin{proof}
(1) The direction $\Leftarrow$ is given by the proposition.

For the converse, assume $f$ continuous at $0$, then $\ball\big(0,1\big)\in \neighbourhood_W(0) = \neighbourhood_W\big(f(0)\big) \subseteq f^{\imf\imf}[\vicinity_\xi(0)]$ by \ref{continuityVicinityFilter}. So there exists $U\in \vicinity_\xi(0)$ such that $f^{\imf}[U] \subseteq \ball(0,1)$, which means that $f$ is bounded by $1$ on $U$.

(2) The continuity of $f$ is equivalent to the continuity of $f$ at $0$, by \ref{shiftHomeomorphism}.
\end{proof}

\begin{proposition} \label{vonNeumannBoundednessInitialSpace}
Let $V$ be a vector space and $B\subseteq V$.
\begin{enumerate}
\item Let $L = \{f_i:V\to \sSet{Y_i,\zeta_i}\}_{i\in I}$ be a set of linear functions to topological vector spaces, then $B$ is von Neumann bounded in the initial convergence w.r.t. $L$ iff $f_i^\imf(B)$ is von Neumann bounded for all $f_i\in L$.
\item Let $S$ be a set of seminorms on $V$, then $B$ is von Neumann bounded in the seminorm topology w.r.t. $S$ \textup{if and only if} $p^\imf(B)$ is von Neumann bounded for all $p\in S$.
\end{enumerate}
\end{proposition}
\begin{proof}
(1) We have, by \ref{initialFinalConvergence}, that $\neighbourhood_\F(0)\cdot\upset \{B\}$ converges to $0$ in the initial convergence if and only if $f_i^{\imf\imf}\big(\neighbourhood_\F(0)\cdot\upset \{B\}\big) = \neighbourhood_\F(0)\cdot\upset \big\{f_i^{\imf}(B)\big\} \overset{\xi_i}{\longrightarrow} 0$. I.e\ $f_i^\imf(B)$ is bounded for each $i\in I$.

(2) Since the vector space convergence on $V$ is topological in this case (by \ref{initialSeminormConvergence}), we have that $B$ is von Neumann bounded if and only if $\upset \{\interval{0,\epsilon}\}_{\epsilon > 0}\cdot\upset\{B\}$ converges to $0$, which, by \ref{initialSeminormConvergence}, is equivalent to $\upset p^{\imf\imf}\big(\{\interval{0,\epsilon}\}_{\epsilon > 0}\cdot\upset\{B\}\big) = \upset \{\interval{0,\epsilon}\}_{\epsilon > 0}\cdot\upset\big\{p^{\imf}(B)\big\} \overset{\R}{\longrightarrow} 0$ for all $p\in S$. I.e.\ $p^{\imf}(B)$ is bounded for all $p\in S$.
\end{proof}

\subsection{Uniform and von Neumann boundedness}
\begin{proposition} \label{boundednessTVS}
Let $\sSet{V,\xi}$ be a topological vector space and $B\subseteq V$. Then
\[ \text{$B$ is totally bounded}\quad\implies\quad\text{$B$ is von Neumann bounded}\quad\implies\quad\text{$B$ is bounded}. \]
If $\xi$ is locally convex, then von Neumann boundedness and boundedness are equivalent.
\end{proposition}
In fact all three are equivalent in the case of weak convergences (\ref{weakBoundedness}).
\begin{proof}
First suppose $B$ is totally bounded and let $U$ be a neighbourhood of $0$. By \ref{TVSconstruction}, we can find a balanced neighbourhood $U'$ of $0$ such that $U'+U'\subseteq U$.

Now $\Delta^\preimf(U')\in \entourage_\xi$ by \ref{entourageConvergenceGroup}. So by total boundedness (and \ref{topologicalBoundednessLemma}), there exists a finite set $S\subseteq V$ such that $B\subseteq \bigcup_{v\in S}\Delta^\preimf(U')v$. By \ref{deltaPreimageLemma}, we have
\[ B\subseteq \bigcup_{v\in S}\Delta^\preimf(U')v = \bigcup_{v\in S} v+U' = S+U'. \]
As $S$ is von Neumann bounded by \ref{vonNeumannBoundedSetLemma}, we can find $a\geq 1$ such that $S\subseteq cU'$ for all $|c|\geq a$.

For all $|c|\geq a$, we have $B\subseteq S+U' \subseteq cU' + U' \subseteq cU'+cU' \subseteq cU$. As $U$ was chosen arbitrarily, von Neumann boundedness follows from \ref{vonNeumannBoundednessAbsorption}.

Now assume $B$ is von Neumann bounded. We show boundedness using \ref{topologicalBoundednessLemma}, so take $A\in \entourage_\xi$. Then there exists $U\in \neighbourhood_\xi(0)$ such that $\Delta^\preimf(U)\subseteq A$.

By von Neumann boundedness, we have $B\subseteq mU$ for some $m\in\N$. So, using \ref{deltaPreimageLemma} and \ref{vectorDeltaLemma},
\[ B\subseteq mU = m\Delta^\preimf(U)_{\{0\}} = \Delta^\preimf(mU)_{\{0\}} \subseteq \Delta^\preimf(U)^m_{\{0\}} \subseteq A^m_{\{0\}}. \]
As $A$ was taken arbitrarily, this shows boundedness.

Finally, let $\sSet{V,\xi}$ be a locally convex TVS and assume $B$ bounded. Now $\xi$ is the initial topology w.r.t. some set $S$ of seminorms on $V$, by \ref{locallyConvexSeminormTopology}.

All the seminorms in $S$ are uniformly continuous by \ref{uniformContinuitySeminorms} and thus bounded on $B$ by \ref{metricBoundedness}. This implies von Neumann boundedness by \ref{vonNeumannBoundednessInitialSpace}.
\end{proof}

\begin{proposition} \label{compactSubsetsVonNeumannBounded}
Let $\sSet{V,\xi}$ be a Hausdorff pseudotopological convergence vector space. Then each compact subset of $V$ is von Neumann bounded.
\end{proposition}
TODO: holds in more generality for countably pseudotopological spaces.
\begin{proof}
Let $K\subseteq V$ be a compact set. We need to show that $\neighbourhood_\F(0)\cdot K$ converges to $0$. Since $\xi$ is pseudotopological, it is enough to show that any larger ultrafilter converges to $0$. Take such an ultrafilter $G\supseteq \neighbourhood_\F(0)\cdot K$. Now $\cball(0,\epsilon)\cdot K = \cdot^{\imf}\big(\cball(0,1)\times K\big)$ is compact for all $\epsilon > 0$ by Tychonoff's theorem \ref{TychonoffsTheorem} and \ref{compactConstructions}.

Now $G$ contains $\cball(0,\epsilon)\cdot K$ for all $\epsilon > 0$, so $G$ must converge and its (unique) limit $x$ must lie in each $\cball(0,\epsilon)\cdot K$. Thus $x\in \bigcap_{\epsilon>0}\cball(0,\epsilon)\cdot K \eqdef K_0$.

By \ref{HausdorffCompactIntersection} $K_0$ is compact. Now 
for all $\lambda\in \F$, we have $x\in \cball(0,\epsilon)\cdot K$ iff $\lambda x \in\cball(0,|\lambda|^{-1}\epsilon)\cdot K$. As $\bigcap_{\epsilon>0}\cball(0,\epsilon)\cdot K = \bigcap_{\epsilon>0}\cball(0,|\lambda|^{-1}\epsilon)\cdot K$, we have $\lambda x\in K_0$. This means that $\F\cdot \{x\} \subseteq K_0$. Thus $K_0$ can only be compact if $x=0$.
\end{proof}

\begin{lemma} \label{vonNeumannBoundedImpliesEquicontinuous}
Let $\sSet{V,\xi}, \sSet{W,\zeta}$ be convergence vector spaces and $H\subseteq \contLin_c(V, W)$. If $V$ is equable and $H$ is von Neumann bounded, then $H$ is equicontinuous.
\end{lemma}
\begin{proof}
Take arbitrary $F\overset{\xi}{\longrightarrow} 0$. Then there exists an equable filter $G\subseteq F$ which also converges to $0$. Then we have
\[ \upset \evalMap^{\imf\imf}(\{H\}\otimes F) \supseteq \upset \evalMap^{\imf\imf}(\{H\}\otimes G) = \upset \evalMap^{\imf\imf}(\{H\}\otimes \neighbourhood_\F(0)\cdot G) = \upset \evalMap^{\imf\imf}(\neighbourhood_\F(0)\cdot \{H\}\otimes G), \]
which converges to $0$ because $H$ is von Neumann bounded, so $\neighbourhood_\F(0)\cdot \{H\}$ continuously converges to $0$.
\end{proof}

\begin{proposition}
Let $\sSet{V,\xi}$ be a convergence vector space and $H\subseteq \contLin_c(V, \F)$. Then
\begin{enumerate}
\item if $H$ is relatively compact, then $H$ is von Neumann bounded;
\item if $V$ is equable, then the converse also holds.
\end{enumerate}
\end{proposition}
\begin{proof}
(1) The space $\contLin_c(V, \F)$ is Hausdorff and pseudotopological by \ref{continuousConvergencePropertiesFromCodomain}. As $\adh(H)$ is compact, it is also von Neumann bounded, by \ref{compactSubsetsVonNeumannBounded}. Thus $H$ is von Neumann bounded by \ref{vonNeumannBoundedSetLemma}.

(2) Assume $H$ is von Neumann bounded. Then $H$ is equicontinuous by \ref{vonNeumannBoundedImpliesEquicontinuous} and thus evenly continuous by \ref{equicontinuityEvenContinuity}. For all $x\in X$, $\evalMap_x^\imf(H)$ is von Neumann bounded by \ref{continuousMappingBoundedSets}, and thus relatively compact. By \ref{evenContinuityRelativeCompactness} $H$ is relatively compact.
\end{proof}

\begin{proposition}
Let $\sSet{V,\xi}$ be a topological vector space and $U\in\neighbourhood_\xi(0)$.
\begin{enumerate}
\item if $U$ is bounded, then $\xi$ is the initial topology w.r.t. some absolutely homogeneous function $f: V\to \R$;
\item if $U$ is bounded and convex, then $\xi$ is the initial topology w.r.t. some seminorm $f$.
\item if $U$ is bounded and convex and $\xi$ is Hausdorff, then $\xi$ is normable.
\end{enumerate}
\end{proposition}
TODO: (1) implies pseudometrisable? Narici/Beckenstein.
This proposition implies that the balls of a metrisable, non-normable LCTVS are not bounded sets.
\begin{proof}
(1) Suppose $U$ bounded. Then $\balanced(U)$ is bounded by \ref{vonNeumannBoundedSetLemma}, so we may take $U$ balanced WLOG. Then $\{\epsilon U\}_{\epsilon >0}$ forms a base of $\neighbourhood_\xi(0)$ by \ref{boundedSetVicinityBase}. Consider the gauge $p_U$, which is absolutely continuous by \ref{gaugeProperties}. By \ref{gaugeClassificationLemma}, each $\epsilon U$ is contained in a preimage of $p_U$, so the initial topology w.r.t. $p_U$ is stronger than $\xi$. We just need to show that $p_U$ is continuous, which immediately follows from \ref{boundedOnVicinityImpliesContinuous} and \ref{gaugeClassificationLemma}.If $p_U(v) = 0$ for some 

(2) In this case $p_U$ is a seminorm by \ref{gaugeProperties}.

(3) TODO.
\end{proof}

\begin{proposition}
Let $\sSet{V, \xi}$ be a Hausdorff LCTVS. If $V$ has a totally bounded neighbourhood, then $V$ is finite dimensional.
\end{proposition}
\begin{proof}
Robertson/Robertson p. 50.
\end{proof}


\subsection{Local von Neumann boundedness}
\begin{definition}
Let $\sSet{V,\xi}$ be a convergence vector space. Then $V$ is \udef{locally von Neumann bounded} if the set of von Neumann bounded sets forms a convergence cover of $V$.
\end{definition}

\begin{proposition}
Let $\sSet{V,\xi}$ be a locally von Neumann bounded convergence vector space of finite depth and $K\subseteq V$ a compact subset. Then $K$ is von Neumann bounded.
\end{proposition}
\begin{proof}
By \ref{compactFiniteSubcover} the convergence cover of von Neumann bounded subsets has a finite subset that covers $K$. This finite union of von Neumann bounded sets is von Neumann bounded by \ref{vonNeumannBoundedSetLemma} and thus the subset $K$ is von Neumann bounded, also by \ref{vonNeumannBoundedSetLemma}.
\end{proof}

\subsection{The Mackey modification}
\begin{definition}
Let $\sSet{V,\xi}$ be a convergence vector space. The convergence associated to the von Neumann bornology is called the \udef{Mackey modification} $\mackeyMod(\xi)$ of $\xi$.

We call $\sSet{V, \xi}$ a \udef{Mackey space} if $\xi = \mackeyMod(\xi)$.
\end{definition}

\begin{lemma} \label{MackeyModPreservesVNBoudnedSets}
Let $\sSet{V,\xi}$ be a convergence vector space. A subset $B\subseteq V$ is $\xi$-von Neumann bounded \textup{if and only if} $B$ is $\mackeyMod(\xi)$-von Neumann bounded.
\end{lemma}
Thus the von Neumann bornology of the Mackey modification $\mackeyMod(\xi)$ is the same as the von Neumann bornology of $\xi$.
\begin{proof}
If $B$ is $\xi$-von Neumann bounded, then $\neighbourhood_\F(0)\cdot\{B\} \subseteq \neighbourhood_\F(0)\cdot\{B\}$, so
\[ \neighbourhood_\F(0)\cdot\{B\} \overset{\mackeyMod(\xi)}{\longrightarrow} 0, \]
which means that $B$ is $\mackeyMod(\xi)$-von Neumann bounded.

Now assume $B$ is $\mackeyMod(\xi)$-von Neumann bounded. Then $\neighbourhood_\F(0)\cdot\{B'\} \subseteq \neighbourhood_\F(0)\cdot\{B\}$ for some $\xi$-von Neumann bounded $B'\subseteq V$. As $\neighbourhood_\F(0)\cdot\{B'\}$ $\xi$-converges to $0$, $\neighbourhood_\F(0)\cdot\{B\}$ must do so as well, which means that $B$ is $\xi$-von Neumann bounded.
\end{proof}
\begin{corollary} \label{MackeyModLocallyBounded}
Let $\sSet{V,\xi}$ be a convergence vector space. Then $\mackeyMod(\xi)$ is locally bounded.
\end{corollary}

\begin{lemma}
The Mackey modificication $\mackeyMod$ is a dual closure function.
\end{lemma}
\begin{proof}
The larger the convergence, then more von Neumann bounded sets and thus the larger the Mackey modification. Thus $\mackeyMod$ is monotone.

If a filter converges in the Mackey modification, it converges in the original convergence by definition of von Neumann boundedness, so $\mackeyMod(\xi)\subseteq \xi$ for all vector space convergences $\xi$.

The function $\mackeyMod$ is idempotent by \ref{MackeyModPreservesVNBoudnedSets}.
\end{proof}

\begin{proposition}
Let $\sSet{V,\xi}, \sSet{W,\zeta}$ be vector convergence spaces and $f:V\to W$ a linear function. Then the following are equivalent:
\begin{enumerate}
\item $f: \sSet{V,\vNborn{\xi}}\to \sSet{W,\vNborn{\zeta}}$ is bounded;
\item $f: \sSet{V,\mackeyMod(\xi)}\to \sSet{W,\mackeyMod(\zeta)}$ is continuous;
\item $f: \sSet{V,\mackeyMod(\xi)}\to \sSet{W,\zeta}$ is continuous.
\end{enumerate}
\end{proposition}
\begin{proof}
$(1) \Rightarrow (2)$ Let $F\overset{\mackeyMod(\xi)}{\longrightarrow} 0$. Then there exists $B\in \vNborn{\xi}$ such that $\neighbourhood_\F(0)\cdot \{B\} \subseteq F$. We have 
\[ \upset f^{\imf\imf}(F) \supseteq \upset f^{\imf\imf}\big(\neighbourhood_\F(0)\cdot \{B\}\big) = \neighbourhood_\F(0)\cdot \{f^\imf(B)\}. \]
As $f^\imf(B)$ is bounded, $f^{\imf\imf}(F)$ converges in $\mackeyMod(\zeta)$.

$(2) \Rightarrow (3)$ Immediate because $\mackeyMod(\zeta)\subseteq \zeta$.

$(3) \Rightarrow (1)$ Take arbitrary $B\in \vNborn{\xi}$, so $\neighbourhood_\F(0)\cdot \{B\}\overset{\mackeyMod(\xi)}{\longrightarrow} 0$. Then
\[ \neighbourhood_\F(0)\cdot \{f^\imf(B)\} = \upset f^{\imf\imf}\big(\neighbourhood_\F(0)\cdot \{B\}\big) \overset{\zeta}{\longrightarrow} 0, \]
so $f^\imf(B)\in\vNborn{\zeta}$.
\end{proof}

\subsubsection{Mackey spaces}
\begin{proposition}
Let $\sSet{V,\xi}$ be a vector convergence space. Then $V$ is a Mackey space \textup{if and only if} it is locally bounded and equable.
\end{proposition}
\begin{proof}
A Mackey space is clearly equable. It is locally bounded by \ref{MackeyModLocallyBounded}.

Now let $\sSet{V,\xi}$ be a locally bounded and equable vector convergence space. Since $\mackeyMod(\xi)\leq \xi$, it is enough to prove the converse inequality. Assume $F\overset{\xi}{\longrightarrow} 0$. Then we can find an equable $G\subseteq F$ that also converges to $0$. Choose a bounded set $B\in G$, which means that $\upset \{B\}\subseteq G$. Then $\neighbourhood_\F(0)\cdot \{B\} \subseteq \neighbourhood_\F(0)\cdot G = G \subseteq F$, so $F$ converges to $0$ in $\mackeyMod(\xi)$.
\end{proof}


\section{Banach-Steinhaus pairs}
\begin{definition}
Let $\sSet{V,\xi}, \sSet{W,\zeta}$ be convergence vector spaces. The pair $(V,W)$ is called a \udef{Banach-Steinhaus pair} if every von Neumann bounded subset of $\contLin_p(V,W)$ is equicontinuous.
\end{definition}
TODO is von Neumann the right bornology?

\begin{lemma} \label{BanachSteinhausPairDomainInclusion}
Let $\sSet{V,\xi}, \sSet{W,\zeta}$ be convergence vector spaces and $\xi'$ a vector space convergence on $V$ such that $\xi'\leq \xi$. If $\big(\sSet{V,\xi}, \sSet{W,\zeta}\big)$ is a Banach-Steinhaus pair, then so is $\big(\sSet{V,\xi'}, \sSet{W,\zeta}\big)$.
\end{lemma}
\begin{proof}
The pointwise convergence on $\contLin_p(V,W)$ does not depend on the convergence on $V$. Thus the bounded subsets of $\contLin_p(V,W)$ are the same for both convergences. The larger the convergence on $V$, the fewer equicontinuous sets, which implies the lemma.
\end{proof}

\begin{proposition}
Let $\sSet{V,\xi}, \sSet{W,\zeta}$ be convergence vector spaces. Then $\big(\sSet{V,\xi}, \sSet{W,\zeta}\big)$ is a Banach-Steinhaus pair \textup{if and only if} $\big(\sSet{V,\lconvMod(\xi)}, \sSet{W,\zeta}\big)$ is a Banach-Steinhaus pair.
\end{proposition}
\begin{proof}
The pointwise convergence on $\contLin_p(V,W)$ does not depend on the convergence on $V$, so $\contLin_p(V,W)$ and $\contLin_p\big(\lconvMod(V),W\big)$ have the same bouded sets. They also have the same equicontinuous sets by \ref{equicontinuousSetsLconvMod}.
\end{proof}

\subsection{Barrelled spaces}
\begin{definition}
Let $\sSet{V,\xi}$ be a convergence vector space. Then $V$ is called \udef{barrelled} if every bounded subset of $\dual{V}_p$ is equicontinuous.
\end{definition}

\begin{theorem}
BB 7.2.17
\end{theorem}
\begin{corollary}[Banach-Steinhaus]
Let $\sSet{V,\xi}$ be a barrelled convergence vector space and $\sSet{W,\zeta}$ an LCTVS. Then $(V,W)$ is a Banach-Steinhaus pair.
\end{corollary}

\subsection{Quasi-completeness}
\begin{definition}
Let $\sSet{V,\xi}$ be a convergence vector space. Then $V$ is called \udef{quasi-complete} if every Cauchy filter that contains a vN bounded set is convergent.
\end{definition}

\begin{proposition} \label{quasiCompleteImpliesSequentiallyComplete}
Let $\sSet{V,\xi}$ be a quasi-complete topological convergence vector space. Then $V$ is sequentially complete.
\end{proposition}
\begin{proof}
Let $\seq{v_n}\in V^\N$ be a Cauchy sequence. Then $\{v_n\}_{n\in\N}$ is totally bounded by \ref{imageCauchySequenceTotallyBounded} and thus vN bounded by \ref{boundednessTVS}. Finally $\seq{v_n}$ is convergent by quasi-completeness.
\end{proof}

\begin{proposition} \label{quasiCompletenessFunctionSpaces}
Let $\sSet{V,\xi}, \sSet{W,\zeta}$ be convergence vector spaces. Such that
\begin{itemize}
\item $(V,W)$ is a Banach-Steinhaus pair;
\item $W$ is quasi-complete, regular and Hausdorff.
\end{itemize}
Then $\contLin_p(V,W)$ is quasi-complete.
\end{proposition}
\begin{proof}
Let $F\in \powerfilters\big(\contLin_p(V,W)\big)$ be a Cauchy filter that contains a vN bounded set. For all $v\in V$, $\evalMap_{v}: \contLin_p(V,W) \to W$ is uniformly continuous, by \ref{uniformContinuityGroupHomomorphism}. Thus $\upset \evalMap_v^{\imf\imf}(F)$ is Cauchy and contains a vN bounded set by \ref{continuousMappingBoundedSets}. Since $W$ was assumed quasi-complete, $\upset \evalMap_v^{\imf\imf}(F)$ converges to some $y_x\in W$.

Consider the function $T: x\mapsto y_x$. Then $F\to T$ in $(V\to W)_p$. Since $(V,W)$ is a Banach-Steinhaus pair, $F$ contains an equicontinuous set. This set is evenly continuous by \ref{equicontinuityEvenContinuity} and thus $F\to T$ in $(V\to W)_c$. Then $F\in \contLin(V,W)$ by \ref{linearFunctionsClosedSubset}.
\end{proof}

\chapter{General duality theory}
\section{Paired spaces}
\begin{definition}
A \udef{pairing} is a triple $\sSet{V,W, \pair{\cdot,\cdot}}$ where $V,W$ are vector spaces over $\mathbb{F}$ and $\pair{\cdot,\cdot}: V\times W\to \mathbb{F}$ is a bilinear form. Often we will write the pairing as just $\sSet{V,W}$.

We say $W$
\begin{itemize}
\item \udef{distinguishes} points of $V$; or
\item is \udef{separating} on $V$; or
\item \udef{separates} $V$
\end{itemize}
if 
\[ \forall v\in V\setminus\{0\}: \exists w\in W: \pair{v,w} \neq 0. \]

A \udef{dual system}, \udef{dual pair} or \udef{duality} over a field $\mathbb{F}$ is a pairing $\sSet{V,W,\pair{\cdot,\cdot}}$ such that $V$ distinguishes points of $W$ and $W$ distinguishes points of $V$.
\end{definition}

\begin{lemma} \label{dualSystemInjective}
Let $\sSet{V,W, \pair{\cdot,\cdot}}$ be a pairing. Then $W$ separates $V$ \textup{if and only if} $v\mapsto \pair{v, \cdot}$ is injective.
\end{lemma}
\begin{proof}
Suppose $W$ separates $V$ and take arbitrary $v,v'\in V$ such that $\pair{v, \cdot} = \pair{v', \cdot}$. Then $\pair{v-v',\cdot} = \underline{0}$, which by assumption means $v-v' = 0$, or $v= v'$.

Now suppose $v\mapsto \pair{v, \cdot}$ is injective and take arbitrary $v\in V\setminus\{0\}$. If $\pair{v,w} = 0$ for all $w\in W$, then $\pair{v,\cdot} = \pair{0,\cdot}$ and thus $v=0$ by injectivity. This was disallowed by assumption.
\end{proof}

\begin{lemma}
Let $\sSet{V,W,\pair{\cdot,\cdot}}$ be a dual pair. Then $\sSet{W,V,\pair{\cdot,\cdot}^d}$ is also a dual pair.
\end{lemma}

\begin{example}
Let $\sSet{V, \xi}$ be a convergence vector space. Then $\sSet{\dual{V}, V, \evalMap(\cdot,\cdot)}$ is a pairing.
\end{example}

\begin{proposition} \label{HausdorffLCTVSdualSystem}
Let $\sSet{V,\xi}$ be a Hausdorff locally convex convergence vector space. Then $\sSet{\dual{V}, V, \evalMap(\cdot,\cdot)}$ is a dual pair.
\end{proposition}
\begin{proof}
It is immediate that $V$ separates $\dual{V}$. We have that $\dual{V}$ separates $V$ by \ref{locallyConvexDualPair}
\end{proof}

\section{The weak topology}
\begin{definition}
Let $(X,Y,\pair{\cdot,\cdot})$ be paired vector spaces. 
The initial convergence on $Y$ w.r.t. the set of linear functionals $\setbuilder{\pair{x,\cdot}}{x\in X}$ is called the \udef{weak topology} $\sigma(X,Y)$ on $Y$ for the pair $\sSet{X,Y}$\footnote{What we denote $\sigma(X,Y)$ is usually denoted $\sigma(Y,X)$.}.

The \udef{weak-$*$ topology} $\sigma^*(X,Y)$ on $X$ is the weak topology $\sigma(Y,X)$ (i.e. the weak topology for the pairing $(Y,X,\pair{\cdot,\cdot}^d)$).
\end{definition}

\begin{lemma} \label{weakTopologyLCTVS}
Let $\sSet{X,Y,\pair{\cdot,\cdot}}$ be a pairing. The weak topology $\sigma(X,Y)$ on $Y$ 
\begin{enumerate}
\item is the same as the initial convergence w.r.t. the set of seminorms $\setbuilder{\abspair{x,\cdot}}{x\in X}$;
\item is a locally convex vector space topology;
\item is Hausdorff \textup{if and only if} $X$ separates $Y$;
\item has the neighbourhood filter
\[ \neighbourhood_{\sigma(X,Y)}(0) = \mathfrak{F}\setbuilder{y\in Y}{\exists x\in X: \abspair{x,y} \leq 1}. \]
\end{enumerate}
\end{lemma}
\begin{proof}
(1, 2) Let $\sigma_1$ be the initial convergence w.r.t. $\setbuilder{\pair{x,\cdot}}{x\in X}$ and $\sigma_2$ the initial convergence w.r.t. $\setbuilder{\abspair{x,\cdot}}{x\in X}$.

By \ref{continuityAbsFunctional}, all functionals of the form $\abspair{x,\cdot}: \sSet{V,\sigma_1} \to \R$, for some $x\in X$, are continuous. Thus $\sigma_1 \subseteq \sigma_2$.

By \ref{locallyConvexSeminormTopology}, $\sigma_2$ is a locally convex vector space topology. Thus all functionals of the form $\pair{x,\cdot}: \sSet{V,\sigma_2} \to \R$, for some $x\in X$, are continuous. This means that $\sigma_2 \subseteq \sigma_1$.

(3) If $X$ separates $Y$, then the weak topology is Hausdroff by \ref{T2initialConvergence}. If $X$ does not separate $Y$, then there exists $y\in Y\setminus\{0\}$ such that $\pair{x,y} = 0$ for all $x\in X$. Then $\pair{x,-}^{\imf\imf}(\pfilter{y}) = \pfilter{0} \to 0 = \pair{x,0}$ for all $x\in X$ and so $\pfilter{y}$ converges weakly to both $y$ and $0$. Since these are different point, the convergence is not Hausdorff.

(4) The form of the neighbourhood filter also follows from \ref{locallyConvexSeminormTopology}. It is enough to show that $\setbuilder{\abspair{x,\cdot}^{\preimf}(\,[0,\epsilon]\,)}{x\in X, \epsilon > 0} = \setbuilder{y\in Y}{\exists x\in X: \abspair{x,y} \leq 1}$. Then
\begin{align*}
y \in \setbuilder{\abspair{x,\cdot}^{\preimf}(\,[0,\epsilon]\,)}{x\in X, \epsilon > 0} &\iff \exists x\in X: \exists \epsilon > 0: \; \abspair{x,y} \leq \epsilon \\
&\iff \exists x\in X: \exists \epsilon > 0: \; \epsilon^{-1}\abspair{x,y} \leq 1 \\
&\iff \exists x\in X: \exists \epsilon > 0: \; \abspair{\epsilon^{-1}x,y} \leq 1 \\
&\iff \exists x\in X: \; \abspair{x,y} \leq 1 \\
&\iff y\in \setbuilder{y\in Y}{\exists x\in X: \abspair{x,y} \leq 1}.
\end{align*}
\end{proof}

\begin{lemma}
Let $(X,Y,\pair{\cdot,\cdot})$ be a pairing and $x\in X$. Then $\abspair{x,\cdot} = p_{^\pol\{x\}}$.
\end{lemma}

\begin{proposition} \label{functionalContinuityWeakTopology}
Let $\sSet{X,Y,\pair{\cdot,\cdot}}$ be a pairing and $f: Y\to \F$ a linear functional. Then
\[ \dual{(Y,\sigma(X,Y))} = \setbuilder{\pair{x,\cdot}}{x\in X}. \]
\end{proposition}
\begin{proof}
This inclusion $\supseteq$ is immediate by the definition of initial topology.

Now take $f\in \dual{(Y,\sigma(X,Y))}$. By \ref{dualSeminormedConvergence}, there exists a finite set $\{x_1, \ldots, x_n\}$ such that $|f(v)| \leq C\max_{i}\abspair{x_i, \cdot}$ for all $v\in V$. By \ref{linearDependenceLinearFunctionals}, this means that $f$ is some linear combination of the $\pair{x_i,\cdot}$, say $\sum_{i=0}^n \alpha_i \pair{x_i, \cdot}$. So
\[ f = \sum_{i=0}^n \alpha_i \pair{x_i, \cdot} = \pair{\sum_{i=0}^n \alpha_i x_i, \cdot}. \]
\end{proof}
\begin{corollary} \label{dualSystemBijection}
Let $\sSet{X,Y,\pair{\cdot,\cdot}}$ be a pairing such that $Y$ separates $X$. Then
\begin{enumerate}
\item $X\to \sSet{Y, \sigma}^*: x\mapsto \pair{x,\cdot}$ is a bijection that preserves the pairing;
\item it is a homeomorphism if $X$ has the weak-$*$ topology and $(Y,\sigma)^*$ the pointwise topology.
\end{enumerate} 
\end{corollary}
In particular this holds if $\sSet{X,Y,\pair{\cdot,\cdot}}$ is a dual system. We then have
\[ \sSet{X,\sigma^*(X,Y)} \cong \dual{\sSet{Y,\sigma(X,Y)}}. \]
\begin{proof}
(1) Injectivity is given by \ref{dualSystemInjective}. Surjectivity by the proposition.

(2) Take arbitrary $y\in Y$. We have, for all $y\in Y$,
\[ \pair{\cdot, y}(x) = \pair{x,y} = \evalMap_y\big(\pair{x, \cdot}\big), \]
so $\pair{\cdot, y} = \evalMap_y \circ (x\mapsto \pair{x,\cdot})$ and $\evalMap_y = \pair{\cdot, y} \circ (x\mapsto \pair{x,\cdot})^{-1}$. Thus both $(x\mapsto \pair{x,\cdot})$ and its inverse are continuous by \ref{characteristicPropertyInitialFinalConvergence}.
\end{proof}
\begin{corollary} \label{weakestConvergenceOfTheDualPair}
Let $V$ be a convergence vector space. Then $V^* = \sSet{V, \sigma(V^*, V)}^*$.

The weak topology is the weakest convergence $\tau$ such that $V^* = \sSet{V, \tau}^*$.
\end{corollary}
Thus for any convergence vector space $\sSet{V,\xi}$, the space $\sSet{V, \sigma(V^*, V)}$ is a locally convex TVS with the same continuous functionals.

\subsection{The pairing $\sSet{V^*, V, \evalMap}$}
\subsubsection{Convergences of the dual pair}
\begin{definition}
Let $V$ be a vector space and $V^*$ the continuous dual under some convergence. Any convergence on $V$ such that the continuous dual is still $V^*$ is called a \udef{convergence of the dual pair}.
\end{definition}
Any property that only depends on continuous linear functionals is the same for any convergence of the dual pair.

\begin{proposition}
The weak topology is the weakest convergence of the dual pair.
\end{proposition}
\begin{proof}
Reformulation of \ref{weakestConvergenceOfTheDualPair}.
\end{proof}

\subsubsection{Weak continuity}
\begin{definition}
Let $T: \sSet{V,\xi} \to \sSet{W,\zeta}$ be a linear operator between convergence vector spaces. Then $T$ is called \udef{weakly continuous} if
\[ T: \sSet{V, \sigma(V^*,V)} \to \sSet{W, \sigma(W^*,W)} \]
is continuous.
\end{definition}

\begin{lemma} \label{continuityImpliesWeakContinuity}
If $T: V\to W$ is continuous, then it is weakly continuous.
\end{lemma}
\begin{proof}
By the characteristic property of the initial convergence \ref{characteristicPropertyInitialFinalConvergence}, the weak continuity of $T$ is equivalent to the continuity of $f\circ T$ for all $f\in W^*$. This holds because the composition of continuous functions is continuous, \ref{continuityComposition}.
\end{proof}

\subsubsection{Properties of subsets}

\begin{proposition} \label{weakClosureConvexSubsets}
Let $V$ be a vector space and $A\subseteq V$ a convex subset. Let $\xi$ be a locally convex vector space convergence on $V$. Then $\closure_\xi(A) = \closure_{\sigma(V^*, V)}(A)$.
\end{proposition}
In other words the closure of convex sets is the same for all locally convex convergences of the dual pair.
\begin{proof}
From \ref{principalInherenceAdherenceProperties}, we have $\closure_\xi(A) \subseteq \closure_{\sigma(V^*, V)}(A)$.

Now assume $x\notin \closure_\xi(A)$. Then, by \ref{locallyConvexHahnBanachSeparationClosedSet}, we can find a continuous functional $f\in V^*$ such that $f(x) \notin \overline{f^{\imf}\big(\closure_\xi(A)\big)}$. So not every neighbourhood of $f(x)$ meshes with $f^{\imf}\big(\closure_\xi(A)\big)$; take some neighbourhood $U$ that is disjoint from it. By continuity and \ref{continuityVicinityFilter}, we have that $f^{\preimf}(U)$ is a $\sigma(V^*, V)$-neighbourhood of $x$ that must be disjoint from $\closure_\xi(A)$ (we have used that $\sigma(V^*, V)$ is topological, \ref{weakTopologyLCTVS}). This implies that $A$ is also disjoint from $f^{\preimf}(U)$.

Thus $x\notin \closure_{\sigma(V^*, V)}(A)$ by \ref{interiorClosureMembership}.
\end{proof}

\begin{proposition} \label{weaklyBoundedIffBounded}
Let $\sSet{V,\xi}$ be a LCTVS and $A\subseteq V$. Then $A$ is bounded \textup{if and only if} $A$ is weakly bounded.
\end{proposition}
\begin{proof}
As both $\xi$ and $\sigma(\dual{V}, V)$ are LCTVSs, boundedness is equivalent to von Neumann boundedness, by \ref{boundednessTVS}. Now $\xi$ is the initial convergence w.r.t. all continuous seminorms on $V$ (by \ref{locallyConvexSeminormTopology}) and $\sigma(\dual{V}, V)$ is the initial convergence w.r.t. all continuous linear functionals on $V$.

Now consider \ref{vonNeumannBoundednessInitialSpace}. We need to show that $f^\imf(A)$ is bounded for all continuous linear functionals $f$ iff $p^\imf(A)$ is bounded for all seminorms $p$. The direction $\Leftarrow$ is immediate because $f$ is both continuous and bounded on $A$ iff the seminorm $|f|$ is (see \ref{continuityAbsFunctional}).

Now suppose $f^\imf(A)$ is bounded for all continuous linear functionals $f$. 

TODO use uniform boundedness, Voigt p 45.
\end{proof}


\subsection{Weak boundedness and completeness}
TODO: emphasise that weakly bounded = weakly von Neumann bounded.


\begin{proposition} \label{weakBoundedness}
Let $\sSet{X,Y,\pair{\cdot,\cdot}}$ be a pairing such that $Y$ separates $X$. A subset $B\subseteq Y$ is weakly bounded \textup{if and only if} $B$ is weakly totally bounded.
\end{proposition}
\begin{proof}
The direction $\Leftarrow$ is immediate.

For the other direction, take arbitrary $x\in X$. Then $\pair{x,\cdot}: \sSet{Y,\sigma}\to \F$ is a continuous linear functional and thus uniformly continuous by \ref{uniformContinuityGroupHomomorphism}. 

By \ref{imageBoundedSet}, we have that $\pair{x, B}$ is bounded and thus $\overline{\pair{x,B}}$ is also bounded by \ref{adherenceBoundedSet}. As $\F$ has the Heine-Borel property (TODO ref), $\overline{\pair{x,B}}$ is in fact compact.

Now $\prod_{x\in X}\overline{\pair{x, B}}$ is compact by Tychonoff's theorem \ref{TychonoffsTheorem}, and in particular totally bounded by \ref{precompactTotallyBounded} and \ref{compactPrecompactComplete}.

By \ref{dualSystemBijection} and \ref{pointwiseConvergenceProductSpace}, we have
\[ Y \cong \sSet{X,\sigma^*}^* \hookrightarrow (X\to \F)_p \cong \prod_{x\in X}\F, \]
so we have an embedding $f: Y \hookrightarrow \prod_{x\in X}\F$, with respect to which $Y$ has the initial topology (TODO ref). We restrict $f$ to its range, such that it is a bijection and thus a homeomorphism by \ref{initialBijectionHomeomorphism}. As the function $f$ is linear, it is uniformly continuous by \ref{uniformContinuityGroupHomomorphism}.
We have
\[ B = (f^{-1})^\imf\circ f^\imf(B) \subseteq (f^{-1})^\imf\Big(\prod_{x\in X}\overline{\pair{x, B}}\Big), \]
and thus $B$ is totally bounded by \ref{imageBoundedSet} and \ref{boundedSetsIdeal}.
\end{proof}

\section{Polars}
\subsection{Polar sets}
\begin{definition}
Let $\sSet{X,Y,\pair{\cdot,\cdot}}$ be a pairing and $B\subseteq Y$ a subset. The \udef{polar} of $B$ is the polar w.r.t. the relation $\pol$ on $(Y,X)$ defined by
\[ y\pol x \qquad\iff\qquad \abspair{x, y} \leq 1. \]
Conventionally, we also use $\pol$ to denote $\pol^\transp$. Thus the \udef{bipolar} $B^{\pol\pol}$ of $B$ is $B^{\pol\pol^\transp}$.
\end{definition}

\begin{lemma} \label{polarLemma}
Let $\sSet{X,Y,\pair{\cdot,\cdot}}$ be a pairing and $B\subseteq Y$ a subset. Then
\begin{align*}
B^{\pol} &= \setbuilder{x\in X}{\sup_{y\in B}\abspair{x,y} \leq 1} \\
&= \bigcap_{y\in B}\setbuilder{x\in X}{\abspair{x,y}\leq 1}.
\end{align*}
and for the gauge of $B^\pol$, we have $p_{B^\pol}(x) = \sup_{y\in B}\abspair{x,y}$.
\end{lemma}
\begin{proof}
The expression for the polar is straightforward. For the gauge we calculate
\begin{align*}
p_{B^\pol}(x) &= \inf \setbuilder{\lambda\in\overline{\R^+}}{x\in \lambda B^\pol} \\
&= \inf \setbuilder{\lambda\in\overline{\R^+}}{\lambda^{-1}x\in B^\pol} \\
&= \inf \setbuilder{\lambda\in\overline{\R^+}}{\sup_{y\in B}\abspair{\lambda^{-1}x,y}\leq 1} \\
&= \inf \setbuilder{\lambda\in\overline{\R^+}}{\sup_{y\in B}\abspair{x,y}\leq \lambda} \\
&= \sup_{y\in B}\abspair{x,y}.
\end{align*}
\end{proof}

\begin{lemma} \label{polarPropertiesLemma}
Let $\sSet{X,Y,\pair{\cdot,\cdot}}$ be a pairing and $B\subseteq Y$ a subset. Then
\begin{enumerate}
\item for all $\lambda \neq 0$: $(\lambda B)^{\pol} = \lambda^{-1}B^{\pol}$;
\item $B^{\pol}$ is absolutely convex;
\item $B^{\pol}$ is $\sigma^*(X,Y)$-closed.
\end{enumerate}
\end{lemma}
\begin{proof}
(1) We calculate
\begin{align*}
x\in (\lambda B)^{\pol} &\iff \forall y\in B:\; \abspair{x,\lambda y} \leq 1 \\
&\iff \forall y\in B:\; \abspair{\lambda x,y} \leq 1 \\
&\iff \lambda x\in B^\pol \\
&\iff x\in \lambda^{-1}B^\pol.
\end{align*}

(2) Take $x,y\in B^\pol$ and $r,r'\in \R$ such that $|r| + |r'|\leq 1$. We need to show that $rx + r'y\in B^\pol$. Indeed, for arbitrary $z\in B$ we have
\begin{align*}
\abspair{rx + r'y, z} &= |r\pair{x,z} + r'\pair{y,z}| \\
&\leq |r|\;\abspair{x,z} + |r'|\;\abspair{y,z} \\
&\leq |r| + |r'| \\
&\leq 1,
\end{align*}
where the last inequality follows from $1 = |1 - r + r| \leq |1-r| + |r|$. Since $z\in B$ was chosen arbitrarily, we have $rx + (1-r)y\in B^\pol$.

(3) We have, using \ref{polarOfUnion},
\begin{align*}
B^\pol &= \left(\bigcup_{y\in B}\{y\}\right)^\pol \\
&= \bigcap_{y\in B} \{y\}^\pol \\
&= \bigcap_{y\in B}\setbuilder{x\in X}{\abspair{x,y}\leq 1} \\
&= \bigcap{y\in B}\abspair{\cdot, y}^{\preimf}[\cball(0,1)],
\end{align*}
which is an intersection of closed sets (as each $\abspair{\cdot, y}$ is continuous in the $\sigma(X,Y)$ topology) and thus closed.
\end{proof}

\begin{proposition}[Bipolar theorem] \label{bipolarTheorem}
Let $\sSet{X,Y,\pair{\cdot,\cdot}}$ be a pairing and $B\subseteq Y$ a subset. Then
\[ B^{\pol\pol} = \overline{\disked(B)}^{\sigma(X,Y)}. \]
\end{proposition}
TODO: clean up proof.
\begin{proof}
We have $B\subseteq B^{\pol\pol}$ by \ref{reflexiveGaloisCorollary}. Then $\disked(B) \subseteq B^{\pol\pol}$ because $B^{\pol\pol}$ is absolutely convex by \ref{polarPropertiesLemma}. Similarly $\overline{\disked(B)}^{\sigma(X,Y)} \subseteq B^{\pol\pol}$, because $B^{\pol\pol}$ is $\sigma(X,Y)$-closed.

The other inclusion is proved by contradiction. Assume, to this end, that $y_0\in B^{\pol\pol} \setminus \overline{\disked(B)}^{\sigma(X,Y)}$. Since $\overline{\disked(B)}^{\sigma(X,Y)}$ is closed and convex by \ref{inherenceAdherenceConvex}, we can apply Hahn-Banach separation \ref{locallyConvexHahnBanachSeparationClosedSet} to obtain a continuous functional $f$ such that $f^\imf\left[\overline{\disked(B)}\right]$ is disjoint from some open neighbourhood $U$ of $f(y_0)$.

Now $\overline{\disked(B)}$ is absolutely convex by \ref{inherenceAdherenceBalanced} and \ref{inherenceAdherenceConvex}, so $f^\imf\left[\overline{\disked(B)}\right]$ is also absolutely convex by linearity. Then, because $|U|$ is open, there exists $t\in |U|$ such that $|f(b)| < t < |f(y_0)|$ for all $b\in \overline{\disked(B)}$. Thus $\sup_{b\in \overline{\disked(B)}}|f(b)| < |f(y_0)|$. By rescaling $f$ we can take $\sup_{b\in \overline{\disked(B)}}|f(b)| \leq 1 < |f(y_0)|$.

By \ref{functionalContinuityWeakTopology}, we can find some $x\in X$ such that $f = \pair{x,\cdot}$. Then $\sup_{b\in \overline{\disked(B)}}\abspair{x, b} \leq 1$ implies $x\in \overline{\disked(B)}^\pol \subseteq B^\pol$. Finally $1 < \abspair{x, y_0}$ implies $y_0\notin B^{\pol\pol}$. This is a contradiction.
\end{proof}

\subsection{The pairing $\sSet{\Lin(V,\F), V, \evalMap}$}
\begin{proposition} \label{dualFromPolars}
Let $\sSet{V,\xi}$ be a topological vector space. Then
\[ \dual{\sSet{V, \xi}} = \bigcup_{U\in \neighbourhood_\xi(0)}U^\pol, \]
where the polars are taken in $\sSet{\Lin(V, \F), V, \evalMap}$.
\end{proposition}
\begin{proof}
First take $f\in \dual{\sSet{V, \xi}}$. Then
\[ \{f\}^\pol = \setbuilder{v\in V}{\abspair{f,x} = |f(x)| \leq 1} = f^\preimf\big(\cball_\F(0,1)\big), \]
which is equal to some $U\in \neighbourhood_\xi(0)$ by continuity of $f$ and \ref{continuityVicinityFilter}. By \ref{polarsGaloisConnection} this implies $\{f\} \subseteq U^\pol$ and thus
\[ \dual{\sSet{V, \xi}} = \bigcup \setbuilder[\big]{\{f\}}{f\in \dual{\sSet{V, \xi}}} \subseteq \bigcup_{U\in \neighbourhood_\xi(0)}U^\pol. \]
For the converse, take some $U\in \neighbourhood_\xi(0)$ and $f\in U^\pol$. Then $U\subseteq \{f\}^\pol = f^\preimf\big(\cball_\F(0,1)\big)$, \ref{polarsGaloisConnection} and the calculation above, so $f^\imf(U) \subseteq \cball_\F(0,1)$ by \ref{functionImagePreimageGaloisConnection}. This implies that $f$ is continuous by \ref{boundedOnVicinityImpliesContinuous}, so $f\in \dual{\sSet{V, \xi}}$.
\end{proof}

\subsection{Polar topologies}

\begin{lemma} \label{weaklyBoundedLemma}
Let $\sSet{X,Y,\pair{\cdot,\cdot}}$ be a pairing and $A\subseteq Y$ a subset. The following are equivalent:
\begin{enumerate}
\item $A$ is weakly bounded;
\item $\sup_{y\in A}\abspair{x, y}$ is finite for all $x\in X$;
\item $A^\pol$ is an absorbent subset of $X$.
\end{enumerate}
\end{lemma}
\begin{proof}
$(1) \Leftrightarrow (2)$ Immediate from \ref{vonNeumannBoundednessInitialSpace} and \ref{weakTopologyLCTVS}, noting that
\[ \sup_{y\in A}\abspair{x, y} = \sup\Big(\abspair{x,-}^{\imf}(A)\Big) \]
and that a subset of $\R$ is bounded iff its supremum is finite.

$(2) \Leftrightarrow (3)$ By \ref{polarLemma}, point (2) is equivalent to the finiteness of $p_{A^\pol}(x)$ for all $x\in X$. By \ref{gaugeWellDefined} (and the fact that $A^\pol$ is balanced, \ref{polarPropertiesLemma}) we have that this is equivalent to the absorbence of $A^\pol$.
\end{proof}

\begin{proposition}
Let $\sSet{X,Y,\pair{\cdot,\cdot}}$ be a pairing and $\mathcal{A}\subseteq\powerset(Y)$ a set of weakly bounded sets. Then the filter
\[ N = \mathfrak{F}\setbuilder{\epsilon A^\pol}{\epsilon>0, A\in \mathcal{A}} \]
is the neighbourhood filter of $0$ in a locally convex vector space topology on $X$.
\end{proposition}
\begin{proof}
We use \ref{TVSbase} to verify that this filter is the neighbourhood filter of a topological vector space. Indeed, the sets $\epsilon A^\pol$ are absorbent by \ref{vonNeumannBoundedSetLemma} and \ref{weaklyBoundedLemma}. They are balanced by \ref{polarPropertiesLemma}. Finaly by convexity (\ref{polarPropertiesLemma}), we have $\frac{1}{2}\epsilon A^\pol + \frac{1}{2} \epsilon A^\pol \subseteq \epsilon A^\pol$.

As noted, the basis sets are convex, so the topology is locally convex.
\end{proof}
\begin{corollary}
Let $\sSet{X,Y,\pair{\cdot,\cdot}}$ be a pairing and $\mathcal{A}\subseteq\powerset(Y)$ a set of weakly bounded sets.
If $\mathcal{A}$ is upwards directed and closed under scalar multiplication, then $\upset\{A^\pol\}_{A\in \mathcal{A}}$ is the neighbourhood filter of $0$ in a locally convex vector space topology on $X$.
\end{corollary}
\begin{proof}
For all $A\in \mathcal{A}$ and $\epsilon>0$, we have $\epsilon A^\pol = (\epsilon^{-1}A)^\pol$. Thus $\mathfrak{F}\setbuilder{\epsilon A^\pol}{\epsilon>0, A\in \mathcal{A}} = \mathfrak{F}\setbuilder{A^\pol}{A\in \mathcal{A}}$.

We just need to show that $\upset\setbuilder{A^\pol}{A\in \mathcal{A}}$ is closed under intersections. Take $A,B\in \mathcal{A}$. By the upwards direction, we have $C\in \mathcal{A}$ such that $A\cup B \subseteq C$.
Then, using \ref{polarOfUnion}, $A^\pol\cap B^\pol = (A\cup B)^\pol \supseteq C^\pol$.
\end{proof}

\begin{definition}
Let $\sSet{X,Y,\pair{\cdot,\cdot}}$ be a pairing and $\mathcal{A}\subseteq\powerset(Y)$ a set of weakly bounded sets. The topology on $X$ with neighbourhood filter
\[ \neighbourhood(0) \defeq \mathfrak{F}\setbuilder{\lambda A^\pol}{\lambda\in \F, A\in \mathcal{A}} \]
is called the \udef{polar topology} determined by $\mathcal{A}$. We denote it $\polarTop{\mathcal{A}}$.
\end{definition}

\begin{lemma} \label{setsDeterminingTheSamePolarTopologies}
Let $\sSet{X,Y,\pair{\cdot,\cdot}}$ be a pairing and $\mathcal{A}\subseteq\powerset(Y)$ a set of weakly bounded sets. Then $\polarTop{\mathcal{A}}$ is equal to the polar topology determined by
\begin{enumerate}
\item $\mathfrak{I}(\mathcal{A})$;
\item $(\closure_{\sigma(X,Y)}\circ \disked)^{\imf}(\mathcal{A})$.
\end{enumerate}
\end{lemma}
\begin{proof}
(1) If $B\subseteq A\in \mathcal{A}$, then $B^\pol \supseteq A^\pol$, which means that $B^\pol$ is a neighbourhood of $0$ in the polar topology.

If $A,B\in \mathcal{A}$, then $A^\pol\cap B^\pol = (A\cup B)^\pol$ (by \ref{polarOfUnion}) is a neighbourhood (because the neighbourhoods form a filter).

(2) We have $A^\pol = A^{\pol\pol\pol} = \Big(\overline{\disked(B)}^{\sigma(X,Y)}\Big)^\pol$ by the bipolar theorem \ref{bipolarTheorem}.
\end{proof}

\begin{lemma} \label{polarTopologyOrdering}
Let $\sSet{X,Y,\pair{\cdot,\cdot}}$ be a pairing and $\mathcal{A}, \mathcal{B}\subseteq\powerset(Y)$ sets of weakly bounded sets. Then $\polarTop{\mathcal{A}} \subseteq \polarTop{\mathcal{B}}$ \textup{if and only if} $\mathfrak{I}\big((\closure_{\sigma(X,Y)}\circ \disked)^{\imf}(\mathcal{A})\big) \supseteq \mathfrak{I}\big((\closure_{\sigma(X,Y)}\circ \disked)^{\imf}(\mathcal{B})\big)$.
\end{lemma}
\begin{proof}
TODO
\end{proof}

\begin{proposition}
Let $\sSet{X,Y,\pair{\cdot,\cdot}}$ be a dual system and  $\mathcal{A}\subseteq\powerset(Y)$ a set of weakly bounded sets. Then the polar topology $\polarTop{\mathcal{A}}$ is Hausdorff \textup{if and only if} $\Span\Big(\bigcup \mathcal{A}\Big)$ is $\sigma(X,Y)$-dense in $Y$.
\end{proposition}
\begin{proof}
TODO! 8.5.1 Narici / Beckenstein
\end{proof}

\begin{proposition} \label{weak*topologyPolarTopology}
Let $\sSet{X,Y,\pair{\cdot,\cdot}}$ be a pairing. The polar topology determined by the singletons in $Y$ is $\sigma^*(X,Y)$. Thus
\[ \neighbourhood_{\sigma^*(X,Y)}(0) = \mathfrak{F}\setbuilder{\{y\}^\pol}{y\in Y} = \upset\setbuilder{F^\pol}{\text{$F\subseteq Y$ finite}} \]
\end{proposition}
\begin{proof}
Comparing with \ref{weakTopologyLCTVS}, we have, for all $y\in Y$, that $\setbuilder{x\in X}{\abspair{x,y}\leq 1} = \{y\}^\pol$ by \ref{polarLemma}.
\end{proof}


\begin{lemma} \label{dualPolarTopology}
Let $\sSet{X,Y,\pair{\cdot,\cdot}}$ be a pairing and $\mathcal{A}$ a set of weakly bounded subsets of $Y$. Then
\[ \dual{\sSet{X, \polarTop{\mathcal{A}}}} = \bigcup_{A\in \mathcal{A}, \epsilon>0}\epsilon A^{\pol\pol'}, \]
where $\pol'$ is the polar relation in the pair $\sSet{\Lin(X, \F), X, \evalMap}$.
\end{lemma}
\begin{proof}
From \ref{dualFromPolars} we have
\begin{align*}
\dual{\sSet{X, \polarTop{\mathcal{A}}}} &= \bigcup_{U\in \neighbourhood_{\polarTop{\mathcal{A}}}(0)}U^{\pol'} \\
&= \bigcup \big((-)^{\pol'}\big)^\imf\Big(\mathfrak{F}\setbuilder{(\epsilon A)^\pol}{\epsilon>0, A\in \mathcal{A}}\Big) \\
&= \bigcup \downset\big((-)^{\pol'}\big)^\imf\Big(\mathfrak{F}\setbuilder{(\epsilon A)^\pol}{\epsilon, A\in \mathcal{A}}\Big) \\
&= \bigcup \mathfrak{I}\setbuilder{(\epsilon A)^{\pol\pol'}}{\epsilon>0, A\in \mathcal{A}} \\
&= \bigcup \setbuilder{(\epsilon A)^{\pol\pol'}}{\epsilon>0, A\in \mathcal{A}} \\
&= \bigcup_{A\in \mathcal{A}, \epsilon>0}\epsilon A^{\pol\pol'}.
\end{align*}
\end{proof}

\subsubsection{Strong topology}
\begin{definition}
Let $\sSet{X,Y,\pair{\cdot,\cdot}}$ be a pairing. The polar topology on $X$ determined by the set of all weakly bounded subsets of $Y$ is called the \udef{strong topology} and is denoted $\beta(X,Y)$.
\end{definition}
The set of all weakly bounded subsets is upwards directed and closed under scalar multiplication by \ref{vonNeumannBoundedSetLemma}, so the neighbourhood filter of $0$ in the strong topology on $X$ is given by
\[ \upset\setbuilder{A^\pol}{\text{$A$ is weakly bounded}}. \]

\subsubsection{The Mackey topology}
\begin{definition}
Let $\sSet{X,Y,\pair{\cdot,\cdot}}$ be a pairing. The polar topology determined by weak-compact and absolutely convex subsets of $Y$ is called the \udef{Mackey topology} and is denoted $\tau(X, Y)$.
\end{definition}

TODO: for Banach space Mackey topology determined by weak-compact sets

\begin{proposition}
Let $\sSet{X,Y,\pair{\cdot,\cdot}}$ be a pairing. Then $\dual{\sSet{X, \tau(X,Y)}} = \dual{\sSet{X, \sigma^*(X,Y)}}$.
\end{proposition}
\begin{proof}
Let $\pol$ be the polar relation in the pairing $\sSet{X,Y,\pair{\cdot,\cdot}}$ and $\pol'$ the polar relation in the pairing $\sSet{\dual{\sSet{X, \tau(X,Y)}}, X, \evalMap}$. Let $K$ be the set of weak-compact and absolutely convex subsets of $Y$. As $K$ is closed under scalar multiplication (by \ref{diskedHullHomogeneous} and \ref{continuityLemmaVectorConvergence}), we have, by \ref{dualPolarTopology}
\[ \dual{\sSet{X, \tau(X,Y)}} = \bigcup_{A\in K}A^{\pol\pol'} \]
TODO
\end{proof}


\subsection{Weak-$*$-compactness}
\begin{theorem}[Banach/Bourbaki-Alaoglu] \label{alaogluTheorem}
Let $(X,Y,\pair{\cdot,\cdot})$ be a pairing such that $Y$ separates $X$ and $A\subseteq Y$ a neighbourhood of the origin in the weak topology. Then $A^\pol$ is $\sigma^*(X,Y)$-compact.
\end{theorem}
\begin{proof}
Since $A$ is absorbent, by \ref{TVSconstruction}, we have that $A\subseteq A^{\pol\pol}$ is also absorbent. Then $A^\pol = A^{\pol\pol\pol}$ is weak-$*$ bounded by \ref{weaklyBoundedLemma} and thus weak-$*$ totally bounded by \ref{weakBoundedness}.

By \ref{dualSystemBijection} we have $\sSet{X, \sigma^*(X,Y)} \cong \dual{\sSet{Y, \sigma(X,Y)}} \subseteq \Lin(Y, \F)_p$. Let $\pol'$ be the relation defined on the pairing $\sSet{\dual{\sSet{Y, \sigma(X,Y)}},Y, \evalMap}$ and $\pol^{\prime\prime}$ the relation defined on the pairing $\sSet{\Lin(Y, \F),Y, \evalMap}$. Clearly $A^{\pol'} = A^{\pol^{\prime\prime}} \cap \dual{\sSet{Y, \sigma(X,Y)}}$, so $A^{\pol'} = A^{\pol^{\prime\prime}}$ by \ref{dualFromPolars}.

Now $\sigma^*(\Lin(Y, \F), Y)$-convergence is exactly pointwise convergence and in this convergence, $A^{\pol^{\prime\prime}}$ is totally bounded by \ref{subspaceBoundedness} and closed by \ref{polarPropertiesLemma}, complete by \ref{algebraicDualComplete} and \ref{closedComplete} and thus finally compact by \ref{compactPrecompactComplete}.
\end{proof}

\subsubsection{The Mackey-Arens theorem}


\ref{HausdorffLCTVSdualSystem}

\begin{theorem}[Mackey-Arens]
Let $\sSet{V,\xi}$ be a Hausdorff convergence vector space and consider the pairing $\sSet{\dual{V}, V, \evalMap}$. Let $\zeta$ be a locally convex topological vector space convergence on $V$. Then the following are equivalent:
\begin{enumerate}
\item $\dual{\sSet{V,\xi}} = \dual{\sSet{V,\zeta}}$;
\item $\zeta$ is a polar topology determined by a cover of absolutely convex $\sigma(\dual{V}, V)$-compact subsets of $\dual{V}$;
\item $\tau(\dual{V}, V) \subseteq \zeta \subseteq \sigma(\dual{V}, V)$.
\end{enumerate}
\end{theorem}
\begin{proof}
$(1) \Rightarrow (2)$ By \ref{lconvModPolarTopology}, $\zeta$ is the polar topology determined by $\setbuilder{U^\pol}{U\in \neighbourhood_\zeta(0)}$, which is a set of absolutely convex $\sigma(\dual{V}, V)$-compact (= $\sigma(\dual{\sSet{V, \xi}}, V)$-compact = $\sigma(\dual{\sSet{V, \zeta}}, V)$-compact) subsets of $\dual{V}$, by \ref{polarPropertiesLemma} and \ref{alaogluTheorem}.

$(2) \Rightarrow (3)$ Clearly $\tau(\dual{V}, V)$ is the strongest such convergence. That $\sigma(\dual{V}, V)$ is such a convergence is given by \ref{weak*topologyPolarTopology}. 

Now take any such convergence $\polarTop{\mathcal{A}}$, where $\mathcal{A}$ covers $\dual{V}$. By \ref{setsDeterminingTheSamePolarTopologies}, $\polarTop{\mathcal{A}}$ is also determined by $\mathfrak{I}(\mathcal{A})$ and $\setbuilder[\big]{\{f\}}{f\in \dual{V}} \subseteq \mathfrak{I}(\mathcal{A})$, so $\polarTop{\mathcal{A}} \subseteq \sigma(\dual{V}, V)$ by \ref{polarTopologyOrdering}.

$(3) \Rightarrow (1)$ It is enough to show $\dual{\sSet{V,\sigma(\dual{V}, V)}} = \dual{\sSet{V,\xi}} = \dual{\sSet{V,\tau(\dual{V}, V)}}$. The first equality is given by \ref{functionalContinuityWeakTopology}. 
\end{proof}


\subsection{The pairing $\sSet{V^*, V, \evalMap}$}
\subsubsection{Equicontinuity and polars}
\begin{proposition} \label{equicontinuityTopologicalFunctionals}
Let $\sSet{V, \xi}$ be a topological convergence space. A subset $H\subseteq \dual{V}$ is equicontinuous \textup{if and only if} there exists $U\in \neighbourhood_\xi(0)$ such that $H\subseteq U^\pol$.
\end{proposition}
The polar is taken in either $\sSet{\Lin(V,\F), V,\evalMap}$ or
$\sSet{\dual{V}, V,\evalMap}$.
\begin{proof}
By \ref{equicontinuityGroupHomomorphisms}, the equicontinuity of $H$ is equivalent to (using pretopologicity, it is enough to consider $\neighbourhood_\xi(0)$)
\begin{align*}
\upset\evalMap^{\imf\imf}\big(\{H\}\otimes \neighbourhood_\xi(0)\big) \overset{\F}{\longrightarrow} 0 &\iff \forall \epsilon >0: \exists U\in \neighbourhood_\xi(0): \; \evalMap^\imf(H\times U) \subseteq \cball(0,\epsilon) \\
&\iff \forall \epsilon >0: \exists U\in \neighbourhood_\xi(0): \forall f\in H: \forall x\in U: \; \abspair{f,x}\leq \epsilon \\
&\iff \forall \epsilon >0: \exists U\in \neighbourhood_\xi(0): \forall f\in H: \forall x\in U: \; \abspair{f,\epsilon^{-1}x}\leq 1 \\
&\iff \forall \epsilon >0: \exists U\in \neighbourhood_\xi(0): \;H\subseteq (\epsilon^{-1}U)^\pol = \epsilon U^\pol.
\end{align*}
In particular, setting $\epsilon =1$, there exists $U\in \neighbourhood_\xi(0)$ such that $H\subseteq U^\pol$. 

Conversely, if $H\subseteq U^\pol$, then for all $\epsilon > 0$, we have $H \subseteq \epsilon (\epsilon U)^\pol$ and $\epsilon U\in \neighbourhood_\xi(0)$ by \ref{vicinityFilterAtOrigin}.
\end{proof}
\begin{corollary}
Let $(X,Y,\pair{\cdot,\cdot})$ be a dual system and $A\subseteq X$. If $A$ is weakly equicontinuous, then $\overline{A}^{\sigma(X,Y)}$ is $\sigma(X,Y)$-compact.
\end{corollary}
\begin{proof}
The equicontinuity of $A$ is equivalent to being a subset of $U^\pol$ for some $U\in \neighbourhood_{\sigma(X,Y)}(0)$ by the proposition. Now $U^\pol$ is compact by \ref{alaogluTheorem}, and so $\overline{A}^{\sigma(X,Y)}$ is compact by \ref{compactClosedSets}.
\end{proof}

\begin{proposition} \label{lconvModPolarTopology}
Let $\sSet{V, \xi}$ be a convergence vector space. Then $\lconvMod(\xi)$ is the polar topology w.r.t. the equicontinuous subsets of $\dual{V}$.
\end{proposition}
Note we are implicitly considering the pairing $\sSet{V,\dual{V},\evalMap^d}$.
\begin{proof}
By \ref{vicinityFilterAtOrigin}, we $\neighbourhood_{\lconvMod(\xi)}(0)$ has a base of closed balanced sets. These sets are also weakly closed by \ref{weakClosureConvexSubsets}. For each $U$ in this base, we have $U = U^{\pol\pol}$ by the bipolar theorem \ref{bipolarTheorem} (applied to the pairing $\sSet{V,\dual{V}, \evalMap}$).

By \ref{equicontinuityTopologicalFunctionals} this base consist of polars of equicontinuous subsets of $\dual{\sSet{V, \lconvMod(\xi)}}$, which, by \ref{equicontinuousSetsLconvMod}, are the same as the equicontinuous subsets of $\dual{V}$.

Conversely, every equicontinuous subset of $\dual{V}$ is contained in the upwards closure of this base, by \ref{equicontinuityTopologicalFunctionals}.
\end{proof}



\subsection{Orthogonal complements}
\begin{definition}
Let $\sSet{X,Y,\pair{\cdot,\cdot}}$ be a pairing and $B\subseteq Y$ a subset. The \udef{orthogonal complement} of $B$ is the polar w.r.t. the relation $\perp$ on $(Y,X)$ defined by
\[ y\perp x \qquad\iff\qquad \pair{x, y} = 0. \]
\end{definition}

\begin{proposition} \label{perpAsPolar}
Let $\sSet{X,Y,\pair{\cdot,\cdot}}$ be a pairing and $B\subseteq Y$ a subset. Then $B^\perp = \Span(B)^\pol$.
\end{proposition}
\begin{proof}
We show both inclusions. First assume $x\in B^\perp$. We need to show that $\abspair{x,y} \leq 1$ for all $y\in \Span(B)$. Indeed we can write $y = \sum_{j=1}^n c_jy_j$ for some $c_j\in\F$ and $y_j\in B$. Then
\[ \abspair{x,y} = \abspair{x, \sum_{j=1}^n c_jy_j} \leq \sum_{j=1}^n |c_j|\;\abspair{x, y_j} = 0 \leq 1, \]
as $y_j\perp x$.

Now assume $x\in \Span(B)^\pol$. If $\abspair{x,y} = 0$ for all $y\in B$, then $x\in B^\perp$ and we are done. Now assume, towards a contradiction, that this is not the case. Then there exists $y\in B$ such that $\abspair{x,y} = \epsilon \neq 1$. Then $\abspair{x,2\epsilon^{-1}y} = 2$, but $2\epsilon^{-1}y\in\Span(B)$, so $\abspair{x,2\epsilon^{-1}y} \leq 1$. This is a contradiction.
\end{proof}
\begin{corollary} \label{corollaryPerpAsPolar}
Let $\sSet{X,Y,\pair{\cdot,\cdot}}$ be a pairing and $B\subseteq Y$ a subset. Then
\begin{enumerate}
\item $B^\perp$ is a subspace;
\item $B^\perp = \Span(B)^\perp = \Span(B^\perp)$;
\item $\{0\}^\perp = X$;
\item $Y$ separates $X$ \textup{if and only if} $Y^\perp = \{0\}$;
\item $B^\perp$ is $\sigma^*(X,Y)$-closed.
\item $B^{\perp\perp} = \overline{\Span(B)}^{\sigma(X,Y)}$;
\item if $\Span(B)$ is $\sigma(X,Y)$-dense in $Y$ and $Y$ separates $X$, then $B^\perp = \{0\}$.
\end{enumerate}
\end{corollary}
\begin{proof}
(1) As $\Span(B)^\pol$ is convex by \ref{polarPropertiesLemma}, we just need to show it is closed under multiplication by $2$, \ref{convexSubspace}. Now $2\cdot \Span(B)^\pol = (2^{-1}\cdot\Span(B))^\pol = \Span(B)^\pol$ by \ref{polarPropertiesLemma}.

(2) We have $B^\perp = \Span(B)^\pol = \Span\big(\Span(B)\big)^\pol = \Span(B)^\perp$. The second equality follows straight from (1).

(3) For all $y\in Y$ we have $y\perp 0$.

(4) We have $0\in Y$, as for all $y\in Y$ we have $y\perp 0$. For all $x\in X\setminus\{0\}$, the following are equivalent: $x\notin Y^\perp$ and $\exists y\in Y: \pair{x,y}\neq 0$. Thus $Y$ being separating on $X$ is equivalent to $\forall x\in X\setminus \{0\}: x\notin Y^\perp$, which is equivalent to $Y^\perp \subseteq \{0\}$.

(5) By \ref{polarPropertiesLemma}.

(6) We have 
\[ B^{\perp\perp} = \Span(B^\perp)^\pol = \big(B^\perp\big)^\pol = \Span(B)^{\pol\pol} = \overline{\disked(\Span(B))}^{\sigma(X,Y)} = \overline{\Span(B)}^{\sigma(X,Y)}, \]
using the bipolar theorem \ref{bipolarTheorem}.

(7) In this case we have $B^\perp = B^{\perp\perp\perp} = \left(\overline{\Span(B)}^{\sigma(X,Y)}\right)^\perp = Y^\perp = \{0\}$.
\end{proof}


\begin{proposition}
Let $\sSet{X,Y,\pair{\cdot,\cdot}}$ be a pairing and $W_1,W_2$ subspaces of $Y$. Then
\[ (W_1+W_2)^\perp = W_1^\perp \cap W_2^\perp. \]
\end{proposition}
\begin{proof}
For a vector $v\in X$,
\begin{align*}
v\in (W_1+W_2)^\perp = (W_1 \cup W_2)^\perp &\implies \forall x\in W_1\cup W_2: \inner{v,x} = 0 \\
&\implies v\in W_1^\perp \cap W_2^\perp
\end{align*}
and
\begin{align*}
v\in W_1^\perp \cap W_2^\perp &\implies \forall x\in W_1, y\in W_2: \inner{v,x} = 0 = \inner{v,y} \\
&\implies \forall x\in W_1, y\in W_2:\inner{v, x+y} = 0 \implies v\in (W_1+W_2)^\perp.
\end{align*}
\end{proof}
TODO: dual result for closed subspaces.


\section{Adjoints}
\begin{definition}
Let $\sSet{X,Y,\pair{\cdot,\cdot}}$ and $\sSet{X',Y',\pair{\cdot,\cdot}'}$ be two pairings and $T: Y\to Y'$ a linear function. An \udef{adjoint} or \udef{transpose} is a linear operator $S: X'\to X$ such that
\[ \pair{x, Ty}' = \pair{Sx, y} \qquad \forall x\in X', \; y\in Y. \]
\end{definition}

\subsection{Adjoints for pairings}
\subsubsection{Existence of adjoints}
\begin{proposition} \label{existenceAdjointWeaklyContinuousFunction}
Let $\sSet{X,Y,\pair{\cdot,\cdot}}$ and $\sSet{X',Y',\pair{\cdot,\cdot}'}$ be two pairings, $T: Y\to Y'$ a linear function. Then $T$ has an adjoint $S: X'\to X$ \textup{if and only if} it is weakly continuous, i.e.\ continuous as a function
\[ T: \sSet{Y, \sigma(X,Y)} \to \sSet{Y', \sigma(X',Y')}. \]
The adjoint is unique if $Y$ separates $X$.
\end{proposition}
\begin{proof}
The weak continuity of $T$ is, by the characteristic property of the initial convergence \ref{characteristicPropertyInitialFinalConvergence}, equivalent to the continuity of
\[ \pair{x', \cdot}' \circ T = \pair{x', T(\cdot)}' : \sSet{Y, \sigma(X,Y)} \to \F \]
for all $x'\in X'$. In other words, $\pair{x', T(\cdot)}' \in \dual{\sSet{Y, \sigma(X,Y)}}$ for all $x'\in X'$. By \ref{functionalContinuityWeakTopology}, this is equivalent to the existence of some $x\in X$ such that $\pair{x', T(\cdot)}' = \pair{x,\cdot}$. We set $S(x') = x$.

Thus the weak continuity of $T$ is equivalent to the existence of some adjoint function $S$, without the requirement that it be linear. We now just need to show that $S$ can always be taken to be linear.

Indeed, pick some Hamel basis $\mathcal{X}$ of $X'$ and define $S'$ by setting $S'|_{\mathcal{X}} = S|_{\mathcal{X}}$ and extending linearly to the whole of $X'$. It is clear that $S'$ is an adjoint: set $x' = \sum_{x\in\mathcal{X}}\lambda_x x$ (with only finitely many $\lambda_x$ non-zero) and calculate
\begin{multline*}
\pair{x', T(\cdot)}' = \pair{\sum_{x\in\mathcal{X}}\lambda_x x, T(\cdot)}' = \sum_{x\in\mathcal{X}}\lambda_x \pair{x, T(\cdot)} = \sum_{x\in\mathcal{X}}\lambda_x \pair{S(x), \cdot} = \\ \sum_{x\in\mathcal{X}}\lambda_x \pair{S'(x), \cdot} = \pair{S'\left(\sum_{x\in\mathcal{X}}\lambda_x x\right), \cdot} = \pair{S'(x'), \cdot}.
\end{multline*}

Finally we note that the choice of $S(x')$ is unique if $Y$ separates $X$ by \ref{dualSystemBijection}.
\end{proof}
\begin{corollary} \label{weak*continuityAdjoint}
The adjoint of a weakly continuous operator is weak-$*$ continuous.
\end{corollary}
\begin{proof}
Suppose $S$ is an adjoint of $T$. By symmetry of the definition, $T$ is an adjoint of $S$ when considering the dual pairings. Thus $S$ is weak-$*$ continuous by the proposition.
\end{proof}
\begin{corollary} \label{adjointContinuousFunction}
Let $\sSet{V,\xi}, \sSet{W,\zeta}$ be vector convergence spaces and $T: V\to W$ a continuous linear function. Then $T$ has a unique adjoint
\[ S: \dual{W}\to \dual{V}: f\mapsto f\circ T. \]
\end{corollary}
Thus $S$ is the pre-composition $T^\star$. We can summarise this by $T^* = T^\star$.
\begin{proof}
The function $T$ is weakly continuous by \ref{continuityImpliesWeakContinuity} and $V$ separates $\dual{V}$ in the pairing $\sSet{\dual{V}, V, \evalMap}$. Finally the adjoint relation gives $f\big(T(x)\big) = S(f)(x)$, which determines $S$.
\end{proof}

\subsubsection{Calculating with adjoints}
\begin{proposition}
Let $\sSet{X,Y,\pair{\cdot,\cdot}}$, $\sSet{X',Y',\pair{\cdot,\cdot}'}$ be pairings, $T, T': Y\to Y'$ weakly continuous linear functions with adjoints $S, S': X'\to X$ and $\lambda\in \F$. Then
\begin{enumerate}
\item $S+S'$ is an adjoint of $T+T'$;
\item $\lambda S$ is an adjoint of $\lambda T$;
\item $\id_X$ is an adjoint of $\id_Y$.
\end{enumerate}
\end{proposition}
\begin{proof}
TODO
\end{proof}

\begin{proposition} \label{adjointPreimagePolarLemma}
Let $\sSet{X,Y,\pair{\cdot,\cdot}}$, $\sSet{X',Y',\pair{\cdot,\cdot}'}$ be pairings and $T: Y\to Y'$ a weakly continuous linear function with adjoint $S: X'\to X$. Then for all subsets $A\subseteq Y$, we have
\[ \big(T^{\imf}(A)\big)^\pol = S^{\preimf}(A^\pol). \]
\end{proposition}
\begin{proof}
We calculate, using \ref{polarLemma},
\begin{align*}
S^{\preimf}(A^\pol) &= S^{\preimf}\Big(\bigcap_{y\in A}\setbuilder{x\in X}{\abspair{x,y}\leq 1}\Big) \\
&= \bigcap_{y\in A}S^{\preimf}\Big(\setbuilder{x\in X}{\abspair{x,y}\leq 1}\Big) \\
&= \bigcap_{y\in A}\Big(\setbuilder{x'\in X'}{\abspair{S(x'),y}\leq 1}\Big) \\
&= \bigcap_{y\in A}\Big(\setbuilder{x'\in X'}{\abspair{x',T(y)}\leq 1}\Big) \\
&= \bigcap_{y'\in T^{\imf}(A)}\Big(\setbuilder{x'\in X'}{\abspair{x',y'}\leq 1}\Big) \\
&= \big(T^{\imf}(A)\big)^\pol.
\end{align*}
\end{proof}
\begin{corollary}
Let $\sSet{X,Y,\pair{\cdot,\cdot}}$, $\sSet{X',Y',\pair{\cdot,\cdot}'}$ be pairings and $T: Y\to Y'$ a weakly continuous linear function with adjoint $S: X'\to X$. Then
\begin{enumerate}
\item $\ker(S) = \im(T)^\perp$;
\item if $T$ is surjective and $Y'$ separates $X'$, then $S$ is injective.
\end{enumerate}
\end{corollary}
TODO: can we strengthen (2)?
\begin{proof}
(1) Set $A = Y$. Then $S^\preimf(Y^\pol) = S^\preimf(\{0\}) = \ker(S)$ and $T^\imf(Y) = \im(T)$.

(2) In this case $\ker(S) = (Y')^\perp = \{0\}$ by \ref{corollaryPerpAsPolar}, which means that $S$ is injective by \ref{injectivityKernelTriviality}.
\end{proof}

\begin{proposition}
Let $\sSet{X,Y,\pair{\cdot,\cdot}}$, $\sSet{X',Y',\pair{\cdot,\cdot}'}$ and $\sSet{X^{\prime\prime},Y^{\prime\prime},\pair{\cdot,\cdot}^{\prime\prime}}$ be pairings and $T: Y\to Y'$, $T': Y'\to Y^{\prime\prime}$ linear functions with adjoints $S: X'\to X$ and $S': X^{\prime\prime}\to X'$. Then $S\circ S': X^{\prime\prime}\to X$ is an adjoint of $T'\circ T: Y\to Y^{\prime\prime}$.
\end{proposition}
\begin{proof}
For all $x\in X^{\prime\prime}$ and $y\in Y$ we have
\[ \pair{x, (T'\circ T)(y)}^{\prime\prime} = \pair{S'(x), T(y)}' = \pair{(S\circ S')(x), y}. \]
\end{proof}
\begin{corollary}
Let $\sSet{X,Y,\pair{\cdot,\cdot}}$ and $\sSet{X',Y',\pair{\cdot,\cdot}'}$be a pairing and $T: Y\to Y'$ a weakly continuous linear bijection with adjoint $S: X'\to X$. Then $S$ has a (left/right?) inverse $R$ which is an adjoint of $T^{-1}$.
\end{corollary}
\begin{proof}
TODO
\end{proof}

\subsubsection{Continuity of adjoints}
\begin{proposition}
Let $\sSet{X,Y,\pair{\cdot,\cdot}}$, $\sSet{X',Y',\pair{\cdot,\cdot}'}$ be pairings and $T: Y\to Y'$ a weakly continuous linear function with adjoint $S: X'\to X$. Let $\mathcal{A}\subseteq\powerset(Y)$ be a set of weakly bounded sets. Then $S$ is continuous as a function
\[ S: \sSet{X',\polarTop{T^\imf(\mathcal{A})}} \to \sSet{X, \polarTop{\mathcal{A}}}. \]
\end{proposition}
\begin{proof}
By \ref{pretopologicalContinuityVicinities} and \ref{shiftHomeomorphism}, it is enough to check that $S^\preimf(U)\in \mathfrak{F}\setbuilder{\lambda T^\imf(A)^\pol}{\lambda\in \F, A\in \mathcal{A}}$ for all $U \in \mathfrak{F}\setbuilder{\lambda A^\pol}{\lambda\in \F, A\in \mathcal{A}}$.

Any such $U$ a subset $\lambda_0 A_0^\pol \cap \ldots \cap \lambda_n A_n^\pol$ for some $A_0,\ldots, A_n\in \mathcal{A}$ and $\lambda_0,\ldots, \lambda_n\in \F$. Now, by \ref{adjointPreimagePolarLemma},
\begin{align*}
S^\preimf\big(\lambda_0 A_0^\pol \cap \ldots \cap \lambda_n A_n^\pol\big) &= \lambda_0 S^\preimf(A_0^\pol) \cap \ldots \cap \lambda_n S^\preimf(A_n^\pol) \\
&= \lambda_0 T^\imf(A_0)^\pol \cap \ldots \cap \lambda_n T^\imf(A_n)^\pol \in \mathfrak{F}\setbuilder{\lambda T^\imf(A)^\pol}{\lambda\in \F, A\in \mathcal{A}}.
\end{align*}
\end{proof}
This proposition, together with \ref{weak*topologyPolarTopology}, provides another way to prove the weak-$*$ continuity of the adjoint $S$ (\ref{weak*continuityAdjoint}).
\begin{corollary}
Let $\sSet{V,\norm{\cdot}}$ and $\sSet{W,\norm{\cdot}}$ be normed spaces. Let $\dual{V}$ and $\dual{W}$ have their norm topologies and consider the pairings $\sSet{\dual{V}, V, \evalMap}$ and $\sSet{\dual{W}, W, \evalMap}$. Let $T: V\to W$ be a linear function with adjoint $S: \dual{W} \to \dual{V}$. Then $T$ is continuous \textup{if and only if} $S$ is continuous.
\end{corollary}
\begin{proof}
TODO with \ref{dualNormTopologyStrong}
\end{proof}


\section{The continuous dual}
\begin{definition}
Let $\sSet{V,\xi}$ be a convergence vector space. The \udef{continuous dual} of $V$ is the vector convergence space $\dual{V}_c = \contLin_c(V, \F)$.
\end{definition}
The continuous dual is a convergence vector space by \ref{continuousConvergenceVectorSpace}.

\begin{proposition}
Let $\sSet{V,\xi}$ be a convergence vector space and $H\in \powerfilters(\dual{V})$. Then $H$ converges continuously to $f$ \textup{if and only if} $H$ converges pointwise to $f$ and for all $F \overset{\xi}{\longrightarrow} 0$, there exists $A\in F$ such that $A^\pol\in H$.
\end{proposition}
\begin{proof}
First assume $H\overset{c}{\longrightarrow} f$. Then $H$ converges pointwise to $f$ by \ref{strengthContinuousPointwiseConvergence}.

Take $F \overset{\xi}{\longrightarrow} 0$. Then, by continuous convergence, $\upset\evalMap^{\imf\imf}(H\otimes F) \overset{\F}{\longrightarrow} f(0) = 0$. In particular there exist $B\in H, A\in F$ such that $\evalMap^\imf(B\times A) \subseteq \cball(0,1)$. This means that $B\subseteq A^\pol$ and thus $A^\pol\in H$.

Now consider the direction $\Leftarrow$. Take some $G\overset{\xi}{\longrightarrow} x$. We need to show that $\upset\evalMap^{\imf\imf}(H\otimes G) \overset{\F}{\longrightarrow} f(x)$. Now
\[ \upset\evalMap^{\imf\imf}(H\otimes G) = \upset\evalMap^{\imf\imf}\big(H\otimes (G-\pfilter{x})\big) + \upset\evalMap^{\imf\imf}(H\otimes \pfilter{x}) \]
and $\upset\evalMap^{\imf\imf}(H\otimes \pfilter{x})$ converges to $f(x)$ by pointwise convergence, so we need to show that $\upset\evalMap^{\imf\imf}\big(H\otimes (G-\pfilter{x})\big)$ converges to $0$.

For all $\epsilon > 0$, $\epsilon(G-\pfilter{x})$ converges to $0$ and thus there exists $A\in G-\pfilter{x}$ such that $B\subseteq (\epsilon A)^\pol = \epsilon^{-1}A^\pol$ for some $B\in H$. This implies $\evalMap^{\imf}(B\times A) \subseteq \cball(0,\epsilon)$ and thus $\cball(0,\epsilon)\in \upset\evalMap^{\imf\imf}\big(H\otimes (G-\pfilter{x})\big)$.

As $\epsilon$ was taken arbitrarily, we have $\neighbourhood_\F(0) \subseteq \upset\evalMap^{\imf\imf}\big(H\otimes (G-\pfilter{x})\big)$ and thus $\upset\evalMap^{\imf\imf}\big(H\otimes (G-\pfilter{x})\big) \overset{\F}{\longrightarrow} 0$.
\end{proof}
\begin{corollary} \label{continuousDualTopologicalSpace}
If $\sSet{V, \xi}$ is topological, then
\begin{enumerate}
\item $H$ converges continuously to $f$ \textup{if and only if} $H$ converges pointwise to $f$ and there exists $A\in \neighbourhood_\xi(0)$ such that $A^\pol\in H$;
\item $\dual{V}_c$ carries the specified sets convergence on $\dual{V}_p$ determined by $\setbuilder{A^\pol}{A\in \neighbourhood_\xi(0)}$;
\item $\dual{V}_c$ carries the specified sets convergence on $\dual{V}_p$ determined by the equicontinuous subsets of $\dual{V}$.
\end{enumerate}
\end{corollary}
\begin{corollary}
Let $\sSet{V,\xi}$ be a convergence vector space and $H\in \powerfilters(\dual{V})$. Then $H$ converges continuously to $0$ \textup{if and only if} $H$ contains all polars of finite sets and for all $F \overset{\xi}{\longrightarrow} 0$, there exists $A\in F$ such that $A^\pol\in H$.
\end{corollary}
\begin{proof}
By \ref{weak*topologyPolarTopology}.
\end{proof}

\begin{proposition}
Let $\sSet{V,\xi}$ be a locally convex topological space. If there exists a vector space topology on $\dual{V}$ such that $\evalMap: \dual{V}\times V\to \F$ is continuous, then $\sSet{V,\xi}$ is seminormable.
\end{proposition}
\begin{proof}
Since $\cball(0,1)\subseteq \F$ is a neighbourhood of $0$, $\evalMap^{\preimf}\big(\cball(0,1)\big)$ is a neighbourhood of $(0,0)\in \dual{V}\times V$ by \ref{continuityVicinityFilter}. Thus we can find $A\subseteq \neighbourhood_{\dual{V}}(0)$ and $B\subseteq \neighbourhood_V(0)$ such that $\evalMap^\imf(A\times B)\subseteq \cball(0,1)$, which means that $A\subseteq B^\pol$. By local convexity we may take $B$ to be convex.

Now $A$ is absorbent by \ref{vicinityFilterAtOrigin}, so $B^\pol$ is absorbent by \ref{absorbingSetProperties} and thus $B$ weakly bounded by \ref{weaklyBoundedLemma}. It is bounded by \ref{weaklyBoundedIffBounded}. This implies $\sSet{V,\xi}$ is seminormable by \ref{metrisableBoundedNeighbourhood}.
\end{proof}

\begin{lemma} \label{continuousDualArzelaAscoli}
Let $\sSet{V,\xi}$ be a vector convergence space and $H\subseteq \dual{V}_c$. Then
\begin{enumerate}
\item $H$ is relatively compact \textup{if and only if} it is equicontinuous;
\item $H$ is compact \textup{if and only if} it is equicontinuous and closed.
\end{enumerate}
\end{lemma}
\begin{proof}
(1) If $H$ is relatively compact, then $H$ is evenly continuous by \ref{compactoidImpliesEvenContinuity} and thus equicontinuous by \ref{equicontinuityEvenContinuity}.

For the converse, equicontinuity implies even continuity by \ref{equicontinuityEvenContinuity} and so we can apply \ref{evenContinuityRelativeCompactness}. We just need to show that $\evalMap_x^\imf(H)$ is relatively compact in $\F$ for all $x\in V$.

Since $\neighbourhood_\F(0)\cdot v\to 0$, we have, by equicontinuity \ref{equicontinuityGroupHomomorphisms}, that
\[ \neighbourhood_\F(0)\cdot \{\evalMap_x^\imf(H)\} = \neighbourhood_\F(0)\cdot \evalMap^{\imf\imf}(\{H\}\otimes \pfilter{x}) = \evalMap^{\imf\imf}\big(\{H\}\otimes \neighbourhood_\F(0)\cdot \pfilter{x}\big) \to 0, \]
so $\evalMap_x^\imf(H)\subseteq \F$ is bounded. By Heine-Borel (TODO ref) it is relatively compact.

(2) Follows from (1) by \ref{compactClosedSets} and the fact that $\dual{V}_c$ is Hausdorff, by \ref{continuousConvergencePropertiesFromCodomain}.
\end{proof}

\begin{proposition}
Let $\sSet{V,\xi}$ be a topological vector convergence space and $U\in \neighbourhood_\xi(0)$. Then $U^\pol$ is a compact subset of $\dual{V}_c$.
\end{proposition}
\begin{proof}
We apply \ref{continuousDualArzelaAscoli} by noting that $U^\pol$ is equicontinuous (\ref{equicontinuityTopologicalFunctionals}) and closed (\ref{polarPropertiesLemma}, \ref{openClosedConvergenceInclusions} and \ref{strengthContinuousPointwiseConvergence}).
\end{proof}
\begin{corollary} \label{continuousDualTVSLocallyCompact}
Let $\sSet{V,\xi}$ be a topological vector convergence space. Then $\dual{V}_c$ is locally compact.
\end{corollary}
\begin{proof}
By \ref{continuousDualTopologicalSpace}, each continuously convergence filter in $\powerfilters(\dual{V})$ contains $A^\pol$ for some $A\in\neighbourhood_\xi(0)$.
\end{proof}
TODO if $V$ is infinite dimensional, then no topological vector convergence on $\dual{V}$ makes it locally compact (?)

\subsection{Continuous adjoints}
\begin{proposition}
Let $\sSet{V,\xi}$, $\sSet{W, \zeta}$ be convergence vector spaces and $T: V\to W$ a continuous linear function with adjoint $S: \dual{W}\to \dual{V}: f\mapsto f\circ T$. Then $S$ is continuous as a function $S: \dual{W}_c\to \dual{V}_c$.
\end{proposition}
\begin{proof}
We have that $(\id_{\dual{W}}, \underline{T}): f\mapsto (f, T)$ is continuous by \ref{embeddingInContinuousConvergence} and \ref{continuityFunctionTuple}. The adjoint $S$ is this function composed with the composition operator, which is continuous by \ref{compositionContinuouslyContinuous}.
\end{proof}

\subsection{Reflexivity}
Let $\sSet{V,\xi}$ be a convergence vector space. Then the map $\evalMap_-: V\to \dual{(\dual{V}_c)}_c: v\mapsto \evalMap_v$ is linear and continuous by \ref{curriedEvalMapContinuous}.

\begin{definition}
Let $\sSet{V,\xi}$ be a convergence vector space. It is called \udef{reflexive} if $\evalMap_-$ is an isomorphism.
\end{definition}

\begin{proposition}
Let $\sSet{X,\xi}$ be a convergence space. Then $\cont_c(X)$ is reflexive.
\end{proposition}
\begin{proof}
TODO Beattie / Butzmann p. 125
\end{proof}

\begin{proposition}
Let $\sSet{V,\xi}$ be a reflexive convergence vector space. Then $\lconvMod(\dual{V}_c) = \contLin_\text{co}(V)$.
\end{proposition}
\begin{proof}
We have that $\id: \dual{V}_c \to \contLin_\text{co}(V)$ is continuous by \ref{continuousConvergenceCompactOpenComparison}.

TODO  BB 4.2.19
\end{proof}

\subsection{The continuous dual of particular topological vector spaces}
\subsubsection{The continuous dual of a locally comapct TVS}
\begin{proposition} \label{continuousDualLocallyCompactTVS}
Let $\sSet{V,\xi}$ be a locally compact vector convergence space. Then $\dual{V}_c = \contLin_\text{co}(V)$ and
\[ \neighbourhood_{\dual{V}_c}(0) = \upset\setbuilder{K^\pol}{\text{$K\subseteq V$ compact}}. \]
\end{proposition}
\begin{proof}
The equality of the continuous and compact open convergences follows from \ref{continuousConvergenceCompactOpen}.

Pick compact $K\subseteq V$ and open $U\subseteq \F$ with $0\in U$. Then $\cball(0,\epsilon)\subseteq U$ for some $\epsilon >0$. If $f^\imf\big(K\big) \subseteq \cball(0,\epsilon) = \epsilon \cball(0,1)$, then $f\in (\epsilon^{-1}K)^\pol = \epsilon K^\pol$, which is compact. Thus every basis element of the neighbourhood filter of the compact open topology contains a polar of a compact set.

We further not that the set of polars of compact sets is closed under finite intersections: by \ref{polarOfUnion} $K_1^\pol \cap K_2^\pol = (K_1\cup K_2)^\pol$ and $K_1\cup K_2$ is compact by \ref{compactConstructions}.

Conversely $f\in K^\pol$ implies $f^{\imf}(K) \subseteq \ball(0, 1.1)$, so $K^\pol$ is contained in an open neighbourhood of the origin in the compact open topology.
\end{proof}

\subsubsection{The continuous dual of a locally convex TVS}
\begin{proposition}
Let $\sSet{V,\xi}$ be a locally convex, Hausdorff topological vector space. Then
\begin{enumerate}
\item $\dual{(\dual{V}_c)}_c$ is a complete, locally convex and Hausdorff TVS;
\item $\evalMap_-: V\to \dual{(\dual{V}_c)}_c$ is an embedding into a dense subspace.
\end{enumerate}
\end{proposition}
\begin{proof}
(1) Completeness is given by \ref{continuousDualComplete}; Hausdorffness by \ref{continuousConvergencePropertiesFromCodomain}.

The space $\dual{V}_c$ is locally compact by \ref{continuousDualTVSLocallyCompact} and then \ref{continuousDualLocallyCompactTVS} gives that $\dual{(\dual{V}_c)}_c$ is a locally convex TVS.

(2) TODO BB 4.3.19 
\end{proof}
\begin{corollary}
Let $\sSet{V,\xi}$ be a locally convex, Hausdorff topological vector space. Then $\dual{(\dual{V}_c)}_c$ is the completion of $V$.
\end{corollary}
\begin{corollary}
A Hausdorff locally convex topological vector space is reflexive \textup{if and only if} it is complete.
\end{corollary}

\section{Adjoints TODO}
\begin{definition}
Let $\sSet{X,Y,\pair{\cdot,\cdot}}$ and $\sSet{X',Y',\pair{\cdot,\cdot}'}$ be two pairings and $T: Y\not\to Y'$ a linear operator. An \udef{adjoint} or \udef{transpose} is a linear operator $S: X'\not\to X$ such that
\[ \pair{x, Ty}' = \pair{Sx, y} \qquad \forall x\dom(S), \; y\in\dom(T). \]
\end{definition}

\begin{lemma}
Let $\sSet{X,Y,\pair{\cdot,\cdot}}$ and $\sSet{X',Y',\pair{\cdot,\cdot}'}$ be two pairings and $T: Y\not\to Y'$ a linear operator.

Let $S_1, S_2: X'\not\to X$ be adjoints of $T$ then for all $x\in \dom(S_1)\cap\dom(S_2)$ we have $S_1(x) - S_2(x) \in \dom(T)^\perp$.

Conversely, let $S$ be an adjoint of $T$ and $x\in\dom(S)$. Then for all $v\in \dom(T)^\perp$ there exists an adjoint $S'$ such that $S'(x) = S(x) + v$.
\end{lemma}
\begin{proof}
For all $u\in \dom(T)$ we have
\[ \pair{S_1(x) - S_2(x), u} = \pair{S_1(x), u} - \pair{S_2(x), u} = \pair{x, Tu}' - \pair{x, Tu}' = 0. \]
So $(S_1(x) - S_2(x)) \in \dom(T)^\perp$.

For the converse, pick some $x\in X$ and set $S' = S + \pair{x,\cdot}v$. This is an adjoint: for all $a\in \dom(T), b\in \dom(S') = \dom(S)$ we have
\[  \inner{S'b,a} = \inner{Sb, a} + \pair{x,b}\pair{v,a} = \inner{Sb, a} = \inner{b,Ta}'. \]
\end{proof}
\begin{corollary}
Let $\sSet{X,Y,\pair{\cdot,\cdot}}$ and $\sSet{X',Y',\pair{\cdot,\cdot}'}$ be two pairings, $T: Y\not\to Y'$ a linear operator and $S_1, S_2$ two adjoints of $T$.

If $Y$ separates $X$ and $T$ is $\sigma(X,Y)$-densely defined, then for all $x\in \dom(S_1)\cap\dom(S_2)$ we have $S_1(x) = S_2(x)$.
\end{corollary}
\begin{proof}
We have, by \ref{corollaryPerpAsPolar}, $\dom(T)^\perp = \{0\}$. So $S_1(x) - S_2(x) = 0$.
\end{proof}
\begin{corollary}
Let $\sSet{X,Y,\pair{\cdot,\cdot}}$ and $\sSet{X',Y',\pair{\cdot,\cdot}'}$ be two pairings, such that $Y$ separates $X$.

Let $T: Y\not\to Y'$ be a linear operator. Then
\[ \bigcup\setbuilder{\graph(S)}{\text{$S\in (X'\not\to X)$ is an adjoint of $T$}} \]
is the graph of an operator \textup{if and only if} $T$ is $\sigma(X,Y)$-densely defined.
\end{corollary}

\begin{definition}
Let $\sSet{X,Y,\pair{\cdot,\cdot}}$ and $\sSet{X',Y',\pair{\cdot,\cdot}'}$ be two pairings. Let $T: Y\not\to Y'$ be a linear operator.

We define the adjoint $T^*$ as the \emph{relation} on $(X',X)$ with graph
\[ \graph(T^*) \defeq \bigcup\setbuilder{\graph(S)}{\text{$S\in (X'\not\to X)$ is an adjoint of $T$}}. \]
\end{definition}

\begin{lemma}
Let $\sSet{X,Y,\pair{\cdot,\cdot}}$ and $\sSet{X',Y',\pair{\cdot,\cdot}'}$ be two pairings, such that $Y$ separates $X$. Let $T: Y\not\to Y'$ be a linear operator with $\sigma(X,Y)$-dense domain.

If $S$ is an adjoint of $T$ that is defined everywhere, then $T^* = S$.
\end{lemma}

\begin{lemma} \label{pairAdjointRelationLemma}
Let $\sSet{X,Y,\pair{\cdot,\cdot}}$ and $\sSet{X',Y',\pair{\cdot,\cdot}'}$ be two pairings. Let $T: Y\not\to Y'$ be a linear operator and $(x,y)\in X'\times X$.

Then $(x, y)\in T^*$ \textup{if and only if}
\[ \forall z\in\dom(T): \; \pair{x, T(z)} = \pair{y, z}. \]
\end{lemma}
\begin{proof}
$\boxed{\Rightarrow}$ If $(x, y)\in T^*$, then there exists an adjoint $S: X'\not\to X$ such that $S(x) = y$. For all $z\in \dom(T)$ we have $\pair{x, T(z)}' = \pair{S(x), z} = \pair{y, z}$.

$\boxed{\Leftarrow}$ The function defined by $S(x) = y$ and extended to $\Span\{x\}$ by linearity is an adjoint.
\end{proof}

\begin{proposition} \label{pairAdjointDomain}
Let $\sSet{X,Y,\pair{\cdot,\cdot}}$ and $\sSet{X',Y',\pair{\cdot,\cdot}'}$ be two pairings. Let $T: Y\not\to Y'$ be a linear operator. Then
\[ \dom(T^*) = \setbuilder{x\in X'}{\text{$\dom(T)\to \F: u\mapsto \pair{x, Tu}$ is a $\sigma(X,Y)$-continuous functional}}. \]
\end{proposition}
\begin{proof}
$\boxed{\subseteq}$ If $\omega_x: u\mapsto \inner{x, Tu}$ is $\sigma(X,Y)$-continuous, then it can be extended to a continuous functional on all of $Y$ by 


------ TODO: rework from here!! ------

its domain can be extended by continuity to $\overline{\dom(T)}$, which is a Hilbert space. This extended functional has a Riesz vector $x^*$ such that $\omega_x = u\mapsto \inner{x^*, u}$. The linear operator with domain $\Span\{x\}$ that maps $x$ to $x^*$ is then an adjoint.

$\boxed{\supseteq}$ If $x\in\dom(T^*)$, then, using the Cauchy-Schwarz inequality,
\[ |\inner{x,Tu}| = |\inner{T^*x,u}| \leq \norm{T^*x}\;\norm{u}, \]
so the functional $u\mapsto \inner{x, Tu}$ is bounded.
\end{proof}
\begin{corollary}
The domain $\dom(T^*)$ is a vector space and in particular contains $0$.
\end{corollary}

\begin{proposition} \label{HilbertAdjointGaloisConnection}
Let $H, K$ be Hilbert spaces. Take $T\in (H\not\to K)$ and $S\in (K\not\to H)$. Then
\[ S \subseteq T^* \iff T\subseteq S^*. \]
Thus $(*,*)$ is an antitone Galois connection between $\sSet{(H\not\to K), \subseteq}$ and $\sSet{(K\not\to H), \subseteq}$.
\end{proposition}
\begin{proof}
We have $S \subseteq T^*$ iff $S$ is an adjoint of $T$ iff $T$ is an adjoint of $S$ (by \ref{adjointRequirementSymmetric}) iff $T\subseteq S^*$.
\end{proof}
\begin{corollary} \label{HilbertAdjointAntitone}
Let $S,T: H\not\to K$ be operators between Hilbert spaces such that $S\subseteq T$. Then $T^* \subseteq S^*$.
\end{corollary}
\subsection{Properties of the adjoint relation}

\begin{proposition}
Let $T$ be an operator between Hilbert spaces and $\lambda\in\C$. If $\lambda \neq 0$, then
\[ \begin{pmatrix}
\id & 0 \\ 0 & \overline{\lambda}\id
\end{pmatrix} \graph(T^*) = (\lambda T)^*. \]
\end{proposition}
Note that if $T^*$ is a function (i.e.\ if $T$ is densely defined), then $\begin{pmatrix}
\id & 0 \\ 0 & \overline{\lambda}\id
\end{pmatrix} \graph(T^*) = \overline{\lambda}T^*$. We write the former in the proposition, because we have not made this assumption.

If $\lambda = 0$ and $T: H\not\to K$, then
\[ \begin{pmatrix}
\id & 0 \\ 0 & 0
\end{pmatrix} \graph(T^*) = \big(0: \dom(T^*)\to H\big) \subseteq \big(0: K\to H\big) = (0 T)^*, \]
where the last equality is given by \ref{adjointBoundedEverywhereDefined}.
\begin{proof}
For the inclusion $\subseteq$, take $f$ to be an adjoint of $T$. It is enough to show that $\overline{\lambda}f$ is an adjoint of $\lambda T$. This follows from
\[ \inner{\overline{\lambda}f(w), v} = \lambda\inner{f(w), v} = \lambda\inner{w,Tv} = \inner{w,\lambda Tv} \qquad \forall w\in \dom(f), v\in \dom(T). \]

For the other inclusion, let $f$ be an adjoint of $\lambda T$. It is enough to show that $\overline{\lambda^{-1}}f$ is an adjoint of $T$, because then $f = \overline{\lambda}\cdot\overline{\lambda^{-1}}f \subseteq \begin{pmatrix}
\id & 0 \\ 0 & \overline{\lambda}\id
\end{pmatrix} \graph(T^*)$. Indeed
\[ \inner{\overline{\lambda^{-1}}f(w), v} = \lambda^{-1}\inner{f(w),v} = \inner{w,\lambda^{-1}\lambda Tv} = \inner{w,Tv} \quad \forall w\in \dom(f), v\in \dom(T). \]
\end{proof}

\begin{proposition} \label{adjointGraph}
Let $T: H\not\to K$ be an operator between Hilbert spaces. Then
\begin{align*}
\graph(T^*) &= \left( \begin{pmatrix}
0 & -\id \\ \id & 0
\end{pmatrix}\graph(T) \right)^\perp 
=  \begin{pmatrix}
0 & -\id \\ \id & 0
\end{pmatrix}\graph(T)^\perp.
\end{align*}
If $T$ is densely defined, then $T^*$ is a closed operator.
\end{proposition}
\begin{proof}
We have
\[ \graph(T^*) = \bigcup\setbuilder{\graph(S)}{\text{$S\in (K\not\to H)$ is an adjoint of $T$}}. \]
Take an adjoint $S$ and $(w, Sw)$ in $\graph(S)$. Then for all $v\in\dom(T)$:
\[ 0 = \inner{w, Tv}_K - \inner{Sw, v}_H = \inner{w, Tv}_K + \inner{Sw, -v}_H = \inner{(w, Sw), (Tv,-v)}_{K\oplus H}. \]
So $(Tv,-v) = \begin{pmatrix}
0 & -\id \\ \id & 0
\end{pmatrix} (v,Tv) \in \graph(S)^\perp $.

The final equality follows from \ref{perpUnderIsometry}, using the fact that $\begin{pmatrix}
0 & -\id \\ \id & 0
\end{pmatrix}$ is a surjective isometry.

If $T$ is densely defined, then $T^*$ is a function by \ref{maximalAdjointIsOperator}. It is closed by \ref{orthogonalComplementClosed}.
\end{proof}
\begin{corollary} \label{adjointDenselyDefinedClosable}
Let $T: H\not\to K$ be a densely defined operator between Hilbert spaces.
Then
\begin{enumerate}
\item $\graph(T^{**}) = \overline{\graph(T)}$;
\item $T^*$ is densely defined \textup{if and only if} $T$ is closable;
\item If $T$ is closable, then $\overline{T} = T^{**}$.
\end{enumerate}
\end{corollary}
\begin{proof}
From the proposition we have
\begin{align*}
\graph(T^{**}) &=  \begin{pmatrix}
0 & -\id \\ \id & 0
\end{pmatrix}\graph(T^*)^\perp 
=  \begin{pmatrix}
0 & -\id \\ \id & 0
\end{pmatrix}\left(\begin{pmatrix}
0 & -\id \\ \id & 0
\end{pmatrix}\graph(T)^\perp\right)^\perp \\
&= \begin{pmatrix}
0 & -\id \\ \id & 0
\end{pmatrix}^2\graph(T)^{\perp\perp} = -\graph(T)^{\perp\perp}
= \overline{\graph(T)}.
\end{align*}
The right-hand side is the graph of an operator iff $T$ is closable and the left-hand side is the graph of an operator iff $T^*$ is densely defined, by \ref{maximalAdjointIsOperator}.

For a closable operator, the closure is defined by $\overline{\graph(T)} = \graph(\overline{T})$.
\end{proof}

\begin{proposition} \label{adjointBoundedEverywhereDefined}
Let $T: H\to K$ be a densely defined operator between Hilbert spaces. Then $\dom(T^*) = K$ \textup{if and only if} $T$ is bounded.
\end{proposition}
\begin{proof}
The direction $\Leftarrow$ is given by \ref{adjointDomain}.

For the other direction, note that $T^*$ is closed by \ref{adjointGraph}. Then $T^*$ is bounded by the closed graph theorem \ref{BanachClosedGraphTheorem}. We use the direction $\Leftarrow$ to see that $\dom(T^{**}) = H$. Similarly, $T^{**}$ is closed by \ref{adjointGraph} and bounded by the closed graph theorem \ref{BanachClosedGraphTheorem}. Thus $T\subseteq \overline{T} = T^{**}$ is bounded.
\end{proof}

An important application of this proposition is the Hellinger-Toeplitz theorem \ref{HellingerToeplitz}.

\begin{proposition} \label{adjointAlgebraicProperties}
Let $T,S$ be compatible operators between Hilbert spaces. Then
\begin{enumerate}
\item $S^* + T^* \subseteq (S+T)^*$;
\item $S^*T^* \subseteq (TS)^*$.
\end{enumerate}
\end{proposition}
\begin{proof}
(1) Let $f$ be an adjoint of $S$ and $g$ an adjoint of $T$. It is enough to see that $f+g$ is an adjoint of $S+T$. Indeed $\forall w\in \dom(f + g), v\in \dom(S+T)$
\[ \inner{(f + g)(w), v} = \inner{f(w),v} + \inner{g(w), Tv} = \inner{w,Sv} + \inner{w,Tv} = \inner{w,(S+T)v}. \]

(2) Let $f$ be an adjoint of $T$ and $g$ an adjoint of $S$. It is enough to see that $gf$ is an adjoint of $TS$. Indeed
\[ \inner{g\circ f(w), v} = \inner{f(w), Sv} = \inner{w,TSv} \qquad \forall w\in \dom(g\circ f), v\in \dom(TS). \]
\end{proof}


There exist various conditions that make the inclusions in \ref{adjointAlgebraicProperties} equalities.
\begin{proposition} \label{equalityAlgebraicPropertiesAdjoint}
Let $T,S$ be compatible operators between Hilbert spaces.
\begin{enumerate}
\item if $T$ is densely defined, $\dom(S) \subseteq \dom(T)$ and $\dom\big((S+T)^*\big) \subseteq \dom(T^*)$, then $S^* + T^* = (S+T)^*$;
\item if $T$ is densely defined, $\im(S)\subseteq \dom(T)$ and $\dom\big((TS)^*)\subseteq \dom(T^*)$, then $S^*T^* = (TS)^*$;
\item if $S$ is densely defined and $\im(S)$ has finite codimension, then $S^*T^* = (TS)^*$.
\end{enumerate}
\end{proposition}
\begin{proof}
(1) By \ref{adjointAlgebraicProperties}, we have
\[ (S+T)^* - T^* \subseteq (S+T-T)^* = S^*, \]
where the last equality is due to $\dom(S) \subseteq \dom(T)$. Now take $x,y$ such that $x\in \dom\big((S+T)^*\big)$. Then $T^*(x)$ exists and we have the implications
\begin{align*}
x(S+T)^*y \iff& x\big((S+T)^* - T^* + T^*\big)y \\
\iff& \exists z: \; x\big((S+T)^* - T^*\big)z \land (z+T^*(x) = y) \\
\implies& \exists z: \; x(S^*)z \land (z+T^*(x) = y) \\
\iff& x(S^* + T^*)y.
\end{align*}
Thus $(S+T)^* \subseteq S^* + T^*$.

(2) We need to prove $(TS)^* \subseteq S^*T^*$. Assume $(x,y)\in (TS)^*$. By \ref{adjointRelationLemma}, we have
\[ \forall z\in \dom(TS):\; \inner{x, TS(z)} = \inner{y, z}. \]
Because $\im(S)\subseteq \dom(T)$, we have $\dom(TS) = \dom(S)$. Also, by assumption, $x\in \dom(T^*)$. So we have
\[ \forall z\in \dom(S):\; \inner{x, TS(z)} = \inner{T^*(x), S(z)} = \inner{y, z}, \]
which means that $\big(T^*(x), y\big)\in S^*$, so $(x,y)\in S^*T^*$.

(3)
\end{proof}
\begin{corollary}
If $T$ is bounded and everywhere defined, then
\[ S^* + T^* = (S+T)^* \qquad\text{and}\qquad S^*T^* = (TS)^*. \]
\end{corollary}


\begin{lemma} \label{HilbertAdjointLemma}
Let $S,T\in\Bounded(H,K)$ and $\lambda \in \mathbb{F}$.
\begin{enumerate}
\item $(T^*)^* = T$;
\item $(S+T)^* = S^* + T^*$;
\item $(\lambda T)^* = \bar{\lambda}T^*$;
\item $\id_V^* = \id_V$.
\end{enumerate}
Let $T\in\Bounded(H_1,H_2), S\in\Bounded(H_2,H_3)$
\begin{enumerate}
\setcounter{enumi}{4}
\item $(ST)^* = T^*S^*$.
\end{enumerate}
\end{lemma}

\begin{note}
Useful exercise: The identities of \ref{HilbertAdjointLemma} can also be proven by elementary manipulations. For example, to prove (1), we take arbitrary $v\in H$ and $w\in K$, Then
\[ \inner{w,Tv} = \inner{T^*w,v} = \overline{\inner{v,T^*w}} = \overline{\inner{(T^*)^*v,w}} = \inner{w, (T^*)^*v}. \]
By lemma \ref{elementaryOrthogonality} we have $Tv = (T^*)^*v$ for all $v\in V$. 
\end{note}

\subsection{Adjoints of densely defined operators}
The adjoint of an operator is a function if and only the operator is densely defined.

\begin{proposition} \label{adjointRangeCriterion}
Let $S: K\not\to H$ and $T: H\not\to K$ be linear operators between Hilbert spaces. If
\[ \im(S\cap T^*) = H \qquad\text{and}\qquad \im(T\cap S^*) = K, \]
then $S$ and $T$ are densely defined with $S^* = T$ and $T^* = S$.
\end{proposition}
\begin{proof}
Notice that $S\cap T^*$ and $T\cap S^*$ are linear operators that are adjoints of each other.

We claim that they are densely defined: take $x\in \dom(S\cap T^*)^\perp$. Then there exists some $y\in H$ such that $x = (T\cap S^*)y$ because of surjectivity. Now for all $z\in \dom(S\cap T^*)$
\[ 0 = \inner{z,x} = \inner{z, (T\cap S^*)y} = \inner{(S\cap T^*)z, y}, \]
so $\inner{z',y} = 0$ for all $z'\in H$, by surjectivity. This means, by \ref{elementaryOrthogonality}, that $y=0$ and thus also $x = (T\cap S^*)y = 0$. We conclude that $\dom(S\cap T^*)^\perp = \{0\}$, meaning $(S\cap T^*)$ is densely defined. The argument for $(T\cap S^*)$ is similar.

It follows that $S$ and $T$ must be densely defined. We have, by \ref{kernelImageAdjoint},
\[ \ker(S) = \im(S^*)^\perp \subseteq \im(T\cap S^*)^\perp = \{0\}. \]
Similarly $\ker(T) = \ker(S^*) = \ker(T^*) = \{0\}$.

So we have $\ker(S) = \ker(T^*)$, $\im(S)\subseteq \im(S\cap T^*)$ and $\im(T^*)\subseteq \im(S\cap T^*)$. The equality $S = T^*$ follows from \ref{partialFunctionSubset}. The equality $T = S^*$ is similar.
\end{proof}


\begin{proposition} \label{kernelImageAdjoint}
Let $T: H\not\to K$ be an operator between Hilbert spaces. Then
\[ \forall v\in K: \; (v,0)\in T^* \iff v\in \im(T)^\perp. \]
If $T^*$ is densely defined, this reduces to
\begin{enumerate}
\item $\ker(T^*) = \im(T)^\perp$;
\item $\ker(T) \subseteq \im(T^*)^\perp$;
\item if $T$ is closed, then $\ker(T) = \im(T^*)^\perp$
\end{enumerate}
\end{proposition}
\begin{proof}
(1) Because $\dom(T)$ is dense in $H$, we have $\dom(T)^\perp = \{0\}$ by \ref{orthogonalComplementDenseSpace}. Take $v\in K$. We have the equivalences
\begin{align*}
v\in \im(T)^\perp &\iff \forall x \in\dom(T): \inner{v, T(x)} = 0 \\
&\iff \forall x \in\dom(T): \inner{v, T(x)} = \inner{v, 0} \\
&\iff (v,0)\in T^*,
\end{align*}
using \ref{adjointRelationLemma}.

Point (1) is a direct translation in the case that $T^*$ is a function.

For point (2) note that $T\subseteq T^{**}$ (by \ref{adjointDenselyDefinedClosable}) implies that $(v,0)\in T \implies (v,0)\in T^{**}$.

For point (3): in this case $\ker(T) = \ker(T^{**}) = \im(T^*)^\perp$.
\end{proof}
\begin{corollary}[Closed range theorem for Hilbert spaces]
Let $T$ be a closed, densely defined operator between Hilbert spaces. Then the following are equivalent:
\begin{enumerate}
\item $\im(T)$ is closed;
\item $\im(T^*)$ is closed;
\item $\im(T) = \ker(T^*)^\perp$;
\item $\im(T^*) = \ker(T)^\perp$.
\end{enumerate}
\end{corollary}
\begin{proof}
By the proposition and \ref{orthogonalComplementClosed}, we have $\overline{\im(T)} = \ker(T^*)^\perp$. This shows $(1) \Leftrightarrow (3)$ and $(2) \Leftrightarrow (4)$.

TODO equivalence $(1)\Leftrightarrow (2)$.
\end{proof}
TODO ref closed range theorem for Banach spaces. This is, e.g., the case when $T$ is bounded below, see \ref{boundedBelowClosedRange}.

\begin{proposition}
Let $T: H\not\to K$ be a densely defined operator between Hilbert spaces. Then
\begin{enumerate}
\item $\im(T)$ is dense in $K$ \textup{if and only if} $T^*$ is injective;
\item if $T$ and $T^*$ are injective, then $(T^*)^{-1} = (T^{-1})^*$;
\item if $T$ is closable and $\overline{T}$ is injective, then $\overline{T}^{\,-1} = \overline{T^{-1}}$.
\end{enumerate}
\end{proposition}
\begin{proof}
(1) This is immediate from \ref{kernelImageAdjoint} and \ref{injectivityKernelTriviality}:
\[ \text{$\im(T)$ is dense} \quad\iff\quad \{0\} = \im(T)^\perp = \ker(T^*). \]

(2) We have $\graph(T^{-1}) = \begin{pmatrix}
0 & \id \\ \id & 0
\end{pmatrix}\graph(T)$. Also note that $\begin{pmatrix}
0 & \id \\ \id & 0
\end{pmatrix}$ and $\begin{pmatrix}
0 & -\id \\ \id & 0
\end{pmatrix}$ commute. Then we compute using \ref{adjointGraph}:
\begin{align*}
\graph((T^*)^{-1}) &= \begin{pmatrix}
0 & \id \\ \id & 0
\end{pmatrix}\begin{pmatrix}
0 & -\id \\ \id & 0
\end{pmatrix}\graph(T)^\perp \\
&= \begin{pmatrix}
0 & -\id \\ \id & 0
\end{pmatrix}\begin{pmatrix}
0 & \id \\ \id & 0
\end{pmatrix}\graph(T)^\perp \\
&= \begin{pmatrix}
0 & -\id \\ \id & 0
\end{pmatrix}\left(\begin{pmatrix}
0 & \id \\ \id & 0
\end{pmatrix}\graph(T)\right)^\perp = \graph((T^{-1})^*).
\end{align*}
The penultimate equality follows from \ref{perpUnderIsometry}, using the fact that $\begin{pmatrix}
0 & \id \\ \id & 0
\end{pmatrix}$ is a surjective isometry.
\end{proof}

\subsection{Adjoints of bounded operators}
\begin{proposition}
Let $T: H\to K$ be a densely defined operator between Hilbert spaces. Then
\begin{enumerate}
\item if $T\in\Bounded(H,K)$, then $T^*\in\Bounded(K,H)$;
\item if $T^*\in\Bounded(K,H)$, then $T$ is bounded. If $T$ is closed, then $T$ is defined everywhere.
\end{enumerate}
Assume $T\in\Bounded(H,K)$. Then
\begin{enumerate} \setcounter{enumi}{2}
\item $\norm{T} = \norm{T^*}$;
\item $T^* = C_H^{-1}T^tC_K$, where $C_K$ is the Riesz isometry from \ref{RieszIsometry}.
\end{enumerate}
\end{proposition}
\begin{proof}
(1) Assume $T\in\Bounded(H,K)$. Then $u\mapsto \inner{x,Tu}$ is a bounded functional for all $x\in K$, so $\dom(T^*) = K$ by \ref{adjointDomain}. Also $T^*$ is closed by \ref{adjointGraph}, so it is bounded by the closed graph theorem \ref{BanachClosedGraphTheorem}.

(2) Assume $T^*\in\Bounded(K,H)$. By the previous argument $T \subseteq \overline{T} = T^{**}\in\Bounded(H,K)$.

(3) The function $(x,u)\mapsto \inner{x,Tu}$ is a bounded sesquilinear form. By proposition \ref{sesquilinearRepresentation}, $T^*$ must be the unique $S$ from the proposition, which has norm $\norm{T}$.

(4) Finally we note that $C_H^{-1}T^tC_K$ is an adjoint with domain $K$ and conclude by \ref{everywhereDefinedAdjointLemma}.
\end{proof}

\begin{lemma}
The adjoint defines a map $*:\Bounded(H,K)\to \Bounded(K,H)$ that is anti-linear and continuous in the weak and uniform operator topologies. It is continuous in the strong operator topology \textup{if and only if} finite dimensional.
\end{lemma}
\begin{proof}
By the proposition the adjoint map is anti-linear. It is also bounded with norm $1$. Then by corollary \ref{boundedAntiLinearMaps} it must be bounded.

TODO
\end{proof}

\begin{proposition}
Let $H,K$ be Hilbert spaces and $T:H\to K$ a bijective bounded linear operator with bounded inverse. Then $(T^*)^{-1}$ exists and
\[ (T^*)^{-1} = (T^{-1})^*. \]
\end{proposition}
\begin{proof}
We prove $(T^{-1})^*$ is both a left- and a right-inverse of $T^*$: $\forall x\in H, y\in K$
\begin{align*}
\inner{T^*(T^{-1})^*x,y} &= \inner{x,T^{-1}Ty} = \inner{x,y} \\
\inner{x,(T^{-1})^*T^*y} &= \inner{TT^{-1}x,y} = \inner{x,y}
\end{align*}
So, by lemma \ref{elementaryOrthogonality}, $T^*(T^{-1})^* = \id_H$ and $(T^{-1})^*T^* = \id_K$.
\end{proof}

\begin{proposition} \label{normOfSquare}
Let $T\in \Bounded(H,K)$ with $H,K$ Hilbert spaces. Then
\[ \norm{T^*T}= \norm{T}^2 = \norm{TT^*}. \]
\end{proposition}
\begin{proof}
For $\norm{T^*T}= \norm{T}^2$ first observe that
\[ \norm{T^*T} \leq \norm{T^*}\cdot\norm{T} = \norm{T}^2. \]
Conversely, $\forall x\in H$:
\[ \norm{T(x)}^2 = \inner{Tx,Tx} = \inner{T^*Tx,x} \leq \norm{T^*Tx}\cdot \norm{x} \leq \norm{T^*T}\cdot\norm{x}^2. \]
The other equality follows by applying the first to $T^*$ and using $\norm{T^*}=\norm{T}$.
\end{proof}

\chapter{Operators on convergence vector spaces}

\section{Continuous operators}

\section{Discontinuous linear operators}
\subsection{Closed operators}
\begin{definition}
Let $T:\dom(T)\subseteq X\to Y$ be an operator. Then $T$ is a \udef{closed operator} if $\graph(T)$ is closed in $X\oplus Y$.
\end{definition}
This is not the same as a closed map in the convergence sense! It is studied in the convergence setting as well, see \ref{secFunctionsClosedGraph}.

\url{https://en.wikipedia.org/wiki/Unbounded_operator#Closed_linear_operators}
\url{https://en.wikipedia.org/wiki/Closed_graph_theorem_(functional_analysis)}

{closedGraphFunctionConstructions}

\begin{lemma} \label{algebraClosedOperators}
Let $T$ be a closed and $S$ a bounded operator, then $S+T$ is closed.
\end{lemma}
\begin{proof}
TODO
\end{proof}

\begin{lemma} \label{closedOperatorKernelClosed}
Let $T$ be a closed operator, then $\ker(T)$ is closed.
\end{lemma}
\begin{proof}
Let $F\in\powerfilters\big(\ker(T)\big)$ be a convergent filter that converges to $x$. Then $T^{\imf\imf}(F) = \pfilter{0}$ and thus converges to $0$. By closedness of $T$, $Tx = 0$ and thus $x\in\ker(T)$. 
\end{proof}
\begin{proof}[Alternate proof.]
Let $T: X\to Y$. Then $\ker(T)\times\{0\} = \graph(T)\cap X\times\{0\}$. As $\graph(T)$ is closed and $X\times\{0\}$ is closed by \ref{productOpenClosed}, we have that $\ker(T)$ is closed by \ref{productOpenClosed}.
\end{proof}
We have already proven this for bounded operators, see \ref{kerClosed}.

\section{Convergences on spaces of continuous operators}
\subsection{Strong operator topology}
\begin{definition}
Let $\sSet{V,\xi}, \sSet{W,\zeta}$ be a convergence vector spaces. The pointwise convergence on $\contLin(V,W)$ is called the \udef{strong operator convergence} in this case.
\end{definition}

\subsection{Weak operator topology}
\begin{definition}
Let $\sSet{V, \xi}, \sSet{W,\zeta}$ be convergence vector spaces. Consider the pairing $\sSet{\dual{W}\otimes V, \contLin(V,W), \pair{\cdot, \cdot}}$, where $\pair{f\otimes v, T} = f(Tx)$.

Then the \udef{weak operator topology} (or WOT) on $\contLin(V,W)$ is the weak topology $\sigma\big(\dual{W}\otimes V, \contLin(V,W)\big)$.
\end{definition}
\begin{lemma}
Let $\sSet{V, \xi}$ be a convergence vector space and $\sSet{W,\zeta}$ a Hausdorff locally convex convergence vector space. Then $\sSet{\dual{W}\otimes V, \contLin(V,W), \pair{\cdot, \cdot}}$ is a dual system. 
\end{lemma}
\begin{proof}
First take $T\in \contLin(V,W)\setminus \{0\}$. Then there exists $v\in V$ such that $Tv \neq 0$. By \ref{locallyConvexDualPair} there exists $f\in \dual{W}$ such that $f(Tv) \neq 0$, i.e. $\pair{f\otimes v, T} \neq 0$.

Now take $\sum_{i=1}^nf_i\otimes v_i \in \dual{W}\otimes V$. By \ref{tensorProductLinearlyIndependentBasis} we can take both the $f_i$s and the $v_i$s linearly independent. Since $\sum_{i=1}^n f_i \neq 0$, there exists $v\in V$ such that $\sum_{i=1}^n f_i(v) \neq 0$. Since $\{v_i\}_{i=1}^n$ is linearly independent, can construct a linear operator $T$ that maps all $v_i$ to $v$. Then $\pair{\sum_{i=1}^nf_i\otimes v_i, T} \neq 0$.
\end{proof}

\begin{lemma}
Let $\sSet{V, \xi}, \sSet{W,\zeta}$ be convergence vector spaces. Then the weak operator topology on $\contLin(V,W)$ is weaker than the strong topology.
\end{lemma}
\begin{proof}
Let $F\in\powerfilters\big(\contLin(V,W)\big)$ be a filter that converges in then strong topology. Take $f\otimes v\in \dual{W}\otimes V$. Then $\pair{f\otimes v,\cdot}^{\imf\imf}(F) = (f\circ \evalMap_v)^{\imf\imf}(F)$ converges since the strong topology makes $\evalMap_v$ continuous.
\end{proof}

\begin{lemma}
Let $\sSet{V, \xi}, \sSet{W,\zeta}$ be convergence vector spaces. Then the weak operator topology is the topology of pointwise convergence on $\contLin\big(V, \sSet{W, \sigma{\dual{W}, W}}\big)$.
\end{lemma}

\section{Compact operators}
\begin{definition}
A linear operator $T:X\to Y$ between TVSs is \udef{compact} if it maps a neighbourhood of the origin to a precompact set, i.e.\ 
\[ \exists U \in \neighbourhood(0): \;  \text{$\overline{T[U]}$ is compact.} \]
The set of compact linear operators in $(X\to Y)$ is denoted $\Compact(X,Y)$.
\end{definition}

\begin{lemma} \label{compactOperatorEquivalents}
Let $X$ be a normed space and $Y$ a TVS and $T:X\to Y$ a linear operator. Then the following are equivalent:
\begin{enumerate}
\item $T$ is a compact operator;
\item there exists a neighbourhood $U \subset X$ of the origin and a compact set $V\subset Y$ such that $T[U] \subset V$;
\item the image of the unit ball of $X$, $T[B(\vec{0},1)]$, is precompact in $Y$;
\item the image of any bounded set in $X$ is precompact in $Y$.
\end{enumerate}
If $Y$ is a normed space, these are also equivalent to
\begin{enumerate} \setcounter{enumi}{4}
\item for any bounded sequence $(x_{n})_{n\in \mathbb{N}}$ in $X$, the sequence $(Tx_{n})_{n\in \mathbb{N} }$ contains a converging subsequence.
\end{enumerate}
\end{lemma}
\begin{proof}
TODO
\end{proof}


\begin{lemma}
Let $X,Y$ be TVSs.
\begin{enumerate}
\item Then $\Compact(X, Y)$ is a vector space.
\item If $X,Y$ are normed spaces, then $\Compact(X, Y)$ is a subspace of $\Bounded(X, Y)$.
\end{enumerate}
\end{lemma}
\begin{proof}
(1) Let $K,K':X\to Y$ be compact operators. Then, by \ref{closureGroupOperation} (TODO opposite inclusion!),
\[ \overline{K[B(0, 1)]+K'[B(0, 1)]} \subseteq \overline{K[B(0, 1)]}+\overline{K'[B(0, 1)]}, \qquad \overline{K[\lambda B(0, 1)]} = \lambda\overline{K[B(0, 1)]}. \]

(2) Let $K\in\Compact(X, Y)$. Then the image of the unit ball is precompact, meaning it is bounded. So $K$ is bounded by \ref{existenceOperatorNorm}.
\end{proof}

\begin{lemma}
Let $T:V\to W$ be a bounded operator. If $W$ has the Heine-Borel property, then $T$ is compact.
\end{lemma}
\begin{proof}
The set $T[B(\vec{0},1)]$ is bounded because $T$ is. By the Heine-Borel (TODO ref) property of $W$, $\overline{B(\vec{0},1)}$ is compact.
\end{proof}
\begin{corollary}
Bounded operators with as image a finite dimensional normed space are compact.
\end{corollary}
\begin{corollary}
The identity on a normed space $X$ is compact \textup{if and only if} $X$ is finite-dimensional.
\end{corollary}
\begin{proof}
TODO ref. 
\end{proof}

\begin{proposition}
Compact operators map weakly convergent sequences to strongly convergent sequences. TODO! + remove from Hilbert section.
\end{proposition}
\begin{corollary} \label{limitCompactImageOrthonormalSequence}
Let $V$ be an inner product space and $\seq{e_n}$ a sequence of orthonormal vectors in $V$. If $K$ is a compact operator, then $\lim_{n\to\infty}Ke_n = 0$.
\end{corollary}
\begin{proof}
Any sequence of orthonormal vectors $\seq{e_n}$ converges weakly to $0$. Because $K$ is compact, $\seq{Ke_n}$ converges strongly to zero. TODO ref.
\end{proof}
\begin{corollary}
If $V$ is infinite-dimensional and $K$ is invertible, then its inverse is unbounded.
\end{corollary}
\begin{proof}
Due to $\lim_{n\to\infty}Ke_n = 0$ the operator $K$ cannot be bounded below, so $K^{-1}$ is not bounded by \ref{boundedBelow}.
\end{proof}



\part{Functional Analysis}
\setcounter{chapter}{0} % Reset chapter counter
\chapter{Vector space convergence}

\section{Vector space convergence}
\begin{definition}
Let $\sSet{\F,V,+}$ be a vector space and $\xi$ a convergence on $V$. Then $\sSet{\F,V,+, \xi}$ is a \udef{convergence vector space} if
\begin{itemize}
\item vector addition $+: V\times V \to V$ is continuous;
\item scalar multiplication $\cdot: \F\times V \to V$ is continuous.
\end{itemize}
\end{definition}

\begin{lemma}
If $\sSet{\F,V,+, \xi}$ is a convergence vector space, then $\sSet{V,+, 0, \xi}$ is a convergence group.
\end{lemma}
\begin{proof}
We just need to show that $v\mapsto -v$ is continuous, but this scalar multiplication and thus continuous by assumption.
\end{proof}

\begin{lemma} \label{continuityLemmaVectorConvergence}
If $\sSet{\F,V,+, \xi}$ is a convergence vector space, then
\begin{enumerate}
\item the function $\F \to \Span\{v\}: \lambda \mapsto \lambda\cdot v$ is a homeomorphism for all $v\in V$;
\item the function $V \to V: v \mapsto \lambda\cdot v$ is a homeomorphism (?) for all $\lambda\in \F$.
\end{enumerate}
\end{lemma}
\begin{proof}
The functions $\lambda \mapsto (\lambda, v)$ and $v \mapsto (\lambda, v)$ are continuous by \ref{continuousEmbeddingProduct}. Composition with the continuous scalar product gives the result by continuity of composition (\ref{continuityComposition}).

They are both clearly invertible (for the first, note that the kernel is $\{0\}$). The inverse of the second is of the same form and thus immediately continuous.

For the inverse of the first TODO!?!
\end{proof}

\begin{proposition}
Let $V$ be a vector space, $\{V_i\}_{i\in I}$ a set of convergence vector spaces and $\{L_i: V \to V_i\}_{i\in I}$ a set of linear maps. Then the initial convergence on $V$ w.r.t. $\{L_i: V \to V_i\}_{i\in I}$ makes $V$ a convergence vector space.
\end{proposition}
\begin{proof}
Continuity of vector addition follows from \ref{initialConvergenceGroup}.

We verify continuity of scalar multiplication $m: \F\times V \to V: (\lambda, v) \mapsto \lambda v$. Using \ref{characteristicPropertyInitialFinalConvergence}, we need to verify that $L_i\circ m$ is continuous for all $i\in I$. Because the $L_i$ are linear, we have
\[ L_i(\lambda v) = \lambda L_i(v) \]
for all $\lambda \in \F, v \in V$. This means that $L_i\circ m = m_i \circ L_i$, where $m_i$ is scalar multiplication in $V_i$, and thus continuous. So $L_i \circ m$ is continuous.
\end{proof}

\begin{proposition} \label{vectorSpaceConvergenceConstruction}
Let $V$ be a vector space over a field $\F$. And $\mathcal{F} \subseteq \powerfilters(V)$ a family of filters. There exists a vector space convergence $\xi$ on $V$ such that $\mathcal{F} = \lim^{-1}_\xi(0)$ \textup{if and only if}
\begin{enumerate}
\item if $F \in \mathcal{F}$ and $G\supseteq F$, then $G\in \mathcal{F}$;
\item if $F,G \in \mathcal{F}$, then $F + G\in \mathcal{F}$;
\item if $F\in \mathcal{F}$, then $\vicinity_\F(0)\cdot F \in \mathcal{F}$;
\item if $v\in V$, then $\vicinity_\F(0)\cdot v \in \mathcal{F}$;
\item if $F\in \mathcal{F}$ and $\lambda\in \F$, then $\lambda\cdot F \in \mathcal{F}$.
\end{enumerate}
\end{proposition}
Note the similarity with \ref{groupConvergenceConstruction} for convergence groups. A group convergence is completely determined by $\lim^{-1}_\xi(0)$ due to the translation homeomorphisms \ref{shiftHomeomorphism}.
\begin{proof}
Assume first that $\mathcal{F} = \lim^{-1}_\xi(0)$ for some vector space convergence $\xi$.
\begin{enumerate}
\item This is just the monotonicity of the convergence.
\item If $F,G\to 0$, then $F\otimes G \to (0,0)$ by \ref{convergenceFiniteProductFilter}. By continuity of addition we have $F+G\to 0$.
\item The convergence on the scalar field is pretopological, so $\vicinity_\F(0)\to 0$. By \ref{convergenceFiniteProductFilter} $\vicinity_\F(0)\otimes F \to (0,0)$ and by continuity of the scalar multiplication $\vicinity_\F(0)\cdot F \to 0$.
\item By \ref{continuityLemmaVectorConvergence}.
\item By \ref{continuityLemmaVectorConvergence}.
\end{enumerate}

Now assume the five points hold. Define the convergence $\xi$ by $F\to v$ iff $F-v \in \mathcal{F}$. We need to show that this is a convergence and that it makes both the vector addition and scalar multiplication continuous.

Monotonicity is guaranteed by (1). To show the convergence is centered, note that $\mathcal{F} \neq \emptyset$ by (4), so long as $V\neq \emptyset$. Then for any $F\in \mathcal{F}$, $\big\{\{0\}\big\} = 0\cdot F \in \mathcal{F}$ by (5).

To show that the vector addition is continuous, take $F\to (v_1, v_2)$. Then $p_1^{\imf\imf}(F) = F_1\to v_1$ and $p_2^{\imf\imf}(F) = F_2 \to v_2$, i.e.\ $F_1-v_1 \in \mathcal{F}$ and $F_2-v_2 \in \mathcal{F}$. By (1), $(F_1-v_1) + (F_2-v_2) = (F_1+F_2) - (v_1 + v_2) \in \mathcal{F}$, so $F_1+F_2 \to v_1 + v_2$. Thus by \ref{filterFactorisationInequality}, $F_1+F_2 = +^{\imf\imf}[F_1\otimes F_2] \subseteq +^{\imf\imf}[F] \to v_1+v_2$ and the addition is continuous.

Let $G \to (\lambda, v)$. Then $G_1 = p_1^{\imf\imf}(G) \to \lambda$ and $G_2 = p_2^{\imf\imf}(G) \to v$, so $G_1 \supseteq \vicinity_\F(\lambda)$. We have
\begin{align*}
\cdot^{\imf\imf}[G] - \lambda\cdot v &\supseteq \cdot^{\imf\imf}[G_1\otimes G_2] - \lambda\cdot v = G_1\cdot G_2 - \lambda\cdot v \\
&\supseteq \vicinity_\F(\lambda) \cdot G_2 - \lambda\cdot v \\
&= (\vicinity_\F(0) + \lambda)\cdot((G_2 - v) + v) - \lambda\cdot v \\
&\supseteq \lambda\cdot (G_2 - v) + \vicinity_\F(0)\cdot(G_2-v) + \lambda\cdot v + \vicinity_\F(0)\cdot v - \lambda\cdot v \\
&= \lambda\cdot (G_2 - v) + \vicinity_\F(0)\cdot(G_2-v) + \vicinity_\F(0)\cdot v \in \mathcal{F}.
\end{align*}
So $\cdot^{\imf\imf}[G] \to \lambda\cdot v$, making the scalar multiplication continuous. Note the last inclusion is not an equality because we go from one instance of $G_2$ and $\vicinity_\F(0)$ to two!
\end{proof}

\begin{proposition} \label{initialVectorConvergenceLinearFunctions}
Let initial convergence w.r.t. a set of linear functions is a vector space convergence.
\end{proposition}

\begin{proposition}
Let $\sSet{V,\xi}$ be a vector space convergence and $A\subseteq V$. Then
\begin{enumerate}
\item if $A$ is balanced, then $\adh_\xi(A)$ is balanced;
\item if $A$ is convex, then $\adh_\xi(A)$ is convex;
\item if $A$ is a subspace, then $\adh_\xi(A)$ is a subspace.
\end{enumerate}
\end{proposition}
\begin{proof}
(1) We use \ref{productAdherence} and \ref{adherenceContinuity} to compute
\begin{align*}
\cball(0,1)\cdot \adh_\xi(A) &= \cdot^\imf[\adh_\F(\cball(0,1))\times \adh_\xi(A)] = \cdot^\imf[\adh_{\F\otimes \xi}(\cball(0,1)\times A)] \\
&\subseteq \adh_{\xi}\big(\cdot^\imf[\cball(0,1)\times A]\big) = \adh_{\xi}(\cball(0,1)\cdot A) = \adh_\xi(A).
\end{align*}
\textit{Alternative proof:}
Take $|\lambda|\leq 1$ and $v\in \adh_\xi(A)$, then we need to show that $\lambda v \in \adh_\xi(A)$. We have $A\in \vicinity_\xi(v)^{\mesh}$ and $A\subseteq \lambda^{-1}A$. So for all $B\in \vicinity_\xi(v)$:
\[ A\mesh B \quad\implies\quad \lambda^{-1}A\mesh B \quad\implies\quad A\mesh \lambda B. \]
Thus $A\in \vicinity_\xi(\lambda v)^{\mesh}$, which is what we needed to show by \ref{principalAdherenceInherence}.

(2) Take $0\leq r \leq 1$ and $v,w\in \adh_\xi(A)$, then we need to show that $rv + (1-r)w \in \adh_\xi(A)$. To that end, take some arbitrary $B\in \vicinity_\xi(rv+(1-r)w)$. Now $B - (rv+(1-r)w) \in \vicinity_\xi(0)$, so by \ref{vicinityFactorisation} we can find a $U\in\vicinity_\xi(0)$ such that $U+U \subseteq B - (rv+(1-r)w)$, which means $r\big(r^{-1}U + v\big) + (1-r)\big((1-r)^{-1}U + w\big) \subseteq B$. Now $r^{-1}U, (1-r)^{-1}U\in \vicinity_\xi(0)$ because $v\mapsto\lambda\cdot v$ is a homeomorphism (\ref{continuityLemmaVectorConvergence}) and \ref{homeomorphismPreservation}, so $r^{-1}U + v \mesh A$ and $(1-r)^{-1}U + w \mesh A$. We take $v'\in r^{-1}U + v \cap A$ and $w'\in (1-r)^{-1}U + w \cap A$. Then $rv'+(1-r)w'$ is in $A$ by convexity and in $B$ by construction, so $A\mesh B$.

(3) Clearly $\adh_\xi(A)$ is not empty. It is then enough to verify that $\adh_\xi(A)+\adh_\xi(A)\subseteq \adh_\xi(A)$ and $\F\cdot \adh_\xi(A) \subseteq \adh_\xi(A)$. We use \ref{productAdherence} and \ref{adherenceContinuity} to compute
\begin{align*}
\adh_\xi(A) + \adh_\xi(A) &= +^\imf[\adh_\xi(A)\times \adh_\xi(A)] = +^\imf[\adh_{\xi\otimes \xi}(A\times A)] \\
&\subseteq \adh_{\xi}\big(+^\imf[A\times A]\big) = \adh_{\xi}(A + A) = \adh_\xi(A)
\end{align*}
and
\begin{align*}
\F\cdot \adh_\xi(A) &= \cdot^\imf[\adh_\F(\F)\times \adh_\xi(A)] = \cdot^\imf[\adh_{\F\otimes \xi}(\F\times A)] \\
&\subseteq \adh_{\xi}\big(\cdot^\imf[\F\times A]\big) = \adh_{\xi}(\F\cdot A) = \adh_\xi(A).
\end{align*}
\end{proof}
\begin{corollary} \label{hyperplaneClosedDense}
A hyperplane in a convergence vector space is either closed or dense.
\end{corollary}
\begin{proof}
Let $H$ be a hyperplane in a vector space $V$. Then $H \subseteq \adh(H)$ and $\adh(H)$ is a subspace. Because $H$ is a coatom, we have either $\adh(H) = H$ or $\adh(H) = V$. In the first case $H$ is closed, in the second dense.
\end{proof}

\begin{lemma}
Let $V$ be a convergence vector space over a field $K$ and $F\subseteq K$ a subfield. Then the $F$-vector space $V_F$ with the same convergence structure is also a convergence vector space.
\end{lemma}
\begin{proof}
It is enough that the restriction of the scalar multiplication to $F$ remains continuous. Alternatively, we can use \ref{vectorSpaceConvergenceConstruction} and note that passing to the field $F$ simply represents a weakening of condition $(5)$.
\end{proof}

\subsection{Algebraic convergence}
\begin{definition}
Let $V$ be a vector space over a field $\F$. The \udef{algebraic convergence} on $V$ is the strongest vector space convergence on $V$. We denote this convergence $\mathfrak{a}$ and thus write $F \overset{\mathfrak{a}}{\longrightarrow} x$ and $x\in \lim_\mathfrak{a} F$.
\end{definition}
Note that the algebraic convergence is not just the discrete convergence because, for example, $\seq{n^{-1}v} \to 0$ for all $v\in V$ by continuity of the scalar product and the fact that $\pfilter{v} \to v$.


We need to show that the definition makes sense.
\begin{proposition} \label{algebraicConvergence}
Let $V$ be a vector space over a field $\F$. There exists a strongest vector space convergence on $V$ and this convergence is defined by
\[ \forall F\in \powerfilters(V): \qquad F\in {\lim}^{-1}(0) \iff \exists v\in V: \;F \supseteq \vicinity_\F(0)\cdot v.  \]
\end{proposition}
\begin{proof}
Using \ref{vectorSpaceConvergenceConstruction} it is easy to see that this convergence is a vector space convergence. Conversely, this is the strongest possible convergence by point (4) of \ref{vectorSpaceConvergenceConstruction}.
\end{proof}
Note that the algebraic convergence of a non-trivial vector space is not topological!

\begin{lemma} \label{constructionsInAlebraicConvergence}
Let $V$ be a vector space over a field $\F$, $v\in V$ and $A\subseteq V$ a subset. Then
\begin{enumerate}
\item $\begin{aligned}[t]
\vicinity_\mathfrak{a}(0) &= \bigcap_{v\in V} \upset \vicinity_\F(0)\cdot v \\
&= \setbuilder{B\in \powerset(V)}{\forall v\in V: \exists \Gamma_v\in \vicinity_\F(0):\; \Gamma_v\cdot v\subseteq B} \\
&= \setbuilder{\bigcup_{v\in V} \Gamma_v\cdot V}{\forall v\in V:\; \Gamma_v \in \vicinity_\F(0)};
\end{aligned}$
\item $\inh_\mathfrak{a}(A) = \setbuilder{x\in V}{\forall v\in V:\exists \Gamma_v \in \vicinity_\F(0):\; x + \Gamma_v\cdot v \subseteq A}$;
\item $\adh_\mathfrak{a}(A) = \setbuilder{x\in V}{\exists v\in V: \forall \Gamma\in\vicinity_\F(0):\; (x+\Gamma\cdot v)\mesh A}$.
\end{enumerate}
\end{lemma}
\begin{proof}
(1) The first equality follows straight from \ref{algebraicConvergence}.

(2) We have $\inh_\mathfrak{a}(A) = \setbuilder{x}{A\in \vicinity_\mathfrak{a}(x)} = \setbuilder{x}{A-x\in \vicinity_\mathfrak{a}(0)}$. From (1) we get 
\begin{align*}
\inh_\mathfrak{a}(A) &= \setbuilder{x}{\forall v\in V: \exists \Gamma_v\in \vicinity_\F(0): \Gamma_v\cdot v \subseteq A-x} \\
&= \setbuilder{x}{\forall v\in V: \exists \Gamma_v\in \vicinity_\F(0): x + \Gamma_v\cdot v \subseteq A}.
\end{align*}

(3) We calculate
\begin{align*}
\adh_\mathfrak{a}(A) &= \big(\inh_\mathfrak{a}(A^c)\big)^c \\
&= \setbuilder{x\in V}{\exists v\in V:\forall \Gamma \in \vicinity_\F(0):\; \neq(x + \Gamma\cdot v \subseteq A^c)} \\
&= \setbuilder{x\in V}{\exists v\in V:\forall \Gamma \in \vicinity_\F(0):\; (x + \Gamma\cdot v) \mesh A}.
\end{align*}
\end{proof}

\begin{lemma}
Let $V$ be a vector space. Then every subspace $U\subseteq V$ is algebraically closed.
\end{lemma}
\begin{proof}
We need to show that $\adh_\mathfrak{a}(U)\subseteq U$. Take $x\in \adh_\mathfrak{a}(U)$. Then take a $v\in V$ such that $\forall \Gamma\in\vicinity_\F(0):\; (x+\Gamma\cdot v)\mesh U$.

Pick some $\Gamma\in\vicinity_\F(0)$. Then $x+\lambda v\in U$ for some $\lambda\in \Gamma$. Then take $\ball(0,|\lambda|/2)\in \vicinity_\F(0)$, so $x+\mu v\in U$ for some $\mu\in \ball(0,|\lambda|/2)$. In particular $\lambda \neq \mu$. If either $\lambda =0$ or $\mu = 0$, then $x\in U$ and we are done. Suppose $\lambda\neq 0 \neq \mu$. Then
\[ \lambda^{-1}(x+\lambda v) - \mu^{-1}(x+\mu v) = (\lambda^{-1} - \mu^{-1})x \in U. \]
So $x\in U$.
\end{proof}

\subsubsection{The algebraic interior or core}
\begin{definition}
Let $V$ be a vector space and $A\subseteq V$ a subset. Then algebraic inherence $\inh_\mathfrak{a}(A)$ is also called the \udef{algebraic interior} or \udef{core} of $A$.
\end{definition}

\begin{proposition} \label{coreProperties}
Let $V$ be a vector space and $A \subseteq V$ a subset. Then
\begin{enumerate}
\item $A$ is absorbing \textup{if and only if} $0\in \inh_\mathfrak{a}(A)$;
\item if $A$ is convex, then $\inh_\mathfrak{a}(A)$ is convex and open.
\end{enumerate}
\end{proposition}
\begin{proof}
(1) We have that
\begin{align*}
\text{$A$ is absorbing} &\iff \forall v\in V: \exists \epsilon >0: \; \ball(0,\epsilon)\cdot v\subseteq A \\
&\iff \forall v\in V: \exists \Gamma \in \vicinity_\F(0): \; \Gamma\cdot v\subseteq A \\
&\iff 0\in \inh_\mathfrak{a}(A).
\end{align*}

(2) We first show convexity: take $x,y \in \inh_\mathfrak{a}(A)$. Then there exist relevant $v,w,\Gamma_v,\Gamma_w$ such that $x+ \Gamma_v\cdot v \subseteq A$ and $y+ \Gamma_w\cdot w \subseteq A$. By \ref{convexCriteria} we have $\lambda \big(x+ \Gamma_v\cdot v\big) + (1-\lambda)\big(y+ \Gamma_w\cdot w\big)\subseteq A$ for all $0\leq \lambda \leq 1$, so
\[ \lambda x+(1-\lambda)y + (\Gamma_v\cap\Gamma_w)\big(\lambda v+(1-\lambda)w\big) \subseteq \lambda x+(1-\lambda)y + \Gamma_v\cdot \lambda v+\Gamma_w\cdot (1-\lambda)w \subseteq A. \]

To show $\inh_\mathfrak{a}(A)$ is open, we use \ref{openClosedCriteria}. Take $x\in \inh_\mathfrak{a}(A)$. Then for all $v\in V$ we can find a $\Gamma_v\in\vicinity_\F(0)$ such that $x+\Gamma_v\cdot v \subseteq A$. This means $x+\bigcup_{v\in V}\Gamma_v\cdot v \subseteq A$. Because the convergence on $\F$ is topological, we may take the $\Gamma_v$ open. To conclude with \ref{openClosedCriteria} it is enough to show that $x+\bigcup_{v\in V}\Gamma_v\cdot v \subseteq \inh_\mathfrak{a}(A)$.

Pick some $y = x+ c_w w \in x+ \bigcup_{v\in V}\Gamma_v\cdot v \subseteq A$, meaning $c_w\in\Gamma_w$. We can find an $0<\epsilon_w<|c_w|$ such that $c_w + \ball(0,\epsilon_w) \subseteq \Gamma_w$ by \ref{openClosedCriteria}. Then for all $1-\frac{\epsilon_w}{|c_w|}<\delta<1$ we have $|c_w - \delta c_w| = (1-\delta)|c_w| < \frac{\epsilon_w}{|c_w|}|c_w| = \epsilon_w$ and so $x+ \delta c_w w \in A$.

Now pick an arbitrary $u\in V$. We have $x+\Gamma_u\cdot u \subseteq A$. By convexity we have
\[ \delta^{-1}(x+ \delta c_w w) + (1-\delta^{-1})\big(x+\Gamma_u\cdot u\big) = x + c_w w + (1-\delta^{-1})\Gamma_u\cdot u \subseteq A. \]
This means that for all $v\in V$ we have $y + (1-\delta^{-1})\Gamma_v\cdot v \subseteq A$ and thus $y\in \inh_\mathfrak{a}(A)$.
\end{proof}

\begin{proposition} \label{algebraicallyOpen}
Let $V$ be a vector space and $A \subseteq V$ an algebraically open subset. Then
\begin{enumerate}
\item $A+U$ is algebraically open for any subspace $U\subseteq V$;
\end{enumerate}
\end{proposition}
\begin{proof}
TODO
\end{proof}

\subsection{Topological vector spaces}
\begin{definition}
A \udef{topological vector space} (or TVS) is a convergence vector space that is topological.
\end{definition}
As with convergence groups, any pretopological vector space convergence is topological, see \ref{pretopologicalGroupConvergence}.

\begin{lemma} \label{continuityToNormedSpace}
Let $V$ be a TVS and $W$ a normed space. A linear function $f: V\to W$ is continuous \textup{if and only if} there exists a neighbourhood $U\in \neighbourhood_V(0)$ such that $f$ is bounded on $U$.
\end{lemma}
\begin{proof}
Assume $f$ continuous, then $\ball(0,1)\in \neighbourhood_W(0) = \neighbourhood_W(f(0))$ implies $U = f^{\preimf}(\ball(0,1)) \in \neighbourhood_V(0)$ by \ref{pretopologicalContinuityVicinities} and $f$ is bounded on $U$ by construction.

Now assume there exists a neighbourhood $U\in \neighbourhood_V(0)$ such that $f$ is bounded on $U$. Then $f^{\imf}(U)\subseteq \ball_W(0,C)$. It is then enough to show that for all $\ball_W(0,\epsilon)$, $f^{\preimf}(\ball_W(0,\epsilon)) \in \neighbourhood_V(0)$. Indeed
\[ f^{\preimf}(\ball_W(0,\epsilon)) = \frac{\epsilon}{C} f^{\preimf}(\ball_W(0,C)) = \frac{\epsilon}{C}U \in \neighbourhood_V(0). \]
\end{proof}

\subsubsection{Neighbourhoods and base}
\begin{proposition} \label{TVSconstruction}
Let $V$ be a vector space and $N\in\powerfilters(V)$. Then $N = \neighbourhood_\xi(0)$ for some topological convergence on $V$ \textup{if and only if}
\begin{enumerate}
\item for all $A\in N$ and $\lambda\in \F$: $\lambda A\in N$;
\item for all $A\in N$, there exists some $B\in N$ such that $B+B\subseteq A$;
\item each $A \in N$ is absorbent;
\item $N$ has a balanced base.
\end{enumerate}
\end{proposition}
\begin{proof}
We adapt \ref{vectorSpaceConvergenceConstruction} to the present situation.

First assume $N$ has a balanced and absorbent base. We check the five conditions for $\mathcal{F} = \pfilter{N}$.

\begin{enumerate}
\item Immediate because $\mathcal{F} = \pfilter{N}$.
\item Take $F,G\in \pfilter{N}$. We need to show that $\upset (F + G) \supseteq N$, which means that for all $A \in N$ there exist $B\in F$ and $C\in G$ such that $B+C\subseteq A$. We can take $B = C$ equal to the $B$ of point (2).
\item Take $F\in \pfilter{N}$. We need to show that $\upset(\vicinity_\F(0)\cdot F) \supseteq N$, which means that for all $A\in N$ there exists a $\Gamma\in \vicinity_\F(0)$ and $B\in F$ such that $\Gamma \cdot B\subseteq A$. We can take $B = \balancedCore(A) \in N \subseteq F$ and $\Gamma = B(0,1)$.
\item Take $v\in V$. We need to show that all $A\in N$ contain $\Gamma\cdot v$ for some $\Gamma \in \vicinity_\F(0)$. Because $A$ is absorbent, there exists an $r>0$ such that $v\in cA$ for all $|c|\geq r$. Conversely $c^{-1}v \in A$ for all $|c^{-1}| \leq r^{-1}$. So $\ball(0,r^{-1})\cdot v \subseteq A$ and $B(0,r^{-1}) \in \vicinity_\F(0)$.
\item Take $F\in \pfilter{N}$ and $\lambda\in \F$. We need to show that for all $A\in N$ there exists a $B\in F$ such that $\lambda\cdot B\subseteq A$. We can take $B = \lambda^{-1}A \in N\subseteq F$.
\end{enumerate}

Now assume $\xi$ is a topological vector space convergence and $N = \neighbourhood_\xi(0)$.
\begin{enumerate}
\item By point (5) of \ref{vectorSpaceConvergenceConstruction}, we have that for all $\lambda\in \F$, $\lambda\cdot N \supseteq N$, so for all $A\in N$, there exists a $B\in N$ such that $A = \lambda B$. This means $\lambda^{-1}A \in N$ for all $\lambda\in \F, A\in N$.
\item By \ref{vicinityFactorisation};
\item For absorbence, take $A\in N$ and $v\in V$. Then there exists a $\Gamma \in \vicinity_\F(0)$ such that $\Gamma\cdot v \subseteq A$. Now we can find a $r>0$ such that $\ball(0,r)\subseteq \Gamma$, so for all $|c|\geq r^{-1}$ we have $v\in cA$.
\item By point (3) of \ref{vectorSpaceConvergenceConstruction}, $\vicinity_\F(0)\cdot N \supseteq N$. Take $A\in N$. Then there exists a $\Gamma\in\vicinity_\F(0)$ and $B\in N$ such that $\Gamma\cdot B \subseteq A$. We can find sume ball $\ball(0,\epsilon) \subseteq \Gamma$, so $\ball(0,1)\cdot \epsilon^{-1}B\subseteq \epsilon^{-1}B \subseteq A$. Thus $\epsilon^{-1}B$ is balanced and a neighbourhood by point(1). So every $A\in N$ contains a balanced set in $N$.
\end{enumerate}
\end{proof}


\subsubsection{Locally convex convergence}
\begin{definition}
A convergence vector space $\sSet{V,\xi}$ is called \udef{locally convex} if $\xi$ is topological and based in the convex sets.
\end{definition}

\begin{lemma} \label{locallyConvexNeighbourhoodsLemma}
Let $\sSet{V,\xi}$ be a TVS. Then the following are equivalent:
\begin{enumerate}
\item $\xi$ is locally convex;
\item $\neighbourhood_\xi(0)$ is based in the convex sets;
\item $\neighbourhood_\xi(0)$ is based in the absolutely convex sets.
\end{enumerate}
\end{lemma}
\begin{proof}
$(1) \Leftrightarrow (2)$ One direction is immediate, for the other it is enough to note that if $U$ is convex, then so is the translated set $x+U$ for all $x\in V$.

$(2) \Leftrightarrow (3)$ One direction is immediate, the other follows because the balanced core of a convex set is convex by \ref{balancedCoreConvexSet}.
\end{proof}

\begin{proposition}
Let $V$ be a vector space and $N\in\powerfilters(V)$. Then $N = \neighbourhood_\xi(0)$ for some topological convergence on $V$ \textup{if and only if}
\begin{enumerate}
\item for all $A\in N$ and $\lambda\in \F$: $\lambda A\in N$;
\item each $A \in N$ is absorbent;
\item $N$ has an absolutely convex base.
\end{enumerate}
\end{proposition}
\begin{proof}
This almost completely follows from \ref{TVSconstruction} and \ref{locallyConvexNeighbourhoodsLemma}. We just need to show that for all $A\in N$, there exists some $B\in N$ such that $B+B\subseteq A$. We may take $B = \frac{1}{2}A'$, where $A'$ is a convex subset of $A$, because for all $v,w\in A'$ we have $\frac{1}{2}v + \frac{1}{2}w \in A'$ by convexity.
\end{proof}


\section{Functionals}
\begin{definition}
Let $V$ be a vector space over a field $\mathbb{F}$.
\begin{enumerate}
\item A \udef{functional} on $V$ is a map $V\to \F$;
\item A \udef{linear functional} on $V$ is a linear map from $V$ to $\mathbb{F}$;
\item A \udef{real functional} on $V$ is a map $V\to \R$.
\end{enumerate}
\end{definition}

\begin{lemma} \label{continuityDominatedFunctional}
Let $V$ be a TVS and $f:V\to \F$ a continuous functional. If $g:V\to \F$ is a functional such that $|g(v)|\leq |f(v)|$ for all $v\in V$, then $g$ is continuous.
\end{lemma}
\begin{proof}
We use \ref{pretopologicalContinuityVicinities} to show continuity. To that end take $K\in \neighbourhood_\F(0)$. Then there exists $\epsilon >0$ such that $\ball(0,\epsilon)\subseteq K$ and so
\[ g^{\preimf}(K) \supseteq g^\preimf[\ball(0,\epsilon)] \supseteq f^\preimf[\ball(0,\epsilon)] \in \neighbourhood_V(0). \]
\end{proof}

\subsection{Linear functionals}
\begin{lemma} \label{kernelHyperplane}
Let $V$ be a vector space and $U\subseteq V$ a subspace. Then $U$ is a hyperplane \textup{if and only if} it is the kernel of a linear functional.
\end{lemma}

\begin{lemma} \label{functionalBoundedNeighbourhood}
Let $f: V\to \F$ be a linear functional and $x\notin \ker(f)$. Let $A\subseteq V$ be a balanced set. Then $(x+A)\perp \ker(f)$ \textup{if and only if} $A \subseteq f^{\preimf}(\ball(0,|f(x)|))$.
\end{lemma}
\begin{proof}
Suppose $A \subseteq f^{\preimf}(\ball(0,|f(x)|))$. Then for all $a\in A$: $f(x+a) = f(x) + f(a) \neq 0$.

Conversely, suppose $A \not\subseteq f^{\preimf}(\ball(0,|f(x)|))$, i.e.\ there exists $a\in A$ such that $|f(a)| \geq |f(x)|$. Then $v= -\frac{f(x)}{f(a)}a\in A$, because $A$ is balanced and so $f(x+ v) = f(x)-\frac{f(x)}{f(a)}f(a) = 0$ and so $(x+A) \mesh \ker(f)$.
\end{proof}

\begin{proposition} \label{linearFunctionalOpen}
Let $V$ be a convergence vector space and $f:V\to \F$ a non-zero linear functional. Then $f$ is an open map.
\end{proposition}
\begin{proof}
It is enough to show $f$ is open when $V$ is equipped with the algebraic convergence.

Let $A$ be an algebraically open map. We use \ref{openClosedCriteria} to show $f^\imf[A]$ is also open. Because $f$ is non-zero, there exists a $v\in V$ such that $f(v) \neq 0$. Take some $y\in f^\imf[A]$. Then there exists an $x\in A$ such that $f(x) = y$. Because $A$ is open, $x\in \inh_\mathfrak{x}(A)$ and there exists $x+ \Gamma_v\in \vicinity_\F(0)$ such that $x+\Gamma_v\cdot v \subseteq A$ by \ref{constructionsInAlebraicConvergence}.

Now $f^\imf[x+ \Gamma_v\in \vicinity_\F(0)] = y+\Gamma_v \cdot f(v) \subseteq f^\imf[A]$ and $y+\Gamma_v \cdot f(v)$ is a vicinity of $y$, so we are done.
\end{proof}

\begin{lemma} \label{complexRangeExtensionRealFunctional}
Let $V$ be a complex vector space and $g: V_\R\to \R$ a linear functional. Then there exists a unique linear functional $f: V\to \C$ such that $g = \Re(f)$.
\end{lemma}
\begin{proof}
We can write $f = g + ih$ for some function $h: V\to \R$. Then for all $x\in V$
\[ g(ix)+ih(ix) = f(ix) = if(x) = ig(x) - h(x). \]
Comparing real parts gives $h(x) = - g(ix)$. So $f$ must be given by $f(x) = g(x) - ig(ix)$. Clearly $f$ is real-linear. We just need to verify that this makes $f$ complex-linear. Indeed, take $\lambda = a +ib \in \C = \R+i\R$ arbitrarily. Then for all $v\in V$
\begin{align*}
f(\lambda v) &= f\big((a+ib)v\big) \\
&= af(v) + bf(iv) \\
&= af(v) + b\big(g(iv) - ig(i^2v)\big) \\
&= af(v) + b\big(g(iv) + ig(v)\big) \\
&= af(v) + ib\big(-ig(iv) + g(v)\big) \\
&= af(v) + ibf(v) = (a+ib)f(v) = \lambda f(v).
\end{align*}
\end{proof}

\begin{lemma}
Let $V$ be a vector space and $f_0,\ldots, f_n, f$ linear functionals in $(V\to \F)$. Then a linear funcctional $f: V\to \F$ is a linear combination of $f_0,\ldots, f_n$ \textup{if and only if} .
\end{lemma}
\begin{proof}
The direction $\Rightarrow$ is immediate: if $f_i(v) = 0$ for all $0\leq i\leq n$, then any linear combination is also zero.

Now assume 
\end{proof}

\begin{lemma} \label{linearDependenceLinearFunctionals}
Let $V$ be a vector space and $f_0,\ldots, f_n, f$ linear functionals in $(V\to \F)$. Then the following are equivalent:
\begin{enumerate}
\item $f$ is a linear combination of $f_0,\ldots, f_n$;
\item there exists a $C>0$ such that $f(v) \leq C \max_{0\leq i\leq n}|f_i(v)|$;
\item $\ker(f) \supseteq \bigcap_{0\leq i \leq n}\ker(f_i)$;
\end{enumerate}
\end{lemma}
\begin{proof}
The implications $(1) \Rightarrow (2) \Rightarrow (3)$ are clear.

Now assume $(3)$ holds and consider the linear function $f_0\times \ldots \times f_n: V\to \F^{n+1}$. Because of (3), the functional
\[ F: \im(f_0\times \ldots \times f_n) \to \F: x\mapsto \sunp\circ f^\imf\circ (f_0\times \ldots \times f_n)^\preimf(\{x\}) \]
is well-defined. Extend $F$ to the whole of $\F^{n+1}$. By \ref{ellIsomorphism} we can represent this as:
\[ F: \F^{n+1}\to \F: (u_0, \ldots, u_n)\mapsto \sum_{i=0}^n \alpha_iu_i. \]
From $f(v) = F\big((f_0\times \ldots \times f_n)(v)\big)$, we get
\[ f(v) = \sum_{i=0}^n \alpha_i f_i(v) \qquad \forall v\in V. \]
So $f = \sum_{i=0}^n \alpha_i f_i$.
\end{proof}
\begin{corollary}
Let $V$ be a vector space and $f_0,\ldots, f_n$ linearly independent linear functionals in $(V\to \F)$. Then there exist $v_0, \ldots, v_n$ such that $f_i(v_j) = \delta_{i,j}$.
\end{corollary}
\begin{proof}
The proof is by induction. The case $n=1$ is clear: if there was no such $a_1$, then $f_1$ would be zero and thus not linearly independent.

Suppose the statement holds for $n-1$ and take $f_0,\ldots, f_n$ linearly independent linear functionals with corresponding $v_0,\ldots, v_{n-1}$. By point (3) of the proposition we can find $v_n \in \bigcap_{0\leq i \leq n}\ker(f_i)\setminus \ker(f_n)$, which after rescaling can be taken to be such that $f_n(v_n) = 1$. By construction $f_i(v_n) = 0$ for $i<n$.

Now replace $v_i$ with $v_i-f_n(v_i)v_n$ and rescale.
\end{proof}

\subsection{The dual space}
\begin{definition}
Let $\sSet{V,\xi}$ be a convergence vector space over a field $\mathbb{F}$.

The \udef{dual} of $V$ is the vector space of all continuous linear functionals on $V$.

The dual is denoted $\sSet{V,\xi}^{*}$ (or just $V^*$ is the convergence is clear from the context).
\end{definition}

\begin{proposition} \label{continuityLinearFunctionals}
Let $V$ be a TVS and $f:V\to \F$ a linear functional on $V$. Then the following are equivalent:
\begin{enumerate}
\item $f\in V^{*}$, i.e.\ $f$ is continuous;
\item there exists a neighbourhood $U\in \neighbourhood_V(0)$ such that $f$ is bounded on $U$
\item $\ker(f)$ is closed;
\item $\ker(f)$ is not dense.
\end{enumerate}
\end{proposition}
\begin{proof}
$(1) \Leftrightarrow (2)$ By \ref{continuityToNormedSpace}.

$(1) \Rightarrow (3)$ Because $\ker(f) = f^{\preimf}(\{0\})$ and $\{0\}$ is closed in $\F$, $\ker(f)$ is closed by \ref{continuity}.

$(3) \Rightarrow (1)$ Now assume $\ker(f)$ closed. If $\ker(f) = V$, then $f$ is constant and thus continuous by \ref{continuityConstructions}. If $\ker(f) \neq V$, we can find some some $x\in \ker(f)^c$, which is open. Thus $\ker(f)^c - x$ is a neighbourhood of the origin, meaning we can take a balanced subset $A$ by \ref{TVSconstruction}. Now $(x+A)\perp \ker(f)$ by construction, so $f$ is bounded on $A$ by \ref{functionalBoundedNeighbourhood} and thus $f$ is continuous by \ref{continuityToNormedSpace}.

$(3) \Leftrightarrow (4)$ By \ref{kernelHyperplane} $\ker(f)$ is a hyperplane and by \ref{hyperplaneClosedDense} this hyperplane is either closed or dense.
\end{proof}


\begin{lemma} Let $X$ be a normed space and
let $x\in X$ $\omega\in \tdual{X}$ be a bounded linear functional. Then
\begin{align*}
\norm{\omega} &= \sup\setbuilder{|\omega(v)|}{\norm{v}=1 } \\
&= \sup\setbuilder{\frac{|\omega(v)|}{\norm{v}}}{v\neq 0} \\
&= \inf\setbuilder{c>0} {|\omega(v)|\leq c\norm{v}\forall v\in X}
\end{align*}
and
\begin{align*}
\norm{x} &= \sup\setbuilder{|\varphi(x)|}{ \norm{\varphi}=1} \qquad\qquad\quad\\ %TODO: fragile spacing!
&= \sup\setbuilder{\frac{|\varphi(x)|}{\norm{\varphi}}}{\varphi\neq 0}.
\end{align*}
\end{lemma}
TODO move??
\begin{proof}
We prove the third equality. Let $\alpha$ be the infimum. Let $\epsilon>0$, then by the definition $|\omega[(\norm{x}+\epsilon)^{-1}x]|\leq \norm{\omega}$. Hence $|\omega(x)|\leq \norm{\omega}(\norm{x}+\epsilon)$. Letting $\epsilon\to 0$ gives $|\omega(x)|\leq \norm{\omega}\norm{x}$ for all $x$. So $\alpha\leq \norm{\omega}$. On the other hand, $|\omega(x)|\leq c$ for all $x$ with $\norm{x}=1$. Hence $\norm{\omega}\leq \alpha$.
\end{proof}

\subsubsection{The algebraic dual}
\begin{definition}
Let $V$ be a vector space. The \udef{algebraic dual} of $V$ is the dual of $\sSet{V,\mathfrak{a}}$, where $\mathfrak{a}$ is the algebraic convergence.

If no convergence on $V$ has been mentioned, then $V^*$ means the algebraic dual.
\end{definition}

\begin{proposition} \label{algebraicDual}
Let $V$ be a vector space. Then the algebraic dual of $V$ is the set of all linear functionals: $V^* = \Lin(V,\F)$.

Thus $V^* \supseteq \sSet{V,\xi}^*$ for all vector space convergences $\xi$ on $V$.
\end{proposition}
\begin{proof}
We need to show that all linear functionals are continuous when $V$ is equipped with the algebraic convergence. Assume $F\overset{\mathfrak{a}}{\longrightarrow} x$. Then there exists a $v\in V$ such that $\vicinity_\F(0)\cdot v+x \subseteq F$ and so $\vicinity_\F(0)\cdot f(v)+f(x) \subseteq f^\imf[F]$, meaning $f^\imf[F] \overset{\F}{\longrightarrow} f(x)$. Thus $f$ is continuous.
\end{proof}


\begin{proposition} \label{dualBasisDimension}
Let $V$ be a vector space. Then $\dim V^* \geq \dim V$ and
\[ \dim V^* = \dim V \iff \text{$V$ is finite-dimensional}. \]
If $V$ is finite-dimensional with a basis $v_1, \ldots, v_n$, then the \udef{dual basis} $\varphi_1, \ldots, \varphi_n$ is the set of linear functionals on $V$ such that
\[ \varphi_j(v_k) = \begin{cases}
1 & (k=j), \\ 0 & (k\neq j)
\end{cases}. \]
This dual basis is indeed a basis of $V^*$.
\end{proposition}
\begin{proof}
We first assume $V$ is finite-dimensional and prove the dual basis is a basis, which proves $\dim V^* = \dim V$. We then assume $V$ is infinite-dimensional and prove $\dim V^* \neq \dim V$.\footnote{Reference: \url{https://mathoverflow.net/questions/13322/slick-proof-a-vector-space-has-the-same-dimension-as-its-dual-if-and-only-if-i}}
\begin{enumerate}
\item Assume $V$ is finite-dimensional. To show the dual basis spans $V^*$, take a linear functional $\varphi$. Now define $a_i = \varphi(v_i)$. It is clear that $\varphi = \sum_{i=1}^n a_i\varphi_i$. To show linear independence, take a combination
\[ b_1\varphi_1 + \ldots +b_n\varphi_n =0. \]
Filling in all basis vectors $v_i$ in turn, gives $b_i=0$ for all $i$.
\item Assume $V$ is infinite-dimensional. At first let us assume $\dim_{\mathbb{F}}V \geq |\mathbb{F}|$. Then we can apply lemma \ref{vsCardinality} to obtain $\dim_{\mathbb{F}}V = |V|$. Let $\beta$ be a basis for $V$. The elements of $V^*$ correspond bijectively to functions from $\beta$ to $\mathbb{F}$. Thus
\[ |V^*| = |\mathbb{F}^\beta| = |\mathbb{F}|^{|\beta|} > |\beta| = |V|. \]
Now we relax the condition $\dim_{\mathbb{F}}V \geq |\mathbb{F}|$. We first note that every field contains a subfield that is at most denumerable. Take such a field $K\subset \mathbb{F}$. We introduce the new vector space $W = \Span_K(\beta)$. Every functional from $W$ to $K$ extends to a functional from $V$ to $\mathbb{F}$. Hence
\[ \dim_\mathbb{F} V = \dim_K W < \dim_K W^* \leq \dim_{\mathbb{F}} V^* \]
using $\dim_{K}W \geq |K| \geq \aleph_0$.
\end{enumerate}
\end{proof}
\begin{corollary}
Let $\sSet{V,\xi}$ be a convergence vector space. If $V$ is finite-dimensional, then $\sSet{V,\xi}^* = \sSet{V,\mathfrak{a}}^*$.
\end{corollary}
\begin{proof}
We have $\sSet{V,\xi}^* \subseteq \sSet{V,\mathfrak{a}}^*$ by \ref{algebraicDual}. Because $V$ is finite-dimensional, we obtain equality equality of space from equality of dimension by \ref{vectorSpaceEquality}.
\end{proof}

\subsubsection{The bidual space}
TODO!
\begin{definition}
Let $V$ be a convergence vector space. The \udef{bidual space} is the dual of the dual $\abidual{V} = \adual{(\adual{V})}$.
\end{definition}
TODO continuous convergence!!

\begin{definition}
Let $V$ be a vector space over $\mathbb{F}$ and $v\in V$. The \udef{evaluation map} $\evalMap: V\to \abidual{V}: v\mapsto \evalMap_v$ is given by
\[ \evalMap_v: \adual{V} \to \mathbb{F}: l\mapsto l(v). \]
\end{definition}

\begin{lemma}
Let $V$ be a vector space. The evaluation map $\evalMap: V\to \abidual{V}: v\mapsto \evalMap_v$ is linear:
\[ \forall v,w\in V, a\in\mathbb{F}: \quad \evalMap_{av + w} = a\evalMap_v + \evalMap_w. \]
\end{lemma}
\begin{lemma}
Let $V$ be vector space over $\mathbb{F}$. The evaluation map is injective.
\end{lemma}
\begin{proof}
Assume $\evalMap_v = \evalMap_w$ for some $v,w\in V$. Then
\[ 0 = \evalMap_v - \evalMap_w  = \evalMap_{v-w}. \]
So $\forall l\in \adual{V}: \evalMap_{v-w}(l) = l(v-w) = 0$. Now define the sublinear functional by
\[ p(x) = \begin{cases}
\alpha & x = \alpha(v-w) \\
0 & \text{else}.
\end{cases} \]
Then the functional $f$ defined on $\Span\{v-w\}$ by $f(\alpha(v-w)) = \alpha$ is bounded by $p$ and can be extended to a functional on all $V$ by the Hahn-Banach theorem \ref{sublinearHahnBanach} if $v-w\neq 0$. Then $f(v-w) \neq 0$, which contradicts our assumptions. Thus $v=w$.
\end{proof}

\begin{proposition}
The mapping $\evalMap: V\to \abidual{V}: v\mapsto \evalMap_v$ is an isomorphism \textup{if and only if} $V$ is finite-dimensional.
\end{proposition}
\begin{proof}
Assume $V$ finite dimensional. As the evaluation map is injective, it is an isomorphism by \ref{invertibleFiniteDim}.
The other direction is a dimensional argument by proposition \ref{dualBasisDimension}.
\end{proof}


Just like for algebraic duality, we can define a topological bidual space (or second dual space) $\tbidual{V}$.

\begin{proposition}
Let $V$ be a normed space. 
For each $v\in V$
\[ \evalMap_v: \tdual{V} \to \mathbb{F}: \omega \mapsto \omega(v) \]
is bounded and thus an element of $\tbidual{V}$.

The evaluation map $\evalMap: V \to \tbidual{V}$ is
\begin{enumerate}
\item isometric (and thus injective): $\norm{\evalMap_v} = \norm{v}$;
\item bounded with norm $\norm{\evalMap} = 1$.
\end{enumerate}
\end{proposition}
\begin{proof}
Let $v\in V$. Then
\[ \norm{\evalMap_v} = \sup\setbuilder{\norm{\evalMap_v(\omega)}}{\norm{\omega}=1} = \sup\setbuilder{\norm{\omega(v)}}{\norm{\omega}=1} \leq \sup\setbuilder{\norm{v}\;\norm{\omega}}{\norm{\omega}=1} = \norm{v}. \]

(1) Setting $\omega = \inner{v/\norm{v}, \cdot}$, we get
\[ \norm{\evalMap_v} \leq |\evalMap_v(\omega)| = |\inner{v/\norm{v}, v}| = \norm{v}. \]
Together with the calculation above, this gives $\norm{\evalMap_v} = \norm{v}$.

(2) $\norm{\evalMap} = \sup\setbuilder{\norm{\evalMap_v}}{\norm{v}=1} = \sup\setbuilder{\norm{v}}{\norm{v}=1} = 1$.
\end{proof}

\begin{lemma}
Let $V$ be normed space over $\mathbb{F}$ and $v\in V$. For each $v\in V$
\[ \evalMap_v: \tdual{V} \to \mathbb{F}: \omega \mapsto \omega(v) \]
is bounded with norm $\norm{v}$ and thus $\evalMap\in \tbidual{V}$ with $\norm{\evalMap} = 1$.
\end{lemma}


\subsubsection{Reflexive spaces}
\begin{definition}
A normed space $V$ is \udef{reflexive} if the evaluation map $\evalMap:V\to \tbidual{V}$ is surjective:
\[ \im\evalMap = \tbidual{V}. \]
\end{definition}
If $V$ is reflexive, then $\tbidual{V}$ is isometrically isomorphic to $V$. The converse is not necessarily true.

\begin{lemma}
Every finite-dimensional space is reflexive.
\end{lemma}

\begin{proposition}
A separable normed space $X$ with a non-separable dual space $\tdual{X}$ cannot be reflexive. 
\end{proposition}
\begin{proof}
TODO
\end{proof}
Thus $l^1$ is not reflexive.

\begin{proposition}
If the dual space $\tdual{X}$ of a  normed space $X$ is separable, then $X$ itself is separable. 
\end{proposition}
\begin{proof}
TODO
\end{proof}

\subsubsection{Transposition}
\begin{definition}
Let $f:V\to W \in \Hom_{\mathbb{F}}(V,W)$. The \udef{dual map}\footnote{The dual map $f^t$ is often denoted $f^*$ or $f'$. We avoid this because it clashes with the notation of the Hilbert adjoint.} or \udef{transpose} $f^t$ is the linear map
\[ f^t:W^* \to V^*: l\mapsto f^t(l) = l\circ f. \]
\end{definition}
\begin{lemma}
Let $f\in \Hom(U,V)$ and $g\in \Hom(V,W)$.
\begin{itemize}
\item $(g\circ f)^t = f^t\circ g^t$;
\item $\id^t_V = \id_{\adual{V}}$;
\item $f$ is an isomorphism \textup{if and only if} $f^t$ is an isomorphism;
\item $(f^t)^{-1} = (f^{-1})^t$ 
\end{itemize}
\end{lemma}
TODO: merge
\begin{lemma}
Let $S,T\in\Hom(V,W)$ and $\alpha\in\mathbb{F}$. Then
\begin{enumerate}
\item $(S+T)^t = S^t+T^t$;
\item $(\alpha T)^t = \alpha T^t$
\item if $T$ is invertible, then $T^t$ is invertible and
\[ (T^t)^{-1} = (T^{-1})^t. \]
\end{enumerate}
\end{lemma}

\begin{proposition}
Let $U\subset V$ be a subspace and $T\in \Hom(V,W)$, where $V,W$ are \emph{finite-dimensional}.
\begin{enumerate}
\item $\dim\ker T^t = \dim \ker T + \dim W - \dim V$;
\item $\dim\im T^t = \dim \im T$;
\item $\im T^t = (\ker T)^0$
\item $T$ is injective \textup{if and only if} $T^t$ is surjective.
\end{enumerate}
\end{proposition}
\begin{proof}
\mbox{}
\begin{enumerate}
\item Using $\dim \adual{V} = \dim V$, we have
\begin{align*}
\dim \ker T^t &= \dim(\im T)^0 = \dim W-\dim \im T \\
&= \dim W - (\dim V - \dim \ker T) = \dim \ker T + \dim W - \dim V
\end{align*}
where the equalities come from proposition \ref{annihilatorSpace} and the dimension theorem for linear maps, theorem \ref{dimensionLinearMaps}.
\item Still using these results, we can calculate
\begin{align*}
\dim \im T^t &= \dim \adual{W} - \dim \ker T^t = \dim \adual{W} - \dim (\im T)^0 \\
&= \dim \adual{(\im T)} = \dim \im T.
\end{align*}
\item Take $\varphi = T^t(\psi) \in \im T^t$ where $\psi \in \adual{W}$. If $v\in \ker T$, then
\[ \varphi(v) = \left(T^t(\psi)\right)v = (\psi\circ T)(v) = \psi(Tv) = \psi(0) = 0. \]
Hence $\varphi \in (\ker T)^0$ and $\im T^t\subset (\ker T)^0$. We prove the equality by showing the dimensions are the same. Indeed:
\[ \dim \im T^t = \dim \im T = \dim V - \dim \ker T = \dim(\ker T)^0. \]
\item $T\in\Hom(V,W)$ is injective iff $\ker T = \{0\}$ iff $(\ker T)^0 = \adual{V}$ iff $\im T^t = \adual{V}$ iff $T^t$ is surjective.
\end{enumerate}
\end{proof}

\begin{proposition} \label{transpDual}
Let $f\in\Hom(V,W)$ and $\mathcal{V}, \mathcal{W}$ bases of $V,W$. The
\[ (f^*)^{\mathcal{V}^*}_{\mathcal{W}^*} = ((f)^{\mathcal{W}}_{\mathcal{V}})^\transp. \] 
\end{proposition}

\begin{definition}
Let $T\in\Bounded(V,W)$. The dual map $T^t: \tdual{W}\to \tdual{V}$ is called the \udef{adjoint} or the \udef{transpose} of $T$.
\end{definition}
The notation $T^t$ is consistent for maps on both the algebraic and topological duals: if $T$ is bounded, $T^t:\adual{W}\to \adual{V}$ restricts to $T^t|_{\tdual{W}} = T^t:\tdual{W}\to \tdual{V}$.

\begin{proposition}
Let $T\in\Bounded(V,W)$. Then the transpose $T^t$ is a bounded operator in $\Bounded(W,V)$ with $\norm{T^t} = \norm{T}$.
\end{proposition}
\begin{proof}
The operator $T^t$ is linear since $\forall f_1,f_2\in \tdual{W}, \forall a\in\mathbb{F}, \forall x\in V:$
\[ (T^t(af_1 + f_2))(x) = (af_1 + f_2)(Tx) = af_1(Tx) + f_2(Tx) = a(T^tf_1)(x) + (T^tf_2)(x). \]
For the equality of norms, we prove two inequalities. First $\forall x\in V, f\in \tdual{W}$
\[ |f(Tx)|\leq \norm{f}\norm{Tx}\leq \norm{f}\norm{x}\norm{T} \implies \frac{|f(Tx)|}{\norm{x}} \leq \norm{f}\norm{T}. \]
taking the supremum over $x\in V$, we get $\norm{T^tf} = \norm{f\circ T}\leq \norm{f}\norm{T}$ and taking the supremum over $f\in \tdual{W}$ gives $\norm{T^t}\leq \norm{T}$. This shows that $T^t$ is bounded.

For the other inequality, we use corollary \ref{existenceBoundedFunctionalOfSameNorm} to the Hahn-Banach theorem: for every $x\in V$, there exists a bounded functional $\omega_x$ such that $\norm{\omega_x}=1$ and $\omega_x(x) = \norm{x}$. Then we can calculate:
\begin{align*}
\norm{Tx} = \omega_{Tx}(Tx) = (T^t\omega_{Tx})(x) \leq \norm{T^t\omega_{Tx}}\norm{x} \leq \norm{T^t}\norm{\omega_{Tx}}\norm{x} = \norm{T^t}\norm{x}
\end{align*}
So $\norm{T}\leq\norm{T^t}$. Combining gives $\norm{T^t}=\norm{T}$.
\end{proof}
\begin{corollary}
The map $T\mapsto T^t$ is an isometric isomorphism in $(\Bounded(X,Y)\to \Bounded(\tdual{Y}, \tdual{X}))$.
\end{corollary}

\begin{lemma}
Let $S,T\in\Bounded(V,W)$ and $\alpha\in\mathbb{F}$. Then
\begin{enumerate}
\item $(S+T)^t = S^t+T^t$;
\item $(\alpha T)^t = \alpha T^t$
\item if $T$ is invertible, then $T^t$ is invertible and
\[ (T^t)^{-1} = (T^{-1})^t. \]
\end{enumerate}
Let $T\in\Bounded(U,V)$ and $S\in\Bounded(V,W)$. Then
\begin{enumerate}
\setcounter{enumi}{3}
\item $(ST)^t = T^tS^t$
\end{enumerate}
\end{lemma}


\subsection{Annihilator subspace}
\begin{definition}
Let $U\subset V$ be a subspace. The \udef{annihilator} of $U$, denoted $U^0$, is the set of functionals that are identically zero on $U$:
\[ U^0 = \left\{ \varphi\in V^*\;|\; \forall u\in U:\varphi(u) = 0 \right\}. \]
\end{definition}
\begin{proposition} \label{annihilatorSpace}
Let $U\subset V$ be a subspace and $T\in \Hom(V,W)$.
\begin{enumerate}
\item $U^0$ is a subspace of $\adual{V}$;
\item $\dim \adual{U} + \dim U^0 = \dim \adual{V}$;
\item $\ker T^t = (\im T)^0$
\item $T$ is surjective \textup{if and only if} $T^t$ is injective.
\end{enumerate}
\end{proposition}
\begin{proof}
\mbox{}
\begin{enumerate}
\item Elementary application of subspace criterion, proposition \ref{subspaceCriterion}.
\item Consider the inclusion $\iota: U\hookrightarrow V$. Then the dimension theorem \ref{dimensionLinearMaps} applied to $\iota'$ gives
\[ \dim \im \iota' + \dim \ker\iota' = \dim V^*. \]
Now $\dim \ker\iota'$ are $\varphi\in V^*$ such that $\varphi \circ \iota = 0$. These are exactly the elements of the annihilator. Any functional on $U$ can be extended to a functional on $V$, so $\iota'$ is surjective and $\dim \im \iota' = \dim U^*$.
\item There are two inclusions. First assume $\varphi \in \ker T'$, so $\forall v\in V$
\[ 0 = (\varphi\circ T)(v) = \varphi(Tv). \]
Thus $\varphi\in(\im T)^0$. The other inclusion uses the same equality.
\item $T\in\Hom(V,W)$ is surjective iff $\im T = W$ iff $(\im T)^0 = \{0\}$ iff $\ker T' = \{0\}$ iff $T'$ is injective.
\end{enumerate}
\end{proof}



\subsection{Real functionals}
\begin{definition}
Let $V$ be a real or complex vector space. Let $f: V\to \R$ be a real functional. We say
\begin{itemize}
\item $f$ is \udef{subadditive} or satisfies the \udef{triangle inequality} if $\forall x,y\in V: f(x+y) \leq f(x) + f(y)$;
\item $f$ is \udef{convex} if $\forall x,y\in V, \lambda\in[0,1]: f(\lambda x + (1-\lambda)y) \leq \lambda f(x) + (1-\lambda)f(y)$;
\item $f$ is \udef{positively homogeneous} if $\forall x\in V,\lambda\geq 0: f(\lambda x) = \lambda f(x)$;
\item $f$ is \udef{absolutely homogeneous} if $\forall x\in V,\lambda\in\F: f(\lambda x) = |\lambda| f(x)$;
\item $f$ \udef{separates points} if $\forall v\in V: f(v) = 0 \implies v = 0$.
\end{itemize}
We call $f$
\begin{itemize}
\item \udef{sublinear} if it is subadditive and positively homogeneous;
\item a \udef{seminorm} if it is subadditive and absolutely homogeneous;
\item a \udef{norm} if it is a point-separating seminorm.
\end{itemize}
\end{definition}
TODO general valued fields.

\begin{lemma}
Let $V$ be a real or complex vector space and $f: V\to \R$ be a real functional. Then
\begin{enumerate}
\item absolute homogeneity $\implies$ positive homogeneity;
\item subadditivity+positive homogeneity $\implies$ convexity $\implies$ subadditivity.
\end{enumerate}
\end{lemma}
Thus norms and seminorms are sublinear.

\begin{lemma}
A subadditive, absolutely homogenous function $f:V\to \R$ is non-negative:
\[ f: V\to \R_{\geq 0}. \]
Thus norms and seminorms are functions $V\to \R_{\geq 0}$.
\end{lemma}
\begin{proof}
For all $v\in V$ we have $0 = f(v-v) \leq f(v)+f(-v) = 2f(v)$, so $f(v) \geq 0$.
\end{proof}

\begin{proposition}[Reverse triangle inequality] \label{reverseTriangleInequality}
Let $V$ be a vector space and $\norm{\cdot}: V\to \R$ a function that satisfies the triangle inequality and has $\norm{-v} = \norm{v}$ for all $v\in V$. Then $\forall v,w\in V$:
\begin{enumerate}
\item $|\norm{v}-\norm{w}|\leq \norm{v-w}$;
\item $|\norm{v}-\norm{w}|\leq \norm{v+w}$.
\end{enumerate}
In particular this holds if $\norm{\cdot}$ is a norm or seminorm.
\end{proposition}
\begin{proof}
We calculate $\norm{v} = \norm{v-w+w} \leq \norm{v-w} + \norm{w}$, so $\norm{v}-\norm{w}\leq \norm{v-w}$. By swapping $v\leftrightarrow w$ we also get $-\norm{v}+\norm{w}\leq \norm{w-v} = \norm{v-w}$ and thus the first inequality is established.

For the second inequality, set $w\to -w$ and use $\norm{-w} = \norm{w}$.
\end{proof}

\subsubsection{Epigraphs}
\begin{definition}
Let $V$ be a vector space and $f: V\to \R$ a real functional on $V$. Then \udef{epigraph} of $f$ is defined as
\[ \epigraph(f) \defeq \setbuilder{(v,r)\in V\times \R}{f(v)\leq r}. \]
\end{definition}

\begin{lemma} \label{epigraphLemma}
Let $V$ be a vector space and $f: V\to \R$ a real functional on $V$. Then for all $v\in V$:
\[ f(v) = \inf\setbuilder{r}{(v,r)\in \epigraph(f)}. \]
\end{lemma}

\begin{proposition}
Let $V$ be a real vector space and $f: V\to \R$ a functional. Then
\begin{enumerate}
\item $f$ is convex \textup{if and only if} $\epigraph(f)$ is a convex subset of $V\oplus \R$;
\item $f$ is positively homogeneous \textup{if and only if} $\epigraph(f)$ is a cone in $V\oplus \R$.
\end{enumerate}
\end{proposition}
\begin{proof}
(1) First assume $f$ convex and pick $(v, s), (w,t)\in \epigraph(f)$ and $\lambda\in [0,1]$. Then we need to show that $(\lambda v + (1-\lambda)w, \lambda s + (1-\lambda)t) \in \epigraph(f)$. This is equivalent to saying $f(\lambda v + (1-\lambda)w) \leq \lambda s + (1-\lambda)t$. Indeed we have $f(\lambda v + (1-\lambda)w) \leq \lambda f(v) + (1-\lambda)f(w) \leq \lambda s + (1-\lambda)t$ by the convexity of $f$.

Conversely, assume $\epigraph(f)$ convex. Then $(v, f(v)), (w,f(w))\in \epigraph(f)$, $(\lambda v + (1-\lambda)w, \lambda f(v) + (1-\lambda)f(w)) \in \epigraph(f)$ for all $\lambda\in [0,1]$. This implies $f(\lambda v + (1-\lambda)w) \leq \lambda f(v) + (1-\lambda)f(w)$.

(2) First assume $f$ is positively homogeneous, take $(v,s)\in \epigraph(f)$ and $r>0$. Then we need to show that $r(v,s) = (rv,rs)\in \epigraph(f)$. This follows because of the implications $f(v)\leq s \implies rf(v) \leq rs \implies f(rv) \leq rs$.

Conversely, assume that $\epigraph(f)$ is a cone. Then $\lambda\cdot \epigraph(f) = \epigraph(f)$ for all $\lambda>0$ by \ref{coneEqualityLemma}. We then calculate using \ref{epigraphLemma}:
\begin{align*}
f(\lambda v) &= \inf\setbuilder{r}{(\lambda v,r)\in \epigraph(f)} \\
&= \inf\setbuilder{r}{(\lambda v,r)\in \lambda\cdot\epigraph(f)} \\
&= \inf\setbuilder{r}{\lambda(v,\lambda^{-1}r)\in \lambda\cdot\epigraph(f)} \\
&= \inf\setbuilder{r}{(v,\lambda^{-1}r)\in \epigraph(f)} \\
&= \inf\setbuilder{\lambda r}{(v,r)\in \epigraph(f)} = \lambda f(v).
\end{align*} 
\end{proof}
\begin{corollary}
A functional on a real vector space is sublinear \textup{if and only if} its epigraph is a convex cone.
\end{corollary}

\subsubsection{Convex functionals}

\begin{proposition}
Let $p: V\to\R$ be convex functional. Then
\[ P: V\to\R: x\mapsto \inf_{t>0} t^{-1}p(tx) \]
is sublinear and $P(x)\leq p(x)$.

Also, if $f:V\to \R$ is a linear functional, then $f\leq p \iff f\leq P$.
\end{proposition}
\begin{proof}
For sublinearity: let $x,y\in V$, then for all $s,t>0$
\[ P(x+y) \leq \frac{s+t}{st}p\left(\frac{st}{s+t}(x+y)\right) = \frac{s+t}{st}p\left(\frac{s}{s+t}(tx)+\frac{t}{s+t}(sy)\right) \leq t^{-1}p(tx) + s^{-1}p(sy). \]
This implies that $P(x+y)\leq P(x)+P(y)$.

For positive homogeneity: let $x\in V,\lambda\geq 0$
\[ P(\lambda x) = \inf_{t>0} t^{-1}p(t\lambda x) = \inf_{t\lambda>0} \lambda (t\lambda)^{-1}p(t\lambda x) = \inf_{t>0} \lambda (t)^{-1}p(tx) = \lambda P(x). \]

Finally we prove that $f\leq p \implies f\leq P$ for linear functionals $f$. For all $t>0$ we have $f(tx) \leq p(tx)$, which implies $f(x) = t^{-1}f(tx) \leq t^{-1}p(tx) \leq P(x)$. So $f\leq P$.
\end{proof}

\subsubsection{Seminorms}
\begin{lemma}
The kernel of a seminorm is a vector space.
\end{lemma}
Note this does not follow from \ref{kernelSubspace} because seminorms are not linear.
\begin{proof}
Let $p:V\to \R$ be a seminorm. We verify the subspace criterion \ref{subspaceCriterion}. First $0\in\ker(p)$ because $p(0) = p(0\cdot 0) = |0|p(0) = 0$.

Now take $v,w\in \ker(p)$ and $\lambda\in \F$. Then $0\leq p(v+\lambda w) \leq p(v)+|\lambda|p(w) = 0$, so $v+\lambda w\in\ker(p)$.
\end{proof}

\begin{proposition} \label{gaugeSeminorms}
Let $V$ be a vector space, $p: V\to \R$ a seminorm and $\lambda\in \F$. Then
\[ \setbuilder{v\in V}{p(v) < \lambda} \qquad\text{and}\qquad \setbuilder{v\in V}{p(v) \leq \lambda} \]
are absolutely convex and absorbent.
\end{proposition}
\begin{proof}
Take $|\mu| + |\nu| \leq 1$ and $v,w\in\setbuilder{v\in V}{p(v) \leq \lambda}$. Then $p(|\mu|v + |\nu|w)\leq |\mu|p(v) + |\nu|p(w) \leq (|\mu|+|\nu|)\lambda \leq \lambda$.

For absorbence, take $v\in V$. Then $\frac{\lambda}{2p(v)} v\in \setbuilder{v\in V}{p(v) < \lambda}$.
\end{proof}

\subsubsection{Gauges}
\begin{definition}
Let $V$ be a vector space and $A\subseteq V$ an absorbent subset. The function
\[ p_A: V\to \R^{\geq 0}: v\mapsto \inf\setbuilder{\lambda\in \R^{\geq 0}}{v\in \lambda A} \]
is called the \udef{gauge} or \udef{Minkowski functional} of $A$.
\end{definition}
The function $p_A$ is well-defined (i.e.\ $p_A(v)$ is finite for all $v\in V$) because $A$ is absorbent.

\begin{lemma} \label{gaugeLemma}
Let $V$ be a vector space, $A\subseteq V$ an absorbent subset and $\lambda\in \R^{> 0}$. Then
\begin{enumerate}
\item $\lambda > p_A(v) \implies \lambda^{-1}v\in A$;
\item $\lambda^{-1}v\in A \implies \lambda \geq p_A(v)$.
\end{enumerate}
In particular, we have
\[ p_A^{\preimf}[\ball(0,1)] = \setbuilder{v\in V}{p_A(v) < 1} \subseteq A \subseteq \setbuilder{v\in V}{p_A(v) \leq 1} = p_A^{\preimf}[\cball(0,1)]. \]
\end{lemma}
\begin{proposition} \label{gaugeInherenceAdherence}
Let $V$ be a vector space and $A\subseteq V$ an absorbent subset. Then
\begin{enumerate}
\item $p_A^{\preimf}[\ball(0,1)] = \inh_\mathfrak{a}(A)$;
\item $p_A^{\preimf}[\cball(0,1)] = \adh_\mathfrak{a}(A)$.
\end{enumerate}
\end{proposition}
\begin{proof}
TODO
\end{proof}
We can summarise \ref{gaugeLemma} and \ref{gaugeInherenceAdherence} as
\[ \inh_\mathfrak{a}(A) = \setbuilder{v\in V}{p_A(v) < 1} \subseteq A \subseteq \setbuilder{v\in V}{p_A(v) \leq 1} = \adh_\mathfrak{a}(A). \]

\begin{proposition} \label{gaugeClassification}
Let $V$ be a vector space and $f: V\to \R^{\geq 0}$ a function.
Then the following are equivalent:
\begin{enumerate}
\item $f$ is positively homogenous;
\item for all $A\subseteq V$ such that $f^{\preimf}(\ball(0,1)) \subseteq A \subseteq f^{\preimf}(\cball(0,1))$, $A$ is absorbent and $f = p_A$;
\item $f = p_A$ for some absorbent subset $A$.
\end{enumerate}
\end{proposition}
Note that positive homogeneity is equivalent to strictly positive homogeneity.
\begin{proof}
$(1) \Rightarrow (2)$ To show absorbence, take some $v\in V$. Then for any $\epsilon>0$, $f\big((f(v)+\epsilon)^{-1}v\big) = \frac{f(v)}{f(v)+\epsilon} < 1$, so $(f(v)+\epsilon)^{-1}v\in A$.

Now take some $A$ and fix some arbitrary $v\in V$. We have $p_A(v) = \inf\setbuilder{\lambda\in \R^{\geq 0}}{v\in \lambda A}$, so
\[ \begin{aligned}
p_A(v) &\leq \inf\setbuilder{\lambda\in \R^{\geq 0}}{v\in \lambda f^{\preimf}(\ball(0,1))} \\
&= \inf\setbuilder{\lambda\in \R^{\geq 0}}{v\in f^{\preimf}(\ball(0,\lambda))} \\
&= \inf\setbuilder{\lambda\in \R^{\geq 0}}{f(v) < \lambda} = f(v)
\end{aligned} \quad\text{and}\quad \begin{aligned}
p_A(v) &\geq \inf\setbuilder{\lambda\in \R^{\geq 0}}{v\in \lambda f^{\preimf}(\cball(0,1))} \\
&= \inf\setbuilder{\lambda\in \R^{\geq 0}}{v\in f^{\preimf}(\cball(0,\lambda))} \\
&= \inf\setbuilder{\lambda\in \R^{\geq 0}}{f(v) \leq \lambda} = f(v).
\end{aligned} \]
We conclude that $f(v) = p_A(v)$.

$(2) \Rightarrow (3)$ Immediate.

$(3) \Rightarrow (1)$ We calculate for $t \geq 0$
\begin{align*}
f(tv) &= \inf\setbuilder{\lambda\in \R^{\geq 0}}{tv\in \lambda A} = \inf\setbuilder{\lambda\in \R^{\geq 0}}{v\in t^{-1}\lambda A} \\
&= \inf\setbuilder{t\lambda\in \R^{\geq 0}}{v\in \lambda A} = t\inf\setbuilder{\lambda\in \R^{\geq 0}}{v\in \lambda A} = tf(v).
\end{align*}
\end{proof}

\begin{lemma} \label{gaugeZeroLemma}
Let $V$ be a vector space, $A\subseteq V$ an absorbent subset and $a\in A$. If there exists a subspace $U\subseteq A$ such that $a\in U$, then $p_A(a) = 0$.
\end{lemma}
\begin{proof}
For all $\epsilon > 0$, $\epsilon^{-1}a\in A$, so $a\in \epsilon A$.
\end{proof}

\begin{proposition} \label{gaugeProperties}
Let $V$ be a vector space and $A\subseteq V$ an absorbent subset. Then
\begin{enumerate}
\item $p_A$ is absolutely homogenous if $A$ is balanced;
\item $p_A$ is sublinear if $A$ is convex;
\item $p_A$ is point-separating if $A$ is balanced and contains only the trivial subspace.
\end{enumerate}
\end{proposition}
\begin{proof}
(1) By \ref{balancedLemma} we have $\mu A = |\mu| A$ and thus
\begin{align*}
p_A(\mu\cdot v) &= \inf\setbuilder{\lambda\in \R^{\geq 0}}{\mu\cdot v\in \lambda A} = \inf\setbuilder{\lambda\in \R^{\geq 0}}{v\in \frac{\lambda}{\mu} A} \\
&= \inf\setbuilder{\lambda\in \R^{\geq 0}}{v\in \frac{\lambda}{|\mu|} A} = \inf\setbuilder{|\mu|\lambda\in \R^{\geq 0}}{v\in \lambda A} = |\mu|\cdot p_A(v).
\end{align*}

(2) We just need to show subadditivity. Positive homogeneity is automatic by \ref{gaugeClassification}. Take $v,w\in V$. Now take arbitrary $\epsilon > 0$, so $(p_A(v)+\epsilon)^{-1}v \in A$ and $(p_A(w)+\epsilon)^{-1}w \in A$ by \ref{gaugeLemma}. By convexity of $A$, we have
\[ \frac{v+w}{p_A(v)+p_A(w)+2\epsilon} = \frac{p_A(v)+\epsilon}{p_A(v)+p_A(w)+2\epsilon}(p_A(v)+\epsilon)^{-1}v + \frac{p_A(w)+\epsilon}{p_A(v)+p_A(w)+2\epsilon}(p_A(w)+\epsilon)^{-1}w \in A. \]
By \ref{gaugeLemma} this means $p_A(v)+p_A(w)+2\epsilon \geq p_A(v+w)$ and because $\epsilon$ was arbitrary, we conclude that $p_A(v+w) \leq p_A(v)+p_A(w)$.

(3) Assume $A$ contains only the trivial subspace. Then for all $v\in V$ there exists some $\lambda\in \F$ such that $\lambda\cdot v\notin A$. Now for all $|c|\geq |\lambda|$, $c\cdot v\notin A$ because $A$ is balanced. Then $p_A(2\lambda\cdot v) \neq 0$ and because $p_A$ is absolutely homogeneous we have $p_A(v) = (2\lambda)^{-1}p_A(2\lambda\cdot v) \neq 0$.
\end{proof}
\begin{corollary}
The gauge of an absolutely convex and absorbent subset is a seminorm. If the subset contains only the trivial subspace, then the gauge is a norm.
\end{corollary}

\begin{proposition} \label{gaugeConstructions}
Let $V$ be a vector space, $A,B\subseteq V$ absolutely convex and absorbent subsets and $\mathcal{E}$ a set of absolutely convex and absorbent subsets.
\begin{enumerate}
\item The gauge of $\lambda A$ is $|\lambda|^{-1}p_A$ for all balanced sets $A$ and $\lambda\in \F\setminus\{0\}$.
\item The gauge of $A\cap B$ is $\max\{p_A, p_B\}$ if $A,B$ are balanced.
\item If $A\subseteq B$, then $p_B \leq p_A$.
\end{enumerate}
\end{proposition}
\begin{proof}
(1) Take $v\in V$. Then
\[ p_{\lambda A} = \inf\setbuilder{\mu\in\R^{\geq 0}}{v\in \mu\lambda A} = \inf\setbuilder{\mu \in\R^{\geq 0}}{\lambda^{-1}v\in \mu A} = p_A(\lambda^{-1}v) = |\lambda|^{-1}p_A(v), \]
where we have used that $p_A$ is absolutely homogeneous because $A$ is balanced.

(2) TODO
\end{proof}

\begin{proposition} \label{seminormContinuity}
Let $\sSet{V,\xi}$ be a TVS and $p_K:V\to \R^{\geq 0}$ the gauge of some absorbent set $K$. Then the following are equivalent:
\begin{enumerate}
\item $p_K\in \sSet{V,\xi}^*$;
\item $p_K$ is continuous at $0$;
\item $K\in \neighbourhood_\xi(0)$.
\end{enumerate}
\end{proposition}
\begin{proof}
$(1) \Rightarrow (2)$ Immediate.

$(2) \Rightarrow (3)$ We have $p_K^{\preimf}[\ball(0,1)] \subseteq K$ by \ref{gaugeClassification}. If $p_K$ is continuous at $0$, then $K$ is a neighbourhood of $0$ by \ref{pretopologicalContinuityVicinities}.

$(3) \Rightarrow (1)$ Assume $K$ a neighbourhood of $0$ and take some neighbourhood $\Gamma$ of $0$ in $\R$. Then $\Gamma$ contains a ball $\ball(0,\epsilon)$. By \ref{pretopologicalContinuityVicinities}, it is enough to show that $p_K^{\preimf}[\ball(0,\epsilon)]$ is a neighbourhood. Indeed
\[ p_K^{\preimf}[\ball(0,\epsilon)] \supseteq p_K^{\preimf}\left[\cball\left(0,\frac{\epsilon}{2}\right)\right] = p_K^{\preimf}\left[\frac{\epsilon}{2}\cball(0,1)\right] = \frac{\epsilon}{2} p_K^{\preimf}[\cball(0,1)] \supseteq \frac{\epsilon}{2} K, \]
and the result follows because $\frac{\epsilon}{2}K$ is a neighbourhood of $0$ by \ref{TVSconstruction}.
\end{proof}

\begin{proposition} \label{gaugeMajorisation}
Let $\sSet{V,\xi}$ be a TVS and $f:V\to \F$ a linear functional. Then $f\in V^*$ \textup{if and only if} $|f| \leq p_K$ for some $K\in \neighbourhood_\xi(0)$.
\end{proposition}
\begin{proof}
If $|f| \leq p_K$ for some $K\in \neighbourhood_\xi(0)$, then $f$ is bounded on $K$ and thus continuous by \ref{continuityToNormedSpace}.

Now assume $f\in V^*$. By \ref{continuityToNormedSpace} $f$ is bounded by some $C\in \R$ on some neighbourhood $M$. By \ref{continuity} we have that $f^\imf[\adh_\xi(M)] \subseteq \cball(0,C)$ and by \ref{inherenceAdherenceInclusion} and \ref{gaugeInherenceAdherence}, $f^\imf[p_M^\preimf(\cball(0,1))] = f^\imf[\adh_\mathfrak{a}(M)] \subseteq f^\imf[\adh_\xi(M)]$. This means that $\left|f\left(\frac{x}{p_M(x)}\right)\right| \leq C$ for all $x\in V$. Then $p_M(x)^{-1}|f(x)| \leq C$ and, using \ref{gaugeConstructions},
\[ |f(x)| \leq C p_M(x) = p_{C^{-1}\cdot M}(x). \]
Thus we can take $K = C^{-1}\cdot M$, which is a neighbourhood of the origin by \ref{TVSconstruction}.
\end{proof}


\subsection{Initial convergence w.r.t. seminorms}

\begin{proposition} \label{initialSeminormConvergence}
Let $V$ be a vector space and $S$ a set of seminorms on $V$. Let $\xi$ be the initial convergence on $V$ w.r.t. $S$. Then
\begin{enumerate}
\item $\xi$ is a topological vector space convergence;
\item $\begin{aligned}[t]
\neighbourhood_\xi(0) &= \mathfrak{F}\setbuilder{p^\preimf[\ball(0,\epsilon)]}{p\in S, \epsilon > 0} \\
&= \mathfrak{F}\setbuilder{p^\preimf[\cball(0,\epsilon)]}{p\in S, \epsilon > 0}
\end{aligned}$
\item $f\in \sSet{V, \xi}^*$ \textup{if and only if}
\begin{itemize}
\item $f\in \sSet{V,\mathfrak{a}}^*$;
\item there exists a finite subset $A\subseteq S$ and $C>0$ such that $|f(v)| \leq C\max_{g\in A} g(v)$ for all $v\in V$.
\end{itemize}
\end{enumerate}
\end{proposition}
\begin{proof}
(1, 2) That $\xi$ is topological follows from \ref{topologicalInitialFinalConvergence}. Point (2) follows from \ref{pretopologicalInitialFinalConvergence}.

To show $\xi$ is a vector space convergence, we verify the conditions in \ref{TVSconstruction}:
\begin{enumerate}
\item Take $\lambda\in \F$ and $U\in \neighbourhood_\xi(0)$. Then there exist $p\in S$ and $\epsilon > 0$ such that $p^\preimf[\ball(0,\epsilon)] \subseteq U$. Then
\[ \lambda U \subseteq \lambda p^\preimf[\ball(0,\epsilon)] = p^\preimf[|\lambda|\ball(0,\epsilon)] = p^\preimf[\ball(0,|\lambda|\cdot \epsilon)], \]
so $\lambda U\in \neighbourhood_\xi(0)$.
\item Take $U\in \neighbourhood_\xi(0)$, so there exist $p\in S$ and $\epsilon > 0$ such that $p^\preimf[\ball(0,\epsilon)] \subseteq U$. Then
\[ p^\preimf[\ball(0,\epsilon/2)] + p^\preimf[\ball(0,\epsilon/2)] \subseteq p^\preimf[\ball(0,\epsilon)] \subseteq U. \]
\item By \ref{coreProperties}.
\item We claim $p^\preimf[\ball(0,\epsilon)]$ is balanced for all $p\in S$ and $\epsilon > 0$. Indeed for all $|r|\leq 1$
\[ rp^\preimf[\ball(0,\epsilon)] = p^\preimf[|r|\cdot \ball(0,\epsilon)] = p^\preimf[\ball(0,|r|\cdot \epsilon)] \subseteq p^\preimf[\ball(0,\epsilon)]. \]
\end{enumerate}

(3) We have by \ref{continuityLinearFunctionals}
\[ f\in \sSet{V, \xi}^* \iff \exists D>0: \exists U\in \neighbourhood_\xi(0):\; f^\imf[U] \subseteq \cball(0,D). \]
Now $U\in \neighbourhood_\xi(0)$ iff there exists a finite $A = \{p_n^\preimf[\ball(0,\epsilon_n)]\}_{n=0}^N \subseteq \setbuilder{p^\preimf[\ball(0,\epsilon)]}{p\in S, \epsilon > 0}$ such that $\bigcap A\subseteq U$. So WLOG we may take $U$ of this form.
Now 
\begin{align*}
f^\imf\Big[\bigcap A\Big] \subseteq \cball(0,D) &\iff \forall v\in V: \; \Big(\forall n\leq N: p_n(v)\leq \epsilon_n \Big) \implies |f(v)|\leq D \\
&\iff \forall v\in V: \; \max_{n\leq N}\epsilon_n^{-1}p_n(v)\leq 1 \implies |f(v)|\leq D \\
&\iff \forall v\in V: \; \max_{n\leq N}\epsilon_n^{-1}p_n\left(\frac{\max_{n\leq N}\epsilon_n^{-1}p_n(v)}{\max_{n\leq N}\epsilon_n^{-1}p_n(v)} v\right)\leq 1 \implies |f(v)|\leq D \\
&\iff \forall v\in V: \; \left(\max_{n\leq N}\epsilon_n^{-1}p_n(v)\right)^{-1}\max_{n\leq N}\epsilon_n^{-1}p_n(v)\leq 1 \implies \left(\max_{n\leq N}\epsilon_n^{-1}p_n(v)\right)^{-1}|f(v)|\leq D \\
&\iff \forall v\in V: \; 1\leq 1 \implies |f(v)|\leq D\max_{n\leq N}\epsilon_n^{-1}p_n(v) \\
&\iff \forall v\in V: \; |f(v)|\leq D\max_{n\leq N}\epsilon_n^{-1}p_n(v).
\end{align*}
WLOG we may take all $\epsilon_n = \epsilon = \min_{n\leq N}\epsilon_n$. We may then take $C = D/\epsilon$.
\end{proof}

\begin{proposition} \label{locallyConvexSeminormTopology}
Let $V$ be a vector space. A convergence on $V$ is locally convex \textup{if and only if} it is the initial convergence w.r.t. some set $S$ of seminorms on $V$.
\end{proposition}
\begin{proof}
If $V$ has the initial convergence w.r.t. $S$, then $V$ is locally convex by \ref{initialSeminormConvergence} because $p^\preimf[\ball(0,\epsilon)]$ is convex.

Now let $V$ be a locally convex TVS, meaning there exists a convex base $\mathcal{B}$ of $\neighbourhood(0)$. Then $S = \setbuilder{p_B}{B\in \mathcal{B}}$ is a set of continuous seminorms, by \ref{seminormContinuity}. 

In order to show that the convergence on $V$ is initial w.r.t. $S$, we verify the form of $\neighbourhood(0)$ given in \ref{initialSeminormConvergence}.

We may take $\mathcal{B}$ to consist of algebraically open convex sets by replacing it with $\inh_\mathfrak{a}^\imf[\mathcal{B}]$, which contains open convex sets by \ref{coreProperties}. Then
\begin{align*}
\neighbourhood(0) &= \mathfrak{F}(\mathcal{B}) \\
&= \mathfrak{F}\setbuilder{p_B^\preimf[\ball(0,1)]}{B\in \mathcal{B}} \\
&= \mathfrak{F}\setbuilder{\epsilon^{-1} p_B^\preimf[\ball(0,1)]}{B\in \mathcal{B}} \\
&= \mathfrak{F}\setbuilder{p_B^\preimf[\ball(0,\epsilon)]}{B\in \mathcal{B}, \epsilon > 0} \\
&= \mathfrak{F}\setbuilder{p^\preimf(\ball(0,\epsilon))}{p\in S, \epsilon > 0},
\end{align*}
where we have used \ref{gaugeInherenceAdherence} and the fact that the convergence is a vector space convergence, so $\epsilon B \in \upset\mathcal{B}$.
\end{proof}

\begin{proposition}
Let $V$ be a vector space. The functions
\begin{align*}
\powerset\setbuilder{A\subseteq V}{\text{$A$ is convex}} &\to \setbuilder{p: V\to \R}{\text{$p$ is a seminorm}}: &&\mathcal{B}\mapsto \setbuilder{p_K}{K\in \mathcal{B}} \\
\setbuilder{p: V\to \R}{\text{$p$ is a seminorm}} &\to \powerset\setbuilder{A\subseteq V}{\text{$A$ is convex}}: &&S\mapsto \setbuilder{p^\preimf[U]}{p\in S, U\in \neighbourhood_\R(0)}
\end{align*}
form an antitone Galois connection, where we order the seminorms pointwise.
\end{proposition}
\begin{proof}
TODO + previous as corollary
\end{proof}

Note: metrisable is not equivalent to normable!






\subsection{Hahn-Banach extension theorems}
\begin{theorem}[Hahn-Banach majorised by convex functionals] \label{convexHahnBanach}
Let $V$ be a real vector space, $U\subset V$ a subspace and $p$ a convex functional on $V$. Let $f:U\to\R$ be a linear functional that is bounded by $p$:
\[ \forall u\in U: \quad f(u) \leq p(u). \]
Then $f$ has an extension $\tilde{f}: V\to \R$ such that $\tilde{f}$ is a linear functional on $V$ bounded by $p$:
\[ \forall v\in V: \tilde{f}(v) \leq p(v) \qquad \text{and} \qquad \forall u\in U: \tilde{f}(u) = f(u). \]
\end{theorem}
\begin{proof}
As a first step, we want to extend $f$ to a functional $g$ on a space that is one dimension larger than $U$. This means $g$ is of the form
\[ g: U\oplus\Span\{v_1\}\to\R: v + \alpha v_1 \mapsto f(v) + \alpha c \]
for some $v_1\in V\setminus U$.

If we want $g$ to be majorised by $p$, then we need to find a $c$ such that
\[ \forall v\in U: \forall \alpha\in\R: \; g(\alpha v_1 + v) = \alpha c + f(v) \leq p(\alpha v_1 + v) \]
this means that we need
\[ \forall v\in U: \forall \alpha\in\R:\; \frac{-p(v - |\alpha|v_1) + f(v)}{|\alpha|} \leq c \leq \frac{p(v + |\alpha|v_1) - f(v)}{|\alpha|} \]
and we can find such a $c$ if and only if
\[ \forall v\in U: \forall \alpha\in\R:\; -p(v - |\alpha|v_1) + f(v) \leq p(v + |\alpha|v_1) - f(v), \]
which is equivalent to $2f(v) \leq p(v+|\alpha|v_1)+p(v-|\alpha|v_1)$. This follows from
\begin{align*}
f(v) \leq p(v) &= p(\tfrac{1}{2}(v+|\alpha|v_1) + \tfrac{1}{2}(v-|\alpha|v_1)) \\
&\leq \tfrac{1}{2}p(v+|\alpha|v_1) + \tfrac{1}{2}p(v-|\alpha|v_1).
\end{align*}
So we can extend the domain of $f$ by one dimension such that it is still majorised by $p$.

We can iterate the construction to extend $f$ by multiple dimensions. Each extension can be viewed as a subset of $V\times \R$, by identifying it with its graph.
Consider the family of all such subsets that determine a majorised extension of $f$ (not just those obtained by iteration of the previous construction!). This is a family of finite character. We apply the Teichmüller-Tukey lemma, \ref{ZornEquivalents}, to obtain a maximal element.

This maximal element has domain $V$, because if it did not, it could be extended and was not a maximal element.
\end{proof}
Clearly if $V$ has a well-ordered Hamel basis, we do not need choice as we can just take successive $v$s in the basis and find $c$s constructively.
\begin{corollary}[Hahn-Banach majorised by sublinear functionals] \label{sublinearHahnBanach}
Any majorant $p$ that is sublinear is also convex and can be used in the Hahn-Banach theorem.
\end{corollary}
\begin{corollary}[Hahn-Banach majorised by seminorms] \label{seminormHahnBanach}
Let $(\mathbb{F},V,+)$ be a real or complex vector space, $U\subset V$ a subspace and $p$ a seminorm on $V$. Let $f:U\to\mathbb{F}$ be a linear functional that is bounded by $p$:
\[ \forall u\in U: \quad |f(u)| \leq p(u). \]
Then $f$ has an extension $\tilde{f}: V\to \R$ such that $\tilde{f}$ is a linear functional on $V$ bounded by $p$:
\[ \forall v\in V: |\tilde{f}(v)| \leq p(v) \qquad \text{and} \qquad \forall u\in U: \tilde{f}(u) = f(u). \]
\end{corollary}
\begin{proof}
First assume $V$ is a \emph{real} vector space. Because every seminorm is a sublinear function, we can use \ref{sublinearHahnBanach} to find an extension $\tilde{f}$. We then just need to check it satisfies $\forall v\in V: |\tilde{f}(v)| \leq p(v)$.
From \ref{sublinearHahnBanach} we know $\forall v\in V: \tilde{f}(v) \leq p(v)$.
To prove $-\tilde{f}(v) \leq p(v)$, we calculate
\[ -\tilde{f}(v) = \tilde{f}(-v) \leq p(-v) = |-1|p(v) = p(v). \]

If $V$ is a \emph{complex} vector space, consider the realification $V_\R$ and apply the preceding proof to obtain a linear functional $g: V_\R \to \R$ that extends $f$ and is majorised by $p$. Then by \ref{complexRangeExtensionRealFunctional} we can find a complex-linear functional $\tilde{f}:V \to \C$ such that $\Re(\tilde{f}) = g$.

We just need to show that $f$ is bounded by $p$. Take arbitrary $v\in V$ and write $\tilde{f}(v) = |\tilde{f}(v)|e^{i\theta}$ then
\[ |\tilde{f}(v)| = \Re|\tilde{f}(v)| = \Re\Big(e^{-i\theta}\tilde{f}(v)\Big) = \Re\Big(\tilde{f}(e^{-i\theta}v)\Big) = g(e^{-i\theta}v) \leq p(e^{-i\theta}v) = |e^{-i\theta}|p(v) = p(v). \]
\end{proof}
\begin{corollary}
Let $V$ be a locally convex vector space, $U\subseteq V$ a subspace and $f:U\to \F$ a continuous functional. Then $f$ has a continuous extension to $V$.
\end{corollary}
\begin{proof}
We have that $|f|\leq p_K$ for some $K\in \neighbourhood_U(0)$ by \ref{gaugeMajorisation}. By continuity of the inclusion map, we can find an $M \in \neighbourhood_V(0)$ such that $M\cap U = K$. Then $|f|\leq p_M$ and $f$ can be extended by the Hahn-Banach extension theorem. By \ref{gaugeMajorisation} this extension is continuous.
\end{proof}
\begin{corollary} \label{locallyConvexTVSDualPair}
Let $V$ be a Hausdorff locally convex vector space and $v\in V$. If $f(v) = 0$ for all $f\in V^*$, then $v = 0$.
\end{corollary}
\begin{corollary}
Let $X$ be a normed space and $Z\subset X$ a subspace. Any bounded linear functional in $\tdual{Z}$ can be extended to a bounded linear functional in $\tdual{X}$ with the same norm.
\end{corollary}
\begin{proof}
Let $f:Z\to \mathbb{F}$ be such a functional. Extend $f$ by the previous theorem, \ref{seminormHahnBanach}, using $p(x) = \norm{f}_Z\norm{x}$.
\end{proof}
\begin{corollary} \label{existenceBoundedFunctionalOfSameNorm}
Let $X$ be a normed space and $x_0\neq 0$ an element of $X$. Then there exists a bounded linear functional $\omega_{x_0}$ such that
\[ \norm{\omega_{x_0}} = 1 \qquad \text{and} \qquad \omega_{x_0}(x_0)=\norm{x_0}. \]
\end{corollary}
\begin{proof}
Extend the functional $f: \Span\{x_0\}\to \mathbb{F}$ defined by
\[ f(x) = f(ax_0) = a\norm{x_0}. \]
\end{proof}
\begin{corollary}
Let $X$ be a normed space. Then $\forall x\in X:$
\[ \norm{x} = \sup_{\substack{f\in X' \\ f\neq 0}}\frac{|f(x)|}{\norm{f}}. \]
\end{corollary}
\begin{proof}
We calculate
\[ \norm{x} \geq \sup_{\substack{f\in X' \\ f\neq 0}}\frac{|f(x)|}{\norm{f}} \geq \frac{|\omega_{x}(x)|}{\norm{\omega_{x}}} = \frac{\norm{x}}{1} = \norm{x} \]
where the first inequality follows from $|f(x)|\leq \norm{f}\norm{x}$ for all $f\in X', x\in X$.
\end{proof}

\subsubsection{Hahn-Banach separation}

\begin{lemma} \label{gaugeSeparationLemma}
Let $V$ be a real vector space, $A$ an absorbent set and $x_0 \notin A$. Consider the functional $f_{x_0}: \Span\{x_0\}\to \F: tx_0 \mapsto t$. Then $f_{x_0}(x)\leq p_A(x)$ for all $x\in \Span\{x_0\}$.
\end{lemma}
\begin{proof}
Let $x = tx_0$. If $t\leq 0$, then the inequality is immediate. Suppose $t>0$. Because $p_A(x_0) \geq 1$ (by the converse of \ref{gaugeLemma}), we have
\[ f_{x_0}(x) = f_{x_0}(tx_0) = t \leq tp_A(x_0) = p_A(tx_0) = p_A(x)  \]
using positive homogeneity (\ref{gaugeClassification}).
\end{proof}

\begin{theorem}[Mazur] \label{MazurTheorem}
Let $V$ be a real or complex convergence vector space and $A$ an open and convex set. If $U$ is a subspace such that $A\perp U$, then there exists a closed hyperplane $H \supseteq U$ such that $A\perp H$.
\end{theorem}
\begin{proof}
First suppose $V$ is a \emph{real} vector space. Because $A$ is open, it is algebraically open. Take $a\in A$. Then $0\in a-A = \inh_{\mathfrak{a}}(a-A)$, so $a-A$ is absorbing by \ref{coreProperties}. 

Then we have
\[ U\perp A \iff 0\notin U-A \iff a \notin a-A+U. \]
Consider the functional $f_{a}$ of \ref{gaugeSeparationLemma}, which is majorised by the gauge $p_{a-A+U}$, which is sublinear by \ref{gaugeProperties}. Then $f_a$ can be extended as an $\R$-linear function to all $V$ by the Hahn-Banach extension theorem \ref{sublinearHahnBanach}.

We note that $U\subseteq \ker(f_a)$, because $p_{a-A+U}(u) = 0$ by \ref{gaugeZeroLemma}.

In order to conclude with \ref{functionalBoundedNeighbourhood}, we need to show that $A-a \subseteq f_a^{\preimf}(\ball(0,|f_a(a)|)) = f_a^{\preimf}(\ball(0,1))$.
Indeed $A-a \subseteq U+A-a = \inh_\mathfrak{a}(U+A-a) = p_{U+A-a}^\preimf[\ball(0,1)] \subseteq f_{a}^\preimf[\ball(0,1)]$ by \ref{algebraicallyOpen} and \ref{gaugeInherenceAdherence}.

Note that $\ker(f_a)^c$ contains the open set $A$ and thus $\ker(f_a)$ is not dense by \ref{openDensityLemma}. By \ref{hyperplaneClosedDense} this means that $\ker(f_a)$ is closed.

Now suppose $V$ is a \emph{complex} vector space. We can consider the realification $V_\R$ with the same convergence, which is a real convergence vector space by \ref. So we can use the preceding proof to find a real hyperplane $K$ in $V$. Then \ref{realComplexHyperplane} gives that $H = K\cap iK$ is a complex hyperplane in $V$. Now $H$ and $A$ must be disjoint because $K$ and $A$ are disjoint and $H \subseteq K$.

Also $H$ is closed because $K$ and $iK$ are closed (the first by the preceding proof, the second because multiplication by $i$ is a homeomorphism \ref{continuityLemmaVectorConvergence}) and the intersection of two closed sets is closed.
\end{proof}

\begin{theorem}[Hahn-Banach separation theorem]
Let $V$ be a convergence vector space. Suppose $A,B$ are disjoint, non-empty, convex sets and that $A$ is open. Then there exists a continuous linear functional $f:V\to \F$ such that $f^\imf[A]$ and $f^\imf[B]$ are disjoint.
\end{theorem}
\begin{proof}
The set $A-B = \bigcup_{b\in B}A-b$ is convex and a union of open sets and thus open by \ref{completeClosureTopology}.
The set $A-B$ and the vector space $\{0\}$ are disjoint, so by \ref{MazurTheorem} we can find a closed hyperplane that is disjoint with $A-B$.

By \ref{kernelHyperplane} and \ref{continuityLinearFunctionals} this is the kernel of a continuous linear functional $f$.
\end{proof}
\begin{corollary}
Let $V$ be a real or complex convergence vector space and $A,B$ as in the proposition. Then there exists a continuous linear functional $f:V\to \F$ and $t\in \R$ such that
\[ \Re f(a) < t \leq \Re f(b) \]
for all $a\in A$ and $b\in B$.
\end{corollary}
This means $A$ and $B$ are separated by a closed affine hyperplane $\ker(f)+v$, where $v \in f^\preimf[\{t\}]$.
\begin{proof}
Apply the proposition to the realification $V_\R$. This gives us an $\R$-linear functional $g: V\to \R$ such that $g^\imf[A]$ and $g^\imf[B]$ are disjoint convex sets. Additionally $g^{\imf}[A]$ is open in $\R$ by \ref{linearFunctionalOpen}.

Because $g^\imf[A]$ and $g^\imf[B]$ are convex, we either have $g^\imf[A]\leq g^\imf[B]$ or $g^\imf[A]\geq g^\imf[B]$. In the second case we simply replace $g$ by $-g$ to obtain the first case. We may take $t= \sup g^\imf[A]$. This is not in $g^\imf[A]$ because it is open.

If $V$ is a real vector space we take $f=g$ are done. If $V$ is complex, we can find a suitable $f$ by \ref{complexRangeExtensionRealFunctional}.
\end{proof}
\begin{corollary}
Let $V$ be a locally convex TVS. Suppose $A,B$ are disjoint, non-empty, convex sets and that $A$ is compact, $B$ is closed. Then there exists a continuous linear functional $f:V\to \F$ and $s,t\in \R$ such that
\[ \Re f(a) < t < s < \Re f(b) \]
for all $a\in A$ and $b\in B$.
\end{corollary}
\begin{proof}
TODO
\end{proof}

\subsubsection{Banach limits}
\begin{proposition}
There exists a linear map $L:l^\infty(\N) \to \C$ satisfying
\begin{enumerate}
\item $\displaystyle L(x) = \lim_{n\to \infty}x_n$ if the limit exists;
\item $L((x_{n+1})_{n\in\N}) = L((x_n)_{n\in\N})$;
\item if $\forall n\in\N:x_n\geq 0$, then $L(x) \geq 0$;
\item $\norm{L} = 1$.
\end{enumerate}
Such a linear map is called a \udef{Banach limit}.
\end{proposition}
\begin{proof}
TODO, after Cesàro means.
\end{proof}

\subsection{Continuous functionals}

TODO???
\begin{proposition}
Let $V$ and $W$ be TVSs and $f: V\to W$ a linear function.
\begin{enumerate}
\item If $f$ is continuous and $W$ is Hausdorff, then $\ker(f)$ is closed.
\item If $f$ has closed kernel and finite-dimensional image, then $f$ is continuous.
\end{enumerate}
\end{proposition}
\begin{proof}
(1) Because $W$ is Hausdorff, it is also $T_1$ and thus $\{0\}$ is closed by \ref{FrechetCharacterisation}. Then $\ker(f) = f^{\preimf}(\{0\})$ is closed by \ref{continuity}.

(2) 
\end{proof}
??



\section{General duality theory}
\subsection{Paired spaces}
\begin{definition}
A \udef{pairing} is a triple $\sSet{V,W, \pair{\cdot,\cdot}}$ where $V,W$ are vector spaces over $\mathbb{F}$ and $\pair{\cdot,\cdot}: V\times W\to \mathbb{F}$ is a bilinear form. Often we will write the pairing as just $\sSet{V,W}$.

We say $W$ \udef{distinguishes} points of $V$ or is \udef{separating} on $V$ if $\pair{v,\cdot}$ is injective for all $v\in V$.

A \udef{dual system}, \udef{dual pair} or \udef{duality} over a field $\mathbb{F}$ is a pairing $\sSet{V,W, b}$ such that $V$ distinguishes points of $W$ and $W$ distinguishes points of $V$.
\end{definition}
We have that $W$ is separating on $V$ \textup{if and only if}
\[ \forall v\in V: \exists w\in W: \pair{v,w} \neq 0. \]

\begin{example}
\begin{itemize}
\item Let $V$ be a vector space. Then $\sSet{V^*, V, \pair{\cdot,\cdot}}$ with $\pair{\cdot,\cdot}:V^*\times V: (f,v)\mapsto \pair{f,v} = f(v)$ is a dual pair.
\item Let $\sSet{V,\xi}$ be a Hausdorff locally convex TVS. Then $\sSet{\sSet{V,\xi}^*, V, \pair{\cdot,\cdot}}$ is a dual pair by \ref{locallyConvexTVSDualPair}.
\end{itemize}
\end{example}

\begin{lemma}
Let $\sSet{V,W,\pair{\cdot,\cdot}}$ be a dual pair. Then $\sSet{W,V,\pair{\cdot,\cdot}^d}$ is also a dual pair.
\end{lemma}

\subsection{The weak topology}
\begin{definition}
Let $(X,Y,\pair{\cdot,\cdot})$ be paired vector spaces. Then for each $y\in Y$, the map
\[ \abspair{x,\cdot}: Y\to \R_{\geq 0}: y\mapsto \abspair{x,y} \]
determines a seminorm on $Y$.

The initial topology on $Y$ w.r.t. $\setbuilder{\abspair{x,\cdot}}{x\in X}$ is called the \udef{weak topology} $\sigma(X,Y)$ on $Y$ for the pair $\sSet{X,Y}$\footnote{What we denote $\sigma(X,Y)$ is usually denoted $\sigma(Y,X)$.}.
\end{definition}
\begin{lemma}
Let $\sSet{X,Y,\pair{\cdot,\cdot}}$ be a pairing. The weak topology $\sigma(X,Y)$ on $Y$ is locally convex and
\[ \neighbourhood_{\sigma(X,Y)}(0) = \mathfrak{F}\setbuilder{y\in Y}{\exists x\in X: \abspair{x,y} \leq 1}. \]
\end{lemma}
\begin{proof}
The weak topology is locally convex by \ref{locallyConvexSeminormTopology}. From this we also get the neighbourhood filter: it is enough to show that $\setbuilder{\abspair{x,\cdot}^{\preimf}(\,[0,\epsilon]\,)}{x\in X, \epsilon > 0} = \setbuilder{y\in Y}{\exists x\in X: \abspair{x,y} \leq 1}$. Then
\begin{align*}
y \in \setbuilder{\abspair{x,\cdot}^{\preimf}(\,[0,\epsilon]\,)}{x\in X, \epsilon > 0} &\iff \exists x\in X: \exists \epsilon > 0: \; \abspair{x,y} \leq \epsilon \\
&\iff \exists x\in X: \exists \epsilon > 0: \; \epsilon^{-1}\abspair{x,y} \leq 1 \\
&\iff \exists x\in X: \exists \epsilon > 0: \; \abspair{\epsilon^{-1}x,y} \leq 1 \\
&\iff \exists x\in X: \; \abspair{x,y} \leq 1 \\
&\iff y\in \setbuilder{y\in Y}{\exists x\in X: \abspair{x,y} \leq 1}.
\end{align*}
\end{proof}

\begin{lemma} \label{functionalContinuityWeakTopology}
Let $\sSet{X,Y,\pair{\cdot,\cdot}}$ be a pairing and $f: Y\to \F$ a linear functional. Then
\[ f\in \sSet{Y, \sigma(X,Y)}^* \iff \exists \{x_0,\ldots, x_n\}\subseteq X: \exists \{\alpha_0,\ldots, \alpha_n\}\subseteq \F: f = \sum_{i=0}^n \alpha_i \pair{x_i, \cdot}.  \]
\end{lemma}
\begin{proof}
By a comparison of \ref{linearDependenceLinearFunctionals} and point (3) of \ref{initialSeminormConvergence}.
\end{proof}

\begin{proposition}
Let $V$ be a locally convex TVS. Then $V^* = \sSet{V, \sigma(V^*, V)}^*$.
\end{proposition}
\begin{proof}
This follows easily from \ref{functionalContinuityWeakTopology}:

If $f\in V^*$, then $f = \pair{f,\cdot}$ and thus $f\in \sSet{V, \sigma(V^*, V)}^*$.

If $f\in \sSet{V, \sigma(V^*, V)}^*$, then $f$ is a linear combination of continuous functions and thus continuous, because the composition of linear functions is continuous.
\end{proof}

\subsubsection{Weak-$*$ topology}

\begin{proposition} \label{weak*continuousFunctional}
Let $X$ be a Banach space and let $X'$ have the weak-$*$ topology. Then a linear functional $\theta: X'\to \C$ is continuous \textup{if and only if}
\[ \exists x\in X: \forall \omega\in X': \quad \theta(\omega) = \omega(x). \]
\end{proposition}
\begin{proof}
TODO 9.2 in lecture notes.
\end{proof}

\subsection{Polar sets}
\begin{definition}
Let $\sSet{X,Y,\pair{\cdot,\cdot}}$ be a pairing and $B\subseteq Y$ a subset. The \udef{polar} of $B$ is the polar w.r.t. the relation $\pol$ on $(Y,X)$ defined by
\[ y\pol x \qquad\iff\qquad \abspair{x, y} \leq 1. \]
\end{definition}

\begin{lemma}
Let $\sSet{X,Y,\pair{\cdot,\cdot}}$ be a pairing and $B\subseteq Y$ a subset. Then
\[ B^\pol = \setbuilder{x\in X}{\sup_{y\in B}\abspair{x,y} \leq 1}. \]
\end{lemma}

\begin{lemma}
Let $\sSet{X,Y,\pair{\cdot,\cdot}}$ be a pairing and $B\subseteq Y$ a subset. Then
\begin{enumerate}
\item for all $\lambda \neq 0$: $(\lambda B)^\pol = |\lambda|^{-1}B^\pol$.
\end{enumerate}
\end{lemma}

\begin{proposition}
Let $\sSet{X,Y,\pair{\cdot,\cdot}}$ be a pairing and $B\subseteq Y$ a subset. Then $B^\pol$ is absolutely convex and $\sigma(X,Y)$-closed.
\end{proposition}

TODO $(\bigcup A)^\pol = \bigcap A^\pol$.



\subsubsection{Annihilator subspaces}
\begin{lemma}
Let $\sSet{X,Y,\pair{\cdot,\cdot}}$ be a pairing and $V\subseteq Y$ a subspace. Then
\[ V^\pol = \setbuilder{x\in X}{\forall y\in V:\;\pair{x,y} = 0}. \]
\end{lemma}


\subsection{The pair $(V^*, V)$}
\begin{definition}
Let $V$ be a vector space and $V^*$ the continuous dual under some convergence. Any convergence on $V$ such that the continuous dual is still $V^*$ is called a \udef{convergence of the dual pair}.
\end{definition}
Any property that only depends on continuous linear functionals is the same for ant convergence of the dual pair.

\begin{proposition}
Let $V$ be a vector space and $A\subseteq V$ a convex subset. Let $\xi,\zeta$ be two vector space convergences on $V$. Then
\[ \sSet{V,\xi}^* = \sSet{V,\zeta}^* \qquad\implies\qquad \adh_\xi(A) = \adh_\zeta(A). \]
\end{proposition}
TODO closures? or only topological convergences? 
\begin{proof}
TODO Robertson p34
\end{proof}


\subsection{Mackey topology}

\begin{theorem}[Mackey-Arens]
\end{theorem}

\section{Operators on topological vector spaces}

\subsection{Continuous operators}
\subsubsection{Closed graph theorem}

\begin{theorem}[Closed graph theorem] \label{closedGraphTheorem}
Let $f:X\to Y$ be a map from a topological space $X$ into a Hausdorff space $Y$.
\begin{enumerate}
\item if $f$ is continuous, then $f$ has closed graph;
\item if $X$ is compact, then the converse also holds.
\end{enumerate}
\end{theorem}
\begin{proof}
TODO
\end{proof}
TODO: for Banach spaces $X$ complete enough!!!!!!


\subsection{Compact operators}
\begin{definition}
A linear operator $T:X\to Y$ between TVSs is \udef{compact} if it maps a neighbourhood of the origin to a precompact set, i.e.\ 
\[ \exists U \in \neighbourhood(0): \;  \text{$\overline{T[U]}$ is compact.} \]
The set of compact linear operators in $(X\to Y)$ is denoted $\Compact(X,Y)$.
\end{definition}
TODO: doesn't the neighbourhood need to be bounded in some way?????

\begin{proposition}
Let $X$ be a normed space and $Y$ a TVS and $T:X\to Y$ a linear operator. Then the following are equivalent:
\begin{enumerate}
\item $T$ is a compact operator;
\item there exists a neighbourhood $U \subset X$ of the origin and a compact set $V\subset Y$ such that $T[U] \subset V$;
\item the image of the unit ball of $X$, $T[B(\vec{0},1)]$, is precompact in $Y$;
\item the image of any bounded set in $X$ is precompact in $Y$.
\end{enumerate}
If $Y$ is a normed space, these are also equivalent to
\begin{enumerate} \setcounter{enumi}{4}
\item for any bounded sequence $(x_{n})_{n\in \mathbb{N}}$ in $X$, the sequence $(Tx_{n})_{n\in \mathbb{N} }$ contains a converging subsequence.
\end{enumerate}
\end{proposition}
\begin{proof}
TODO
\end{proof}


\begin{lemma}
Let $X,Y$ be TVSs.
\begin{enumerate}
\item Then $\Compact(X, Y)$ is a vector space.
\item If $X,Y$ are normed spaces, then $\Compact(X, Y)$ is a subspace of $\Bounded(X, Y)$.
\end{enumerate}
\end{lemma}
\begin{proof}
(1) Let $K,K':X\to Y$ be compact operators. Then, by \ref{closureGroupOperation} (TODO opposite inclusion!),
\[ \overline{K[B(0, 1)]+K'[B(0, 1)]} \subseteq \overline{K[B(0, 1)]}+\overline{K'[B(0, 1)]}, \qquad \overline{K[\lambda B(0, 1)]} = \lambda\overline{K[B(0, 1)]}. \]

(2) Let $K\in\Compact(X, Y)$. Then the image of the unit ball is precompact, meaning it is bounded. So $K$ is bounded by \ref{existenceOperatorNorm}.
\end{proof}

\begin{lemma}
Let $T:V\to W$ be a bounded operator. If $W$ has the Heine-Borel property, then $T$ is compact.
\end{lemma}
\begin{proof}
The set $T[B(\vec{0},1)]$ is bounded because $T$ is. By the Heine-Borel (TODO ref) property of $W$, $\overline{B(\vec{0},1)}$ is compact.
\end{proof}
\begin{corollary}
Bounded operators with as image a finite dimensional normed space are compact.
\end{corollary}
\begin{corollary}
The identity on a normed space $X$ is compact \textup{if and only if} $X$ is finite-dimensional.
\end{corollary}
\begin{proof}
TODO ref. 
\end{proof}

\begin{proposition}
Compact operators map weakly convergent sequences to strongly convergent sequences. TODO! + remove from Hilbert section.
\end{proposition}
\begin{corollary} \label{limitCompactImageOrthonormalSequence}
Let $V$ be an inner product space and $\seq{e_n}$ a sequence of orthonormal vectors in $V$. If $K$ is a compact operator, then $\lim_{n\to\infty}Ke_n = 0$.
\end{corollary}
\begin{proof}
Any sequence of orthonormal vectors $\seq{e_n}$ converges weakly to $0$. Because $K$ is compact, $\seq{Ke_n}$ converges strongly to zero. TODO ref.
\end{proof}
\begin{corollary}
If $V$ is infinite-dimensional and $K$ is invertible, then its inverse is unbounded.
\end{corollary}
\begin{proof}
Due to $\lim_{n\to\infty}Ke_n = 0$ the operator $K$ cannot be bounded below, so $K^{-1}$ is not bounded by \ref{boundedBelow}.
\end{proof}

\section{Continuity}
\url{https://en.wikipedia.org/wiki/Bilinear_map#Continuity_and_separate_continuity}



















\chapter{Functionals on vector spaces}

\begin{theorem}[Riesz-Markov-Kakutani representation theorem]
Let $X$ be a locally compact Hausdorff space. For any positive linear functional $\psi$ on $C_c(X)$, there is a unique Radon measure $\mu$ on $X$ such that
\[ \forall f\in C_c(X): \quad \psi(f) = \int_X f(x)\diff{\mu(x)}. \]
\end{theorem}














\chapter{Banach spaces}
\begin{definition}
\begin{itemize}
\item A \udef{Banach space} is a normed vector space that is complete as a metric space.
\item A \udef{Hilbert space} is an inner product space that is complete as a metric space.
\end{itemize}
\end{definition}

A finite-dimensional normed / inner product space is automatically a Banach / Hilbert space by proposition \ref{finiteDimComplete}.

Every proper subspace $U$ of a normed vector space $V$ has empty interior.
A nice consequence of this is that any closed proper subspace is necessarily nowhere dense. So if V is a Banach space, the Baire category theorem implies that V cannot be a countable union of closed proper subspaces. In particular, an infinite dimensional Banach space cannot be a countable union of finite dimensional subspaces. This means, for example, that a vector space of countable dimension (e.g\ the space of polynomials) cannot be equipped with a complete norm.

The space $\Bounded(V,W)$ is a Banach space.

TODO: quotient of Banach spaces.

Complemented subspace problem: \url{https://arxiv.org/pdf/math/0501048v1.pdf}

\section{Function spaces}
\subsection{The spaces $L^p(X,\diff{\mu})$}

\subsubsection{The spaces $\mathcal{L}^p(X,\diff{\mu})$}
\begin{proposition}
Let $0< p<q \leq \infty$ and $\sSet{X,\mathcal{A},\mu}$ be a measure space. Then
\begin{enumerate}
\item $\mathcal{L}^p(X,\mu) \supseteq \mathcal{L}^q(X,\mu)$ \textup{if and only if} $X$ does not contain sets of finite but arbitrary large measure;
\item $\mathcal{L}^p(X,\mu) \subseteq \mathcal{L}^q(X,\mu)$ \textup{if and only if} $X$ does not contain sets of non-zero but arbitrary small measure.
\end{enumerate}
\end{proposition}
\begin{proof}
TODO \url{https://en.wikipedia.org/wiki/Lp_space#Embeddings}
\end{proof}
\begin{corollary}
For all $0< p<q \leq \infty$ and $a,b\in\R$, we have $\mathcal{L}^p([a,b],\lambda) \supseteq \mathcal{L}^q([a,b],\lambda)$.
\end{corollary}

\subsubsection{Indentifying functions that are equal a.e.}

\begin{proposition}
TODO every equivalence class contains exactly one continuous functions (for Borel measures?)
\end{proposition}

\begin{theorem}[Riesz-Fisher]
The space $L^p(X,\diff{\mu})$ is complete.
\end{theorem}

For $L^\infty$: essential supremum.

\begin{proposition}
Let $X$ be a set. For all $1\leq p < \infty$, the set of $p$-integrable simple functions is dense in $L^p(\diff{\mu})$.
\end{proposition}


\begin{proposition}
Let $X$ be a locally compact Hausdorff space and $\mu$ a Radon measure on $X$. Then $\cont_c(X)$ is dense in $L^p(\diff{\mu})$ for all $1\leq p < \infty$.
\end{proposition}
\begin{proof}

\end{proof}


\subsubsection{Locally integrable spaces}
\begin{definition}
Let $(\Omega, \mathcal{A}, \mu)$ be a measure space. The \udef{locally $L^p$} space is the space
\[ L^p_\text{loc}(\Omega) \defeq \setbuilder{f \in (\Omega\to\C)}{\text{$f\in L^p(K)$ for all compact $K\subset \Omega$}}. \]
The functions in $L^1_\text{loc}(\Omega)$ are called \udef{locally integrable} on $\Omega$.
\end{definition}
TODO: deal with equivalence classes??

\subsection{Sequence spaces}
TODO:  $L^p(A,\mu)$ with $\mu$ counting measure.

Let $J$ be a countable index set and $x:J\to \mathbb{F}$ a sequence indexed by $J$. We define
\[ \norm{x}_p := \left(\sum_{j\in J}|x(j)|^p\right)^{1/p} \qquad\text{and}\qquad \norm{x}_\infty = \sup_{j\in J}|x(j)|. \]
So $\norm{\cdot}_1$ is the standard norm on $\mathbb{F}^n$. For general sequences there is no guarantee that these norms do not diverge.
\begin{definition}
Let $J$ be an index set, $D$ a directed set and $p\geq 1$,
\begin{align*}
\ell^p(J) &= \setbuilder{x:J\to \F}{\norm{x}_p < +\infty},\\
\ell^\infty(J) &= \setbuilder{x:J\to \F}{\norm{x}_\infty < +\infty},\\
c_0(D) &= \setbuilder{x:D\to \F}{\lim_{n\to\infty}|x(n)| = 0}, \\
c_{00}(D) &= \setbuilder{x:D\to \F}{\setbuilder{n\in D}{x(n)\neq 0}\;\text{has finite cardinality}}.
\end{align*}
unless specified we equip $c_0$ and $c_{00}$ with the norm $\norm{\cdot}_\infty$.
\end{definition}

\begin{lemma}
$c_{00}$ is dense in $\ell^p$ if it is equipped with the norm $\norm{\cdot}_p$ and dense in $c_0$ if it is equipped with the norm $\norm{\cdot}_\infty$.
\end{lemma}

Let $1<p,q<\infty$ satisfy $\frac{1}{p}+\frac{1}{q}$. We have the inequalities
\begin{align*}
\norm{xy}_1 &\leq \norm{x}_p\norm{y}_q\qquad\text{(Hölder inequality)} \\
\norm{x+y}_p &\leq \norm{x}_p+\norm{y}_p\qquad\text{(Minkowski inequality)}
\end{align*}
which follow from the general cases (TODO ref) by applying the counting measure.

\begin{proposition}
The continuous dual of $l^p(J)$ is $l^q(J)$ where $1<p,q<\infty$ satisfy $\frac{1}{p}+\frac{1}{q}$.
Also, the continuous dual of $l^1$ is $l^\infty$.
\end{proposition}

\subsubsection{Operators on sequence spaces}
TODO Gribanov's theorems

3.7.1, 3.7.2 of Hanson / Yakovlev.

\section{Series in Banach spaces}
TODO
\url{https://link.springer.com/content/pdf/10.1007%2F978-0-8176-4687-5_3.pdf}
\begin{definition}
Let $\seq{x_n}$ be a sequence in a Banach space $X$. As for series of scalars, we say a series $\sum_{n=1}^\infty x_n$ is
\begin{itemize}
\item \udef{unconditionally convergent} if $\sum_{n=1}^\infty x_{\sigma(n)}$ converges for every permutation $\sigma$ of $\N$;
\item \udef{absolutely convergent} if $\sum_{n=1}^\infty \norm{x_n} < \infty$.
\end{itemize}
\end{definition}

\begin{proposition} \label{absoluteUnconditionalConvergenceBanach}
Let $\seq{x_n}$ be a sequence in a Banach space $X$. If $\sum_{n=1}^\infty$ converges absolutely, then it converges unconditionally.
\end{proposition}
\begin{proof}
Assume absolute convergence, so $\sum\norm{x_i}<\infty$. Then (for $m< n$)
\[ \norm{\sum_{i=1}^n x_i - \sum_{i=1}^m x_i} = \norm{\sum_{i=m+1}^n x_i} \leq \sum_{i=m+1}^n\norm{x_i} = \sum_{i=1}^n \norm{x_i} - \sum_{i=1}^m \norm{x_i}, \]
and because $\sum\norm{x_i}$ converges, it is a Cauchy sequence and by the inequality so is $\sum x_i$. By completeness this sequence is convergent.

By (TODO ref) $\sum\norm{x_{\sigma(i)}}$ converges for any permutation $\sigma$ of $\N$. We can then repeat the argument to show $\sum x_{\sigma(i)}$ is also convergent and thus unconditionally convergent.
\end{proof}

\subsection{Fourier series}
TODO Sacks 7.1

\section{Completions and constructions}

\begin{proposition}
The completions of a space with respect to two different norms are isomorphic \textup{if and only if} the norms are equivalent.
\end{proposition}

TODO move down
\subsection{Tensor products}
TODO Ryan
\url{https://math.stackexchange.com/questions/2712906/does-mathcalb-mathcalh-mathcalh-otimes-mathcalh-in-infinite-dime}
\url{https://math.stackexchange.com/questions/35191/operator-norm-and-tensor-norms?noredirect=1&lq=1}

\subsection{Direct sums}

For arbitrary direct sums we can generalise: now that we have a concept of limits, we can relax the requirement that all but finitely many terms be zero. Instead we require that the sequence of norms is bounded in some way. This gives a whole family of related concepts of direct sum, named for which sequence space the sequence of norms belongs to.
\begin{definition}
Let $\{V_i\}_{i\in I}$ be an arbitrary family of Banach spaces over a field $\F$ and let $\ell(I,\F)$ be a space of sequences in $\F$ indexed by $I$. Then the \udef{$\ell$-direct sum} is the vector space with as field
\[ \bigoplus_{i\in I}^\ell V_i = \setbuilder{(v_i)_{i\in I}}{\forall i\in I: v_i\in V_i \quad\text{and}\quad (\norm{v_i}_{V_i})_{i\in I}\in \ell(I,\F) }. \]
In particular we have, for all $1\leq p<\infty$, the \udef{$\ell^p$-direct sum}
\[ \bigoplus_{i\in I}^p V_i \defeq \setbuilder{(v_i)_{i\in I}}{\forall i\in I: v_i\in V_i \quad\text{and}\quad \sqrt[p]{\sum_{i\in I}\norm{v_i}_{V_i}^p}<\infty} \]
and the \udef{$\ell^\infty$-direct sum}
\[ \bigoplus_{i\in I}^\infty V_i \defeq \setbuilder{(v_i)_{i\in I}}{\forall i\in I: v_i\in V_i \quad\text{and}\quad \sup_{i\in I}\norm{v_i}_{V_i}<\infty}. \]
\end{definition}

\begin{proposition}
For any sequence space that is a Banach space the direct sum is a Banach space. TODO: in particular algebraic direct sum as $c_{00}$? (one possible norm)? and finite direct sums?
\end{proposition}

\subsubsection{Direct sum of identical spaces}
\begin{proposition}
Let $V$ be a Banach space over $\F$, $I$ an arbitrary index set and $\ell(I,\F)$ a banach sequence space.
\[ \bigoplus_{i\in I}^\ell V \cong \ell\otimes V \]
\end{proposition}


\section{Operators on Banach spaces}
\subsection{Closed operators}
\begin{proposition}
Let $X,Y$ be Banach spaces and $S,T\in \Lin(X,Y)$ with $\dom(S) = \dom(T)$. If $S$ is a closed operator and there exist $\alpha,\beta,\gamma\in \R^+$ such that $0 < \gamma \leq 1$ and $\beta < 1/\gamma$ and
\[ \norm{(S-T)u} \leq \alpha \norm{u} + \beta\norm{Su}^\gamma{u}^{1-\gamma} \qquad\text{for all $u\in \dom(S) = \dom(T)$,} \]
then $T$ is also closed.
\end{proposition}
\begin{proof}
TODO Jeribi.
\end{proof}

\section{Bounded operators}
\begin{proposition}
Let $V,W$ be normed spaces. The vector space $\Bounded(V,W)$ with the operator norm is a Banach space \textup{if and only if} $W$ is a Banach space.
\end{proposition}
\begin{corollary}
Let $V$ be a normed space. The continuous dual $X'$ is a Banach space.
\end{corollary}
\begin{corollary}
Topologically reflexive spaces are Banach spaces.
\end{corollary}

\begin{proposition}[Bounded linear extension] \label{BLT}
Let $T:\dom(T)\subseteq X\to Y$ be a bounded operator between normed spaces. Then $T$ has a unique extension
\[ \widetilde{T}:\overline{\dom(T)}\to Y \]
where $\widetilde{T}$ is a bounded operator with $\norm*{\widetilde{T}} = \norm{T}$.
\end{proposition}
\begin{proof}
Normed vector spaces have the unique extension property because they are Hausdorff, \ref{uniqueExtensionHausdorff}. We just need to show the norm stays the same:

Clearly $\norm*{\tilde{T}} \geq \norm{T}$. For the converse take any $x\in X$. As $\overline{\dom(T)} = X$, there exists a sequence $\seq{x_i}\subset \dom(T)$ that converges to $x$. Then
\[ \norm*{\tilde{T}(x)}_Y = \norm{T\left(\lim_{i\to\infty}x_i\right)}_Y = \lim_{i\to\infty}\norm{T(x_i)}_Y \leq \lim_{i\to\infty}\norm{T}\;\norm{x_i}_X = \norm{T}\;\norm{x}_X. \]
\end{proof}

\subsection{Contractions}
A linear operator $T$ on a normed space is a contraction if and only if it is bounded and $\norm{T}<1$. 

\subsubsection{Neumann series}
\begin{lemma}
Let $T$ be a bounded linear operator on a normed space $X$ with $\norm{T}<1$. Then the series $\sum^\infty_{i=1}T^i(b)$ converges for all $b\in X$ and is the unique fixed point of $F(x) = T(x)+b$.
\end{lemma}
\begin{proof}
The function $F$ is a contraction if and only if $\norm{T}<1$. So it has a unique fixed point. Starting the fixed point iteration at $b$ yields the series:
\begin{align*}
F(b) &= Tb + b \\
F(Tb+b) &= T^2b + Tb + b \\
&\hdots.
\end{align*}
Alternatively we could have used the inequality $\norm{T^nb} \leq \norm{T}^n\norm{b}$, the convergence of the geometric series and \ref{absoluteUnconditionalConvergenceBanach} to prove convergence. Proving it is a fixed point is then elementary.
\end{proof}
\begin{corollary}[Neumann series] \label{operatorNeumannSeries}
Let $T$ be a bounded linear operator with $\norm{T}<1$. Then
\[ (\id - T)^{-1} = \sum_{i=1}^\infty T^i \]
with uniform convergence. Also
\[ \norm{(\id - T)^{-1}} \leq \frac{1}{1-\norm{T}}. \]
\end{corollary}
\begin{proof}
Let $x\in X$. Then set $(\id - T)^{-1}x = y$. This is equivalent to $x = y-Ty$ and means $y$ is the fixed point of $y\mapsto Ty+x$. So $y = \sum_{i=1}^\infty T^ix$.

The convergence is uniform by TODO ref.

Finally we have
\[ \norm{(\id - T)^{-1}} = \norm{\sum_{i=1}^\infty T^i} \leq \sum_{i=1}^\infty \norm{T^i} = \frac{1}{1-\norm{T}} \]
by the geometric series.
\end{proof}
TODO ref: uniform convergence if $\sum_i \norm{T_i} < \infty$??

\subsection{The uniform boundedness principle}
TODO: if a family of bounded operators on a Banach space is pointwise bounded, then it is uniformly bounded.
\begin{theorem}[Uniform boundedness principle] \label{uniformBoundednessPrinciple}
Let $\mathcal{F}\subset \Bounded(X,Y)$ be a family of bounded operators where $X$ is a Banach space and $Y$ a normed space, such that
\[ \sup\setbuilder{\norm{Tx}}{T\in\mathcal{F}} < \infty \qquad \text{for all $x\in X$}. \]
Then $\sup\setbuilder{\norm{T}}{T\in\mathcal{F}} < \infty$.
\end{theorem}
\begin{proof}
The proof is an application of the Baire category theorem. Define the closed subsets $K_n$ as
\[ K_n = \setbuilder{x\in X}{\forall T\in\mathcal{F}: \norm{Tx}\leq n}. \]
These are closed because the functional $f_T: X\to \R: x\mapsto \norm{Tx}$ is bounded and
\[ K_n = \bigcap_{T\in\mathcal{F}}f_T^{-1}[\,[0,n]\,]. \]
By assumption, $X=\bigcup_{n\in\N} K_n$. As $X$ is a Banach space, and thus a complete metric space, we can apply the Baire category theorem, \ref{BaireCategory}, to conclude that there is a $K_n$ with non-empty interior (by contraposition of the Baire condition). Take $x_0\in K_n^\circ$, then $-x_0+K_n^\circ \subset K_{n2}$. So $\vec{0}\in (K_{2n})^\circ$ and we can find a $\rho$ such that $B(\vec{0},\rho)\subset K_{2n}$. By proposition \ref{existenceOperatorNorm} we have $\norm{T}\leq 2n/\rho$ for all $T\in\mathcal{F}$.
\end{proof}
\begin{corollary}[Banach-Steinhaus] \label{BanachSteinhaus}
Let $X$ be a Banach space and $Y$ a normed space. Let $T_n: X\to Y$ be a sequence of bounded operators. If $T_n$ converges pointwise to $T:X\to Y: Tx = \lim_n T_n x$, then $\sup_n\norm{T_n} <\infty$ and thus $T$ is bounded.
\end{corollary}
\begin{proof}
Any convergent sequence in a normed space is bounded, so we can apply the uniform boundedness principle.
\end{proof}

\subsection{Open mapping and closed graph theorems}

\begin{proposition} \label{openUnitBall}
Let $X,Y$ be Banach spaces and $T:X\to Y$ a surjective bounded operator.  Then the image of the open unit ball $B(\vec{0},1)\subset X$ contains an open ball about $\vec{0}\in Y$.
\end{proposition}
\begin{proof}
We first prove $0\in \overline{T[B(\vec{0},r)]}^\circ$ for every $r>0$: (TODO: make computations lemma.)
\begin{itemize}
\item Using $X = \bigcup_{n=1}^\infty B(\vec{0},n)$, we see by surjectivity
\[ Y = T[X] = T\left[\bigcup_{n=1}^\infty B(\vec{0},n)\right] = \bigcup_{n=1}^\infty T[B(\vec{0},n)]. \]
Because $Y$ has the Baire property (theorem \ref{BaireCategory}) and $Y$ is both open and non-empty, it may not be meagre, by lemma \ref{BaireEquivalents}. So for some $n\in\N$, $T[B(\vec{0},n)]$ is non-rare, meaning that $\overline{T[B(\vec{0},n)]}$ has non-empty interior.
\item Because
\[ \overline{T[B(\vec{0},n)]} = \overline{2nT[B(\vec{0},1/2)]} = 2n\overline{T[B(\vec{0},1/2)]}, \]
$\overline{T[B(\vec{0},1/2)]}$ must have non-empty interior. Let $B(y_0,\epsilon)\subset \overline{T[B(\vec{0},1/2)]}$.
\item Note $B(0,\epsilon) = y_0 - B(y_0,\epsilon) \subset \overline{T[B(\vec{0},1)]}$ and thus $B(0,r\epsilon) \subset \overline{T[B(\vec{0},r)]}$.
\end{itemize}
We then prove $\overline{T[B(\vec{0},1/2)]} \subset T[B(\vec{0}, 1)]$, proving the proposition.
\begin{itemize}
\item Choose some $y_0\in \overline{T[B(\vec{0},1/2)]}$. Then every neighbourhood $B(y_0,\epsilon/4)$ intersects $T[B(\vec{0},1/2)]$.
\item Then
\[ B(y_0,\epsilon/4) = y_0 - B(\vec{0},\epsilon/4) \subset y_0 - \overline{T[B(\vec{0},1/4)]}, \]
so $y_0 - \overline{T[B(\vec{0},1/4)]}$ intersects $T[B(\vec{0},1/2)]$. Take a $y_1 \in \overline{T[B(\vec{0},1/4)]}$ such that $y_0-y_1$ is in this intersection. Then we have an $x_0\in B(\vec{0},1/2)$ such that $T(x_0) = y_0-y_1$.
\item We can continue recursively choosing $y_{n+1}\in \overline{T[B(\vec{0}, 2^{-(n+1)})]}$ and $x_n \in B(\vec{0}, 2^{-n})$ such that $y_n-y_{n+1} = T(x_n)$.
\item Consider the sequence $\sum_{k=0}^nx_k$. It is a Cauchy sequence in $X$. Call its limit $x$. Then $x\in B(\vec{0},1)$.
\item Because $\norm{y_n}\leq 2^{-n}\norm{T}$, $(y_n)$ converges to zero. Then
\[ \left( T\left(\sum^n_{k=1}x_k\right) \right)_{n\in\N} = \left( y_0-y_{n+1} \right)_{n\in\N} \]
converges to $y_0$. Thus $T(x) = y_0 \in T[B(\vec{0},1)]$.
\end{itemize}
\end{proof}

\begin{proposition} \label{zeroInInteriorOfImageImpliesOpen}
Let $X,Y$ be normed spaces and $T: X\to Y$ a linear map. If $\vec{0}$ lies in the interior of $T[B(\vec{0},r)]$ for some $r>0$, then $T$ is open.
\end{proposition}
\begin{proof}
TODO: make computations lemma.
Given the assumption, $0$ lies in the interior of $T[B(\vec{0},\epsilon)]$ for all $\epsilon>0$.
Because $T[B(x,\epsilon)] = T(x) + T[B(\vec{0},\epsilon)]$, $T(x)$ lies in the interior of $T[B(x,\epsilon)]$, for all $x\in X$.
Thus for all neighbourhoods $U(x)\subset X$, $T(x)\subset T[U]^\circ$ and so $T[U] \subset T[U]^\circ$, so $T[U]$ is open.
\end{proof}

\begin{theorem}[Open mapping]
Let $X,Y$ be Banach spaces and $T:X\to Y$ a surjective bounded operator. Then $T$ is an open map.
\end{theorem}
\begin{proof}
This is the consequence of propositions \ref{openUnitBall} and \ref{zeroInInteriorOfImageImpliesOpen}.
\end{proof}
\begin{corollary}[Bounded inverse theorem] \label{boundedInverse}
Let $X,Y$ be Banach spaces. If $T:X\to Y$ is is continuous, linear and bijective, then $T$ is a homeomorphism.
\end{corollary}


\begin{proposition}
Let $T: \dom(T)\subset X\to Y$ be a bounded linear operator. Then
\begin{enumerate}
\item if $\dom(T)$ is a closed subset of $X$, then $T$ has closed graph;
\item if $T$ has closed graph and $Y$ is complete, then $\dom(T)$ is a closed subset of $X$.
\end{enumerate}
\end{proposition}
\begin{proof}
We use proposition \ref{closedGraphEquivalence} twice: First assume $(x_n)$ and $(Tx_n)$ converge to $x$ and $y$, respectively. Then $x\in\dom(T)$ by closure and $y = Tx$ by continuity.

Now assume $T$ has closed graph and $Y$ is complete. Take $x\in\overline{\dom(T)}$ and $(x_n)\subset \dom(T)$ converging to $x$. Since $T$ is bounded:
\[ \norm{Tx_n - Tx_m} = \norm{T(x_n-x_m)} \leq \norm{T}\norm{x_n-x_m}, \]
so $(Tx_n)$ is Cauchy by \ref{CauchyCriterion} and thus by completeness has a limit, say $y$. Then $Tx=y$ by continuity. Since $T$ has closed graph, $x\in\dom(T)$. So $\overline{\dom(T)}\subseteq \dom(T)$ and $\dom(T)$ is closed. 
\end{proof}

For closed graph theorem, see TVS.


\subsection{Compact operators}
\begin{proposition}
Let $L\in\Hom(V,W)$ with $V,W$ Banach spaces. Then $L$ is compact \textup{if and only if} the image of any bounded subset of $V$ under $L$ is totally bounded in $W$.
\end{proposition}
TODO proof


\subsubsection{Calkin algebra}
\begin{proposition}
Let $X$ be a Banach space. Then $\Compact(X)$ is a closed two-sided ideal in $\Bounded(X)$.
\end{proposition}
\begin{proof}
TODO + $*$-ideal for Hilbert spaces.
\end{proof}

\begin{definition}
Let $X$ be a Banach space. The \udef{Calkin algebra} is the quotient $\Bounded(X)/\Compact(X)$.
\end{definition}
TODO: quotient algebra ($[A][B] = [AB]$)

\begin{proposition}
Let $[T]\in\Bounded(X)/\Compact(X)$. Then the following are equivalent:
\begin{enumerate}
\item $[T]$ is invertible in the Calkin algebra;
\item $\exists S\in\Bounded(X):$ both $\vec{1}-TS$ and $\vec{1}-ST$ are compact;
\item $T$ has closed range and finite-dimensional kernel and cokernel. 
\end{enumerate}
\end{proposition}
\begin{proof}
Point 1. and 2. are easily equivalent: $[S]$ is an inverse of $[T]$ if and only if $[\vec{1}] = [S][T] = [ST]$ and $[\vec{1}] = [T][S] = [TS]$. Then
\[ [\vec{1}] = [ST] \iff [ST - \vec{1}] = [0] \qquad [\vec{1}] = [TS] \iff [TS - \vec{1}] = [0] \]
and $[F]=[0]$ if and only if $F$ is compact.

TODO
\end{proof}

\section{Unbounded operators}

\section{Bochner integration}

TODO: the Bochner integral is the unique extension of the integral of simple functions to the set of Bochner measurable functions???? (i.e.\ simple functions dense in Bochner space, with $L^1$ metric)
\begin{definition}
Let $(\Omega, \mathcal{A},\mu)$ be a measure space and $Y$ a normed vector space. Then a Bochner measurable function $f:\Omega\to Y$ is called \udef{Bochner integrable} if there exists a sequence of integrable simple functions $\seq{s_n}\subset\SF(\Omega,Y)$ such that
\[ \lim_{n\to\infty}\int_\Omega \norm{f-s_n}\diff{\mu} = 0. \]
Take such a sequence $\seq{s_n}$. The \udef{Bochner integral} of $f$ on $\Omega$ w.r.t. $\mu$ is defined as
\[ \int_\Omega f\diff{\mu} \defeq \lim_{n\to\infty}\int_\Omega s_n\diff{\mu}. \]
\end{definition}

\begin{lemma}
The Bochner integral is well-defined: let $\seq{s_n},\seq{t_n}\in \prescript{\N}{}{\SF(\Omega,Y)}$ be sequences such that
\[ \lim_{n\to\infty}\int_\Omega \norm{f-s_n}\diff{\mu} = 0 = \lim_{n\to\infty}\int_\Omega \norm{f-t_n}\diff{\mu}.  \]
Then
\begin{enumerate}
\item the limits $\lim_{n\to\infty}\int_\Omega s_n\diff{\mu}$ and $\lim_{n\to\infty}\int_\Omega t_n\diff{\mu}$ exist;
\item $\lim_{n\to\infty}\int_\Omega s_n\diff{\mu} = \lim_{n\to\infty}\int_\Omega t_n\diff{\mu}$.
\end{enumerate}
\end{lemma}
\begin{proof}
TODO
\end{proof}

\begin{proposition}[Bochner integrability criterion] \label{BochnerIntegrabilityCondition}
Let $(\Omega, \mathcal{A},\mu)$ be a measure space and $Y$ a normed vector space.

A Bochner measurable function $f$ is Bochner integrable \textup{if and only if}
\[ \int_\Omega \norm{f} \diff{\mu} < \infty. \]
\end{proposition}

\begin{proposition}
Linearity and monotonicity.
\end{proposition}

\begin{theorem}[Hille's theorem] \label{HilleTheorem}
Let $(\Omega, \mathcal{A},\mu)$ be a measure space, $X,Y$ normed vector spaces and $T: X\not\to Y$ a closed operator. If $T\circ f$ is integrable, then
\[ \int_\Omega (T\circ f)\diff{\mu} = T\left(\int_\Omega f\diff{\mu}\right). \]
\end{theorem}
\begin{proof}
TODO
\end{proof}
\begin{corollary} \label{boundedOperatorUnderIntegral}
If $T$ is bounded, then $T\circ f$ is integrable and
\[ \int_\Omega (T\circ f)\diff{\mu} = T\left(\int_\Omega f\diff{\mu}\right). \]
\end{corollary}
\begin{proof}
TODO: show that $T\circ f$ is integrable!
\end{proof}

TODO Dominated convergence.

\subsection{Integration of bounded operators}
\begin{lemma} \label{integralBoundedOperator}
Let $X$ be a normed space and $(\Omega, \mathcal{A},\mu)$ a measure space. Let $T: \Omega \to \Bounded(X)$ be a function. If $T$ is integrable, then for all $x\in X$, $Tx$ is integrable and
\[ \left(\int_\Omega T\diff{\mu}\right)x = \int_\Omega Tx\diff{\mu}. \]
\end{lemma}
\begin{proof}
The evaluation map $\evalMap_x$ is linear and bounded by $\norm{x}$ for all $x\in X$, so we can use \ref{boundedOperatorUnderIntegral}.
\end{proof}


\chapter{Spectral theory and functional calculus}
\section{Invariant subspaces}
\begin{definition}
Let $L\in \Hom(V)$ be an endomorphism. A subspace $U$ of $V$ is \udef{invariant} under $L$ if $T|_U$ is an endomorphism on $U$. In other words, $u\in U$ implies $Tu\in U$.
\end{definition}
Clearly this definition only works for endomorphisms, not for linear maps in general. This is true for the rest of the theory about eigenvalues and eigenvectors.
\begin{example}
Let $L\in \Hom(V)$. The following are invariant under $L$:
\begin{itemize}
\item $\{0\}$;
\item $\ker L$;
\item $\im L$.
\end{itemize}
\end{example}

\section{The spectrum}
TODO: eigenvalue problem $Lx = \lambda x$

generalised eigenvalue problem $Lx = \lambda T x$

nonstandard eigenvalue problem $A(\gamma)x = 0$.

TODO: consistency $\lambda \id - L$, not $L-\lambda \id$.
TODO: everything is now in $\C$.

\begin{definition}
Let $L: \dom(L)\subset V \to V$ be an operator on a complex normed vector space $V$.

For $\lambda\in\C$ the \udef{resolvent} $R_L(\lambda): \im(\lambda \id_V - L)\to\dom(L)$ is the left inverse of $\lambda \id_V - L$, if this inverse exists (i.e.\ if $\lambda \id_V - L$ is injective).
\begin{itemize}
\item The \udef{resolvent set} $\res(L)$ is the set
\begin{align*}
\res(L) &\defeq \setbuilder{\lambda\in \C}{R_L(\lambda)\in\Bounded(V)} \\
&= \setbuilder{\lambda\in \C}{\text{$R_L(\lambda)$ exists, has domain $V$ and is bounded}}.
\end{align*}
\item The \udef{spectrum} of $L$ is the complement of the resolvent set: $\spec(L) \defeq \C\setminus\rho(L)$.
\item The \udef{spectral radius} $\spr(L)$ is $\sup_{\lambda\in\spec(L)} |\lambda|$.
\end{itemize}
\end{definition}

\begin{lemma} \label{elementResolventSetNormedSpace}
Let $T$ be an operator on a normed vector space $V$. Then $\lambda \in \res(T)$ \textup{if and only if} $\lambda \id_V - T$ is surjective and bounded from below.
\end{lemma}
\begin{proof}
By \ref{boundedBelow}, $\lambda \id_V - T$ has a bounded inverse $(\lambda \id_V - T)^{-1}: \im(\lambda \id_V - T)\to V$ if and only if it is bounded below. In order for $\lambda$ to be in the resolvent set, we need $(\lambda \id_V - T)^{-1}$ to be defined everywhere, i.e. $\im(\lambda \id_V - T) = V$.
\end{proof}



\subsection{The three-way classification of the spectrum}
\begin{definition}
Let $L: \dom(L)\subset V \to V$ be an operator on a complex vector space $V$.

\begin{itemize}
\item The \udef{point spectrum} or \udef{discrete spectrum} $\pspec(L)$ contains the values of $\lambda$ where $\lambda \id_V - L$ fails to be injective, so the resolvent fails to exist. These values are called the \udef{eigenvalues} of $L$.

We call
\begin{itemize}
\item $\ker(\lambda \id_V - L)$ the \udef{multiplicity space} or \udef{geometric eigenspace} of $\lambda$; and
\item $\dim\ker(\lambda \id_V - L)$ the \udef{(geometric) multiplicity} of $\lambda$.
\end{itemize}
\item The \udef{continuous spectrum} $\cspec(L)$ is the set of all values of $\lambda\in\spec(L)$ such that the resolvent $R_L(\lambda)$ exists and is densely defined.
\item The \udef{residual spectrum} $\rspec(L)$ is the set of all values of $\lambda\in\spec(L)$ such that the resolvent $R_L(\lambda)$ exists, but is not densely defined.

We call
\begin{itemize}
\item $\im(\lambda \id_V - L)^\perp$ the \udef{deficiency subspace} of $\lambda$; and 
\item $\dim(\im(\lambda \id_V - L)^\perp)$ the \udef{deficiency} of $\lambda$.
\end{itemize}
\end{itemize}
The sets $\pspec(T), \cspec(T)$ and $\rspec(T)$ are disjoint.
\end{definition}
In finite dimensions we know that
\[ \text{$\lambda \id_V - L$ is surjective} \quad\iff\quad \text{$\lambda \id_V - L$ is injective} \]
and all linear operator are bounded.
So in this case there can only ever be a point spectrum.

\begin{proposition}
If $T$ is an operator on a Banach space that is not closed, then $\spec(T) = \C$.
\end{proposition}
\begin{proof}
We can find a sequence $x_n \to x$ such that $Tx_n \to y$, but $Tx \neq y$. Then for all $\lambda\in\C$ we have $z_n = (\lambda\id - T)x_n \to \lambda x - y$. If $R_T(\lambda)$ was a bounded inverse of $(\lambda\id - T)$, then $R_T(\lambda)\circ(\lambda\id - T)x_n \to R_T(\lambda)(\lambda x - y)$. We need to show that $R_T(\lambda)(\lambda x - y) \neq x$. Indeed
\begin{align*}
R_T(\lambda)(\lambda x - y) &= R_T(\lambda)(\lambda x - Tx + Tx - y) \\
&= R_T(\lambda)(\lambda x - Tx) + R_T(\lambda)(Tx - y) \\
&= x + R_T(\lambda)(Tx - y),
\end{align*}
and $R_T(\lambda)(Tx - y) \neq 0$, because $Tx - y \neq 0$ and the kernel of $R_T(\lambda)$ is trivial because it is injective. 
\end{proof}

\begin{example}
Closed operators may also have empty resolvent set. \url{https://math.stackexchange.com/questions/3262168/closed-operator-with-trivial-resolvent-set}
\end{example}

So spectral theory is only interesting for closed operators. In this case the three-way classification exhausts the possibilities: (only on Banach spaces??)

\begin{proposition} \label{closedOperatorBanachSpaceSpectrumCriterion}
Let $X$ be a Banach space and $T$ a closed linear operator on $X$. Then $\lambda \in \spec(T)$ \textup{if and only if} $\lambda \id_X - T: \dom(T) \to V$ is not bijective.
\end{proposition}
\begin{proof}
If $\lambda \id_X - T$ is not bijective, then clearly $\lambda \in \spec(T)$.

Conversely, assume $\lambda \id_X - T$ is bijective. Then $(\lambda \id_X - T)^{-1}: X\to \dom(T)$ is closed by \ref{algebraClosedOperators} and has as domain a Banach space, so it is bounded by the closed graph theorem \ref{closedGraphTheorem}.
\end{proof}
\begin{corollary}
Let $T$ a closed operator on a Banach space. Then
\[ \spec(T) = \pspec(T) \cup \cspec(T) \cup \rspec(T). \]
\end{corollary}


\begin{proposition}
Let $T:X\to X$ be an operator on a Banach space and $\lambda\in\cspec$, then $R_\lambda(T)$ is unbounded.
\end{proposition}
\begin{proof}
If $R_\lambda(T)$ is bounded, $\lambda \id_V - T$ then is bounded below by lemma \ref{boundedBelow} and has closed range by proposition \ref{boundedBelowClosedRange}. Then because $\im(\lambda \id_V - T)$ is dense, this means $T$ is surjective, which is a contradiction because then $\lambda\in\res(T)$.
\end{proof}

\subsection{Resolvents}

\begin{proposition}[Resolvent identity]
Let $T$ be a linear operator and $\lambda,\mu\in\C$ such that $R_T(\lambda), R_T(\mu)$ exist. Then $R_T(\lambda)$ and $R_T(\mu)$ commute and
\[ \frac{R_T(\lambda) - R_T(\mu)}{\lambda-\mu} = -R_T(\lambda)R_T(\mu). \]
\end{proposition}
\begin{proof}
The commutativity of the resolvents follows from \ref{commutationInverse}.

We calculate
\begin{align*}
R_T(\lambda) - R_T(\mu) &= R_T(\lambda)R_T(\mu)(\mu\id - T) - R_T(\mu)R_T(\lambda)(\lambda\id - T) \\
&= \mu R_T(\lambda)R_T(\mu) - R_T(\lambda)R_T(\mu) T - \lambda R_T(\mu)R_T(\lambda) + R_T(\mu)R_T(\lambda) T \\
&= (\mu - \lambda)R_T(\lambda)R_T(\mu).
\end{align*}
\end{proof}
\begin{corollary}
Let $T$ ba a linear operator. Then
\begin{enumerate}
\item $R_T(\lambda)$ is holomorphic in $\rho(T)$;
\item $R_T'(\lambda) = -R_T(\lambda)^2$;
\item $R_T^{(n)}(\lambda) = n!(-1)^n R_T(\lambda)^{n+1}$ for all $n\in \N$.
\end{enumerate}
\end{corollary}
\begin{corollary} \label{firstNeumannSeries}
Let $T$ ba a linear operator and $\lambda_0\in \rho(T)$. Then the Taylor expansion of $R_T(\lambda)$ around $\lambda_0$ is
\begin{align*}
R_T(\lambda) &= \sum_{n=0}^\infty (\lambda-\lambda_0)^n(-1)^nR_T(\lambda_0)^{n+1} \\
&= R_T(\lambda_0) \sum_{n=0}^\infty \Big((\lambda_0-\lambda)R_T(\lambda_0)\Big)^n \\
&= \frac{R_T(\lambda_0)}{1+(\lambda-\lambda_0)R_T(\lambda_0)},
\end{align*}
with convergence radius $1/\norm{R_T(\lambda_0)}$.
\end{corollary}
\begin{proof}
If $|(\lambda_0-\lambda)| \leq 1/\norm{R_T(\lambda_0)}$, then $(\lambda_0-\lambda)R_T(\lambda_0)$ is a contraction and we can use the Neumann series expansion \ref{operatorNeumannSeries}, which gives the last equality.
\end{proof}
\begin{corollary} \label{resolventNormDistanceToSpectrum}
For all $\lambda\in\res(T)$, we have $d(\lambda, \spec(T)) \geq \norm{R_T(\lambda)}^{-1}$.
\end{corollary}
\begin{corollary}
The resolvent set $\res(T)$ is open.
\end{corollary}


\begin{proposition} \label{boundedPerturbationClosedOperator}
Let $S: \dom(S)\subseteq X\to X$ be a closed, bijective linear operator on a Banach space $X$. Let $T\in\Bounded(X)$ be a bounded operator with $\norm{T} < \norm{S^{-1}}^{-1}$, then $S-T$ is invertible and
\[ (S-T)^{-1} = S^{-1}\sum_{n=0}^\infty \left(S^{-1}T\right)^n. \]
Also $\dom((S-T)^{-1}) = X$.
\end{proposition}
The assumptions are enough to guarantee $S^{-1}\in \Bounded(X)$, so $\norm{S^{-1}}$ makes sense.
\begin{proof}
By injectivity, we can define $S^{-1}$. It is closed by \ref{algebraClosedOperators} and has closed domain, so it is bounded by the closed graph theorem \ref{closedGraphTheorem}. Since $\norm{S^{-1}T} \leq \norm{S^{-1}}\;\norm{T} < 1$, it follows that by \ref{operatorNeumannSeries} that $\id - S^{-1}T$  has bounded inverse and that we can expand its bounded inverse as a Neumann power series. So we can calculate
\[ (S-T)^{-1} = S^{-1}(\id - S^{-1}T)^{-1} = S^{-1}\sum_{n=0}^\infty \left(S^{-1}T\right)^n. \]
\end{proof}
\begin{corollary} \label{secondNeumannSeries}
Let $T$ be a bounded operator on a Banach space $X$. For $|\lambda|>\norm{T}$ the resolvent $R_T(\lambda)$ is bounded and given by
\[ R_T(\lambda) = \sum_{n=0}^\infty\frac{T^n}{\lambda^{n+1}} \]
with uniform convergence. The norm is bounded by
\[ \norm{R_T(\lambda)} \leq \frac{1}{|\lambda|-\norm{T}}. \]
\end{corollary}
\begin{proof}
We use the proposition with $S = \lambda\id$. The norm bound follows from the Neumann series expansion \ref{operatorNeumannSeries}.
\end{proof}
\begin{corollary}
Let $T$ be a bounded linear operator on a Banach space. Then
\begin{enumerate}
\item $\spec(T)\subset [-\norm{T}, \norm{T}]$;
\item $\spec(T)$ is compact.
\end{enumerate}
\end{corollary}
For the point spectrum a simpler argument also leads to $\pspec(T)\subset [-\norm{T}, \norm{T}]$: let $\lambda$ be an eigenvalue with eigenvector $x$. Then
\[ |\lambda|\;\norm{x} = \norm{\lambda x} = \norm{Tx} \leq \norm{T}\;\norm{x}. \]

\begin{proposition}
Let $T$ be an injective operator with dense range. Then for all $\lambda\neq 0$
\[ R_{T^{-1}}(\lambda^{-1}) = -\lambda T R_{T}(\lambda) = \lambda -\lambda^2 R_T(\lambda). \]
\end{proposition}
\begin{proof}
This is a reformulation of the calculation
\[ \frac{1}{\lambda^{-1} - T^{-1}} = \frac{\lambda T}{\lambda T}\frac{1}{\lambda^{-1} - T^{-1}} = \frac{\lambda T}{T - \lambda} = \frac{\lambda T - \lambda^2 + \lambda^2}{T - \lambda} = \frac{\lambda\cancel{(T - \lambda)}}{\cancel{T - \lambda}} + \frac{\lambda^2}{T - \lambda} = \lambda - \lambda^2 R_T(\lambda). \]
TODO: make rigourous!!
\end{proof}
\begin{corollary}
Let $T$ be an injective operator with dense range. Then for all $\lambda\neq 0$
\begin{enumerate}
\item $\spec(T^{-1})\setminus\{0\} = (\spec(T)\setminus \{0\})^{-1}$;
\item $\pspec(T^{-1})\setminus\{0\} = (\pspec(T)\setminus \{0\})^{-1}$.
\end{enumerate}
\end{corollary}

\subsection{Approximate spectrum and Weyl sequences}
\begin{definition}
The set of all $\lambda$ such that $T-\lambda \id_V$ is not bounded from below is called the \udef{approximate point spectrum} $\apspec$.

If $\lambda\in\apspec(T)$, then $\lambda$ is an \udef{approximate eigenvalue} of $T$.
\end{definition}

\begin{proposition} \label{approximateSpectrum}
Let $T$ be an operator. Then
\begin{enumerate}
\item $\apspec(T) \subset \spec(T)$;
\item if $T$ is closed, then $\pspec(T)\cup\cspec(T)\subset\apspec(T)$.
\end{enumerate}
\end{proposition}
\begin{proof}
(1) Assume $\lambda \notin \spec(T)$. Then $(T-\lambda \id_V)^{-1}$ is bounded, so its inverse $T-\lambda \id_V$ is bounded below by \ref{boundedBelow} and $\lambda\in \apspec(T)$.

(2) Assume $\lambda\notin \apspec(T)$, 
so $T-\lambda \id_V$ is bounded below. Then $T-\lambda \id_V$ is injective by \ref{boundedBelow} and $\lambda\notin\pspec(T)$. By proposition \ref{boundedBelowClosedRange} the range $\im(T-\lambda \id_V)$ is closed, so it cannot be a proper dense subset of $X$ and $\lambda\notin\cspec(T)$.
\end{proof}

\begin{proposition}[Weyl sequences] \label{WeylSequence}
Let $T$ be an operator on a normed vector space $V$. Then $\lambda \in \apspec(T)$ \textup{if and only if} there exists a sequence of unit vectors $(e_n)_{n\in\N}$ for which
\[ \lim_{n\to\infty}\norm{\lambda e_n - Te_n} = 0. \]
\end{proposition}
\begin{proof}
Assume there is such a sequence $(e_n)_{n\in\N}$. Then for all $\epsilon>0$, we can find a unit  vector $e_k$ such that $\norm{(\lambda \id_V - T)e_n} \leq \epsilon = \epsilon \norm{e_n}$. This is clearly not bounded below.

This other direction is just an inversion of this argument.
\end{proof}
A sequence as described in \ref{WeylSequence} is called a \udef{Weyl sequence} for $\lambda$. This gives meaning to the name ``approximate eigenvalue''.

\begin{corollary}
Let $T$ be an operator. Then $\sigma(T)\cap \overline{\res(T)} \subseteq \apspec(T)$.
\end{corollary}
\begin{proof}
Let $\lambda \in \sigma(T)\cap \overline{\res(T)}$. We show there is a Weyl sequence for $\lambda$.

We can find a sequence $\seq{\lambda_n}\subseteq \res(T)$ such that $\lambda_n \to \lambda$.
Now $d(\lambda_n, \spec(T)) \to 0$, so by \ref{resolventNormDistanceToSpectrum}, we can find a sequence of unit vectors $\seq{x_n}$ such that $\norm{R_T(\lambda_n)x_n} \to \infty$. Now we can rescale $\seq{x_n}$ such that $\norm{R_T(\lambda_n)x_n} = 1$.

Then $\norm{x_n}\to 0$, and hence
\begin{align*}
\norm{(\lambda\id - T)R_T(\lambda_n)x_n} &= \norm{\frac{\lambda\id - T}{\lambda_n\id - T}x_n} \\
&= \norm{\frac{(\lambda\id - T) + (\lambda_n\id - T) - (\lambda_n\id - T)}{\lambda_n\id - T}x_n} \\
&= \norm{\left(\id + \frac{(\lambda\id - T) - (\lambda_n\id - T)}{\lambda_n\id - T}\right)x_n} \\
&= \norm{\left(\id + \frac{\lambda\id - \lambda_n\id}{\lambda_n\id - T}\right)x_n} \\
&= \norm{x_n + (\lambda\id - \lambda_n\id)R_T(\lambda_n)x_n} \\
&\leq \norm{x_n} + |\lambda - \lambda_n|\;\norm{R_T(\lambda_n)x_n} \to 0.
\end{align*}
Thus $\seq{R_T(\lambda_n)x_n}$ is the kind of sequence we were looking for.
\end{proof}


\subsection{Parts of the spectrum}

\begin{example}
Operator with empty spectrum. TODO \url{https://math.stackexchange.com/questions/1344287/example-operator-with-empty-spectrum}.
\end{example}

\subsubsection{The point spectrum: eigenvalue and eigenvectors}
In this section we study invariant subspaces with dimension $1$, i.e.\ subspaces $U= \Span\{v\}$ such that
\[ Lv = \lambda v. \]
\begin{definition}
Suppose $L\in \Hom_{\mathbb{F}}(V)$.
\begin{itemize}
\item  A scalar $\lambda\in \mathbb{F}$ is called an \udef{eigenvalue} of $L$ if there exists a $v\in V$ such that $v\neq 0$ and $Lv = \lambda v$.
\item Such a vector $v$ is called an \udef{eigenvector}.
\item The set of all eigenvectors associated with an eigenvalue $\lambda$ is called the \udef{eigenspace} $E_\lambda(L)$. Because
\[ E_\lambda(L) = \ker(L-\lambda \id_V) \]
it is indeed a vector space.

The dimension of $E_\lambda(L)$ is the \udef{geometric multiplicity} of $\lambda$.
\end{itemize}
\end{definition}
\begin{proposition}
Let $L\in \Hom_\mathbb{F}(V)$ and $\lambda\in \mathbb{F}$, then
\[ \text{$\lambda$ is an eigenvalue of $L$} \qquad \iff \qquad \text{$\lambda$ is in the point spectrum $\pspec(L)$.} \]
\end{proposition}
\begin{proof}
The equation $Lv = \lambda v$ is equivalent to $(L-\lambda \id_V)v = 0$.
\end{proof}

\begin{proposition}
Let $L\in\Hom(V)$ be an operator on some vector space. Suppose $\lambda_1, \ldots, \lambda_m$ are distinct eigenvalues of $L$ and $v_1,\ldots, v_m$ are corresponding eigenvectors. Then $\{v_1,\ldots, v_m\}$ is linearly independent.
\end{proposition}
\begin{proof}
The proof goes by contradiction. Assume $\{v_1,\ldots, v_m\}$ is linearly dependent. Let $k$ be the smallest positive integer such that
\[ v_k \in \Span\{v_1,\ldots, v_{k-1}\}. \]
So there exists a nontrivial linear combination
\[ v_k = a_1v_1+\ldots +a_{k-1}v_{k-1}. \]
Applying $L$ to both sides gives
\[ \lambda_kv_k = a_1\lambda_kv_1+\ldots +a_{k-1}\lambda_kv_{k-1}. \]
Multipliying the previous combination by $\lambda_k$ and subtracting both equations gives
\[ 0= a_1(\lambda_k-\lambda_1)v_1 +\ldots + a_{k-1}(\lambda_k - \lambda_{k-1})v_{k-1}. \]
By assumption of linear independence of $\{v_1,\ldots, v_{k-1}\}$ this combination must be trivial, however none of the $(\lambda_k-\lambda_i)$ can be zero, so all the $a_i$ must be zero. This is a contradiction with the assumption of linear dependence.
\end{proof}
\begin{corollary}
For each operator on $V$, the set of distinct eigenvalues has at most cardinality $\dim V$.
\end{corollary}
\begin{corollary}
Let $L\in\Hom(V)$. Suppose $\lambda_1, \ldots, \lambda_m$ are distinct eigenvalues of $L$. Then
\[ E_{\lambda_1}(L) \oplus \ldots \oplus E_{\lambda_m}(L) \]
is a direct sum. Furthermore, the sum of geometric multiplicities is less than or equal to the dimension of $V$:
\[ \dim E_{\lambda_1}(L) + \ldots + \dim E_{\lambda_m}(L) \leq \dim V. \]
\end{corollary}


\subsubsection{Residual spectrum}
\begin{proposition}
Let $L$ be a densely defined linear operator on a Hilbert space. If $\lambda$ is in the residual spectrum of $L$ with deficiency $m$, then $\overline{\lambda}$ is in the point spectrum of $L^*$ with multiplicity $m$.
\end{proposition}
\begin{proof}
By \ref{kernelImageAdjoint} we have
\[ \im(\lambda \id - L)^\perp = \ker(\lambda\id - L)^* = \ker(\overline{\lambda}\id - L^*). \]
\end{proof}

\subsubsection{Compression spectrum}
\begin{definition}
The set of $\lambda$ for which $T-\lambda I$ does not have dense range is the \udef{compression spectrum} $\cpspec(T)$ of $T$.
\end{definition}
Then $\rspec(T) = \cpspec(T)\setminus\pspec(T)$.

\subsubsection{The essential spectrum}
TODO \url{https://en.wikipedia.org/wiki/Spectrum_(functional_analysis)#Classification_of_points_in_the_spectrum}


\subsection{The spectral radius}
\begin{definition}
The \udef{spectral radius} $\spr(T)$ of a operator $T$ is given by
\[ \spr(T) \defeq \sup_{\lambda\in\spec(T)}|\lambda|. \]
\end{definition}


\subsection{The spectrum of operators on Hilbert spaces}
\begin{proposition}
Let $T \in \Bounded(H)$ for some Hilbert space $H$. Then
\begin{enumerate}
\item $\spec(T) \neq \emptyset$;
\item $\rho(T) = \overline{\rho(T^*)}$, where the bar denotes complex conjugation.
\end{enumerate}
\end{proposition}
\begin{proof}
(1) Let $x,y\in H$ and define
\[ f(\lambda) = \inner{x,R_\lambda(T)y}. \]
If $\spec(T) = \emptyset$, then $f$ is an entire function. Now
\[ \norm{R_\lambda(T)} \leq \frac{1}{|\lambda| - \norm{T}} \to 0 \quad\text{as}\quad |\lambda| \to \infty. \]
By Liouville's theorem (TODO ref) we must have $f\equiv 0$. Because the $x,y$ we arbitrary we must have $R_\lambda(T)y = 0$ for all $y\in H$, such that $R_\lambda(T)$ is not injective, which is impossible as it is an inver\begin{proposition}
Let $K$ be a compact operator on a Banach space. Then
\[ \spec(K)\setminus\{0\} = \pspec(K)\setminus\{0\}. \]
\end{proposition}
\begin{proof}
For all $\lambda\neq 0$, we have that $\lambda\id - K$ is Fredholm with index zero (and thus bounded). Then by the Fredholm alternative \ref{FredholmAlternative} $\lambda\id - K$ is either bijective or neither injective nor surjective, meaning $\lambda$ is either in $\rho(T)$ or in $\pspec(T)$. 
\end{proof}

\begin{proposition} \label{spectrumCompactOperator}
Let $K$ be a compact operator on a Banach space $X$. Then
\begin{enumerate}
\item for all $\lambda\in\spec(K)\setminus\{0\}$ there exists a least $m$ such that $\ker(\lambda\id- K)^m = \ker(\lambda\id- K)^{m+1}$. This space is finite dimensional and reducing for $K$;
\item for $\alpha > 0$ the number of eigenvalues $\lambda$ such that $|\lambda|\geq \alpha$ is finite;
\item $0$ is the only accumulation point; if $X$ is infinite dimensional, then $0\in\spec(K)$;
\item $\spec(K)$ is at most countably infinite;
\item every $\lambda \in \spec(K)\setminus \{0\}$ is a pole of the resolvent $R_K$.
\end{enumerate}
\end{proposition}
\begin{proof}
\url{https://en.wikipedia.org/wiki/Spectral_theory_of_compact_operators}
\end{proof}
TODO: if $K$ is a self-adjoint compact operator on a Hilbert space $H$, then $H$ has an orthonormal basis of eigenvectors of $K$.
se.

(2) Take $\lambda\in\rho(T)$. Then
\[ ((\lambda\id - A)^{-1})^* = (\overline{\lambda}\id - A^*)^{-1} \]
so $\overline{\lambda}\in\rho(T^*)$ iff $((\lambda\id - A)^{-1})^*$ is bounded iff $(\lambda\id - A)^{-1}$ is bounded iff $\lambda\in \rho(T)$.
\end{proof}

\begin{lemma} \label{eigenspaceOrthogonalAdjoint}
Let $L$ be a densely defined operator on a Hilbert space $H$. Take $\lambda\in \pspec(L)$ and $\mu\in\pspec(L^*)$. If $\lambda \neq \overline{\mu}$, then
\[ \ker(\lambda\id - L)\perp \ker(\mu\id - L^*). \]
\end{lemma}
\begin{proof}
Take non-zero eigenvectors $x,y$ such that $Ax = \lambda x$ and $A^*y = \mu y$. Then
\[ \lambda \inner{y,x} = \inner{y,\lambda x} = \inner{y, Ax} = \inner{A^*y,x} = \inner{\mu y,x} = \overline{\mu}\inner{y,x}. \]
So we have $(\lambda - \overline{\mu})\inner{y,x} = 0$.
\end{proof}

\begin{proposition} \label{adjointSpectrumNoResidual}
Let $L$ be a densely defined operator on a Hilbert space $H$. Then the following are equivalent:
\begin{enumerate}
\item the residual spectrum of $L$ is empty;
\item $\overline{\pspec(L^*)} \subseteq \pspec(L)$;
\end{enumerate}
as are the following:
\begin{enumerate}
\item the residual spectrum of $L^*$ is empty;
\item $\pspec(L) \subseteq \overline{\pspec(L^*)}$.
\end{enumerate}
In particular all these statements hold if $L$ is normal.
\end{proposition}
\begin{proof}
Consider, for all $x\in \dom(L), y\in\dom(L^*)$, the equality
\[ \inner{(\lambda\id-L)x,y} = \inner{x,(\overline{\lambda}\id-L^*)y}. \]
We can make the following inferences:
\begin{itemize}
\item If $\lambda\in \overline{\pspec(L^*)}$, then the equality holds in particular for all eigenvectors $y$. This implies $\inner{(\lambda\id-L)x,y} = 0$. By \ref{perpToDenseSet} $\im(\lambda\id-L)$ may then not be dense, so it cannot be injective because the residual spectrum of $L$ is empty.
\item Assume $\lambda\id-L$ injective and take  $y\perp \im(\lambda\id-L)$. Then by the equality $\inner{x, (\overline{\lambda}\id - L^*)y} = 0$ for all $x\in\dom(L)$, which is dense. So $(\overline{\lambda}\id - L^*)y = 0$ by \ref{perpToDenseSet}. Now $\lambda\notin \pspec(L)$, so $\overline{\lambda}\notin \pspec(L^*)$. Thus $y = 0$ and $\im(\lambda\id-L)^\perp = \{0\}$, meaning $\im(\lambda\id-L)$ is dense.
\end{itemize}
The arguments for the second set of statements are similar.

If $L$ is normal, then $\ker(\lambda \id - L) = \ker{\overline{\lambda}\id -L^*}$ by \ref{equalityKernelAdjointNormal}, so $\pspec(L) = \overline{\pspec(L^*)}$.
\end{proof}

\begin{proposition}
Let $T$ be a closed, densly defined operator on a Hilbert space.
\begin{enumerate}
\item If $\lambda\in\rho(T)$, then $\overline{\lambda}\in\rho(T^*)$.
\item If $\lambda\in\rspec(T)$, then $\overline{\lambda}\in\pspec(T^*)$.
\item If $\lambda\in\pspec(T)$, then $\overline{\lambda}\in\rspec(T^*)\cup\pspec(T^*)$.
\end{enumerate}
\end{proposition}
\begin{proof}
TODO Compare with \ref{adjointSpectrumNoResidual}. CLosure necessary?
\end{proof}


\begin{proposition}
Let $T$ be a unitary operator. Then
\begin{enumerate}
\item $\rspec(T) = \emptyset$;
\item $\spec(T) \subset \setbuilder{\lambda\in\C}{|\lambda| = 1}$.
\end{enumerate}
\end{proposition}
TODO: move to more general place??

\begin{lemma}
The eigenvalues of a bounded dissipative linear operator
lie in the half-plane $\Im\lambda \geq 0$.
\end{lemma}

\subsubsection{Rayleigh quotient}
\begin{lemma}
Let $L$ be an operator on a Hilbert space. If $x$ is an eigenvector with eigenvalue $\lambda$, then
\[ J_L(x) = \lambda. \]
\end{lemma}
\begin{proof}
Let $x$ be an eigenvector with eigenvalue $\lambda$, then
\[ J_L(x) = \frac{\inner{x,Lx}}{\inner{x,x}} = \lambda \frac{\inner{x,x}}{\inner{x,x}} = \lambda. \]
\end{proof}

\begin{proposition}
If $U$ is unitary, then $\spec(U)\subset \mathbb{T}$.
\end{proposition}

\section{Spectral theory for types of operators}
\subsection{Compact operators}

\begin{proposition}
Let $K$ be a compact operator on a Banach space. Then
\[ \spec(K)\setminus\{0\} = \pspec(K)\setminus\{0\}. \]
\end{proposition}
\begin{proof}
For all $\lambda\neq 0$, we have that $\lambda\id - K$ is Fredholm with index zero (and thus bounded). Then by the Fredholm alternative \ref{FredholmAlternative} $\lambda\id - K$ is either bijective or neither injective nor surjective, meaning $\lambda$ is either in $\rho(T)$ or in $\pspec(T)$. 
\end{proof}

\begin{proposition} \label{spectrumCompactOperator}
Let $K$ be a compact operator on a Banach space $X$. Then
\begin{enumerate}
\item for all $\lambda\in\spec(K)\setminus\{0\}$ there exists a least $m$ such that $\ker(\lambda\id- K)^m = \ker(\lambda\id- K)^{m+1}$. This space is finite dimensional and reducing for $K$;
\item for $\alpha > 0$ the number of eigenvalues $\lambda$ such that $|\lambda|\geq \alpha$ is finite;
\item $0$ is the only accumulation point; if $X$ is infinite dimensional, then $0\in\spec(K)$;
\item $\spec(K)$ is at most countably infinite;
\item every $\lambda \in \spec(K)\setminus \{0\}$ is a pole of the resolvent $R_K$.
\end{enumerate}
\end{proposition}
\begin{proof}
\url{https://en.wikipedia.org/wiki/Spectral_theory_of_compact_operators}
\end{proof}
TODO: if $K$ is a self-adjoint compact operator on a Hilbert space $H$, then $H$ has an orthonormal basis of eigenvectors of $K$.


\subsection{Multiplication operators}
\begin{definition}
Let $(\Omega, \mathcal{A}, \mu)$ be a measure space. A \udef{multiplication operator} is an operator of the form
\[ T: L^p(\Omega, \mu) \to L^p(\Omega, \mu): u(x) \mapsto a(x)u(x) \]
for some $a\in L^\infty(\Omega,\mu)$
\end{definition}

\begin{proposition}
Let $T: L^p(\Omega, \mu) \to L^p(\Omega, \mu): u \mapsto a\cdot u$ be a multiplication operator. Then
\[ \norm{T} = \norm{a}_{L^\infty}. \]
\end{proposition}
\begin{proof}
From the inequality $\norm{Tu}_{L^p}\leq \norm{a}_{L^\infty}\norm{u}_{L^p}$ we get $\norm{T} \leq \norm{a}_{L^\infty}$.

TODO
\end{proof}

\begin{lemma}
Let $T: L^2(\Omega, \mu) \to L^2(\Omega, \mu): u \mapsto a\cdot u$ be a multiplication operator with $a\in L^\infty(\Omega,\mu)$. Then $T^*$ is the multiplication operator
\[ T^*: L^2(\Omega, \mu) \to L^2(\Omega, \mu): u \mapsto \overline{a}\cdot u. \]
\end{lemma}
\begin{proof}
From 
\[ \inner{Tu,v} = \int_\Omega a\cdot u \cdot \overline{v}\diff{\mu} = \int_\Omega u \cdot \overline{\overline{a}\cdot v}\diff{\mu} \]
it follows that $T^*v = \overline{a}\cdot v$.
\end{proof}
\begin{corollary}
Then
\begin{enumerate}
\item $T$ is self-adjoint if $a$ is real-valued;
\item $T$ is skew-adjoint if $a$ is purely imaginary;
\item $T$ is unitary if $|a(x)| \equiv 1$.
\end{enumerate}
\end{corollary}

Let $E_\lambda$ be the level set
\[ E_\lambda = \setbuilder{x\in\Omega}{a(x) = \lambda} \]

\begin{proposition}
Let $T: L^2(\Omega, \mu) \to L^2(\Omega, \mu): u\mapsto a\cdot u$ be a multiplication operator with $a\in \cont(\Omega)$. Then
\begin{enumerate}
\item $\pspec(T) = \setbuilder{\lambda\in \im(a)}{\mu(E_\lambda)>0}$;
\item $\cspec(T) = \setbuilder{\lambda\in \overline{\im(a)}}{\mu(E_\lambda) = 0}$;
\item $\rspec(T) = \emptyset$;
\item $\rho(T) = \C\setminus \overline{\im(T)}$.
\end{enumerate}
\end{proposition}
\begin{proof}
TODO
\end{proof}

\subsection{Dissipative operators}
\begin{definition}
Let $T\in \Lin(V, W)$ be a linear operator between Banach spaces. Then $T$ is called \udef{dissipative} if $\lambda\id-T$ is bounded below by $\lambda$ for all $\lambda>0$:
\[ \norm{(\lambda\id-T)x} \geq \lambda\norm{x} \]
for all $x\in\dom(T)$.
\end{definition}

\begin{lemma} \label{dissipativeResolventBound}
Let $T\in \Lin(V, W)$ be an operator between Banach spaces. Then $T$ is dissipative \textup{if and only if} for all $\lambda>0$ the resolvent $R_T(\lambda): \im(T)\to V$ exists and is bounded by $\norm{R_T(\lambda)} \leq \lambda^{-1}$.
\end{lemma}
\begin{proof}
If $T$ is dissipative, then the result is given by \ref{boundedBelow}.

Assume $R_T(\lambda): \im(T)\to V$ exists. Then
\[ \lambda\norm{x} = \lambda\norm{R_T(\lambda)(\lambda\id-T)x} \leq \lambda \norm{R_T(\lambda)}\,\norm{(\lambda\id-T)x} \leq \lambda\lambda^{-1}\norm{(\lambda\id-T)x} = \norm{(\lambda\id-T)x}. \]
\end{proof}

Thus $\lambda>0$ is in $\res(T)$ if and only if $\lambda\id - T$ is surjective.

\begin{proposition} \label{spectrumDissipativeOperator}
Let $T\in \Lin(V, W)$ be a dissipative operator between Banach spaces. Then either $]0,+\infty[\,\perp \res(T)$ or $]0,+\infty[\,\subseteq \res(T)$.
\end{proposition}
\begin{proof}
We need to show that $\lambda\in\res(T)$ for some $\lambda >0$, then $\lambda\id - T$ is surjective for all $\lambda>0$.

Assume $\lambda\in\res(T)$ for some $\lambda >0$. Then \ref{dissipativeResolventBound} and \ref{firstNeumannSeries} combine the give $]0, 2\lambda[ \subseteq \res(T)$. We can repeat this to cover the whole of $]0,+\infty[$.
\end{proof}

\begin{proposition} \label{closureDissipativeOperator}
Let $T\in \Lin(V, W)$ be a dissipative operator between Banach spaces. Then the following are equivalent:
\begin{enumerate}
\item $T$ is closed;
\item $\im(\lambda\id - T)$ is closed for some $\lambda > 0$;
\item $\im(\lambda\id - T)$ is closed for all $\lambda > 0$.
\end{enumerate}
\end{proposition}
\begin{proof}
For all $\lambda\in\R$, we have that $T$ is closed iff $\lambda\id - T$ is closed iff $(\lambda\id - T)^{-1}: \im(\lambda\id - T) \to V$ is closed by \ref{algebraClosedOperators}.

Now closedness of $\lambda\id - T$ implies $\im(\lambda\id - T)$ is closed by \ref{boundedBelowClosedRange}. Conversely, if $\im(\lambda\id - T)$ is closed, then $\lambda\id - T$ is closed by the closed graph theorem \ref{closedGraphTheorem}.
\end{proof}

\begin{proposition}
Let $T\in \Lin(V)$ be a dissipative operator on a Banach space $V$. If $\im(T)\subseteq \overline{\dom(T)}$, then
\begin{enumerate}
\item $T$ is closable;
\item its closure $\overline{T}$ is dissipative;
\item $\im(\lambda\id - \overline{T}) = \overline{\im(\lambda\id - T)}$ for all $\lambda >0$.
\end{enumerate}
\end{proposition}
In particular $\im(T)\subseteq \overline{\dom(T)}$ holds whenever $T$ is densely defined.
\begin{proof}
(1) We use \ref{closableCriterion}. Assume $\seq{x_n}\to 0$ and $\seq{Tx_n}\to v$. We need to show that $v=0$. Because $T$ is dissipative, we have
\[ \norm{\lambda(\lambda\id-T)x_n + (\lambda\id-T)w} = \norm{(\lambda\id-T)(\lambda x_n -w)} \geq \lambda\norm{\lambda x_n + w} \]
for all $w\in \in\dom(T)$ and all $\lambda>0$. Taking the limit $n\to \infty$ gives
\[ \norm{-\lambda v +(\lambda\id- T)w} \geq \lambda\norm{w}, \qquad\text{and hence}\qquad \norm{w - v - \frac{1}{\lambda}Tw} \geq w. \]
Taking the limit $\lambda \to \infty$ gives $\norm{w-v}\geq \norm{w}$. Now $y\in \overline{\im(T)} \subseteq \overline{\dom(T)}$. Thus we can find a sequence $\seq{w_n}\to y$ in $\dom(T)$. This sequence then satisfies $\norm{w_n-y} \geq \norm{w_n}$. Taking the limit gives $0\geq \norm{y}$, so $y = 0$.

(2) For all $x\in\dom(\overline{T})$ there exists a sequence $\seq{x_n}\to x$ in $\dom(T)$ such that $\seq{Tx_n} \to \overline{T}x$ by \ref{graphNormConvergenceLemma}. Now for all $n\in \N$,
\[ \norm{(\lambda\id-T)x_n} \geq \lambda\norm{x_n}. \]
Taking the limit $n\to\infty$ gives $\norm{(\lambda\id-\overline{T})x} \geq \lambda\norm{x}$, meaning $\overline{T}$ is dissipative.

(3) By \ref{domImClosureOperator}, $\im(\lambda\id - T)$ is dense in $\im(\lambda\id-\overline{T})$ and by \ref{closureDissipativeOperator}, $\im(\lambda\id-\overline{T})$ is closed.
\end{proof}



\section{The spectral theorem}
\url{https://link.springer.com/content/pdf/10.1007%2F978-1-4614-7116-5.pdf}

\url{http://individual.utoronto.ca/jordanbell/notes/SVD.pdf}
\url{https://digitalcommons.mtu.edu/cgi/viewcontent.cgi?article=2133&context=etdr}

\url{https://web.ma.utexas.edu/mp_arc/c/09/09-32.pdf}


\section{Functional calculus}
\subsection{Holomorphic functional calculus}

\begin{theorem}[Holomorphic functional calculus] 
\label{holomorphicFunctionalCalculus} \label{holomorphicSpectralMapping}
Let $A$ be a Banach algebra and $x\in A$. Consider the function
\[ \Phi_x: \cont^\infty(\spec(x),\C) \to A: f\mapsto f(x)\defeq \oint_\Gamma f(z)R_x(z)\diff{z}. \]
Here $\Gamma$ is any simple Jordan curve that contains $\spec(x)$ such that $f$ is holomorphic in a region that contains $\Gamma$ and its interior. Then
\begin{enumerate}
\item $\Phi_x$ is well-defined: it does not depend on the particular curve $\Gamma$;
\item $\Phi_x$ is a homomorphism;
\item for any polynomial $p\in \C[X]$, we have $\Phi_x(p) = p(x)$; \\
in particular $\Phi_x(\id_\C) = x$ and $\Phi_x(\underline{1}) = \id_A$;
\item $\spec(\Phi_x(f)) = f[\spec(x)]$;
\item $\Phi_x$ is continuous if $\cont^\infty(\spec(x),\C)$ is equipped with continuous convergence (?).
\end{enumerate}
\end{theorem}
TODO: $\cont^\infty(\spec(x))$ should be the space of functions that are analytic in some neighbourhood of $\spec(x)$. Is it??
\begin{proof}
TODO
\end{proof}

TODO unbounded operators

\subsubsection{Riesz eigenprojections}
Holomorphic functional calculus applied to
\[ \chi_{S,\delta}: A\to \{0,1\}: x\mapsto \begin{cases}
1 & d(x,S) \leq \delta \\
0 & \text{otherwise}.
\end{cases} \]

TODO: spectral measure with only disconnected parts in $\sigma$-algebra??

TODO: $P_\Delta$ and $E_\Delta \defeq \im P_\Delta$.

\begin{lemma}
$\spec(T|_{E_\Delta}) = \spec(T)\cap\Delta$.
\end{lemma}

\begin{definition}
We call $\dim E_\lambda$ the \udef{algebraic multiplicity} of $\lambda$.
\end{definition}

\subsubsection{Frobenius covariants}
TODO $P_\lambda$ is a Frobenius covariant. \url{https://en.wikipedia.org/wiki/Frobenius_covariant}

TODO cfr. Lagrange polynomial??

\section{Jordan decomposition}
\subsection{Eigennilpotent}
\begin{definition}
Let $a$ be a finite element in a semisimple Banach algebra and $\lambda\in \spec(a)$. The \udef{eigennilpotent operator} of $a$ at $\lambda$ is defined as
\[ D_{\lambda} \defeq (a-\lambda)P_{\lambda}. \]
\end{definition}
This definition works because we can find a $\delta < d(\lambda, \spec(a)\setminus\{\lambda\})$.

\begin{lemma}
Let $a$ be a finite element in a semisimple Banach algebra and $\lambda\in \spec(a)$. The eigennilpotent operator $D_\lambda$ is nilpotent.
\end{lemma}
\begin{proof}
By spectral mapping \ref{holomorphicSpectralMapping}, $D_\lambda$ is quasinilpotent. Because $a$ is finite, it is nilpotent by \ref{nilpotentQuasinilpotent}.
\end{proof}



\subsection{Jordan vectors}
\begin{definition}
Let $V$ be a finite dimensional vector space and $T$ an operator on $V$. A \udef{Jordan vector} of $T$ belonging to the eigenvalue $\lambda$ is a vector $x\in V$ such that
\[ (\lambda\id_V - T)^kx = 0 \]
for some $k\in \N$. The least such $k$ is called the \udef{degree} of $x$ and is denoted $\deg_J(x)$.
\end{definition}
Eigenvectors are Jordan vectors of degree $1$.

\begin{proposition}
Let $V$ be a finite dimensional vector space, $T$ an operator on $V$ $\lambda\in\spec(T)$ and $x\in V$. Then $x$ is a Jordan vector of $T$ belonging to the eigenvalue $\lambda$ \textup{if and only if} $x\in E_\lambda$.
\end{proposition}
\begin{proof}
Let $x\in E_\lambda$. Then $x = P_\lambda x$ and thus
\[ (\lambda\id_V - T)^kx = (\lambda\id_V - T)^kP_\lambda x = \big((\lambda\id_V - T)P_\lambda\big)^k x = D_\lambda^k x, \]
which is zero for some $k$ because $D_\lambda$ is nilpotent.

Conversely, assume $x$ is a Jordan vector of $T$ belonging to the eigenvalue $\lambda$. We can write $x = x_1+x_2 \in E_{\lambda}\oplus E_{\C\setminus\{\lambda\}}$.
Then (because $E_\lambda$ is reducing for $T-\lambda\id_V$)
\[ 0 = (\lambda\id_V - T)^kx = (\lambda\id_V - T)^kx_1 + (\lambda\id_V - T)^kx_2 \in E_{\lambda}\oplus E_{\C\setminus\{\lambda\}} \]
Thus we have $(\lambda\id_V - T)^kx_1 = 0$ and $(\lambda\id_V - T)^kx_2 = 0$ separately.
Now $T-\lambda\id_V$ is invertible on $E_{\C\setminus\{\lambda\}}$, so $x_2 = 0$ (TODO ref). This means that $x = x_1 \in E_\lambda$.
\end{proof}

\begin{definition}
Let $m = \deg_N(D_\lambda)$. Then we have
\[ \{0\} \subsetneq \ker(\lambda\id_V - T) \subsetneq \ker(\lambda\id_V - T)^2 \subsetneq \ldots \subsetneq \ker(\lambda\id_V - T)^{m-1} \subsetneq \ker(\lambda\id_V - T)^m = V. \]
We define $E^k_\lambda \defeq \ker(\lambda\id_V - T)^k$. In particular
\begin{itemize}
\item $E^1_\lambda$ is the \udef{geometric eigenspace};
\item $E^{m-1}_\lambda$ is the \udef{algebraic eigenspace}.
\end{itemize}
\end{definition}

\begin{lemma}
Let $V$ be a finite dimensional vector space, $T$ an operator on $V$, $\lambda\in\spec(T)$ and $x\in E_\lambda$. Then
\begin{enumerate}
\item $1 \leq \dim\ker(\lambda\id_V - T) \leq \dim E_\lambda$;
\item $1 \leq \deg_J(x) \leq \dim_E\lambda$.
\end{enumerate}
\end{lemma}
The lemma says the geometric multiplicity is smaller than the algebraic multiplicity.
\begin{proof}
Every eigenvector is a Jordan vector, so $\ker(\lambda\id_V - T) \subseteq E_\lambda$.

For all $k\in\N$ smaller then the degree of $x$, $(\lambda\id_V - T)^kx$ is a Jordan vector and thus in $E_\lambda$. TODO all $(\lambda\id_V - T)^kx$ are linearly independent (like in \ref{nilpotentQuasinilpotent})
\end{proof}

\begin{definition}
Let $V$ be a finite dimensional vector space, $T$ an operator on $V$ and $\lambda\in\spec(T)$. The eigenvalue $\lambda$ is called
\begin{itemize}
    \item \udef{simple} if the algebraic multiplicity is $1$;
    \item \udef{semisimple} if every Jordan vector in $E_\lambda$ has degree $1$;
    \item \udef{prime} if the geometric multiplicity is $1$.
\end{itemize}
If all eigenvalues of $T$ are semisimple, then $T$ is called a \udef{diagonal operator}.
\end{definition}

\begin{lemma}
An operator $T$ is diagonal iff $T$ is of the form $\sum_j a_jP_j$, where $a_j\in \F$ and $P_j$ are projectors that commute pairwise.
\end{lemma}

\subsection{Characteristic polynomial and equation}
\begin{definition}
Let $V$ be a finite dimensional vector space and $T$ an operator on $V$. The \udef{characteristic polynomial} $p_T(x)$ of $T$ is the polynomial
\[ p_T(x) \defeq \det(x\id_V - T). \]
\end{definition}

\begin{proposition}
Let $V$ be a finite dimensional vector space, $T$ an operator on $V$ and $\spec(T) = \{\lambda_j\}_{j=1}^r$. Then
\[ p_T(x) = \prod_{j=1}^r(x - \lambda_j)^{\dim E_{\lambda_j}}. \]
\end{proposition}
\begin{proof}
TODO
\end{proof}
\begin{corollary}
A number $\lambda\in \C$ is an eigenvalue of $T$ \textup{if and only if} it is a root of $p_T(x)$.
\end{corollary}

\begin{definition}
The equation $p_T(x) = 0$ is the \udef{characteristic equation} of $T$.
\end{definition}

\subsection{Spectral representation}
\begin{proposition}
Let $V$ be a finite dimensional complex vector space and $T$ an operator on $V$. There exists a unique decomposition $T = S + D$ such that
\begin{itemize}
\item $S$ is diagonal;
\item $D$ is nilpotent;
\item $SD = DS$.
\end{itemize}
If $\spec(T) = \{\lambda_j\}_{j=1}^r$, this decomposition is given by
\[ T = \sum_{j=1}^r \lambda_r P_{\lambda_r} + \sum_{j=1}^r D_{\lambda_r}. \]
\end{proposition}

\subsection{Partial fraction decomposition of the resolvent}
For any operator $T$ on a vector space $V$ with eigenvalue $\lambda_0$, the resolvent $R_T(\lambda)$ has a pole at $\lambda_0$.

\begin{proposition}
Let $T$ be an operator on a finite dimensional vector space $V$ and $\lambda_0\in\spec(T)$. Then the Laurent expansion of $R_T(\lambda)$ around $\lambda_0$ is of the form
\[ R_T(\lambda) = \frac{P_0}{\lambda-\lambda_0} + \sum_{n=1}^{\deg_N(D_0)-1}\frac{D_0^{n}}{(\lambda - \lambda_0)^{n+1}} + \sum_{n=0}^\infty(-1)^n S_0^{n+1}(\lambda - \lambda_0)^n, \]
where $P_0\defeq P_{\lambda_0}, D_0\defeq D_{\lambda_0}$ and $S_0$ is some fixed operator.
\end{proposition}
\begin{proof}
TODO
\end{proof}

\begin{definition}
The holomorphic part of the Laurent expansion of $R_T(\lambda)$ at $\lambda_0$ is called the \udef{reduced resolvent} of $T$ w.r.t. $\lambda_0$:
\[ S_{T,\lambda_0}(\lambda) \defeq \sum_{n=0}^\infty(-1)^n S_0^{n+1}(\lambda - \lambda_0)^n = R_T(\lambda) - \left(\frac{P_0}{\lambda-\lambda_0} + \sum_{n=1}^{\deg_N(D_0)-1}\frac{D_0^{n}}{(\lambda - \lambda_0)^{n+1}}\right). \]
\end{definition}

\begin{proposition}
Let $T$ be an operator on a finite dimensional vector space $V$ and $\lambda_0\in\spec(T)$. Then
\[ R_{T|_{(\id_V-P_0)}}(\lambda) = S_{T,\lambda_0}|_{\id_V-P_0}(\lambda). \]
\end{proposition}

\begin{proposition}
Let $T$ be an operator on a finite dimensional vector space $V$ with $\spec(T) = \{\lambda_j\}_{j=1}^r$. The partial fraction decomposition of $R_T(\lambda)$ is given by
\[ R_T(\lambda) = \sum_{j=1}^r\left(\frac{P_{\lambda_j}}{\lambda - \lambda_j} +\sum_{n=1}^{\deg_N(D_{\lambda_j})-1}\frac{D_{\lambda_j}^n}{(\lambda - \lambda_j)^{n+1}}\right). \]
The partial fraction decomposition of $S_{T,\lambda_k}(\lambda)$ is given by
\[ S_{T,\lambda_k}(\lambda) = \sum_{\substack{j=1 \\ j\neq k}}^r\left(\frac{P_{\lambda_j}}{\lambda - \lambda_j} +\sum_{n=1}^{\deg_N(D_{\lambda_j})-1}\frac{D_{\lambda_j}^n}{(\lambda - \lambda_j)^{n+1}}\right). \]
\end{proposition}
\begin{proof}
The poles of $R_T(\lambda)$ are exactly the eigenvalues of $T$. There are finitely many of them, so we can use partial fraction decomposition, \ref{partialFractionDecomposition}. We just need to show that the holomorphic part is zero. For that we note that $\lim_{\lambda \to \infty} R_T(\lambda) = 0$ and all principal parts tend to $0$ at infinity as well. Thus the holomorphic part also tends to $0$, making it bounded. By Liouville's theorem, \ref{liouvilleTheoremAnalysis}, we get that it is identically zero.
\end{proof}
\begin{corollary}[Sylvester-Lagrange formula]
Let $f$ be a holomorphic function on an open set that contains $\spec(T)$. Then
\[ f(T) = \sum_{j=1}^r\left(f(\lambda_j)P_{\lambda_j} +\sum_{n=1}^{\deg_N(D_{\lambda_j})-1}\frac{f^{(n)}(\lambda_j)D_{\lambda_j}^n}{n!}\right). \] 
\end{corollary}
\begin{proof}
We have
\[ f(T) = \oint_\Gamma f(\lambda)R_T(\lambda)\diff{\lambda} = 2\pi i\sum_{j=1}^r \Res_{\lambda_j}f(\lambda)R_T(\lambda) \]
by the residue theorem (TODO ref for operators).
\end{proof}
\begin{corollary}[Cayley-Hamilton]
Let $p_T(x)$ be the characteristic polynomial of $T$. Then $p_T(T) = 0$.
\end{corollary}
\begin{proof}
Since $p_T(x) = \prod_{j=1}^r(x - \lambda_j)^{\dim E_{\lambda_j}}$ and $\dim E_{\lambda_j} \geq \deg_N(D_{\lambda_j})$, we see that $p_T(\lambda)R_T(\lambda)$ has no poles and is holomorphic, meaning that $oint_\Gamma f(\lambda)R_T(\lambda)\diff{\lambda} = 0$ by Cauchy's theorem (TODO ref for operators).
\end{proof}

\subsection{Normal operators}


\begin{proposition}
If $T$ is a normal operator, then $P_\lambda = P^*_\lambda$ and $D_\lambda = D^*_\lambda = 0$.
\end{proposition}
This means normal operators are diagonalisable.
\begin{proof}
TODO
\end{proof}
\begin{corollary}
Let $V$ be a finite dimensional complex vector space and $T$ a normal operator
on $V$ with $\spec(T) = \{\lambda_j\}_{j=1}^r$.
\begin{enumerate}
\item We have the spectral decompositions
\[ T = \sum_{j=1}^r \lambda_r P_{\lambda_r} \qquad\text{and}\qquad T^* = \sum_{j=1}^r \overline{\lambda_r} P_{\lambda_r}. \]
\item We have
\[ R_T(\lambda) = \sum_{j = 1}^r \frac{P_{\lambda_j}}{\lambda - \lambda_j} \qquad \text{and} \qquad S_{T,\lambda_k}(\lambda) = \sum_{\substack{j = 1 \\ j\neq k}}^r \frac{P_{\lambda_j}}{\lambda - \lambda_j} \]
\end{enumerate}
\end{corollary}

\subsection{Jordan decomposition}
TODO matrix representation + matrix representation of Lagrange-Sylvester. See Baumgärtel







\chapter{Hilbert spaces}

\section{Examples}
\subsection{The $\ell^2$ spaces}
Sequence spaces $\ell^p$ Hilbert iff $p=2$. (TODO: other sequence spaces?)

\subsection{Direct sum}
Let $(V_i)_{i\in I}$ be a family of Hilbert spaces. By considering them as Banach spaces we can take the $\ell^2$-direct sum. (TODO: other sequence spaces?)
\begin{proposition}
Let $(V_i)_{i\in I}$ be a family of Hilbert spaces. The $\ell^2$-direct sum is a Hilbert space.
\end{proposition}
This gives the conventional interpretation of the \udef{Hilbert space direct sum}: it is the $\ell^2$-direct sum of the summands as Banach spaces.

\section{Strong and weak convergence}
\subsection{Weak convergence of vectors}
\begin{definition}
Let $\mathcal{H}$ be a Hilbert space. The \udef{weak convergence} on $\mathcal{H}$ is the initial convergence w.r.t.
\[ \setbuilder{\inner{x,-}: \mathcal{H}\to \F}{x\in\mathcal{H}}. \]
Thus $F \overset{w}{\longrightarrow} v$ \textup{if and only if} $\inner{x, F} \longrightarrow \inner{x,v}$ for all $x\in\mathcal{H}$.

We denote the weak convergence $\overset{w}{\longrightarrow}$ or $\lim^w$ and write $\mathcal{H}^w$ to denote the convergence space $\sSet{\mathcal{H},\lim^w}$.
\end{definition}
By \ref{initialVectorConvergenceLinearFunctions} we have that the weak convergence is a vector space convergence.

\begin{lemma} \label{normConvergenceFinerThenWeakConvergence}
Norm convergence is finer than weak convergence.
\end{lemma}
\begin{proof}
The functions in $\setbuilder{\inner{x,-}: \mathcal{H}\to \F}{x\in\mathcal{H}}$ are also continuous in the norm convergence.
\end{proof}

\begin{example}
Let $\seq{e_n}$ be an orthonormal basis. Then the Riemann-Lebesgue lemma, \ref{RiemannLebesgueLemma}, gives $e_n \overset{w}{\longrightarrow} 0$, even though clearly $e_n \not\to 0$.
\end{example}

So norm convergence is strictly finer.

\begin{proposition}
Let $\mathcal{H}$ be a Hilbert space, $x\in \mathcal{H}$ and $\seq{x_n}\subseteq \mathcal{H}$. Then
\begin{enumerate}
\item $x_n \overset{w}{\longrightarrow} x$ implies $\norm{x} \leq \liminf \norm{x_n}$;
\item $x_n \longrightarrow x$ \textup{if and only if} $x_n \overset{w}{\longrightarrow} x$ and $\norm{x_n}\to \norm{x}$.
\end{enumerate}
\end{proposition}
\begin{proof}
(1) We have
\[ \norm{x}^2 = \inner{x,x} = \lim \inner{x,x_n} = \liminf\inner{x,x_n} \leq \norm{x}\liminf\norm{x_n}, \]
where the limit is in $\R$. We use that for convergent sequences, $\lim = \liminf$.

(2) The direction $\Rightarrow$ is clear form \ref{normConvergenceFinerThenWeakConvergence} and the continuity of the norm.

For the converse, we have
\[ \norm{x-x_n}^2 = \norm{x}-2\Re(\inner{x,x_n}) + \norm{x_n} \to 0, \]
because $\norm{x_n} \to \norm{x}$ and $\inner{x,x_n} \to \inner{x,x} = \norm{x}^2$.
\end{proof}

\url{https://math.stackexchange.com/questions/1461363/closed-unit-ball-of-hilbert-space-sequentially-compact-in-weak-topology}

\subsubsection{Weak Cauchy filters}
\begin{definition}
Cauchy filters in $\mathcal{H}^w$ are called \udef{weak Cauchy filters}.
\end{definition}


\begin{proposition}
Let $\mathcal{H}$ be a real or complex Hilbert space.
A filter $F$ in $\powerfilters(\mathcal{H})$ is a weak Cauchy filter \textup{if and only if} $\inner{x,F}$ converges in $\F$ for all $x$.
\end{proposition}
\begin{proof}
The following steps are equivalent:
\begin{itemize}
\item $F$ is weak Cauchy;
\item $F-F \overset{w}{\longrightarrow} 0$;
\item $\inner{x, F} - \inner{x,F} \to 0$ for all $x\in \mathcal{H}$;
\item $\inner{x, F}$ is Cauchy in the scalar field for all $x\in\mathcal{H}$;
\item $\inner{x, F}$ converges for all $x\in\mathcal{H}$.
\end{itemize}
The last equivalence follows because $\R$ and $\C$ are complete.
\end{proof}
\begin{corollary}
Every weak Cauchy filter is bounded in norm.
\end{corollary}
\begin{proof}
Consider the family of bounded linear operator $\inner{F,-}$. Then $\inner{F,-}$ converges pointwise, by the proposition. By Banach-Steinhaus, \ref{BanachSteinhaus}, we have $\sup_{y\in F}\norm{\inner{y,-}} = \sup_{y\in F}\norm{y} <\infty$.
\end{proof}

Note that the previous proposition does not mean a weak Cauchy filter is necessarily weakly convergent, as there may not exist a vector $v$ such that $\inner{x,F}$ converges to $\inner{x,v}$ for all $x\in\mathcal{H}$. However, we have the following proposition:

\begin{proposition}
Let $\mathcal{H}$ be a Hilbert space. Then $\mathcal{H}^w$ is Cauchy complete.
\end{proposition}
\begin{proof}
Let $F$ be a weak Cauchy filter. Let $\seq{e_i}_{i\in I}$ be an orthonormal basis of $\mathcal{H}$. Let $c_i = \lim \inner{e_i, F}$. Then $F \overset{w}{\longrightarrow} \sum_{i\in I}c_i e_i$.

First we check this sum is well-defined. TODO!!
\end{proof}

TODO: $\inner{F,F}$ weak convergent?????


\subsection{Strong and weak convergence of operators}
\begin{definition}
Let $\mathcal{H}$ be a Hilbert space. Then the
\begin{itemize}
\item \udef{strong operator topology} is the topology of pointwise convergence on $\Lin(\mathcal{H})$;
\item \udef{weak operator topology} is the topology of pointwise convergence on $\Lin(\mathcal{H}^w)$.
\end{itemize}
We write $F \overset{SOT}{\longrightarrow}A$ and $F \overset{WOT}{\longrightarrow}A$ for the convergence of $F$ to $A$ in the strong and weak operator topologies.
\end{definition}


TODO introduce shifts earlier.
\begin{example}
Consider the left and right shifts $S_l$ and $S_r$ on $\ell^2(\N)$. Then
\begin{itemize}
\item $S_l \overset{SOT}{\longrightarrow} 0$, but $S_l \overset{norm}{\not\longrightarrow} 0$;
\item $S_r \overset{WOT}{\longrightarrow} 0$, but $S_r \overset{SOT}{\not\longrightarrow} 0$.
\end{itemize}
In general taking adjoints is not continuous w.r.t. strong operator convergence, because $S_l \overset{SOT}{\longrightarrow} 0$, but $S_r = S_l^* \overset{SOT}{\not\longrightarrow} 0$.
\end{example}

\url{https://math.stackexchange.com/questions/1054288/the-set-of-all-normal-operators-on-a-hilbert-space-is-not-strongly-closed}


Note normal operators not $SOT$-closed!
\begin{proposition}
If $\seq{A_n}$ is a sequence of normal operators that converges to a normal operator $A$ in the strong operator topology, then $A_n^* \overset{SOT}{\longrightarrow} A^*$.
\end{proposition}

\begin{proposition}
Let $\seq{A_n}$ be a sequence of bounded operators on a Hilbert space and $A\in\Lin(\mathcal{H})$. Then
\begin{enumerate}
\item if $A_n \overset{SOT}{\longrightarrow} A$, then $\norm{A}\leq \liminf\norm{A_n}$;
\item If $A_nx \longrightarrow Ax$ for all $x$ in a dense subset of $\mathcal{A}$ and $\seq{A_n}$ is a bounded sequence, then $A_n \overset{SOT}{\longrightarrow} A$.
\end{enumerate}
\end{proposition}
\begin{proof}
(1) For all unit vectors $x$ we have
\[ \norm{Ax} = \norm{\lim_{SOT}A_nx} = \lim_{n\to\infty}\norm{A_nx} \leq  \]
\end{proof}
TODO: same for WOT.

TODO: Cauchy sequences are bounded. Does this follow form general Banach theory?


\section{Tools to study operators}
\subsection{Spectrum}
\subsection{Numerical range}
\begin{lemma}
Let $T$ be an operator on a Hilbert space. If $\lambda\in\cspec(T)$, then there exists a Weyl sequence for $\lambda$. 
\end{lemma}
\begin{proof}
If $\lambda\in\cspec(T)$, then $R_T(\lambda)$ is densely defined.
\end{proof}

\begin{proposition}[Spectral inclusion property of numerical range] \label{spectralInclusionNumericalRange}
Let $T$ be an operator on a Hilbert space. Then
\begin{enumerate}
\item $\pspec(T)\subseteq \NumRange(T)$;
\item $\rspec(T) \subseteq \NumRange(T)$;
\item $\apspec(T) \subseteq \overline{\NumRange(T)}$.
\end{enumerate}
In particular $\cspec(T) \subseteq \overline{\NumRange(T)}$.
\end{proposition}
\begin{proof}
(1) Let $\lambda \in \pspec(T)$. Then $Tx = \lambda x$ for some unit vector $x$ and so
\[ \inner{x,Tx} = \inner{x,\lambda x} = \lambda \inner{x,x} = \lambda\norm{x}^2 = \lambda, \]
which means that $\lambda \in \NumRange(A)$.

(2) Let $\lambda \in \rspec(T)$. Then there exists a unit vector $x \in \im(\lambda\id - T)^\perp$, so
\[ 0 = \inner{x,(\lambda\id-T)x} = \lambda\inner{x}^2 - \inner{x,Tx} = \lambda - \inner{x,Tx}, \]
which means that $\lambda \in \NumRange(A)$.

(3) Let $\lambda \in \apspec(T)$. Then there exists a Weyl sequence $\seq{e_n}$ for $\lambda$ by \ref{WeylSequence}. Then
\[ \norm{(\lambda\id - T)e_n} = \norm{e_n}\;\norm{(\lambda\id - T)e_n} \geq |\inner{e_n, (\lambda\id - T)e_n}| = |\lambda - \inner{e_n,Te_n}| \to 0. \]
Thus $\lambda\in\overline{\NumRange(T)}$.

Finally we have $\cspec(T)\subseteq\apspec(T)$ by \ref{approximateSpectrum} and so $\cspec(T)\subseteq\overline{\NumRange(T)}$.
\end{proof}
\begin{corollary}
If $T$ is a closed operator on a Hilbert space, then $\spec(T)\subseteq \overline{\NumRange(T)}$.
\end{corollary}

\section{Projectors and minimisation problems}
Every subspace is a convex, non-empty subset.
\begin{theorem}[Hilbert projection theorem]
Let $\mathcal{H}$ be a Hilbert space, $K$ a closed, convex, non-empty subset of $\mathcal{H}$.
\begin{enumerate}
\item There exists a unique element of $K$ of least norm. i.e.\ there exists a unique $k_0\in K$ such that
\[ \norm{k_0} = \inf\setbuilder{\norm{k}}{k\in K}. \]
i.e.\ $\min\setbuilder{\norm{k}}{k\in K}$ exists.
\item For any $h\in\mathcal{H}$ there exists a unique point $k_0$ in $K$ such that
\[ \norm{h-k_0} = \inf\{\norm{h-k}\;|\; k\in K\}. \]
We use this to define the distance $d(h,K) \defeq \norm{h-k_0}$.
\item If $K$ is a (closed) subspace, then $k_0$ is also the unique point in $K$ such that $(h-k_0)\perp K$.
\end{enumerate}
\end{theorem}
The idea for the first part of the proof is to take a sequence $\seq{\norm{k_i}}\to \inf\setbuilder{\norm{k}}{k\in K}$. By the parallelogram law $\seq{k_i}$ is Cauchy and by completeness it has a limit $k_0$.
\begin{proof}
(1) We can find a sequence $\seq{k_i}$ in $K$ such that $\norm{k_i}$ converges to $d = \inf\setbuilder{\norm{k}}{k\in K}$ by \ref{sequenceToSupInf}. By the parallelogram law
\begin{align*}
\norm{k_i-k_j}^2 &= 2\norm{k_i}^2 + 2\norm{k_j}^2 - 4\norm{\frac{1}{2}(k_i+k_j)}^2 \\
&\leq 2\norm{k_i}^2 + 2\norm{k_j}^2 - 4d^2
\end{align*}
the sequence $\seq{k_i}$ is Cauchy. So it converges to some $k_0$ in $K$ because $K$ is a closed subset of a complete space.

To prove uniqueness, take another $k_0'\in K$ such that $\norm{k_0'}=d$. By convexity $\tfrac{1}{2}(k_0 +k_0')\in K$, hence
\[ d\leq \norm{\tfrac{1}{2}(k_0+k_0')}\leq \tfrac{1}{2}(\norm{k_0}+\norm{k_0'}) = d. \]
So $\norm{\tfrac{1}{2}(k_0+k_0')} = d$. The parallelogram law gives
\[ d^2 = \norm{\frac{k_0+k_0'}{2}}^2 = d^2- \norm{\frac{k_0-k_0'}{2}}^2; \]
hence $\norm{k_0 - k_0'}^2 = 0$ and thus $h_0=k_0$.

(2) The element $k_0$ considered in point 1. is the point closest to a particular choice for $h$, namely $h=0$. For other $h$ consider the set $K-h$, which is again closed and convex.

(3) For all $k\in K$ and $a\in \mathbb{F}$, we have
\[ \norm{h-k_0}\leq \norm{h-k_0+ak} \]
and thus, by lemma \ref{orthogonality}, $(h-k_0)\perp k$, meaning $(h-k_0)\perp K$.

For the converse (i.e.\ uniqueness), suppose $f_0\in K$ such that $(h-f_0)\perp K$. Then for all $f\in K$ we have $(h-f_0)\perp (f_0 -f)$ so that
\begin{align*}
\norm{h-f}^2 &= \norm{(h-f_0) + (f_0-f)}^2 \\
&= \norm{h-f_0}^2 + \norm{f_0 - f}^2 \geq \norm{h-f_0}^2.
\end{align*}
So $\norm{h-f_0}=\inf\{\norm{h-k}\;|\; k\in K\} = d(h,K)$ and thus $f_0=k_0$.
\end{proof}
\begin{corollary}
Let $\mathcal{H}$ be a Hilbert space and $K$ a closed vector subspace. Then $\mathcal{H} = K^\perp \oplus K$.
\end{corollary}
\begin{proof}
We need to prove every vector $x\in \mathcal{H}$ has a unique decomposition of the form
\[ x = y+z \qquad y\in K,\; z\in K^\perp. \]

Such a decomposition exists: we can take $y=k_0$ and $z = x-k_0$. We have already proved uniqueness. We can also give another argument for uniqueness: assume another such decomposition $x=y'+z'$. Then $y-y'= z-z'$ where the left side is in $K$ and the right in $K^\perp$. The only element in $K\cap K^\perp$ is $0$, so $y=y'$ and $z=z'$.
\end{proof}
The ability to make such decompositions in general is unique to Hilbert spaces, see theorem \ref{criterionHilbertSpace}.

\subsection{Orthogonal projection and decomposition}
\begin{definition}
Let $\mathcal{H}$ be a Hilbert space. Given a subspace $K$ and an element $x \in \mathcal{H}$, we call the unique element $y\in K$ of the decomposition $K\oplus K^\perp$ the \udef{orthogonal projection} of $x$ on $K$. It is denoted $P_K(x)$. This defines a function $P_K:\mathcal{H}\to K$ called the \udef{orthogonal projection} on $K$.
\end{definition}

\begin{proposition}
Let $P$ be the orthogonal projection on a closed subspace $K$. Then
\begin{enumerate}
\item $P$ is a linear operator on $\mathcal{H}$;
\item $\norm{Px}\leq \norm{x}$ for all $x\in\mathcal{H}$;
\item $P^2 = P$;
\item $\ker P = K^\perp$ and $\im P = K$;
\item $\id_\mathcal{H} - P$ is the orthogonal projection of $\mathcal{H}$ onto $K^\perp$.
\end{enumerate}
\end{proposition}
\begin{proof}
These are mostly direct results of the decomposition. In particular 5. follows if we know $K^\perp$ is closed, which it is by proposition \ref{orthogonalComplementClosed}.
\end{proof}
\begin{corollary} \label{HilbertClosedSpaceOrthogonalDecomposition}
Let $\mathcal{H}$ ba a Hilbert space and $K$ a closed subspace, then $\mathcal{H} = K\oplus K^\perp$.
\end{corollary}
\begin{proof}
Let $P$ be the orthogonal projection on $K$. Then by \ref{directSumKernelImageIdempotent}
\[ \mathcal{H} = \im P \oplus \ker P = K\oplus K^\perp. \]
\end{proof}
\begin{corollary} \label{doubleComplementClosure}
Let $\mathcal{H}$ be a Hilbert space.
\begin{enumerate}
\item If $K$ is a subspace, then $(K^\perp)^\perp = \overline{K}$ is the closure of $K$.
\item If $A$ is a subset, then $(A^\perp)^\perp$ is the closed linear span of $A$.
\end{enumerate}
\end{corollary}
\begin{proof}
(1) Assume $K$ is closed. Then using $0=(I-P_K)x\;\; \Leftrightarrow \;\; x=P_Kx$, we see
\[ (K^\perp)^\perp = \ker(I-P_K) = \im P_K = K. \]
Then, if $K$ is not closed, $(K^\perp)^\perp = (\overline{K}^\perp)^\perp = \overline{K}$, by proposition \ref{orthogonalComplementClosed}.

(2) Using \ref{OrthogonalComplementProperties} we calculate $(A^\perp)^\perp = (\Span(A)^\perp)^\perp = \overline{\Span(A)}$.
\end{proof}
\begin{corollary} \label{denseZeroComplement}
Let $A$ be a subset of a Hilbert space $\mathcal{H}$. Then $\Span(A)$ is dense in $\mathcal{H}$ \textup{if and only if} $A^\perp = \{0\}$.
\end{corollary}
\begin{proof}
The subspace $\Span(A)$ is dense in $\mathcal{H}$ iff $\overline{\Span(A)} = \mathcal{H}$ iff $(\Span(A)^\perp)^\perp = (A^\perp)^\perp = \mathcal{H}$ iff $A^\perp = \{0\}$.

In the last step we have used that $A^\perp$ is closed so that $((A^\perp)^\perp)^\perp = \overline{A^\perp} = A^\perp$, see \ref{orthogonalComplementClosed}.
\end{proof}

\subsubsection{Existence of orthonormal bases}
\begin{corollary}
Let $D$ be an orthonormal subset of a Hilbert space $\mathcal{H}$, then $D$ is an ortonormal basis \textup{if and only if} it is maximal.
\end{corollary}
\begin{proof}
This is a restatement of the previous corollary in the language of \ref{characterisationMaximalOrthonormalSet}.
\end{proof}
\begin{corollary}
Every Hilbert space has an orthonormal basis.
\end{corollary}
\begin{proof}
Every inner product space has a maximal orthonormal set by \ref{exitenceMaximalOrthonormalSet}. This maximal orthonormal set is an orthonormal set by the proposition.
\end{proof}
\begin{corollary} \label{HilbertOnBasisMaximal}
An orthonormal subset of a Hilbert space is an orthonormal basis \textup{if and only if} it is maximal.
\end{corollary}

\begin{lemma}
Let $\mathcal{H}$ be a Hilbert space and $K$ a closed subspace. Let $\{e_i\}_{i\in I}$ be an orthonormal basis of $K$. Then
\[ P_K(x) = \sum_{i\in I} \inner{e_i,x}e_i. \]
\end{lemma}
\begin{proof}
We can extend $\{e_i\}_{i\in I}$ to an orthonormal basis $\{e_i\}_{i\in J}$ of $\mathcal{H}$. Then
\[ x = \sum_{i\in J}\inner{e_i,x}e_i = \sum_{i\in I}\inner{e_i,x}e_i + \sum_{i\notin I}\inner{e_i,x}e_i, \]
which is clearly a decomposition in $K\oplus K^\perp$. This is unique, so we have found $P_K(x)$.
\end{proof}

\subsubsection{When are inner product spaces complete?}
Notice that some of the results obtained for Hilbert spaces have one direction that is generally true for inner product spaces: in any inner product space we have
\begin{itemize}
\item $\overline{K}\subset (K^\perp)^\perp$;
\item $\Span(A)$ dense in $\mathcal{H}$ implies $A^\perp = \{0\}$;
\item if $D$ is an orthonormal basis, then it is maximal.
\end{itemize}
See \ref{orthogonalComplementClosed}, \ref{orthogonalComplementDenseSpace} and \ref{characterisationMaximalOrthonormalSet}.

The converses are only true for Hilbert spaces.
\begin{theorem} \label{criterionHilbertSpace}
Let $H$ be an inner product space. If any of the following hold, $H$ is a Hilbert space:
\begin{enumerate}
\item For any orthonormal set $D$,
\[ \text{$D$ is maximal} \quad\implies\quad \text{$D$ is an orthonormal basis.} \]
\item For any subset $A$, $A^\perp = \{0\}$ implies $\Span(A)$ is dense in $H$.
\item For any subspace $K$, we have $(K^\perp)^\perp = \overline{K}$.
\item For all closed subspaces $K$ we can decompose $H = K\oplus K^\perp$.
\end{enumerate}
\end{theorem}
\begin{proof}
We prove the first statement implies $H$ is a Hilbert space. The other three imply the first and thus that $H$ is a Hilbert space.
\begin{enumerate}
\item We prove the contrapositive: assume $H$ is not complete, we wish to show that 1. does not hold, i.e.\ there exists a maximal orthonormal subset of $H$ that is not an orthonormal basis.

Let $\mathcal{H}$ be the completion of $H$ and take a unit vector $v\in \mathcal{H}\setminus H$. Now working in the completion, we have the decomposition $\Span\{v\}\oplus \Span\{v\}^\perp$. Consider the subspace $\Span\{v\} + H = \Span\{v\}\oplus(H\cap \Span\{v\}^\perp)$. We can extend $\{v\}$ to a maximal orthonormal set $\{v\}\cup D$ by \ref{exitenceMaximalOrthonormalSet}.

We claim $D$ is the orthonormal set we want:

Firstly it is maximal.
Assume, towards a contradiction, that $D$ is not maximal in $H$, so there exists an orthonormal set $D'\supsetneq D$. Take $w\in D'\setminus D$ and let $w'$ be the normalisation of $w - \inner{v,w}v$. Then $w' \perp v$ and $w' \perp D$, so $\{v\}\cup D\cup\{w'\}$ is an orthonormal set in $\Span\{v\} + H$, which contradicts the maximality of $\{v\}\cup D$.

Secondly it cannot be total. Indeed if $\Closure_H(\Span(D)) = H\cap\overline{\Span(D)}$ were equal to $H$, then $H \subseteq \overline{\Span(D)}$ and thus $\mathcal{H} = \overline{H} \subseteq \overline{\Span(D)} \subseteq \mathcal{H}$, meaning $\overline{\Span(D)} = \mathcal{H}$. But $v\notin \overline{\Span(D)}$, so $\overline{\Span(D)} \neq \mathcal{H}$.

\item 2. clearly implies 1. We can also adapt the proof above to show 2. implies $H$ is a Hilbert space:
Assume $H$ is not complete and let $\mathcal{H}$ be the completion of $H$. There exists a $v\in \mathcal{H}\setminus H$. All orthogonal complements are taken in the completion.
The set
\[ U \defeq H\cap\{v\}^\perp \]
is not dense in $\mathcal{H}$ for the same reason $D$ was not total above. We claim that the orthogonal complement of $U$ in $H$ is $\{0\}$:
\[ U^\perp\cap H = \{0\}. \]
First we claim $U$ is dense in $\{v\}^\perp$: take a $w\in \{v\}^\perp$ and let $(x_n)_{n\in\N}\subseteq H$ converge to $w$ (this is possible because $w\in\mathcal{H}$ and $H$ is dense in $\mathcal{H}$). Fix some $x\in H$ such that $\inner{x,v}\neq 0$, then we have the following sequence in $U$ that converges to $w$:
\[ n\mapsto x_n - \inner{x_n,v}\frac{x}{\inner{x,v}}. \]
Then because $U$ is dense in $\{v\}^\perp$,
\[ U^\perp\cap H = \overline{U}^\perp\cap H = (\{v\}^\perp)^\perp \cap H = \Span\{v\}\cap H = \{0\}. \]
\item Assume 3. Let $D$ be a maximal orthonormal set. Then
\[ \overline{\Span(D)} = (\Span(D)^\perp)^\perp = (D^\perp)^\perp = \{0\}^\perp = H, \]
so $D$ is an orthonormal basis.
\item Assume 4. Let $D$ be a maximal orthonormal set. Then $D^\perp$ is a closed subspace, so
\[ H  = D^\perp \oplus (D^\perp)^\perp = \{0\} \oplus (D^\perp)^\perp = (\Span(D)^\perp)^\perp = \overline{\Span(D)}. \]
\end{enumerate}
\end{proof}

\subsubsection{Orthogonal decomposition}
\begin{theorem}
 A Banach space such all of its closed subspaces are complemented is isomorphic to a Hilbert space.
\end{theorem}
\begin{proof}
TODO Lindestrauss and Tzafriri in 1971. Only real??
\end{proof}

\begin{proposition} \label{directSumOrthogonalClosed}
Let $\mathcal{H}$ be a Hilbert space and let $\{V_i\}_{i\in I}$ be a family of closed, (pairwise) orthogonal subspaces. Then
\[ \bigoplus_{i\in I}V_i \qquad \text{is a closed subspace of $\mathcal{H}$.} \]
\end{proposition}
\begin{proof}
Let $(v_n)$ be a Cauchy sequence in $\bigoplus_{i\in I}V_i$ which converges to $w$. Let $v_{i,n}$ be the component of $v_n$ in $V_i$. By orthogonality we have
\[ \norm{v_n-v_m}^2 = \sum_{i\in I}\norm{v_{i,n}-v_{i,m}}^2. \]
Then
\[ \norm{v_{i,n}-v_{i,m}} \leq \norm{v_n-v_m} \]
which implies $(v_{i,n})_n$ is a Cauchy sequence in the closed space $V_i$ which therefore converges to $w_i\in V_i$. Now there are only a finite number of $i$ for which there exist non-zero $v_{i,n}$ (TODO proof!!!!). So then
\[ \lim_n v_n = \lim_n \sum_{i\in I}v_{i,n} = \sum_{i\in I}w_i \in \bigoplus_{i\in I}V_i \]
where the interchange of limits and last equality follow because the sums are finite.
\end{proof}

\begin{lemma} \label{cancellationOminus}
Let $\mathcal{H}$ be a Hilbert space and $A\supseteq B \supseteq C$ subspaces with $B$ closed. Then
\[ (A\ominus B)\oplus (B\ominus C) = A\ominus C.\]
\end{lemma}
\begin{proof}
Take $v\in(A\ominus B)\oplus (B\ominus C)$. Then either $\{v\}\perp C$ or $\{v\}\perp B$, but this implies $\{v\}\perp C$, so $v\in A\ominus C$.

Take $v\in A\ominus C$. We can uniquely write $v = v_1 + v_2 \in (A\ominus B)\oplus B = A$. We just need to show that $v_2\in B\ominus C$. Indeed assume $\inner{c,v_2}\neq 0$ for some $c\in C$. Then
\[ \inner{c, v} = \inner{c, v_1+v_2} = \inner{c,v_1}+\inner{c,v_2} = \inner{c,v_2} \neq 0, \]
so $v\notin A\ominus C$, a contradiction.
\end{proof}

\subsection{Projection and minimisation in finite-dimensional spaces}

\begin{lemma}
Let $K$ be a subspace of $\F^n$ spanned by the orthonormal basis $\{\vec{u}_i\}_{i=1}^k$. Then
\[ P_K = QQ^* \qquad\text{where}\qquad Q = \begin{bmatrix}
\vec{u}_1 & \vec{u}_2 & \hdots & \vec{u}_k
\end{bmatrix}. \]
\end{lemma}
\begin{proof}
$P_K(\vec{x}) = \sum_{i=1}^k\vec{u}_i\inner{\vec{u}_i,\vec{x}} = \sum_{i=1}^k\vec{u}_i \vec{u}_i^*\vec{x} = \left(\sum_{i=1}^k\vec{u}_i \vec{u}_i^*\right)\vec{x} = QQ^*\vec{x}$.
\end{proof}
\begin{corollary}
For any matrix $A$ with QR factorisation $A=QR$, we have
\[ P_{\Col(A)} = QQ^*. \]
\end{corollary}
In general $P_{\Col(A)} = A(A^*A)^{-1}A^*$.

\begin{proposition}[Normal equations]
Let $\{\vec{v}_i\}_{i=1}^k$ be linearly independent set of vectors in $\F^n$. Set $K = \Span\{\vec{v}_i\}_{i=1}^k$. Then for all $\vec{x}\in \F^n$
\[ P_K(\vec{x}) = \sum_{i=1}^k c_i \vec{v}_i, \]
where $\begin{bmatrix}
c_1 & c_2 & \hdots & c_k
\end{bmatrix}^\transp$ is the solution of
\[ \begin{bmatrix}
\inner{\vec{v}_1,\vec{v}_1} & \inner{\vec{v}_1,\vec{v}_2} & \hdots & \inner{\vec{v}_1,\vec{v}_k} \\
\inner{\vec{v}_2,\vec{v}_1} & \inner{\vec{v}_2,\vec{v}_2} & \hdots & \inner{\vec{v}_2,\vec{v}_k} \\
\vdots & \vdots & \ddots & \vdots \\
\inner{\vec{v}_k,\vec{v}_1} & \inner{\vec{v}_k,\vec{v}_2} & \hdots & \inner{\vec{v}_k,\vec{v}_k} \\
\end{bmatrix}\begin{bmatrix}
c_1 \\ c_2 \\ \vdots \\ c_k
\end{bmatrix} = \begin{bmatrix}
\inner{\vec{v}_1,\vec{x}} \\ \inner{\vec{v}_2,\vec{x}} \\ \vdots \\ \inner{\vec{v}_k,\vec{x}}
\end{bmatrix}. \]
This system of linear equations is consistent, yielding a unique solution.
\end{proposition}
The equations in this proposition are known as \udef{normal equations} and the matrix
\[ G(\vec{v}_1, \ldots, \vec{v}_k) \defeq \begin{bmatrix}
\vec{v}_1^* \\ \vec{v}_2^* \\ \vdots \\ \vec{v}_k^*
\end{bmatrix}\begin{bmatrix}
\vec{v}_1 & \vec{v}_2 & \hdots & \vec{v}_k
\end{bmatrix} = \begin{bmatrix}
\inner{\vec{v}_1,\vec{v}_1} & \inner{\vec{v}_1,\vec{v}_2} & \hdots & \inner{\vec{v}_1,\vec{v}_k} \\
\inner{\vec{v}_2,\vec{v}_1} & \inner{\vec{v}_2,\vec{v}_2} & \hdots & \inner{\vec{v}_2,\vec{v}_k} \\
\vdots & \vdots & \ddots & \vdots \\
\inner{\vec{v}_k,\vec{v}_1} & \inner{\vec{v}_k,\vec{v}_2} & \hdots & \inner{\vec{v}_k,\vec{v}_k} \\
\end{bmatrix} \]
is known as the \udef{Gram matrix} or \udef{Grammian}.
\begin{proof}
TODO
\end{proof}

\begin{proposition}
Let $A\in\F^{m\times n}$, $\vec{b}\in\F^m$ and $\vec{x}_0\in\F^n$. Then
\[ \min_{\vec{x}\in\F^n}\norm{\vec{b}-A \vec{x}} = \norm{\vec{b} - A \vec{x}_0} \]
if and only if
\[ A^*A \vec{x}_0 = A^* \vec{b}. \]
\end{proposition}
We regard $\vec{x}_0$ as the ``best approximate solution'' to the (not necessarily consistent) system $A \vec{x} = \vec{b}$.
\begin{proof}
TODO
\end{proof}

\subsection{Riesz representation}
\begin{theorem}[Riesz-Fréchet representation theorem] \label{rieszRepresentation}
Let $\mathcal{H}$ be a Hilbert space. For every continuous linear functional $\omega\in \mathcal{H}'$, there exists a unique $v_\omega\in\mathcal{H}$ such that
\[ \omega(x) = \inner{v_\omega, x} \qquad \forall x\in\mathcal{H}. \]
Moreover, $\norm{v_\omega}_\mathcal{H} = \norm{\omega}_{\mathcal{H}'}$.
\end{theorem}  

The idea of the proof is as follows: consider $\mathcal{H} \cong \ker\omega \oplus \im\omega$. So we can find a subspace $U\subseteq \mathcal{H}$ such that $\mathcal{H} = \ker\omega\oplus U$. Clearly $\dim U = \dim\im\omega = \dim\F = 1$. Between $1$-dimensional spaces there can only be one linear map, up to rescaling. This map is given by $x\mapsto \inner{v,x}$ for some $v\in U$, where the scaling determines the $v$. So we choose $v$ such that $\omega|_U = x\mapsto \inner{v,x}$.

Now we want extend this form of $\omega|_U$ to the whole of $\mathcal{H}$. This works exactly if $v\in(\ker\omega)^\perp$. So we need $U=(\ker\omega)^\perp$ which is true if and only if $\mathcal{H} = \ker\omega\oplus U = \ker\omega\oplus (\ker\omega)^\perp$, which only works in general if $\ker\omega$ is closed and $\mathcal{H}$ is a Hilbert space. Now $\ker\omega$ is closed if and only if it is continuous, by \ref{continuousMapCriterion}.

With this idea we give a full proof:
\begin{proof}
If $\ker\omega = \mathcal{H}$, we can take $v_\omega = 0$.

Assume $\ker\omega\neq \mathcal{H}$, then $(\ker\omega)^\perp\neq \{0\}$ by \ref{denseZeroComplement}, because $\ker\omega$ is closed (\ref{continuousMapCriterion}). So we can take a non-zero $u\in (\ker\omega)^\perp$. We can choose it such that $\omega(u) = 1$, by rescaling. Now let $h\in\mathcal{H}$. We can write $h = (h - \omega(h)u)+\omega(h)u\in\ker\omega\oplus (\ker\omega)^\perp$, because $\omega(h - \omega(h)u) = 0$. So
\[ 0 = \inner{u,h - \omega(h)u} = \inner{u,h} - \omega(u)\norm{u}^2. \]
If $v_\omega = \norm{u}^{-2}u$, then $\omega(h) = \inner{v_\omega, h}$ for all $h\in\mathcal{H}$.

For uniqueness: assume we can find two vectors $v_\omega,v_\omega'$ such that for all $h\in\mathcal{H}$ we have $\omega(h) = \inner{v_\omega, h} = \inner{v_\omega', h}$. Then $v_\omega - v_\omega'\perp \mathcal{H}$, so $v_\omega - v_\omega'= 0$.
\end{proof}
Together with lemma \ref{innerBoundedFunctionals} this gives:
\begin{corollary} \label{RieszIsometry}
The map $C_\mathcal{H}:\mathcal{H}\to \tdual{\mathcal{H}}: v\mapsto \inner{v,\cdot}$ is a bijective anti-linear isometry.
\end{corollary}
\begin{corollary}
Every Hilbert space is reflexive.
\end{corollary}
\begin{proof}
TODO
\end{proof}
\begin{corollary}
Every bounded functional defined on a closed subspace of $\mathcal{H}$ can be extended to a functional on $\mathcal{H}$ with the same norm.
\end{corollary}
\begin{proof}
The functional on the closed subspace, say $K$, can be represented as $x\mapsto \inner{v,x}_K$ for some $v\in K$. The extended functional is then simply given by $x\mapsto \inner{v,x}_\mathcal{H}$.
\end{proof}

\begin{proposition}[Representation of sesquilinear forms] \label{sesquilinearRepresentation}
Let $\mathcal{H}_1,\mathcal{H}_2$ be Hilbert spaces over $\mathbb{F}$ and $h:\mathcal{H}_1,\mathcal{H}_2\to\mathbb{F}$ a bounded sesquilinear form. Then there exists a unique bounded operator $S:\mathcal{H}_1 \to \mathcal{H}_2$ such that
\[ h(x,y) = \inner{Sx,y}. \]
This operator has the property $\norm{S} = \norm{h}$.
\end{proposition}
\begin{proof}
For fixed $x$, $y\mapsto h(x,y)$ is a bounded linear functional, so by the Riesz representation theorem \ref{rieszRepresentation} this can be represented by a unique $v_x$. Let $S$ be the function $x\mapsto v_x$. Then $h(x,y) = \inner{Sx,y}$.

To prove this function $S$ is linear, take arbitrary $x_1,x_2\in \mathcal{H}_1;y\in \mathcal{H}_2$ and $\lambda \in \mathbb{F}$. Then
\begin{align*}
\inner{S(\lambda x_1+ x_2),y} &= h(\lambda x_1+ x_2, y) = \overline{\lambda} h(x_1,y)+h(v,y_2) \\
&= \overline{\lambda} \inner{Sx_1, y} + \inner{Sx_2, y} = \inner{\lambda Sx_1 + Sx_2,y},
\end{align*}
so $S$ is linear by lemma \ref{elementaryOrthogonality}.

The equality of norms follows from
\begin{align*}
\norm{h} = \sup_{\substack{x\neq 0 \\ y\neq 0}}\frac{|\inner{Sx,y}|}{\norm{x}\norm{y}} &\geq \sup_{\substack{x\neq 0 \\ Sx\neq 0}}\frac{|\inner{Sx,Sx}|}{\norm{x}\norm{Sx}} = \sup_{x\neq 0}\frac{\norm{Sx}}{\norm{x}} = \norm{S} \\
&\leq \sup_{\substack{x\neq 0 \\ y\neq 0}}\frac{\norm{Sx}\norm{y}}{\norm{x}\norm{y}} = \sup_{x\neq 0}\frac{\norm{Sx}}{\norm{x}} = \norm{S}
\end{align*}
where the second inequality is Cauchy-Schwarz.
\end{proof}

\section{Orthonormal bases}

Hamel basis / Schauder basis / Hilbert basis

Every Hilbert basis is Schauder basis if $V$ is separable.

Hamel basis too big in Banach space??

Necessity of completeness for existence of complete orthonormal system, i.e.\ orthonormal system $\{a_i\}_{i\in I}$ (so $a_i \cdot a_j = \delta_{ij}$) with
\[ v = \sum_{i\in I}(a_i \cdot v)a_i \]
for all $v$. This is equivalent with
\[ v \cdot w = \sum_{i\in I}(v\cdot a_i)(a_i \cdot w) \]
for all $v,w$.


\begin{theorem}[Riesz-Fischer]
Let $\{e_i\}_{i\in I}$ be an orthonormal basis of a Hilbert space $H$ and $\alpha: I\to \C$ a net. Then
\[ \sum_{i\in I}\alpha_i e_i \]
converges \textup{if and only if} $\sum_{i\in I}|\alpha_i|^2 < \infty$. 
\end{theorem}
\begin{proof}
If $\sum_{i\in I}\alpha_i e_i$ converges, then $\sum_{i\in I}|\alpha_i|^2$ is bounded by the Bessel inequality \ref{BesselInequality}.

By monotone convergence, $\sum_{i\in I}|\alpha_i|^2 < \infty$ is equivalent to saying the sum converges. By (ref TODO) $\alpha$ has finite support. So $\sum_{i\in I}\alpha_i e_i$ can be expressed as the series
\[ \sum_{k\in \N}\alpha_{i_k} e_{i_k}. \]
By completeness it is enough to show that $\seq{s_n} = \seq{\sum_{k=0}^n\alpha_{i_k} e_{i_k}}$ is Cauchy. Let $n < m$, then
\[ \norm{s_n - s_m}^2 = \norm{\sum_{k=m+1}^n\alpha_{i_k} e_{i_k}}^2 = \sum_{k=m+1}^n\norm{\alpha_{i_k} e_{i_k}}^2 = \sum_{k=m+1}^n |\alpha_{i_k}|^2 = \sum_{k=0}^n|\alpha_{i_k}|^2 -\sum_{k=0}^m|\alpha_{i_k}|^2.  \]
Since $\seq{\sum_{k=0}^n |\alpha_{i_k}|^2}$ is convergent, it is Cauchy and thus so is $\seq{s_n}$.
\end{proof}
\begin{corollary}
Let $\mathcal{H}$ be a Hilbert space and $D$ be an orthonormal basis of $\mathcal{H}$. Then $\mathcal{H}$ is isometrically isomorphic to $\ell^2(D)$.
\end{corollary}
\begin{corollary}
Hilbert spaces whose orthonormal bases have the same cardinality are isometrically isomorphic.
\end{corollary}

??
\begin{lemma}
Let $(\Omega,\mathcal{A}, \mu)$ be a measure space. Then $L^2(\Omega, \mu)$ is separable \textup{if and only if} $\mu$ is $\sigma$-finite.
\end{lemma}

\begin{lemma}
Let $\{\phi_n(x)\}^\infty_{n=0}$ be an orthonormal basis of $L^2(\Omega, \mu)$ and $\{\psi_n(x)\}^\infty_{n=0}$ be an orthonormal basis of $L^2(\Lambda, \nu)$, then $\{\phi_n(x)\psi_m(y)\}^\infty_{n,m=0}$ is an orthonormal basis of $L^2(\Omega\times\Lambda, \mu\times\nu)$.
\end{lemma}
\begin{proof}
The set $\{\phi_n(x)\psi_m(y)\}^\infty_{n,m=0}$ is orthonormal:
\[ \iint_{\Omega\times\Lambda} \phi_n(x)\psi_m(y)\overline{\phi_{n'}(x)\psi_{m'}(y)}\diff{\mu(x)}\diff{\nu(y)} = \int_\Omega\phi_n(x)\overline{\phi_{n'}(x)}\diff{\mu(x)} \cdot \int_\Lambda\psi_m(y)\overline{\psi_{m'}(y)}\diff{\nu(y)} = \delta_{n,n'}\delta_{m,m'}, \]
using Fubini's theorem and the Hölder inequality (TODO refs).

To show $D = \{\phi_n(x)\psi_m(y)\}^\infty_{n,m=0}$ is an orthonormal basis, we verify point 5. of \ref{totalONBParsevalEquivalence}: if $f\perp D$, then $f = 0$.

If $f\perp D$, then for all $m,n\in \N$
\[ 0 = \inner{f, \phi_n\psi_m} = \iint_{\Omega\times\Lambda}f(x,y)\overline{\phi_n(x)\psi_m(y)}\diff{\mu(x)}\diff{\nu(y)} = \int_\Omega \left(\int_\Lambda f(x,y)\overline{\psi_m(y)}\diff{\nu(y)} \right)\overline{\phi_n(x)}\diff{\mu(x)}.  \]
Using point 5. of \ref{totalONBParsevalEquivalence} in $L^2(\Omega,\mu)$, we see that for all $m$ the function $x\mapsto\int_\Lambda f(x,y)\overline{\psi_m(y)}\diff{\nu(y)}$ is $0$ as an element of $L^2(\Omega, \mu)$, i.e.\ it is $0$ a.e. as a function of $x$. Let
\[ E_m = \setbuilder{x\in\Omega}{ \int_\Lambda f(x,y)\overline{\psi_m(y)}\diff{\nu(y)} \neq 0} \]
and set $E = \bigcup_{m\in\N}E_m$.
Then
\[ \mu(E) =  \mu\left(\bigcup_{m\in \N}E_m\right) \leq \sum_{m\in\N}\mu(E_m) = 0. \]

For $x\notin E$, we have $\int_\Lambda f(x,y)\overline{\psi_m(y)}\diff{\nu(y)} = 0$, so by the same logic $f(x,y) = 0$ for almost all $y$. 

Now $|f|^2$ is integrable and
\[ \iint_{\Omega\times \Lambda}|f(x,y)|^2\diff{\mu(x)}\diff{\nu(y)} = \int_{\Omega\setminus E}\int_\Lambda |f(x,y)|^2\diff{\mu(x)}\diff{\nu(y)} = 0, \]
so $f=0$ in $L^2(\Omega\times\Lambda, \mu\times\nu)$.
\end{proof}


\section{Adjoints of operators}
\begin{definition}
Let $H,K$ be Hilbert spaces and $T: H\not\to K$ an operator. An \udef{adjoint} of $T$ is an operator $S: K\not\to H$ such that
\[ \inner{w,Tv}_K = \inner{S w,v}_H \quad \forall v\in \dom(T),\; \forall w\in \dom(S). \]
\end{definition}

\begin{theorem}[Hellinger-Toeplitz]
Let $T: H\to K$ be an operator between Hilbert spaces (which is defined everywhere), then $T$ has an adjoint that is defined everywhere \textup{if and only if} it is bounded.
\end{theorem}
\begin{proof}
The ``if'' will be shown below by explicit construction. For the ``only if'', take such an operator $T$.

First we show $T$ has closed graph, by using proposition \ref{closedGraphEquivalence}: assume $(x_n)$ converges to $x$ and $(Tx_n)$ converges to $y$. Then
\[ \inner{z, Tx} = \inner{Sz,x} = \lim_n\inner{Sz, x_n} = \lim_n\inner{z, Tx_n} = \inner{z, y} \]
where we have used the boundedness of $x\mapsto\inner{z,x}$. By the non-degeneracy of the inner product, $Tx = y$. So the graph of $T$ is closed. Similarly the graph of $S$ is closed. Applying the closed graph theorem \ref{closedGraphTheorem}, yields the boundedness of $T$ and $S$.
\end{proof}
\begin{corollary}
Everywhere-defined symmetric operators are bounded.
\end{corollary}

\begin{example}
The adjoint of the left-shift operator
\[ S_L: \ell^2(\N) \to \ell^2(\N): (x_n)_{n\in\N} \mapsto (x_{n+1})_{n\in\N} \]
is the right-shift operator
\[ S_R: \ell^2(\N) \to \ell^2(\N): (x_n)_{n\in\N} \mapsto \left(\begin{cases}
x_{n-1} & (n\geq 1) \\ 0 &(n=0)
\end{cases}\right)_{n\in \N}. \]
\end{example}

\subsection{The adjoint as a relation}
\begin{lemma}
Let $T: H\not\to K$ be an operator between Hilbert spaces. Let $S_1, S_2$ be adjoints of $T$ then for all $x\in \dom(S_1)\cap\dom(S_2)$ we have $S_1(x) - S_2(x) \in \dom(T)^\perp$.

Conversely, let $S$ be an adjoint of $T$ and $x\in\dom(S)$. Then for all $v\in \dom(T)^\perp$ there exists an adjoint $S'$ such that $S'(x) = S(x) + v$.
\end{lemma}
\begin{proof}
For all $u\in \dom(T)$ we have
\[ \inner{S_1(x) - S_2(x), u}_H = \inner{S_1(x), u}_H - \inner{S_2(x), u}_H = \inner{x, Tu}_K - \inner{x, Tu}_K = 0. \]
So $\{S_1(x) - S_2(x)\} \in \dom(T)^\perp$.

For the converse, set $S' = S + \frac{\inner{x,\cdot}_K}{\inner{x,x}_K}v$. This is an adjoint: for all $a\in \dom(T), b\in \dom(S') = \dom(S)$ we have
\[  \inner{S' b,a}_H = \inner{Sb, a}_H + \frac{\inner{x,b}_K}{\inner{x,x}_K}\inner{v,a}_H = \inner{Sb, a}_H = \inner{b,Ta}_K. \]
\end{proof}
\begin{corollary} \label{agreementAdjoints}
Let $T: H\not\to K$ be a densely defined operator between Hilbert spaces. Let $S_1, S_2$ be adjoints of $T$ then for all $x\in \dom(S_1)\cap\dom(S_2)$ we have $S_1(x) = S_2(x)$.
\end{corollary}
\begin{proof}
We have $\dom(T)^\perp = \overline{\dom(T)}^\perp = H^\perp = \{0\}$. So $S_1(x) - S_2(x) = 0$.
\end{proof}
\begin{corollary} \label{maximalAdjointIsOperator}
Let $T: H\not\to K$ be an operator between Hilbert spaces. Then
\[ \bigcup\setbuilder{\graph(S)}{\text{$S\in (K\not\to H)$ is an adjoint of $T$}} \]
is the graph of an operator \textup{if and only if} $T$ is densely defined.
\end{corollary}

\begin{definition}
Let $T: H\not\to K$ be an operator between Hilbert spaces. We define the adjoint $T^*$ as the \emph{relation} on $(H,K)$ with graph
\[ \graph(T^*) \defeq \bigcup\setbuilder{\graph(S)}{\text{$S\in (K\not\to H)$ is an adjoint of $T$}}. \]
\begin{itemize}
\item If $T^* = T$, we say $T$ is \udef{self-adjoint}.
\item If $T^* = -T$, we say $T$ is \udef{skew-adjoint}.
\end{itemize}
We denote the set of self-adjoint operators on $H$ by $\SelfAdjoints(H)$.
\end{definition}
Note that, by \ref{maximalAdjointIsOperator}, the adjoint is a function if and only if $T$ is densely defined.

\begin{lemma} \label{everywhereDefinedAdjointLemma}
Let $T: H\not\to K$ be a densely defined operator between Hilbert spaces. If $S$ is an adjoint of $T$ that is defined everywhere, then $T^* = S$.
\end{lemma}
\begin{corollary}
Let $H$ be a Hilbert space. Then $\id_H^* = \id_H$.
\end{corollary}

\begin{proposition} \label{adjointDomain}
Let $T: H\not\to K$ be an operator between Hilbert spaces. Then
\[ \dom(T^*) = \setbuilder{x\in K}{\text{$\dom(T)\to \F: u\mapsto \inner{x, Tu}$ is a bounded functional}}. \]
\end{proposition}
\begin{proof}
$\boxed{\subseteq}$ If $\omega_x: u\mapsto \inner{x, Tu}$ is bounded, then its domain can be extended by linearity to all of $H$ and it has a Riesz vector $x^*$ such that $\omega_x = u\mapsto \inner{x^*, u}$. The linear operator with domain $\Span\{x\}$ that maps $x$ to $x^*$ is then an adjoint.

$\boxed{\supseteq}$ If $x\in\dom(T^*)$, then, using the Cauchy-Schwarz inequality,
\[ |\inner{x,Tu}| = |\inner{T^*x,u}| \leq \norm{T^*x}\;\norm{u}, \]
so the functional $u\mapsto \inner{x, Tu}$ is bounded.
\end{proof}
\begin{corollary}
The domain $\dom(T^*)$ is a vector space and in particular contains $0$.
\end{corollary}
\begin{corollary} \label{HilbertAdjointAntitone}
Let $S,T: H\not\to K$ be operators between Hilbert spaces such that $S\subseteq T$. Then $T^* \subseteq S^*$.
\end{corollary}
\begin{corollary}
Let $H, K$ be Hilbert spaces. Then $(*,*)$ is a Galois connection between $\sSet{(H\not\to K), \subseteq}$ and $\sSet{(K\not\to H), \subseteq}$.
\end{corollary}
\begin{proof}
TODO
\end{proof}

\begin{proposition}[Algebraic properties of the adjoint] \label{adjointAlgebraicProperties}
Let $T,S$ be compatible operators between Hilbert spaces and $\lambda\in\C$. Then
\begin{enumerate}
\item $S^* + T^* \subseteq (S+T)^*$;
\item $S^*T^* \subseteq (TS)^*$;
\item $\begin{pmatrix}
\id & 0 \\ 0 & \overline{\lambda}\id
\end{pmatrix} \graph(T^*) \subseteq (\lambda T)^*$;
\item if $\lambda \neq 0$, then $\begin{pmatrix}
\id & 0 \\ 0 & \overline{\lambda}\id
\end{pmatrix} \graph(T^*) = (\lambda T)^*$;
\item $(T+\lambda\id)^* = T^*+\overline{\lambda}\id$.
\end{enumerate}
\end{proposition}
\begin{proof}
(1) Let $f$ be an adjoint of $S$ and $g$ an adjoint of $T$. It is enough to see that $f+g$ is an adjoint of $S+T$. Indeed $\forall w\in \dom(f + g), v\in \dom(S+T)$
\[ \inner{(f + g)(w), v} = \inner{f(w),v} + \inner{g(w), Tv} = \inner{w,Sv} + \inner{w,Tv} = \inner{w,(S+T)v}. \]

(2) Let $f$ be an adjoint of $T$ and $g$ an adjoint of $S$. It is enough to see that $gf$ is an adjoint of $TS$. Indeed
\[ \inner{g\circ f(w), v} = \inner{f(w), Sv} = \inner{w,TSv} \qquad \forall w\in \dom(g\circ f), v\in \dom(TS). \] 

(3) Let $f$ be an adjoint of $T$. It is enough to see that $\overline{\lambda}f$ is an adjoint of $\lambda T$. Indeed
\[ \inner{\overline{\lambda}f(w), v} = \lambda\inner{f(w), v} = \lambda\inner{w,Tv} = \inner{w,\lambda Tv} \qquad \forall w\in \dom(f), v\in \dom(T). \]

(4) One inclusion has already been proved. For the other inclusion, let $f$ be an adjoint of $\lambda T$. It is enough to see that $\overline{\lambda^{-1}}f$ is an adjoint of $T$, because then $f = \overline{\lambda}\cdot\overline{\lambda^{-1}}f \subseteq \begin{pmatrix}
\id & 0 \\ 0 & \overline{\lambda}\id
\end{pmatrix} \graph(T^*)$. Indeed
\[ \inner{\overline{\lambda^{-1}}f(w), v} = \lambda^{-1}\inner{f(w),v} = \inner{w,\lambda^{-1}\lambda Tv} = \inner{w,Tv} \quad \forall w\in \dom(f), v\in \dom(T). \]

(5) From (1) and (3), we have $T^*+\overline{\lambda}\id \subseteq (T+\lambda\id)^*$. Conversely, let $f$ be an adjoint of $(T+\lambda\id)$. It is enough to see that $f-\overline{\lambda}\id$ is an adjoint of $T$. Indeed, $\forall w\in\dom(f-\overline{\lambda}\id), v\in \dom(T)$,
\begin{align*}
\inner{(f-\overline{\lambda}\id)(w),v} &= \inner{f(w),v}-\lambda \inner{w,v} \\
&= \inner{w,(T+\lambda\id)(v)}-\lambda \inner{w,v} \\
&= \inner{w,Tv} - \lambda\inner{w,v} + \lambda\inner{w,v} = \inner{w,Tv}.
\end{align*}
\end{proof}

\begin{proposition} \label{adjointGraph}
Let $T: H\not\to K$ be an operator between Hilbert spaces. Then
\begin{align*}
\graph(T^*) &= \left( \begin{pmatrix}
0 & -\id \\ \id & 0
\end{pmatrix}\graph(T) \right)^\perp 
=  \begin{pmatrix}
0 & -\id \\ \id & 0
\end{pmatrix}\graph(T)^\perp.
\end{align*}
If $T$ is densely defined, then $T^*$ is a closed operator.
\end{proposition}
\begin{proof}
We have
\[ \graph(T^*) = \bigcup\setbuilder{\graph(S)}{\text{$S\in (K\not\to H)$ is an adjoint of $T$}}. \]
Take an adjoint $S$ and $(w, Sw)$ in $\graph(S)$. Then for all $v\in\dom(T)$:
\[ 0 = \inner{w, Tv}_K - \inner{Sw, v}_H = \inner{w, Tv}_K + \inner{Sw, -v}_H = \inner{(w, Sw), (Tv,-v)}_{K\oplus H}. \]
So $(Tv,-v) = \begin{pmatrix}
0 & -\id \\ \id & 0
\end{pmatrix} (v,Tv) \in \graph(S)^\perp $.

The final equality follows from \ref{perpUnderIsometry}, using the fact that $\begin{pmatrix}
0 & -\id \\ \id & 0
\end{pmatrix}$ is a surjective isometry.

If $T$ is densely defined, then $T^*$ is a function by \ref{maximalAdjointIsOperator}. It is closed by \ref{orthogonalComplementClosed}.
\end{proof}
\begin{corollary} \label{adjointDenselyDefinedClosable}
Let $T: H\not\to K$ be a densely defined operator between Hilbert spaces. Then $T^*$ is densely defined \textup{if and only if} $T$ is closable. In this case $\overline{T} = T^{**}$.
\end{corollary}
\begin{proof}
From the proposition we have
\begin{align*}
\graph(T^{**}) &=  \begin{pmatrix}
0 & -\id \\ \id & 0
\end{pmatrix}\graph(T^*)^\perp 
=  \begin{pmatrix}
0 & -\id \\ \id & 0
\end{pmatrix}\left(\begin{pmatrix}
0 & -\id \\ \id & 0
\end{pmatrix}\graph(T)^\perp\right)^\perp \\
&= \begin{pmatrix}
0 & -\id \\ \id & 0
\end{pmatrix}^2\graph(T)^{\perp\perp} = -\graph(T)^{\perp\perp}
= \overline{\graph(T)} = \graph(\overline{T}).
\end{align*}
The right-hand side is the graph of an operator iff $T$ is closable and the left-hand side is the graph of an operator iff $T^*$ is densely defined, by \ref{maximalAdjointIsOperator}.

By definition, the left-hand side is equal to $\graph(T^{**})$, so $T^{**} = \overline{T}$.
\end{proof}

\begin{proposition} \label{kernelImageAdjoint}
Let $T: H\not\to K$ be a densely defined operator between Hilbert spaces. Then
\begin{enumerate}
\item $\ker(T^*) = \im(T)^\perp$;
\item $\ker(T) = \im(T^*)^\perp$.
\end{enumerate}
\end{proposition}
\begin{proof}
First take $v\in \ker(T^*)$, then $T^*(v) = 0$ which implies
\[ \forall x \in\dom(T): \inner{T^*(v), x} = 0 \;\implies\; \forall x \in\dom(T): \inner{v, T(x)} = 0 \;\implies\; v\perp \im(T).  \]
Next take $v\perp \im(T)$ TODO complete.
\end{proof}
TODO: link with previous? + Drop densely defined.

\begin{proposition} \label{adjointRangeCriterion}
Let $S: K\not\to H$ and $T: H\not\to K$ be linear operators between Hilbert spaces. If
\[ \im(S\cap T^*) = H \qquad\text{and}\qquad \im(T\cap S^*) = K, \]
then $S$ and $T$ are densely defined with $S^* = T$ and $T^* = S$.
\end{proposition}
\begin{proof}
Notice that $S\cap T^*$ and $T\cap S^*$ are linear operators that are adjoints of each other.

We claim that they are densely defined: take $x\in \dom(S\cap T^*)^\perp$. Then there exists some $y\in H$ such that $x = (T\cap S^*)y$ because of surjectivity. Now for all $z\in \dom(S\cap T^*)$
\[ 0 = \inner{z,x} = \inner{z, (T\cap S^*)y} = \inner{(S\cap T^*)z, y}, \]
so $\inner{z',y} = 0$ for all $z'\in H$, by surjectivity. This means, by \ref{elementaryOrthogonality}, that $y=0$ and thus also $x = (T\cap S^*)y = 0$. We conclude that $\dom(S\cap T^*)^\perp = \{0\}$, meaning $(S\cap T^*)$ is densely defined. The argument for $(T\cap S^*)$ is similar.

It follows that $S$ and $T$ must be densely defined. We have, by \ref{kernelImageAdjoint},
\[ \ker(S) = \im(S^*)^\perp \subseteq \im(T\cap S^*)^\perp = \{0\}. \]
Similarly $\ker(T) = \ker(S^*) = \ker(T^*) = \{0\}$.

So we have $\ker(S) = \ker(T^*)$, $\im(S)\subseteq \im(S\cap T^*)$ and $\im(T^*)\subseteq \im(S\cap T^*)$. The equality $S = T^*$ follows from \ref{partialFunctionSubset}. The equality $T = S^*$ is similar.
\end{proof}

\begin{proposition}
Let $T: H\not\to K$ be a densely defined operator between Hilbert spaces. Then
\begin{enumerate}
\item $\im(T)$ is dense in $K$ \textup{if and only if} $T^*$ is injective;
\item if $T$ and $T^*$ are injective, then $(T^*)^{-1} = (T^{-1})^*$;
\item if $T$ is closable and $\overline{T}$ is injective, then $\overline{T}^{\,-1} = \overline{T^{-1}}$.
\end{enumerate}
\begin{proof}
(1) This is immediate from \ref{kernelImageAdjoint} and \ref{injectivityKernelTriviality}:
\[ \text{$\im(T)$ is dense} \quad\iff\quad \{0\} = \im(T)^\perp = \ker(T^*). \]

(2) We have $\graph(T^{-1}) = \begin{pmatrix}
0 & \id \\ \id & 0
\end{pmatrix}\graph(T)$. Also note that $\begin{pmatrix}
0 & \id \\ \id & 0
\end{pmatrix}$ and $\begin{pmatrix}
0 & -\id \\ \id & 0
\end{pmatrix}$ commute. Then we compute using \ref{adjointGraph}:
\begin{align*}
\graph((T^*)^{-1}) &= \begin{pmatrix}
0 & \id \\ \id & 0
\end{pmatrix}\begin{pmatrix}
0 & -\id \\ \id & 0
\end{pmatrix}\graph(T)^\perp \\
&= \begin{pmatrix}
0 & -\id \\ \id & 0
\end{pmatrix}\begin{pmatrix}
0 & \id \\ \id & 0
\end{pmatrix}\graph(T)^\perp \\
&= \begin{pmatrix}
0 & -\id \\ \id & 0
\end{pmatrix}\left(\begin{pmatrix}
0 & \id \\ \id & 0
\end{pmatrix}\graph(T)\right)^\perp = \graph((T^{-1})^*).
\end{align*}
The penultimate equality follows from \ref{perpUnderIsometry}, using the fact that $\begin{pmatrix}
0 & \id \\ \id & 0
\end{pmatrix}$ is a surjective isometry.
\end{proof}

\end{proposition}

\url{https://arxiv.org/pdf/1507.08418.pdf}
\url{https://link.springer.com/article/10.1007/s43036-020-00068-4}

\subsection{Bounded operators}
\begin{proposition}
Let $T: H\to K$ be a closed densely defined operator between Hilbert spaces. Then
\begin{enumerate}
\item $T\in\Bounded(H,K)$ \textup{if and only if} $T^*\in\Bounded(K,H)$.
\end{enumerate}
In this case
\begin{enumerate} \setcounter{enumi}{1}
\item $\norm{T} = \norm{T^*}$;
\item $T^* = C_H^{-1}T^tC_K$, where $C_K$ is the Riesz isometry from \ref{RieszIsometry}.
\end{enumerate}
\end{proposition}
\begin{proof}
First assume $T\in\Bounded(H,K)$. Then $u\mapsto \inner{x,Tu}$ is a bounded functional for all $x\in K$, so $\dom(T^*) = K$ by \ref{adjointDomain}. Also $T^*$ is closed by \ref{adjointGraph}, so it is bounded by the closed graph theorem \ref{closedGraphTheorem}.

Now assume $T^*\in\Bounded(K,H)$. By the previous argument $T = \overline{T} = T^{**}\in\Bounded(H,K)$.

The function $(x,u)\mapsto \inner{x,Tu}$ is a bounded sesquilinear form. By proposition \ref{sesquilinearRepresentation}, $T^*$ must be the unique $S$ from the proposition, which has norm $\norm{T}$.

Finally we note that $C_H^{-1}T^tC_K$ is an adjoint with domain $K$ and conclude by \ref{everywhereDefinedAdjointLemma}.
\end{proof}

\begin{lemma}
The adjoint defines a map $*:\Bounded(H,K)\to \Bounded(K,H)$ that is anti-linear and continuous in the weak and uniform operator topologies. It is continuous in the strong operator topology \textup{if and only if} finite dimensional.
\end{lemma}
\begin{proof}
By the proposition the adjoint map is anti-linear. It is also bounded with norm $1$. Then by corollary \ref{boundedAntiLinearMaps} it must be bounded.

TODO
\end{proof}

\begin{lemma} \label{HilbertAdjointLemma}
Let $S,T\in\Bounded(H,K)$ and $\lambda \in \mathbb{F}$.
\begin{enumerate}
\item $(T^*)^* = T$;
\item $(S+T)^* = S^* + T^*$;
\item $(\lambda T)^* = \bar{\lambda}T^*$;
\item $\id_V^* = \id_V$.
\end{enumerate}
Let $T\in\Bounded(H_1,H_2), S\in\Bounded(H_2,H_3)$
\begin{enumerate}
\setcounter{enumi}{4}
\item $(ST)^* = T^*S^*$.
\end{enumerate}
\end{lemma}
\begin{proof}
These follow straight from \ref{adjointAlgebraicProperties} and the fact that the operators and adjoints are everywhere defined.
\end{proof}

\begin{note}
Useful exercise: The identities of \ref{HilbertAdjointLemma} can also be proven by elementary manipulations. For example, to prove 1., we take arbitrary $v\in H$ and $w\in K$, Then
\[ \inner{w,Tv} = \inner{T^*w,v} = \overline{\inner{v,T^*w}} = \overline{\inner{(T^*)^*v,w}} = \inner{w, (T^*)^*v}. \]
By lemma \ref{elementaryOrthogonality} we have $Tv = (T^*)^*v$ for all $v\in V$. 
\end{note}

\begin{proposition}
Let $H,K$ be Hilbert spaces and $T:H\to K$ a bijective bounded linear operator with bounded inverse. Then $(T^*)^{-1}$ exists and
\[ (T^*)^{-1} = (T^{-1})^*. \]
\end{proposition}
\begin{proof}
We prove $(T^{-1})^*$ is both a left- and a right-inverse of $T^*$: $\forall x\in H, y\in K$
\begin{align*}
\inner{T^*(T^{-1})^*x,y} &= \inner{x,T^{-1}Ty} = \inner{x,y} \\
\inner{x,(T^{-1})^*T^*y} &= \inner{TT^{-1}x,y} = \inner{x,y}
\end{align*}
So, by lemma \ref{elementaryOrthogonality}, $T^*(T^{-1})^* = \id_H$ and $(T^{-1})^*T^* = \id_K$.
\end{proof}

\begin{proposition}
Let $T\in\Bounded(H,K)$. Then
\[ \ker T = (\im T^*)^\perp \qquad \text{and thus} \qquad \overline{\im(T)} \subseteq \ker(T^*)^\perp. \]
\end{proposition}
\begin{proof}
\[ x\in \ker T \iff Tx = 0 \iff \forall y\in K: \inner{y, Tx}=0 \iff \forall y\in K: \inner{T^*y, x}=0 \iff x\perp T^*[K]. \]
\end{proof}
In particular $\im(T)$ is closed iff it is equal to $\ker(T^*)^\perp$. This is sometimes known as the closed range theorem. This is, e.g\, the case when $T$ is bounded below, see \ref{boundedBelowClosedRange}.

\begin{proposition} \label{normOfSquare}
Let $T\in \Bounded(H,K)$ with $H,K$ Hilbert spaces. Then
\[ \norm{T^*T}= \norm{T}^2 = \norm{TT^*}. \]
\end{proposition}
\begin{proof}
For $\norm{T^*T}= \norm{T}^2$ first observe that
\[ \norm{T^*T} \leq \norm{T^*}\cdot\norm{T} = \norm{T}^2. \]
Conversely, $\forall x\in H$:
\[ \norm{T(x)}^2 = \inner{Tx,Tx} = \inner{T^*Tx,x} \leq \norm{T^*Tx}\cdot \norm{x} \leq \norm{T^*T}\cdot\norm{x}^2. \]
The other equality follows by applying the first to $T^*$ and using $\norm{T^*}=\norm{T}$.
\end{proof}

\subsection{Normal operators}
\begin{definition}
A densely defined linear operator $T$ on a Hilbert space $H$ is \udef{normal} if it is closed and $TT^* = T^*T$.
\end{definition}
Self-adjoint and unitary operators are normal.

TODO 3.10 Self-Adjoint, Unitary and Normal Operators from Kreyszig.


\begin{proposition}
Let $T: H\not\to H$ be a densely defined operator. Then $T$ is normal \textup{if and only if} $\dom(T) = \dom(T^*)$ and $\forall x\in H: \norm{Tx} = \norm{T^*x}$.
\end{proposition}
\begin{proof}
First, assume $T$ normal. Then
\[ \norm{Tx}^2 = |\inner{Tx,Tx}| = |\inner{T^*Tx,x}| = |\inner{TT^*x,x}| = |\inner{T^*x,T^*x}| = \norm{T^*x}^2. \]

For the converse, we have $\inner{Tx, Ty} = \inner{T^*x, T^* y}$ for all $x,y\in H$ by polarisation. From this we have $\inner{T^*Tx, y} = \inner{TT^*x, y}$ and normality follows from \ref{equalityOfMapsInnerProductSpaces}.

TODO question of domain.
\url{https://www.math.drexel.edu/faculty/mjz55/wp-content/uploads/sites/8/2017/01/normalnotes.pdf}.
\end{proof}
\begin{corollary} \label{equalityKernelAdjointNormal}
If $T$ is a normal operator, then $\ker T = \ker T^*$.
\end{corollary}
\begin{proof}
We have $x\in\ker(T) \iff \norm{Tx} = 0 \iff \norm{T^*x} = 0 \iff x\in\ker(T^*)$. 
\end{proof}
\begin{corollary}
If $T$ is a normal operatorm then
\begin{enumerate}
\item $\rspec(T) = \emptyset$;
\item $\spec(T) = \apspec(T)$.
\end{enumerate} 
\end{corollary}
\begin{proof}
If $T$ is normal, then so is $\lambda\id-T$. Now $\lambda\in\rspec(T)$ iff $\ker(\lambda\id - T) = \{0\}$ and $\im(\lambda\id-T)^\perp \neq \{0\}$, but $\im(\lambda\id-T)^\perp = \ker(\lambda\id-T)^* = \ker(\lambda\id-T)$. By \ref{kernelImageAdjoint} and the previous corollary. This is a contradiction.

(2) then follows straight from (1).
\end{proof}

\begin{theorem} \label{closureNumericRangeConvexHullSpectrum}
The closure of the numerical range of a normal operator is the
convex hull of its spectrum.
\end{theorem}
\begin{proof}
Normal operators $T$ are by definition closed, so $\spec(T)\subseteq \overline{\NumRange(T)}$ by \ref{spectralInclusionNumericalRange}. TODO
\end{proof}

\begin{lemma} \label{normalSpectralRadiusEqualsNorm}
For normal elements the spectral radius equals the norm.
\end{lemma}

\begin{lemma}
A normal operator on a Hilbert space is invertible \textup{if and only if} it is bounded below.
\end{lemma}

\subsection{Symmetric and self-adjoint operators}
\subsubsection{Domain related matters}
A symmetric operator $A$ is self-adjoint if and only if $\dom(A) = \dom(A^*)$.
\begin{lemma} \label{symmetricOperatorAdjointInclusion}
Let $A$ be a symmetric densely defined operator on a Hilbert space $H$. Then
\begin{enumerate}
\item $\dom(A) \subseteq \dom(A^*)$;
\item $A = A^*|_{\dom(A)}$;
\item $A$ is closable and $\overline{A} = A^{**}$.
\end{enumerate}
\end{lemma}
\begin{proof}
(1, 2) This is immediate from symmetry: $\inner{Ax,y} = \inner{x,Ay}$ for all $x,y\in\dom(A)$.

(3) From (1) we see that $A^*$ is densely defined, because the superset of a dense set is dense. The result follows by \ref{adjointDenselyDefinedClosable}.
\end{proof}
\begin{corollary}
A closed and densely defined symmetric operator is self-adjoint \textup{if and only if} $A^*$ is symmetric.
\end{corollary}
\begin{proof}
If $A^*$ is not symmetric, $A$ can clearly not be self-adjoint.

Assume $A^*$ is symmetric. Then $\dom(A) \subseteq \dom(A^*) \subseteq \dom(\overline{A}) = \dom(A)$.
\end{proof}

\begin{proposition} \label{selfAdjointMaximal}
A self-adjoint operator cannot have a proper symmetric extension.
\end{proposition}
\begin{proof}
Assume $A$ self-adjoint and $A\subseteq B$ for some symmetric operator $B$. Then
\[ A \subseteq B \subseteq B^* \subseteq A^* = A, \]
so $A = B$. We have used \ref{symmetricOperatorAdjointInclusion} and \ref{HilbertAdjointAntitone}.
\end{proof}
\begin{corollary}
Let $A$ be a densely defined symmetric operator. If $\overline{A}$ is self-adjoint, then it is the unique self-adjoint extension of $A$.
\end{corollary}
Note that $\overline{A}$ is always an operator by \ref{symmetricOperatorAdjointInclusion}.
\begin{proof}
Let $B$ be a self-adjoint extension of $A$. Then $\overline{A} = A^{**}\subseteq B^{**} = B$, by \ref{HilbertAdjointAntitone}. This means that $B$ is symmetric extension of the self-adjoint operator $\overline{A}$, which, by the proposition, implies $B = \overline{A}$.
\end{proof}
In general it is possible for an unbounded,
symmetric operator to not have a self-adjoint extension or have multiple self-adjoint extensions, even if it is densely defined. (TODO example)

\begin{definition}
Let $A$ be a densely defined symmetric operator whose closure is self-adjoint. We call $A$
\begin{itemize}
\item \udef{essentially self-adjoint};
\item a \udef{core} for $A$.
\end{itemize}
\end{definition}

\begin{example}
Consider the operator
\[ A: L^2(a,b) \to L^2(a,b): f\mapsto i\od{f}{x} \]
with domain
\[ \dom(A) = \setbuilder{f\in L^2(a,b)}{\od{f}{x}\in L^2(a,b),\; f(a) = 0 = f(b)}. \]
Then
\begin{align*}
\inner{g, Af} &= \int_{a}^b \overline{g(x)}i\od{f(x)}{x}\diff{x} \\
&= \overline{g(b)}f(b) - \overline{g(a)}f(a) - \int_{a}^b \Big(i \od{}{x}\overline{g(x)}\Big)f(x)\diff{x} \\
&= \int_a^b \overline{i \od{g(x)}{x}} f(x) \diff{x} = \inner{Ag, f}.
\end{align*}
So $A$ is symmetric and $\dom(A^*) = \setbuilder{f\in L^2(a,b)}{\od{f}{x}\in L^2(a,b)}$. We cannot extend $\dom(A)$ while keeping $\dom(A^*)$ the same, because $A$ would no longer be symmetric due to boundary terms.

There are, however, multiple ways we can extend $A$ to a self-adjoint operator (in each case $\dom(A^*)$ must shrink).

Let $A_\alpha$, for $\alpha\in \R$, be the operator $A$ with domain
\[ \dom(A_\alpha) = \setbuilder{f\in L^2(a,b)}{\od{f}{x}\in L^2(a,b),\; f(b) = e^{i\alpha}f(b)}. \]
We must have $\forall f\in \dom(A_\alpha)$ and $g\in\dom(A^*_\alpha)$ that
\[ \overline{g(b)}f(b) - \overline{g(a)}f(a) = f(a)\Big(e^{i\alpha}\overline{g(b)} - \overline{g(a)}\Big) = 0, \]
so we have $e^{-i\alpha}g(b) = g(a)$ and thus $g(b) = e^{i\alpha}g(a)$, which means $\dom(A_\alpha^*) = \dom(A_\alpha)$. So $A_\alpha$ is a self-adjoint extension of $A$ for all $\alpha\in \R$.

TODO: compare Aharonov-Bohm
\end{example}
Notice that the operator
\[ T: L^2(a,b) \to L^2(a,b): f\mapsto i\od{f}{x} \]
with domain
\[ \dom(T) = \setbuilder{f\in L^2(a,b)}{\od{f}{x}\in L^2(a,b)} \]
is not symmetric. In this case
\[  \dom(T^*) = \setbuilder{f\in L^2(a,b)}{\od{f}{x}\in L^2(a,b),\; f(a)=0=f(b)}, \]
so $\dom(T^*) \subseteq \dom(T)$.

\subsubsection{Spectrum and related criteria}
TODO: $iA$ dissipative!
\begin{lemma}
Let $A$ be a symmetric operator on a complex Hilbert space $H$. If $\exists z \in \C\setminus\R: \; \im(A+z\id) = H$, then $A$ is densely defined.
\end{lemma}
\begin{proof}
Let $A+z\id$ be surjective and suppose, towards a contradiction that there exists an $y\perp \dom(A)$. Then $y = (A+z\id)x$ for some $x\in\dom(A)$ by surjectivity. Then
\[ 0 = \Im\inner{x,y} = \Im\inner{x, (A+z\id)x} = \cancel{\Im\inner{x,Ax}} + \Im \inner{x,zx} = \Im(z)\norm{x}^2. \]
By assumption, $\Im(z) \neq 0$, so $x=0$, meaning $y = (A+z\id)x = 0$ and thus $\dom(A)^\perp = \{0\}$.
\end{proof}

\begin{proposition}
Let $A$ be a symmetric operator on a complex Hilbert space $H$. Then $A + z\id_H$ is bounded below by $|\Im \lambda|$ for all $\lambda \in \C\setminus\R$.
\end{proposition}
\begin{proof}
We first calculate, $\forall x\in H$:
\[ \Im\inner{x, (A+ z\id_H)x} = \cancel{\Im\inner{x,Ax}} + \Im z\norm{x}^2. \]
Thus
\[ |\Im\lambda|\;\norm{x}^2 = |\Im\inner{x, (A + z\id_H)x}| \leq |\inner{x, (A + z\id_H)x}| \leq \norm{x}\;\norm{(A + z\id_H)x}, \]
which means that $\norm{(A + z\id_H)x} \geq |\Im\lambda|\;\norm{x}$, so $A + z\id_H$ is bounded below by $|\Im \lambda|$.
\end{proof}
\begin{corollary} \label{approximateSpectrumSymmetricOperator}
Let $A$ be a symmetric operator on a complex Hilbert space $H$. Then $\apspec(A) \subseteq \R$.
\end{corollary}
\begin{corollary}
The eigenvalues of a symmetric operator are real.
\end{corollary}
\begin{proof}
This is immediate using $\pspec(A)\subseteq \apspec(A)$. We can also give a direct calculation:

Assume there exists an $x\in \ker(\lambda\id_H - A)\setminus\{0\}$. Then $Ax = \lambda x$ and thus
\[ \lambda\norm{x}^2 = \lambda\inner{x,x} = \inner{x, \lambda x} = \inner{x,Ax} = \inner{Ax,x} = \inner{\lambda x, x} = \overline{\lambda}\inner{x,x} = \overline{\lambda}\norm{x}^2. \]
Because $\norm{x}^2 \neq 0$, we have $\lambda = \overline{\lambda}$, meaning $\lambda$ is real.
\end{proof}
\begin{corollary} \label{symmetricResolvent}
Let $A$ be a symmetric operator on a complex Hilbert space $H$. Then for all $\lambda\in\C\setminus\R$, the resolvent $R_A(\lambda)$ well-defined and bounded by $\norm{R_A(\lambda)}\leq 1/|\Im \lambda|$.
\end{corollary}
Note this does not mean $\C\setminus\R\subseteq \res(A)$, as $\dom(R_A(\lambda))$ may not be all of $H$.
\begin{proof}
This is an application of \ref{boundedBelow}.
\end{proof}

\begin{proposition} \label{rangeSelfAdjointCriterion}
Let $A$ be a symmetric operator on a Hilbert space $H$. The following are equivalent:
\begin{enumerate}
\item $\exists z \in \C: \; \im(A+z\id) = H = \im(A+\overline{z}\id)$;
\item $A$ is self-adjoint;
\item $\rspec(A) = \emptyset$.
\end{enumerate}
In this case $\spec(A) = \apspec(A)$.
\end{proposition}
Notice that in (1) we include $\R$ and in (3) we exclude $\R$.
\begin{proof}
$(1) \Rightarrow (2)$ From \ref{symmetricOperatorAdjointInclusion}, we have $A\subseteq A^*$ and thus $A+z\id = (A^* + z\id)\cap(A+z\id)$. From point (5) of \ref{adjointAlgebraicProperties}, we have $A+z\id = (A^* + z\id)\cap(A+z\id) = (A+\overline{z}\id)^*\cap (A+z\id)$.

We then use \ref{adjointRangeCriterion} with $S = A+z\id$ and $T = A+\overline{z}\id$ to obtain $(A+z\id)^* = A+\overline{z}\id$. Subtracting $\overline{z}\id$ from each side yields the result.

$(2) \Rightarrow (3)$ Fix some $z \in \spec(A) \setminus\pspec(A)$ we need to show that $\overline{\im(\lambda\id - A)} = H$. Indeed
\[ \overline{\im(A+z\id)} = \ker(A^* + \overline{z}\id)^\perp = \ker(A+\overline{z}\id)^\perp = \{0\}^\perp = H. \]

$(3) \Rightarrow (1)$ We have $\spec(A) = \apspec(A)$. Because $\apspec\subseteq \R$, by \ref{approximateSpectrumSymmetricOperator}, we have that $A+z\id$ is surjective for all $\C\setminus\spec(A) = \C\setminus\apspec(A) \supseteq \C\setminus\R$.
\end{proof}
\begin{corollary}
Every surjective symmetric operator is self-adjoint.
\end{corollary}
\begin{proof}
Take $z=0$ in point (1).
\end{proof}

\begin{proposition}
Let $A$ be a closed symmetric operator. Then one of the following cases holds:
\begin{itemize}
\item $A$ is self-adjoint, in which case $\spec(A) \subseteq \R$;
\item $\spec(A) = \overline{\C^{\uparrow}}$;
\item $\spec(A) = \overline{\C^{\downarrow}}$;
\item $\spec(A) = \C$.
\end{itemize}
If $A$ is not densely-defined, then the last case holds.
\end{proposition}
We have denoted the closed upper half plane $\overline{\C^{\uparrow}}$ and the closed lower half plane $\overline{\C^{\downarrow}}$.
\begin{proof}
First assume $A$ self-adjoint, then $\spec(A)\subseteq \R$ by a combination of \ref{approximateSpectrumSymmetricOperator} and \ref{rangeSelfAdjointCriterion}.

Now note that if there exists a real $\lambda\in\R$ such that $\lambda \in \res(A)$, then in particular $\lambda\id -A$ is surjective, so $A$ is self-adjoint by \ref{rangeSelfAdjointCriterion}.

Now assume $A$ not self-adjoint and pick a $\lambda\in \C^{\uparrow}$. From \ref{rangeSelfAdjointCriterion} we must have either $\lambda\in\spec(A)$ or $\overline{\lambda}\in\spec(A)$ (or both).

If $\lambda\in \res(A)$, then $\C^\uparrow \subseteq \res(A)$ and if $\overline{\lambda}\in\res(A)$, then $\C^\downarrow \subseteq \res(A)$.

Indeed take some $\mu\in\C$.
By \ref{symmetricResolvent} we only need to check surjectivity of $\mu\id - A$. Now note that
\begin{align*}
(\mu\id - A)R_A(\lambda) &= (\mu\id -\lambda\id+\lambda\id - A)R_A(\lambda) \\
&= (\mu-\lambda)R_A(\lambda) + (\lambda\id-A)R_A(\lambda) \\
&= (\mu-\lambda)R_A(\lambda) + \id,
\end{align*}
which we can consider as a bounded perturbation of $\id$. Thus by \ref{boundedPerturbationClosedOperator}, $(\mu\id - A)R_A(\lambda)$ is surjective if $\norm{(\mu-\lambda)R_A(\lambda)}< 1$, i.e.\ $|\mu-\lambda| < |\Im(\lambda)|$ using \ref{symmetricResolvent}.

We can iterate this construction to cover the whole of $\C^\uparrow$. The argument for $\overline{\lambda}$ is similar.
\end{proof}

\begin{example}
Spectrum half plane TODO \url{https://math.stackexchange.com/questions/893899/spectrum-of-symmetric-non-selfadjoint-operator-on-hilbert-space}

\url{https://math.stackexchange.com/questions/925097/spectrum-of-self-adjoint-operator-on-hilbert-space-real}
\end{example}

\begin{proposition}
Let $A$ be a closed symmetric operator on a Hilbert space. Then $A$ is positive \textup{if and only if} $\spec(A)\subseteq [0,\infty[$.
\end{proposition}
\begin{proof}
\ref{closureNumericRangeConvexHullSpectrum}
\end{proof}

\begin{proposition}
Let $A$ be a self-adjoint operator. Then
\begin{enumerate}
\item $\inf \sigma(A) = \inf\NumRange(A)$;
\item $\sup \sigma(A) = \sup\NumRange(A)$.
\end{enumerate}
\end{proposition}
\begin{proof}
\ref{closureNumericRangeConvexHullSpectrum}
\end{proof}

\begin{proposition}
Let $T$ be a densely defined self-adjoint operator. Then
\begin{enumerate}
\item $\rspec(T) = \emptyset$;
\item let $\lambda_1,\lambda_2 \in \pspec(T)$ and $\lambda_1\neq \lambda_2$, then 
\[ \Null(\lambda_1\id - T) \perp \Null(\lambda_2 \id - T). \]
\end{enumerate}
\end{proposition}
\begin{proof}
TODO
\end{proof}


\begin{proposition}
Let $T$ be a symmetric operator on a Hilbert space $H$. Then
\begin{enumerate}
\item the eigenvalues of $T$ are real;
\item the eigenvectors corresponding to distinct eigenvalues are orthogonal.
\end{enumerate}
\end{proposition}
\begin{proof}
This is an application of \ref{eigenspaceOrthogonalAdjoint} and \ref{adjointSpectrumNoResidual}.
\end{proof}

\subsubsection{Compact self-adjoint operators}
\begin{proposition}
Every compact self-adjoint operator $L$ on a nontrivial Hilbert space has an eigenvalue $\lambda$ with $|\lambda| = \norm{L}$.
\end{proposition}

\begin{proposition}
Let $A$ be a compact self-adjoint operator. Then the only possible accumulation point of $\spec(A)$ is $0$.
\end{proposition}
TODO self-adjoint not necessary? See \ref{spectrumCompactOperator}?
\begin{proof}
Assume $\spec(A)$ is infinite. Then take $\seq{\lambda_n}\subset \spec(A)$. Any associated sequence $\seq{x_n}$ of eigenvectors is orthogonal. We can take it to be orthonormal. By \ref{limitCompactImageOrthonormalSequence} we have
\[ 0 = \lim_{n\to\infty} \norm{Ax_n}^2 = \lim_{n\to\infty}\inner{Ax_n,Ax_n} = \lim_{n\to\infty}\lambda_n^2\inner{x_n,x_n} = \lim_{n\to\infty}\lambda_n^2, \]
so $\seq{\lambda_n}$ converges to $0$.
\end{proof}

\begin{theorem}
Every spectral value $\lambda\neq 0$ of a compact self-adjoint linear
operator $A : H \to H$ is an eigenvalue of finite multiplicity that can only
accumulate at $\lambda = 0$. Conversely, a self-adjoint operator having these
properties is compact.
\end{theorem}
\begin{proof}
TODO See \ref{spectrumCompactOperator}
\end{proof}

\subsubsection{Self-adjoint extensions of symmetric operators}
\paragraph{Cayley transform}
Consider the Möbius transform
\[ \C\setminus\{\overline{\lambda}\} \to \C: x\mapsto \frac{x - \lambda}{x-\overline{\lambda}} \qquad \text{for some $\lambda\in\C\setminus\R$.} \]
This transform maps
\begin{itemize}
\item the real line to $\T\setminus\{1\}$;
\item the half-plane above / below the real line containing $\lambda$ to the interior of the unit disk;
\item the half plane containing $\overline{\lambda}$ to the exterior of the unit disk;
\item in particular $\lambda \mapsto 0$ and $\overline{\lambda} \mapsto \infty$.
\end{itemize}
Conventional choice: $\lambda = i$.

\paragraph{Defect indices}
Or deficiency(?)
\url{https://link-springer-com.ezproxy.ulb.ac.be/content/pdf/10.1007/978-94-007-4753-1.pdf}

Cfr. dilation theory through Cayley transform.

\subsubsection{Positive self-adjoint extensions of symmetric operators}
\begin{theorem}[Friedrich's extension]
Let $A$ be a positive symmetric operator on a Hilbert space $H$. Then $A$ has a unique positive self-adjoint extension $\widetilde{A}$ with domain $\dom(\widetilde{A}) \subseteq \Closure_{\norm{\cdot}_{A+\id}}(\dom(A))$.
\end{theorem}
\begin{proof}
Set $H_A \defeq \Closure_{\norm{\cdot}_{A+\id}}(\dom(A))$. By \ref{energyNormTopology}, we have $H_A \subseteq \Closure_{\norm{\cdot}}(\dom(A))$.

For \undline{existence}, we can construct the operator $\widetilde{A}$ as follows:
\begin{align*}
\dom(\widetilde{A}) &\defeq \setbuilder{x\in H_A}{\exists x'\in H:\forall y\in H_A:\; \inner{y,x}_{A+\id} = \inner{y,x'}} \\
\widetilde{A}x &\defeq x' - x.
\end{align*}
Now $\widetilde{A}$ is an extension of $A$, because for all $x\in \dom(A)$, we can take $x' = Ax + x$. So $\widetilde{A}x = Ax$.

But $\dom(\widetilde{A})$ may be larger than $\dom(A)$, because we can extended $\inner{y,x}_{A+\id}$ to be defined on all of $H_A$ by continuity.

By construction $\dom(\widetilde{A}) \subseteq \Closure_{\norm{\cdot}_{A+\id}}(\dom(A))$.

Now we claim $\im(\widetilde{A} + \id) = H$. Indeed for any $x'\in H$, the functional $H_A \to H_A: y\mapsto \inner{y,x'}$ is bounded. By Riesz representiation \ref{rieszRepresentation}, we can find an $x\in H_A$ such that $\inner{y,x}_{A+\id} = \inner{y,x'}$. Thus $(\widetilde{A} + \id)x = x'$.

By \ref{rangeSelfAdjointCriterion} we conclude that $\widetilde{A}$ is self-adjoint. 

For \undline{uniqueness}, assume there exists a second such extension $\widehat{A}$. For all $y\in \dom(A)$ and $x\in \dom(\widehat{A})$, we have
\[ \inner{y, (\widehat{A}+\id)x} = \inner{(\widehat{A}+\id)y, x} = \inner{(A+\id)y, x} = \overline{\inner{x, (A+\id)y}} = \overline{\inner{x, y}_{A+\id}} = \inner{y, x}_{A+\id}. \]
By continuity this holds for all $y\in H_A$. And thus by definition $\widetilde{A}x = \widehat{A}x$ for all $x\in\dom(\widetilde{A})$. Thus $\widetilde{A} \subseteq \widehat{A}$, but self-adjoint operators are maximal by \ref{selfAdjointMaximal}, so $\widetilde{A} = \widehat{A}$.
\end{proof}

\subsubsection{Bounded self-adjoint operators}
\begin{lemma}
Let $A, B\in\Bounded(H)$. Then
\begin{enumerate}
\item $A^*A, AA^*$ and $A+A^*$ are self-adjoint;
\item if $A,B$ are self-adjoint, then $AB$ is self-adjoint \textup{if and only if} $A,B$ commute.
\end{enumerate}
\end{lemma}
\begin{corollary}
Let $A\in\Bounded(H)$. Then there exist unique self-adjoint operators $S,T$ such that
\[ A = S+iT \qquad A^* = S-iT. \]
\end{corollary}
\begin{proof}
Indeed $S = (A+A^*)/2$ and $T = (A-A^*)/2i$ are self-adjoint.
\end{proof}
\begin{corollary}
The operator $A$ is normal \textup{if and only if} $S,T$ commute.
\end{corollary}
\begin{proof}
We calculate the commutator
\[ [S,T] = \left[\frac{A+A^*}{2}, \frac{A-A^*}{2i}\right] = \frac{A^*A - AA^*}{2i} = \frac{1}{2i}[A^*, A]. \]
\end{proof}

\begin{proposition}
The set of bounded self-adjoint operators forms an anti-lattice.
\end{proposition}
\begin{proof}
TODO + generalised to self-adjoint operators??
\end{proof}

\subsection{Orthogonal projections}
\url{https://planetmath.org/latticeofprojections}

\url{https://zfn.mpdl.mpg.de/data/Reihe_A/35/ZNA-1980-35a-0437.pdf}

We denote the set op projections on a Hilberts space $\mathcal{H}$ by $\Projections(\mathcal{H})$.

TODO: $\im(P) = \ker{P^*}^\perp$ shows that we need $P= P^*$ for orthogonality.

\begin{proposition}
Let $P$ be a bounded operator $P$ on a Hilbert space $\mathcal{H}$. Then the following are equivalent:
\begin{enumerate}
\item $P$ is an orthogonal projection onto a closed subspace of $\mathcal{H}$;
\item $P^2 = P$ and $P=P^*$;
\item $P^2 = P$ and $\norm{P}\in \{0,1\}$;
\item $P^2 = P$ and $\norm{P}\leq 1$;
\end{enumerate}
\end{proposition}
\begin{proof}
$\boxed{(1)\Rightarrow (2)}$  Suppose first that $P$ is the orthogonal projection operator onto a closed subspace $K$. Clearly $P^2 = P$. Let $x,y\in\mathcal{H}$ and write $x= x_1+x_2, y = y_1+y_2$ where $x_1,y_1\in K$ and $x_2,y_2\in K^\perp$. Then
\[ \inner{Px, y} = \inner{x_1, y_1+y_2} = \inner{x_1, y_1} + \inner{x_1,y_2} = \inner{x_1,y_1} = \inner{x_1+x_2, y_2} = \inner{x,Py}. \]
So $P = P^*$.

$\boxed{(2)\Rightarrow (3)}$ We calculate $\norm{P} = \norm{P^2} = \norm{P^*P} = \norm{P}^2$ using \ref{normOfSquare}. The solutions to this equation are $\{0,1\}$.

$\boxed{(3)\Rightarrow (4)}$ This is clear.

$\boxed{(4)\Rightarrow (1)}$ Define $K=\im P$, then $K$ is closed because $x\in K$ iff $Px=x$ and thus for any converging sequence $(x_n)_n\subset K$: $\lim x_n = \lim Px_n = P\left(\lim x_n\right)$, so the limit is in $K$.

We just need to show orthogonality: $Px \perp x- Px$. For this we use \ref{orthogonality}: for all $a\in\F$
\[ \norm{Px} = \norm{Px + aPx - aPx} = \norm{P(Px + a(x-Px))} \leq \norm{P}\cdot \norm{Px + a(x-Px)} \leq \norm{Px + a(x-Px)}. \]
We conclude $Px \perp x- Px$.
\end{proof}

\begin{proposition} \label{projectorOrthogonalComplement}
Let $\mathcal{H}$ be a Hilbert space and let $P$ be an orthogonal projector on a closed subspace $K$. Then $\id-P$ is the orthogonal projector on $K^\perp$.
\end{proposition}
\begin{proof}
Any $x\in \mathcal{H}$ can be uniquely decomposed as $x_1 + x_2\in K\oplus K^\perp$. If $Px = x_1$, then $(\id - P)x = x_1 +x_2 - x_1 = x_2$.
\end{proof}
\begin{corollary} \label{projectorsIn01}
The set of projectors $\Projections(\mathcal{H})$ is a subset of $[0,\id]$.
\end{corollary}
\begin{proof}
Let $P\in\Projections(\mathcal{H})$. Then $P\geq 0$ follows from $P = P^2 = P^*P$.
\end{proof}

\begin{proposition} \label{commutingProjectors}
Let $\mathcal{H}$ be a Hilbert space and $P,Q$ be projections. The following are equivalent:
\begin{enumerate}
\item $PQ = QP$;
\item $PQ$ is a projection;
\item $QP$ is a projection;
\item $P+Q-PQ$ is a projection;
\item $\im(PQP) = \im(P) \cap \im(Q)$;
\item $PQP = QP$;
\item $\mathcal{H} = \big(\im(P)\cap\im(Q)\big)\oplus \big(\im(P)\cap\im(Q)^\perp\big) \oplus \big(\im(P)^\perp\cap\im(Q)\big) \oplus \big(\im(P)^\perp\cap\im(Q)^\perp\big)$.
\end{enumerate}
\end{proposition}
\begin{proof}
Points (1), (2), (3) are equivalent by the equation $(PQ)^* = Q^*P^* = QP$, and the fact that (1) implies $(PQ)^2 = PQPQ = PPQQ = PQ$.

(4) If $P,Q$ commute, then
\begin{align*}
(P+Q-PQ)^* &= P+Q-(PQ)^* = P+Q-Q^*P^* =P+Q-QP = P+Q-PQ \\
(P+Q-PQ)^2 &= P^2 + PQ -P^2Q + QP+Q^2 - QPQ - PQP -PQP +PQPQ \\
&= P + Q + 3PQ - 4PQ= P+Q-PQ.
\end{align*}
Assume (4), then $(P+Q-PQ)^* = P+Q-QP = P+Q-PQ$. This implies $PQ=QP$.

$\boxed{(1)\Rightarrow (5)}$ Clearly $\im(PQP) \subseteq \im(P) \cap \im(Q)$.
For the inverse inequality, take $x\in im(P)\cap\im(Q)$. Then $PQP(x) = PQ(x) = P(x) = x$, so $x\in\im(PQP)$.

$\boxed{(5)\Rightarrow (6)}$ We decompose $\mathcal{H} = \im(PQP) \oplus \ker(PQP)$ and show that the operators are the same on both parts. For all $x\in \mathcal{H}$ we have
\[ x\in \ker(PQP) \iff \inner{x,PQPx} = 0 \iff \inner{QPx,QPx} = 0 \iff \norm{QPx} = 0 \iff x\in\ker{QP}.  \]
Now let $x\in\im(PQP) = \im(P)\cap\im(Q)$. Then $QPx = Qx = x = PQPx$.

$\boxed{(6)\Rightarrow (3)}$ $PQP$ is always a projection.

$\boxed{(6)\Rightarrow (7)}$ Take some $x\in \mathcal{H}$. Then we can uniquely decompose $x = P(x) + (x-P(x)) = x_P + x_{P^\perp} \in \im(P)\oplus \im(P)^\perp$. We can then further decompose $x_P = x_{P,Q} + x_{P,Q^\perp}$ and $x_{P^\perp} = x_{P^\perp, Q} + x_{P^\perp, Q^\perp}$. In order to have the decomposition of the proposition, we need to show that $x_{P,Q},x_{P,Q^\perp}\in \im(P)$ and $x_{P^\perp, Q},x_{P^\perp, Q^\perp}\in\im(P)^\perp$.

First take $x_{P,Q} = QPx$. From (6) we have $P(QPx) = PQPx = QPx$, so $x_{P,Q}\in \im(P)$. For the others we have similar calculations (also using the identity $PQP = PQ$):
\begin{align*}
P(x_{P,Q^\perp}) &= P\big((\id-Q)P\big)x = Px - PQPx = Px - QPx = (\id-Q)Px = x_{P,Q^\perp} \\
(\id-P)(x_{P^\perp,Q}) &= (\id-P)\big(Q(\id-P)\big)x = (Q-QP-PQ+PQP)x = (Q-QP)x = Q(\id-P)x = x_{P^\perp,Q} \\
(\id-P)(x_{P^\perp,Q^\perp}) &= (\id-P)\big((\id-Q)(\id-P)\big)x = (\id-P-Q+QP-P+P+PQ-PQP)x \\
&= (\id-Q-P+QP)x = (\id-Q)(\id-P)x = x_{P^\perp,Q^\perp}.
\end{align*}
$\boxed{(7)\Rightarrow (1)}$ Take $x\in \mathcal{H}$ and decompose it as $x_{P,Q} + x_{P,Q^\perp} + x_{P^\perp, Q} + x_{P^\perp, Q^\perp}$. Then $PQx = P(x_{P,Q} + x_{P^\perp, Q}) = x_{P,Q}$ and $QP = Q(x_{P,Q} + x_{P, Q^\perp}) = x_{P,Q}$, so $PQ = QP$. 
\end{proof}

\begin{proposition} \label{perpendicularProjections} \label{subspaceProjections}
Let $P,Q$ be orthogonal projections onto subspaces $\im(P)$ and $\im(Q)$ of $\mathcal{H}$.
\begin{enumerate}
\item The following are equivalent to $\im(P) \perp \im(Q)$:
\begin{enumerate}
\item $QP = 0$;
\item $PQ = 0$;
\item $Q+P$ is an orthogonal projection.
\end{enumerate}
\item The following are equivalent to $\im(P) \subseteq \im(Q)$:
\begin{enumerate}
\item $QP = P$;
\item $PQ = P$;
\item $Q-P$ is an orthogonal projection;
\item $P\leq Q$;
\item $\norm{Px} \leq \norm{Qx}$ for all $x \in \mathcal{H}$.
\end{enumerate}
\end{enumerate}
\end{proposition}
\begin{proof}
(1) We have:

$\boxed{(a)\Leftrightarrow (b) \Leftrightarrow \im(P) \perp \im(Q)}$ By \ref{commutingProjectors}.

$\boxed{(a, b)\Leftrightarrow (c)}$ We know $(P+Q)^* = P^*+Q^* =P+Q$ and we can write
\[ (P+Q)^2 = P^2 + Q^2 + PQ + QP = P+Q+ PQ+QP,  \]
So clearly (a) or (b) imply (c). Conversely, assume $PQ + QP = 0$, implying $PQ=-QP$. By left- and right-multiplication by $P$ this implies both
\[ PPQ = PQ = -PQP \qquad \text{and} \qquad PQP = -QPP = -QP. \]
So $PQ = -PQP = QP$, meaning $PQ = 1/2(PQ+QP) = 0$.

(2) We prove the following:

$\boxed{(a)\Leftrightarrow (b) \Leftrightarrow \im(P) \subseteq \im(Q)}$ By \ref{commutingProjectors}.

$\boxed{(a,b)\Rightarrow (c)}$ Obviously $(Q-P)^*= Q-P$. Also
\[ (Q-P)^2 = Q+P-PQ-QP= Q+P-2P = Q-P. \]

$\boxed{(c)\Rightarrow (a,b)}$ Now from
\[ Q-P = (Q-P)^2 = Q+P-PQ-QP \]
we obtain $2P = PQ+QP$. The result then follows if we can show that $PQ=QP$. This follows by multiplying the equality on the left and on the right by $P$ to obtain $QP = 2P-PQP$ and $PQ = 2P-PQP$, respectively. 

$\boxed{(c)\Rightarrow (d)}$ This follows because all projections are positive.

$\boxed{(d)\Rightarrow (a, b)}$ Assume, towards a contradiction, that $\im(P)\nsubseteq \im(Q)$. Then we can take $v\in\im(P)\setminus \im(Q)$. Then
\[ \inner{v,(Q-P)v} = \inner{v,Qv} - \inner{v,v} = \inner{Qv,Qv} - \inner{Qv,Qv} - \inner{v-Qv, v-Qv} = -\norm{v-Qv}^2. \]
Because $v\notin \im(Q)$, $\norm{v-Qv}$ is not zero and thus $Q-P$ is not positive.

$\boxed{(d)\Leftrightarrow (e)}$ By the equivalence
\[ \norm{Px} \leq \norm{Qx} \iff \inner{Px,Px} \leq \inner{Qx,Qx} \iff \inner{Px,x}\leq \inner{Qx,x} \iff \inner{(Q-P)x,x}\geq 0. \]
\end{proof}

We can generalise part 2(d) of the previous proposition to a slightly larger class of operators.
\begin{lemma} \label{comparisonSelfAdjointProjection}
Let $P\in \Projections(\mathcal{H})$ and $T \in [0,\id]$, then the following are equivalent:
\begin{enumerate}
\item $\im(T) \subseteq \im(P)$;
\item $T\leq P$.
\end{enumerate}
\end{lemma}
\begin{proof}
As $T$ is self-adjoint, we have $\norm{T} = \nr(T) \leq 1$ by \ref{normNumRadius}.

Assume (1) so that for all $x\in \mathcal{H}$ we get
\[ \inner{x,Tx} = \inner{x, PTx} = \inner{Px,PTx} \leq \norm{Px}^2\nr(T) \leq \norm{Px}^2 = \inner{Px,Px} = \inner{x,Px}. \]
So $\inner{x, (P-T)x}\geq 0$ and thus $T\leq P$.

Assume (2). The energy form associated with $T$ is a pre-inner product by \ref{positiveOperatorPositiveEnergyForm}. The Cauchy-Schwarz inequality \ref{CauchySchwarz} gives
\[ |\inner{v,Tw}|^2 \leq \inner{v,Tv}\inner{w,Tw} \leq \inner{v,Pv}\inner{w,Pw}. \]
So if $v\in\im(P)^\perp$, then $\inner{v,Tw} = 0$ for all $w\in \mathcal{H}$. So $\im(T)\perp \im(P)^\perp$, implying $\im(T)\subseteq \im(P)^{\perp\perp} = \im(P)$.
\end{proof}

\begin{proposition}
Let $\mathcal{H}$ be a Hilbert space. Let $\{P_i\}_{i\in I}$ be an arbitrary subset of $\Projections(\mathcal{H})$ and let $K_i = \im(P_i)$ for all $i\in I$. Then, as a subset of $[0,\id]$,
\begin{enumerate}
\item $\inf \{P_i\}_{i\in I} = P_M$ where $M = \bigcap_{i\in I}K_i$;
\item $\sup \{P_i\}_{i\in I} = P_N$ where $N = \bigcap\setbuilder{K \subseteq \mathcal{H}}{\text{$K$ is closed} \land \forall i\in I: K_i \subseteq K}$.
\end{enumerate}
The set of projections on $\mathcal{H}$ is thus a complete lattice as a subset of $[0,\id]$.

If $I$ is finite, then $N = \Span(\bigcup_{i\in I}K_i)$. TODO: always closure of this $N$????
\end{proposition}
In particular this means $\Projections(\mathcal{H})$ is a complete lattice as itself, with the same suprema and infima. It is not a lattice as a subset of $\SelfAdjoints(\mathcal{H})$ (TODO + example ??).
\begin{proof}
(1) By \ref{subspaceProjections} $P_M$ is a lower bound of $\{P_i\}_{i\in I}$ in $[0,\id]$. Let $T$ be a lower bound of $\{P_i\}_{i\in I}$ in $[0,\id]$. By \ref{comparisonSelfAdjointProjection} $\im(T)\subseteq K_i$ for all $i\in I$, so $\im(T)\subseteq M$ and thus $T\leq P$ again by \ref{comparisonSelfAdjointProjection}. This means $P$ is the greatest lower bound.

(2) The mapping $T\mapsto \id-T$ keeps $[0,\id]$ invariant and inverts the order. Then $\inf \{\id - P_i\}_{i\in I}$ is a projection due to the previous point and so $\sup \{P_i\}_{i\in I}$ is also a projection. The expression for $N$ is clear from \ref{subspaceProjections}.
\end{proof}

\subsubsection{Sets of pairwise disjoint projections}
TODO!

\subsubsection{Derivatives of orthogonal projections}



\begin{proposition}
Let $\{P_i\}_{i\in I}$ be a set of pairwise disjoint orthogonal projectors which have derivatives and take $i\neq j$ in $I$. Then
\begin{enumerate}
\item $P_i'P_j = - P_iP_j'$;
\item if $\id \in \upset \{P_i\}_{i\in I}$, then
\[ P_iP_i' = \sum_{j\neq i}P'_iP_j \qquad\text{and}\qquad P_i'P_i = \sum_{j\neq i}P_jP_i'. \]
\end{enumerate}
\end{proposition}
\begin{proof}
(1) We have $P_iP_j = 0$, so $0 = P_i'P_j + P_iP_j'$.

(2) We calculate, using $\id = \sum_{j\in I}P_j$ and \ref{derivativeIdempotent}:
\[ P_iP_i' = P_iP_i'\left(\sum_{j\in I}P_j\right) = P_iP_i'P_i + \sum_{j\neq i}P_iP_i'P_j = 0 - \sum_{j\neq i}P_iP_iP_j' = -\sum_{j\neq i}P_iP_j' = \sum_{j\neq i}P_i'P_j. \]
\end{proof}
\begin{corollary}
Let $P_1, P_2$ be orthogonal projections such that $P_1 + P_2 = \id$. Then
\[ P_1P_1'= P_1'P_2 \qquad \text{and}\qquad P_1'P_1 = P_2P_1'. \]
\end{corollary}


\subsection{Isometries}
We recall that isometries are injective and continuous. On Hilbert spaces they are also closed. See \ref{isometryInjective}, \ref{isometryContinuous} and \ref{isometryClosed}.

\begin{proposition} \label{isometryCharacterisation}
Let $T\in \Bounded(H,K)$ with $H,K$ Hilbert spaces. Then
\begin{enumerate}
\item $T$ is an isometry \textup{if and only if} $T^*T = \id_H$;
\item $T$ is unitary \textup{if and only if} $T^*T = \id_H$ and $TT^* = \id_K$, i.e.\ $T^{-1} = T^*$.
\end{enumerate}
\end{proposition}
\begin{proof}
(1) For all $v,w\in H$ we have
\[ \inner{Tv,Tw} = \inner{T^*Tv,w}. \]
The left-hand side is equal to $\inner{v,w}$ iff $T$ is an isometry. The right-hand side is equal to $\inner{v,w}$ iff $T^*T = \id_H$, by \ref{equalityOfMapsInnerProductSpaces}.

(2) If $T$ is invertible, it must have a left and right inverse. By lemma \ref{leftRightInverse} they must be the same.
\end{proof}
\begin{corollary}
An isometry $T\in\Bounded(H)$ is unitary \textup{if and only if} it is normal.
\end{corollary}

\begin{lemma} \label{isometryRangeProjection}
Let $T$ be an isometry between Hilbert spaces $H$ and $K$. Then $TT^*$ is an orthogonal projection.
\end{lemma}
\begin{proof}
Clearly $(TT^*)^* = TT^*$. Also $(TT^*)^2 = T(T^*T)T^* = T\id_HT^* = TT^*$.
\end{proof}


\subsubsection{Wandering spaces and unilateral shifts}
\begin{definition}
Let $\mathcal{H}$ be a Hilbert space, $\mathcal{V}\subseteq \mathcal{H}$ a closed subspace and $T:\mathcal{H}\to \mathcal{H}$ a linear map. Then $\mathcal{V}$ is called a \udef{wandering space} for $T$ if $T^p[\mathcal{V}]\perp T^q[\mathcal{V}]$ for every $p\neq q\in\N$.
\end{definition}

\begin{lemma} \label{WoldLemma1}
Let $\mathcal{H}$ be a Hilbert space, $\mathcal{V}\subseteq \mathcal{H}$ a closed subspace and $T:\mathcal{H}\to \mathcal{H}$ a linear isometry.
\begin{enumerate}
\item $\mathcal{V}$ is a wandering space for $T$ \textup{if and only if} $T^n[\mathcal{V}]\perp \mathcal{V}$ for all $n\in\N$;
\item $T[\mathcal{H}]^\perp$ is a wandering subspace for $T$;
\item if $\mathcal{V}$ is a wandering space for $T$, then $T^n[\mathcal{V}] \cong \mathcal{V}$ for all $n\in N$.
\end{enumerate}
\end{lemma}
\begin{proof}
(1) The direction $\Rightarrow$ is clear. For the converse, assume $T^n[\mathcal{V}]\perp \mathcal{V}$ for all $n\in\N$. We need to show that $T^p[\mathcal{V}]\perp T^q[\mathcal{V}]$ for every $p\neq q\in\N$. WLOG we may assume $p\leq q$. Take arbitrary $x\in T^p[\mathcal{V}]$ and $y\in T^q[\mathcal{V}]$. Then
\[ \inner{x,y} = \inner{T^p(u), T^q(v)} = \inner{u, T^{q-p}(v)} = 0 \]
because $\mathcal{V} \perp T^{q-p}[\mathcal{V}]$.

(2) For all $n\geq 1$ we have
\[ T^{n}\big[T[\mathcal{H}]^\perp\big] \subset T^{n}[\mathcal{H}] = T\big[T^{n-1}[\mathcal{H}]\big] \subset T[\mathcal{H}] \perp T[\mathcal{H}]^\perp. \]

(3) For all $n\in \N$ the operator $T^n$ is an isometry. It is injective by \ref{isometryInjective}, and thus maps its domain bijectively to its image.
\end{proof}

\begin{definition}
An isometry $T$ on a Hilbert space $\mathcal{H}$ is called a \udef{unilateral shift} if there is a closed subspace $\mathcal{V}\subseteq \mathcal{H}$ that is wandering for $T$ such that
\[ \mathcal{H} = \bigoplus_{n=0}^\infty T^n[\mathcal{V}]. \]
We call the subspace $\mathcal{V}$ \udef{generating} for $T$ and $\dim(\mathcal{V})$ the \udef{multiplicity} of $T$.
\end{definition}

By \ref{WoldLemma1}, we see that any isometry $T:\mathcal{H}\to\mathcal{H}$ is a unilateral shift when restricted to $\bigoplus_{n=0}^\infty T^n\big[T[\mathcal{H}]^\perp\big]$.



\begin{lemma} \label{WoldLemma2}
Let $T$ be an isometry on $\mathcal{H}$. If $T$ is a unilateral shift, then it is generated by $T[\mathcal{H}]^\perp$.
\end{lemma}
\begin{proof}
Let $\mathcal{V}$ be the generating subspace of the unilateral shift $T$. We calculate
\[ T[\mathcal{H}] = T\left[\bigoplus_{n=0}^\infty T^n[\mathcal{V}]\right] = \bigoplus_{n=1}^\infty T^n[\mathcal{V}] = \bigoplus_{n=0}^\infty T^n[\mathcal{V}] \ominus \mathcal{V} = \mathcal{H}\ominus \mathcal{V} = \mathcal{V}^\perp, \]
so $\mathcal{V} = T[\mathcal{H}]^\perp$.
\end{proof}

A unilateral shift is determined up to unitary equivalence by its multiplicity:
\begin{lemma}
Let $T: \mathcal{H}\to\mathcal{H}$ and $T':\mathcal{H}'\to\mathcal{H}'$ be unilateral shifts generated by $\mathcal{V}$ and $\mathcal{V}'$ such that $\dim(\mathcal{V}) = \dim(\mathcal{V}')$. Then there exists an unitary $U:\mathcal{H}'\to\mathcal{H}$ such that
\[ T' = U^*TU \]
\end{lemma}
\begin{proof}
Choose an isometric isomorphism $u:\mathcal{V}'\to\mathcal{V}$. Then any $x\in\mathcal{H}'$ can be written as $x = \sum_{n=0}^\infty T^n(x_n)$. Then define
\[ Ux = \sum_{n=0}^\infty T^n(ux_n). \]
\end{proof}

\begin{theorem}[Wold decomposition]
Let $\mathcal{H}$ be a Hilbert space and $T\in\Bounded(\mathcal{H})$ an isometry. Then $\mathcal{H}$ decomposes into an orthogonal sum $\mathcal{H} = \mathcal{H}_0\oplus \mathcal{H}_1$such that $\mathcal{H}_0, \mathcal{H}_1$ reduce $T$ and
\[ T|_{\mathcal{H}_0}\;\text{is unitary} \quad\text{and}\quad T|_{\mathcal{H}_1}\;\text{is a unilateral shift}. \]
This decomposition is uniquely determined and given by
\[ \mathcal{H}_0 = \bigcap_{n=0}^\infty T^n[\mathcal{H}] \qquad\text{and}\qquad \mathcal{H}_1 = \bigoplus_{n=0}^\infty T^n[\mathcal{V}] \qquad\text{where}\qquad \mathcal{V} = T[\mathcal{H}]^\perp. \]
\end{theorem}
\begin{proof}
The subspace $\mathcal{V} = T[\mathcal{H}]^\perp$ is wandering by \ref{WoldLemma1}. Then $T$ is a unilateral shift in the subspace
\[ \mathcal{H}_1 = \bigoplus_{n=0}^\infty T^n[\mathcal{V}]. \]
Now $v\in\mathcal{H}_0 = \mathcal{H}_1^\perp$ if and only if it is perpendicular to $\bigoplus_{i=0}^n T^i[\mathcal{V}]$ for all $n$ and we have
\begin{align*}
\bigoplus_{i=0}^n T^i[\mathcal{V}] &= \bigoplus_{i=0}^n T^i[\mathcal{H}\ominus T[\mathcal{H}]] = \bigoplus_{i=0}^n T^i[\mathcal{H}]\ominus T^{i+1}[\mathcal{H}] \\
&= (\mathcal{H}\ominus T[\mathcal{H}])\oplus(T[\mathcal{H}]\ominus T^2[\mathcal{H}])\oplus \ldots \oplus (T^n[\mathcal{H}]\ominus T^{n+1}[\mathcal{H}])  = \mathcal{H} \ominus T^{n+1}[\mathcal{H}] 
\end{align*}
using \ref{perpUnderIsometry} and \ref{cancellationOminus}, which is applicable because $T^i[\mathcal{V}]$ is closed by \ref{isometryClosed}. So $\mathcal{H}_0\subseteq T^n[\mathcal{H}]$ for all $n$.

Finally $T|_{\mathcal{H}_0}$ is unitary because it is an isometry and surjective on $\mathcal{H}_0$.
\end{proof}

\subsubsection{Left and right shifts on $\ell^2$}
\begin{definition}
Consider the space $\ell^2(\N)$ with o.n. basis $\seq{e_i}$. Then
\begin{itemize}
\item the \udef{right shift operator} $S_r$ is the operator that maps $e_i \mapsto e_{i+1}$;
\item the \udef{left shift operator} $S_l$ is the operator that maps $e_i \mapsto \begin{cases}
e_{i-1} & i \geq 1 \\ 0 & i = 0
\end{cases}$.
\end{itemize}
\end{definition}

\begin{lemma}
$S_r$ is a unilateral shift
\end{lemma}

\begin{proposition}
$S_r = S^*_l$ (also converse?)
\end{proposition}

\subsubsection{Partial isometries}
\begin{definition}
An operator $T\in \Lin(H, H')$ is called a \udef{partial isometry} if there is a closed subspace $K\subseteq H$ such that
\begin{itemize}
\item $T|_K$ is an isometry;
\item $T|_{K^\perp} = 0$.
\end{itemize}
\end{definition}

Clearly every partial isometry is bounded.

\begin{lemma}
An operator $T\in \Lin(H, H')$ is a partial isometry \textup{if and only if} $T|_{\ker(T)^\perp}$ is an isometry.
\end{lemma}

\begin{proposition} \label{partialIsometryEquivalences}
Let $T\in \Bounded(H,H')$. The following are equivalent:
\begin{enumerate}
\item $T$ is a partial isometry;
\item $T^*TT^* = T^*$;
\item $TT^*T = T$;
\item $TT^*: H' \to H'$ is a projection;
\item $T^*T: H \to H$ is a projection;
\item $T^*$ is a partial isometry.
\end{enumerate}
Moreover,
\begin{enumerate}
\item $T^*T$ is the projection onto $\ker(T)^\perp$;
\item $\im(T)$ is closed and $TT^*$ is the projection onto $\im(T)$.
\end{enumerate}
\end{proposition}
\begin{proof}

$\boxed{(1)\Rightarrow (2)}$ By \ref{elementaryOrthogonality} it is enough to show that $\inner{T^*TT^*x,y} = \inner{T^*x,y}$ for all $x\in H', y\in H$. Take such $x,y$. We decompose $y = y_1\oplus y_2 \ker(T)\oplus \ker(T)^\perp$. Then
\[ \inner{T^*TT^*x, y_1} = \inner{TT^*x, Ty} = 0 = \inner{x,Ty_1} = \inner{T^*x, y_1} \]
and
\[ \inner{T^*TT^*x, y_2} = \inner{TT^*x, Ty_2} = \inner{T^*x,y_2}, \]
where we have used the fact that both $y_2$ and $T^*x$ are elements of $\ker(T)^\perp = \overline{\im(T^*)}$, and $T$ is an isometry on this space. In conclusion, we have
\[ \inner{T^*TT^*x,y} = \inner{T^*TT^*x,y_1} + \inner{T^*TT^*x,y_2} = \inner{T^*x,y_1} + \inner{T^*x,y_2} = \inner{T^*x,y} \]
for all $x\in H', y\in H$, so $T^*TT^* = T^*$.

$\boxed{(2) \Leftrightarrow (3)}$ By taking adjoints: $(TT^*T)^* = T^*TT^*$.

$\boxed{(2) \Rightarrow (4,5)}$ Clearly $T^*T$ and $TT^*$ are self-adjoint. We just need to show idempotency:
\[ (T^*T)^2 = (T^*T)(T^*T) = (T^*TT^*)T = T^*T \qquad (TT^*)^2 = (TT^*)(TT^*) = T(T^*TT^*) = TT^*. \]

$\boxed{(4) \Rightarrow (1)}$ Assume $TT^*$ a projection. Let $v\in \ker(T)^\perp = \overline{\im(T^*)}$. Then there exists a sequence $\seq{v_n}\in H^{\prime\N}$ such that $\lim_{n\to\infty}T^*v_n = v$. Then
\begin{align*}
\norm{Tv}^2 &= \lim_{n\to\infty}\norm{TT^*v_n}^2 = \lim_{n\to\infty}\inner{TT^*v_n,TT^*v_n} \\
&= \lim_{n\to\infty}\inner{(TT^*)^2v_n,v_n} = \lim_{n\to\infty}\inner{TT^*v_n,v_n} \\
&= \lim_{n\to\infty}\inner{T^*v_n,T^*v_n} = \lim_{n\to\infty}\norm{T^*v_n}^2 = \norm{v}^2,
\end{align*}
so $T$ is a partial isometry.

$\boxed{(5,6)}$ Applying the proposition to $T^*$ instead of $T$ yields the equivalences with $T=TT^*T$, and thus with the rest of the statements.

TODO + $\im(T^*) = \ker(T)^\perp$ means support and range are exchanged between $T$ and $T^*$.
\end{proof}

\begin{definition}
Let $T$ be a partial isometry. We call
\begin{itemize}
\item $T^*T$ the \udef{support projection} or \udef{initial projection} of $T$;
\item $TT^*$ the \udef{range projection} or \udef{final projection} of $T$.
\end{itemize}
\end{definition}

\begin{proposition}
Let $H,H'$ be Hilbert spaces with $K\subseteq H$ and $L\subseteq H'$ closed subspaces. Then the following are equivalent:
\begin{enumerate}
\item $T$ is a partial isometry with support $K$ and range $L$;
\item $(T,T^*)$ is a Galois connection between $\sSet{H, \perp_K}$ and $\sSet{H', \perp_L}$.
\end{enumerate}
Here $\perp_K$ is defined by
\[ x \perp_K y \quad\defequiv\quad P_K(x)\perp P_{K}(y). \]
\end{proposition}
\begin{proof}
The direction $(2) \Rightarrow (1)$ is immediate from \ref{partialIsometryEquivalences}, because $T,T^*$ are generalised inverses.

For the other direction, we first prove $T$ preserves the relational structure. Take arbitrary $x= x_1+x_2$ and $y=y_1+y_2$ in $K\oplus K^\perp$ such that $x\perp_K y$. Then
\[ \inner{T(x), T(y)} = \inner{T(x_1), T(y_1)} = \inner{x_1, y_1} = 0. \]
So $T(x)\perp T(y)$ and, because $T(x), T(y) \in L$, we have $T(x)\perp_L T(y)$. The argument for $T^*$ is similar.

For the Galois condition, we need to show that $T^*T(x)\perp_K y \implies x\perp_K y$. Indeed
\begin{align*}
T^*T(x)\perp_K y &\iff T^*T(x_1)\perp y_1 \\
&\iff 0= \inner{T^*T(x_1), y_1} = \inner{T(x_1), T(y_1)} = \inner{x_1,y_1} \\
&\iff P_K(x)\perp P_K(y).
\end{align*}
\end{proof}
\begin{corollary}
Let $T: H\to H'$ be a partial isometry with support $K$ and range $L$. Then
\[ T(x) \perp P_L(y) \iff P_K(x) \perp T^*(y) \]
for all $x\in H, y\in H'$.
\end{corollary}
\begin{proof}
This is the Galois identity \ref{GaloisIdentity}, although the direct proof is also very simple.
\end{proof}

\subsubsection{Unitaries}
\paragraph{Bilateral shifts}


\section{Dirac notation}
\url{https://core.ac.uk/download/pdf/25263496.pdf}
\url{https://michael-herbst.com/talks/2014.07.22_Mathematical_Concept_Dirac_Notation.pdf}
\url{http://galaxy.cs.lamar.edu/~rafaelm/webdis.pdf}
\url{https://plato.stanford.edu/entries/qt-nvd/}
\url{file:///C:/Users/user/Downloads/Abdus%20Salam,%20E.P.%20Wigner%20(Ed.)%20-%20Aspects%20of%20Quantum%20Theory%20-%20Dedicated%20to%20Dirac%E2%80%99s%2070th%20Birthday-Cambridge%20University%20Press%20(1972).pdf}
\url{https://aip.scitation.org/doi/pdf/10.1063/1.1705001}

\begin{lemma}
\begin{enumerate}
\item $T\ketbra{\varphi}{\psi} = \ketbra{T\varphi}{\psi} = \ketbra{\varphi}{\psi}T = \ketbra{\varphi}{T^*\psi}$;
\item $\ketbra{\varphi}{\psi}\ketbra{\xi}{\eta} = \inner{\psi, \xi}\ketbra{\varphi}{\eta}$;
\item $(\ketbra{\varphi}{\psi})^* = \ketbra{\psi}{\varphi}$.
\end{enumerate}
\end{lemma}

\begin{lemma}
Let $H$ be a Hilbert space and $\seq{e_i}_{i\in I}$ a basis for $H$. Then
\[ \id_H = \sum_{i\in I}\ketbra{e_i}{e_i} \qquad\text{in the strong limit.} \]
\end{lemma}
\begin{proof}
TODO!!
\end{proof}
\begin{lemma} \label{operatorBraketExpansion}
Let $H$ be a Hilbert space, $\seq{e_i}_{i\in I}$ a basis for $H$ and $T$ an operator on $H$. Then
\[ T = \sum_{i,j\in I}\braket[T]{e_i}{e_j}\; \ketbra{e_i}{e_j}. \]
in the strong limit.
\end{lemma}
\begin{proof}
TODO!! Tannery.
\end{proof}

\section{Hilbert space ideals}

\subsection{Finite-rank operators}
Remember that finite-rank operators are bounded by definition (this is not automatic, cfr. \ref{continuousMapCriterion}).

\begin{proposition}[Finite rank singular value decomposition] \label{finiteRankSingularValues}
Let $V$ be an inner product space and $T\in\Hom(V)$. Then $T$ is a finite-rank operator \textup{if and only if} $T$ can be written in the form
\[ T = \sum_{i=1}^N \lambda_i \ketbra{v_i}{w_i}, \]
where $(v_i)_{i=1}^N$ and $(w_i)_{i=1}^N$ are finite sets of vectors and $(\lambda_i)_{i=1}^N$ are positive (non-zero) numbers.

The numbers $(\lambda_i)_{i=1}^N$ in this decomposition are uniquely determined by the operator.
\end{proposition}
The numbers $(\lambda_i)_{i=1}^N$ are called the \udef{singular values} of the operator.
\begin{proof}
Because $\im(T)$ is finite-dimensional, we can find an orthonormal basis $(v_i)_{i=1}^N$ for it. Then we can write
\begin{align*}
Tx &= \sum_{i=1}^N \ket{v_i}\braket{v_i}{Tx} = \sum_{i=1}^N \ket{v_i}\braket{T^*v_i}{Tx} = \sum_{i=1}^N \ket{v_i}\braket{\lambda_i w_i}{Tx}  = \sum_{i=1}^N \lambda_i\ket{v_i}\braket{w_i}{Tx}
\end{align*}
where $\lambda_i = \norm{T^*v_i}$ and $w_i = \frac{T^*v_i}{\lambda_i}$.

We just need to show that the $\lambda_i$ are independent of the chosen basis $(v_i)_{i=1}^N$. TODO!!!!
\end{proof}
\begin{corollary}
Every finite-rank operator on a Hilbert space is a finite sum of rank-1 operators.
\end{corollary}

\begin{lemma}
Let $H$ be Hilbert space. The set of finite rank operators on $H$ is a $*$-ideal in $H$.
\end{lemma}

\subsection{Compact operators}
\begin{proposition}
Let $T\in\Bounded(H)$. Then the following are equivalent:
\begin{enumerate}
\item $T$ is compact;
\item $T^*$ is compact;
\item there exists a sequence $(T_n)_{n\in\N}$ of finite rank operators such that $\norm{T-T_n}\to 0$.
\end{enumerate}
\end{proposition}
This is false in Banach spaces. (TODO Enflo, approximation property, goose problem)
\begin{proof}
TODO
\end{proof}
\begin{corollary}[Canonical expansion]
Any compact operator $T$ on a Hilbert space $\mathcal{H}$ can be written in the form
\[ T = \sum_{i=1}^\infty \lambda_i \ketbra{v_i}{w_i}, \]
where $(v_i)_{i=1}^\infty$ and $(w_i)_{i=1}^\infty$ are orthonormal sets and $(\lambda_i)_{i=1}^\infty$ is a monotonically decreasing sequence of positive numbers with $\lim_{i\to\infty}\lambda_i = 0$.
\end{corollary}
As in \ref{finiteRankSingularValues} for finite-rank operators we call $(\lambda_i)_{i=1}^\infty$ the \udef{singular values} of $T$. They are uniquely determined by the operator.
\begin{proof}
TODO (one way is with polar decomposition and spectral theorem. Are there others?)
\end{proof}
Compare with \ref{operatorBraketExpansion}.

\begin{lemma}
Let $H$ be a Hilbert space. Then the set of compact operators on $H$, $\Compact(H)$ is a two-sided $*$-ideal of $H$. 
\end{lemma}

\begin{proposition}
Let $H$ be a Hilbert space with orthonormal basis $(e_i)_{i\in I}$. If $T\in\Bounded(H)$ and
\[ \sum_{i\in I}\norm{Te_i}^2  < \infty, \]
then $T$ is a compact operator. + Converse??
\end{proposition}
\begin{proof}
TODO + weaken $T\in\Bounded(H)$?
\end{proof}
\begin{corollary}
An integral operator defined by a square integrable kernel $K\in L^2(A\times A, \mu)$ is compact.
\end{corollary}

\begin{proposition}
Let $T$ be an operator on a Hilbert space. Then the following are equivalent:
\begin{enumerate}
\item $T$ is compact;
\item for all sequences $\seq{x_n}$, weak convergence $x_n \overset{w}{\to} x$ implies the strong convergence $Ax_n \to Ax$;
\item for any two weakly convergent sequences $x_n\overset{w}{\to} x$ and $y_n\overset{w}{\to} y$ the energy form is continuous in both arguments:
\[ \lim_{n\to\infty}\inner{x_n,y_n}_T = \lim_{n\to\infty}\inner{x_n,Ty_n} = \inner{x,Ty} = \inner{x,y}_T. \]
\end{enumerate} 
\end{proposition}

\begin{lemma}
Let $H$ be a Hilbert space and $P\in\Projections(H)$. If $P$ is compact, then $P$ has finite rank.
\end{lemma}

\subsection{Positive operators}

\subsubsection{Polar decomposition}
\begin{proposition}
Let $H$ be a Hilbert space and $T\in \Bounded(H)$. There exists a unique partial isometry $V$ such that $T = V|T|$ and $\ker(V) = \supp(T)$.
\end{proposition}
TODO: should this be $\ker(V) = \ker(T)$??
\begin{proof}
TODO
\end{proof}
\begin{lemma}
There is only one positive operator $A$ such that $T = VA$ for some partial isometry.
\end{lemma}
\begin{proof}
TODO uniqueness positive squared root.
\end{proof}

\url{https://encyclopediaofmath.org/wiki/Polar_decomposition}

\subsection{Trace class operators}
TODO Simon


\section{Dilation theory}
\subsection{Dilations, $N$-dilations and power dilations}
\begin{definition}
Let $\mathcal{H} \subseteq \mathcal{H}'$ be Hilbert spaces and let $P_\mathcal{H}$ be the projection on $\mathcal{H}$. If a pair of linear maps $S: \mathcal{H}'\to\mathcal{H}'$ and $T: \mathcal{H}\to \mathcal{H}$ satisfy the relation
\[ T = P_\mathcal{H} S |_\mathcal{H} \]
then $T$ is called a \udef{compression} of $S$ and $S$ a \udef{dilation} of $T$. This is abbreviated $T\prec U$.

\begin{itemize}
\item Let $N\in\N$. If $T^k = P_\mathcal{H} S^k |_\mathcal{H}$ for all $k\leq N$, then $S$ is called an \udef{$N$-dilation}.
\item If this holds for all $k\in\N$, then $S$ is called a \udef{power dilation}.
\item If $T^* = P_\mathcal{H} S^* |_\mathcal{H}$, we call TODO??
\end{itemize}
We call $\mathcal{H}'$ \udef{minimal} if the only reducing subspace for $S$ that contains $\mathcal{H}$ is $\mathcal{H}'$.
\end{definition}

If $S$ is a dilation of $T$, then we clearly have $T = P_\mathcal{H} S P_\mathcal{H}|_\mathcal{H}$.

\begin{lemma}
Let $S:\mathcal{H}'\to\mathcal{H}'$ be an $N$-dilation of $T: \mathcal{H}\to \mathcal{H}$ and $p$ a polynomial of degree at most $N$. Then
\[ p(T) = P_\mathcal{H}p(S)|_\mathcal{H}. \]
\end{lemma}

Let $\mathcal{H}$ be a Hilbert space. We call $T\in\Bounded(\mathcal{H})$ a \udef{contraction} if $\norm{T}\leq 1$.
\begin{proposition} \label{dilationOfContraction}
Let $\mathcal{H} \cong \mathcal{H}\oplus \{0\} \subseteq \mathcal{H}\oplus \mathcal{H} = \mathcal{H}^2$ be a Hilbert space. Every contraction $T$ on $\mathcal{H}$ has a unitary dilation $U$ on $\mathcal{H}^2$.
\end{proposition}
\begin{proof}
From $\norm{T}\leq 1$ (and the fact that $T^*T$ is normal), we have that $\vec{1}-T^*T\geq 0$ by spectral mapping. We can define $D_T = \sqrt{\vec{1}-T^*T}$. Then
\[ U = \begin{pmatrix}
T & D_{T^*} \\ D_T & -T^*
\end{pmatrix} \]
is a dilation of $T$ and it is unitary:
\begin{align*}
UU^* &= \begin{pmatrix}
TT^* + D_{T^*}^2 & TD_T^* - D_{T^*}T \\
D_TT^* - T^*D_{T^*}^* & D^2_{T} + T^*T
\end{pmatrix} = \begin{pmatrix}
\vec{1} & TD_T - D_{T^*}T \\
D_TT^* - T^*D_{T^*} & \vec{1}
\end{pmatrix} \\
U^*U &= \begin{pmatrix}
T^*T + D_{T}^2 & T^*D_{T^*} - D_{T}^*T^* \\
D_{T^*}^*T - TD_{T} & D^2_{T^*} + TT^*
\end{pmatrix} = \begin{pmatrix}
\vec{1} & T^*D_{T^*} - D_{T}T^* \\
D_{T^*}T - TD_{T} & \vec{1}.
\end{pmatrix}
\end{align*}
We have used that $D_T$ is self-adjoint for all contractions $T$. We just need to show that $TD_T = D_{T^*}T$. Clearly we have
\[ T(D_T)^2 = T(\vec{1} - T^*T) = T - TT^*T = (\vec{1} - TT^*)T = (D_{T^*}T)^2T. \]
By functional-like calculus (TODO!!) we have $TD_T = D_{T^*}T$.
\end{proof}
The operator $D_T$ in the previous proof is sometimes called the \udef{defect operator} of $T$. It measures in some sense how far $T$ is from being a unitary operator. If $T$ is unitary, then $D_T = 0 = D_{T^*}$. If $T$ is an isometry, then $D_T = 0$ (by \ref{isometryRangeProjection}) and $D_{T^*}$ is a projector ($TT^*$ is a projector by \ref{isometryCharacterisation}, so $\vec{1} - TT^*$ is too by \ref{projectorOrthogonalComplement} and $D_{T^*} = \sqrt{\vec{1}-TT^*} = \sqrt{(\vec{1}-TT^*)^2} = \vec{1}-TT^*$).

\begin{proposition}
Let $\mathcal{H} \cong \mathcal{H}\oplus \{0\} \subseteq \mathcal{H}\oplus \mathcal{H} = \mathcal{H}^2$ be a Hilbert space. Every isometry $T$ on $\mathcal{H}$ has a unitary power dilation $U$ on $\mathcal{H}^2$.
\end{proposition}
\begin{proof}
Consider the unitary dilation of \ref{dilationOfContraction}. When $T$ is an isometry this reduces to
\[ U = \begin{pmatrix}
T & D_{T^*} \\ 0 & -T^*
\end{pmatrix} = \begin{pmatrix}
T & \vec{1}-TT^* \\ 0 & -T^*
\end{pmatrix}, \]
where we have used that $D_{T^*} = \sqrt{\vec{1}-TT^*} = \sqrt{(\vec{1}-TT^*)^2} = \vec{1}-TT^*$ is a projector.

Now for all $n\in\N$ we have $U^n = \begin{pmatrix}
T^n & * \\ 0 & (-T^*)^n
\end{pmatrix}$, so in particular $P_\mathcal{H}U^n|_\mathcal{H} = T^n$, meaning $U$ is a power dilation of $T$. 
\end{proof}

\begin{lemma}
Let $T$ a contraction on a Hilbert space $\mathcal{H}$. Then $V_T: \mathcal{H} \to \mathcal{H}\oplus\mathcal{H}: x\mapsto (Tx, D_Tx)$ is an isometry.
\end{lemma}
\begin{proof}
For all $x\in \mathcal{H}$ we have
\[ \norm{V_Tx} = \sqrt{\norm{Tx}^2 + \norm{D_Tx}^2} = \sqrt{\inner{Tx,Tx} + \inner{D_Tx,D_Tx}} = \sqrt{\inner{T^*Tx,x} + \inner{D_T^2x,x}} = \sqrt{\inner{x,x}} = \norm{x}. \]
\end{proof}

\begin{proposition}
Let $\mathcal{H} \cong \mathcal{H}\oplus \{0\}^N \subseteq \mathcal{H}^{N+1}$ be a Hilbert space. Every contraction $T$ on $\mathcal{H}$ has a unitary $N$-dilation $U$ on $\mathcal{H}^{N+1}$.
\end{proposition}
\begin{proof}
Let $U'$ be a unitary dilation of $T$ on $\mathcal{H}^2$. Let $C_1 = U'_{-,1}$ and $C_2 = U'_{-,2}$ denote the columns. Then
\[ U = \begin{pmatrix}
C_1 & \mathbb{0}^{2\times N-1} & C_2 \\
\mathbb{0}^{N-1\times 1} & \mathbb{1}^{N-1\times N-1} & \mathbb{0}^{N-1\times 1}
\end{pmatrix} \]
is unitary by
\[ \begin{pmatrix}
C_1^* & \mathbb{0} \\
\mathbb{0} & \mathbb{1} \\
C_2^* & \mathbb{0}
\end{pmatrix}\begin{pmatrix}
C_1 & \mathbb{0} & C_2 \\
\mathbb{0} & \mathbb{1} & \mathbb{0}
\end{pmatrix} = \begin{pmatrix}
C_1^*C_1 & \mathbb{0} & C_1^*C_2 \\
\mathbb{0} & \mathbb{1} & \mathbb{0} \\
C_2^*C_1 & \mathbb{0} & C_2^*C_2
\end{pmatrix} = \mathbb{1}^{N+1\times N+1}. \]
We just need to show that the (1,1)-component of $U^k$ is $T^k$ for all $k\in 1:N$. In order to perform the multiplication, we rewrite $U$ such that the row and column partitions are the same, i.e.\ $(2|(N-3)|2)\times (2|(N-3)|2)$:
\[ U = \begin{pmatrix}
\begin{bmatrix}
T & 0 \\ D_T & 0
\end{bmatrix} & \mathbb{0} & \begin{bmatrix}
0 & D_{T^*} \\ 0 & -T^*
\end{bmatrix} \\
\begin{bmatrix}
0 & 1 \\ \mathbb{0} & \mathbb{0}
\end{bmatrix} & \begin{bmatrix}
\mathbb{0} & 0 \\ \mathbb{1} & \mathbb{0}
\end{bmatrix} & \mathbb{0} \\
\begin{bmatrix}
0 & 0 \\ 0 & 0
\end{bmatrix} & \begin{bmatrix}
\mathbb{0} & 1 \\ \mathbb{0} & 0
\end{bmatrix} & \begin{bmatrix}
0 & 0 \\ 1 & 0
\end{bmatrix}
\end{pmatrix} \]
TODO
\end{proof}

\begin{proposition}[von Neumann's inequality]
Let $T$ be a contraction on some Hilbert space $\mathcal{H}$. Then, for every polynomial $p\in\C[z]$,
\[ \norm{p(T)}\leq \sup_{|z|=1}|p(z)|. \]
\end{proposition}
\begin{proof}
Suppose the degree of $p$ is $N$. Let $U$ be a unitary $N$-dilation of $T$. Then
\[ \norm{p(T)} = \norm{P_\mathcal{H}p(U)|_\mathcal{H}}\leq \norm{p(U)} = \sup_{z\in\sigma(U)}|p(z)| \leq \sup_{|z|=1}|p(z)| \]
since the spectrum of $U$ is contained in the unit circle.
\end{proof}

\begin{theorem}[Sz.-Nagy's dilation theorem]
Let $\mathcal{H} \subseteq \ell^2(\N)\otimes\mathcal{H}$ be Hilbert spaces. Every contraction on $\mathcal{H}$ has a unitary power dilation on $\ell^2(\N)\otimes\mathcal{H}$.
\end{theorem}




\section{Constructions}
\subsection{Direct sum}
\subsection{Tensor product}
\url{https://web.ma.utexas.edu/mp_arc/c/14/14-2.pdf}



\chapter{Types of operators}
\section{Fredholm operators}
\begin{definition}
An operator $T\in\Bounded(X,Y)$ between Banach spaces is called a \udef{Fredholm operator} if $T$ has a finite-dimensional kernel and cokernel.

The \udef{Fredholm index} of $T$ is defined as
\[ \Index T \defeq \dim\ker T - \dim\coker T.  \]

We denote the space of Fredholm operators from $X$ to $Y$ as $\Fred(X,Y)$. If $X=Y$, we write $\Fred(X)$.
\end{definition}

\begin{example}
\begin{enumerate}
\item If $X=Y$ is finite-dimensional, then all operators are Fredholm with index $0$.
\item The left shift $S_l:\ell^2(\N)\to\ell^2(\N): (x_n)_n\mapsto (x_{n+1})_n$ has index $1$.
\item The right shift $S_r = S_l^*$ has index $-1$.
\end{enumerate}
\end{example}

\begin{lemma}
A Fredholm operator has closed range.
\end{lemma}

\begin{lemma}
Let $T\in\Bounded(H)$ be a bounded operator on a Hilbert space. Then $\dim\coker T = \dim\ker T^*$.
\end{lemma}
\begin{proof}
TODO (is it correct?) $\ker(T^*) = \im(T)^\perp$.
\end{proof}


\begin{proposition}
Let $S,T\in\Fred(X)$, $\lambda\in\F$ and $K\in\Compact(X)$. Then
\begin{enumerate}
\item $\Index(ST) = \Index(S)+\Index(T)$;
\item $\Index(T+K) = \Index(T)$;
\item $\Index(\lambda T) = \Index(T)$, if $\lambda \neq 0$;
\item $\Index(T) = 0$ \textup{if and only if} $T=K'+L$ for some compact $K'$ and invertible $L$.
\end{enumerate}
Let $T\in\Fred(H)$ for some Hilbert space $H$. Then
\begin{enumerate} \setcounter{enumi}{4}
\item $\Index(T^*) = -\Index(T)$.
\end{enumerate}
\end{proposition}
TODO: integrate with corollary??

\begin{lemma}
Let the commutative diagram
\[ \begin{tikzcd}
0 \rar & X \dar{T} \rar & Y \dar{S} \rar & Z \dar{R} \rar & 0 \\
0 \rar & X \rar & Y \rar & Z \rar & 0
\end{tikzcd} \]
have short exact rows. If any two of $T,S,R$ are Fredholm, then so is the third and
\[ \Index S = \Index T + \Index R. \]
\end{lemma}
\begin{proof}
TODO snake lemma to obtain long exact
\[ 0\to \ker T \to \ker S\to \ker R \to \coker T \to \coker S \to \coker R \to 0. \]
\end{proof}
\begin{corollary} \mbox{}
\begin{enumerate} 
\item Let $T\in\Fred(X)$ and $S\in\Fred(Y)$ be Fredholm, then so is $T\oplus S$ with
\[ \Index(T\oplus S) = \Index(T)+\Index(S). \]
\item Let $T\in\Fred(X,Y)$ and $S\in\Fred(Y,Z)$ be Fredholm, then so is $ST$ with
\[ \Index(ST) = \Index(T)+\Index(S). \]
\item Let $K\in\Compact(X)$ be compact, then $\id_X+K$ is Fredholm with
\[ \Index(\id_X+K) = 0. \]
\end{enumerate}
\end{corollary}


\begin{lemma}[Fredholm alternative] \label{FredholmAlternative}
Let $T$ be a Fredholm operator of index zero. Then either $T$ is bijective, or it is neither injective nor surjective.
\end{lemma}
\begin{proof}
The operator $T$ is injective iff $\dim\ker(T) = 0$ and surjective iff $\dim\coker(T) = 0$.
\end{proof}


\section{Integral operators and transforms}
\begin{definition}
Let $(\Omega, \mathcal{A}, \mu)$ be a measure space. Then an \udef{integral operator} or \udef{integral transform} is a map of the form
\[ T: U\subset (\Omega\to\C) \to (\Omega\to\C): f \mapsto \int_\Omega K(x,y)f(y) \diff{\mu(y)} \]
where $K\in (\Omega\times \Omega \to \C)$ is the \udef{kernel} or \udef{nucleus} of $T$.

The kernel is called
\begin{itemize}
\item \udef{symmetric} if $K(x,y) = \overline{K(y,x)}$;
\item \udef{Volterra} if $\Omega = \R$ and $K(x,y) = 0$ for $y>x$;
\item \udef{convolutional} if $\Omega$ is a group and $K(x,y) = F(x-y)$ for some function $F$;
\item \udef{Hilbert-Schmidt} if $K\in L^2(\Omega\times \Omega)$, i.e.\
\[ \int_{\Omega\times \Omega}|K(x,y)|^2\diff{x}\diff{y} < \infty; \]
\item \udef{singular} if $K(x,y)$ is unbounded on $\Omega\times \Omega$.
\end{itemize}
\end{definition}

\begin{lemma}
Hilbert-Schmidt integral operators are compact operators on $L^2(\Omega\times \Omega)$.
\end{lemma}
\begin{proof}
A Hilbert-Schmidt integral operator $T$ maps $L^2(\Omega)$ to $L^2(\Omega)$ functions:
\begin{align*}
\norm{Tu}^2_{L^2} &= \int_\Omega \left|\int_{\Omega} K(x,y)u(y)\diff{\mu(y)}\right|^2\diff{\mu(x)} \\
&\leq \int_\Omega \left(\int_{\Omega} |K(x,y)|^2\diff{\mu(y)}\right) \bigg( |u(y)|^2\diff{\mu(y)}\bigg)\diff{\mu(x)} \\
&= \left(\int_\Omega \int_{\Omega} |K(x,y)|^2\diff{\mu(y)}\diff{\mu(x)}\right) \bigg( |u(y)|^2\diff{\mu(y)}\bigg) < \infty
\end{align*}
where we have used the Cauchy-Schwarz inequality. This also immediately shows Hilbert-Schmidt integral operators are bounded.

TODO Compact
\end{proof}

\begin{proposition}
Let $T$ be an integral operator with kernel $K(x,y)$, then $T^*$ is the integral operator with kernel $\overline{K(y,x)}$.
\end{proposition}
\begin{proof}
TODO
\end{proof}

\begin{proposition}
Let $A$ be a Borel set and $K:A\times A\to \C$ a measurable function such that the integral operator with kernel $K$ is bounded. Then the adjoint of the integral operator is again an integral operator with kernel $K^*(x,y) = \overline{K(y,x)}$.
\end{proposition}

\begin{proposition}
Let $T$ be a Volterra integral operator. Then $\spec(T) = \cspec(T) = \{0\}$.
\end{proposition}
\begin{proof}
TODO
\end{proof}

\subsection{Integral equations}
\begin{definition}
Let $(\Omega, \mathcal{A}, \mu)$ be a measure space. An \udef{integral equation} is an equation containing an unknown function on $\Omega$ and an integral over $\Omega$.

An integral equation is 
\begin{itemize}
\item \udef{of the first kind} if it is of the form
\[ \int_\Omega K(x,y)u(y)\diff{\mu(y)} = f(x) \qquad x\in \Omega \]
where $f$ is a given function and $u$ is the unknown function;
\item \udef{of the second kind} if it is of the form
\[ \lambda u(x) - \int_\Omega K(x,y)u(y)\diff{\mu(y)} = f(x) \qquad x\in \Omega \]
where $f$ is a given function, $\lambda$ is a scalar and $u$ is the unknown function.
\end{itemize}
\end{definition}

\begin{proposition}
Let
\[ \lambda u(x) - \int_\Omega K(x,y)u(y)\diff{\mu(y)} = f(x)\]
be an integral equation of the second kind. This integral equation has a unique solution $u$ if
\[ |\lambda| > \sup_{x\in \Omega} \int_{\Omega}|K(x,y)|\diff{\mu(y)}. \]
\end{proposition}
\begin{proof}
Let the map $T$ be defined by
\[ T(u) = x\mapsto \frac{1}{\lambda}\left(\int_\Omega K(x,y)u(y)\diff{\mu(y)} + f(x)\right) \]
so that solutions of the integral equation are exactly the fixed points of $T$. Then
\[ \norm{Tu-Tv}_\infty = \sup_{x\in\Omega} \frac{1}{|\lambda|} \left|\int_\Omega K(x,y)(u(y)- v(y))\diff{\mu(y)}\right| \leq \frac{1}{|\lambda|} \sup_{x\in \Omega} \int_{\Omega}|K(x,y)|\diff{\mu(y)} \cdot \norm{u-v}_\infty. \]
So $T$ is a contraction if $|\lambda| > \sup_{x\in \Omega} \int_{\Omega}|K(x,y)|\diff{\mu(y)}$. The result follows from \ref{contractionFixedPoint}.
\end{proof}

\section{Convolution operators}





\chapter{Fourier transforms}

\section{Types of Fourier transform}

\subsection{Discrete Fourier transform}
\begin{definition}
Then $N$-dimensional \udef{discrete Fourier transform} (DFT) is the linear transformation $\C^N \to \C^N$ defined by the matrix $DFT_N$ with components
\[ [DFT_N]_{j,k} = \frac{1}{\sqrt{N}}\omega_N^{(j-1)(k-1)}, \]
where $\omega_N$ is the $N^\text{th}$ root of unity.
\end{definition}

\begin{lemma} \mbox{}
\begin{enumerate}
\item The $DFT_N$ matrix is the Vandermonde matrix of the roots of unity, up to the normalisation factor $1/\sqrt{N}$.
\item The $DFT_N$ matrix is unitary.
\end{enumerate}
\end{lemma}
\begin{proof}
(1) Just an observation.

(2) We calculate
\[ [DFT_N\cdot DFT_N]_{j,l} = \frac{1}{N}\sum_{k=1}^N\omega_N^{jk}\overline{\omega_N}^{kl} = \frac{1}{N}\sum_{k=1}^N\omega_N^{k(j-l)} = \delta_{j,l}. \]
\end{proof}


\part{Analysis}
\setcounter{chapter}{0} % Reset chapter counters
\url{file:///C:/Users/user/Downloads/0-8176-4442-3.pdf}
\url{https://link.springer.com/content/pdf/10.1007%2F978-0-387-84895-2.pdf}
\url{https://zr9558.files.wordpress.com/2014/08/a-guide-to-distribution-theory-and-fourier-transforms.pdf}
\url{file:///C:/Users/user/Downloads/978-0-8176-4675-2.pdf}

Integral mean value theorem \url{https://en.wikipedia.org/wiki/Mean_value_theorem#Mean_value_theorems_for_definite_integrals}

TODO: Hölder, Monkowski, Lyapounov

\chapter{Limits}

\section{Bachmann-Landau notation}
\subsection{Asymptotic bounds: $O, \Theta, \Omega$}
\begin{definition}
Let $(X,\mathcal{T})$ be a topological space and $(V,\norm{\cdot})$ a normed space. Let $x_0 \in X$ and $f,g: X\setminus\{x_0\}\to V$ be functions. The statement
\begin{itemize}
\item ``$f(x) = O(g(x))$ as $x\to x_0$'' means there exists a neighbourhood $S$ of $x_0$ and a constant $M\in\R$ such that
\[ \forall x\in S:\; \norm{f(x)} \leq M\norm{g(x)}; \]
\item ``$f(x) = \Omega(g(x))$ as $x\to x_0$'' means there exists a neighbourhood $S$ of $x_0$ and a constant $M\in\R$ such that
\[ \forall x\in S:\; \norm{f(x)} \geq M\norm{g(x)}; \]
\item ``$f(x) = \Theta(g(x))$ as $x\to x_0$'' means there exists a neighbourhood $S$ of $x_0$ and constants $M_1,M_2\in\R$ such that
\[ \forall x\in S:\; M_1\norm{g(x)} \leq \norm{f(x)} \leq M_2\norm{g(x)}. \]
\end{itemize}
We may add the word ``uniformly'' to these statements to mean we can take $S=X$.

We may suppress the $x$ dependence for legibility and write e.g. $f = O(g)$ instead.
\end{definition}

\begin{lemma}
Let $(X,\mathcal{T})$ be a topological space and $(V,\norm{\cdot})$ a normed space. Let $x_0 \in X$ and $f,g: X\setminus\{x_0\}\to V$ be functions. Then
\begin{enumerate}
\item $f = O(g)$ \textup{if and only if} $g = \Omega(f)$ as $x\to x_0$;
\item $f = \Theta(g)$ \textup{if and only if} $f = O(g)$ and $f = \Omega(g)$ as $x\to x_0$.
\end{enumerate}
\end{lemma}

\begin{lemma}
Let $(X,\mathcal{T})$ be a topological space and $(V,\norm{\cdot})$ a normed space. Let $x_0 \in X$ and $f: X\setminus\{x_0\}\to V$ be functions. Then ``being $\Theta(f)$ as $x\to x_0$'' is an equivalence relation.
\end{lemma}

\begin{lemma}
Let $(X,\mathcal{T})$ be a topological space and $(V,\norm{\cdot})$ a normed space. Let $x_0 \in X$ and $f,g: X\setminus\{x_0\}\to V$ be functions. Then
\begin{enumerate}
\item $f(x) = O(g(x))$ as $x\to x_0$ \textup{if and only if} there exists a neighbourhood $S$ of $x_0$ such that $\norm{f(x)}/\norm{g(x)}$ is bounded on $S$;
\item $f(x) = \Omega(g(x))$ as $x\to x_0$ \textup{if and only if} there exists a neighbourhood $S$ of $x_0$ such that $\norm{f(x)}/\norm{g(x)}$ is bounded below on $S$ by a strictly positive constant.
\end{enumerate}
\end{lemma}

\subsection{Asymptotic domination and equality: $o,\sim,\omega$}
\begin{definition}
Let $(X,\mathcal{T})$ be a topological space and $(V,\norm{\cdot})$ a normed space. Let $x_0 \in X$ and $f,g: X\setminus\{x_0\}\to V$ be functions. The statement
\begin{itemize}
\item ``$f(x) = o(g(x))$ as $x\to x_0$'' means $\lim_{x\to x_0} \frac{\norm{f(x)}}{\norm{g(x)}} = 0$;
\item ``$f(x) \sim_{x_0} g(x)$'' means $\lim_{x\to x_0} \frac{\norm{f(x)}}{\norm{g(x)}} = 1$;
\item ``$f(x) = \omega(g(x))$ as $x\to x_0$'' means $\lim_{x\to x_0} \frac{\norm{f(x)}}{\norm{g(x)}} = \infty$.
\end{itemize}
\end{definition}

\begin{lemma}
Let $(X,\mathcal{T})$ be a topological space and $(V,\norm{\cdot})$ a normed space. Let $x_0 \in X$ and $f,g: X\setminus\{x_0\}\to V$ be functions.

Then $f = o(g)$ \textup{if and only if} $g = \omega(f)$ as $x\to x_0$.
\end{lemma}

\begin{lemma}
Let $(X,\mathcal{T})$ be a topological space and $(V,\norm{\cdot})$ a normed space. Let $x_0 \in X$ and $f,g: X\setminus\{x_0\}\to V$ be functions. Then
\begin{enumerate}
\item $f\sim_{x_0} g \iff (f-g)\in o(g)$ as $x\to x_0$;
\item $\sim_{x_0}$ is an equivalence relation;
\item $f \sim_{x_0} g \implies f = \Theta(g)$ as $x\to x_0$.
\end{enumerate}
\end{lemma}

\begin{lemma}
Let $(X,\mathcal{T})$ be a topological space and $(V,\norm{\cdot})$ a normed space. Let $x_0 \in X$ and $f,g, h,k: X\setminus\{x_0\}\to V$ be functions. Then
\begin{enumerate}
\item if $f = o(h)$ and $g = O(k)$, then $fg = o(hk)$ as $x\to x_0$;
\item if $f = O(h)$ and $g = O(k)$, then $fg = O(hk)$ as $x\to x_0$;
\item if $f = o(h)$ and $h = O(k)$, then $f = o(k)$ as $x\to x_0$.
\end{enumerate}
\end{lemma}

\chapter{Differentiation}
\url{file:///C:/Users/user/Downloads/978-1-4614-3894-6.pdf}
\url{file:///C:/Users/user/Downloads/2011_Bookmatter_TheRicciFlowInRiemannianGeomet.pdf}

\section{For real functions}

\section{For real normed vector spaces}
TODO: directional / Gateaux derivative for locally convex TVSs?
\subsection{Directional derivatives}
\begin{definition}
Let $V,W$ be normed vector spaces and $f:U\subseteq V\to W$ a function defined on an open subset $U$. For $a,u\in V$, we call
\[ \partial_u f|_a \defeq \lim_{t\to 0} \frac{f(a+tu) - f(a)}{t} \]
the \udef{directional derivative} of $f$ at $a$ in the direction $u$,if it exists.

\begin{itemize}
\item If $V= \R^n$, then we define $\pd{f}{x^i}f \defeq \partial_{\vec{e}_i}f$, where $\mathcal{E} = \seq{\vec{e}_i}_{i=1}^n$ is the standard basis of $\R^n$. These directional derivatives are called the \udef{partial derivatives} w.r.t. the basis $\mathcal{E}$.
\item If $V = \R$, then there is, up to scalar multiplication, only one direction $u$. We denote the directional derivative $f'(a) \defeq \partial_u f|_a$.
\end{itemize}
\end{definition}
For a given function $f:V\to W$, the directional derivative is a partial function of both a direction and a point:
\[ (V\times V) \not\to W:\quad (u,a) \mapsto \partial_u f(a)  \]

Partial application in the first argument gives a function
\[ \partial_u f:\; V\not\to W:\; a\mapsto \partial_u f(a) \defeq \partial_u f|_a \]
that is also referred to as the \udef{directional derivative} of $f$ in the direction $u$.

\begin{lemma}
Let $f,g: V\to W$, $u\in V$ and $\lambda\in\F$, then
\begin{enumerate}
\item $\partial_u(f+g) = \partial_uf + \partial_u g$;
\item $\partial_u(fg) = (\partial_uf)g + f(\partial_u g)$;
\item $\partial_u(\lambda f) = \lambda \partial_uf$.
\end{enumerate}
\end{lemma}

\begin{proposition} \label{derivativeBilinearForm}
Let $B: V_1 \oplus V_2 \to W$ be a bilinear form. Then, for $(x,y),(a,b)\in V_1\oplus V_2$
\[ \partial_{(x,y)}B|_{(a,b)} = B(x,b) + B(a,y). \]
\end{proposition}
\begin{proof}
We calculate
\begin{align*}
\partial_{(x,y)}B|_{(a,b)} &= \lim_{t\to 0} \frac{B(a+tx, b+ty) - B(a,b)}{t} \\
&= \lim_{t\to 0} \frac{1}{t} (B(a,b) + tB(a,y) + tB(x,b) + t^2B(x,y) - B(a,b)) \\
&=B(x,b) + B(a,y) + \lim_{t\to 0} tB(x,y) \\
&= B(x,b) + B(a,y).
\end{align*}
\end{proof}

\subsubsection{Partial derivatives}
TODO notation $D^\alpha$ for multiindex $\alpha$. Also $|\alpha| = \sum_i \alpha_i$.

\subsubsection{Gateaux derivative}
\begin{definition}
Partial application of the directional derivative in the second argument gives a function
\[ \diff{_af}: V\not\to W: u\mapsto \diff{_af}(u) \defeq \partial_u f|_a = \lim_{t\to 0} \frac{f(a+tu) - f(a)}{t} \]
that is referred to as the \udef{Gateaux differential} of $f$ at the point $a$.

If $\diff{_af}: V\not\to W$ is a bounded linear map, we will refer to it as the \udef{Gateaux derivative}.
\end{definition}
The Gateaux differential is homogeneous even if it is not linear:
\begin{lemma}
Let $f:V\to W$ be a function between normed spaces and $a,u\in V$. If $\partial_u f$ is defined at $a$, then
\[ \diff{_a f}(\lambda u) = \partial_{\lambda u}f(a) = \lambda\partial_u f(a) = \lambda \diff{_a f}(u) \qquad \forall \lambda\in\F. \]
\end{lemma}
\begin{proof}
$\partial_{\lambda u}f(a) = \lim_{t\to 0} \frac{f(a+t\lambda u) - f(a)}{t} = \lim_{t\lambda\to 0} \frac{f(a+t\lambda u) - f(a)}{t \lambda / \lambda} = \lambda\partial_u f(a)$.
\end{proof}

TODO mean value theorem?

\subsection{Hadamard derivative}

\subsection{Fréchet derivative}
\begin{definition}
If a function has a (bounded linear) Gateaux derivative at $a$ and the limit in the definition of the derivative
\[ \diff{_af}: V\not\to W: u\mapsto \diff{_af}(u) \defeq \partial_u f|_a = \lim_{t\to 0} \frac{f(a+tu) - f(a)}{t} \]
is uniform in all $u$ on the $S(\vec{0},1)$, then we say the function is \udef{(Fréchet) differentiable} at $a$ and has \udef{Fréchet derivative} $\diff{_af}$.

We may also write $\diff{f}$, leaving the $a$ implicit.
\end{definition}

\begin{proposition}
Let $V,W$ be normed vector spaces and $f:U\subseteq V\to W$ a function defined on an open subset $U$. Let $a\in V$.

Then $f$ is Fréchet differentiable at $a$ \textup{if and only if} there exists a bounded linear map $A: V\to W$ such that $f(a+x)$ can be written as
\[ f(a+x) = f(a) + A(x) + o(x) \qquad \text{as} \qquad x\to 0. \]
In this case $A = \diff{_af}$.
\end{proposition}
\begin{proof}
First assume $f$ is Fréchet differentiable at $a$. Then
\begin{multline*}
\forall \varepsilon>0:\exists \delta>0: \; \forall u\in S(\vec{0},1): \forall t\in\R: \; t< \delta \implies \varepsilon > \\ \norm{\frac{f(a+tu) - f(a)}{t} - \diff{_af}(u)} = \frac{\norm{f(a+tu) - f(a)- \diff{_af}(tu)}}{|t|} = \frac{\norm{f(a+tu) - f(a)- \diff{_af}(tu)}}{\norm{tu}}.
\end{multline*}

Now each vector $x$ in $V$ can be written as $tu$ for some $t\in\R$ and $u\in S(\vec{0},1)$, so this can be written as
\[ \forall \varepsilon>0:\exists \delta>0: \; \forall x\in V: \; \norm{x}< \delta \implies  \varepsilon > \frac{\norm{f(a+x) - f(a)- \diff{_af}(x)}}{\norm{x}} \]
which is exactly the statement $f(a+x) = f(a) + \diff{_af}(x) + o(x)$ as $x\to 0$.

The logic can be reversed to obtain the equivalence.
\end{proof}

\begin{proposition}
If a function is Fréchet differentiable at a point $a$, then it is continuous at $a$.
\end{proposition}
\begin{proof}
Assume $f$ is has Fréchet derivative $A$. Then
\[ 0 = \lim_{x\to a} \norm{f(x) - f(a) - \diff{_af}(x-a)} = \norm{\lim_{x\to a}f(x) - f(a) - \diff{_af}(\lim_{x\to a} x-a)} = \norm{\lim_{x\to a}f(x) - f(a)}. \]
\end{proof}

\begin{lemma}
The Fréchet derivative is the same for equivalent norms.
\end{lemma}

\subsubsection{Link with Gateaux derivative}
\url{https://link.springer.com/content/pdf/bbm%3A978-3-642-16286-2%2F1.pdf}
\url{http://www.m-hikari.com/ams/ams-password-2008/ams-password17-20-2008/behmardiAMS17-20-2008.pdf}
\begin{proposition}
If a function between subsets of normed spaces is Fréchet differentiable, it is also Gateaux differentiable and the Fréchet derivative is equal to the Gateaux derivative.
\end{proposition}
\begin{proof}
Let $A$ be the Fréchet derivative of $f: U\subseteq V\to W$. Then for all $u\in V$
\begin{align*}
0 &= \lim_{t\to 0} \frac{\norm{f(a+tu) - f(a) - A(tu)}}{\norm{tu}} = \lim_{t\to 0} \frac{\norm{(f(a+tu) - f(a))/t - A(u)}}{\norm{u}} \\
&= \frac{\norm{\lim_{t\to 0}(f(a+tu) - f(a))/t - A(u)}}{\norm{u}} = \frac{\norm{\diff{_af}(u) - A(u)}}{\norm{u}}. 
\end{align*}
\end{proof}
For this reason we will also denote the Fréchet derivative of $f$ at $a$ as $\diff{_a f}$. We will sometimes also write $f'(a)$.

\begin{example}
TODO!

There are functions that have a Gateaux derivative, but not a Fréchet derivative at certain points. For example
\[ f: \R^2\to \R: (x,y) \mapsto \begin{cases}
\frac{xy}{x^2+y^2} & (x,y)\neq (0,0) \\0 & (x,y) = (0,0)
\end{cases} \]
which has $\partial_{\vec{u}}f(\vec{0}) = 0$ for all $\vec{u}\in \R^2$ and thus the Gateaux derivative at zero is $\diff{f} = 0$.

Composing $f$ with $t\mapsto (t,t^2)$ yields the function $t\mapsto \begin{cases}
t^{-2} & t\neq 0 \\ 0 & t=0
\end{cases}$, which is not continuous at $0$. So $f$ is not continuous at zero and a fortiori is not Fréchet differentiable.
\end{example}

\begin{proposition}
If there exists a basis $\beta$ of $V$ such that the partial derivatives of $f:U\subseteq V\to W$ w.r.t. $\beta$ exist and are continuous in $a\in V$, then $f$ is Fréchet differentiable in $a$.
\end{proposition}
\begin{proof}

\end{proof}
This is not a necessary condition for the Fréchet derivative to exist!
\begin{example}

\end{example}

\subsubsection{The Jacobian}
\begin{definition}
Let $f:U\subseteq\R^m\to\R^n$ be a function. Then $A_{\diff{f}}$ is a matrix with
\[ [A_{\diff{f}}]_{ij} = [\diff{f}\vec{e}_j]_i = \left[\pd{f}{x^j}\right]_i. \]
This matrix is called the \udef{Jacobian} $J_f$.
\end{definition}

\subsection{Differentiation of a normed algebra}

\begin{proposition}[Leibniz rule]
Let $A$ be a normed algebra and $a,b\in (\R \to A)$ elements that have derivatives. Then
\[ (ab)' = a'b + b'a. \]
\end{proposition}
\begin{proof}
We calculate
\begin{align*}
0 &= 0\cdot a'(t)b'(t) = \lim_{\epsilon \to 0} \epsilon a'(t)b'(t) \\
&= \lim_{\epsilon \to 0} \epsilon \frac{a(t+\epsilon) - a(t)}{\epsilon}\frac{b(t+\epsilon) - b(t)}{\epsilon} \\
&= \lim_{\epsilon \to 0}\frac{a(t+\epsilon)b(t+\epsilon) - a(t+\epsilon)b(t) - a(t)b(t+\epsilon) + a(t)b(t)}{\epsilon} + \frac{a(t)b(t)}{\epsilon} - \frac{a(t)b(t)}{\epsilon} \\
&= \lim_{\epsilon \to 0} \frac{a(t+\epsilon)b(t+\epsilon) - a(t)b(t)}{\epsilon} - \frac{a(t+\epsilon) - a(t)}{\epsilon}b(t) - a(t)\frac{b(t+\epsilon) - b(t)}{\epsilon} \\
&= (ab)' - a'b - ab'.
\end{align*}
\end{proof}

\begin{proposition} \label{derivativeIdempotent}
Let $A$ be an algebra and $p\in A$ such that $p^2 = p$ and $p'$ exists. Then
\begin{enumerate}
\item $p' = pp'+ p'p$;
\item $pp'p = 0$;
\item $(p')^2 = p'pp' + p(p')^2p$;
\item $p^{\prime\prime} = 2(p')^2 + pp^{\prime\prime} + p^{\prime\prime}p$;
\item $pp^{\prime\prime}p = -2p(p')^2p$.
\end{enumerate}
\end{proposition}
\begin{proof}
(1) We calculate $p' = (p^2)' = pp'+ p'p$.

(2) Multiply (1) by $p$ on the left and right.

(3) We calculate $(p')^2 = (pp'+ p'p)(pp'+ p'p) = pp'pp' + pp'p'p + p'ppp' + p'pp'p = 0p + pp'p'p + p'pp' + p'0$.

(4) Take derivative of (1).

(5) Multiply (3) by $p$ on the left and right.
\end{proof}

\section{Taylor expansion}
Radius of convergence

\section{Classification of spaces}
\begin{definition}
Let $X,Y$ be subsets of normed vector spaces and $X$ be open. We call a function $f: X\to Y$
\begin{itemize}
\item \udef{smooth} at $x_0\in V$ if all derivatives of $f$ at $x_0$ exist;
\item \udef{analytic} at $x_0\in V$ if the Taylor series of $f$ at $x_0$ exists and has non-zero radius of convergence.
\end{itemize}
\end{definition}
\begin{lemma}
Let $f: X\to Y$ be a smooth function. Then all derivatives are continuous.
\end{lemma}

\begin{definition}
Let $X,Y$ be subsets of normed vector spaces and $X$ be open.
\begin{itemize}
\item $\cont^r(X,Y)$ is the space of functions in $(X \to Y)$ whose first $r$ derivatives exist and are continuous;
\item $\cont^\infty(X,Y)$ is the space of functions in $(X \to Y)$ that are smooth at all points in $X$;
\item $\cont^\omega(X,Y)$ is the space of functions in $(X \to Y)$ that are analytic at all points in $X$.
\end{itemize}
If $Y = \C$, we write $\cont^r(X), \cont^\infty(X)$ and $\cont^\omega(X)$. We can also use subscripts $_0$ and $_c$ to denote the extra conditions of vanishing at infinity and having compact support.
\end{definition}



\chapter{Non-standard analysis}
\url{http://www.lightandmatter.com/calc/}

\begin{proposition}
Let $f:\R\to\R$ be a real function. Then $f$ is continuous at $x\in\R$ \textup{if and only if} for all infinitesimal $\delta$ there exists an infinitesimal $\epsilon$ such that
\[ f(x+\delta) = f(x) + \epsilon. \]
Alternatively we can state this as
\[ f(x+\delta) \approx f(x) \]
for all infinitesimal $\delta$.
\end{proposition}

Clearly continuity is a requirement for differentiability: if $f(x+\delta) - f(x)$ is not infinitesimal, then $\frac{f(x+\delta) - f(x)}{\delta}$ will not be finite.

\begin{lemma} \label{chainLemma}
Let $f:\R\to\R$ be a real function and $y:\R\to\R$ a continuous function. Assume $f$ differentiable at $y_0\in\im(y)$. Consider $f$ as depending on $y$ and $y$ as depending on $x$. Then
\[ \left.\dod{f}{y}\right|_{y_0} = \st\left(\frac{\Delta_y f}{\Delta y}\right) = \st\left(\frac{\Delta_x f\circ y}{\Delta_x y}\right). \]
\end{lemma}
\begin{proof}
We calculate, setting $y_0 = y(x_0)$ and using continuity of $y$,
\begin{align*}
\st\left(\frac{\Delta_x f\circ y}{\Delta_x y}\right) &= \st\left(\frac{f(y(x_0+\Delta x)) -f(y(x_0))}{y(x_0+\Delta x) - y(x_0)}\right) = \st\left(\frac{f(y(x_0)+\delta) -f(y(x_0))}{y(x_0)+\delta - y(x_0)}\right) \\
&= \st\left(\frac{f(y(x_0)+\delta) -f(y(x_0))}{\delta}\right) = \left.\dod{f}{y}\right|_{y_0}.
\end{align*}
\end{proof}

\begin{proposition}[Chain rule]
Let $y,f$ be real functions, differentiable at points $x_0$ and $y_0=y(x_0)$, respectively. Then
\[ \left.\dod{f}{y}\right|_{y_0} \left.\dod{y}{x}\right|_{x_0} = \left.\dod{f\circ y}{x}\right|_{x_0}. \]
\end{proposition}
\begin{proof}
We calculate, using \ref{chainLemma},
\[ \left.\dod{f}{y}\right|_{y_0} \left.\dod{y}{x}\right|_{x_0} = \st\left(\frac{\Delta_y f}{\Delta y}\frac{\Delta_x y}{\Delta x}\right) = \st\left(\frac{\Delta_x f\circ y}{\Delta_x y}\frac{\Delta_x y}{\Delta x}\right) = \st\left(\frac{\Delta_x f\circ y}{\Delta x}\right) = \left.\dod{f\circ y}{x}\right|_{x_0}. \]
\end{proof}


\chapter{Real functions}
\section{Exponentiation}
\begin{definition}
\udef{base} and \udef{exponent} or \udef{power}.
\end{definition}


\subsection{The square of a number}
TODO
!! square of real nonnegative

\subsection{$n^\text{th}$ roots}
TODO in particular of one.


\subsection{Exponential functions}
Corresponds to power for rational numbers.

Assume $a>0$ and $b>0$, and $x$ and $y$ are any real numbers, then the exponential function has the following properties:
\begin{enumerate}
\item $a^0 = 1$
\item $a^{x+y} = a^x a^y$
\item $a^{-x} = \frac{1}{a^x}$
\item $(a^x)^y = a^{xy}$
\item $(ab)^x = a^x b^x$
\end{enumerate}
Limits:
\begin{enumerate}
\item If $a>1$, then $\lim_{x\to -\infty} a^x = 0$ and $ \lim_{x\to \infty} a^x = \infty$
\item If $0 < a < 1$, then $\lim_{x\to -\infty} a^x = \infty$ and $ \lim_{x\to \infty} a^x = 0$
\end{enumerate}

\section{Logarithms}
The logarithm, denoted
\[ \log_a: [0,\infty[ \to [-\infty, \infty] \]
is defined as the inverse of the exponential function with base $a$. Thus
\[ \log_a(a^x) = x \quad \forall x\in\R \qquad \text{and} \qquad a^{\log_a(x)}=x \forall x>0 \]
If $x>0, y>0, a>0, b>0$ and $a \neq 1, b\neq 1$, then
\begin{enumerate}
\item $\log_a 1 = 0$
\item $\log_a(xy) = \log_a x + \log_a y$
\item $\log_a(\frac{1}{x}) = -\log_a x$
\item $\log_a(x^y) = y\log_a x$
\item $\log_a x = \frac{\log_b x}{\log_b a}$
\end{enumerate}


\section{Polynomial and rational functions}
The \udef{polynomial functions} are a very important class of functions. They can be specified by equations of the form
\[ f(x) = \sum_{i=0}^n a_i x^i \]
where the numbers $a_i$ specified by the index $i$ are called the \udef{coefficients} corresponding to the $i^\text{th}$ power of $x$. We can assume that $a_n$ is not zero (if it is we reduce $n$ until it isn't). That way $n$ gives the \udef{degree} of the polynomial expression.

Polynomial functions with a degree of one are called \udef{linear}.

TODO def rational functions

\subsection{Linear functions}
\subsection{Quadratic functions}
focus, directrix, vertex, axis

root formula

\subsection{Fundamental theorem of algebra}

\section{Absolute value}
TODO + standard definition of distance
\section{Transformations}
TODO + transforming the graph (translation x/y, scaling x/y)

\chapter{Real Analysis}
Dual numbers

Dini theorem

TODO: all versions of homogeneity.

The objects of study in real analysis are real functions, and more generally functions between Euclidean spaces, i.e. real, finite-dimensional, normed vector spaces.

\section{Limits}
In this section infinities can appear because we assume $\N$ and $\R$ are embedded in their Dedekind-MacNeille completions $\overline{\N} = \N\cup\{\infty\}$ and $\overline{\R} = \R\cup\{-\infty, +\infty\}$.

\url{https://en.wikipedia.org/wiki/Interchange_of_limiting_operations}

\subsection{Sequences}
Applying the definition of convergent sequences to sequences in $\R$ gives the so-called $\varepsilon-n_0$ criterion for convergence:
\begin{proposition}
Let $(x_n)$ be a real sequence. Then $(x_n)$ converges to $L$ \textup{if and only if}
\[ \forall \varepsilon> 0: \exists n_0\in\N: \forall n\in\N: n\geq n_0 \implies |x_n-L|<\varepsilon. \]
\end{proposition}
The definition for divergence to $\pm\infty$ is identical.

TODO Bolzano-Weierstrass

\subsubsection{Examples of sequences}
\begin{proposition}
Let $p\in\R$. Then
\[ \lim_{n\to\infty} n^p = \begin{cases}
+\infty & (p>0) \\
1 & (p=0) \\
0 & (p<0)
\end{cases}. \]
\end{proposition}

\begin{proposition}
Let $r\in\R$. Then
\[ \lim_{n\to\infty} r^n = \begin{cases}
+\infty & (r>1) \\
1 & (r=1) \\
0 & (-1<r<1) \\
\text{does not exist} & (r\leq -1)
\end{cases}. \]
\end{proposition}

\subsection{Series}
TODO:move up!!
\begin{theorem}[Tannery's theorem] \label{tannery}
Let $s_n = \sum_{k=0}^\infty a_{n,k}$ be a convergent series for each $n$ such that $\lim_{n\to \infty}a_{n,k}$ converges to $a_k$ for all $n$. If there exists a sequence $M_k$ such that $|a_{n,k}|\leq M_k$ for all $n,k$ and $\sum_{k=0}^\infty M_k<\infty$, then
\[ \lim_{n\to\infty} s_n = \lim_{n\to\infty}\sum_{k=0}^\infty a_{n,k} = \sum_{k=0}^\infty\lim_{n\to\infty} a_{n,k} = \sum_{k=0}^\infty a_k.   \]
\end{theorem}
\begin{proof}
Choose an arbitrary $\varepsilon>0$. For any $n,N\in\N$ we can write
\[ \left|s_n - \sum_{k=0}^\infty a_k\right| \leq \sum_{k=0}^\infty\left|a_{n,k} -  a_k\right|\leq \sum_{k=0}^N\left|a_{n,k} -  a_k\right| + 2\sum_{k>N}M_k \leq N\max_{k<N}\left|a_{n,k} -  a_k\right| + 2\sum_{k>N}M_k. \]
So we aim to find some $N_0$ such that $2\sum_{k>N}M_k \leq \varepsilon/2$ for all $N\geq N_0$, which of course we can. Then we choose an $n_0$, in function of this $N_0$ and $\varepsilon$, such that $\max_{k<N_0}\left|a_{n,k} -  a_k\right|\leq \varepsilon/(2N_0)$ for all $n\geq n_0$. It is clear we can do so for each $k$ separately, but there are only finitely many $k$s so we take the largest $n_0$. Then
\[ \left|s_n - \sum_{k=0}^\infty a_k\right| \leq \varepsilon \qquad \text{for all $n\geq n_0$, implying the limit is zero.} \]
\end{proof}
\url{https://www.coloradomesa.edu/math-stat/documents/JohnGillresearchnoteTanneryTheorem.pdf}

\subsection{Real functions}
Applying the definition of limits to functions in $\R$ gives the so-called $\varepsilon-\delta$ definition of limits:
\begin{proposition}
Let $f:A\subseteq\R\to\R$ be a real function and $p$ a limit point o $A$. Then $L\in \R$ is the limit of $f(x)$ as $x$ approaches $p$ \textup{if and only if}
\[ \forall \varepsilon>0: \exists \delta> 0: \forall x\in A: |x-p| < \delta \implies |f(x)-L|<\varepsilon. \]
\end{proposition}
TODO: criteria for limits involving infinities.

In addition we can define some special types of limits.
\begin{definition}
Let $f:A\subseteq\R\to\R$ be a function and $p$ a limit point of $A$. Then
\begin{itemize}
\item the \udef{left limit} of $f(x)$ as $x$ approaches $p$ is the limit of $f|_{A\cap]-\infty,p]}$ as $x\to p$, denoted
\[ \lim_{x\overset{<}{\to} p}f(x) \qquad \text{or} \qquad \lim_{x\to p^-}f(x); \]
\item the \udef{right limit} of $f(x)$ as $x$ approaches $p$ is the limit of $f|_{A\cap[p,+\infty[}$ as $x\to p$, denoted
\[ \lim_{x\overset{>}{\to} p}f(x) \qquad \text{or} \qquad \lim_{x\to p^+}f(x). \]
\end{itemize}
\end{definition}

\begin{lemma}
Let $f:A\subseteq\R\to\R$ be a function and $p$ a limit point of both $A\cap]-\infty,p]$ and $A\cap[p,+\infty[$. Then $\lim_{x\to p} f(x)$ exists if the left and right limit exist and are equal to each other. In this case
\[ \lim_{x\to p} f(x) = \lim_{x\to p^-} f(x) = \lim_{x\to p^+} f(x). \]
\end{lemma}

\subsection{Properties of limits}
TODO: for functions $X\to \R$.

We enumerate some of the properties for limits here. These properties are valid for all types of limit of real functions, so long as all limits in the equation are limits to the same point (or infinity, TODO elaborate!).  In fact the properties also hold true for sequences of real numbers.
\begin{itemize}
\item Taking the limit is a linear operation:
\[\lim  (f(x) + g(x)) = \lim  f(x) + \lim g(x)\]
\[\lim  (a\cdot f(x)) = a\cdot \lim  f(x) \qquad \forall a \in \R\]
\item The limit of the product:
\[ \lim (f(x)\cdot g(x)) = \left(\lim  f(x)\right) \cdot \left(\lim  g(x)\right) \]
\item The limit of the quotient (assuming $\lim  g(x) \neq 0$):
\[ \lim \left(\frac{f(x)}{g(x)}\right) = \frac{\lim  f(x)}{\lim  g(x)}  \]
\item The limit of a power (with $m$ an integer and $n$ a positive integer):
\[ \lim  [f(x)]^{m/n} = \left(\lim  f(x)\right)^{m/n} \]
\item Partial order is preserved. Assume $f(x) \leq g(x)$ on some interval containing $x_0$. Then
\[ \lim f(x) \leq \lim g(x) \]
The same is not true for the strict order $<$!
\end{itemize}

\subsubsection{The squeeze theorem}


\section{Continuity}
\begin{proposition}
Let $f:A\subseteq\R\to\R$ be a function and $p\in A\cap A'$, where $A'$ is the set of limit points of $A$. Then
\[ \text{$f$ is continuous at $p$} \quad\iff\quad \lim_{x\to p}f(x) = f(x). \]
\end{proposition}

\subsection{Left and right continuity}
TODO

\subsection{Discontinuities}
If a function $f:A\subseteq\R\to\R$ is not continuous at $p$, then $p$ is a limit point of $A$ by \ref{continuityAtIsolatedPoint} and \ref{notLimitPointSingletonOpen}.
\begin{definition}
Let $f:A\subseteq\R\to\R$ be a function and $p\in A$ such that $f$ is not continuous at $p$. Then
\begin{itemize}
\item if $\lim_{x\to p}f(x)$ exists and is finite, we call $p$ a \udef{removable discontinuity};
\item if both $\lim_{x\to p^-}f(x)$ and $\lim_{x\to p^+}f(x)$ exist and are finite, but are different, we call $p$ a \udef{jump discontinuity};
\item if either $\lim_{x\to p^-}f(x)$ or $\lim_{x\to p^+}f(x)$ do not exist, we call $p$ an \udef{essential discontinuity}.
\end{itemize}
Removable and jump discontinuities are also called \udef{discontinuities of the first kind}. Essential discontinuities are also called \udef{discontinuities of the second kind}.
\end{definition}
TODO: should discontinuities of type 1/x be considered essential?

\begin{proposition} \label{monotoneDiscontinuities}
Let $f$ be a monotone real-valued function on an interval $I$. Then all discontinuities are jump discontinuities.
\end{proposition}
\begin{theorem}[Darboux-Froda] \label{DarbouxFroda}
Let $f$ be a monotone real-valued function on an interval $I$. Then the set of discontinuities is at most countable.
\end{theorem}

TODO: intervals must be closed / open?? In \ref{monotoneDiscontinuities} and \ref{DarbouxFroda}.

\section{Functions on closed, finite intervals}
\subsection{Min-max theorem} $f(p) \leq f(x) \leq f(q)$

\subsection{Intermediate value theorem}.

\section{Derivatives}

\section{Stone-Weierstrass}
\begin{theorem}[Stone-Weierstrass] \label{StoneWeierstrass}
Let $X$ be a compact Hausdorff space. Let $A\subseteq \mathcal{C}(X)$ be a unital $*$-subalgebra. Suppose that $A$ separates points, i.e. for all $x\neq y$ in $X$ there exists $f\in A$ with $f(x) \neq f(y)$. Then $A$ is dense in $\mathcal{X}$ with respect to $\norm{\cdot}_\infty$.
\end{theorem}



\chapter{Measure theory}
TODO (bounded) finitely additive signed measures form Riesz spaces.
\section{Pre-measures}
\begin{definition}
Let $\mathcal{S}$ be a semi-ring on a set $\Omega$. A \udef{(positive) pre-measure} on $\mathcal{S}$ is a map $\mu: \mathcal{S} \to [0,\infty]$ satisfying
\begin{itemize}
\item $\mu(\emptyset) = 0$;
\item $\mu$ is \udef{(finitely) additive}: if $A,B \in \mathcal{S}$ are disjoint, then
\[ \mu(A\cup B)) = \mu(A)+\mu(B). \]
\end{itemize}
\end{definition}
\begin{proposition}
Let $\mathcal{S}$ be a semi-ring on $\Omega$. Every pre-measure $\mu_\mathcal{S}$ on $\mathcal{S}$ extends uniquely to a pre-measure $\mu$ on $\mathfrak{R}\{\mathcal{S}\}$, the ring generated by $\mathcal{S}$.
\end{proposition}

\begin{lemma} \label{emptysetNullset}
If there exists an $A\in \mathcal{S}$ such that $\mu(A) < \infty$, then the requirement $\mu(\emptyset) = 0$ is redundant.
\end{lemma}
\begin{proof}
We calculate $\mu(A) = \mu(A \cup \emptyset) = \mu(A) + \mu(\emptyset)$, so $\mu(\emptyset) = 0$.
\end{proof}


\subsection{Outer measures}
\begin{definition}
Let $\Omega$ be a non-empty set. An \udef{outer measure} on $\Omega$ is a map $\nu: \powerset(\Omega)\to [0,\infty]$ satisfying
\begin{itemize}
\item $\nu(\emptyset) = 0$;
\item if $E\subseteq F\subseteq \Omega$, then $\nu(E)\leq \nu(F)$;
\item $\nu$ is \udef{$\sigma$-subadditive}: for every sequence $(E_n)$ of pairwise disjoint subsets of $\Omega$, we have
\[ \nu\left(\biguplus_{n\in\N}E_n\right) \leq \sum_{n\in\N}\nu(E_n). \]
\end{itemize}
\end{definition}
It is important to note that outer measures are not in general measures or pre-measures.

\begin{proposition}
Let $\mathcal{R}$ be a ring with universe set $\Omega$ and $\mu$ a measure on $\mathcal{R}$. Then
\[ \mu^*: \mathcal{R}\to [0,\infty]: E\mapsto \inf\setbuilder{\sum^\infty_{n=1}\mu(E_n)}{(E_n)_{n\in\N}\subseteq \mathcal{R}\;\text{with}\;E\subseteq \bigcup_{n\in\N}E_n} \]
with the convention that $\inf\emptyset = \infty$, defines an outer measure on $\Omega$.
\end{proposition}

\begin{definition}
Let $\nu$ be an outer measure on a set $\Omega$. We say that a set $E\subseteq\Omega$ is $\nu$-measurable, if
\[ \forall A\subseteq \Omega: \; \nu(A) = \nu(A\cap E)+\nu(A\setminus E). \]
\end{definition}

\section{$\sigma$-algebras and measurable spaces}
\begin{definition}
Let $\Omega$ be a non-empty set. A \udef{$\sigma$-algebra} $\mathcal{A}$ on $\Omega$ is a subset of $\powerset(\Omega)$ such that
\begin{itemize}
\item $\emptyset \in \mathcal{A}$;
\item if $E\in \mathcal{A}$, then $E^c\in \mathcal{A}$;
\item for any sequence $(E_n)_{n\in\N}$ in $\mathcal{A}$, one has $\bigcup_{n\in\N}E_n \in \mathcal{A}$.
\end{itemize}
The elements of a $\sigma$-algebra are called \udef{events}.

A pair $(\Omega, \mathcal{A})$ of a set and a $\sigma$-algebra on the set is called a \udef{measurable space}.
\end{definition}

\begin{example}
For any non-empty set $\Omega$, the following are $\sigma$-algebras:
\begin{itemize}
\item $\{\emptyset, \Omega\}$;
\item $\powerset(\Omega)$.
\end{itemize}
\end{example}

\begin{lemma}
A $\sigma$-algebra is closed w.r.t. all countable operations involving complements, unions, intersections and differences.
\end{lemma}
\begin{proof}
All can be expressed in terms of unions and complements:
\[ E\cap F = (E^c\cup F^c)^c\qquad E\setminus F = (E^c\cup F)^c. \]
\end{proof}

\begin{lemma}
Let $\Omega$ be a non-empty set. Let $\{\mathcal{A}_i\}_{i\in I}$ be an arbitrary family of $\sigma$-algebras. Then the intersection $\bigcap_{i\in I}\mathcal{A}_i$ is a $\sigma$-algebra.
\end{lemma}
\begin{corollary}
Let $\mathcal{S}\subset \powerset(\Omega)$. Then there exists a smallest $\sigma$-algebra containing $\mathcal{S}$.
\end{corollary}
This $\sigma$-algebra is called the $\sigma$-algebra generated by $\mathcal{S}$.

\subsection{Measurable functions}
\begin{definition}
Let $(\Omega_1, \mathcal{A}_1)$ and $(\Omega_2, \mathcal{A}_2)$ be measurable spaces.

A function $f:\Omega_1 \to \Omega_2$ is called \udef{measurable} if
\[ \forall E\in\mathcal{A}_2: f^{-1}[E] \in\mathcal{A}_1. \]
\end{definition}
Suppose we are moving around in $\Omega_1$ and tracking the output of the function in $\Omega_2$. We would like to be able to explore the contents of an event in $\Omega_2$ from within an event in $\Omega_1$.

\begin{proposition} \label{measurableFromGeneratingSet}
Let $(\Omega_1, \mathcal{A}_1)$ and $(\Omega_2, \mathcal{A}_2)$ be measurable spaces and $\mathcal{A}_2$ is generated by $\mathcal{S}$.

Then $f: \Omega_1\to \Omega_2$ is measurable \textup{if and only if} $\forall S\in\mathcal{S}: f^{-1}[S] \in \mathcal{A}_1$.
\end{proposition}

\subsection{Borel-$\sigma$-algebras}
\begin{definition}
Let $(X,\mathcal{T})$ be a topological space. The $\sigma$-algebra on $X$ generated by $\mathcal{T}$ is called the \udef{Borel-$\sigma$-algebra} of $(X,\mathcal{T})$. The measurable space consisting of $X$ equipped with the Borel-$\sigma$-algebra is called the \udef{Borel-measurable space}.
\end{definition}

\begin{lemma}
Let $(X,\mathcal{T})$ be a topological space. If $\mathcal{T}$ has a countable basis $\mathcal{B}$, then the Borel-$\sigma$-algebra is generated by the basis $\mathcal{B}$.
\end{lemma}

TODO:
\begin{align*}
\mu(B) &= \sup\setbuilder{\mu(C)}{C\subseteq B, \; \text{$C$ compact}} \\
&= \inf\setbuilder{\mu(O)}{B\subseteq O, \; \text{$O$ open}}.
\end{align*}
cfr compact open topology??

\begin{lemma}
Every continuous function between topological spaces is a measurable function between Borel-measurable spaces.
\end{lemma}

\begin{proposition} \label{pointWiseConvergenceMeasurable}
Let $(\Omega,\mathcal{A})$ be a measurable space and $(Y,d)$ a metric space. We equip $Y$ with the Borel-$\sigma$-algebra associated to the metric topology.

Suppose that a sequence of measurable functions $f_n : \Omega \to Y$ converges pointwise to a function $f:\Omega\to Y$. Then $f$ is measurable.
\end{proposition}
\begin{proof}
By \ref{measurableFromGeneratingSet} it is enough to show that for all closed sets $C\subset Y$, the inverse image $f^{-1}[C]$ is in $\mathcal{A}_1$.

We now use that in a metric space the closure of any space can be written as the countable intersection of a sequence of open sets, by TODOref. So
\[ C = \overline{C} = \bigcap_{k\in\N}O_k. \]
Now we combine this with the fact that $f_n(x)\to f(x)$ for every $x\in\Omega$, and that every metric space is sequential, to obtain the equivalences
\begin{align*}
x\in f^{-1}[C] &\iff f(x)\in C \\
&\iff \forall k\in\N: f(x)\in O_k\\
&\iff \forall k\in\N: \exists n_0\in\N: \forall n\geq n_0: f_n(x)\in O_n \\
&\iff x\in \bigcap_{k\in\N}\bigcup_{n_0\in\N}\bigcap_{n\geq n_0}f^{-1}_n[O_n]
\end{align*}
As all $O_n$ are open, they are in $\mathcal{A}_2$ and thus $f^{-1}_n[O_n]\in\mathcal{A}_1$. Finally $\sigma$-algebras are closed under countable unions and intersections.
\end{proof}
TODO restate proof: prove that liminf and limsup are measurable.

\section{Measure spaces}
\begin{definition}
Let $(\Omega,\mathcal{A})$ be a measurable space. A \udef{(positive) measure} on $(\Omega,\mathcal{A})$ is a map $\mu: \mathcal{A} \to [0,\infty]$ satisfying
\begin{itemize}
\item $\mu(\emptyset) = 0$;
\item $\mu$ is \udef{$\sigma$-additive}: for every sequence $(E_n)$ in $\mathcal{A}$ of pairwise disjoint sets, we have
\[ \mu\left(\biguplus_{n\in\N}E_n\right) = \sum_{n\in\N}\mu(E_n). \]
\end{itemize}
The triple $(\Omega, \mathcal{A}, \mu)$ is called a \udef{measure space}. If
\begin{itemize}
\item $\mu(\Omega) < \infty$, we call $\mu$ a \udef{finite measure};
\item $\mu(\Omega) = 1$, we call $\mu$ a \udef{probability measure} and we call $(\Omega, \mathcal{A}, \mu)$ a \udef{probability space};
\item there exists a sequence $(E_n)$ in $\mathcal{A}$ with $\Omega = \bigcup_{n\in\N}E_n$ and $\mu(E_n)<\infty$ for all $n\in\N$, then we call $\mu$ \udef{$\sigma$-finite}.
\end{itemize}
\end{definition}
We typically consider $\mathcal{A}$ as ordered by inclusion.

\begin{lemma}
Finite additivity is implied by $\sigma$-additivity.
\end{lemma}
\begin{proof}
Let $E_1,E_2\in\mathcal{A}$ be disjoint sets. Then
\begin{align*}
\mu(E_1\uplus E_2) &= \mu(E_1\uplus E_2 \uplus \emptyset \uplus \emptyset \uplus \ldots) \\
&= \mu(E_1) + \mu(E_2) + \mu(\emptyset) + \mu(\emptyset) + \ldots \\
&= \mu(E_1) + \mu(E_2).
\end{align*}
\end{proof}

As with pre-measures, if there exists an event $E$ such that $\mu(E)< \infty$, then $\mu(\emptyset) = 0$ is redundant. See \ref{emptysetNullset}.

\begin{lemma} \label{submeasurespace}
Let $(\Omega,\mathcal{A},\mu)$ be a measure space and let $E\in\mathcal{E}$. Then $\mathcal{A}' = \powerset(E)\cap\mathcal{A}$ is a $\sigma$-algebra on $E$ and $(E,\mathcal{A}',\mu|_{\mathcal{A}'})$is a measure space.
\end{lemma}
\begin{proof}
We need to show that $\mathcal{A}'$ is a $\sigma$-algebra. Now complements are w.r.t. $E$. These are still in the $\sigma$-algebra because $E\setminus A = \Omega\setminus((\Omega\setminus E)\cup A)$.
\end{proof}

\begin{lemma} \label{measuresPositiveLinear}
Let $(\Omega,\mathcal{A})$ be a measurable space. Positive linear combinations of measures are measures: let $\mu_1,\mu_2$ be measures on $(\Omega,\mathcal{A})$ and $0 \leq c \in \R$. Then $c\mu_1 + \mu_2$ is a measure on $(\Omega,\mathcal{A})$.
\end{lemma}

\begin{example}
\begin{itemize}
\item For any non-empty set $\Omega$ define
\[ \mu: \powerset(\Omega)\to [0,\infty]: E\mapsto \begin{cases}
\#(E) & \text{($E$ is finite)} \\
\infty & \text{($E$ is infinite)}
\end{cases}. \]
Then $(\Omega, \powerset(\Omega),\mu)$ is a measure space and $\mu$ is called the \udef{counting measure}.
\item Let $(\Omega,\mathcal{A})$ be a measurable space and $x\in\Omega$ and define
\[ \delta_x: \mathcal{A}\to [0,\infty]: E\mapsto \begin{cases}
1 & (x\in E) \\ 0 & (x\notin E)
\end{cases}. \]
Then $(\Omega,\mathcal{A},\delta_x)$ is a measure space and $\delta_x$ is called a \udef{Dirac measure}.
\end{itemize}
\end{example}

\begin{proposition}[Pushforward measure] \label{pushforwardMeasure}
Let $\sSet{\Omega_1, \mathcal{A}, \mu}$ be a measure space and $\sSet{Omega_2, \mathcal{B}}$ a measurable space. Let $f: \Omega_1 \to \Omega_2$ be a measurable function. Then
\[ \nu = \mu\circ f^{-1}|_{\mathcal{B}}: \mathcal{B}\to [0,\infty]: B \mapsto \mu(f^{-1}[B]) \]
is a measure on $\sSet{Omega_2, \mathcal{B}}$.
\end{proposition}
\begin{proof}
Clearly this is well-defined due to $f$ being measurable. We have
\[ \nu(\emptyset) = \mu(f^{-1}[\emptyset])  = \mu(\emptyset) = 0 \]
for every sequence $(E_n)$ in $\mathcal{B}$ of pairwise disjoint sets, we have
\[ \nu\left(\biguplus_{n\in\N}E_n\right) = \mu\left(f^{-1}\left[\biguplus_{n\in\N}E_n\right]\right) = \mu\biguplus_{n\in\N} f^{-1}[E_n]  = \sum_{n\in\N}\mu(f^{-1}[E_n]) = \sum_{n\in\N}\nu(E_n). \]
\end{proof}

\begin{proposition} \label{measures}
Let $(\Omega, \mathcal{A}, \mu)$ be a measure space. Then
\begin{enumerate}
\item $\mu$ is order-preserving (if $\mathcal{A}\subset \powerset(\Omega)$ is ordered by inclusion);
\item $\mu$ is $\sigma$-subadditive, i.e. for any sequence $(E_n)$ in $\mathcal{A}$, we have
\[ \mu\left(\bigcup_{n\in\N}E_n\right) \leq \sum_{n\in\N}\mu(E_n); \]
\item if $(E_n)$ is a converging sequence in $\mathcal{A}$, then $(\mu(E_n))$ also converges with
\[ \lim_{n\to\infty}\mu(E_n) = \mu\left(\lim_{n\to\infty}E_n\right). \]
\end{enumerate}
\end{proposition}
\begin{proof}
TODO + upper  and lower convergence + non-sequential.
\end{proof}
\begin{corollary}
Let $\seq{\Omega, \mathcal{A}, \mu}$ be a measure space. Then
\begin{enumerate}
\item if $(E_n)$ is an increasing sequence in $\mathcal{A}$, then $\mu(E_n)$ is also increasing and
\[ \mu\left(\bigcup_{n\in\N}E_n\right) = \sup_{n\in\N}\mu(E_n) = \lim_{n\to\infty}\mu(E_n); \]
\item if $(E_n)$ is a decreasing sequence in $\mathcal{A}$ and $\mu(E_1)<\infty$, then $\mu(E_n)$ is also decreasing and
\[ \mu\left(\bigcap_{n\in\N}E_n\right) = \inf_{n\in\N}\mu(E_n) = \lim_{n\to\infty}\mu(E_n). \]
\end{enumerate}
\end{corollary}
\begin{corollary}
Let $\mu,\nu$ be measures defined on the same measure space $\seq{\Omega, \mathcal{A}}$. Assume $\mathcal{A} = \sigma\{\mathcal{F}\}$ for some $\pi$-system $\mathcal{F}$ which contains $\Omega$. Then $\mu = \nu$ \textup{if and only if} $\mu(A) = \nu(A)$ for all $A\in\mathcal{F}$.
\end{corollary}
\begin{proof}
Define
\[ \mathcal{E} = \setbuilder{A\in\mathcal{A}}{\mu(A) = \nu(A)}. \]
By the $\pi-\lambda$ theorem \ref{piLambdaTheorem} it is enough to show that $\mathcal{E}$ is a Dynkin system.
\begin{itemize}
\item By assumption $\Omega\in \mathcal{E}$.
\item Assume $A\subset B$ are sets in $\mathcal{E}$. Then
\[ \mu(B) = \mu(A \cup (B\setminus A)) = \mu(A) + \mu(B\setminus A) \]
which implies
\[ \mu(B\setminus A) = \mu(B) - \mu(A) = \nu(B) - \nu(A) = \nu(B\setminus A). \]
\item Let $\seq{A_i}$ be a monotonically increasing family of sets in $\mathcal{E}$. Then
\[ \mu\left(\bigcup_{i\in\N}A_i\right) = \sup_{i\in\N}\mu(A_i) = \sup_{i\in\N}\nu(A_i) = \nu\left(\bigcup_{i\in\N}A_i\right). \]
\end{itemize}
\end{proof}

\subsection{Null sets and completeness}
\begin{definition}
Let $\sSet{\Omega, \mathcal{A}, \mu}$ be a measure space. A set $A\subseteq \Omega$ is a \udef{null set} if there exists a measurable set $B\in\mathcal{A}$ such that $A\subseteq B$ and $\mu(B) = 0$.

A measure space is called \udef{complete} if very null set is
measurable.

A proposition $P(x)$ referencing some $x\in\Omega$ is said to be true \udef{almost everywhere} (or a.e.) if $\setbuilder{x\in\Omega}{\text{$P(x)$ is false}}$ is a null set.
\end{definition}

\begin{lemma}
Let $\sSet{\Omega, \mathcal{A}, \mu}$ be a measure space. If $A$ is a measurable null set, then $\mu(A) = 0$.
\end{lemma}
\begin{proof}
Let $B\supseteq A$ be a measurable set with $\mu(B) = 0$. Then $\mu(A) \leq \mu(B) = 0$.
\end{proof}

\begin{lemma}
Let $\seq{\Omega, \mathcal{A}, \mu}$ be a measure space and $\seq{A_i}$ a sequence of measurable null sets. Then
\[ \mu\left(\bigcup_{i\in\N}A_i\right) = 0. \]
\end{lemma}
\begin{proof}
This follows by $\sigma$-sub-additivity:
\[ \mu\left(\bigcup_{i\in\N}A_i\right) \leq \sum_{i=1}^\infty \mu(A_i) = 0. \]
\end{proof}

\begin{proposition}
Every measure space can be completed (TODO)
\end{proposition}

\subsection{Decompositions of measures}

\section{Carathéodory's construction of measures}

\chapter{Integration theory}
\section{Riemann integration}
See also Reed/Simon and \ref{BLT}.
\subsection{Riemann-Stieltjes}

\section{Lebesgue integration}
\url{https://math.stackexchange.com/questions/2218114/theoretical-advantages-of-lebesgue-integration}
\url{https://math.stackexchange.com/questions/3202630/what-are-the-advantages-of-the-riemann-vs-lebesgue-integral}
\subsection{Simple functions}
TODO order on function spaces

\begin{definition}
Let $\Omega$ be a set. A \udef{simple function} (or \udef{step function}) on $\Omega$ is a function with a range of finite cardinality.

Let $B$ be a set. We denote the subset of simple functions in $(\Omega\to B)$ by $\SF(\Omega,B)$
\end{definition}

\begin{lemma}
Let $s:\Omega\to Y$ be a simple function into a vector space. Then $s$ can be written as
\[ s(x) = \sum_{\lambda \in s[\Omega]}\lambda\cdot \chi_{s^{-1}[\lambda]}(x). \]
Additionally for some $k\in\N$ we can write $s[\Omega] = \bigcup_{i=1}^k\{\lambda_i\}$ and $A_i = s^{-1}[\lambda_i]$. Then the $A_i$ form a partition of $\Omega$ and
\[ s(x) = \sum_{i=1}^k\lambda_i\cdot \chi_{A_i}(x). \]
This is known as the \udef{canonical form} of $s$. Conversely every function of this form is a simple function.
\end{lemma}

\begin{lemma}
Let $(\Omega, \mathcal{A})$ and $(Y, \mathcal{B})$ be measurable spaces such that $\mathcal{B}$ contains all singleton sets. Let $s:\Omega\to Y$ be a simple function.

Then $s$ is measurable \textup{if and only if} $s^{-1}[\lambda]\in\mathcal{A}$ for all $\lambda\in s[\Omega]$.

This is equivalent to saying the partition $\{A_i\}_{i=1}^k$ is a subset of $\mathcal{A}$.
\end{lemma}
\begin{proof}
The direction $\boxed{\Rightarrow}$ is clear, since $\mathcal{B}$ is assumed to contain all singleton sets.

For the $\boxed{\Leftarrow}$ direction, let $B\in \mathcal{B}$. Then
\[ s^{-1}[B] = \bigcup_{\lambda\in s[\Omega]}s^{-1}[B\cap \{\lambda\}], \]
which is a finite union of measurable sets.
\end{proof}

\begin{lemma}
Let $\Omega$ be a set, $G$ an abelian group and $s,t\in\SF(\Omega,G)$. If $s$ and $t$ have canonical forms
\[ s = \sum_{i=1}^k a_i\cdot\chi_{A_i} \qquad\text{and}\qquad t = \sum_{j=1}^l b_j\cdot\chi_{B_j}, \]
then $s+t$ has canonical form
\[ s+t = \sum_{i=1}^k\sum_{j=1}^l (a_i+b_j)\chi_{A_i}\chi_{B_j} = \sum_{i,j \in (1:k)\times(1:l)}(a_i+b_j)\chi_{A_i\cap B_j}. \]
\end{lemma}

\begin{proposition}
Let $(\Omega,\mathcal{A})$ be a measurable set. Every measurable function $f:\Omega\to \R$ is the pointwise limit of a sequence of simple functions.
\end{proposition}
TODO: generalise??
\begin{proof}
\url{https://proofwiki.org/wiki/Measurable_Function_is_Pointwise_Limit_of_Simple_Functions}
\end{proof}

\subsubsection{Integration of simple functions}
\begin{definition}
Let $(\Omega, \mathcal{A}, \mu)$ be a measure space and $Y$ a vector space. Let
\[ s:\Omega \to Y: x\mapsto \sum_{\lambda \in s[\omega]}\lambda\cdot \chi_{s^{-1}[\lambda]}(x) = \sum_{i=1}^k\lambda_i\cdot\chi_{A_i}(x)  \]
be a measurable simple function. We define the \udef{integral} of $s$ over $\Omega$ w.r.t. $\mu$ as
\[ \int_\Omega s\diff{\mu} \defeq \sum_{\lambda \in s[\omega]}\lambda\cdot \mu(s^{-1}[\lambda]) = \sum_{i=1}^k\mu(A_i)\cdot\lambda_j. \]
Using the convention that $0\times \infty = 0$. (TODO: clarify)

We call $s$ \udef{integrable} if the integral is finite (TODO clarify + below).
\end{definition}

The integral can be seen as a map $\SF(\Omega,Y)\to \overline{Y}$, where $\overline{Y}$ is the Dedekind-MacNeille completion (TODO!).

For any $E\in\mathcal{A}$ we also define the integral over $E$ as the map
\[ \int_E\diff{\mu}: \SF(\Omega,Y) \to \overline{Y}: s\mapsto \int_E s|_E \diff{\mu|_E}. \]
Where the last integral is taken over the measure space $(E,\mathcal{A}',\mu|_{\mathcal{A}'})$ as defined in \ref{submeasurespace}.

\begin{proposition} \label{integrationLinear} \label{integrationOrderPreserving}
Let $(\Omega, \mathcal{A}, \mu)$ be a measure space and $Y$ a vector space over $\F$. Then
\begin{enumerate}
\item the integral is linear: $\forall c\in \F$ and $\forall s,t\in\SF(\Omega, Y)$:
\[ \int_\Omega (c\cdot s + t)\diff{\mu} = c\cdot \int_\Omega s\diff{\mu} + \int_\Omega t\diff{\mu}. \]
\end{enumerate}
If $Y$ is a normed space, then
\begin{enumerate} \setcounter{enumi}{1}
\item for all $s\in\SF(\Omega, Y)$:
\[ \int_\Omega \norm{s}\diff{\mu} \leq \norm{\int_\Omega s \diff{\mu}}; \]
\item if $E_1\subseteq E_2$ are events in $\mathcal{A}$, then
\[ \int_{E_1}\norm{s}\diff{\mu} \leq \int_{E_2}\norm{s}\diff{\mu}. \]
\end{enumerate} \setcounter{enumi}{3}
If $Y$ is an ordered space, then
\begin{enumerate}
\item if $s(x)\leq t(x)$ for all $x\in\Omega$, then
\[ \int_\Omega s\diff{\mu} \leq \int_\Omega t\diff{\mu}. \]
\end{enumerate}
\end{proposition}
\begin{proof}
(1) Let $s,t$ have canonical forms
\[ s = \sum_{i=1}^k a_i\cdot\chi_{A_i} \qquad\text{and}\qquad t = \sum_{i=1}^k b_i\cdot\chi_{B_i}. \]
Then we calculate
\begin{align*}
\int_\Omega (c\cdot s + t)\diff{\mu} &= \sum_{i=1}^k\sum_{j=1}^l\mu(A_i\cap B_j)\cdot(c a_i + b_j) \\
&= c\sum_{i=1}^k\sum_{j=1}^l\mu(A_i\cap B_j)\cdot a_i +  \sum_{i=1}^k\sum_{j=1}^l\mu(A_i\cap B_j) \cdot b_j \\
&= c\sum_{i=1}^k\mu(A_i)\cdot a_i +  \sum_{j=1}^l\mu(B_j) \cdot b_j \\
&= c\int_\Omega s\diff{\mu} + \int_\Omega t\diff{\mu}.
\end{align*}

The property (2) is just the triangle inequality. The other properties can be proven in a similar fashion to (1).
\end{proof}

\begin{proposition}
Let $(\Omega, \mathcal{A}, \mu)$ be a measure space and $s:\Omega\to [0,+\infty[$ a measurable simple function. The map
\[ \nu:\mathcal{A}\to [0,+\infty]: E\mapsto \int_E s\diff{\mu} = \int_\Omega s\cdot \chi_E\diff{\mu} \]
defines a measure.
\end{proposition}
\begin{proof}
From the positive linearity of both measures and integration (\ref{measuresPositiveLinear}, \ref{integrationLinear}) it is enough to consider $s = \chi_A$ for some $A\in\mathcal{A}$. In this case
\[ \nu(E) = \int_\Omega\chi_A\chi_B \diff{\mu} = \int_\Omega\chi_{A\cap B} \diff{\mu} = \mu(A\cap E). \]
It is clear that $\nu(\emptyset) = 0$. For $\sigma$-additivity, let $(E_n)$ be a sequence of disjoint sets in $\mathcal{A}$ and calculate
\[ \nu\left(\biguplus_{n\in\N}E_n\right) = \mu\left(A\cap \biguplus_{n\in\N}E_n\right) = \mu\left(\biguplus_{n\in\N}(A\cap E_n)\right) = \sum_{n\in\N}\mu(A\cap E_n) = \sum_{n\in\N}\nu(E_n). \]
\end{proof}
\begin{corollary} \label{integralContinuousInDomain}
Let $(\Omega, \mathcal{A}, \mu)$ be a measure space, $s:\Omega\to[0,+\infty]$a measurable simple function and $(E_n)$ a converging sequence in $\mathcal{A}$. Then
\[ \lim_{n\to\infty}\int_{E_n}s\diff{\mu} = \int_{\lim_{n\to\infty} E_n}s\diff{\mu}. \]
\end{corollary}
\begin{proof}
Define the measure $\nu: E\mapsto \int_E s\diff{\mu}$. Then by \ref{measures}
\[ \lim_{n\to\infty}\int_{E_n}s\diff{\mu} = \lim_{n\to\infty}\nu(E_n) = \nu(\lim_{n\to\infty}E_n) =\int_{\lim_{n\to\infty}E_n}s\diff{\mu}. \]
\end{proof}

\subsection{Positive real functions}
\begin{definition}
Let $(\Omega, \mathcal{A}, \mu)$ be a measure space and let $f:\Omega\to[0,+\infty]$ be a positive measurable function. Define

We define the \udef{Lebesgue integral} of $f$ on $\Omega$ w.r.t. $\mu$ as
\[ \int_\Omega f \diff{\mu} \defeq \sup\setbuilder{\int_\Omega s \diff{\mu}}{s\in\SF(\Omega, [0,+\infty[)\;\land\; s\leq f}. \]
We call $f$ \udef{integrable} when $\int_\Omega f \diff{\mu} < \infty$.
\end{definition}
For simple functions this definition corresponds to the previous one by \ref{integrationOrderPreserving}.

This definition a priori makes sense even when $f$ is not assumed to be measurable. However the integral has undesirable properties in this case, such as not being additive.

\begin{example}
If $\delta_x$ is the Dirac measure associated to a point $x\in\Omega$ in a measurable space, then
\[ \int_\Omega f \diff{\delta_x} = f(x). \]
\end{example}

\begin{lemma} \label{integralOverSubset}
Let $(\Omega, \mathcal{A}, \mu)$ be a measure space, $E\in\mathcal{A}$ and $f:\Omega\to[0,+\infty]$ be a measurable function. Then
\[ \int_E f \diff{\mu} = \int_\Omega f\cdot\chi_{E} \diff{\mu} \qquad\text{and}\qquad \int_\Omega f\diff{\mu} = \int_{\Omega\setminus E} f\diff{\mu}+\int_E f\diff{\mu}. \]
\end{lemma}

\begin{proposition}
Let $(\Omega, \mathcal{A}, \mu)$ be a measure space and let $f:\Omega\to[0,+\infty]$ be a positive function.

If $f$ is measurable, then there exists an increasing sequence of positive measurable step functions $(s_n)$ that converges point-wise to $f$.
\end{proposition}
The converse is also true and is given by \ref{pointWiseConvergenceMeasurable}.
\begin{proof}
Assume $f$ measurable. If we can find an increasing sequence of positive measurable step functions $(t_n)$ that converges point-wise to $\id:[0,+\infty]\to[0,+\infty]$, then
\[ f = \id\circ f = \lim_{n\to\infty} t_n\circ f = \sup_{n\in\N}(t_n\circ f) \]
and so $s_n = t_n\circ f$ gives the sequence we are looking for. And we can find such a sequence $(t_n)$. For example
\[ t_n = n\chi_{[n,+\infty[}+\sum_{k=1}^{n2^n}\frac{k-1}{2^n}\chi_{[\frac{k-1}{2^n},\frac{k}{2^n}[}. \]
\end{proof}

\begin{proposition} \label{propertiesIntegralPositiveFunctions}
Let $(\Omega, \mathcal{A}, \mu)$ be a measure space and let $f,g:\Omega\to[0,+\infty]$ be positive measurable functions. Then
\begin{enumerate}
\item if $f\leq g$, then $\int_\Omega f\diff{\mu} \leq \int_\Omega g\diff{\mu}$;
\item if $E_1\subseteq E_2$ are events in $\mathcal{A}$, then $\int_{E_1}f\diff{\mu} \leq \int_{E_2}f\diff{\mu}$;
\item \textup{(Beppo Levi's lemma)} if $(f_n)$  is an increasing sequence of positive functions that converges to $f$ point-wise, then $(\int_\Omega f_n\diff{\mu})_n$ is an increasing sequence and
\[ \lim_{n\to\infty}\int_\Omega f_n\diff{\mu} = \int_\Omega \lim_{n\to\infty}f_n\diff{\mu} = \int_\Omega f\diff{\mu}; \]
\item the integral is positive linear: $\forall c\geq 0$:
\[ \int_\Omega(cf+g)\diff{\mu} = c\int_\Omega f\diff{\mu} + \int_\Omega g\diff{\mu}. \]
\end{enumerate}
\end{proposition}
\begin{proof}
(1) For all $s\in \SF(\Omega, [0,+\infty[)$ we have that $s\leq f$ implies $s\leq g$.

(2) This follows from $f\cdot\chi_{E_1}\leq f\cdot\chi_{E_2}$ and \ref{integralOverSubset}.

(3) That the sequence $(\int_\Omega f_n\diff{\mu})_n$ is increasing follows from point 1. For increasing sequences the limits are suprema, by monotone convergence \ref{sequenceMonotoneConvergence}. Also, for all $m\in\N$, we have $f_m\leq\sup_{n\in\N}f_n$, which implies, by point 1., that $\int_\Omega f_m\diff{\mu}\leq \int_\Omega \sup_{n\in\N}f_n\diff{\mu}$. So
\[ \lim_{n\to\infty}\int_\Omega f_n\diff{\mu} = \sup_{n\in\N}\int_\Omega f_n\diff{\mu} \leq \int_\Omega \sup_{n\in\N}f_n\diff{\mu} = \int_\Omega \lim_{n\to\infty}f_n\diff{\mu}. \]
For the other inequality, it is enough to prove that $c\int_\Omega s\diff{\mu} \leq \lim_{n\to\infty}\int_\Omega f_n\diff{\mu}$ for all $0<c<1$ and $s\in\SF(\Omega,[0,+\infty[)$ such that $s\leq f$. Fix such a $c$ and $s = \sum_{i=1}^k\lambda_i\chi_{A_i}$. Consider the sets
\[ E_n = \setbuilder{x\in\Omega}{cs(x)\leq f_n(x)} = \bigcup_{i=1}^k \left(f^{-1}_n[\,[c\lambda_i, +\infty]\,]\cap A_i\right). \]
Then $(E_n)$ is an increasing sequence in $\mathcal{A}$ with $\Omega= \bigcup_{n\in\N}E_n$ and
\[ c\int_\Omega s\diff{\mu} = \int_{\bigcup_n E_n} cs\diff{\mu} = \lim_{n\to\infty}\int_{E_n} cs\diff{\mu} \]
by \ref{integralContinuousInDomain}. Also 
\[ \int_{E_n} cs\diff{\mu} \leq \int_{E_n} f_n\diff{\mu} \leq \int_{\Omega} f_n\diff{\mu} \]
by the previous points and so the result follows from the fact that limits preserve inequalities, \ref{limitPreservesInequality}.

(4) Take sequences $(s_n)$ and $(t_n)$ of positive measurable step functions
that increase pointwise to $f$ and $g$, respectively. Then $cs_n+t_n$ converges pointwise to $cf+g$ by the linearity of the limit and
\begin{align*}
\int_\Omega(cf+g)\diff{\mu} &= \lim_{n\to\infty}\int_\Omega(cs_n+t_n)\diff{\mu} \\
&= \lim_{n\to\infty}\left(c\int_\Omega s_n\diff{\mu}+\int_\Omega t_n\diff{\mu}\right) \\
&= c\lim_{n\to\infty}\int_\Omega s_n\diff{\mu}+\lim_{n\to\infty}\int_\Omega t_n\diff{\mu} \\
&= c\int_\Omega f\diff{\mu} + \int_\Omega g\diff{\mu}.
\end{align*}
\end{proof}

\begin{proposition}[Fatou's lemma] \label{FatouLemma}
Let $(\Omega, \mathcal{A}, \mu)$ be a measure space and $(h_n)$ any sequence of positive measurable functions in $(\Omega\to[0,+\infty])$. Then
\[ f: \Omega\to[0,+\infty]: x\mapsto \liminf_{n\to\infty}h_n(x) \]
is measurable and
\[ \int_\Omega \liminf_{n\to\infty}h_n\diff{\mu} \leq \liminf_{n\to\infty}\int_\Omega h_n\diff{\mu}. \]
\end{proposition}
\begin{proof}
Consider the sequence $(f_n)$ defined by $f_n(x) = \inf_{k\geq n}h_k(x)$. This is an increasing sequence of positive functions and $f_n \leq h_n$ for all $n\in\N$. Each $f_n$ is measurable because
\[ f_n^{-1}[\,[t,+\infty]\,] = \bigcap_{k=n}^\infty h_k^{-1}[\,[t,+\infty]\,]. \]
This shows that $f$ is measurable by \ref{pointWiseConvergenceMeasurable}. Then we can use Beppo Levi's lemma \ref{propertiesIntegralPositiveFunctions} to obtain
\[ \int_\Omega\liminf_{n\to\infty}h_n\diff{\mu} = \lim_{n\to\infty}\int_\Omega f_n\diff{\mu} = \liminf_{n\to\infty}\int_\Omega f_n\diff{\mu} \leq \liminf_{n\to\infty}\int_\Omega h_n\diff{\mu} \]
using monotonicity of the integral \ref{propertiesIntegralPositiveFunctions} and liminf \ref{propertiesIntegralPositiveFunctions} for the last inequality.
\end{proof}

\begin{proposition}
Let $\seq{\Omega, \mathcal{A}, \mu}$ be a measure space and $f:\Omega\to[0,+\infty]$ a positive measurable function. Then
\begin{enumerate}
\item $\int_\Omega f\diff{\mu} = 0$ \textup{if and only if} $f(x) = 0$ a.e.;
\item if $f$ is integrable, then $f(x)< +\infty$ a.e.
\end{enumerate}
\end{proposition}
\begin{proof}(1) Set $E = \setbuilder{x\in\R}{f(x) \neq 0}$. Then $f(x) = 0$ a.e. is equivalent to $\mu(E) = 0$.

Assume $\mu(E) = 0$, then 
\[ \int_\Omega f\diff{\mu} = \int_{\Omega\setminus E} f\diff{\mu} + \int_E f\diff{\mu} = 0+\int_E f\diff{\mu} \leq \sup_{x\in E}(f(x)). \]
Now for all $s\in\SF(\Omega\to [0,+\infty[)\cap \downset f$ we have
\[ 0\leq \int_E s \diff{\mu}\leq \max_x(s(x))\mu(E) = 0  \]
and so the supremum $\int_E f\diff{\mu}$ is zero as well.

Now assume $\int_\Omega f\diff{\mu} = 0$. Consider the sets $E_n = f^{-1}[\,]\frac{1}{n},+\infty]\,]$. Then $\frac{1}{n}\chi_{E_n}\in \downset f$ is simple. Hence $\frac{1}{n}\mu(E_n)\leq \int_\Omega f\diff{\mu} = 0$, meaning $\mu(E_n) = 0$ for all $n$. So $\mu(E) = \sup_{n\in\N}\mu(E_n) = 0$.

(2) Towards contraposition, assume the set $E = \setbuilder{x\in\R}{f(x) = +\infty}$ has non-zero measure. Then $a/\mu(E)\chi_E\in\SF(\Omega, [0,+\infty[)\cap \downset f$ for all real $a>0$ and
\[  \int_\Omega f\diff{\mu} \geq \int_\Omega \frac{a\chi_E}{\mu(E)}\diff{\mu} = a  \]
so $f$ is not integrable.
\end{proof}

\subsubsection{Product measures}
Fubini!

\subsection{Real functions}
Let $\Omega$ be a set and $f:\Omega \to \C$ a function.  Then we can uniquely decompose $f$ into $f= u+iv$ where $u,v: \Omega\to \R$. We can further decompose
\[ \begin{cases}
u = u^+ - u^- & \text{such that $u^+u^- = 0$} \\
v = v^+ - v^- & \text{such that $v^+v^- = 0$.}
\end{cases} \]
If $\Omega $ carries a $\sigma$-algebra such that $f$ is measurable, then $u^+,u^-,v^+,v^-$ are also measurable.
\begin{definition}
Let $\seq{\Omega, \mathcal{A},\mu}$ be a measure space. We say a measurable function $f: \Omega \to \C$ is \udef{integrable}, if $|f|:\Omega\to [0,\infty[$ is integrable. In this case we define the \udef{integral} of $f$ as
\[ \int_\Omega f\diff{\mu} = \int_\Omega u^+\diff{\mu} - \int_\Omega u^-\diff{\mu} + i\int_\Omega v^+\diff{\mu} - i\int_\Omega v^-\diff{\mu}. \]
The set of all integrable functions in $(\Omega \to \C)$ is denoted $\mathcal{L} ^1(\Omega,\mathcal{A},\mu)$ or $\mathcal{L} ^1(\mu)$.
\end{definition}

TODO
\begin{proposition}[Reverse Fatou lemma]
Let $(\Omega, \mathcal{A}, \mu)$ be a measure space and $(h_n)$ any sequence of measurable functions that is dominated by a positive integrable function $g$ (i.e. $h_n\leq g$ for all $n\in\N$). Then
\[ f: \Omega\to[0,+\infty]: x\mapsto \limsup_{n\to\infty}h_n(x) \]
is measurable and
\[ \int_\Omega f\diff{\mu} \geq \limsup_{n\to\infty}\int_\Omega h_n\diff{\mu}. \]
\end{proposition}
\begin{proof}
By the corollary $g(x)<+\infty$ a.e. and so also $g-h_n<+\infty$ a.e. Applying Fatou's lemma \ref{FatouLemma} gives
\[ \int\liminf g-h_n \leq \liminf\int g-h_n\]
\end{proof}


\subsection{Integration of vector-valued functions}
\subsubsection{Weak and strong measurability}
TODO is Bochner measurable measurable with $Y$ given Borel $\sigma$-algebra?

A function $f:X\to B$ is called Bochner-measurable if it is equal $\mu$-almost everywhere to a function $g$ taking values in a separable subspace $B_{0}$ of $B$, and such that the inverse image $g^{-1}[U]$ of every open set $U$ in $B$ belongs to $\Sigma$. Equivalently, $f$ is limit $\mu$-almost everywhere of a sequence of simple functions. 

\begin{theorem}[Pettis measurability theorem]
Let $(\Omega, \mathcal{A},\mu)$ be a measure space and $Y$ a normed vector space. Then $f$ is strongly measurable \textup{if and only if} $f$ is weakly measurable and almost surely separably valued.
\end{theorem}

\subsubsection{Bochner integration}
TODO: the Bochner integral is the unique extension of the integral of simple functions to the set of Bochner measurable functions???? (I.e. simple functions dense in Bochner space, with $L^1$ metric)
\begin{definition}
Let $(\Omega, \mathcal{A},\mu)$ be a measure space and $Y$ a normed vector space. Then a Bochner measurable function $f:\Omega\to Y$ is called \udef{Bochner integrable} if there exists a sequence of integrable simple functions $\seq{s_n}\subset\SF(\Omega,Y)$ such that
\[ \lim_{n\to\infty}\int_\Omega \norm{f-s_n}\diff{\mu} = 0. \]
Take such a sequence $\seq{s_n}$. The \udef{Bochner integral} of $f$ on $\Omega$ w.r.t. $\mu$ is defined as
\[ \int_\Omega f\diff{\mu} \defeq \lim_{n\to\infty}\int_\Omega s_n\diff{\mu}. \]
\end{definition}

\begin{lemma}
The Bochner integral is well-defined: let $\seq{s_n},\seq{t_n}\in \prescript{\N}{}{\SF(\Omega,Y)}$ be sequences such that
\[ \lim_{n\to\infty}\int_\Omega \norm{f-s_n}\diff{\mu} = 0 = \lim_{n\to\infty}\int_\Omega \norm{f-t_n}\diff{\mu}.  \]
Then
\begin{enumerate}
\item the limits $\lim_{n\to\infty}\int_\Omega s_n\diff{\mu}$ and $\lim_{n\to\infty}\int_\Omega t_n\diff{\mu}$ exist;
\item $\lim_{n\to\infty}\int_\Omega s_n\diff{\mu} = \lim_{n\to\infty}\int_\Omega t_n\diff{\mu}$.
\end{enumerate}
\end{lemma}
\begin{proof}
TODO
\end{proof}

\begin{proposition}[Bochner integrability criterion]
Let $(\Omega, \mathcal{A},\mu)$ be a measure space and $Y$ a normed vector space.

A Bochner measurable function $f$ is Bochner integrable \textup{if and only if}
\[ \int_\Omega \norm{f} \diff{\mu} < \infty. \]
\end{proposition}

\begin{proposition}
Linearity and monotonicity.
\end{proposition}

\begin{proposition}
Let $(\Omega, \mathcal{A},\mu)$ be a measure space $Y$ a normed vector space and $T$ a closed operator on $Y$. If $T\circ f$ is integrable, then
\[ \int_\Omega (T\circ f)\diff{\mu} = T\left(\int_\Omega f\diff{\mu}\right). \]
\end{proposition}
\begin{proof}
TODO
\end{proof}
\begin{corollary}
If $T$ is bounded, then $T\circ f$ is integrable and
\[ \int_\Omega (T\circ f)\diff{\mu} = T\left(\int_\Omega f\diff{\mu}\right). \]
\end{corollary}

TODO Dominated convergence.

\subsubsection{Pettis integration}


\section{Further topics}
TODO rename!

\subsection{Absolute continuity and mutual singularity}
\begin{definition}
Let $\mu,\nu$ be measures on the measurable space $(\Omega,\mathcal{A})$. We say
\begin{itemize}
\item $\nu$ is \udef{absolutely continuous} w.r.t. $\mu$ if $\mu(A)=0\implies \nu(A) = 0$ for all $A\in\mathcal{A}$;
\item $\mu$ and $\nu$ are \udef{mutually singular} if there exists a set $A\in\mathcal{A}$ with $\mu(A) = 0$ and $\nu(A^c) = 0$.
\end{itemize}
\end{definition}

\begin{theorem}[Radon-Nikodym]
Let $\mu,\nu$ be measures on the measurable space $(\Omega,\mathcal{A})$. Then $\nu$ is absolutely continuous w.r.t. $\mu$ \textup{if and only if} there exists a measurable function $f:\Omega\to\R$ (or $\C$?) such that
\[ \nu(A) = \int_\Omega f\cdot \chi_A \diff{\mu} \qquad \forall A\in\mathcal{A}. \]
The function $f$ is uniquely determined a.e. (w.r.t. $\mu$).
\end{theorem}

\subsection{Lebesgue decomposition}
\begin{theorem}[Lebesgue decomposition theorem]
Let $\mu, \nu$ be two measures on a measurable space $(\Omega, \mathcal{A})$. Then $\nu$ can be written uniquely as
\[ \nu = \nu_\text{ac} + \nu_\text{sing} \]
where $\mu$ and $\nu_\text{sing}$ are mutually singular and $\nu_\text{ac}$ is absolutely continuous w.r.t. $\mu$. 

\end{theorem}

\subsection{Convolution}
TODO Young's convolution inequality

\section{Duality in integration}
Distributions with kernels will be example.

\chapter{Complex analysis}
\begin{definition}
A \udef{complex function} is a function in $(U\subseteq \C \to \C)$.
\end{definition}
\section{Holomorphic functions}
\begin{definition}
Let $f:U\subseteq \C \to \C$ be a complex function. We say
\begin{enumerate}
\item $f$ is \udef{holomorphic  at $z\in U$} if the limit $\lim_{h\to 0} \dfrac{f(z+h) - f(z)}{h}$
exists;
\item $f$ is \udef{holomorphic in $S\subset U$} if it is holomorphic at every point in $S$;
\item $f$ is \udef{holomorphic} if it is holomorphic at every point in $U$;
\item $f$ is \udef{entire} if $U=\C$ and it is holomorphic at every point in $\C$.
\end{enumerate}
\end{definition}

\begin{lemma}
Holomorphic functions are continuous.
\end{lemma}
\begin{lemma}
Let $f,g$ be holomorphic. Then
\begin{enumerate}
\item $f+g$ is holomorphic and $(f+g)' = f'+g'$;
\item $fg$ is holomorphic and $(fg)' = f'g+fg'$;
\item if $g(z_0)\neq 0$, then $f/g$ is holomorphic at $z_0$ and
\[ (f/g)' = \frac{f'g - fg'}{g^2}; \]
\item the chain rule holds.
\end{enumerate}
\end{lemma}

\subsubsection{Cauchy-Riemann equations}
The space of complex numbers $\C$ is a real $2$-dimensional vector space. So any 


isomorphic to $\R^2$ as a real vector space by $x+iy \mapsto \begin{pmatrix}
x & y
\end{pmatrix}^\transp$.

\subsection{Power series}

\section{Meromorphic functions}

\section{Conformal mappings}

\chapter{Calculus}
\section{Exploring the concept of change}
TODO: diffeomorphism

In physics how things change is quite important. Much of physics is concerned with the question of, given a particular system at a particular time, how that system will evolve.

We have not yet really introduced a mathematical construct that expresses an idea of change. We will do so here.

In particular we will consider ways to express how the output of a function changes if we (slightly) change its input.

To motivate the discussion below, consider the function represented by the graph in figure TODO. 

Locally at any one point the rate of change of the function can be described using the slope at that point. That makes intuitive sense; when walking up a mountain the slope is a measure for how quickly the altitude changes.

The slope between two points can be calculated by dividing the vertical distance by the horizontal distance. This definition of slope obviously depends on two points. We would quite like to be able to talk about the slope at a single point (the way we would intuitively when walking up a hill). To do that we can just bring both points very close together.

As can be seen on the picture this procedure gives the slope of the tangent line at that point (straight lines have a constant slope).

\subsection{Speed}

At this point we can give an important physical motivating example, namely the speed of an object. Say we throw an apple straight up into the air. Its vertical movement is plotted in figure TODO.

We may want to know its speed at different times. We can calculate speed by taking the displacement and dividing it by the time it takes traverse that distance. We can now make an important distinction between average speed (the slope between two distinct points) and instantaneous speed (the limit when we bring both points together).

\section{The derivative}
As motivated above, the rate of change of a (real) function is the difference in output divided by the difference in input of two points:
\[ \frac{f(y) - f(x)}{y-x}. \]
We conventionally call $h = y-x$. We can then write the above quantity (which is called the \udef{Newton quotient}) as
\[ \frac{f(x+h) - f(x)}{x+h-x} = \frac{f(x+h) - f(x)}{h}. \]
The \udef{derivative} of $f$ at $x$ is then just the limit of the Newton quotient with $h$ going to zero.

This limit does not always exist. If the limit exists for all $x$, the function is called \udef{differentiable}. A function may also be differentiable in some points and not in others.

We can now use the definition and properties of limits to calculate derivatives, such as in the following example. This process is slow and laborious even for relatively simple functions. Luckily the derivative has some important properties that lets us calculate the derivative of many functions with relative ease.

We can define a (real) function that, for any input, calculates the derivative of a particular fixed function $f$ at that point and gives that as its output. This new function is often called the derivative of the function $f$. 

There are many ways to write the derivative of $f$:
\[ \lim_{h\to 0} \frac{f(x+h)-f(h)}{h} \equiv f'(x) \equiv \od{f}{x} \equiv \od{f(x)}{x} \]
A mathematician would want me to emphasize that the expression $\od{f}{x}$ should be read as a whole and is technically \emph{not} a division, but a physicist would say that (in some situations) it can be viewed as such, where $\div{f}$ and $\div{x}$ are (the in this context relevant) infinitesimal variations of $f$ and $x$. Do not tell any mathematicians I said this.

\begin{example}
TODO derivative of polynomial function using limits.
\end{example}

In certain situations a dot is used to indicate a derivative with respect to time (i.e.\ the derivative of a quantity in function of time). So we might for example use $x(t)$ to denote the position in function of time (here $x$ is \emph{not} used to refer to a variable but to a function, the notation is standard and usually it clear from the context what $x$ refers to). We can then use the notation
\[ x'(t) \equiv \dot{x}(t). \]
In fact $\dot{x}(t)$, is just the speed.

When using the notation $\od{f}{x}$, this usually refers to the function that is the derivative of $f$. If we want to evaluate this function in a particular point (say $x_0$), we can write something like this
\[ \left.\od{f}{x}\right|_{x=x_0}. \]

\subsection{Slope of a curve}
\[ \text{slope of the normal} = \frac{-1}{\text{slope of the tangent}} \]

\subsection{Properties of the derivative}
Here we give some properties of the derivative:
\begin{itemize}
\item The derivative is a linear operation:
\[ (f+g)'(x) = f'(x) + g'(x) \]
and
\[ (c\cdot f)'(x) = c\cdot f'(x) \qquad \forall c \in \R \]
\item \ueig{Product rule}
\[ (f\cdot g)'(x) = f(x)\cdot g'(x) + f'(x)g(x) \]
\item The derivative of $\frac{1}{f(x)}$, assuming $f(x) \neq 0$:
\[ \left(\frac{1}{f(x)}\right)' = \frac{f'(x)}{f(x)^2}. \]
\item Combining the previous two properties, we get the quotient rule (assuming $g(x)\neq 0$)
\[ \left(\frac{f(x)}{g(x)}\right)' = \frac{g(x)f'(x) - f(x)g'(x)}{g(x)^2}. \]
\item Finally we have the very important \ueig{chain rule}. This tells us how to take the derivative of composite functions:
\[ (f \circ g)'(x) = f'(g(x))g'(x). \]
We can also write this as
\[ \od{f(g(x))}{x} = \od{f}{g}\od{g}{x}. \]
TODO example
\item Derivative of an inverse
\[ \od{f^{-1}(x)}{x} = \frac{1}{f'(f^{-1}(x))} \]
\end{itemize}

TODO Faà di Bruno

\subsection{Derivatives of some common functions}
Using the results below together with the properties above we can calculate the derivative of a large number of functions.

\begin{itemize}
\item Let $n$ be an integer larger than or equal to $1$ and let $f(x) = x^n$. Then
\[ f'(x) = n x^{n-1}. \]
Using this result together with the property of linearity, we can easily calculate the derivative of any polynomial function. TODO: general exponent
\item The derivatives of the trigonometric functions can be derived from
\[ \sin'(x) = \cos \qquad \text{and} \qquad \cos'(x) = -\sin(x) \]
TODO: list
\item TODO cyclometric
\item TODO hyperbolic
\end{itemize}

\subsubsection{The exponential and logarithm}
Define natural logarithm $\ln$ and \textit{the} exponential function $\exp$.
\[ \od{\ln x}{x} = \frac{1}{x} \]
\[ a^x = e^{x\ln a} \qquad (a>0, x\in \R) \]

\[ e^x = \lim_{n\to \infty}\left(1+\frac{x}{n}\right)^n \]
and growth.

\subsection{Applications of differentiation}
\subsubsection{Extreme values}
Link increasing, decreasing and derivatives. + derivative zero everywhere = constant.
critical points. singulat points. concavity and inflections
\subsubsection{Rolle's lemma}
\subsubsection{Mean-value theorem}
\subsubsection{L'Hôpital's rules}


\subsection{Higher order derivatives}
When we take the derivative of a function, we we get a new function. We can now take the derivative of this new function. This is called taking the second order derivative. This process can be repeated for as long as the derivatives exist. We write the $n$-th order derivative as
\[ f^{(n)}(x) = \od[n]{f}{n}. \]
So for example $f''(x) = f^{(2)}(x)$.

\subsection{Implicit differentiation}

\subsection{Partial derivatives}
\subsubsection{Definition}
\subsubsection{Geometric interpretation}

\subsection{Meaning of the differential $\div{}$}
TODO conventional use + examples with nabla

TODO: put series here!

\subsection{Generalisations and types of derivatives}
TODO: Liebnitz rule!!! + linear.

\section{Integration}
TODO intuition, solving strategies, solving intelligently 
SEE: The electric field (first write all quantities, then )

\subsection{Areas as limits of sums}
\subsubsection{Sums and sigma notation}
\subsubsection{Trapezoid rule}
\subsubsection{Midpoint rule}
\subsubsection{Simpson's rule}

\subsection{The definite integral}
\subsection{Computing different areas and volumes}
\subsubsection{Rotation bodies}
\subsubsection{Surface bounded by function of polar coordinate $\theta$}
\[ \frac{1}{2}\int_{\theta_1}^{\theta_2}[f(\theta)]^2\div{\theta} \]

\subsection{The fundamental theorem of calculus}
\subsubsection{Indefinite integrals}
anti-derivative $+C$
\subsubsection{Some elementary integrals}

\subsection{Properties of integrals}
\subsubsection{Linearity}
\subsubsection{Mean-value theorem}
\subsubsection{Integrals of piece-wise continuous functions}

\subsection{Techniques of integration}
\subsubsection{Integrals of rational functions}
\subsubsection{Substitutions}
+ inverse substitutions
\subsubsection{Integration by parts}


\subsection{Improper integrals}


\subsection{Different types of integrals}
\subsubsection{Riemann}
\subsubsection{Lebesgue}
\subsubsection{Stieltjes}
\subsubsection{Cauchy}

\subsection{From infinite sum to integral}
Using measure

\section{Complex analysis}
holomorphic functions, residue theorem

\subsection{Complex integration and analyticity}
\subsection{Laurent series and isolated singularities}
\subsection{Residue calculus}
\subsection{Conformal mapping}


TODO
Solving intelligently (later using physics): Green functions, method of mirrors (+ cfr. general section on equations)
charge distributions

separation of variables (Legendre polynomials)

going from discrete sum to integral (also opposite with dirac delta). volume int using $\mathcal{V}$ and surface $\mathcal{S}$

surface int goes to zero at infinity.

\section{Dirac delta}
\subsection{In one dimension}
\subsection{In three dimensions}
\subsection{Properties}

\begin{eigenschap}
Composition of the Dirac $\delta$ with a smooth, continuously differentiable function $g$ follows from the following relation
\[ \int_\R \delta(g(x))f(g(x))|g'(x)|\div{x} = \int_{g(\R)} \delta(u)f(u)\div{u} \]
Thus we say that
\[ \delta(g(x)) = \sum_i \frac{\delta(x-x_i)}{|g'(x_i)|}\]
Where $x_i$ are the simple roots of $g$.
\end{eigenschap}

\section{Silly integrals}
\[ \int x^{\diff x}-1 = x\ln(x) - x +c \]
\chapter{Taylor and other series}
TODO: move
TODO MacLaurin, propto, other expansions (multipolar, binomial etc) 
Taylor polynomials, big O, possible big O types

\section{Definition of series}
\section{Examples of series}
\subsection{Geometric series}
\subsection{Harmonic series}

\section{Convergence tests for positive series}

\section{Absolute and conditional convergence}

\section{Power series}

\section{Taylor and Maclaurin series}
\subsection{Taylor's theorem}


\section{Matrix exponential}
The matrix exponential is quite simply defined as the MacLaurin series of the normal exponential applied to square matrices.

\begin{definition}
Let $X$ be an $n\times n$ matrix. The exponential of $X$ is given by the power series
\[ e^X = \sum^\infty_{m=0} \frac{X^m}{m!}. \]
\end{definition}

It's nice to know that for any $n \times n$ real or complex matrix $X$, this series does actually converge. The matrix exponential is also a \ueig{continuous} function of $X$.

Here are also some elementary properties of the matrix exponential, that may be useful for somebody somewhere.
\begin{eigenschap}
Let $X$ be an arbitrary $n\times n$ matrix. Let $C$ be invertible.
\begin{enumerate}
\item $e^0 = \mathbb{1}_n$.
\item $\left(e^X\right)^\dagger = e^{X^\dagger}$.
\item $e^X$ is invertible and $\left(e^X\right)^{-1} = e^{-X}$.
\item $\left(e^X\right)^* = e^{A^*}$
\item $\left(e^X\right)^\dagger = e^{A^\dagger}$
\item $\left(e^X\right)^\intercal = e^{A^\intercal}$
\item $e^{CXC^{-1}} = Ce^XC^{-1}$
\end{enumerate}
\end{eigenschap}
It is in general \textbf{not} true that $e^{X+Y} = e^Xe^Y$; this is only true if $X$ and $Y$ commute.

Here are a number of properties of the matrix exponential that will be useful later. We will assume $X, Y$ are complex $n\times n$ matrix.
\begin{eigenschap}
The map $\R \to \C^{n\times n}: t \mapsto e^{tX}$ is a smooth curve in $\C^{n\times n}$ and
\[ \od{}{t}e^{tX} = Xe^{tX} = e^{tX}X. \]
In particular,
\[ \left.\od{}{t}e^{tX}\right|_{t=0} = X. \]
\end{eigenschap}

\begin{eigenschap}
\ueig{Lie product formula} 
\[ e^{X+Y} = \lim_{m\to\infty} \left(e^{\frac{X}{m}}e^{\frac{Y}{m}}\right)^m \]
\end{eigenschap}
Finally for the determinant we have:
\begin{eigenschap}
\[\det \left(e^X\right) = e^{\Tr(X)}\]
\end{eigenschap}

\section{Binomial theorem and binomial series}

\chapter{Distributions}
\section{The space of test functions}
\subsection{Canonical LF topology}
\subsubsection{Convergence}
\subsection{The space of test functions}
\begin{definition}
Let $X$ be an open subset of a normed vector space. The space of \udef{test functions} on $X$ is $\cont^\infty_c(X)$ equipped with the canonical LF topology. This space is also commonly denoted $\testFuncs(X)$.
\end{definition}

\begin{example}
The \udef{bump function}
\[ \phi: \R\to\R: x\mapsto \begin{cases}
e^{1/(x^2-1)} & |x|<1 \\ 0 & |x| \geq 1
\end{cases} \]
is a test function in $\testFuncs(\R)$.
\end{example}

\begin{lemma}
Every test function is bounded.
\end{lemma}
\begin{proof}
TODO
\end{proof}

\section{The module of distributions}
\begin{definition}
    Let $X$ be an open subset of a normed vector space.
    A linear map $T: \testFuncs(X) \to \C$ is a \udef{distribution} on $X$ is an element of the topological dual of $\testFuncs(X)$ equipped with the strong dual topology. It is denoted $\dists(X)$.
\end{definition}

\begin{proposition}
    The space of distributions $\dists(X)$ is a module over the ring $\cont^\infty(X)$.
    
    If $T\in \dists(X)$ and $a\in \cont^\infty(X)$, then the multiplication is defined by
    \[ (aT)(\phi) = T(a\phi) \qquad \forall \phi \in \testFuncs(X). \]
\end{proposition}

\subsection{Types of distributions}
\subsubsection{Regular distributions}
\begin{lemma}
Let $X$ be an open subset of a normed vector space and $f: X\to\C$ a locally integrable function in $L^1_\text{loc}(X)$. Then
\[ T_f: \testFuncs(X) \to \C: \phi \mapsto \int_X f(x)\phi(x)\diff{x} \]
is a distribution.
\end{lemma}
\begin{proof}
TODO We just need to show continuity (from boundedness)
\end{proof}

\begin{definition}
Distributions of the form $T_f$ for some $f\in L^1_\text{loc}(X)$ are called \udef{regular distributions}.
\end{definition}


\begin{lemma} \label{uniquenessIntegratedFunction}
Let $f,g: X\to \C$ be locally, absolutely integrable functions. If $T_f = T_g$, then $f(x) = g(x)$ a.e.
\end{lemma}

\subsubsection{Dirac delta distribution}
\begin{definition}
    Let $X$ be an open subset of a normed vector space and $x_0\in X$. The \udef{Dirac delta distribution} at $x_0$ is the distribution
    \[ \delta_{x_0}: \testFuncs(X) \to \C: \phi \mapsto \phi(x_0). \]
    We write $\delta \defeq \delta_0$.
\end{definition}

\begin{proposition}
Let $X$ be an open subset of a normed vector space containing $0$. Suppose $\seq{f_n: X\to \C}$ is a sequence of functions such that
\begin{enumerate}
\item $\int_X f_n(x)\diff{x} = 1$ for all $n$;
\item there exists a constant $C$ such that $\int_X |f_n(x)|\diff{x} \leq C$ for all $n$;
\item $\lim_{n\to \infty} \int_{|x|>r}|f_n(x)|\diff{x} = 0 $ for all $r > 0$.
\end{enumerate}
If $\phi$ is bounded on $X$ and continuous at $0$, then
\[ \lim_{n\to \infty} \int_X f_n(x)\phi(x)\diff{x} = \phi(0) \]
and in particular $f_n\to \delta$ in $\dists(X)$.
\end{proposition}
Notice that the condition of $\phi$ being bounded on $X$ and continuous at $0$ is much less strong than $\phi\in\testFuncs$.
\begin{proof}
TODO + iff? Then we could define delta sequences as sequences that converge to $\delta$.

TODO link with (and formulation of integral mean value theorem)
\end{proof}
Sequences $\seq{f_n}$ satisfying the assumptions of the proposition are called \udef{delta sequences}.

\begin{lemma}
If $f_n$ is such that $\int f_n(x) \diff{x} = 1$ and the support shrinks to zero, then $f_n$ is a delta sequence.
\end{lemma}

\begin{lemma}
Let $f: \R^N\to\C$ be an integrable function with $\int_{\R^N}f(x)\diff{x} = 1$. Then $\seq{n^Nf(nx)}_{n\in \N}$ is a delta sequence.
\end{lemma}

\subsection{The derivative of a distribution}
\begin{definition}
Let $X$ be an open subset of a normed vector space $V$ and let $T\in \dists(X)$. For $u\in V$ we define the directional derivative of $T$ as
\[ \partial_u(T) \defeq - T\circ \partial_u. \]

In particular, if $V = \R$, we define
\[ T' \defeq \od{}{x}(T) \defeq - T\circ \od{}{x}. \]
\end{definition}

\begin{lemma}
Let $T_f$ be an integral distribution such that $\partial_u(f)$ is well-defined on $X$, then
\[ \partial_u(T_f) = T_{\partial_{u}(f)}. \]
\end{lemma}
\begin{proof}
TODO by partial integration.
\end{proof}

\begin{lemma}
Let $X$ be an open subset of $\R^N$ and let $T\in \dists(X)$. Then
\[ D^\alpha T = (-1)^{|\alpha|}T\circ D^\alpha. \]
\end{lemma}

\begin{proposition}
    Let $X$ be an open subset of a normed vector space $V$, $T\in \dists(X)$ and $u\in V$. Then
    \begin{enumerate}
    \item $\partial_{u}T \in \dists(X)$;
    \item $\partial_{u}$ is linear on the module $\dists(X)$.
    \end{enumerate}
\end{proposition}

\begin{proposition}
Let $\theta$ be the Heaviside function. Then
\[ (T_\theta)' = \delta. \]
\end{proposition}

\subsubsection{Jump discontinuities in integral distributions}
\begin{proposition}
Let $f$ be a function that is $\cont^1$ on $]-\infty, x_0[$ and $]x_0, +\infty[$ for some $x_0\in\R$. Then
the derivative of $f$ as distribution is given by
\[ f' = (f|_{\R\setminus\{x_0\}})' + (\Delta_{x_0}f)\delta_{x_0}. \]
\end{proposition}
Note that if $f$ is continuous at $x_0$, this gives $f' = (f|_{\R\setminus\{x_0\}})'$ as distributions.
\begin{proof}
Let $\phi\in\testFuncs(\R)$. Then we calculate
\begin{align*}
f'(\phi) &= \int_{-\infty}^\infty f(x)\phi'(x)\diff{x} = \int_{-\infty}^{x_0} f(x)\phi'(x)\diff{x} + \int_{x_0}^\infty f(x)\phi'(x)\diff{x} \\
&= \Big[f(x)\phi(x)\Big]_{-\infty}^{x_0} - \int_{-\infty}^{x_0} f'(x)\phi(x)\diff{x} + \Big[f(x)\phi(x)\Big]_{x_0}^{+\infty} - \int_{x_0}^{+\infty} f'(x)\phi(x)\diff{x} \\
&= -\int_{\R\setminus\{x_0\}} f'(x)\phi(x)\diff{x} + \lim_{x\to x_0+}f(x)\phi(x)-\lim_{x\to x_0-}f(x)\phi(x) \\
&= \int_{\R}f|_{\R\setminus\{x_0\}}\phi'(x)\diff{x} + \phi(x_0)\lim_{x\to x_0+}f(x)- \phi(x_0)\lim_{x\to x_0-}f(x) = \int_{\R}(f|_{\R\setminus\{x_0\}})'\phi(x)\diff{x} + (\Delta_{x_0}f)\delta_{x_0}.
\end{align*}
\end{proof}
\begin{corollary}
Let $f$ be a function that is $\cont^1$ on $]-\infty, x_0[$ and $]x_0, +\infty[$ for some $x_0\in\R$. Then
the $k^\text{th}$ derivative of $f$ as distribution is given by
\[ f^{(k)} = (f|_{\R\setminus\{x_0\}})^{(k)} + \sum_{j=0}^{k-1}(\Delta_{x_0}f^{(j)})\delta^{(k-1-j)}_{x_0}. \]
\end{corollary}

\subsection{Convolution}


\section{Sobolev spaces}
TODO: after $L^p$ spaces.
\subsection{Weak and strong derivatives}
\begin{definition}
    Let $X$ be an open subset of a normed vector space, $f\in L^p(X)$ and $D^\alpha$ a derivative on $X$. If there exists $g\in L^q(X)$ such that $D^\alpha T_f = T_g$, then $g$ is the \udef{weak $\alpha$ derivative} of $f$ in $L^q(X)$.
\end{definition}
The weak $\alpha$ derivative is unique if it exists, due to \ref{uniquenessIntegratedFunction}.

The definition of weak derivative translates to the requirement
\[ \int_X f D^\alpha \phi \diff{x} = (-1)^{|\alpha|}\int_X g\phi\diff{x} \qquad \forall \phi\in\testFuncs(X). \]

\begin{definition}
    Let $X$ be an open subset of a normed vector space, $f\in L^p(X)$ and $D^\alpha$ a derivative on $X$. We call $g\in L^q(X)$ a \udef{strong $\alpha$ derivative} if there exists a sequence $\seq{f_n} \subset \cont^\infty(X)$  such that
    \begin{itemize}
    \item $f_n \to f$ in $L^p(X)$; and
    \item $D^{\alpha}f_n \to g$ in $L^q(X)$.
    \end{itemize}
\end{definition}

\begin{theorem}
Let $f\in L^p(X)$. Then $g\in L^q(X)$ is a weak $\alpha$ derivative \textup{if and only if} it is a strong $\alpha$ derivative.
\end{theorem}

\subsection{Sobolev spaces}
\begin{definition}
Let $X$ be an open subset of a normed vector space, $1\leq p \leq \infty$ and $k\in\N$. Then the \udef{Sobolev space} $W^{k,p}(X)$ is defined as
\[ W^{k,p}(X) \defeq \setbuilder{T\in \dists(X)}{\text{$T$ has a weak $\alpha$ derivative in $L^p(X)$ for all $|\alpha|\leq k$}}.
 \]
\end{definition}
In particular each distribution in $W^{k,p}(X)$ is an integral distribution $T_f$ for some $f$ in $L^p(X)$.

\begin{lemma}
    Let $X$ be an open subset of a normed vector space, $1\leq p \leq \infty$ and $k\in\N$. Then
    \[ \testFuncs(X) \subset W^{k,p}(X) \subset L^p(X), \]
    so that $W^{p,k}(X)$ is a dense subspace of $L^p(X)$.
\end{lemma}

\begin{proposition}
A Sobolev space $W^{k,p}(X,\diff{\mu})$ is a Banach space with norm
\[ \norm{f}_{W^{k,p}(X,\diff{\mu})} = \begin{cases}
\left(\sum_{|\alpha|\leq k}\norm{D^\alpha f}^p_{L^{p}(X, \diff{\mu})}\right)^{1/p} & 1\leq p < \infty \\
\max_{|\alpha|\leq k} \norm{D^\alpha f}_{L^{\infty}(X, \diff{\mu})} & p = \infty.
\end{cases} \]
If the measure is clear, we may also write $W^{k,p}(X)$.

In particular $W^{k,2}(X, \diff{\mu})$ is a Hilbert space with inner product
\[ \inner{f,g} \sum_{|\alpha| \leq k} \int_X D^\alpha f(x) \overline{D^\alpha g(x)}\diff{\mu(x)}. \]
The Hilbert space $W^{k,2}(X, \diff{\mu})$ is more commonly denoted $H^k(X, \diff{\mu})$.
\end{proposition}
TODO: alternative definition: use ordinary derivative and take completion???

\begin{proposition}
Let $X$ be an open subset of a normed vector space, $1\leq p < \infty$ and $k\in\N$. Then $W^{k,p}(X)$ coincides with the closure of $\cont^\infty(X) \cap W^{k,p}(X)$ in the $W^{k,p}(X)$-norm.
\end{proposition}

\begin{definition}
Let $X$ be an open subset of a normed vector space, $1\leq p < \infty$ and $k\in\N$. We define $W_0^{k,p}(X)$ to be the closure of $\cont^\infty_c(X)$ in the $W^{k,p}(X)$-norm.
\end{definition}

\part{Hilbert Spaces and $C^*$ Algebras}
\setcounter{chapter}{0} % Reset chapter counter
\chapter{Inner product spaces}

\begin{definition}
An \udef{inner product} on a vector space $V$ is a function
\[ \inner{\cdot,\cdot}: V\times V \to \mathbb{F}  \]
that has the following properties:
\begin{itemize}[leftmargin=4.5cm]
\item[\textbf{Linearity}] in the \emph{second}\footnote{Some authors take linearity in the first component.} component
\[\inner{v,\lambda_1 w_1 + \lambda_2 w_2} = \lambda_1\inner{v,w_1} + \lambda_2\inner{v,w_2},\]
where $\lambda_1,\lambda_2 \in \mathbb{F}$ and $v,w_1,w_2\in V$.
\item[\textbf{Conjugate symmetry}\footnote{This is for $\mathbb{F} = \C$. For $\mathbb{F} = \R$ this reduces to normal symmetry $\inner{v,w} = \inner{w,v}$.}] $\inner{v,w} = \overline{\inner{w,v}}$ for all $v,w\in V$.
\item[\textbf{Positivity}\footnote{By conjugate symmetry we know that $\inner{v,v}$ is a real number, so this condition makes sense.}] $\inner{v,v} \geq 0$ for all $v\in V$.
\item[\textbf{Definiteness}]$\inner{v,v} = 0$ if and only if $v= 0$.
\end{itemize}
An \udef{inner product space} or \udef{pre-Hilbert space} $(\mathbb{F}, V,+,\inner{\cdot,\cdot})$ is a vector space $(\mathbb{F}, V,+)$ together with an inner product $\inner{\cdot,\cdot}$ on $V$.

\begin{itemize}
\item A real finite-dimensional inner product space is called a \udef{Euclidean space}.
\item A \udef{Hilbert space} is an inner product space that is complete as a metric space.
\end{itemize}
\end{definition}

A finite-dimensional inner product space is automatically a Hilbert space by proposition \ref{finiteDimComplete}.


\section{Inner product spaces}
\begin{lemma}
An inner product over a complex vector space $V$ is anti-linear in the first component.
\end{lemma}

\begin{lemma} \label{nonDegeneracyInnerProduct}
Definiteness implies the inner product on $V$ is non-degenerate:
\[ [\forall u\in V:\inner{u,v} = 0] \implies v = 0. \]
\end{lemma}
The converse is not true.

There are some generalised notions of inner product:
\begin{definition}
Let $V$ be a complex vector space.
\begin{enumerate}
\item A \udef{sesquilinear form} is a function $V\times V\to \C$ that is linear in the second component and anti-linear in the first.
\item A \udef{Hermitian form} is a conjugate symmetric sesquilinear form.
\item A \udef{pre-inner product} is a positive Hermitian form, i.e.\ an inner product without the requirement of definiteness.
\end{enumerate}
\end{definition}

\begin{example}
\begin{enumerate}
\item The \udef{standard inner product} on $\R^n$ is given by
\[ \inner{a,b} = \inner{\begin{bmatrix}
a_1 \\ \vdots \\ a_n
\end{bmatrix},\begin{bmatrix}
b_1 \\ \vdots \\ b_n
\end{bmatrix}} = \begin{bmatrix}
a_1 & \hdots & a_n
\end{bmatrix}\begin{bmatrix}
b_1 \\ \vdots \\ b_n
\end{bmatrix} = a^\transp b \]
This is also known as the \udef{dot product} $a\cdot b$.
\item The \udef{standard inner product} on $\C^n$ is given by
\[ \inner{a,b} = \inner{\begin{bmatrix}
a_1 \\ \vdots \\ a_n
\end{bmatrix},\begin{bmatrix}
b_1 \\ \vdots \\ b_n
\end{bmatrix}} = \begin{bmatrix}
\bar{a}_1 & \hdots & \bar{a}_n
\end{bmatrix}\begin{bmatrix}
b_1 \\ \vdots \\ b_n
\end{bmatrix} = \bar{a}^\transp b \]
\item The \udef{Frobenius inner product} on $\C^{m\times n}$ is given by
\[ \inner{A,B}_F =  \Tr(\overline{A}^\transp B) = \overline{\vectorisation_C(A)}^\transp \vectorisation_C(B)\]
\item On the vector space $\mathcal{C}[a,b]$ of continuous real functions on $[a,b]$, we can take the inner product
\[ \inner{f,g} = \int_a^b f(x)\cdot g(x) \diff{x}. \]
\end{enumerate}
\end{example}

\begin{definition}
Two vectors $u,v \in V$ are \udef{orthogonal} if $\inner{u,v} =0$. This is denoted $u\perp v$.
\end{definition}
\begin{lemma} \label{elementaryOrthogonality}
Let $V$ be an inner product space.
\begin{enumerate}
\item $0$ is the only vector orthogonal to itself.
\item $0$ is orthogonal to all $v\in V$;
\item Let $x,y\in V$. If, for all $v\in V$, $\inner{v,x} = \inner{v,y}$, then $x=y$.
\end{enumerate}\end{lemma}
\begin{proof}
The first is a consequence of definiteness, the second a consequence of linearity: $\inner{v,0} = \inner{v,0\cdot0} = 0\inner{v,0} = 0$.

The third is also a consequence of linearity: assume $\forall v\in V: \inner{v,x} = \inner{v,y}$, then $\inner{v,x-y}=0$ and $x-y$ is orthogonal to all $v\in V$ and in particular to $0$. Thus $x-y$ must be zero.
\end{proof}

\begin{proposition}
Every inner product gives rise to a norm, defined by
\[ \norm{\cdot} = \sqrt{\inner{\cdot,\cdot}}. \]
\end{proposition}
\begin{proof}
The only non-trivial part is the triangle inequality. This will be proved later using the Cauchy-Schwarz inequality.
\end{proof}


\begin{lemma}
Let $V$ be an inner product space. Then
\[ \norm{v+w}^2 = \norm{v}^2+\norm{w}^2+2\Re\inner{v,w} \]
\end{lemma}
\begin{lemma} \label{orthogonalDecomposition}
Let $v,w\in V$, with $w\neq 0$. We can decompose $v$ as a multiple of $w$ and a vector $u$ orthogonal to $w$:
\[ v = cw+u = \left(\frac{\inner{v,w}}{\norm{w}^2}\right)w + \left( v- \frac{\inner{v,w}w}{\norm{w}^2} \right). \]
\end{lemma}
\begin{proof}
The only thing to check is $\inner{w, v- \frac{\inner{v,w}w}{\norm{w}^2}} = 0$, which is a simple calculation.
\end{proof}

\subsection{Pythagoras and Cauchy-Schwarz}
\begin{theorem}[Pythagorean theorem] \label{Pythagoras}
Suppose $u\perp v$. Then $\norm{u+v}^2 = \norm{u}^2 + \norm{v}^2$.
\end{theorem}
\begin{proof}
\[ \norm{u+v}^2 = \inner{u+v,u+v} = \inner{u,u}+ \inner{u,v} + \inner{v,u} + \inner{v,v} = \norm{u}^2 + \norm{v}^2. \]
\end{proof}

\begin{theorem}[Cauchy-Schwarz-Bunyakovsky inequality.] \label{CauchySchwarz}
Let $V$ be a vector space with a pre-inner product $\inner{\cdot,\cdot}$. Let $v,w\in V$. Then
\[ |\inner{v,w}|^2\leq \inner{v,v}\cdot\inner{w,w}. \]
Suppose $\inner{\cdot,\cdot}$ is definite (i.e.\ an inner product), then
this is an equality \textup{if and only if} $v$ and $w$ are scalar multiples.
\end{theorem}
This result is also known as the Cauchy-Schwarz inequality, or the CSB inequality.
\begin{proof}
Consider 
\[ \inner{v-\lambda w, v-\lambda w} = \inner{v,v}-\lambda\inner{v,w}-\overline{\lambda}\inner{w,v} + |\lambda|^2 \inner{w,w} \geq 0. \]
Suppose $\inner{v,w}=re^{i\theta}$ (if $\mathbb{F} = \R$, then $\theta=0$ or $\theta = \pi$). The inequality must still hold for all $\lambda$ of the form $te^{-i\theta}$ for some $t\in \R$. The inequality thus becomes
\[ 0\leq \inner{v,v}-te^{-i\theta}re^{i\theta}-te^{i\theta}re^{-i\theta} + t^2 \inner{w,w} = \inner{v,v}-2rt + t^2 \inner{w,w}. \]
On the right we have a quadratic formula in $t$. This may never be negative and the discriminant may therefore not be positive. Calculating the discriminant gives $(2r)^2 - 4\inner{v,v}\inner{w,w}$. Thus
\[ 0\geq r^2 - \inner{v,v}\inner{w,w} = |\inner{v,w}|^2 - \inner{v,v}\inner{w,w}. \]
\end{proof}
In the case of an inner product, there is a simpler proof:
\begin{proof}
Take the decomposition from lemma \ref{orthogonalDecomposition} and apply the Pythagorean theorem to obtain
\[ \norm{v}^2 = \frac{|\inner{v,w}|^2}{\norm{w}^2} + \norm{u}^2 \geq \frac{|\inner{v,w}|^2}{\norm{w}^2}. \]
This also shows the claim about scalar multiples.
\end{proof}
\begin{corollary} \label{innerBoundedFunctionals}
Let $V$ be an inner product space. The functions
\[\inner{v,\cdot}: V\to \mathbb{F}: x\mapsto \inner{v,x} \]
are bounded linear functionals for all $v\in V$.
\end{corollary}
\begin{corollary} \label{preInnerProductCSBZero}
Let $V$ be a vector space with a pre-inner product $\inner{\cdot,\cdot}$. Then
\[ \inner{x,x}=0\lor\inner{y,y}=0 \quad\implies\quad \inner{x,y} = 0. \]
\end{corollary}
\begin{definition}
The Cauchy-Schwarz inequality allows us to define the \udef{angle} $\theta$ between two vectors $v,w$ by
\[ \cos\theta = \frac{\inner{v,w}}{\norm{v}\norm{w}}.\]
\end{definition}
\begin{lemma}
If $v\perp w$, then the angle between them is $\pi/2 + k\pi$.
\end{lemma}

TODO CS special case of Hölder inequality.

\begin{theorem}[Triangle inequality]
Let $v,w\in V$. Then
\[ \norm{v+w} \leq \norm{v}+\norm{w} \]
This inequality is an equality if and only if one of $u,v$ is a nonnegative multiple of the other. Also
\begin{enumerate}
\item $\big|\norm{v}-\norm{w}\big|\leq \norm{v-w}$;
\item $\big|\norm{v}-\norm{w}\big| \leq \norm{v+w} \leq \norm{v}+\norm{w}$.
\end{enumerate}
\end{theorem}
\begin{proof}
We calculate
\begin{align*}
\norm{v+w}^2 &= \norm{v}^2+\norm{w}^2+2\Re\inner{v,w} \\
&\leq \norm{v}^2+\norm{w}^2+2|\inner{v,w}| \\
&\leq \norm{v}^2+\norm{w}^2+2\norm{v}\norm{w} \\
&= (\norm{v}+\norm{w})^2.
\end{align*}
The other inequalities are the reverse triangle inequalities \ref{reverseTriangleInequality}.
\end{proof}

\subsection{Parallelogram law and polarisation}
\begin{theorem}[Parallelogram law] \label{parallelogramLaw}
Let $V$ be an inner product space and $v,w\in V$. Then
\[ \norm{v+w}^2 + \norm{v-w}^2 = 2 (\norm{v}^2+\norm{w}^2). \]
\end{theorem}
\begin{proof}
We calculate
\begin{align*}
\norm{v+w}^2 + \norm{v-w}^2 = \inner{v+w, v+w}+\inner{v-w,v-w} = 2(\norm{v}^2 + \norm{w}^2).
\end{align*}
\end{proof}
\begin{corollary}[Appolonius' identity] \label{AppoloniusIdentity}
Let $V$ be an inner product space and $x,y,z\in V$. Then
\[ \norm{z-x}^2 + \norm{z-y}^2 = \frac{1}{2}\norm{x-y}^2 + 2\norm*{z-\frac{1}{2}(x+y)}^2. \]
\end{corollary}
\begin{proof}
Apply the parallelogram law to $u = \frac{1}{2}(z-x)$ and $v = \frac{1}{2}(z-y)$.
\end{proof}

\begin{proposition}[Ptolemy's inequality] \label{PtolemyInequality}
Let $V$ be an inner product space. Then the norm satisfies $\forall u,v,w\in V$
\[ \norm{u-v}\;\norm{w} + \norm{v-w}\;\norm{u} \geq \norm{u-w}\;\norm{v}. \]
\end{proposition}

Polarisation identities allow us to recover the inner product from the norm.
\begin{theorem}[Polarisation identities] \label{polarisationIdentities}
\mbox{}
\begin{enumerate}
\item For real inner product spaces, $\mathbb{F} = \R$:
\begin{align*}
\inner{v,w} &= \frac{1}{2}(\norm{v+w}^2 - \norm{v}^2-\norm{w}^2) \\
&= \frac{1}{2}(\norm{v}^2 + \norm{w}^2-\norm{v-w}^2) \\
&= \frac{1}{4}(\norm{v+w}^2 - \norm{v-w}^2) = \frac{1}{4}\sum_{k=0}^1 (-1)^k\norm{v+(-1)^k w}^2.
\end{align*}
\item For complex inner product spaces, $\mathbb{F} = \C$:
\[ \inner{x,y} = \frac{1}{4}\sum_{k=0}^3 i^k\norm{i^k x+y}^2. \]
\item For general sesquilinear forms:
\[ S(x,y) = \frac{1}{4}\sum_{k=0}^3 i^k S(i^k x+y, i^k x+y). \]
\end{enumerate}
\end{theorem}
\begin{proof}
We prove (3):
\begin{align*}
\sum_{k=0}^3 i^k S(i^k x+y, i^k x+y) &= \sum_{k=0}^3 i^ki^k(-1)^k S(x, x) + i^k(-i)^kS(x,y) + i^ki^kS(y,x) + i^kS(y,y) \\
&= \sum_{k=0}^3 i^k S(x, x) + S(x,y) + (-1)^kS(y,x) + i^kS(y,y) \\
&= \sum_{k=0}^3 S(x,y) \\
&= 4 S(x,y),
\end{align*}
where we have used that $\sum_{k=0}^3 i^k = 0$ and $\sum_{k=0}^3 (-1)^k = 0$.
\end{proof}

\begin{proposition} \label{HermitianRealQuadratic}
A sesquilinear form $S$ is Hermitian (i.e.\ conjugate symmetric) \textup{if and only if} $S(v,v)$ is real for all $v\in V$.

In particular, all positive sesquilinear forms are pre-inner products.
\end{proposition}
\begin{proof}
The direction $\boxed{\Rightarrow}$ is immediate: conjugate symmetry gives $S(v,v) = \overline{S(v,v)}$.

For the other direction, assume $S(v,v)$ is real for all $v\in V$ and consider arbitrary $u,v\in V$. Then
\[ \begin{cases}
S(u+iv, u+iv) = S(u,u) + S(v,v) + i\Big(S(u,v) - S(v,u)\Big) \\
S(u+v, u+v) = S(u,u) + S(v,v) + \Big(S(u,v) + S(v,u)\Big).
\end{cases} \]
Taking the imaginary part gives
\[ \begin{cases}
0 = \Im S(u+iv, u+iv) = \cancel{\Im S(u,u)} + \cancel{\Im S(v,v)} + \Re\Big(S(u,v) - S(v,u)\Big) \\
0 = \Im S(u+v, u+v) = \cancel{\Im S(u,u)} + \cancel{\Im S(v,v)} + \Im\Big(S(u,v) - S(v,u)\Big).
\end{cases} \]
Thus $\Re S(u,v) = \Re S(v,u)$ and $\Im S(u,v) = - \Im S(v,u)$, which means $S(u, v) = \overline{S(v,u)}$.
\end{proof}
\begin{proof}[Alternate proof]
We can also prove the direction $\Leftarrow$ by a direct calculation using the polarisation identity \ref{polarisationIdentities}:
\begin{align*}
\overline{S(u,v)} &= \frac{1}{4}\sum^3_{k=0}(-i)^kS(u+i^kv,u+i^kv) & &\text{Using the fact that $S(u+i^kv,u+i^kv)$ is real} \\
&= \frac{1}{4}\sum^3_{k=0}(-i)^kS\Big((i^k)(v+(-i)^ku),(i^k)(v+(-i)^ku)\Big) & &\text{Using $i^k(-i)^k=1$}\\
&= \frac{1}{4}\sum^3_{k=0}(-i)^k\cancel{(i^k)}\cancel{(-i^k)}S(v+(-i)^ku,v+(-i)^ku) & &\text{Using (conjugate) linearity}\\
&= \frac{1}{4}\sum^3_{k=0}i^kS(v+i^ku,v+i^ku) & &\text{Substituting $k\to k+2$}\\
&= S(v,u).
\end{align*}
\end{proof}
Not all norms on vector spaces can be obtained from an inner product. If a norm can be obtained from an inner product, we can use polarisation to recover the inner product. If a norm cannot be obtained from an inner product, the putative inner product suggested by polarisation will turn out not to be an inner product.
\begin{theorem}[Jordan-von Neumann] \label{JordanVonNeumann}
Let $\sSet{V,\norm{\cdot}}$ be a real or complex normed space. The following are equivalent:
\begin{enumerate}
\item the polarisation yields an inner product;
\item the parallelogram law holds;
\item Appolonius' identity holds;
\item Ptolemy's inequality holds.
\end{enumerate}
\end{theorem}
TODO! (For other fields??)
\begin{proof}
If polarisation yields an inner product, then we have an inner product space and thus the parallelogram law and Ptolemy's inequality hold by \ref{parallelogramLaw} and \ref{PtolemyInequality}.

The polarisation identities immediately imply
\begin{itemize}
\item Conjugate symmetry:
\[ \inner{x,y} = \frac{1}{4}\sum_{k=0}^3i^k\norm{i^k x+ y}^2 = \frac{1}{4}\sum_{k=0}^3i^k\norm{x+ i^{-k}y}^2 = \frac{1}{4}\sum_{k'=0}^3\overline{i^{k'}}\norm{i^{k'}y+ x}^2 = \overline{\inner{y,x}}. \]
\item Positivity and definiteness:
\[ \inner{x,x} = \frac{1}{4}\sum_{k=0}^3i^k\norm{i^k x+ x}^2 = \frac{1}{4}\sum_{k=0}^3i^k\norm{(i^k+1)x}^2 = \frac{\norm{x}^2}{4}\sum_{k=0}^3 i^k\cdot |1+i^k|^2 = \norm{x}^2 \]
\end{itemize}
Now assume Appolonius' identity, \ref{AppoloniusIdentity}, holds. We need to show linearity in second component.
We can calculate
\begin{align*}
\inner{x,y_1} + \inner{x,y_2} &= \frac{1}{4}\sum_{k=0}^3i^k\norm{i^k x+ y_1}^2 + \frac{1}{4}\sum_{k=0}^3i^k\norm{i^k x+ y_2}^2 = \frac{1}{4}\sum_{k=0}^3i^k\Big(\norm{i^k x+ y_1}^2\frac{1}{4} + \norm{i^k x+ y_2}^2\Big) \\
&= \frac{1}{4}\sum_{k=0}^3i^k\left(\frac{1}{2}\norm{y_1-y_2}^2 + 2\norm{i^k+\frac{y_1+y_2}{2}}\right) \\
&= 2\frac{1}{4}\sum_{k=0}^3i^k\left(\norm{i^k+\frac{y_1+y_2}{2}}\right) \\
&= 2\inner{x,\frac{y_1+ y_2}{2}}.
\end{align*}
Setting $y_2 = 0$ and $y_1 = 2y$, this yields $\inner{x,2y} = 2\inner{x,y}$, which also means that
\[ \inner{x,y_1+y_2} = 2\inner{x,\frac{y_1+ y_2}{2}} = \inner{x,y_1} + \inner{x,y_2}. \]
By induction, we can prove that this putative inner product is linear for all positive rational scalars. By continuity this result extends to all positive scalars.

Finally we check
\begin{align*}
\inner{x,-y} &= \frac{1}{4}\sum_{k=0}^3i^k\norm{i^k x - y}^2 = \frac{1}{4}\sum_{k=0}^3i^k\norm{i^{k-2} x + y}^2 = -\frac{1}{4}\sum_{k'=0}^3i^{k'}\norm{i^{k'} x + y}^2 = -\inner{x,y} \\
\inner{x,iy} &= \frac{1}{4}\sum_{k=0}^3i^k\norm{i^k x + iy}^2 = \frac{1}{4}\sum_{k=0}^3i^k\norm{i^{k-1} x + y}^2 = i\frac{1}{4}\sum_{k'=0}^3i^{k'}\norm{i^{k'} x + y}^2 = i\inner{x,y}.
\end{align*}
TODO Ptolemy inequality.
\end{proof}
\begin{corollary}
The space $l^p$ is an inner product space \textup{if and only if} $p=2$.
\end{corollary}
\begin{proof}
The inner product on $l^2$ is defined by $\inner{x_n, y_n} = \sum_{n=1}^\infty \overline{x_n}y_n$.

If $p\neq 2$ we can find a counterexample to the parallelogram law: let $x=(1,1,0,0,\ldots)\in l^p$ and $y = (1,-1,0,0,\ldots)\in l^p$. Then
\[ \norm{x}_p = \norm{y}_p = 2^{1/p} \qquad \text{and} \qquad \norm{x+y} = \norm{x-y} = 2 \]
and the parallelogram law is then not valid if $p\neq 2$.
\end{proof}

\subsection{Topology of inner product spaces}
\begin{lemma}[Continuity of inner product]
Let $V$ be an inner product space. Then the inner product is a continuous function $V\times V \to \mathbb{F}$.
\end{lemma}
\begin{proof}
We show that if $x_n \to x$ and $y_n \to y$, then $\inner{x_n,y_n}\to \inner{x,y}$. By the triangle and Cauchy-Schwarz inequalities
\begin{align*}
|\inner{x_n,y_n}-\inner{x,y}| &= |\inner{x_n,y_n}-\inner{x_n,y}+\inner{x_n,y} - \inner{x,y}| \\
&\leq |\inner{x_n, y_n-y}| + |\inner{x_n-x, y}| \\
&\leq \norm{x_n}\norm{y_n-y} + \norm{x_n-x}\norm{y}.
\end{align*}
Because the right-hand side converges to $0$, the left-hand side must too.
\end{proof}

\subsubsection{Completion}
\begin{lemma}
Let $V$ be an inner product space. The inner product on $V$ can be extended to its completion by continuity. The completion is a Hilbert space.
\end{lemma}
\begin{proof}
\ref{uniformlyContinuousExtensionToCompletion}

The completion of $V$ exists by \ref{existenceMetricCompletion} and the 
\end{proof}

\section{Orthogonal and orthonormal sets of vectors}
\subsection{Orthogonal complements}
\begin{definition}
Let $A$ be a subset of an inner product space $V$. The \udef{orthogonal complement} $A^\perp$ of $A$ is the set of vectors in $V$ that are orthogonal to every vector in $A$:
\[ A^\perp = \{ v\in V\;|\; \inner{v,a}=0\; \forall a\in A \}. \]
\end{definition}

\begin{proposition} \label{OrthogonalComplementProperties}
Let $A,B$ be \emph{subsets} of an inner product space $V$.
\begin{enumerate}
\item $A^\perp$ is a subspace of $V$;
\item $A^\perp = \Span(A)^\perp$;
\item $\{0\}^\perp = V$;
\item $V^\perp = \{0\}$;
\item $A\cap A^\perp \subset \{0\}$;
\item If $A\subset B$, then $B^\perp \subset A^\perp$.
\end{enumerate}
\end{proposition}

We can also consider the orthogonal complement of a subspace with respect to another subspace, not the full space.
\begin{definition}
Let $A\subseteq B$ be subsets of an inner product space $V$. The \udef{orthogonal complement} of $A$ with respect to $B$ is the set of vectors in $B$ that are orthogonal to every vector in $A$:
\[ B\ominus A = \{ b\in B\;|\; \inner{b,a}=0\; \forall a\in A \}. \]
\end{definition}

\begin{lemma} \label{ominusSubspace}
Let $V$ be an inner product space and $A\subseteq B \subseteq V$ subsets. Then $B\ominus A = B\cap A^\perp$.
\end{lemma}
\begin{proof}
Take $v\in B\ominus A$. This is equivalent to $v\in B \land \forall u\in A: \inner{v,u} =0$ and thus equivalent to $v\in B \land v\in A^\perp$.
\end{proof}

\begin{proposition} \label{perpUnderIsometry}
Let $V$ be an inner product space, let $A\subseteq B\subseteq V$ be subsets and $T:V\to V$ an isometry. Then
\begin{enumerate}
\item if $A\perp B$, then $T[A]\perp T[B]$;
\item $T[B\ominus A] = T[B]\ominus T[A]$.
\end{enumerate}
\end{proposition}
\begin{proof}
(1) If $\inner{a,b}=0$ for all $a\in A, b\in B$, then $\inner{T(a), T(b)} =0$, meaning $T[A]\perp T[B]$.

(2) Take $v\in T[B\ominus A]$. This is equivalent to the existence of $x\in B$ such that $T(x) = v$ and $\inner{x,y}=0$ for all $y\in A$. By isometry $\inner{x,y}=0 \iff \inner{T(x), T(y)}=0$ for all $y\in A$. So, equivalently, $v\in T[B]\ominus T[A]$. 
\end{proof}

\begin{proposition} \label{orthogonalComplementClosed}
Let $A$ be a \emph{subset} of an inner product space $V$. Then $A^\perp$ is closed and $\overline{A}^\perp = A^\perp$. This can be rephrased as
\[ \overline{A}^\perp = \overline{A^\perp} = A^\perp. \]
Also
\[\overline{A} \subset (A^\perp)^\perp. \]
\end{proposition}
\begin{proof}
Let $x\in \overline{A^\perp}$. Then there exists a sequence $(x_i)$ in $A^\perp$ that converges to $x$. For all $a\in A$, the functional $\inner{a,\cdot}:y\mapsto \inner{a,y}$ is bounded (by Cauchy-Schwarz). Thus all these functionals are continuous. Applying any one to the sequence $x_i$ gives a sequence of zeros. Thus $\inner{a,x} = 0$ for all $a\in A$. Thus $x\in A^\perp$ and hence $A^\perp \supset \overline{A^\perp}$ meaning $A^\perp$ is closed.

Now $\overline{A}\supset A$, so $\overline{A}^\perp \subset A^\perp$. For the other inclusion, take an $x\in A^\perp$. Take an arbitrary $y\in \overline{A}$. Then there exists a sequence $(y_i)$ in $A$ that converges to $y$. Apply the bounded functional $\inner{x,\cdot}$ to the sequence $(y_i)$, yielding a sequence of zeros. Thus $\inner{x,y}=0$. Thus $x\in \overline{U}^\perp$.

Finally let $a\in \overline{A}$. Take a sequence $a_i\to a$. Take an arbitrary element $x\in A^\perp$. As before $\inner{x,a} = \lim_i\inner{x,a_i} = 0$. So $a\in (A^\perp)^\perp$.
\end{proof}
\begin{corollary} \label{orthogonalComplementDenseSpace}
Let $A$ be a subset of $V$. If $\Span(A)$ is dense in $V$, then $A^\perp = \{0\}$. 
\end{corollary}
\begin{proof}
\[ A^\perp = \Span(A)^\perp = \overline{\Span(A)}^\perp = V^\perp = \{0\}. \]
\end{proof}
\begin{corollary} \label{perpToDenseSet}
Let $x\in V$. If there exists a dense set $S$ such that $x\perp y$ for all $y\in S$, then $x=0$.
\end{corollary}
\begin{proof}
If such an $S$ exists, then $x\in S^\perp = \overline{S}^\perp = V^\perp = \{0\}$.
\end{proof}
\begin{proposition}
Let $U$ be a finite-dimensional subspace of an inner product space $V$.
\begin{enumerate}
\item $V=U\oplus U^\perp$;
\item $U = (U^\perp)^\perp$.
\end{enumerate}
\end{proposition}
Notice that $V$ may be infinite dimensional!
\begin{proof}
We start with the first point. The sum $U + U^\perp$ is definitely direct, $U\oplus U^\perp$, by proposition \ref{OrthogonalComplementProperties} and the criterion for a direct sum, proposition \ref{directSumCriterion}. Clearly $U\oplus U^\perp\subseteq V$, so we just need to show that $V \subseteq U\oplus U^\perp$.

To that end, take a vector $v\in V$. Let $\{e_i\}_{i=1}^n$ be an orthonormal basis of $U$. We can write
\[ v = \left(v - \sum_{i=1}^n\inner{v,e_i}e_i\right) + \left(\sum_{i=1}^n\inner{v,e_i}e_i\right). \]
The first part is an element of $U^\perp$, the second of $U$, so $v\in U\oplus U^\perp$.

For the second point: any finite-dimensional subspace $U$ is automatically closed, so $U = \overline{U} \subset (U^\perp)^\perp$, by proposition \ref{orthogonalComplementClosed}. For the other inclusion, take $v\in (U^\perp)^\perp$. By the first point, we can write $v = v_1 + v_2$ where $v_1\in U$ and $v_2\in U^\perp$. Because $v\in (U^\perp)^\perp$ and $v_2\in U^\perp$, we must have
\[ 0 = \inner{v_2, v} = \inner{v_2, v_1+v_2} = \inner{v_2, v_1} + \inner{v_2,v_2} = \norm{v_2}. \]
So $v=v_1\in U$.
\end{proof}

TODO all projection results for projection onto finite dim? See proposition before Bessel inequality. In fact better: projection onto summand of direct sum! Put under decompositions.

\begin{proposition} \label{linearDeMorgan}
Let $W_1,W_2$ be subspaces of an inner product space $V$. Then
\[ (W_1+W_2)^\perp = W_1^\perp \cap W_2^\perp. \]
\end{proposition}
\begin{proof}
For a vector $v\in V$,
\[  v\in (W_1+W_2)^\perp \implies \forall x\in W_1\cup W_2: \inner{v,x} = 0 \implies v\in W_1^\perp \cap W_2^\perp \]
and
\begin{align*}
v\in W_1^\perp \cap W_2^\perp &\implies \forall x\in W_1, y\in W_2: \inner{v,x} = 0 = \inner{v,y} \\
&\implies \forall x\in W_1, y\in W_2:\inner{v, x+y} = 0 \implies v\in (W_1+W_2)^\perp.
\end{align*}
\end{proof}

A result dual to proposition \ref{linearDeMorgan} holds for finite-dimensional spaces:
\begin{proposition}
Let $W_1,W_2$ be subspaces a finite-dimensional space $V$. Then
\[ (W_1\cap W_2)^\perp = W_1^\perp + W_2^\perp. \]
\end{proposition}
\begin{proof}
We start by applying proposition \ref{linearDeMorgan} to $W_1^\perp$ and $W_2^\perp$:
\[ (W_1^\perp+W_2^\perp)^\perp = (W_1^\perp)^\perp \cap (W_2^\perp)^\perp = W_1 \cap W_2. \]
Taking the orthogonal complement of both sides gives the result. In infinite dimensions $(W_1^\perp+W_2^\perp)$ is not necessarily closed. 
\end{proof}

\subsection{Orthogonal sets and sequences}
\begin{definition}
\begin{itemize}
\item A set of vectors $D$ is called \udef{orthogonal} if for any two vectors $v,w\in D$, $v\perp w$ \textup{if and only if} $v\neq w$.
\item A set of vectors $D$ is called \udef{orthonormal} if for any two vectors $v,w\in D$,
\[ \inner{v,w} = \begin{cases}
0 & (v\neq w) \\ 1 & (v=w)
\end{cases}. \]
\end{itemize}
In particular an orthonormal set is an orthogonal set of unit vectors.
\end{definition}

\begin{lemma} \label{orthogonalLinearlyIndependent}
Every orthogonal set of vectors is linearly independent.
\end{lemma}
\begin{lemma}
Every subset of an orthogonal (resp. orthonormal) set is orthogonal (resp. orthonormal).
\end{lemma}

\begin{theorem}[Gram-Schmidt procedure]
Every finite set of linearly independent vectors $D = \{v_1,\ldots, v_n\}$ can be transformed into an orthonormal set $D' = \{e_1,\ldots,e_n\}$ with the same number of vectors such that the spans are the same: $\Span(D') = \Span(D)$.
\end{theorem}
\begin{proof}
The procedure goes as follows:
\begin{align*}
e_1 &= \frac{v_1}{\norm{v_1}} \\
e_2 &= \frac{v_2 - \inner{e_1,v_2}e_1}{\norm{v_2 - \inner{e_1,v_2}e_1}} \\
&\hdots \\
e_j &= \frac{v_j - \inner{e_1,v_j}e_1- \ldots - \inner{e_{j-1},v_j}e_{j-1}}{\norm{v_2 - \inner{e_1,v_2}e_1- \ldots - \inner{e_{j-1},v_j}e_{j-1}}} \\
&\hdots
\end{align*}
\end{proof}

If we only need an orthogonal set $\{y_1,\ldots,y_n\}$, not an orthonormal one, we can use the procedure
\[ y_{k+1} = v_{k+1} - \sum_{i=1}^k \frac{\inner{v_{k+1}, y_i}}{\inner{y_i,y_i}}y_i. \]

\begin{lemma} \label{orthogonality}
Let $(\mathbb{F}, V,+,\inner{\cdot,\cdot})$ be an inner product space. Then
\[ \inner{v,w}=0 \qquad \iff \qquad \forall a\in\mathbb{F}:\;\norm{v}\leq\norm{v+aw}.  \]
\end{lemma}
\begin{proof}
The implication $\Rightarrow$ is a consequence of the Pythagorean theorem. For the other implication, assume $\forall a\in\mathbb{F}:\;\norm{v}\leq\norm{v+aw}$. Then
\[ \norm{v}^2 \leq \norm{v-aw}^2 = \norm{v}^2 - 2\Re\inner{v,aw} + \norm{aw}^2 \]
which implies $2\Re\inner{v,aw} \leq a^2\norm{w}^2$. Let $\inner{v,w} = re^{i\theta}$. (If $\mathbb{F} = \R$, then $\theta=0$.) Then in particular the inequality holds for all $a=te^{i\theta}$ with $t\in\R$. This yields
\[ 2\Re(te^{-i\theta}re^{i\theta}) \leq t^2\norm{w}^2 \qquad \text{or}\qquad 2rt\leq t^2\norm{w}^2. \]
Letting $t\geq 0$, we can divide out a $t$: $2r\leq t\norm{w}^2$. Then letting $t\to 0$ gives $r=0$ and thus $\inner{v,w}=0$.
\end{proof}

\begin{proposition}
Let $V$ be an inner product space and $D = \{e_1,\ldots, e_n\}$ a finite orthonormal set of vectors. Then $\forall v\in V$
\[ \inf_{c_i\in\mathbb{F}}\norm{v-\sum_{i=1}^nc_ie_i} = \norm{v-\sum_{i=1}^n\inner{e_i,v}e_i} \]
\end{proposition}
\begin{proof}
We calculate
\begin{align*}
\norm{v-\sum_{i=1}^nc_ie_i}^2 &= \inner{v-\sum_{i=1}^nc_ie_i,v-\sum_{j=1}^nc_je_j} \\
&= \norm{v} - \sum_{j=1}^n c_j\inner{v,e_j} - \sum_{i=1}^n\bar{c}_i\inner{e_i,v} + \sum_{i,j=1}^n\bar{c}_ic_j\inner{e_i,e_j} \\
&= \norm{v} - 2\Re\left(\sum_{i=1}^nc_i\overline{\inner{e_i,v}}\right) + \sum_{i=1}^n|c_i|^2 \\
&= \sum_{i=1}^n\left(|c_i|^2 - 2\Re\left(\sum_{i=1}^nc_i\overline{\inner{e_i,v}}\right) + |\inner{e_i,v}|^2\right) +\norm{v} - \sum_{i=1}^n|\inner{e_i;v}|^2 \\
&= \sum_{i=1}^n|c_i - \inner{e_i,v}|^2 +\norm{v} - \sum_{i=1}^n|\inner{e_i,v}|^2.
\end{align*}
This is clearly minimised when $c_i = \inner{e_i,v}$.
\end{proof}
\begin{corollary}
Let $v\in\Span(D)$, then $v = \sum_{i=1}^n \inner{e_i,v}e_i$.
\end{corollary}
We call the numbers $\inner{e_i,v}$ the \udef{Fourier coefficients} of $v$ w.r.t. $D$.
\begin{proof}
In this case $\inf_{c_i\in\mathbb{F}}\norm{v-\sum_{i=1}^nc_ie_i} = 0$.
\end{proof}
\begin{corollary}[Bessel inequality] \label{BesselInequality}
Let $\{e_i\}_{i\in I}$ be an orthonormal family and $v\in V$, then
\[ \sum_{i\in I}|\inner{e_i,v}|^2 = \sup \left\{\sum_{\substack{i\in I' \subset I\\ I' \;\text{finite}}} |\inner{e_i,v}|^2 \right\} \leq \norm{v}^2. \]
\end{corollary}
\begin{proof}
In the previous proof,
\[ 0 \leq \norm{v-\sum_{i=1}^nc_ie_i}^2 = \sum_{i=1}^n|c_i - \inner{e_i,v}|^2 +\norm{v} - \sum_{i=1}^n|\inner{e_i,v}|^2 = \norm{v} - \sum_{i=1}^n|\inner{e_i,v}|^2. \]
Where we have set $c_i = \inner{e_i,v}$. Thus the supremum must also be $\leq \norm{v}$.
\end{proof}
\begin{corollary}
For any $v\in V$, $\inner{e_i,v} = 0$ except for countably many $i\in I$. \label{countableComponents}
\end{corollary}
\begin{proof}
Immediate from \ref{finiteSumsAreCountable}.
\end{proof}
TODO: link with metric topology being sequential?

\begin{corollary}[Riemann-Lebesgue lemma] \label{RiemannLebesgueLemma}
For any sequence $\seq{e_i}_{i\in J \subset I}$, we have
\[ \lim_{i\in J} \inner{e_i,v} = 0. \]
\end{corollary}

\begin{corollary}
We can also obtain the Cauchy-Schwarz inequality from the Bessel inequality.
\end{corollary}
\begin{proof}
Let $x,y\in V$. Then $\{x/\norm{x}\}$ is an orthonormal set. Applying the Bessel inequality for $y$ gives $\norm{y}^2 \geq |\inner{x/\norm{x}, y}|^2 \implies |\inner{x,y}|^2\leq \norm{x}^2\norm{y}^2 \implies |\inner{x,y}| \leq \norm{x}\;\norm{y}$.
\end{proof}

\subsection{Orthonormal bases}
\begin{definition}
Let $D$ be an orthonormal set of vectors in an inner product space $V$, then $D$ is said to be
\begin{enumerate}
\item \udef{maximal}, if it is a maximal element in the set of orthonormal sets ordered by inclusion;
\item \udef{total}, if the smallest closed subspace that includes $D$ is $V$ (i.e.\ $\Span(D)$ is dense in $V$);
\item an \udef{orthonormal basis} (o.n. basis) or a \udef{Hilbert basis} if any vector in $V$ can be written as a (possibly infinite) linear combination of elements of $D$.
\end{enumerate}
\end{definition}
\begin{note}
Hilbert bases are in general not Hamel bases.  e.g\, take $\R^\mathbb{N}$. Then 
\begin{align*}
(1,0,0,&\ldots), \\
(0,1,0,&\ldots), \\
(0,0,1,&\ldots), \\
&\ldots
\end{align*}
is an orthonormal basis, but not a Hamel basis (consider $(1,1,1,\ldots)$).
\end{note}

\begin{proposition} \label{exitenceMaximalOrthonormalSet}
\begin{itemize}
\item Every vector inner product space has a maximal orthonormal set.
\item Every orthonormal set can be extended to a maximal orthonormal set.
\end{itemize}
\end{proposition}
\begin{proof}
The first statement follows easily from the second. The second statement is proved using Zorn's lemma. Let $S$ be an orthonormal set. Define
\[ \mathcal{A} = \{ D\subset V \;|\; S\subset D \; \text{and $D$ is orthonormal} \} \]
ordered by inclusion. It is easy to see that any chain on $\mathcal{A}$ has an upper bound on $\mathcal{A}$, by just taking the union which is still orthonormal. It follows from Zorn's lemma that $\mathcal{A}$ has a maximal element $R$. This is by definition an orthonormal basis.

In the finite-dimensional case this can also be proved using the Gram-Schmidt procedure.
\end{proof}

\begin{proposition}
If $V$ is finite-dimensional, then the notions of maximal orthonormal set, total orthonormal set and orthonormal set coincide. Such an orthonormal set is also a (Hamel) basis of $V$.
\end{proposition}
\begin{proof}
Corollaries of Gram-Schmidt.
\end{proof}

\begin{lemma} \label{characterisationMaximalOrthonormalSet}
Let $V$ be an inner product space and $D$ an orthonormal set. Then
\begin{enumerate}
\item $D$ is maximal \textup{if and only if} $D^\perp = \{0\}$;
\item if $D$ is an orthonormal basis, then $D$ is maximal.
\end{enumerate}
\end{lemma}
\begin{proof}
(1) All possible vectors with which to extend $D$ are elements of $D^\perp$. 

(2) Assume $D$ an o.n. basis. Then $D^\perp = (\Span(D))^\perp = V^\perp = \{0\}$, using \ref{OrthogonalComplementProperties} and \ref{orthogonalComplementDenseSpace}.
\end{proof}
There are maximal orthonormal families that are not bases.
\begin{example}
Consider the space $l^2(\N)$ and take the subspace $X$ generated by the family of elements
\[ \left( \sum_{n=1}^\infty n^{-1}e_n, e_2,e_3,e_4,\ldots \right) \]
with the inner product induced by the inner product of $l^2$. In this space $F=\{e_2,e_3,\ldots\}$ is orthonormal and maximal, but not an orthonormal basis.
\end{example}

Maximal orthonormal families are easy to construct, but do not have the nice properties of orthonormal bases (see below). We would really like the concepts of orthonormal basis and maximal orthonormal family to coincide. We will see they coincide exactly in Hilbert spaces (see \ref{criterionHilbertSpace}).

\begin{proposition} \label{totalONBParsevalEquivalence} \label{plancherel}
Let $V$ be an inner product space and $D = \{e_i\}_{i\in I}$ an orthonormal set. The following are equivalent:
\begin{enumerate}
\item $D$ is an orthonormal basis of $V$;
\item $D$ is total in $V$;
\item for all $v,w\in V$,
\[ \inner{v,w} = \sum_{i\in I}\inner{v,e_i}\inner{e_i,w}; \]
\item \textup{(Parseval's identity)} for all $v\in V$,
\[ \norm{v}^2 = \sum_{i\in I}|\inner{e_i,v}|^2; \]
\item for all $v\in V$: if $v\perp D$, then $v=0$;
\item \textup{(Plancherel formula)} for all $v\in V$,
\[ v = \sum_{i\in I}\inner{e_{i},v}e_{i}. \]
\end{enumerate}
\end{proposition}
\begin{proof}
We proceed round-robin-style.
\begin{itemize}[leftmargin=2cm]
\item[$\boxed{(1) \Rightarrow (2)}$] Assume $D$ an o.n. basis. Then there exists a net of partial sums converging to any element $v\in V$. Each of these partial sums is a finite linear combination of elements in $D$ and thus this net is a net in $\Span(D)$. This means $v\in\overline{\Span(D)}$.
\item[$\boxed{(2) \Rightarrow (3)}$] Fix $v,w\in V$. Because $V$ is a metric spaces and thus sequential, we can find sequences $(v_j)_{j\in J}$ and $(w_k)_{k\in K}$ in $\Span(D)$ converging to $v$ and $w$. Now the linear maps $u\mapsto \overline{\inner{u, e_i}}$ and $u\mapsto \inner{e_i, u}$ are bounded by Cauchy-Schwarz and thus continuous by theorem \ref{boundedLinearMaps} (TODO corollary CSB). Then we can calculate, using the fact that each $v_j$ and $w_k$ is a finite linear combination of $e_i$,
\begin{align*}
\inner{v,w} &= \inner{\lim_{j}v_j, \lim_k w_k} = \lim_{j}\lim_{k}\inner{v_j,w_k} \\
&= \lim_{j}\lim_{k}\inner{\sum_{i=1}^{N_{j}}\inner{e_i,v_j}e_i,\sum_{i'=1}^{N_k}\inner{e_{i'},w_k}e_{i'}} \\
&= \lim_{j}\lim_{k}\sum_{i=1}^{N_{j}}\sum_{i'=1}^{N_k}\inner{v_j,e_i}\inner{e_{i'},w_k}\inner{e_i,e_{i'}} = \lim_{j}\lim_{k}\sum_{i=1}^{N_{j}}\sum_{i'=1}^{N_k}\inner{v_j,e_i}\inner{e_{i'},w_k}\delta_{i,i'} \\
&= \lim_{j}\lim_{k}\sum_{i=1}^{\min\{N_{j},N_{k}\}}\inner{v_j,e_i}\inner{e_i,w_k} \\
&= \lim_{j}\lim_{k}\sum_{i\in I}\inner{v_j,e_i}\inner{e_i,w_k} \\
&= \sum_{i\in I}\lim_{j}\lim_{k}\inner{v_j,e_i}\inner{e_i,w_k} \\
&= \sum_{i\in I}\inner{v,e_i}\inner{e_i,w}.
\end{align*}
For the interchange of the limits and the summation in the penultimate equality we can use Tannery's theorem, \ref{tannery}. Indeed $|\inner{e_i,w_k}|$ is bounded by $\norm{w_k}$ by the Bessel inequality. By the continuity of the norm we have $\lim_k \norm{w_k} = \norm{w}$, so the sequence $\norm{w_k}$ is bounded.
\item[$\boxed{(3) \Rightarrow (4)}$] Set $v=w$.
\item[$\boxed{(4) \Rightarrow (5)}$] If $v\perp D$, then
\[ \norm{v}^2 = \sum_{i\in I}|\inner{e_i,v}|^2 = 0 \qquad\text{which implies $v=0$.} \]
\item[$\boxed{(5) \Rightarrow (6)}$] The vector $v-\sum_{i\in I}\inner{e_i,v}e_i$ is perpendicular to $D$:
\[ \forall e_j\in D: \quad \inner{e_j, v-\sum_{i\in I}\inner{e_i,v}e_i} = \inner{e_j, v}-\sum_{i\in I}\inner{e_i,v}\inner{e_j,e_i} = \inner{e_j, v} - \inner{e_j, v} = 0. \]
So $v-\sum_{i\in I}\inner{e_i,v}e_i = 0$ and the Plancherel formula holds.
\item[$\boxed{(6) \Rightarrow (1)}$] By definition of o.n. basis.
\end{itemize}
\end{proof}

\begin{lemma}
Let $V$ be an inner product space. If $D$ is an orthonormal basis of $V$, then it is also an orthonormal basis of $\overline{V}$, the completion of $V$.
\end{lemma}
\begin{proof}
Let $D$ be an o.n. basis. By \ref{totalONBParsevalEquivalence} $\Span(D)$ is dense in $V$, meaning it is also dense in $\overline{V}$, by \ref{denseSubsetOfDenseSubspaceIsDense}. Thus $D$ is total in $\overline{V}$ and an o.n. basis by \ref{totalONBParsevalEquivalence}.
\end{proof}

\subsubsection{Cardinality and separable inner product spaces}
\url{https://arxiv.org/pdf/1606.03869.pdf}
\begin{definition}
An inner product space is \udef{separable} if it is separable as a metric space, i.e.\ it admits a countable dense subset.
\end{definition}

\begin{proposition}
Given a vector space $V$, any two maximal orthonormal sets have the same cardinality.
\end{proposition}
\begin{proof}
Take $D = \{e_i\}_{i\in I}$ and $D' = \{f_j\}_{j\in J}$ maximal orthonormal sets.
\end{proof}

\begin{proposition}
An inner product space is separable \textup{if and only if} it admits an orthonormal basis with at most countably many vectors.
\end{proposition}
\begin{proof}
TODO infinite-dimensional analog of the Gram-Schmidt process
\end{proof}
\begin{corollary}
Any separable inner product space has an orthonormal basis.
\end{corollary}

\begin{proposition}
Not every inner product space has an orthonormal basis.
\end{proposition}
\begin{proof}
\url{https://en.wikipedia.org/wiki/Inner_product_space#Orthonormal_sequences}
\url{https://groups.google.com/g/sci.math.research/c/1SA_3h1whQo?pli=1}
\url{https://www.angelfire.com/journal/mathematics/innerproduct.pdf}
\url{https://arxiv.org/pdf/1009.1441.pdf}
\end{proof}


\section{Maps on inner product spaces}

\begin{lemma} \label{equalityOfMapsInnerProductSpaces}
Let $V$ be an inner product space and $S,T\in\Hom(V)$. Then $S=T$ \textup{if and only if}
\[ \forall v,w\in V: \inner{Tv,w} = \inner{Sv,w}. \]
\end{lemma}
\begin{proof}
The direction $\boxed{\Rightarrow}$ is obvious. For the other direction, use
\[ 0 = \inner{Tv,w} - \inner{Sv,w} = \inner{(T-S)v,w} \]
for all $v,w$. In particular set $w$ equal to $(T-S)v$. Then $\norm{(T-S)v} = 0$ for all $v\in V$. By the definiteness of the norm we have $(T-S)v = 0$, meaning $Tv = Sv$.
\end{proof}

\subsection{Bounded operators}
\begin{lemma} \label{operatorNormInnerProduct}
Let $T\in\Bounded(V,W)$, then
\begin{align*}
\norm{T} &= \sup_{w\in \im(T),v \in \dom(T)} \frac{|\inner{w,Tv}|}{\norm{w}\,\norm{v}} \\
&= \sup\setbuilder{|\inner{w,Tv}|}{w\in \im(T)\;\land\; v\in\dom{T}\;\land\; \norm{w} = 1 = \norm{v}} \\
&= \sup_{w\in W,v \in \dom(T)} \frac{|\inner{w,Tv}|}{\norm{w}\,\norm{v}} \\
&= \sup\setbuilder{|\inner{w,Tv}|}{w\in W\;\land\; v\in\dom{T}\;\land\; \norm{w} = 1 = \norm{v}}.
\end{align*}
\end{lemma}
\begin{proof}
We prove
\[ \norm{T} \leq \sup_{w\in \im(T),v \in \dom(T)} \frac{|\inner{w,Tv}|}{\norm{w}\,\norm{v}} \leq \sup_{w\in W,v \in \dom(T)} \frac{|\inner{w,Tv}|}{\norm{w}\,\norm{v}} \leq \norm{T}. \]
The first two inequalities follow from the characterisation \ref{operatorNorm}
\[ \norm{T} = \sup_{v \in \dom(T)} \frac{\norm{Tv}}{\norm{v}} = \sup_{v \in \dom(T)} \frac{\inner{Tv,Tv}}{\norm{Tv}\,\norm{v}} \]
and the inclusions
\begin{align*}
\setbuilder{\frac{|\inner{w,Tv}|}{\norm{w}\,\norm{v}}}{v\in\dom(T), w = Tv} &\subseteq \setbuilder{\frac{|\inner{w,Tv}|}{\norm{w}\,\norm{v}}}{v\in\dom(T), w\in\im(T)}\\
&\qquad\quad\subseteq \setbuilder{\frac{|\inner{w,Tv}|}{\norm{w}\,\norm{v}}}{v\in\dom(T), w\in V}.
\end{align*}
The last equality follows from the Cauchy-Schwarz inequality \ref{CauchySchwarz}:
\[ \frac{|\inner{w,Tv}|}{\norm{w}\,\norm{v}} \leq \frac{\norm{w}\,\norm{Tv}}{\norm{w}\,\norm{v}} = \frac{\norm{Tv}}{\norm{v}} \leq \frac{\norm{T}\,\norm{v}}{\norm{v}} = \norm{T} \]
for all $v\in\dom(T), w\in V$. 
\end{proof}

\subsection{Isometries}
\begin{lemma}
Let $V,W$ be inner product spaces. Let $f:V\to W$ be a function. Then $f$ preserves the metric (i.e.\ is an isometry) \textup{if and only if} $f$ also preserves the inner product:
\[ \forall x,y \in V: \quad \inner{f(x),f(y)}_W = \inner{x,y}_V. \]
\end{lemma}
The proof is a simple application of the polarisation identities.

\begin{definition}
Let $V,W$ be an inner product spaces. A linear map $U\in\Hom(V,W)$ is called \udef{unitary} if it is an isometry and invertible.

Unitary operators on real vector spaces are also called \udef{orthogonal operators}.
\end{definition}
Because every isometry is injective (see lemma \ref{isometryLemma}), it is enough for a linear map to be isometric and surjective to be unitary.

\begin{lemma}
Every unitary map is bounded and has norm $1$.
\end{lemma}
\begin{proof}
Let $U: V\to W$ be a unitary map between inner product spaces. Then $\forall v\in V: \norm{U(v)} = \norm{v}$.
\end{proof}

Unitary operators transform orthonormal bases to orthonormal bases:
\begin{proposition}
Let $T\in \Hom(V,W)$ with $V,W$ inner product spaces and let $V$ have an orthonormal basis $\{e_i\}_{i\in I}$. Then $T$ is unitary \textup{if and only if} $\{Te_i\}_{i\in I}$ is an orthonormal basis of $W$.
\end{proposition}
\begin{proof}
Assume $T$ unitary. The family $\{Te_i\}_{i\in I}$ is certainly orthonormal, by preservation of the inner product. Now let $w\in W$ and so $T^{-1}w\in V$. By the Plancherel formula, proposition \ref{plancherel}, we can write
\[ T^{-1}w = \sum_{n=1}^\infty \inner{e_{i_n},T^{-1}w}e_{i_n} = \lim_{N\to\infty}\sum_{n=1}^N \inner{e_{i_n},T^{-1}w}e_{i_n} \]
and so
\[ w = TT^{-1}w = T\lim_{N\to\infty}\sum_{n=1}^N \inner{e_{i_n},T^{-1}w}e_{i_n} = \lim_{N\to\infty}\sum_{n=1}^N \inner{e_{i_n}T^{-1}w}Te_{i_n} \]
because $T$ is bounded and thus continuous, by theorem \ref{boundedLinearMaps}.
Thus $\{Te_i\}_{i\in I}$ is an orthonormal basis of $W$.

Conversely, assume $\{Te_i\}_{i\in I}$ is an orthonormal basis of $W$. We first prove $T$ is bounded, which is a simple application of Parseval's identity, proposition \ref{totalONBParsevalEquivalence}:
\[ \norm{Tv}^2 = \sum_{i\in I}|\inner{Te_i,Tv}|^2 = \sum_{i\in I}|\inner{e_i,v}|^2 = \norm{v}^2. \]
The rest of the proof is again an application of the Plancherel formula.
\end{proof}

\begin{lemma}
Let $U$ be a unitary map. If $\lambda$ is an eigenvalue of $U$, then $|\lambda| = 1$.
\end{lemma}
\begin{proof}
Let $v$ be an eigenvector associated to the eigenvalue $\lambda$. Then
\[ \inner{v,v} = \inner{L(v),L(v)} = \inner{\lambda v, \lambda v} = \lambda^2\inner{v,v},  \]
so $\lambda^2 = 1$.
\end{proof}

\subsection{Symmetric operators}
\begin{definition}
Let $(\mathbb{F},V,+,\inner{\cdot,\cdot})$ be an inner product space. A linear operator $L$ is called \udef{symmetric} if, $\forall v,w\in \dom(L)$
\[ \inner{L(v),w} = \inner{v,L(w)}. \]
\end{definition}

\begin{proposition}
Let $V$ be an inner product space and $L$ a symmetric operator on $V$. Then eigenvectors of $L$ associated to different eigenvalues are orthogonal.
\end{proposition}
\begin{proof}
Let $v,w$ be eigenvectors of $L$ with eigenvalues $\lambda, \mu$ such that $\lambda \neq \mu$. Then
\[ \lambda\inner{v,w} = \inner{\lambda v,w}=\inner{L(v),w} = \inner{v,L(w)} = \inner{v,\mu w} = \mu \inner{v,w} \]
and consequently $\inner{v,w} =0$.
\end{proof}

\subsection{Impact on subspaces}
\subsubsection{Invariant and reducing subspaces}
\begin{definition}
Let $V$ be an inner product space and $T$ a linear operator on $V$.
\begin{itemize}
\item A subspace $U\subseteq V$ is said to be \udef{invariant} under $T$ if $T[U] \subset U$.
\item A subspace $U\subseteq V$ is said to be \udef{reducing} for $T$ if both $U$ and $U^\perp$ are invariant under $T$.
\end{itemize}
\end{definition}

\section{Energy forms}
\begin{definition}
Let $T$ be an operator on an inner product space $V$. The \udef{energy form} of $T$ is the map
\[ \inner{\cdot, \cdot}_T: \dom(T)\times \dom(T) \to \F: (x,y) \mapsto \inner{x,Ty}. \]
We also define the associated quadratic form
\[ Q_T: \dom(T)\to \F: x\mapsto \inner{x,x}_T = \inner{x,Tx}. \]
\end{definition}
Energy forms are clearly sesquilinear.

\begin{lemma} \label{quadraticFormInverseOperator}
Let $T$ be an invertible operator. Then
\[ Q_{T^{-1}}(x) = \overline{Q_T(T^{-1}(x))}. \]
\end{lemma}
\begin{proof}
For all $x\in V$
\[ Q_{T^{-1}}(x) = \inner{x,T^{-1}x} = \inner{TT^{-1}x, T^{-1}x} = \overline{\inner{T^{-1}x, T(T^{-1}x)}} = \overline{Q_T(T^{-1}(x))}. \]
\end{proof}

\begin{lemma} \label{sameEnergyFormSameOperator}
Two operators $T_1,T_2\in \Lin(V)$ have the same energy form \textup{if and only if} $T_1 = T_2$.
\end{lemma}
\begin{proof}
This is a consequence of \ref{elementaryOrthogonality}.
\end{proof}

\begin{lemma} \label{energyFormHermitianSymmetric}
Let $V$ be an inner product space and $T$ an operator on $V$. Then the following are equivalent:
\begin{enumerate}
\item $T$ is symmetric;
\item $\inner{\cdot,\cdot}_T$ is Hermitian;
\item $Q_T$ is real-valued.
\end{enumerate}
\end{lemma}
\begin{proof}
$(1) \Leftrightarrow (2)$ For all $x,y\in \dom(T)$ we have $\inner{x,Ty} = \inner{Tx, y}$ if and only if $\inner{x,Ty} = \overline{\inner{y, Tx}}$.

$(2) \Leftrightarrow (3)$ Given by \ref{HermitianRealQuadratic}.
\end{proof}

So the energy form associated to a symmetric operator is Hermitian. We typically would like our energy forms to be pre-inner products. This is exactly the case for positive operators.

\subsection{Positive operators}
\begin{definition}
Let $T$ be an operator on an inner product space $V$. Then $T$ is called \udef{positive} if the associated energy form is positive: for all $x\in V$
\[ Q_T(x) = \inner{x,x}_T = \inner{x,Tx} \geq 0. \]
We write $A \geq 0$. We also say
\begin{itemize}
\item $A$ is \udef{strictly positive}, denoted $A > 0$, if $Q_T(u) > 0$;
\item $A$ is \udef{negative} if $-A$ is positive;
\item $A$ is \udef{positive definite}, \udef{strongly positive} or \udef{coercive} if there exists a constant $k>0$ such that
\[ Q_T(x) \geq k\norm{x}^2 > 0. \]
\end{itemize}
\end{definition}

\begin{lemma} \label{positiveOperatorSymmetric}
If $T$ is a positive operator on a complex inner product space, then $T$ is symmetric.
\end{lemma}
\begin{proof}
Follows immediately from \ref{energyFormHermitianSymmetric}.
\end{proof}

On a real inner product space there may exist positive operators that are not symmetric.
\begin{example}
Let $V= \R^2$ and $T: \R^2 \to\R^2: (x,y)\mapsto (y,-x)$. Then
\[ \forall (x,y)\in V: \quad Q_T\big((x,y)\big) = \inner{(x,y), (y,-x)} = xy -xy = 0 \geq 0, \]
so $T$ is positive. But $T$ is not symmetric. Indeed $\inner{(0,y), T(x,0)} = -xy$ and $\inner{T(0,y), (x, 0)} = xy$.
\end{example}

\begin{lemma}
Let $A$ be an bounded operator on a Hilbert space $H$. Then $A^*A$ and $AA^*$ are positive. Also $A^*A$ is strictly positive \textup{if and only if} $A$ is injective.
\end{lemma}
\begin{proof}
For all $x\in H$:
\[ \inner{A^*Ax,x} = \inner{Ax,Ax} = \norm{Ax}^2 \geq 0 \qquad \inner{AA^*x,x} = \inner{A^*x,A^*x} = \norm{A^*x}^2 \geq 0. \]
If $A$ is injective, then its kernel is $\{0\}$ and thus $\norm{Ax}^2 > 0$ for all $x\in H\setminus\{0\}$.
\end{proof}

\begin{lemma}
Let $T$ be an invertible operator on an inner product space $V$. Then $Q_T[V] = Q_{T^{-1}}[V]$.
\end{lemma}
\begin{proof}
Immediate from \ref{quadraticFormInverseOperator}.
\end{proof}
\begin{corollary}
Let $T$ be an invertible operator. Then $T$ is positive (definite) \textup{if and only if} $T^{-1}$ is positive (definite).
\end{corollary}

\begin{lemma} \label{positiveOperatorPositiveEnergyForm}
Let $T$ be an operator. The energy form $\inner{\cdot,\cdot}_T$ is positive (and thus a pre-inner product) \textup{if and only if} $T$ is a positive operator.
\end{lemma}

\subsubsection{Energy norm}
\begin{definition}
Let $T$ be a positive operator. Then
\[ \norm{\cdot}_{\inner{}_T}: \dom(T) \to \interval[co]{0,+\infty}: x\mapsto \norm{x}_{\inner{}_T} = \sqrt{\inner{x,x}_T} = \sqrt{Q_T(x)} \]
is the \udef{energy norm} associated to $T$.
\end{definition}

\begin{lemma}
The energy norm of a positive operator determines a pseudometric topology.
\end{lemma}

\begin{definition}
The topology generated by the energy norm is called the \udef{energy topology} and convergence in the energy topology is called \udef{convergence in energy}.
\end{definition}

\begin{proposition}
Let $T$ be a positive operator. Then
\begin{enumerate}
\item the energy topology is coarser than the norm topology;
\item the topologies are the same on $\dom(T)$ if $T$ is positive definite.
\end{enumerate}
\end{proposition}

\subsubsection{The partial order on operators}
\begin{definition}
We define an \udef{operator partial order} by
\[ A\leq B \qquad\iff\qquad B-A \geq 0. \]
\end{definition}
TODO: restrict to bounded operators??

\begin{lemma}
The operator partial order is a partial order on the set of operators on an inner product space.
\end{lemma}

\subsubsection{Induced topology}
We consider the topology induced by an energy norm $\norm{\cdot}_{\inner{}_T}$.

\begin{proposition} \label{energyNormTopology}
Let $V$ be an inner product space and $T$ a positive operator on $V$. Then
\begin{enumerate}
\item if $T$ is bounded, then $\norm{\cdot}_{\inner{}_T}$ is bounded by $\norm{\cdot}$;
\item $\norm{\cdot}$ is bounded by $\norm{\cdot}_{\inner{}_{T+\id}}$.
\end{enumerate}
\end{proposition}
\begin{proof}
(1) If $T$ is bounded, then $\forall v\in V$
\[ \norm{v}_{\inner{}_T} = \sqrt{\inner{v,Tv}} = \sqrt{|\inner{v,Tv}|} \leq \sqrt{\norm{v}^2\norm{T}} = \sqrt{\norm{T}}\norm{v}, \]
where we have used the Cauchy-Schwarz inequality \ref{CauchySchwarz}.

(2) For all $v\in V$ we have
\[ \norm{v} \leq \norm{v} + \norm{v}_{\inner{}_T} = \inner{v,v} + \inner{v,Tv} = \inner{v,(T+\id)v} = \norm{v}_{\inner{}_{T+\id}}. \]
\end{proof}
\begin{corollary}
Let $V$ be an inner product space and $T$ a positive operator on $V$. Then
\begin{enumerate}
\item if $T$ is bounded, then the topology induced by $\norm{\cdot}$ is finer than the topology induced by $\norm{\cdot}_{\inner{}_T}$;
\item the topology on $\dom(T)$ induced by $\norm{\cdot}_{\inner{}_{T+\id}}$ is finer than the topology induced by $\norm{\cdot}$.
\end{enumerate}
\end{corollary}
\begin{proof}
This follows straight from \ref{normComparison}.
\end{proof}
\begin{corollary}
Let $H$ be a Hilbert space and $T$ a positive operator on $H$. Then $\inner{\cdot,\cdot}_{T+\id}$ is an inner product on $\dom(T)$.
\end{corollary}
\begin{proof}
We have that $\inner{\cdot,\cdot}_{T+\id}$ is a pre-inner product by \ref{positiveOperatorPositiveEnergyForm}. We just need to check definiteness. This follows because $\norm{\cdot} \leq \norm{\cdot}_{\inner{}_{T+\id}}$ and $\norm{\cdot}$ is definite.
\end{proof}




\ref{existenceMetricCompletion}

$H_T \defeq \Closure_{\inner{}_{T+\id}}\big(\dom(T)\big)$ is a Hilbert space with inner product
\[ \inner{x,y}_{H_T} \defeq \lim_{n\to \infty} \inner{x_n,y_n}_{T+\id} \qquad \text{for any $\seq{x_n} \overset{\inner{}_{T+\id}}{\longrightarrow} x$ and $\seq{y_n} \overset{\inner{}_{T+\id}}{\longrightarrow} y$.} \]


\subsection{Dissipative operators}
\begin{definition}
Let $T$ be an operator on $H$. Then $T$ is \udef{dissipative} if, for all $x\in\dom(T)$
\[ \Im \inner{x,Tx} \geq 0. \]
\end{definition}

\subsection{Rayleigh quotient}
\begin{definition}
Let $T$ be a linear operator on an inner product space $V$. The \udef{Rayleigh quotient} for $T$ is 
\[ J_T: \dom(T)\setminus\{0\}\to \F: u\mapsto \frac{Q(u)}{\norm{u}^2} = \frac{\inner{u,Tu}}{\norm{u}^2}. \]
We may also write just $J$ if the intended operator $T$ is clear.
\end{definition}

\begin{lemma}
Let $T\in\Lin(V)$ be a linear operator and $J_T$ the associated Rayleigh quotient. Then for all $u\in V$:
\[ J_T(u) = J_T\left(\frac{u}{\norm{u}}\right). \]
\end{lemma}

\subsubsection{Numerical range}
\url{https://users.math.msu.edu/users/shapiro/pubvit/downloads/numrangenotes/numrange_notes.pdf}

\url{https://pskoufra.info.yorku.ca/files/2016/07/Numerical-Range.pdf}

\url{http://www.math.wm.edu/~ckli/nrnote}

\url{https://link-springer-com.ezproxy.ulb.ac.be/content/pdf/10.1007%2F978-3-319-01448-7.pdf}

\url{https://projecteuclid.org/journalArticle/Download?urlId=10.1307%2Fmmj%2F1028997958}

\begin{definition}
Let $T$ be a linear operator on an inner product space $V$ and $J_T$ the Rayleigh quotient of $T$. The range $\NumRange(T) \defeq \im(J_T)$ is known as the \udef{numerical range}.
\end{definition}

The numerical range of $T$ can equivalently be defined as the image of the unit sphere under the quadratic form associated to $T$.

\begin{lemma}
Let $T$ be a linear operator on an inner product space $V$ and $J_T$ the Rayleigh quotient of $T$. Then
\begin{align*}
\NumRange(T) &= J_T[\setbuilder{u\in V}{\norm{u} = 1} \cap \dom(T)] \\
&= Q_T[\setbuilder{u\in V}{\norm{u} = 1}\cap \dom(T)].
\end{align*}
\end{lemma}

\begin{lemma}
Let $V$ be an inner product space over a field $\F$, $\lambda,\mu\in \F$ and $T$ an operator on $V$. Then
\[ W(\lambda T + \mu) = \lambda W(T) + \mu. \]
\end{lemma}

\begin{theorem}[Toeplitz-Hausdorff theorem]
Let $V$ be an inner product space and $T$ an operator on $V$. Then $W(T)$ is convex.
\end{theorem}
\begin{proof}
TODO \url{https://www.ams.org/journals/proc/1970-025-01/S0002-9939-1970-0262849-9/S0002-9939-1970-0262849-9.pdf}

\url{https://www.cambridge.org/core/services/aop-cambridge-core/content/view/BA251EBB1E1DE08DBD3D84964F65938B/S0008439500058197a.pdf/the-toeplitz-hausdorff-theorem-explained.pdf}
\end{proof}

\begin{proposition}
Let $V$ be an inner product space and $T$ an operator on $V$. If $V$ is finite dimensional, then $W(T)$ is compact.
\end{proposition}
\begin{proof}
Heine-Borel. TODO.
\end{proof}

\begin{lemma}
Let $V$ be an inner product space and $T$ a bounded symmetric operator on $V$. Then
\begin{enumerate}
\item the directional derivative $\partial_v(J_T(u))$ exists if $u\neq 0$ and is equal to (TODO remove and place in proof?)
\[ \partial_v(J_T)|_u = \frac{\inner{u,u}\Big( \inner{v,Tu} + \inner{u,Tv} \Big) - \inner{u,Tu}\Big(\inner{u,v}+\inner{v,u}\Big)}{\inner{u,u}^2}; \]
\item $u\in V\setminus \{ 0 \}$ is a critical point of $J_T$ \textup{if and only if} $u$ is an eigenvector of $T$ with corresponding eigenvalue $\lambda = J_T(u)$.
\end{enumerate}
\end{lemma}
\begin{proof}
TODO: critical point in $\C$ v $\R$?? (For symmetric operators $J$ is real valued)
\ref{derivativeBilinearForm}
\end{proof}

\subsubsection{Numerical radius}
\begin{definition}
Let $T$ be a linear operator on an inner product space $V$. Then
\[ \nr(T) \defeq \sup_{u\in \dom(T)\setminus\{0\}} |J_T(u)| \]
is the \udef{numerical radius}.
\end{definition}
If $Q_T$ is the quadratic form associated to an operator $T$, we have
\[ |Q_T(u)| \leq \norm{u}^2\nr(T). \]

\begin{lemma}
Let $T$ be a linear operator on an inner product space $V$ and $J_T$ the Rayleigh quotient of $T$. Then
\begin{align*}
\nr(T) &= \sup_{\substack{u\in \dom(T) \setminus\{0\}\\ \norm{u} = 1}} |J_T(u)| \\
&= \sup_{\substack{u\in \dom(T) \setminus\{0\}\\ \norm{u} = 1}} |Q_T(u)|.
\end{align*}
\end{lemma}

\begin{proposition} \label{normNumRadius}
Let $T$ be an operator on an inner product space $V$.
\begin{enumerate}
\item If $T$ is bounded, then $\forall u\in \dom(T)\setminus\{0\}$
\[ |J_T(u)| \leq \nr(T) \leq \norm{T}. \]
\item If $T$ is symmetric, then $T$ is bounded with $\norm{T} = \nr(T)$.
\end{enumerate}
\end{proposition}
\begin{proof}
(1) The first claim follows simply from the Cauchy-Schwarz inequality \ref{CauchySchwarz}
\[ |J(u)| \leq \frac{\norm{u}\,\norm{Tu}}{\norm{u}^2} = \frac{\norm{Tu}}{\norm{u}} \leq \frac{\norm{T}\norm{u}}{\norm{u}} = \norm{T}. \]

(2) For the second claim we need to also show the inverse inequality. By \ref{operatorNormInnerProduct} it is enough to show that $|\inner{w,Tv}| \leq \nr(T)$ for all $v\in \dom(T)$ and $w\in\im(T)$ with $\norm{v} = 1 = \norm{w}$.

Take arbitrary unit vectors $v,w\in V$ and let $\theta$ be such that $|\inner{w,Tv}| = e^{i\theta}\inner{w,Tv}$. Then $\inner{e^{-i\theta}w,Tv}$ is real, so, viewing it as a sesquilinear form, the imaginary parts of the polarisation identity \ref{polarisationIdentities} cancel:
\begin{align*}
\inner{e^{-i\theta}w,Tv} &= \frac{1}{4}\sum_{k=0}^3i^k \inner{(i^ke^{-i\theta}w + v), T((i^ke^{-i\theta}w + Tv))} \\
&= \frac{1}{4}\Big( \inner{v+e^{-i\theta}w, T(v+e^{-i\theta}w)} - \inner{v-e^{-i\theta}w, T(v-e^{-i\theta}w)} \Big),
\end{align*}
where we have used that the quadratic form is real by \ref{energyFormHermitianSymmetric}.

Thus
\begin{align*}
|\inner{w,Tv}| &= |\inner{e^{-i\theta}w,Tv}| \\
&= \frac{1}{4}\Big(\inner{v+e^{-i\theta}w, T(v+e^{-i\theta}w)} - \inner{v-e^{-i\theta}w, T(v-e^{-i\theta}w)} \Big) \\
&\leq \frac{1}{4}\Big( |\inner{v+e^{-i\theta}w, T(v+e^{-i\theta}w)}| + |\inner{v-e^{-i\theta}w, T(v-e^{-i\theta}w)}| \Big) \\
&\leq \frac{1}{4}\nr(T)\Big( \norm{v+e^{-i\theta}w}^2 + \norm{v-e^{-i\theta}w}^2 \Big) \\
&= \frac{1}{4}\nr(T)\Big( 2\norm{v}^2 + 2\norm{w}^2 \Big) = \nr(T),
\end{align*}
where we have used the fact that $v,w$ are unit vectors and the parallelogram law \ref{parallelogramLaw}.
\end{proof}
\begin{corollary}
If $T$ is a symmetric operator; it is bounded iff $J_T$ is bounded above and below:
\[ \forall u\in\dom(T): \; k \leq J_T(u) \leq K \]
for some $k,K\in \R$.
\end{corollary}
\begin{corollary}
If $T$ is symmetric and bounded, then
\[ \norm{T} = \sup_{\norm{u}\leq 1} |\inner{u,Tu}|. \]
\end{corollary}






\chapter{Hilbert spaces}

\section{Tools to study operators}
\subsection{Numerical range}
\begin{lemma}
Let $T$ be an operator on a Hilbert space. If $\lambda\in\cspec(T)$, then there exists a Weyl sequence for $\lambda$. 
\end{lemma}
\begin{proof}
If $\lambda\in\cspec(T)$, then $R_T(\lambda)$ is densely defined.
\end{proof}

\begin{proposition}[Spectral inclusion property of numerical range] \label{spectralInclusionNumericalRange}
Let $T$ be an operator on a Hilbert space. Then
\begin{enumerate}
\item $\pspec(T)\subseteq \NumRange(T)$;
\item $\rspec(T) \subseteq \NumRange(T)$;
\item $\apspec(T) \subseteq \overline{\NumRange(T)}$.
\end{enumerate}
In particular
\begin{enumerate} \setcounter{enumi}{3}
\item $\cspec(T) \subseteq \overline{\NumRange(T)}$;
\item if $T$ is closed, then $\spec(T) \subseteq \overline{\NumRange(T)}$.
\end{enumerate}
\end{proposition}
\begin{proof}
(1) Let $\lambda \in \pspec(T)$. Then $Tx = \lambda x$ for some unit vector $x$ and so
\[ \inner{x,Tx} = \inner{x,\lambda x} = \lambda \inner{x,x} = \lambda\norm{x}^2 = \lambda, \]
which means that $\lambda \in \NumRange(A)$.

(2) Let $\lambda \in \rspec(T)$. Then there exists a unit vector $x \in \im(\lambda\id - T)^\perp$, so
\[ 0 = \inner{x,(\lambda\id-T)x} = \lambda\inner{x}^2 - \inner{x,Tx} = \lambda - \inner{x,Tx}, \]
which means that $\lambda \in \NumRange(A)$.

(3) Let $\lambda \in \apspec(T)$. Then there exists a Weyl sequence $\seq{e_n}$ for $\lambda$ by \ref{WeylSequence}. Then
\[ \norm{(\lambda\id - T)e_n} = \norm{e_n}\;\norm{(\lambda\id - T)e_n} \geq |\inner{e_n, (\lambda\id - T)e_n}| = |\lambda - \inner{e_n,Te_n}| \to 0. \]
Thus $\lambda\in\overline{\NumRange(T)}$.

Finally we have $\cspec(T)\subseteq\apspec(T)$ by \ref{approximateSpectrum} and so $\cspec(T)\subseteq\overline{\NumRange(T)}$.

(4) We have $\cspec(T) \subseteq \apspec(T) \subseteq \overline{\NumRange(T)}$ by \ref{approximateSpectrum}.

(5) Immediate from \ref{spectrumDecomposition}.
\end{proof}

\subsection{Graph norm and inner product}
Let $T: H\not\to H'$ be a linear map between Hilbert spaces. Then $\graph(T)\subseteq H\oplus H'$. If we want to $H\oplus H'$ to be a Hilbert space, we set
\[ \inner{x\oplus x', y\oplus y'}_{H\oplus H'} = \inner{x,y}_H + \inner{x',y'}_{H'}, \]
then $\norm{x\oplus y}_{H\oplus H'} = \sqrt{\norm{x}_H^2 + \norm{y}_{H'}}$.

We define, for $x,y\in \dom(T)$,
\[ \inner{x,y}_{\graph(T)} = \inner{x,y}_H + \inner{Tx,Ty}_{H'}. \]

\section{Projectors and minimisation problems}
Every subspace is a convex, non-empty subset.
\begin{theorem}[Hilbert projection theorem]
Let $\mathcal{H}$ be a Hilbert space, $K$ a closed, convex, non-empty subset of $\mathcal{H}$.
\begin{enumerate}
\item There exists a unique element of $K$ of least norm. i.e.\ there exists a unique $k_0\in K$ such that
\[ \norm{k_0} = \inf\setbuilder{\norm{k}}{k\in K}. \]
i.e.\ $\min\setbuilder{\norm{k}}{k\in K}$ exists.
\item For any $h\in\mathcal{H}$ there exists a unique point $k_0$ in $K$ such that
\[ \norm{h-k_0} = \inf\{\norm{h-k}\;|\; k\in K\}. \]
We use this to define the distance $d(h,K) \defeq \norm{h-k_0}$.
\item If $K$ is a (closed) subspace, then $k_0$ is also the unique point in $K$ such that $(h-k_0)\perp K$.
\end{enumerate}
\end{theorem}
The idea for the first part of the proof is to take a sequence $\seq{\norm{k_i}}\to \inf\setbuilder{\norm{k}}{k\in K}$. By the parallelogram law $\seq{k_i}$ is Cauchy and by completeness it has a limit $k_0$.
\begin{proof}
(1) We can find a sequence $\seq{k_i}$ in $K$ such that $\norm{k_i}$ converges to $d = \inf\setbuilder{\norm{k}}{k\in K}$ by \ref{sequenceToSupInf}. By the parallelogram law
\begin{align*}
\norm{k_i-k_j}^2 &= 2\norm{k_i}^2 + 2\norm{k_j}^2 - 4\norm{\frac{1}{2}(k_i+k_j)}^2 \\
&\leq 2\norm{k_i}^2 + 2\norm{k_j}^2 - 4d^2
\end{align*}
the sequence $\seq{k_i}$ is Cauchy. So it converges to some $k_0$ in $K$ because $K$ is a closed subset of a complete space.

To prove uniqueness, take another $k_0'\in K$ such that $\norm{k_0'}=d$. By convexity $\tfrac{1}{2}(k_0 +k_0')\in K$, hence
\[ d\leq \norm{\tfrac{1}{2}(k_0+k_0')}\leq \tfrac{1}{2}(\norm{k_0}+\norm{k_0'}) = d. \]
So $\norm{\tfrac{1}{2}(k_0+k_0')} = d$. The parallelogram law gives
\[ d^2 = \norm{\frac{k_0+k_0'}{2}}^2 = d^2- \norm{\frac{k_0-k_0'}{2}}^2; \]
hence $\norm{k_0 - k_0'}^2 = 0$ and thus $h_0=k_0$.

(2) The element $k_0$ considered in point 1. is the point closest to a particular choice for $h$, namely $h=0$. For other $h$ consider the set $K-h$, which is again closed and convex.

(3) For all $k\in K$ and $a\in \mathbb{F}$, we have
\[ \norm{h-k_0}\leq \norm{h-k_0+ak} \]
and thus, by lemma \ref{orthogonality}, $(h-k_0)\perp k$, meaning $(h-k_0)\perp K$.

For the converse (i.e.\ uniqueness), suppose $f_0\in K$ such that $(h-f_0)\perp K$. Then for all $f\in K$ we have $(h-f_0)\perp (f_0 -f)$ so that
\begin{align*}
\norm{h-f}^2 &= \norm{(h-f_0) + (f_0-f)}^2 \\
&= \norm{h-f_0}^2 + \norm{f_0 - f}^2 \geq \norm{h-f_0}^2.
\end{align*}
So $\norm{h-f_0}=\inf\{\norm{h-k}\;|\; k\in K\} = d(h,K)$ and thus $f_0=k_0$.
\end{proof}
\begin{corollary}
Let $\mathcal{H}$ be a Hilbert space and $K$ a closed vector subspace. Then $\mathcal{H} = K^\perp \oplus K$.
\end{corollary}
\begin{proof}
We need to prove every vector $x\in \mathcal{H}$ has a unique decomposition of the form
\[ x = y+z \qquad y\in K,\; z\in K^\perp. \]

Such a decomposition exists: we can take $y=k_0$ and $z = x-k_0$. We have already proved uniqueness. We can also give another argument for uniqueness: assume another such decomposition $x=y'+z'$. Then $y-y'= z-z'$ where the left side is in $K$ and the right in $K^\perp$. The only element in $K\cap K^\perp$ is $0$, so $y=y'$ and $z=z'$.
\end{proof}
The ability to make such decompositions in general is unique to Hilbert spaces, see theorem \ref{criterionHilbertSpace}.

\subsection{Orthogonal projection and decomposition}
\begin{definition}
Let $\mathcal{H}$ be a Hilbert space. Given a subspace $K$ and an element $x \in \mathcal{H}$, we call the unique element $y\in K$ of the decomposition $K\oplus K^\perp$ the \udef{orthogonal projection} of $x$ on $K$. It is denoted $P_K(x)$. This defines a function $P_K:\mathcal{H}\to K$ called the \udef{orthogonal projection} on $K$.
\end{definition}

\begin{proposition}
Let $P$ be the orthogonal projection on a closed subspace $K$. Then
\begin{enumerate}
\item $P$ is a linear operator on $\mathcal{H}$;
\item $\norm{Px}\leq \norm{x}$ for all $x\in\mathcal{H}$;
\item $P^2 = P$;
\item $\ker P = K^\perp$ and $\im P = K$;
\item $\id_\mathcal{H} - P$ is the orthogonal projection of $\mathcal{H}$ onto $K^\perp$.
\end{enumerate}
\end{proposition}
\begin{proof}
These are mostly direct results of the decomposition. In particular 5. follows if we know $K^\perp$ is closed, which it is by proposition \ref{orthogonalComplementClosed}.
\end{proof}
\begin{corollary} \label{HilbertClosedSpaceOrthogonalDecomposition}
Let $\mathcal{H}$ ba a Hilbert space and $K$ a closed subspace, then $\mathcal{H} = K\oplus K^\perp$.
\end{corollary}
\begin{proof}
Let $P$ be the orthogonal projection on $K$. Then by \ref{directSumKernelImageIdempotent}
\[ \mathcal{H} = \im P \oplus \ker P = K\oplus K^\perp. \]
\end{proof}
\begin{corollary} \label{doubleComplementClosure}
Let $\mathcal{H}$ be a Hilbert space.
\begin{enumerate}
\item If $K$ is a subspace, then $(K^\perp)^\perp = \overline{K}$ is the closure of $K$.
\item If $A$ is a subset, then $(A^\perp)^\perp$ is the closed linear span of $A$.
\end{enumerate}
\end{corollary}
\begin{proof}
(1) Assume $K$ is closed. Then using $0=(I-P_K)x\;\; \Leftrightarrow \;\; x=P_Kx$, we see
\[ (K^\perp)^\perp = \ker(I-P_K) = \im P_K = K. \]
Then, if $K$ is not closed, $(K^\perp)^\perp = (\overline{K}^\perp)^\perp = \overline{K}$, by proposition \ref{orthogonalComplementClosed}.

(2) Using \ref{OrthogonalComplementProperties} we calculate $(A^\perp)^\perp = (\Span(A)^\perp)^\perp = \overline{\Span(A)}$.
\end{proof}
\begin{corollary} \label{denseZeroComplement}
Let $A$ be a subset of a Hilbert space $\mathcal{H}$. Then $\Span(A)$ is dense in $\mathcal{H}$ \textup{if and only if} $A^\perp = \{0\}$.
\end{corollary}
\begin{proof}
The subspace $\Span(A)$ is dense in $\mathcal{H}$ iff $\overline{\Span(A)} = \mathcal{H}$ iff $(\Span(A)^\perp)^\perp = (A^\perp)^\perp = \mathcal{H}$ iff $A^\perp = \{0\}$.

In the last step we have used that $A^\perp$ is closed so that $((A^\perp)^\perp)^\perp = \overline{A^\perp} = A^\perp$, see \ref{orthogonalComplementClosed}.
\end{proof}

\subsubsection{Existence of orthonormal bases}
\begin{corollary}
Let $D$ be an orthonormal subset of a Hilbert space $\mathcal{H}$, then $D$ is an ortonormal basis \textup{if and only if} it is maximal.
\end{corollary}
\begin{proof}
This is a restatement of the previous corollary in the language of \ref{characterisationMaximalOrthonormalSet}.
\end{proof}
\begin{corollary}
Every Hilbert space has an orthonormal basis.
\end{corollary}
\begin{proof}
Every inner product space has a maximal orthonormal set by \ref{exitenceMaximalOrthonormalSet}. This maximal orthonormal set is an orthonormal set by the proposition.
\end{proof}
\begin{corollary} \label{HilbertOnBasisMaximal}
An orthonormal subset of a Hilbert space is an orthonormal basis \textup{if and only if} it is maximal.
\end{corollary}

\begin{lemma} \label{sumExpansionOrthogonalProjector}
Let $\mathcal{H}$ be a Hilbert space and $K$ a closed subspace. Let $\{e_i\}_{i\in I}$ be an orthonormal basis of $K$. Then
\[ P_K(x) = \sum_{i\in I} \inner{e_i,x}e_i. \]
\end{lemma}
\begin{proof}
We can extend $\{e_i\}_{i\in I}$ to an orthonormal basis $\{e_i\}_{i\in J}$ of $\mathcal{H}$. Then
\[ x = \sum_{i\in J}\inner{e_i,x}e_i = \sum_{i\in I}\inner{e_i,x}e_i + \sum_{i\notin I}\inner{e_i,x}e_i, \]
which is clearly a decomposition in $K\oplus K^\perp$. This is unique, so we have found $P_K(x)$.
\end{proof}

\subsubsection{When are inner product spaces complete?}
Notice that some of the results obtained for Hilbert spaces have one direction that is generally true for inner product spaces: in any inner product space we have
\begin{itemize}
\item $\overline{K}\subset (K^\perp)^\perp$;
\item $\Span(A)$ dense in $\mathcal{H}$ implies $A^\perp = \{0\}$;
\item if $D$ is an orthonormal basis, then it is maximal.
\end{itemize}
See \ref{orthogonalComplementClosed}, \ref{orthogonalComplementDenseSpace} and \ref{characterisationMaximalOrthonormalSet}.

The converses are only true for Hilbert spaces.
\begin{theorem} \label{criterionHilbertSpace}
Let $H$ be an inner product space. If any of the following hold, $H$ is a Hilbert space:
\begin{enumerate}
\item For any orthonormal set $D$,
\[ \text{$D$ is maximal} \quad\implies\quad \text{$D$ is an orthonormal basis.} \]
\item For any subset $A$, $A^\perp = \{0\}$ implies $\Span(A)$ is dense in $H$.
\item For any subspace $K$, we have $(K^\perp)^\perp = \overline{K}$.
\item For all closed subspaces $K$ we can decompose $H = K\oplus K^\perp$.
\end{enumerate}
\end{theorem}
\begin{proof}
We prove the first statement implies $H$ is a Hilbert space. The other three imply the first and thus that $H$ is a Hilbert space.
\begin{enumerate}
\item We prove the contrapositive: assume $H$ is not complete, we wish to show that 1. does not hold, i.e.\ there exists a maximal orthonormal subset of $H$ that is not an orthonormal basis.

Let $\mathcal{H}$ be the completion of $H$ and take a unit vector $v\in \mathcal{H}\setminus H$. Now working in the completion, we have the decomposition $\Span\{v\}\oplus \Span\{v\}^\perp$. Consider the subspace $\Span\{v\} + H = \Span\{v\}\oplus(H\cap \Span\{v\}^\perp)$. We can extend $\{v\}$ to a maximal orthonormal set $\{v\}\cup D$ by \ref{exitenceMaximalOrthonormalSet}.

We claim $D$ is the orthonormal set we want:

Firstly it is maximal.
Assume, towards a contradiction, that $D$ is not maximal in $H$, so there exists an orthonormal set $D'\supsetneq D$. Take $w\in D'\setminus D$ and let $w'$ be the normalisation of $w - \inner{v,w}v$. Then $w' \perp v$ and $w' \perp D$, so $\{v\}\cup D\cup\{w'\}$ is an orthonormal set in $\Span\{v\} + H$, which contradicts the maximality of $\{v\}\cup D$.

Secondly it cannot be total. Indeed if $\Closure_H(\Span(D)) = H\cap\overline{\Span(D)}$ were equal to $H$, then $H \subseteq \overline{\Span(D)}$ and thus $\mathcal{H} = \overline{H} \subseteq \overline{\Span(D)} \subseteq \mathcal{H}$, meaning $\overline{\Span(D)} = \mathcal{H}$. But $v\notin \overline{\Span(D)}$, so $\overline{\Span(D)} \neq \mathcal{H}$.

\item 2. clearly implies 1. We can also adapt the proof above to show 2. implies $H$ is a Hilbert space:
Assume $H$ is not complete and let $\mathcal{H}$ be the completion of $H$. There exists a $v\in \mathcal{H}\setminus H$. All orthogonal complements are taken in the completion.
The set
\[ U \defeq H\cap\{v\}^\perp \]
is not dense in $\mathcal{H}$ for the same reason $D$ was not total above. We claim that the orthogonal complement of $U$ in $H$ is $\{0\}$:
\[ U^\perp\cap H = \{0\}. \]
First we claim $U$ is dense in $\{v\}^\perp$: take a $w\in \{v\}^\perp$ and let $(x_n)_{n\in\N}\subseteq H$ converge to $w$ (this is possible because $w\in\mathcal{H}$ and $H$ is dense in $\mathcal{H}$). Fix some $x\in H$ such that $\inner{x,v}\neq 0$, then we have the following sequence in $U$ that converges to $w$:
\[ n\mapsto x_n - \inner{x_n,v}\frac{x}{\inner{x,v}}. \]
Then because $U$ is dense in $\{v\}^\perp$,
\[ U^\perp\cap H = \overline{U}^\perp\cap H = (\{v\}^\perp)^\perp \cap H = \Span\{v\}\cap H = \{0\}. \]
\item Assume 3. Let $D$ be a maximal orthonormal set. Then
\[ \overline{\Span(D)} = (\Span(D)^\perp)^\perp = (D^\perp)^\perp = \{0\}^\perp = H, \]
so $D$ is an orthonormal basis.
\item Assume 4. Let $D$ be a maximal orthonormal set. Then $D^\perp$ is a closed subspace, so
\[ H  = D^\perp \oplus (D^\perp)^\perp = \{0\} \oplus (D^\perp)^\perp = (\Span(D)^\perp)^\perp = \overline{\Span(D)}. \]
\end{enumerate}
\end{proof}

\subsubsection{Orthogonal decomposition}
\begin{theorem}
 A Banach space such all of its closed subspaces are complemented is isomorphic to a Hilbert space.
\end{theorem}
\begin{proof}
TODO Lindestrauss and Tzafriri in 1971. Only real??
\end{proof}

\begin{proposition} \label{directSumOrthogonalClosed}
Let $\mathcal{H}$ be a Hilbert space and let $\{V_i\}_{i\in I}$ be a family of closed, (pairwise) orthogonal subspaces. Then
\[ \bigoplus_{i\in I}V_i \qquad \text{is a closed subspace of $\mathcal{H}$.} \]
\end{proposition}
\begin{proof}
Let $(v_n)$ be a Cauchy sequence in $\bigoplus_{i\in I}V_i$ which converges to $w$. Let $v_{i,n}$ be the component of $v_n$ in $V_i$. By orthogonality we have
\[ \norm{v_n-v_m}^2 = \sum_{i\in I}\norm{v_{i,n}-v_{i,m}}^2. \]
Then
\[ \norm{v_{i,n}-v_{i,m}} \leq \norm{v_n-v_m} \]
which implies $(v_{i,n})_n$ is a Cauchy sequence in the closed space $V_i$ which therefore converges to $w_i\in V_i$. Now there are only a finite number of $i$ for which there exist non-zero $v_{i,n}$ (TODO proof!!!!). So then
\[ \lim_n v_n = \lim_n \sum_{i\in I}v_{i,n} = \sum_{i\in I}w_i \in \bigoplus_{i\in I}V_i \]
where the interchange of limits and last equality follow because the sums are finite.
\end{proof}

\begin{lemma} \label{cancellationOminus}
Let $\mathcal{H}$ be a Hilbert space and $A\supseteq B \supseteq C$ subspaces with $B$ closed. Then
\[ (A\ominus B)\oplus (B\ominus C) = A\ominus C.\]
\end{lemma}
\begin{proof}
Take $v\in(A\ominus B)\oplus (B\ominus C)$. Then either $\{v\}\perp C$ or $\{v\}\perp B$, but this implies $\{v\}\perp C$, so $v\in A\ominus C$.

Take $v\in A\ominus C$. We can uniquely write $v = v_1 + v_2 \in (A\ominus B)\oplus B = A$. We just need to show that $v_2\in B\ominus C$. Indeed assume $\inner{c,v_2}\neq 0$ for some $c\in C$. Then
\[ \inner{c, v} = \inner{c, v_1+v_2} = \inner{c,v_1}+\inner{c,v_2} = \inner{c,v_2} \neq 0, \]
so $v\notin A\ominus C$, a contradiction.
\end{proof}

\subsection{Projection and minimisation in finite-dimensional spaces}

\begin{lemma}
Let $K$ be a subspace of $\F^n$ spanned by the orthonormal basis $\{\vec{u}_i\}_{i=1}^k$. Then
\[ P_K = QQ^* \qquad\text{where}\qquad Q = \begin{bmatrix}
\vec{u}_1 & \vec{u}_2 & \hdots & \vec{u}_k
\end{bmatrix}. \]
\end{lemma}
\begin{proof}
$P_K(\vec{x}) = \sum_{i=1}^k\vec{u}_i\inner{\vec{u}_i,\vec{x}} = \sum_{i=1}^k\vec{u}_i \vec{u}_i^*\vec{x} = \left(\sum_{i=1}^k\vec{u}_i \vec{u}_i^*\right)\vec{x} = QQ^*\vec{x}$.
\end{proof}
\begin{corollary}
For any matrix $A$ with QR factorisation $A=QR$, we have
\[ P_{\Col(A)} = QQ^*. \]
\end{corollary}
In general $P_{\Col(A)} = A(A^*A)^{-1}A^*$.

\begin{proposition}[Normal equations]
Let $\{\vec{v}_i\}_{i=1}^k$ be linearly independent set of vectors in $\F^n$. Set $K = \Span\{\vec{v}_i\}_{i=1}^k$. Then for all $\vec{x}\in \F^n$
\[ P_K(\vec{x}) = \sum_{i=1}^k c_i \vec{v}_i, \]
where $\begin{bmatrix}
c_1 & c_2 & \hdots & c_k
\end{bmatrix}^\transp$ is the solution of
\[ \begin{bmatrix}
\inner{\vec{v}_1,\vec{v}_1} & \inner{\vec{v}_1,\vec{v}_2} & \hdots & \inner{\vec{v}_1,\vec{v}_k} \\
\inner{\vec{v}_2,\vec{v}_1} & \inner{\vec{v}_2,\vec{v}_2} & \hdots & \inner{\vec{v}_2,\vec{v}_k} \\
\vdots & \vdots & \ddots & \vdots \\
\inner{\vec{v}_k,\vec{v}_1} & \inner{\vec{v}_k,\vec{v}_2} & \hdots & \inner{\vec{v}_k,\vec{v}_k} \\
\end{bmatrix}\begin{bmatrix}
c_1 \\ c_2 \\ \vdots \\ c_k
\end{bmatrix} = \begin{bmatrix}
\inner{\vec{v}_1,\vec{x}} \\ \inner{\vec{v}_2,\vec{x}} \\ \vdots \\ \inner{\vec{v}_k,\vec{x}}
\end{bmatrix}. \]
This system of linear equations is consistent, yielding a unique solution.
\end{proposition}
The equations in this proposition are known as \udef{normal equations} and the matrix
\[ G(\vec{v}_1, \ldots, \vec{v}_k) \defeq \begin{bmatrix}
\vec{v}_1^* \\ \vec{v}_2^* \\ \vdots \\ \vec{v}_k^*
\end{bmatrix}\begin{bmatrix}
\vec{v}_1 & \vec{v}_2 & \hdots & \vec{v}_k
\end{bmatrix} = \begin{bmatrix}
\inner{\vec{v}_1,\vec{v}_1} & \inner{\vec{v}_1,\vec{v}_2} & \hdots & \inner{\vec{v}_1,\vec{v}_k} \\
\inner{\vec{v}_2,\vec{v}_1} & \inner{\vec{v}_2,\vec{v}_2} & \hdots & \inner{\vec{v}_2,\vec{v}_k} \\
\vdots & \vdots & \ddots & \vdots \\
\inner{\vec{v}_k,\vec{v}_1} & \inner{\vec{v}_k,\vec{v}_2} & \hdots & \inner{\vec{v}_k,\vec{v}_k} \\
\end{bmatrix} \]
is known as the \udef{Gram matrix} or \udef{Grammian}.
\begin{proof}
TODO
\end{proof}

\begin{proposition}
Let $A\in\F^{m\times n}$, $\vec{b}\in\F^m$ and $\vec{x}_0\in\F^n$. Then
\[ \min_{\vec{x}\in\F^n}\norm{\vec{b}-A \vec{x}} = \norm{\vec{b} - A \vec{x}_0} \]
if and only if
\[ A^*A \vec{x}_0 = A^* \vec{b}. \]
\end{proposition}
We regard $\vec{x}_0$ as the ``best approximate solution'' to the (not necessarily consistent) system $A \vec{x} = \vec{b}$.
\begin{proof}
TODO
\end{proof}

\section{Orthonormal bases}

Hamel basis / Schauder basis / Hilbert basis

Every Hilbert basis is Schauder basis if $V$ is separable.

Hamel basis too big in Banach space??

Necessity of completeness for existence of complete orthonormal system, i.e.\ orthonormal system $\{a_i\}_{i\in I}$ (so $a_i \cdot a_j = \delta_{ij}$) with
\[ v = \sum_{i\in I}(a_i \cdot v)a_i \]
for all $v$. This is equivalent with
\[ v \cdot w = \sum_{i\in I}(v\cdot a_i)(a_i \cdot w) \]
for all $v,w$.


\begin{theorem}[Riesz-Fischer]
Let $\{e_i\}_{i\in I}$ be an orthonormal basis of a Hilbert space $H$ and $\alpha: I\to \C$ a net. Then
\[ \sum_{i\in I}\alpha_i e_i \]
converges \textup{if and only if} $\sum_{i\in I}|\alpha_i|^2 < \infty$. 
\end{theorem}
\begin{proof}
If $\sum_{i\in I}\alpha_i e_i$ converges, then $\sum_{i\in I}|\alpha_i|^2$ is bounded by the Bessel inequality \ref{BesselInequality}.

By monotone convergence, $\sum_{i\in I}|\alpha_i|^2 < \infty$ is equivalent to saying the sum converges. By (ref TODO) $\alpha$ has finite support. So $\sum_{i\in I}\alpha_i e_i$ can be expressed as the series
\[ \sum_{k\in \N}\alpha_{i_k} e_{i_k}. \]
By completeness it is enough to show that $\seq{s_n} = \seq{\sum_{k=0}^n\alpha_{i_k} e_{i_k}}$ is Cauchy. Let $n < m$, then
\[ \norm{s_n - s_m}^2 = \norm{\sum_{k=m+1}^n\alpha_{i_k} e_{i_k}}^2 = \sum_{k=m+1}^n\norm{\alpha_{i_k} e_{i_k}}^2 = \sum_{k=m+1}^n |\alpha_{i_k}|^2 = \sum_{k=0}^n|\alpha_{i_k}|^2 -\sum_{k=0}^m|\alpha_{i_k}|^2.  \]
Since $\seq{\sum_{k=0}^n |\alpha_{i_k}|^2}$ is convergent, it is Cauchy and thus so is $\seq{s_n}$.
\end{proof}
\begin{corollary}
Let $\mathcal{H}$ be a Hilbert space and $D$ be an orthonormal basis of $\mathcal{H}$. Then $\mathcal{H}$ is isometrically isomorphic to $\ell^2(D)$.
\end{corollary}
\begin{corollary}
Hilbert spaces whose orthonormal bases have the same cardinality are isometrically isomorphic.
\end{corollary}

??
\begin{lemma}
Let $(\Omega,\mathcal{A}, \mu)$ be a measure space. Then $L^2(\Omega, \mu)$ is separable \textup{if and only if} $\mu$ is $\sigma$-finite.
\end{lemma}

\begin{lemma}
Let $\{\phi_n(x)\}^\infty_{n=0}$ be an orthonormal basis of $L^2(\Omega, \mu)$ and $\{\psi_n(x)\}^\infty_{n=0}$ be an orthonormal basis of $L^2(\Lambda, \nu)$, then $\{\phi_n(x)\psi_m(y)\}^\infty_{n,m=0}$ is an orthonormal basis of $L^2(\Omega\times\Lambda, \mu\times\nu)$.
\end{lemma}
\begin{proof}
The set $\{\phi_n(x)\psi_m(y)\}^\infty_{n,m=0}$ is orthonormal:
\[ \iint_{\Omega\times\Lambda} \phi_n(x)\psi_m(y)\overline{\phi_{n'}(x)\psi_{m'}(y)}\diff{\mu(x)}\diff{\nu(y)} = \int_\Omega\phi_n(x)\overline{\phi_{n'}(x)}\diff{\mu(x)} \cdot \int_\Lambda\psi_m(y)\overline{\psi_{m'}(y)}\diff{\nu(y)} = \delta_{n,n'}\delta_{m,m'}, \]
using Fubini's theorem and the Hölder inequality (TODO refs).

To show $D = \{\phi_n(x)\psi_m(y)\}^\infty_{n,m=0}$ is an orthonormal basis, we verify point 5. of \ref{totalONBParsevalEquivalence}: if $f\perp D$, then $f = 0$.

If $f\perp D$, then for all $m,n\in \N$
\[ 0 = \inner{f, \phi_n\psi_m} = \iint_{\Omega\times\Lambda}f(x,y)\overline{\phi_n(x)\psi_m(y)}\diff{\mu(x)}\diff{\nu(y)} = \int_\Omega \left(\int_\Lambda f(x,y)\overline{\psi_m(y)}\diff{\nu(y)} \right)\overline{\phi_n(x)}\diff{\mu(x)}.  \]
Using point 5. of \ref{totalONBParsevalEquivalence} in $L^2(\Omega,\mu)$, we see that for all $m$ the function $x\mapsto\int_\Lambda f(x,y)\overline{\psi_m(y)}\diff{\nu(y)}$ is $0$ as an element of $L^2(\Omega, \mu)$, i.e.\ it is $0$ a.e. as a function of $x$. Let
\[ E_m = \setbuilder{x\in\Omega}{ \int_\Lambda f(x,y)\overline{\psi_m(y)}\diff{\nu(y)} \neq 0} \]
and set $E = \bigcup_{m\in\N}E_m$.
Then
\[ \mu(E) =  \mu\left(\bigcup_{m\in \N}E_m\right) \leq \sum_{m\in\N}\mu(E_m) = 0. \]

For $x\notin E$, we have $\int_\Lambda f(x,y)\overline{\psi_m(y)}\diff{\nu(y)} = 0$, so by the same logic $f(x,y) = 0$ for almost all $y$. 

Now $|f|^2$ is integrable and
\[ \iint_{\Omega\times \Lambda}|f(x,y)|^2\diff{\mu(x)}\diff{\nu(y)} = \int_{\Omega\setminus E}\int_\Lambda |f(x,y)|^2\diff{\mu(x)}\diff{\nu(y)} = 0, \]
so $f=0$ in $L^2(\Omega\times\Lambda, \mu\times\nu)$.
\end{proof}

\section{Self-duality of Hilbert spaces}
\subsection{Riesz representation}
\begin{theorem}[Riesz-Fréchet representation theorem] \label{rieszRepresentation}
Let $\mathcal{H}$ be a Hilbert space. For every continuous linear functional $\omega\in \dual{\mathcal{H}}$, there exists a unique $v_\omega\in\mathcal{H}$ such that
\[ \omega(x) = \inner{v_\omega, x} \qquad \forall x\in\mathcal{H}. \]
Moreover, $\norm{v_\omega}_\mathcal{H} = \norm{\omega}_{\dual{\mathcal{H}}}$.
\end{theorem}  

The idea of the proof is as follows: consider $\mathcal{H} \cong \ker\omega \oplus \im\omega$. So we can find a subspace $U\subseteq \mathcal{H}$ such that $\mathcal{H} = \ker\omega\oplus U$. Clearly $\dim U = \dim\im\omega = \dim\F = 1$. Between $1$-dimensional spaces there can only be one linear map, up to rescaling. This map is given by $x\mapsto \inner{v,x}$ for some $v\in U$, where the scaling determines the $v$. So we choose $v$ such that $\omega|_U = x\mapsto \inner{v,x}$.

Now we want extend this form of $\omega|_U$ to the whole of $\mathcal{H}$. This works exactly if $v\in(\ker\omega)^\perp$. So we need $U=(\ker\omega)^\perp$ which is true if and only if $\mathcal{H} = \ker\omega\oplus U = \ker\omega\oplus (\ker\omega)^\perp$, which only works in general if $\ker\omega$ is closed and $\mathcal{H}$ is a Hilbert space. Now $\ker\omega$ is closed if and only if it is continuous, by \ref{continuousMapCriterion}.

With this idea we give a full proof:
\begin{proof}
If $\ker\omega = \mathcal{H}$, we can take $v_\omega = 0$.

Assume $\ker\omega\neq \mathcal{H}$, then $(\ker\omega)^\perp\neq \{0\}$ by \ref{denseZeroComplement}, because $\ker\omega$ is closed (\ref{continuousMapCriterion}). So we can take a non-zero $u\in (\ker\omega)^\perp$. We can choose it such that $\omega(u) = 1$, by rescaling. Now let $h\in\mathcal{H}$. We can write $h = (h - \omega(h)u)+\omega(h)u\in\ker\omega\oplus (\ker\omega)^\perp$, because $\omega(h - \omega(h)u) = 0$. So
\[ 0 = \inner{u,h - \omega(h)u} = \inner{u,h} - \omega(u)\norm{u}^2. \]
If $v_\omega = \norm{u}^{-2}u$, then $\omega(h) = \inner{v_\omega, h}$ for all $h\in\mathcal{H}$.

For uniqueness: assume we can find two vectors $v_\omega,v_\omega'$ such that for all $h\in\mathcal{H}$ we have $\omega(h) = \inner{v_\omega, h} = \inner{v_\omega', h}$. Then $v_\omega - v_\omega'\perp \mathcal{H}$, so $v_\omega - v_\omega'= 0$.
\end{proof}
Together with lemma \ref{innerBoundedFunctionals} this gives:
\begin{corollary} \label{RieszIsometry}
Let $\mathcal{H}$ be a Hilbert space, $x, y\in \mathcal{H}$ and $f,g\in\dual{\mathcal{H}}$. Then
\begin{enumerate}
\item the map $C_\mathcal{H}:\mathcal{H}\to \dual{\mathcal{H}}: v\mapsto \inner{v,\cdot}$ is a bijective anti-linear isometry;
\item $C_\mathcal{H}(x)(y) = \inner{x,y}$ and $\inner{C_\mathcal{H}^{-1}(f), x} = f(x)$;
\item $\dual{\mathcal{H}}$ is a Hilbert space with $\inner{f,g}_{\dual{H}} = \inner{C_\mathcal{H}^{-1}(g), C_\mathcal{H}^{-1}(f)}$;
\item $C_{\dual{\mathcal{H}}}\circ C_{\mathcal{H}} = \evalMap$;
\item $\mathcal{H}$ is reflexive.
\end{enumerate}
\end{corollary}
\begin{proof}
(1) Restatement of the theorem.

(2) The first equation is a restatement of the definition of $C_\mathcal{H}$. For the second, we use the first and calculate
\[ \inner{C_\mathcal{H}^{-1}(f), x} = C_\mathcal{H}\big(C_\mathcal{H}^{-1}(f)\big)(x) = \big((C_\mathcal{H}\circ C_\mathcal{H}^{-1})(f)\big)(x) = f(x). \]

(3) Since $C_\mathcal{H}$ is an isometry, the parallelogram law holds in $\im(C_{\mathcal{H}}) = \dual{H}$. Then the Jordan-von Neumann theorem \ref{JordanVonNeumann} implies that the norm of $\dual{H}$ comes from an inner product. The inner product can be recovered from the norm by the polarisation identity \ref{polarisationIdentities}.

We can then calculate
\begin{align*}
\inner{f,g}_{\dual{\mathcal{H}}} &= \frac{1}{4}\sum_{k=0}^3 i^k\norm{i^kf+ g}^2_{\dual{\mathcal{H}}} \\
&= \frac{1}{4}\sum_{k=0}^3 i^k\norm{i^k(C_\mathcal{H}\circ C_\mathcal{H}^{-1})(f)+ (C_\mathcal{H}\circ C_\mathcal{H}^{-1})(g)}^2_{\dual{\mathcal{H}}} \\
&= \frac{1}{4}\sum_{k=0}^3 i^k\norm{C_\mathcal{H}\Big((-i)^kC_\mathcal{H}^{-1}(f)+ C_\mathcal{H}^{-1}(g)\Big)}^2_{\dual{\mathcal{H}}} \\
&= \frac{1}{4}\sum_{k=0}^3 i^k\norm{(-i)^kC_\mathcal{H}^{-1}(f)+ C_\mathcal{H}^{-1}(g)}^2_{\mathcal{H}} \\
&= \frac{1}{4}\sum_{k=0}^3 i^k\norm{i^{k+2}C_\mathcal{H}^{-1}(f)+ C_\mathcal{H}^{-1}(g)}^2_{\mathcal{H}} \\
&= \frac{1}{4}\sum_{k=0}^3 i^{k-2}\norm{i^{k}C_\mathcal{H}^{-1}(f)+ C_\mathcal{H}^{-1}(g)}^2_{\mathcal{H}} \\
&= \frac{1}{4}\sum_{k=0}^3 (-i)^{k}\norm{i^{k}C_\mathcal{H}^{-1}(f)+ C_\mathcal{H}^{-1}(g)}^2_{\mathcal{H}} \\
&= \overline{\frac{1}{4}\sum_{k=0}^3 i^{k}\norm{i^{k}C_\mathcal{H}^{-1}(f)+ C_\mathcal{H}^{-1}(g)}^2_{\mathcal{H}}} \\
&= \overline{\inner{C_\mathcal{H}^{-1}(f), C_\mathcal{H}^{-1}(g)}_{\mathcal{H}}} \\
&= \inner{C_\mathcal{H}^{-1}(g), C_\mathcal{H}^{-1}(f)}_{\mathcal{H}}.
\end{align*}

(4) We use (2) and (3) to calculate, for arbitrary $x\in \mathcal{H}$ and $f\in\dual{\mathcal{H}}$,
\[ \big((C_{\dual{\mathcal{H}}}\circ C_\mathcal{H})(x)\big)(f) = \inner{C_\mathcal{H}(x), f}_{\dual{\mathcal{H}}} = \inner{C^{-1}_{\mathcal{H}}(f), x} = f(x). \]
Thus $(C_{\dual{\mathcal{H}}}\circ C_\mathcal{H})(x) = \evalMap_x|_{\dual{\mathcal{H}}}$ and $C_{\dual{\mathcal{H}}}\circ C_\mathcal{H} = \evalMap$.

(5) Let $H$ be a Hilbert space. By \ref{normedReflexiveIFFSemireflexive}, it is enough to show that $\evalMap: H\to \bidual{H}$ is surjective.

Since both $C_{\dual{\mathcal{H}}}$ and $C_\mathcal{H}$ are surjective and $\evalMap = C_{\dual{\mathcal{H}}}\circ C_\mathcal{H}$, we have that $\evalMap$ is surjective.
\end{proof}
\begin{corollary}
Every bounded functional defined on a closed subspace of $\mathcal{H}$ can be extended to a functional on $\mathcal{H}$ with the same norm.
\end{corollary}
\begin{proof}
The functional on the closed subspace, say $K$, can be represented as $x\mapsto \inner{v,x}_K$ for some $v\in K$. The extended functional is then simply given by $x\mapsto \inner{v,x}_\mathcal{H}$.
\end{proof}

\begin{proposition}[Representation of sesquilinear forms] \label{sesquilinearRepresentation}
Let $\mathcal{H}_1,\mathcal{H}_2$ be Hilbert spaces over $\mathbb{F}$ and $h:\mathcal{H}_1,\mathcal{H}_2\to\mathbb{F}$ a bounded sesquilinear form. Then there exists a unique bounded operator $S:\mathcal{H}_1 \to \mathcal{H}_2$ such that
\[ h(x,y) = \inner{Sx,y}. \]
This operator has the property $\norm{S} = \norm{h}$.
\end{proposition}
\begin{proof}
For fixed $x$, $y\mapsto h(x,y)$ is a bounded linear functional, so by the Riesz representation theorem \ref{rieszRepresentation} this can be represented by a unique $v_x$. Let $S$ be the function $x\mapsto v_x$. Then $h(x,y) = \inner{Sx,y}$.

To prove this function $S$ is linear, take arbitrary $x_1,x_2\in \mathcal{H}_1;y\in \mathcal{H}_2$ and $\lambda \in \mathbb{F}$. Then
\begin{align*}
\inner{S(\lambda x_1+ x_2),y} &= h(\lambda x_1+ x_2, y) = \overline{\lambda} h(x_1,y)+h(v,y_2) \\
&= \overline{\lambda} \inner{Sx_1, y} + \inner{Sx_2, y} = \inner{\lambda Sx_1 + Sx_2,y},
\end{align*}
so $S$ is linear by lemma \ref{elementaryOrthogonality}.

The equality of norms follows from
\begin{align*}
\norm{h} = \sup_{\substack{x\neq 0 \\ y\neq 0}}\frac{|\inner{Sx,y}|}{\norm{x}\norm{y}} &\geq \sup_{\substack{x\neq 0 \\ Sx\neq 0}}\frac{|\inner{Sx,Sx}|}{\norm{x}\norm{Sx}} = \sup_{x\neq 0}\frac{\norm{Sx}}{\norm{x}} = \norm{S} \\
&\leq \sup_{\substack{x\neq 0 \\ y\neq 0}}\frac{\norm{Sx}\norm{y}}{\norm{x}\norm{y}} = \sup_{x\neq 0}\frac{\norm{Sx}}{\norm{x}} = \norm{S}
\end{align*}
where the second inequality is Cauchy-Schwarz.
\end{proof}

\subsection{Hilbert space adjoints}

\subsection{Weak convergence of vectors}
\begin{lemma}
Let $\mathcal{H}$ be a Hilbert space. The weak convergence $\sSet{\mathcal{H}, \sigma(\dual{\mathcal{H}}, \mathcal{H})}$ on $\mathcal{H}$ is the initial convergence w.r.t.
\[ \setbuilder{\inner{x,-}: \mathcal{H}\to \F}{x\in\mathcal{H}}. \]
\end{lemma}
\begin{proof}
By Riesz representation \ref{rieszRepresentation} every $f\in \dual{H}$ is of the form $\inner{v_f, -}$ for some $v_f\in \mathcal{H}$. Conversely, $\inner{x,-}\in \dual{\mathcal{H}}$, for all $x\in \mathcal{H}$.
\end{proof}
\begin{definition}
Let $\mathcal{H}$ be a Hilbert space. We call the weak convergence $\sSet{\mathcal{H}, \sigma(\dual{\mathcal{H}}, \mathcal{H})}$ on $\mathcal{H}$ simply the \udef{weak convergence} and abbreviate $\sigma(\dual{\mathcal{H}}, \mathcal{H})$ by $\sigma$.
\end{definition}

\begin{example}
Let $\seq{e_n}$ be an orthonormal basis. Then the Riemann-Lebesgue lemma, \ref{RiemannLebesgueLemma}, gives $e_n \overset{\sigma}{\longrightarrow} 0$, even though clearly $e_n \not\to 0$.
\end{example}

So norm convergence is strictly stronger than weak convergence.

\begin{proposition} \label{weakHilbertSpaceConvergence}
Let $\mathcal{H}$ be a Hilbert space, $x\in \mathcal{H}$ and $\seq{x_n}\subseteq \mathcal{H}$. Then
\begin{enumerate}
\item $x_n \overset{\sigma}{\longrightarrow} x$ implies $\norm{x} \leq \liminf \norm{x_n}$;
\item $x_n \longrightarrow x$ \textup{if and only if} $x_n \overset{\sigma}{\longrightarrow} x$ and $\norm{x_n}\to \norm{x}$.
\end{enumerate}
\end{proposition}
\begin{proof}
(1) We have, using the CSB inequality \ref{CauchySchwarz},
\[ \norm{x}^2 = |\inner{x,x}| = \lim_n |\inner{x,x_n}| = \liminf_n|\inner{x,x_n}| \leq \norm{x}\liminf_n\norm{x_n}. \]
We use that for convergent sequences, $\lim = \liminf$.

(2) The direction $\Rightarrow$ is clear, using the continuity of the norm.

For the converse, we have
\[ \norm{x-x_n}^2 = \norm{x}-2\Re(\inner{x,x_n}) + \norm{x_n} \to 0, \]
because $\norm{x_n} \to \norm{x}$ and $\inner{x,x_n} \to \inner{x,x} = \norm{x}^2$.
\end{proof}


\subsection{Strong and weak convergence of operators}
\begin{lemma}
Let $V_1,V_2$ be inner product spaces. Then the weak operator topology on $\Bounded(H_1,H_2)$ is the initial topology w.r.t.\ $\big\{\inner{y, (-)x}: T\mapsto \inner{y,Tx}\big\}_{x\in H_1, y\in H_2}$.
\end{lemma}

TODO introduce shifts earlier. \label{adjointMapNotSOTContinuous}
\begin{example}
Consider the left and right shifts $S_l$ and $S_r$ on $\ell^2(\N)$. Then
\begin{itemize}
\item $S_l^n \overset{SOT}{\longrightarrow} 0$, but $S_l^n \overset{norm}{\not\longrightarrow} 0$;
\item $S_r^n \overset{WOT}{\longrightarrow} 0$, but $S_r^n \overset{SOT}{\not\longrightarrow} 0$.
\end{itemize}
\end{example}

\begin{example}
The adjoint map is not continuous w.r.t. the strong operator convergence.
\begin{itemize}
\item Consider the left and right shifts $S_l$ and $S_r$ on $\ell^2(\N)$. Then $S_l^n \overset{SOT}{\longrightarrow} 0$, but $S_r^n = (S_l^*)^n \overset{SOT}{\not\longrightarrow} 0$.
\item In any infinite-dimensional Hilbert space, take some orthonormal set $\{e_n\}_{n\in \N}$. And consider the sequence of operators $\seq{\ketbra{e_1}{e_n}}$. This converges strongly to $0$, indeed
\[ \lim_n\norm{\ketbra{e_1}{e_n}\ket{x}} = \lim_n\norm{\ket{e_1}\inner{e_n, x}} = \norm{\ket{e_1}}\lim_n\inner{e_n, x} = \lim_n\inner{e_n, x} = 0, \]
for all vectors $x$ by the Bessel inequality \ref{BesselInequality}. The sequence of adjoints $\seq{\big(\ketbra{e_1}{e_n}\big)^*} = \seq{\ketbra{e_n}{e_1}}$ does not converge strongly to $0$:
\[ \lim_n\norm{\ketbra{e_n}{e_1}\ket{x}} = \lim_n\norm{\ket{e_n}\inner{e_1, x}} = \lim_n\norm{\ket{e_n}}\inner{e_1, x} = \lim_n\inner{e_1, x} = \inner{e_1, x} \neq 0. \]
\end{itemize}
\end{example}

\url{https://math.stackexchange.com/questions/1054288/the-set-of-all-normal-operators-on-a-hilbert-space-is-not-strongly-closed}


Note normal operators not $SOT$-closed!
\begin{proposition}
If $\seq{A_n}$ is a sequence of normal operators that converges to a normal operator $A$ in the strong operator topology, then $A_n^* \overset{SOT}{\longrightarrow} A^*$.
\end{proposition}

\begin{proposition}
Let $\seq{A_n}$ be a sequence of bounded operators on a Hilbert space and $A\in\Bounded(\mathcal{H})$. Then
\begin{enumerate}
\item if $A_n \overset{WOT}{\longrightarrow} A$, then $\norm{A}\leq \liminf\norm{A_n}$;
\item if $A_nx \longrightarrow Ax$ for all $x$ in a dense subset of $\mathcal{A}$ and $\seq{A_n}$ is a bounded sequence, then $A_n \overset{SOT}{\longrightarrow} A$.
\end{enumerate}
\end{proposition}
\begin{proof}
(1) For all unit vectors $x$ we have $A_nx \overset{\sigma}{\longrightarrow} Ax$, so, by \ref{weakHilbertSpaceConvergence}, $\norm{Ax} \leq \liminf_n\norm{A_nx}$. Then
\begin{align*}
\norm{A} = \sup_{\norm{x}=1}\norm{Ax} &\leq \sup_{\norm{x}=1}\liminf_n\norm{A_nx} \\
&= \sup_{\norm{x}=1}\sup_{k\in \N}\inf_{n\geq k}\norm{A_nx} \\
&\leq \sup_{k\in \N}\inf_{n\geq k}\sup_{\norm{x}=1}\norm{A_nx} = \liminf{n}\norm{A_n}.
\end{align*}

(2) TODO!!
\end{proof}


\section{Adjoints of operators}
\begin{definition}
Let $H,K$ be Hilbert spaces and $T: H\not\to K$ an operator. An \udef{adjoint} of $T$ is an operator $S: K\not\to H$ such that
\[ \inner{w,Tv}_K = \inner{S w,v}_H \quad \forall v\in \dom(T),\; \forall w\in \dom(S). \]
\end{definition}

\begin{lemma} \label{adjointRequirementSymmetric}
Let $H,K$ be Hilbert spaces, $T\in (H\not\to K)$ and $S\in(K\not\to H)$. Then $T$ is an adjoint of $S$ \textup{if and only if} $S$ is an adjoint of $T$.
\end{lemma}
\begin{proof}
The definition of adjoint is symmetric in $S$ and $T$.
\end{proof}

\subsection{The adjoint}
\subsubsection{The adjoint as a relation}
\begin{lemma}
Let $T: H\not\to K$ be an operator between Hilbert spaces. Let $S_1, S_2$ be adjoints of $T$ then for all $x\in \dom(S_1)\cap\dom(S_2)$ we have $S_1(x) - S_2(x) \in \dom(T)^\perp$.

Conversely, let $S$ be an adjoint of $T$ and $x\in\dom(S)$. Then for all $v\in \dom(T)^\perp$ there exists an adjoint $S'$ such that $S'(x) = S(x) + v$.
\end{lemma}
\begin{proof}
For all $u\in \dom(T)$ we have
\[ \inner{S_1(x) - S_2(x), u}_H = \inner{S_1(x), u}_H - \inner{S_2(x), u}_H = \inner{x, Tu}_K - \inner{x, Tu}_K = 0. \]
So $(S_1(x) - S_2(x)) \in \dom(T)^\perp$.

For the converse, set $S' = S + \frac{\inner{x,\cdot}_K}{\inner{x,x}_K}v$. This is an adjoint: for all $a\in \dom(T), b\in \dom(S') = \dom(S)$ we have
\[  \inner{S' b,a}_H = \inner{Sb, a}_H + \frac{\inner{x,b}_K}{\inner{x,x}_K}\inner{v,a}_H = \inner{Sb, a}_H = \inner{b,Ta}_K. \]
\end{proof}
\begin{corollary} \label{agreementAdjoints}
Let $T: H\not\to K$ be a densely defined operator between Hilbert spaces. Let $S_1, S_2$ be adjoints of $T$ then for all $x\in \dom(S_1)\cap\dom(S_2)$ we have $S_1(x) = S_2(x)$.
\end{corollary}
\begin{proof}
We have $\dom(T)^\perp = \overline{\dom(T)}^\perp = H^\perp = \{0\}$. So $S_1(x) - S_2(x) = 0$.
\end{proof}
\begin{corollary} \label{maximalAdjointIsOperator}
Let $T: H\not\to K$ be an operator between Hilbert spaces. Then
\[ \bigcup\setbuilder{\graph(S)}{\text{$S\in (K\not\to H)$ is an adjoint of $T$}} \]
is the graph of an operator \textup{if and only if} $T$ is densely defined.
\end{corollary}

\begin{definition}
Let $T: H\not\to K$ be an operator between Hilbert spaces. We define the adjoint $T^*$ as the \emph{relation} on $(H,K)$ with graph
\[ \graph(T^*) \defeq \bigcup\setbuilder{\graph(S)}{\text{$S\in (K\not\to H)$ is an adjoint of $T$}}. \]
\end{definition}
Note that, by \ref{maximalAdjointIsOperator}, the adjoint is a function if and only if $T$ is densely defined.

\begin{lemma} \label{everywhereDefinedAdjointLemma}
Let $T: H\not\to K$ be a densely defined operator between Hilbert spaces. If $S$ is an adjoint of $T$ that is defined everywhere, then $T^* = S$.
\end{lemma}
\begin{corollary}
Let $H$ be a Hilbert space. Then $\id_H^* = \id_H$.
\end{corollary}

\begin{example}
Consider the left- and right-shift operators
\begin{align*}
&S_L: \ell^2(\N) \to \ell^2(\N): (x_n)_{n\in\N} \mapsto (x_{n+1})_{n\in\N} \\
&S_R: \ell^2(\N) \to \ell^2(\N): (x_n)_{n\in\N} \mapsto \left(\begin{cases}
x_{n-1} & (n\geq 1) \\ 0 &(n=0)
\end{cases}\right)_{n\in \N}.
\end{align*}
Then $S_L^* = S_R$ and $S_R^* = S_L$. To see this, take $\seq{x_n}, \seq{y_n}\in \ell^2(\N)$. Then
\[ \inner{S_L\seq{x_n}, \seq{y_n}} = \sum_{n\in\N}\overline{x_{n+1}}y_n = \overline{x_0}\cdot 0 + \sum_{n\in\N\setminus\{0\}}\overline{x_{n}}y_{n-1} = \inner{\seq{x_n}, S_R\seq{y_n}}. \]
Thus $S_L$ is an adjoint of $S_R$ and $S_R$ is an adjoint of $S_L$. We conclude with \ref{everywhereDefinedAdjointLemma}.
\end{example}

\begin{lemma} \label{adjointRelationLemma}
Let $T: H\not\to K$ be an operator between Hilbert spaces and $(x,y)\in K\times H$. Then $(x, y)\in T^*$ \textup{if and only if}
\[ \forall z\in\dom(T): \; \inner{x, T(z)} = \inner{y, z}. \]
\end{lemma}
\begin{proof}
$\boxed{\Rightarrow}$ If $(x, y)\in T^*$, then there exists an adjoint $f: K\not\to H$ such that $f(x) = y$. Then for all $z\in \dom(T)$ we have $\inner{x, T(z)} = \inner{f(x), z} = \inner{y, z}$.

$\boxed{\Leftarrow}$ The function defined by $f(x) = y$ and extended to $\Span\{x\}$ by linearity is an adjoint.
\end{proof}

\begin{proposition} \label{adjointDomain}
Let $T: H\not\to K$ be an operator between Hilbert spaces. Then
\[ \dom(T^*) = \setbuilder{x\in K}{\text{$\dom(T)\to \F: u\mapsto \inner{x, Tu}$ is a bounded functional}}. \]
\end{proposition}
\begin{proof}
$\boxed{\subseteq}$ If $\omega_x: u\mapsto \inner{x, Tu}$ is bounded, then its domain can be extended by continuity to $\overline{\dom(T)}$, which is a Hilbert space. This extended functional has a Riesz vector $x^*$ such that $\omega_x = u\mapsto \inner{x^*, u}$. The linear operator with domain $\Span\{x\}$ that maps $x$ to $x^*$ is then an adjoint.

$\boxed{\supseteq}$ If $x\in\dom(T^*)$, then, using the Cauchy-Schwarz inequality,
\[ |\inner{x,Tu}| = |\inner{T^*x,u}| \leq \norm{T^*x}\;\norm{u}, \]
so the functional $u\mapsto \inner{x, Tu}$ is bounded.
\end{proof}
\begin{corollary}
The domain $\dom(T^*)$ is a vector space and in particular contains $0$.
\end{corollary}

\begin{proposition} \label{HilbertAdjointGaloisConnection}
Let $H, K$ be Hilbert spaces. Take $T\in (H\not\to K)$ and $S\in (K\not\to H)$. Then
\[ S \subseteq T^* \iff T\subseteq S^*. \]
Thus $(*,*)$ is an antitone Galois connection between $\sSet{(H\not\to K), \subseteq}$ and $\sSet{(K\not\to H), \subseteq}$.
\end{proposition}
\begin{proof}
We have $S \subseteq T^*$ iff $S$ is an adjoint of $T$ iff $T$ is an adjoint of $S$ (by \ref{adjointRequirementSymmetric}) iff $T\subseteq S^*$.
\end{proof}
\begin{corollary} \label{HilbertAdjointAntitone}
Let $S,T: H\not\to K$ be operators between Hilbert spaces such that $S\subseteq T$. Then $T^* \subseteq S^*$.
\end{corollary}
\subsubsection{Properties of the adjoint relation}

\begin{proposition} \label{adjointScalarMultiple}
Let $T$ be an operator between Hilbert spaces and $\lambda\in\C$. If $\lambda \neq 0$, then
\[ \begin{pmatrix}
\id & 0 \\ 0 & \overline{\lambda}\id
\end{pmatrix} \graph(T^*) = (\lambda T)^*. \]
\end{proposition}
Note that if $T^*$ is a function (i.e.\ if $T$ is densely defined), then $\begin{pmatrix}
\id & 0 \\ 0 & \overline{\lambda}\id
\end{pmatrix} \graph(T^*) = \overline{\lambda}T^*$. We write the former in the proposition, because we have not made this assumption.

If $\lambda = 0$ and $T: H\not\to K$, then
\[ \begin{pmatrix}
\id & 0 \\ 0 & 0
\end{pmatrix} \graph(T^*) = \big(0: \dom(T^*)\to H\big) \subseteq \big(0: K\to H\big) = (0 T)^*, \]
where the last equality is given by \ref{adjointBoundedEverywhereDefined}.
\begin{proof}
For the inclusion $\subseteq$, take $f$ to be an adjoint of $T$. It is enough to show that $\overline{\lambda}f$ is an adjoint of $\lambda T$. This follows from
\[ \inner{\overline{\lambda}f(w), v} = \lambda\inner{f(w), v} = \lambda\inner{w,Tv} = \inner{w,\lambda Tv} \qquad \forall w\in \dom(f), v\in \dom(T). \]

For the other inclusion, let $f$ be an adjoint of $\lambda T$. It is enough to show that $\overline{\lambda^{-1}}f$ is an adjoint of $T$, because then $f = \overline{\lambda}\cdot\overline{\lambda^{-1}}f \subseteq \begin{pmatrix}
\id & 0 \\ 0 & \overline{\lambda}\id
\end{pmatrix} \graph(T^*)$. Indeed
\[ \inner{\overline{\lambda^{-1}}f(w), v} = \lambda^{-1}\inner{f(w),v} = \inner{w,\lambda^{-1}\lambda Tv} = \inner{w,Tv} \quad \forall w\in \dom(f), v\in \dom(T). \]
\end{proof}

\begin{proposition} \label{adjointGraph}
Let $T: H\not\to K$ be an operator between Hilbert spaces. Then
\begin{align*}
\graph(T^*) &= \left( \begin{pmatrix}
0 & -\id \\ \id & 0
\end{pmatrix}\graph(T) \right)^\perp 
=  \begin{pmatrix}
0 & -\id \\ \id & 0
\end{pmatrix}\graph(T)^\perp.
\end{align*}
If $T$ is densely defined, then $T^*$ is a closed operator.
\end{proposition}
\begin{proof}
We have
\[ \graph(T^*) = \bigcup\setbuilder{\graph(S)}{\text{$S\in (K\not\to H)$ is an adjoint of $T$}}. \]
Take an adjoint $S$ and $(w, Sw)$ in $\graph(S)$. Then for all $v\in\dom(T)$:
\[ 0 = \inner{w, Tv}_K - \inner{Sw, v}_H = \inner{w, Tv}_K + \inner{Sw, -v}_H = \inner{(w, Sw), (Tv,-v)}_{K\oplus H}. \]
So $(Tv,-v) = \begin{pmatrix}
0 & -\id \\ \id & 0
\end{pmatrix} (v,Tv) \in \graph(S)^\perp $.

The final equality follows from \ref{perpUnderIsometry}, using the fact that $\begin{pmatrix}
0 & -\id \\ \id & 0
\end{pmatrix}$ is a surjective isometry.

If $T$ is densely defined, then $T^*$ is a function by \ref{maximalAdjointIsOperator}. It is closed by \ref{orthogonalComplementClosed}.
\end{proof}
\begin{corollary} \label{adjointDenselyDefinedClosable}
Let $T: H\not\to K$ be a densely defined operator between Hilbert spaces.
Then
\begin{enumerate}
\item $\graph(T^{**}) = \overline{\graph(T)}$;
\item $T^*$ is densely defined \textup{if and only if} $T$ is closable;
\item If $T$ is closable, then $\overline{T} = T^{**}$.
\end{enumerate}
\end{corollary}
\begin{proof}
From the proposition we have
\begin{align*}
\graph(T^{**}) &=  \begin{pmatrix}
0 & -\id \\ \id & 0
\end{pmatrix}\graph(T^*)^\perp 
=  \begin{pmatrix}
0 & -\id \\ \id & 0
\end{pmatrix}\left(\begin{pmatrix}
0 & -\id \\ \id & 0
\end{pmatrix}\graph(T)^\perp\right)^\perp \\
&= \begin{pmatrix}
0 & -\id \\ \id & 0
\end{pmatrix}^2\graph(T)^{\perp\perp} = -\graph(T)^{\perp\perp}
= \overline{\graph(T)}.
\end{align*}
The right-hand side is the graph of an operator iff $T$ is closable and the left-hand side is the graph of an operator iff $T^*$ is densely defined, by \ref{maximalAdjointIsOperator}.

For a closable operator, the closure is defined by $\overline{\graph(T)} = \graph(\overline{T})$.
\end{proof}

\begin{proposition} \label{adjointBoundedEverywhereDefined}
Let $T: H\to K$ be a densely defined operator between Hilbert spaces. Then $\dom(T^*) = K$ \textup{if and only if} $T$ is bounded.
\end{proposition}
\begin{proof}
The direction $\Leftarrow$ is given by \ref{adjointDomain}.

For the other direction, note that $T^*$ is closed by \ref{adjointGraph}. Then $T^*$ is bounded by the closed graph theorem \ref{BanachClosedGraphTheorem}. We use the direction $\Leftarrow$ to see that $\dom(T^{**}) = H$. Similarly, $T^{**}$ is closed by \ref{adjointGraph} and bounded by the closed graph theorem \ref{BanachClosedGraphTheorem}. Thus $T\subseteq \overline{T} = T^{**}$ is bounded.
\end{proof}

An important application of this proposition is the Hellinger-Toeplitz theorem \ref{HellingerToeplitz}.

\begin{proposition} \label{adjointAlgebraicProperties}
Let $T,S$ be compatible operators between Hilbert spaces. Then
\begin{enumerate}
\item $S^* + T^* \subseteq (S+T)^*$;
\item $S^*T^* \subseteq (TS)^*$.
\end{enumerate}
\end{proposition}
\begin{proof}
(1) Let $f$ be an adjoint of $S$ and $g$ an adjoint of $T$. It is enough to see that $f+g$ is an adjoint of $S+T$. Indeed $\forall w\in \dom(f + g), v\in \dom(S+T)$
\[ \inner{(f + g)(w), v} = \inner{f(w),v} + \inner{g(w), Tv} = \inner{w,Sv} + \inner{w,Tv} = \inner{w,(S+T)v}. \]

(2) Let $f$ be an adjoint of $T$ and $g$ an adjoint of $S$. It is enough to see that $gf$ is an adjoint of $TS$. Indeed
\[ \inner{g\circ f(w), v} = \inner{f(w), Sv} = \inner{w,TSv} \qquad \forall w\in \dom(g\circ f), v\in \dom(TS). \]
\end{proof}

\begin{example}
The inclusions in \ref{adjointAlgebraicProperties} are, in general, not equalities.
\begin{itemize}
\item If $S,T$ are densely defined, but $\dom(S+T) = \dom(S)\cap \dom(T)$ is not dense, then there can clearly not be an equality.
\item Let $T: H\to K$ be a densely defined unbounded operator. Then $\dom(T^*) \neq K$ by \ref{adjointBoundedEverywhereDefined}. Now we have
\[ T^* - T^* = \big(0: \dom(T^*) \to H\big) \subsetneq \big(0: K\to H\big) = \big(0: \dom(T) \to K\big)^* = (T-T)^*. \]
The penultimate equality follows from \ref{adjointBoundedEverywhereDefined}. In this case the domain of the sum is dense, but still there is no equality.
\end{itemize}
\end{example}

There exist various conditions that make the inclusions in \ref{adjointAlgebraicProperties} equalities.
\begin{proposition} \label{equalityAlgebraicPropertiesAdjoint}
Let $T,S$ be compatible operators between Hilbert spaces.
\begin{enumerate}
\item if $T$ is densely defined, $\dom(S) \subseteq \dom(T)$ and $\dom\big((S+T)^*\big) \subseteq \dom(T^*)$, then $S^* + T^* = (S+T)^*$;
\item if $T$ is densely defined, $\im(S)\subseteq \dom(T)$ and $\dom\big((TS)^*)\subseteq \dom(T^*)$, then $S^*T^* = (TS)^*$;
\item if $S$ is densely defined and $\im(S)$ has finite codimension, then $S^*T^* = (TS)^*$.
\end{enumerate}
\end{proposition}
\begin{proof}
(1) By \ref{adjointAlgebraicProperties}, we have
\[ (S+T)^* - T^* \subseteq (S+T-T)^* = S^*, \]
where the last equality is due to $\dom(S) \subseteq \dom(T)$. Now take $x,y$ such that $x\in \dom\big((S+T)^*\big)$. Then $T^*(x)$ exists and we have the implications
\begin{align*}
x(S+T)^*y \iff& x\big((S+T)^* - T^* + T^*\big)y \\
\iff& \exists z: \; x\big((S+T)^* - T^*\big)z \land (z+T^*(x) = y) \\
\implies& \exists z: \; x(S^*)z \land (z+T^*(x) = y) \\
\iff& x(S^* + T^*)y.
\end{align*}
Thus $(S+T)^* \subseteq S^* + T^*$.

(2) We need to prove $(TS)^* \subseteq S^*T^*$. Assume $(x,y)\in (TS)^*$. By \ref{adjointRelationLemma}, we have
\[ \forall z\in \dom(TS):\; \inner{x, TS(z)} = \inner{y, z}. \]
Because $\im(S)\subseteq \dom(T)$, we have $\dom(TS) = \dom(S)$. Also, by assumption, $x\in \dom(T^*)$. So we have
\[ \forall z\in \dom(S):\; \inner{x, TS(z)} = \inner{T^*(x), S(z)} = \inner{y, z}, \]
which means that $\big(T^*(x), y\big)\in S^*$, so $(x,y)\in S^*T^*$.

(3)
\end{proof}
\begin{corollary}
If $T$ is bounded and everywhere defined, then
\[ S^* + T^* = (S+T)^* \qquad\text{and}\qquad S^*T^* = (TS)^*. \]
\end{corollary}

\url{https://arxiv.org/pdf/1507.08418.pdf}
\url{https://link.springer.com/article/10.1007/s43036-020-00068-4}

\begin{lemma} \label{HilbertAdjointLemma}
Let $S,T\in\Bounded(H,K)$ and $\lambda \in \mathbb{F}$.
\begin{enumerate}
\item $(T^*)^* = T$;
\item $(S+T)^* = S^* + T^*$;
\item $(\lambda T)^* = \bar{\lambda}T^*$;
\item $\id_V^* = \id_V$.
\end{enumerate}
Let $T\in\Bounded(H_1,H_2), S\in\Bounded(H_2,H_3)$
\begin{enumerate}
\setcounter{enumi}{4}
\item $(ST)^* = T^*S^*$.
\end{enumerate}
\end{lemma}

\begin{note}
Useful exercise: The identities of \ref{HilbertAdjointLemma} can also be proven by elementary manipulations. For example, to prove (1), we take arbitrary $v\in H$ and $w\in K$, Then
\[ \inner{w,Tv} = \inner{T^*w,v} = \overline{\inner{v,T^*w}} = \overline{\inner{(T^*)^*v,w}} = \inner{w, (T^*)^*v}. \]
By lemma \ref{elementaryOrthogonality} we have $Tv = (T^*)^*v$ for all $v\in V$. 
\end{note}

\subsubsection{Adjoints of densely defined operators}
The adjoint of an operator is a function if and only the operator is densely defined.

\begin{proposition} \label{adjointRangeCriterion}
Let $S: K\not\to H$ and $T: H\not\to K$ be linear operators between Hilbert spaces. If
\[ \im(S\cap T^*) = H \qquad\text{and}\qquad \im(T\cap S^*) = K, \]
then $S$ and $T$ are densely defined with $S^* = T$ and $T^* = S$.
\end{proposition}
\begin{proof}
Notice that $S\cap T^*$ and $T\cap S^*$ are linear operators that are adjoints of each other.

We claim that they are densely defined: take $x\in \dom(S\cap T^*)^\perp$. Then there exists some $y\in H$ such that $x = (T\cap S^*)y$ because of surjectivity. Now for all $z\in \dom(S\cap T^*)$
\[ 0 = \inner{z,x} = \inner{z, (T\cap S^*)y} = \inner{(S\cap T^*)z, y}, \]
so $\inner{z',y} = 0$ for all $z'\in H$, by surjectivity. This means, by \ref{elementaryOrthogonality}, that $y=0$ and thus also $x = (T\cap S^*)y = 0$. We conclude that $\dom(S\cap T^*)^\perp = \{0\}$, meaning $(S\cap T^*)$ is densely defined. The argument for $(T\cap S^*)$ is similar.

It follows that $S$ and $T$ must be densely defined. We have, by \ref{kernelImageAdjoint},
\[ \ker(S) = \im(S^*)^\perp \subseteq \im(T\cap S^*)^\perp = \{0\}. \]
Similarly $\ker(T) = \ker(S^*) = \ker(T^*) = \{0\}$.

So we have $\ker(S) = \ker(T^*)$, $\im(S)\subseteq \im(S\cap T^*)$ and $\im(T^*)\subseteq \im(S\cap T^*)$. The equality $S = T^*$ follows from \ref{partialFunctionSubset}. The equality $T = S^*$ is similar.
\end{proof}


\begin{proposition} \label{kernelImageAdjoint}
Let $T: H\not\to K$ be an operator between Hilbert spaces. Then
\[ \forall v\in K: \; (v,0)\in T^* \iff v\in \im(T)^\perp. \]
If $T^*$ is densely defined, this reduces to
\begin{enumerate}
\item $\ker(T^*) = \im(T)^\perp$;
\item $\ker(T) \subseteq \im(T^*)^\perp$;
\item if $T$ is closed, then $\ker(T) = \im(T^*)^\perp$
\end{enumerate}
\end{proposition}
\begin{proof}
(1) Because $\dom(T)$ is dense in $H$, we have $\dom(T)^\perp = \{0\}$ by \ref{orthogonalComplementDenseSpace}. Take $v\in K$. We have the equivalences
\begin{align*}
v\in \im(T)^\perp &\iff \forall x \in\dom(T): \inner{v, T(x)} = 0 \\
&\iff \forall x \in\dom(T): \inner{v, T(x)} = \inner{v, 0} \\
&\iff (v,0)\in T^*,
\end{align*}
using \ref{adjointRelationLemma}.

Point (1) is a direct translation in the case that $T^*$ is a function.

For point (2) note that $T\subseteq T^{**}$ (by \ref{adjointDenselyDefinedClosable}) implies that $(v,0)\in T \implies (v,0)\in T^{**}$.

For point (3): in this case $\ker(T) = \ker(T^{**}) = \im(T^*)^\perp$.
\end{proof}
\begin{corollary}[Closed range theorem for Hilbert spaces]
Let $T$ be a closed, densely defined operator between Hilbert spaces. Then the following are equivalent:
\begin{enumerate}
\item $\im(T)$ is closed;
\item $\im(T^*)$ is closed;
\item $\im(T) = \ker(T^*)^\perp$;
\item $\im(T^*) = \ker(T)^\perp$.
\end{enumerate}
\end{corollary}
\begin{proof}
By the proposition and \ref{orthogonalComplementClosed}, we have $\overline{\im(T)} = \ker(T^*)^\perp$. This shows $(1) \Leftrightarrow (3)$ and $(2) \Leftrightarrow (4)$.

TODO equivalence $(1)\Leftrightarrow (2)$.
\end{proof}
TODO ref closed range theorem for Banach spaces. This is, e.g., the case when $T$ is bounded below, see \ref{boundedBelowClosedRange}.

\begin{proposition}
Let $T: H\not\to K$ be a densely defined operator between Hilbert spaces. Then
\begin{enumerate}
\item $\im(T)$ is dense in $K$ \textup{if and only if} $T^*$ is injective;
\item if $T$ and $T^*$ are injective, then $(T^*)^{-1} = (T^{-1})^*$.
\end{enumerate}
\end{proposition}
\begin{proof}
(1) This is immediate from \ref{kernelImageAdjoint} and \ref{injectivityKernelTriviality}:
\[ \text{$\im(T)$ is dense} \quad\iff\quad \{0\} = \im(T)^\perp = \ker(T^*). \]

(2) We have $\graph(T^{-1}) = \begin{pmatrix}
0 & \id \\ \id & 0
\end{pmatrix}\graph(T)$. Also note that $\begin{pmatrix}
0 & \id \\ \id & 0
\end{pmatrix}$ and $\begin{pmatrix}
0 & -\id \\ \id & 0
\end{pmatrix}$ commute. Then we compute using \ref{adjointGraph}:
\begin{align*}
\graph((T^*)^{-1}) &= \begin{pmatrix}
0 & \id \\ \id & 0
\end{pmatrix}\begin{pmatrix}
0 & -\id \\ \id & 0
\end{pmatrix}\graph(T)^\perp \\
&= \begin{pmatrix}
0 & -\id \\ \id & 0
\end{pmatrix}\begin{pmatrix}
0 & \id \\ \id & 0
\end{pmatrix}\graph(T)^\perp \\
&= \begin{pmatrix}
0 & -\id \\ \id & 0
\end{pmatrix}\left(\begin{pmatrix}
0 & \id \\ \id & 0
\end{pmatrix}\graph(T)\right)^\perp = \graph((T^{-1})^*).
\end{align*}
The penultimate equality follows from \ref{perpUnderIsometry}, using the fact that $\begin{pmatrix}
0 & \id \\ \id & 0
\end{pmatrix}$ is a surjective isometry.
\end{proof}

\subsubsection{Adjoints of bounded operators}
\begin{proposition}
Let $T: H\to K$ be a densely defined operator between Hilbert spaces. Then
\begin{enumerate}
\item if $T\in\Bounded(H,K)$, then $T^*\in\Bounded(K,H)$;
\item if $T^*\in\Bounded(K,H)$, then $T$ is bounded. If $T$ is closed, then $T$ is defined everywhere.
\end{enumerate}
Assume $T\in\Bounded(H,K)$. Then
\begin{enumerate} \setcounter{enumi}{2}
\item $\norm{T} = \norm{T^*}$;
\item $T^* = C_H^{-1}T^tC_K$, where $C_K$ is the Riesz isometry from \ref{RieszIsometry}.
\end{enumerate}
\end{proposition}
\begin{proof}
(1) Assume $T\in\Bounded(H,K)$. Then $u\mapsto \inner{x,Tu}$ is a bounded functional for all $x\in K$, so $\dom(T^*) = K$ by \ref{adjointDomain}. Also $T^*$ is closed by \ref{adjointGraph}, so it is bounded by the closed graph theorem \ref{BanachClosedGraphTheorem}.

(2) Assume $T^*\in\Bounded(K,H)$. By the previous argument $T \subseteq \overline{T} = T^{**}\in\Bounded(H,K)$.

(3) The function $(x,u)\mapsto \inner{x,Tu}$ is a bounded sesquilinear form. By proposition \ref{sesquilinearRepresentation}, $T^*$ must be the unique $S$ from the proposition, which has norm $\norm{T}$.

(4) Finally we note that $C_H^{-1}T^tC_K$ is an adjoint with domain $K$ and conclude by \ref{everywhereDefinedAdjointLemma}.
\end{proof}

\begin{lemma}
The adjoint defines a map $*:\Bounded(H,K)\to \Bounded(K,H)$ that is anti-linear and continuous in the weak and uniform operator topologies. It is continuous in the strong operator topology \textup{if and only if} finite dimensional.
\end{lemma}
\begin{proof}
By the proposition the adjoint map is anti-linear. It is also bounded with norm $1$. Then by corollary \ref{boundedAntiLinearMaps} it must be bounded.

TODO
\end{proof}

\begin{proposition}
Let $H,K$ be Hilbert spaces and $T:H\to K$ a bijective bounded linear operator with bounded inverse. Then $(T^*)^{-1}$ exists and
\[ (T^*)^{-1} = (T^{-1})^*. \]
\end{proposition}
\begin{proof}
We prove $(T^{-1})^*$ is both a left- and a right-inverse of $T^*$: $\forall x\in H, y\in K$
\begin{align*}
\inner{T^*(T^{-1})^*x,y} &= \inner{x,T^{-1}Ty} = \inner{x,y} \\
\inner{x,(T^{-1})^*T^*y} &= \inner{TT^{-1}x,y} = \inner{x,y}
\end{align*}
So, by lemma \ref{elementaryOrthogonality}, $T^*(T^{-1})^* = \id_H$ and $(T^{-1})^*T^* = \id_K$.
\end{proof}

\begin{proposition} \label{normOfSquare}
Let $T\in \Bounded(H,K)$ with $H,K$ Hilbert spaces. Then
\[ \norm{T^*T}= \norm{T}^2 = \norm{TT^*}. \]
\end{proposition}
\begin{proof}
For $\norm{T^*T}= \norm{T}^2$ first observe that
\[ \norm{T^*T} \leq \norm{T^*}\cdot\norm{T} = \norm{T}^2. \]
Conversely, $\forall x\in H$:
\[ \norm{T(x)}^2 = \inner{Tx,Tx} = \inner{T^*Tx,x} \leq \norm{T^*Tx}\cdot \norm{x} \leq \norm{T^*T}\cdot\norm{x}^2. \]
The other equality follows by applying the first to $T^*$ and using $\norm{T^*}=\norm{T}$.
\end{proof}

\subsection{Symmetric and self-adjoint operators}
\begin{definition}
Let $A$ be an operator on a Hilbert space.
\begin{itemize}
\item If $A^* = A$, we say $A$ is \udef{self-adjoint}.
\item If $A^* = -A$, we say $A$ is \udef{skew-adjoint}.
\end{itemize}
We denote the set of self-adjoint operators on a Hilbert space $H$ by $\SelfAdjoints(H)$.
\end{definition}

\begin{lemma} \label{selfAdjointLemma}
Let $A$ be a self-adjoint or skew-adjoint operator on a Hilbert space. Then
\begin{enumerate}
\item $A$ is densely defined;
\item $A$ is normal;
\item $A$ is closed;
\item $\lambda A$ is self-adjoint (resp.\ skew-adjoint) for all $\lambda\in \R$.
\end{enumerate}
\end{lemma}
\begin{proof}
(1) From $A = A^*$ or $-A = A^*$, we have that $A^*$ is a function. This implies that $A$ is densely defined by \ref{maximalAdjointIsOperator}.

(2) Sicne $A$ is densely defined, we can apply \ref{normalCriterion}.

(3) For any self-adjoint operator $A$, we have $A = A^* = A^{**} = \overline{A}$. Alternatively, note that all normal operators are closed (by definition).

(4) Since $A$ is densely defined, we have $(\lambda A)^* = \overline{\lambda}A^* = \lambda A^* = \pm\lambda A^*$, by \ref{adjointScalarMultiple}.
\end{proof}

\begin{lemma}
Let $A$ be an operator on a Hilbert space. Then $A$ is self-adjoint \textup{if and only if} $iA$ is skew-adjoint.
\end{lemma}
\begin{proof}
First suppose $A$ is self-adjoint, then $A$ is densely defined by \ref{selfAdjointLemma}, so $(iA)^* = \overline{i}A^* = -iA^*$ by \ref{adjointScalarMultiple}.

Now suppose $iA$ is skew-adjoint. Then $iA$ is densely defined by \ref{selfAdjointLemma}, so $A^* = \big(-i(iA)\big)^* = \overline{-i}(iA)^* = i(-iA) = A$, by \ref{adjointScalarMultiple}.
\end{proof}

\subsubsection{Domain related matters}
\begin{lemma} \label{symmetricOperatorAdjointInclusion}
Let $A$ be a densely defined operator on a Hilbert space. Then
\begin{enumerate}
\item $A$ is symmetric \textup{if and only if} $A\subseteq A^*$;
\item if $A$ is symmetric, then is $A$ closable and $\overline{A} = A^{**}$ is symmetric.
\end{enumerate}
\end{lemma}
\begin{proof}
(1) $A$ is symmetric iff it is an adjoint of itself, iff $A\subseteq A^*$.

(2) From (1) we see that $A^*$ is densely defined, because the superset of a dense set is dense. Then $A$ is closable by \ref{adjointDenselyDefinedClosable}.

To show symmetry of $\overline{A}$, we have (using the properties implied by \ref{HilbertAdjointGaloisConnection}) $A^{**}\subseteq A^*$ from $A\subseteq A^*$ and thus
\[ \overline{A} = A^{**} \subseteq A^* = A^{***} = \overline{A}^*. \]
\end{proof}

A symmetric operator $A$ is self-adjoint if and only if $\dom(A) = \dom(A^*)$.

\begin{corollary}
A closed and densely defined symmetric operator $A$ is self-adjoint \textup{if and only if} $A^*$ is also symmetric.
\end{corollary}
\begin{proof}
If $A$ is self-adjoint, then $A^*$ is self-adjoint and thus symmetric,

If $A^*$ is symmetric, then $A\subseteq A^* \subseteq A^{**} = A$.
\end{proof}

\begin{example}
TODO example of closed symmetric operator that is not self-adjoint (see differential operator below)
\end{example}

\begin{theorem}[Hellinger-Toeplitz] \label{HellingerToeplitz}
Everywhere-defined symmetric operators are bounded.
\end{theorem}
\begin{proof}
Assume $A: H\to H$ an everywhere-defined symmetric operator. Then $\dom(A) = H$. Also $A\subseteq A^*$ by \ref{symmetricOperatorAdjointInclusion}. Thus $H = \dom(A) \subseteq \dom(A^*) \subseteq H$. So $\dom(A^*) = H$. By \ref{adjointBoundedEverywhereDefined}, $A$ is bounded. 
\end{proof}

\begin{proposition} \label{selfAdjointMaximal}
A self-adjoint operator cannot have a proper symmetric extension.
\end{proposition}
\begin{proof}
Assume $A$ self-adjoint and $A\subseteq B$ for some symmetric operator $B$. Then
\[ A \subseteq B \subseteq B^* \subseteq A^* = A, \]
so $A = B$. We have used \ref{symmetricOperatorAdjointInclusion} and \ref{HilbertAdjointAntitone}.
\end{proof}
\begin{corollary}
Let $A$ be a densely defined symmetric operator. If $\overline{A}$ is self-adjoint, then it is the unique self-adjoint extension of $A$.
\end{corollary}
Note that $\overline{A}$ is always an operator by \ref{symmetricOperatorAdjointInclusion}.
\begin{proof}
Let $B$ be a self-adjoint extension of $A$. Then $\overline{A} = A^{**}\subseteq B^{**} = B$, by \ref{HilbertAdjointAntitone}. This means that $B$ is symmetric extension of the self-adjoint operator $\overline{A}$, which, by the proposition, implies $B = \overline{A}$.
\end{proof}
In general it is possible for an unbounded,
symmetric operator to not have a self-adjoint extension or have multiple self-adjoint extensions, even if it is densely defined. (TODO example)

\begin{definition}
Let $A$ be a densely defined symmetric operator whose closure is self-adjoint. We call $A$
\begin{itemize}
\item \udef{essentially self-adjoint};
\item a \udef{core} for $A$.
\end{itemize}
\end{definition}

\begin{example}
Consider the operator
\[ A: L^2(a,b) \to L^2(a,b): f\mapsto i\od{f}{x} \]
with domain
\[ \dom(A) = \setbuilder{f\in L^2(a,b)}{\od{f}{x}\in L^2(a,b),\; f(a) = 0 = f(b)}. \]
Then
\begin{align*}
\inner{g, Af} &= \int_{a}^b \overline{g(x)}i\od{f(x)}{x}\diff{x} \\
&= \overline{g(b)}f(b) - \overline{g(a)}f(a) - \int_{a}^b \Big(i \od{}{x}\overline{g(x)}\Big)f(x)\diff{x} \\
&= \int_a^b \overline{i \od{g(x)}{x}} f(x) \diff{x} = \inner{Ag, f}.
\end{align*}
So $A$ is symmetric and $\dom(A^*) = \setbuilder{f\in L^2(a,b)}{\od{f}{x}\in L^2(a,b)}$. We cannot extend $\dom(A)$ while keeping $\dom(A^*)$ the same, because $A$ would no longer be symmetric due to boundary terms.

There are, however, multiple ways we can extend $A$ to a self-adjoint operator (in each case $\dom(A^*)$ must shrink).

Let $A_\alpha$, for $\alpha\in \R$, be the operator $A$ with domain
\[ \dom(A_\alpha) = \setbuilder{f\in L^2(a,b)}{\od{f}{x}\in L^2(a,b),\; f(b) = e^{i\alpha}f(b)}. \]
We must have $\forall f\in \dom(A_\alpha)$ and $g\in\dom(A^*_\alpha)$ that
\[ \overline{g(b)}f(b) - \overline{g(a)}f(a) = f(a)\Big(e^{i\alpha}\overline{g(b)} - \overline{g(a)}\Big) = 0, \]
so we have $e^{-i\alpha}g(b) = g(a)$ and thus $g(b) = e^{i\alpha}g(a)$, which means $\dom(A_\alpha^*) = \dom(A_\alpha)$. So $A_\alpha$ is a self-adjoint extension of $A$ for all $\alpha\in \R$.

TODO: compare Aharonov-Bohm TODO show closure \url{https://math.stackexchange.com/questions/214218/uniform-convergence-of-derivatives-tao-14-2-7}.
\end{example}
Notice that the operator
\[ T: L^2(a,b) \to L^2(a,b): f\mapsto i\od{f}{x} \]
with domain
\[ \dom(T) = \setbuilder{f\in L^2(a,b)}{\od{f}{x}\in L^2(a,b)} \]
is not symmetric. In this case
\[  \dom(T^*) = \setbuilder{f\in L^2(a,b)}{\od{f}{x}\in L^2(a,b),\; f(a)=0=f(b)}, \]
so $\dom(T^*) \subseteq \dom(T)$.

\subsubsection{Spectrum and related criteria}
TODO: $iA$ dissipative!
\begin{lemma}
Let $A$ be a symmetric operator on a complex Hilbert space $H$. If $\exists z \in \C\setminus\R: \; \im(A+z\id) = H$, then $A$ is densely defined.
\end{lemma}
\begin{proof}
Let $A+z\id$ be surjective and suppose, towards a contradiction that there exists an $y\perp \dom(A)$. Then $y = (A+z\id)x$ for some $x\in\dom(A)$ by surjectivity. Then
\[ 0 = \Im\inner{x,y} = \Im\inner{x, (A+z\id)x} = \cancel{\Im\inner{x,Ax}} + \Im \inner{x,zx} = \Im(z)\norm{x}^2. \]
By assumption, $\Im(z) \neq 0$, so $x=0$, meaning $y = (A+z\id)x = 0$ and thus $\dom(A)^\perp = \{0\}$.
\end{proof}

\begin{proposition} \label{symmetricPlusiBoundedBelow}
Let $A$ be a symmetric operator on a complex Hilbert space $H$. Then $A + z\id_H$ is bounded below by $|\Im z|$ for all $z \in \C\setminus\R$.
\end{proposition}
\begin{proof}
We first calculate, $\forall x\in H$:
\[ \Im\inner{x, (A+ z\id_H)x} = \cancel{\Im\inner{x,Ax}} + \Im z\norm{x}^2. \]
Thus
\[ |\Im z|\;\norm{x}^2 = |\Im\inner{x, (A + z\id_H)x}| \leq |\inner{x, (A + z\id_H)x}| \leq \norm{x}\;\norm{(A + z\id_H)x}, \]
which means that $\norm{(A + z\id_H)x} \geq |\Im z|\;\norm{x}$, so $A + z\id_H$ is bounded below by $|\Im z|$.
\end{proof}
\begin{corollary} \label{approximateSpectrumSymmetricOperator}
Let $A$ be a symmetric operator on a complex Hilbert space $H$. Then $\apspec(A) \subseteq \R$.
\end{corollary}
\begin{corollary}
The eigenvalues of a symmetric operator are real.
\end{corollary}
\begin{proof}
This is immediate using $\pspec(A)\subseteq \apspec(A)$. We can also give a direct calculation:

Assume there exists an $x\in \ker(\lambda\id_H - A)\setminus\{0\}$. Then $Ax = \lambda x$ and thus
\[ \lambda\norm{x}^2 = \lambda\inner{x,x} = \inner{x, \lambda x} = \inner{x,Ax} = \inner{Ax,x} = \inner{\lambda x, x} = \overline{\lambda}\inner{x,x} = \overline{\lambda}\norm{x}^2. \]
Because $\norm{x}^2 \neq 0$, we have $\lambda = \overline{\lambda}$, meaning $\lambda$ is real.
\end{proof}
\begin{corollary} \label{symmetricResolvent}
Let $A$ be a symmetric operator on a complex Hilbert space $H$. Then for all $\lambda\in\C\setminus\R$, the resolvent $R_A(\lambda)$ well-defined and bounded by $\norm{R_A(\lambda)}\leq 1/|\Im \lambda|$.
\end{corollary}
Note this does not mean $\C\setminus\R\subseteq \res(A)$, as $\dom(R_A(\lambda))$ may not be all of $H$.
\begin{proof}
This is an application of \ref{boundedBelow}.
\end{proof}

\begin{proposition} \label{rangeSelfAdjointCriterion}
Let $A$ be a symmetric operator on a Hilbert space $H$. The following are equivalent:
\begin{enumerate}
\item $\forall z \in \C\setminus\R: \; \im(A+z\id) = H = \im(A+\overline{z}\id)$;
\item $\exists z \in \C: \; \im(A+z\id) = H = \im(A+\overline{z}\id)$;
\item $A$ is self-adjoint;
\item $\rspec(A) = \emptyset$;
\item $A$ is closed and $\forall z \in \C\setminus \R: \; \ker(A^*+z\id) = \{0\} = \ker(A^*+\overline{z}\id)$;
\item $A$ is closed and $\exists z \in \C\setminus \R: \; \ker(A^*+z\id) = \{0\} = \ker(A^*+\overline{z}\id)$.
\end{enumerate}
In this case $\spec(A) = \apspec(A)$.
\end{proposition}
Notice that in (2) we include $\R$ and in (6) we exclude $\R$.
\begin{proof}
$(1) \Rightarrow (2)$ Trivial.

$(2) \Rightarrow (3)$ From \ref{symmetricOperatorAdjointInclusion}, we have $A\subseteq A^*$ and thus $A+z\id = (A^* + z\id)\cap(A+z\id)$. Similarly, $A+\overline{z}\id = (A^* + \overline{z}\id)\cap(A+\overline{z}\id)$.

From point (1) of \ref{equalityAlgebraicPropertiesAdjoint}, we have $(A+z\id)^* = A^* + \overline{z}\id$ and $(A+\overline{z}\id)^* = A^* + z\id$.

We use \ref{adjointRangeCriterion} with $S = A+z\id$ and $T = A+\overline{z}\id$, which is applicable since
\begin{align*}
\im(S\cap T^*) &= \im\big((A+z\id)\cap (A+\overline{z}\id)^*\big) = \im\big((A+z\id)\cap (A^*+z\id)\big) = \im(A+z\id) = H \\
\im(T\cap S^*) &= \im\big((A+\overline{z}\id)\cap (A+z\id)^*\big) = \im\big((A+\overline{z}\id)\cap (A^*+\overline{z}\id)\big) = \im(A+\overline{z}\id) = H.
\end{align*}
Thus we have $A^* + \overline{z}\id = (A+z\id)^* = S^* = T = A+\overline{z}\id$. Subtracting $\overline{z}\id$ from each side yields the result.

$(3) \Rightarrow (4)$ Because self-adjoint operators are normal, we can use \ref{equalityKernelAdjointNormal}.

$(4) \Rightarrow (1)$ We have $\spec(A) = \apspec(A)$. Because $\apspec\subseteq \R$, by \ref{approximateSpectrumSymmetricOperator}, we have that $A+z\id$ is surjective for all $\C\setminus\spec(A) = \C\setminus\apspec(A) \supseteq \C\setminus\R$.

$(1,3) \Rightarrow (5)$ The closedness of $A$ follows from its self-adjointness.

Pick arbitrary $z \in \C\setminus\R$. Using \ref{kernelImageAdjoint}, we have
\[ \ker(A^* + z\id) = \ker(A+\overline{z}\id)^* =\im(A+\overline{z}\id)^\perp = H^\perp = \{0\}, \]
and something similar for $\ker(A^* + \overline{z}\id)$.

$(5) \Rightarrow (6)$ Trivial.

$(6) \Rightarrow (2)$ Pick some $z\in\C\setminus \R$ for which the statement holds. We have
\[ \overline{\im(A+z\id)} = \im(A+z\id)^{\perp\perp} = \ker\big((A^*+\overline{z})\big)^\perp = \{0\}^\perp = H. \]
We now just need to show that $\im(A+z\id)$ is closed. This follows because $A+z\id$ is bounded below by \ref{symmetricPlusiBoundedBelow} and thus we can apply \ref{boundedBelowClosedRange}.
\end{proof}
\begin{corollary}
Let $A$ be a symmetric operator on a Hilbert space $H$. The following are equivalent:
\begin{enumerate}
\item $A$ is essentially self-adjoint;
\item $\exists z \in \C\setminus\R: \; \overline{\im(A+z\id)} = H = \overline{\im(A+\overline{z}\id)}$;
\item $\exists z \in \C\setminus\R: \; \ker(A^*+z\id) = \{0\} = \ker(A^*+\overline{z}\id)$.
\end{enumerate}
\end{corollary}
\begin{corollary}
Every surjective symmetric operator is self-adjoint.
\end{corollary}
\begin{proof}
Take $z=0$ in point (1).
\end{proof}

\begin{proposition}
Let $A$ be a closed symmetric operator. Then one of the following cases holds:
\begin{itemize}
\item $A$ is self-adjoint, in which case $\spec(A) \subseteq \R$;
\item $\spec(A) = \overline{\C^{\uparrow}}$;
\item $\spec(A) = \overline{\C^{\downarrow}}$;
\item $\spec(A) = \C$.
\end{itemize}
If $A$ is not densely-defined, then the last case holds.
\end{proposition}
We have denoted the closed upper half plane $\overline{\C^{\uparrow}}$ and the closed lower half plane $\overline{\C^{\downarrow}}$.
\begin{proof}
First assume $A$ self-adjoint, then $\spec(A)\subseteq \R$ by a combination of \ref{approximateSpectrumSymmetricOperator} and \ref{rangeSelfAdjointCriterion}.

Now note that if there exists a real $\lambda\in\R$ such that $\lambda \in \res(A)$, then in particular $\lambda\id -A$ is surjective, so $A$ is self-adjoint by \ref{rangeSelfAdjointCriterion}.

Now assume $A$ not self-adjoint and pick a $\lambda\in \C^{\uparrow}$. From \ref{rangeSelfAdjointCriterion} we must have either $\lambda\in\spec(A)$ or $\overline{\lambda}\in\spec(A)$ (or both).

If $\lambda\in \res(A)$, then $\C^\uparrow \subseteq \res(A)$ and if $\overline{\lambda}\in\res(A)$, then $\C^\downarrow \subseteq \res(A)$.

Indeed take some $\mu\in\C$.
By \ref{symmetricResolvent} we only need to check surjectivity of $\mu\id - A$. We calculate
\begin{align*}
(\mu\id - A)R_A(\lambda) &= (\mu\id -\lambda\id+\lambda\id - A)R_A(\lambda) \\
&= (\mu-\lambda)R_A(\lambda) + (\lambda\id-A)R_A(\lambda) \\
&= (\mu-\lambda)R_A(\lambda) + \id \\
&= \id - (\lambda-\mu)R_A(\lambda).
\end{align*}
Now, using \ref{symmetricResolvent}, we have
\[ \norm{(\lambda-\mu)R_A(\lambda)} \leq |\mu-\lambda| \, |\Im(\lambda)|^{-1}. \]
If $|\mu-\lambda| < |\Im(\lambda)|$, then $(\lambda-\mu)R_A(\lambda)$ is a contraction and we can apply the Neumann series formula \ref{NeumannSeries} to see that $(\mu\id - A)R_A(\lambda)$ is bijective. In particular $\mu\id - A$ is surjective.

We can iterate this construction to cover the whole of $\C^\uparrow$. The argument for $\overline{\lambda}$ is similar.
\end{proof}

\begin{example}
Spectrum half plane TODO \url{https://math.stackexchange.com/questions/893899/spectrum-of-symmetric-non-selfadjoint-operator-on-hilbert-space}

\url{https://math.stackexchange.com/questions/925097/spectrum-of-self-adjoint-operator-on-hilbert-space-real}
\end{example}

\begin{proposition} \label{symmetryAdjointConstructions}
Let $T$ be a densely defined operator on a Hilbert space $H$. Then
\begin{enumerate}
\item $T+T^*$ is symmetric;
\item $T^*T$ and $TT^*$ are symmetric.
\end{enumerate}
\end{proposition}
\begin{proof}
(1) We use \ref{adjointAlgebraicProperties} to get
\[ T+T^* \subseteq T^{**} + T^* \subseteq (T+T^*)^*. \]
We conclude by \ref{symmetricOperatorAdjointInclusion}.

(2) We use \ref{adjointAlgebraicProperties} to get
\[ T^*T \subseteq T^*T^{**} \subseteq (T^*T)^* \qquad\text{and}\qquad TT^* \subseteq T^{**}T^* \subseteq (TT^*)^*, \]
which means that $T^*T$ and $TT^*$ are symmetric by \ref{symmetricOperatorAdjointInclusion}.
\end{proof}

\begin{theorem}[von Neumann] \label{vonNeumannTheoremSquareSelfAdjoint}
Let $T$ be a densely defined and closed operator on a Hilbert space $H$. Then
\begin{enumerate}
\item both $T^*T$ and $TT^*$ are self-adjoint;
\item both $\dom(T^*T)$ and $\dom(TT^*)$ are essential domains of $T$.
\end{enumerate}
\end{theorem}
\begin{proof}
(1) Because $T^*$ is closed, $\graph(T^*)$ is closed in $H\oplus H$. Thus
\begin{align*}
H\oplus H &= \graph(T^*) \oplus \graph(T^*)^\perp \\
&= \graph(T^*) \oplus \left(\begin{pmatrix}
0 & -\id \\ \id & 0
\end{pmatrix}\graph{T}\right)^{\perp\perp} \\
&= \graph(T^*) \oplus \begin{pmatrix}
0 & -\id \\ \id & 0
\end{pmatrix}\graph{T}.
\end{align*}
The last equality holds because $\graph(T)$ is closed (and $\begin{pmatrix}
0 & -\id \\ \id & 0
\end{pmatrix}$ is a homeomorphism).

Then for all $v\in H$, we can write
\[ \begin{pmatrix}
0 \\ v
\end{pmatrix} = \begin{pmatrix}
y \\ T^*y
\end{pmatrix} + \begin{pmatrix}
0 & -\id \\ \id & 0
\end{pmatrix}\begin{pmatrix}
x \\ Tx
\end{pmatrix} = \begin{pmatrix}
y - Tx \\ T^*y + x
\end{pmatrix}. \]
So $y = Tx$ and $v = T^*y + x = T^*Tx + x = (T^*T +\id)x$, which means that $T^*T +\id$ is surjective. Since $T^*T$ is symmetric, by \ref{symmetryAdjointConstructions}, it is self-adjoint by \ref{rangeSelfAdjointCriterion}.

We can show $TT^* + \id$ is surjective by writing
\[ \begin{pmatrix}
v \\ 0
\end{pmatrix} = \begin{pmatrix}
y \\ T^*y
\end{pmatrix} + \begin{pmatrix}
0 & -\id \\ \id & 0
\end{pmatrix}\begin{pmatrix}
x \\ Tx
\end{pmatrix} = \begin{pmatrix}
y - Tx \\ T^*y + x
\end{pmatrix}, \]
so $x = -T^*y$ and $v = y - Tx = y+TT^*y = (TT^* + \id)y$.

(2) We need to show that $\dom(T^*T)$ is dense in $\dom(T)$ w.r.t. the graph norm. Take $h\in \dom(T^*T)^{\perp_{\graph(T)}}$. Then, for all $x\in \dom(T^*T)$, we have
\[ 0 = \inner{x,h}_{\graph(T)} = \inner{x,h} + \inner{Tx,Th} = \inner{x,h} + \inner{T^*Tx,h} = \inner{(\id + T^*T)x,h}. \]
Thus $h\in \im(\id+T^*T)^\perp = H^\perp = \{0\}$ and thus $h=0$ by the calculations in (1). So the orthogonal complement of $\dom(T^*T)$ w.r.t.\ the graph inner product is $\{0\}$, which shows density.

The argument for $\dom(TT^*)$ is similar.
\end{proof}

\begin{example}
Let $T$ be a densely defined operator. Then $T+T^*$ and $T^*T$ are in general not self-adjoint. Closedness of $T$ is enough to make $T^*T$ self-adjoint. This is not the case for $T+T^*$.
\begin{itemize}
\item If $T$ is not closed, then $T+T^* \subsetneq T^{**} + T^* \subseteq (T+T^*)^*$.
\item It is even not necessarily self-adjoint if $T$ is closed. Let $T$ be a closed, symmetric, but not self-adjoint operator, for example.
\end{itemize}
\end{example}

\begin{proposition}
Let $A$ be a self-adjoint operator on a Hilbert space. Then $A$ is positive \textup{if and only if} $\spec(A)\subseteq \interval[co]{0,\infty}$.
\end{proposition}
\begin{proof}
\ref{closureNumericRangeConvexHullSpectrum}
\end{proof}

\begin{proposition}
Let $A$ be a self-adjoint operator. Then
\begin{enumerate}
\item $\inf \sigma(A) = \inf\NumRange(A)$;
\item $\sup \sigma(A) = \sup\NumRange(A)$.
\end{enumerate}
\end{proposition}
\begin{proof}
\ref{closureNumericRangeConvexHullSpectrum}
\end{proof}

\begin{proposition}
Let $T$ be a densely defined self-adjoint operator. Then
\begin{enumerate}
\item $\rspec(T) = \emptyset$;
\item let $\lambda_1,\lambda_2 \in \pspec(T)$ and $\lambda_1\neq \lambda_2$, then 
\[ \Null(\lambda_1\id - T) \perp \Null(\lambda_2 \id - T). \]
\end{enumerate}
\end{proposition}
\begin{proof}
TODO
\end{proof}


\begin{proposition}
Let $T$ be a symmetric operator on a Hilbert space $H$. Then
\begin{enumerate}
\item the eigenvalues of $T$ are real;
\item the eigenvectors corresponding to distinct eigenvalues are orthogonal.
\end{enumerate}
\end{proposition}
\begin{proof}
This is an application of \ref{eigenspaceOrthogonalAdjoint} and \ref{adjointSpectrumNoResidual}.
\end{proof}

\subsubsection{Compact self-adjoint operators}
\begin{proposition}
Every compact self-adjoint operator $L$ on a nontrivial Hilbert space has an eigenvalue $\lambda$ with $|\lambda| = \norm{L}$.
\end{proposition}

\begin{proposition}
Let $A$ be a compact self-adjoint operator. Then the only possible accumulation point of $\spec(A)$ is $0$.
\end{proposition}
TODO self-adjoint not necessary? See \ref{spectrumCompactOperator}?
\begin{proof}
Assume $\spec(A)$ is infinite. Then take $\seq{\lambda_n}\subset \spec(A)$. Any associated sequence $\seq{x_n}$ of eigenvectors is orthogonal. We can take it to be orthonormal. By \ref{limitCompactImageOrthonormalSequence} we have
\[ 0 = \lim_{n\to\infty} \norm{Ax_n}^2 = \lim_{n\to\infty}\inner{Ax_n,Ax_n} = \lim_{n\to\infty}\lambda_n^2\inner{x_n,x_n} = \lim_{n\to\infty}\lambda_n^2, \]
so $\seq{\lambda_n}$ converges to $0$.
\end{proof}

\begin{theorem}
Every spectral value $\lambda\neq 0$ of a compact self-adjoint linear
operator $A : H \to H$ is an eigenvalue of finite multiplicity that can only
accumulate at $\lambda = 0$. Conversely, a self-adjoint operator having these
properties is compact.
\end{theorem}
\begin{proof}
TODO See \ref{spectrumCompactOperator}
\end{proof}

\subsubsection{Self-adjoint extensions of symmetric operators}
\paragraph{Cayley transform}
Consider the Möbius transform
\[ \C\setminus\{\overline{\lambda}\} \to \C: x\mapsto \frac{x - \lambda}{x-\overline{\lambda}} \qquad \text{for some $\lambda\in\C\setminus\R$.} \]
This transform maps
\begin{itemize}
\item the real line to $\T\setminus\{1\}$;
\item the half-plane above / below the real line containing $\lambda$ to the interior of the unit disk;
\item the half plane containing $\overline{\lambda}$ to the exterior of the unit disk;
\item in particular $\lambda \mapsto 0$ and $\overline{\lambda} \mapsto \infty$.
\end{itemize}
Conventional choice: $\lambda = i$.

\paragraph{Defect indices}
Or deficiency(?)
\url{https://link-springer-com.ezproxy.ulb.ac.be/content/pdf/10.1007/978-94-007-4753-1.pdf}

Cfr. dilation theory through Cayley transform.

See also Conway.

\subsubsection{Positive operators}
All positive operators are symmetric, \ref{positiveOperatorSymmetric}. They are not all self-adjoint (as in the bounded case), but can always be extended to a self-adjoint operator (which is not true in general for symmetric operators).

\begin{definition}
Let $A$ be a positive operator on a Hilbert space $H$. We call
\[ H_A \defeq \setbuilder{x\in H}{\exists \seq{x_n}\subseteq \dom(A): \; \text{$\seq{x_n}$ is $\inner{}_{A+\id}$-Cauchy and $\seq{x_n}\overset{\inner{}}{\longrightarrow} x$}} \]
the \udef{form domain} of $A$.
\end{definition}

\begin{lemma}
Let $A$ be a positive operator on a Hilbert space $H$ with form domain $H_A$. Then
\begin{enumerate}
\item $H_A$ is the completion of $\sSet{\dom(A), \inner{\cdot, \cdot}_{A+\id}}$;
\item $\inner{\cdot, \cdot}_A$ can be extended to an inner product on $H_A$.
\end{enumerate}
\end{lemma}
\begin{proof}
(1) By \ref{energyNormTopology}, every $\inner{}_{A+\id}$-Cauchy sequence is $\inner{}$-Cauchy

TODO: what is convergence on $H_A$??

----

We use \ref{completenessCriterion} to show completeness. Take some Cauchy sequence $\seq{x_n}$ in $\sSet{\dom(A), \inner{\cdot, \cdot}_{A+\id}}$. Then $\seq{x_n}$ is a Cauchy sequence in $\sSet{H,\inner{\cdot, \cdot}}$ by \ref{energyNormTopology} and \ref{uniformContinuityGroupHomomorphism}. 

(2) We extend $\inner{\cdot, \cdot}_A$ using the fact that we can extend $\inner{\cdot, \cdot}_{A+\id}$ to $H_A$ and the equation
\[ \inner{x, y}_A = \inner{x, y}_{A+\id} - \inner{x,y}. \]
\end{proof}




\begin{lemma}
Let $A$ be a positive operator on a Hilbert space $H$. Consider the operator $B$ defined by
\begin{align*}
\dom(B) &\defeq \setbuilder{x\in \Closure_{\inner{}_{A+\id_H}}\big(\dom(A)\big)}{\exists x'\in H: \forall y\in \dom(A): \; \inner{x, y}_{A+\id} = \inner{x', y}} \\
Bx &= x' - x 
\end{align*}
Then
\begin{enumerate}
\item $A\subseteq B$;
\item $B$ is positive;
\item $\im(B+\id_H) = H$.
\end{enumerate}
\end{lemma}
\begin{proof}

\end{proof}

\begin{theorem}[Friedrich's extension]
Let $A$ be a positive symmetric operator on a Hilbert space $H$. Then $A$ has a unique positive self-adjoint extension $\widetilde{A}$ with domain $\dom(\widetilde{A}) \subseteq \Closure_{\inner{}_{A+\id_H}}(\dom(A))$.
\end{theorem}
By \ref{energyNormTopology}, we have
\[ \dom(\widetilde{A}) \subseteq \Closure_{\inner{}_{A+\id_H}}(\dom(A)) \subseteq \Closure_{\norm{\cdot}}(\dom(A)). \]
\begin{proof}
Set $H_A \defeq \Closure_{\inner{}_{A+\id_H}}(\dom(A))$.

For \undline{existence}, we can construct the operator $\widetilde{A}$ as follows:
\begin{align*}
\dom(\widetilde{A}) &\defeq \setbuilder{x\in H_A}{\exists x'\in H:\forall y\in H_A:\; \inner{y,x}_{A+\id} = \inner{y,x'}} \\
\widetilde{A}x &\defeq x' - x.
\end{align*}
Now $\widetilde{A}$ is an extension of $A$, because for all $x\in \dom(A)$, we can take $x' = Ax + x$. So $\widetilde{A}x = Ax$.

But $\dom(\widetilde{A})$ may be larger than $\dom(A)$, because we can extended $\inner{y,x}_{A+\id}$ to be defined on all of $H_A$ by continuity.

By construction $\dom(\widetilde{A}) \subseteq \Closure_{\norm{\cdot}_{E(A+\id)}}(\dom(A))$.

Now we claim $\im(\widetilde{A} + \id) = H$. Indeed for any $x'\in H$, the functional $H_A \to H_A: y\mapsto \inner{y,x'}$ is bounded. By Riesz representiation \ref{rieszRepresentation}, we can find an $x\in H_A$ such that $\inner{y,x}_{A+\id} = \inner{y,x'}$. Thus $(\widetilde{A} + \id)x = x'$.

By \ref{rangeSelfAdjointCriterion} we conclude that $\widetilde{A}$ is self-adjoint. 

For \undline{uniqueness}, assume there exists a second such extension $\widehat{A}$. For all $y\in \dom(A)$ and $x\in \dom(\widehat{A})$, we have
\[ \inner{y, (\widehat{A}+\id)x} = \inner{(\widehat{A}+\id)y, x} = \inner{(A+\id)y, x} = \overline{\inner{x, (A+\id)y}} = \overline{\inner{x, y}_{A+\id}} = \inner{y, x}_{A+\id}. \]
By continuity this holds for all $y\in H_A$. And thus by definition $\widetilde{A}x = \widehat{A}x$ for all $x\in\dom(\widetilde{A})$. Thus $\widetilde{A} \subseteq \widehat{A}$, but self-adjoint operators are maximal by \ref{selfAdjointMaximal}, so $\widetilde{A} = \widehat{A}$.
\end{proof}

\subsubsection{Bounded self-adjoint operators}
\begin{lemma}
Let $A, B\in\Bounded(H)$. Then
\begin{enumerate}
\item $A^*A, AA^*$ and $A+A^*$ are self-adjoint;
\item if $A,B$ are self-adjoint, then $AB$ is self-adjoint \textup{if and only if} $A,B$ commute.
\end{enumerate}
\end{lemma}
\begin{corollary}
Let $A\in\Bounded(H)$. Then there exist unique self-adjoint operators $S,T$ such that
\[ A = S+iT \qquad A^* = S-iT. \]
\end{corollary}
\begin{proof}
Indeed $S = (A+A^*)/2$ and $T = (A-A^*)/2i$ are self-adjoint.
\end{proof}
\begin{corollary}
The operator $A$ is normal \textup{if and only if} $S,T$ commute.
\end{corollary}
\begin{proof}
We calculate the commutator
\[ [S,T] = \left[\frac{A+A^*}{2}, \frac{A-A^*}{2i}\right] = \frac{A^*A - AA^*}{2i} = \frac{1}{2i}[A^*, A]. \]
\end{proof}

\begin{proposition}
The set of bounded self-adjoint operators forms an anti-lattice.
\end{proposition}
\begin{proof}
TODO + generalised to self-adjoint operators??
\end{proof}

\subsection{Normal operators}
\begin{definition}
A densely defined linear operator $T$ on a Hilbert space $H$ is \udef{normal} if it is closed and $TT^* = T^*T$.
\end{definition}
Self-adjoint and unitary operators are normal.

\begin{proposition} \label{normalCriterion}
Let $T: H\not\to H$ be a densely defined operator. Then $T$ is normal \textup{if and only if} $\dom(T) = \dom(T^*)$ and $\forall x\in \dom(T): \norm{Tx} = \norm{T^*x}$.
\end{proposition}
\begin{proof}
First, assume $T$ normal. Then, for all $x\in \dom(T^*T) = \dom(TT^*)$, we have $x\in \dom(T)$ and $x\in \dom(T^*)$ and
\[ \norm{Tx}^2 = |\inner{Tx,Tx}| = |\inner{T^*Tx,x}| = |\inner{TT^*x,x}| = |\inner{T^*x,T^*x}| = \norm{T^*x}^2. \]
By \ref{vonNeumannTheoremSquareSelfAdjoint}, $\dom(T^*T)$ is $\graph(T)$-dense in $\dom(T)$. Thus, for all $x\in \dom(T)$, there exists a sequence $\seq{x_n}\in \dom(T^*T)^\N$ such that $x_n \overset{\graph(T)}{\longrightarrow} x$.

In particular, $Tx_n \to Tx$, which means $\seq{Tx_n}$ is a Cauchy sequence. Since $x_n,x_m\in \dom(T^*T)$, we have shown that $\norm{Tx_n - Tx_m} = \norm{T^*x_n - T^*x_m}$ and thus $\seq{T^*x_n}$ is Cauchy by \ref{CauchyCriterion}. It converges to some $y\in H$ by completeness and so $x\in \dom(T^*)$ and $T^*x = y$ by \ref{closedGraphEquivalence}, since $T^*$ is closed \ref{adjointGraph}. 

This shows that $\dom(T) \subseteq \dom(T^*)$. The same reasoning with $T^*$ gives the opposite inclusion.

Finally, we calculate
\[ \norm{Tx} = \lim_n\norm{Tx_n} = \lim_n \norm{T^*x_n} = \norm{T^*x}. \]

For the converse, we first prove that $T$ is closed, using \ref{closedGraphEquivalence}. Suppose $\seq{x_n}\in \dom(T)^\N$ converges to $x$ and $\seq{Tx_n}$ also converges. Then $\seq{Tx_n}$ is Cauchy and, since $\norm{Tx_n - Tx_m} = \norm{T^*x_n - T^*x_m}$, the sequence $\seq{T^*x_n}$ is also Cauchy and thus convergent. Since $T^*$ is closed \ref{adjointGraph}, we have $x\in \dom(T^*) = \dom(T)$ and $T^*x_n \to T^*x$ by \ref{closedGraphEquivalence}. Thus
\[ \norm{Tx_n - Tx} = \norm{T^*x_n - T^*x} \to 0, \]
so $Tx_n \to Tx$.

For all $x,y\in\dom(T) = \dom(T^*)$, we have
\[ \inner{Tx,Ty} = \frac{1}{4}\sum_{k=0}^3\norm{i^kTx+ i^kTy}^2 = \frac{1}{4}\sum_{k=0}^3\norm{i^kT^*x+ i^kT^*y}^2 = \inner{T^*x, T^*y} \]
by the polarisation identity \ref{polarisationIdentities}.
Using \ref{adjointDomain}, we have
\begin{align*}
x\in \dom(T^*T) &\iff Tx \in \dom(T^*) \\
&\iff \text{$y\mapsto \inner{Tx, Ty}$ is a bounded functional} \\
&\iff \text{$y\mapsto \inner{T^*x, T^*y}$ is a bounded functional} \\ 
&\iff T^*x \in \dom(T^{**}) = \dom(T) \\
&\iff x\in \dom(TT^*), 
\end{align*}
where we have used $T^{**} = T$ by \ref{adjointDenselyDefinedClosable}.

Finally, take $x\in \dom(T^*T) = \dom(TT^*)$ and $y\in \dom(T) = \dom(T^*)$. Then
\[ \inner{Tx,Ty} = \inner{T^*x, T^*y} \implies \inner{T^*Tx,y} = \inner{TT^*x, y} \implies \inner{(T^*Tx - TT^*)x, y} = 0, \]
so $(T^*Tx - TT^*)x \in \dom(T)^\perp = \{0\}$, so $T^*Tx = TT^*x$.
\end{proof}
\begin{corollary} \label{equalityKernelAdjointNormal}
If $T$ is a normal operator, then $\ker T = \ker T^*$.
\end{corollary}
\begin{proof}
We have $x\in\ker(T) \iff \norm{Tx} = 0 \iff \norm{T^*x} = 0 \iff x\in\ker(T^*)$. 
\end{proof}
\begin{corollary}
If $T$ is a normal operator then
\begin{enumerate}
\item $\rspec(T) = \emptyset$;
\item $\spec(T) = \apspec(T)$.
\end{enumerate} 
\end{corollary}
\begin{proof}
If $T$ is normal, then so is $\lambda\id-T$. Now $\lambda\in\rspec(T)$ iff $\ker(\lambda\id - T) = \{0\}$ and $\im(\lambda\id-T)^\perp \neq \{0\}$, but $\im(\lambda\id-T)^\perp = \ker(\lambda\id-T)^* = \ker(\lambda\id-T)$. By \ref{kernelImageAdjoint} and the previous corollary. This is a contradiction.

(2) then follows straight from (1).
\end{proof}

\begin{theorem} \label{closureNumericRangeConvexHullSpectrum}
The closure of the numerical range of a normal operator is the
convex hull of its spectrum.
\end{theorem}
\begin{proof}
Normal operators $T$ are by definition closed, so $\spec(T)\subseteq \overline{\NumRange(T)}$ by \ref{spectralInclusionNumericalRange}. TODO
\end{proof}

\begin{lemma} \label{normalSpectralRadiusEqualsNorm}
For normal elements the spectral radius equals the norm.
\end{lemma}

\begin{lemma}
A normal operator on a Hilbert space is invertible \textup{if and only if} it is bounded below.
\end{lemma}

\begin{theorem}[Fuglede's theorem]
Let $H$ be a Hilbert space $N$ a normal operator on $H$ and $A\in\Bounded(H)$. Then $AN\subseteq NA$ implies $AN^* \subseteq N^*A$.
\end{theorem}
\begin{proof}
TODO
\end{proof}
\begin{corollary}[Putnam-Fuglede theorem]
Let $H, K$ be a Hilbert spaces, $A\in\Bounded(K, H)$ and both $N: H\not\to K$ and $M: K\not\to H$ normal operators. Then $AN\subseteq MA$ implies $AN^* \subseteq M^*A$.
\end{corollary}
\begin{proof}
Consider
\[ L \defeq \begin{pmatrix}
N & 0 \\ 0 & M
\end{pmatrix} \qquad\text{and}\qquad A' \defeq \begin{pmatrix}
0 & A \\ 0 & 0
\end{pmatrix}. \]
Then $L$ is normal (TODO) and $A'$ is bounded. Then we can apply Fuglede's theorem (TODO details).
\end{proof}

\subsubsection{Spectral measures}
\begin{definition}
Let $H$ be a Hilbert space and $\sSet{\Omega,\mathcal{A}}$ a measurable space.
\begin{itemize}
\item A measure $E$ from $\sSet{\Omega,\mathcal{A}}$ to the lattice of projectors on $H$ is called a \udef{projector-valued measure}.
\item A projector-valued measure $E$ such that $E(\Omega) = \id_H$ is called a \udef{spectral measure}.
\end{itemize}
Let $E$ be a projector-valued measure on $\sSet{\Omega,\mathcal{A}}$. For any measurable function $f: X\to \C$ we define
\[ \int_\Omega f\diff{E}: \dom\Big(\int_\Omega f\diff{E}\Big) \to H: x\mapsto \int_\Omega f\diff{E_x}, \]
where the integral is a Bochner integral,
\[ E_x: \mathcal{A}\to H: A\mapsto E(A)(x) \]
and
\[ \dom\Big(\int_\Omega f\diff{E}\Big) = \setbuilder{x\in H}{\text{$f$ if $E_x$-integrable}}. \]
\end{definition}
TODO: can we make definition of Bochner integral general enough such that this integral can be  considered as a Bochner integral??

\begin{lemma}
Let $H$ be a Hilbert space, $E$ a projector-valued measure on the measurable space $\sSet{\Omega,\mathcal{A}}$ and $x\in H$. Then $E_x$ is a vector-valued measure.
\end{lemma}
\begin{proof}
TODO
\end{proof}

\begin{lemma} \label{projectorIntegrableSolid}
Let $H$ be a Hilbert space and $E$ a projector-valued measure on the measurable space $\sSet{\Omega,\mathcal{A}}$. Let $f,g: \Omega\to \C$ be measurable functions such that $|f|\leq |g|$. Then $\dom\Big(\int_\Omega f \diff{E}\Big) \supseteq \dom\Big(\int_\Omega g \diff{E}\Big)$.
\end{lemma}
\begin{proof}
TODO
\end{proof}

\begin{proposition} \label{integralProjectorValuedMeasure}
Let $H$ be a Hilbert space and $E$ a projector-valued measure on the measurable space $\sSet{\Omega,\mathcal{A}}$. Let $f,g: \Omega\to \C$ be measurable functions. Then
\begin{enumerate}
\item $\int_\Omega f\diff{E}$ is a normal operator;
\item $\Big(\int_\Omega f\diff{E}\Big)^* = \int_\Omega \overline{f}\diff{E}$;
\item $\int_\Omega f\diff{E}\circ \int_\Omega g\diff{E} \subseteq \int_\Omega f\cdot g\diff{E}$ and $\dom\Big(\int_\Omega f\diff{E}\circ\int_\Omega g\diff{E}\Big) = \dom\Big(\int_\Omega f\cdot g\diff{E}\Big)\cap \dom\Big(\int_\Omega g\diff{E}\Big)$.
\end{enumerate}
\end{proposition}
\begin{proof}
TODO
\end{proof}
\begin{corollary}
Let $H$ be a Hilbert space and $E$ a projector-valued measure on the measurable space $\sSet{\Omega,\mathcal{A}}$. Let $f,g: \Omega\to \C$ be measurable functions. Then
\begin{enumerate}
\item $\Big(\int_\Omega f\diff{E}\Big)^*\circ \int_\Omega f\diff{E} = \int_\Omega |f|^2\diff{E}$;
\item if $g$ is bounded, then $\int_\Omega f\diff{E}\circ \int_\Omega g\diff{E} = \int_\Omega f\cdot g\diff{E} = \int_\Omega g\diff{E}\circ \int_\Omega f\diff{E}$.
\end{enumerate}
\end{corollary}

\subsubsection{Spectral theorem}
\begin{theorem}[Spectral theorem]
Let $H$ be a Hilbert space and $N$ a normal operator on $H$. Then there exists a unique spectral measure $E$ on $\C$ such that
\begin{enumerate}
\item $N = \int z\diff{E(z)}$;
\item $E(A) = 0$ for all Borel sets $A$ such that $A\perp \spec(N)$;
\item if $U\subseteq \C$ is an open set such that $U\mesh \spec(N)$, then $E(U) \neq 0$;
\item if $T\in \Bounded(H)$ is such that $TN\subseteq NT$, then $A\big(\int_\C f\diff{E}\big) \subseteq \big(\int_\C f\diff{E}\big)A$ for all measurable $f:\C\to \C$.
\end{enumerate}
\end{theorem}
Clearly $\int z\diff{E(z)}$ is shorthand for $\int_\C\id_\C\diff{E}$.
\begin{proof}
TODO
\end{proof}
\begin{corollary} \label{realSpectrumSelfAdjoint}
Let $H$ be a Hilbert space and $N$ a normal operator on $H$. If $\spec(N) \subseteq \R$, then $N = N^*$.
\end{corollary}
\begin{proof}
We have $\C\setminus\R \perp \spec(N)$, so $N = \int_\C z\diff{E(z)} = \int_\R z\diff{E(z)}$. By \ref{integralProjectorValuedMeasure}, we have
\[ N = \int_\R z\diff{E(z)} = \int_\R \overline{z}\diff{E(z)} = \Big(\int_\R z\diff{E(z)}\Big)^* = N^*. \]
\end{proof}

\begin{definition}
Let $H$ be a Hilbert space, $N$ a normal operator on $H$ and $f: \C\to \C$ a measurable function. Then we define
\[ f(N) \defeq \int_\C f\diff{E}, \]
where $E$ is the spectral measure associated to $N$.
\end{definition}

\begin{theorem} \label{spectralTheoremFunctionalCalculus}
Let $H$ be a Hilbert space, $N$ a normal operator on $H$ and $f,g: \C\to \C$ measurable functions. Then
\begin{enumerate}
\item $(f\circ g)(N) = f\big(g(N)\big)$;
\item $\spec(f(N)) = f^\imf\big(\spec(N)\big)$.
\end{enumerate}
\end{theorem}
\begin{proof}
TODO!
\end{proof}

\begin{proposition}
Let $T$ be a normal operator on a Hilbert space. If $\lambda$ is an isolated point of the spectrum, then $\lambda$ is an eigenvalue.
\end{proposition}
\begin{proof}
Because $\lambda$ is isolated, the function
\[ f: \spec(T)\to \C: x\mapsto \begin{cases}
1 & (x=\lambda) \\
0 & (x\neq \lambda)
\end{cases} \]
is continuous.

Set $P = f(T)$ by continuous functional calculus (TODo ref!!). This is a projector by (TODO ref).

For all $t\in \spec(T)$, we have $tf(t) = \lambda f(t)$. By functional calculus, this gives $TP = \lambda P$.
\end{proof}


\subsection{Orthogonal projections}
\url{https://planetmath.org/latticeofprojections}

\url{https://zfn.mpdl.mpg.de/data/Reihe_A/35/ZNA-1980-35a-0437.pdf}

We denote the set op projections on a Hilberts space $\mathcal{H}$ by $\Projections(\mathcal{H})$.

TODO: $\im(P) = \ker{P^*}^\perp$ shows that we need $P= P^*$ for orthogonality.

\begin{proposition}
Let $P$ be a bounded operator $P$ on a Hilbert space $\mathcal{H}$. Then the following are equivalent:
\begin{enumerate}
\item $P$ is an orthogonal projection onto a closed subspace of $\mathcal{H}$;
\item $P^2 = P$ and $P=P^*$;
\item $P^2 = P$ and $\norm{P}\in \{0,1\}$;
\item $P^2 = P$ and $\norm{P}\leq 1$;
\end{enumerate}
\end{proposition}
\begin{proof}
$\boxed{(1)\Rightarrow (2)}$  Suppose first that $P$ is the orthogonal projection operator onto a closed subspace $K$. Clearly $P^2 = P$. Let $x,y\in\mathcal{H}$ and write $x= x_1+x_2, y = y_1+y_2$ where $x_1,y_1\in K$ and $x_2,y_2\in K^\perp$. Then
\[ \inner{Px, y} = \inner{x_1, y_1+y_2} = \inner{x_1, y_1} + \inner{x_1,y_2} = \inner{x_1,y_1} = \inner{x_1+x_2, y_2} = \inner{x,Py}. \]
So $P = P^*$.

$\boxed{(2)\Rightarrow (3)}$ We calculate $\norm{P} = \norm{P^2} = \norm{P^*P} = \norm{P}^2$ using \ref{normOfSquare}. The solutions to this equation are $\{0,1\}$.

$\boxed{(3)\Rightarrow (4)}$ This is clear.

$\boxed{(4)\Rightarrow (1)}$ Define $K=\im P$, then $K$ is closed because $x\in K$ iff $Px=x$ and thus for any converging sequence $(x_n)_n\subset K$: $\lim x_n = \lim Px_n = P\left(\lim x_n\right)$, so the limit is in $K$.

We just need to show orthogonality: $Px \perp x- Px$. For this we use \ref{orthogonality}: for all $a\in\F$
\[ \norm{Px} = \norm{Px + aPx - aPx} = \norm{P(Px + a(x-Px))} \leq \norm{P}\cdot \norm{Px + a(x-Px)} \leq \norm{Px + a(x-Px)}. \]
We conclude $Px \perp x- Px$.
\end{proof}

\begin{proposition} \label{projectorOrthogonalComplement}
Let $\mathcal{H}$ be a Hilbert space and let $P$ be an orthogonal projector on a closed subspace $K$. Then $\id-P$ is the orthogonal projector on $K^\perp$.
\end{proposition}
\begin{proof}
Any $x\in \mathcal{H}$ can be uniquely decomposed as $x_1 + x_2\in K\oplus K^\perp$. If $Px = x_1$, then $(\id - P)x = x_1 +x_2 - x_1 = x_2$.
\end{proof}
\begin{corollary} \label{projectorsIn01}
The set of projectors $\Projections(\mathcal{H})$ is a subset of $[0,\id]$.
\end{corollary}
\begin{proof}
Let $P\in\Projections(\mathcal{H})$. Then $P\geq 0$ follows from $P = P^2 = P^*P$.
\end{proof}

\begin{proposition} \label{commutingProjectors}
Let $\mathcal{H}$ be a Hilbert space and $P,Q$ be projections. The following are equivalent:
\begin{enumerate}
\item $PQ = QP$;
\item $PQ$ is a projection;
\item $QP$ is a projection;
\item $P+Q-PQ$ is a projection;
\item $\im(PQP) = \im(P) \cap \im(Q)$;
\item $PQP = QP$;
\item $\mathcal{H} = \big(\im(P)\cap\im(Q)\big)\oplus \big(\im(P)\cap\im(Q)^\perp\big) \oplus \big(\im(P)^\perp\cap\im(Q)\big) \oplus \big(\im(P)^\perp\cap\im(Q)^\perp\big)$.
\end{enumerate}
\end{proposition}
\begin{proof}
Points (1), (2), (3) are equivalent by the equation $(PQ)^* = Q^*P^* = QP$, and the fact that (1) implies $(PQ)^2 = PQPQ = PPQQ = PQ$.

(4) If $P,Q$ commute, then
\begin{align*}
(P+Q-PQ)^* &= P+Q-(PQ)^* = P+Q-Q^*P^* =P+Q-QP = P+Q-PQ \\
(P+Q-PQ)^2 &= P^2 + PQ -P^2Q + QP+Q^2 - QPQ - PQP -PQP +PQPQ \\
&= P + Q + 3PQ - 4PQ= P+Q-PQ.
\end{align*}
Assume (4), then $(P+Q-PQ)^* = P+Q-QP = P+Q-PQ$. This implies $PQ=QP$.

$\boxed{(1)\Rightarrow (5)}$ Clearly $\im(PQP) \subseteq \im(P) \cap \im(Q)$.
For the inverse inequality, take $x\in im(P)\cap\im(Q)$. Then $PQP(x) = PQ(x) = P(x) = x$, so $x\in\im(PQP)$.

$\boxed{(5)\Rightarrow (6)}$ We decompose $\mathcal{H} = \im(PQP) \oplus \ker(PQP)$ and show that the operators are the same on both parts. For all $x\in \mathcal{H}$ we have
\[ x\in \ker(PQP) \iff \inner{x,PQPx} = 0 \iff \inner{QPx,QPx} = 0 \iff \norm{QPx} = 0 \iff x\in\ker{QP}.  \]
Now let $x\in\im(PQP) = \im(P)\cap\im(Q)$. Then $QPx = Qx = x = PQPx$.

$\boxed{(6)\Rightarrow (3)}$ $PQP$ is always a projection.

$\boxed{(6)\Rightarrow (7)}$ Take some $x\in \mathcal{H}$. Then we can uniquely decompose $x = P(x) + (x-P(x)) = x_P + x_{P^\perp} \in \im(P)\oplus \im(P)^\perp$. We can then further decompose $x_P = x_{P,Q} + x_{P,Q^\perp}$ and $x_{P^\perp} = x_{P^\perp, Q} + x_{P^\perp, Q^\perp}$. In order to have the decomposition of the proposition, we need to show that $x_{P,Q},x_{P,Q^\perp}\in \im(P)$ and $x_{P^\perp, Q},x_{P^\perp, Q^\perp}\in\im(P)^\perp$.

First take $x_{P,Q} = QPx$. From (6) we have $P(QPx) = PQPx = QPx$, so $x_{P,Q}\in \im(P)$. For the others we have similar calculations (also using the identity $PQP = PQ$):
\begin{align*}
P(x_{P,Q^\perp}) &= P\big((\id-Q)P\big)x = Px - PQPx = Px - QPx = (\id-Q)Px = x_{P,Q^\perp} \\
(\id-P)(x_{P^\perp,Q}) &= (\id-P)\big(Q(\id-P)\big)x = (Q-QP-PQ+PQP)x = (Q-QP)x = Q(\id-P)x = x_{P^\perp,Q} \\
(\id-P)(x_{P^\perp,Q^\perp}) &= (\id-P)\big((\id-Q)(\id-P)\big)x = (\id-P-Q+QP-P+P+PQ-PQP)x \\
&= (\id-Q-P+QP)x = (\id-Q)(\id-P)x = x_{P^\perp,Q^\perp}.
\end{align*}
$\boxed{(7)\Rightarrow (1)}$ Take $x\in \mathcal{H}$ and decompose it as $x_{P,Q} + x_{P,Q^\perp} + x_{P^\perp, Q} + x_{P^\perp, Q^\perp}$. Then $PQx = P(x_{P,Q} + x_{P^\perp, Q}) = x_{P,Q}$ and $QP = Q(x_{P,Q} + x_{P, Q^\perp}) = x_{P,Q}$, so $PQ = QP$. 
\end{proof}

\begin{proposition} \label{perpendicularProjections} \label{subspaceProjections}
Let $P,Q$ be orthogonal projections onto subspaces $\im(P)$ and $\im(Q)$ of $\mathcal{H}$.
\begin{enumerate}
\item The following are equivalent to $\im(P) \perp \im(Q)$:
\begin{enumerate}
\item $QP = 0$;
\item $PQ = 0$;
\item $Q+P$ is an orthogonal projection.
\end{enumerate}
\item The following are equivalent to $\im(P) \subseteq \im(Q)$:
\begin{enumerate}
\item $QP = P$;
\item $PQ = P$;
\item $Q-P$ is an orthogonal projection;
\item $P\leq Q$;
\item $\norm{Px} \leq \norm{Qx}$ for all $x \in \mathcal{H}$.
\end{enumerate}
\end{enumerate}
\end{proposition}
\begin{proof}
(1) We have:

$\boxed{(a)\Leftrightarrow (b) \Leftrightarrow \im(P) \perp \im(Q)}$ By \ref{commutingProjectors}.

$\boxed{(a, b)\Leftrightarrow (c)}$ We know $(P+Q)^* = P^*+Q^* =P+Q$ and we can write
\[ (P+Q)^2 = P^2 + Q^2 + PQ + QP = P+Q+ PQ+QP,  \]
So clearly (a) or (b) imply (c). Conversely, assume $PQ + QP = 0$, implying $PQ=-QP$. By left- and right-multiplication by $P$ this implies both
\[ PPQ = PQ = -PQP \qquad \text{and} \qquad PQP = -QPP = -QP. \]
So $PQ = -PQP = QP$, meaning $PQ = 1/2(PQ+QP) = 0$.

(2) We prove the following:

$\boxed{(a)\Leftrightarrow (b) \Leftrightarrow \im(P) \subseteq \im(Q)}$ By \ref{commutingProjectors}.

$\boxed{(a,b)\Rightarrow (c)}$ Obviously $(Q-P)^*= Q-P$. Also
\[ (Q-P)^2 = Q+P-PQ-QP= Q+P-2P = Q-P. \]

$\boxed{(c)\Rightarrow (a,b)}$ Now from
\[ Q-P = (Q-P)^2 = Q+P-PQ-QP \]
we obtain $2P = PQ+QP$. The result then follows if we can show that $PQ=QP$. This follows by multiplying the equality on the left and on the right by $P$ to obtain $QP = 2P-PQP$ and $PQ = 2P-PQP$, respectively. 

$\boxed{(c)\Rightarrow (d)}$ This follows because all projections are positive.

$\boxed{(d)\Rightarrow (a, b)}$ Assume, towards a contradiction, that $\im(P)\nsubseteq \im(Q)$. Then we can take $v\in\im(P)\setminus \im(Q)$. Then
\[ \inner{v,(Q-P)v} = \inner{v,Qv} - \inner{v,v} = \inner{Qv,Qv} - \inner{Qv,Qv} - \inner{v-Qv, v-Qv} = -\norm{v-Qv}^2. \]
Because $v\notin \im(Q)$, $\norm{v-Qv}$ is not zero and thus $Q-P$ is not positive.

$\boxed{(d)\Leftrightarrow (e)}$ By the equivalence
\[ \norm{Px} \leq \norm{Qx} \iff \inner{Px,Px} \leq \inner{Qx,Qx} \iff \inner{Px,x}\leq \inner{Qx,x} \iff \inner{(Q-P)x,x}\geq 0. \]
\end{proof}

We can generalise part 2(d) of the previous proposition to a slightly larger class of operators.
\begin{lemma} \label{comparisonSelfAdjointProjection}
Let $P\in \Projections(\mathcal{H})$ and $T \in [0,\id]$, then the following are equivalent:
\begin{enumerate}
\item $\im(T) \subseteq \im(P)$;
\item $T\leq P$.
\end{enumerate}
\end{lemma}
\begin{proof}
As $T$ is self-adjoint, we have $\norm{T} = \nr(T) \leq 1$ by \ref{normNumRadius}.

Assume (1) so that for all $x\in \mathcal{H}$ we get
\[ \inner{x,Tx} = \inner{x, PTx} = \inner{Px,PTx} \leq \norm{Px}^2\nr(T) \leq \norm{Px}^2 = \inner{Px,Px} = \inner{x,Px}. \]
So $\inner{x, (P-T)x}\geq 0$ and thus $T\leq P$.

Assume (2). The energy form associated with $T$ is a pre-inner product by \ref{positiveOperatorPositiveEnergyForm}. The Cauchy-Schwarz inequality \ref{CauchySchwarz} gives
\[ |\inner{v,Tw}|^2 \leq \inner{v,Tv}\inner{w,Tw} \leq \inner{v,Pv}\inner{w,Pw}. \]
So if $v\in\im(P)^\perp$, then $\inner{v,Tw} = 0$ for all $w\in \mathcal{H}$. So $\im(T)\perp \im(P)^\perp$, implying $\im(T)\subseteq \im(P)^{\perp\perp} = \im(P)$.
\end{proof}

\begin{proposition}
Let $\mathcal{H}$ be a Hilbert space. Let $\{P_i\}_{i\in I}$ be an arbitrary subset of $\Projections(\mathcal{H})$ and let $K_i = \im(P_i)$ for all $i\in I$. Then, as a subset of $[0,\id]$,
\begin{enumerate}
\item $\inf \{P_i\}_{i\in I} = P_M$ where $M = \bigcap_{i\in I}K_i$;
\item $\sup \{P_i\}_{i\in I} = P_N$ where $N = \bigcap\setbuilder{K \subseteq \mathcal{H}}{\text{$K$ is closed} \land \forall i\in I: K_i \subseteq K}$.
\end{enumerate}
The set of projections on $\mathcal{H}$ is thus a complete lattice as a subset of $[0,\id]$.

If $I$ is finite, then $N = \Span(\bigcup_{i\in I}K_i)$. TODO: always closure of this $N$????
\end{proposition}
In particular this means $\Projections(\mathcal{H})$ is a complete lattice as itself, with the same suprema and infima. It is not a lattice as a subset of $\SelfAdjoints(\mathcal{H})$ (TODO + example ??).
\begin{proof}
(1) By \ref{subspaceProjections} $P_M$ is a lower bound of $\{P_i\}_{i\in I}$ in $[0,\id]$. Let $T$ be a lower bound of $\{P_i\}_{i\in I}$ in $[0,\id]$. By \ref{comparisonSelfAdjointProjection} $\im(T)\subseteq K_i$ for all $i\in I$, so $\im(T)\subseteq M$ and thus $T\leq P$ again by \ref{comparisonSelfAdjointProjection}. This means $P$ is the greatest lower bound.

(2) The mapping $T\mapsto \id-T$ keeps $[0,\id]$ invariant and inverts the order. Then $\inf \{\id - P_i\}_{i\in I}$ is a projection due to the previous point and so $\sup \{P_i\}_{i\in I}$ is also a projection. The expression for $N$ is clear from \ref{subspaceProjections}.
\end{proof}

\begin{proposition}
Let $P,Q$ be idempotents such that $\norm{P-Q}<1$. Then $\im(P) \cong \im(Q)$.
\end{proposition}
\begin{proof}
Kato p.34 TODO
\end{proof}

\subsubsection{Sets of pairwise disjoint projections}
TODO!

\subsubsection{Derivatives of orthogonal projections}



\begin{proposition}
Let $\{P_i\}_{i\in I}$ be a set of pairwise disjoint orthogonal projectors which have derivatives and take $i\neq j$ in $I$. Then
\begin{enumerate}
\item $P_i'P_j = - P_iP_j'$;
\item if $\id \in \upset \{P_i\}_{i\in I}$, then
\[ P_iP_i' = \sum_{j\neq i}P'_iP_j \qquad\text{and}\qquad P_i'P_i = \sum_{j\neq i}P_jP_i'. \]
\end{enumerate}
\end{proposition}
\begin{proof}
(1) We have $P_iP_j = 0$, so $0 = P_i'P_j + P_iP_j'$.

(2) We calculate, using $\id = \sum_{j\in I}P_j$ and \ref{derivativeIdempotent}:
\[ P_iP_i' = P_iP_i'\left(\sum_{j\in I}P_j\right) = P_iP_i'P_i + \sum_{j\neq i}P_iP_i'P_j = 0 - \sum_{j\neq i}P_iP_iP_j' = -\sum_{j\neq i}P_iP_j' = \sum_{j\neq i}P_i'P_j. \]
\end{proof}
\begin{corollary}
Let $P_1, P_2$ be orthogonal projections such that $P_1 + P_2 = \id$. Then
\[ P_1P_1'= P_1'P_2 \qquad \text{and}\qquad P_1'P_1 = P_2P_1'. \]
\end{corollary}


\subsection{Isometries}
We recall that isometries are injective and continuous. On Hilbert spaces they are also closed. See \ref{isometryLemma}, \ref{isometryLemma} and \ref{isometryLemma}.

\begin{proposition} \label{isometryCharacterisation}
Let $T\in \Bounded(H,K)$ with $H,K$ Hilbert spaces. Then
\begin{enumerate}
\item $T$ is an isometry \textup{if and only if} $T^*T = \id_H$;
\item $T$ is unitary \textup{if and only if} $T^*T = \id_H$ and $TT^* = \id_K$, i.e.\ $T^{-1} = T^*$.
\end{enumerate}
\end{proposition}
\begin{proof}
(1) For all $v,w\in H$ we have
\[ \inner{Tv,Tw} = \inner{T^*Tv,w}. \]
The left-hand side is equal to $\inner{v,w}$ iff $T$ is an isometry. The right-hand side is equal to $\inner{v,w}$ iff $T^*T = \id_H$, by \ref{equalityOfMapsInnerProductSpaces}.

(2) If $T$ is invertible, it must have a left and right inverse. By lemma \ref{leftRightInverse} they must be the same.
\end{proof}
\begin{corollary}
An isometry $T\in\Bounded(H)$ is unitary \textup{if and only if} it is normal.
\end{corollary}

\begin{lemma} \label{isometryRangeProjection}
Let $T$ be an isometry between Hilbert spaces $H$ and $K$. Then $TT^*$ is an orthogonal projection.
\end{lemma}
\begin{proof}
Clearly $(TT^*)^* = TT^*$. Also $(TT^*)^2 = T(T^*T)T^* = T\id_HT^* = TT^*$.
\end{proof}


\subsubsection{Wandering spaces and unilateral shifts}
\begin{definition}
Let $\mathcal{H}$ be a Hilbert space, $\mathcal{V}\subseteq \mathcal{H}$ a closed subspace and $T:\mathcal{H}\to \mathcal{H}$ a linear map. Then $\mathcal{V}$ is called a \udef{wandering space} for $T$ if $T^p[\mathcal{V}]\perp T^q[\mathcal{V}]$ for every $p\neq q\in\N$.
\end{definition}

\begin{lemma} \label{WoldLemma1}
Let $\mathcal{H}$ be a Hilbert space, $\mathcal{V}\subseteq \mathcal{H}$ a closed subspace and $T:\mathcal{H}\to \mathcal{H}$ a linear isometry.
\begin{enumerate}
\item $\mathcal{V}$ is a wandering space for $T$ \textup{if and only if} $T^n[\mathcal{V}]\perp \mathcal{V}$ for all $n\in\N$;
\item $T[\mathcal{H}]^\perp$ is a wandering subspace for $T$;
\item if $\mathcal{V}$ is a wandering space for $T$, then $T^n[\mathcal{V}] \cong \mathcal{V}$ for all $n\in N$.
\end{enumerate}
\end{lemma}
\begin{proof}
(1) The direction $\Rightarrow$ is clear. For the converse, assume $T^n[\mathcal{V}]\perp \mathcal{V}$ for all $n\in\N$. We need to show that $T^p[\mathcal{V}]\perp T^q[\mathcal{V}]$ for every $p\neq q\in\N$. WLOG we may assume $p\leq q$. Take arbitrary $x\in T^p[\mathcal{V}]$ and $y\in T^q[\mathcal{V}]$. Then
\[ \inner{x,y} = \inner{T^p(u), T^q(v)} = \inner{u, T^{q-p}(v)} = 0 \]
because $\mathcal{V} \perp T^{q-p}[\mathcal{V}]$.

(2) For all $n\geq 1$ we have
\[ T^{n}\big[T[\mathcal{H}]^\perp\big] \subset T^{n}[\mathcal{H}] = T\big[T^{n-1}[\mathcal{H}]\big] \subset T[\mathcal{H}] \perp T[\mathcal{H}]^\perp. \]

(3) For all $n\in \N$ the operator $T^n$ is an isometry. It is injective by \ref{isometryLemma}, and thus maps its domain bijectively to its image.
\end{proof}

\begin{definition}
An isometry $T$ on a Hilbert space $\mathcal{H}$ is called a \udef{unilateral shift} if there is a closed subspace $\mathcal{V}\subseteq \mathcal{H}$ that is wandering for $T$ such that
\[ \mathcal{H} = \bigoplus_{n=0}^\infty T^n[\mathcal{V}]. \]
We call the subspace $\mathcal{V}$ \udef{generating} for $T$ and $\dim(\mathcal{V})$ the \udef{multiplicity} of $T$.
\end{definition}

By \ref{WoldLemma1}, we see that any isometry $T:\mathcal{H}\to\mathcal{H}$ is a unilateral shift when restricted to $\bigoplus_{n=0}^\infty T^n\big[T[\mathcal{H}]^\perp\big]$.



\begin{lemma} \label{WoldLemma2}
Let $T$ be an isometry on $\mathcal{H}$. If $T$ is a unilateral shift, then it is generated by $T[\mathcal{H}]^\perp$.
\end{lemma}
\begin{proof}
Let $\mathcal{V}$ be the generating subspace of the unilateral shift $T$. We calculate
\[ T[\mathcal{H}] = T\left[\bigoplus_{n=0}^\infty T^n[\mathcal{V}]\right] = \bigoplus_{n=1}^\infty T^n[\mathcal{V}] = \bigoplus_{n=0}^\infty T^n[\mathcal{V}] \ominus \mathcal{V} = \mathcal{H}\ominus \mathcal{V} = \mathcal{V}^\perp, \]
so $\mathcal{V} = T[\mathcal{H}]^\perp$.
\end{proof}

A unilateral shift is determined up to unitary equivalence by its multiplicity:
\begin{lemma}
Let $T: \mathcal{H}\to\mathcal{H}$ and $T':\mathcal{H}'\to\mathcal{H}'$ be unilateral shifts generated by $\mathcal{V}$ and $\mathcal{V}'$ such that $\dim(\mathcal{V}) = \dim(\mathcal{V}')$. Then there exists an unitary $U:\mathcal{H}'\to\mathcal{H}$ such that
\[ T' = U^*TU \]
\end{lemma}
\begin{proof}
Choose an isometric isomorphism $u:\mathcal{V}'\to\mathcal{V}$. Then any $x\in\mathcal{H}'$ can be written as $x = \sum_{n=0}^\infty T^n(x_n)$. Then define
\[ Ux = \sum_{n=0}^\infty T^n(ux_n). \]
\end{proof}

\begin{theorem}[Wold decomposition]
Let $\mathcal{H}$ be a Hilbert space and $T\in\Bounded(\mathcal{H})$ an isometry. Then $\mathcal{H}$ decomposes into an orthogonal sum $\mathcal{H} = \mathcal{H}_0\oplus \mathcal{H}_1$such that $\mathcal{H}_0, \mathcal{H}_1$ reduce $T$ and
\[ T|_{\mathcal{H}_0}\;\text{is unitary} \quad\text{and}\quad T|_{\mathcal{H}_1}\;\text{is a unilateral shift}. \]
This decomposition is uniquely determined and given by
\[ \mathcal{H}_0 = \bigcap_{n=0}^\infty T^n[\mathcal{H}] \qquad\text{and}\qquad \mathcal{H}_1 = \bigoplus_{n=0}^\infty T^n[\mathcal{V}] \qquad\text{where}\qquad \mathcal{V} = T[\mathcal{H}]^\perp. \]
\end{theorem}
\begin{proof}
The subspace $\mathcal{V} = T[\mathcal{H}]^\perp$ is wandering by \ref{WoldLemma1}. Then $T$ is a unilateral shift in the subspace
\[ \mathcal{H}_1 = \bigoplus_{n=0}^\infty T^n[\mathcal{V}]. \]
Now $v\in\mathcal{H}_0 = \mathcal{H}_1^\perp$ if and only if it is perpendicular to $\bigoplus_{i=0}^n T^i[\mathcal{V}]$ for all $n$ and we have
\begin{align*}
\bigoplus_{i=0}^n T^i[\mathcal{V}] &= \bigoplus_{i=0}^n T^i[\mathcal{H}\ominus T[\mathcal{H}]] = \bigoplus_{i=0}^n T^i[\mathcal{H}]\ominus T^{i+1}[\mathcal{H}] \\
&= (\mathcal{H}\ominus T[\mathcal{H}])\oplus(T[\mathcal{H}]\ominus T^2[\mathcal{H}])\oplus \ldots \oplus (T^n[\mathcal{H}]\ominus T^{n+1}[\mathcal{H}])  = \mathcal{H} \ominus T^{n+1}[\mathcal{H}] 
\end{align*}
using \ref{perpUnderIsometry} and \ref{cancellationOminus}, which is applicable because $T^i[\mathcal{V}]$ is closed by \ref{isometryLemma}. So $\mathcal{H}_0\subseteq T^n[\mathcal{H}]$ for all $n$.

Finally $T|_{\mathcal{H}_0}$ is unitary because it is an isometry and surjective on $\mathcal{H}_0$.
\end{proof}

\subsubsection{Left and right shifts on $\ell^2$}
\begin{definition}
Consider the space $\ell^2(\N)$ with o.n. basis $\seq{e_i}$. Then
\begin{itemize}
\item the \udef{right shift operator} $S_r$ is the operator that maps $e_i \mapsto e_{i+1}$;
\item the \udef{left shift operator} $S_l$ is the operator that maps $e_i \mapsto \begin{cases}
e_{i-1} & i \geq 1 \\ 0 & i = 0
\end{cases}$.
\end{itemize}
\end{definition}

\begin{lemma}
$S_r$ is a unilateral shift
\end{lemma}

\begin{proposition}
$S_r = S^*_l$ (also converse?)
\end{proposition}

\subsubsection{Partial isometries}
\begin{definition}
An operator $T\in \Lin(H, H')$ is called a \udef{partial isometry} if there is a closed subspace $K\subseteq H$ such that
\begin{itemize}
\item $T|_K$ is an isometry;
\item $T|_{K^\perp} = 0$.
\end{itemize}
\end{definition}

Clearly every partial isometry is bounded.

\begin{lemma}
An operator $T\in \Lin(H, H')$ is a partial isometry \textup{if and only if} $T|_{\ker(T)^\perp}$ is an isometry.
\end{lemma}

\begin{proposition} \label{partialIsometryEquivalences}
Let $T\in \Bounded(H,H')$. The following are equivalent:
\begin{enumerate}
\item $T$ is a partial isometry;
\item $T^*TT^* = T^*$;
\item $TT^*T = T$;
\item $TT^*: H' \to H'$ is a projection;
\item $T^*T: H \to H$ is a projection;
\item $T^*$ is a partial isometry.
\end{enumerate}
Moreover,
\begin{enumerate}
\item $T^*T$ is the projection onto $\ker(T)^\perp$;
\item $\im(T)$ is closed and $TT^*$ is the projection onto $\im(T)$.
\end{enumerate}
\end{proposition}
\begin{proof}

$\boxed{(1)\Rightarrow (2)}$ By \ref{elementaryOrthogonality} it is enough to show that $\inner{T^*TT^*x,y} = \inner{T^*x,y}$ for all $x\in H', y\in H$. Take such $x,y$. We decompose $y = y_1\oplus y_2 \ker(T)\oplus \ker(T)^\perp$. Then
\[ \inner{T^*TT^*x, y_1} = \inner{TT^*x, Ty} = 0 = \inner{x,Ty_1} = \inner{T^*x, y_1} \]
and
\[ \inner{T^*TT^*x, y_2} = \inner{TT^*x, Ty_2} = \inner{T^*x,y_2}, \]
where we have used the fact that both $y_2$ and $T^*x$ are elements of $\ker(T)^\perp = \overline{\im(T^*)}$, and $T$ is an isometry on this space. In conclusion, we have
\[ \inner{T^*TT^*x,y} = \inner{T^*TT^*x,y_1} + \inner{T^*TT^*x,y_2} = \inner{T^*x,y_1} + \inner{T^*x,y_2} = \inner{T^*x,y} \]
for all $x\in H', y\in H$, so $T^*TT^* = T^*$.

$\boxed{(2) \Leftrightarrow (3)}$ By taking adjoints: $(TT^*T)^* = T^*TT^*$.

$\boxed{(2) \Rightarrow (4,5)}$ Clearly $T^*T$ and $TT^*$ are self-adjoint. We just need to show idempotency:
\[ (T^*T)^2 = (T^*T)(T^*T) = (T^*TT^*)T = T^*T \qquad (TT^*)^2 = (TT^*)(TT^*) = T(T^*TT^*) = TT^*. \]

$\boxed{(4) \Rightarrow (1)}$ Assume $TT^*$ a projection. Let $v\in \ker(T)^\perp = \overline{\im(T^*)}$. Then there exists a sequence $\seq{v_n}\in H^{\prime\N}$ such that $\lim_{n\to\infty}T^*v_n = v$. Then
\begin{align*}
\norm{Tv}^2 &= \lim_{n\to\infty}\norm{TT^*v_n}^2 = \lim_{n\to\infty}\inner{TT^*v_n,TT^*v_n} \\
&= \lim_{n\to\infty}\inner{(TT^*)^2v_n,v_n} = \lim_{n\to\infty}\inner{TT^*v_n,v_n} \\
&= \lim_{n\to\infty}\inner{T^*v_n,T^*v_n} = \lim_{n\to\infty}\norm{T^*v_n}^2 = \norm{v}^2,
\end{align*}
so $T$ is a partial isometry.

$\boxed{(5,6)}$ Applying the proposition to $T^*$ instead of $T$ yields the equivalences with $T=TT^*T$, and thus with the rest of the statements.

TODO + $\im(T^*) = \ker(T)^\perp$ means support and range are exchanged between $T$ and $T^*$.
\end{proof}

\begin{definition}
Let $T$ be a partial isometry. We call
\begin{itemize}
\item $T^*T$ the \udef{support projection} or \udef{initial projection} of $T$;
\item $TT^*$ the \udef{range projection} or \udef{final projection} of $T$.
\end{itemize}
\end{definition}

\begin{proposition}
Let $H,H'$ be Hilbert spaces with $K\subseteq H$ and $L\subseteq H'$ closed subspaces. Then the following are equivalent:
\begin{enumerate}
\item $T$ is a partial isometry with support $K$ and range $L$;
\item $(T,T^*)$ is a Galois connection between $\sSet{H, \perp_K}$ and $\sSet{H', \perp_L}$.
\end{enumerate}
Here $\perp_K$ is defined by
\[ x \perp_K y \quad\defequiv\quad P_K(x)\perp P_{K}(y). \]
\end{proposition}
\begin{proof}
The direction $(2) \Rightarrow (1)$ is immediate from \ref{partialIsometryEquivalences}, because $T,T^*$ are generalised inverses.

For the other direction, we first prove $T$ preserves the relational structure. Take arbitrary $x= x_1+x_2$ and $y=y_1+y_2$ in $K\oplus K^\perp$ such that $x\perp_K y$. Then
\[ \inner{T(x), T(y)} = \inner{T(x_1), T(y_1)} = \inner{x_1, y_1} = 0. \]
So $T(x)\perp T(y)$ and, because $T(x), T(y) \in L$, we have $T(x)\perp_L T(y)$. The argument for $T^*$ is similar.

For the Galois condition, we need to show that $T^*T(x)\perp_K y \implies x\perp_K y$. Indeed
\begin{align*}
T^*T(x)\perp_K y &\iff T^*T(x_1)\perp y_1 \\
&\iff 0= \inner{T^*T(x_1), y_1} = \inner{T(x_1), T(y_1)} = \inner{x_1,y_1} \\
&\iff P_K(x)\perp P_K(y).
\end{align*}
\end{proof}
\begin{corollary}
Let $T: H\to H'$ be a partial isometry with support $K$ and range $L$. Then
\[ T(x) \perp P_L(y) \iff P_K(x) \perp T^*(y) \]
for all $x\in H, y\in H'$.
\end{corollary}
\begin{proof}
This is the Galois identity \ref{GaloisIdentity}, although the direct proof is also very simple.
\end{proof}

\subsubsection{Unitaries}
\paragraph{Bilateral shifts}


\section{Dirac notation}
\url{https://core.ac.uk/download/pdf/25263496.pdf}
\url{https://michael-herbst.com/talks/2014.07.22_Mathematical_Concept_Dirac_Notation.pdf}
\url{http://galaxy.cs.lamar.edu/~rafaelm/webdis.pdf}
\url{https://plato.stanford.edu/entries/qt-nvd/}
\url{file:///C:/Users/user/Downloads/Abdus%20Salam,%20E.P.%20Wigner%20(Ed.)%20-%20Aspects%20of%20Quantum%20Theory%20-%20Dedicated%20to%20Dirac%E2%80%99s%2070th%20Birthday-Cambridge%20University%20Press%20(1972).pdf}
\url{https://aip.scitation.org/doi/pdf/10.1063/1.1705001}

\begin{lemma}
\begin{enumerate}
\item $T\ketbra{\varphi}{\psi} = \ketbra{T\varphi}{\psi} = \ketbra{\varphi}{\psi}T = \ketbra{\varphi}{T^*\psi}$;
\item $\ketbra{\varphi}{\psi}\ketbra{\xi}{\eta} = \inner{\psi, \xi}\ketbra{\varphi}{\eta}$;
\item $(\ketbra{\varphi}{\psi})^* = \ketbra{\psi}{\varphi}$.
\end{enumerate}
\end{lemma}

\begin{lemma}
Let $H$ be a Hilbert space and $\seq{e_i}_{i\in I}$ a basis for $H$. Then
\[ \id_H = \sum_{i\in I}\ketbra{e_i}{e_i} \qquad\text{in the strong limit.} \]
\end{lemma}
\begin{proof}
TODO!!
\end{proof}
\begin{lemma} \label{operatorBraketExpansion}
Let $H$ be a Hilbert space, $\seq{e_i}_{i\in I}$ a basis for $H$ and $T$ an operator on $H$. Then
\[ T = \sum_{i,j\in I}\braket[T]{e_i}{e_j}\; \ketbra{e_i}{e_j}. \]
in the strong limit.
\end{lemma}
\begin{proof}
TODO!! Tannery.
\end{proof}

\section{Ideals of operators on a Hilbert space}

\subsection{Finite-rank operators}
Remember that finite-rank operators are bounded by definition (this is not automatic, cfr. \ref{continuousMapCriterion}).

\begin{proposition}[Finite rank singular value decomposition] \label{finiteRankSingularValues}
Let $V$ be an inner product space and $T\in\Hom(V)$. Then $T$ is a finite-rank operator \textup{if and only if} $T$ can be written in the form
\[ T = \sum_{i=1}^N \lambda_i \ketbra{v_i}{w_i}, \]
where $(v_i)_{i=1}^N$ and $(w_i)_{i=1}^N$ are finite sets of vectors and $(\lambda_i)_{i=1}^N$ are positive (non-zero) numbers.

The numbers $(\lambda_i)_{i=1}^N$ in this decomposition are uniquely determined by the operator.
\end{proposition}
The numbers $(\lambda_i)_{i=1}^N$ are called the \udef{singular values} of the operator.
\begin{proof}
Because $\im(T)$ is finite-dimensional, we can find an orthonormal basis $(v_i)_{i=1}^N$ for it. Then we can write
\begin{align*}
Tx &= \sum_{i=1}^N \ket{v_i}\braket{v_i}{Tx} = \sum_{i=1}^N \ket{v_i}\braket{T^*v_i}{Tx} = \sum_{i=1}^N \ket{v_i}\braket{\lambda_i w_i}{Tx}  = \sum_{i=1}^N \lambda_i\ket{v_i}\braket{w_i}{Tx}
\end{align*}
where $\lambda_i = \norm{T^*v_i}$ and $w_i = \frac{T^*v_i}{\lambda_i}$.

We just need to show that the $\lambda_i$ are independent of the chosen basis $(v_i)_{i=1}^N$. TODO!!!!
\end{proof}
\begin{corollary}
Every finite-rank operator on a Hilbert space is a finite sum of rank-1 operators.
\end{corollary}

\begin{lemma}
Let $H$ be Hilbert space. The set of finite rank operators on $H$ is a $*$-ideal in $H$.
\end{lemma}

\subsection{Compact operators}

\url{https://math.stackexchange.com/questions/4198074/space-of-compact-operators-is-the-only-proper-closed-two-sided-ideal-of-the-spac}

\begin{lemma}
Let $K\in\Compact(H)$ be a compact operator. For all $\lambda\in\spec(K)\setminus\{0\}$, the eigenspace $E_\lambda$ is finite-dimensional.
\end{lemma}
Compare with \ref{spectrumCompactOperator}. Note that each such $\lambda$ is indeed an eigenvalue, by \ref{pointSpectrumCompactOperatorBanachSpace}.
\begin{proof}
Suppose, towards a contradiction, $E_\lambda$ is infinite dimensional, then we can find a sequence $\seq{x_n}$ of orthonormal vectors in $E_\lambda$. Since $\norm{x_n-x_m} = \sqrt{\norm{x_n}^2+\norm{x_m}^2} = \sqrt{2}$ for all $n,m\in\N$, by the Pythagorean theorem \ref{Pythagoras}, $\seq{x_n}$ has not Cauchy subsequence and thus also no convergent subsequence. This implies that $\seq{Kx_n} = \lambda\seq{x_n}$ also has no convergent subsequence, which implies that $K$ is not compact by \ref{compactOperatorEquivalents}.
\end{proof}

\begin{proposition}
Let $H_1,H_2$ be Hilbert spaces and $T\in \Bounded(H_1,H_2)$. If $T$ is bounded, then $T^*$ is bounded.
\end{proposition}
\begin{proof}
Let $\seq{x_n}$ be a bounded sequence in $H_2$ with $\norm{x_n}\leq M\in \R$ for all $n\in\N$. Since $T^*$ is bounded, we have that $\seq{T^*x_n}$ is also bounded and thus $\seq{TT^*x_n}$ has a convergent subsequence $\seq{TT^*x_{n_k}}_{k\in \N}$ by \ref{compactOperatorEquivalents}. Then, using the Cauchy-Schwarz inequality \ref{CauchySchwarz}, we calculate
\begin{align*}
\norm{T^*x_{n_k} - T^*x_{n_l}}^2 &= \inner{TT^*(x_{n_k} - x_{n_l}), x_{n_k} - x_{n_l}} \\
&\leq \norm{TT^*(x_{n_k} - x_{n_l})}\;\norm{x_{n_k} - x_{n_l}} \\
&\leq \norm{TT^*(x_{n_k} - x_{n_l})}\;\big(\norm{x_{n_k}} +\norm{x_{n_l}}\big) \\
&\leq 2M\norm{TT^*(x_{n_k} - x_{n_l})} \to 0.
\end{align*}
Thus $\seq{T^*x_{n_k}}_{k\in\N}$ is a Cauchy sequence in $H_1$, which means it converges. This means that $\seq{T^*x_n}_{n\in\N}$ has a convergent subsequence and thus that $T^*$ is compact by \ref{compactOperatorEquivalents}.
\end{proof}


\begin{proposition}
Let $T\in\Bounded(H)$. Then the following are equivalent:
\begin{enumerate}
\item $T$ is compact;
\item there exists a sequence $(T_n)_{n\in\N}$ of finite rank operators such that $\norm{T-T_n}\to 0$.
\end{enumerate}
\end{proposition}
This is false in Banach spaces. (TODO Enflo, approximation property, goose problem)
\begin{proof}
\ref{SchaudersTheorem}
\end{proof}
\begin{corollary}[Canonical expansion]
Any compact operator $T$ on a Hilbert space $\mathcal{H}$ can be written in the form
\[ T = \sum_{i=1}^\infty \lambda_i \ketbra{v_i}{w_i}, \]
where $(v_i)_{i=1}^\infty$ and $(w_i)_{i=1}^\infty$ are orthonormal sets and $(\lambda_i)_{i=1}^\infty$ is a monotonically decreasing sequence of positive numbers with $\lim_{i\to\infty}\lambda_i = 0$.
\end{corollary}
As in \ref{finiteRankSingularValues} for finite-rank operators we call $(\lambda_i)_{i=1}^\infty$ the \udef{singular values} of $T$. They are uniquely determined by the operator.
\begin{proof}
TODO (one way is with polar decomposition and spectral theorem. Are there others?)
\end{proof}
Compare with \ref{operatorBraketExpansion}.

\begin{lemma}
Let $H$ be a Hilbert space. Then the set of compact operators on $H$, $\Compact(H)$ is a two-sided $*$-ideal of $H$. 
\end{lemma}

\begin{proposition}
Let $H$ be a Hilbert space with orthonormal basis $(e_i)_{i\in I}$. If $T\in\Bounded(H)$ and
\[ \sum_{i\in I}\norm{Te_i}^2  < \infty, \]
then $T$ is a compact operator. + Converse??
\end{proposition}
\begin{proof}
TODO + weaken $T\in\Bounded(H)$?
\end{proof}
\begin{corollary}
An integral operator defined by a square integrable kernel $K\in L^2(A\times A, \mu)$ is compact.
\end{corollary}

\begin{proposition}
Let $T$ be an operator on a Hilbert space. Then the following are equivalent:
\begin{enumerate}
\item $T$ is compact;
\item for all sequences $\seq{x_n}$, weak convergence $x_n \overset{w}{\to} x$ implies the strong convergence $Ax_n \to Ax$;
\item for any two weakly convergent sequences $x_n\overset{w}{\to} x$ and $y_n\overset{w}{\to} y$ the energy form is continuous in both arguments:
\[ \lim_{n\to\infty}\inner{x_n,y_n}_T = \lim_{n\to\infty}\inner{x_n,Ty_n} = \inner{x,Ty} = \inner{x,y}_T. \]
\end{enumerate} 
\end{proposition}

\begin{lemma}
Let $H$ be a Hilbert space and $P\in\Projections(H)$. If $P$ is compact, then $P$ has finite rank.
\end{lemma}

\subsection{Positive operators}

\subsubsection{Polar decomposition}
\begin{proposition}
Let $H$ be a Hilbert space and $T$ a closed operator on $H$. There exists a unique pair of operators $V,|T|$ on $H$ such that
\begin{enumerate}
\item $T = V|T|$
\item $|T|$ is positive self-adjoint and $V$ is a partial isometry;
\item $\dom(T) = \dom(|T|)$, $\ker(T) = \ker(|T|)$;
\item the support projection of $V$ is $\ker(T)^\perp$ and range is $\overline{\im(T)}$.
\end{enumerate}
\end{proposition}
The unique $|T|$ is given by $\sqrt{T^*T}$.
\begin{proof}
By \ref{vonNeumannTheoremSquareSelfAdjoint}, we have that $T^*T$ is self-adjoint. By \ref{spectralTheoremFunctionalCalculus} we have $\spec(\sqrt{T^*T}) \subseteq \interval[co]{0,+\infty}$ and $\sqrt{T^*T}^2 = T^*T$. By \ref{realSpectrumSelfAdjoint} we have that $\sqrt{T^*T}$ is self-adjoint.

Now, for all $x\in $

\ref{projectorIntegrableSolid}
\end{proof}

\subsubsection{Square root}
\begin{proposition} \label{squareRootUnboundedOperator}
Let $H$ be a Hilbert space and $A$ a positive operator on $H$. Then there exists a unique self-adjoint operator $\sqrt{A}$ on $H$ with $\dom(\sqrt{A}) = \dom(A)$ and $\sqrt{A}^2 = A$.
\end{proposition}


\subsubsection{Trace-class operators}
TODO: move away from ideal heading.

\begin{definition}
Let $H$ be a Hilbert space with orthonormal basis $\seq{e_i}_{i\in I}$. Let $T\in\Bounded(H)$ be a bounded positive operator. We define the \udef{trace} of $T$ as
\[ \Tr(T) \defeq \sum_{i\in I}\inner{e_i, Te_i}. \]
If $\Tr(T) < \infty$, we say $T$ is \udef{trace-class}.
\end{definition}
Since $T$ is positive, the trace is well-defined.

\begin{lemma}
The trace of an operator does not depend on the choice of orthonormal basis.
\end{lemma}
\begin{proof}
Let $H$ be a Hilbert space and $\seq{e_i}_{i\in I}, \seq{f_i}_{i\in I}$ two orthonormal bases for $H$. Let $T$ be a bounded positive operator on $H$. Then $T$ has a square root $\sqrt{T}$ by (TODO ref) and we can calculate, using the Parseval identity \ref{totalONBParsevalEquivalence},
\begin{align*}
\sum_{i\in I}\inner{e_i, Te_i} &= \sum_{i\in I}\inner{\sqrt{T}e_i, \sqrt{T}e_i} \\
&= \sum_{i\in I}\norm{\sqrt{T}e_i}^2 \\
&= \sum_{i\in I}\sum_{j\in I}\left|\inner{f_j, \sqrt{T}e_i}\right|^2 \\
&= \sup_{F\subseteq I \text{finite}}\sup_{F'\subseteq I \text{finite}}\sum_{i\in F}\sum_{j\in F'}\left|\inner{f_j, \sqrt{T}e_i}\right|^2 \\
&= \sup_{F'\subseteq I \text{finite}}\sup_{F\subseteq I \text{finite}}\sum_{j\in F'}\sum_{i\in F}\left|\inner{f_j, \sqrt{T}e_i}\right|^2 \\
&= \sum_{j\in I}\sum_{i\in I}\left|\inner{\sqrt{T}f_j, e_i}\right|^2 \\
&= \sum_{j\in I}\norm{\sqrt{T}f_i}^2 \\
&= \sum_{i\in I}\inner{\sqrt{T}f_i, \sqrt{T}f_i} \\
&= \sum_{i\in I}\inner{e_i, Te_i}.
\end{align*} 
\end{proof}


\begin{proposition} \label{traceCommutatorCompactSA}
Let $H$ be a Hilbert space and $A,B\in\Lin(X)$ such that $B$ is compact self-adjoint, then $\Tr[A,B] = 0$.
\end{proposition}
\begin{proof}
Let $\seq{e_n}$ be an orthonormal basis of eigenvectors of $B$, with corresponding real eigenvalues $\lambda_n$. This exists by the spectral theorem (TODO ref). Then
\begin{align*}
\Tr[A,B] &= \sum_n\inner{e_n, [A,B]e_n} \\
&= \sum_n\inner{e_n, ABe_n} - \inner{e_n, BAe_n} \\
&= \sum_n\inner{e_n, ABe_n} - \inner{Be_n, Ae_n} \\
&= \sum_n\lambda_n\inner{e_n, Ae_n} - \lambda_n\inner{e_n, Ae_n} \\
&= 0.
\end{align*}
\end{proof}
\begin{corollary}
Let $H$ be a Hilbert space and $A,B\in\Lin(X)$ such that $A$ is self-adjoint and $B$ compact. If $[A,B]$ is trace-class, then $\Tr[A,B] = 0$.
\end{corollary}
\begin{proof}
If $[A,B]$ is trace-class, then $-[A,B]^* = [A,B^*]$ is also 
\end{proof}


\section{Dilation theory}
\subsection{Dilations, $N$-dilations and power dilations}
\begin{definition}
Let $\mathcal{H} \subseteq \mathcal{H}'$ be Hilbert spaces and let $P_\mathcal{H}$ be the projection on $\mathcal{H}$. If a pair of linear maps $S: \mathcal{H}'\to\mathcal{H}'$ and $T: \mathcal{H}\to \mathcal{H}$ satisfy the relation
\[ T = P_\mathcal{H} S |_\mathcal{H} \]
then $T$ is called a \udef{compression} of $S$ and $S$ a \udef{dilation} of $T$. This is abbreviated $T\prec U$.

\begin{itemize}
\item Let $N\in\N$. If $T^k = P_\mathcal{H} S^k |_\mathcal{H}$ for all $k\leq N$, then $S$ is called an \udef{$N$-dilation}.
\item If this holds for all $k\in\N$, then $S$ is called a \udef{power dilation}.
\item If $T^* = P_\mathcal{H} S^* |_\mathcal{H}$, we call TODO??
\end{itemize}
We call $\mathcal{H}'$ \udef{minimal} if the only reducing subspace for $S$ that contains $\mathcal{H}$ is $\mathcal{H}'$.
\end{definition}

If $S$ is a dilation of $T$, then we clearly have $T = P_\mathcal{H} S P_\mathcal{H}|_\mathcal{H}$.

\begin{lemma}
Let $S:\mathcal{H}'\to\mathcal{H}'$ be an $N$-dilation of $T: \mathcal{H}\to \mathcal{H}$ and $p$ a polynomial of degree at most $N$. Then
\[ p(T) = P_\mathcal{H}p(S)|_\mathcal{H}. \]
\end{lemma}

Let $\mathcal{H}$ be a Hilbert space. We call $T\in\Bounded(\mathcal{H})$ a \udef{contraction} if $\norm{T}\leq 1$.
\begin{proposition} \label{dilationOfContraction}
Let $\mathcal{H} \cong \mathcal{H}\oplus \{0\} \subseteq \mathcal{H}\oplus \mathcal{H} = \mathcal{H}^2$ be a Hilbert space. Every contraction $T$ on $\mathcal{H}$ has a unitary dilation $U$ on $\mathcal{H}^2$.
\end{proposition}
\begin{proof}
From $\norm{T}\leq 1$ (and the fact that $T^*T$ is normal), we have that $\vec{1}-T^*T\geq 0$ by spectral mapping. We can define $D_T = \sqrt{\vec{1}-T^*T}$. Then
\[ U = \begin{pmatrix}
T & D_{T^*} \\ D_T & -T^*
\end{pmatrix} \]
is a dilation of $T$ and it is unitary:
\begin{align*}
UU^* &= \begin{pmatrix}
TT^* + D_{T^*}^2 & TD_T^* - D_{T^*}T \\
D_TT^* - T^*D_{T^*}^* & D^2_{T} + T^*T
\end{pmatrix} = \begin{pmatrix}
\vec{1} & TD_T - D_{T^*}T \\
D_TT^* - T^*D_{T^*} & \vec{1}
\end{pmatrix} \\
U^*U &= \begin{pmatrix}
T^*T + D_{T}^2 & T^*D_{T^*} - D_{T}^*T^* \\
D_{T^*}^*T - TD_{T} & D^2_{T^*} + TT^*
\end{pmatrix} = \begin{pmatrix}
\vec{1} & T^*D_{T^*} - D_{T}T^* \\
D_{T^*}T - TD_{T} & \vec{1}.
\end{pmatrix}
\end{align*}
We have used that $D_T$ is self-adjoint for all contractions $T$. We just need to show that $TD_T = D_{T^*}T$. Clearly we have
\[ T(D_T)^2 = T(\vec{1} - T^*T) = T - TT^*T = (\vec{1} - TT^*)T = (D_{T^*}T)^2T. \]
By functional-like calculus (TODO!!) we have $TD_T = D_{T^*}T$.
\end{proof}
The operator $D_T$ in the previous proof is sometimes called the \udef{defect operator} of $T$. It measures in some sense how far $T$ is from being a unitary operator. If $T$ is unitary, then $D_T = 0 = D_{T^*}$. If $T$ is an isometry, then $D_T = 0$ (by \ref{isometryRangeProjection}) and $D_{T^*}$ is a projector ($TT^*$ is a projector by \ref{isometryCharacterisation}, so $\vec{1} - TT^*$ is too by \ref{projectorOrthogonalComplement} and $D_{T^*} = \sqrt{\vec{1}-TT^*} = \sqrt{(\vec{1}-TT^*)^2} = \vec{1}-TT^*$).

\begin{proposition}
Let $\mathcal{H} \cong \mathcal{H}\oplus \{0\} \subseteq \mathcal{H}\oplus \mathcal{H} = \mathcal{H}^2$ be a Hilbert space. Every isometry $T$ on $\mathcal{H}$ has a unitary power dilation $U$ on $\mathcal{H}^2$.
\end{proposition}
\begin{proof}
Consider the unitary dilation of \ref{dilationOfContraction}. When $T$ is an isometry this reduces to
\[ U = \begin{pmatrix}
T & D_{T^*} \\ 0 & -T^*
\end{pmatrix} = \begin{pmatrix}
T & \vec{1}-TT^* \\ 0 & -T^*
\end{pmatrix}, \]
where we have used that $D_{T^*} = \sqrt{\vec{1}-TT^*} = \sqrt{(\vec{1}-TT^*)^2} = \vec{1}-TT^*$ is a projector.

Now for all $n\in\N$ we have $U^n = \begin{pmatrix}
T^n & * \\ 0 & (-T^*)^n
\end{pmatrix}$, so in particular $P_\mathcal{H}U^n|_\mathcal{H} = T^n$, meaning $U$ is a power dilation of $T$. 
\end{proof}

\begin{lemma}
Let $T$ a contraction on a Hilbert space $\mathcal{H}$. Then $V_T: \mathcal{H} \to \mathcal{H}\oplus\mathcal{H}: x\mapsto (Tx, D_Tx)$ is an isometry.
\end{lemma}
\begin{proof}
For all $x\in \mathcal{H}$ we have
\[ \norm{V_Tx} = \sqrt{\norm{Tx}^2 + \norm{D_Tx}^2} = \sqrt{\inner{Tx,Tx} + \inner{D_Tx,D_Tx}} = \sqrt{\inner{T^*Tx,x} + \inner{D_T^2x,x}} = \sqrt{\inner{x,x}} = \norm{x}. \]
\end{proof}

\begin{proposition}
Let $\mathcal{H} \cong \mathcal{H}\oplus \{0\}^N \subseteq \mathcal{H}^{N+1}$ be a Hilbert space. Every contraction $T$ on $\mathcal{H}$ has a unitary $N$-dilation $U$ on $\mathcal{H}^{N+1}$.
\end{proposition}
\begin{proof}
Let $U'$ be a unitary dilation of $T$ on $\mathcal{H}^2$. Let $C_1 = U'_{-,1}$ and $C_2 = U'_{-,2}$ denote the columns. Then
\[ U = \begin{pmatrix}
C_1 & \mathbb{0}^{2\times N-1} & C_2 \\
\mathbb{0}^{N-1\times 1} & \mathbb{1}^{N-1\times N-1} & \mathbb{0}^{N-1\times 1}
\end{pmatrix} \]
is unitary by
\[ \begin{pmatrix}
C_1^* & \mathbb{0} \\
\mathbb{0} & \mathbb{1} \\
C_2^* & \mathbb{0}
\end{pmatrix}\begin{pmatrix}
C_1 & \mathbb{0} & C_2 \\
\mathbb{0} & \mathbb{1} & \mathbb{0}
\end{pmatrix} = \begin{pmatrix}
C_1^*C_1 & \mathbb{0} & C_1^*C_2 \\
\mathbb{0} & \mathbb{1} & \mathbb{0} \\
C_2^*C_1 & \mathbb{0} & C_2^*C_2
\end{pmatrix} = \mathbb{1}^{N+1\times N+1}. \]
We just need to show that the (1,1)-component of $U^k$ is $T^k$ for all $k\in 1:N$. In order to perform the multiplication, we rewrite $U$ such that the row and column partitions are the same, i.e.\ $(2|(N-3)|2)\times (2|(N-3)|2)$:
\[ U = \begin{pmatrix}
\begin{bmatrix}
T & 0 \\ D_T & 0
\end{bmatrix} & \mathbb{0} & \begin{bmatrix}
0 & D_{T^*} \\ 0 & -T^*
\end{bmatrix} \\
\begin{bmatrix}
0 & 1 \\ \mathbb{0} & \mathbb{0}
\end{bmatrix} & \begin{bmatrix}
\mathbb{0} & 0 \\ \mathbb{1} & \mathbb{0}
\end{bmatrix} & \mathbb{0} \\
\begin{bmatrix}
0 & 0 \\ 0 & 0
\end{bmatrix} & \begin{bmatrix}
\mathbb{0} & 1 \\ \mathbb{0} & 0
\end{bmatrix} & \begin{bmatrix}
0 & 0 \\ 1 & 0
\end{bmatrix}
\end{pmatrix} \]
TODO
\end{proof}

\begin{proposition}[von Neumann's inequality]
Let $T$ be a contraction on some Hilbert space $\mathcal{H}$. Then, for every polynomial $p\in\C[z]$,
\[ \norm{p(T)}\leq \sup_{|z|=1}|p(z)|. \]
\end{proposition}
\begin{proof}
Suppose the degree of $p$ is $N$. Let $U$ be a unitary $N$-dilation of $T$. Then
\[ \norm{p(T)} = \norm{P_\mathcal{H}p(U)|_\mathcal{H}}\leq \norm{p(U)} = \sup_{z\in\sigma(U)}|p(z)| \leq \sup_{|z|=1}|p(z)| \]
since the spectrum of $U$ is contained in the unit circle.
\end{proof}

\begin{theorem}[Sz.-Nagy's dilation theorem]
Let $\mathcal{H} \subseteq \ell^2(\N)\otimes\mathcal{H}$ be Hilbert spaces. Every contraction on $\mathcal{H}$ has a unitary power dilation on $\ell^2(\N)\otimes\mathcal{H}$.
\end{theorem}




\section{Constructions}
\subsection{Direct sum}

Let $(V_i)_{i\in I}$ be a family of Hilbert spaces. By considering them as Banach spaces we can take the $\ell^2$-direct sum. (TODO: other sequence spaces?)
\begin{proposition}
Let $(V_i)_{i\in I}$ be a family of Hilbert spaces. The $\ell^2$-direct sum is a Hilbert space.
\end{proposition}
This gives the conventional interpretation of the \udef{Hilbert space direct sum}: it is the $\ell^2$-direct sum of the summands as Banach spaces.


\subsection{Tensor product}
\url{https://web.ma.utexas.edu/mp_arc/c/14/14-2.pdf}

TODO: tensor product of operators defined on algebraic tensor product of their domains!

\begin{proposition}
If $S,T$ are closable operators, then $S\otimes_a T$ is closable.
\end{proposition}
\begin{proof}
TODO
\end{proof}

For closable operators, we define $S\otimes T \defeq \overline{S\otimes_a T}$.

\begin{proposition}
For bounded operators $S,T$, we have $\norm{S\otimes T} = \norm{S}\,\norm{T}$.
\end{proposition}

\section{Spectral properties of operators on Hilbert spaces}
\begin{lemma}
Let $T \in \Bounded(H)$ for some Hilbert space $H$. Then
$\rho(T) = \overline{\rho(T^*)}$, where the bar denotes complex conjugation.
\end{lemma}
\begin{proof}
TODO?? Take $\lambda\in\rho(T)$. Then
\[ ((\lambda\id - A)^{-1})^* = (\overline{\lambda}\id - A^*)^{-1} \]
so $\overline{\lambda}\in\rho(T^*)$ iff $((\lambda\id - A)^{-1})^*$ is bounded iff $(\lambda\id - A)^{-1}$ is bounded iff $\lambda\in \rho(T)$.
\end{proof}

\begin{lemma} \label{eigenspaceOrthogonalAdjoint}
Let $L$ be a densely defined operator on a Hilbert space $H$. Take $\lambda\in \pspec(L)$ and $\mu\in\pspec(L^*)$. If $\lambda \neq \overline{\mu}$, then
\[ \ker(\lambda\id - L)\perp \ker(\mu\id - L^*). \]
\end{lemma}
\begin{proof}
Take non-zero eigenvectors $x,y$ such that $Ax = \lambda x$ and $A^*y = \mu y$. Then
\[ \lambda \inner{y,x} = \inner{y,\lambda x} = \inner{y, Ax} = \inner{A^*y,x} = \inner{\mu y,x} = \overline{\mu}\inner{y,x}. \]
So we have $(\lambda - \overline{\mu})\inner{y,x} = 0$.
\end{proof}

\begin{proposition} \label{adjointSpectrumNoResidual}
Let $L$ be a densely defined operator on a Hilbert space $H$. Then the following are equivalent:
\begin{enumerate}
\item the residual spectrum of $L$ is empty;
\item $\overline{\pspec(L^*)} \subseteq \pspec(L)$;
\end{enumerate}
as are the following:
\begin{enumerate}
\item the residual spectrum of $L^*$ is empty;
\item $\pspec(L) \subseteq \overline{\pspec(L^*)}$.
\end{enumerate}
In particular all these statements hold if $L$ is normal.
\end{proposition}
\begin{proof}
Consider, for all $x\in \dom(L), y\in\dom(L^*)$, the equality
\[ \inner{(\lambda\id-L)x,y} = \inner{x,(\overline{\lambda}\id-L^*)y}. \]
We can make the following inferences:
\begin{itemize}
\item If $\lambda\in \overline{\pspec(L^*)}$, then the equality holds in particular for all eigenvectors $y$. This implies $\inner{(\lambda\id-L)x,y} = 0$. By \ref{perpToDenseSet} $\im(\lambda\id-L)$ may then not be dense, so it cannot be injective because the residual spectrum of $L$ is empty.
\item Assume $\lambda\id-L$ injective and take  $y\perp \im(\lambda\id-L)$. Then by the equality $\inner{x, (\overline{\lambda}\id - L^*)y} = 0$ for all $x\in\dom(L)$, which is dense. So $(\overline{\lambda}\id - L^*)y = 0$ by \ref{perpToDenseSet}. Now $\lambda\notin \pspec(L)$, so $\overline{\lambda}\notin \pspec(L^*)$. Thus $y = 0$ and $\im(\lambda\id-L)^\perp = \{0\}$, meaning $\im(\lambda\id-L)$ is dense.
\end{itemize}
The arguments for the second set of statements are similar.

If $L$ is normal, then $\ker(\lambda \id - L) = \ker{\overline{\lambda}\id -L^*}$ by \ref{equalityKernelAdjointNormal}, so $\pspec(L) = \overline{\pspec(L^*)}$.
\end{proof}

\begin{proposition}
Let $T$ be a closed, densly defined operator on a Hilbert space.
\begin{enumerate}
\item If $\lambda\in\rho(T)$, then $\overline{\lambda}\in\rho(T^*)$.
\item If $\lambda\in\rspec(T)$, then $\overline{\lambda}\in\pspec(T^*)$.
\item If $\lambda\in\pspec(T)$, then $\overline{\lambda}\in\rspec(T^*)\cup\pspec(T^*)$.
\end{enumerate}
\end{proposition}
\begin{proof}
TODO Compare with \ref{adjointSpectrumNoResidual}. CLosure necessary?
\end{proof}


\begin{proposition}
Let $T$ be a unitary operator. Then
\begin{enumerate}
\item $\rspec(T) = \emptyset$;
\item $\spec(T) \subset \setbuilder{\lambda\in\C}{|\lambda| = 1}$.
\end{enumerate}
\end{proposition}
TODO: move to more general place??

\begin{lemma}
The eigenvalues of a bounded dissipative linear operator
lie in the half-plane $\Im\lambda \geq 0$.
\end{lemma}


\subsubsection{Residual spectrum}
\begin{proposition}
Let $L$ be a densely defined linear operator on a Hilbert space. If $\lambda$ is in the residual spectrum of $L$ with deficiency $m$, then $\overline{\lambda}$ is in the point spectrum of $L^*$ with multiplicity $m$.
\end{proposition}
\begin{proof}
By \ref{kernelImageAdjoint} we have
\[ \im(\lambda \id - L)^\perp = \ker(\lambda\id - L)^* = \ker(\overline{\lambda}\id - L^*). \]
\end{proof}

\subsection{Rayleigh quotient}
\begin{lemma}
Let $L$ be an operator on a Hilbert space. If $x$ is an eigenvector with eigenvalue $\lambda$, then
\[ J_L(x) = \lambda. \]
\end{lemma}
\begin{proof}
Let $x$ be an eigenvector with eigenvalue $\lambda$, then
\[ J_L(x) = \frac{\inner{x,Lx}}{\inner{x,x}} = \lambda \frac{\inner{x,x}}{\inner{x,x}} = \lambda. \]
\end{proof}

\begin{proposition}
If $U$ is unitary, then $\spec(U)\subset \mathbb{T}$.
\end{proposition}

\section{Examples of Hilbert spaces}

\subsection{The $L^2$ spaces}

\subsubsection{Multiplication operators}
\begin{lemma} \label{multiplicationOperatorDenselyDefined}
Let $(\Omega, \mathcal{A}, \mu)$ be a measure space and $f\in \meas(\Omega, \C)$. Then $M_f: L^2(\Omega,\diff\mu)\to L^2(\Omega,\diff\mu)$ is densely defined.
\end{lemma}
\begin{proof}
Suppose $g\in \dom(M_f)^\perp$. Now $\underline{0}\leq (|f|+1)^2 = |f|^2 + 2|f| + 1$, so $|f|\leq \frac{1}{2}(|f|^2 + 1)$ and thus $\frac{2|f|}{|f|^2 + 1} \leq \underline{1}$. Now $\frac{2f}{|f|^2 + 1}g\in L^2(\Omega,\diff\mu)$ by \ref{propertiesIntegralPositiveFunctions}, using
\[ \Big|\frac{2f}{|f|^2 + 1}g\Big| = \frac{2|f|}{|f|^2 + 1}|g| \leq \underline{1}|g| = |g|. \]
Thus $\frac{2g}{|f|^2+1} \in \dom(M_f)$. Then
\[ 0 = \inner{g, \frac{2g}{|f|^2+1}} = \int_\Omega |g|^2 \frac{2}{|f|^2+1}\diff{\mu}, \]
which, by \ref{functionPropertiesFromIntegral}, implies $|g|^2 \frac{2}{|f|^2+1} = 0$ a.e. and thus $g = 0$ a.e.

We have shown that $\dom(M_f)^\perp = \{0\}$, which means that $\dom(M_f)$ is dense in $L^2(\Omega,\diff\mu)$.
\end{proof}

\begin{proposition}
Let $(\Omega, \mathcal{A}, \mu)$ be a measure space and $f\in \meas(\Omega, \C)$. Consider the multiplication operator $M_f: L^2(\Omega,\diff\mu)\to L^2(\Omega,\diff\mu)$. Then $M_f^* = M_{\overline{f}}$ and $\dom(M_f*) = \dom(M_f)$.
\end{proposition}
\begin{proof}
For any $g\in \dom(M_{\overline{f}})$ and $h\in \dom(M_f)$, we have
\[ \inner{g, M_fh} = \int_\Omega \overline{g}fh \diff\mu = \int_\Omega \overline{\overline{f}g}h \diff\mu = \inner{M_{\overline{f}}g, h}, \]
so $M_{\overline{f}}$ is an adjoint of $M_f$ and thus $M_{\overline{f}}\subseteq M_f^*$.

Now take $g\in \dom(M_f^*)$. Then, for all $h\in \dom(M_f)$, we have
\[ \int_\Omega \overline{M_f^*g}h \diff\mu = \inner{M_f^*g,h} = \inner{g,M_f h} = \int_\Omega \overline{g}fh\diff\mu, \]
so $\int_\Omega \overline{(M_f^*g - g\overline{f})}h\diff\mu = 0$. Set $A_n \defeq \{|f| \leq n\}$. Then $\charFunc{A_n}h'\in \dom(M_f)$ for all $h\in L^2(\Omega,\diff\mu)$. Thus $\int_\Omega \overline{(M_f^*g - g\overline{f})}\charFunc{A_n}h\diff\mu = 0$, which implies $\overline{(M_f^*g - g\overline{f})}\charFunc{A_n} = 0$ a.e. Since this holds for all $n\in \N$, we have $\overline{M_f^*g - g\overline{f}} = 0$. Thus $g\in \dom(M_{\overline{f}})$.

It is clear that $\dom(M_f) = \dom(M_{\overline{f}}) = \dom(M_f^*)$.
\end{proof}
\begin{corollary}
Let $(\Omega, \mathcal{A}, \mu)$ be a measure space and $f\in \meas(\Omega, \C)$. Then $M_f$ is a closed operator.
\end{corollary}
\begin{proof}
Since $M_{f} = M_{\overline{f}}^*$, the closedness of $M_f$ follows from \ref{adjointGraph}.
\end{proof}

\subsection{The $\ell^2$ spaces}
Sequence spaces $\ell^p$ Hilbert iff $p=2$. (TODO: other sequence spaces?)

\chapter{Types of operators}

\section{Integral operators and transforms}
\begin{definition}
Let $(\Omega, \mathcal{A}, \mu)$ be a measure space. Then an \udef{integral operator} or \udef{integral transform} is a map of the form
\[ T: U\subset (\Omega\to\C) \to (\Omega\to\C): f \mapsto \int_\Omega K(x,y)f(y) \diff{\mu(y)} \]
where $K\in (\Omega\times \Omega \to \C)$ is the \udef{kernel} or \udef{nucleus} of $T$.

The kernel is called
\begin{itemize}
\item \udef{symmetric} if $K(x,y) = \overline{K(y,x)}$;
\item \udef{Volterra} if $\Omega = \R$ and $K(x,y) = 0$ for $y>x$;
\item \udef{convolutional} if $\Omega$ is a group and $K(x,y) = F(x-y)$ for some function $F$;
\item \udef{Hilbert-Schmidt} if $K\in L^2(\Omega\times \Omega)$, i.e.\
\[ \int_{\Omega\times \Omega}|K(x,y)|^2\diff{x}\diff{y} < \infty; \]
\item \udef{singular} if $K(x,y)$ is unbounded on $\Omega\times \Omega$.
\end{itemize}
\end{definition}

\begin{lemma}
Hilbert-Schmidt integral operators are compact operators on $L^2(\Omega\times \Omega)$.
\end{lemma}
\begin{proof}
A Hilbert-Schmidt integral operator $T$ maps $L^2(\Omega)$ to $L^2(\Omega)$ functions:
\begin{align*}
\norm{Tu}^2_{L^2} &= \int_\Omega \left|\int_{\Omega} K(x,y)u(y)\diff{\mu(y)}\right|^2\diff{\mu(x)} \\
&\leq \int_\Omega \left(\int_{\Omega} |K(x,y)|^2\diff{\mu(y)}\right) \bigg( |u(y)|^2\diff{\mu(y)}\bigg)\diff{\mu(x)} \\
&= \left(\int_\Omega \int_{\Omega} |K(x,y)|^2\diff{\mu(y)}\diff{\mu(x)}\right) \bigg( |u(y)|^2\diff{\mu(y)}\bigg) < \infty
\end{align*}
where we have used the Cauchy-Schwarz inequality \ref{CauchySchwarz}. This also immediately shows Hilbert-Schmidt integral operators are bounded.

TODO Compact
\end{proof}

\begin{proposition}
Let $T$ be an integral operator with kernel $K(x,y)$, then $T^*$ is the integral operator with kernel $\overline{K(y,x)}$.
\end{proposition}
\begin{proof}
TODO
\end{proof}

\begin{proposition}
Let $A$ be a Borel set and $K:A\times A\to \C$ a measurable function such that the integral operator with kernel $K$ is bounded. Then the adjoint of the integral operator is again an integral operator with kernel $K^*(x,y) = \overline{K(y,x)}$.
\end{proposition}

\begin{proposition}
Let $T$ be a Volterra integral operator. Then $\spec(T) = \cspec(T) = \{0\}$.
\end{proposition}
\begin{proof}
TODO
\end{proof}

\subsection{Integral equations}
\begin{definition}
Let $(\Omega, \mathcal{A}, \mu)$ be a measure space. An \udef{integral equation} is an equation containing an unknown function on $\Omega$ and an integral over $\Omega$.

An integral equation is 
\begin{itemize}
\item \udef{of the first kind} if it is of the form
\[ \int_\Omega K(x,y)u(y)\diff{\mu(y)} = f(x) \qquad x\in \Omega \]
where $f$ is a given function and $u$ is the unknown function;
\item \udef{of the second kind} if it is of the form
\[ \lambda u(x) - \int_\Omega K(x,y)u(y)\diff{\mu(y)} = f(x) \qquad x\in \Omega \]
where $f$ is a given function, $\lambda$ is a scalar and $u$ is the unknown function.
\end{itemize}
\end{definition}

\begin{proposition}
Let
\[ \lambda u(x) - \int_\Omega K(x,y)u(y)\diff{\mu(y)} = f(x)\]
be an integral equation of the second kind. This integral equation has a unique solution $u$ if
\[ |\lambda| > \sup_{x\in \Omega} \int_{\Omega}|K(x,y)|\diff{\mu(y)}. \]
\end{proposition}
\begin{proof}
Let the map $T$ be defined by
\[ T(u) = x\mapsto \frac{1}{\lambda}\left(\int_\Omega K(x,y)u(y)\diff{\mu(y)} + f(x)\right) \]
so that solutions of the integral equation are exactly the fixed points of $T$. Then
\[ \norm{Tu-Tv}_\infty = \sup_{x\in\Omega} \frac{1}{|\lambda|} \left|\int_\Omega K(x,y)(u(y)- v(y))\diff{\mu(y)}\right| \leq \frac{1}{|\lambda|} \sup_{x\in \Omega} \int_{\Omega}|K(x,y)|\diff{\mu(y)} \cdot \norm{u-v}_\infty. \]
So $T$ is a contraction if $|\lambda| > \sup_{x\in \Omega} \int_{\Omega}|K(x,y)|\diff{\mu(y)}$. The result follows from \ref{contractionFixedPoint}.
\end{proof}

\section{Convolution operators}






\chapter{Fourier transforms}

\section{Types of Fourier transform}

\subsection{Discrete Fourier transform}
\begin{definition}
Then $N$-dimensional \udef{discrete Fourier transform} (DFT) is the linear transformation $\C^N \to \C^N$ defined by the matrix $DFT_N$ with components
\[ [DFT_N]_{j,k} = \frac{1}{\sqrt{N}}\omega_N^{(j-1)(k-1)}, \]
where $\omega_N$ is the $N^\text{th}$ root of unity.
\end{definition}

\begin{lemma} \mbox{}
\begin{enumerate}
\item The $DFT_N$ matrix is the Vandermonde matrix of the roots of unity, up to the normalisation factor $1/\sqrt{N}$.
\item The $DFT_N$ matrix is unitary.
\end{enumerate}
\end{lemma}
\begin{proof}
(1) Just an observation.

(2) We calculate
\[ [DFT_N\cdot DFT_N]_{j,l} = \frac{1}{N}\sum_{k=1}^N\omega_N^{jk}\overline{\omega_N}^{kl} = \frac{1}{N}\sum_{k=1}^N\omega_N^{k(j-l)} = \delta_{j,l}. \]
\end{proof}


\chapter{$C^*$-algebras}
\section{$*$-algebras}
TODO: move higher: $*$-algebras much higher and Banach stuff with Banach stuff.
\begin{definition}
A \udef{$*$-algebra} is a $*$-r(i)ng $(A,+,\cdot, *)$, with involution $*$, that is an associative algebra over a commutative $*$-ring $(R,+,\cdot, ')$, with involution $'$,
such that
\[ \forall r\in R, x\in A: \quad (rx)^* = r'x^*. \]

A \udef{complex $*$-algebra} is a $*$-algebra where the $*$-ring $R$ is $\C$ with complex conjugation as the involution $'$.

A \udef{real $*$-algebra} is a $*$-algebra where the $*$-ring $R$ is $\R$ with the identity map as the involution $'$.

If a $*$-algebra is also a Banach algebra and for all elements $\norm{x^*} =\norm{x}$, then it is called a \udef{Banach-$*$-algebra}.
\end{definition}

\begin{lemma} \label{starIsometry}
Let $A$ be a Banach-$*$-algebra. Then the map $x\mapsto x^*$ is an isometry and thus continuous.
\end{lemma}

\begin{lemma}
Let $A$ be a $*$-algebra. The unitisation $A^\dagger = A\oplus \F$ can also be seen as a $*$-algebra with the involution defined by
\[ (a, \lambda)^* = (a^*, \overline{\lambda}) \qquad \forall a\in A, \lambda\in \F.\]
\end{lemma}
\begin{lemma} \label{elementaryStarLemma}
Let $A$ be a unital $*$-algebra and $x\in A$. Then
\begin{enumerate}
\item $\vec{1}^* = \vec{1}$;
\item $x$ is invertible \textup{if and only if} $x^*$ is invertible; in this case with $(x^*)^{-1} = (x^{-1})^*$;
\item $\sigma(x^*) = \setbuilder{\overline{\lambda}}{\lambda \in \sigma(x)}$.
\end{enumerate}
\end{lemma}
\begin{proof}
(1) We have $\vec{1}^* x = (x^*\cdot \vec{1})^* = x^{**} = x$.
Similarly $x\vec{1}^* = x$, so $\vec{1}^*$ is an identity, which is unique by \ref{identityAbsorbingElementUnique}.

(2) First assume $x$ is invertible. We have
\[ x^*\cdot (x^{-1})^* = (x^{-1}x)^* = \vec{1}^* = \vec{1} \qquad\text{and}\qquad (x^{-1})^*\cdot x^* = (x\cdot x^{-1})^* = \vec{1}^* = \vec{1}. \]
This means that $x^*$ is invertible with $(x^*)^{-1} = (x^{-1})^*$.

Now assume $x^*$ is invertible. By the previous calculation, we have that $x = x^{**}$ is invertible.
\end{proof}

\begin{proposition} \label{smallestBanach*Algebra}
Let $A$ be a Banach-$*$-algebra and $S\subset A$ a subset. Then
\[ \mathcal{B}^*(S) \defeq \mathcal{B}(S\cup S^*) \]
is the smallest Banach-$*$-subalgebra in $A$ that contains $S$, where $\mathcal{B}$ is defined as in \ref{smallestBanachAlgebra} and $S^* = \setbuilder{s^*\in A}{s\in S}$.
\end{proposition}

\begin{definition}
Let $A$ be a $*$-algebra and $x\in A$. We say that $x$ is
\begin{enumerate}
\item \udef{normal}, if $x^*x = xx^*$;
\item \udef{self-adjoint}, if $x=x^*$;
\item a \udef{projection}, if $x=x^*=x^2$;
\item a \udef{partial isometry} if $x^*x$ and $xx^*$ are both projections. Then $x^*x$ is called the \udef{support projection} and $xx^*$ is called the \udef{range projection}.
\end{enumerate}
If $A$ is unital, we call $x$
\begin{itemize}
\item an \udef{isometry}, if $x^*x = \vec{1}$;
\item a \udef{co-isometry}, if $xx^* = \vec{1}$;
\item \udef{unitary}, if $x^*x = xx^* = \vec{1}$.
\end{itemize}
The set of all
\begin{enumerate}
\item normal elements in $A$ is denoted $\Normals(A)$;
\item self-adjoint elements in $A$ is denoted $\SelfAdjoints(A)$;
\item unitaries in $A$ is denoted $\Unitaries(A)$;
\item projections in $A$ is denoted $\Projections(A)$.
\end{enumerate}
\end{definition}
\begin{lemma}
We have the following implications:
\[ \text{projection} \Rightarrow \text{self-adjoint} \Rightarrow \text{normal} \Leftarrow \text{unitary}, \]
where the right-most arrow is only valid in unital $*$-algebras.
\end{lemma}

\begin{lemma}
Let $A$ be a unital $*$-algebra and $x\in A$. If $x$ is either an isometry or a co-isometry, then $x$ is a partial isometry.
\end{lemma}
\begin{proof}
First assume $x$ is an isometry. Then $x^*x = \vec{1}$ is clearly a projection. We calculate $(xx^*)^* = x^{**}x^* = xx^*$ and $(xx^*)^2 = x(x^*x)x^* = x\vec{1}x^* = xx^*$, which means that $xx^*$ is a projection.
\end{proof}

\begin{lemma} \label{orthogonalProjection}
Let $A$ be a unital $*$-algebra and $p\in\Projections(A)$. Then $\vec{1}-p$ is a projection.
\end{lemma}
\begin{proof}
We simply calculate
\[ (\vec{1}-p)^2 = (\vec{1}-p)(\vec{1}-p) = \vec{1} - p -p + p = \vec{1}-p = (\vec{1}-p)^*. \]
\end{proof}

\begin{proposition} \label{closedSubsetsBanachStarAlgebra}
Let $A$ be a Banach-$*$-algebra. Then
\begin{enumerate}
\item $\Normals(A)$ is a closed subset of $A$;
\item $\SelfAdjoints(A)$ is a closed subset of $A$;
\item $\Projections(A)$ is a closed subset of $A$;
\item if $A$ is unital, then $\Unitaries(A)$ is a closed subset of $A$.
\end{enumerate}
\end{proposition}
\begin{proof}
Note that the map $x\mapsto x^*$ is continuous, \ref{starIsometry}.

(1) Consider the continuous function $g: A\to A: x\mapsto x^*x - xx^*$. Then $\Normals(A) = g^\preimf(\{0\})$ and so is closed by \ref{preimageOpenClosed}.

(2) Consider the continuous function $g: A\to A: x\mapsto x - x^*$. Then $\SelfAdjoints(A) = g^\preimf(\{0\})$ and so is closed by \ref{preimageOpenClosed}.

(3) Consider the continuous functions $g_1: A\to A: x\mapsto x - x^*$ and $g_2: A\to A: x\mapsto x-x^2$. Then $\SelfAdjoints(A) = g_1^\preimf(\{0\}) \cap g_2^\preimf(\{0\})$ and so is closed by \ref{preimageOpenClosed} and \ref{propertiesTopology}.

(4) Consider the continuous functions $g_1: A\to A: x\mapsto xx^*$ and $g_2: A\to A: x\mapsto x^*x$. Then $\SelfAdjoints(A) = g_1^\preimf(\{\vec{1}\}) \cap g_2^\preimf(\{\vec{1}\})$ and so is closed by \ref{preimageOpenClosed} and \ref{propertiesTopology}.
\end{proof}

\begin{lemma} \label{realImaginaryParts}
Let $A$ be a $*$-algebra and $x\in A$. Then there are unique self-adjoint elements $x_1,x_2\in A$ such that $x = x_1+i\cdot x_2$. They are given by
\[ x_1 = \frac{x+x^*}{2} \qquad \text{and} \qquad x_2 = \frac{x-x^*}{2i}. \]
\end{lemma}
We call $x_1$ and $x_2$ the \udef{real part} and \udef{imaginary part} of $x$, respectively.

\subsection{Ideals}
\begin{definition}
Let $A$ be a $*$-algebra. A $*$-ideal is an (algebra) ideal that is closed under the $*$ operation.
\end{definition}

\begin{lemma} \label{starIdealCongruence}
Let $A$ be a $*$-algebra. A translation invariant binary relation $\mathfrak{q}$ on $A$ is a
$\{+, ·, (z\cdot -), *\}_{z\in \C}$-congruence if and only if $\widetilde{\mathfrak{q}}$ is a $*$-ideal.
\end{lemma}
\begin{proof}
Compare \ref{congruenceRingIdeals} and \ref{congruenceSubspace}.

For all $(a,b)\in \mathfrak{q}$, we have
\[ (a,b)^* = (a^*, b^*) \in \mathfrak{q} \iff a^* - b^* = (a-b)^*\in \widetilde{\mathfrak{q}}. \]
\end{proof}

\subsection{$*$-homomorphisms}
\begin{definition}
Let $A,B$ be $*$-algebras. A \udef{$*$-homomorphism} is a linear, multiplicative, $*$-preserving map $\Psi: A \to B$.

If $A,B$ are unital and $\Psi(\vec{1}_A) = \vec{1}_B$, then we say $\Psi$ is \udef{unital}.
\end{definition}
\begin{lemma}
Let $A$ be a $*$-algebra, then $*$-homomorphisms map
\begin{enumerate}
\item normal elements to normal elements;
\item self-adjoints to self-adjoints;
\item projections to projections;
\item unitaries to unitaries, if the $*$-homomorphism is unital.
\end{enumerate}
\end{lemma}

\subsection{$*$-matrix algebras}
TODO define matrix algebra.
\begin{definition}
Let $A$ be a $*$-algebra. Then the matrix algebra $A^{n\times n}$ is considered a $*$-algebra with the star operation given defined by
\[ [a^*]_{i,j} \defeq [a]_{j,i}^*. \]
for all components of $a\in A^{n\times n}$.
\end{definition}
Notice that the $*$-operation acts as the element-wise $*$-operation composed with the transpose.
\section{$C^*$-algebras}
\begin{definition}
A (complex) \udef{$C^*$-algebra} is a complex Banach-$*$-algebra $A$ such that
\[\forall x\in A: \quad \norm{x^*x} = \norm{x}^2.\]
This identity is known as the \udef{$C^*$-identity}, the \udef{$C^*$-property}, the \udef{$C^*$-condition} or the \udef{$C^*$-axiom}.
\end{definition}

\begin{definition}
A \udef{real} $C^*$-algebra is a real Banach-$*$-algebra such that 
\end{definition}

\begin{example}
TODO: Concrete $C^*$-algebras.

$\mathcal{C}(X)$ for some compact $X$ (need Hausdorff?). TODO: norm well defined (i.e.\ bounded) and for $g\in \mathcal{C}(X)$, $\sigma(g) = g[X]$.
\end{example}

\begin{proposition}
The $C^*$-identity is equivalent to
\[\forall x\in A: \quad \norm{x^*x} = \norm{x^*}\cdot\norm{x}.\]
\end{proposition}
\begin{proof}
TODO. Highly non-trivial. TODO: move later.
\end{proof}


\begin{lemma} \label{C*identityEquivalent}
The $C^*$-identity is equivalent to
\[\forall x\in A: \quad \norm{x^*x} \geq \norm{x}^2.\]
\end{lemma}
\begin{proof}
Clearly the $C^*$-identity implies this inequality.

For the converse, let $x\in A$, then $\norm{x}^2 \leq \norm{x^*x} \leq \norm{x}\cdot\norm{x^*}$ and so $\norm{x}\leq \norm{x^*}$. By replacing $x$ with $x^*$ and using $x^{**}=x$ we also get $\norm{x}\geq \norm{x^*}$. Then $\norm{x^*x}\leq \norm{x^*}\cdot\norm{x} = \norm{x}^2$. Together with the original inequality this implies the $C^*$-identity.
\end{proof}

\begin{definition}
Let $A$ be a $C^*$-algebra and $D$ a subset of $A$. The $C^*$-algebra \udef{generated} by $D$, $C^*(D)$, is the smallest $C^*$-subalgebra of $A$ containing $D$.
TODO refine def.
\end{definition}
\begin{lemma}
$C^*(\vec{1},a)$ is commutative.
\end{lemma}

\begin{lemma} \label{consequencesC*}
Let $A$ be a $C^*$-algebra. The $C^*$-identity implies
\begin{enumerate}
\item  $\norm{\vec{1}} = 1$.
\item the involution $*$ is isometric: $\norm{x^*} = \norm{x}$;
\item the involution $*$ is continuous.
\end{enumerate}
\end{lemma}

\begin{lemma}
Let $A$ be a $C^*$-algebra. Then the sets $\Normals(A), \SelfAdjoints(A), \Unitaries(A)$ and $\Projections(A)$ are closed in $A$. 
\end{lemma}
\begin{proof}
This follows from the continuity of the multiplication and the involution $*$.
\end{proof}

\begin{proposition} \label{normNormalElement}
Let $A$ be a $C^*$-algebra and $x\in A$ a normal element. Then $\spr(x) = \norm{x}$.
\end{proposition}
\begin{proof}
We compute
\[ \norm{x}^4 = \norm{x^*x}^2 = \norm{x^*xx^*x} = \norm{(x^*)^2x^2} = \norm{x^2}^2 \]
where we have repeatedly applied the $C^*$-identity and used normality once. We conclude that $\norm{x^2} = \norm{x}^2$. Inductively we obtain $\norm{x^{(2^n)}} = \norm{x}^{2^n}$. By the spectral radius formula, \ref{spectralRadiusFormula}, we get
\[ \spr(x) = \lim_{n\to\infty}\norm{x^{(2^n)}}^{1/2^n} = \lim_{n\to\infty}\norm{x} = \norm{x}. \]
\end{proof}
\begin{corollary} \label{atMostOneNorm}
Let $A$ be a $*$-algebra. There exists at most one norm on $A$ turning it into a $C^*$-algebra. If there is such a norm, it is given by $\norm{x} = \sqrt{\spr(x^*x)}$.
\end{corollary}
\begin{proof}
By the $C^*$-identity $\norm{x} = \sqrt{\norm{x^*x}}$ and $x^*x$ is normal, so we can apply the proposition.
\end{proof}
It is important to note that the spectral radius is a purely algebraic property and is independent of the norm.

\begin{lemma}
Let $A$ be a $C^*$-algebra and $a\in A$. Then $\norm{a} = \sup_{\norm{x}\leq 1}\norm{ax}$.
\end{lemma}
\begin{proof}
It is clear that the map $a\mapsto \sup_{\norm{x}\leq 1}\norm{ax}$ is a norm. By \ref{atMostOneNorm}, it is enough to verify that it satisfies the $C^*$-identity.

We calculate, using $\norm{x^*} = \norm{x}$ (\ref{consequencesC*}),
\begin{align*}
\sup_{\norm{x}\leq 1}\norm{a^*ax} &\geq \sup_{\norm{x}\leq 1}\norm{x^*}\cdot\norm{a^*ax} \\
&\geq \sup_{\norm{x}\leq 1}\norm{x^*a^*ax} = \sup_{\norm{x}\leq 1}\norm{(ax)^*ax} = \sup_{\norm{x}\leq 1}\norm{ax}^2 = \big(\sup_{\norm{x}\leq 1}\norm{ax}\big)^2.
\end{align*}
The $C^*$-identity follows from \ref{C*identityEquivalent}.
\end{proof}

\subsection{Properties of elements}
\subsubsection{Self-adjoint elements}
\begin{lemma} \label{normSelfAdjoint}
Let $A$ be a unital $C^*$-algebra and $x\in A$ a self-adjoint element. Then
\[ \forall t\in\R: \quad \norm{x+it}^2 = \norm{x^2+t^2} \leq \norm{x}^2 + t^2. \]
\end{lemma}
\begin{proof}
Since $(x+it\cdot\vec{1})^* = x^* - it\cdot\vec{1}^* = x - it\cdot\vec{1}$ (using \ref{elementaryStarLemma}), we have
\begin{align*}
\norm{x+it}^2 &= \norm{(x+it)(x+it)^*} \\
&= \norm{(x+it)(x-it)} \\
&= \norm{x^2+t^2} \\
&\leq \norm{x^2}+t^2\norm{\vec{1}} = \norm{x^2}+t^2 \leq \norm{x}^2+t^2.
\end{align*}
\end{proof}

\begin{proposition}
Let $A$ be a unital $C^*$-algebra and $\varphi: A\to \C$ a linear functional satisfying $\norm{\varphi} = \varphi(\vec{1})$. If $a\in A$ is self-adjoint, then $\varphi(a) \in \R$.
\end{proposition}
\begin{proof}
We may assume $\varphi \neq 0$ and $\varphi(\vec{1}) = 1$. Using \ref{normSelfAdjoint}, we calculate
\[ |\varphi(x)+it|^2 = |\varphi(x+it)|^2 \leq \norm{x+it}^2 \leq \norm{x}^2 + t^2. \]
By \ref{boundedThenReal} this means $\varphi(x)\in\R$.
\end{proof}
TODO: alternate proof in Fillmore using exponential map.
\begin{corollary} \label{selfAdjointSpectrumReal}
Let $x\in A$ be self-adjoint. Then $\sigma(x) \subseteq \R$.
\end{corollary}
\begin{proof}
By \ref{commutativeSameSpectrum}, and the fact that $x$ is self-adjoint (TODO ref) we may assume $A$ commutative. Let $\lambda \in \sigma(x)$. By \ref{spectrumFromSpectrum} there is a $\varphi\in\hat{A}$ such that $\lambda = \varphi(x)$. By \ref{charactersUnital}, $\norm{\varphi} = \varphi(\vec{1}) = 1$. Then by the proposition $\lambda = \varphi(x) \in \R$.
\end{proof}
\begin{corollary}
Let $\varphi:A\to \C$ be a \emph{linear} functional satisfying $\norm{\varphi} = \varphi(\vec{1})$. Then $\varphi$ is $*$-preserving, i.e.\ for all $x\in A$
\[\varphi(x^*) = \overline{\varphi(x)}. \] 
\end{corollary}
\begin{proof}
By \ref{realImaginaryParts} we can write $x= x_1+ix_2$. Then $\varphi(x^*) = \varphi(x_1-ix_2) = \varphi(x_1) - i \varphi(x_2)$. By the proposition $\varphi(x_1), \varphi(x_2)\in \R$.
\end{proof}
\begin{corollary} \label{characters*Preserving}
Every character on $A$ is $*$-preserving.
\end{corollary}


\subsubsection{Partial isometries}
\begin{lemma} \label{partialIsometries}
Let $A$ be a $C^*$-algebra and $v\in A$.
If either $v^*v$ or $vv^*$ is a projection, then
\begin{enumerate}
\item $v=vv^*v$ and $v^* = v^*vv^*$;
\item the other is also a projection.
\end{enumerate}
Thus either condition is sufficient for $v$ to be a partial isometry.
\end{lemma}
\begin{proof}
First assume $v^*v$ is a projection. Put $z = (\vec{1}- vv^*)v$. Then
\[ z^*z = v^*(\vec{1}-vv^*)(\vec{1}-vv^*)v = v^*v - v^*vv*v -v^*vv^*v + v^*vv^*vv^*vv^*v = v^*v - v^*v - v^*v + v^*v = 0. \]
Now by the $C^*$-identity $\norm{z} = \sqrt{\norm{z^*z}} = 0$, so $0=z= v - vv^*v$. Multiplying this on the right by $v^*$ gives $vv^* = (vv^*)^2$. Because $vv^*$ is self-adjoint, this shows us it is a projection.

If we assume $vv^*$ is a projection, we put $z = (\vec{1}- v^*v)v^*$ instead and again $z^*z = 0$.
\end{proof}

\begin{lemma}
Let $v\in\Bounded(\mathcal{H})$ be a partial isometry. Then
\[\ker v = (I - v^*v)\mathcal{H} \qquad \ker v^* = (I - vv^*)\mathcal{H}\]
and
\[ v\mathcal{H} = vv^*\mathcal{H} \qquad v^*\mathcal{H} = v^*v\mathcal{H}. \]
\end{lemma}
\begin{proof}
First assume $x\in\ker v$. Then $(I-vv^*)x = Ix = x$, so $x\in (I - vv^*)\mathcal{H}$. Conversely, $v(I-v^*v)\mathcal{H} = (v-vv^*v)\mathcal{H} = 0\mathcal{H} = \{0\}$ by \ref{partialIsometries}.

Also by \ref{partialIsometries}, we have
\[ v\mathcal{H} \subseteq vv^*\mathcal{H} \subseteq vv^*v\mathcal{H} = v\mathcal{H}. \]
\end{proof}

\subsection{Algebraic aspects}
\subsubsection{Ideals}

\begin{proposition} \label{closedIdealIsStarIdeal}
Let A be a $C^*$-algebra and $I\subseteq A$ a closed two-sided ideal. Then $I$ is closed under the $*$ operation.
\end{proposition}
\begin{proof}
page 41 of \url{https://www.math.uvic.ca/faculty/putnam/ln/C*-algebras.pdf}
\end{proof}

\subsubsection{The lattice of $C^*$-algebras}
\begin{lemma}
The $C^*$-algebraic closure is given by the norm-closure of the algebraic closure.
\end{lemma}

\subsection{$C^*$-homomorphisms}

\begin{proposition} \label{starHomomorphismCstarProperties}
Let $A,B$ be $C^*$-algebras and $\Psi: A\to B$ a $*$-homomorphism. Then
\begin{enumerate}
\item $\Psi$ is bounded (and thus continuous) with $\norm{\Psi} \leq 1$;
\item if $\Psi$ is injective, then it is isometric;
\item $\ker(\Psi)$ is a closed $*$-ideal;
\item $\im(\Psi)$ is closed and thus a $C^*$ algebra;
\item if $A,B$ and $\Psi$ are unital, then $\norm{\Psi} = 1$.
\end{enumerate}
\end{proposition}
\begin{proof}
(1) WLOG we may assume $\Psi$ is unital, since we may replace it by $\Psi^\dagger$ and this operation preserves relevant properties, by \ref{DaggerMorphismProperties}. Because $x^*x$ and $\Psi(x^*x)$ are normal, we calculate using \ref{normNormalElement}
\[ \norm{x}^2 = \norm{x^*x} = \spr(x^*x) \geq \spr(\Psi(x^*x)) = \norm{\Psi(x^*x)} = \norm{\Psi(x)^*\Psi(x)} = \norm{\Psi(x)}^2, \]
where the inequality follows from an application of lemma \ref{spectrumOfImage} to $x^*x$ (using unitality of $\Psi$). Hence $\norm{\Psi}\leq 1$.

(2) By the calculation in (1), it is enough to prove $\spr(x^*x) \leq \spr(\Psi(x^*x))$. We can define a $*$-homomorphism $\Psi^{-1}: \im(\Psi)\to A$. Then
\[ \spr(\Psi(x^*x)) = \spr_{\im(\Psi)}(\Psi(x^*x)) \geq \spr\big((\Psi^{-1}\circ\Psi)(x^*x)\big) = \spr(x^*x) \]
by \ref{spectrumIndependentOfSurroundingAlgebra} and \ref{spectrumOfImage}.

(3) By \ref{kernelIsIdeal} $\ker(\Psi)$ is an ideal. By \ref{closedIdealIsStarIdeal} it is enough to observe that $\ker(\Psi)$ is closed, which it is by \ref{preimageOpenClosed}.

(4) By the factor theorem \ref{factorTheorem} and \ref{starIdealCongruence}, we obtain an injective $*$-homomorphism $\Psi': A/\ker(\Psi) \to B$ with $\im(\Psi') = \im(\Psi)$. Now $\Psi'$ is isometric by point (2), so $\im(\Psi) = \im(\Psi')$ is closed by \ref{isometryLemma}.

(5) In this case we have $\norm{\Psi(\vec{1})} = \norm{\vec{1}}$, so $\norm{\Psi} \geq 1$.
\end{proof}


\begin{lemma}
A surjective $*$-homomorphism from a unital $C^*$-algebra is unital.
\end{lemma}
\begin{proof}
Let $\Psi: A \to B$ be a surjective $*$-homomorphism with $A$ unital. For all $a\in A$:
\[ \Psi(a)\Psi(\vec{1}) = \Psi(a \vec{1}) = \Psi(a) \qquad \text{and} \qquad \Psi(\vec{1})\Psi(a) = \Psi(\vec{1} a) = \Psi(a). \]
As all elements of $B$ are of the form $\Psi(a)$, $\Psi(\vec{1})$ is a multiplicative identity for $B$.
\end{proof}

\subsubsection{Lifts}
TODO: move to $*$-algebra homomorphisms?
\begin{proposition}
Let $\Psi: A \to B$ be a surjective $*$-homomorphism between $C^*$-algebras. Then
\begin{enumerate}
\item every self-adjoint element $b\in B$ has a self-adjoint lift $a\in A$, such that $\norm{a} = \norm{b}$;
\item every positive element $b\in B$ has a positive lift $a\in A$, such that $\norm{a} = \norm{b}$;
\item every element $b\in B$ has a lift $a\in A$ such that $\norm{a} = \norm{b}$.
\end{enumerate}
\end{proposition}
In general normal elements, unitaries and projections do not lift to normal elements, unitaries and projections, unless $\Psi$ is injective.
\begin{lemma} \label{injectiveLifts}
Let $\Psi: A \to B$ be an injective $*$-homomorphism between $C^*$-algebras.
\begin{enumerate}
\item if $\Psi(a)$ is a normal element, then $a$ is a normal element;
\item if $\Psi(p)$ is a projection, then $p$ is a projection;
\item if $\Psi(u)$ is a unitary element, then $u$ is a unitary element.
\end{enumerate}
\end{lemma}
\begin{proof}
If $\Psi(a)\Psi(a)^* = \Psi(a)^*\Psi(a)$, then $\Psi(a^*a) = \Psi(aa^*)$ and $a^*a = aa^*$ by injectivity.

The other conditions are verified similarly.
\end{proof}

\subsection{Direct sums of $C^*$-algebras}
Note that multiplication in direct sum is pointwise: $(a,b)(c,d) = (ac, bd)$.

\begin{proposition}
Let $A,B$ be $C^*$-algebras. Then there exists a norm on $A\oplus B$ that makes it a $C^*$-algebra. This norm is given by
\[ \norm{(a,b)}_{A\oplus B} \defeq \max\{\norm{a}_A, \norm{b}_B\}. \]
\end{proposition}
\begin{proof}
TODO it is a norm.

We check that is satisfies the $C^*$-identity. Take $(a,b)\in A\oplus B$. Then
\begin{align*}
\norm{(a,b)^*(a,b)}_{A\oplus B} &= \norm{(a^*a,b^*b)}_{A\oplus B} \\
&= \max\{\norm{a^*a}_A, \norm{b^*b}_B\} \\
&= \max\{\norm{a}_A^2, \norm{b}_B^2\} \\
&= \big(\max\{\norm{a}_A, \norm{b}_B\}\big)^2 \\
&= \norm{(a,b)}_{A\oplus B}^2.
\end{align*}
\end{proof}

\subsubsection{Unitisation of $C^*$-algebras}
For Banach-$*$-algebras we may have a choice of norms to put on the unitisation. For $C^*$-algebras there is exactly one.

Note that the norm of $A^\dagger$ is not the norm of $A\oplus \C$. This is because the multiplication in $A^\dagger$ is not the multiplication in $A\oplus \C$.
\begin{proposition}
Let $A$ be a $C^*$-algebra. Then there exists a unique norm on $A^\dagger$ that turns it into a $C^*$-algebra: the operator norm
\[ \norm{(a,\lambda)} \defeq \sup\setbuilder{\norm{ax + \lambda x}}{x\in A \land \norm{x}\leq 1} .\]
\end{proposition}
\begin{proof}
There is at most one such norm, by \ref{atMostOneNorm}. Because the operator norm is a suitable norm for $A^\dagger$ by \ref{normsOfUnitisation}, we just need to verify the $C^*$-identity:
\begin{align*}
\norm{(a,\lambda)^*(a,\lambda)} &= \sup_{\norm{x}\leq 1} \norm{(a,\lambda)^*(a,\lambda)x} \\
&\geq \sup_{\norm{x}\leq 1} \norm{x^*}\cdot\norm{(a,\lambda)^*(a,\lambda)x} \geq \sup_{\norm{x}\leq 1} \norm{x^*(a,\lambda)^*(a,\lambda)x} \\
&= \sup_{\norm{x}\leq 1} \norm{((a,\lambda)x)^*(a,\lambda)x} = \sup_{\norm{x}\leq 1} \norm{(a,\lambda)x}^2 = \norm{(a,\lambda)}^2,
\end{align*}
where we have used the $C^*$-identity in $A$ because $(a,\lambda)x = ax + \lambda x \in A$.
\end{proof}

\section{Continuous functional calculus}

\begin{theorem}[Stone-Weierstrass] \label{StoneWeierstrass}
Let $X$ be a compact Hausdorff space. Let $A\subseteq \mathcal{C}(X)$ be a unital $*$-subalgebra. Suppose that $A$ separates points, i.e.\ for all $x\neq y$ in $X$ there exists $f\in A$ with $f(x) \neq f(y)$. Then $A$ is dense in $\mathcal{X}$ with respect to $\norm{\cdot}_\infty$.
\end{theorem}
TODO: can we relate this to canonical mapping to second dual, and general duality theory?

\subsection{Spectral stability of $C^*$-algebras}
\begin{proposition}
Let $A$ be a unital $C^*$-algebra, $B\subseteq A$ a unital $C^*$-subalgebra and $x\in B$. Then $x$ is invertible in $B$ \textup{if and only if} $x$ is invertible in $A$.
\end{proposition}
\begin{proof}
If an inverse exists in $B$, said inverse will also be in $A$.

Conversely, suppose $x$ not invertible in $B$. Then either $x^*x$ or $xx^*$ is not invertible in $B$ by \ref{elementaryStarLemma} and \ref{productInvertibility}. Let $y$ be one of the two that is not invertible. Because $y$ is self-adjoint, $\sigma_B(y)\subset \R$, by \ref{selfAdjointSpectrumReal}. Thus $(y_n) = (y+\frac{i}{n})$ is a sequence of invertibles converging to $y$. By \ref{openSetInvertibles}, $\norm{y_n - y}\geq \norm{y_n^{-1}}^{-1}$ must hold for all $n$, otherwise $y$ would be invertible. Thus $\norm{y_n^{-1}}^{-1}$ must converge to zero and $\norm{y_n^{-1}}$ must diverge (TODO ref). Since the inversion map is continuous on $\GL(A)$, by \ref{inverseMapContinuous}, it follows that $y$ cannot be invertible in $A$. Since $x$ is invertible if and only if $x^*$ is invertible, by \ref{elementaryStarLemma}, $y$ being non-invertible implies $x$ is not invertible, by \ref{productInvertibility}.
\end{proof}
\begin{corollary} \label{spectrumIndependentOfSurroundingAlgebra}
Let $A$ be a $C^*$-algebra, $B\subseteq A$ a $C^*$-subalgebra and $x\in B$.

The spectrum of $x$ is independent of the surrounding algebra:
\[ \sigma_B(x) = \sigma_A(x). \]
\end{corollary}
\begin{proof}
Apply the proposition to $\tilde{A}$ and $\tilde{B}$. Note that if $A,B$ are non-unital, the unit of $B^\dagger$ is the same as that of $A^\dagger$.
\end{proof}
This does not hold in general for Banach-$*$-algebras!

TODO example


\subsection{The Gelfand-Naimark theorem}
\begin{theorem}[Unital Gelfand-Naimark] \label{GelfandNaimarkCommutative}
Let $A$ be a unital commutative $C^*$-algebra. Then the Gelfand transform
\[ \evalMap_{-}|_{\hat{A}}: A\to\cont(\hat{A}): x\mapsto \evalMap_{x} \]
is an isometric $*$-isomorphism.
\end{theorem}
Here we view $\cont(\hat{A})$ as a $C^*$-algebra with involution $f^*(\varphi) = \overline{f(\varphi)}$ for all $\varphi\in\hat{A}$.
\begin{proof}
We already know the Gelfand transform is a homomorphism by \ref{GelfandTransformHomomorphism}.

We first prove it is a $*$-homomorphism: $\forall x\in A: (\evalMap_{x})^* = \evalMap_{x^*}$. To that end, take some $\varphi\in\hat{A}$. Then
\[ (\evalMap_{x})^*(\varphi) = \overline{\evalMap_{x}(\varphi)} = \overline{\varphi(x)} = \varphi(x^*) = \evalMap_{x^*}(\varphi), \]
where the third equality is due to \ref{characters*Preserving}.

For isometry, notice that every element is normal due to commutativity. By \ref{normNormalElement} and \ref{GelfandTransformHomomorphism}, we have
\[ \norm{x} = \spr(x) = \norm{\evalMap_{x}}. \]

Finally, to show that the Gelfand transform is an isomorphism, we just need to show that it is bijective by \ref{inverseHomomorphism}. Injectivity follows from isometry, \ref{isometryLemma}.

For surjectivity, note that $\im(\evalMap_{-})$ separates $\hat{A}$ (indeed, for $\varphi \neq \psi$ in $\hat{A}$, there exists $x\in A$ such that $\varphi(x) \neq \psi(x)$and thus $\evalMap_x(\varphi) \neq \evalMap_x(\psi)$).
By the Stone-Weierstrass theorem \ref{StoneWeierstrass}, $\im(\evalMap_{-})$ is dense in $\cont(\hat{A})$. By \ref{isometryLemma} the image of an isometry from a complete space is closed. Thus the image of $A$ is all of $\cont(\hat{A})$.
\end{proof}
\begin{corollary}
Every commutative unital $C^*$-algebra is isomorphic to $\cont(X)$ for some compact Hausdorff space $X$.
This space $X$ is unique up to homeomorphism.
\end{corollary}
\begin{proof}
The first part follows from the Gelfand-Naimark theorem and the fact that $\hat{A}$ is compact, \ref{characterSpaceLocallyCompact}.


For the second part, we know that $\widehat{C(X)}\cong X$ for all compact topological spaces by \ref{charactersFunctionAlgebraCompactSpace}. Suppose that $X,Y$ are suitable spaces, then $X \cong\cont(X) \cong \cont(Y)\cong Y$.
\end{proof}

\begin{theorem}[Non-unital Gelfand-Naimark] \label{nonUnitalGelfandNaimarkCommutative}
Let $A$ be a commutative $C^*$-algebra. Then the Gelfand transform
\[ \evalMap_{-}|_{\hat{A}}: A\to\cont_0(\hat{A}): x\mapsto \evalMap_{x} \]
is an isometric $*$-isomorphism.
\end{theorem}
\begin{proof}
By \ref{GelfandNaimarkCommutative}, the function $\evalMap_{-}|_{\hat{A^\dagger}}: A^\dagger\to\cont(\hat{A^\dagger}): x\mapsto \evalMap_{x}$ is an isometric $*$-isomorphism. Let $h: \hat{A}^\dagger \to \widehat{A^\dagger}$ be the homeomorphism from \ref{compactificationOfCharacterSpaceIsCharacterSpaceOfUnitisation}, so $h^\star$ is an algebra $*$-isomorphism. Let $\Phi: \cont_0(\hat{A}^\dagger) \to \cont(\hat{A}^\dagger)$ be the $*$-isomorphism from \ref{unitisationOnePointCompactificationIsomorphism}. Then
\[ \Phi^{-1}\circ h^\star\circ \big(\evalMap|_{\widehat{A^\dagger}}\big): A^\dagger \to \cont(\widehat{A^\dagger}) \to \cont(\hat{A}^\dagger) \to \cont_0(\hat{A})^\dagger \]
is a $*$-isomorphism by construction.

Now we claim that $\Phi^{-1}\circ h^\star\circ \big(\evalMap|_{\widehat{A^\dagger}}\big) = \big(\evalMap_{-}|_{\hat{A}}\big)^\dagger$. Indeed, for all $(a, \lambda)\in A^\dagger$, we have
\begin{align*}
\Big(\Phi^{-1}\circ h^\star\circ \big(\evalMap|_{\widehat{A^\dagger}}\big)\Big)(a, \lambda) &= \Phi^{-1}\big(\evalMap_{(a, \lambda)}|_{\widehat{A^\dagger}}\circ h\big) \\
&= \Big(\big(\evalMap_{(a, \lambda)}|_{\widehat{A^\dagger}}\circ h\big)|_{\hat{A}} - \constant{\big(\evalMap_{(a, \lambda)}|_{\widehat{A^\dagger}}\circ h\big)(\infty)}, \big(\evalMap_{(a, \lambda)}|_{\widehat{A^\dagger}}\circ h\big)(\infty)\Big) \\
&= \Big(\big(\evalMap_{(a, \lambda)}|_{\widehat{A^\dagger}}\circ h\big)|_{\hat{A}} - \constant{\evalMap_{(a, \lambda)}|_{\widehat{A^\dagger}}(\proj_2)}, \evalMap_{(a, \lambda)}|_{\widehat{A^\dagger}}(\proj_2)\Big) \\
&= \Big(\big(\evalMap_{(a, \lambda)}|_{\widehat{A^\dagger}}\circ h\big)|_{\hat{A}} - \constant{\lambda}, \lambda\Big) \\
&= \Big(\evalMap_{(a, \lambda)}|_{\widehat{A^\dagger}}\circ \widetilde{(-)} - \constant{\lambda}, \lambda\Big) \\
&= \Big(\evalMap_{a}|_{\hat{A}} + \constant{\lambda} - \constant{\lambda}, \lambda\Big) \\
&= \big(\evalMap_{a}|_{\hat{A}}, \lambda\big) \\
&= \big(\evalMap_{-}|_{\hat{A}}\big)^\dagger(a,\lambda).
\end{align*}
Thus the restriction $\evalMap_{-}|_{\hat{A}}$ is a $*$-isomorphism and it is isometric by \ref{starHomomorphismCstarProperties}.
\end{proof}



\begin{proposition}
If $A$ (commutative) generated by one element $a$, then $A$ is isomorphic to the $C^*$-algebra of continuous functions on the spectrum of $a$ which vanish at $0$.
\end{proposition}

TODO non-unital Gelfand-Naimark!

\subsection{Continuous functional calculus}

\begin{lemma} \label{generatedAlgebraSpectrumHomeomorphism}
Let $A$ be a unital $C^*$-algebra and $x\in A$ a normal element. Then the map
\[ \evalMap_x|_{\widehat{C^*(x,\vec{1})}}: \widehat{C^*(x,\vec{1})}\to \spec(x): \varphi \mapsto \varphi(x) \]
is a homeomorphism.
\end{lemma}
\begin{proof}
By \ref{spectrumIndependentOfSurroundingAlgebra} the $\spec_A(x) = \spec_{C^*(x,\vec{1})}$. As this is the only part of the function definition that could have depended on the rest of $A$, not just $C^*(x,\vec{1})$, we may take $A = C^*(x,\vec{1})$ WLOG. Because $x$ is normal, $C^*(x,\vec{1})$ is commutative (TODO ref: closure of commutative subset commutative?). By \ref{spectrumFromSpectrum} the map is surjective and well-defined, in that it maps into the codomain $\sigma(x)$. It is also continuous by definition of the weak-$*$ topology on $\widehat{C^*(x,\vec{1})}$.

To show injectivity, suppose $\varphi(x) = \psi(x)$ for some $\varphi,\psi\in\widehat{C^*(x,\vec{1})}$. Because (TODO ref)
\[ C^*(x,\vec{1}) = \overline{\Span}\setbuilder{x^n(x^*)^m}{n,m\geq 0} \]
and characters are continuous homomorphisms, \ref{charactersUnital}, we see that $\varphi = \psi$.

Finally we need to show the map is open. Because $\sigma(x)$ is compact, this follows from \ref{compactToHausdorffHomeomorphism}.
\end{proof}


TODO: relocate.
\begin{lemma} \label{WeierstrassApproximation}
Let $D\subseteq \C$. Then
\[ \cont(D) = C^*(\id_D, \constant{1}_D). \]
\end{lemma}
\begin{proof}
TODO ref to Stone-Weierstrass.
\end{proof}

\begin{theorem}[Continuous functional calculus for unital $C^*$-algebras] \label{continuousFunctionalCalculus}
Let $A$ be a unital $C^*$-algebra and $x\in A$ a normal element. There exists a unique $*$-homomorphism
\[ \Phi_x: \cont(\spec(x))\to A \qquad \text{such that} \qquad \text{$\Phi_x(\id_{\spec(x)}) = x$ and $\Phi_x(\constant{1}_{\spec(x)}) = \vec{1}$.} \]
This $*$-homomorphism is isometric.
\end{theorem}
We will usually write $f(x)$ to mean $\Phi_x(f)$.
\begin{proof}
For uniqueness, note that $\cont(\spec(x)) = C^*(\id_{\spec(x)}, \constant{1}_{\spec(x)})$ by the Weierstrass approximation theorem \ref{WeierstrassApproximation}. Thus any $*$-homomorphism defined on $\cont(\spec(x))$ is completely determined by its value at $\id_{\spec(x)}$ and $\constant{1}_{\spec(x)}$.

For existence, let $B \defeq C^*(x,\vec{1})$, which is commutative because $x$ is normal. Then $\evalMap_x: \hat{B}\to \sigma(x): \varphi\mapsto \varphi(x)$ is a homeomorphism by \ref{generatedAlgebraSpectrumHomeomorphism} and so the pre-composition $\evalMap_x^\star: \mathcal{C}(\sigma(x))\to \mathcal{C}(\hat{B})$ is an isometric $*$-isomorphism. The we have the $*$-homomorphism
\[ \Phi_x \defeq \iota_B\circ\big(\evalMap_{-}|_{\hat{B}}\big)^{-1}\circ \evalMap_x^\star: \begin{tikzcd} \mathcal{C}(\sigma(x)) \ar[r, "{\evalMap_x^\star}"] & \mathcal{C}(\hat{B}) \ar[r, "\big(\evalMap_{-}|_{\hat{B}}\big)^{-1}"] & B \ar[r , hook, "\iota_B"] & A \end{tikzcd} \]
where the inverse Gelfand transform $\big(\evalMap_{-}|_{\hat{B}}\big)^{-1}$ exists and is an isometric isomorphism by the Gelfand-Naimark theorem \ref{GelfandNaimarkCommutative}, since $B$ is commutative.

We verify
\begin{align*}
\Phi_x(\id_{\sigma(x)}) &= \Big(\iota_B\circ\big(\evalMap_{-}|_{\hat{B}}\big)^{-1}\circ\evalMap_x^\star\Big)(\id_{\sigma(x)}) \\
&= (\iota_B\circ\big(\evalMap_{-}|_{\hat{B}}\big)^{-1})(\id_{\sigma(x)}\circ\evalMap_x) \\
&= (\iota_B\circ\big(\evalMap_{-}|_{\hat{B}}\big)^{-1})(\evalMap_x) \\
&= \iota_B(x) = x
\end{align*}
using the definition of the transpose, cancellation of $I_{\sigma}(x)$ and inverse of Gelfand transform. We also verify
\begin{align*}
\Phi_x(\constant{1}_{\sigma(x)}) &= (\iota_B\circ\big(\evalMap_{-}|_{\hat{B}}\big)^{-1}\circ\evalMap_x^\star)(\constant{1}_{\sigma(x)}) \\
&= (\iota_B\circ\big(\evalMap_{-}|_{\hat{B}}\big)^{-1})(\constant{1}_{\sigma(x)}\circ\evalMap_x) \\
&= (\iota_B\circ\big(\evalMap_{-}|_{\hat{B}}\big)^{-1})(\constant{1}_{\hat{B}}) \\
&= \iota_B(\vec{1}) = \vec{1}
\end{align*}
using the fact that $\evalMap_{\vec{1}}|_{\hat{B}}(\varphi) = \varphi(\vec{1}) = 1$ for all $\varphi\in \hat{B}$ by \ref{charactersUnital}, so $\evalMap_{\vec{1}}|_{\hat{B}} = \constant{1}_{\hat{B}}$.
\end{proof}
For polynomials in $z,\overline{z}$, this functional calculus works as expected, because it is a $*$-homomorphism.

In fact we can apply functional calculus to any continuous defined on a superset of the spectrum: restricting to the spectrum still yields a continuous function by \ref{continuousConstructions}.

\begin{proposition} \label{compositionPropertiesCFC}
Let $A$ be a unital $C^*$-algebra and $x\in A$ be a normal element with functional calculus $\Phi_x$. Then
\begin{enumerate}
\item if $B$ is a unital $C^*$-algebra and $\Psi: A\to B$ a unital $*$-homomorphism, then $\Psi\circ\Phi_x = \Phi_{\Psi(x)}$;
\item if $f\in\cont(\spec(x))$ and $g\in\cont\big(f^{\imf}(\spec(x))\big)$, then $\Phi_x(g\circ f) = \Phi_{\Phi_x(f)}(g)$.
\end{enumerate}
\end{proposition}
The first point means that $\Psi(f(x)) = f(\Psi(x))$ for all $f\in \cont(\spec(x))$. The second that $(g\circ f)(x) = g\big(f(x)\big)$.
\begin{proof}
(1) The claim is well-defined because $\spec(\Psi(x))\subseteq \spec(x)$, by \ref{spectrumOfImage}. It is easy to check both sides are unital $*$-homomorphisms from $\cont(\spec(x))$ to $B$, sending the identity function to $\Psi(x)$. The claim then follows from uniqueness of the functional calculus \ref{continuousFunctionalCalculus}.

(2) We need to prove $\Phi_x\circ f^\star = \Phi_{\Phi_x(f)}$. We calculate
\[ (\Phi_x\circ f^\star)(\id_{f^\imf(\spec(x))}) = \Phi_x(\id_{f^\imf(\spec(x))}\circ f) = \Phi_x(f) \qquad\text{and}\qquad (\Phi_x\circ f^\star)(\constant{1}_{f^\imf(\spec(x))}) = \Phi_x(\constant{1}_{f^\imf(\spec(x))}\circ f) = \Phi_x(\constant{1}_{\spec(x)}) = \vec{1}, \]
so the equality holds by uniqueness of the continuous functional calculus \ref{continuousFunctionalCalculus}.
\end{proof}


\begin{theorem}[Continuous functional calculus for non-unital $C^*$-algebras] \label{nonUnitalContinuousFunctionalCalculus}
Let $A$ be a $C^*$-algebra and $x\in A$ a normal element. There exists a unique $*$-homomorphism
\[ \Phi'_x: \setbuilder{f\in \cont\big(\spec(x)\big)}{f(0) = 0} \to A \qquad \text{such that $\Phi'_x(\id_{\spec(x)}) = x$.} \]
Also $\Phi'_x$ is isometric.
\end{theorem}
\begin{proof}
We first show existence. By \ref{continuousFunctionalCalculus}, there exists a isometric $*$-homomorphism $\Phi_x: \cont\big(\spec(x)\big)\to A^\dagger$ that maps $\id_{\spec(x)}$ to $x$. We can then take $\Phi'_x = \proj_1\circ \Phi_x|_{\setbuilder{f\in \cont\big(\spec(x)\big)}{f(0) = 0}}$. By \ref{unitalProjectionsAlgebraHomomorphisms} this is an isometric algebra homomorphism if $\proj_2\big(\Phi_x(f)\big) = 0$ for all $f\in \cont\big(\spec(x)\big)$ such that $f(0) = 0$. Indeed, by \ref{compositionPropertiesCFC} (and the fact that $\proj_2$ is unital \ref{unitalProjectionsAlgebraHomomorphisms}), 
\[ \proj_2\big(\Phi_x(f)\big) = \Phi_{\proj_2(x)}(f) = \Phi_{0}(f) = f(0) = 0. \]
For uniqueness, note that the unitisation $(\Phi'_x)^\dagger = \cont\big(\spec(x)\big)\to A^\dagger$ is a continuous functional calculus for $A^\dagger$ and thus unique by \ref{continuousFunctionalCalculus}. This means that the restriction is also unique.
\end{proof}


\begin{proposition} \label{compositionPropertiesCFC}
Let $A$ be a unital $C^*$-algebra and $x\in A$ be a normal element with functional calculus $\Phi_x$. Then
\begin{enumerate}
\item if $B$ is a unital $C^*$-algebra and $\Psi: A\to B$ a unital $*$-homomorphism, then $\Psi\circ\Phi_x = \Phi_{\Psi(x)}$;
\item if $f\in\cont(\spec(x))$ and $g\in\cont\big(f^{\imf}(\spec(x))\big)$, then $\Phi_x(g\circ f) = \Phi_{\Phi_x(f)}(g)$.
\end{enumerate}
\end{proposition}
The first point means that $\Psi(f(x)) = f(\Psi(x))$ for all $f\in \cont(\spec(x))$. The second that $(g\circ f)(x) = g\big(f(x)\big)$.
\begin{proof}
(1) The claim is well-defined because $\spec(\Psi(x))\subseteq \spec(x)$, by \ref{spectrumOfImage}. It is easy to check both sides are unital $*$-homomorphisms from $\cont(\spec(x))$ to $B$, sending the identity function to $\Psi(x)$. The claim then follows from uniqueness of the functional calculus \ref{continuousFunctionalCalculus}.

(2) We need to prove $\Phi_x\circ f^\star = \Phi_{\Phi_x(f)}$. We calculate
\[ (\Phi_x\circ f^\star)(\id_{f^\imf(\spec(x))}) = \Phi_x(\id_{f^\imf(\spec(x))}\circ f) = \Phi_x(f) \qquad\text{and}\qquad (\Phi_x\circ f^\star)(\constant{1}_{f^\imf(\spec(x))}) = \Phi_x(\constant{1}_{f^\imf(\spec(x))}\circ f) = \Phi_x(\constant{1}_{\spec(x)}) = \vec{1}, \]
so the equality holds by uniqueness of the continuous functional calculus \ref{continuousFunctionalCalculus}.
\end{proof}

\begin{proposition}[Spectral mapping] \label{spectralMappingCFC}
Let $A$ be a unital $C^*$-algebra and $x\in A$ be a normal element and $f\in\mathcal{C}(\sigma(x))$. Then
\[ \sigma(f(x)) = f^\imf(\sigma(x)). \]
\end{proposition}
\begin{proof}
Let $B = C^*(x,\vec{1})$, which is commutative because $x$ is normal. We calculate
\[ \sigma(f(x)) = \setbuilder{\varphi(f(x))}{\varphi\in\hat{B}} = \setbuilder{f(\varphi(x))}{\varphi\in\hat{B}} = f(\sigma(x)). \]
using \ref{spectrumFromSpectrum} and \ref{compositionPropertiesCFC}.
\end{proof}


\begin{proposition} \label{commutativityFunctionalCalculus}
Let $A$ be a unital $C^*$-algebra and $x\in A$ be a normal element. Let $f$ be a continuous function on the spectrum of $x$ and let $y\in A$ commute with $x$. Then $f(x)$ commutes with $y$.
\end{proposition}
TODO proof + restructure + define joint functional calculus? (but we still need case where $y$ is not necessarily normal)


\begin{lemma}
Let $X$ be a compact space and view $\mathcal{C}(X)$ as a unital commutative $C^*$-algebra. Fix $g\in\mathcal{C}(X)$. The functional calculus is then given simply by composition:
\[ \mathcal{C}(g[X]) \to \mathcal{C}(X): f\mapsto f(g) = f\circ g. \]
\end{lemma}
\begin{proof}
Composition is a unital $*$-homomorphism with the right properties. The claim follows from uniqueness of the functional calculus.
\end{proof}


\begin{proposition} \label{continuityContinuousFunctionalCalculus}
Let $K\subset \R$ be non-empty and compact; $f:K\to \C$ a continuous function; $A$ a unital $C^*$-algebra and $\Omega_K$ the set of self-adjoint elements in $A$ with spectrum contained in $K$. The function
\[ f: \Omega_K\subset A \to A: a\mapsto f(a) \]
is continuous.
\end{proposition}
\begin{proof}
The map $A\to A$ given by $a\mapsto a^n$ is continuous for every $n\geq 0$, because multiplication is continuous. Then every polynomial $f$ induces a continuous map $A\to A$.
Then $\epsilon/3$ by Stone-Weierstrass.
\end{proof}
TODO: also for non-compact $K$ and $K\subseteq \C$. Then $\Omega_K$ set of normal elements.

\begin{proposition}
Let $A$ be a unital $C^*$-algebra, $x\in A$ a normal element and $f\in \cont(\sigma(x))$. Then
\[ \norm{f(x)} = \sup_{\lambda\in \sigma(x)}|f(\lambda)|. \]
\end{proposition}
\begin{proof}
We calculate
\[ \norm{f(x)} = \spr(f(x)) = \sup |\sigma(f(x))| = \sup |f(\sigma(x))| = \sup_{\lambda\in \sigma(x)}|f(\lambda)|. \]
\end{proof}
\begin{corollary}
Let $x$ be an normal element in a unital $C^*$-algebra and $\lambda_0 \in \rho(x)$. Then $d(\lambda_0, \sigma(x)) = \norm{R_x(\lambda_0)}^{-1}$.
\end{corollary}
\begin{proof}
We calculate
\[ \norm{R_x(\lambda_0)} = \norm{\frac{1}{x-\lambda_0\cdot \vec{1}}} = \sup_{\lambda\in \sigma(x)}\left|\frac{1}{\lambda - \lambda_0}\right| = \frac{1}{\inf_{\lambda\in \sigma(x)}|\lambda- \lambda_0|} = \frac{1}{d(\lambda_0, \sigma(x))}. \]
\end{proof}
TODO: for general closed operators $d(\lambda_0, \sigma(x)) \geq \norm{R_x(\lambda_0)}^{-1}$??

\begin{lemma}
If two polynomials in $z,\overline{z}$ agree on the spectrum of a normal element, they give an equation the element obeys.
\end{lemma}
The proof is the unicity of the functional calculus.
\begin{corollary} \label{propertiesFromSpectrum}
Let $A$ be a unital $C^*$-algebra and $x\in A$ a normal element. Then
\begin{enumerate}
\item $x$ is self-adjoint \textup{if and only if} $\sigma(x)\subseteq \R$;
\item $x$ is unitary \textup{if and only if} $\sigma(x)\subseteq \mathbb{T}$;
\item $x$ is a projection \textup{if and only if} $\sigma(x)\subseteq \{0,1\}$.
\end{enumerate}
\end{corollary}
\begin{proof}
\hspace{1em} TODO spectral mapping
\begin{enumerate}
\item By \ref{selfAdjointSpectrumReal} we have that self-adjoint implies real spectrum. The converse follows from the lemma applied to $z = \overline{z}$.
\item Assume $x$ unitary. By the $C^*$-identity $\norm{x} = \sqrt{\norm{x^*x}} = 
\sqrt{\norm{\vec{1}}} = 1$, by \ref{consequencesC*}. By \ref{normNormalElement}, $\spr(x) = \norm{x} = 1$. Also $\norm{x^{-1}}^{-1} = 1$. So $\sigma(x)\subseteq \mathbb{T}$ by \ref{openSetInvertibles}. The converse follows from the lemma applied to $1 = z\overline{z} = \overline{z}z$.
\item Assume $x$ a projection. Then $x-\lambda$ has an inverse given by $-\lambda^{-1} + (1 - \lambda)^{-1}\lambda^{-1}x$ if $\lambda \notin \{0,1\}$:
\[ (x-\lambda)\left( - \frac{1}{\lambda} + \frac{x}{(1-\lambda)\lambda} \right) = \vec{1} - \frac{x}{\lambda} + \frac{x-x\lambda}{(1-\lambda)\lambda} =  \vec{1} - \frac{x}{\lambda} + \frac{x}{\lambda} = \vec{1}. \]
The converse follows from the lemma applied to $z = \overline{z} = z^2$.
\end{enumerate}
\end{proof}



\begin{proposition}
Let $A$ be a unital $C^*$-algebra. Then every element in $A$ can be written as a linear combination of at most four unitaries.
\end{proposition}
\begin{proof}
Let $x\in A$. By \ref{realImaginaryParts} we can write $x=x_1+ix_2$ for some self-adjoint $x_1,x_2$.
TODO
\end{proof}

\subsection{Constructions}
\subsubsection{Pseudoinverse}
\begin{definition}
Let $A$ be a unital $C^*$-algebra and $x\in A$ a normal element. Then the \udef{pseudo-inverse} of $x$ is 
\[ x^+ \defeq f(x) \qquad\text{where}\qquad f: y \mapsto \begin{cases}
y^{-1} & (y \neq 0) \\ 0 & (y = 0).
\end{cases} \]
\end{definition}

\section{Positivity}
\url{https://link.springer.com/content/pdf/10.1023/A:1009717500980.pdf}
\subsection{Positive elements}
\begin{definition}
Let $A$ be a $C^*$-algebra. An element $a\in A$ is \udef{positive} if it is normal and $\sigma(a) \subset \interval[co]{0,+\infty}$. We write $a\geq 0$.

The set of all positive elements of $A$ is the \udef{positive cone} of $A$
\[ A^+ \defeq \setbuilder{a\in A}{a\geq 0}. \]
\end{definition}
By \ref{propertiesFromSpectrum} every projection is positive and every positive element is in fact self-adjoint.

By \ref{normalSpectralRadiusEqualsNorm} we have for all positive $a\in A$:
\[ \norm{a} = \sup\sigma(a) = \max\sigma(a). \]

By definition, $A^+ = \widetilde{A}^+ \cap A$.

\begin{proposition}
Let $A$ be a unital $C^*$-algebra and $a\in A$ a self-adjoint element. Then
\[ a\in A^+ \qquad\iff\qquad \exists b\in \SelfAdjoints(A):\quad a = b^2. \]
If we further require $b$ to be positive, then it is unique.
\end{proposition}
\begin{proof}
Let $a\in A^+$. Then define $b = \sqrt{a}$ by spectral calculus. Then we have
\[ b^2 = (\Phi_a(\sqrt{\mbox{\;\;}}))^2 = \Phi_a(\sqrt{\mbox{\;\;}})\cdot \Phi_a(\sqrt{\mbox{\;\;}}) = \Phi_a(\sqrt{\mbox{\;\;}}^2) = \Phi_a(I_{\sigma(a)}) = a. \]
Converse by spectral mapping \ref{spectralMappingCFC}.
\end{proof}

\begin{lemma} \label{positivityDistanceToNorm}
Let $A$ be a unital $C^*$-algebra and $a\in A$ self-adjoint. Then the following are equivalent:
\begin{enumerate}
\item $a$ is positive;
\item $\norm{\frac{1}{2}\vec{1} - a / \norm{a}} \leq \frac{1}{2}$;
\item $\norm{r\vec{1} - a / \norm{a}} \leq r$ for all $r\geq 1/2$.
\item $\norm{r\vec{1} - a / \norm{a}} \leq r$ for some $r\geq 1/2$.
\end{enumerate}
\end{lemma}
\begin{proof}
The proof is cyclic:

$(1.\Rightarrow 2.)$ Assume $a$ positive. Then $\sigma(a) \subseteq [0,\norm{a}]$. By spectral mapping, \ref{spectralMappingCFC}, we have $\spec(\frac{1}{2}\vec{1} - a / \norm{a}) \subseteq [-1/2, 1/2]$ and thus $\norm{\frac{1}{2}\vec{1} - a / \norm{a}} \leq \frac{1}{2}$, by \ref{normNormalElement}.

$(2.\Rightarrow 3.)$ Write $r = 1/2 + r'$, so $r'\geq 0$. Then
\[ \norm{r\vec{1} - \frac{a}{\norm{a}}} = \norm{\frac{1}{2}\vec{1} + r'\vec{1} - \frac{a}{\norm{a}}} \leq \norm{r'\vec{1}}+ \norm{\frac{1}{2}\vec{1} - \frac{a}{\norm{a}}} \leq r' + \frac{1}{2} = r. \]

$(3.\Rightarrow 4.)$ Clear.

$(4.\Rightarrow 1.)$ By \ref{normNormalElement}, $\sigma(r\vec{1} - a / \norm{a}) \subseteq [-r, r]$. By spectral mapping, this means $\sigma(a) \subseteq [0, 2r\norm{a}]$ and thus $a$ is positive.
\end{proof}
\begin{corollary} \label{CstarPositiveConeClosed}
Let $A$ be a $C^*$-algebra. The positive cone $A^+$ is closed.
\end{corollary}
\begin{proof}
It is enough to prove this for unital $C^*$-algebras since $A^+ = \widetilde{A}^+ \cap A$.

Set $g: A\setminus\{0\}\to \R^+: a\mapsto \norm{\vec{1} - a/\norm{a}\big.}$, which is continuous by construction. By the proposition, a non-zero self-adjoint vector $a$ is positive iff $g(a) \in \interval{0,1}$. So the set of positive non-zero vectors is $X \defeq g^\preimf\big(\interval{0,1}\big) \cap \SelfAdjoints(A)$. This is a closed subset of $A\setminus\{0\}$ by \ref{preimageOpenClosed}, \ref{closedSubsetsBanachStarAlgebra} and \ref{propertiesTopology}.

By \ref{subspaceAdherence}, we have, by \ref{propertiesTopology},
\[ A^+\setminus\{0\} = \closure_{A\setminus\{0\}}\big(A^+\setminus\{0\}\big) = \closure_{A}\big(A^+\setminus\{0\}\big)\cap \big(A\setminus\{0\}\big) = \closure_{A}\big(A^+\setminus\{0\}\big)\setminus \{0\}. \]
Since $0\in A^+$ (because $\spec(0) = \{0\}$), we have
\begin{align*}
A^+ &= \big(A^+ \setminus \{0\}\big) \cup \{0\} = \Big(\closure_{A}\big(A^+\setminus\{0\}\big)\setminus \{0\}\Big) \cup \{0\} \\
&= \closure_{A}\big(A^+\setminus\{0\}\big) \cup \closure_A\big(\{0\}\big) = \closure_{A}\big(A^+\setminus\{0\} \cup \{0\}\big) = \closure_A(A^+).
\end{align*}
Thus $A^+$ is closed.
\end{proof}

\begin{proposition}\label{existenceSquareRoot}
Let $A$ be a $C^*$-algebra and $a\in A$. Then $a$ is positive \textup{if and only if} $a = b^*b$ for some $b\in A$.
\end{proposition}
\begin{proof}
TODO!!!!
\end{proof}
TODO: link with $\sqrt{a}$?
\begin{corollary}
Every positive element is self-adjoint (and, in particular, normal).
\end{corollary}
\begin{corollary}
Let $A$ be a concrete $C^*$-algebra of bounded operators on some Hilbert space $\mathcal{H}$ and $a\in A$. Then $a$ is positive as an element of the $C^*$ algebra \textup{if and only if} $a$ is positive as an operator on $\mathcal{H}$. for all $x\in \mathcal{H}: \inner{x,ax}\geq 0$.
\end{corollary}
\begin{proof}
If $a$ is positive, then $a = b^*b$ and thus
\[ \forall x\in \mathcal{H}: \inner{x,ax} = \inner{x, b^*bx} = \inner{bx,bx} = \norm{bx}^2 \geq 0, \]
meaning $a$ is a positive operator.

Conversely, the spectrum is contained in the closure of the numerical range (TODO ref), which is a subset of $\interval[co]{0,\infty}$.
\end{proof}
Consequently if $\inner{x,ax}\geq 0$ for all $x\in\mathcal{H}$, then $a$ is self-adjoint.

\begin{lemma} \label{selfAdjointUnitSphereFromOrder}
Let $A$ be a unital $C^*$-algebra and $a\in \SelfAdjoints(A)$. Then $\norm{a} \leq 1$ \textup{if and only if} $\{\vec{1}+a, \vec{1}- a\} \subseteq A^+$. 
\end{lemma}
\begin{proof}
By \ref{normNormalElement}, we have $\norm{a} = \spr(a)$, so, by spectral mapping \ref{spectralMappingCFC},
\begin{align*}
\norm{a}\leq 1 \iff \spec(a) \subseteq \interval{-1,1} \quad&\iff\quad \big(\spec(a) \subseteq \interval{-1,+\infty}\big) \land \big(\spec(a) \subseteq \interval{-\infty,1}\big) \\
&\iff \quad \big(\spec(\vec{1} + a) \subseteq \interval{0,+\infty}\big) \land \big(\spec(a - \vec{1}) \subseteq \interval{-\infty,0}\big) \\
&\iff \quad \big(\spec(\vec{1} + a) \subseteq \interval{0,\infty}\big) \land \big(\spec(\vec{1}-a) \subseteq \interval{0, \infty}\big) \\
&\iff \quad \big(\vec{1} + a\in A^+\big) \land \big(\vec{1}-a\in A^+\big).
\end{align*}
\end{proof}

\begin{lemma} \label{orderSubUnitVectorLemma}
Let $A$ be a $C^*$-algebra and $a, b\in A^+$ with $\norm{a}\leq 1$ and $\norm{b}\leq 1$. Then
\begin{enumerate}
\item $a^2 \leq a$;
\item $\norm{a-b} \leq 1$.
\end{enumerate}
\end{lemma}
\begin{proof}
(1) Since the function $x\mapsto x-x^2$ is positive on $\interval{0,1}$ and $\spec(a) \subseteq \interval{0,1}$, we have that $a - a^2$ is positive by spectral mapping \ref{spectralMappingCFC}.

(2) Since $\norm{a}_{\widetilde{A}} = \norm{a} \leq 1$ and $\norm{b}_{\widetilde{A}} = \norm{b} \leq 1$, we have $0\leq \vec{1}-a$ and $0\leq \vec{1}-b$ from \ref{selfAdjointUnitSphereFromOrder}. Thus $0\leq \vec{1}-a+b = \vec{1} - (a-b)$ and $0\leq \vec{1} -b +a = \vec{1}+ (a-b)$. Then $\norm{a-b} = \norm{a-b}_{\widetilde{A}} \leq 1$ by \ref{selfAdjointUnitSphereFromOrder}.
\end{proof}

\subsection{Partial order on self-adjoint elements}
\begin{proposition}
Let $A$ be a $C^*$-algebra. The set $A^+$ is a salient pointed convex cone.
\end{proposition}
\begin{proof}
That $A^+$ is a cone follows from spectral mapping (\ref{spectralMappingCFC}), as does the salience of $A^+$.

For convexity, we verify additive closure (see \ref{convexityAdditiveClosure}) Take $a,b\in A$ and set $p= \frac{\norm{a}+\norm{b}}{\norm{a+b}} \geq 1 \geq 1/2$. Then
\[ \norm{p\vec{1} - \frac{a+b}{\norm{a+b}}} \leq \frac{1}{\norm{a+b}}\left(\norm{\norm{a}-a} + \norm{\norm{b}-b}\right) \leq \frac{\norm{a}+\norm{b}}{\norm{a+b}} = p \]
using the triangle inequality and point 3. of \ref{positivityDistanceToNorm} with $r=1$.
\end{proof}
By \ref{positiveCone}, we have:
\begin{corollary}
The relation $\leq$ defined by
\[ a\leq b \qquad\iff\qquad b-a\in A^+ \]
is a vector partial order on $A$.
\end{corollary}
TODO in general $\sSet{A, \leq}$ is not a Riesz space. (See e.g.\ the absolute value)

\begin{lemma}
If $\alpha,\beta\in\R$ and $\alpha\leq \beta$, then $\alpha\vec{1}\leq \beta\vec{1}$.
\end{lemma}

\begin{lemma}
Let $0\leq a \leq b$ and $a$ invertible, then $b$ is invertible and $0\leq a^{-1}\leq b^{-1}$.
\end{lemma}

\begin{lemma} \label{CstarOrderLemma}
Let $A$ be a $C^*$-algebra, $x\leq y$ and $a,b\in A$. Then
\begin{enumerate}
\item $a^*xa \leq a^*ya$;
\item $0\leq b^*a^*ab \leq \norm{a}^2b^*b$.
\end{enumerate}
\end{lemma}
\begin{proof}
(1) By \ref{existenceSquareRoot}, there exists $c\in A$ such that $y-x = c^*c$. Then
\[ 0 \leq (ac)^*ac = a^*c^*ca = a^*(y-x)a = a^*ya - a^*xa. \]
This implies $a^*xa \leq a^*ya$.

(2) Since $\norm{a^*a} = \norm{a}^2$, we have $\norm{\frac{a^*a}{\norm{a}^2}} \leq 1$ and thus $\frac{a^*a}{\norm{a}^2} \leq \vec{1}$ by \ref{selfAdjointUnitSphereFromOrder}. Point (1) then implies $b^*\frac{a^*a}{\norm{a}^2}b \leq b^*\vec{1}b = b^*b$, so $b^*a^*ab \leq \norm{a}^2b^*b$. We also have $0\leq (ab)^*ab = b^*a^*ab$.
\end{proof}


\subsubsection{Lattice properties of self-adjoint operators}
\url{https://www.ams.org/journals/proc/1951-002-03/S0002-9939-1951-0042064-2/S0002-9939-1951-0042064-2.pdf}

\begin{proposition}
Let $A$ be a $C^*$-algebra. The real vector space of self-adjoint operators $\SelfAdjoints(A)$ is a Riesz space \textup{if and only if} $A$ is commutative.
\end{proposition}
\begin{proof}
TODO
\end{proof}
\begin{corollary} \label{positiveNegativeParts}
Let $A$ be a unital $C^*$-algebra, and $a\in A$ a self-adjoint element. Then there exists a unique decomposition $a=a^+ - a^-$ where $a^+,a^-\in A^+$ and $a^+a^- = 0$.
\end{corollary}
\begin{proof}
TODO
\end{proof}
The corollary \ref{positiveNegativeParts} can also be proved using functional calculus, by setting $a^+ = f^+(a)$ and $a^- = f^-(a)$ where
\[ f^+: x\mapsto \begin{cases}
x & (x\geq 0)\\ 0 & (x < 0)
\end{cases} \qquad\text{and}\qquad f^-: x\mapsto \begin{cases}
0 & (x\geq 0)\\ -x & (x < 0)
\end{cases} \]

\begin{corollary}[Cartesian decomposition] \label{CartesianDecomposition}
Let $A$ be a $C^*$-algebra and $a\in A$. Then we can decompose $a$ as
\[ (p_1-p_2) + i(p_3-p_4) \]
where $p_1,p_2,p_3,p_4$ are positive.
\end{corollary}
\begin{proof}
By \ref{realImaginaryParts}.
\end{proof}
Thus $\Span_\C(A^+) = A$.

\begin{proposition}
The set of all bounded self-adjoint operators on a Hilbert space is an anti-lattice.
\end{proposition}

\subsubsection{Operator monotonicity}
Delicate!
\begin{proposition}
Assume $0\leq a\leq b$. Then
\begin{enumerate}
\item $\sqrt{a}\leq \sqrt{b}$;
\item $0\leq ab$ if $a,b$ commute.
\end{enumerate}
\end{proposition}
It is not true that $a^2\leq b^2$ or that $ab$ is positive in general!

\subsubsection{Increasing approximate units}
\begin{proposition} \label{approximateUnitInIdeal}
Let $A$ be a $C^*$-algebra and $J \subseteq A$ a two-sided $*$-ideal. Then there exists a net $(e_\lambda)_{\lambda\in\Lambda}$ in $J$ that that is an increasing approximate unit of $\overline{J}$.
\end{proposition}
\begin{proof}
See Dana Williams
\end{proof}

\subsection{Absolute value}
\begin{definition}
Let $A$ be a unital $C^*$-algebra. For each $a\in A$, the \udef{absolute value} of $a$ is $|a| = (a^*a)^{1/2}$.
\end{definition}
The square root is well defined using functional calculus on the self-adjoint element $a^*a$.

\begin{lemma} \label{propertiesAbsoluteValue}
Let $A$ be a unital $C^*$-algebra.
\begin{enumerate}
\item Let $u\in\Unitaries$, then $|u| = \vec{1}$.
\item Let $a\in A$, then $|a|$ is positive and thus self-adjoint.
\item The map $a\mapsto |a|$ is continuous.
\item If $b\in \GL(A)$, then $|b|\in \GL(A)$.
\end{enumerate}
\end{lemma}
\begin{proof}
(1) We have $|u| = (u^*u)^{1/2} = \vec{1}^{1/2} = \vec{1}$.

(2) This follows from spectral mapping \ref{spectralMappingCFC} using the fact that $z\mapsto \overline{z}z$ has positive image in $\C$.

(3) We have that $a\mapsto a^*a$ is continuous by \ref{multiplicationContinuous} and \ref{consequencesC*}. Then $a\mapsto |a|$ is continuous by \ref{continuityContinuousFunctionalCalculus}.

(4) Suppose $b\in \GL(A)$. Then $b^*\in \GL(A)$ by \ref{elementaryStarLemma}, so $b^*b$ is invertible. Thus $0\notin \spec(b^*b)$, which implies $0\notin \spec(\sqrt{b^*b}) = \spec(|b|)$ by spectral mapping \ref{spectralMappingCFC}. Thus $|b|$ is invertible.
\end{proof}

\begin{lemma}
Let $T$ be a bounded operator on a Hilbert space $\mathcal{H}$. Then $|T|$ is the only positive operator $A$ in $\Bounded(\mathcal{H})$ such that $\norm{Ax} = \norm{Tx}$ for all $x\in\mathcal{H}$.
\end{lemma}
\begin{proof}
We have for all $x\in\mathcal{H}$,
\[ \inner{Ax,Ax} = \inner{Tx,Tx} \implies \inner{(A^*A-T^*T)x,x}=0, \]
which implies $T^*T = A^*A$. Now $A$ is positive, so $A^*A = A^2$ and taking the squared root give $A = \sqrt{T^*T} = |T|$.
\end{proof}

We do \emph{not}, in general, have a triangle inequality $|a+b| \leq |a| + |b|$.
\begin{example}
Consider the $C^*$ algebra $\C^{2\times 2}$ with
\[ A = \begin{pmatrix}
1 & 1 \\ 1 & 1
\end{pmatrix} = \frac{1}{2}\begin{pmatrix}
1 & 1 \\ -1 & 1
\end{pmatrix}\begin{pmatrix}
0 & 0 \\ 0 & 2
\end{pmatrix}\begin{pmatrix}
1 & -1 \\ 1 & 1
\end{pmatrix} = |A| \qquad\text{and}\qquad B = \begin{pmatrix}
0 & 0 \\ 0 & -2
\end{pmatrix} \]
Then
\[ A + B = \begin{pmatrix}
1 & 1 \\ 1 & -1
\end{pmatrix} = \begin{pmatrix}
1-\sqrt{2} & 1=\sqrt{2} \\ 1 & 1
\end{pmatrix}\begin{pmatrix}
-\sqrt{2} & 0 \\ 0 & \sqrt{2}
\end{pmatrix}\begin{pmatrix}
1-\sqrt{2} & 1=\sqrt{2} \\ 1 & 1
\end{pmatrix}^{-1} \]
So
\[ |A+B| = \begin{pmatrix}
\sqrt{2} & 0 \\ 0 & \sqrt{2}
\end{pmatrix} \qquad\text{and}\qquad |A| + |B| = \begin{pmatrix}
1 & 1 \\ 1 & 3
\end{pmatrix}. \]
Finally $|A| + |B| - |A+B| = \begin{pmatrix}
1 - \sqrt{2} & 1 \\ 1 & 3-\sqrt{2}
\end{pmatrix}$ is not positive:
\[ \begin{pmatrix}
1 & 0
\end{pmatrix}\begin{pmatrix}
1 - \sqrt{2} & 1 \\ 1 & 3-\sqrt{2}
\end{pmatrix}\begin{pmatrix}
1 \\ 0
\end{pmatrix} = 1-\sqrt{2} < 0. \]
\end{example}

\subsubsection{Polar decomposition}
\begin{proposition}[Polar decomposition for invertible elements]
Let $A$ be a unital $C^*$-algebra and $a\in \GL(A)$ an invertible element. Then there exists a unique decomposition
\[ a = u(a) |a| \]
such that $u(a)$ is unitary. The map $u: \GL(A) \to \Unitaries(A)$ is continuous.
\end{proposition}
\begin{proof}
If $a$ is invertible, then $|a|$ is invertible by \ref{propertiesAbsoluteValue}. Put $u(a) = a|a|^{-1}$. Clearly $a = u(a)|a|$ and $u(a)$ is unitary because it is invertible and
\[ u(a)^*u(a) = |a|^{-1}a^*a|a|^{-1} = |a|^{-1}|a|^2|a|^{-1} = \vec{1}. \]

The continuity of u: $a\mapsto a|a|^{-1}$ follows from the continuity of multiplication, \ref{multiplicationContinuous}, the continuity of the absolute value, \ref{propertiesAbsoluteValue} and the continuity of the inverse, \ref{inverseMapContinuous}.
\end{proof}

In some cases we can find a polar decomposition for non-invertible elements.
\begin{definition}
A $C^*$ algebra $A$ is said to have the \udef{polar decomposition property} if for each $a\in A$ there exists a \emph{unique} pair $(v,p)$ of elements in $A$ such that
\begin{enumerate}
\item $a = vb$;
\item $v$ is a partial isometry and $b$ is positive;
\item $a^*a = b^2$;
\item $ap = 0$ implies $vp = 0$ for all projections $p\in\Projections(A)$.
\end{enumerate}
\end{definition}
Von Neumann algebras have the polar decomposition property. (TODO ref).

TODO: Rickart $C^*$ algebras have the polar decomposition property.


\subsection{Positive maps}
\begin{definition}
Let $A,B$ be $C^*$-algebras. Then $f:A\to B$ is a \udef{positive map} if
\[ \forall x\in A: \quad x\geq 0 \implies f(x)\geq 0. \]
\end{definition}
By \ref{existenceSquareRoot}, this is equivalent to the condition that $f(x^*x)\geq 0$ for all $x\in A$.

TODO: \url{https://www-m5.ma.tum.de/foswiki/pub/M5/CQC/Masterarbeit.pdf}
\url{https://iopscience.iop.org/article/10.1088/0305-4470/34/29/308}

\begin{lemma} \label{starHomomorphismPositive}
Let $A,B$ be $C^*$-algebras and $f:A\to B$ a $*$-homomorphism. Then $f$ is positive.
\end{lemma}
\begin{proof}
Take $a\in A^+$. Then there exists $b\in A$ such that $a = b^*b$ and so $f(a) = f(b^*b) = f(b)^*f(b)$, which is positive.
\end{proof}

\subsubsection{Positive functionals}
\begin{definition}
Let $A$ be a $C^*$-algebra. A linear functional $\rho$ on $A$ is positive, written $\rho\geq 0$, if
\[ \forall x\in A: \quad x\geq 0 \implies \rho(x)\geq 0. \]
\end{definition}
Not necessarily multiplicative!

\begin{lemma} \label{positiveLinearFunctionalPreinnerProduct}
Let $\omega$ be a positive linear functional on a $C^*$-algebra $A$. Then the map
\[ \inner{\cdot, \cdot}_\omega: A\times A \to \C: (x,y)\mapsto \omega(x^*y) \]
is a pre-inner product on $A$.
\end{lemma}
\begin{proof}
It is clear that $\omega$ is $\inner{\cdot, \cdot}_\omega$ a sesquilinear form. By \ref{HermitianRealQuadratic}, it is now enough to prove positivity. Since $x^*x$ is self-adjoint for all $x\in A$, the positivity of $\inner{\cdot, \cdot}_\omega$ follows from the positivity of $omega$.
\end{proof}

\begin{proposition} \label{positiveLinearFunctionalBounded}
Let $A$ be a $C^*$-algebra and $f: A\to \C$ a positive linear functional on $A$. Then $f$ is bounded.
\end{proposition}
\begin{proof}
We first show that $f$ is bounded on $A^+$. Suppose, towards a contradiction, that $f$ is not bounded on $A^+$. Then for all $n\in \N$, we can find a unit vector $a_n\in A^+$ such that $f(a_n) \geq n$. Now $\sum_{n=1}^\infty \frac{a_n}{n^2}$ is absolutely convergent by \ref{pseriesConvergence} and thus is convergent by \ref{absoluteConvergenceImpliesConvergence}. We call the limit $a$. Let $b_N \defeq \sum_{n=1}^N \frac{a_n}{n^2}$ be the partial sum for all $N\geq 1$.

For all such $N$ we have $0 \leq b_N \leq a$. Indeed, $a-b_N$ is the limit of $\seq{b_M - b_N}_{M\geq N}$ which is a sequence of positive elements. Since $A^+$ is closed by \ref{CstarPositiveConeClosed} in $A$, $a-b_N$ is also an element of $A^+$. Thus
\[ f(a) \geq f(b_N) = \sum_{n=1}^N \frac{f(a_n)}{n^2} \geq \sum_{n=1}^N \frac{1}{n}. \]
Since, by \ref{pseriesConvergence}, $\sum_{n=1}^N \frac{1}{n}$ grows arbitrarily large, this is a contradiction. We have established that $m \defeq\sup_{\substack{a \geq 0  \\ \norm{a} = 1}} f(a)$ is finite.

Now take an arbitrary unit vector $c\in A$ and consider the Cartesian decomposition $c = (p_1 - p_2) + i(p_3 - p_4)$ (\ref{CartesianDecomposition}). Then
\[ |f(c)| = \big|(p_1 - p_2) + i(p_3 - p_4)\big| \leq |p_1| + |p_2| + |p_3| + |p_4| \leq 4m. \]
This implies that $f$ is bounded.
\end{proof}

\begin{proposition} \label{positiveFunctionalNormValueAtUnit}
Let $A$ be a $C^*$-algebra, $f: A\to \C$ a bounded linear functional on $A$ and $\seq{e_\lambda}_{\lambda\in \Lambda}$ an increasing approximate unit. Then the following are equivalent
\begin{enumerate}
\item $f$ is positive;
\item $\norm{f} = \lim_\lambda f(e_\lambda)$.
\end{enumerate}
In particular, if $A$ is unital, then $\norm{f} = f(\vec{1})$.
\end{proposition}
\begin{proof}
$(1) \Rightarrow (2)$  For all $\lambda \leq \mu$ we have $e_\lambda \leq e_\mu$, so $f(e_\lambda) \leq f(e_\mu) \leq \norm{f}$. Since $\seq{f(e_\lambda)}$ is a bounded monotone sequence, it converges (TODO ref) to a limit less than $\norm{f}$, i.e. $\lim_\lambda f(e_\lambda)$ exists and $\lim_\lambda f(e_\lambda)\leq \norm{f}$.

We now prove the other inequality. Let $a\in A$ be a unit vector. Since $f$ determines a pre-inner product, \ref{positiveLinearFunctionalPreinnerProduct}, the Cauchy-Schwarz inequality \ref{CauchySchwarz} holds. Then
\begin{align*}
|f(a)|^2 &= \big|f\big(\lim_\lambda e_\lambda a\big)\big| \\
&= \lim_\lambda |f(e_\lambda a)|^2 = \lim_\lambda |f(e_\lambda^* a)|^2 \\
&\leq \limsup_\lambda |f(e_\lambda^* a)|^2 \\
&\leq \limsup_\lambda |f(e_\lambda^*e_\lambda)|\, |f(a^*a)| \\
&= \limsup_\lambda f(e_\lambda^*e_\lambda)f(a^*a).
\end{align*}
By positivity of $f$ we have $|f(e_\lambda^*e_\lambda)| = f(e_\lambda^2) \leq f(e_\lambda)$, using \ref{orderSubUnitVectorLemma}. Since $\lim_\lambda f(e_\lambda)$ exists, as was previously established, we have $\limsup_\lambda f(e_\lambda) = \lim_\lambda f(e_\lambda)$. Thus
\[ |f(a)|^2 \leq \limsup_\lambda f(e_\lambda^*e_\lambda)f(a^*a) \leq \limsup_\lambda f(e_\lambda) f(a^*a) = \lim_\lambda f(e_\lambda) f(a^*a). \]
Since $\norm{a^*a} \leq \norm{a^*}\,\norm{a} = 1$, we have $|f(a^*a)|\leq \norm{f}$ and thus $|f(a)|^2 \leq \lim_\lambda f(e_\lambda) \norm{f}$. Taking the supremum over $a$ yields $\norm{f}^2 \leq \norm{f}\lim_\lambda f(e_\lambda)$ or $\norm{f} \leq \lim_\lambda f(e_\lambda)$.

$(2) \Rightarrow (1)$ We first prove that $f(b)\in \R$ for all $b\in \SelfAdjoints(A)$. Indeed, set $f(b) = \alpha + i\beta$. WLOG we may take $\norm{b} \leq 1$ (by potentially rescaling) and $\beta \geq 0$ (by potentially replacing $b\to -b$).

Since $e_\lambda b - be_\lambda \to 0$, we can, for all $n\in \N$ find $\lambda_n\in \Lambda$ such that for all $\lambda \geq \lambda_n$ we have $\norm{e_\lambda b - be_\lambda} \leq n^{-1}$. Then, for $\lambda \geq \lambda_n$, we have
\begin{align*}
\big|f(ne_\lambda - ib)\big|^2 &\leq \norm{f}^2\norm{ne_\lambda - ib}^2 \\
&= \norm{f}^2\norm{(ne_\lambda - ib)(ne_\lambda - ib)^*} \\
&= \norm{f}^2\norm{(ne_\lambda - ib)(ne_\lambda + ib)} \\
&= \norm{f}^2\norm{n^2e_\lambda^2 + b^2 - in(e_\lambda b - be_\lambda)} \\
&\leq \norm{f}^2(n^2 + 2).
\end{align*}
For arbitrary $n\in \N$, we calculate
\begin{align*}
\lim_\lambda \big|f(ne_\lambda - ib)\big|^2 &= \lim_\lambda \big|nf(e_\lambda) - i\alpha + \beta\big|^2 \\
&= \big|n\norm{f} - i\alpha + \beta\big|^2 \\
&= \big(n\norm{f}+\beta\big)^2 + \alpha^2 \\
&= n^2\norm{f}^2 + 2n\beta\norm{f}+\beta^2 + \alpha^2.
\end{align*}
Putting everything together gives
\[ n^2\norm{f}^2 + 2n\beta\norm{f}+\beta^2 + \alpha^2 = \lim_\lambda \big|f(ne_\lambda - ib)\big|^2 \leq \norm{f}^2(n^2 + 2). \]
Cancelling $n^2\norm{f}^2$ from both sides gives $2n\beta\norm{f}+\beta^2 + \alpha^2 \leq 2\norm{f}^2$. This can only hold for all $n\in \N$ if $\beta = 0$. Thus $f(b)\in \R$.

Now take arbitrary $a\in A^+$. WLOG we may take $\norm{a}\leq 1$. Since we have already seen that $f(e_\lambda - a) \in \R$, we have
\[ f(e_\lambda) - f(a) = f(e_\lambda - a) \leq |f(e_\lambda - a)| \leq \norm{f}\,\norm{e_\lambda - a} \leq \norm{f}, \]
where we have used that $\norm{e_\lambda - a} \leq 1$ by \ref{orderSubUnitVectorLemma}.
Thus $f(a) \geq f(e_\lambda) - \norm{f} \to 0$.
\end{proof}


\begin{lemma}
Let $A$ be a $C^*$-algebra, $f: A\to \C$ a bounded linear functional on $A$ and $a,x\in A$. Then
\begin{enumerate}
\item $f(a) = \overline{f(a^*)}$;
\item $|f(a)|^2 \leq \norm{f}f(a^*a) = \norm{f}\inner{a,a}_f$.
\end{enumerate}
\end{lemma}
\begin{proof}
(1) Since $f$ determines a pre-inner product, as in \ref{positiveLinearFunctionalPreinnerProduct}, we have that $\inner{\cdot,\cdot}_f$ has conjugate symmetry. 

Let $\seq{e_\lambda}_{\lambda\in \Lambda}$ be an increasing approximate unit. Since $f$ is bounded \ref{positiveLinearFunctionalBounded}, we have
\[ f(a) = \lim_{\lambda}f(e_\lambda a) = \lim_{\lambda}\inner{e_\lambda, a}_f = \lim_{\lambda}\overline{\inner{a, e_\lambda}_f} = \lim_{\lambda}\overline{f(a^*e_\lambda)} = \overline{f(a^*)}. \]

(2) The inequality $|f(a)|^2 \leq \lim_\lambda f(e_\lambda) f(a^*a)$ appears in the proof of \ref{positiveFunctionalNormValueAtUnit}. Using the result of this proposition, we have $|f(a)|^2 \leq \norm{f} f(a^*a)$.
\end{proof}

\subsubsection{States}
\begin{definition}
A \udef{state} on a $C^*$-algebra $A$ is a positive linear functional of norm $1$. The set $\states(A)$ of all states on $A$ is called the \udef{state space} of $A$.
\end{definition}


\begin{example}
Let $A$ be a concrete $C^*$-algebra of operators acting non-degenerately on $\mathcal{H}$ and $\xi \in \mathcal{H}$. Define
\[ \rho_\xi: A \to \C: x\mapsto \inner{\xi, x\xi}, \]
then $\rho_\xi$ is a positive linear functional on $A$ of norm $\norm{\xi}^2$, so  $\rho_\xi$ is a state if $\norm{\xi} =1$. Such a state is called a \udef{vector state} of $A$.
\end{example}

\begin{lemma}
Let $B\subseteq A$ be unital $C^*$-algebras. State on $B$ extends to a state on $A$.
\end{lemma}
\begin{proof}
Let $\omega\in\states(B)$ be a state on $B$. It can be extended to a bounded functional $\omega'$ of the same norm on $A$ by \ref{existenceBoundedFunctionalOfSameNorm}. Since $\vec{1} \in B\subseteq A$, we have, by \ref{positiveFunctionalNormValueAtUnit},
\[\omega'(\vec{1}) = \omega(\vec{1}) = \norm{\omega} = \norm{\omega'}. \]
This implies that $\omega'$ is positive by \ref{positiveFunctionalNormValueAtUnit}. 
\end{proof}

\begin{lemma} \label{normalElementExtendsToState}
Let $A$ be a $C^*$-algebra and $a\in \Normals(A)\setminus\{0\}$. Then there exists a state $\omega\in \states(A)$ such that $|f(a)| = \norm{a}$.
\end{lemma}
\begin{proof}
TODO
\end{proof}

\begin{proposition}
Let $\omega$ be a positive linear functional over a $C^*$-algebra $A$ and $a,b\in A$, then
\begin{enumerate}
\item $\omega(a^*) = \overline{\omega(a)}$;
\item $|\omega(a)|^2 \leq \omega(a^*a)\norm{\omega}$;
\item $|\omega(a^*ba)| \leq \omega(a^*a)\norm{b}$;
\item $\norm{\omega} = \sup\setbuilder{\omega(a^*a)}{\norm{a} = 1}$
\end{enumerate}
\end{proposition}
\begin{proof}
By \ref{realImaginaryParts} and \ref{positiveNegativeParts} we can write $a\in A$ as
\[ a = x_{1,+} - x_{1,-} + i(x_{2,+} - x_{2,-}).\]
Where $x_{1,+}, x_{1,-}, x_{2,+}, x_{2,-}$ are self-adjoint. Then
\[ \rho(a^*) = \rho(x_{1,+} - x_{1,-} - i(x_{2,+} - x_{2,-})) = \rho(x_{1,+}) - \rho(x_{1,-}) - i(\rho(x_{2,+}) - \rho(x_{2,-})) = \overline{\rho(a)}. \]
Where the last equality follows because $\rho$ takes real values on self-adjoint elements. (TODO!)
\end{proof}
\begin{corollary}
Let $\omega_1$ and $\omega_2$ be positive linear functionals over a $C^*$-algebra $A$. Then $\omega_1+\omega_2$ is a positive linear functional and
\[ \norm{\omega_1+\omega_2} = \norm{\omega_1} + \norm{\omega_2}. \]
Thus the state space is a convex subset of the dual of $A$.
\end{corollary}
\begin{corollary}
Let $X$ be a compact Hausdorff space. Let $\omega$ be a positive linear functional on $\cont(X)$. Then $\omega$ is continuous and $\norm{\omega} = 1$.
\end{corollary}
TODO: move up for Riesz-Markov?

\begin{proposition}
If $A$ is commutative, the pure states are exactly the characters.
\end{proposition}

\subsection{Comparison of projectors}
Lattice
\subsection{General comparison theory}

\section{Matrix $C^*$-algebras}
\begin{proposition}
Let $A$ be a $C^*$-algebra. There exists a norm that makes $A^{n\times n}$ a $C^*$-algebra, which is given by
\begin{align*}
\norm{\begin{pmatrix}
a_{1, 1} & \hdots & a_{1,n} \\
\vdots & \ddots & \vdots \\
a_{n,1} & \hdots & a_{n,n}
\end{pmatrix}}_{A^{n\times n}} &\defeq \sup\setbuilder{\norm{\sum_{i,j=1}^nx_i^*a_{i,j}y_{j}}_A}{\seq{x_i}_{i=0}^n, \seq{y_j}_{j=0}^n\in A^n, \, \sqrt{\sum_{i=1}^n \norm{x_i}^2} = 1 = \sqrt{\sum_{j=1}^n \norm{y_j}^2}}.
\end{align*}
\end{proposition}
\begin{proof}
It is clear that $\norm{\cdot}_{A^{n\times n}}$ is a norm. We just need to verify the $C^*$-identity. Take $[a_{i,j}]\in A^{n\times n}$. Then
\begin{align*}
\norm{[a_{i,j}]}_{A^{n\times n}} &= \sup\setbuilder{\norm{\sum_{i,j=1}^nx_i^*a_{i,j}y_{j}}_A}{\seq{x_i}_{i=0}^n, \seq{y_j}_{j=0}^n\in A^n, \,\forall i,j\leq n: \norm{x_i}_{A} \leq 1, \norm{y_j}_{A}\leq 1}
\end{align*}
\end{proof}

\begin{proposition}
Norm given by representing matrix algebra on Hilbert space. If representation is faithful, then norm independent of representation.
\end{proposition}

\subsection{Dilation tricks}
\begin{lemma}
Let $A$ be a unital $C^*$-algebra and $a\in A$. Then $\norm{a}\leq 1$ \textup{if and only if} $\begin{pmatrix}
\vec{1} & a \\ a^* & \vec{1}
\end{pmatrix}$ is positive in $A^{2\times 2}$.
\end{lemma}
\begin{proof}

\end{proof}

\subsection{Completely positive maps}
\begin{definition}
Let $A,B$ be $C^*$-algebras and $f:A\to B$ a linear function. We call $f$ \udef{completely positive} if for all $n\in\N$ the pointwise extension of $f$ in $(A^{n\times n}\to B^{n\times n})$ is positive.
\end{definition}

\begin{lemma}
Let $A,B$ be $C^*$-algebras and $\phi:A\to B$ a $*$-homomorphism. Then
\begin{enumerate}
\item $\phi_n$ is a $*$-homomorphism;
\item $\phi$ is completely positive.
\end{enumerate}
\end{lemma}
\begin{proof}
(1) Take $X,Y\in A^{n\times n}$. Then we calculate
\begin{align*}
\big[\phi_n(XY)\big]_{i,j} &= \phi\big([XY]_{i,j}\big) \\
&= \phi\Big(\sum_{k=1}^n[X]_{i,k}[Y]_{k,j}\Big) \\
&= \sum_{k=1}^n\phi\big([X]_{i,k}\big)\phi\big([Y]_{k,j}\big) \\
&= \sum_{k=1}^n[\phi_n(X)]_{i,k}[\phi_n(Y)]_{k,j} \\
&= \big[\phi_n(X)\phi_n(Y)\big]_{i,j}.
\end{align*}
We also have
\[ \big[\phi_n(X^*)\big]_{i,j} = \phi\big([X^*]_{i,j}\big) = \phi\big([X]_{j,i}^*\big) = \phi\big([X]_{j,i}\big)^* = [\phi_n(X)]_{j,i}^* = [\phi_n(X)^*]_{i,j}. \]

(2) Immediate from \ref{starHomomorphismPositive} and point (1).
\end{proof}

\subsection{Completely bounded maps}
\begin{definition}
Let $A,B$ be $C^*$-algebras and $f:A\to B$ a linear function. We call $f$ \udef{completely bounded} if for all $n\in\N$ the pointwise extension of $f$ in $(A^{n\times n}\to B^{n\times n})$ is bounded.
\end{definition}

\chapter{Von Neumann Algebras}
TODO: definitions of SOT and WOT!
\begin{definition}
A concrete $C^*$-algebra $A\subseteq \Bounded(\mathcal{H})$ is a \udef{von Neumann algebra} if it is closed in the SOT.
\end{definition}

\section{von Neumann bicommutant theorem}
\begin{proposition} \label{commutantBanachAlgebra}
Let $S\subset \Bounded(\mathcal{H})$ be a set for some Hilbert space $\mathcal{H}$. Then
\begin{enumerate}
\item $\comm{S}$ is a Banach algebra;
\item $\comm{S}$ is a $C^*$-algebra if $S = S^*$;
\item $\comm{S}\subseteq \Bounded(\mathcal{H})$ is WOT-closed.
\end{enumerate}
\end{proposition}

\chapter{Representations and states}
\section{Representations}
\begin{definition}
A \udef{representation} of a $C^*$-algebra $A$ on a Hilbert space $\mathcal{H}$ is a $*$-homomorphism $\pi: A \to \Bounded(\mathcal{H})$.

A \udef{subrepresentation} of $\pi$ is the restriction of $\pi$ to a closed invariant subspace of $\mathcal{H}$.

We say a representation $\pi: A \to \Bounded(\mathcal{H})$ is
\begin{enumerate}
\item \udef{faithful} if it is injective;
\item \udef{non-degenerate} if $\overline{\pi(A)\mathcal{H}} = \mathcal{H}$;
\item \udef{cyclic} w.r.t. a unit vector $\xi\in\mathcal{H}$ if $\overline{\pi(A)\xi} = \mathcal{H}$.
\end{enumerate}
\end{definition}

\begin{lemma}
Let $A$ be a $C^*$-algebra and $\pi: A \to \Bounded(\mathcal{H})$ a representation of $A$. Then $\pi$ is a faithful representation of $A/\ker\pi$.
\end{lemma}

\begin{proposition}
Let $\pi:A\to\mathcal{H}$ be a representation of a $C^*$-algebra, then $\pi$ being faithful is equivalent to any of the following:
\begin{enumerate}
\item $\ker \pi = \{0\}$;
\item $\norm{\pi(a)} = \norm{a}$ for all $a\in A$;
\item $\pi(a) > 0$ for all $a>0$.
\end{enumerate}
\end{proposition}

\begin{lemma}
Let $\pi: A\to\mathcal{H}$ be a representation and $P_1$ be a projector with closed range $\mathcal{H}_1$. Then $\pi|_{\mathcal{H}_1}$ is a subrepresentation \textup{if and only if}
\[ \forall a\in A: \quad \pi(a)P_1 = P_1\pi(a).  \]
\end{lemma}
\begin{proof}
Assume $P_1\pi(a) = \pi(a)P_1$ for all $a\in A$. Then multiplying by $P_1$ gives
\[ P_1\pi(a)P_1 = \pi(a)P_1 \quad \forall a\in A \]
which expresses invariance. Conversely, assume this invariance condition. Then
\[ \pi(a)P_1 = P_1\pi(a)P_1 = (P_1\pi(a)P_1)^{**} = (P_1\pi(a^*)P_1)^{*} = (\pi(a^*)P_1)^* = P_1\pi(a). \]
\end{proof}

\begin{lemma} \label{nonDegeneracyAlgebraRepresentation}
Let $\pi:A\to \mathcal{H}$ be a representation of a $C^*$-algebra $A$. Define
\[ \mathcal{H}_0 \defeq \setbuilder{x\in\mathcal{H}}{\forall a\in A: \pi(a)x = 0}. \]
Then $\pi$ is non-degenerate \textup{if and only if} $\mathcal{H}_0 = \{0\}$.
\end{lemma}
\begin{proof}
It is enough to prove that
\[ (\pi(A)\mathcal{H})^\perp = \mathcal{H}_0. \]
This implies $\overline{\pi(A)\mathcal{H}} = \mathcal{H}_0^\perp$ by \ref{doubleComplementClosure}. The claim then follows from \ref{OrthogonalComplementProperties}.

The proof then rests on the equality
\[ \forall x,y\in \mathcal{H}, a\in A: \quad \inner{x,\pi(a)y} = \inner{\pi(a^*)x,y}. \]
An $x\in\mathcal{H}$ is an element of $(\pi(A)\mathcal{H})^\perp$ iff the left side is zero for all $y\in\mathcal{H},a\in A$. An $x\in\mathcal{H}$ is an element of $\mathcal{H}_0$ iff $\pi(a^*)x = 0$ for all $a^*\in A$. This is equivalent to saying the right side is zero for all $y\in\mathcal{H},a\in A$ by the non-degeneracy of the inner product \ref{nonDegeneracyInnerProduct}.
\end{proof}

\begin{proposition}
Let $\pi:A\to \mathcal{H}$ be a representation. Then $\pi$ is non-degenerate \textup{if and only if} $\pi$ is the direct sum of a family of cyclic representations.
\end{proposition}

\begin{definition}
Two representations $\pi,\rho$ of $A$ on Hilbert spaces $\mathcal{X}$ and $\mathcal{Y}$ respectively are \udef{(unitarily) equivalent} if there is a unitary operator $U\in\Bounded(\mathcal{X}, \mathcal{Y})$ such that
\[ \forall x\in A: \quad U\pi(x)U^* = \rho(x). \]
\end{definition}

\subsection{Irreducible representations}
TODO move: to do with subsets of operators, not really representations.

\begin{definition}
A set $D$ of bounded operators on a Hilbert space $\mathcal{H}$ is called \udef{algebraically irreducible} if the only subspaces of $\mathcal{H}$ invariant under the action of $D$ are the trivial subspaces $\{0\}$ and $\mathcal{H}$.

The set $D$ is called \udef{topologically irreducible} if the only closed invariant subspaces of $\mathcal{H}$ are the trivial subspaces.

A representation $\pi: A\to\mathcal{H}$ is called irreducible if $\pi[A]$ is irreducible.
\end{definition}


Let $D$ be a set of bounded operators on the Hilbert space $\mathcal{H}$. Then $\mathcal{H}$ is a left $D$-module and $\mathcal{H}$ is algebraically irreducible if and only if it is a simple module. The commutant $D^\commute$ can be seen as a set of module morphisms.

\begin{proposition} \label{equivalentsIrreducibleSetsOperatorsHilbertSpace}
Let $D$ be a set of bounded operators on the Hilbert space $\mathcal{H}$. The following are equivalent:
\begin{enumerate}
\item $D$ is topologically irreducible;
\item all non-zero elements of the commutant $D^\commute$ are invertible; 
\item the commutant $D^\commute$ consists of multiples of the identity operator;
\item the commutant $D^\commute$ contains no projectors other than the identity;
\item if $D \neq \{0\}$ is a subalgebra of $\Bounded(\mathcal{H})$, then every non-zero vector $x\in\mathcal{H}$ is cyclic for $D$ in $\mathcal{H}$.
\end{enumerate}
\end{proposition}
If $D = \{0: \C\to \C\}$, then $D$ is irreducible, but no element of $\C$ is cyclic.
\begin{proof}
$(1) \Rightarrow (2)$ Immediate by Schur's lemma \ref{SchursLemma}.

$(2) \Rightarrow (3)$ Immediate by the Gelfand-Mazur theorem \ref{GelfandMazur}. Note that we use the fact that the commutant is a Banach algebra, \ref{commutantBanachAlgebra}.

$(3) \Rightarrow (4)$ Immediate.

$(4) \Rightarrow (1)$ Suppose $D$ were topologically reducible. Then there would be an orthogonal projector $P$ whose image was a closed invariant subspace of $\mathcal{H}$ and $P\neq \id_\mathcal{H}$. Invariance means $P\in D^\commute$, which goes against the assumption.

$(1) \Leftrightarrow (5)$ We have that $V \defeq \closure\setbuilder{d(x)}{d\in D}$ is a closed invariant subspace of $\mathcal{H}$. Since there exists $d\in D$ that is non-zero, there exists a non-zero element in $V$, so $V\neq \{0\}$. The equivalence of (1) and (5) is then immediate.
\end{proof}
\begin{corollary}
For $C^*$-algebras the notions of algebraic and topological irreducibility are equivalent.
\end{corollary}
\begin{proof}
TODO: should distinction algebraic and topological irreducibility be removed?
\end{proof}

\begin{lemma} \label{idealRepresentationExtensionLemma}
Let $A$ be a $C^*$-algebra and $J$ an ideal in $A$. Let $\rho: J\to \Bounded(\mathcal{H})$ be a nondegenerate representation of $J$. For all $b_1,\ldots, b_n\in J$, $h_1,\ldots, h_n\in\mathcal{H}$ and $a\in A$, we have
\[ \norm{\sum_{k=1}^n\rho(ab_k)h_k} \leq \norm{a}\, \norm{\sum_{k=1}^n\rho(b_k)h_k}. \]
\end{lemma}
\begin{proof}
By \ref{approximateUnitInIdeal} we can find an approximate unit $\seq{e_\lambda} \subseteq J$. Then
\begin{align*}
\norm{\sum_{k=1}^n\rho(ab_k)h_k} &= \lim_\lambda\norm{\sum_{k=1}^n\rho(ae_\lambda b_k)h_k} \\
&= \lim_\lambda\norm{\rho(ae_\lambda)\sum_{k=1}^n\rho(b_k)h_k} \\
&\leq \limsup_\lambda\norm{\rho(ae_\lambda)}\norm{\sum_{k=1}^n\rho(b_k)h_k} \\
&\leq \limsup_\lambda\norm{ae_\lambda}\norm{\sum_{k=1}^n\rho(b_k)h_k} \\
&= \lim_\lambda\norm{a}\norm{\sum_{k=1}^n\rho(b_k)h_k},
\end{align*}
where we have used \ref{starHomomorphismCstarProperties}.
\end{proof}

\begin{proposition}
Let $A$ be a $C^*$-algebra and $J$ an ideal in $A$. Let $\rho: J\to \Bounded(\mathcal{H})$ be a nondegenerate representation of $J$. Then there is a unique representation $\overline{\rho}: A\to \Bounded(\mathcal{H})$ of $A$ that extends $\rho$. If $\rho$ and $\sigma$ are equivalent representations of $J$, then $\overline{\rho}$ and $\overline{\sigma}$ are equivalent representations of $A$.
\end{proposition}
\begin{proof}
Set $\mathcal{H}_0 = \rho^\imf(J)\mathcal{H}$. For all $x = \sum_{k=1}^n\rho(b_k)h_k\in \mathcal{H}_0$, we define $\overline{\rho}$ by
\[ \overline{\rho}(a)x \defeq \sum_{k=1}^n\rho(ab_k)h_k. \]
Then \ref{idealRepresentationExtensionLemma} implies that this definition is well-defined. It also implies that $\overline{\rho}$ is bounded and so can be extended to a bounded operator on $\mathcal{H} = \overline{\mathcal{H}_0}$ by (TODO ref).

To show $\overline{\rho}$ is a representation of $A$, take $a_1,a_2\in A$. Then we calculate
\begin{align*}
\overline{\rho}(a_1a_2)(x) &= \sum_{k=1}\rho(a_1a_2b_k)h_k \\
&= \overline{\rho}(a_1)\big(\sum_{k=1}\rho(a_2b_k)h_k\big) \\
&= \overline{\rho}(a_1)\Big(\overline{\rho}(a_2)\big(\sum_{k=1}\rho(b_k)h_k\big)\Big) \\
&= \overline{\rho}(a_1)\big(\overline{\rho}(a_2)(x)\big) = \big(\overline{\rho}(a_1)\overline{\rho}(a_2)\big)(x).
\end{align*}
To show $\overline{\rho}$ is a $*$-homomorphism, take $y = \sum_{l=1}^m \rho(c_l)g_l \in \mathcal{H}_0$. Then we calculate
\begin{align*}
\inner{\overline{\rho}(a)^*x, y} &= \inner{x, \overline{\rho}(a)y} \\
&= \sum_{l=1}^m\inner{x, \rho(ac_l)g_l} \\
&= \sum_{l=1}^m\inner{\rho(c_l^*a^*)x, g_l} \\
&= \sum_{k,l}\inner{\rho(c_l^*a^*)\rho(b_k)h_k, g_l} \\
&= \sum_{k,l}\inner{\rho(c_l^*)\rho(a^*b_k)h_k, g_l} \\
&= \sum_{k,l}\inner{\rho(a^*b_k)h_k, \rho(c_l)g_l} \\
&= \inner{\overline{\rho}(a^*)(x), y}.
\end{align*}

Finally let $\sigma(b) = U\rho(b)U^*$ we need to show that $\overline{\rho}$ and $\overline{\sigma}$ are equivalent representations. We have
\[ U^*x = U^*\sum_{k=1}^n\rho(b_k)h_k = \sum_{k=1}^nU^*\rho(b_k)h_k = \sum_{k=1}^n\rho(b_k)\big(U^*h_k\big), \]
so we can calculate
\begin{align*}
\big(U\overline{\rho}(a)U^*\big)(x) &= U\big(\overline{\rho}(a)(U^*x)\big) \\
&= \sum_{k=1}^nU\Big(\overline{\rho}(a)\big(\rho(b_k)(U^*h_k)\big)\Big) \\
&= \sum_{k=1}^nU\big(\rho(ab_k)(U^*h_k)\big) \\
&= \sum_{k=1}^n\big(U\rho(ab_k)U^*\big)h_k \\
&= \sum_{k=1}^n\sigma(ab_k)h_k = \overline{\sigma}(a)(x).
\end{align*}
\end{proof}

\begin{proposition}
Let $A$ be a $C^*$-algebra and $J$ an ideal in $A$. Let $\rho$ be an irreducible representation of $J$ and $\pi$ an irreducible representation of $A$.
Then
\begin{enumerate}
\item $\overline{\rho}$ is an irreducible representation of $A$;
\item if $\pi^\imf(J)\neq \{0\}$, then $\pi|_J$ is an irreducible representation of $J$.
\end{enumerate}
\end{proposition}
\begin{proof}
(1) Any subspace that is invariant under $\im(\overline{\rho})$ is in particular invariant under $\im(\overline{\rho}|_J) = \im(\rho)$ and thus trivial by assumption.

(2) Suppose $\pi$ maps elements of $A$ to bounded operators on $\mathcal{H}$. By \ref{equivalentsIrreducibleSetsOperatorsHilbertSpace}, it is enough to show that $\overline{\pi^{\imf}(J)h} = \mathcal{H}$ for all $h\in \mathcal{H}\setminus\{0\}$.

Since $J$ is an ideal, $\overline{\pi^{\imf}(J)h}$ is invariant under $\pi^\imf(A)$, which means it is either $\{0\}$ or $\mathcal{H}$. We show that the first case leads to a contradiction, indeed this means that $\inner{\pi(b)x, h} = \inner{x, \pi(b^*)h} = 0$ for all $x\in \mathcal{H}$ and $b\in J$, so $h\in \big(\pi^\imf(J)\mathcal{H}\big)^\perp$. Since $\pi^\imf(J)\mathcal{H}$ is invariant under $\pi^\imf(A)$ and non-zero by assumption, we have $h\in \big(\pi^\imf(J)\mathcal{H}\big)^\perp = \mathcal{H}^\perp = \{0\}$. This is a contradiction.
\end{proof}

\subsection{The spectrum}
\begin{definition}
Let $A$ be a $C^*$-algebra. Then set of equivalence classes of irreducible
representations is called the \udef{spectrum} of A.
\end{definition}

The following allows us to identify the spectrum of a commutative $C^*$-algebra with the set of non-zero complex-valued algebra homomorphisms.

\begin{proposition}
Let $A$ be a commutative $C^*$-algebra and $\pi: A\to \Bounded{\mathcal{H}}$ a representation. Then $\pi$ is irreducible \textup{if and only if} $\mathcal{H}$ is one-dimensional.
\end{proposition}
\begin{proof}
If $\mathcal{H}$ is one-dimensional, then the representation is clearly irreducible.

Now assume $\pi$ is irreducible.
By \ref{starHomomorphismCstarProperties} $\im(\pi)$ is a $C^*$-algebra, which is clearly commutative. Then $\im(\pi) \subseteq \im(\pi)^\commute = \C\cdot \id_\mathcal{H}$ by \ref{equivalentsIrreducibleSetsOperatorsHilbertSpace}, so all subspaces of $\mathcal{H}$ are invariant. This implies that the trivial subspaces are the only subspaces and thus $\mathcal{H}$ is one-dimensional.
\end{proof}

\subsection{The GNS construction}

\begin{lemma} \label{GNSLeftIdeal}
Let $\omega$ be a positive linear functional on a $C^*$-algebra $A$. Then
\[ N_\omega \defeq \setbuilder{x\in A}{\inner{x,x}_\omega = 0} \]
is a closed left ideal in $A$.
\end{lemma}
\begin{proof}
For all $x,y\in N_\omega$, we have $\inner{x,y}_\omega = 0$ by \ref{preInnerProductCSBZero}.
Take $x,y\in N_\omega$ and $\lambda \in C$, then
\[ \inner{x+\lambda y,x+\lambda y}_\omega = \inner{x,x}_\omega + \overline{\lambda}\inner{y,x}_\omega + \lambda\inner{x,y}_\omega + |\lambda|^2\inner{y,y}_\omega = 0. \]
This shows that $N_\omega$ is a subspace by \ref{subspaceCriterion}.

Now, in order to see that $N_\omega$ is a left ideal, take $x\in N_\omega$ and $a\in A$. Then
\[ \inner{ax,ax}_{\omega} = |\inner{ax,ax}_{\omega}| = |\inner{x,a^*ax}_{\omega}| \leq \cancel{\sqrt{\inner{x,x}_\omega}}\, \sqrt{\inner{a^*ax,a^*ax}_\omega} = 0. \]
Thus $ax\in N_\omega$.

That $N_\omega$ is closed follows immediately from the fact that $N_\omega = (x\mapsto \inner{x,x}_\omega)^{\preimf}\big(\{0\}\big)$, \ref{preimageOpenClosed} and the fact that $(x\mapsto \inner{x,x}_\omega)$ is continuous.
\end{proof}
That $N_\omega$ is a left-ideal means that the representation $\pi_\omega$ of \ref{GNSconstruction} is well-defined.

\begin{lemma} \label{GNSlemma}
Let $\omega\in\states(A)$ be a state on a $C^*$-algebra $A$. Then
\begin{enumerate}
\item $A/N_\omega$ is an inner product space with inner product $\inner{[x], [y]}_\omega = \inner{x,y}_\omega$;
\item the quotient map $[\cdot]: A\to A/N_\omega$, when $A/N_\omega$ is equipped with the topology derived from the inner product;
\item $\seq{[e_\lambda]}$ is a Cauchy net for any increasing approximate unit $\seq{e_\lambda}_{\lambda\in\Lambda}$;
\item for any increasing approximate units $\seq{e_\lambda}$ and $\seq{e'_\mu}$, the Cauchy nets $\seq{[e_\lambda]}$ and $\seq{[e'_\mu]}$ are equivalent.
\end{enumerate}
\end{lemma}
We also define $H_\omega$ as the Hilbert space completion of $A/N_\omega$.
\begin{proof}
(1) We have that $A/N_\omega$ is a vector space.

We verify this inner product is well-defined: take $x,x'\in [x]$ and $y,y'\in[y]$. Then $(x'-x) \in N_\omega$ and $(y'-y) \in N_\omega$. Using \ref{preInnerProductCSBZero}, we see
\[ \omega(x^*y) = \omega(x^*y) + \omega(x^*(y'-y)) = \omega(x^*y') = \omega(x^*y') + \omega((x'-x)^*y') = \omega(x^{\prime *}y'). \]

(2) Suppose $x_n \to x$ in $A$, so $\norm{x_n - x}\to 0$. Then we calculate
\begin{align*}
\norm{[x_n] - [x]}_\omega &= \norm{[x_n - x]}_\omega \\
&= \sqrt{\omega\big((x_n - x)^*(x_n-x)\big)} \\
&\leq \sqrt{\norm{\omega}\norm{(x_n - x)^*(x_n-x)}} \\
&= \norm{x_n - x} \to 0.
\end{align*}

(3) Take $\epsilon >0$. Since $\seq{\omega(e_\lambda)}$ converges, by \ref{positiveFunctionalNormValueAtUnit}, it is a Cauchy net and there exists $\lambda\in\Lambda$ such that for all $\mu,\nu \geq \lambda$, $|f(e_\mu) - f(e_\nu)| \leq \epsilon/4$.

Now \ref{orderSubUnitVectorLemma} gives $\norm{e_\mu - e_\lambda} \leq 1$ and thus $(e_\mu - e_\lambda)^2\leq (e_\mu - e_\lambda)$, since $e_\mu \geq e_\lambda$. Similarly $(e_\nu - e_\lambda)^2\leq (e_\nu - e_\lambda)$.

We calculate
\begin{align*}
\norm{[e_\mu] - [e_\nu]}_\omega^2 &\leq \big(\norm{[e_\mu] - [e_\lambda]}_\omega + \norm{[e_\nu] - [e_\lambda]}_\omega\big)^2 \\
&= \big(\sqrt{\omega\big((e_\mu - e_\lambda)^2\big)} + \sqrt{\omega\big((e_\nu - e_\lambda)^2\big)}\big)^2 \\
&\leq \big(\sqrt{\omega(e_\mu - e_\lambda)} + \sqrt{\omega(e_\nu - e_\lambda)}\big)^2 \\
&\leq \Big(\sqrt{\frac{\epsilon}{4}} + \sqrt{\frac{\epsilon}{4}}\Big)^2 = \epsilon^2.
\end{align*}
Thus $\norm{[e_\mu] - [e_\nu]}_\omega \leq \epsilon$, which implies $\seq{[e_\lambda]}$ is a Cauchy net.

(4) TODO
\end{proof}

\begin{theorem}[Gelfand-Naimark-Segal construction] \label{GNSconstruction}
Let $\omega\in\states(A)$ be a state on a $C^*$-algebra $A$. The map $\pi_\omega: A\to \Bounded(A/N_\omega)$ defined by
\[ \pi_\omega(x) : A/N_\omega \to A/N_\omega: [y] \mapsto [xy] \]
is a well-defined $*$-representation of $A$ on $A/N_\omega$. This can be extended to a representation of $A$ on $H_\omega$. There exists $\xi_\omega\in H_\omega$ such that
\begin{enumerate}
\item $\pi_\omega(a)\xi_\omega = [a]$;
\item $\omega(a) = \inner{\xi_\omega, \pi_\omega(a)\xi_\omega}_\omega$ for all $a\in A$;
\item $\pi_\omega$ is cyclic w.r.t. $\xi_\omega$.
\end{enumerate}
\end{theorem}
\begin{proof}
Since $N_\omega$ is a left ideal, \ref{GNSLeftIdeal}, we have by \ref{congruenceRingIdeals} and \ref{quotientAlgebra}, that $A/N_\omega$ is a $\{\lambda_a, +, (\lambda \cdot -)\}_{a\in A, \lambda\in \C}$-algebra. For all $x\in A$, the operator $(\lambda_x)_{A/N_\omega}$ is exactly $\pi_\omega(x)$.

Next we show that $\pi_\omega(x)$ is bounded for all $x\in A$. This follows from \ref{CstarOrderLemma}:
\[ \norm{\pi_\omega(x)\big([y]\big)}_\omega = \sqrt{\inner{[xy], [xy]}_\omega} = \sqrt{\inner{xy,xy}_{\omega}} = \sqrt{\omega(y^*x^*xy)} \leq \norm{x}\sqrt{\omega(y^*y)} = \norm{x}\,\norm{[y]}_\omega. \]
Thus $\pi_\omega(x)$ can be extended to a bounded operator on $\Bounded(H_\omega)$, which we also denote $\pi_\omega(x)$.

Now we consider multiplicativity and $*$-preservation of $\pi_\omega$. Take $a,b\in A$ and $x,y\in H_\omega$. Then there exists $\seq{[x_n]}, \seq{[y_n]} \in (A/N_\omega)^\N$ such that $[x_n]\overset{\inner{}_\omega}{\longrightarrow} x$ and $[y_n]\overset{\inner{}_\omega}{\longrightarrow} y$. We calculate
\begin{align*}
\pi_\omega(ab)(x) &= \lim_n\pi_\omega(ab)\big([x_n]\big) \\
&= \lim_n[abx_n] \\
&= \lim_n\pi_\omega(a)\big([bx_n]\big) \\
&= \lim_n\big(\pi_\omega(a)\circ \pi_\omega(b)\big)\big([x_n]\big) \\
&= \big(\pi_\omega(a)\circ \pi_\omega(b)\big)(x).
\end{align*}
Thus $\pi_\omega(ab) = \pi_\omega(a)\circ \pi_\omega(b)$. We also calculate
\begin{align*}
\inner{x, \pi_\omega(y)}_\omega &= \lim_n \inner{[x_n], \pi_\omega(a)[y_n]}_\omega \\
&= \lim_n \inner{[x_n], [ay_n]}_\omega \\
&= \lim_n \omega(x_n^*ay_n) \\
&= \lim_n \omega\big((a^*x_n)^*y_n\big) \\
&= \lim_n \inner{[a^*x_n], [y_n]}_\omega \\
&= \lim_n \inner{\pi_\omega(a^*)[x_n], [y_n]}_\omega \\
&= \inner{\pi_\omega(a^*)x, y}_\omega.
\end{align*}
This implies $\pi_\omega(a)^* = \pi_\omega(a^*)$.

Let $\seq{e_\lambda}$ be an increasing approximate unit in $A$. Then $\seq{[e_\lambda]}$ is a Cauchy net in $A/N_\omega$ by \ref{GNSlemma}, which means it has a limit in $H_\omega.$ We call this limit $\xi_\omega$.

(1) Now, since the quotien map is continuous, \ref{GNSlemma}, we have
\[ \pi_\omega(a)\xi_\omega = \lim_{\lambda}\pi_\omega(a)[e_\lambda] = \lim_{\lambda}[ae_\lambda] = \big[\lim_{\lambda}ae_\lambda\big] = [a]. \]
(2) We have
\[ \inner{\xi_\omega, \pi_\omega(a)\xi_\omega}_\omega = \inner{\xi_\omega, [a]}_\omega = \lim_\lambda \inner{[e_\lambda], [a]}_\omega = \lim_\lambda \omega(e_\lambda a) = \omega(a). \]
(3) Immediate from (1). 
\end{proof}

\begin{theorem}[Gelfand-Naimark theorem]
Every $C^*$-algebra has a faithful non-degenerate representation.
\end{theorem}
\begin{proof}
By \ref{normalElementExtendsToState}, there exists a state $\omega_a$ such that $\omega_a(a^*a) = \norm{a^*a} = \norm{a}^2$ for all $a\in A$. Let $\pi_a$ be the corresponding GNS-representation \ref{GNSconstruction}. Then $\pi \defeq \bigoplus_{a\in A\setminus\{0\}}\pi_a$ is a faithful representation. It is also non-degenerate (TODO ref).
\end{proof}

\begin{proposition}
Let $\omega$ be a state over the $C^*$-algebra $A$ and $\tau: A\to A$ a $*$-automorphism that leaves $\omega$ invariant:
\[ \forall a\in A: \quad \omega(\tau(a)) = \omega(a). \]
Then there exists a unique unitary operator $U$ on $\mathcal{H}_\omega$ such that
\[ \forall a\in A: \quad U\pi_\omega(a)U^* = \pi_\omega(\tau(a)) \]
and $U\xi_\omega = \xi_\omega$.
\end{proposition}

\begin{definition}
Let $\omega: A\to \C$ be a state.
We call the cyclic representation $(\mathcal{H}_\omega,\pi_\omega,\xi_\omega)$ constructed in the GNS theorem is called the \udef{canonical cyclic representation} of $A$ associated with $\omega$.
\end{definition}

\begin{lemma}
Let $\omega$ be a state over a $C^*$-algebra $A$ and $(\mathcal{H}_\omega,\pi_\omega,\xi_\omega)$ the associated cyclic representation. There is a bijective correspondence 
\[ \omega_T(a) = \inner{T\xi_\omega, \pi_\omega(a)\xi_\omega} \]
between positive functionals $\omega_T$ over $A$ majorised by $\omega$ and positive operators $T$ in the commutant $\comm{\pi_\omega}$ with $\norm{T}\leq 1$.
\end{lemma}

\begin{proposition}
Let $\omega$ be a state over a $C^*$-algebra $A$ and $(\mathcal{H}_\omega,\pi_\omega,\xi_\omega)$ the associated cyclic representation. The following are equivalent:
\begin{enumerate}
\item $(\mathcal{H}_\omega,\pi_\omega)$;
\item $\omega$ is a pure state;
\item $\omega$ is an extremal point of the state space $\mathcal{S}(A)$.
\end{enumerate}
\end{proposition}

\begin{theorem}
Let $A$ be a $C^*$-algebra. Then $A$ is isomorphic to a norm-closed self-adjoint algebra of bounded operators on a Hilbert space.
\end{theorem}

\subsection{Stinespring's dilation theorem}
\begin{theorem}[Stinespring's dilation theorem]
Let $A$ be a unital $C^*$-algebra and $\phi: A\to \Bounded(H)$ a completely positive map. Then there exits a Hilbert space $H'$, a unital representation $\pi: A\to \Bounded(H')$ and a $V\in \Bounded(H,H')$ such that
\[ \forall a\in A: \quad \phi(a) = V^*\pi(a)V. \]
If $\phi$ is unital, then $V$ is an isometry. Then we can identify $H$ with a subspace of $H'$ and have
\[ \forall a\in A: \quad \phi(a) = P_H\pi(a)|_H. \]
\end{theorem}

\begin{proposition}
All Stinespring dilations such that $\big(\pi^\imf(A)V\big)^\imf(H)$ is dense in $H'$ are unitarily equivalent.
\end{proposition}

\section{Multiplier algebras}
\subsection{Essential ideals}
TODO move to section about ideals
\begin{definition}
Let $J$ be an ideal of a $C^*$-algebra $A$. Then $J$ is called \udef{essential} if for all $a\in A$ we have that $aJ = \{0\}$ implies $a=0$.
\end{definition}

\begin{lemma} \label{C*idealSquared}
Let $A$ be a $C^*$-algebra and $I\subset A$ an ideal. Then $I^2 = I$.
\end{lemma}
\begin{proof}
Let $a\in I^+$. Then $a = (a^{1/2})^2\in I^2$. As $I$ and $I^2$ are $C^*$-algebras, they are spanned by their positive elements. So $I^2 
\subset I \subset I^2$.
\end{proof}
\begin{lemma} \label{productC*ideals}
Let $A$ be a $C^*$-algebra and $I,J\subset A$ ideals. Then $IJ = I\cap J$.
\end{lemma}
\begin{proof}
We calculate $I\cap J = (I\cap J)^2 \subset IJ \subset I\cap J$ using \ref{C*idealSquared}.
\end{proof}

\begin{proposition}
Let $J$ be an ideal of a $C^*$-algebra $A$. Then the following are equivalent:
\begin{enumerate}
\item $J$ is essential;
\item $\forall a\in A:\;Ja = \{0\}$ implies $a=0$;
\item every other non-zero ideal in $A$ has a non-zero intersection with $J$.
\end{enumerate}
\end{proposition}
\begin{proof}
Assume (3) and let $a\in A$ such that $aJ=\{0\}$. Let $I= \overline{AaA}$ be the ideal generated by $a$.
\end{proof}

\subsection{Multiplier algebras}
\begin{proposition}
Let $A$ be a $C^*$-algebra and $\pi_1, \pi_2$ faithful, non-degenerate representations. Then the idealisers $I(\pi_1[A]), I(\pi_2[A])$ of $\pi_1[A]$ and $\pi_2[A]$ are isomorphic to each other.
\end{proposition}
\begin{proof}
We need to show
\[ I(\pi_1[A]) = \setbuilder{T\in \Bounded(\mathcal{H}_1)}{T\pi_1[A] \cup \pi_1[A]T \subseteq \pi_1[A]} \cong \setbuilder{T\in \Bounded(\mathcal{H}_2)}{T\pi_2[A] \cup \pi_2[A]T \subseteq \pi_2[A]} = I(\pi_2[A]). \]

We first show $I(\pi[A])$ contains $\pi[A]$ as an essential ideal. That it contains $A$ as an ideal is obvious. Non-degeneracy of the representation means that the only $x\in\mathcal{H}$ that is mapped to $0$ by all $\pi[A]$ is $0$, by \ref{nonDegeneracyAlgebraRepresentation}. 

The idealiser is the largest subalgebra of $\Bounded(\mathcal{H})$ that contains $A$ as an ideal. The ideal is necessarily 
\end{proof}

\begin{definition}
Let $A$ be a $C^*$-algebra. The \udef{multiplier algebra} $M(A)$ of $A$ is the largest $C^*$-algebra that contains $A$ as an essential ideal.
\end{definition}

The multiplier algebra is the non-commutative analogue of Stone–Čech compactification: if $A$ is commutative, then $A\cong C(X)$ and
\[ M(A) \cong C_b(X) \cong C(\beta(X)), \]
where $\beta(X)$ denotes the Stone–Čech compactification of $X$.

\begin{lemma}
If $A$ is a unital $C^*$-algebra, then $M(A) = A$.
\end{lemma}
If we view the $C^*$-algebra $A$ as a Hilbert $A$-module, then $M(A)$ is the set of adjointable operators on $A$.

\begin{proposition}
Let $A$ be a $C^*$-algebra and $\pi_1, \pi_2$ faithful, non-degenerate representations. Then the idealisers $I(\pi_1[A]), I(\pi_2[A])$ of $\pi_1[A]$ and $\pi_2[A]$ are isomorphic to each other and to the multiplier algebra $M(A)$.


$M(A)$ can be realised as the idealiser
\[ M(A) \cong I(\pi[A]) = \setbuilder{T\in \Bounded(\mathcal{H})}{T\pi[A] \cup \pi[A]T \subseteq \pi[A]} \]
of $A$ in $\Bounded(\mathcal{H})$.
\end{proposition}
\begin{proof}
We need to show
\[ I(\pi_1[A]) = \setbuilder{T\in \Bounded(\mathcal{H}_1)}{T\pi_1[A] \cup \pi_1[A]T \subseteq \pi_1[A]} \cong \setbuilder{T\in \Bounded(\mathcal{H}_2)}{T\pi_2[A] \cup \pi_2[A]T \subseteq \pi_2[A]} = I(\pi_2[A]). \]

We first show $I(\pi[A])$ contains $\pi[A]$ as an essential ideal. That it contains $A$ as an ideal is obvious. Non-degeneracy of the representation means that the only $x\in\mathcal{H}$ that is mapped to $0$ by all $\pi[A]$ is $0$, by \ref{nonDegeneracyAlgebraRepresentation}. 

The idealiser is the largest subalgebra of $\Bounded(\mathcal{H})$ that contains $A$ as an ideal. The ideal is necessarily 
\end{proof}

For example, let $\mathcal{H}$ be a Hilbert space. Then $M(\mathcal{K}(\mathcal{H})) = \Bounded(\mathcal{H})$, where $\mathcal{K}(\mathcal{H})$ is the algebra of compact operators on $\mathcal{H}$.

We write $\mathcal{U}M(A)$ to mean the unitary elements of the multiplier algebra.

Let $\pi: A\to M(B)$ be a $*$-homomorphism. If $\overline{\Span}(\pi(A)B) = B$, then $\pi$ can be uniquely extended to $\overline{\pi}: M(A) \to M(B)$. 

\section{Universal $C^*$-algebras}
\begin{definition}
Let $\mathcal{X}$ be a non-empty set. We formally write $\mathcal{X}^* = \setbuilder{x^*}{x\in \mathcal{X}}$ and view it as a set disjoint from $\mathcal{X}$. A noncommutative-$*$-polynomial with variables in $\mathcal{X}$ is a formal expression of the form
\[ \sum_{k=1}^m\lambda_k x_{k,1}x_{k,2}\ldots x_{k,n_k} \]
where $m, n_k\in \N$, $x_{k,n}\in \mathcal{X}\cup\mathcal{X}^*$ and $\lambda_k\in \C$.

a \udef{polynomal relation} $\mathcal{R}$ on $\mathcal{X}$ is a collection of formal statements of the form
\[ \norm{p_j(\mathcal{X})}\leq r_j \]
indexed by some index set $J$ where $r_j \in\R^{\geq 0}$ and $p_j$ is a noncommutative-$*$-polynomial with variables in $\mathcal{X}$.
\end{definition}

\begin{definition}
Let $\mathcal{X}$ be a non-empty set and $\mathcal{R}$ a set of polynomial relations on $\mathcal{X}$.

A \udef{representation} of $(\mathcal{X}\;|\;\mathcal{R})$ is a $C^*$-algebra $A$ together with a map $\pi:\mathcal{X}\to A$ such that $\mathcal{R}$ becomes true in the image of $\pi$.

A representation $\pi_u:\mathcal{X}\to B$ of $(\mathcal{X}\;|\;\mathcal{R})$ is called \udef{universal} if for any other representation $\pi:\mathcal{X}\to A$ of $(\mathcal{X}\;|\;\mathcal{R})$, there exists a unique $*$-homomorphism $\varphi: B\to A$ such that $\varphi\circ \pi_u = \pi$.

In this case we call $B$ the \udef{universal $C^*$-algebra} generated by $(\mathcal{X}\;|\;\mathcal{R})$ and write $B= C^*(\mathcal{X}\;|\;\mathcal{R})$.
\end{definition}

\begin{definition}
A polynomial relation $\mathcal{R}$ on $\mathcal{X}$ is said to be \udef{bounded}, if for every $x\in \mathcal{X}$, we have
\[ \sup\setbuilder{\norm{\pi(x)}}{\pi:\mathcal{X}\to A\;\text{is a representation of}\;(\mathcal{X}\;|\;\mathcal{R})} < \infty. \] 
\end{definition}

\begin{example}
\begin{itemize}
\item The relation $(\mathcal{X}\;|\;\mathcal{R}) = (\{a\}\;|\;\{\norm{a-a^*}\leq 0\})$ is not bounded.
\item The relation $(\mathcal{X}\;|\;\mathcal{R}) = (\{x,y\}\;|\;\{\norm{\vec{1}-x^*x-y^*y}\leq 0\})$ is bounded: writing $x$ for $\pi(x)$, the spectrum of $x^*x = 1-y^*y$ is positive and bounded by $1$ according to the spectral mapping theorem \ref{spectralMappingCFC}. Now $\norm{\pi(x)} = \sqrt{\spr(\pi(x)^*\pi(x))} \leq 1$ by \ref{normNormalElement} and so
\[ \sup\setbuilder{\norm{\pi(x)}}{\pi:\mathcal{X}\to A\;\text{is a representation of}\;(\mathcal{X}\;|\;\mathcal{R})} \leq 1 < \infty. \]
\item The relation $(\mathcal{X}\;|\;\mathcal{R}) = (\mathcal{X}\;|\;\bigcup_{x\in\mathcal{X}}\{\norm{x-x^*}\leq 0,\norm{x-x^2}\leq 0 \})$ is bounded. It gives rise to a universal $C^*$-algebra generated by projections.
\end{itemize}
\end{example}

\begin{proposition}
Let $\mathcal{X}$ be a non-empty set and $\mathcal{R}$ a polynomial relation on $\mathcal{X}$. Then $(\mathcal{X}\;|\; \mathcal{R})$ is bounded if and only if $C^*(\mathcal{X}\;|\;\mathcal{R})$ exists.
\end{proposition}
\begin{proof}
TODO
\end{proof}

\section{Direct limits}
\subsection{AF}
\subsection{UHF}
\subsection{Stable algebras}
\url{http://web.math.ku.dk/~rordam/manus/encyc.pdf}

\section{Tensor products}
\subsection{Algebraic tensor product}
\subsection{Spatial tensor product}

\chapter{Hilbert-Schmidt space}
\begin{proposition}
\begin{enumerate}
\item $\phi$ unital iff $\phi^*$ trace-preserving
\item
\end{enumerate}
\end{proposition}


\chapter{Group algebras and harmonic analysis}
\section{Group $C^*$-algebras}
\subsection{Discrete groups}
\begin{definition}
Let $G$ be a finite group and $R$ a r(i)ng. The \udef{group ring} $RG$ is the set of functions $(G\to R)$ with pointwise addition and the convolution product
\[ (x\star y)(g) = \sum_h x(h)y(h^{-1}g) = \sum_{g=hk}x(h)y(k) \]
for all $x,y\in RG$ and $g\in G$. 
\end{definition}
The a group ring can be seen as a free module generated by $G$. (TODO: this as definition?)

The group algebra $\C G$ has an involution:
\[ x^*(g) = \overline{x(g^{-1})} \qquad \text{for all $x\in \C G$}. \]
And it admits a norm making it a $C^*$-algebra.

\subsection{Locally compact Hausdorff groups}
For topological groups we are not restricted to finite sums.

\begin{definition}
Let $G$ be a locally compact group and $f,g\in L^1(G)$. The \udef{convolution product} $f\star g$ of $f$ and $g$ is the partial function defined by
\[ (f\star g)(x) = \int_Gf(y)g(y^{-1}x)\diff{y}, \]
whenever the integral exists.
\end{definition}

\begin{proposition}
Let $G$ be a locally compact group and $f,g\in L^1(G)$. Then convolution makes $L^1(G)$ a Banach-$*$-algebra.
\end{proposition}
We need to show:
\begin{enumerate}
\item $f\star g$ exists a.e.;
\item $\norm{f\star g}_1 = \int_G |f\star g|\diff{\mu}< \infty$;
\item $\norm{f \star g}_1 \leq \norm{f}_1\norm{g}_1$;
\item the convolution is a bilinear and associative.
\end{enumerate}
\begin{proof}
TODO
\end{proof}

\begin{proposition}
The algebra $L^1(G)$ is commutative \textup{if and only if} $G$ is a commutative group.
\end{proposition}

\begin{lemma}
Let $G$ be a locally compact Hausdorff group. Convolution is a bilinear operation that maps $C_c(G)\times C_c(G)\to C_c(G)$ defined by
\[ (f\star g)(t) \defeq \int_Gf(s)g(s^{-1}t)\diff\mu(s). \]
\end{lemma}
\begin{proof}
Continuity follows from the dominated convergence theorem. Also
\[ \operatorname{supp}(f\star g)\subseteq \operatorname{supp}(f)\cdot\operatorname{supp}(g) \]
where $\cdot$ is the group multiplication.
\end{proof}
\begin{lemma}
The algebra $C_c(G)$ has an involutive anti-linear anti-automorphism
\[ *: f \mapsto f^* = (s\mapsto \overline{f(s^{-1})}\Delta(s^{-1})) \]
where $\Delta$ is the modular function on $G$. This means $C_c(G)$ is a $*$-algebra with $*$ as involution.
\end{lemma}


\section{$C^*$-dynamical systems}
\begin{definition}
A \udef{$C^*$-dynamical system} is a triple $(G,\alpha, A)$ consisting of a locally  
compact group $G$, a $C^*$-algebra $A$ and a homomorphism $\alpha$ of $G$ into $\Aut(A)$, such that $g \mapsto a_g(a)$ is continuous for all $a \in A$. 
\end{definition}

\subsection{Covariant homomorphisms and representations}
\begin{definition}
Let $(G,\alpha, A)$ be a $C^*$-dynamical system. A \udef{covariant homomorphism} into the multiplier algebra $M(D)$ of some $C^*$-algebra $D$ is a pair $(\rho, U)$ where
\begin{itemize}
\item $\rho: A\to M(D)$ is a $*$-homomorphism and
\item $U: G\to \Unitaries M(D)$ is a strictly continuous homomorphism between groups
\end{itemize}
satisfying
\[ \rho(\alpha_g(a))= U_g\rho(a)U_{g}^* \qquad \text{for all $g\in G$.} \]
We say $(\rho, U)$ is non-degenerate if $\rho$ is.
\end{definition}

\subsubsection{Integrated forms}
\begin{definition}
Given a covariant homomorphism $(\rho, U)$ on a $C^*$-dynamical system $(G,\alpha, A)$ into $M(D)$ we can parcel these two functions into one function $C_c(G,A)\to M(D)$, called the \udef{integrated form}
\[ (\rho \rtimes U) (f)  \defeq \int_G\rho(f(r))U_r \diff\mu(r) \]
where $\mu$ is the left Haar measure.
\end{definition}
\begin{lemma}
Let $(\rho, U)$ be a covariant homomorphism. Then $\rho \rtimes U$ is a $*$-homomorphism.
\end{lemma}

\subsubsection{Induced covariant morphisms}
Given a $*$-homomorphism $\rho: A\to M(D)$ we can extend it naturally to a covariant homomorphism.
\begin{definition}
Let $(G, \alpha, A)$ be a $C^*$-dynamical system and $\rho: A\to M(D)$ a $*$-homomorphism. Then the \udef{covariant homomorphism induced from $\rho$} $\Ind \rho$ is the covariant homomorphism $(\widetilde{\rho}, 1\otimes \lambda)$ of $(G, \alpha, A)$ into $M(D\otimes\Compact(L^2(G))$ where
\begin{itemize}
\item $\lambda: G\to \Unitaries(L^2(G))$ is the left regular representation of $G$ given by $(\lambda_s\xi)(t) =\xi(s^{-1}t)$;
\item $\rho$ is the composition
\[ \begin{tikzcd}
A \rar{\widetilde{\alpha}} & C_b(G,A) \ar[r,hook] & M(A\otimes C_0(G)) \rar{\rho\otimes M} & M(D\otimes \Compact(L^2(G)))
\end{tikzcd} \]
where $\widetilde{\alpha}: A\to C_b(G,A)$ is defined by $\widetilde{\alpha}(a)(s)= \alpha_{s^{-1}}(a)$ and 
\[ M: C_0(G)\to \Bounded(L^2(G)) = M(\Compact(L^2(G))) \]
denotes the representation by multiplication operators.
\end{itemize}
\end{definition}
The \udef{regular representation} of $(G, \alpha, A)$ is $\Lambda^G_A \defeq \Ind(\id_A)$.

\begin{lemma}
Let $\rho: A\to M(D)$ be a $*$-homomorphism. Then
\[ \Ind\rho = () \]
\end{lemma}

\subsubsection{Covariant representations}
\begin{definition}
A \udef{(covariant) representation} of a $C^*$-dynamical system $(G,\alpha, A)$ on a Hilbert space $\mathcal{H}$ is a covariant homomorphism $(\pi, U)$ into $M(\Compact(\mathcal{H})) = \Bounded(\mathcal{H})$.
\end{definition}

\begin{definition}
Covariant representations $(\pi, U)$ on $\mathcal{H}$ and $(\pi', U')$ on $\mathcal{H}'$ are \udef{unitarily equivalent} if there is a unitary operator $W: \mathcal{H}\to \mathcal{H}'$ such that
\[ \pi'(a) = W\pi(a)W^* \qquad \text{and} \qquad U_g' =  WU_gW^* \]
for all $a\in A,g\in G$.
\end{definition}

Suppose $(\pi,U)$ and $(\rho, V)$ are covariant representations on $\mathcal{H}$ and $\mathcal{V}$ respectively. Their direct sum $(\pi, U) \oplus (\rho, V )$ is the covariant representation $(\pi \oplus \rho, U \oplus V )$ on $\mathcal{H} \oplus \mathcal{V}$ given by $(\pi \oplus \rho)(a) \defeq \pi(a)\oplus \rho(a)$ and $(U \oplus V)_s \defeq U_s \oplus V_s$.
\subsection{Crossed products}
The crossed product $A \rtimes_\alpha G$ will be defined as the completion of $C_c(G,A)$, viewed as a $*$-algebra in a certain way, with respect to a certain norm.

First the algebra: the set of functions $G\to A$ with compact support naturally comes equipped with scalar multiplication and vectorial addition. We define the multiplication as
\[ f\star g: G \to \C: x\mapsto \int_G f(s)\alpha_s(g(s^{-1}x))\diff\mu(s) \]
and the involution $*$ by
\[ f^*: x\mapsto \Delta(x^{-1})\alpha_x(f(x^{-1})^*). \]
Notice the appearance of $\alpha$ in the definitions.

Next we define a norm. This will be done using integrated forms.
Let $(\pi, U)$ be a covariant representation of a $C^*$-dynamical system $(A,G,\alpha)$ on $\mathcal{H}$. Then
\[ (\pi \rtimes U) (f)  \defeq \int_G\pi(f(r))U_r \diff\mu(r)\]
defines a $*$-representation of the $*$-algebra $C_c(G,A)$ on $\mathcal{H}$, called the \udef{integrated form}.

Then we can define a norm, called the \udef{universal norm}, on $C_c(G,A)$ by
\[ \norm{f} \defeq \sup\setbuilder{\norm{(\pi\rtimes U)(f)}}{(\pi, U)\;\text{is a covariant representation of} \; (A,G,\alpha)} \]
The supremum\footnote{One may worry we are taking the supremum over a class and not a set (the covariant representations do not form a set). Luckily the class is a subclass of the real numbers and thus a set.} is finite because $\norm{f} \leq \norm{f}_1$.

The completion of $C_c(G,A)$ with respect to the universal norm is called the \udef{crossed product} $A \rtimes_\alpha G$.

In fact, when evaluating the supremum for the universal norm, we do not need to consider all representations of $(A,G,\alpha)$: Let $(\pi, U)$ be a covariant representation of $(A,G,\alpha)$ on $\mathcal{H}$. Let 
\[ \mathcal{E} \defeq \overline{\Span}\setbuilder{\pi(a)h}{a\in A; h\in \mathcal{H}} \]
be the \udef{essential subspace} of $\pi$. We call the corresponding subrepresentation $\operatorname{ess} \pi$. Because
\[ U_s h =  \pi(\alpha_{s^{-1}}(a))U_s\pi(a) h \qquad \forall a\in A, h\in\mathcal{H}, \]
it is clear $\mathcal{E}$ is invariant under $U$ as well. Call $U'$ the restriction of $U$ to $\mathcal{E}$.

Then $\norm{(\operatorname{ess}\pi\rtimes U')(f)} = \norm{(\pi\rtimes U)(f)}$ and so
\[ \norm{f} =\sup\setbuilder{\norm{(\pi\rtimes U)(f)}}{(\pi, U)\;\text{is a non-degenerate covariant representation of} \; (A,G,\alpha)} \]

\begin{proposition}
If $(A,G,\alpha)$ is a dynamical system, then the map sending
a covariant pair $(\pi, U)$ to its integrated form $\pi\rtimes U$ is a one-to-one correspondence
between non-degenerate covariant representations of $(A, G, \alpha)$ and non-degenerate
representations of $A\rtimes_\alpha G$. This correspondence preserves direct sums, irreducibility
and equivalence.
\end{proposition}

\subsubsection{Universal property}

In general the crossed product $A\rtimes_\alpha G$ does not contain a copy of either $A$ or $G$. The multiplier algebra $M(A\rtimes_\alpha G)$ does however: There exist injective homomorphisms
\[ i_A: A\to M(A\rtimes_\alpha G) \qquad i_G: G\to \mathcal{U}M(A\rtimes_\alpha G) \]
satisfying
\begin{enumerate}
\item $i_A(\alpha_r(a)) = i_G(r)i_A(a)i_G(r)^*$ for all $a\in A, r\in G$;
\item $A \rtimes_\alpha G = \overline{\Span}\setbuilder{i_A(a)\int_Gf(s)i_G(s)\diff\mu(s)}{a\in A, f\in C_c(G)}$;
\item if $(\pi, U)$ is a covariant representation of $(G,A,\alpha)$, then
\[ \pi = (\pi\rtimes U)\circ i_A \qquad \text{and} \qquad U = (\pi \rtimes U)\circ i_G. \]
\end{enumerate}
This is a universal property. Suppose another $C^*$-algebra $B$ and maps
\[ j_A: A\to M(B) \qquad \text{and} \qquad j_G: G\to \mathcal{U}M(B) \]
satisfy these conditions, then there exists an isomorphism $\Psi: A\rtimes_\alpha G \to B$ such that
\[ \Psi \circ i_A = j_A \qquad \text{and} \qquad \Psi\circ i_G = j_G. \]

The existence of such homomorphisms $i_A,i_G$ is proved by explicitly giving them.
They are defined by
\begin{align*}
(i_A(a)f)(t) &= af(t) & (i_G(s)f)(t) &= \alpha_s(f(s^{-1}t) \\
(fi_A(a))(t) &= f(t)\alpha_t(a) & (fi_G(s))(t) &= f(t^{-1}s)\Delta(s^{-1})
\end{align*}
for all $f\in C_c(G,A), a\in A, t\in G$.

We can use the existence of these embeddings to prove the proposition. We need to show the existence of an inverse of the map from non-degenerate covariant representations to integrated forms.

Let $\omega$ be a non-degenerate covariant representation of $A\rtimes_\alpha G$. Because it is non-degenerate, we can extend it uniquely to a representation of $M(A\rtimes_\alpha G)$. Using $i_A, i_G$ we can restrict this representation to a representation of $A$ and $G$. Together they form a covariant representation and one can check it is exactly the original representation $\omega$.

\subsubsection{Induced representations}

\part{Operator Equations}
\setcounter{chapter}{0} % Reset chapter counters
\chapter{Semigroups and evolution operators}
\section{Semigroups of linear operators}
\begin{definition}
Let $X$ be a Banach space and $\sSet{I, +, 0, \leq}$ a partially ordered normed group with positive cone $I^+$. We call a function $T: I^+ \to \Lin(X)$ a \udef{semigroup of linear operators} (or just \udef{semigroup}) if
\begin{itemize}
\item $T(0) = \id_X$;
\item $T(s+t) = T(s)T(t)$ for all $s,t \in I^+$.
\end{itemize}
A function $T: I \to \Lin(X)$ satisfying the same conditions is called a \udef{group of linear operators}.

A semigroup is called \udef{strongly continuous} if it is continuous when $I$ is equipped with the order convergence and $X$ with the strong operator topology.

We call $T$ \udef{bounded} if $T(t)$ is bounded for all $t\in I^+$.
\end{definition}

\begin{lemma}
A semigroup of linear operators $T:I^+\to \Lin(X)$ is bounded \textup{if and only if} $T[S]$ is a set of bounded operators for some $S\subseteq I+$ that generates $I^+$ as a monoid.
\end{lemma}

\begin{proposition}
Let $T:I^+\to \Lin(X)$ be a semigroup of operators on a Banach space. Then the following are equivalent:
\begin{enumerate}
\item $T$ is strongly continuous;
\item $T$ is strongly continuous at $0$;
\item $T(t)$ is uniformly bounded on some neighbourhood of $0$ and there exists a dense subset $D\subseteq X$ such that $\lim_{t\to 0} T(t)x = x$ for all $x\in D$.
\end{enumerate}
\end{proposition}
\begin{proof}
$(1) \Rightarrow (2)$ If a semigroup is strongly continuous, then it is obviously strongly continuous at $0$.

$(2) \Rightarrow (3)$ Because $T(0) = \id_X$, strong continuity at $0$ means that $\lim_{t\to 0} T(t)x = x$ for all $x\in X$. Thus this also holds for all $x$ in any (dense) subset of $X$.

For the uniform bound we use the uniform boundedness principle \ref{uniformBoundednessPrinciple}. It is then enough to show that there exists a neighbourhood $U$ of $0$ such that $\sup\setbuilder{\norm{T(t)x}}{t\in U} <\infty$ for all $x\in X$.

TODO

$(3) \Rightarrow (1)$

Conversely, assume a semigroup $T: I^+\to \Lin(X)$ is continuous at $0$. Take $t_0\in I^+$. By \ref{leftRightConvergence} it is enough to check that $T$ is left and right continuous at $t_0$.

First let $F$ be a filter in $\powerfilters(\upset t_0)$ that converges to $t_0$. Then $F - t_0 \in \powerfilters(I^+)$ and $F-t_0 \to 0$. Thus $T[F] = T[F-t_0 + t_0] = T[F-t_0]T[t_0] \to T[0]T[t_0] = T[t_0]$, meaning that $T$ is right continuous.

TODO left continuity.
\end{proof}
TODO cfr SOT.

\subsection{Growth bounds}
\begin{proposition}
Let $T: I^+ \to \Lin(X)$ be a strongly continuous semigroup of bounded linear operators. Then there exist contstants $w\in \R$ and $M\geq 1$ such that
\[ \forall t\in I^+:\quad \norm{T(t)} \leq Me^{w\norm{t}}. \]
\end{proposition}
\begin{proof}
Choose $M$ such that for all $0\leq s\leq 1$, $\norm{T(s)}\leq M$.
\end{proof}

\begin{definition}

\end{definition}

\subsection{Differentiability}
\begin{definition}
Let $T: I^+ \to \Lin(X)$ be a semigroup of linear operators. We say $T$ is \udef{differentiable} at $t\in I$ if the limit
\[  \]
\end{definition}

\subsection{$C_0$-semigroups}
\begin{definition}
A $C_0$-semigroup
\end{definition}

\section{One-parameter semigroups on Banach spaces}
In this section we will study functions $T: \R^+\to \Bounded(X)$ that are strongly continuous semigroups on a Banach spaces $X$. We will shorten this description to \udef{operator semigroup}.

\subsection{Orbit maps}
\begin{lemma}
Let $T: \R^+\to \Bounded(X)$ be an operator semigroup. Consider the orbit map $\xi_x: \R^+ \to X: t\mapsto T(t)x$ for some $x\in X$. The following are equivalent:
\begin{enumerate}
\item $\xi_x$ is differentiable;
\item $\xi_x$ is differentiable at $0$.
\end{enumerate}
\end{lemma}
\begin{proof}
Clearly $(1)$ implies $(2)$. Now assume $(2)$, we show $\xi_x$ is left and right differentiable at all points $t\in\R^+$ (TODO cfr. \ref{leftRightConvergence}). For right differentiability we can calculate
\begin{align*}
\lim_{h\downarrow 0} h^{-1}(\xi_x(t+h) - \xi_x(t)) &= \lim_{h\downarrow 0} h^{-1}(T(t+h)x - T(t)x) \\
&= T(t)\lim_{h\downarrow 0} h^{-1}(T(h)x - T(0)x) \\
&= T(t)\lim_{h\downarrow 0} h^{-1}(\xi_x(h) - \xi_x(0)) = \xi_x'(0).
\end{align*}
TODO
\end{proof}

\begin{lemma} \label{integrabilityOrbitMaps}
Let $T: \R^+\to \Bounded(X)$ be an operator semigroup. For all $x\in X$, the orbit map $\xi_x$ is integrable on any compact $K\subseteq [0,\infty]$.
\end{lemma}
\begin{proof}
By \ref{BochnerIntegrabilityCondition} we just need to show $\int_K\norm{\xi_x(s)}\diff{s} < \infty$. We have $\int_K\norm{\xi_x(s)}\diff{s} \leq \sup_{s\in K}\norm{\xi_x(s)}\lambda(K)$. Now $\lambda(K)$ is finite by \ref{HaarConsequences} and $\sup_{s\in K}\norm{\xi_x(s)}$ is finite by (TODO ref: extreme value theorem), as $\R \to \R: t\mapsto \norm{\xi_x(t)}$ is a continuous function.
\end{proof}

\subsection{The generator}
\begin{definition}
Let $T: \R^+\to \Bounded(X)$ be an operator semigroup. The \udef{generator} of $T$ is the operator $A: X\not\to X$ defined by
\[ Ax \defeq \xi_x'(0) = \lim_{h\downarrow 0}\frac{1}{h}\big(T(h)x - x\big) \]
with domain
\[ \dom(A) = \setbuilder{x\in X}{\text{$\xi_x$ is differentiable}}. \]
\end{definition}
Notice that the generator is indeed a linear operator.

\begin{lemma} \label{differentialOperatorSemigroupGenerator}
Let $T: \R^+\to \Bounded(X)$ be an operator semigroup with generator $A$. If $x\in\dom(A)$, then $T(t)x\in\dom(A)$ and
\[ \od{}{t}T(t)x = T(t)Ax = AT(t)x \]
for all $t\in [0,\infty]$.
\end{lemma}

\begin{proposition} \label{integralOperatorSemigroupGenerator}
Let $T: \R^+\to \Bounded(X)$ be an operator semigroup with generator $A$. For all $x\in X$, then $\int_0^t T(s)x\diff{s}\in\dom(A)$ and
\begin{align*}
T(t)x - x &= A\int_0^t T(s)x\diff{s}
\end{align*}
for all $t\in [0,\infty]$. Also
\begin{align*}
T(t)x - x &= \int_0^t T(s)Ax\diff{s}
\end{align*}
for all $x\in\dom(A)$ and $t\in [0,\infty]$.
\end{proposition}
This is an integrated version of \ref{differentialOperatorSemigroupGenerator}, except it holds for all $x\in X$.
\begin{proof}
Firstly $\int_0^t T(s)x\diff{s}$ exists and is finite by \ref{integrabilityOrbitMaps}.

To show $A\int_0^t T(s)x\diff{s}$ is well-defined, we calculate
\begin{align*}
A\int_0^t T(s)x\diff{s} &= \left.\dod{}{h}\right|_{h=0}T(h)\int_0^t T(s)x\diff{s} \\
&= \left.\dod{}{h}\right|_{h=0}\int_0^t T(h)T(s)x\diff{s} \\
&= \left.\dod{}{h}\right|_{h=0}\int_0^t T(s+h)x\diff{s} \\
&= \left.\dod{}{h}\right|_{h=0}\int_h^{t+h} T(s)x\diff{s} \\
&= \left.\dod{}{h}\right|_{h=0}\left(\int_0^{t+h} T(s)x\diff{s}-\int_0^{h} T(s)x\diff{s}\right) \\
&= \left.\dod{}{h}\right|_{h=t}\int_0^{h} T(s)x\diff{s}-\left.\dod{}{h}\right|_{h=0}\int_0^{h} T(s)x\diff{s} \\
&= T(t)x - x,
\end{align*}
where we have used \ref{boundedOperatorUnderIntegral} and (TODO ref: fundamental theorem calculus for Bochner integral+continuity to deal with equality a.e.).

Finally $\int_0^t T(s)Ax\diff{s}$ exists and is finite by \ref{integrabilityOrbitMaps} applied to $Ax$. We conclude using \ref{differentialOperatorSemigroupGenerator} and (TODO ref: second fundamental theorem of calculus).
\end{proof}

\begin{proposition}
The generator of an operator semigroup is closed and densely defined.
\end{proposition}
\begin{proof}
Let $T: \R^+\to \Bounded(X)$ be an operator semigroup with generator $A$. To show $A$ is closed, consider $\seq{x_n}\subseteq \dom(A)$ such that $x_n\to x$ and $Ax_n \to y$. From \ref{integralOperatorSemigroupGenerator} we have $T(t)x_n - x_n = \int_0^tT(s)Ax_n\diff{s}$. Thus
\[ T(t)x-x = \lim_{n\to\infty} T(t)x_n - x_n = \lim_{n\to\infty} \int_0^tT(s)Ax_n\diff{s}. \]
Now $Ax_n \to y$ means the sequence $\seq{Ax_n}$ is bounded by some constant $M$. Then $\norm{T(s)Ax_n} \leq M\cdot\sup_{0\leq s \leq t}\norm{T(s)}$, meaning we can apply the dominated convergence theorem (TODO ref). So $T(t)x-x = \int_0^tT(s)y\diff{s}$ and thus
\[ Ax = \lim_{t\downarrow 0}\frac{1}{t}(T(t)x-x) = \lim_{t\downarrow 0}\frac{1}{t}\int_0^tT(s)y\diff{s} = \left.\dod{}{t}\right|_{t=0}\int_0^tT(s)y\diff{s} = T(0)y = y, \]
by the fundamental theorem of calculus (TODO ref).

For any $x\in X$ we can find a sequence $\seq{x_n}$ in $\dom(A)$ that converges to $x$, namely
\[ \frac{1}{t_n}\int^{t_n}_0T(s)x\diff{s} \qquad \text{for any sequence $\seq{t_n}$ in $[0,\infty[$ converging to $0$.}  \]
This is a sequence in $\dom(A)$ by \ref{integralOperatorSemigroupGenerator} and converges to $x$ by TODO ref. Thus $\dom(A)$ is dense in $X$ by TODO ref.
\end{proof}

\begin{proposition}
An operator is the generator of at most one operator semigroup.
\end{proposition}
\begin{proof}
Let $T: \R^+\to \Bounded(X)$ and $S: \R^+\to \Bounded(X)$ be two operator semigroups with the same generator $A$. Then for any $t\in [0,\infty[$ and $x\in \dom(A)$,
\begin{align*}
\dod{}{s} \left(T(t-s)S(s)x\right) &= T(t-s)S'(s)x - T'(t-s)S(s)x \\
&= T(t-s)As(s)x - T(t-s)As(s)x \\
&= 0
\end{align*}
by (TODO ref!!!). TODO rest.
\end{proof}

\begin{proposition}
Let $T: \R^+\to \Bounded(X)$ be an operator semigroup with generator $A$. Then the following are equivalent:
\begin{enumerate}
\item $A$ is bounded;
\item $A$ is defined everywhere;
\item $T$ is uniformly continuous.
\end{enumerate}
In each case $T(t) = e^{tA}$.
\end{proposition}
\begin{proof}
$(1) \Rightarrow (2)$ Because $A$ is closed, $\dom(A)$ is closed by the closed graph theorem \ref{closedGraphTheorem}. As $\dom(A)$ is dense in $X$, this means $\dom(A) = X$.

$(2) \Rightarrow (1)$ TODO also closed graph.

$(3)$ TODO
\end{proof}

\subsubsection{Cores of the generator}
TODO move:
\begin{lemma} \label{uniformContinuityAverage}
Let $\seq{f_n: [a,b]\to X}$ be a sequence of functions from $[a,b]\subseteq \R$ to a Banach space $X$ that converges uniformly to $f: X\to Y$. Then
\[ [a,b]\to X: t\mapsto \frac{1}{t}\int_{0}^tf_n(s)\diff{s} \quad\longrightarrow\quad [a,b]\to X: t\mapsto \frac{1}{t}\int_{0}^tf(s)\diff{s} \qquad \text{uniformly.} \]
\end{lemma}
\begin{proof}
We have
\begin{align*}
\sup_{t\in[a,b]}\frac{1}{t}\int_0^t \norm{f_n(s) - f(s)}\diff{s} &\leq \sup_{t\in[a,b]}\frac{1}{t}\int_0^t \diff{s}\sup_{s\in [a,b]} \norm{f_n(s) - f(s)} \\
&= \sup_{s\in [a,b]} \norm{f_n(s) - f(s)} \to 0.
\end{align*}
\end{proof}

\begin{lemma} \label{uniformContinuityOrbitMapsConvergentSequence}
Let $T: \R^+\to \Bounded(X)$ be an operator semigroup and $\seq{x_n} \to x$ a convergent sequence in $X$. Then $f_n: [0,1]\to X: t\mapsto T(t)x_n$ converges uniformly to $f: [0,1]\to X: t\mapsto T(t)x$.
\end{lemma}
\begin{proof}
The image of $[0,1]$ under $t\mapsto \norm{T(t)}$ is compact and thus bounded, with bound $M$. Then
\begin{align*}
\sup_{t\in [0,1]}\norm{f_n(t) - f(t)} &= \sup_{t\in [0,1]}\norm{T(t)x_n - T(t)x} \\
&= \sup_{t\in [0,1]}\norm{T(t)(x_n - x)} \\
&\leq M\norm{(x_n - x)} \to 0.
\end{align*}
\end{proof}

\begin{proposition} \label{coreGeneratorCriterion}
Let $T: \R^+\to \Bounded(X)$ be an operator semigroup with generator $A$ and $D$ a subspace of $\dom(A)$. If $D$ is norm dense in $X$ and invariant under $T(t)$ for all $t\in [0,+\infty]$, then $D$ is a core for $A$.
\end{proposition}
Note norm density in $X$ is the same as norm density in $\dom(A)$, because $A$ is densely defined.
\begin{proof}
We use \ref{operatorCoreCriterion}. Take $x\in \dom(A)$; we need to show that $x\in\closure_{\norm{\cdot}_A}(D)$.

Because $D$ is norm dense in $X$, we can find a sequence $\seq{x_n}$ in $D$ that norm converges to $x$. We claim it also converges in the graph norm. To show that, it is enough to show that $\seq{Ax_n} \to Ax$ in norm.

Combining \ref{uniformContinuityOrbitMapsConvergentSequence} and \ref{uniformContinuityAverage} gives that
\[ [0,1] \to X: t\mapsto \frac{1}{t}\int_0^t T(s)x_n\diff{s} \]
converges uniformly in $n$. Thus by \ref{integralOperatorSemigroupGenerator} and Moore-Osgood (TODO ref) we have
\begin{align*}
\lim_{n\to \infty} Ax_n &= \lim_{n\to \infty} \lim_{t\downarrow 0}\frac{1}{t}\int_0^t T(s)Ax_n \diff{s} \\
&= \lim_{n\to \infty} \lim_{t\downarrow 0}\frac{1}{t}(T(t)x_n - x_n) \\
&= \lim_{t\downarrow 0}\frac{1}{t}\lim_{n\to \infty}(T(t)x_n - x_n) \\
&= \lim_{t\downarrow 0}\frac{1}{t}(T(t)x - x) = Ax.
\end{align*}

TODO

Take a sequence $\seq{t_n}\to 0$ in $\R$. Then we claim
\[ \seq{x'_n} = \seq{\frac{1}{t_n}\int_0^{t_n}T(s)x_n\diff{s}} \]
is a sequence in $D$ that converges to $x$ in the graph norm. It is definitely a sequence in $D$ by the invariance of $D$ under $T(t)$. It is now enough to show $\seq{x'_n}$ norm converges to $x$ and $\seq{Ax'_n}$ norm converges to $Ax$.
\end{proof}

\begin{proposition}
Let $T: \R^+\to \Bounded(X)$ be an operator semigroup with generator $A$. Then $\bigcap_{n\in\N}\dom(A^n)$ is a core for $A$.
\end{proposition}
\begin{proof}
The space $\bigcap_{n\in\N}\dom(A^n)$ is invariant under $T$ by \ref{differentialOperatorSemigroupGenerator}. In order to use \ref{coreGeneratorCriterion}, we need to verify that $\bigcap_{n\in\N}\dom(A^n)$ is norm dense in $\dom (A)$. TODO
\end{proof}

\subsubsection{Spectral properties of resolvents}
\begin{proposition}
Let $T: \R^+\to \Bounded(X)$ be an operator semigroup with generator $A$ and growth bound $\norm{T(t)}\leq Me^{wt}$. If $\int_0^\infty e^{-\lambda s}T(s)x\diff{s}$ exists for all $x\in X$, then $\lambda\in\res(A)$ and
\[ R_A(\lambda)x = \int_0^\infty e^{-\lambda s}T(s)x\diff{s} \]
for all $x\in X$.
\end{proposition}
\begin{proof}
We use \ref{elementResolventSetNormedSpace}.
\end{proof}
\begin{corollary}
et $T: \R^+\to \Bounded(X)$ be an operator semigroup with growth bound $\norm{T(t)}\leq Me^{wt}$. Then
\begin{enumerate}
\item $\Re \lambda > w$ implies $\lambda \in\res(A)$;
\item $\norm{R_A(\lambda)} \leq \frac{M}{\Re \lambda - w}$ for all $\lambda$ such that $\Re \lambda > w$.
\end{enumerate}
\end{corollary}
\begin{proof}
We estimate
\[ \norm{\int_0^t e^{-\lambda s}T(s)x\diff{s}} \leq M\int_0^t e^{(w-\Re\lambda)s}\diff{s}\norm{x} \]
\end{proof}


\chapter{Operator equations}
Linear Operator Equations: Approximation and Regularization
\section{Terminology}
\begin{definition}
Let $T: X\not\to Y$ be a (non-)linear operator between normed spaces. An \udef{operator equation} is an equation of the form
\[ T(u) = f \qquad f\in Y\; \text{$u$ unknown.} \]
A \udef{solution} to is operator equation is a vector $a\in X$ such that $T(a) = f$.
\end{definition}
\subsection{Well- and ill-posed problems}
Some natural questions associated to such problems are:
\begin{enumerate}
\item Whether a solution exists for a given $f\in Y$.
\item Whether this solution is unique.
\item Whether this solution depends continuously on $f$, i.e.\ whether the solution is stable under perturbation.
\end{enumerate}
The later questions depend on the affirmative answers of the former. A problem is called \udef{well-posed} if the answer to all three questions is positive and \udef{ill-posed} if not. The terminology is due to Jacques Hadamard (TODO ref Hadamard, Jacques (1902). Sur les problèmes aux dérivées partielles et leur signification physique. Princeton University Bulletin. pp. 49–52.)

\begin{proposition}
Let $ T(u) = f$ be an operator equation where $T$ is a linear operator. Then the problem of solving this operator equation is well-posed for all $f\in\im(T)$ \textup{if and only if} $T$ is bounded below.
\end{proposition}
\begin{proof}
By \ref{boundedBelow}.
\end{proof}

\begin{proposition}
Let $T$ be a linear operator and $\lambda \in \C$. Consider the operator equation
\[  Tu = \lambda u + f. \]
This problem is well-posed if and only if $\lambda\in \rho(T)$.

In particular, 
\begin{enumerate}
\item if $\lambda\in\sigma_\text{p}(T)$, then uniqueness fails.
\item if $\lambda\in\sigma_\text{r}(T)$, then existence fails for some $f$,
\item if $\lambda\in\sigma_\text{c}(T)$, then the solution does not depend continuously on $f$. TODO: verify with (re)definition of continuous spectrum.
\end{enumerate}
\end{proposition}

\subsection{Equations of the first and second kind}
\begin{definition}
\begin{itemize}
\item An operator equation of the form $T(u) = f$ is said to be of the \udef{first kind}
\item An operator equation of the form $\lambda u - T(u) = f$ for some non-zero scalar $\lambda$ is said to be of the \udef{second kind}.
\end{itemize}
\end{definition}

\section{Equations on function spaces}
\subsection{Relevant spaces}

\subsection{Solutions}
classical, weak, distributional.

\subsubsection{Green's functions}


\subsection{Boundary conditions}
It is often useful to identify a subset of functions using boundary conditions.
\begin{definition}
Let $X$ be a topological space and $\Omega$ a closed subset. A \udef{boundary condition} is an operator equation of the form
\[ \big[Tu\big]_{\partial\Omega} = f \]
where $f$ is a function on $\partial\Omega$. The operator $B\defeq u\mapsto \big[Tu\big]_{\partial\Omega}$ is called a \udef{boundary operator}.

The boundary condition is called
\begin{itemize}
\item \udef{Dirichlet} or \udef{first-type} if $T = \id$;
\item \udef{Neumann} or \udef{second-type} if $\Omega$ is a Riemannian manifold and $T = \partial_{\vec{n}}$ where $\vec{n}$ is the vector normal to $\Omega$; (TODO: correct setting??)
\item \udef{Robin} or \udef{third-type} if $T = \id + \alpha \partial_{\vec{n}}$ for some non-zero constant $\alpha$.
\end{itemize}
The boundary condition is called \udef{homogenous} if $f\equiv 0$.

If the boundary condition is specified on a subset of $\partial\Omega$, then we have a \udef{partial boundary condition}.
\begin{itemize}
\item If $\partial\Omega$ is partitioned with a partial boundary condition specified on each partition, we call this a \udef{mixed boundary condition}.
\item A \udef{Cauchy boundary condition} consists of both a partial Dirichlet and a partial Neumann boundary condition.
\end{itemize}
\end{definition}
TODO: An Introduction to the Finite Element Method - Reddy + TODO generalised functions?

\begin{lemma}
A boundary condition is of the first, second or third type \textup{if and only if} it is of the general form
\[ \Big[(\alpha\id + \beta\partial_{\vec{n}})u\Big]_{\partial\Omega} = f \]
where $\alpha, \beta$ are constants that are not both zero.

In particular, the boundary condition is
\begin{enumerate}
\item Dirichlet if $\beta = 0$;
\item Neumann if $\alpha = 0$;
\item Robin if $\alpha \neq 0 \neq \beta$.
\end{enumerate}
\end{lemma}

\begin{lemma}
Functions obeying a homogeneous boundary condition with linear boundary operator form a subspace.
\end{lemma}
\begin{proof}
These functions are exactly the functions in the kernel of the boundary operator.
\end{proof}

\subsection{Periodic boundary conditions}
TODO

\section{Linear dynamical systems}



\section{Approximating well-posed problems}
\section{Regularising ill-posed problems}
\section{Regularised approximation methods}


\chapter{Ordinary differential equations}
\url{https://www.mat.univie.ac.at/~gerald/ftp/book-ode/ode.pdf}
\url{file:///C:/Users/user/Downloads/978-3-030-47849-0.pdf}
\url{file:///C:/Users/user/Downloads/Polyanin%20A.,%20D.,%20Zaitsev%20V.%20F.,%20Handbook%20of%20exact%20solutions%20for%20ordinary%20differential%20equations.pdf}

\section{Classification}
\begin{definition}
An \udef{$n^\text{th}$ order (ordinary) differential equation} (or ODE) is an equation of the form
\[ F(t, u, u', u^{\prime\prime}, \ldots, u^{(n)}) \equiv 0. \]
We call a real function $u: [a,b] \to \R$ a \udef{solution} of this differential equation on $[a,b]$
if it has at least $n$ continuous derivatives such that $F(t, u(t), u'(t), u^{\prime\prime}(t), \ldots, u^{(n)}(t))$ is zero for all $t\in [a,b]$. The set of all solutions is called the \udef{general solution}.

We call the differential equation
\begin{enumerate}
\item \udef{linear} if $\md{F}{2}{u^{(i)}}{}{u^{(j)}}{} \equiv 0$ for all $i,j\in [0,n]$;
\item \udef{homogenous} if $F(t, 0,\ldots, 0) = 0$.
\end{enumerate}
Unless explicitly stated, we will always assume that $F$ can be solved for the highest derivative, such that the ODE can be written as
\[ u^{(n)} \equiv f(t,u,u',\ldots, u^{(n-1)}). \]
\end{definition}

A linear $n^\text{th}$ order differential equation can be written in the form
\[ \sum_{i=0}^n a_{i}(t)u^{(i)}(t) - g(t) = 0 \]
where $a_i, g \in (]a,b[\to\R)$.

\begin{definition}
Let $\sum_{i=0}^n a_{i}(t)u^{(i)}(t) \equiv g(t)$ be a linear differential equation. We call the functions $a_i$ the \udef{coefficients} and we say the ODE has \udef{constant coefficients} if all the $a_i$ are constants.
\end{definition}

In the linear case we can introduce the linear operator
\[ L \defeq \sum_{i=0}^n a_{i} \left(\dod{}{t}\right). \]
Then the differential equation can be written as $Lu = g$.

The differential equation is homogenous if and only if $g = 0$.

\subsection{Differential problems}
\subsubsection{Initial value problems}
\begin{definition}
An \udef{initial value problem} (IVP) is an $n^{\text{th}}$ order ODE together with \udef{initial conditions}
\[ u^{(i)}(t_0) = c_i \qquad i \in 0:n-1 \]
where $t_0$ and the $c_i$ are real numbers.
\end{definition}

TODO: replace derivatives with Lipschitz continuity.
\begin{theorem}
Let $u^{(n)} \equiv f(t,u,u',\ldots, u^{(n-1)})$ be an $n^{\text{th}}$ order ODE with initial conditions
\[ u^{(i)}(t_0) = c_i \qquad i \in 0:n-1. \]
Assume
\[ f, \pd{f}{t}, \pd{f}{u}, \pd{f}{u'}, \ldots, \pd{f}{u^{(n-1)}} \]
are defined and continuous on a neighbourhood of $(t_0, c_0, \ldots, c_{n-1})\in \R^{n+1}$.

Then there exists $\epsilon > 0$ such that the IVP has a unique solution on the interval $]t_0-\epsilon, t_0+\epsilon[$.
\end{theorem}
\begin{proof}
TODO
\end{proof}

It is possible for a solution $u$ to be defined outside the interval $]t_0-\epsilon, t_0+\epsilon[$, but not be a solution to the IVP.
\begin{example}
Consider the IVP
\[ u' = u^2 \qquad u(0) = c. \]
The function
\[ u: \R\setminus\{1/c\} \to \R: t\mapsto \frac{c}{1-ct} \]
is the solution only for $t < 1/c$.
\end{example}

\subsubsection{Boundary value problems}
\begin{definition}
A \udef{boundary value problem} (BVP) is an $n^{\text{th}}$ order ODE together with boundary conditions.
\end{definition}

The questions of existence and uniqueness are less clear than for IVPs.
\begin{example}
Consider the ODE
\[ u''(t) + u(t) = 0 \qquad 0<t<\pi. \]
The general solution is $u(t) = c_1\sin t + c_2 \cos t$. All solutions have $u(0) = u(\pi)$.
So the BVP with boundary conditions
\[ u(0) = 0 \qquad u(\pi) = 1 \]
has no solution and the BVP with boundary conditions
\[ u(0) = 0 \qquad u(\pi) = 0 \]
has infinitely many solutions.
\end{example}


\subsection{Systems of differential equations}

\section{Existence and uniqueness}
\subsection{Existence}
Generally solutions to IVPs exist if $f$ (TODO ref) is continuous. Such existence results are generally only local.

\begin{example}
Consider $u' = u^2$ with condition $u(1) = -1$. A solution is given by $u(t) = -t^{-1}$. This solution does not exist at $0$, even though $u\mapsto u^2$ is continuous.
\end{example}

\subsubsection{Picard-Lindelöf theorem}
\begin{theorem}[Picard-Lindelöf]
Consider $f: \R\times \R^d \to \R^d$ and the differential problem of finding $y:\R\to \R^d$ such that
\[ y' = f(t,y) \qquad y(t_0) = y_0 \qquad (t_0\in \R, y_0\in\R^d). \]
Let $R$ be the rectangle $[t_0,t_0+a]\times B(y_0, b)$ for some $a,b\in \R$.
If $f$ is continuous on $R$  and uniformly Lipschitz continuous w.r.t. $y$, then the differential problem has a unique solution $y(t)$ on $[t_0,t_0+\alpha]$, where $\alpha = \min(a,b/M)$ and $M$ is a bound for $|f(t,y)|$ on the rectangle $R$.
\end{theorem}
Any norm on $\R^d$ can be used to define $B(y_0, b)$, as they are all equivalent.
\begin{proof}
For any potential solution $y(t)$, the function $t\mapsto f(t,y(t))$ is integrable (TODO ref). So the solution must satisfy \[ y(t) = y_0 + \int_{t_0}^tf(s, y(s))\diff{s}. \]
We use this the construct a sequence of approximate solutions. Set $y_0: t\mapsto y_0$ and
\[ y_{n+1}: [t_0, t_0+a] \to \R^d: t\mapsto y_0 + \int_{t_0}^t f(s,y_n(s))\diff{s}. \]
These integrals can be taken because the graph of each $y_n$ lies in the rectangle $R$:
\[ |y_n(t) - y_0| \leq \int_{t_0}^t|f(s,y_{n-1}(s))|\diff{s} \leq M\alpha \leq b. \]

TODO
\end{proof}

\subsubsection{Peano's existence theorem}
\begin{theorem}[Peano's existence theorem]
Consider $f: \R\times \R^d \to \R^d$ and the differential problem of finding $y:\R\to \R^d$ such that
\[ y' = f(t,y) \qquad y(t_0) = y_0 \qquad (t_0\in \R, y_0\in\R^d). \]
Let $R$ be the rectangle $[t_0,t_0+a]\times B(y_0, b)$ for some $a,b\in \R$.

If $f$ is continuous on $R$ and $|f(t,y)|$ is bounded on $R$ with bound $M$, then the differential problem has at least one solution $y(t)$ on $[t_0,t_0+\alpha]$.
\end{theorem}
Any norm on $\R^d$ can be used to define $B(y_0, b)$, as they are all equivalent.
\begin{proof}
TODO
\end{proof}

\subsection{Differential inequalities}
\subsection{Dependence on initial conditions and parameters}

\section{First order differential equations}
\subsection{Existence and uniqueness}
\subsection{Qualitative properties of solutions}
\subsection{A miscellany of solutions for different types of equations}
\subsubsection{Separable equations}
\paragraph{The logistic equation.}
\subsubsection{Exact equations}
\subsubsection{Linear first order differential equations}
\subsubsection{Homogeneous equations}
\subsubsection{Bernoulli equations}

\section{Systems of equations and higher order equations}
\subsection{Problem statement and notation}
Equivalence systems and higher order
\subsection{Existence and uniqueness}
\subsection{Second order equations}
\subsection{Higher order linear equations}
\subsection{Systems of first order equations}

\section{Qualitative analysis}

\section{Solutions by infinite series and Bessel functions}

\section{Second order differential equations}
\subsection{Solutions with Green's functions}
\begin{proposition}
Consider a second order linear differential equation on an interval $[a,b]$, which is of the general form
\[ Lu = a_2u^{\prime\prime} + a_1u' + a_0u = f \]
where $a_2,a_1, a_0, f\in \cont([a,b])$.
Consider mixed homogenous boundary conditions of first, second or third type, i.e.\ of the form
\begin{align*}
B_au = c_1 u(a) + c_2 u'(a) &= 0 \\
B_bu = c_3 u(b) + c_4 u'(b) &= 0
\end{align*}
where at least one of $c_1,c_2$ and $c_3,c_4$ is non-zero.

Assume that $a_2(x) \neq 0$ for all $x\in [a,b]$, $f\in L^2([a,b])$ and the kernel of $L$ is trivial.

Then there exists a unique solution of the form
\[ u(x) = \int_{a}^b G(x,y)f(y)\diff{y} \]
where $G$ is a bounded function in $([a,b]\times [a,b] \to \C)$.
\end{proposition}
\begin{proof}
We want to find a kernel $G$ such that
\[ L(G(\cdot, y)) = \delta_y \]
as distributions for all fixed $y\in [a,b]$ and
\[ B_aG(\cdot, y) = 0 = B_bG(\cdot, y). \]
TODO
\end{proof}

\subsection{Sturm-Liouville theory}
\subsubsection{Strum-Liouville problems and operators}
\begin{definition}
A \udef{Sturm-Liouville equation} is a real second-order ODE of the form
\[ -(pu')' + qu = \lambda \omega u \]
where $\lambda\in\R$ and $p,p',q,\omega\in \cont([a,b])$ for some $a,b\in \R$. Also $p$ is assumed strictly positive on $]a,b[$ and $\omega$ strictly positive on $[a,b]$.

A \udef{Sturm-Liouville operator} is a linear operator on the Sobolev space $W^{2,2}([a,b])$ of the form
\[ L: u\mapsto \frac{1}{\omega}\Big(-(pu')' + qu\Big) \]
where $p,q, \omega$ are as above.

A \udef{Sturm-Liouville problem} is a Sturm-Liouville equation with mixed homogenous boundary conditions of first, second or third type, i.e.\ of the form
\[ c_1 u(t_b) + c_2 u'(t_b) = 0 \]
where $t_b$ is $a$ or $b$ and at least one of $c_1,c_2$ is non-zero.

We call the Sturm-Liouville problem
\begin{itemize}
\item \udef{regular} if $p$ is strictly positive on $[a,b]$ and boundary conditions are specified at $a$ and $b$;
\item \udef{singular} if any of the following hold:
\begin{itemize}
\item $p(a) = 0$ and there is no boundary condition at $a$;
\item $p(b) = 0$ and there is no boundary condition at $a$;
\item $p(a) = 0 = p(b)$ and there are no boundary conditions; or
\item $[a,b]$ is infinite.
\end{itemize}
\end{itemize}
\end{definition}
For finite $[a,b]$, all $u\in\cont([a,b])$ are square integrable: $u$ is necessarily bounded by \ref{imageCompactIsClosedBounded} and the integral is bounded by this bound times $|b-a|$. If $[a,b]$ is infinite we still require solutions to be square integrable.

We may consider $\cont([a,b]) \subset (\R\to\C)$, in which case the Sturm-Liouville operator maps real functions to real functions.

\begin{proposition}
Consider a Sturm-Liouville problem. Take the Sobolev space $W^{2,2}([a,b], \omega(x)\diff{x})$ and let $\mathcal{H}$ be the subspace of functions that obey the boundary conditions. Then the Sturm-Liouville operator $L$ restricted to $\mathcal{H}$ is self-adjoint.
\end{proposition}
We take $Lu$ to be defined if $Lu\in\cont([a,b])$. If $[a,b]$ is infinite, we 
\begin{proof}

\end{proof}

\chapter{Partial differential equations}
Transport equation, Laplace's equation, Heat equation, Wave equation
\section{Classification}
\begin{definition}
An \udef{$n^\text{th}$ order partial differential equation} (or PDE) is an equation of the form
\[ F(x, \{D^\alpha u\}_{|\alpha|\leq m}) \equiv 0. \]
We call a  function $u: \Omega \subset \R^N \to \R$ a \udef{solution} of this differential equation on $\Omega$ if $D^\alpha u$ exists and is continuous for $|\alpha|\leq m$ and $F(x, \{D^\alpha u(x)\}_{|\alpha|\leq m})$ is zero for all $x\in \Omega$.

The set of all solutions is called the \udef{general solution}.

We call the differential equation
\begin{enumerate}
\item \udef{linear} if $\md{F}{2}{(D^\alpha u)}{}{(D^\beta u)}{} \equiv 0$ for $|\alpha|,|\beta|\leq m$;
\item \udef{homogenous} if $F(x, 0,\ldots, 0) = 0$.
\end{enumerate}
\end{definition}

\begin{lemma}
A PDE is linear \textup{if and only if} it can be written in the form
\[ Lu(x) = \sum_{|\alpha|\leq m}a_\alpha(x)D^\alpha u(x) = g(x). \]
A linear PDE is homogeneous \textup{if and only if} $g = 0$.
\end{lemma}

\subsection{Elliptic, Hyperbolic and Parabolic PDEs}

\part{Probability Theory}
\setcounter{chapter}{0} % Reset chapter counter
\url{file:///C:/Users/user/Downloads/Gut2005_Book_ProbabilityAGraduateCourse.pdf}
\url{https://services.math.duke.edu/~rtd/PTE/PTE5_011119.pdf}
TODO: Kolmogorov 0-1 law Gut p21

\chapter{Probability spaces}
\section{Kolmogorov axioms}
\begin{definition}
A measure space $\seq{\Omega, \mathcal{A}, P}$ is called a \udef{probability space} if the measure $P$ is normalised: $P(\Omega) = 1$.
\end{definition}

\begin{lemma}
Let $\seq{\Omega, \mathcal{A}, P}$ be a probability space and $A,B\subseteq \Omega$ measurable sets. Then
\begin{enumerate}
\item $P(A^c)= 1- P(A)$;
\item $P(A) = P(A\setminus B)+P(A\cap B)$
\item $P(A\cup B)+P(A\cap B) = P(A) + P(B)$;
\end{enumerate}
\end{lemma}
\begin{proof}
(1) $\Omega = A\uplus A^c$ is a disjoint union.

(2) $B = (B\setminus A)\uplus (A\cap B)$ is a disjoint union.

(3) Using (2) we get
\[ A\cup B = \Big(A\setminus (A\cap B)\Big) \uplus \Big(B\setminus (A\cap B)\Big) \uplus (A\cap B) = P(A) - P(A\cap B) + P(B)- P(A\cap B)+ P(A\cap B). \]
\end{proof}
\begin{corollary} \mbox{}
\begin{enumerate}
\item $P(A\cup B) \leq P(A) + P(B)$;
\item $A\subset B$ implies $P(B)= P(A) + P(B\setminus A)$;
\item $A\subset B$ implies $P(A) \leq P(B)$.
\end{enumerate}
\end{corollary}

\begin{theorem}[The inclusion-exclusion formula]
Let $\seq{\Omega, \mathcal{A}, P}$ be a probability space and $\seq{A_k}$ a sequence of events. Then
\begin{multline*}
P\left(\bigcup^n_{k=1} A_k\right) = \sum_{k=1}^nP(A_k)\; - \sum_{1\leq i< j \neq n}P(A_i\cap A_j)\; + \sum_{1\leq i<j<k\leq n}P(A_i\cap A_j\cap A_k)\; - \;\ldots \\
+ \;(-1)^{n+1}P(A_1\cap A_2 \cap \ldots \cap A_n).
\end{multline*}
This can also be written as
\[ P\left(\bigcup^n_{k=1} A_k\right) = \sum_{S\subset 1:n}(-1)^{\#(S)+1} P \left(\bigcap_{i\in S}A_i\right). \]
\end{theorem}
\begin{proof}
Set $A = \bigcup^n_{k=1}A_k$ and consider the function
\[ f = \prod_{k=1}^n(\chi_A-\chi_{A_k}) \]
in $(\Omega\to \{0,1\})$. This function is identically zero. Expanding $f=0$ yields the equation
\[ \chi_A = \sum_{k=1}^n\chi_{A_k} - \sum_{1\leq i< j \neq n}\chi_{A_i}\cdot\chi_{A_j} + \sum_{1\leq i<j<k\leq n}\chi_{A_i}\cdot\chi_{A_j}\cdot\chi_{A_k} - \ldots + (-1)^{n+1}\chi_{A_1}\cdot\chi_{A_2}\cdot \ldots \cdot\chi_{A_n}. \]
Integrating both sides of the equation over the measure $P$ gives the result.
\end{proof}
\begin{corollary}[Bonferroni inequalities]
\begin{align*}
P\left(\bigcup^n_{k=1} A_k\right) &\leq \sum_{k=1}^nP(A_k) \\
P\left(\bigcup^n_{k=1} A_k\right) &\geq \sum_{k=1}^nP(A_k) - \sum_{1\leq i< j \neq n}P(A_i\cap A_j) \\
P\left(\bigcup^n_{k=1} A_k\right) &\leq \sum_{k=1}^nP(A_k) - \sum_{1\leq i< j \neq n}P(A_i\cap A_j) + \sum_{1\leq i<j<k\leq n}P(A_i\cap A_j\cap A_k)
\end{align*}
\end{corollary}
\begin{proof}
TODO \url{https://planetmath.org/proofofbonferroniinequalities}
\end{proof}
\begin{corollary}
If the events of the sequence $\seq{A_k}$ are independent, then
\[ P\left(\bigcup^n_{k=1} A_k\right) = 1 - \prod^n_{k=1}(1-P(A_k)). \]
Also
\[ P\left(\bigcup^n_{k=1} A_k\right) \geq 1 - \exp\left(-\sum^n_{k=1}P(A_k)\right). \]
\end{corollary}

\begin{proposition}
Let $\seq{\Omega, \mathcal{A}, \mu}$ be a probability space and $\mathcal{F}$ an algebra that generates the $\sigma$-algebra $\mathcal{A} = \sigma\{\mathcal{F}\}$. For any $A\in\mathcal{A}$ and $\varepsilon > 0$ there exists a set $A_\varepsilon \in \mathcal{F}$ such that
\[ \mu(A\symdiff A_\varepsilon) \leq \varepsilon. \] 
\end{proposition}
\begin{proof}
Let $\varepsilon > 0$ and define
\[ \mathcal{E} = \setbuilder{A\in\mathcal{A}}{\mu(A\symdiff A_\varepsilon) \leq \varepsilon\;\text{for some}\; A_\varepsilon\in\mathcal{F}}. \]
Clearly $\mathcal{F} \subseteq \mathcal{E} \subseteq \mathcal{A}$. So if $\mathcal{E}$ is a $\sigma$-algebra, then it is equal to $\mathcal{A}$ and the proposition is proven.
\begin{itemize}
\item $\Omega \in \mathcal{A}$ because $\Omega \in \mathcal{F}$.
\item Let $A\in \mathcal{E}$. Then $A^c\symdiff (A_\varepsilon)^c = A\symdiff A_\varepsilon < \varepsilon$, so $A^c\in \mathcal{E}$.
\item Let $\seq{A_i}$ be a sequence of sets in $\mathcal{E}$ and set $A = \bigcup_{i= 0}^\infty A_i$. Then
\[ \lim_{n\to\infty}P\left(\bigcup_{i=0}^n A_i\right) = P\left(\lim_{n\to\infty}\bigcup_{i=0}^n A_i\right) = P(A) \]
and there exists an $n_0\in\N$ such that
\[ \varepsilon/2 > P(A) - P\left(\bigcup_{i=0}^{n_0} A_i\right) = P\left(A\setminus \bigcup_{i=0}^{n_0} A_i\right). \]
TODO
\end{itemize}
\end{proof}

\section{Independence}
\begin{definition}
Let $\seq{\Omega, \mathcal{A}, P}$ be a probability space. The events in a set $\{A_i\}$ are called \udef{independent} if for all finite $F\subset \{A_i\}$ we have
\[ P\left(\bigcap_{A_i\in F}A_i\right) = \prod_{A_i\in F}P(A_i). \]
\end{definition}

\begin{lemma}
Let $\seq{\Omega, \mathcal{A}, P}$ be a probability space and $A,B$ independent events. Then $\{A,B^c\}, \{A^c,B\}$ and $\{A^c, B^c\}$ are also independent.
\end{lemma}
\begin{lemma}
Null sets are independent of any event, in particular of themselves.
\end{lemma}
\begin{proof}
Let $A$ be a null set and $B$ any event. Then
\[ 0\leq P(A\cap B) \leq P(A\setminus B) + P(A\cap B) = P(A) = 0, \]
so 
\[ P(A\cap B) = 0 =  P(A)P(B).  \]
\end{proof}

\subsection{Independent collections of events}
\begin{definition}
Let $\seq{\Omega, \mathcal{A}, P}$ be a probability space. Let $\{\mathcal{A}_i\}$ be a countable family of sets of events. The sets of events in this family are called \udef{independent} if (the image of) every section of $(\mathcal{A}_i\mapsto i)$ is independent.
\end{definition}

\begin{proposition}
Let $\{\mathcal{A}_i\}_{i\in I}$ be a countable family of independent sets of events. Then
\begin{enumerate}
\item $\{\mathfrak{D}\{\mathcal{A}_i\}\}$ are independent sets of events;
\item if the $\mathcal{A}_i$ are $\pi$-systems, then $\{\sigma\{\mathcal{A}_i\}\}$ are independent sets of events.
\end{enumerate}
\end{proposition}
\begin{proof}
The second part follows from the first by \ref{generatedDynkinSigma}.

We prove the first part by induction on the cardinality of $I$. For $\#(I) = 1$ any section contains only one set, which is necessarily independent.

For the induction step, let $s: I \to \bigcup \{\mathfrak{D}\{\mathcal{A}_i\}\}$ be a section. Take $i_0\in I$. We need to show that $s[I\setminus \{i_0\}]\cup \{A\}$ is independent for all $A\in \mathfrak{D}\{\mathcal{A}_{i_0}\}$. By the induction hypothesis we may assume that $s[I\setminus \{i_0\}]\cup \{A\}$ is independent for all $A\in \mathcal{A}_{i_0}$.

Let $B\in s[I]$ and define 
\[ \mathcal{E}_B = \setbuilder{A\in \mathfrak{D}\{\mathcal{A}_{i_0}\}}{P(A\cap B) = P(A)P(B)}. \]
TODO
\end{proof}

\subsection{Pair-wise independence}
\begin{definition}
Let $\seq{\Omega, \mathcal{A}, P}$ be a probability space. The events in a set $\{A_k\}_{k\in I}$ are called \udef{pair-wise independent} if for all $i\neq j \in I$ we have
\[ P\left(A_i \cap A_j\right) = P(A_i)\cdot P(A_j). \]
\end{definition}
Clearly independence implies pair-wise independence. The converse is not true.

\begin{example}
Let $\Omega = \{(1, 0, 0), (0, 1, 0), (0, 0, 1), (1, 1, 1)\}$, $\mathcal{A} = \powerset(\Omega)$ and $P = A\mapsto 1/4 \cdot \#(A)$.

Set $A_k = \{\text{the $k^\text{th}$ coordinate equals $1$}\}$ for $k=1,2,3$. Then
\begin{align*}
P(A_k) &= \frac{1}{2} & \forall k \in\{1,2,3\} \\
P(A_i\cap A_j) &= \frac{1}{4} & \forall i\neq j \in\{1,2,3\} \\
P(A_i)P(A_j) &= \frac{1}{4} & \forall i\neq j \in\{1,2,3\} \\
P(A_1\cap A_2 \cap A_3) &= \frac{1}{4} \\
P(A_1)P(A_2)P(A_3) &= \frac{1}{8}.
\end{align*}
The sets $A_1, A_2, A_3$ are pair-wise independent, but not independent.
\end{example}

\section{Conditional probability}
\begin{definition}
Let $\seq{\Omega, \mathcal{A}, P}$ be a probability space, $A$ and $B$ be two events, and suppose that $P(A) > 0$. The \udef{conditional probability} of $B$ given $A$ is defined as
\[ P(B | A) \defeq \frac{P(A\cap B)}{P(A)}. \]
\end{definition}
\begin{lemma}
Let $\seq{\Omega, \mathcal{A}, P}$ be a probability space and $A$ an event with non-zero probability. Then
\[ P(\cdot | A): B\mapsto P(B | A) \]
is a probability measure on the measurable space $\seq{\Omega, \mathcal{A}}$.
\end{lemma}
\begin{lemma}
If $A,B$ are independent events, then $P(B|A) = P(B)$.
\end{lemma}
\begin{proof}
$P(B|A) = \frac{P(A\cap B)}{P(A)} = \frac{P(A)\cdot P(B)}{P(A)} = P(B)$.
\end{proof}

\section{Chain rule and law of total probability}
\begin{theorem}[Law of total probability]
Let $\seq{\Omega, \mathcal{A}, P}$ be a probability space and $\seq{H_i}_{i\in I}$ a (countable) partition of $\Omega$. Then, for any event $A\in \mathcal{A}$
\[ P(A) = \sum_{i\in I}P(A|H_i)\cdot P(H_i). \]
\end{theorem}
\begin{proof}
$A = A\cap \Omega = \biguplus_{i\in I}(A\cap H_i)$ is a disjoint union.
\end{proof}

\section{Bayes' formula}
\begin{theorem}[Bayes' formula]
Let $\seq{\Omega, \mathcal{A}, P}$ be a probability space and $\seq{H_i}_{i\in I}$ a (countable) partition of $\Omega$. Then, for any event $A$ of non-zero probability,
\[ P(H_k|A ) = \frac{P(A|H_k)\cdot P(H_k)}{P(A)} = \frac{P(A|H_k)\cdot P(H_k)}{\sum_{i\in I}P(A|H_i)\cdot P(H_i)}. \]
\end{theorem}
\begin{proof}
$P(H_k|A) = \frac{P(H_k \cap A)}{P(A)} = \frac{P(A|H_k)\cdot P(H_k)}{P(A)}$.
\end{proof}

\section{Sequences of events}
\subsection{Infinitely often events}
\begin{definition}
Let $\sSet{\Omega, \mathcal{A}, P}$ be a probability space and $\seq{A_n}$ be a sequence of events. We say $\omega \in A_n$ \udef{infinitely often} (or i.o.) if $\omega\in \limsup_{n\to\infty}A_n$.

The event $\limsup_{n\to\infty}A_n$ is also denoted $\{A_n \text{i.o.}\}$.
\end{definition}
\subsubsection{Borel-Cantelli lemma}
\begin{proposition}[Borel-Cantelli lemma]
Let $\seq{\Omega, \mathcal{A}, P}$ be a probability space.
\begin{enumerate}
\item If $\seq{A_n}$ is an arbitrary sequence of events, then
\[ \sum_{n=1}^\infty P(A_n) < \infty \implies P(A_n \text{i.o.}) = 0; \]
\item If $\seq{A_n}$ is a sequence of pair-wise independent events, then
\[ \sum_{n=1}^\infty P(A_n) = \infty \implies P(A_n \text{i.o.}) = 1. \]
\end{enumerate}
\end{proposition}
\begin{proof}
(1) We have, for all $k$,
\[ P(A_n \text{i.o.}) = P\left(\bigcap_{n=1}^\infty\bigcup_{m=n}^\infty A_m\right) \leq P\left(\bigcup_{m=k}^\infty A_m\right) \leq \sum_{m=k}^\infty P(A_m), \]
and $\sum_{m=k}^\infty P(A_m) \to 0$ as $k\to \infty$.

(2) TODO
\end{proof}
\begin{corollary}
Let $\sSet{\Omega, \mathcal{A}, P}$ be a probability space and $\seq{A_n}$ be a sequence of pair-wise independent events. Then
\[  P(A_n \text{i.o.}) = \begin{cases}
0 & \sum_{n=1}^\infty P(A_n) < \infty \\
1 & \sum_{n=1}^\infty P(A_n) = \infty.
\end{cases} \]
\end{corollary}

\chapter{Random variables, random vectors and random elements}
\begin{definition}
Let $\sSet{\Omega, \mathcal{A}, P}$ be a probability space and $\sSet{S, d}$ a metric space which we consider as a measurable space $\sSet{X,\mathcal{B}}$, where $\mathcal{B}$ is the Borel $\sigma$-algebra.
\begin{itemize}
\item A measurable function $X:\Omega \to S$ is called a \udef{random element}.
\item If $S$ is a normed vector space, then $X$ is called a \udef{random vector}.
\item If $S = \R$, then $X$ is called a \udef{random variable} (or \udef{r.v.}).
\item If $S = \overline{\R} = [-\infty, +\infty]$, then $X$ is called an \udef{extended random variable}.
\end{itemize}
\end{definition}


\begin{lemma}
Let $X: \sSet{\Omega, \mathcal{A}, P} \to \sSet{S, d}$ be a random element. Then the pushforward measure
\[ \mathbb{P}_X: \mathcal{B}\to [0, +\infty]: B\mapsto P(X^{-1}(B)) = P(\setbuilder{\omega\in \Omega}{X(\omega)\in B}) \]
is a probability measure on $\sSet{V, \mathcal{B}}$.
\end{lemma}
\begin{proof}
The pushforward measure is a measure by \ref{pushforwardMeasure}.
It is normalised because $X^{-1}[V] = \Omega$, so $\mathbb{P}(V) = P(\Omega) = 1$.
\end{proof}

\begin{definition}
The pushforward probability measure $\mathbb{P}_X$ is called the \udef{induced probability measure} or the probability measure \udef{induced} by $X$. The probability space $\sSet{S, \mathcal{B}, \mathbb{P}_X}$ is the \udef{induced probability space}.
\end{definition}

We will often write $P(X\in B)$ for $\mathbb{P}_X(B) = P(X^{-1}(B)) = P(\setbuilder{\omega\in \Omega}{X(\omega)\in B})$

\section{Equivalence relations on random vectors}
\subsection{Almost sure equivalence}
\begin{definition}
Let $X, Y: \sSet{\Omega, \mathcal{A}, P} \to \sSet{S, \mathcal{B}}$ be random elements. We say $X$ and $Y$ are \udef{almost surely (a.s.) equal}, denoted $X\sim Y$, if they differ on at most a null set:
\[ X\sim Y \iff P(\setbuilder{\omega\in \Omega}{X(\omega) \neq Y(\omega)}) = 0 \iff P(\setbuilder{\omega\in \Omega}{X(\omega) = Y(\omega)}) = 1. \]
We also say $X$ and $Y$ are \udef{equivalent} random vectors.
\end{definition}

\subsection{Equivalence in distribution}
\begin{definition}
Let $X, Y: \sSet{\Omega, \mathcal{A}, P} \to \sSet{S, \mathcal{B}}$ be random elements. We say $X$ and $Y$ are \udef{equal in distribution}, denoted $X \overset{d}{=} Y$, if they assign the same probability the each event in $\mathcal{B}$
\[ X \overset{d}{=} Y \iff \forall B\in\mathcal{B}: \; P(X\in B) = P(Y\in B). \]
\end{definition}

\begin{lemma}
Let $X, Y: \sSet{\Omega, \mathcal{A}, P} \to \sSet{S, \mathcal{B}}$ be random elements. If $X$ and $Y$ are a.s. equal, then they are equal in distribution.
\end{lemma}
\begin{proof}
Set $C = \setbuilder{\omega\in \mathcal{A}}{X(\omega)\in B}$, $D = \setbuilder{\omega\in \mathcal{A}}{Y(\omega)\in B}$ and $E = \setbuilder{\omega\in \Omega}{X(\omega)\neq Y(\omega)}$. Clearly $C\cap E$ and $D\cap E$ are measurable null sets, so $P(C\cap E) = 0 = P(D\cap E)$. Also $C\setminus E = D\setminus E$.

We calculate
\[ P(X\in B) = P(C) = P(C\setminus E) +P(C\cap E) = P(D\setminus E) +P(D\cap E) = P(D) = P(Y\in B). \]
\end{proof}
The converse of this lemma is not true.
\begin{example}
Toss a fair coin. The universe set is $\Omega = \{\text{heads}, \text{tails}\}$. Consider the random vectors
\[ X: \omega \mapsto \begin{cases}
1 & \omega = \text{heads} \\ 0 & \omega = \text{tails}
\end{cases} \qquad\text{and}\qquad  Y: \omega \mapsto \begin{cases}
0 & \omega = \text{heads} \\ 1 & \omega = \text{tails}.
\end{cases} \]
Clearly $P(X = 1) = P(X = 0) = P(Y = 1) = P(Y = 0) = 1/2$. Thus we see that $X \overset{d}{=} Y$. But clearly $X$ and $Y$ are not almost surely equal. In fact they are surely unequal.
\end{example}

\section{Distribution functions}
\begin{definition}
Let $X$ be a random variable on a probability space $\sSet{\Omega, \mathcal{A}, P}$. The \udef{distribution function} of $X$ is the function
\[ F_X: \R \to \R: x\mapsto P(X \leq x) = P(\setbuilder{\omega\in \Omega}{X(\omega) \leq x}). \]
\end{definition}

TODO: this is the Riemann-Stieltjes function. Generalise.

\begin{proposition}
Let $X$ be a random variable on a probability space $\sSet{\Omega, \mathcal{A}, P}$ and $F$ the distribution function of $X$. Then
\begin{enumerate}
\item $F$ is monotonically increasing;
\item $\lim_{x\to-\infty} F(x) = 0$ and $\lim_{x\to+\infty} F(x) = 1$; 
\item $F$ is right-continuous at every point.
\end{enumerate}
Conversely, any function $F:\R\to\R$ that satisfies these properties is the distribution function of some random variable.
\end{proposition}
\begin{proof}
(1) Let $x\leq y$. Then $\setbuilder{\omega\in \Omega}{X(\omega) \leq x} \subseteq \setbuilder{\omega\in \Omega}{X(\omega) \leq y}$, so
\[ F(x) =  P(\setbuilder{\omega\in \Omega}{X(\omega) \leq x}) \leq  P(\setbuilder{\omega\in \Omega}{X(\omega) \leq y}) = F(y). \]

(2) Follows from \ref{measures}.

(3) TODO

(Converse) TODO
\end{proof}
\begin{corollary}
Every discontinuity is a jump discontinuity, so all left limits exist. Also there are at most countably many discontinuities.
\end{corollary}
\begin{proof}
By \ref{monotoneDiscontinuities} and \ref{DarbouxFroda}.
\end{proof}

\begin{proposition}
Let $X_1, X_2$ be random variables. Then
\[ F_{X_1+X_2}(u) = \int_{-\infty}^\infty F_{X_1}(u-y)\diff{F_{X_2}(y)}. \]
If both distributions are absolutely continuous, then
\[ f_{X_1+X_2}(u) = \int_{-\infty}^\infty f_{X_1}(u-y)f_{X_2}(y)\diff{y}. \]
\end{proposition}
\begin{proof}
Fubini TODO
\end{proof}

\begin{lemma}
Let $X$ be a random variable. Then $P\{X = x\} = \Delta F_X(x)$.
\end{lemma}

\subsection{Probability density functions}

\subsection{Transformed random variables}

\begin{proposition} \label{transformationRandomVariable}
Let $X:\Omega \to \R$ be a random variable and $g: \R\to\R$ a Borel measurable function.
\begin{enumerate}
\item If $(g, h)$ forms a Galois connection, then $F_{g(X)}(y) = F_X(h(y))$.
\item If $(g, h)$ forms an antitone Galois connection, then $F_{g(X)}(y) = 1-F_X(h(y)) + \Delta F_X(h(y))$.
\item If $g$ is either strictly increasing or strictly decreasing, then
\[ f_{g(X)}(y) = \begin{cases}
f_X(g^{-1}(y))\left|\od{g^{-1}(y)}{y}\right| & y\in \im(g) \\
0 & y\notin \im(g).
\end{cases} \]
\end{enumerate}
\end{proposition}
In particular we can find a Galois connection $(g,h)$ if $g$ is right continuous and increasing.
\begin{proof}
(1) We calculate
\[ F_{g(X)}(y) = P\{g(X) \leq y\} = P\{X \leq h(y) \} = F_X(h(y)). \]
(2) Similarly,
\[ F_{g(X)}(y) = P\{g(X) \leq y\} = P\{X \geq h(y) \} = 1 - P\{X < h(y) \} = 1 - P\{X \leq h(y) \} + P\{X = h(y)\} = 1-F_X(h(y)) + \Delta F_X(h(y)). \]
(3) TODO
\end{proof}


\section{Convergence}
\begin{definition}
Let $\seq{X_n}$ be a sequence of random elements in $(\sSet{\Omega, \mathcal{A}, P} \to \sSet{S,d})$ and $X$ a random element in the same set. We say
\begin{itemize}
\item $\seq{X_n}$ \udef{converges almost surely} to $X$ if
\[ P(\setbuilder{\omega\in\Omega}{X_n(\omega) \to X(\omega)\;\text{as}\; n\to \infty}) = 1. \]
We write $X_n \overset{a.s.}{\longrightarrow} X$.
\end{itemize}
\end{definition}

\begin{lemma}
Suppose $Y,X,X_n$ are random vectors such that $X_n \overset{a.s.}{\longrightarrow} X$, $E[Y]<\infty$ and $|X_n| \leq Y$ for all $n$. Then $E[|X_n -X|] \to 0$ as $n\to \infty$.
\end{lemma}
\begin{proof}
This is just the Lebesgue dominated convergence theorem TODO ref.
\end{proof}

\section{Expected value}
\begin{definition}
Let $X$ be a random  vector on a probability space $\sSet{\Omega, \mathcal{A}, P}$. We define the \udef{exprectated value} of $X$ as
\[ \E{X} \defeq \int_\Omega X(\omega)\diff{P(\omega)}. \]
assuming $X$ is integrable.
\end{definition}



\subsection{Moments}
\subsubsection{Raw and central moments}
\subsubsection{Moment generating function}
\subsubsection{Normalised moments}
\subsubsection{Examples of moments}
\paragraph{Expected value}
\paragraph{Variance and standard deviation}
\paragraph{Skewness}
\paragraph{Kurtosis}

\subsection{Mean and variance}

\begin{proposition}
Let $\sSet{\Omega, \mathcal{A}, P}$ be a probability space and $X,Y:\Omega \to \R$ random variables with finite means $\mu_X,\mu_Y$ and standard deviations $\sigma_X$, $\sigma_Y$. Then
\[ |\E[XY] - \mu_X\mu_Y| \leq \sigma_X\sigma_Y. \]
\end{proposition}
\begin{proof}
This follows from the CSB inequality, \ref{CauchySchwarz}:
\begin{align*}
\sigma_X\sigma_Y &= \sqrt{\E[(X-\mu_X)^2]}\sqrt{\E[(Y-\mu_Y)^2]} \\
&\geq |\E[(X-\mu_X)(Y-\mu_Y)]| \\
&= |\E[XY] + \mu_X\mu_Y -\mu_X\mu_Y -\mu_X\mu_Y| = |\E[XY] - \mu_X\mu_Y|.
\end{align*}
\end{proof}

\subsection{Cumulants}

\subsection{Conditional expectation}

\section{Joint distributions}
\subsection{Marginal distributions}

\section{Independence of random elements}
\begin{definition}
A set of random elements $\{X_i\}_{i\in 1:n}$ is called \udef{independent} if for all sets $\{A_i\}_{i\in 1:n}$ of $n$ Borel measurable sets we have that $\{X_i^{-1}[A_i]\}_{i\in 1:n}$ is a set of independent events. i.e.
\[ P\left(\bigcap_{i=1}^n X_i^{-1}[A_i]\right) = \prod_{i=1}^n P(X_i^{-1}[A_i]). \]
\end{definition}

\begin{proposition}
The random variables $X_1, \ldots X_n$ are independent \textup{if and only if}
\[ F_{X_1,\ldots, X_n}(x_1, \ldots, x_n) = \prod_{i=1}^n F_{X_i}(x_i). \]
\end{proposition}

\begin{proposition}
Let $X_1, \ldots X_n$ be independent random elements and $f_1, \ldots, f_n$ measurable functions. Then $f_1\circ X_1, \ldots, f_n\circ X_n$ are independent.
\end{proposition}
\begin{proof}
Let $A_1, \ldots, A_n$ be Borel measurable sets. Since the $f_i^{-1}[A_i]$ are Borel measurable, we have
\[ P\left(\bigcap_{i=1}^n (f_i\circ X_i)^{-1}[A_i]\right) = P\left(\bigcap_{i=1}^n X_i^{-1}[f^{-1}_i[A_i]]\right) = \prod_{i=1}^n P(X_i^{-1}[f^{-1}_i[A_i]]) = \prod_{i=1}^n P((f_i\circ X_i)^{-1}[A_i]). \]
\end{proof}

\chapter{Distributions}
TODO: uniform distribution

\section{Distributions of discrete random variables}
\subsection{Bernoulli trials}
\subsubsection{Bernoulli distribution}
\subsubsection{Geometric distribution}

\subsection{Iterated Bernoulli trials}
\subsubsection{Binomial distribution}
\subsubsection{Negative binomial or Pascal distribution}
\subsubsection{Hypergeometric distribution}
\subsubsection{Negative hypergeometric distribution}

\subsection{Poisson distribution}
\begin{definition}
Let $\lambda > 0$ be a positive real number. The \udef{Poisson distribution} with parameter $\lambda$ is the distribution of a random variable $X: \sSet{\Omega, \mathcal{A}, P} \to \N$ with probability density
\[ f_\lambda(k) = \begin{cases}
\frac{\lambda^k e^{-\lambda}}{k!} & k\in\N \\
0 & \text{otherwise}
\end{cases}. \]
We write $X \sim \Poisson(\lambda)$.
\end{definition}

\section{Distributions of continuous random variables}
\subsection{Gamma distribution and subfamilies}
\subsubsection{Gamma distribution}
\begin{definition}
Let $\alpha, \beta$ be positive real numbers. The \udef{gamma distribution} with parameters $\alpha, \beta$ is the distribution of a random variable $X: \sSet{\Omega, \mathcal{A}, P} \to \R_+$ with probability density
\[ f_{\alpha,\beta}(x) = \frac{x^{\alpha-1}e^{-\beta x}\beta^\alpha}{\Gamma(\alpha)}, \]
where $\Gamma$ is the Gamma function.
\begin{itemize}
\item We call $\alpha$ the \udef{shape parameter}.
\item We call $\beta$ the \udef{rate parameter}.
\item We call $1/\beta$ the \udef{scale parameter}.
\end{itemize}
We write $X \sim \GammaDist(\alpha,\beta)$.
\end{definition}
TODO lower incomplete gamma function.

\begin{lemma}
Let $\alpha,\beta, k$ be positive real numbers. Let $X \sim \GammaDist(\alpha, \beta)$ be a random variable. Then $kX \sim \GammaDist(\alpha, \beta / k)$.
\end{lemma}

\subsubsection{Erlang distribution}
\begin{definition}
An \udef{Erlang distribution} is a gamma distibution where the shape parameter $\alpha$ is an integer $n$. We wite $\Erlang(n, \beta) \defeq \GammaDist(n, \beta)$.
\end{definition}

\begin{proposition} \label{ErlangCDF}
Let $X \sim \Erlang(n, \beta)$. Then the cumulative distribution function of $X$ is given by
\[ F_{n,\beta}(x) = 1 = \sum_{k=0}^{n-1} \frac{(\beta x)^k}{k!}e^{-\beta x}. \]
\end{proposition}

\subsubsection{Exponential distribution}
\begin{definition}
An \udef{exponential distribution} is an Erlang distribution with shape parameter $n = 1$. Thus $\Exponential(\beta) \defeq \Erlang(1,\beta) = \GammaDist(1, \beta)$.
\end{definition}

\subsubsection{$\chi^2$-distribution}
\begin{definition}
Let $k\in \N^\times$ be strictly positive integer. Then we call a gamma distribution with shape $k/2$ and rate $1$ a \udef{$\chi^2$-distribution} with $k$ \udef{degrees of freedom}. We write $\chi^2_k = \GammaDist(k/2, 1)$.
\end{definition}

\section{Distributions of random vectors}

\chapter{Convergence}
\section{Types of convergence}

\chapter{Stochastic processes}
\url{https://math.stackexchange.com/questions/1309853/proving-galmarinos-test/1596012}

\url{https://link.springer.com/content/pdf/10.1007%2F978-3-319-78768-8.pdf}
\url{https://people.math.harvard.edu/~knill/books/KnillProbability.pdf}

\url{file:///C:/Users/user/Downloads/(Advances%20in%20applied%20mathematics)%20Kirkwood,%20James%20R%20-%20Markov%20Processes-CRC%20Press%20(2015).pdf}
\url{file:///C:/Users/user/Downloads/(De%20Gruyter%20Studies%20in%20Mathematics)%20Kolokoltsov%20V.N.%20-%20Markov%20processes,%20semigroups%20and%20generators-De%20Gruyter%20(2011).pdf}

\section{Processes}
\begin{definition}
Let $\sSet{\Omega, \mathcal{A}, P}$ be a probability space, $\sSet{S,d}$ a metric space and $\sSet{I,\leq}$ a partially ordered index set. A \udef{stochastic process} $X$ is a function
\[ I\times \Omega \to S: (t,\omega) \mapsto X_t(\omega) \]
such that the partial application $X_t$ is a random element for all $t\in I$.
\begin{itemize}
\item A \udef{sample path}, \udef{trajectory} or \udef{realisation} is a partial application
\[ X_-(\omega): t\mapsto X_t(\omega). \]
\item If $I \subseteq \N$, we call the stochastic process a \udef{stochastic sequence}.
\item If $I$ is a subinterval of $\R$, we call the stochastic process a \udef{continuous process}.
\item If $I\subseteq \R^+$, we call the index \udef{time} and the process a \udef{continuous time process}.
\item If $I\subseteq \R^k$, we call the stochastic process a \udef{$k$-parameter} or \udef{multiparameter process}.
\end{itemize}
\end{definition}

\begin{lemma}
Let $\sSet{\Omega, \mathcal{A}, P}$ be a probability space, $\sSet{S,d}$ a metric space, $\sSet{I,\leq}$ a partially ordered index set and $X$ a function $X: I\times \Omega\to S: (t,\omega) \mapsto X_t(\omega)$. The following are equivalent:
\begin{enumerate}
\item $X$ is a stochastic process;
\item $X$ is measurable w.r.t. the $\sigma$-algebra generated by $\bigcup_{\substack{t\in I\\ A\in \mathcal{A}}}\{\{t\}\}\times A$;
\item 
\end{enumerate}

Then $X$ is a stochastic process \textup{if and only if} $X$ is measurable w.r.t. the $\sigma$-algebra $\powerset(I)\otimes \mathcal{A}$.
\end{lemma}
\begin{proof}

\end{proof}

\begin{lemma}
Let $\sSet{\Omega, \mathcal{A}, P}$ be a probability space, $\sSet{S,d}$ a metric space, $\sSet{I,\leq}$ a partially ordered index set and $X$ a function $X: I\times \Omega\to S: (t,\omega) \mapsto X_t(\omega)$.

Then $X$ is a stochastic process \textup{if and only if} $\operatorname{curry}(X): \Omega \to (I\to S)$ is measurable, where $(I\to S)$ is equipped with the $\sigma$-algebra $\bigotimes_{t\in I}\mathcal{B}$ and $\mathcal{B}$ is the Borel-$\sigma$-algebra on $S$.
\end{lemma}
\begin{proof}
TODO!! (is it true??)
\end{proof}

\subsection{Equivalent stochastic processes}
\begin{definition}
Let $\sSet{\Omega, \mathcal{A}, P}$ be a probability space, $\sSet{S,d}$ a metric space and $\sSet{I,\leq}$ a partially ordered index set. Let $X,Y$ be stochastic processes. We call $X$ and $Y$
\begin{itemize}
\item \udef{equivalent} or \udef{indistinguishable} if
\[ P(X_- = Y_-) = 1. \]
\item \udef{stochastically equivalent} or \udef{modifications} of each other if
\[ \forall t\in I: P(X_t = Y_t) = 1.  \]
\end{itemize}
\end{definition}

\begin{lemma}
Equivalence and stochastic equivalence are equivalence relations.
\end{lemma}

\begin{lemma}
Let $\sSet{\Omega, \mathcal{A}, P}$ be a probability space, $\sSet{S,d}$ a metric space, $\sSet{I,\leq}$ a partially ordered index set and $X,Y$ stochastic processes. Then
\begin{enumerate}
\item $X$ and $Y$ are equivalent if and only if
\[ \exists \;\text{null set}\; A\subseteq \Omega: \forall t\in I: \forall \omega\in A^c: \; X_t(\omega) = Y_t(\omega); \]
\item $X$ and $Y$ are stochastically equivalent if and only if
\[ \forall t\in I: \exists \;\text{null set}\; A\subseteq \Omega: \forall \omega\in A^c: \; X_t(\omega) = Y_t(\omega).  \]
\end{enumerate}
In particular equivalence implies stochastic equivalence.
\end{lemma}
The opposite implication does not hold:
\begin{example}
Let $T$ be a random variable that is uniformly distributed over the interval $[0,1]$. Define
\[ X_t = \chi_{\{t=T\}} \qquad\text{and}\qquad Y_t = \omega\mapsto 0. \]
So the sample paths of $Y$ are identically zero and the sample paths of $X$ are zero everywhere except at one point, where it is one.

Now $X$ and $Y$ are stochastically equivalent: for all $t$ we have
\[  P(X_t = Y_t) = P(X_t = 0) = P(T \neq t) = 1. \]

But $X$ and $Y$ are not equivalent: for all $\omega$, the sample paths differ by exactly one point, so the probability of them coinciding is zero.
\end{example}

\begin{proposition}
If the index set $I$ is countable, then stochastic equivalence implies equivalence.
\end{proposition}
\begin{proof}
Let $X,Y$ be stochastically equivalent processes on a countable index set $I$. We calculate
\begin{align*}
P(X_- = Y_-) &= P\left(\bigcap_{i\in I}\{X_i = Y_i\}\right) = 1-P\left(\bigcup_{i\in I}\{X_i \neq Y_i\}\right) \\
&\geq 1 - \sum_{i\in I}P(\{X_i \neq Y_i\}) = 1.
\end{align*}
\end{proof}
So for countable index sets stochastic equivalence and equivalence are equivalent.

\begin{proposition}
Let $X$ be a stochastic process such that all sample paths of $X$ lie in a separable ...
\end{proposition}
\begin{proof}
TODO
\end{proof}
\begin{corollary}
Let $X$ and $Y$ be stochastic processes whose sample paths are almost surely right-continuous (resp. left-continuous). Then $X$ and $Y$ are equivalent \textup{if and only if} they are stochastically equivalent.
\end{corollary}

\subsection{Filtration}
\begin{definition}
Let  $\sSet{\Omega, \mathcal{A}, P}$ be a probability space and $\sSet{I,\leq}$ an ordered index set. A \udef{filtration} is an order-preserving function from $I$ to the lattice of sub-$\sigma$-algebras of $\mathcal{A}$.

If $i$ is mapped to $\mathcal{A}_i$, we call $\sSet{\Omega, \mathcal{A}, \{\mathcal{A}_i\}_{i\in I}, P}$ a \udef{filtered probability space}.

Let $t\in I$. Then we define
\begin{itemize}
\item the \udef{left limit} at $t$ as $\mathcal{A}_{t-} \defeq \bigvee_{s<t}\mathcal{A}_s$;
\item the \udef{right limit} at $t$ as $\mathcal{A}_{t+} \defeq \bigwedge_{t<s}\mathcal{A}_s$;
\item the \udef{limit at infinity} as $\mathcal{A}_{\infty} \defeq \bigvee_{s\in I}\mathcal{A}_s$.
\end{itemize}
\end{definition}
The suprema and infima are taken in the lattice of $\sigma$-algebras on $\Omega$. By \ref{completeLatticeOperationsUnderClosure} we have
\[ \bigwedge\mathcal{E} = \bigcap\mathcal{E} \qquad\text{and}\qquad \bigvee\mathcal{E} = \sigma\left\{\bigcup \mathcal{E}\right\},  \]
where $\mathcal{E}$ is any set of $\sigma$-algebras on $\Omega$.

\begin{lemma}
Let $\sSet{\Omega, \mathcal{A}, \{\mathcal{A}_i\}_{i\in I}, P}$ be a filtered probability space. Then for all $t\in I$
\begin{enumerate}
\item $\mathcal{A}_{t-} \subseteq \mathcal{A}_{t} \subseteq \mathcal{A}_{t+}$;
\item $\mathcal{A}_{t} \subseteq \mathcal{A}_{\infty}$.
\end{enumerate}
\end{lemma}
\begin{proof}
Using the fact that $t\mapsto \mathcal{A}_t$ is order-preserving and \ref{orderPreservingFunctionLatticeOperations}, we have
\[ \mathcal{A}_{t-} = \bigvee\mathcal{A}_{\downset t \setminus\{t\}} \subseteq \mathcal{A}_{\bigvee \downset t \setminus\{t\}} \subseteq \mathcal{A}_{\bigvee \downset t} = \mathcal{A}_t = \mathcal{A}_{\bigwedge \upset t} \subseteq \mathcal{A}_{\bigwedge \upset t \setminus\{t\}} \subseteq \bigwedge \mathcal{A}_{\upset t \setminus\{t\}} = \mathcal{A}_{t+}. \]
\end{proof}

\subsubsection{Adapted processes and natural filtrations}
\begin{definition}
Let $\sSet{\Omega, \mathcal{A}, P}$ be a probability space, $\sSet{I,\leq}$ an ordered index set and $\sSet{S,d}$ a metric space.

\begin{itemize}
\item Given a filtration $\{\mathcal{A}_t\}_{t\in I}$, a stochastic process $X: I\times \Omega \to S$ is \udef{adapted} to the filtration if for all $t\in I$ $X_t$ is measurable w.r.t. $\mathcal{A}_t$.
\item Given a stochastic process $X: I\times \Omega \to S$, the \udef{natural filtration} $\NF(X)$ is the filtration \udef{generated} by $X$:
\[ \mathcal{A}_t \defeq \sigma\setbuilder{X_s^{-1}[B]}{s\leq t, B\in \mathcal{B}(S)}. \]
\end{itemize}

Let $\NF_{I,\Omega}^S$ be the set of natural filtrations generated by stochastic processes in $(I\times\Omega \to S)$.
\end{definition}

\begin{lemma}
Let $\sSet{\Omega, \mathcal{A}, P}$ be a probability space, $\sSet{I,\leq}$ an ordered index set and $\sSet{S,d}$ a metric space. Let $X: I\times \Omega\to S$ be a stochastic process and $\{\mathcal{A}_t\}_{t\in I}$ a filtration. Then
\[ X \;\text{is adapted to}\; \{\mathcal{A}_t\}_{t\in I} \iff \NF(X) \leq \{\mathcal{A}_t\}_{t\in I}. \]
\end{lemma}

\begin{lemma}
Let $\sSet{\Omega, \mathcal{A}, P}$ be a probability space, $\sSet{I,\leq}$ an ordered index set, $\sSet{S,d}$ a metric space and $X: I\times\Omega\to S$ a stochastic process.

For all $t\in I$ we define $\mathcal{A}_i$ as the set of all subsets of $\Omega$ of the form
\[ \bigcap_{j\in J} X_{t_j}^{-1}[B_j] \]
for some (countable) sequence $\seq{t_j}$ in $\downset t$ and $\seq{B_j}$ in $\mathcal{B}(S)$.

Then $\{\mathcal{A}_t\}_{t\in I}$ is the natural filtration $\NF(X)$.
\end{lemma}
\begin{proof}
Clearly $\setbuilder{X_s^{-1}[B]}{s\leq t, B\in \mathcal{B}(S)}_{t\in I} \subseteq \{\mathcal{A}_t\}_{t\in I} \subseteq \NF(X)$. So it is enough to prove that $\mathcal{A}_t$ is a $\sigma$-algebra for all $t\in I$. By the monotone class theorem, \ref{monotoneClassTheorem}, it is in fact enough to show that $\mathcal{A}_t$ is a monotone class for all $t\in I$. Closure under countable (monotone) intersections is clear.


\end{proof}

\begin{proposition}
Let $\sSet{\Omega, \mathcal{A}, P}$ be a probability space, $\sSet{I,\leq}$ an ordered index set and $\sSet{S,d}$ a metric space.

The set $\NF_{I,\Omega}^S$ of natural filtrations forms a sublattice of the lattice of filtrations of $\mathcal{A}$ on $I$.
\end{proposition}
\begin{proof}

\end{proof}

\subsubsection{Usual and natural conditions}
\begin{definition}
Usual conditions, natural conditions
\end{definition}

\subsection{Stopping times}
\begin{definition}
Let $\sSet{\Omega, \mathcal{A}, \{\mathcal{A}_i\}_{i\in I}, P}$ be a filtered probability space, $\sSet{I,\leq}$ an ordered index set and $\overline{I}$ its Dedekind-MacNeille completion. A \udef{stopping time} is a function $\tau: \Omega \to I$ such that $\{\tau \leq t\} \in \mathcal{A}_t$ for all $t\in I$.
\end{definition}

We equip the set of stopping times with pointwise order.

\begin{lemma}
Let $\sSet{\Omega, \mathcal{A}, \{\mathcal{A}_i\}_{i\in I}, P}$ be a filtered probability space and $\sSet{I,\leq}$ an ordered index set. Then
\begin{enumerate}
\item the constant function $\underline{t}$ is a stopping time for all $t\in I$;
\item the set of stopping times is a sublattice of $(\Omega\to I)$.
\end{enumerate}
\end{lemma}
\begin{proof}
(1) For all $i\in I$: we have
\[ \{\underline{t} \leq i\} = \begin{cases}
\Omega & (t\leq i) \\
\emptyset & (t > i)
\end{cases} \]
In both cases $\{\underline{t} \leq i\}$ is an element of each $\sigma$-algebra.

(2)
\end{proof}

\begin{theorem}[Galmarino's test]
Let $\sSet{\Omega, \mathcal{A}, \{\mathcal{A}_i\}_{i\in I}, P}$ be a filtered probability space and $\sSet{I,\leq}$ an ordered index set.

A function $\tau: \Omega\to I$ is a stopping time \textup{if and only if} for all $\omega,\omega'\in \Omega$:
\[ \tau(\omega) = t \land \Big( \forall s\in\downset t: X_s(\omega) = X_s(\omega') \Big) \implies \tau(\omega')=t. \]
\end{theorem}

\subsubsection{Stopped processes}
\begin{definition}
Let $X: I\times \Omega\to S$ be a stochastic process and $\tau:\Omega\to I$ a stopping time. Then the \udef{stopped process} $X^\tau$ is defined by
\end{definition}

\subsection{Measurability requirements}
\subsubsection{Joint measurability}
\begin{definition}
Let $\sSet{\Omega, \mathcal{A}, P}$ be a probability space, $\sSet{S,d}$ a metric space, $\sSet{I,\mathcal{T}, \leq}$ a topological poset and $X$ a stochastic process. We call $X: \Omega\times I\to S$ \udef{jointly measurable} if it is measureable w.r.t. $\mathcal{A}\otimes \mathcal{B}(I)$.
\end{definition}

TODO: order algebra???????

\begin{lemma}
All right-continuous and left-continuous processes are jointly measurable. 
\end{lemma}

\begin{proposition}
If $X$ is a jointly measurable stochastic process and $\tau: \Omega\to I$ a random time, then
\[ X_\tau: \Omega\to S: \omega\mapsto X(\omega)_{\omega} \]
is measurable and thus a random element.
\end{proposition}
\begin{proof}

\end{proof}

\subsubsection{Predictable and optional processes}

\subsubsection{Progressively measurable processes}

\section{Martingales}

\section{Properties and classes of processes}
\subsection{Processes generated by transition probabilities}
\url{https://arxiv.org/pdf/1603.00251.pdf}
\subsubsection{Markov processes}
\subsubsection{Feller processes}
``$C_0$-semigroup''??
\subsubsection{Lévy processes}

\subsection{Processes by distribution}
\subsubsection{Wiener processes}
\subsubsection{Bessel processes}

\subsection{Integer-valued processes}
\subsubsection{Birth-death processes}
\begin{definition}
An integer-valued Markov process is called a \udef{birth-death process}.
\end{definition}

\subsubsection{Counting processes and birth processes}
\begin{definition}
A \udef{counting process} is a stochastic process
\[ N: \sSet{I,\leq}\times \sSet{\Omega, \mathcal{A}, P} \to \N \]
for some probability space $\Omega$ and ordered set $I$ such that for all $s,t\in I$:
\[ \forall s,t\in I: \quad s\leq t \implies N(s)\leq N(t). \]

If the counting process $N$ is a Markov process, then it is called a \udef{birth process}.
\end{definition}

\subsubsection{Poisson processes}
\begin{definition}
A stochastic process
\[ N: \R_+\times \sSet{\Omega, \mathcal{A}, P} \to \Z \]
is called a \udef{Poisson process} if there exists a continuous, increasing function $\Lambda: \R\to \R$ such that
\[ N_t - N_s \sim \Poisson(\Lambda(t) - \Lambda(s)) \]
for all $s\leq t$, independently of $\setbuilder{N_u}{u\leq s}$.

The function $\Lambda$ is called the \udef{cumulative rate}. If $\od{\Lambda(s)}{s}$ exists, it is called the \udef{(instantaneous) rate}.

If $\Lambda$ is of the form $t\mapsto t\lambda$ for some $\lambda \in \R_+$, then we say $N$ is a \udef{homogenous Poisson process}.
\end{definition}

\begin{lemma} \label{PoissonProcessRateTransform}
Let $N_t$ be a Poisson process with cumulative rate $\Lambda$.
\begin{enumerate}
\item Let $\theta: \R_+\to \R$ be a continuous function. Then $N_{\theta(t)}$ is a Poisson process with cumulative rate $\Lambda\circ \theta$, if $\Lambda\circ \theta$ is increasing.
\item Let $c\in \R$. Then $\Lambda + \underline{c}$ is also a cumulative rate function for $N_t$.
\item Let $K$ be a random, integer-valued element. Then $N_t + K$ is also a Poisson process with cumulative rate $\Lambda$.
\end{enumerate}
\end{lemma}
From now on we assume, WLOG, that $\Lambda(0) = 0$.

\begin{lemma}
Let $N$ be a Poisson process with cumulative rate $\Lambda$ such that $N_0 = 0$ and $\Lambda(0) = 0$. Then for each $t\in \R_+$, we have $N_t \sim \Poisson(\Lambda(t))$.
\end{lemma}
\begin{proof}
This follows directly from $N_t = N_t - N_0 \sim \Poisson(\Lambda(t) - \Lambda(0)) = \Poisson(\Lambda(t))$.
\end{proof}

\begin{proposition}
For any continuous, increasing function $\Lambda: \R_+\to \R$, there exists a Poisson processes with cumulative rate $\Lambda$.
\end{proposition}
\begin{proof}
Construct homogenous Poisson process $N_t$ with rate $1$ (TODO ref). Then $N_{\Lambda(t)}$ is a Poisson process with cumulative rate $\Lambda$ by \ref{PoissonProcessRateTransform}.
\end{proof}

\begin{proposition}
Let $N$ be a Poisson process. Then
\begin{enumerate}
\item $N$ is a Markov process;
\item if $N$ is a homogenous Poisson process, then it is a Lévy process.
\end{enumerate}
\end{proposition}
\begin{proof}
TODO
\end{proof}

\begin{proposition}
Let $N$ be a Poisson process. For each $n\in \N$, consider the stopping time
\[ \tau_n: \sSet{\Omega, \mathcal{A}, P} \to \R_+: \omega \mapsto \inf\setbuilder{t}{N_t(\omega) - N_0(\omega) \geq n }. \]
\begin{enumerate}
\item The distribution of $\tau_n$ is given by
\[ F_{\tau_n}(t) = 1- \sum_{k=0}^{n-1}\frac{\Lambda(t)^k}{k!}e^{-\Lambda(t)} = F_{\Erlang(n, 1)}(\Lambda(t)). \]
\item If $N$ is a homogenous Poisson process with rate $\lambda$, then $\tau_n \sim \Erlang(n, \lambda)$. In particular $\tau_1 \sim \Exponential(\lambda)$.
\item If $\Lambda$ is bijective, then $N_{\Lambda^{-1}(t)}$ is a homogenous Poisson process with rate $1$ and stopping times $\Lambda(\tau_n)$.
\end{enumerate}
\end{proposition}
\begin{proof}
(1) We have
\begin{align*}
F_{\tau_n}(t) &= P\{\tau_n \leq t\} = P\{N_t \geq n\} \\
&= \sum_{k=n}^{\infty}P\{N_t = k\} = \sum_{k=n}^{\infty} \frac{\Lambda(t)^k}{k!}e^{-\Lambda(t)} \\
&= (e^{\Lambda(t)} - \sum_{k=0}^{n-1}\frac{\Lambda(t)^k}{k!})e^{-\Lambda(t)} = 1- \sum_{k=0}^{n-1}\frac{\Lambda(t)^k}{k!}e^{-\Lambda(t)}.
\end{align*}
We see this is equal to $F_{\Erlang(n, 1)}(\Lambda(t))$ by comparing with \ref{ErlangCDF}. 

(2) This follows by the substitution $\Lambda(t) = \lambda t$ and comparison with \ref{ErlangCDF}.

(3) We calculate
\[ \inf\setbuilder{t}{N_{\Lambda^{-1}(t)}(\omega) - N_0(\omega) \geq n } = \inf\setbuilder{\Lambda(u)}{N_{u}(\omega) - N_0(\omega) \geq n } = \Lambda(\tau_n), \]
using the substitution $u = \Lambda^{-1}(t)$ and TODO ref inf preserving.
\end{proof}

\begin{proposition}
Let $N$ be a homogenous Poisson process with rate $\lambda$. Then the interarrival times $S_n = \tau_n - \tau_{n-1}$ are i.i.d. random variables with distribution $\Exponential(\lambda)$.
\end{proposition}
\begin{proof}
TODO
\end{proof}

\section{Stochastic integration}
\subsection{Stochastic differential equations}



\part{Statistics}
\setcounter{chapter}{0} % Reset chapter counters
TODO: comparison of two runs of a simulation as measure of accuracy and variance??

\chapter{Descriptive statistics}
TODO: Odds!


\section{Statistics}

\section{Random samples}

\section{Cox's theorem}

\chapter{Some important distributions}
\section{Discrete distributions}
\section{Continuous distributions}

\chapter{Multivariate statistics}

\chapter{Convergence and limits}

\chapter{Statistical models and parametric point estimation}
\section{Point estimators}
\section{Confidence regions}

\chapter{Estimation methods and estimation theory}

\chapter{Hypothesis testing}

\chapter{Bayesian statistical inference and nonparametric statistical inference}

\chapter{Experimental methods}
ML / LS curve fitting
error propagation


\part{Geometry}
\setcounter{chapter}{0} % Reset chapter counter
\chapter{Introduction}
Space is obviously quite important for physics. Finding a mathematical model for space presents some challenges. Many branches of mathematics have tried to model essential characteristics of space in different ways. This has lead to many different mathematical meanings, one of which we have already seen: the vector space.

\begin{displayquote}
SPACE. The word came into English—from Old French from Latin—around 1300. The OED entry distinguishes many meanings. In one sense (under heading 6b) it has room as a synonym. This word derives from the Old English and is related to the modern German Raum. Under heading 17 the OED defines “a space” as “an instance of any of various mathematical concepts, usually regarded as a set of points having some specified structure.” Among the quotations is a nice one from 1932: “The word ‘space’ has gradually acquired a mathematical significance so broad that it is virtually equivalent to the word ‘class’, as used in logic.” (M. H. Stone Linear Transformations in Hilbert Space p. 1.) The space age was well under way by 1914 when Hausdorff’s Grundzüge der Mengenlehre (Fundamentals of Set Theory) gave axioms for a METRIC SPACE (metrischer Raum) and for a TOPOLOGICAL SPACE (topologischer Raum).\footnote{From \textit{Earliest Known Uses of Some of the Words of Mathematics}, \url{http://jeff560.tripod.com/s.html}}
\end{displayquote}

The main branch of mathematics that is relevant for the modeling of space is obviously geometry. Just like the word ``space'', the label geometry is applied to many parts of mathematical reasoning.

\section{Erlangen programm}
In 1872 Felix Klein proposed his \udef{Erlangen programm} (named after the University Erlangen-Nürnberg, where Klein worked) which attempted to classify different geometries based on symmetries. In particular it allowed geometry to be viewed as the study of properties of figures that remain invariant under a certain group of transformations. So for example, in Euclidean geometry may be viewed as the study of properties that remain invariant under isometry transformations (these can roughly be viewed as the translations and rotations). Two shapes are called \udef{congruent} if there is an isometry that maps one onto the other. If we take a rectangle, its corners will always be $90\si{\degree}$ angles no matter how we rotate or translate it. Angles are in general invariant under such transformations. Other invariants include
\begin{itemize}
\item Distances;
\item Areas;
\item Volumes;
\item Whether lines are parallel, or not;
\item Whether points are or other shapes, or not;
\item Whether points are collinear, or not;
\end{itemize}

Another important aspect to the Erlangen program is its hierarchical nature: if we take a larger group of transformations, then fewer aspects will invariants of all the transformations. Conversely, if we restrict the group of transformations, more aspects will be invariants. For example, the isometry transformations are part of a larger group of transformations called affine transformations. In particular they are the affine transformations that preserve distance. Of the bulleted list of invariants above, the last three are affine invariants, but the first three are \textbf{not}.

In this way projective geometry may be seen as the underlying, unifying frame. Restricting it a bit we get affine geometry; restricting it a bit more we get Euclidean geometry.

\section{Analytic–synthetic distinction}
TODO

\chapter{Euclidean and related geometry}
\section{Axiomatic (or synthetic) Euclidean geometry}
Historically it was easy: geometry was the study of shapes, either on a flat plane or in space (or what was somewhat naively thought to be space). By the third century BC Euclid had published his \textit{Elements}. In it he gave gave the postulates of what is now called Euclidean geometry. Euclid held his postulates to be self-evidently applicable the the actual, physical space in the real world.

\subsection{Euclid's \textit{Elements}}
There are 13 books\footnote{It must be remembered that in ancient times book binding technology was not as advanced and works were published in smaller subunits called \textit{books}. Each book was only a few dozen pages long, in today's pages, but published on scrolls.} in the \textit{Elements}.
\begin{itemize}
\item Books I to IV and VI discuss planar geometry.
\begin{itemize}
\item[\textbf{Book I}] lays out the fundamentals of planar geometry involving straight lines.
\item[\textbf{Book II}] contains propositions to do with squares and rectangles. This has a strong link with geometric algebra due to the link with the squaring of a number.  
\item[\textbf{Book III}] lays out the fundamentals of planar geometry involving circles.
\item[\textbf{Book IV}] deals with the intersection of circles and rectilinear figures.
\item[\textbf{Book VI}] discusses similar figures.
\end{itemize}
\item  Books V and VII-X deal with number theory, with numbers treated geometrically as lengths of line segments or areas of regions.
\item Books XI-XIII concern solid geometry.
\item There are apocryphal books XIV and XV, probably written by Hypsicles and Isidore of Miletus, respectively. These are not usually included.
\end{itemize}

What follows is the first part of book I\footnote{Translation by TODO ref}, which contains some definitions and Euclid's postulates.

\begin{displayquote}
{\LARGE \centering BOOK I \par}
{\centering DEFINITIONS\par}
\begin{enumerate}
\item A \textbf{point} is that which has no part.
\item A \textbf{line} is breadthless length.
\item The extremities of a line are points.
\item A \textbf{straight line} is a line which lies evenly with the
points on itself.
\item A \textbf{surface} is that which has length and breadth only.
\item The extremities of a surface are lines.
\item A \textbf{plane surface} is a surface which lies evenly with
the straight lines on itself.
\item A \textbf{plane angle} is the inclination to one another of
two lines in a plane which meet one another and do not lie in
a straight line.
\item And when the lines containing the angle are straight,
the angle is called \textbf{rectilineal}.
\item When a straight line set up on a straight line makes
the adjacent angles equal to one another, each of the equal
angles is \textbf{right}, and the straight line standing on the other is
called a \textbf{perpendicular} to that on which it stands.
\item An \textbf{obtuse angle} is an angle greater than a right
angle.
\item An \text{acute angle} is an angle less than a right angle.
\item A \textbf{boundary} is that which is an extremity of anything.
\item A \textbf{figure} is that which is contained by any boundary
or boundaries.
\item A \textbf{circle} is a plane figure contained by one line such
that all the straight lines falling upon it from one point among
those lying within the figure are equal to one another;
\item And the point is called the \textbf{centre} of the circle.
\item A \textbf{diameter} of the circle is any straight line drawn
through the centre and terminated in both directions by the
circumference of the circle, and such a straight line also
bisects the circle.
\item A \textbf{semicircle} is the figure contained by the diameter
and the circumference cut off by it. And the centre of the
semicircle is the same as that of the circle.
\item \textbf{Rectilineal figures} are those which are contained
by straight lines, \textbf{trilateral} figures being those contained by
three, \textbf{quadrilateral} those contained by four, and \textbf{multilateral}
those contained by more than four straight lines.
\item Of trilateral figures, an \textbf{equilateral triangle} is that
which has its three sides equal, an \textbf{isosceles triangle} that
which has two of its sides alone equal, and a \textbf{scalene
triangle} that which has its three sides unequal.
\item Further, of trilateral figures, a \textbf{right-angled triangle}
is that which has a right angle, an \textbf{obtuse-angled
triangle} that which has an obtuse angle, and an \textbf{acute-angled
triangle} that which has its three angles acute.
\item Of quadrilateral figures, a \textbf{square} is that which is
both equilateral and right-angled; an \textbf{oblong} that which is
right-angled but not equilateral; a \textbf{rhombus} that which is
equilateral but not right-angled; and a \textbf{rhomboid} that which
has its opposite sides and angles equal to one another but is
neither equilateral nor right-angled. And let quadrilaterals
other than these be called \textbf{trapezia}.
\item \textbf{Parallel} straight lines are straight lines which,
being in the same plane and being produced indefinitely in
both directions, do not meet one another in either direction.
\end{enumerate}
{\centering POSTULATES\par}
Let the following be postulated :
\begin{enumerate}
\item To draw a straight line from any point to any point.
\item To produce a finite straight line continuously in a
straight line.
\item To describe a circle with any centre and distance.
\item That all right angles are equal to one another.
\item That, if a straight line falling on two straight lines
make the interior angles on the same side less than two right
angles, the two straight lines, if produced indefinitely, meet
on that side on which are the angles less than the two right
angles.
\end{enumerate}
{\centering COMMON NOTIONS\par}
\begin{enumerate}
\item Things which are equal to the same thing are also
equal to one another.
\item If equals be added to equals, the wholes are equal.
\item If equals be subtracted from equals, the remainders
are equal.
\item Things which coincide with one another are equal to
one another.
\item The whole is greater than the part.
\end{enumerate}
\end{displayquote}

The rest of book I contains propositions. The other books contain definitions and propositions.

When Euclid uses ``equal'', he means equal in magnitude. This is effectively what is now called congruence. This equals is obviously an equivalence relation (the transitivity and reflexivity are asserted as ``common notions'' 1. and 4.). Common notions 2. and 3. assert that the operations of addition and subtraction are compatible with the equality relation.

There is some debate as to whether the common notions were (in part or at all) included by Euclid\footnote{See ref TODO for much insightful commentary}. 

In the postulates, Euclid only explicitly asserts the \textit{existence} of the constructed objects, in his reasoning \textit{uniqueness} is implicitly assumed. 

\subsubsection{The fifth postulate.} Reading through the postulates, it may be obvious that the fifth is of quite a different nature than the rest. It looks like is should be a proposition and for centuries mathematicians tried, unsuccessfully, to prove it as such.

Euclid himself, it seems, mistrusted the postulate and proved the first 28 propositions without using it.

Many equivalent formulations of the fifth postulate exist, the most famous probably being Playfair's axiom:

\begin{displayquote}
In a plane, through a point not on a given straight line, at most one line can be drawn that never meets the given line.
\end{displayquote}

In fact there is always exactly one. It can be shown (and in fact follows from proposition 27) that even in absolute geometry we can always find at least one parallel line.

Only in the beginning of the $19^\text{th}$ century did mathematicians start what Euclidean geometry would look like without the fifth postulate. Leaving out this postulate, one obtains what is known as \udef{absolute geometry}.

If the fifth postulate were provable as a theorem, absolute geometry would be the same as Euclidean geometry. It turns out that this is not the case and including the negation of the fifth postulate leads to a consistent set of axioms, describing a \udef{non-Euclidean geometry}.

\subsection{Other axiomatic systems}
Euclid's axioms do not actually provide the complete, logical foundation he thought they did. Many authors have offered their own sets of axioms that meet modern standards or rigour. Moritz Pasch was the first to accomplish this task in 1882.

\subsubsection{David Hilbert} proposed a set of axioms in 1899 that did not depart too greatly in spirit from Euclid's, but was complete. The result was an axiom system constructed with six primitive notions (point, line, plane, betweenness, congruence and containment) and twenty axioms devided into 5 classes (incidence, order, congruence, parallels and continuity).

\subsubsection{George Birkhoff} created a set of axioms\footnote{TODO ref.} for planar geometry that uses real numbers in 1932. Leveraging the mathematics of real numbers, only four axioms were needed. Birkhoff's reason for introducing them was that they may be readily verified in the real world using a ruler and a protractor. Their simplicity also allowed them to be used in high-school books.

The primitive terms are:
\begin{enumerate}[(a)]
\item \textbf{Points}, designated by $A,B,C, \ldots$
\item Particular sets of points called \textbf{lines}, designated by $l,m, \ldots$
\item The symmetric relation \textbf{distance} between any two points $A,B$, designated by $d(A,B)$
\item The \textbf{angle} determined by three ordered points $A,O,B \;(A\neq O, B\neq O)$, designated $\angle AOB$, is a relation between two lines, $AO$ and $BO$.
\end{enumerate}
Then the axioms may be stated as follows:
\begin{enumerate}
\item[\textbf{Postulate I}] \textit{Postulate of Line Measure}. The points $A, B, \ldots$ of any line can be put into 1:1 correspondence with the real numbers $x$ so that $|x_B - x_A| = d(A, B)$ for all points $A$ and $B$. 
\item[\textbf{Postulate II}] \textit{Point-Line Postulate}. There is one and only one straight line, $l$, that contains any two given distinct points $P$ and $Q$.

\begin{definition}
Before continuing with the rest of the postulates, we must interject with a few definitions.
\begin{itemize}
\item For any three points $A,B,C$ on the same line (i.e.\ \udef{collinear points}), point $B$ is said to be \udef{between} $A$ and $C$ if
\[ d(A,C) = d(A,B) + d(B,C) \]
\item A \udef{line segment} $AC$ is the set of points on the line through $A$ and $C$ that are between $A$ and $C$. Equivalently, it is the set of points $P$ such that
\[ x_A \leq x_P \leq x_C \quad \text{or} \quad x_C \leq x_P \leq x_A \]
\item A \udef{ray} or \udef{half-line} $l'$ with \udef{end-point} $O$ is defined by two distinct points $O,A$ on line $l$ as the set of all points $A'$ on $l$ such that $O$ is not between $A$ and $A'$. TODO fig.
\item A \udef{broken line} $ABC\ldots KL$ consists of a collection of segments $AB, BC, CD, \ldots, KL$. The points $A,B, \ldots, L$ are the \udef{vertices} of the broken line.
\item If the initial point $A$ and the terminal point $L$ coincide, the broken line is called a \udef{polygon}.
\item A polygon with three distinct vertices is called a \udef{triangle}.
\end{itemize}
\end{definition}

\item[\textbf{Postulate III}] \textit{Postulate of Angle Measure}. The rays {$l, m, n, \ldots$} through any point $O$ can be put into 1:1 correspondence with the real numbers $a \mod 2\pi$ so that if $A$ and $B$ are points (not equal to $O$) of $l$ and $m$, respectively, the difference $(a_m - a_l) \mod 2\pi$ of the numbers associated with the lines $l$ and $m$ is the angle $\angle AOB$. Furthermore, if the point $B$ on $m$ varies continuously in a line $r$ not containing the vertex $O$, the number $a_m$ varies continuously also. 
\item[\textbf{Postulate IV}] \textit{Postulate of Similarity}\footnote{This postulate rules out non-Euclidean geometries}. If in two triangles $\triangle ABC$ and $\triangle A'B'C'$  and for some constant $k > 0$,
\[d(A', B' ) = k\cdot d(A, B), \quad d(A', C' ) = k\cdot d(A, C) \quad \text{and} \quad \angle B'A'C'  = \pm \angle BAC, \]
then
\[d(B', C' ) = k\cdot d(B, C), \quad \angle  C'B'A'  = \pm \angle CBA, \quad \text{and}\quad  \angle A'C'B'  = \pm \angle ACB. \]
\end{enumerate}

\begin{definition}
Some more definitions:
\begin{itemize}
\item As a consequence of postulate II, two distinct lines have either one point in common, or none. In the first case they are said to \udef{intersect} in their common point; in the second case, they are said to be \udef{parallel}.
\item Two figures are called \udef{similar} if all corresponding distances are in proportion and all corresponding angles are equal or all negatives of each other.
\item Two figures are called \udef{congruent} if they are similar with a ratio of proportionality equal to one.
\end{itemize}
\end{definition}

\paragraph{Modifications to allow for higher dimensions}

\section{Analytic Euclidean geometry}

In the $17^\text{th}$ century René Descartes and Pierre de Fermat departed from this purely axiomatic (or \udef{synthetic}) approach and introduced the \udef{analytic} approach, explicitly using a coordinate system. This was an important reason for developing linear algebra (as lines and planes can be represented by linear equations) and one of the reasons we talk about vector \textit{spaces}. In fact $n$-dimensional Euclidean space can be modeled using an $n$-dimensional vector space with the standard inner product (that supplies the notions of distance and angle).

For the rest of the geometries mentioned here, we will focus on the analytic side of things, as that approach is more useful for the practicing physicist.

\subsection{Introducing the model}
All this talk of axioms may be frustrating for an engineer of physicist who just wants to be able to calculate. We now introduce a model (in fact a class of models) that can easily be used to calculate with. It is of course important to verify that these models do in fact describe Euclidean geometries. We will do that by verifying that they satisfy Birkhoff's postulates.

The models are based on real vector spaces equipped with the dot product, together with definitions for the primitive terms point, line, distance and angle.

TODO:

position and displacement vector

Zero

dimension

$V$

Using bold $\vec{v}$ for vectors

Taking a look at the axiom for angle measurement, it is obvious that we need a way to determine the angle between two vectors where the angle can be anything from $0$ to $2\pi$. The problem with this is that the dot product only gives us the cosine of the angle $\cos{\theta}$. Inverting that, we get a number between $0$ and $\pi$. In other words, always the smallest angle between two vectors (TODO fig). What we need to do in order to get an number between $0$ and $2\pi$ is fix an \textbf{orientation}. To do that we need a basis (TODO need?). We can then multiply the result of the arc cosine by $-1$ and take the angle mod $2\pi$ if the orientation of the vectors is negative.

\begin{note}
\begin{enumerate}
\item[\textbf{Point}] A point $A$ is modeled by a vector $\vec{v}_A \in V$.
\item[\textbf{Line}] A line is any set of vectors of the following form
\[ l = \{ \vec{v}_A + \lambda(\vec{v}_B - \vec{v}_A) \;|\; \lambda \in \R\} \]
where $\vec{v}_A$ and $\vec{v}_B$ are distinct vectors in $V$. Conventionally this is denoted
\[ l \; \leftrightarrow \; \vec{v}_A + \lambda(\vec{v}_B - \vec{v}_A). \]
\item[\textbf{Distance}] The distance between points $\vec{v}_A$ and $\vec{v}_B$ is given by
\[ d(A,B) = \lVert \vec{v}_B - \vec{v}_A \lVert = \sqrt{(\vec{v}_B - \vec{v}_A)\cdot (\vec{v}_B - \vec{v}_A)} \]
\item[\textbf{Angle}] The angle $\angle AOB$ is given by
\[ \angle AOB = \pm\cos^{-1}\left(\frac{(\vec{v}_A- \vec{v}_O)\cdot (\vec{v}_B- \vec{v}_O)}{\lVert \vec{v}_A - \vec{v}_O\lVert\cdot\lVert \vec{v}_B- \vec{v}_O\lVert}\right) \]
\end{enumerate}
The angle is negative if $\left((\vec{v}_A- \vec{v}_O), (\vec{v}_B- \vec{v}_O)\right)$ has a negative orientation.
\end{note}

\subsection{Compatibility with Birkhoff's postulates}
TODO: lots of figures

It turns out it's easiest to consider the postulates in a different order, so that is what we will do.

\begin{enumerate}
\item[Postulate II] This proof contains two parts:
\begin{enumerate}
\item \textit{Existence}: for any two points a straight line can be found that contains both points.
\item \textit{Uniqueness}: only one such line can be found. We need to show that any such line we can construct is equivalent.
\end{enumerate}
The proof is as follows:
\begin{enumerate}
\item Existence is easy. Take two arbitrary points $P,Q$. Consider the line
\[ l \;\leftrightarrow\; \vec{v}_P + \lambda (\vec{v}_Q - \vec{v}_P) \]
setting $\lambda = 0$ we see that the line contains $\vec{v}_P$; setting $\lambda = 1$ we see that the line contains $\vec{v}_Q$. So this line is a good line.
\item Say we have another straight line $m$ such that $m$ contains $\vec{v}_P$ and $\vec{v}_Q$. The line $m$ can be written as
\[ m \;\leftrightarrow\; \vec{v}_A + \mu (\vec{v}_B - \vec{v}_A)  \]
for some $\vec{v}_A$ and $\vec{v}_B$. We must show that $m = l$. We split this into two parts: $l \subset m$ and $m \subset l$.
\begin{enumerate}
\item[$\boxed{l \subset m}$] Because $\vec{v}_P, \vec{v}_Q \in m$ there must exist $\mu_P, \mu_Q \in \R$ such that
\begin{equation}
\begin{cases}
\vec{v}_A + \mu_P(\vec{v}_B - \vec{v}_A) = \vec{v}_P \\
\vec{v}_A + \mu_Q(\vec{v}_B - \vec{v}_A) = \vec{v}_Q.
\end{cases} \label{vPvQ}
\end{equation}
These expressions for $\vec{v}_P$ and $\vec{v}_Q$ can be filled in in the expression for the line $l$:
\begin{align}
l \;\leftrightarrow\; &\vec{v}_P + \lambda (\vec{v}_Q - \vec{v}_P) \\
&\vec{v}_A + \mu_P(\vec{v}_B - \vec{v}_A) + \lambda ((\vec{v}_A + \mu_Q(\vec{v}_B - \vec{v}_A)) - (\vec{v}_A + \mu_P(\vec{v}_B - \vec{v}_A))) \\
&\vec{v}_A + \mu_P(\vec{v}_B - \vec{v}_A) + \lambda (\mu_Q(\vec{v}_B - \vec{v}_A) - \mu_P(\vec{v}_B - \vec{v}_A)) \\
&\vec{v}_A + [\mu_P + \lambda(\mu_Q - \mu_P)](\vec{v}_B - \vec{v}_A).
\end{align}
For every $\lambda \in \R$, the expression $(\mu_P + \lambda(\mu_Q - \mu_P))$ is a real number and thus a value $\mu$ can take. This means that every point of $l$ is also a point of $m$ and thus $l \subset m$.
\item[$\boxed{m \subset l}$] Because $\vec{v}_P$ and $\vec{v}_Q$ are distinct (and thus $\mu_P \neq \mu_Q$), equations (\ref{vPvQ}) can be inverted to obtain
\[ \begin{cases}
\vec{v}_A = \vec{v}_P + \frac{-\mu_P}{\mu_P - \mu_Q}(\vec{v}_Q - \vec{v}_P) \\
\vec{v}_B = \vec{v}_P + \frac{\mu_P-1}{\mu_P - \mu_Q}(\vec{v}_Q - \vec{v}_P)
\end{cases} \]
With a very similar line of reasoning, we can see that $m \subset l$.
\end{enumerate}
This concludes the proof.
\end{enumerate}
\item[Postulate I] Many such bijections can be found, each corresponding with a different placement and orientation of the ruler used. Assume that the points $A,B,C,D$ are on the line $l$ and are distinct. We elect to place the beginning of our ruler at point $C$and consider the half of the line on which $D$ lies as being in the positive direction. Consider the function
\[ x: l\to \R: A \mapsto x_A = \pm\lVert \vec{v}_A - \vec{v}_C \lVert \]
where the expression for $x_A$ is positive if $C$ is not between $A$ and $D$. We now need to prove two things
\begin{enumerate}
\item The proposed mapping $x$ is a 1:1 correspondence (i.e.\ a bijection) and
\item The distance $d(A,B) = \lVert \vec{v}_B - \vec{v}_A\lVert$ is equal to $|x_B - x_A|$.
\end{enumerate}
We proceed as follows:
\begin{enumerate}
\item We first introduce the special unit vector 
\[ \hat{v}_l \equiv \frac{\vec{v}_D - \vec{v}_C}{\lVert \vec{v}_D - \vec{v}_C\lVert} \]
Because $D$ and $C$ lie on the line and taking into account the second postulate, we see that we can write the line $l$ as
\[ l \;\leftrightarrow\; \vec{v}_C + \lambda \hat{v}_l. \]
\begin{itemize}
\item Now to prove surjectivity, we need to prove that for any real number $y$. There is a point $\vec{v}_E$ on the line such that $x_E = y$. Take an arbitrary real number $y$. The claim is now that the relevant point is given by $\vec{v}_E = \vec{v}_C + y \hat{v}_l$. Indeed
\[ x_E = \pm\lVert (\vec{v}_C + y \hat{v}_l) - \vec{v}_C \lVert = \pm\lVert y \hat{v}_l \lVert = \pm |y| \lVert \hat{v}_l \lVert = \pm |y|. \]
Now this is positive if $\vec{v}_E$ is on the $D$ side of $\vec{C}$, which is exactly the case if $y$ is positive, so $x_E = y$.
\item Injectivity states that if we have two distinct points $A,B\in l$, then $x_A \neq x_B$. To prove this, write $A$ and $B$ as
\[ \begin{cases}
\vec{v}_A = \vec{v}_C + y_A \hat{v}_l \\
\vec{v}_B = \vec{v}_C + y_B \hat{v}_l
\end{cases} \]
which must necessarily be possible for some $y_A, y_B \in \R$ with $y_A\neq y_B$. Reasoning as before we obtain
\[ \begin{cases}
x_A = y_A \\ x_B = y_B.
\end{cases} \]
Thus $x_A \neq x_B$.
\end{itemize}
\item As before we write
\[ \begin{cases}
\vec{v}_A = \vec{v}_C + y_A \hat{v}_l \\
\vec{v}_B = \vec{v}_C + y_B \hat{v}_l
\end{cases} \]
Consequently
\begin{align}
d(A,B) &= \lVert \vec{v}_B - \vec{v}_A\lVert \\
&= \lVert (\vec{v}_C + y_B \hat{v}_l) - (\vec{v}_C + y_A \hat{v}_l)\lVert \\
&= \lVert y_B \hat{v}_l - y_A \hat{v}_l\lVert \\
&= |y_B - y_A|\cdot\lVert \hat{v}_l\lVert \\
&= |y_B - y_A|
\end{align}
and
\begin{align}
|x_B - x_A| = |y_B - y_A|.
\end{align}
\end{enumerate}
This concludes the proof.
\item[Postulate III] The proof that our model satisfies this axiom is similar to the last one. We will again propose a mapping that we will show to be a bijection with the requisite properties. We choose a ray $n$ to act as our reference. For any ray $l$ with a point $A$ on it, we define the unit vector along the ray $\hat{e}_l$ as
\[ \hat{e}_l = \frac{\vec{v}_A - \vec{v}_O}{\lVert \vec{v}_A - \vec{v}_O \lVert}. \]
As a consequence of the second postulate these unit vectors are unique.
Then we define the function $a$ on the rays through $O$ as follows:
\[ a_l = \pm \cos^{-1}(\hat{e}_l \cdot \hat{e}_n) \mod 2\pi \]
where the minus sign appears if $(\hat{e}_l, \hat{e}_n)$ has a negative orientation.

We also define $\hat{e}_t$ as the unique unit vector that makes $(\hat{e}_n, \hat{e}_t)$ a positively oriented orthonormal basis (TODO ?).

Now we need to show that
\begin{enumerate}
\item The mapping $a$ is a bijection.
\item If $A$ and $B$ are points (not equal to $O$) of rays $l$ and $m$ through $O$, then 
\[ \angle AOB = (a_m - a_l)\mod 2\pi \]
\item If $B$ varies continuously, then $a_m$ varies continuously also.
\end{enumerate}
We proceed as follows:
\begin{enumerate}
\item We first prove injectivity and then surjectivity.
\begin{itemize}
\item To prove injectivity we take two rays $l$ and $m$. Assuming that $a_l = a_m$, we need to show that $l=m$. Clearly $a_l = a_m$ implies that
\[ \hat{e}_l \cdot \hat{e}_n = \hat{e}_m \cdot \hat{e}_n \]
and that $(\hat{e}_l, \hat{e}_n)$ and $(\hat{e}_m, \hat{e}_n)$ have the same orientation.
The unit vector $\hat{e}_l$ has the following orthonormal decomposition:
\[ \hat{e}_l = (\hat{e}_l \cdot \hat{e}_n)\hat{e}_n + (\hat{e}_l \cdot \hat{e}_t)\hat{e}_t \]
Using the fact that it is a unit vector, we get
\[ \hat{e}_l^2 = (\hat{e}_l \cdot \hat{e}_n)^2 + (\hat{e}_l \cdot \hat{e}_t)^2 = 1 \]
Thus $\hat{e}_l \cdot \hat{e}_t = \pm\sqrt{1- (\hat{e}_l \cdot \hat{e}_n)^2}$. This shows that $\hat{e}_l$ is one of two vectors. For one of those the orientation of $(\hat{e}_l, \hat{e}_n)$ is positive, for it is negative (TODO: show). Same for $\hat{e}_m$. Thus because the orientation of $(\hat{e}_l, \hat{e}_n)$ and $(\hat{e}_m, \hat{e}_n)$ is the same, $\hat{e}_l$ and $\hat{e}_m$ are the same vector. Then considering the rays through $\vec{v}_O$ and $\vec{v}_O + \hat{e}_l$, and through $\vec{v}_O$ and $\vec{v}_O + \hat{e}_m$, postulate II gives $l=m$ and the sought-after injectivity.
\item For surjectivity we must find a ray $l$ for every angle in $[0,2\pi[$ such that $a_l$ equals that angle. Take an arbitrary angle $\theta \in [0,2\pi[$. TODO
\end{itemize}
\item TODO
\item TODO
\end{enumerate}
This concludes the proof.
\item[Postulate IV] TODO
\end{enumerate}

\subsection{Towards categoricity}
TODO $\mathbb{E}$
\subsection{Spatial analytic geometry}

\section{Projective geometry}

Also in the $17^\text{th}$ mathematicians were trying to see what happened if certain axioms or concepts were left out of Euclids axiomatic system.
In particular Girard Desargues started the systematic study of projective geometry, which does not have any concept of distance or parallel lines. The original motivation for this was an attempt to understand perspective.

There are several axiomatisations of projective geometry (such as those by Whitehead, Coxeter, Hilbert \& Cohn-Vossen and Greenberg). We will not be considering those.

From an analytic point of view, the $n$-dimensional projective space over an arbitrary field $K$ can be constructed as follows:
\begin{itemize}
\item Take an $n$-dimensional vector space $V$ over the field $K$. We define $K_0$ and $V_0$ as resp. the sets $K$ and $V$ without the neutral element for the addition, $0$. i.e.\
\[ V_0 \equiv V\setminus \{0\} \qquad K_0 \equiv K\setminus \{0\} \]
\item We define the equivalence relation $\sim$ on $V_0$:
\[ v \sim w \;\Leftrightarrow\; \exists \lambda \in K_0: v = \lambda w. \]
It should be clear that this is indeed an equivalence relation.
\item We define $[v]$ as the equivalence class that contains $v$. Explicitly this is given by
\[ [v] = \{ \lambda v \; |\; \lambda \in K_0 \}. \]
This gives a partition of $V_0$ (i.e.\ $[v] = [w] \Leftrightarrow v\sim w$ and $v \nsim w \Rightarrow [v] \cap [w] = \emptyset$).
\item The \udef{projective space} $P(V)$ associated to $V$ can now be defined as the set of equivalence classes. In other words, it is a quotient space:
\[ P(V) \, = \,V_0 / \sim\; =  \,\{ [v] \;|\; v \in V_0 \}  \]
Each equivalence class is called a \textit{point} of the projective space. It may be strange to call a set a point (TODO point about models and isomorphism)
\item In the construction above we have collapsed whole lines (containing e.g\ the points $\lambda v$ for all $\lambda \in K$) into single points. It therefore makes sense to \textit{define} the dimension of $P(V)$ to be one less than the dimension of $V$, if $V$ has a finite dimension that is.
\end{itemize}
In order to get the full projective geometry, according to the Erlanger program, we now also need a group of transformations. For the construction proposed above, the projective transformations $\phi: P(V) \to P(W)$ can be constructed as follows.
\begin{itemize}
\item Take the vector space $\GL(V,W)$ of the bijective linear maps from $V$ to $W$. We would like to define the projective transformations as the transformations of the form
\[ P(V) \to P(W): [v] \mapsto [f(v)] \]
where $f$ is an element of $\GL(V,W)$. It is not immediately clear that this well defined. The problem is that $[v]$ is a set of which $v$ is only one element. If we take a different element of $u \in [v]$, how do we know that $f(u) \in [f(v)]$? In other words how do we know that projective transformation does not depend on the (arbitrarily chosen) representative $v$ of the projective point $[v]$?
\item Luckily we can use the result that for all $v\in V_0$, $[f(v)] = [g(v)]$ if and only if $[f] = [g]$. The notation $[f]$ makes sense because $\GL(V,W)$ is itself a vector space.
\end{itemize}


\section{Affine geometry}
Tangent space and bundle

\section{Back to Euclidean geometry}

\section{Non-Euclidean geometry}
The development of non-Euclidean geometries (with different sets of axioms) was another exciting enrichment of the field. These axiomatic systems roughly describe shapes in space that is not flat. In 2D this translates to doing geometry on surface that is not flat, like a sphere. From an analytic viewpoint, vector spaces (embodying linearity) are obviously no longer 

Tangent bundle!!

\chapter{Manifolds}
\section{Definition}
The basic idea is that a manifold is an object $M$ that looks locally like a piece of $\R^n$, i.e. around any point we can find an area small enough that it looks flat.
\begin{definition}
A topological space $M$ is \udef{locally Euclidean} of dimension $n$ is every point $p\in M$ has a neighbourhood $U$ such that there is a homeomorphism $\phi$ from $U$ onto an open subset of $\R^n$. We call
\begin{itemize}
\item the pair $(U,\phi:U\to \R^n)$ a \udef{chart};
\item $U$ a \udef{coordinate neighbourhood} or a \udef{coordinate open set};
\item $\phi$ a \udef{coordinate map} or a \udef{coordinate system} on $U$:
\[ \phi:U\to \R^n: p\mapsto (x^1(p), \ldots, x^n(p)). \]
\end{itemize}
A chart $(U,\phi)$ is \udef{centred at} $p\in U$ if $\phi(p) = \vec{0}$. A \udef{chart about} $p$ is a chart $(U,\phi)$ such that $p\in U$.
\end{definition}
TODO: for well defined: open subsets of $\R^m$ and $\R^n$ cannot be homeomorphic if $n\neq m$.
\begin{definition}
An $n$-dimensional \udef{(topological) manifold} $M$ is a second-countable, Hausdorff topological space that is locally Euclidean of dimension $n$.
\end{definition}
[Picture with manifold and overlapping patches mapping to Rm phi1 phi2 + mappings between R's .]

By requiring the topology to be second-countable (C2) and Hausdorff, we immediately guaranty our manifold has whole load of nice properties. An important one is the ability to embed manifolds in higher-dimensional Euclidean spaces (cfr. Whitney's embedding theorem). Other properties that second-countable Hausdorff spaces have include being metrizable, completely normal and paracompact.

Also note that subspaces of a Hausdorff (resp. C2) space are automatically Hausdorff (resp. C2).

\begin{proposition}
Every discrete space is a $0$-dimensional manifold.
\end{proposition}
\begin{lemma}
Every manifold is locally path-connected.
\end{lemma}
This follows from the homeomorphisms with the path-connected space $\R^n$. Thus for manifolds the notions of connectedness and path-connectedness coincide.

\begin{definition}
Let $(U,\phi), (V,\psi)$ be two charts such that $U\cap V \neq \emptyset$. The maps
\[ \phi\circ\psi^{-1}:\psi(U\cap V)\to \phi(U\cap V) \qquad \text{and} \qquad \psi\circ\phi^{-1}:\phi(U\cap V)\to \psi(U\cap V) \]
are called \udef{transition functions} or \udef{change of coordinate maps}.
\end{definition}
Transition functions are compositions of homeomorphisms and thus homeomorphisms.

\begin{definition}
An \udef{atlas} on a manifold $M$ is a collection of charts $\{(U_\alpha, \phi_\alpha)\}$ that covers $M$, i.e. $M=\bigcup_\alpha U_\alpha$.
\end{definition}

\section{Types of manifolds}
Manifolds are very useful in many areas of physics and mathematics. Consequently there are many extensions and types of manifolds.
\subsection{Topological manifolds}
Topological manifolds are manifolds with no additional structure. The moniker topological is redundant, but can be used to emphasise that there is no additional structure, or, if there was previously additional structure, that one should forget about that additional structure.
\subsection{Differential manifolds} 
Differential manifolds have an atlas that allows differential calculus to be used on the manifold.

Each chart allows the use of calculus using the standard differential structure on linear space. (TODO!) The only difficulty is with the transition maps.
\begin{definition}
Two charts are called \udef{$C^k$-compatible} if the transition functions are in $C^k$.
\end{definition}

\begin{definition}
A $C^k$ \udef{atlas} on a manifold $M$ is a collection  $\{(U_\alpha, \phi_\alpha)\}$ of \udef{$C^k$-compatible} charts that covers $M$, i.e. $M=\bigcup_\alpha U_\alpha$.
\end{definition}

\begin{lemma}
Let $\{(U_\alpha, \phi_\alpha\}$ be an atlas. If two charts $(V,\psi)$ and $(W,\sigma)$ are compatible with the atlas, they are compatible with each other.
\end{lemma}

Two atlases are \udef{$C^k$-equivalent} if the union of their sets of charts forms a $C^k$-atlas.

\begin{lemma}
The $C^k$-equivalence of atlases is an equivalence relation. Each equivalence class is called a distinct \udef{$C^k$ differential structure} of the manifold.
\end{lemma}

A \udef{maximal atlas} is an atlas that is not contained in a larger atlas.

\begin{lemma}
The union of a $C^k$-equivalence class is a maximal atlas. Conversely every maximal atlas is the union of a $C^k$-equivalence class.
\end{lemma}
A differential structure is sometimes defined as a maximal atlas. By this lemma this is equivalent.

\begin{corollary}
Any atlas is contained in a unique maximal atlas.
\end{corollary}

\begin{definition}
A \udef{differential manifold} is a manifold with a $C^k$ differential structure.
\end{definition}
In fact we can even recover the manifold from the differential structure

It is however not meaningful to talk of a $C^k$-manifold as any manifold with a $C^k$-atlas with $k>0$ can be given a $C^\infty$-atlas. In fact every $C^k$-structure is uniquely \textit{smoothable} to a $C^\infty$-structure.

The differential structure allows the definition of the globally differentiable tangent space. This process is discussed in the next section.

A manifold may also be defined as an equivalence class of atlases (TODO).

In dimensions smaller than $4$ every topological manifold has a unique differentiable structure and in dimensions larger than $4$ every compact topological manifold hasa finite number of differentiable structures. Dimension 4 is a mystery.

There are topological manifolds with no differentiable structure.

\subsection{Smooth manifolds}
Smooth manifolds are differentiable manifolds for which all the transition maps are smooth (i.e. infinitely differentiable). All concepts in the previous section relating to differential manifolds apply if $C^k$ is replaced by $C^\infty$ or ``smooth''.

We will see later that:
\begin{proposition}
Every differential manifold is uniquely smoothable. I.e. every $C^k$ differential structure has a unique $C^\infty$ structure that is $C^k$ equivalent.
\end{proposition}
Consequently there is no real difference between differential and smooth manifolds. We will mainly study the latter.


\subsection{Analytic manifolds}
Analytic manifolds are smooth manifolds with the additional condition that each transition map is analytic or $C^\omega$: the Taylor expansion is absolutely convergent and equals the function on some open ball.

\section{Manifolds with boundaries}

\section{Submanifolds}




\chapter{Smooth manifolds}
In this chapter all manifolds are assumed smooth, i.e. $C^\infty$.
\section{The tangent space}
\subsection{Functions on manifolds}
\begin{definition}
Let $M,N$ be manifolds of dimension $m,n$. A map $F: M \to N$ is said to be \udef{smooth}, or $C^\infty$, at a point $p\in M$ if there are charts $(U,\phi)$ about $p\in M$ and $(V,\psi)$ about $F(p) \in N$ such that
\[ \psi \circ F \circ \phi^{-1}: \phi[F^{-1}[V]\cap U]\subset \R^m \to \R^n \]
is smooth at $\phi(p)$.

The function $F: M\to N$ is said to be \udef{smooth} if it is smooth are every point $p\in M$.
\end{definition}
Notice that the smoothness of a function is independent of the charts chosen:
\begin{lemma}
Let $M,N$ be manifolds of dimension $m,n$. If a function $F: M\to N$ is smooth at $p\in M$, then for any charts $(U',\phi')$ about $p\in M$ and $(V',\psi')$ about $F(p)\in N$,
\[ \psi'\circ F \circ (\phi')^{-1}: \phi'[F^{-1}[V']\cap U']\subset \R^m \to \R^n \]
is $C^\infty$ at $\phi'(p)$.
\end{lemma}
\begin{proof}
Let $F: M\to N$ be smooth and $p\in M$, $(U,\phi), (V,\psi)$ be as in the definition. Then
\[ \psi' F\circ (\phi')^{-1} = (\psi'\circ \psi)\circ(\psi F\circ \phi^{-1})\circ (\phi \circ (\phi')^{-1}) \]
is smooth at $\phi'(p)$, because it is a composition of smooth maps.
\end{proof}

\begin{lemma}
Let $F: M\to N$ and $G: N\to P$ be smooth maps of manifolds, then $G\circ F$ is smooth.
\end{lemma}

\begin{definition}
A \udef{diffeomorphism of manifolds} is a bijective $C^\infty$ maps $F:M\to N$ such that $F^{-1}$ is also $C^\infty$.
\end{definition}

\begin{lemma}
If $(U,\phi)$ is a chart on a manifold $M$, then the coordinate map $\phi$ is a diffeomorphism.
\end{lemma}
\begin{proof}
By definition $\phi: U \subset M \to \phi(U) \subset \R^n$ is a homeomorphism, so we just need to check $\phi$ and $\phi^{-1}$ are smooth.  Let $(\R^n, I_{\R^n})$ be a chart of $\R^n$. Then 
\[ I_{\R^n} \circ \phi \circ \phi^{-1} \qquad \text{and} \qquad \phi\circ \phi^{-1}\circ I_{\R^n} \]
are both the identity map and thus smooth, so both $\phi$ and $\phi^{-1}$ are $C^\infty$.
\end{proof}

\begin{lemma}
Let $U$ be an open subset of a manifold $M$ of dimension $n$. If $F: U\to \R^n$ is a diffeomorphism into an open subset of $\R^n$, the $(U,F)$ is a chart in the differentiable structure of $M$.
\end{lemma}


$C^\infty(M)$; $C^\infty(M)_p$ algebra of germs of functions in $C^\infty(M)$ at $p$.
TODO Pullback? Previous chapter?

\subsection{Derivatives of functions on manifolds}
Let $r^1,\ldots,r^n$ denote the standard coordinates on $\R^n$. Then $x^i = r^i\circ \phi$.
\begin{definition}
Let $p\in U$. We define the partial derivative $\pd{f}{{x^i}}$ at the point $p$ as
\[ \left.\pd{}{x^i}\right|_p f \defeq \pd{(f\circ\phi^{-1})}{r^i}(\phi(p)). \]
\end{definition}
Alternatively, as a function on $\phi(U)$, we write
\[ \pd{f}{x^i}\circ\phi^{-1} = \pd{(f\circ\phi^{-1})}{r^i}. \]

\begin{proposition}
Let $(U,(x^1,\ldots, x^n))$ be a chart on a manifold. Then
\[ \pd{x^i}{x^j} = \delta^i_j. \]
\end{proposition}
\begin{proof}
By direct computation:
\[ \pd{x^i}{x^j}(p) = \pd{(x^i\circ\phi^{-1})}{r^j}(\phi(p)) = \pd{(r^i\circ \phi\circ\phi^{-1})}{r^j}(\phi(p)) = \pd{r^i}{r^j}(\phi(p)) = \delta^i_j. \]
\end{proof}

TODO inverse function theorem?

\subsection{Derivations of functions on manifolds}
\begin{definition}
A \udef{derivation} at a point $p$ on a manifold $M$ (or a \udef{point-derivation} of $C^\infty_p(M)$) is a map $D:C^\infty_p(M)\to \R$ that
\begin{itemize}
\item is linear;
\item satisfies the Leibniz identity:
\[ \forall f,g\in C^\infty_p(M): \quad D(fg) = (Df)g(p) + f(p)(Dg). \]
\end{itemize}
\end{definition}

\begin{definition}
A \udef{tangent vector} at a point $p$ in a manifold $M$ is a derivation at $p$. The \udef{tangent space} at $p$ is the vector space of tangent vectors, denoted $T_pM$.
\end{definition}
\begin{lemma}
Let $M$ be a manifold and $p\in M$. The tangent space at $p$ is a vector space.
\end{lemma}
\begin{proof}
The space of linear maps $D:C^\infty_p(M)\to \R$ is a vector space. We check the criterion \ref{prop:subspaceCriterion}:
\begin{itemize}
\item The zero map satisfies the Leibniz identity.
\item Let $D_1,D_2$ be derivations. Then $\forall f,g\in C^\infty_p(M)$ and $\lambda\in\R$:
\begin{align*}
(\lambda D_1 + D_2)(fg) &= \lambda D_1(fg) + D_2(fg) \\
&= \lambda (D_1f)g(p) + \lambda f(p)(D_1g) + (D_2f)g(p) + f(p)(D_2g) \\
&= (\lambda (D_1f) + (D_2f))g(p) + f(p)(\lambda(D_1g)+ (D_2g)) \\
&= (\lambda D_1 + D_2f)(f)g(p) + f(p)(\lambda D_1+ D_2)(g).
\end{align*}
\end{itemize}
\end{proof}

If $U$ is an open subset of $M$ containing $p$, then $C^\infty_p(U) = C^\infty_p(M)$ and thus $T_pU = T_pM$.

\subsubsection{Curves and derivations}

\subsection{The differential of a map between manifolds}

\subsubsection{Computations in coordinates}
$F^j = y^j\circ F$
\begin{align*}
\diff{F}_p \left(\left.\pd{}{x^i}\right|_p\right)f &= \left.\pd{}{x^i}\right|_p(f\circ F) = \pd{f}{y^j}(F(p))\pd{F^j}{x^i}(p) \\
&= \left(\pd{F^j}{x^i}(p)\left.\pd{}{y^j}\right|_{F(p)}\right)f.
\end{align*}

\subsubsection{Computation using curves}

\subsubsection{Critical and regular points}

\subsection{Bases for the tangent space at a point}

\section{Submanifolds}

\subsection{Rank theorems}
\begin{theorem}[Global rank theorem] \label{theorem:globalRank}
Let $F:M\to N$ be a smooth map of constant rank between smooth manifolds. Then
\begin{enumerate}
\item if $F$ is surjective, it is a smooth submersion;
\item if $F$ is injective, it is a smooth immersion;
\item if $F$ is bijective, it is a diffeomorphism;
\end{enumerate}
\end{theorem}

\section{Tangent and cotangent spaces}

As mentioned above, if a manifold $\mathcal{S}$ is a submanifold of some Euclidean space $\R^n$, the tangent space at a point $p$ is the set of vectors that can be expressed as $v = \left.\od{\gamma}{t}\right|_{t=0}$, where $\gamma(t)$ is a smooth curve lying in $\mathcal{S}$ and satisfying $\gamma(0) = p$. Unfortunately this definition of a tangent space only works if the manifold is embedded in a Euclidean space.

We obtain a more abstract definition by generalising the notion of a directional derivative. If we still assume our manifold $\mathcal{S}$ is embedded in $\R^n$ and $f$ is a smooth function on $\mathcal{S}$, we define the \udef{directional derivative} of $f$ at the point $p$ and in the direction $v$ to be
\[ \left(D_vf\right)(p) = \left.\od{}{t}f(\gamma(t))\right|_{t=0}, \]
where $\gamma$ is any smooth curve lying in $\mathcal{S}$ with $\gamma(0) = 0$ and $\left.\od{\gamma}{t}\right|_{t=0} = v$. This directional derivative associates a number to each smooth function $f$ and satisfies the usual product rule for derivatives.

We now define the \udef{tangent space} at $p$ to an arbitrary manifold $\mathcal{M}$, denoted $T_p(\mathcal{M})$, as the set of all \ueig{linear} maps $X$ from $C^\infty(\mathcal{M})$ into $\R$ satisfying the \ueig{product rule}
\[X(fg) = X(f)g(p) + f(p)X(g)\]
for all $f$ and $g$ in $C^\infty(\mathcal{M})$, and \ueig{localisation} which means that if $f$ equals $g$ in a neighbourhood of $p$, then $X(f) = X(g)$. Obviously $T_p(\mathcal{M})$ is a real vector space and an element of $T_p(\mathcal{M})$ is called a \udef{tangent vector} at $p$.

If $x_1,\ldots, x_n$ is a local coordinate system, then one can prove that each tangent vector $X$ at $m$ can be expressed uniquely as
\[ X(f) = \sum^n_{k=1}a_k \pd{f}{x_k}(p) = \sum^n_{k=1}a_k\partial_k(p) \]
for some real constants $a_1, \ldots, a_n$. This means that the tangent space has the same dimension as the manifold at every point.

This particular basis is called the \udef{coordinate basis} and is in general not orthonormal.

===================

Now that we have defined our manifold, we can start adding structures and concepts on top, starting with vectors.

The tangent space $T_p$ can be identified with the space of directional derivative operators along curves through $p$. These operators act on smooth functions on the manifold $M$ (i.e. on $C^\infty$ maps $f: M \to \R$).

This is a vector space. Linearity is definitely ok. We need to check that vector addition closes. To do that we verify the Leibniz (product) rule.

So we want to construct the tangent space using only things intrinsic to the manifold (no embedding). Tangent vectors to curves through $P$? What does that mean?

Directional derivatives. Basis: partial derivatives $\partial_\mu$ (i.e. directional derivatives along curve with constant $x^\nu$ for all $\nu\neq\mu$, parametrized by $x^\mu$).

Actually derivative? OK

\[\od{}{\lambda}f = \od{x^\mu}{\lambda}\partial_{\mu}f\]
So the partial derivatives $\{ \partial_\mu \}$ represent a good basis for the directional derivatives. (Coordinate basis)

\subsection{Transformation under coordinate transformations}
Change of basis immediate through chain rule. New coordinate system $x^{\mu'}$:
\[ \partial_{\mu'} = \pd{x^\mu}{x^{\mu'}}\partial_\mu \]
Thus also transformation rules for vectors: $\vec{v} = v^\mu\partial_\mu$.
\begin{align}
v^\mu \partial_\mu &= v^{\mu'}\partial_{\mu'} \\
& = v^{\mu}\pd{x^\mu}{x^{\mu'}}\partial_\mu
\end{align}
So
\[ v^{\mu'} = \pd{x^{\mu'}}{x^\mu}v^\mu \]
for all changes of coordinates (not just linear transformations)

\subsection{Cotangent space}
Gradient is one-form:
\[ \grad{f}\left(\od{}{\lambda}\right) = \od{f}{\lambda} \]
$f$ itself is not a one-form: one-forms exist only in one point and to get the derivative of $f$ we need a neighbourhood.

Basis for one-forms given by the gradients of the coordinate functions:
\[ \diff{x^\mu}(\partial_\nu) = \pd{x^\mu}{x^\nu} = \delta^\mu_{\nu} \]
with transformation rule
\[ \diff{x^{\mu'}}= \pd{x^{\mu'}}{x^\mu}\diff{x^\mu} \qquad \omega_{\mu'} = \pd{x^\mu}{x^{\mu'}}\omega_\mu \]
where $\omega = \omega_\mu \diff{x^\mu}$ is a one-form.

!! Partial derivative of tensor of higher rank than a scalar is not a tensor. (Show)
We will introduce several alternatives.


\subsection{Vector fields}
Vector field: maps smooth functions to smooth functions all over the manifold by taking derivative at each point.
We can define \udef{commutator} by its action of a function $f(x^\mu)$
\[ [X,Y](f) \equiv X(Y(f)) - Y(X(f)). \]
This is a vector field with components
\[ [X,Y]^\mu = X^\lambda \partial_\lambda Y^\mu - Y^\lambda\partial_\lambda X^\mu \]

\subsection{Tensors and tensor bundles}
\[ T = \tensor{T}{^{\mu_1}^\ldots^{\mu_k}_{\nu_1}_{\ldots}_{\nu_l}}\partial_{\mu_1}\otimes\ldots\otimes\partial_{\mu_k}\otimes\diff{x}^{\nu_1}\otimes\diff{x}^{\nu_l} \]
\[\tensor{T}{^{\mu'_1}^\ldots^{\mu'_k}_{\nu'_1}_{\ldots}_{\nu'_l}} = \pd{x^{\mu'_1}}{x^{\mu_1}}\ldots\pd{x^{\mu'_k}}{x^{\mu_k}}\pd{x^{\nu_1}}{x^{\nu'_1}}\ldots\pd{x^{\nu_l}}{x^{\nu'_l}}\tensor{T}{^{\mu_1}^\ldots^{\mu_k}_{\nu_1}_{\ldots}_{\nu_l}} \]

\chapter{(Pseudo-)Riemannian differential geometry}

TODO tangent space of regular level set (see wiskunde I).

We model spacetime as a pseudo-Riemannian manifold. This is a differentiable manifold equipped with a smooth, symmetric metric tensor that is non-degenerate everywhere. In this section we will see what all these words mean and define some more mathematics that will prove invaluable in the study of general relativity.

\section{Constructions on the manifold}
\subsection{The tangent space}

\subsection{The metric}
We refer to the components of the $(0,2)$-tensor as $g_{\mu\nu}$ (while $\eta_{\mu\nu}$ is reserved specifically for the Minkowski metric). The \udef{inverse metric} can be defined via
\[ g^{\mu\sigma}g_{\sigma\nu} = g_{\lambda\nu}g_{\lambda\mu} = \delta^{\mu}_{\nu} \]
The inverse metric is also symmetric.

We do not require the metric to be positive-definite (this is what makes pseudo-Riemannian geometry pseudo). This metric cannot serve as a metric in the topological sense, but has many other uses. 

As a symmetric $4\times 4$ matrix, it has $10$ independent components. The form of $g_{\alpha\beta}(x)$ will be different in different coordinate systems for the same geometry. Since there are $4$ arbitrary functions involved in transforming $4$ coordinates, there are really only $6$ independent functions associated with a metric. (TODO explain + reword)

\subsubsection{Line element}
For example the line element of flat, Euclidean space in Cartesian coordinates is given by
\[ \diff{s} = \left[(\diff{x})^2 + (\diff{y})^2 + (\diff{z})^2\right]^{1/2} \]
Conventionally the line element is written as a quadratic relation for $\diff{s}^2$. 
The line element is then given by
\[ \diff{s}^2 = \diff{x}^2 + \diff{y}^2 + \diff{z}^2.  \]
The line element of two dimensional Euclidean space in polar coordinates is given by
\[ \diff{s}^2 = \diff{r}^2 + (r \diff{\phi})^2. \]
The line element on a sphere is
\[ \diff{s}^2 = a^2 \left(\diff{\theta}^2 + \sin^2\theta \diff{\phi}^2\right) \]
where $a$ is the radius.

We can write the line element in the following form (TODO why linear)
\[ \diff{s}^2 = g_{\alpha\beta}(x)\diff{x^\alpha}\diff{x^\beta} \]
where $g_{\alpha\beta}(x)$ is the metric.

\subsubsection{Canonical form}
A line element specifies a geometry, but many different line elements describe the same spacetime geometry.

A metric $g_{\mu\nu}$ in its \udef{canonical form} has components:
\[ g_{\mu\nu} = \diag(-1,\ldots,-1,+1,\ldots,+1,0,\ldots,0) \]
The \udef{signature} of the metric is the number of positive and negative eigenvalues. For example the metric $\eta_{\mu\nu}$ has a signature ``minus-plus-plus-plus''. If any of the eigenvalues are zero, the metric is degenerate. If the metric contains one minus and all the rest plus (or all minus and one plus), it is called \udef{Lorentzian}.


In special relativity, the metric is the Minkowski metric $\eta_{\alpha\beta}$. The (Einstein) equivalence principle requires that it be possible at each point $P$ to change to a new set of coordinates $\tilde{x}_P$ such that
\[ \tilde{g}_{\alpha\beta}(\tilde{x}_P) = \eta_{\alpha\beta} \]

In fact we can even make the first derivatives vanish. Second derivatives cannot be made to vanish, they express curvature. In other words we can find a coordinate transformation $x^\mu \to x^{\hat{\mu}}$ such that
\[g_{\hat{\mu}\hat{\nu}}(P) = \eta_{\hat{\mu}\hat{\nu}} \qquad \partial_{\hat{\sigma}}g_{\hat{\mu}\hat{\nu}}(P) = 0.\]
Such coordinates are known as \udef{locally inertial coordinates}, and the associated basis vectors constitute a \udef{local Lorentz frame}. 

Value of locally inertial coordinates: perform calculations and express answer in coordinate independent form. Often just knowing that locally inertial coordinates exist is useful knowing the exact transformations necessary to obtain them. We can perform calculations and express answer in coordinate independent form, so that it is not modified when we transform back.

\begin{note}
TODO: derivation
\end{note}

\subsection{Tensor densities}
Tensor densities are objects that do not transform like tensors, but like tensors multiplied by a power (called the \udef{weight}) of the Jacobian. A prototypical example is given by the Levi-Civita symbol.

The transformation of the Levi-Civita symbol can be inferred by noting that for any $n\times n$ matrix $M$, the determinant is given by
\[ \epsilon_{\mu'_1\mu'_2\ldots \mu'_n}|M| = \epsilon_{\mu_1\mu_2 \ldots \mu_n}\tensor{M}{^{\mu_1}_{\mu'_1}}\tensor{M}{^{\mu_2}_{\mu'_2}}\ldots \tensor{M}{^{\mu_n}_{\mu'_n}} \]
(TODO ref linear algebra; after transform is this OK?) By setting $\tensor{M}{^{\mu}_{\mu'}} = \pd{x^\mu}{x^{\mu'}}$ we have
\[ \epsilon_{\mu'_1\mu'_2\ldots\mu'_n} = \left|\pd{x^{\mu'}}{x^\mu}\right|\epsilon_{\mu_1\mu_2\ldots\mu_n}\pd{x^{\mu_1}}{x^{\mu'_1}}\pd{x^{\mu_2}}{x^{\mu'_2}}\ldots \pd{x^{\mu_n}}{x^{\mu'_n}} \]
So the Levi-Civita symbol has weight $1$.

Another example of tensor densities is given by the determinant of the metric $g = |g_{\mu\nu}|$. The transformation law
\[g(x^{\mu'}) = \left|\pd{x^{\mu'}}{x^\mu}\right|^{-2} g(x^\mu)\]
is easily derived by taking the the determinant of the transformation law for $g_{\mu\nu}$. Here the weight is $-2$.

\subsubsection{Tensors from tensor densities}
A tensor density with weight $w$ can be turned into a tensity by multiplying it with $|g|^{w/2}$. The absolute value is necessary because we have not required $g$to be positive definite, and in fact for Lorentzian metrics $|g| = -g$. This factor cancels the Jacobian factor.

For example we can define the \udef{Levi-Civita tensor}
\[ \varepsilon_{\mu_1\mu_2\ldots\mu_n} \equiv \sqrt{|g|}\epsilon_{\mu_1\mu_2\ldots\mu_n} \]
This can also be defined with upper indices:
\[ \varepsilon^{\mu_1\mu_2\ldots\mu_n} \equiv \frac{1}{\sqrt{|g|}}\epsilon^{\mu_1\mu_2\ldots\mu_n} \]

The indices of the Levi-Civita tensor can be contracted as follows:
\[ \varepsilon^{\mu_1\mu_2\ldots\mu_p\alpha_1\alpha_2\ldots\alpha_{n-p}}\varepsilon_{\mu_1\mu_2\ldots\mu_p\beta_1\beta_2\ldots\beta_{n-p}} = (-1)^sp!(n-p)!\delta_{\beta_1}^{[\alpha_1}\ldots\delta_{\beta_{n-p}}^{\alpha_{n-p}]} \]

TODO: also definable as pseudo-tensor. Then transition SR to GR: $\epsilon \to \varepsilon$?

\subsubsection{Variational calculus for tensor densities}
TODO $g$

\subsection{Differential forms}
Differential forms are interesting (and called as such) because they can be differentiated and integrated without additional geometric structure.

\subsubsection{Exterior derivative}
The \udef{exterior derivative $\diff{}$} maps $(0,p)$-forms to $(0,p+1)$-forms and is defined as
\[ \diff{}: \Lambda^p \to \Lambda^{p+1}: \omega \mapsto \diff{\omega} = \frac{1}{p!}\partial_\nu\omega_{\mu_1,\ldots,\mu_p} \diff{x^\nu}\wedge \diff{x^{\mu_1}}\wedge \ldots \wedge \diff{x^{\mu_p}}. \]
This can also be written in components as
\[ (\diff{\omega})_{\mu_1\ldots\mu_{p+1}} = (p+1)\partial_{[\mu_1}\omega_{\mu_2\ldots\mu_{p+1}]} \]
In particular we can take the exterior derivative of a scalar ($0$-form), which gives the gradient.
\[ (\diff{f})_\mu = \partial_\mu f \qquad \text{so} \qquad \diff{f} = \diff{x^\mu}\partial_\mu f \]
The exterior derivative has some interesting properties:
\begin{enumerate}
\item A modified version of the Leibniz rule applies. Say $\omega$ is a $p$-form an $\eta$ a $q$-form, then
\[ \diff{(\omega \wedge \eta)} = (\diff{\omega})\wedge \eta + (-1)^p\omega\wedge (\diff{\eta}) \] 
\item If we try applying the exterior derivative twice (TODO factor), we get zero because the simple derivative commutes.
\begin{align}
\diff{(\diff{\omega})} &= \frac{1}{p!}\partial_\rho\partial_\nu\omega_{\mu_1,\ldots,\mu_p} \diff{x^\rho}\wedge\diff{x^\nu}\wedge \diff{x^{\mu_1}}\wedge \ldots \wedge \diff{x^{\mu_p}} \\
&= 0
\end{align}
\item It \textit{is} a tensor. (TODO)
\end{enumerate}
We say a $p$-form $\omega$ is
\begin{itemize}
\item \udef{closed} if $\diff{\omega} = 0$, and
\item \udef{exact} if $\omega = \diff{\eta}$ for some $\eta$.
\end{itemize}
Exact forms are closed, but the inverse is not necessarily true.

\subsubsection{Cohomology classes}

\subsubsection{Hodge duality}
The idea for Hodge duality stems from the dimensionality of spaces of $p$-forms. On an $n$-dimensional manifold $M$ the space of $p$-forms has dimension 
\[ \text{dim}(\Lambda^p(M)) = \frac{n!}{p!(n-p)!} \]

For example, if $n=4$, then
\begin{align}
\Lambda^0(M) = C^\infty \qquad &\text{dim} 1 \\
\diff{x^\mu}A_\mu \in \Lambda^1(M) = T^*(M) \qquad &\text{dim} 4 \\
\frac{1}{2}\diff{x^\mu}\wedge \diff{x^\nu}\omega_{\mu\nu} \in \Lambda^2(M) \qquad &\text{dim} 6 \\
\frac{1}{3!}\diff{x^\mu}\wedge \diff{x^\nu}\wedge \diff{x^\rho}\alpha_{\mu\nu\rho} \in \Lambda^3(M) \qquad &\text{dim} 4 \\
\frac{1}{4!}\diff{x^\mu}\wedge \diff{x^\nu}\wedge \diff{x^\rho}\wedge \diff{x^\sigma}\beta_{\mu\nu\rho\sigma} \in \Lambda^4(M) \qquad &\text{dim} 1 \\
\end{align}
In general we have
\[ \text{dim}(\Lambda^p) = \text{dim}(\Lambda^{n-p}), \]
which suggests some idea of duality (in the sense of transforming twice gives the same result). This duality is called \udef{Hodge duality}. It is embodied by the \udef{Hodge star operator $*$}, which is defined on an $n$-dimensional manifold as a map from $p$-forms to $(n-p)$-forms.
\[ *:\Lambda^p \to \Lambda^{n-p}: A \mapsto *A \]
where
\begin{align}
(*A)_{\mu_1\ldots\mu_{n-p}} &= \frac{\sqrt{|g|}}{p!}g^{\mu_1\nu_1}\ldots g^{\mu_{n-p}\nu_{n-p}}\epsilon_{\nu_1\ldots\nu_p\mu_1\ldots\mu_{n-p}}A_{\nu_1\ldots\nu_p} \\
&= \frac{1}{p!}\tensor{\varepsilon}{^{\nu_1}^{\ldots}^{\nu_p}_{\mu_1\ldots\mu_{n-p}}}A_{\nu_1\ldots\nu_p}
\end{align}
Because the Levi-Civita tensor is a proper tensor, its indices can be raised and lowered using the metric (TODO!!). This is why the second expression makes sense and is equal to the first.

Applying the Hodge star twice returns $\pm$ the original form.
\[ **A = (-1)^{s+p(n-p)} \]
where $s$ is the number of minus signs in the signature of the metric.

For example, in three dimensional Euclidean space the Levi-Civita tensor is equal to the Levi-Civita symbol ($\varepsilon = \epsilon$). The Hodge dual of the wedge product of two 1-forms gives another 1-form
\[ *(U\wedge V)_i = \epsilon_i^{ij}U_iV_k \]
which is exactly the cross product.

\paragraph{In electrodynamics} Maxwells equations can be written as
\[ \begin{cases}
\diff{F} = 0 \\ \diff{*F} = *J
\end{cases} \]
In Minkowski space all closed forms are exact, so we can find an $A$ such that
\[ F = \diff{A}. \]
This is the vector potential.
\paragraph{Strongly and weakly coupled theories.} TODO


\subsubsection{Integration of differential forms \& volumes}
TODO redo post calculus.

On an $n$-dimensional manifold $M$, integration over a region $\Sigma \subset M$ may be seen as a map from an $n$-form field $\omega$, i.e. the integrand, to the real numbers
\[ \int_\Sigma: \omega \to \R. \]
(TODO example of line integral?)

To properly motivate this however, we need to show that
\begin{enumerate}
\item the integrand is a $(0,n)$ tensor and
\item the integrand is antisymmetric.
\end{enumerate}
The first point can intuitively be motivated by viewing the $\diff{^nx}$ as taking an infinitesimal (parallelipedal) area \textit{defined by three vectors} and outputting its (infinitesimal) volume. See fig TODO. This mapping is linear.
The second second point follows because the volume is oriented.

This leads us to propose the identification
\begin{equation}
\diff{^nx} \equiv \diff{x^0}\wedge \diff{x^1}\wedge \ldots \wedge \diff{x^{n-1}} \label{eq:diffTransfRule}
\end{equation}


This is actually a tensor density (TODO: why + as opposed to $\diff{x^{\mu_1}}\wedge \ldots \wedge \diff{x^{\mu_n}}$)

This is consistent with the appearance of the Jacobian under change of coordinates (in the calculations in Euclidean space we have done so far) 
\[ \diff{^nx'} = \left|\pd{x^{\mu'}}{x^\mu}\right|\diff{^nx}. \]

To see that this is the case, we note that
\[ \diff{x^0}\wedge \ldots \wedge \diff{x^{n-1}} = \frac{1}{n!}\epsilon_{\mu_1\ldots\mu_n}\diff{x^{\mu_1}}\wedge \ldots \wedge \diff{x^{\mu_n}} \]
The factor $1/n!$ takes care of the overcounting by summing over the permutations of the indices. The Levi-Civita symbol $\epsilon$ does not change under coordinate transformations, so
\begin{align}
\epsilon_{\mu_1\ldots\mu_n}\diff{x^{\mu_1}}\wedge \ldots \wedge \diff{x^{\mu_n}} &= \epsilon_{\mu_1\ldots\mu_n} \pd{x^{\mu_1}}{x^{\mu'_1}}\ldots\pd{x^{\mu_n}}{x^{\mu'_n}}\diff{x^{\mu'_1}}\wedge \ldots \wedge \diff{x^{\mu'_n}} \\
&= \left|\pd{x^\mu}{x^{\mu}}\right|\epsilon_{\mu'_1\ldots\mu'_n}\diff{x^{\mu'_1}}\wedge \ldots \wedge \diff{x^{\mu'_n}}
\end{align}
This clearly implies equation \ref{eq:diffTransfRule}.

Recognising its nature as a tensor density, we can construct the \udef{invariant volume element} by multiplying by $\sqrt{|g|}$:
\[ \sqrt{|g'|}\diff{^nx'} = \sqrt{|g'|}\diff{x^{0'}}\wedge \ldots \wedge \diff{x^{(n-1)' }} = \sqrt{|g|}\diff{x^0}\wedge \ldots \wedge \diff{x^{n-1}} = \sqrt{|g|}\diff{^nx} \]

The invariant volume element can in fact be identified with the Levi-Civita tensor $\varepsilon$:
\begin{align}
\varepsilon &= \varepsilon_{\mu_1\ldots\mu_n}\diff{x^\mu_1}\otimes \ldots\otimes \diff{x^\mu_n} \\
&= \frac{1}{n!}\varepsilon_{\mu_1\ldots\mu_n}\diff{x^\mu_1}\wedge \ldots\wedge \diff{x^\mu_n} \\
&= \frac{1}{n!}\sqrt{|g|}\epsilon_{\mu_1\ldots\mu_n}\diff{x^\mu_1}\wedge \ldots\wedge \diff{x^\mu_n} \\
&= \sqrt{|g|}\diff{x^0}\wedge \ldots\wedge \diff{x^{n-1}} \\
&= \sqrt{|g|}\diff{^nx}
\end{align}

Finally the main result is that the integral $I$ of a scalar function $\phi(x)$ over a region $\Sigma$ of an $n$-manifold is written as
\[ \boxed{I = \int_\Sigma\phi(x) \sqrt{|g|} \diff{^n}x. } \]
This can be evaluated with the usual rules of calculus.

In a more abstract notation, we can also write
\[ I = \int_\Sigma\phi(x)\epsilon. \]

One problem with this conception of integrals is that it is not well suited to integral of vectors. And as such things like Stokes' theorem are hard to formulate.

\subsection{Vielbeins}
\begin{align}
e^a(x) &= \diff{x^\mu}e_\mu^{\;a}(x) \\
e^a(\tilde{x}) &= \diff{\tilde{x}^\mu}e_\mu^{\;a}(\tilde{x}) \\
e^a(x) = \Lambda^a_{\;b}(x)e^b(x) \qquad \text{where} \Lambda^\intercal\eta\Lambda = \eta
\end{align}

\[ \diff{x^0}\wedge\diff{x^1}\wedge\diff{x^2}\wedge\diff{x^3} = \diff{t}\wedge\diff{r}\wedge\diff{\theta}\wedge\diff{\phi}r^2\sin\theta \]
\[ \begin{cases}
x^0 = t \\ x^1 = r\cos\theta \\ x^2 = r\sin\theta\cos\phi \\ x^3 = r\sin\theta\sin\phi
\end{cases} \]

\[ \frac{1}{4!}e^a\wedge e^b\wedge e^c\wedge e^d\epsilon_{abcd} = \frac{1}{4!}\diff{x^\mu}\wedge\diff{x^\nu}\wedge\diff{x^\sigma}\wedge\diff{x^\rho}\underbrace{e_\mu^{\;a}e_\nu^{\;b}e_\rho^{\;c}e_\sigma^{\;d}\epsilon_{abcd}}_{\det e \epsilon{\mu\nu\rho\sigma}} = \diff{^4x}\sqrt{-g} = \frac{1}{4!}\sqrt{-g} \diff{x^\mu}\wedge\diff{x^\nu}\wedge\diff{x^\rho}\wedge\diff{x^\sigma}\epsilon_{\mu\nu\rho\sigma} \]
\[g_{\mu\nu} = e_\mu^{\;a}e_\nu^{\;b}\eta_{ab}\]
\[ \det g = (\det e)^2 \det \eta \]
\[ \det e = \sqrt{-\det g} \]

\section{Curvature}
TODO how to characterise curvature. TODO: why connection through derivative.

\subsection{Connection}
\[ V\in TM, \qquad V = V^\mu(x)\pd{}{x^\mu} = \tilde{V}^\mu \pd{}{\tilde{x}{^\mu}}\]
With
\[ \tilde{V}^\mu = \pd{\tilde{x}^\mu}{x^\nu}V^\nu \]
If we change coordinates
\begin{align}
\partial_\mu V^\nu &= \pd{}{x^\mu}\left(\pd{x^\nu}{\tilde{x}{^\rho}}\tilde{V}^\rho\right) \\
&= \pd{\tilde{x}^\sigma}{x^\mu}\pd{}{\tilde{x}{^\sigma}}\left(\pd{x^\nu}{\tilde{x}{^\rho}}\tilde{V}^\rho\right) \\
&= \pd{\tilde{x}^\sigma}{x^\mu}\pd{x^\nu}{\tilde{x}{^\rho}}\tilde{\partial}_\sigma\tilde{V}^\rho + \pd{x^\sigma}{x^\mu}\pd{x^\nu}{\tilde{x}{^\sigma}}{\tilde{x}^\rho}\tilde{V}^\rho
\end{align}

We want to introduce a ``new derivative'' that transforms as a tensor.
\[ (\partial_\mu \; \to \; \nabla_\mu) \]
\subsubsection{Covariant derivative}
In general the covariant derivative is a map from $(p,q)$-tensors to $(p,q+1)$-tensors with the following properties
\begin{enumerate}
\item It is linear:
\[ \nabla(aT + bS) = a\nabla T + b\nabla S\]
where $a,b\in\R$ and $T,S$ are tensors.
\item Liebnitz:
\[ \nabla(T\otimes S) = (\nabla T) \otimes S + T \otimes (\nabla S) \]
\end{enumerate}

These properties mean that $\nabla$ needs to take the form (TODO: why + notes about indices and placement)
\[ \nabla_\mu V^\nu = \partial_\mu V^\nu + \Gamma^{\nu}_{\mu\lambda}V^\lambda \]
The objects $\Gamma^{\nu}_{\mu\lambda}$ are called \udef{connection coefficients}.

The behaviour of $\nabla$ under coordinate transformations depends on how the connection coefficients transform. They need to transform in a precisely non-tensorial way to negate the non-tensorial behaviour of $\partial_\mu$.

In particular the covariant derivative needs to transform as follows:
\[ \nabla_{\mu'}V^{\nu'} = \pd{x^\mu}{x^{\mu'}}\pd{x^{\nu'}}{x^\nu}\nabla_\mu V^\nu \]
The left side of this equation gives
\begin{align}
\nabla_{\mu'}V^{\nu'} &= \partial_{\mu'}V^{\nu'} + \Gamma^{\nu'}_{\mu'\lambda'}V^{\lambda'} \\
&= \pd{x^\mu}{x^{\mu'}}\pd{x^{\nu'}}{x^\nu}\partial_\mu V^\nu + \pd{x^\mu}{x^{\mu'}}V^\nu \pd{}{x^\mu}\pd{x^{\nu'}}{x^\nu} + \Gamma^{\nu'}_{\mu'\lambda'} \pd{x^{\lambda'}}{x^\lambda}V^\lambda
\end{align}
The right side gives
\[ \pd{x^\mu}{x^{\mu'}}\pd{x^{\nu'}}{x^\nu}\nabla_\mu V^\nu = \pd{x^\mu}{x^{\mu'}}\pd{x^{\nu'}}{x^\nu}\partial_\mu V^\nu + \pd{x^\mu}{x^{\mu'}}\pd{x^{\nu'}}{x^\nu}\Gamma^\nu_{\mu\lambda} V^\lambda \]
These two equations are the same for any $V^\nu$ if the connection coefficients transform as
\[ \Gamma^{\nu'}_{\mu'\lambda'} = \pd{x^\mu}{x^{\mu'}}\pd{x^\lambda}{x^{\lambda'}}\pd{x^{\nu'}}{x^\nu}\Gamma^\nu_{\mu\lambda} - \pd{x^\mu}{x^{\mu'}}\pd{x^\lambda}{x^{\lambda'}}\pd[2]{x^{\nu'}}{x^\mu}{x^\lambda} \]
which still leaves plenty of room for choosing a specific form of the connection.

So far in this calculation we have calculated the covariant derivative of vectors. What happens when computing the covariant derivative of tensors in general? This matter can be settled by looking at the covariant derivative acting on a one-form $\omega_\nu$. Following the same reasoning as before, we get
\[ \nabla_\mu\omega_nu = \partial_\mu\omega_nu + \tilde{\Gamma}^\lambda_{\mu\nu}\omega_\lambda \]
where $\tilde{\Gamma}^\lambda_{\mu\nu}$ has the same transformation properties as $\Gamma^\lambda_{\mu\nu}$, but does not generally have to share any other similarity.

The two connections can be related if we require the covariant derivative to have some more useful properties:
\begin{enumerate}
\item[3.] Commutes with contractions
\[ \nabla_\mu(\tensor{T}{^\lambda_\lambda_\rho}) = \tensor{(\nabla T)}{_\mu^\lambda_\lambda_\rho} \]
\item[4.] Reduces to the partial derivative on scalars
\[ \nabla_\mu\phi = \partial_\mu\phi \]
\end{enumerate}

To see the effect of these new properties, we can take the covariant derivative of the scalar field $\omega_\lambda V^\lambda$:
\begin{align}
\nabla_\mu(\omega_\lambda V^\lambda) &= (\nabla_\mu \omega_\lambda)V^\lambda + \omega_\lambda(\nabla_\mu V^\lambda) \\
&= (\partial_\mu\omega_\lambda)V^\lambda + \tilde{\Gamma}^\sigma_{\mu\lambda}\omega_\sigma V^\lambda + \omega_\lambda(\partial_\mu V^\lambda) + \omega_\lambda\Gamma^\lambda_{\mu\rho}V^\rho
\end{align}
From property 4., this covariant derivative must just be partial derivative, meaning that the connection coefficients must cancel. In general
\[ \tilde{\Gamma}^\sigma_{\mu\lambda} = - \Gamma^\sigma_{\mu\lambda} \]
because both $\omega_\sigma$ and $V^\lambda$ were completely arbitrary. In conclusion
\[ \nabla_\mu\omega_\nu = \partial_\mu\omega_\nu - \Gamma^\lambda_{\mu\nu}\omega_\lambda \]

This leads us to the following formula for an arbitrary tensor $T$:
\begin{align}
\nabla_\sigma \tensor{T}{^{\mu_1\ldots\mu_k}_{\nu_1\ldots\nu_l}} = {}&\partial_\sigma \tensor{T}{^{\mu_1\ldots\mu_k}_{\nu_1\ldots\nu_l}} \\
&+ \Gamma^{\mu_1}_{\sigma\lambda}\tensor{T}{^{\sigma\mu_2\ldots\mu_k}_{\nu_1\nu_2\ldots\nu_l}} + \Gamma^{\mu_2}_{\sigma\lambda}\tensor{T}{^{\mu_1\sigma\ldots\mu_k}_{\nu_1\nu_2\ldots\nu_l}} + \ldots + \Gamma^{\mu_k}_{\sigma\lambda}\tensor{T}{^{\mu_1\mu_2\ldots\sigma}_{\nu_1\nu_2\ldots\nu_l}} \\
&- \Gamma^{\lambda}_{\sigma\nu_1}\tensor{T}{^{\mu_1\mu_2\ldots\mu_k}_{\lambda\nu_2\ldots\nu_l}} - \Gamma^{\lambda}_{\sigma\nu_2}\tensor{T}{^{\mu_1\mu_2\ldots\mu_k}_{\nu_1\lambda\ldots\nu_l}} - \ldots - \Gamma^{\lambda}_{\sigma\nu_l}\tensor{T}{^{\mu_1\mu_2\ldots\mu_k}_{\nu_1\nu_2\ldots\lambda}}
\end{align}

\subsubsection{Christoffel connection}
There are still many possible connection coefficients that satisfy the above requirements. There is however a unique one that is defined by the metric.

The first thing to note is that the difference between two connection coefficients is a $(1,2)$-tensor. This is because the non-tensorial part in the transformation rule cancels. Say $\nabla_\mu$ and $\hat{\nabla}_\mu$ are two different covariant derivatives with connection coefficients $\Gamma^\lambda_{\mu\nu}$ and $\hat{\Gamma}^\lambda_{\mu\nu}$. Then the difference
\[ \tensor{S}{^\lambda_\mu_\nu} = \Gamma^\lambda_{\mu\nu} - \hat{\Gamma}^\lambda_{\mu\nu}\]
is the $(1,2)$-tensor.

From any given connection $\Gamma^\lambda_{\mu\nu}$, a new one can be formed by permuting the lower indices. This new object still satisfies the transformation requirement and is thus a good connection. The difference between these two is a tensor known as the \udef{torsion tensor $\tensor{T}{^\lambda_\mu_\nu}$}
\[ \tensor{T}{^\lambda_\mu_\nu} = \Gamma^\lambda_{\mu\nu} - \Gamma^\lambda_{\nu\mu} = 2\Gamma^\lambda_{[\mu\nu]} \]

Now a unique connection can be defined on a manifold with a metric $g_{\mu\nu}$ by introducing two additional properties:

\begin{itemize}
\item The connection is torsion free, meaning
\[ \Gamma^\lambda_{\mu\nu} - \Gamma^\lambda_{\nu\mu} = 0 \qquad \text{or, equivalently} \qquad \tensor{T}{^\lambda_\mu_\nu} = 0 \]
The lower indices of a torsion free connection commute.
\item The connection is \udef{metric compatible}. This means that the covariant derivative of the metric is zero everywhere.
\[ \nabla_\rho g_{\mu\nu} = 0 \]
\end{itemize}
Such a covariant derivative has a number of nice properties:
\begin{eigenschap}
\begin{itemize}
\item The covariant derivative of the inverse metric is zero:
\[ \nabla_\rho g^{\mu\nu} = 0 \]
\item As a consequence the covariant derivative commutes with the raising and lowering of indices, e.g.
\[ g_{\mu\lambda}\nabla_\rho V^\lambda = \nabla_\rho (g_{\mu\lambda}V^\lambda) = \nabla_\rho V_\mu \]
\item The covariant derivative of the Levi-Civita tensor is also zero:
\[ \nabla_\lambda \varepsilon_{\mu\nu\rho\sigma} = 0 \]
\end{itemize}
\end{eigenschap}

To see that the stated properties define a unique connection, we proceed as follows:
The covariant derivative of the metric (which is zero) can be written out as
\[ \nabla_\mu g_{\nu\rho} = \partial_\mu g_{\nu\rho} - \Gamma_{\mu\nu}^\sigma g_{\sigma\rho} - \Gamma_{\mu\rho}^\sigma g_{\nu\sigma} = 0 \]
because it has two lower indices. Then we consider the sum
\[ \nabla_\mu g_{\nu\rho} + \nabla_\nu g_{\mu\rho} - \nabla_\rho g_{\mu\nu} = 0 \]
Expanding this and using the fact that that the lower indices commute, gives
\begin{align}
0 = {} &\partial_\mu g_{\nu\rho} - \Gamma_{\mu\nu}^\sigma g_{\sigma\rho} - \cancel{\Gamma_{\mu\rho}^\sigma g_{\nu\sigma}} \\
+ &\partial_\nu g_{\mu\rho} - \Gamma_{\nu\mu}^\sigma g_{\sigma\rho} - \bcancel{\Gamma_{\nu\rho}^\sigma g_{\mu\sigma}} \\
- &\partial_\rho g_{\mu\nu} + \cancel{\Gamma_{\rho\mu}^\sigma g_{\sigma\nu}} + \bcancel{\Gamma_{\rho\nu}^\sigma g_{\mu\sigma}}
\end{align}
multiplying with $-1$, this becomes
\[ \partial_{\rho}g_{\mu\nu} - \partial_{\mu}g_{\nu\rho} - \partial_{\nu}g_{\rho\mu} + \Gamma_{\mu\nu}^\sigma g_{\sigma\rho} = 0 \]

This can be solved for the connection by multiplying by $g^{\lambda\rho}$:
\[ \boxed{\Gamma^\lambda_{\mu\nu} = \frac{1}{2}g^{\lambda\rho}\left(\partial_\mu g_{\nu\rho} + \partial_\nu g_{\rho\mu}-\partial_\rho g_{\mu\nu}\right)} \]

This connection is so important it is known by several different names: \udef{Christoffel connection}, \udef{Riemannian connection} and sometimes \udef{Levi-Civita connection}. The associated coefficients are known as the \udef{Christoffel symbols}.

When considering cartesian coordinates in flat space, the connection coefficients vanish. This is not the case for curvilinear coordinates.

Because it is possible to make the metric vanish in a single point even in curved space (as explained before), the Christoffel symbols can be made to vanish in a point by a change of coordinates. In a neighbourhood this is in general not possible.

\paragraph{Divergence and Stokes' theorem.} The covariant divergence of a vector $V^\mu$ is given by
\[ \nabla_\mu V^\mu = \partial_\mu V^\mu + \Gamma^\mu_{\mu\lambda}V^\lambda \]
This contraction of the Christoffel symbol can be calculated as follows:
\begin{align}
\Gamma^\mu_{\mu\lambda} &= \frac{1}{2}g^{\mu\rho}\left(\partial_\mu g_{\lambda\rho} + \partial_\lambda g_{\rho\mu}-\partial_\rho g_{\mu\lambda}\right) \\
&= \frac{1}{2}g^{\mu\rho}\partial_\lambda g_{\rho\mu} \\
&= \frac{1}{2}\Tr(g^{-1}\partial_\lambda g) \\
&= \frac{1}{2}\Tr(\partial_\lambda \ln g) \\
&= \frac{1}{2} \partial_\lambda \Tr(\ln g) \\
&= \frac{1}{2} \partial_\lambda \ln |g| \\
&= \partial_\lambda \ln |g|^{1/2} \\
&= \frac{1}{\sqrt{|g|}}\partial_\lambda \sqrt{|g|}
\end{align}
where the first and third terms of the first equality cancel because the metric is symmetric and we have used the matrix identity $\ln\det M=\Tr\ln M$. (TODO transfer to log requires positive definiteness?). Thus the covariant divergence can be written as
\[ \nabla_\mu V^\mu = \frac{1}{\sqrt{|g|}}\partial_\mu \left(\sqrt{|g|}V^\mu\right). \]

TODO stokes' theorem.

\paragraph{Relation to other derivatives.}
If $\nabla_\mu$ is a torsion free covariant derivative, $\omega_\mu$ is a one-form, and $X^\mu$ and $Y^\mu$ are vector fields, then
\[ (\diff{\omega})_{\mu\nu} = \partial_{[\mu}\omega_{\nu]} = \nabla_{[\mu}\omega_{\nu]} \]
and
\[ [X, Y]^\mu = X^\lambda\partial_\lambda Y^\mu - Y^\lambda\partial_\lambda X^\mu = X^\lambda\nabla_\lambda Y^\mu - Y^\lambda\nabla_\lambda X^\mu. \]


\subsection{Parallel transport}
In flat space the derivative along a curve is given by
\[\od{V^\mu}{\tau} = \od{x^\mu}{\tau} \partial_\mu V^\mu\]
For parallel transport this derivative should vanish. In curved space the partial derivative is replaced by a covariant one. This gives the \udef{coraviant directional derivative}
\[ \frac{D}{d\tau} \equiv \od{x^\mu}{\tau}\nabla_\mu. \]
For the parallel transport of a vector, it is applied to a vector and set to zero:
\[ \frac{DV^\mu}{d\tau} = \od{x^\nu}{\tau} \nabla_\nu V^\mu = \od{V^\mu}{\tau} + \Gamma^\mu_{\nu\rho} \od{x^\nu}{\tau} V^\rho = 0 \]
or
\[ \boxed{ \od{x^\nu}{\tau}\partial_\nu V^\mu + \Gamma^\mu_{\nu\rho} \od{x^\nu}{\tau} V^\rho = 0 } \]
This is the \textbf{equation of parallel transport}.

\begin{tikzpicture}
\draw (0,0) circle (1cm);
\foreach \rot in {36,72,...,360}{
\draw[->] ({sin(\rot)},{cos(\rot)}) -- +(.4,0);}
\end{tikzpicture}

\begin{tikzpicture}
\draw (0,0) -- (1,0) arc (0:300:1cm) -- cycle;
\end{tikzpicture}

If the connection is metric compatible, the \ueig{metric is parallel transported}. This means parallel transport preserves properties like norm, orthogonality etc.

\subsection{Geodesics}
Two way of seeing it:
\begin{enumerate}
\item Parallel transports it's own tangent vector
\item Shortest distance between two points.
\end{enumerate}

\subsubsection{As a curve that parallel transports it's tangent vector}
Setting directional covariant derivative of $\od{x^\mu}{\lambda}$ to zero gives
\[ \frac{D}{d\lambda}\od{x^\mu}{\lambda} = \boxed{ \od[2]{x^\mu}{\lambda} + \Gamma^\mu_{\rho\sigma}\od{x^\rho}{\lambda}\od{x^\sigma}{\lambda} = 0 } \]
This is the \textbf{geodesic equation}. In flat space using Cartesian coordinates, this reduces to the equation of straight lines.

Actually this procedure constrains the parametrisation of the curve. In general a curve may be thought of as a geodesic if it parallel transports the \emph{unit} tangent vector. The in formula above the norm of the tangent vector was also forced to be constant. This constrains the parametrisation to be an \udef{affine parameter}, which is a parameter of the form
\[ \lambda = a\tau + b \]
with $\tau$ the arc length (i.e. proper time) and $a$ and $b$ constants.

Now say $\alpha$ is an arbitrary parametrisation and $v(\alpha) = \left|\od{x^\mu}{\alpha}\right|$ the norm of the tangent vector $\od{x^\mu}{\alpha}$. The unit tangent vector is then $v^{-1}\od{x^\mu}{\alpha}$. Requiring that this be parallel transported gives
\begin{align}
\frac{D}{d\alpha}\left(v^{-1}\od{x^\mu}{\alpha}\right) &= \od{x^\rho}{\alpha}\nabla_\rho\left[v^{-1}\od{x^\mu}{\alpha}\right] \\
&= \od{x^\rho}{\alpha} \left[\partial_\rho \left(v^{-1}\od{x^\mu}{\alpha}\right) + \Gamma^\mu_{\rho\sigma}v^{-1}\od{x^\sigma}{\alpha}\right] \\
&= \od{x^\rho}{\alpha} \left[\od{x^\mu}{\alpha}\partial_\rho v^{-1} + v^{-1}\partial_\rho \od{x^\mu}{\alpha} + \Gamma^\mu_{\rho\sigma}v^{-1}\od{x^\sigma}{\alpha}\right] = 0.
\end{align}
Multiplying both sides by $v$ gives
\[ 0 = \od{x^\mu}{\alpha}v\od{x^\rho}{\alpha}\partial_\rho v^{-1} + \od[2]{x^\mu}{\alpha} + \Gamma^\mu_{\rho\sigma}\od{x^\rho}{\alpha}\od{x^\sigma}{\alpha} \]
which is the geodesic equation derived above plus an extra term of the form $f(\alpha)\od{x^\mu}{\alpha}$, with
\begin{align}
f(\alpha) &= v \od{v^{-1}}{\alpha} \\
&= -v^{-1}\od{v}{\alpha} \\
&= - \left(\od[2]{\tau}{\alpha}\right)\left(\od{\tau}{\alpha}\right)^{-1}
\end{align}
using $v = \od{\tau}{\alpha}$, because the proper time is the arc length (TODO rephrase?). The factor $f(\alpha)$ is obviously zero for affine parameters.

For timelike paths we can write the geodesic equation in terms of the four-velocity $u^\mu = \od{x^\mu}{\tau}$:
\[ u^\lambda\nabla_\lambda u^\mu = 0 \]

For massive particles the four-momentum $p^\mu$ is $mu^\mu$, making this equivalent to
\[ p^\lambda\nabla_\lambda p^\mu = 0 \]

\subsubsection{As the shortest distance between two points}
\paragraph{For timelike paths.} Consider the proper time functional for a timelike path parametrised by $\lambda$:
\[ \tau_{AB} = \int \left(-g_{\mu\nu} \od{x^\mu}{\lambda}\od{x^\nu}{\lambda}\right)^{1/2} \diff{\lambda} \]
We want to find the stationary paths, with $\delta \tau = 0$. Computing the variation and setting $f = g_{\mu\nu} \od{x^\mu}{\lambda}\od{x^\nu}{\lambda}$ gives
\begin{align}
\delta \tau &= \int \delta\sqrt{-f}\diff{\lambda} \\
&= -\int \frac{1}{2}(-f)^{-1/2}\delta f \diff{\lambda}.
\end{align}

Now we can reparametrise the path, taking the proper time as the new parameter. This means the tangent vector is the four-velocity $u^\mu$, which fixes $f$
\[ f = g_{\mu\nu} \od{x^\mu}{\tau}\od{x^\nu}{\tau} = g_{\mu\nu}u^\mu u^\nu = -1, \]
so
\[ \delta \tau = - \frac{1}{2}\int\delta f \diff{\tau} \]

This means stationary paths of the proper time functional are also stationary paths of the simpler integral
\[ I = \frac{1}{2}\int f \diff{\tau} = \frac{1}{2}\int g_{\mu\nu} \od{x^\mu}{\tau}\od{x^\nu}{\tau} \]
and vice versa.

We can explicitly vary this integral with (TODO why)
\[ \begin{cases}
x^\mu \to x^\mu _ \delta x^\mu \\
g_{\mu\nu} \to g_{\mu\nu} + (\partial_\sigma g_{\mu\nu})\delta x^\sigma.
\end{cases} \]
Plugging this into the expression for $I$ and keeping only terms that are first order in $\delta x^\mu$, we get
\[ \delta I = \frac{1}{2}\int \left[\partial_\sigma g_{\mu\nu}\od{x^\mu}{\tau}\od{x^\nu}{\tau}\delta x^\sigma + g_{\mu\nu}\od{(\delta x^\mu)}{\tau}\od{x^\nu}{\tau} +g_{\mu\nu}\od{x^\mu}{\tau}\od{(\delta x^\nu)}{\tau}\right]\diff{\tau} \]
The last two terms can be integrated by parts; for example,
\begin{align}
\int \left[ g_{\mu\nu}\od{x^\mu}{\tau}\od{(\delta x^\nu)}{\tau} \right]\diff{\tau} &= \left. g_{\mu\nu}\od{x^\mu}{\tau}\delta x^\nu \right|_\text{at boundary} - \int \od{}{\tau}\left(g_{\mu\nu}\od{x^\mu}{\tau}\right)\delta x^\nu \diff{\tau} \\
&= - \int \left[g_{\mu\nu} \od[2]{x^\mu}{\tau} + \od{g_{\mu\nu}}{\tau}\od{x^\mu}{\tau}\right]\delta x^\nu \diff{\tau} \\
&= - \int \left[g_{\mu\nu} \od[2]{x^\mu}{\tau} + \partial_\sigma g_{\mu\nu}\od{x^\sigma}{\tau}\od{x^\mu}{\tau}\right]\delta x^\nu \diff{\tau}
\end{align}
where we have used that $\delta x^\nu$ vanishes at the boundary.

The total variation is then
\[ \delta I = - \int \left[g_{\mu\sigma}\od[2]{x^\mu}{\tau} + \frac{1}{2}\left(\partial_\mu g_{\nu\sigma} + \partial_\nu g_{\sigma\mu} - \partial_\sigma g_{\mu\nu}\right)\od{x^\mu}{\tau}\od{x^\nu}{\tau}\right]\delta x^\sigma \diff{\tau}. \]
Since we are searching for stationary points, we want $\delta I$ to vanish for any variation $\delta x^\sigma$. This implies that the expression inside the square brackets must vanish. Multiplying it with the inverse metric $g^{\rho\sigma}$ yields
\[ \od[2]{x^\rho}{\tau} + \frac{1}{2}g^{\rho\sigma}\left(\partial_\mu g_{\nu\sigma} + \partial_\nu g_{\sigma\mu} - \partial_\sigma g_{\mu\nu}\right)\od{x^\mu}{\tau}\od{x^\nu}{\tau} = 0 \]
Which is exactly the geodesic equation with the Christoffel symbols as the connection.

This procedure provides a convenient way to calculate the Christoffel symbols for a given metric: by explicitly varying the integral $I$ with the metric of interest plugged in.

\paragraph{Null geodesics.} The geodesic formula was found using a very specific parametrisation, which is not a problem because all regular curves can be arc parametrised. Unfortunately we also restricted ourselves to timelike paths, because for null paths $\tau = 0$.

Now the geodesic equation derived above is still perfectly valid, even if $\tau$ can no longer be considered a valid parameter. An affine parameter is now any parameter such that the geodesic equation is satisfied, but now there is no special one. They are all related by the fact that if $\lambda$ is an affine parameter, any parameter of the form $a\lambda + b$ is as well. 

It is often convenient to normalise the affine parameter $\lambda$ along a null geodesic such that
\[ p^\mu = \od{x^\mu}{\lambda} \]

\paragraph{Timelike geodesics are maxima.} Locally that is. We can see that this is true because any timelike path can be arbitrarily well approximated by a null curve.

\subsubsection{Exponential map}
TODO

Geodesically incomplete

Riemann normal coordinates

\subsection{Riemann curvature tensor}
We now want an object that embodies our idea of curvature. Seeing as we are going to define a new object to fit out intuition, we need to flesh it out a bit first. We begin by naming some properties of flat spacetime:
\begin{itemize}
\item Parallel transport around a closed loop leaves vectors unchanged;
\item Covariant derivatives of tensors commute;
\item Initially parallel geodesics remain parallel.
\end{itemize}
TODO motivation.
\udef{Riemann tensor $\tensor{\mathcal{R}}{^\rho_{\sigma\mu\nu}}$} is a $(1,3)$-tensor. It is antisymmetric in the last two indices.

\begin{align}
[\nabla_\mu,\nabla_\nu]V^\rho &= \nabla_\mu\nabla_\nu V^\rho - \nabla_\nu\nabla_\mu V^\rho \\
&= \left(\partial_\mu(\nabla_\nu V^\rho) - \Gamma^\lambda_{\mu\nu}\nabla_\lambda V^\rho + \Gamma^\rho_{\mu\sigma}\nabla_\nu V^\rho \right) - \left(\partial_\nu(\nabla_\mu V^\rho) - \Gamma^\lambda_{\nu\mu}\nabla_\lambda V^\rho + \Gamma^\rho_{\nu\sigma}\nabla_\mu V^\rho \right) \\
&= 2\left(\partial_{[\mu}(\nabla_{\nu]} V^\rho) - \Gamma^\lambda_{[\mu\nu]}\nabla_\lambda V^\rho + \Gamma^\rho_{[\mu|\sigma|}\nabla_{\nu]} V^\rho \right) \\
&= 2\left(\partial_{[\mu}\partial_{\nu]} V^\rho + \partial_{[\mu}(\Gamma^\rho_{\nu]\sigma} V^\sigma) - \Gamma^\lambda_{[\mu\nu]}\nabla_\lambda V^\rho + \Gamma^\rho_{[\mu|\sigma|}\partial_{\nu]} V^\sigma + \Gamma^\rho_{[\mu|\sigma|}\Gamma^\rho_{\nu]\sigma} V^\sigma \right) \\
&= \cancel{[\partial_{\mu}, \partial_{\nu}] V^\rho} + 2\left(\partial_{[\mu}(\Gamma^\rho_{\nu]\sigma} V^\sigma) - \Gamma^\lambda_{[\mu\nu]}\nabla_\lambda V^\rho + \Gamma^\rho_{[\mu|\sigma|}\partial_{\nu]} V^\sigma + \Gamma^\rho_{[\mu|\sigma|}\Gamma^\rho_{\nu]\sigma} V^\sigma \right) \\
&= 2\left(\partial_{[\mu}(\Gamma^\rho_{\nu]\sigma}) V^\sigma + \cancel{\partial_{[\mu}( V^\sigma)\Gamma^\rho_{\nu]\sigma}} - \Gamma^\lambda_{[\mu\nu]}\nabla_\lambda V^\rho + \cancel{\Gamma^\rho_{[\mu|\sigma|}\partial_{\nu]} V^\sigma} + \Gamma^\rho_{[\mu|\sigma|}\Gamma^\rho_{\nu]\sigma} V^\sigma \right) \\
&= 2\left(\partial_{[\mu}\Gamma^\rho_{\nu]\sigma} + \Gamma^\rho_{[\mu|\sigma|}\Gamma^\rho_{\nu]\sigma} \right) V^\sigma - 2\Gamma^\lambda_{[\mu\nu]}\nabla_\lambda V^\rho \\
&= \left(\partial_{\mu}\Gamma^\rho_{\nu\sigma} - \partial_{\nu}\Gamma^\rho_{\mu\sigma} + \Gamma^\rho_{\mu\sigma}\Gamma^\rho_{\nu\sigma} - \Gamma^\rho_{\nu\sigma}\Gamma^\rho_{\mu\sigma} \right) V^\sigma - \tensor{T}{^\lambda_{\mu\nu}}\nabla_\lambda V^\rho \\
&\equiv \tensor{\mathcal{R}}{^\rho_{\mu\nu\sigma}}V^\sigma - \tensor{T}{^\lambda_{\mu\nu}}\nabla_\lambda V^\rho
\end{align}

Where $[\;,\;]$ is the antisymmetrisation, $[\mu|\sigma|\nu]$ means that only $\mu$ and $\nu$ are antisymmetrised and $T$ is the torsion tensor. So the Riemann tensor is identified as
\[ \boxed{\tensor{\mathcal{R}}{^\rho_{\mu\nu\sigma}} \equiv \partial_{\mu}\Gamma^\rho_{\nu\sigma} - \partial_{\nu}\Gamma^\rho_{\mu\sigma} + \Gamma^\rho_{\mu\sigma}\Gamma^\rho_{\nu\sigma} - \Gamma^\rho_{\nu\sigma}\Gamma^\rho_{\mu\sigma}} \]

\subsubsection{Properties of the curvature tensor}
To investigate the properties of the Riemann tensor, the upper index is lowered
\[ \mathcal{R}_{\rho\sigma\mu\nu} = g_{\rho\lambda}\tensor{\mathcal{R}}{^\lambda_{\sigma\mu\nu}} \]
and an explicit expression is obtained in locally inertial coordinates. In locally inertial coordinates the metric and its first derivatives vanish, so the Christoffel symbols do as well, but not their derivatives.
\begin{align}
\mathcal{R}_{\hat{\rho}\hat{\sigma}\hat{\mu}\hat{\nu}} &= g_{\hat{\rho}\hat{\lambda}} \left(\partial_{\hat{\mu}}\Gamma^{\hat{\lambda}}_{\hat{\nu}\hat{\sigma}} - \partial_{\hat{\nu}}\Gamma^{\hat{\lambda}}_{\hat{\mu}\hat{\sigma}}\right) \\
&= \frac{1}{2}g_{\hat{\rho}\hat{\lambda}}g^{\hat{\lambda}\hat{\tau}}\left[\left(\partial_{\hat{\mu}}\partial_{\hat{\nu}}g_{\hat{\sigma}\hat{\tau}} + \partial_{\hat{\mu}}\partial_{\hat{\sigma}}g_{\hat{\tau}\hat{\nu}} - \partial_{\hat{\mu}}\partial_{\hat{\tau}}g_{\hat{\nu}\hat{\sigma}}\right) - \left(\partial_{\hat{\nu}}\partial_{\hat{\mu}}g_{\hat{\sigma}\hat{\tau}} + \partial_{\hat{\nu}}\partial_{\hat{\sigma}}g_{\hat{\tau}\hat{\mu}} - \partial_{\hat{\nu}}\partial_{\hat{\tau}}g_{\hat{\mu}\hat{\sigma}}\right)\right] \\
&= \frac{1}{2}\left[\partial_{\hat{\mu}}\partial_{\hat{\sigma}}g_{\hat{\rho}\hat{\nu}} - \partial_{\hat{\mu}}\partial_{\hat{\rho}}g_{\hat{\nu}\hat{\sigma}} - \partial_{\hat{\nu}}\partial_{\hat{\sigma}}g_{\hat{\rho}\hat{\mu}} + \partial_{\hat{\nu}}\partial_{\hat{\rho}}g_{\hat{\mu}\hat{\sigma}}\right]
\end{align}
This derivation was done in a special coordinate system, but all tensorial equations that follow from it must be true in any coordinate system. A few such equations are now listed:
\begin{enumerate}
\item The Riemann tensor is antisymmetric in its first two indices.
\[ \boxed{ \mathcal{R}_{\rho\sigma\mu\nu} = - \mathcal{R}_{\sigma\rho\mu\nu} } \]
\item The Riemann tensor is antisymmetric in its last two indices.
\[ \boxed{ \mathcal{R}_{\rho\sigma\mu\nu} = - \mathcal{R}_{\rho\sigma\nu\mu} } \]
\item The Riemann tensor is invariant under exchange of the first and last pair of indices.
\[ \boxed{ \mathcal{R}_{\rho\sigma\mu\nu} = \mathcal{R}_{\mu\nu\rho\sigma} } \]
\item Thus sum of cyclic permutations of the last three indices vanishes.
\[ \mathcal{R}_{\rho\sigma\mu\nu} + \mathcal{R}_{\rho\mu\nu\sigma} + \mathcal{R}_{\rho\nu\sigma\mu} = 0 \]
This is equivalent to the vanishing of the antisymmetric part of the last three indices.
\[ \boxed{  \mathcal{R}_{\rho[\sigma\mu\nu]} = 0 } \]
\end{enumerate}
\paragraph{Number of parameters.} TODO
\paragraph{Bianchi identity.} TODO
\[ \nabla_{[\lambda}\mathcal{R}_{\rho\sigma]\mu\nu} = 0 \]

\subsubsection{Derived quantities}
Tricks for decomposition: taking contractions and taking (anti)symmetric parts.

\paragraph{Ricci tensor.} This \udef{Ricci tensor} is defined as
\[ \mathcal{R}_{\mu\nu} \equiv \tensor{\mathcal{R}}{^\lambda_{\mu\lambda\nu}}. \]
The Ricci tensor associated with the Christoffel connection is automatically symmetric:
\begin{align}
\mathcal{R}_{\mu\nu} &= g^{\rho\lambda}\mathcal{R}_{\rho\mu\lambda\nu} = g^{\rho\lambda}\mathcal{R}_{\lambda\nu\rho\mu} = \tensor{\mathcal{R}}{^\rho_{\nu\rho\mu}} \\
&= \mathcal{R}_{\nu\mu}
\end{align}
The trace of the Ricci tensor is called the \udef{Ricci scalar} (or curvature scalar):
\[ R \equiv \tensor{\mathcal{R}}{^\mu_\mu} = g^{\mu\nu}\mathcal{R}_{\mu\nu} \]

Now a useful form of the Bianchi identity can be obtained by multiplying it by $3g^{\nu\sigma}g^{\mu\lambda}$:
\begin{align}
0 &= 3g^{\nu\sigma}g^{\mu\lambda}\nabla_{[\lambda}\mathcal{R}_{\rho\sigma]\mu\nu} \\
&= \frac{3g^{\nu\sigma}g^{\mu\lambda}}{6!}\left[\left(\nabla_{\lambda}\mathcal{R}_{\rho\sigma\mu\nu} + \nabla_{\rho}\mathcal{R}_{\sigma\lambda\mu\nu} + \nabla_{\sigma}\mathcal{R}_{\lambda\rho\mu\nu}\right) - \left(\nabla_{\lambda}\mathcal{R}_{\sigma\rho\mu\nu} + \nabla_{\rho}\mathcal{R}_{\lambda\sigma\mu\nu} + \nabla_{\sigma}\mathcal{R}_{\rho\lambda\mu\nu}\right)\right] \\
&= g^{\nu\sigma}g^{\mu\lambda}\left[\nabla_{\lambda}\mathcal{R}_{\rho\sigma\mu\nu} + \nabla_{\rho}\mathcal{R}_{\sigma\lambda\mu\nu} + \nabla_{\sigma}\mathcal{R}_{\lambda\rho\mu\nu}\right] \\
&= \nabla^\mu\mathcal{R}_{\rho\mu} - \nabla_\rho R + \nabla^\nu\mathcal{R}_{\rho\nu}
\end{align}
or
\[ \boxed{\nabla^\mu\mathcal{R}_{\rho\mu} = \frac{1}{2}\nabla_\rho R.} \]


\paragraph{Weyl tensor.} TODO Why. In $n$ dimensions the \udef{Weyl tensor} is given by
\[ C_{\rho\sigma\mu\nu} \equiv \mathcal{R}_{\rho\sigma\mu\nu} - \frac{2}{n-2}\left(g{\rho[\mu}\mathcal{R}_{\nu]\sigma} - g{\sigma[\mu}\mathcal{R}_{\nu]\rho}\right) + \frac{2}{(n-1)(n-2)}g{\rho[\mu}\mathcal{R}_{\nu]\sigma} \]
TODO properties.

\paragraph{Einstein tensor.} The \udef{Einstein tensor} is defined as
\[ G_{\mu\nu} \equiv \mathcal{R}_{\mu\nu} - \frac{1}{2}Rg_{\mu\nu}. \]
Now the Bianchi identity reduces to
\[ \boxed{\nabla^{\mu}G_{\mu\nu} = 0} \]


\section{Isometries}
\subsection{About isometries}
TODO post geometry.
\[ g'_{\mu\nu}(y) = g_{\mu\nu}(y) \]

Symmetries of arbitrary tensor fields.
\subsection{Lie derivatives}
In general $g_{\mu\nu}(x_p)$ is different from $g_{\mu\nu}(x_q)$. Are there directions we can move in on the manifold so that the metric (or any other function on the manifold) doesn't change.
So we want
\[ g_{\mu\nu}(\tilde{x}) = \pd{x^\rho}{\tilde{x}{^\mu}}\pd{x^\sigma}{\tilde{x}{^\nu}}g_{\rho\sigma}(x) \]
to equal the original metric.

We consider an infinitessimal transformation:
\[ \tilde{x}^\mu = x^\mu+\epsilon V^\mu \]
\[ \delta y^\mu = \pd{y^\mu}{x^\nu}\delta x^\nu \]

\subsubsection{Lie derivative on a scalar}
We are comparing
\[  \begin{cases}
\phi(\tilde{x}) = \phi(x + \epsilon V) = \phi(x) + \epsilon V^\mu\partial_\mu \phi(x) + \mathcal{O}(\epsilon^2) \\
\tilde{\phi}(\tilde{x}) = \phi(x)
\end{cases}\]

\begin{align}
L_V\phi &\equiv \lim_{\epsilon\to 0}\frac{\phi(\tilde{x}) - \tilde{\phi}(\tilde{x})}{\epsilon} \\
&= \lim_{\epsilon\to 0}\frac{\phi(x+\epsilon V^\mu)-\phi(x)}{\epsilon} \\
&= \lim_{\epsilon\to 0}\frac{\cancel{\phi(x)}+\epsilon V^\mu\partial_\mu\phi(x)+\mathcal{O}(\epsilon^2)-\cancel{\phi(x)}}{\epsilon} \\
&= V^\mu\partial_\mu\phi(x)
\end{align}

Ordinary directional derivative.

In general:
\[ L_V: (p,q) \text{forms} \to (p,q) \text{forms} \]

\subsubsection{Lie derivative of a vector field}
Again we define
\[ L_VW^\mu \equiv \lim_{\epsilon\to 0}\frac{W^\mu(\tilde{x}) - \tilde{W^\mu}(\tilde{x})}{\epsilon} \]
Again we are comparing two quantities
\begin{align}
W(\tilde{x}) &= W(x + \epsilon V) = W(x) + \epsilon V^\mu\partial_\mu W(x) + \mathcal{O}(\epsilon^2) \\
\tilde{W}(\tilde{x}) &= \pd{\tilde{x}^\mu}{x^\nu}W^\nu (x) \\
&= \pd{(x^\mu+\epsilon V^\mu)}{x^\nu}W^\nu (x) \\
&= \left(\delta^\mu_\nu+\epsilon \pd{V^\mu}{x^\nu}\right)W^\nu (x) \\
&= W^\mu (x)+\epsilon W^\nu (x)\partial_\nu V^\mu
\end{align}

Putting everything together, we get
\[ L_V W^\mu = V^\nu\partial_\nu W^\mu - W^\nu \partial_\nu V^\mu \]

Normal derivative not covariant, but here same as covariant derivative (extra bits cancel).

Not defined with respect to any particular metric.

Some properties:
\begin{itemize}
\item Partial derivatives can be replaced by covariant ones.
\item The Lie derivative is antisymmetric in $V$ and $W$ and defines a commutator
\[ [V,W]^\mu \equiv L_VW^\mu = - L_WV^\mu \]
This satisfies the Jacobi identity 
\[ [V,[W,X]]^\mu + [X,[V,W]]^\mu + [W,[X,V]]^\mu \]
and thus is a Lie bracket.  The Jacobi identity is equivalent to
\[ L_V[W,X]^\mu = [L_VW,X]^\mu + [W,L_VX]^\mu \]
\end{itemize}


\subsubsection{Lie derivative of other tensor fields}
The definitions above are readily generalised
\[ \tilde{x}^\mu(x) = x^\mu +\epsilon V^\mu(x) \]
\[ L_VT = \lim_{\epsilon\to 0} \frac{T(\tilde{x})-\tilde{T}(\tilde{x})}{\epsilon} \]
where we need
\[ \pd{\tilde{x}^\mu}{x^\rho} = \delta^\mu_\rho + \epsilon \partial_\rho V^\mu + \mathcal{O}(\epsilon^2) \qquad \text{and}\qquad \pd{x^\mu}{\tilde{x}{^\rho}} = \delta^\mu_\rho - \epsilon \partial_\rho V^\mu + \mathcal{O}(\epsilon^2) \]

Again partial derivatives can be replaced by covariant derivatives. We illustrate with a $(0,2)$-tensor $T_{\mu\nu}$.
\[ \begin{cases}
T_{\mu\nu}(\tilde{x}) = T_{\mu\nu}(x) + \epsilon V^\rho\partial_\rho T_{\mu\nu}+ \mathcal{O}(\epsilon^2) \\
\tilde{T}_{\mu\nu}(\tilde{x}) = \pd{x^\mu}{\tilde{x}{^\mu}}\pd{x^\nu}{\tilde{x}{^\nu}}T_{\mu\nu} = T_{\mu\nu} - \epsilon\partial_\mu V^\rho T_{\rho\nu}(x) - \epsilon\partial_\nu V^\sigma T_{\mu\sigma} + \mathcal{O}(\epsilon^2)
\end{cases} \]
Filling this in gives
\begin{align}
L_V T_{\mu\nu} &= \lim_{\epsilon\to 0} \frac{T_{\mu\nu}(\tilde{x})-\tilde{T}_{\mu\nu}(\tilde{x})}{\epsilon} \\
&= \lim_{\epsilon\to 0} \frac{\cancel{T_{\mu\nu}(x)}+\epsilon V^\rho\partial_\rho T_{\mu\nu}+ \mathcal{O}(\epsilon^2)-\cancel{T_{\mu\nu}}+\epsilon T_{\rho\nu} \partial_\mu V^\rho +\epsilon T_{\mu\sigma} \partial_\nu V^\sigma + \mathcal{O}(\epsilon^2)}{\epsilon} \\
&= V^\rho\partial_\rho T_{\mu\nu} + T_{\rho\nu} \partial_\mu V^\rho + T_{\mu\rho} \partial_\nu V^\rho \\
&= V^\rho \left(\nabla_\rho T + \Gamma^\lambda_{\rho \mu}T_{\lambda \nu} + \Gamma^\lambda_{\rho \nu}T_{\mu \lambda} \right) + T_{\rho\nu} \left(\nabla_\mu V^\rho - \Gamma^\rho_{\mu\lambda}V^\lambda\right) + T_{\mu\rho} \left(\nabla_\nu V^\rho - \Gamma^\rho_{\nu\lambda}V^\lambda\right) \\
&= V^\rho\nabla_\rho T + T_{\rho\nu}\nabla_\mu V^\rho + T_{\mu\rho}\nabla_\nu V^\rho + \left(\Gamma^\lambda_{\rho \mu}V^\rho T_{\lambda \nu} - \Gamma^\rho_{\mu\lambda}V^\lambda T_{\rho\nu}\right) + \left(\Gamma^\lambda_{\rho \nu}V^\rho T_{\mu \lambda} - \Gamma^\rho_{\nu\lambda}V^\lambda T_{\mu\rho}\right) \\
&= V^\rho\nabla_\rho T + T_{\rho\nu}\nabla_\mu V^\rho + T_{\mu\rho}\nabla_\nu V^\rho
\end{align}

Isometries from an algebra
\[ \left[L_V,L_W\right] = L_{[V,W]} \]
Enough to verify scalars and vectors.

\subsubsection{Lie derivative of tensor densities}
TODO

\subsection{Killing vectors}
The Lie derivative of the metric tensor. This is $(0,2)$-tensor, so the formula is the one given above.
Due to metric compatibility, the first term is zero
\begin{align}
L_V g_{\mu\nu} &= V^\rho\nabla_\rho g_{\mu\nu} + g_{\lambda\nu}\nabla_\mu V^\lambda + g_{\mu\lambda}\nabla_\nu V^\lambda \\
&= g_{\lambda\nu}\nabla_\mu V^\lambda + g_{\mu\lambda}\nabla_\nu V^\lambda \\
&= \nabla_\mu V_\nu + \nabla_\nu V_\nu \\
\end{align}

An infinitesimal coordinate transformation is a symmetry of the metric if $L_V g_{\mu\nu} = 0$, which is equivalent to requiring $V$ to satisfy the equations
\[ \nabla_\mu V_\nu + \nabla_\nu V_\nu = 0 = \nabla_{(\mu} V_{\nu)}. \]
Such vectors are called \udef{Killing vectors}. These equations are equivalent to
\[ \nabla_\mu V_\nu = \nabla_{[\mu}V_{\nu]} \]


Properties:
\begin{enumerate}
\item Killing vectors form a Lie algebra. If $V$ and $W$ are Killing vectors, i.e. $L_Vg_{\mu\nu} = L_Wg_{\mu\nu} = 0$, then $[V,W]$ is a Killing vector because
\[ L_{[V,W]}g_{\mu\nu} = L_VL_Wg_{\mu\nu} - L_WL_V g_{\mu\nu} = 0  \]
\item If all the components of the metric are independent of a particular coordinate, say $y$
\[ \partial_y g_{\mu\nu} \qquad \forall \mu,\nu \]
Then $V=\partial_y$ is a Killing vector. A coordinate system in which a Killing vector is a partial derivative is said to be \textit{adapted} to the Killing vector (or isometry) in question. TODO derive Killing equations from this.
\item Two Killing vectors commute if and only if there is a coordinate system that is adapted to both of them.
\end{enumerate}

You can use the equations in 2 ways:
\begin{itemize}
\item Impose symmetries on the metric.
\item Find the Killing vectors for a given metric, which gives the symmetries
\end{itemize}

\begin{example}
Algebra of Killing vectors in Minkowski and two-sphere.
\end{example}

\subsection{Conserved quantities}
\subsubsection{Conserved charges along geodesics}
Let $K^\mu$ be a Killing vector field and $x^\mu(\tau)$ a geodesic with four-velocity $x^\mu$. Then the quantity
\[ Q_K = K_\mu u^\mu \]
is constant along the geodesic. Indeed,
\begin{align}
\od{}{\tau}Q_K = \od{K_\mu u^\mu}{\tau} &= u^\mu \od{K_\mu}{\tau} + K_\mu \od{}{\tau}u^\mu \\
&= u^\mu u^\nu \nabla_\nu K_\mu + 0\\
&= \frac{1}{2}\left(\nabla_\nu K_\mu + \nabla_\mu K_\nu\right)u^\mu u^\nu = 0
\end{align}
where the last equality is due to the Killing equations.

\subsubsection{Conserved currents from the energy-momentum tensor}
Let $K^\mu$ be a Killing vector field and $T^{\mu\nu}$ the covariantly conserved symmetric energy-momentum tensor ($\nabla_\mu T^{\mu\nu} = 0$). Then the current
\[ J_K^\mu = T^{\mu\nu}K_\nu \]
is covariantly conserved. Indeed,
\begin{align}
\nabla_\mu J^\mu_K &= (\nabla_\mu T^{\mu\nu})K_\nu + T^{\mu\nu} \nabla_\mu K_\nu \\
&= 0 + \frac{1}{2}T^{\mu\nu}\left(\nabla_\mu K_\nu + \nabla_\nu K_\mu\right) = 0
\end{align}

\subsubsection{Komar currents}
The Einstein tensor is symmetric and conserved (from the Bianchi identity), so we have the conserved current
\[ J^\mu_1 = \tensor{G}{^\mu_\nu}K^\nu =  \]

\subsection{Killing tensors}

\subsubsection{Killing(-Stäckel) tensors}
A \udef{Killing tensor $K_{\beta_1\ldots\beta_n}$} is a totally symmetric tensor satisfying
\[ \nabla_{(\alpha}K_{\beta_1\ldots\beta_n)} = 0. \]

The charge
\[ Q_K = K_{\beta_1\ldots\beta_n}u^{\beta_1}\ldots u^{\beta_n} \]
is constant along the geodesic.

\subsubsection{Killing-Yano tensors}
A \udef{Killing-Yano tensor $Y_{\beta_1\ldots\beta_n}$} is a totally anti-symmetric tensor satisfying
\[ \nabla_{(\alpha} Y_{\beta_1)\ldots\beta_n} = 0 \qquad \text{or, equivalenty} \qquad \nabla_\alpha Y_{\beta_1\ldots\beta_n} = \nabla_{[\alpha}Y_{\beta_1\ldots\beta_n]} \]

The tensorial charges
\[ Z_{\beta_1\ldots\beta_{n-1}} = u^\beta Y_{\beta\beta_1\ldots\beta_{n-1}} \]
are conserved along geodesics.


\subsubsection{Symmetries and conserved charges (Komar integrals)}
\subsubsection{Conservation laws}
Conservation laws
\begin{itemize}
\item for geodesics
\item for spacetime
\end{itemize}

\[ Q = V^\mu \od{x^\nu}{\tau}g_{\mu\nu} \]
$\od{}{\tau}Q$ if $V$ is Killing and $\dot{x}^\mu$ is a geodesic.
\begin{align}
\od{}{\tau}Q = \od{}{\tau}\left(V_\mu\dot{x}^\mu\right) &= \left(\frac{DV_\mu}{D\tau}\right)\dot{x}^\mu + V_\mu \frac{D\dot{x}^\mu}{\tau} \\
&= \underbrace{\dot{x}^\rho D_\rho V_\mu \dot{x}^\mu}_{=0 \text{because Killing}} + \underbrace{V_\mu \frac{D\dot{x}^\mu}{D\tau}}_{=0 \text{geodesic}}
\end{align}

\subsection{Maximally symmetric spaces}
In D=4, maximally 10 symmetries. Minkowski maximally symmetric, but not uniquely so.
(Depends on number of killing vectors (which also form an algebra and can commute or not))

\[ K^\mu(x) \text{Killing} \quad \Leftrightarrow \quad D_{[\mu}K_{\nu]} = 0 \]
\[ K_\mu(x) = K_\mu(x^*) + \partial_\nu K_\mu(x^*)(x^\nu-x^{\nu*}) + \frac{1}{2}\partial_\rho\partial_\nu K_\mu(x^*)(x^\nu-x^{\nu *}(x^\rho - x^{\rho*}) + \ldots \]

\[ D_\mu K_\nu = \partial_\mu K_\nu - \Gamma^\rho_{\mu\nu}K_\rho \]
\[ \partial_\nu K_\mu(x^*) = D_\nu K_\mu(x^*) + \Gamma_{\nu\mu}^{\;\;\rho}K_\rho(x^*) \]
With
\[ D_\nu K_\mu(x^*) = \underbrace{\cancel{D_{(\nu}K_{\mu)}}}_{0 \text{because Killing}} + D_{[\nu}K_{\mu]}(x^*) \]
So
\[ \frac{D(D-1)}{2} \qquad \text{for} D=4 \quad \Rightarrow \quad 6 \text{coeff.} \]

Maximally symmetric:
\begin{itemize}
\item Minkowski: 4 translations, 6 Lorentz ($\Lambda = 0$)
\[ [P_\mu, P_\nu] = 0, \qquad [M_{\mu\nu}, P_\rho] = 2 \eta_{\rho[\mu}P_{\nu]}, \qquad [M_{\mu\nu}, M^{\rho\sigma}] = 4 \delta_{[\mu}^{\;\;[\rho}M_{\nu]}^{\;\;\sigma]} \]
\[M^{\mu\nu} = x^\mu\partial^\nu - x^\nu\partial^\mu \qquad G=\R^4 \rtimes \SO(1,3)\]
\item de Sitter $G = \SO(1,4)$ ($\Lambda>0$)
\item Anti-de Sitter $G=\SO(2,3)$
\end{itemize}
On a side note
\[ dS_4 \equiv \frac{\SO(1,4)}{\SO(1,3)} \qquad AdS_4 \equiv \frac{\SO(2,3)}{\SO(1,3)} \]
Confer:
\[ S^p = \frac{\SO(p+1)}{\SO(p)} \]

riemann normal coordinates + freely falling frames

\section{Conformal transformations}
\subsection{Conformal Killing vectors}
\[ L_C g_{\mu\nu} = \nabla_\mu C_\nu + \nabla_\nu C_\mu = 2\omega(x)g_{\mu\nu} \]

A Killing vector for a metric is at least a conformal Killing vector for any conformally rescaled metric.

\subsubsection{Conserved charges along null geodesics}
\[ Q_C = C_\mu u^\mu \]
\subsubsection{Conserved currents from the energy-momentum tensor}
energy-momentum tensor is traceless
\[ J_C^\mu = T^{\mu\nu}C_\nu \]

\subsection{Conformal Killing(-Yano) tensors}


\chapter{Lie groups and algebras}

\section{Lie Group}
A Lie group is a topological group that is also a differential manifold. This means we can apply differentials, which is of course very important. So important in fact that Sophus Lie called Lie groups infinitesimal groups when he first introduced them. Not only that, but it means we can consider tangent spaces, which will also be important later. TODO better justification

Bearing in mind the link between the topology and group properties explored in the section on topological groups, we quite naturally arrive at the following definition:
\begin{definition}
A \udef{Lie group} is a smooth manifold $G$ which is also a group and such that both the group product $G\times G \to G$ and the inverse map $G \to G$ are smooth.
\end{definition}

There is a particular type of Lie group that will be of particular importance to us, namely the matrix Lie group. In fact we will almost exclusively consider matrix Lie groups.

\subsection{Matrix Lie group}
For matrix groups there is a simpler condition to see whether it is a Lie group or not:
\begin{eigenschap}
All \ueig{closed subgroups} of $\GL(n, \C)$ are matrix Lie groups.
\end{eigenschap}
The condition that it be closed means that for every sequence in the Lie group the limit needs to be in the Lie group as well, if there is one. (Or you can say every Cauchy sequence in the Lie group has to have a limit in the Lie group). This is a technicality and is satisfied for most of the interesting subgroups of $\GL(n, \C)$. 

We have already seen that all subgroups of $\GL(n, \C)$ are topological groups. To prove the assertion then we must only verify that it is a smooth manifold. Because $\C^{n\times n}$ is a manifold and a matrix Lie group is a subset of $\C^{n\times n}$, the matrix Lie group inherits Hausdorffness and second-countability from $\C^{n\times n}$. To show it is smooth and locally homeomorphic to $\R^{m}$ in every point, we will explicitly construct such homeomorphisms using the matrix exponential.

\subsubsection{Exponential maps}
The homeomorphisms will be constructed based on the exponential map.
\[ \exp: \GL(n,\C) \to \GL(n,\C): X \mapsto e^X \]
This map is not a bijection, however if we restrict it to a neighbourhood of $\mathbb{0}$, it is locally a bijection. In fact it maps that neighbourhood to a neighbourhood of $\mathbb{1}$. More formally
\begin{eigenschap}
There exists a neighbourhood $U$ of $\mathbb{0}$ and a neighbourhood $V$ of $\mathbb{1}$ such that the exponential mapping takes $U$ homeomorphically onto $V$.
\end{eigenschap}
This result should not be surprising. For $X$ close to $\mathbb{0}$ we have the approximation $e^{X} \approx \mathbb{1} + X + \mathcal{O}(X^2)$. So for matrices in a small neighbourhood $U$ around $\mathbb{0}$ the exponential mapping can be seen as approximately linear, which is injective. In order to get surjectivity, we restrict the codomain of the mapping to the image of $U$ under the exponential mapping. This is a neighbourhood of $\mathbb{1}$ because $e^0 = \mathbb{1}$. 

We have obtained a bijection and because the matrix exponential is continuous, this restriction of it is also continuous. We would now like to show that the map maps open sets in our matrix Lie group, which we shall now call $G$, to open sets of $\R^{m}$. Unfortunately it doesn't. There is no reason why $\mathbb{0}$ or any matrices in $U$ should be elements of $G$. (Remember that the relevant group operation for matrix groups is the matrix multiplication, for which the neutral element is $\mathbb{1}$; the matrix $\mathbb{0}$ is of no particular importance in this context.) Being a group, the matrix Lie group must contain $\mathbb{1}$; being a topological group, it must contain a neighbourhood of $\mathbb{1}$; being a subspace of $\GL(n,\C)$ endowed with the subspace topology, the intersection of $V$ with that neighbourhood is an open set in $G$ which we will call $V'$.

So if we invert the restricted matrix exponential, we get a homeomorphism from \undline{one} neighbourhood of $G$ to $\GL(n,\C)$, which can then be composed with a homeomorphism to $\R^m$.

\begin{definition}
The inverse map $\exp^{-1}: V' \to U$ is called the \udef{logarithm}.
\end{definition}

From this we can construct a homeomorphism from a neighbourhood of any element $A$ of $G$. By multiplying each element of $V'$ with $A$ we get a neighbourhood $V_A$ of $A$. We define the following homeomorphism on $V_A$: multiply by $A^{-1}$ (this is bijective due to associativity of the group operation and continuous due to the definition of topological groups) and then send through the inverted, restricted matrix exponential. This composition of homeomorphisms is a homeomorphism. So for each element $A \in G$ we can find a neighbourhood $V_A$ that is homeomorphic to $\R^m$ thanks to this homeomorphism.

\begin{example}
TODO Finite Lie group
\end{example}


\subsubsection{Lie algebra of a matrix Lie group}
TODO: justification

\begin{definition}
Let $G$ be a matrix Lie group. The \udef{Lie algebra} of $G$, denoted $\mathfrak{g}$, is the set of all matrices $X_t$ such that $e^{itX_t}$ is in $G$ for all \undline{\textbf{real}} numbers $t$. We call the matrices $X_t$ \udef{generators} of the group.
\end{definition}

\begin{note}
Now here we have a complication. There are actually two conventions. The definition above is the convention most often used in physics. In the mathematics literature the Lie algebra in usually defined using $e^{tX_t}$, not $e^{itX_t}$. The physics convention gives rise to Hermitian generators in the algebras of $\U(n)$ and $\SU(n)$. This is useful because we are often interested in turning them into quantum operators, which correspond to observables only if they are Hermitian. The downside of this convention however is that it makes our life much more difficult in other places, and it even means that some definitions don't make any sense. In what follows we will generally be using the physics convention. We will however make use of the mathematics convention when the need arises. Also if there are interesting differences in the mathematics definition, we will mention those as well.
\end{note}

To try to grasp why the definition given above is useful, we introduce the notion of parametrization of group elements.
\subsubsection{Parametrization of group elements.}
When first introducing the matrix groups, we pointed out how the elements could be written in function of real parameters. We now make this notion more concrete and begin by defining a one-parameter subgroup.
\begin{definition}
A function $A : \R \to \GL(n, \C)$ is called a \udef{one-parameter subgroup} of $\GL(n, \C)$ if
\begin{enumerate}
\item $A$ is continuous,
\item $A(0) = \mathbb{1}_n$,
\item $A(t+s) = A(t)A(s)$ for all $t,s \in \R$.
\end{enumerate}
\end{definition}
If $A$ is a one-parameter subgroup of $\GL(n,\C)$, then it has the following property:
\begin{eigenschap}
There exists a unique $n\times n$ complex matrix $X$ such that
\[ A(t) = e^{tX} \]
\end{eigenschap}

So $X_t$ is in $\mathfrak{g}$ if and only if the one-parameter subgroup generated by $X_t$ lies in G. Conversely for any one-parameter subgroup that is a subgroup of G, there exists a generator and that generator is by definition part of the algebra.

Before continuing we shall consider some examples of algebras of matrix Lie groups. In general we shall call the algebra of a Lie group the lowercase version of the name of the Lie group. e.g\, the Lie algebra of $\GL(n, \C)$ is $\glAlg(n,\C)$.

\begin{example}
\begin{enumerate}
\item If $X$ is any $n\times n$ complex matrix, then $e^{itX}$ is invertible. Thus the Lie algebra, $\glAlg(n,\C)$, of the invertible matrices, $\GL(n,\C)$, is the space of all complex $n\times n$ matrices.
\item If we use the mathematical convention, then the Lie algebra of $\GL(n,\R)$ is the space of all real $n\times n$ matrices, denoted $\glAlg(n,\R)$. To prove this we first remark that is $X$ is any real $n\times n$ matrix, then $e^{tX}$ will be invertible and real. Conversely, if $e^{tX}$ is real for all real $t$, then $X=\left.\od{}{t}e^{tX}\right|_{t=0}$ will also be real. Obviously in the physics convention the above no longer holds true.

\item The Lie algebra $\slAlg(n,\C)$ of $\SL(n,\C)$ is the space of all complex $n \times n$ matrices with zero trace. To prove this we use that
\[ \det(e^X) = e^{\Tr(X)}. \]
If $\Tr(X) = 0$, then $\det(e^{itX}) = 1$ for all real numbers $t$. On the other hand, if $X$ is any $n\times n$ matrix such that $\det(e^{itX}) =1$ for all $t$, then $e^{it\Tr(X)} = 1$ for all $t$. This means that $it\Tr(X)$ is an integer multiple of $2\pi i$ for all $t$, which is only possible if $\Tr(X) = 0$.

\item Lie algebra of $\U(N)$. If $X$ is to be a generator in our algebra, we need $e^{itX}$ to be unitary. So
\[ \left(e^{itX}\right)^\dagger = \left(e^{itX}\right)^{-1} = e^{-itX}. \]
We also have that
\[ \left(e^{itX}\right)^\dagger = e^{-itX^\dagger}. \]
Which gives us
\[ e^{-itX} = e^{-itX^\dagger}. \]
Differentiating at $t=0$ we see that the generators have to be Hermitian ($X = X^\dagger$).

We can also prove this by writing out the definition of the matrix exponential. 
\[ U(N) \ni U = e^{it_iX_i} \]
\begin{align}
\mathbb{1} = U^\dagger U &= (\mathbb{1}-it_i X_i^\dagger + \ldots )(\mathbb{1}+it_i X_i + \ldots) \\
&= \mathbb{1} + it_i(X_i^\dagger - X_i) + \ldots = \mathbb{1}
\end{align}
So we require the generators to be Hermitian matrices ($X_i^\dagger = X_i$). We have $N^2$ independent $X_i$ that are Hermitian. 
\[ \uAlg(N) = \{ H \in \GL(N,\C), H^\dagger = H \} \]
In the mathematics convention this condition becomes that the generators have to be skew-Hermitian, i.e.\ $X_i^\dagger = -X_i$.

\item Lie algebra of $\SU(N)$. Combining the arguments for the algebras of the unitary and special linear group, we see that the generators must be unitary and of trace zero. In other words the algebra is given by
\[ \suAlg(N) = \{ H\in\uAlg(N), \Tr[H] = 0 \} \]
and has dimension $N^2-1$.

For $N=2$ we have:
\[ \begin{cases}
\suAlg(2) = \{\sigma_1, \sigma_2, \sigma_3\} \\
\uAlg(2) = \{\sigma_1, \sigma_2, \sigma_3, \mathbb{1}\}
\end{cases} \]
Where
\[ \sigma_1 = \begin{pmatrix}
0 & 1 \\ 1 & 0
\end{pmatrix}, \qquad \sigma_2 = \begin{pmatrix}
0 & -i \\ i & 0
\end{pmatrix}, \qquad \sigma_3 = \begin{pmatrix}
1 & 0 \\ 0 & -1
\end{pmatrix}\]

\item Lie algebra of $\Ogroup(N)$. As explained above, if we want the algebra to be real, we need to make use of the mathematical convention. So $O=e^{t_iX_i}$.
\begin{align}
\mathbb{1} = O^\intercal O = e^{t_iX_i^\intercal}e^{t_iX_i} &= (\mathbb{1}+t_i X_i^\intercal + \ldots )(\mathbb{1}+t_i X_i + \ldots) \\
&= \mathbb{1} + t_i(X^\intercal_i + X_i) + \ldots
\end{align}
So we require the generators to be antisymmetric matrices ($X_i^\intercal = -X_i$).
\[ \oAlg(N) = \{ X \in \GL(N,\R), X^\intercal = -X \} = \soAlg(N) \]
The dimension of $\oAlg(N)$ is $\frac{N(N-1)}{2}$.
\end{enumerate}
\end{example}

The Lie algebra as defined above is in some way prototypical. i.e.\ when we make this notion more abstract, we want the abstract notion to behave in a similar fashion and have many of the same properties. Of course to do that we first need an idea of what properties these Lie algebras actually have. This is what we will be exploring next.

\begin{eigenschap}
If $G$ is a \textit{connected} matrix Lie group, then every $A \in G$ can be written in the form
\[ A = e^{X_1}e^{X_2}\ldots e^{X_m} \]
for some $X_1, X_2, \ldots, X_m$ in $\mathfrak{g}$
\end{eigenschap}

\begin{eigenschap}
Every continuous homomorphism between two matrix Lie groups is smooth.
\end{eigenschap}

\begin{eigenschap}
A matrix $X$ is in $\mathfrak{g}$ if and only if there exists a smooth curve $\gamma$ in $\C^{n\times n}$ such that
\begin{enumerate}
\item $\gamma(t)$ lies in $G$ for all $t$;
\item $\gamma(0) = \mathbb{1}$;
\item $\left.\od{\gamma}{t}\right|_{t=0} = X$
\end{enumerate}
Thus $\mathfrak{g}$ is the tangent space at the identity to $G$.
\end{eigenschap}


\begin{itemize}
\item If we assume $X \in \mathfrak{g}$, we can take $\gamma(t) = \exp(tX)$. This $\gamma(t)$ satisfies the points of the proposition above.
\item We now assume $\gamma(t)$ is a smooth curve in $G$ with $\gamma(0) = \mathbb{1}$.
\begin{align}\od{\gamma(t)}{t} &= \lim_{\delta t \to 0} \frac{\gamma(t+\delta t)-\gamma(t)}{\delta t} = \gamma(t)\left(\lim_{\delta t \to 0}\frac{\gamma(\delta t)-\gamma(0)}{\delta t}\right) \\ &= \gamma(t)\left.\od{\gamma}{t}\right|_{t=0} = \gamma(t)X \end{align}
From which we get that
\[ \gamma(t) = \exp{tX} \]
\end{itemize}

Now this is interesting, so interesting in fact that we use this last proposition to construct a general definition of a Lie algebra associated to a Lie group.

We can now also reintroduce the physics convention. We just divide all elements of any algebra by the imaginary unit $i$. The elements of this algebra may not be closed under the bracket operation, but that does not matter as we have a different definition to work from now: they are elements of the tangent space at identity to $G$, rescaled with a fractor $-i$.

We have already seen that in the mathematics convention the commutator belongs to the algebra (remembering to sum according to Einstein notation):
\[ [X_i, X_j] = f_{ij}^k X_k \]
Where we call $f_{ij}^k$ a \udef{structure constant} (with respect to the chosen basis of course). In the physics convention, we obviously need to deal with the factor $i$:
\[ [X_i, X_j] = if_{ij}^k X_k \]
\begin{eigenschap}
From the antisymmetry of the bracket we get:
\[ f^k_{ij} + f^k_{ji} = 0 \]
From the Jacobi identity we get:
\[ f^m_{ie}f^e_{jk} + f^m_{je}f^e_{ki} + f^m_{ke}f^e_{ij} = 0 \]
\end{eigenschap}

And lastly a final property of Lie algebras of matrix Lie groups follows straight from the \ueig{Baker-Campbell-Hausdorff formula}: 
\[ e^{A}e^{B} = e^C \qquad \text{with} \qquad C=A+B+\frac{1}{2}[A,B] + \frac{1}{12}([A,[A,B]]+[B,[B,A]]) + \ldots  \]
\begin{eigenschap}
To know the (local) structure of a Lie group close to the identity one \ueig{only} needs to know the commutator of the generators $[X_i,X_j]$
\end{eigenschap}

\section{Lie Algebra}
Again we will start by restricting our attention to Lie algebra's of matrix Lie groups. That way we can give some examples that will (hopefully) aid in the understanding of the general case.


\subsection{Definition}

\begin{eigenschap}
Let $G$ be a matrix Lie group, with Lie algebra $\mathfrak(g)$. Let $X$ and $Y$ be elements of $\mathfrak{g}$. Then
\begin{enumerate}
\item $sX \in \mathfrak{g}$, for all \undline{real} numbers $s$,
\item $X+Y \in \mathfrak{g}$,
\item $-i(XY - YX)\in \mathfrak{g}$.
\end{enumerate}
\end{eigenschap}
The first two points mean that the Lie algebra is actually a vector space over the \undline{real} numbers. This is important and serves as the crux of our first generalisation of Lie algebras, so we will have a quick look at the proofs of the statements above. 

\begin{enumerate}
\item This first point is fairly straightforward, since $e^{t(sX)} = e^{(ts)X}$, which must be in $G$ if $X$ is in $\mathfrak{g}$.
\item If $X$ and $Y$ commute, this is again immediate. If they don't however we need to do a little more work. We start from the Lie product formula:
\[ e^{t(X+Y)} = \lim_{m\to\infty} \left(e^{\frac{tX}{m}}e^{\frac{tY}{m}}\right)^m \]
Clearly if $X, Y \in \mathfrak{g}$,  for every $m$, $e^{\frac{tX}{m}}$ and $e^{\frac{tY}{m}}$ are elements of $G$. Since $G$ is a group, $\left(e^{\frac{tX}{m}}e^{\frac{tY}{m}}\right)^m$ is in $G$. Now because $G$ is a matrix Lie group, and thus \textit{closed} in $\GL(n, \C)$, the limit must also be in $G$. (If that is the limit is in $\GL(n, \C)$, which it is because $e^{t(X+Y)}$ is invertible). This shows that $X+Y$ is in $\mathfrak{g}$.
\item The third point follows from the product rule of the differential operator. Alternatively we can use the Baker-Campbell-Hausdorff formula:
\[e^{tX}e^{sY} = e^{tX+sY+ \frac{ts}{2}[X,Y] + \ldots}\]
This together with the first two points shows the third point.
\end{enumerate}

We have shown that the Lie algebra is a vector space over the real numbers, but crucially a Lie algebra is in general not a vector space over complex numbers, even if it consists of matrices with complex entries. For an example we consider the algebra $\suAlg(n)$, which consists of Hermitian matrices with zero trace. Assume $X$ is such a matrix. Now because $(iX)^\dagger = -iX^\dagger = -iX$, $iX$ is not Hermitian. As a consequence it cannot be an element of $\suAlg(n)$ and thus $\suAlg(n)$ is not a complex vector space.

If we follow the mathematical definition, the third point becomes $XY - YX \in \mathfrak{g}$. This will be important later. In fact it's so important we will give it name.
\begin{definition}
Given two $n \times n$ matrices $A$ and $B$, the \udef{bracket} (or \udef{commutator}) of $A$ and $B$, denoted $[A,B]$ is defined to be
\[ [A,B] = AB - BA \]
\end{definition}

Using the mathematical convention, the Lie algebra of any matrix Lie group is closed under brackets. This is in general not the case using the physics convention. Take for example the algebra $\suAlg(2)$, generated by $\{\sigma_1, \sigma_2, \sigma_3\}$. Then
\[ [\sigma_1,\sigma_2] = \begin{pmatrix}
2i & 0 \\ 0 & -2i
\end{pmatrix} \]
which is not Hermitian and thus not an element of $\suAlg(2)$! This also means that $\suAlg(n)$ (with $n>1$) is not an algebra in according to the definition we are about to give.

Despite the problems with the conventions, these properties seem nice. We would like to study things that exhibit these properties in general. So based on this we define a Lie algebra in general in the following way.

\begin{definition}
A (finite-dimensional) real or complex \udef{Lie algebra $\mathfrak{g}$} is an $n$-dim (real or complex) vector space with the following map:
\[[\cdot,\cdot]: \mathfrak{g}\times\mathfrak{g} \to \mathfrak{g}: (X,Y) \mapsto [X,Y]\]
that has the following properties
\begin{enumerate}
\item Bilinear: $\forall X,Y,Z \in \mathfrak{g}, \qquad a,b \in \R \quad (\text{or} \C)$:
\[ [aX + bY, Z] = a[X,Z] + b[Y,Z] \]
\item Antisymmetric: $\forall X,Y \in \mathfrak{g}$
\[ [X,Y] = -[Y,X] \]
\item Satisfies the \udef{Jacobi identity}: $\forall X,Y,Z \in \mathfrak{g}$
\[ [X,[Y,Z]] + [Y,[Z,X]] + [Z,[X,Y]] = 0 \]
\end{enumerate}
\end{definition}
The only surprising thing in this definition is the appearance of the Jacobi identity. It can be thought of as a condition that takes the place of associativity, but is weaker. In fact every Lie algebra can be embedded in into some associative algebra so that the bracket corresponds to the operation $XY - YX$. 

If we follow the mathematical convention, the Lie algebra of a matrix Lie group is a real Lie algebra in the sense of the above definition. Unfortunately this is not true for the physics convention.

As noted above, this notably means $\suAlg(n)$ (with $n>1$) is not an algebra in according this definition. There are several ways to solve this problem. The obvious one would be to redefine the bracket operator when using the physics convention (i.e.\ say that $[A,B] = -i \left(AB - BA\right)$). This is usually not done. We could also extend $\suAlg(n)$ to include $iX$ for every $X \in \suAlg(n)$ (this is called the \udef{complexification} of $\suAlg(n)$), which would mean that $\suAlg(n)$ is actually $\slAlg(n)$ (i.e.\ we drop the condition that the elements of $\suAlg(n)$ have to be Hermitian). This is apparently actually done sometimes in the physics literature. Or finally we can do what we will do in these notes, namely forget about this definition, use the definition we will motivate in the next section and write the extra $i$ whenever it pops up.

Furthermore for every finite-dimensional real or complex vector space $V$, let $\glAlg(V)$ denote the space of linear maps of $V$ into itself. Then $\glAlg(V)$ is a real or complex Lie algebra with the bracket operation $[A,B] = AB - BA$.

\subsection{Lie algebra of a Lie group}

We finally define the Lie algebra:
\begin{definition}
The \udef{Lie algebra} of a Lie group $G$ is the tangent space at the identity with the bracket operation defined by
\[ [v,w] = [X^v, X^w]_e. \]
\end{definition}



\section{Representations of Lie algebras}
TODO: representations of Lie groups: Representation vs linear group action. Continuous groups must be represented on the physical Hilbert space by unitary operators $U(T(\theta))$.


We start with some definitions.

\begin{definition}
A \udef{homomorphism} between two algebras $\mathfrak{g}_1, \mathfrak{g}_2$ is a map that preserves $[,]$:
\[ \phi: \mathfrak{g}_1 \to \mathfrak{g}_2: [X_1,X_2] \mapsto \phi([X_1,X_2]) = [\phi(X_1), \phi(X_2)] \]
If the map is invertible, it is called an \udef{isomorphism}.
\end{definition}

Every Lie group homomorphism gives rise to a Lie algebra homomorphism.
\begin{eigenschap}
Let $G$ and $H$ be matrix Lie groups, with Lie algebras $\mathfrak{g}$ and $\mathfrak{h}$ respectively. Suppose that $\Phi: G \to H$ is a Lie group homomorphism. Then there exists a unique real linear \ueig{homomorphism} $\phi: \mathfrak{g} \to \mathfrak{h}$ such that
\[\Phi\left(e^{X}\right) = e^{\phi(X)}\]
for all $X \in \mathfrak{g}$. The map $\phi$ has the following additional properties:
\begin{enumerate}
\item $\phi\left(AXA^{-1}\right) = \Phi(A)\phi(X)\Phi(A)^{-1}$, for all $X\in\mathfrak{g}, A \in G$
\item $\phi(X) = \left.\od{}{t}\Phi \left(e^{tX}\right)\right|_{t=0}$, for all $X \in \mathfrak{g}$
\end{enumerate}
\end{eigenschap}

\begin{definition}
An \udef{algebra representations} is a homomorphism between an abstract algebra and the space of linear operators.
\[ D: \mathfrak{g} \to \GL(n,\R) \; \text{or} \; \GL(n,\C): X \mapsto D(X) \]
A representation is said to be faithful if it is injective.
\end{definition}

\begin{eigenschap}
\ueig{Ado's theorem}:
Any finite dimensional Lie algebra admits a faithful matrix representation.
\end{eigenschap}
This nontrivial theorem means that every Lie algebra can be viewed as a subalgebra of $\glAlg(n,\C)$, and thus as an algebra of a matrix Lie group.

\subsection{Adjoint representation}

\begin{eigenschap}
Let $G$ be a matrix Lie group, with Lie algebra $\mathfrak(g)$. Let $X$ be an element of $\mathfrak{g}$ and $A$ an element of $G$.
\[ AXA^{-1} \in \mathfrak{g} \]
\end{eigenschap}
This means that the following definition makes sense:
\begin{definition}
Let $G$ be a matrix Lie group with algebra $\mathfrak{g}$. Then for each $A \in G$ we define the linear map $\Ad_A: \mathfrak{g} \to \mathfrak{g}$ by the formula
\[ \Ad_A(X) = AXA^{-1} \]
\end{definition}
\begin{eigenschap}
\begin{itemize}
\item $\Ad_A^{-1} = \Ad_{A^{-1}}$.
\item The map $A \to \Ad_A$ is a group homomorphism of $G$ into $\GL(\mathfrak{g})$.
\item $\Ad_A([X,Y]) = [\Ad_A(X),\Ad_A(Y)] \qquad \forall A\in G, X,Y \in \mathfrak{g}$.
\end{itemize}
\end{eigenschap}
Because $A \to \Ad_A$ is a group homomorphism, we have an associated algebra homomorphism, $X \mapsto \ad_X$.
\begin{eigenschap}
The associated Lie algebra map $\ad: \mathfrak{g} \to \glAlg(\mathfrak{g})$ is given by
\[ \ad_X(Y) = [X,Y] \]
\end{eigenschap}
This last property generalises well and we can use it to define the adjunct map for a Lie algebra in general.

The maps $\Ad$ and $\ad$ give the \udef{adjoint representations} of $G$ and $\mathfrak{g}$.

The adjoint representation $\ad_{X_i}$ is linear, and thus can be represented as a matrix. So for every $X_i$ in the basis, we have a $T_i$ that maps the coordinates of a $Y \in \mathfrak{g}$ to $[X_i, Y]$. If we write $Y = c_1X_1 + c_2X_2 + c_3X_3 + \ldots$, then
\[ T_i \begin{pmatrix}
c_1 \\ c_2 \\ \vdots
\end{pmatrix} = \begin{pmatrix}
if^1_{11}c_1 + if^1_{12}c_2 + \hdots \\
if^2_{11}c_1 + if^2_{12}c_2 + \hdots \\
\vdots
\end{pmatrix} \]
So
\[ T_i = \begin{pmatrix}
if^1_{11} & if^1_{12} & \ldots \\
if^2_{11} & if^2_{12} & \ldots \\
\vdots
\end{pmatrix} \qquad \text{or} \qquad \left(T_i\right)^k_j = if^k_{ij}\]
These matrices have the following property (derived from the Jacobi identity):
\begin{eigenschap}
\[ [T_i,T_j] = -if_{ij}^k T_k \]
\end{eigenschap}
 

\begin{definition}
The \udef{Cartan-Killing form} 
\begin{align}
g_{ij} &\equiv \Tr[T_i\cdot T_j] \\
&=-f_{ik}^ef_{je}^k
\end{align}
\end{definition}

\begin{definition}
The \udef{quadratic Casimir} in a given representation of an algebra is given by
\[ C_2 = g^{ij}X_iX_j \]
This is an \ueig{invariant} for a specific representation.
\end{definition}

\begin{eigenschap}
The quadratic Casimir \ueig{commutes} with any element $X$ of the algebra:
\[ [C_2,X] = 0 \]
In general $C_2 \notin \mathfrak{g}$
\end{eigenschap}
A \udef{Casimir} is an operator that commutes with all generators. 
\begin{example}
The angular momentum operators have to structure of $\suAlg(2)$
\begin{align}
[L_i,L_j] &= i\epsilon_{ijk}L_k \qquad (L_i \; \text{generators}) \\
[L^2, L_i] &= 0
\end{align}
\end{example}

\begin{example}
Find the Casimir operator of the fundamental representation of $\suAlg(2)$.

We call $\tau_i = \frac{\sigma_i}{2}$, so that
\[ [\tau_i, \tau_j] = i\epsilon_{ijk}\tau_k \]
We then compute
\begin{align}
C_2 &= \sum_{i,j}g^{ij}\tau_i\tau_j = \frac{1}{2}\sum_{i,j}\delta_{ij}\tau_i\tau_j = \frac{3}{8}\mathbb{1} \\
&= \frac{1}{2}s(s+1) \mathbb{1} \qquad \Rightarrow \qquad s= \tfrac{1}{2}
\end{align}
Where we used that $g^{ij} = (g_{ij})^{-1}$ and $g_{ij} = \epsilon_{ike}\epsilon_{jke} = 2\delta_{ij}$
\end{example}

\subsection{Representations of $\suAlg(2)$}
\subsubsection{The algebras $\suAlg(2)$ and $\soAlg(3)$} are isomorphic
\[ \soAlg(3) = \{ X\in \GL(3,\R), X^\intercal = -X \} \]
\[ X_1 = \begin{pmatrix}
0 & 0 & 0 \\ 0 & 0 & 1 \\ 0 & -1 & 0
\end{pmatrix}, \qquad X_2 = \begin{pmatrix}
0 & 0 & -1 \\ 0&0&0 \\ 1&0&0
\end{pmatrix}, \qquad X_3 = \begin{pmatrix}
0&1&0 \\ -1&0&0 \\ 0&0&0
\end{pmatrix} \]

\begin{example}
Show that $[X_i,X_j] = -\epsilon_{ijk}X_k$
\end{example}

Let $J_i$ be
\[ J_i = -iX_i \]
so that $[J_i,J_j] = i\epsilon_{ijk}J_k$. Then $J_i$ are generators of $\suAlg(2)$.
\remark{The groups have the same algebra, which means they are the same around identity}
\subsubsection{Building the $\suAlg(2)$ representation} we run into the problem that the $J_i$ cannot be diagonalised simultaneously, i.e.\ they don't commute.
So we choose a basis in which $J_3$ is diagonal, then we define
\[ J_\pm \equiv J_1 \pm iJ_2 \]
With the following properties:
\[ \begin{cases}
[J_3,J_\pm] = [J_3,J_1]\pm i[J_3,J_2] = iJ_2 \pm J_1 = \pm J_\pm \\
[J_+, J_-] = i[J_2,J_1] - i[J_1,J_2] = 2J_3
\end{cases} \]
We now notate a basis of states $V$ with $\ket{j,m}$
\[ J_3\ket{j,m} = m\ket{j,m} \]
where $m$ is an eigenvalue of $J_3$ and $j$ is the biggest eigenvalue ($m\leq j$). We can find enough eigenvectors to make the basis because $J_3$ is diagonal.

From the relation 
\begin{align}
J_3 \left(J_\pm\ket{j,m}\right) &= \left(J_\pm J_3 + [J_3,J_\pm]\right)\ket{j,m} \\
&= J_\pm m \ket{j,m} \pm J_\pm \ket{j,m} \\
&= (m\pm 1)\left(J_\pm\ket{j,m}\right)
\end{align}
we get the following
\[ \begin{cases}
J_+\ket{j,j} = 0 \qquad (j>0) \\
J_-\ket{j,j_-} = 0 \qquad (j_- \;\text{smallest eigenvalue of}\; J_3)
\end{cases} \]

We also see that the eigenvalues are spaced an integer apart, from $j_-$ to $j$.


Because the trace of a commutator is zero (as the trace is cyclic), we also have that 
\[ \Tr[J_3] = \frac{1}{2}\Tr([J_+,J_-]) = 0 = \sum_{j_-}^j m \]
which means that
\[ 0 = j + (j-1) + \ldots + (j_-+1) +j_- \quad \Rightarrow \quad j+j_- = 0 \quad \Rightarrow \quad j_- = -j \]

So the dimension of $V$ is $2j+1$, which must be an integer, meaning that $j$ must be half-integer.

We also impose the following normalisation:
\[ \braket{j,m} = 1 \qquad \braket{j,j} = 1\]

For a generic state $\ket{j,m}$ we get the following:
\begin{align}
J_3\ket{j,m} &= m\ket{j,m} \\
J_+\ket{j,m} &= [(j+1+m)(j+m)]^{1/2} \ket{j,m+1} \\
J_-\ket{j,m} &= [(j+1-m)(j+m)]^{1/2} \ket{j,m-1}
\end{align}

\begin{example}
Fundamental representation of $\suAlg(2)$ (i.e.\ of dimension $2$).

\[ j=1/2 \qquad \begin{cases}
\ket{1/2,+1/2} = \begin{pmatrix} 1 \\ 0 \end{pmatrix} \\
\ket{1/2,-1/2} = \begin{pmatrix} 0 \\ 1 \end{pmatrix}
\end{cases} \]
\[ J_3 = - \frac{1}{2} \begin{pmatrix}
-1 & 0 \\ 0 & 1
\end{pmatrix} = \frac{\sigma_3}{2} \]
\[ \begin{cases}
J_+\ket{1/2, 1/2} = 0 \\ J_+\ket{1/2, -1/2} = \ket{1/2,1/2}
\end{cases} \qquad \begin{cases}
J_-\ket{1/2, 1/2} = \ket{1/2,-1/2} \\ J_-\ket{1/2, -1/2} = 0
\end{cases} \]
\[J_+ = \begin{pmatrix}
0&1\\0&0
\end{pmatrix} \qquad J_- = \begin{pmatrix}
0&0\\1&0
\end{pmatrix}\]
\[ J_1 = \frac{1}{2}\begin{pmatrix}
0&1\\1&0
\end{pmatrix} = \frac{\sigma_2}{2} \qquad J_2 = \frac{1}{2} \begin{pmatrix}
0&-i\\i&0
\end{pmatrix} = \frac{\sigma_2}{2}\]
\end{example}

\begin{example}
The $j=1$ representation of $\suAlg(2)$
\[\ket{1,1} = \begin{pmatrix} 1\\0\\0 \end{pmatrix}, \quad \ket{1,0} = \begin{pmatrix} 0\\1\\0 \end{pmatrix}, \quad \ket{1,-1} = \begin{pmatrix} 0\\0\\1 \end{pmatrix} \quad J_3 = \begin{pmatrix}
1&0&0\\0&0&0\\0&0&-1
\end{pmatrix}\]
\[ \begin{cases}
J_+\ket{1,1} = 0 \\
J_+\ket{1,0} = \sqrt{2}\ket{1,1} \\
J_+\ket{1,-1} = \sqrt{2}\ket{1,0}
\end{cases} \qquad \begin{cases}
J_-\ket{1,1} = \sqrt{2}\ket{1,0} \\
J_-\ket{1,0} = \sqrt{2}\ket{1,-1} \\
J_-\ket{1,-1} = 0
\end{cases} \]
\[J_+ = \sqrt{2}\begin{pmatrix}
0&1&0\\0&0&1\\0&0&0
\end{pmatrix} \qquad J_- = \sqrt{2}\begin{pmatrix}
0&0&0\\1&0&0\\0&1&0
\end{pmatrix}\]
\[ J_1 = \frac{1}{\sqrt{2}}\begin{pmatrix}
0&1&0\\1&0&1\\0&1&0
\end{pmatrix} \qquad J_2 = \frac{1}{\sqrt{2}} \begin{pmatrix}
0&-i&0\\i&0&-i\\0&i&0
\end{pmatrix}\]
\end{example}

\begin{example}
Exercise: calculate $C_2$
\[ C_2 = \frac{1}{2}l(l+1) \qquad (l=1) \]
\end{example}

\chapter{Riemannian manifolds}
\section{Riemannian metrics}
\begin{definition}
Let $M$ be a smooth manifold. A \udef{Riemannian metric} on $M$ is a smooth covariant $2$-tensor field $g\in T^*M\otimes T^*M$ whose value $g_p$ at each point $p\in M$ is an inner product on $T_pM$. We often use the notation
\[ \inner{v,w}_g \defeq g_p(v,w) \]
where $p\in M$ and $v,w\in T_pM$. Similarly we write $\norm{v}_g \defeq \sqrt{\inner{v,v}_g}$.

A \udef{Riemannian manifold} is a pair $(M,g)$ where $M$ is a smooth manifold and $g$ is a Riemannian metric on $M$.
\end{definition}

Most things work with boundary as well.

\begin{lemma}
Every smooth manifold admits a Riemannian metric.
\end{lemma}
\begin{proof}
TODO with partition of unity.
\end{proof}

\begin{example}
The \udef{Euclidean metric} is the Riemannian metric $g_E$ on the manifold $\R^n$ whose value at each $x\in\R^n$ is the standard inner product on $T_x\R^n$.
\end{example}

\subsection{Isometries}
\begin{definition}
Let $(M_1, g_1)$ and $(M_2, g_2)$ be Riemannian manifolds. An \udef{isometry} from $(M_1, g_1)$ to $(M_2, g_2)$ is a diffeomorphism $\varphi: M_1\to M_2$ such that $\varphi^* g_2 = g_1$.

We say $\varphi: M_1\to M_2$ is a \udef{local isometry} if for each point $p\in M_1$ their is a neighbourhood $U(p)$ such that $\varphi|_U$ is an isometry onto an open subset of $M_2$.

A Riemannian $n$-manifold is called \udef{flat} if it is locally isometric to a Euclidean space.
\end{definition}
An isometry from $(M,g)$ to itself is called an isometry of $(M,g)$. The set of isometries of $(M,g)$ is a group under composition, the \udef{isometry group} of $(M,g)$, denoted $\Iso(M,g)$.

\begin{lemma}
All Riemannian $1$-manifolds are flat.
\end{lemma}

\begin{lemma}
A mapping $\varphi:M\to M'$ between smooth manifolds is an isometry \textup{if and only if} $\varphi$ is a smooth bijection and each differential $\diff\varphi_p:T_pM\to T_{\varphi(p)}M'$ is a linear isometry.
\end{lemma}
\begin{proof}
The only part to prove is that $\varphi$ is automatically a diffeomorphism if it is a smooth bijection. This follows from the global rank theorem \ref{theorem:globalRank} because $F$ has contant rank (equal to the dimension of $M$ and $M'$).
\end{proof}

\subsection{Local representations for metrics}
Let $(x^1, \ldots, x^n)$ be smooth local coordinates on the neighbourhood $U\subseteq M$. Then $g|_U$ can be written as
\[ g|_U = g_{ij}\diff{x^i}\otimes\diff{x^j}  \]
The definition of inner product translates to the requirement that $[g(p)]_{ij}$ be a symmetric, non-singular matrix. Using symmetry we get the symmetric product
\[ g|_U = g_{ij}\diff{x^i}\diff{x^j} \]

\begin{example}
The Euclidean metric can be expressed as
\[ g_E = \sum_i\diff{x^i}\diff{x^i} = \delta_{ij}\diff{x^i}\diff{x^i} \]
so $g_{ij} = \delta_{ij}$.
\end{example}

\begin{proposition}
Given a smooth local frame for $TM$ we can construct a smooth orthonormal frame with the same span.
\end{proposition}
\begin{proof}
Gram-Schmidt.
\end{proof}

\subsection{Constructing Riemannian metrics}
\subsection{Riemannian immersions}
\begin{proposition}
Let $(M,g)$ be a Riemannian manifold, $M'$ a smooth manifold and $F:M'\to M$ a smooth map. Then $g' = F^*g$ is a Riemannian metric on $M$ \textup{if and only if} $F$ is an immersion.
\end{proposition}
\begin{proof}
The only reason $g' = F^*g$ may fail to be a metric is if it is not definite. First assume $F$ is not an immersion. Then there exist $p\in M'$ and $v,w\in T_pM'$ such that $\diff{F}_p(v) = \diff{F}_p(w)$ and $v\neq w$. Then $v-w \neq 0$, but
\[ \inner{v-w,v-w}_{g'}= \inner{\diff{F}(v-w),\diff{F}(v-w)}_{g} = \inner{0,0}_g = 0. \]
Conversely, assume $g'$ not definite. Then there exists a $v\neq 0$ such that $0 = \norm{v}_{g'}= \norm{\diff{F}(v)}_{g}$, implying $\diff{F}(v) = 0$. Thus the kernel of $\diff{F}$ is not $\{0\}$, meaning it is not injective by \ref{prop:injectivityKernelTriviality} and thus $F$ is not an immersion by definition. 
\end{proof}
The metric $g' = F^*g$ of the proposition is called the \udef{metric induced by $F$}.

An immersion (resp. embedding) $F: (M,g)\to (M',g')$ is called an \udef{isometric immersion} (resp. \udef{isometric embedding}) if $g' = F^*g$.

\begin{lemma}
Existence of adapted orthonormal frames.
\end{lemma}

\begin{definition}
Let $(M,g)$ be a Riemannian manifold and $M'\subseteq M$ a smooth submanifold. A vector $v\in T_pM$, for some $p\in M'$, is called \udef{normal} to $M'$ if $\inner{v,w}_g = 0$ for every $w\in T_pM'$.

The space of all vectors normal to $M'$ at $p\in M'$ is called the \udef{normal space} $N_pM'$ at $p$.
\end{definition}
Clearly $N_pM' = (T_pM')^\perp$ and
\[ T_pM = T_pM' \oplus N_pM'. \]

\begin{proposition}[Normal bundle]
Let $(M,g)$ be a Riemannian $m$-manifold without boundary and $M'\subseteq M$ a an immersed $n$-submanifold. The set
\[ NM' = \bigsqcup_{p\in M'}N_pM' \]
is a smooth subbundle of $TM|_{M'}$ of rank $(m-n)$.
\end{proposition}
The vector bundle $NM'$ is called the \udef{normal bundle} of $M'$.

A section of the normal bundle $NM'$ is called a \udef{normal vector field} along $M'$.

The \udef{tangential projection} $\pi^\top: TM|_{M'}\to TM'$ and the \udef{normal projection} $\pi^\perp: TM|_{M'}\to NM'$ are the maps that for each $p\in M'$ restrict to the orthogonal projections $T_pM\to T_pM'$ and $T_pM\to N_pM'$.

\begin{lemma}
The tangential and normal projections are smooth bundle homomorphisms
\end{lemma}

\subsection{Riemannian products}
\begin{definition}
Let $(M_1,g_1)$ and $(M_2,g_2)$ be Riemannian manifolds. The product manifold $M_1\times M_2$ has a natural Riemannian metric $g=g_1\oplus g_2$ called the \udef{product metric} defined by
\[ g_{p_1,p_2}: (T_{p_1}M_1\oplus T_{p_2}M_2)^2 \to \R: (v_1+v_2, w_1+w_2) \mapsto g_1|_{p_1}(v_1,w_1) + g_2|_{p_2}(v_2,w_2) \]
where we have identified $T_{(p_1,p_2)}(M_1\times M_2)$ with $T_{p_1}M_1\oplus T_{p_2}M_2$.
\end{definition}

\subsection{Riemannian submersions}
\subsubsection{Horizontal and vertical tangent spaces}
Suppose $M,M'$ are smooth manifolds, $\pi:M\to M'$a smooth submersion and $g$ a Riemannian metric on $M$.

TODO: we can view $M$ as a fibre bundle with as fibres the properly embedded smooth manifolds $M_y = \pi^{-1}(y)$.

At each point $x\in M$ we can split $T_xM$ into two subspaces $V_x \oplus H_x$, the \udef{horizontal} and \udef{vertical tangent spaces} at $x$, defined by
\[ V_x \defeq \ker\diff{\pi}_x = T_x(M_{\pi(x)}) \qquad \text{and} \qquad H_x = (V_x)^\perp. \]
Where the equality $\ker\diff{\pi}_x = T_x(M_{\pi(x)})$ is due to (TODO tangent space to a submanifold). Notice that the definition of $V_x$ does not depend on the metric, but the definition of $H_x$ does.

A \udef{horizontal vector field} on $M$ consists of vectors in the horizontal tangent space and a \udef{vectical vector field} on $M$ consists of vectors in the vertical tangent tangent space on $M$.

A vector field $X$ on $M$ is a \udef{horizontal lift} of a vector field $X'$ on $M'$ if $X$ is horizontal and $\pi$-related to $X$, which means that
\[ \forall x\in M: \; \diff{\pi}_x(X_x) = X'_{\pi(x)} .\]

\begin{proposition}
Let $M,M'$ be smooth manifolds, $\pi:M\to M'$ a smooth submersion and $g$ a Riemannian metric on $M$.
\begin{enumerate}
\item Every smooth vector field $W$ on $M$ can uniquely be expressed as the sum of a smooth horizontal and a smooth vertical vector field:
\[ W = W^H + W^V. \]
\item Every smooth vector field on $M'$ has a unique smooth horizontal lift to $M$.
\item For every $x\in M$ and $v\in H_x$, there is a vector field $X'\in\mathfrak{X}(M')$ whose horizontal lift $X$ satisfies $X_x = v$.
\end{enumerate}
\end{proposition}
The last part of the previous proposition says that any horizontal vector can be extended to a horizontal lift on all of $M$.

Importantly, it is \emph{not true} that every horizontal vector field on $M$ is a horizontal lift.
\begin{example}
Take $\pi: \R^2\to\R: (x,y)\mapsto x$. Let $W$ be the smooth vector field $y\partial_x$ on $\R^2$. At any point $V_p= \Span\{\partial_y\}$ and $H_p= \Span\{\partial_x\}$, so $W$ is horizontal. But there is no vector field on $\R$ whose horizontal lift is $W$. Indeed $\diff{\pi}_p(W) = y\partial_x$ is not constant on $\pi^{-1}(p)$ because it depends on $y$.
\end{example}

\subsubsection{Riemannian submersions}
\begin{definition}
Let $\pi: (M,g)\to (M',g')$ be a smooth submersion between Riemannian manifolds. Then $\pi$ is a \udef{Riemannian submersion} if $\diff{\pi}_x|_{H_x}: H_x\to T_{\pi(x)}M'$ is a (bijective) linear isometry for all $x\in M$.
\end{definition}
Equivalently, the submersion $\pi$ is a Riemannian submersion if the metrics satisfy
\[ \forall x\in M:\; \forall v,w\in H_x:\; g_x(v,w) = g'_{\pi(x)}(\diff{\pi}_x(v), \diff{\pi}_x(w)). \]

\subsubsection{Riemannian coverings}

\subsection{Basic constructions derived from the metric}
\subsubsection{Raising and lowering indices}
Let $M$ be a smooth manifold. Given a Riemannian metric $g$ in $M$, we define a bundle homomorphism
\[ \hat{g}: TM \to T^*M: v\mapsto g_p(v,\cdot). \]
In other words we have $\hat{g}(v)(w) = g_p(v,w)$ for all $p\in M$ and $v,w\in T_pM$.

Musical isomorphisms

\subsubsection{Inner products of tensors}
We define $\inner{\omega, \eta}_g \defeq \inner{\omega^\sharp, \eta^\sharp}$.
Then
\[ \inner{\omega,\eta}= g_{kl}(g^{ki}\omega_i)(g^{lj}\eta_j) = \delta^i_lg^{lj}\omega_i\eta_j = g^{ij}\omega_i\eta_j. \]



\section{Connections}
\subsection{Affine connection}
\begin{definition}
Let $\pi: E\to M$ be a smooth vector bundle over a smooth manifold $M$ and let $\Gamma(E)$ denote the space of sections of $E$. A \udef{connection} in $E$ is a map
\[ \nabla: \mathfrak{X}(M)\times \Gamma(E) \to \Gamma(E): (X,Y)\mapsto \nabla_X Y \]
satisfying the following properties:
\begin{enumerate}
\item $\nabla_X Y$
\end{enumerate}
\end{definition}

\section{Geodesics}


\section{Curvature}




\chapter{Algebraic geometry}
\chapter{Geometric topology}
\section{Vector bundles}
\subsection{Definition}
\subsection{Operations on vector bundles}

\part{Number Theory}
\setcounter{chapter}{0} % Reset chapter counter
\chapter{Primes}

\chapter{Irrational numbers}

\chapter{Congruences and modular arithmetic}



\part{$K$-theory}
\setcounter{chapter}{0} % Reset chapter counter
\chapter{$K$-theory for additive categories}
Let $\cat{C}$ be an additive category. Consider the isomorphism classes $[E]$ of objects $E$ in $\cat{C}$ with an addition operation given by
\[ [E]+[F] = [E\oplus F] \qquad E,F\in \cat{C}. \]
\begin{lemma}
The isomorphism classes of $\cat{C}$ form an abelian monoid $M(\cat{C})$ under this addition operation:
\[ E\oplus (F\oplus G) \cong (E\oplus F)\oplus G, \qquad  E\oplus F \cong F\oplus E, \qquad \text{and}\qquad E\oplus 0 \cong E. \]
\end{lemma}
Note that it is necessary to use isomorphism classes, because in general
\[ E\oplus (F\oplus G) \neq (E\oplus F)\oplus G \qquad \text{even though} \qquad E\oplus (F\oplus G) \cong (E\oplus F)\oplus G. \]
\begin{definition}
The \udef{$K$ functor} is in this case just the Grothendieck functor $G$ applied after making the category a monoid:
\[ K(-) \defeq G(M(-)). \]
\end{definition}


\chapter{Topological $K$-theory}
\section{The group $K(X)$}
\begin{definition}
Let $X$ be a compact topological space. The category $\cat{Vect}(X)$ of vector bundles over $X$ with the direct sum is an additive category. We define the $K$ group of $X$ as
\[ K(X) \defeq K(\cat{Vect}(X)). \] 
\end{definition}
Let $\mathcal{E}_n$ be the trivial bundle of rank $n$ over a compact space $X$.
\begin{proposition}
Every element $x$ of $K(X)$ can be written as $[E]-[\mathcal{E}_n]$ for some $n$ and some vector bundle $E$ over $X$.

Moreover, $[E]-[\mathcal{E}_p]=[F]-[\mathcal{E}_q]$ \textup{if and only if} there exists an integer $n$ such that $E\oplus\mathcal{E}_{q+n}\cong F\oplus\mathcal{E}_{p+n}$.
\end{proposition}
\begin{proof}
Immediate from Grothendieck construction and the fact that for all vector bundles $E$ there exists a vector bundle $F$ such that $E\oplus F \cong \mathcal{E}_n$.
\end{proof}
\begin{corollary}
Let $E,F$ be vector bundles over $X$. Then $[E]=[F]$ in $K(X)$ \textup{if and only if} $E\oplus \mathcal{E}_n \cong F\oplus \mathcal{E}_n$ for some $n$.
\end{corollary}

\begin{proposition}
For topological spaces $K$ is a contravariant functor on the category of compact spaces.
\end{proposition}

\section{The group $\widetilde{K}(X)$ for pointed spaces}
\begin{definition}
Let $(X, x_0)$ be a pointed compact space. The projection $\pi: X\to \{x_0\}$ induces a homomorphism $K(\{x_0\})\cong \Z\to K(X)$. The \udef{reduced $K$-theory} $\widetilde{K}(X)$ of $X$ is the cokernel of this homomorphism:
\[ \begin{tikzcd}
0 \rar & \Z \rar & K(X) \rar & \widetilde{K}(X) \rar & 0.
\end{tikzcd} \]
\end{definition}

\begin{lemma}
There is a canonical splitting generated by the inclusion $\{x_0\}\hookrightarrow X$ so that
\[ K(X) \cong \Z\oplus \widetilde{K}(X) \qquad \text{and} \qquad \widetilde{K}(X) \cong\ker(K(\{x_0\}\hookrightarrow X)). \]
\end{lemma}

\begin{proposition}
The composition $\gamma: M(\cat{Vect}(X))\to K(X) \to \widetilde{K}(X)$ is a surjective homomorphism.
Moreover, $\gamma([E])=\gamma([F])$ \textup{if and only if} $E\oplus \mathcal{E}_p \cong F\oplus \mathcal{E}_q$ for some $p,q$.
\end{proposition}
This gives a more direct definition of $\widetilde{K}(X)$ as the quotient of $M(\cat{Vect}(X))$ by the equivalence relation
\[ [E] \sim [F] \quad\iff\quad \exists p,q\in \N:\; E\oplus\mathcal{E}_p \cong F\oplus \mathcal{E}_q.  \]


\section{The relative $K$-group $K(X,Y)$}
\begin{definition}
Let $X$ be a compact topological space and $Y$ a closed subspace. Consider the triples $(E,F,\alpha)$ where $E,F$ are vector bundles over $X$ and $\alpha$ is an isomorphism $E_Y \to F_Y$ where $E_Y$ and $F_Y$ are the vector bundles $E,F$ restricted to $Y$.

We define the sum of two triples to be
\[ (E,F,\alpha) + (E',F',\alpha') = (E\oplus E',F\oplus F',\alpha\oplus\alpha'). \]

We call two triples $(E,F,\alpha), (E',F',\alpha')$ isomorphic if there exist isomorphisms $f:E\to E'$ and $g:F\to F'$ such that the diagram
\[ \begin{tikzcd}
E|_Y \rar{\alpha} \dar[swap]{f|_Y} & F|_Y \dar{g|_Y} \\
E'|_Y \rar{\alpha'} & F'|_Y
\end{tikzcd} \qquad \text{commutes.}\]

We consider the equivalence relation of ``stable isomorphism'' on these triples, that is two triples $(E,F,\alpha), (E',F',\alpha')$ are equivalent if and only if there exist triples $(G,G,I_{G_Y})$ and $(G',G',I_{G'_Y})$ such that
\[ \begin{cases}
(E,F,\alpha)+(G,G,I_{G_Y}) = (E\oplus G,F\oplus G, \alpha\oplus I_{G_Y}) & \text{and} \\ (E',F',\alpha')+(G',G',I_{G'_Y}) = (E'\oplus G',F'\oplus G', \alpha'\oplus I_{G'_Y})
\end{cases} \]
are isomorphic.

Then $K(X,Y)$ is the set of equivalence classes of such triples. We denote the equivalence class of a triple $(E,F,\alpha)$ by $d(E,F,\alpha)$.
\end{definition}

\begin{proposition}
Let $X$ be a compact space and $Y$ a closed subspace. Then
\begin{enumerate}
\item $K(X,Y)$ is an abelian group with as neutral element
\[ 0 = d(G,G,I_{G_Y}) \]
and
\[ d(E,F,\alpha) + d(F,E,\alpha^{-1}) = 0; \]
\item $K(X) \cong K(X,\emptyset)$;
\item $d(E,F,\alpha)+d(F,G,\beta) = d(E,G,\beta\circ \alpha)$.
\end{enumerate}
\end{proposition}
\begin{proof}

\end{proof}

\begin{proposition}
Let $i$ be the homomorphism
\[ i:K(X,Y)\to K(X): d(E,F\alpha) \mapsto [E]-[F] \]
and $j$ the homomorphism
\[ i:K(X)\to K(Y): [E]-[F] \mapsto [E|_Y]-[F|_Y]. \]
Then we have the exact sequence
\[ \begin{tikzcd}
K(X,Y) \rar{i} & K(X) \rar{j} & K(Y).
\end{tikzcd} \]
Moreover, if $Y$ is a retract of $X$ (i.e.\ the inclusion $Y\hookrightarrow X$ admits a left-inverse), then we have the split exact sequence
\[ \begin{tikzcd}
0 \rar & K(X,Y) \rar & K(X) \rar & K(Y) \rar & 0.
\end{tikzcd} \]
\end{proposition}
\begin{corollary}
Let $(X,x_0)$ be a pointed space. Then $\{x_0\}$ is a retract of $X$ and thus
\[ K(X, \{x_0\}) \cong \ker(K(X)\to K(\{x_0\})) \cong \widetilde{K}(X) \]
\end{corollary}

\begin{proposition}
The projection $\pi: X\to X/Y$ induces an isomorphism $K(X/Y,\{y\}) \to K(X,Y)$.
\end{proposition}

\begin{proposition}
Let $Y$ be a closed subspace of a compact space $X$. Then we have the 
exact sequence 
\[ \begin{tikzcd}
\widetilde{K}(X/Y) \rar & \widetilde{K}(X) \rar & \widetilde{K}(Y).
\end{tikzcd} \]
\end{proposition}

\begin{theorem}[Atiyah-Jänich]
Let $X$ be a compact Hausdorff space and $H$ a Hilbert space. Then
\[ [X,\Fred(H)] \cong K(X). \]
\end{theorem}

\section{Clifford modules and the functor $K^{p,q}$}
\begin{proposition}
Let $A,B$ be $\R$-algebras. 
\[ (\cat{C}^A)^B \cong \cat{C}^{A\otimes_\R B} \]
\[ \cat{C}^{A\oplus B}\simeq \cat{C}^A\times \cat{C}^B. \]
\end{proposition}

\begin{definition}
Let $\cat{Vect}(X)^{p,q}$ be the category of $\Cl^{p,q}$-modules $W$ such that $W$ is a vector bundle over $X$.

We define $K^{p,q}(X)$ as the Grothendieck group of the functor
\[ \cat{Vect}(X)^{p,q+1}\to \cat{Vect}(X)^{p,q}  \]
\end{definition}

\begin{theorem}
Let $X$ be a compact space. Then $K^{0,0}$ and $K(0,1)$ are canonically isomorphic to $K(X)$ and $K^{-1}(X)$.
\end{theorem}

\subsection{Description via gradings}
\begin{definition}
Let $E\in \cat{Vect}(X)^{p,q}$. A \udef{grading} of $E$ is an endomorphism $\eta$ of $E$ regarded as an object of $\cat{Vect}(X)$ such that
\begin{enumerate}
\item $\eta^2 = I$;
\item $\eta\rho(e_i) = -\rho(e_i)\eta$.
\end{enumerate}
Equivalently, a grading on $E$ is a $\Cl^{p,q+1}$-structure on $E$ extending the $\Cl^{p,q}$-structure where $\eta = \rho(e_{p+q+1})$.
\end{definition}

\chapter{$K$-theory for $C^*$-algebras}
\section{Homotopy equivalence of unitaries}
TODO ref on homotopy + $\sim_h$.

\begin{lemma} \label{productHomotopy}
If $u_1 \sim_h v_1$ and $u_2 \sim_h v_2$, then $ u_1u_2\sim_h v_1v_2$.
\end{lemma}

\begin{definition}
Let $A$ be a unital $C^*$-algebra. We let $\mathcal{U}_0(A)\subseteq \mathcal{U}(A)$ denote the set of all unitaries homotopic with $\vec{1}$ in $\mathcal{U}(A)$.
\end{definition}

\begin{lemma} \label{homotopyOfUnitaries}
Let $A$ be a unital $C^*$-algebra.
\begin{enumerate}
\item For each self-adjoint element $h\in A$, $\exp(ih)\in\mathcal{U}_0(A)$.
\item If $u\in \mathcal{U}(A)$ with $\sigma(u) \neq \mathbb{T}$, then $u\in\mathcal{U}_0(A)$.
\item If $u,v\in\mathcal{U}(A)$ with $\norm{u-v}<2$, then $u\sim_h v$.
\end{enumerate}
\end{lemma}
\begin{proof} \hspace{1em}
\begin{enumerate}
\item By spectral mapping, \ref{spectralMappingCFC}, and \ref{propertiesFromSpectrum} we see that $\exp(ih)$ is unitary. The homotopy is given by $t\mapsto \exp(ith)$.
\item If $\sigma(u) \neq \mathbb{T}$, then for some real $\theta$, $\exp(i\theta)\notin \sigma(u)$. This means the exponential has a well defined inverse $f: \sigma(u) \to \interval[o]{\theta, \theta+2\pi}: \exp(it)\mapsto t$. By spectral mapping, \ref{spectralMappingCFC}, and \ref{propertiesFromSpectrum} we see that $f(u)$ is self-adjoint. It follows that $u = \exp(if(u))$, so we can conclude using (1).
\item Assume $\norm{u-v}<2$. Then
\[ 2 > \norm{u-v} = \norm{v^*}\norm{u-v} \geq \norm{v^*u - 1} \]
so $-2\notin \sigma(v^*u-1)$ and $-1 \notin \sigma(v^*u)$ by spectral mapping. By (2) $v^*u \sim_h \vec{1}$ and hence $u\sim_h v$ be \ref{productHomotopy}.
\end{enumerate}
\end{proof}

\begin{corollary}
The unitary group $\U(\C^{n\times n})$ is connected for all $n\in\N$.
\end{corollary}
\begin{proof}
Every element in $\C^{n\times n}$ has finite spectrum (the eigenvalues). So we conclude by (2) of \ref{homotopyOfUnitaries}.
\end{proof}
Because $\norm{u-v}\leq \norm{u}+\norm{v} = 2$ for all $u,v\in \Unitaries(A)$, two unitaries are only not homotopic if they lie at a distance of exactly 2.

TODO generalise to $\GL$:
\begin{lemma} \label{sectionConnectedToIdentity}
Let $A$ be a unital $C^*$-algebra. Then
\begin{enumerate}
\item $\mathcal{U}_0(A)$ is a normal subgroup of $\mathcal{U}(A)$;
\item $\mathcal{U}_0(A)$ is open and closed relative to $\mathcal{U}(A)$;
\item an element $u\in \mathcal{U}(A)$ belongs to $\mathcal{U}_0(A)$ \textup{if and only if} for some self-adjoint elements $h_1,\ldots, h_n \in A$
\[ u = \exp(ih_1)\cdot \ldots \cdot \exp(ih_n). \]
\end{enumerate}
\end{lemma}
\begin{proof}
TODO ref.
\begin{enumerate}
\item First note $\mathcal{U}_0(A)$ is closed under multiplication by \ref{productHomotopy}. Let $u_t$ be a continuous path from $\vec{1}$ to $u$. Then $u_t^{-1}$ and (for all $v\in \mathcal{U}(A)$) $v^*u_t v$ are continuous paths from $\vec{1}$ to $u^{-1}$ and $v^*uv$, respectively.
\item TODO.
\end{enumerate}
\end{proof}

\begin{lemma}[Whitehead]
Let $A$ be a unital $C^*$-algebra and $u,v\in\mathcal{U}(A)$. Then
\[ \begin{pmatrix}
u & 0 \\ 0 & v
\end{pmatrix} \sim_h \begin{pmatrix}
uv & 0 \\ 0 & \vec{1}
\end{pmatrix} \sim_h \begin{pmatrix}
vu & 0 \\ 0 & \vec{1}
\end{pmatrix} \sim_h \begin{pmatrix}
v & 0 \\ 0 & u
\end{pmatrix} \qquad \text{in}\;\mathcal{U}(A^{2\times 2}). \]
It follows in particular that
\[ \begin{pmatrix}
u & 0 \\ 0 & u^*
\end{pmatrix} \sim_h \begin{pmatrix}
\vec{1} & 0 \\ 0 & \vec{1}
\end{pmatrix}. \]
\end{lemma}
\begin{proof}
First
\[ \sigma_A \begin{pmatrix}
0 & \vec{1} \\ \vec{1} & 0
\end{pmatrix} = \sigma_\C \begin{pmatrix}
0 & 1 \\ 1 & 0
\end{pmatrix} = \{1\}, \qquad \text{so} \qquad \begin{pmatrix}
0 & \vec{1} \\ \vec{1} & 0
\end{pmatrix} \sim_h \begin{pmatrix}
\vec{1} & 0 \\ 0 & \vec{1}
\end{pmatrix} \]
by (2) of \ref{homotopyOfUnitaries}.
Hence
\[ \begin{pmatrix}
u & 0 \\ 0 & v
\end{pmatrix} = \begin{pmatrix}
u & 0 \\ 0 & \vec{1}
\end{pmatrix}\begin{pmatrix}
0 & \vec{1} \\ \vec{1} & 0
\end{pmatrix}\begin{pmatrix}
v & 0 \\ 0 & \vec{1}
\end{pmatrix}\begin{pmatrix}
0 & \vec{1} \\ \vec{1} & 0
\end{pmatrix} \sim_h \begin{pmatrix}
u & 0 \\ 0 & \vec{1}
\end{pmatrix}\begin{pmatrix}
v & 0 \\ 0 & \vec{1}
\end{pmatrix} = \begin{pmatrix}
uv & 0 \\ 0 & \vec{1}
\end{pmatrix}.
 \]
 The other claims follow in a similar way.
\end{proof}

\begin{lemma} \label{unitaryLifting}
Let $A,B$ be unital $C^*$-algebras and let $\Psi: A \to B$ be a surjective $*$-homomorphism. Then
\begin{enumerate}
\item $\Psi\left(\mathcal{U}_0(A)\right) = \mathcal{U}_0(B)$;
\item if $u\in\Unitaries(B)$, and if $u \sim_h \Psi(v)$ for some $v\in\Unitaries(A)$, then $u$ lifts to a unitary element in $A$.
\end{enumerate}
\end{lemma}
\begin{proof}
TODO
\end{proof}

\begin{proposition} \label{unitariesRetractionOfGL}
Let $A$ be a unital $C^*$-algebra and for all $a\in\GL(A)$, let $u(a)|a|$ be the polar decomposition of $a$. Then
\begin{enumerate}
\item $a\sim_h u(a)$ in $\GL(A)$;
\item for all $v_1,v_2\in\Unitaries(A)$
\[ v_1\sim_h v_2 \quad \text{in}\quad \GL(A) \qquad \iff \qquad v_1\sim_h v_2 \quad \text{in}\quad \Unitaries(A). \]
\end{enumerate}
\end{proposition}
\begin{proof}
TODO
\end{proof}
Thus the polar decomposition gives a deformation retract of $\GL(A)$ onto $\Unitaries(A)$.
TODO also for matrices!!


\section{Projections}
\begin{definition}
Let $A$ be a $C^*$-algebra. Two projections $p,q\in\Projections(A)$ are \udef{orthogonal} if $pq = 0$. We write $p \perp q$.
\end{definition}
TODO: $pq = 0$ iff $qp = 0$.

\begin{lemma}
Let $A$ be a $C^*$-algebra and $p,q\in\Projections(A)$. Then the following are equivalent:
\begin{enumerate}
\item $p+q \in \Projections(A)$;
\item $p$ and $q$ are orthogonal;
\item $p+q \leq 1$.
\end{enumerate}
\end{lemma}
\begin{proof}
TODO
\end{proof}
Every ``almost-idempotent'' can be approximated by a projection:
TODO



\subsection{Equivalence of projections}
\begin{definition}
Let $A$ be a $C^*$-algebra and $p,q\in A$. We write
\begin{enumerate}
\item $p\sim q$ if there exists $v\in A$ such that $p = v^*v$ and $q=vv^*$. This is \udef{(Murray-von Neumann) equivalence}.
\item $p \sim_u q$ if there exists $u\in \mathcal{U}(\tilde{A})$ such that $q = upu^*$. This is \udef{unitary equivalence}.
\end{enumerate}
\end{definition}
TODO: We can also take $A^\dagger$ ipv $\tilde{A}$.

\begin{lemma}
Both Murray-von Neumann equivalence and unitary equivalence are equivalence relations.
\end{lemma}
\begin{proof}
TODO transitivity.
\end{proof}

\begin{lemma}
Let $A$ be a unital $C^*$-algebra and $p,q\in\Projections(A)$. Then
\[ p \sim_u q \qquad \iff p \sim q \quad \text{and}\quad \vec{1} - p \sim \vec{1} - q. \]
\end{lemma}
\begin{proof}
Assume $p \sim_u q$, so we can write $q = upu^*$ for some $u\in\mathcal{U}(A)$. Put $v = up$ and $w = u(\vec{1}-p)$. Then
\begin{align*}
v^*v &= p^*u^*up = p, & vv^* &= upp^*u^* = upu^* = q \\
w^*w &= (\vec{1}-p)u^*u(\vec{1}-p) = \vec{1} - p & ww^* &= u(\vec{1}-p)(\vec{1}-p)u^* = u(\vec{1}-p)u^* = \vec{1} - q.
\end{align*}

Assume the converse. TODO
\end{proof}

\begin{lemma}
Let $p,q\in \Projections(A)$. If $\exists z\in \GL(\tilde{A})$ such that $q = zpz^{-1}$, then $p\sim_u q$.
\end{lemma}
\begin{proof}
We have $zp = qz$ and $pz^* = z^*q$, so $p$ commutes with $z^*z$:
\[ pz^*z = z^*qz = z^*zp. \]
Now put $u = z|z|^{-1}$ and calculate
\[ upu^* = z|z|^{-1}p|z|^{-1}z^* = zp|z|^{-2}z^* = qz(z^*z)^{-1}z^* = q. \]
TODO: clarify rules of calculation.
\end{proof}

\begin{proposition}
Let  $p,q\in \Projections(A)$. If $\norm{p-q}< 1$, then $p \sim_h q$.
\end{proposition}

\begin{proposition} \label{implicationsBeweenEquivalences}
Let $A$ be a $C^*$-algebra and $p,q\in\Projections(A)$. Then
\begin{enumerate}
\item if $p\sim_h q$, then $p \sim_u q$;
\item if $p\sim_u q$, then $p \sim q$;
\end{enumerate}and
\begin{enumerate}
\setcounter{enumi}{2}
\item if $p\sim q$, then $\begin{pmatrix}
p & 0 \\ 0 & 0
\end{pmatrix} \sim_u \begin{pmatrix}
q & 0 \\ 0 & 0
\end{pmatrix}$ in $A^{2\times 2}$;
\item if $p\sim_u q$, then $\begin{pmatrix}
p & 0 \\ 0 & 0
\end{pmatrix} \sim_h \begin{pmatrix}
q & 0 \\ 0 & 0
\end{pmatrix}$ in $A^{2\times 2}$.
\end{enumerate}
\end{proposition}
\begin{proof}
TODO
\end{proof}
\subsubsection{Decomposition into matrix algebras}
\begin{proposition}
Let $A$ be a unital $C^*$-algebra. Let $p_1,\ldots ,p_n$ be pairwise orthogonal and Murray-von Neumann equivalent projections for which $p_1 + \ldots + p_n = \vec{1}$. Then $A \cong (p_1Ap_1)^{n\times n}$.
\end{proposition}
\begin{proof}
TODO
\end{proof}

\subsection{Semigroups of projections}
\begin{definition}
Let $A$ be a $C^*$-algebra. We define $\Projections_\infty(A)$ as
\[ \Projections_n(A) = \Projections(A^{n\times n}) \qquad \Projections_\infty(A) = \bigcup_{n=1}^\infty \Projections_n(A) \]
and equip it with the binary operation $\oplus$:
\[ \forall p,q\in \Projections_\infty(A): \quad p\oplus q = \diag(p,q) = \begin{pmatrix}
p & 0 \\ 0 & q
\end{pmatrix}. \]
The involution on $\Projections_\infty(A)$ is the transposed pointwise application of $*$.
\end{definition}

\begin{lemma}
Let $A$ be a $C^*$-algebra.
\begin{enumerate}
\item If $p\in\Projections_n(A)$ and $q\in\Projections_m(A)$, then $p\oplus q\in\Projections_{n+m}(A)$.
\item The operation $\oplus$ is associative, making $\Projections_\infty(A)$ a semigroup.
\item If $p,q,r, p+q\in\Projections_\infty(A)$, then
\[ (p+q)\oplus r = p\oplus r + q\oplus r \qquad \text{and} \qquad r\oplus(p+q) = r\oplus p +r\oplus q. \]
\end{enumerate}
\end{lemma}
\begin{proof}
The first point follows from
\[ (p\oplus q)^* = \begin{pmatrix}
p^* & 0 \\ 0 & q^*
\end{pmatrix} = \begin{pmatrix}
p & 0 \\ 0 & q
\end{pmatrix} = p\oplus q \]
and
\[ (p\oplus q)^2 = \begin{pmatrix}
p & 0 \\ 0 & q
\end{pmatrix}\begin{pmatrix}
p & 0 \\ 0 & q
\end{pmatrix} = \begin{pmatrix}
p^2 & 0 \\ 0 & q^2
\end{pmatrix} = p\oplus q. \]
The others are easy.
\end{proof}

\begin{definition}
We define a relation $\sim_0$ on $\Projections_\infty(A)$ as follows: for $p\in \Projections_n(A)$ and $p\in \Projections_m(A)$,
\[ p \sim_0 q \defequiv \exists v\in A^{m\times n}: \quad p = v^*v\;\land \; q = vv^*. \]
\end{definition}
If $m=n$, then $\sim_0$ equivalence is Murray-von Neumann equivalence.
\begin{lemma}
The relation $\sim_0$ is an equivalence relation on $\Projections_\infty(A)$.
\end{lemma}

\begin{lemma} \label{sim0properties}
Let $A$ be a $C^*$-algebra and $p,q,r,p',q'\in \Projections_\infty(A)$. Then
\begin{enumerate}
\item $p\sim_0 p\oplus 0^{n\times n}$; in particular, $0 \sim_0 0^{n\times n}$;
\item if $p\sim_0 p'$ and $q\sim_0 q'$, then $p\oplus q \sim_0 p'\oplus q'$;
\item $p\oplus q \sim_0 q\oplus p$;
\item if $p,q\in\Projections_n(A)$ such that $pq = 0$, then $p+q\in\Projections_n(A)$ and $p+q \sim_0 p\oplus q$.
\end{enumerate}
\end{lemma}
\begin{proof}
TODO
\end{proof}

\begin{definition}
We set
\[ \mathcal{V}(A) = \Projections_\infty(A) / \sim_0 \]
and let $[p]_\mathcal{V}$ denote the equivalence class containing $p$. We define addition on $\mathcal{V}(A)$ by
\[ [p]_\mathcal{V} + [q]_\mathcal{V} = [p\oplus q]_\mathcal{V} \qquad \forall p,q\in\Projections_\infty(A). \]
\end{definition}
Clearly $\mathcal{V}(A)$ is a commutative monoid with identity $[0]_0$. The $\mathcal{V}$ comes from ``vector bundle''.

The addition $[p]_\mathcal{V} + [q]_\mathcal{V}$ is well-defined for all projections $p,q$. If $p\perp q$, then 
\[ [p]_\mathcal{V} + [q]_\mathcal{V} = [p+q]_\mathcal{V} \]
by \ref{sim0properties}. In general this does not work, which is essentially the reason we work with matrices in $\Projections_\infty(A)$, not just with projections.

\begin{lemma}
Then $\mathcal{V}(-): \cat{C^*alg} \to \cat{CMon}$ is a functor that sends morphisms $f: A \to B$ in $\cat{C^*alg}$,i.e.\ $*$-homomorphisms, to
\[  \mathcal{V}(f):\mathcal{V}(A) \to \mathcal{V}(B): [p]_\mathcal{V} \mapsto [f(p)]_\mathcal{V}. \]
\end{lemma}
\begin{proof}
We need to check the mapping of morphisms is well-defined. Then the functorial properties are immediate.

First we note that $*$-homomorphisms map projections to projections.

Let $[p]_\mathcal{V} = [q]_\mathcal{V}$. Then $p\sim_0 q$ and thus $\exists v\in A^{m\times n}$ such that $p=v^*v$ and $q = vv^*$. Thus
\[ f(p) = f(v^*v) = f(v)^*f(v) \qquad \text{and}\qquad f(q) = f(vv^*) = f(v)f(v)^*. \]
So $f(p) \sim_0 f(q)$ and thus $[f(p)]_\mathcal{V} = [f(q)]_\mathcal{V}$, meaning the mapping of morphisms is well-defined.
\end{proof}

\begin{definition}
Define a relation $\sim_s$ on $\Projections_\infty(A)$ as follows: for $p,q\in\Projections_\infty(A)$,
\[ p\sim_s q \defequiv \exists r\in\Projections_\infty:\quad p\oplus r \sim_0 q\oplus r. \]
The relation $\sim_s$ is called \udef{stable equivalence}.
\end{definition}

\begin{lemma} \label{stableEquivalence}
Let $A$ be unital. Then for all $p,q\in\Projections_\infty(A)$
\[ p\sim_s q \iff p\oplus \vec{1}_n \sim_0 q\oplus \vec{1}_n \]
for some integer $n$.
\end{lemma}
\begin{proof}
Assume $ p\sim_s q$, so $p\oplus r \sim_0 q\oplus r$ for some $r\in\Projections_n(A)$. By \ref{orthogonalProjection}, we know $(\vec{1}-r)$ is a projection. By the second point of \ref{sim0properties}, we have $p\oplus r \oplus (\vec{1}_n - r) \sim_0 q\oplus r \oplus (\vec{1}_n - r)$.

Now $r(\vec{1}_n - r) = r-r = 0$, so $r\oplus(\vec{1}_ - r) \sim_0 r\oplus(\vec{1}_n - r)$ by the fourth point of \ref{sim0properties}. By the second point
\[ p\oplus \vec{1}_n \sim_0 p\oplus r \oplus (\vec{1}_n - r) \sim_0 q\oplus r \oplus (\vec{1}_n - r) \sim_0 q\oplus \vec{1}_n. \]

The converse is immediate.
\end{proof}

\section{The $K_{00}$ functor}
\begin{definition}
Let $A$ be a $C^*$-algebra. Then we define the functor $K_{00}$ as the composition of two functors
\[ K_{00} = G \circ \mathcal{V}: \cat{C^*alg} \to \cat{Ab} \]
and let $[\cdot]_{0}$ be the mapping
\[ \begin{tikzcd}
\Projections_\infty(A) \rar["{[\cdot]_\mathcal{V}}"] & \mathcal{V}(A) \rar["g_0"] & K_{00}(A)
\end{tikzcd} \]
where $g_0$ is the Grothendieck map $x\mapsto (x,0)$.
\end{definition}
Note that $[\cdot]_{0}$ is not surjective and $g_0$ is not necessarily injective.

\begin{proposition}[The standard picture of $K_{00}$] \label{StandardPictureK00}
Let $A$ be a $C^*$-algebra, then
\begin{align*}
K_{00}(A) &= \setbuilder{[p]_0 - [q]_0}{p,q \in \Projections_\infty(A)} \\
&= \setbuilder{[p]_0 - [q]_0}{p,q \in \Projections_n(A), n\in \N}.
\end{align*}
Moreover,
\begin{enumerate}
\item $[p\oplus q]_0 = [p]_0+[q]_0$ for all projections $p,q\in\Projections_\infty(A)$;
\item $[0_A]_0 = 0$;
\item if $p,q\in\Projections_n(A)$ and $p\sim_h q \in \Projections_n(A)$, then $[p]_0 = [q]_0$.
\end{enumerate}
and
\begin{enumerate}
\setcounter{enumi}{3}
\item if $p,q\in\Projections_n(A)$ and $pq =0$, then $[p+q]_0 = [p]_0+[q]_0$;
\item for all $p,q\in\Projections_\infty(A)$, $[p]_0 = [q]_0 \iff p\sim_s q$.
\end{enumerate}
For $*$-homomorphisms $f,g: K_{00}(A)\to K_{00}(B)$:
\begin{enumerate}
\item $K_{00}(f)([p]_0) = [f(p)]_0$;
\item $\forall p\in\Projections_\infty(A): f([p]_0) = g([p]_0) \implies  f = g$;
\item If $f,g$ are orthogonal, i.e.\ for all $a\in A$: $f(a)\cdot g(a) = 0$, then
\[ K_{00}(f+g) = K_{00}(f)+K_{00}(g). \]
\end{enumerate}
\end{proposition}
\begin{proof}
The first equality is a property of the Grothendieck map. For the second equality, take $p\in\Projections_m(A)$ and $p\in\Projections_n(A)$, then $[p]_0 - [q]_0 = [p\oplus 0^{n\times n}]_0 - [q \oplus^{m\times m}]_0$.
Then
\begin{enumerate}
\item This follows because $[p\oplus q]_\mathcal{V} = [p]_\mathcal{V} + [q]_\mathcal{V}$ and the Grothendieck map is a homomorphism. TODO refs.
\item $[0_A]_0+[0_A]_0 = [0_A\oplus 0_A]_0 = [0_A]_0$, since $0_A\oplus 0_A \sim_0 0_A$.
\item By \ref{implicationsBeweenEquivalences},
\[ p\sim_h q \implies p\sim q \implies p\sim_0 q \implies [p]_\mathcal{V} = [q]_\mathcal{V} \implies [p]_0 = [q]_0. \]
\end{enumerate}
and
\begin{enumerate}
\setcounter{enumi}{3}
\item This is just point (4) of \ref{sim0properties} combined $[p\oplus q]_0 = [p]_0+[q]_0$.
\item If $p\sim_s q$, then $p\oplus r \sim_0 q\oplus r$, so $[p]_0 +[r]_0 = [q]_0 + [r]_0$ and $[p]_0 = [q]_0$ because $K_{00}(A)$ is a group.

Conversely, if $[p]_0 = [q]_0$, then there is an $[r]_\mathcal{V}$ such that $[p]_\mathcal{V} + [r]_\mathcal{V} = [q]_\mathcal{V} + [r]_\mathcal{V}$. Hence
\[ [p\oplus r]_\mathcal{V} = [q\oplus r]_\mathcal{V} \implies p\oplus r \sim_0 q\oplus r \implies p\sim_s q. \]
Note that we cannot conclude $p \sim_0 q$ from $[p]_0 = [q]_0$, because the Grothedieck map is not necessarily injective.
\end{enumerate}
For the morphisms,
\begin{enumerate}
\item $K_{00}(f)([p]_0) = G(\mathcal{V}(f))([p]_0) = \mathcal{V}(f)([p]_0) = [f(p)]_0$.
\item Assume $ \forall p\in\Projections_\infty(A): f([p]_0) = g([p]_0)$. Take an arbitrary $[p]_0-[q]_0\in K_{00}(A)$, then
\[ f([p]_0-[q]_0) = f([p]_0)-f([q]_0) = g([p]_0)-g([q]_0) = g([p]_0-[q]_0). \]
\item For all $p\in\Projections_\infty(A)$:
\begin{align*}
K_{00}(f + g)([p]_0) &= K_{00}([f(p) + g(p)]_0) = K_{00}([f(p)]_0 + [g(p)]_0) =  (K_{00}(f) + K_{00}(g))([p]_0)
\end{align*}
where we have used point $(4)$ of \ref{sim0properties}.
\end{enumerate}
\end{proof}


\begin{proposition}[Universal property of $K_{00}$]
Let $A$ be a $C^*$-algebra and $G$ an Abelian group. Suppose $\nu:\Projections_\infty(A)\to G$ is a function that satisfies
\begin{enumerate}
\item $\nu(p\oplus q) = \nu(p)+\nu(q)$ for all projections $p,q\in\Projections_\infty(A)$;
\item $\nu(0_A) = 0$;
\item if $p,q\in\Projections_n(A)$ and $p\sim_h q \in \Projections_n(A)$, then $\nu(p) = \nu(q)$.
\end{enumerate}
Then there is a unique group homomorphism $\alpha: K_{00}(A) \to G$ which makes the diagram
\[ \begin{tikzcd}
\Projections_\infty(A) \ar[d,"{[\cdot]_0}"] \ar[dr, "\nu"] & \\
K_{00}(A) \ar[r, dashed, swap, "\exists! \alpha"] & G
\end{tikzcd} \qquad \text{commute.}\]
\end{proposition}
In the third point we can also use $\sim_u, \sim_0$ or $\sim_s$:
\begin{proposition}
Let A be a $C^*$-algebra, $G$ an Abelian group, and
$\nu: \Projections_\infty(A) \to G$ a function that satisfies $\nu(0_A) = 0$ and $\nu(p\oplus q) = \nu(p) + \nu(q)$ for all projections $p,q\in\Projections_\infty(A)$. Then the following are equivalent:
\begin{itemize}
\item[$3.$] for all $n$ and all $p,q\in \Projections_n(A)$: if $p\sim_h q$, then $\nu(p) = \nu(q)$;
\item[$3'.$]for all $n$ and all $p,q\in \Projections_n(A)$: if $p\sim_u q$, then $\nu(p) = \nu(q)$;
\item[$3^{\prime\prime}.$] for all  $p,q\in \Projections_\infty(A)$: if $p\sim_0 q$, then $\nu(p) = \nu(q)$;
\item[$3^{\prime\prime\prime}.$] for all  $p,q\in \Projections_\infty(A)$: if $p\sim_s q$, then $\nu(p) = \nu(q)$;
\end{itemize}
\end{proposition}
\begin{proof}
We cyclically prove $(3^{\prime\prime\prime}) \Rightarrow (3^{\prime\prime}) \Rightarrow (3') \Rightarrow (3) \Rightarrow (3^{\prime\prime\prime})$:
\begin{enumerate}
\item[$\boxed{(3^{\prime\prime\prime}) \Rightarrow (3^{\prime\prime})}$] Assume $(3^{\prime\prime\prime})$ and take arbitrary $p,q\in\Projections_\infty(A)$ such that $p\sim_0 q$. Then $p\oplus 0 \sim_0 q\oplus 0$, so $p\sim_s q$ and $\nu(p)=\nu(q)$ by $(3^{\prime\prime\prime})$.
\item[$\boxed{(3^{\prime\prime}) \Rightarrow (3')}$] Assume $(3^{\prime\prime})$ and take arbitrary $p,q\in\Projections_n(A)$ for some $n$ such that $p\sim_u q$. Then $p\sim q$ by \ref{implicationsBeweenEquivalences} and thus $p\sim_0 q$, so $\nu(p)=\nu(q)$ by $(3^{\prime\prime})$.
\item[$\boxed{(3') \Rightarrow (3)}$] Assume $(3')$ and take arbitrary $p,q\in\Projections_n(A)$ for some $n$ such that $p\sim_h q$. Then $p\sim_u q$ by \ref{implicationsBeweenEquivalences}, so $\nu(p)=\nu(q)$ by $(3')$.
\item[$\boxed{(3) \Rightarrow (3^{\prime\prime\prime})}$] Assume $(3)$ and take arbitrary $p,q\in\Projections_\infty(A)$ such that $p\sim_s q$. Then there exists an $r\in\Projections_\infty(A)$ such that $p\oplus r \sim_0 q\oplus r$. If $p\oplus r\in\Projections_m$ and $q\oplus r\in\Projections_n$, then
\[ p \oplus r\oplus 0^{n\times n} \sim_0 p\oplus r \sim_0 q\oplus r \sim_0 q\oplus r\oplus 0^{m\times m}. \]
And since both sides are in $\Projections_{n+m}(A)$, we have $p \oplus r\oplus 0^{n\times n} \sim q\oplus r\oplus 0^{m\times m}$. By \ref{implicationsBeweenEquivalences} this implies
\[ p \oplus r\oplus 0^{n\times n}\oplus 0^{(m+n)\times(m+n)} \sim_h q\oplus r\oplus 0^{m\times m}\oplus 0^{(m+n)\times(m+n)}. \]
Using $(3)$ this gives
\[ \nu(p) + \nu(r) = \nu(p \oplus r\oplus 0^{n\times n}\oplus 0^{(m+n)\times(m+n)}) = \nu(q\oplus r\oplus 0^{m\times m}\oplus 0^{(m+n)\times(m+n)}) = \nu(q) +\nu(r) \]
which implies $\nu(p) = \nu(q)$.
\end{enumerate}
\end{proof}


\begin{proposition}[Homotopy invariance of $K_{00}$] \label{homotopyInvarianceK00}
Let $A,B$ be $C^*$-algebras. If $\varphi,\psi:A\to B$ are homotopic $*$-homomorphisms, then 
\[ K_{00}(\varphi) = K_{00}(\psi). \]
\end{proposition}
\begin{proof}
For every $p\in\Projections_\infty(A)$, $\varphi(p)\sim_h \psi(p)$. So
\[ K_{00}(\varphi)([p]_0) = [\varphi(p)]_0 = [\psi(p)]_0 = K_{00}(\psi)([p]_0), \]
meaning $K_{00}(\varphi) = K_{00}(\psi)$
\end{proof}
\begin{corollary}
If $A$ and $B$ are homotopy equivalent, then $K_{00}(A)\cong K_{00}(B)$.
\end{corollary}
\begin{proof}
If $\psi\circ\varphi \sim_h I_A$, then
\[ K_{00}(\psi)\circ K_{00}(\varphi) = K_{00}(I_A) = I_{K_{00}(A)}. \]
Similarly $\varphi\circ\psi \sim_h I_B$ implies
\[ K_{00}(\varphi)\circ K_{00}(\psi) = K_{00}(I_B) = I_{K_{00}(B)}. \]
So $K_{00}(\varphi): A\to B$ is invertible with inverse $K_{00}(\psi)$.
\end{proof}

\begin{proposition}
The functor $K_{00}$ is not half exact for non-unital $C^*$-algebras.
\end{proposition}
This is essentially the motivation to work with $K_0$, not $K_{00}$.

\section{The $K_{0}$ functor}
\begin{definition}
Let $A$ be a $C^*$-algebra and $\pi: A^\dagger \to \C$ the projection of the second component of $A^\dagger$onto $\C$. Then we define $K_0(A)$ as the kernel of $K_{00}(\pi): K_{00}(A^\dagger)\to K_{00}(\C)$.
\[ \begin{tikzcd}
0 \rar & A \rar[hook, "\iota"] & A^\dagger \rar[shift left, "\pi"] & \lar[hook, shift left, "\lambda"] \C \rar & 0
\end{tikzcd} \]
\end{definition}
\begin{proposition} \label{K00embedsIntoK0}
Let $A$ be a $C^*$-algebra, then $K_{00}(\iota)$ in an embedding $K_{00}(\iota): K_{00}(A) \hookrightarrow K_{0}(A)$.
\end{proposition}
\begin{proof}
To show injectivity, it is enough to note that $\iota$ is split monic. Then $K_{00}(\iota)$ is also split monic in the category $\cat{Ab}$ and thus injective.

To show the image is a subset of $K_0(A)$, take some arbitrary $[p]_0-[q]_0\in K_{00}(A)$ with  $p,q\in\Projections_\infty(A)$. Then we claim 
\[K_{00}(\iota)([p]_0-[q]_0) = K_{00}(\iota)([p]_0)-K_{00}(\iota)([q]_0) =  [\iota(p)]_0-[\iota(q)]_0\]
maps to zero under $K_{00}(\pi)$. Indeed:
\[ K_{00}(\pi)([\iota(p)]_0-[\iota(q)]_0) = K_{00}(\pi)([\iota(p)]_0) - K_{00}(\pi)([\iota(q)]_0) = [\pi(\iota(p))]_0 - [\pi(\iota(q))]_0 = 0 \]
using that fact that $\pi\circ\iota = 0$. So $K_{00}(\iota)([p]_0-[q]_0)\in\ker(K_{00}(\pi)) = K_0(A)$.
\end{proof}
So we can identify $K_{00}(A) \cong \im K_{00}(\iota) \subseteq K_0(A)$ and we can naturally extend $[\cdot]_{0}$ to a function
\[ \Projections_\infty(A)\to K_{00}(A) \hookrightarrow K_0(A). \]

\begin{proposition}
Let $A$ be a unital $C^*$-algebra. Then $K_{00}(A) \cong K_0(A)$.
\end{proposition}
\begin{proof}
By \ref{K00embedsIntoK0} we have $K_{00}(\iota): K_{00}(A) \hookrightarrow K_{0}(A)$. We just need to show $K_{00}(\iota)$ is surjective. To do this we are going to decompose $I_{A^\dagger}$ into the form
\[ I_{A^\dagger} = \iota\circ \mu + \mu'\circ \pi \]
with the property that $\iota\circ \mu \cdot \mu'\circ \pi$ is the zero map, i.e.\ $\iota\circ \mu$ and $\mu'\circ \pi$ are orthogonal. Such a decomposition is given by
\[ \mu: A^\dagger \to A: (a,\alpha) \mapsto a+\alpha \qquad \text{and}\qquad \mu': \C \to A^\dagger: \alpha \mapsto (-\alpha, \alpha),\]
which works because $A$ is unital.

We verify
\begin{align*}
&(\iota\circ \mu)(a,\alpha) + (\mu'\circ \pi)(a,\alpha) = (a+\alpha, 0) + (-\alpha, a\alpha) = (a,\alpha) \\
&(\iota\circ \mu)(a,\alpha) \cdot (\mu'\circ \pi)(a,\alpha) = (a+\alpha,0) \cdot (-\alpha, \alpha) = (-\alpha(a+\alpha) + \alpha(a+\alpha) + 0,0) = (0,0).
\end{align*}

Then take some $s\in K_0(A) = \ker(K_{00}(\pi))$. We calculate
\begin{align*}
s &= I_{K_{00}(A^\dagger)}(s) = K_{00}(I_{A^\dagger})(s) = K_{00}(\iota\circ \mu + \mu'\circ \pi)(s) \\
&= K_{00}(\iota\circ \mu)(s) + K_{00}(\mu'\circ \pi)(s) = (K_{00}(\iota)\circ K_{00}(\mu))(s) + (K_{00}(\mu')\circ K_{00}(\pi))(s) \\
&= K_{00}(\iota)\circ K_{00}(\mu))(s) + 0 \in \im K_{00}(\iota).
\end{align*}
\end{proof}
For this reason $K_0(A)$ is often defined as $K_{00}(A)$ for unital algebras.

\begin{lemma}
Let $A,B$ be $C^*$-algebras. For every morphism $f: A\to B$, we can view $K_{00}(f)$ as a morphism on a subgroup of $K_0(A)$ by identifying it with
\[ K_{00}(\iota)K_{00}(f)K_{00}(\iota)^{-1}. \]
Then $K_{00}(f)$ can uniquely be extended to a morphism $K_0(f): K_0(A)\to K_0(B)$. 

This makes $K_0$ a functor.
\end{lemma}
\begin{proof}
Consider the diagram
\[ \begin{tikzcd}
K_{00}(A) \rar{K_{00}(\iota_A)} \dar{K_{00}(f)} & K_0(A) \rar{\subseteq}\dar[dashed]{K_0(f)} & K_{00}(A^\dagger) \rar{K_{00}(\pi_A)}\dar{K_{00}(f^\dagger)} & K_{00}(\C) \ar[d, equals] \\
K_{00}(B) \rar[swap]{K_{00}(\iota_B)} & K_0(B) \rar[swap]{\subseteq} & K_{00}(B^\dagger) \rar[swap]{K_{00}(\pi_B)} & K_{00}(\C)
\end{tikzcd} \]
which is commutative because functors preserve commutative diagrams. Uniqueness is immediate from the commutativity of the middle square. To show existence, we must show that
\[ K_{00}(f^\dagger)[K_0(A)] \subseteq K_0(B). \]
Take $r \in K_0(A) = \ker K_{00}(\pi_A)$. Then using the fact that $f^\dagger$ commutes with $\pi$, i.e.\ 
\[ \pi_B \circ f^\dagger = f^\dagger\circ \pi_A,\]
we get 
\[ (K_{00}(\pi_B)\circ K_{00}(f^\dagger))(r) = (K_{00}(f^\dagger)\circ K_{00}(\pi_A))(r) = K_{00}(f^\dagger)(0) = 0. \]
So $K_{00}(f^\dagger)(r) \in \ker K_{00}(\pi_B) = K_0(B)$.
\end{proof}

\begin{proposition}[The standard picture of $K_{0}$] \label{StandardPictureK0}
Let $A$ be a $C^*$-algebra, then
\begin{align*}
K_{0}(A) &= \setbuilder{[p]_0 - [s(p)]_0}{p \in \Projections_\infty(A^\dagger)}.
\end{align*}
Moreover, for all $p,q\in\Projections_\infty(A^\dagger)$, the following are equivalent:
\begin{enumerate}
\item $[p]_0-[s(p)]_0 = [q]_0 - [s(q)]_0$,
\item $p\oplus \vec{1}_k \sim_0 q \oplus \vec{1}_l$ in $\Projections_\infty(A^\dagger)$ for some $k,l\in \N$,
\item there exist scalar projections $r_1, r_2$ such that $p\oplus r_1 \sim_0 q \oplus r_2$ in $\Projections_\infty(A^\dagger)$.
\end{enumerate}
If $\varphi:A\to B$ is a $*$-homomorphism, then
\[ K_0(\varphi)([p]_0 - [s(p)]_0) = [\varphi^\dagger(p)]_0 - [s(\varphi^\dagger(p))]_0 \]
for all $p\in \Projections_\infty(A^\dagger)$.
\end{proposition}
\begin{proof}
For all $p\in\Projections_\infty(A^\dagger)$, $[p]_0 - [s(p)]_0$ is in $K_0(A) = \ker K_{00}(\pi)$:
\[ K_{00}(\pi)([p]_0 - [s(p)]_0) = [\pi(p)]_0 - [(\pi\circ s)(p)]_0 = [\pi(p)]_0 - [\pi(p)]_0 = 0. \]
Conversely, let $g\in K_0(A)$, then by the standard picture of $K_{00}(A^\dagger)$ there are $e,f\in \Projections((A^\dagger)^{n\times n})$ such that $g = [e]_0 - [f]_0$. Put
\[ p = \begin{pmatrix}
e & 0 \\ 0 & \vec{1}_n - f
\end{pmatrix}, \qquad q = \begin{pmatrix}
0 & 0 \\ 0 & \vec{1}_n
\end{pmatrix}. \]
Then $(\vec{1}_n - f)$ and $f$ are orthogonal,
\[ (\vec{1}_n - f)f = f - f^2 = f-f = 0. \]
So, by \ref{StandardPictureK00}, $[\vec{1}_n]_0 = [\vec{1}_n -f + f]_0 = [\vec{1}_n -f]_0 + [f]_0$ and we get
\[ [p]_0 - [q]_0 = [e]_0 + [\vec{1}_n - f]_0 - [\vec{1}_n]_0 = [e]_0 - [f]_0 = g. \]
Using $q = s(q)$ and $K_{00}(\pi)(g) = 0$, we get
\[ [s(p)]_0 - [q]_0 = [s(p)]_0 - [s(q)]_0 = K_{00}(s)(g) = (K_{00}(\lambda)\circ K_{00}(\pi))(g) = 0. \]
So $[s(p)]_0 = [q]_0$ and we get $g = [p]_0 - [s(p)]_0$.

We prove the equivalent statements cyclically:
\begin{itemize}
\item[$\boxed{(1) \Rightarrow (3)}$] Suppose $[p]_0-[s(p)]_0 = [q]_0 - [s(q)]_0$ for some $p,q\in\Projections_\infty(A^\dagger)$ this implies
\[ [p\oplus s(q)]_0 = [q\oplus s(p)]_0 \implies p\oplus s(q) \sim_s q\oplus s(p) \qquad \text{in $\Projections_\infty(A^\dagger)$} \]
by \ref{StandardPictureK00}. By \ref{stableEquivalence} this implies $p\oplus s(q)\oplus \vec{1}_n \sim_0 q \oplus s(p) \oplus \vec{1}_n$. Putting $r_1 = s(q)\oplus \vec{1}_n$ and $r_2 = s(p) \oplus \vec{1}_n$, which are scalar projections, this is exactly $(3)$.
\item[$\boxed{(3) \Rightarrow (2)}$] If $r_1$ is a scalar projection in $\Projections_k(A)$ and $r_2$ in $\Projections_l(A)$, then $r_1\sim_0 \vec{1}_k$ and $r_2 \sim_0 \vec{1}_l$, by TODO ref. Hence $p\oplus \vec{1}_k \sim_0 q \oplus \vec{1}_l$.
\item[$\boxed{(2) \Rightarrow (1)}$] 
\end{itemize}
\end{proof}

\begin{proposition}[Half exactness of $K_0$]
Every short exact sequence of $C^*$-algebras
\[ \begin{tikzcd}
0 \rar & I \rar{\varphi} & A \rar{\psi} & B \rar & 0
\end{tikzcd} \]
induces an exact sequence of Abelian groups
\[ \begin{tikzcd}
K_0(I) \rar{K_0(\varphi)} & K_0(A) \rar{K_0(\psi)} & K_0(B).
\end{tikzcd} \]
\end{proposition}

\begin{proposition}
Every split exact sequence of $C^*$-algebras
\[ \begin{tikzcd}
0 \rar & I \rar{\varphi} & A \rar[shift left]{\psi} & B \rar \lar[shift left]{\lambda} & 0
\end{tikzcd} \]
induces a split exact sequence of Abelian groups
\[ \begin{tikzcd}
0 \rar & K_0(I) \rar{K_0(\varphi)} & K_0(A) \rar[shift left]{K_0(\psi)} & K_0(B) \rar \lar[shift left]{K_0(\lambda)} & 0
\end{tikzcd} \]
\end{proposition}

\begin{proposition}
For every pair $A,B$ of $C^*$-algebras,
\[ K_0(A\oplus B) \cong K_0(A)\oplus K_0(B).  \]
\end{proposition}

\subsection{Homotopy, suspensions and cones}
\begin{proposition}[Homotopy invariance of $K_0$]
Let $A,B$ be $C^*$-algebras. If $\varphi,\psi:A\to B$ are homotopic $*$-homomorphisms, then 
\[ K_{0}(\varphi) = K_{0}(\psi). \]
If $A$ and $B$ are homotopy equivalent, then $K_{0}(A)\cong K_{0}(B)$.
\end{proposition}
\begin{proof}
If $\varphi$ is homotopic to $\psi$, then $\varphi^\dagger$ is homotopic to $\psi^\dagger$ and thus $K_{00}(\varphi^\dagger) = K_{00}(\psi^\dagger)$, by \ref{homotopyInvarianceK00}. Restricting to $K_0(A)$ yields the result.
\end{proof}
In particular, $K_0(A) = 0$ for every contractible $C^*$-algebra $A$.

\begin{definition}
Let $A$ be a $C^*$-algebra. The \udef{cone} over $A$ is
\[ CA \defeq \setbuilder{f\in C([0,1], A)}{f(0) = 0}. \]
The \udef{suspension} of $A$ is
\[ SA \defeq \setbuilder{f\in C([0,1], A)}{f(0) = f(1) = 0}. \]
\end{definition}

\begin{lemma}
If the operations are pointwise and the norm the supremum norm, then $CA$ and $SA$ are $C^*$-algebras.
\end{lemma}

\begin{lemma}
Let $A$ be a $C^*$-algebra. The cone $CA$ is contractible. The suspension $SA$ is contractible if $A$ is contractible.
\end{lemma}
\begin{proof}
Let $\gamma_t: CA\to CA$ be defined by $\gamma_t(f)(s) = f(st)$ for all $t\in [0,1]$. Then $\gamma$ defines a contraction of $CA$.

For any contraction $\beta_t: A\to A$ of $A$, $\gamma_t: SA \to SA: f\mapsto \beta_t\circ f$ is a contraction of $SA$.
\end{proof}

\begin{lemma} \label{exactSequenceSuspensionCone}
Let $A$ be a $C^*$-algebra. Then $SA$ is a closed ideal of $CA$ and $A \cong CA/SA$.

We have the short exact sequence
\[ \begin{tikzcd}
 0 \rar & SA \rar[hook, "\iota"] & CA \rar & A \rar & 0.
\end{tikzcd} \]
\end{lemma}
\begin{proof}
Consider the map
\[ CA \to A: f\mapsto f(1). \]
This is a surjective morphism with kernel $SA$. So $SA$ is a closed ideal by ref TODO.
\end{proof}


\begin{definition}
Let $A,B$ be $C^*$-algebras and $\alpha: A\to B$ a morphism. The \udef{mapping cone} for $\alpha$ is
\[ \begin{tikzcd}
C_\alpha \defeq \setbuilder{(a, f) \in A\oplus CB}{f(1) = \alpha(a)}.
\end{tikzcd} \]
\end{definition}
The mapping cone is a $C^*$-algebra.

\begin{lemma}
The mapping cone $\alpha$ is related to $A$ and $B$ in the short exact sequence
\[ \begin{tikzcd}
0 \rar & SB \rar{\iota: f\mapsto (0,f)} & C_\alpha \rar{\pi: (a,f) \mapsto a} & A \rar & 0.
\end{tikzcd} \]
Moreover, the sequence
\[ \begin{tikzcd}
K_0(C_\alpha) \rar{\pi_*} & K_0(A) \rar{\alpha_*} & K_0(B)
\end{tikzcd} \]
is exact.
\end{lemma}
\begin{proof}
TODO
\end{proof}


\section{The $K_1$ functor}
\begin{definition}
As with the projections, we define, for some unital $C^*$-algebra,
\[ \Unitaries_n(A) = \Unitaries(A^{n\times n}), \qquad \Unitaries_\infty(A) = \bigcup^\infty_{n=1}\Unitaries_n(A) \]
as well as the binary operations $\oplus$ on $\Unitaries_\infty(A)$
\[ \forall u,v \in \Unitaries_\infty(A): \quad u\oplus v = \diag(u,v) = \begin{pmatrix}
u & 0 \\ 0 & v
\end{pmatrix}. \]
\end{definition}
\begin{lemma}
Let $A$ be a unital $C^*$-algebra. If $u\in\Unitaries_n(A)$ and $v\in\Unitaries_m(A)$, then $u\oplus v\in\Unitaries_{n+m}(A)$.
\end{lemma}
\subsection{Normalised matrices}
\begin{definition}
The group of \udef{normalised invertible matrices} is
\[ \GL_n^\dagger(A) \defeq \setbuilder{a\in \GL_n(A^\dagger)}{\pi(a) = \vec{1}_n} \]
and the group of \udef{normalised unitary matrices} is
\[ \Unitaries_n^\dagger(A) \defeq \setbuilder{u\in \Unitaries_n(A^\dagger)}{\pi(u) = \vec{1}_n}. \]
The group operation for both is multiplication. We also define
\[ \GL_\infty^\dagger(A) \defeq \bigcup_{n=1}^\infty\GL_n^\dagger(A) \qquad \text{and}\qquad \Unitaries_\infty^\dagger(A) \defeq \bigcup_{n=1}^\infty\Unitaries_n^\dagger(A). \]
\end{definition}

\begin{proposition} \label{normalisedQuotientsIsomorphisms}
Let $A$ be a $C^*$-algebra and $n\in \N\cup\{\infty\}$. Then the groups
\begin{align*}
\GL_n^\dagger(A) / \GL_n^\dagger(A)_0,& & &\Unitaries_n^\dagger(A) / \Unitaries_n^\dagger(A)_0 \\
\GL_n(A^\dagger) / \GL_n(A^\dagger)_0,& & &\Unitaries_n(A^\dagger) / \Unitaries_n(A^\dagger)_0
\end{align*}
are pairwise isomorphic. If $A$ is unital, then 
\[ \GL_n^\dagger(A) \cong \GL_n(A) \qquad \text{and} \qquad \Unitaries_n^\dagger(A) \cong \Unitaries_n(A). \]
\end{proposition}
\begin{proof}
First note that by \ref{sectionConnectedToIdentity}, all quotients are normal subgroups.

Consider the continuous map
\[ \phi: \GL_n^\dagger(A) \to \Unitaries_n^\dagger(A): z\mapsto z|z|^{-1}. \]
The induced map
\[ \psi: \GL_n^\dagger(A) / \GL_n^\dagger(A)_0 \to \Unitaries_n^\dagger(A) / \Unitaries_n^\dagger(A)_0: [z] \mapsto [\phi(z)] \]
is a group homomorphism by (TODO ref) and is bijective: it is clearly surjective. For injectivity, we prove the kernel is trivial.
Indeed let $[\phi(z)] = \vec{1}$, then $z|z|^{-1} \sim_h \vec{1}_n$. By \ref{unitariesRetractionOfGL}, $z|z|^{-1}\sim_h z$, so $z\sim_h \vec{1}_n$ by transitivity.

To prove $\GL_n(A^\dagger) / \GL_n(A^\dagger)_0 \cong \GL_n^\dagger(A) / \GL_n^\dagger(A)_0$, consider the isomorphism $[z] \mapsto [z\pi(z^{-1})]$.

Restricting to unitaries gives the last isomorphism.
\end{proof}

\subsection{Equivalence of unitaries}
\begin{definition}
We define a relation $\sim_1$ on $\Unitaries_\infty(A)$ as follows: for $u\in \Unitaries_n(A)$ and $v\in \Unitaries_m(A)$,
\[ u \sim_1 v \defequiv \exists k\geq \max\{m,n\}: \quad u\oplus \vec{1}_{k-n} \sim_h v\oplus \vec{1}_{k-m} \quad\text{in $\Unitaries_{k}(A)$}. \]
With the convention that $w\oplus 1_0 = w$ for all $w\in\Unitaries_\infty(A)$.
\end{definition}
\begin{lemma}
Let $A$ be a unital $C^*$-algebra. Then for all $u,v,u',v'\in \Unitaries(A)$
\begin{enumerate}
\item $\sim_1$ is an equivalence relation on $\Unitaries_\infty(A)$;
\item $u \sim_1 u\oplus \vec{1}_k$
\item $u\oplus v \sim_1 v\oplus u$;
\item if $u\sim_1 u'$ and $v\sim_1 v'$, then $u\oplus v \sim_1 u'\oplus v'$;
\item if $u,v\in \Unitaries_n(A)$, then $uv\sim_1 vu \sim_1 u\oplus v$.
\end{enumerate}
\end{lemma}

\subsection{The $K_1$ functor}
\begin{definition}
For each $C^*$-algebra $A$, we define
\[ K_1(A) = \Unitaries_\infty(A^\dagger) / \sim_1. \]
Define the binary operation $+$ on $K_1(A)$ by $[u]_1 + [v]_1 = [u\oplus v]_1$.
\end{definition}
Because
\[ [u]_1 + [v]_1 = [u\oplus v]_1 = [uv]_1, \]
the $K_1(A)$ group is naturally a multiplicative group that we are writing in additive notation for uniformity with other $K$ groups.
Notice in particular that
\[ [\vec{1}]_1 = 0. \]

In fact we could have defined
\[ [u]_1 + [v]_1 = [uv]_1, \]
so that there is less dependence on matrices. This is different than for projections where the definition
\[ [p]_0 + [q]_0 = [p+q]_0 \]
did not work because $p+q$ was not necessarily a projection.

\begin{proposition}
Let $A$ be a $C^*$-algebra. Then $K_1(A)$ is isomorphic to any of the following:
\begin{align*}
\GL_\infty^\dagger(A) / \GL_\infty^\dagger(A)_0,& & &\Unitaries_\infty^\dagger(A) / \Unitaries_\infty^\dagger(A)_0 \\
\GL_\infty(A^\dagger) / \GL_\infty(A^\dagger)_0,& & &\Unitaries_\infty(A^\dagger) / \Unitaries_\infty(A^\dagger)_0.
\end{align*}
\end{proposition}
\begin{proof}
The four groups are isomorphic by \ref{normalisedQuotientsIsomorphisms}. Consider $\Unitaries_\infty(A^\dagger) / \Unitaries_\infty(A^\dagger)_0$. We just need to see that $\forall u\in \Unitaries_\infty(A^\dagger)$
\[ u\sim_1 \vec{1} \iff u \in \Unitaries_\infty(A^\dagger)_0\]
by TODO ref. This is clear.
\end{proof}

\begin{proposition}[Universal property of $K_{1}$] \label{univeralPropertyK1}
Let $A$ be a $C^*$-algebra and $G$ an Abelian group. Suppose $\nu:\Unitaries_\infty(A^\dagger)\to G$ is a function that satisfies
\begin{enumerate}
\item $\nu(u\oplus v) = \nu(u)+\nu(v)$ for all unitaries $u,v\in\Unitaries_\infty(A^\dagger)$;
\item $\nu(\vec{1}_A) = 0$;
\item if $u,v\in\Unitaries_n(A^\dagger)$ and $u\sim_h v \in \Unitaries_n(A^\dagger)$, then $\nu(p) = \nu(q)$.
\end{enumerate}
Then there is a unique group homomorphism $\alpha: K_{1}(A) \to G$ which makes the diagram
\[ \begin{tikzcd}
\Unitaries_\infty(A^\dagger) \ar[d,"{[\cdot]_1}"] \ar[dr, "\nu"] & \\
K_{1}(A) \ar[r, dashed, swap, "\exists! \alpha"] & G
\end{tikzcd} \qquad \text{commute.}\]
\end{proposition}

The functor $K_1$ is homotopy invariant, half exact, split exact and respects direct sums.

\section{Exact sequences of $K$-groups}
\subsection{Suspensions}
\begin{lemma}
Let $A$ be a $C^*$-algebra. We have the isomorphisms
\begin{align*}
SA &\cong A\otimes C_0(\R) \\ &\cong C_0(\R, A) \\ &\cong C_0(]0,1[, A) \\ &\cong \setbuilder{f\in C(\mathbb{T}, A)}{f(1)=0}.
\end{align*}
\end{lemma}

\begin{lemma}
Suspension is a functor $S: \cat{C^*alg} \to \cat{C^*alg}$.
\end{lemma}
\begin{proof}
We know the suspension maps $C^*$-algebras to $C^*$-algebras. Let $f: A\to B$ be a $*$-homomorphism. Then $f_*: SA \to SB$ is well-defined and a $*$-homomorphism. The functorial properties are clearly satisfied.
\end{proof}

\begin{theorem}
The functors $K_1$ and $K_0\circ S$ are naturally isomorphic.
\end{theorem}
\begin{proof}
For every $C^*$-algebra $A$ we define
\end{proof}


\subsection{The index map}
Suppose a short exact sequence of $C^*$-algebras
\[  \begin{tikzcd}
0 \rar & I \rar{\varphi} & A \rar{\psi} & B\rar & 0
\end{tikzcd}.\]
Then we want to define a map $\delta_1:  K_1(B)\to K_0(I)$, called the \udef{index map}, such that the sequence
\[ \begin{tikzcd}
K_1(I) \rar{K_1(\varphi)} & K_1(A) \rar{K_1(\psi)} & K_1(B) \dar{\delta_1} \\
K_0(B) & K_0(A) \lar{K_0(\psi)} & K_0(I) \lar{K_0(\varphi)}
\end{tikzcd} \]
is exact.

\subsubsection{Constructing the index map}
Take an element $[u]\in K_1(B) = \Unitaries_\infty(B^\dagger) / \Unitaries_\infty(B^\dagger)_0$. Now the elements of $u\cdot \Unitaries_\infty(B^\dagger)_0$ do not in general lift to unitaries in $A^\dagger$.

We wish to measure to what degree this lifting is not possible. We expect such a map to be well-defined, because the elements of $\Unitaries_\infty(B^\dagger)_0$ should not impede the lifting, by \ref{unitaryLifting}.



To do this, we define a function $\nu': \Unitaries_\infty(B^\dagger)\to \Projections(I^\dagger)$ such that $\nu \defeq [\cdot]_0\circ \nu': \Unitaries_\infty(B^\dagger)\to K_0(I)$ satisfies the universal property of the $K_1$ functor, \ref{univeralPropertyK1}, meaning it uniquely factors through $K_1(B)$, giving a group homomorphism $\delta_1: K_1(B)\to K_0(I)$ satisfying $\delta_1([u]_1) = \nu(u)$ for each $u\in\Unitaries_\infty(B^\dagger)$.


We define the map $\nu': \Unitaries_\infty(B^\dagger)\to \Projections(I^\dagger)$ as follows:

Take $u\in \Unitaries_\infty(B^\dagger)$. First we would like to lift this to a unitary 

\begin{lemma}
Suppose a short exact sequence of $C^*$-algebras
\[  \begin{tikzcd}
0 \rar & I \rar{\varphi} & A \rar{\psi} & B\rar & 0
\end{tikzcd}\]
and let $u\in \Unitaries_n(B^\dagger)$.
\begin{enumerate}
\item There exists a unitary $v\in\Unitaries_{2n}(A^\dagger)$ and a projection $p\in\Projections_{2n}(I^\dagger)$ such that
\[ \psi^\dagger(v) = \begin{pmatrix}
u & 0 \\ 0 & u^*
\end{pmatrix}, \qquad \varphi^\dagger(p) = v \begin{pmatrix}
\vec{1}_n & 0 \\ 0 & 0
\end{pmatrix}v^*, \qquad s(p) = \begin{pmatrix}
\vec{1}_n & 0 \\ 0 & 0
\end{pmatrix}. \]
\item Given these $v,p$, if $w\in\Unitaries_{2n}(A^\dagger)$ and $q\in\Projections_{2n}(I^\dagger)$ satisfy
\[ \psi^\dagger(w) = \begin{pmatrix}
u & 0 \\ 0 & u^*
\end{pmatrix}, \qquad \varphi^\dagger(q) = w \begin{pmatrix}
\vec{1}_n & 0 \\ 0 & 0
\end{pmatrix}w^*, \]
then $s(q) = \diag(\vec{1}_n, 0_n)$ and $p\sim_u q$ in $\Projections_{2n}(I^\dagger)$.
\end{enumerate}
\end{lemma}
\begin{proof}
(1). Because $\diag(u,u^*)\sim_h \diag(1,1)$, we can use the first point of \ref{unitaryLifting} to see that $\diag(u,u^*)$ lifts to a unitary $v\in (A^\dagger)^{2n\times 2n}$, giving the first equation. Also
\[ \psi^\dagger( v \begin{pmatrix}
\vec{1}_n & 0 \\ 0 & 0
\end{pmatrix}v^* ) = \psi^\dagger(v) \begin{pmatrix}
\vec{1}_n & 0 \\ 0 & 0
\end{pmatrix} \psi^\dagger(v^*) = \begin{pmatrix}
u & 0 \\ 0 & u^*
\end{pmatrix}\begin{pmatrix}
\vec{1}_n & 0 \\ 0 & 0
\end{pmatrix}\begin{pmatrix}
u^* & 0 \\ 0 & u
\end{pmatrix} = \begin{pmatrix}
\vec{1}_n & 0 \\ 0 & 0
\end{pmatrix}. \]
Letting $\pi_1: A^\dagger \to A$ be the projection as in \ref{projectionOnACommutes}, this means
\[ \pi_1(\psi^\dagger( v \begin{pmatrix}
\vec{1}_n & 0 \\ 0 & 0
\end{pmatrix}v^* )) = \psi(\pi_1( v \begin{pmatrix}
\vec{1}_n & 0 \\ 0 & 0
\end{pmatrix}v^* )) = 0. \]
By the exactness of the sequence, 
\[ \pi_1( v \begin{pmatrix}
\vec{1}_n & 0 \\ 0 & 0
\end{pmatrix}v^* ) \in \im (\varphi), \]
meaning there is a $p\in (I^\dagger)^{2n\times 2n}$ such that $\varphi^\dagger(p) = v\diag(\vec{1}_n,0)v^*$ and $p$ is a projection by \ref{injectiveLifts}.
For the third equality, $s(p) = \psi^\dagger(\varphi^\dagger(p)) = \diag(\vec{1}_n, 0)$.

(2). That $s(q) = \diag(\vec{1}_n, 0)$ follows from $\psi^\dagger(\varphi^\dagger(p)) = \diag(\vec{1}_n, 0)$ as before.

Then $\psi^\dagger(wv^*) = \vec{1}_{2n}$, so $\psi(\pi_1(wv^*)) = 0$ and $\pi_1(wv^*)\in \im\varphi$ by exactness. So we can find a $z\in (I^\dagger)^{2n\times 2n}$ such that $\varphi^\dagger(z) = wv^*$ and $z$ is unitary by \ref{injectiveLifts}. From $\varphi^\dagger(zpz^*) = \varphi^\dagger(q)$ and the injectivity of $\varphi^\dagger$, we get $q = zpz^*$, meaning $p \sim_u q$ in $\Projections_{2n}(I^\dagger)$.
\end{proof}
We use this lemma to define a function
\[ \nu: \Unitaries_\infty(B^\dagger)\to K_0(I) \]
which maps $u\in\Unitaries_\infty(B^\dagger)$ to $\nu(u) = [p]_0 - [s(p)]_0$ where $p\in\Projections_{2n}(I^\dagger)$ is as in the lemma. This map is well-defined by the lemma.

\begin{lemma}
The map $\nu: \Unitaries_\infty(B^\dagger)\to K_0(I)$ satisfies the universal property of $K_1(B)$:
\begin{enumerate}
\item $\nu(u_1\oplus u_2) = \nu(u_1)+\nu(u_2)$ for all unitaries $u_1,u_2\in\Unitaries_\infty(B^\dagger)$;
\item $\nu(\vec{1}) = 0$;
\item if $u_1,u_2\in\Unitaries_n(B^\dagger)$ and $u_1\sim_h u_2 \in \Unitaries_n(B^\dagger)$, then $\nu(u_1) = \nu(u_2)$.
\end{enumerate}
\end{lemma}
\begin{proof}
(1). For $j=1,2$, let $u_j$ be given. Choose $v_j \in \Unitaries_{2n_j}(A^\dagger)$ and $p_j \in \Projections_{2n_j}(I^\dagger)$ as in the definition of the index map, i.e.\ $\nu(u_j) = [p_j]_0 - [s(p_j)]_0$. Then introduce
\[ y = \begin{pmatrix}
\vec{1}_{n_1} & 0 & 0 & 0 \\
0 & 0 & \vec{1}_{n_2} & 0 \\
0 & \vec{1}_{n_1} & 0 & 0 \\
0 & 0 & 0 & \vec{1}_{n_2}
\end{pmatrix}\in \Unitaries_{2(n_1+n_2)}(\C) \]
\end{proof}

Because the map $\nu$ satisfies the universal property of the $K_1$ functor, \ref{univeralPropertyK1}, it uniquely factors throught $K_1(B)$, giving a group homomorphism $\delta_1: K_1(B)\to K_0(I)$ satisfying $\delta_1([u]_1) = \nu(u)$ for each $u\in\Unitaries_\infty(B^\dagger)$.

The map $\delta_1$ is called the \udef{index map} associated with the short exact sequence.

\subsubsection{Properties of the index map}
With the index map we have the exact sequence:
\[ \begin{tikzcd}
K_1(I) \rar{K_1(\varphi)} & K_1(A) \rar{K_1(\psi)} & K_1(B) \dar{\delta_1} \\
K_0(B) & K_0(A) \lar{K_0(\psi)} & K_0(I) \lar{K_0(\varphi)}
\end{tikzcd} \]

\begin{proposition}[Naturality of the index map] \label{naturalityIndexMap}
Let
\[ \begin{tikzcd}
0 \rar & I \dar{\gamma} \rar{\varphi} & A \dar{\alpha} \rar{\psi} & B \dar{\beta} \rar & 0 \\
0 \rar & I' \rar{\varphi'} & A' \rar{\psi'} & B' \rar & 0
\end{tikzcd} \]
be a commutative diagram of short exact rows of $C^*$-algebras. Let
\[ \delta_1: K_1(B) \to K_0(I) \qquad \text{and} \qquad \delta'_1: K_1(B') \to K_0(I') \]
be the index maps associated with both rows. Then the diagram
\[ \begin{tikzcd}
K_1(B) \rar{\delta_1} \dar{K_1(\beta)} & K_0(I) \dar{K_0(\gamma)} \\
K_1(B') \rar{\delta_1'} & K_0(I')
\end{tikzcd} \qquad \text{commutes.} \]
\end{proposition}

\subsection{Higher $K$-groups}
\begin{proposition}
There is a natural isomorphism between $K_1(A)$ and $K_0(SA)$.
\end{proposition}
\begin{proof}
From the short exact sequence, \ref{exactSequenceSuspensionCone}
\[ \begin{tikzcd}
 0 \rar & SA \rar & CA \rar & A \rar & 0.
\end{tikzcd} \]
we get the long exact sequence, ref TODO.
\[ \begin{tikzcd}
K_1(SA) \rar & K_1(CA) \rar & K_1(A) \rar & K_0(SA) \rar & K_0(CA) \rar & K_0(A)
\end{tikzcd}. \]
Because $CA$ is contractible, we have
\[ \begin{tikzcd}
K_1(SA) \rar & 0 \rar & K_1(A) \rar & K_0(SA) \rar & 0 \rar & K_0(A)
\end{tikzcd}. \]
By exactness this gives $K_1(A) \cong K_0(SA)$. The naturality is given by the naturality of the index map, \ref{naturalityIndexMap}.
\end{proof}

\subsection{Bott periodicity}
\begin{theorem}[Bott periodicity]
The functors $K_0$ and $K_1\circ S$ are naturally isomorphic.
\end{theorem}
\subsection{The six-term exact sequence}

\chapter{$K$-theory for graded $C^*$-algebras}
\section{Van Daele's picture}

\section{Karoubi's picture}
\begin{definition}
Let $A$ be a graded $C^*$-algebra. 
\end{definition}


\chapter{$K$-theory for group $C^*$-algebras}


\part{Applied Mathematics}
\setcounter{chapter}{0} % Reset chapter counters
\chapter{Optimisation}
\input{optimisation}

\chapter{Curves and surfaces in Euclidean space}
\section{Curves and parametrisations in Euclidean space}
\subsection{Definition of curves in $\mathbb{E}^n$}
A curve can be describes with a mapping of the form
\[ \gamma: I\subseteq \R \to \mathbb{E}^n: t \mapsto \gamma(t) = (\gamma_1(t), \ldots, \gamma_n(t)) \]
where $I$ is an interval and each of the components $\gamma_i(t)$ are real functions.

We call the curve described by $\gamma$ \udef{differentiable} if each component is infinitely differentiable.

We call $t$ the \udef{parameter} of the curve and $\gamma$ is called the \udef{parametrisation} of the curve. The parametrisation contains information not only about the shape of the curve, but also about how it is traversed. We will often simply use the word curve when we mean parametrisation.

\begin{example}
\begin{itemize}
\item A line is a type of curve and can written as
\[ \gamma: \R \to \mathbb{E}^n: t \mapsto p + tv \]
where $p\in \mathbb{E}^n$ is a point and $v\in \R^n$ is a vector.
\item A circle is a curve and can be written as
\[ \gamma: [0,2\pi]\to \mathbb{E}^2: t\mapsto (m_1 + R\cos t, m_2 + R\sin t) \]
where $m = (m_1,m_2) \in \mathbb{E}^2$ is a point and $R$ is a positive number called the radius.
\item A helix is a curve and can be written as
\[ \gamma: \R\to \mathbb{E}^3: t\mapsto (a\cos t, a\sin t, bt) \]
where $a$ and $b$ are real numbers.
\end{itemize}
\end{example}

\begin{definition}
Let $\gamma: I \to \mathbb{E}^n$ be a curve. A \udef{vector field along $\gamma$} is a map of the form
\[ Y: I \to T\mathbb{E}^n: t\mapsto Y(t) \in T_{\gamma(t)}\mathbb{E}^n \]
\end{definition}

TODO specify components after geometry!!!!

\subsection{Velocity and arc length}
We define the \udef{velocity vector field along the curve} as the map
\[ \vec{\gamma'}: I \to T\mathbb{E}^n: t \mapsto \vec{\gamma'(t)}\equiv (\gamma'_1(t),\ldots, \gamma'_n(t))_{\gamma(t)} \in T_{\gamma(t)}\mathbb{E}^n \]
where $\vec{\gamma_i'}(t)$ means the derivative of $\gamma_i(t)$.

The \udef{speed} of $\gamma$ can then be defined as the function
\[ v: I \to \R: t \mapsto v(t) \equiv \lVert\vec{\gamma'}(t)\lVert \]
and for $a,b \in I$ with $a\leq b$, we call
\[ \int^b_a v(t) \diff{t} = \int^b_a \lVert \vec{\gamma'}(t) \lVert \diff{t} \]
the \udef{length} of the stretch of $\gamma$ between $\gamma(a)$ and $\gamma(b)$. This corresponds to our intuitive notion of length of a curve.

\begin{note}
Another way to define the length of a curve, is by dividing the interval $[a,b]$ into $k$ sections, each of length $\Delta$. Thus we can write
\[ a = t_0 < t_1 < \ldots < t_{k-1} < t_k = b \qquad \text{with} \; t_i - t_{i-1} = \Delta. \]
This defines a broken line with length
\[ \sum^k_{i=1} \lVert \gamma(t_i) - \gamma(t_{i-1})\lVert  = \sum^{k-1}_{i=0} \lVert (\gamma(t_i+ \Delta) - \gamma(t_{i})) / \Delta \lVert \Delta. \]

We could define the length of the curve as the length of such a broken line in the limit of $k \to \infty$ (which also means that $\Delta$ goes to $0$). So
\begin{align*}
\text{length} &= \lim_{k\to\infty, \Delta\to 0} \sum^{k-1}_{i=0} \lVert (\gamma(t_i+ \Delta) - \gamma(t_{i})) / \Delta \lVert \Delta \\
&= \lim_{k\to\infty, \Delta\to 0}\sum^{k-1}_{i=0} \lVert (\gamma'(t_i)) \lVert \Delta \\
&= \int^b_a \lVert \gamma'(u) \lVert \diff{u}
\end{align*}
which is the definition we gave before.
\end{note}

We can also define the \udef{arc length} as the function
\[ s: I \to \R: t\mapsto s(t) \equiv \int_a^t v(u) \diff{u} = \int_a^t \lVert \vec{\gamma'}(u)\lVert \diff{u} \]
for a given $a\in I$. This is quite simply the length of the curve between $\gamma(a)$ and $\gamma(t)$, if $t\geq a$ and minus the length otherwise.

\subsection{Tangent vectors}
A \udef{tangent line} to the curve $\gamma$ in $t_0$ is a line through the point $\gamma(t_0)$ in the direction of $\vec{\gamma'}(t_0)$. This is obviously only defined if $\vec{\gamma'}(t_0) \neq 0$.

Any multiple of $\vec{\gamma'}(t_0)$ in $T_{\gamma(t_0)}\mathbb{E}^n$ is called a \udef{tangent vector} in $\gamma(t_0)$. In particular a tangent vector with norm one is called a \udef{unit tangent vector}.

\begin{note}
With the right assumptions of differentiability etc. it is possible to write down a Taylor expansion of the curve $\gamma$.
\[ \gamma(t) = \gamma(t_0) + (t-t_0)\vec{\gamma'}(t_0) + \ldots + \frac{1}{k!}(t-t_0)^k\vec{\gamma^{(k)}}(t_0) + (t-t_0)^{k+1}R_{k+1}(t) \]
We recognise the expression for the tangent line as the first order approximation
\[ \gamma(t) \approx \gamma(t_0) + (t-t_0)\vec{\gamma'}(t_0). \]
\end{note}

\subsection{Reparametrisations and arc length parametrisations}
Two different parametrisations may look the same when drawn in space.

\begin{definition}
Let $I, \tilde{I} \subseteq \R$ be intervals and $\gamma: I \to \mathbb{E}^n$ a curve.
If $h: \tilde{I} \to I$ is a diffeomorphism, then 
\[ \beta \equiv \gamma \circ h: \tilde{I} \to \mathbb{E}^n \]
is a curve \textit{with the same image as $\gamma$} (i.e. it looks the same in space). We call $\beta$ a \udef{reparametrisation} of $\gamma$.
\end{definition}
Because
\[ \vec{\beta'}(t) = (\gamma\circ h)'(t) = \vec{\gamma}'(h(t))h'(t), \]
$\vec{\beta'}(t)$ and $\vec{\gamma'}(h(t))$ are proportional to each other and thus the tangent lines are the same. Also
\[ v_\beta = |h'|(v_\gamma\circ h). \]

Because the arc length is also a geometric quantity, we would expect it to be the same for both parametrisations. Actually it turns out to be the same up to the sign, because the new parametrisation may traverse the curve in the opposite direction.
\[ s_\beta(t) = \pm s_\gamma(h(t)) \]

\subsubsection{Arc length parametrisation}
We would now like to find a reparametrisation such that the speed is always unity (i.e. one). This turns out to always be possible is the curve is regular.
\begin{definition}
A curve is called \udef{regular} if $v(t) > 0$ for all $t$.
\end{definition}

If the curve is regular, the the arc length is a diffeomorfism, as is it's inverse. Using the inverse in the place of the diffeomorphism $h$, we get exactly the reparametrisation we were looking for.

It turns out that this reparametrisation is relatively unique: If $\beta_1$ and $\beta_2$ are reparametrisations of the same curve, both with speed 1, then $\beta_1(t) = \beta_2(\pm t + c)$, for a constant $c\in\R$. In other words, if we want a reparametrisation with speed 1 everywhere, then that reparametrisation is unique once we have chosen a direction and origin.

\begin{definition}
We call a curve with speed 1 everywhere an \udef{arc length parametrisation}.
\end{definition}

Let $\beta$ be an arc length parametrisation, then the arc length is
\[ s_\beta(t) = \int_a^t \diff{u} = t - a \]

\section{Curves in flat Euclidean space}
\subsection{Frenet frame for regular curves}
Given a curve $\gamma: I \to \mathbb{E}^2$, we wish to introduce a useful basis for $T_{\gamma(t)}\mathbb{E}^2$. By useful, we mean that it can serve as a natural reference frame for a particle traveling along the curve. The Frenet frame also leads to a natural definition of the curvature of a curve (as well as torsion in 3 dimensional space).

A first obvious vector to introduce in our basis is the unit tangent vector, which we will call $\vec{T}$. Then there is only one way to extend this to a positively oriented orthonormal basis, which we call the normal unit vector $\vec{N}$. If the unit tangent vector is given by $\vec{T} = \left(T_1, T_2\right)$, then\footnote{Effectively we are applying the two-dimensional complex structure to $\vec{T}$. Where a \udef{linear complex structure} is a linear transformation that squares to minus identity.}
\[ \vec{N} = \left(-T_2, T_1\right) \]
\begin{definition}
In two dimensions, the \udef{Frenet frame}, for a curve $\gamma(t)$ at a point $t_0$, is given by $(\vec{T}, \vec{N})$ where
\begin{itemize}
\item $\vec{T}$ is the unit tangent vector to $\gamma(t)$ at $t_0$;
\[ \vec{T} = \frac{\vec{\gamma'}}{\lVert \vec{\gamma'} \lVert} \]
\item $\vec{N}$ is the unique vector such that $(\vec{T}, \vec{N})$ is a positively oriented orthonormal basis, called the \udef{normal unit vector}.
\end{itemize}
\end{definition}

From $(\vec{T}\cdot \vec{T}) = 1$, we see that $(\vec{T}\cdot \vec{T})' = 0$. Using the product rule, we see that it is also equal to $(\vec{T}\cdot \vec{T})' = \vec{T}'\cdot\vec{T} + \vec{T}\cdot\vec{T}' = 2(\vec{T}\cdot \vec{T}')$. Thus we see that $\vec{T}\cdot \vec{T}' = 0$, meaning that the derivative of $\vec{T}$ must be perpendicular to $\vec{T}$. This is true for any unit vector. The unit normal vector is also perpendicular to $\vec{T}$, so we can find a $k \in \R$ such that $\vec{T}' = k \vec{N}$. Because $\vec{N}$ is a unit vector, $k$ is given by $k = \vec{T}'\cdot \vec{N}$.

Following a similar line of reasoning, we see that $\vec{N}'$ must be proportional to $\vec{T}$, with a factor of proportionality equal to $\vec{T}\cdot \vec{N}'$. From
\[ (\vec{T}\cdot \vec{N})' = \vec{T}'\cdot \vec{N} + \vec{T}\cdot \vec{N}' = 0 \]
we see that the factor must equals $-k$.

The quantity $k$ correspond to our intuitive notion of curvature (as we will see), but \textbf{only if the curve is arc length parametrised}. We also want the curvature to be a purely geometric quantity that does not depend on which parametrisation we choose (as $k$ does). So, for regular curves, it makes sense to define the curvature $\kappa$ as the factor $\vec{T}'\cdot \vec{N}$ for an arc length parametrisation of the curve. This uniquely determines the curvature $\kappa$ of the curve up to the sign, which depends on the direction of traversal.

Now for a general regular curve with arc length $s(t)$, we denote the quantities associated with an arc length parametrisation with a tilde:
\[ \begin{cases}
\vec{T}(t) = \vec{\tilde{T}}(s(t)) \\
\vec{N}(t) = \vec{\tilde{N}}(s(t)) \\
\kappa(t) = \tilde{k}(s(t))
\end{cases} \]
We can calculate
\[ \vec{T}'(t) = (\vec{\tilde{T}}(s(t)))' = \vec{\tilde{T}}'(s(t))s'(t) = v(t)\tilde{k}(s(t))\vec{\tilde{N}}(s(t)) = v(t)\kappa(t)N(t) \]

We now have a general expression for $\kappa$:
\[ \kappa = \frac{\vec{T}'\cdot \vec{N}}{v} \]

Summarising
\begin{eigenschap}
The Frenet-Serret formulae are
\[ \begin{cases}
\vec{T}' = \kappa v \vec{N} \\
\vec{N}' = -\kappa v \vec{T}
\end{cases} \]
where $\kappa = \frac{\vec{T}'\cdot \vec{N}}{v}$ is called the \udef{(oriented) curvature} and $v$ is the speed at which the curve is traversed in that point; $v=1$ for arc parametrised curves.
\end{eigenschap}

The curvature $\kappa$ of any regular curve $\gamma$ can be calculated directly from the first and second derivatives of the curve:
\[ \kappa = \frac{\lVert \vec{\gamma'} \times \vec{\gamma''}\lVert}{\lVert\vec{\gamma'}\lVert^3} \]
Or
\[ \kappa = \frac{| \vec{\gamma'} \quad \vec{\gamma''}|}{\lVert\vec{\gamma'}\lVert^3} \]
where $| \vec{\gamma'} \quad \vec{\gamma''}|$ is the determinant with $\vec{\gamma'},\vec{\gamma''}$ seen as column vectors.

The associated collection $\vec{T}, \vec{N}, \kappa$ is called the \udef{Frenet-Serret apparatus}. In three dimensions this also includes the binormal unit vector $\vec{B}$ and torsion $\tau$.

\subsection{Curvature}
First some examples to support the idea that our definition of curvature makes sense.
\begin{example}
\begin{itemize}
\item Let $\gamma$ be a straight line with arc length parametrisation
\[ \gamma: \R \to \mathbb{E}^2: s\mapsto p + s \vec{s} \]
where $\lVert \vec{v}\lVert = 1$. Then $\vec{T}(s) = \vec{v}$ for all $s$ and consequently $\vec{T}' = 0$. Thus a straight line has zero curvature.
\item It can be proven that any curve with zero curvature is a straight line.
\item Let $\gamma$ be a circle centered at $m= (m_1,m_2)$ with radius $R$ and arc length parametrisation
\[ \gamma: \R \to \mathbb{E}^2: s\mapsto \left(m_1 + R\cos \left(\frac{s}{R}\right), m_2 + R\sin \left(\frac{s}{R}\right)\right). \]
Then the Frenet frame, for all $s\in\R$, is given by
\[ \begin{cases}
\vec{T}(s) = \left(-\sin \left(\frac{s}{R}\right), \cos \left(\frac{s}{R}\right)\right) \\
\vec{N}(s) = \left(-\cos \left(\frac{s}{R}\right), -\sin \left(\frac{s}{R}\right)\right)
\end{cases} \]
Thus
\[ \vec{T}'(s) = \left(- \frac{1}{R}\cos \left(\frac{s}{R}\right), - \frac{1}{R}\sin \left(\frac{s}{R}\right)\right) = \frac{1}{R} \vec{N}(s), \]
meaning that $\kappa(s) = \frac{1}{R}$ for all $s\in \R$. A circle with radius $R$ has a constant curvature $1/R$.
\item Every curve with constant, non-zero, curvature is a (part of a) circle with radius $1/|\kappa|$.
\end{itemize}
\end{example}
\subsubsection{Osculating parabola}
Let $\beta$ be an arc length parametrised curve. Then we can write the Taylor expansion:
\begin{align*}
\beta(s) &= \beta(s_0) + (s-s_0)\vec{\beta'}(s_0) + \frac{1}{2}(s-s_0)^2\vec{\beta''}(s_0) + (s-s_0)^3R_3(s) \\
&= \beta(s_0) + (s-s_0)\vec{T}(s_0) + \frac{1}{2}(s-s_0)^2\kappa(s_0)\vec{N}(s_0) + \ldots
\end{align*}
The second order approximation is a parabola, called the \udef{osculating parabola} (which comes from the latin word osculans meaning kissing). Intuitively it may be thought of as the parabola with it's top in $\beta(s_0)$ that most closely matches the curve.

TODO Figure. 

If $\kappa$ is positive, $\beta$ curves towards $\vec{N}$ and $\beta$ locally lies on the same side of $\vec{T}$ as $\vec{N}$. If $\kappa$ is negative, $\beta$ curves away from $\vec{N}$ and $\beta$ locally lies on the opposite side of $\vec{T}$ from $\vec{N}$.

\subsubsection{Osculating circle}
Again let $\beta$ be an arc length parametrised curve.
\begin{definition}
\begin{itemize}
\item We call $1/|\kappa(s_0)|$ the \udef{radius of curvature} of the curve at $s_0$.
\item We call the point
\[ m \equiv \beta(s_0) + (1/\kappa(s_0))\vec{N}(s_0) \]
the \udef{centre of curvature} of the curve at $s_0$.
\item The circle which has as its centre in the centre of curvature and a radius that is the same as the radius of curvature, is called the \udef{osculating circle} of the curve at $s_0$.
\end{itemize}
\end{definition}
The osculating circle may be parametrised as
\[ c(s) = m + R\cos \left(\frac{s-s_0}{R}\right)(- \vec{N}(s_0)) + R\sin \left(\frac{s-s_0}{R}\right)\vec{T}(s_0) \]
where $m$ is the centre of curvature and $R = \frac{1}{|\kappa(s_0)|}$ is the radius of curvature.

We can easily calculate that
\[ \begin{cases}
c(s_0) = \beta(s_0) \\
\vec{c'}(s_0) = \vec{\beta'}(s_0) \\
\vec{c''}(s_0) = \vec{\beta''}(s_0).
\end{cases} \]
We say that the osculating circle approximates the curve to second order. It is the only circle that does that.

The osculating circle gives us quite a useful intuitive interpretation of the curvature:
it is the inverse of the radius of the ``best fitting'' circle.

\subsection{Intrinsic equations}
An \udef{intrinsic equation} of a curve is an equation that defines the curve using a relation between geometrical properties that are intrinsic to the curve and do not depend on the exact parametrisation.

Examples of such intrinsic quantities are: arc length $s$, tangential angle $\theta$ and curvature $\kappa$.
\subsubsection{Tangential angle}
The \udef{tangential angle} $\theta$ is the angle of the unit tangent vector $\vec{T}$ with the line through points $(0,0)$ and $(0,1)$ in $\mathbb{E}^2$ (i.e. the ``$x$-axis''). It is a function $\theta: I \to \R$ such that
\[ \vec{T}(s) = (\cos(\theta(s)), \sin(\theta(s))). \]
The normal unit vector is then given by
\[ \vec{N}(s) = (-\sin(\theta(s)), \cos(\theta(s))). \]
From $\vec{T}' = (-\sin(\theta)\theta', \cos(\theta)\theta') = \theta'\vec{N}$, we see that
\[ \kappa(s) = \theta'(s). \]

\subsubsection{Whewell equations}
Suppose we have an equation for the tangential angle of a curve in function of the arc length ($\theta(s) = \ldots$). We would now like to find a parametrisation for that curve.

Consider the following curve:
\[ \beta(s) = \left(\int_{s_0}^s\cos\theta(u) \diff{u},\, \int^s_{s_0}\sin\theta(u) \diff{u}\right) \]
Then $\vec{\beta'}(s) = (\cos(\theta(s)),\sin(\theta(s)))$ for all $s\in I$, so $\beta$ is arc length parametrised and tangential angle $\theta$ at all points. So $\beta$ is exactly the parametrisation we were looking for.

\begin{example}
\begin{itemize}
\item Straight lines are determined by $\theta = c$ for some constant $c\in\R$.
\item Circles are determined by $\theta(s) = \frac{s}{R}$ where $R\in \R$ is the radius.
\item Catenary curves are determined by $\theta = \arctan \left(\frac{s}{R}\right)$.
\end{itemize}
\end{example}
\subsubsection{Cesàro equations}
Now suppose we have an equation for the curvature of a curve in function of the arc length ($\kappa(s) = \ldots$).

Because $\theta' = \kappa$, the Cesàro equation of a curve can be obtained from the Whewell equation by differentiating it.

The parametrisation is then given by
\[ \beta(s) = \left(\int_{s_0}^s\cos\left(\int_{s_0}^u\kappa(t)\diff{t}\right) \diff{u},\, \int^s_{s_0}\sin\left(\int_{s_0}^u\kappa(t)\diff{t}\right) \diff{u}\right) \]

\begin{example}
\begin{itemize}
\item Line: $\kappa = 0$
\item Circle: $\kappa = 1/R$
\item Logarithmic spiral: $\kappa = C / s$
\item Circle involute: $\kappa = C / \sqrt{s}$
\item Cornu spiral (or clothoid): $\kappa = Cs$
\item Catenary: $\kappa = \frac{a}{s^2 + a^2}$
\end{itemize}
\end{example}

\subsection{Global properties of flat curves}
So far we have mainly looked at \textit{local} properties of curves, like curvature. Now we take a look at some global properties.

\begin{definition}
We call a curve $\gamma:\R\to\mathbb{E}^n$ \udef{closed} if there is a strictly positive number $\omega \in R_0^+$ such that $\gamma(t+\omega) = \gamma(t)$ for all $t\in \R$. We call $\omega$ the \udef{period} of $\gamma$.

If $\omega$ is a period of a curve, then any multiple of $\omega$ is also a period. We call the smallest period the \udef{real period} $\omega$.
\end{definition}

\begin{definition}
A \udef{simple} curve is a curve that does not cross itself. 
\end{definition}
A closed curve $\gamma$ with real period $\omega$ is said to be simple if the restriction $\gamma|_{[0,\omega[}: [0,\omega[ \to \mathbb{E}^n $ is injective.

\begin{definition}
Let $\beta:\R \to \mathbb{E}^2$ be an arc parametrised, closed curve with period $L$.
\begin{itemize}
\item We call
\[ \int_0^L \kappa(s) \diff{s} \]
the \udef{total curvature} of $\beta$.
\item We define the \udef{rotation index} $i_\beta$ of $\beta$ as
\[ i_\beta \equiv \frac{1}{2\pi}\left(\theta(L) - \theta(0)\right) \]
\end{itemize}
\end{definition}
The rotation index must always be an integer, because $\vec{T}(0)$ must be the same as $\vec{T}(L)$. So the angle tangential angles must be the same, modulus $2\pi$.

\begin{eigenschap}
The total curvature is related to the rotation index:
\[ \int_0^L\kappa(s)\diff{s} = 2\pi i_\beta \]
\end{eigenschap}

Finally we formulate three theorems for simple, closed, flat curves (also called \udef{Jordan curves}).
\begin{eigenschap}
\begin{itemize}
\item \textit{Umlaufsatz}. The rotation index of a simple closed curve is $1$ or $-1$.
\item \textit{Jordan's theorem}. A simple closed curve divides the plain onto two parts: a bounded interior and an unbounded exterior.
\item \textit{Isoperimetric inequality}. The surface area of the bounded interior $A$ satisfies the inequality
\[ L^2 \geq 4\pi A \]
where $L$ is a period. This is an equality only if the curve is a circle (and $L$ the real period).
\end{itemize}
\end{eigenschap}

\section{Curves in three dimensional Euclidean space}
\subsection{The Frenet frame}
Let $\gamma$ be a regular curve. Again we define $\vec{T}$ as the unit tangent vector. For the same reason as before, $\vec{T'}$ is perpendicular to $\vec{T}$, so we can use that to define $\vec{N}$. In three dimensions we need a third basis vector. There is only one vector that can be added to the orthonormal vectors $\vec{T}$ and $\vec{N}$ to make \ueig{positively oriented orthonormal basis} of $T_{\gamma(t)}\mathbb{E}^3$, namely $\vec{B} = \vec{T}\times \vec{N}$.
\begin{definition}
In three dimensions, the \udef{Frenet frame}, for a curve $\gamma(t)$ at a point $t_0$, is given by $(\vec{T}, \vec{N}, \vec{B})$ where
\begin{itemize}
\item $\vec{T}$ is the \udef{tangent unit vector} to $\gamma(t)$ at $t_0$
\[ \vec{T} \equiv \frac{\vec{\gamma'}}{\lVert \vec{\gamma'} \lVert} \]
\item $\vec{N}$ is the \udef{normal unit vector}
\[ \vec{N} \equiv \frac{\vec{T'}}{\lVert \vec{T'} \lVert} \]
\item $\vec{B}$ is the \udef{binormal unit vector}
\[ \vec{B} \equiv \vec{T}\times \vec{N} \]
\end{itemize}
\end{definition}

From this definition it is obvious that $\vec{T}'$ is a multiple of $\vec{N}$. As in the one dimensional case, the factor connecting ($k = \lVert \vec{T'} \lVert$) them has geometric significance, so long as the speed is fixed. The big difference is that now the factor $k$ is always positive.

Again we define the curvature $\kappa$ of a curve as the factor $k$ for an arc length parametrisation of the curve. As in the two dimensional case (and following the same reasoning), we have for general regular curves
\[ \vec{T}' = k \vec{N} = v \kappa \vec{N} \]

We now prove that $\vec{B}'= l \vec{N}$ for some factor $l(t)\in \R$. First we write an orthonormal expansion of $\vec{B}'$
\[ \vec{B}' = (\vec{B}'\cdot \vec{T})\vec{T} + (\vec{B}'\cdot \vec{N})\vec{N} + (\vec{B}'\cdot \vec{B})\vec{B}. \]
Now $\vec{B}'\cdot \vec{B}=0$, which we have already shown to be true in the previous section because $\vec{B}$ is a unit vector (like $\vec{T}$). If we derive $\vec{B}\cdot \vec{T} = 0$, we get $\vec{B}'\cdot \vec{T} + \vec{B}\cdot \vec{T}' = \vec{B}'\cdot \vec{T} + k \vec{B}\cdot \vec{N} = \vec{B}'\cdot \vec{T} = 0$. Thus 
\[ \vec{B}' = (\vec{B}'\cdot \vec{N})\vec{N} = l \vec{N} \]

Again, to give $l = \vec{B}'\cdot \vec{N}$ geometric significance, we define the torsion $\tau$ as $-l$ for arc length parametrisations. The minus sign is a classical convention. The torsion \textit{can} be negative. Again, following the same reasoning as we have twice before, we get for general regular curves
\[ \vec{B}' = l \vec{N} = - v \tau \vec{N} \]

\begin{eigenschap}
The Frenet-Serret formulae are
\begin{alignat*}{4}
\vec{T}' &= & & & v\kappa&\vec{N} & & \\
\vec{N}' &= & -v\kappa&\vec{T} & & & +v\tau&\vec{B} \\
\vec{B}' &= & & & -v\tau&\vec{N} & &
\end{alignat*}
where $v = \lVert \vec{\gamma'} \lVert$ is the speed at which the curve is traversed in that point ($v=1$ for arc parametrised curves) and
\begin{itemize}
\item $\kappa = \frac{\vec{T}'\cdot \vec{N}}{v}$ is called the \udef{curvature}
\item $\tau = \frac{- \vec{B}'\cdot \vec{N}}{v}$ is called the \udef{torsion}
\end{itemize}
\end{eigenschap}
The only formula we have not yet proven is the second one. It can easily be seen to be correct if we take the orthonormal expansion $\vec{N}' = (\vec{N}'\cdot \vec{T})\vec{T} + (\vec{N}'\cdot \vec{N})\vec{N} + (\vec{N}'\cdot \vec{B})\vec{B}$ and calculate
\begin{align*}
\vec{N}'\cdot \vec{T} &= - \vec{N}\cdot \vec{T}' = -v\kappa \vec{N}\cdot \vec{N} = -v\kappa \\
\vec{N}'\cdot \vec{N} &= 0 \\
\vec{N}'\cdot \vec{B} &= - \vec{N}\cdot \vec{B}' = v\tau \vec{N}\cdot \vec{N} = v\tau
\end{align*}

The curvature $\kappa$ and torsion $\tau$ of any regular curve $\gamma$ can be calculated directly from the first, second and third derivatives of the curve. As in two dimensions, we have
\[ \kappa = \frac{\lVert \vec{\gamma'} \times \vec{\gamma''}\lVert}{\lVert\vec{\gamma'}\lVert^3} \]
For the torsion we have
\[ \tau = \frac{\vec{\gamma'}\times\vec{\gamma''}\cdot\vec{\gamma'''}}{\lVert\vec{\gamma'}\times\vec{\gamma''}\lVert^2} \qquad \text{or}\qquad \frac{|\vec{\gamma'}\quad\vec{\gamma''}\quad\vec{\gamma'''}|}{\lVert\vec{\gamma'}\times\vec{\gamma''}\lVert^2} \]
where $| \vec{\gamma'} \quad \vec{\gamma''}\quad \vec{\gamma'''}|$ is the determinant with $\vec{\gamma'},\vec{\gamma''}, \vec{\gamma'''}$ seen as column vectors.

The associated collection $\vec{T}, \vec{N}, \vec{B}, \kappa, \tau$ is called the \udef{Frenet-Serret apparatus}.

\subsubsection{Osculating plane}
We define the \udef{osculating plane} in each point as the plane that contains $\vec{T}$ and $\vec{N}$.

Because the Taylor expansion of an arc length parametrised curve $\beta$ is still given by
\begin{align*}
\beta(s) &= \beta(s_0) + (s-s_0)\vec{\beta'}(s_0) + \frac{1}{2}(s-s_0)^2\vec{\beta''}(s_0) + (s-s_0)^3R_3(s) \\
&= \beta(s_0) + (s-s_0)\vec{T}(s_0) + \frac{1}{2}(s-s_0)^2\kappa(s_0)\vec{N}(s_0) + \ldots
\end{align*}
the osculating plane can be seen as the tangent plane and it contains the osculating parabola.

If $\beta$ lies in a plane, then all osculating planes are equal to this plane. That means that $\vec{B}$ does not change, so $\vec{B}'= 0$ and $\tau = 0$. In fact we can state that for any regular curve $\gamma$ with curvature $\kappa > 0$, $\gamma$ lies in a plane if and only if $\tau = 0$.

If a space curve $\gamma$ lies in a plane, then the curvature $\kappa$ of the curve is the absolute value of the curvature of the curve seen as a planar curve in two dimensions.

\section{Surfaces in Euclidean space}
In this section we introduce some of the key concepts
\chapter{Vector and tensor calculus}
TODO analysis in vector notation. See Stat inf AQFT

\section{Fields}
\subsection{What is a field?}
People mean different things when they say field. We have already encountered the algebraic structure (e.g. the fields $\R$ and $\mathbb{Q}$). In physics the term field usually means we associate a value or an object to each point in space. The vector field along a curve that we have already seen can be seen as a field in one dimensional space. In this section we will restrict our attention to Euclidean space. Details for other geometries will follow. In modern high energy physics the term field often refers specifically to fields of operators.

\begin{definition}
A field $F$ associates an element of a set $X$ to each point in space.
\[ F: \mathbb{E}^n \to X \]
Where $n$ is typically 2 or 3. Depending on $X$ we call the field differently:
\begin{itemize}
\item For $X \subset \R$ we call $F$ a \udef{scalar field}.
\item For $X$ a three-dimensional vector field, we call $F$ a \udef{vector field}.
\end{itemize}
\end{definition}

Examples of scalar fields include temperature in space and pressure distribution in a fluid. The flow of a fluid can be modeled using a vector field.

\section{Differential calculus}
There are two important cases: a field may be two or three dimensional. Because two dimensional fields can be seen as a special case of three dimensional ones, we will assume $n=3$ for the rest of this section. This means that the field is essentially a function with three variables:
\[ F(x,y,z) \]
This means there are three differential operators $\pd{}{x}, \pd{}{y}, \pd{}{z}$. We also fix and orthonormal basis for $\mathbb{E}^3$ with basisvectors
\[\vhat{x},\; \vhat{y},\; \vhat{z}\]
in that order.

\subsection{Nabla, the vector differential operator}
The main operations we can do in scalar and vector fields (taking the gradient, divergence and curl) can be expressed in terms of the \udef{nabla} operator $\vnabla$ (also known as the \udef{del} operator), which can be seen as a \textit{vector} operator with three components:
\[ \vnabla = \pd{}{x}\vhat{x} + \pd{}{y}\vhat{y} + \pd{}{z}\vhat{z} \]
Formally this means it is an operator that when acting on a number produces a vector. 

More intuitively it means that is also makes sense to use it in conjunction with the dot and cross products. It must however be remembered that $\vnabla$ is not really a vector and we cannot just use vector identities with it (even if they do sometimes turn out to be correct). When in doubt, write out the components.

\subsubsection{Gradient}
Say $F$ is a scalar field. A first, obvious, question for calculus to solve is: how fast does $F$ vary?

TODO after Taylor expansion. TODO directional derivative + intuition

\[ \diff{F} = (\grad F)\cdot (\diff{\vec{l}}) \]
where
\[ \grad F = \left(\pd{}{x}\vhat{x} + \pd{}{y}\vhat{y} + \pd{}{z}\vhat{z}\right)F = \pd{F}{x}\vhat{x} + \pd{F}{y}\vhat{y} + \pd{F}{z}\vhat{z} \]

Now
\[ \diff{F} = \grad F \cdot \diff{\vec{l}} = |\grad F||\diff{\vec{l}}|\cos\theta. \]
From this it is clear that $\diff{F}$ is largest if $\theta = 0$ and smallest (i.e. zero) if $\theta = \pi$. Fixing $\theta = 0$ and viewing $F$ as a one dimensional function (with variable $l = |\vec{l}|$) along this line, we see
\[ \od{F}{l} = |\grad F|. \]
This leads us to a geometrical interpretation of the gradient:
\begin{itemize}
\item The gradient $\grad F$ points in the direction of maximum increase of the function $F$.
\item Locally the field does not vary perpendicular to $\grad F$.
\item The magnitude $|\grad F|$ gives the slope (rate of increase) along this maximal direction.
\end{itemize}

We call a point $(x,y,z)$ a \udef{stationary point} if $\grad F = 0$ at $(x,y,z)$. As for single variable calculus, local maxima and minima are stationary points.

\subsubsection{Divergence}
Assume we have a vector field
\[ \vec{v}: \mathbb{E}^3 \to \mathbb{R}^3: (x,y,z) \mapsto \vec{v}(x,y,z) = v_x \vhat{x} + v_y \vhat{y} + v_z \vhat{z} \]
we can then define the \udef{divergence} as
\begin{align*}
\nabla\cdot \vec{v} &= \left(\pd{}{x}\vhat{x} + \pd{}{y}\vhat{y} + \pd{}{z}\vhat{z}\right)\cdot (v_x \vhat{x} + v_y \vhat{y} + v_z \vhat{z}) \\
&= \pd{v_x}{x} + \pd{v_y}{y} + \pd{v_z}{z}
\end{align*}
The divergence of a vector function is a \textit{scalar}.

Intuitively the divergence can be thought of as the amount the vector field spreads out (diverges) from the point in question. If we think of the vector field as modeling the flow of a fluid, then a point of positive divergence is a source and a point of negative divergence is a drain.

TODO figure (like 18 in electro)

\subsubsection{Curl}
For a vector field $\vec{v}$, the \udef{curl} can be defined as follows:
\begin{align*}
\curl \vec{v} &= \begin{vmatrix}
\vhat{x} & \vhat{y} & \vhat{z} \\
\pd{}{x} & \pd{}{y} & \pd{}{z} \\
v_x & v_y & v_z
\end{vmatrix} \\
&= \left(\pd{v_z}{y} - \pd{v_y}{z}\right)\vhat{x} + \left(\pd{v_x}{z} - \pd{v_z}{x}\right)\vhat{y} + \left(\pd{v_y}{x} - \pd{v_x}{y}\right)\vhat{z}.
\end{align*}
The curl of a vector function is a \textit{vector}.

Intuitively the curl is a measure of how much the vector swirls around the point in question. Again viewing the vector field as the flow of some liquid, the curl indicates how much a paddle wheel fixed at that point would rotate (TODO fig, like 19 in electro + paddle wheel).

\subsection{Properties of vector derivatives}
\begin{note}
We now assume
\begin{itemize}
\item $k\in \R$ is a constant.
\item $f$ and $g$ are scalar fields.
\item $\vec{A}$ and $\vec{B}$ are vector fields.
\end{itemize}
\end{note}
All the vector derivatives are linear:
\[ \begin{cases}
\grad(kf+g) = k\grad f + \grad g \\
\nabla\cdot(k\vec{A}+ \vec{B}) = k\nabla\cdot\vec{A} + \nabla\cdot \vec{B} \\
\curl(k \vec{A}+ \vec{B}) = k\curl \vec{A} + \curl \vec{B}
\end{cases} \]
\subsubsection{Product rules}
There are several relevant products to consider: scalar times scalar ($fg$), scalar times vector ($f \vec{A}$), dot product ($\vec{A}\cdot \vec{B}$) and cross product ($\vec{A}\times \vec{B}$). Accordingly, there are six product rules. Each can easily be verified by writing out the components and using the standard product rule from single variable calculus.
\begin{itemize}
\item For gradients
\begin{itemize}
\item[(a)] $\grad(fg) = f\grad g + g\grad f$
\item[(b)] $\grad(\vec{A}\cdot \vec{B}) = \vec{A}\times(\curl \vec{B}) + \vec{B}\times(\curl\vec{A}) + (\vec{A}\cdot\vnabla)\vec{B} +(\vec{B}\cdot\vnabla)\vec{A}$
\end{itemize}
\item For divergences
\begin{itemize}
\item[(c)] $\nabla\cdot(f\vec{A}) = f(\nabla\cdot\vec{A}) + \vec{A}\cdot(\grad f)$
\item[(d)] $\nabla\cdot(\vec{A}\times \vec{B}) = \vec{B}\cdot(\curl \vec{A}) - \vec{A}\cdot(\curl\vec{B})$
\end{itemize}
\item For curls
\begin{itemize}
\item[(e)] $\curl(f\vec{A}) = f(\curl \vec{A}) - \vec{A}\times (\grad f)$
\item[(f)] $\curl(\vec{A}\times \vec{B}) = (\vec{B}\cdot\vnabla)\vec{A} - (\vec{A}\cdot\vnabla)\vec{B} + \vec{A}(\nabla\cdot \vec{B}) - \vec{B}(\nabla\cdot \vec{A})$
\end{itemize}
\end{itemize}
One strange feature of these product rules is the occurrence of terms of the form $(\vec{A}\cdot\vnabla)\vec{B}$. This is clearer (at least to my mind) when written out in components (in three dimensions)
\[ (\vec{A}\cdot\vnabla)\vec{B} = \left(A_x  \pd{B_x}{x}\right)\vhat{x} + \left(A_y  \pd{B_y}{y}\right)\vhat{y} + \left(A_z  \pd{B_z}{z}\right)\vhat{z}. \]

\subsubsection{Quotient rules}
These can be easily obtained from the product rules.
\begin{align*}
\grad \left(\frac{f}{g}\right) &= \frac{g\grad f - f\grad g}{g^2} \\
\nabla\cdot \left(\frac{\vec{A}}{g}\right) &= \frac{g(\nabla\cdot \vec{A}) - \vec{A}\cdot (\grad g)}{g^2} \\
\curl \left(\frac{\vec{A}}{g}\right) &= \frac{g(\curl\vec{A}) + \vec{A}\times(\grad g)}{g^2}
\end{align*}

\subsection{Second derivatives}
\subsubsection{The Laplacian}
We introduce a second order operator, the \udef{Laplacian} $\nabla^2$:
\begin{align*}
\nabla^2 &= \vnabla \cdot \vnabla \\
&= \left(\pd{}{x}\vhat{x} + \pd{}{y}\vhat{y} + \pd{}{z}\vhat{z}\right)\cdot \left(\pd{}{x}\vhat{x} + \pd{}{y}\vhat{y} + \pd{}{z}\vhat{z}\right) \\
&= \pd[2]{}{x}\vhat{x} + \pd[2]{}{y}\vhat{y} + \pd[2]{}{z}\vhat{z}
\end{align*}
This can be applied to a scalar field:
\[ \nabla^2 F \equiv \pd[2]{F}{x}\vhat{x} + \pd[2]{F}{y}\vhat{y} + \pd[2]{F}{z}\vhat{z} \]
Or to a vector field by applying the Laplacian to each component individually:
\[ \nabla^2 \vec{v} \equiv (\nabla^2 v_x)\vhat{x} + (\nabla^2 v_y)\vhat{y} + (\nabla^2 v_z)\vhat{z} \]
\subsubsection{Constructing second derivatives from first order derivatives}
There are five ways we can make second derivative operators by mixing gradient, divergence and curl:
\begin{enumerate}
\item Divergence of gradient:
\begin{align*}
\nabla\cdot (\grad F) &= \left(\pd{}{x}\vhat{x} + \pd{}{y}\vhat{y} + \pd{}{z}\vhat{z}\right)\cdot \left(\pd{F}{x}\vhat{x} + \pd{F}{y}\vhat{y} + \pd{F}{z}\vhat{z}\right) \\
&= \pd[2]{F}{x}\vhat{x} + \pd[2]{F}{y}\vhat{y} + \pd[2]{F}{z}\vhat{z} = \nabla^2 F
\end{align*}
So this is just the Laplacian.
\item The curl of a gradient:
\begin{align*}
\curl(\grad F) &= \left(\pd{}{y}\pd{F}{z} - \pd{}{z}\pd{F}{y}\right)\vhat{x} + \left(\pd{}{z}\pd{F}{x} - \pd{}{x}\pd{F}{z}\right)\vhat{y} + \left(\pd{}{x}\pd{F}{y} - \pd{}{y}\pd{F}{x}\right)\vhat{z} \\
&= 0
\end{align*}
This is an important fact that hinges on the fact that cross derivatives commute.
\item The gradient of the divergence $\grad(\nabla\cdot \vec{v})$ is \textit{not} the same as the Laplacian of a vector. It does not have a special name.
\item The divergence of a curl:
\begin{align*}
\nabla\cdot(\curl \vec{v}) &= \pd{}{x}\left(\pd{v_z}{y} - \pd{v_y}{z}\right) + \pd{}{y}\left(\pd{v_x}{z} - \pd{v_z}{x}\right) + \pd{}{z}\left(\pd{v_y}{x} - \pd{v_x}{y}\right) \\
&= \pd{}{y}\pd{v_x}{z} - \pd{}{z}\pd{v_x}{y} + \pd{}{x}\pd{v_z}{y} - \pd{}{z}\pd{v_y}{x} + \pd{}{x}\pd{v_z}{y} - \pd{}{z}\pd{v_z}{x}
&= 0
\end{align*}
Again this hinges on the fact the cross derivatives commute.
\item The curl of a curl gives nothing new:
\[ \curl(\curl \vec{v}) = \grad(\nabla\cdot \vec{v}) - \nabla^2 \vec{v} \]
\end{enumerate}
We repeat two important facts for future reference:
\begin{eigenschap}
\begin{itemize}
\item The curl of a gradient is always \textbf{zero}:
\[ \curl(\grad F) = 0 \]
\item The divergence of a curl is always \textbf{zero}:
\[ \nabla\cdot(\curl \vec{v}) = 0 \]
\end{itemize}
\end{eigenschap}

\subsection{With respect to which coordinates?}
Sometimes we will deal with maps that look like fields (they depend on $x,y$ and $z$ coordinate), but also depend on other variables, like time. We can still use all the results from this section. All derivatives are partial, so we just calculate as if the other variables were constant.

Sometimes we will deal with maps that depend on two or more sets of spatial coordinates. For example, the electric field in a point may depend on the locations of various charged particles. In this case we can still use the notation and result from this section, we just need to specify with respect to which set of coordinates we are applying the derivative.

For example, using the compact notation $\vec{r_1} = (x_1,y_1,z_1)$ and $\vec{r_2} = (x_2,y_2,z_2)$, we may have a quantity $T(\vec{r_1}, \vec{r_2}) = T(x_1,y_1,z_1,x_2,y_2,z_2)$.
We can now write $\vnabla_{\vec{r_1}}T$ to mean
\[ \vnabla_{\vec{r_1}}T =  \pd{T}{x_1}\vhat{x} + \pd{T}{y_1}\vhat{y} + \pd{T}{z_1}\vhat{z} \]
and $\vnabla_{\vec{r_2}}T$ to mean
\[ \vnabla_{\vec{r_2}}T =  \pd{T}{x_2}\vhat{x} + \pd{T}{y_2}\vhat{y} + \pd{T}{z_2}\vhat{z} \]

\subsection{Miscellaneous identities}
\[ \vec{a}\times (\curl \vec{a}) = \grad \left(\frac{a^2}{2}\right) - (\vec{a}\cdot \nabla)\vec{a} \]

\subsection{Tensor derivatives}

\section{Integral calculus}
\subsection{Line integrals}
Given a curve $\gamma$ we define the \udef{line integral} between $\gamma(a)$ and $\gamma(b)$ as

\begin{definition}
\begin{itemize}
\item The line integral of a scalar field $F$ is given by
\[ \int_a^b F[\gamma(t)]\;|\vec{\gamma'(t)}|\; \diff{t}  \]
\item The line integral of a vector field $\vec{v}$ is given by
\[ \int_a^b \vec{v}[\gamma(x)]\cdot \vec{\gamma'}(x) \diff{x}. \]
This is usually written in the following way:
\[ \int_{\vec{a}}^{\vec(b)} \vec{v}\cdot \diff{\vec{l}} \qquad \text{or} \qquad \int_\gamma \vec{v}\cdot \diff{\vec{l}} \]
where $\vec{a} = \gamma(a)$ and $\vec{b} = \gamma(b)$
\end{itemize}
\end{definition}

If $\gamma$ is a closed loop (so $\vec{a} = \vec{b}$) we write
\[ \oint \vec{v}\cdot \diff{\vec{l}} \]

In general the value of the line integral depends on the curve $\gamma$. There exist some vector fields such that line integrals only depend on the endpoints, not on the path. Such vector fields are called \udef{conservative}. For any conservative field $\vec{u}$:
\[ \oint \vec{u}\cdot \diff{\vec{l}} = 0. \]

\subsection{Surface integrals}
For a given surface $\mathcal{S}$ and vector field $\vec{v}$, we define the \udef{surface integral}
TODO
\[ \iint_{\mathcal{S}} \vec{v}\cdot \diff{\vec{a}} \]
\[ \oiint \vec{v}\cdot \diff{\vec{a}} \]

\subsection{Volume integrals}
TODO
\[ \iiint_{\mathcal{V}} F \diff{\tau} \]

\section{Fundamental theorems of vector calculus}
In this section we will state three very important results that are analoguous to the fundamental theorem of calculus:
\[ \int_a^b \left(\od{f}{x}\right)\diff{x} = f(b) - f(a) \]

That is we will be inversing the operations of gradient divergence and curl using integrals.

\subsection{Fundamental theorem for gradient}
The fundamental theorem for gradients states that for any scalar field $F$
\[ \boxed{ \int_{\vec{a}}^{\vec{b}}(\grad F)\cdot \diff{\vec{l}} = F(\vec{b}) - F(\vec{a})  } \]

An intuitive explanation can be given as follows: TODO figure.

This fundamental theorem is valid for any curve. Thus for any scalar field, the vector field $\grad F$ is \ueig{conservative}.

\subsection{Fundamental theorem for divergence}
The fundamental theorem for divergences is also known as \textbf{Gauss's theorem}, \textbf{Green's theorem} or simply the \textbf{divergence theorem}.
\[ \boxed{ \iiint_{\mathcal{V}}(\nabla\cdot \vec{v}\diff{\tau}) = \oiint_{\mathcal{S}} \vec{v}\cdot \diff{\vec{a}} } \]
Where $\mathcal{S}$ is the surface of the volume $\mathcal{V}$.

If we view the vector field as the flow of a fluid, the divergence theorem can be interpreted as
\[ \iiint(\text{sources within the volume}) = \oiint (\text{flow out through the surface}). \]
\subsection{Fundamental theorem for curl}
The fundamental theorem for curls is known as \textbf{Stokes' theorem}.
\[ \boxed{ \iint_{\mathcal{S}}(\curl \vec{v})\cdot \diff{\vec{a}} = \oint_{\mathcal{P} \vec{v}\cdot \diff{\vec{l}}} } \]
Where $\mathcal{P}$ is the perimeter of the path $\mathcal{S}$.

Two comments
\begin{itemize}
\item The expressions on both sides of the equals sign have a sign ambiguity: The surface integral changes sign if you orient the surface differently and the line integral changes sign if you change direction of traversal.
\item The surface integral does not depend on the exact surface chosen, only on its perimeter. This also means that
\[ \oiint(\curl \vec{v})\cdot \diff{\vec{a}} = 0. \]
\end{itemize}

\section{Integrating by parts}
Integration by parts in vector calculus is analogous to the one dimensional case:
Use the product rule, integrate both sides and invoke the fundamental theorem. For vector calculus we have more product rules to exploit. Assume $f$ is a scalar field and $\vec{A}$ and $\vec{B}$ are vector fields.
\[ \iiint_{\mathcal{V}}f(\nabla\cdot \vec{A}) \diff{\tau} = - \iiint_{\mathcal{V}}\vec{A}\cdot(\grad f) \diff{\tau} + \oiint_{\mathcal{S}}f\cdot (\vec{A}\cdot \diff{\vec{a}}) \]
\[ \iint_\mathcal{S} f(\curl \vec{A})\cdot \diff{\vec{a}} = \iint_\mathcal{S}[\vec{A}\times(\grad f)]\cdot \diff{\vec{a}} + \oint_\mathcal{P} f\cdot (\vec{A}\cdot \diff{\vec{l}}) \]
\[ \iiint_\mathcal{V} \vec{B}\cdot (\curl \vec{A})\diff{\tau} = \iiint_\mathcal{V} \vec{A}\cdot(\curl \vec{B})\diff{\tau} + \oiint_{\mathcal{S}}(\vec{A}\times \vec{B})\cdot \diff{\vec{a}} \]

\section{Other coordinate systems}
\subsection{General coordinate transformations}
TODO


The formula for $\nabla\cdot X$ is incorrect. The notation with the 'usual' dot product is misleading. Properly it is for a diagonal metric:
\[\nabla\cdot F = \frac 1\rho\frac{\partial(\rho F^i)}{\partial x^i}\]
where $\rho=\sqrt{\det g}$ is the coefficient of the differential volume element $dV=\rho\, dx^1\wedge\ldots \wedge dx^n$, meaning $\rho$ is also the Jacobian determinant, and where $F^i$ are the components of $F$ with respect to an unnormalized basis.

In polar coordinates we have $\rho=\sqrt{\det g}=r$, and:
\[\nabla\cdot X = \frac 1r \frac{\partial(r X^r)}{\partial r} 
+ \frac 1r\frac{\partial(r X^\theta)}{\partial \theta}\]

In the usual normalized coordinates $X=\vhat X^{r}\frac{\partial}{\partial r} + \vhat X^{\theta}\frac 1r\frac{\partial}{\partial\theta}$ this becomes:
\[\nabla\cdot X = \frac 1r \frac{\partial(r \vhat X^{r})}{\partial r} 
+ \frac 1r\frac{\partial \vhat X^{\theta}}{\partial \theta}\]
which agrees with the usual formula given in calculus books.

\subsection{Spherical coordinates}
\begin{itemize}
\item \textit{Gradient}
\[ \grad F = \pd{F}{r}\vhat{r} + \frac{1}{r} \pd{F}{\theta}\vhat{\theta} + \frac{1}{r\sin\theta}\pd{F}{\phi}\vhat{\phi} \]
\item \textit{Divergence}
\[ \nabla\cdot \vec{v} = \frac{1}{r^2}\pd{r^2v_r}{r} + \frac{1}{r\sin\theta}\pd{\sin\theta v_\theta}{\theta} + \frac{1}{r\sin\theta}\pd{v_\phi}{\phi} \]
\item \textit{Curl}
\[ \curl \vec{v} = \frac{1}{r\sin\theta}\left(\pd{\sin\theta v_\theta}{\theta} - \pd{v_\theta}{\phi}\right)\vhat{r} + \frac{1}{r}\left(\frac{1}{\sin\theta}\pd{v_r}{\phi} - \pd{rv_\phi}{r}\right)\vhat{\theta} + \frac{1}{r}\left(\pd{rv_\theta}{r} - \pd{v_r}{\theta}\right)\vhat{\phi}  \]
\item \textit{Laplacian}
\[ \nabla^2 F = \frac{1}{r^2} \pd{}{r}\left(r^2 \pd{F}{r}\right) + \frac{1}{r^2\sin\theta} \pd{}{\theta}\left(\sin\theta \pd{F}{\theta}\right) + \frac{1}{r^2\sin^2}\theta\pd[2]{F}{z} \]
\end{itemize}
\subsection{Cylindrical coordinates}
\begin{itemize}
\item \textit{Gradient}
\[ \grad F = \pd{F}{s}\vhat{s} + \frac{1}{s} \pd{F}{\phi}\vhat{\phi} + \pd{F}{z}\vhat{z} \]
\item \textit{Divergence}
\[ \nabla\cdot \vec{v} = \frac{1}{s}\pd{sv_s}{s} + \frac{1}{s}\pd{v_\phi}{\phi} + \pd{v_z}{z} \]
\item \textit{Curl}
\[ \curl \vec{v} = \left(\frac{1}{s}\pd{v_z}{\phi} - \pd{v_\phi}{z}\right)\vhat{s} + \left(\pd{v_s}{z} - \pd{v_z}{s}\right)\vhat{\phi} + \frac{1}{s}\left(\pd{sv_\phi}{s} - \pd{v_s}{\phi}\right)\vhat{z}  \]
\item \textit{Laplacian}
\[ \nabla^2 F = \frac{1}{s} \pd{}{s}\left(s \pd{F}{s}\right) + \frac{1}{s^2} \pd[2]{F}{\phi} + \pd[2]{F}{z} \]
\end{itemize}
\subsection{Polar coordinates}
This is just like for cylindrical coordinates, except nothing depends on the $z$ coordinate and vectors of a vector field do not have a component in the $z$ direction. Thus we may set $v_z$, and anything that is operated on by $\pd{}{z}$, to zero.
\begin{itemize}
\item \textit{Gradient}
\[ \grad F = \pd{F}{s}\vhat{s} + \frac{1}{s} \pd{F}{\phi}\vhat{\phi} \]
\item \textit{Divergence}
\[ \nabla\cdot \vec{v} = \frac{1}{s}\pd{sv_s}{s} + \frac{1}{s}\pd{v_\phi}{\phi} \]
\item \textit{Curl}
\[ \curl \vec{v} = \frac{1}{s}\left(\pd{sv_\phi}{s} - \pd{v_s}{\phi}\right)\vhat{z}  \]
\item \textit{Laplacian}
\[ \nabla^2 F = \frac{1}{s} \pd{}{s}\left(s \pd{F}{s}\right) + \frac{1}{s^2} \pd[2]{F}{\phi} \]
\end{itemize}

\section{Potentials}
\subsection{Irrotational fields}
\udef{Irrotational fields} are fields where the curl vanishes everywhere.
\begin{eigenschap}
For a vector field $\vec{F}$ the following conditions are equivalent:
\begin{enumerate}[(a)]
\item $\curl \vec{F} = 0 \;$ everywhere.
\item $\int_{\vec{a}}^{\vec{b}}\vec{F}\cdot \diff{\vec{l}}$ is independent of the path, for any given end points.
\item $\oint \vec{F}\cdot \diff{\vec{l}} = 0$ for any closed loop.
\item $\vec{F}$ is the gradient of some scalar function, called the \udef{potential}
\[ \vec{F} = - \grad V \]
\end{enumerate}
\end{eigenschap}
The potential is not unique, any constant can be added to $V$ without changing $\grad V$. The minus sign in the definition is conventional.
\subsection{Solenoidal fields}
\udef{Solenoidal fields} are fields where the divergence vanishes everywhere.
\begin{eigenschap}
For a vector field $\vec{F}$ the following conditions are equivalent:
\begin{enumerate}[(a)]
\item $\nabla\cdot \vec{F} = 0 \;$ everywhere.
\item $\iint \vec{F}\cdot \diff{\vec{a}}$ is independent of the exact path, for a given perimeter.
\item $\oiint \vec{F}\cdot \diff{\vec{a}} = 0$ for any closed surface.
\item $\vec{F}$ is the curl of some vector function, called the \udef{vector potential}
\[ \vec{F} = \curl \vec{A} \]
\end{enumerate}
\end{eigenschap}
\subsection{Helmholtz theorem}
\subsubsection{Decomposition in irrotational and solenoidal field}
An arbitrary vector field $\vec{F}$ can always be written as the sum of an irrotational and a solenoidal field, and thus also as the sum of the gradient of a scalar and the curl of a vector.
\[ \vec{F} = - \grad V + \curl \vec{A} \]
This is sometimes known as the \textbf{Helmholtz decomposition}.
\subsubsection{Fields with prescribed divergence and curl}
Say we have a scalar field $D$ and a solenoidal vector fields $\vec{C}$. Can we find a vector field $\vec{F}$ such that the divergence and curl are given by $D$ and $\vec{C}$ respectively?
\[ \begin{cases}
\nabla\cdot \vec{F} = D \\ \curl \vec{F} = \vec{C}
\end{cases} \]
The answer is \textit{yes} if $\vec{C}(\vec{r})$ and $D(\vec{r})$ go to zero at infinity faster than $\frac{1}{r^2}$. They even define $\vec{F}$ uniquely if $\vec{F}$ goes to zero at infinity.

The vector field field can be constructed as follows:
\[ \vec{F} = -\nabla\cdot U + \curl \vec{W} \]
where
\[ \begin{cases}
U(\vec{r}) = \frac{1}{2\pi}\int \frac{D(\vec{r'})}{|\vec{r} - \vec{r'}|}\diff{\tau'} \\
\vec{W}(\vec{r}) = \frac{1}{2\pi}\int \frac{\vec{C}(\vec{r'})}{|\vec{r} - \vec{r'}|}\diff{\tau'}
\end{cases} \]

\section{Laplace's equation}
TODO intro + see electro
\subsection{Uniqueness theorems}
\subsection{Method of images}
\subsection{Seperation of variables}





\appendix

\chapter{Symbols}


\listoftheorems[title={List of named results}, swapnumber,ignoreall,onlynamed]

%\printindex[definition]


\chapter{Bibliography}

\url{https://en.wikipedia.org/wiki/Philosophy_of_mathematics}

Categories for the working mathematician, Saunders Mac Lane

Mathematical Foundations of Game Theory
An Introduction to Decision Theory ?

Philosophy of Mathematics (Princeton Foundations of Contemporary Philosophy)
\url{https://plato.stanford.edu/entries/philosophy-mathematics/#Cat}
An Introduction to the Philosophy of Mathematics (Cambridge Introductions to Philosophy)
The Oxford Handbook of Philosophy of Mathematics and Logic, Stewart Shapiro
BELIEVING THE AXIOMS. PENELOPE MADDY
Philosophy of Mathematics, Jeremy Avigad
Thinking about mathematics, Shapiro


\begin{itemize}
\item Logicaboek, Batens
\item A friendly introduction to mathematical logic, Christopher C. Leary
\item A course in mathematical logic, John Bell and Moshé Machover
\item Mathematical Logic, Lightstone
\item Handbook of Mathematical Logic, Jon Barwise ed.
\item \url{http://www2.hawaii.edu/~robertop/Courses/TMP/7_Peano_Axioms.pdf}
\item Axiom of choice, Jech
\item \url{https://arxiv.org/abs/0904.4205v1}
\item \url{http://philsci-archive.pitt.edu/2703/1/reconstruction2.pdf}
\item Raymond Smullyan on Self Reference
\item Anticipatory Systems; Philosophical, Mathematical, and Methodological Foundations
\item A Shorter model theory Hodges
\item Marker's Model Theory; An Introduction
\item Equivalence: an attempt at a history of the idea
\item A book of Set Theory, Charles C. Pinter
\item Enderton's "Elements of Set Theory
\item I. V. Volovich, ``Number theory as the ultimate physical theory'', Preprint No. TH 4781/87, CERN, Geneva, (1987)
\item Weyl, Philosophy of Mathematics and Natural Science
\item Heath, Thomas L. (1956). The Thirteen Books of Euclid's Elements (2nd ed. [Facsimile. Original publication: Cambridge University Press, 1925] ed.). New York: Dover Publications. In 3 vols.: vol. 1 ISBN 0-486-60088-2, vol. 2 ISBN 0-486-60089-0, vol. 3 ISBN 0-486-60090-4. Heath's authoritative translation of Euclid's Elements, plus his extensive historical research and detailed commentary throughout the text.
\item Birkhoff, George David (1932), "A Set of Postulates for Plane Geometry (Based on Scale and Protractors)", Annals of Mathematics, 33: 329–345, doi:10.2307/1968336
\item Birkhoff's Axioms for Space Geometry
Roland Brossard
The American Mathematical Monthly
Vol. 71, No. 6 (Jun. - Jul., 1964), pp. 593-606 (14 pages)
\item Backgrounds Of Arithmetic And Geometry: An Introduction
By Branzei Dan, Miron Radu
\item \url{http://lya.fciencias.unam.mx/gfgf/ga20101/material/analyticgeometry_chap1.pdf}
\item Topology, James Munkres
\item Counterexamples in topology
\item Real analysis in reverse: \url{https://arxiv.org/pdf/1204.4483.pdf}
\item A garden of integrals, Frank E. Burk
\item A textbook on ordinary differential equations, Shair and Ambrosetti
\item Partial differential equations, Lawrence C. Evans
\item Partial differential equations in action, Sandro Salsa
\item Wikipedia: real numbers,
\item Categories for the Working Mathematician, Saunders Mac Lane
\item Category Theory for the Sciences (The MIT Press): David I. Spivak
\item \url{https://www.math.mcgill.ca/triples/Barr-Wells-ctcs.pdf}
\item Abraham Robinson (1996):Non-Standard Analysis, revised edition. Princeton University Press, Princeton,New Jersey.
\item Lectures on the Hyperreals
\item The way of the Infinitesimal, Fabrizio Genovese
\item A guide to distribution theory and fourier transforms, Robert S. Strichartz
\item Hall, Brian C. (2003). \textit{Lie Groups, Lie Algebras, and Representations}. New York: Springer.
\item Machì, Antonio. (2012). \textit{Groups : an introduction to ideas and methods of the theory of groups}. Milan: Springer.
\item Brian G. Wybourne, Classical groups for physicists
\item Varadarajan, Lie groups
\item Nastase, Classical field theory
\item Introduction to Differentiable Manifolds, Serge Lang
\item A Comprehensive Introduction to Differential Geometry, Michael Spivak
\item W. Schläg - Complex Analysis and Riemann Surfaces
\item L.W. Tu - An Introduction to Manifolds
\item (Lee)
\item Blanché, Axiomatics
\item Kunen, Set theory, An introduction to independence proofs
\item Feldman, Epistemology
\item BonJour, Epistemology
\item Doets, Basic model theory
\item Hodges shorter model theory
\item Moschovakis, Notes on set theory
\item Theory of formal systems, Smullyan
\item How  to  Prove  It:  A  Structured  Approach
\item \url{https://www.drmaciver.com/2015/12/you-can-tell-about-zorns-lemma/}
\item Istrăţescu, Inner product structures
\item Axler, Linear Algebra Done Right
\item Kreyszig - Introductory Functional Analysis with Applications
\item Conway, A Course in Functional Analysis
\item Helemskii - Lectures and exercises on functional analysis
\item Steven Roman - Advanced Linear Algebra
\item Narici, Beckenstein - Topological Vector Spaces
\item Schmidt, relational mathematics
\item \url{https://www.math.ksu.edu/~nagy/real-an/}
\item Jordan, Smith - Mathematical techniques
\item Dunford N., Schwartz J. - Linear operators
\item Fillmore - A user's guide to operator algebras.
\item Blackadar - Operator Algebras.
\item M.  Rørdam,  F.  Larsen,  N.  Laustsen - An  introduction  to  K-theory  for $C^*$-algebras.
\item Dana P. Williams - Crossed products of C-star algebras
\item Pedersen, Gert Kjaergård - C-star-algebras and their automorphism groups
\item Spin geometry - Lawson
\item Notes on Category Theory - Paolo Perrone
\item Introduction to Tensor Products of Banach Spaces - Ryan
\item Postmodern Analysis - Jost
\item Harmonic Analysis of Operators on Hilbert Space
\item Sheaves in Geometry and Logic
\item Introduction to set theory - Monk
\item Infinite Dimensional Analysis: A Hitchhiker’s Guide
\item Handbook of mathematics - Thierry Vialar
\item Probability: A Graduate Course - Allan Gut
\item Probability: Theory and Examples - Rick Durrett
\item A second course in linear algebra - Garcia, Horn
\item Matrix Analysis - Horn, Johnson
\item Markov processes, semigroups and generators - Kolokoltsov V.N.
\item Topics in Linear and Nonlinear Functional Analysis - Gerald Tesch
\item Calculus on Normed Vector Spaces - Coleman
\item Differential Calculus in Topological Linear Spaces - Yamamuro
\item Galois Connections and Applications
\item Term Rewriting and All That - Franz Baader, Tobias Nipkow
\item Term Rewriting Systems - Terese
\item Ordinary Differential equations - Hartman
\item Convergence foundations of topology - Dolecki, Mynard
\item Techniques of Functional Analysis for Differential and Integral Equations - Sacks
\item A Hilbert Space Problem Book - Halmos
\item Introduction to Operator Theory in Riesz Spaces - Zaanen
\item \url{https://www.math.uwaterloo.ca/~snburris/htdocs/UALG/univ-algebra2012.pdf}
\item Lattices and Ordered Algebraic Structures - T.S. Blyth
\item \url{https://www.mat.univie.ac.at/~gerald/ftp/book-fa/fa.pdf}
\item A Course on Topological Vector Spaces - Voigt
\item Pre-Riesz spaces - Kalauch, van Gaans
\item Convergence Structures and Applications to Functional Analysis - Beattie, Butzmann
\item Continuous Convergence on C(X) - Binz
\item Introduction to Spectral Theory - Hislop, Sigal
\item A Course in Topological Combinatorics - de Longueville
\item \url{https://people.bath.ac.uk/mw2319/ma30252/lecturenotes.html}
\item Frames and Locales, Topology without points - Picado, Pultr
\item \url{https://arxiv.org/pdf/0907.5356.pdf}
\item Clifford Algebras. An Introduction. - Garling
\item An Introduction to Clifford Algebras and Spinors - Vaz, da Rocha
\item Topological vector spaces - Robertson, Robertson
\item Principles of Harmonic Analysis - Deitmar, Echterhoff
\item One-Parameter Semigroups for Linear Evolution Equations - Engel, Nagel
\item Equations of Evolution - Tanabe
\item \url{https://link.springer.com/article/10.1007/s11225-012-9465-0}
\item \url{https://www.bioinf.uni-leipzig.de/~studla/Publications/PREPRINTS/01-pfs-007-subl1.pdf}
\item Fourier Analysis - Körner
\item Fourier Analysis - Folland
\item Classical Fourier Analysis - Grafakos
\item Modern Fourier Analysis - Grafakos
\item Measure Theory - Bogachev
\item Measure Theory - Cohn
\item Measure Theory - Tao
\item Measure and Integration Theory - Bauer
\item Lectures on Analysis on Metric Spaces - Heinonen
\item Geometric Measure Theory - Federer
\item Geometric Measure Theory, A Beginner's Guide - Morgan
\item Measure Theory and Fine Properties of Functions - Evans, Gariepy
\item Axioms for infinite matroids
\item Graph Theory - Wilson
\item Intro to convergence theory with filters: 
url{https://www.bioinf.uni-leipzig.de/~studla/Publications/PREPRINTS/01-pfs-007-subl1.pdf}
\end{itemize}

\printbibliography


\end{document}
