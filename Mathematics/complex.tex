TODO
\[ -1 = i^2 = \sqrt{-1}\sqrt{-1} = \sqrt{(-1)^2} = \sqrt{1} = 1 \]


The set of the complex numbers is denoted $\C$.

\begin{lemma} \label{boundedThenReal}
Let $z\in\C$. Suppose there is a $C\geq 0$ such that
\[ \forall t\in\R: \quad |z+it|^2\leq C+t^2, \]
then $z\in \R$.
\end{lemma}
\begin{proof}
Write $z = a+bi$ for some $a,b\in \R$. Then
\[ |z+it|^2-t^2 = a^2 + (b+t)^2 - t^2 = a^2+b^2+2bt. \]
The left side is bounded by $C$ for all $t\in\R$. If $b>0$, the right side is unbounded for $t\to +\infty$. If $b<0$, the right side is unbounded for $t\to -\infty$. So we need $b=0$ and thus $z=a\in\R$.
\end{proof}

\section{Solutions to quadratic equations}
 We define\footnote{In engineering $i$ is often called $j$, because $i$ is used to denote electric current}
\[ i \equiv \sqrt{-1} \]
We call $i$ the \udef{imaginary unit}. The choice of terminology is not great here; it creates an artificial divide between the ``real'' numbers and this new ``imaginary'' thing. Many a maths teacher has taken great pains to explain that imaginary and complex numbers are no more imaginary than real numbers. I would tend to make the opposite case: all numbers are imaginary mathematical constructs that happen to be quite useful. We are just more familiar with real numbers.

So let us, arguendo, accept this new mathematical object. We are now faced with two questions: is it useful and what are the consequences? We will start by exploring the consequences.

We have already remarked upon the fact that the real numbers form a field. It makes some sense to try to find a field that contains both the real numbers and this new imaginary unit. There may be many such fields. We will try to find the smallest, which we will call the \udef{complex numbers}, notated $\C$. Obviously this new field contains $\R \cup \{i\}$ as a subset, but this in itself is \emph{not} a field. As stated above, in a field both multiplication and addition work well and in particular give results that are still part of the field.  For instance, imagine multiplying $i$ with a real number, like $2$ (something that we can do in a field by definition). We can show that $2i$ is not an element of $\R \cup \{i\}$. Now $2i$ cannot be a real number, because
\begin{align*}
(2i)^2 &= 2i2i \\
&= 4i^2 \qquad \text{by commutativity} \\
&= -4
\end{align*}
and we know that the square of a real number cannot be negative. It might be the case that $2i = i$, but subtracting $i$ from both sides of the equation would give us $i=0$, which can quite easily be seen to be contradictory. This means $2i$ is an entirely new number not yet considered. We could give it a new name (like we did for $\sqrt{-1}$) if we wanted to, but given we can multiply $i$ by any real number and get a new object, we will not start naming them just yet. So we can sum up our current knowledge as follows
\[ \C \; \supset \; \R \cup \{a\cdot i \; | \; a \in \R_0\} \]
We call the set $\{a\cdot i \; | \; a \in \R_0\}$ the \udef{imaginary numbers}. We are not finished yet, because there are still more numbers that must be included in the set of complex numbers. Take for instance $i + 1$. Again this must be a complex number if we wish the complex numbers to be a field. Performing simple calculations as before we can see that $i + 1$ cannot be either a real number (as that would mean that $i$ was a real number minus $1$) or a purely imaginary one (assuming $i + 1 = a\cdot i$ imaginary, we would see that $1 = (a-1)i$ which can easily be seen to be absurd). Consequently we can see that if we add a real number to an imaginary number, the result is a new complex number that we have not considered yet. The complex numbers must therefore include all numbers of the form $a + bi$ and for each $a,b \in \R$ we have a unique complex number.

Finally we can remark that the set of complex numbers that we have created so far is in fact a field. We can verify for example that the result of addition or multiplication of two numbers of the form $a+bi$ can also be written in that form (making use of the fact that the addition and multiplication operations fulfill the requirements to be part of a field):
\begin{align*}
(a_1 + b_1i) + (a_2 + b_2i) &= (a_1 + a_2) + (b_1 + b_2)i \\
(a_1 + b_1i)(a_2 + b_2i) &= (a_1a_2 - b_1b_2) + (a_1b_2 + a_2b_1)i
\end{align*}
Because we were looking for the smallest field that contains both the real numbers and the imaginary unit, we can stop here.

The treatment so far has included the essential arguments necessary to prove that all complex numbers can be written as
\[a + bi \qquad \text{with} \qquad a,b \in \R.\]
In other words we can say
\[ \C = \{a + bi \; | \; a,b \in \R \}. \]

\section{How to represent complex numbers}
In the previous section we saw that every complex number can be written as $a + bi (a,b \in \R)$. Conversely for every $a,b $ in $\R$ there is a unique complex number $a + bi$. Thus we can see that every complex number can be constructed using two real numbers. We give those real numbers special names: For a complex number $z = a + bi$, we call $a$ the real part (denoted $\Re(z)$) and $b$ the complex part (denoted $\Im(z)$).

TODO complex plane

modulus argument
Euler formula??
Conversions

\section{Practical calculations}
The following methods give a practical way to perform calculations with complex numbers. Assume we have two complex numbers $z_1$ and $z_2$. 
\subsection{Addition} is usually easiest if the complex numbers are in the form $a+bi$. Then we have
\begin{align*}
z_1 + z_2 &= (a_1 + b_1i) + (a_2 + b_2i) \\
&= (a_1+a_2) + (b_1+b_2)i
\end{align*} 
\subsection{Multiplication} is usually easiest if the complex numbers are in the form $re^{i\phi}$. Then we have
\begin{align*}
z_1 \cdot z_2 &= r_1e^{i\phi_1}\cdot r_2e^{i\phi_2} \\
&= (r_1\cdot r_2)e^{i(\phi_1+\phi_2)}
\end{align*}
So we multiply the moduli and add the arguments.
\subsection{Exponentiation} with an integer (or real) exponent is again usually easiest if the complex number is in the form $re^{i\phi}$.
\begin{align*}
z^n &= (re^{i\phi})^n \\
&= r^n e^{in\phi}
\end{align*}
\section{Trigonometry revisited}
\subsection{Waves and complex numbers}

Cayley-Dickinson