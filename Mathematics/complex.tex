TODO
\[ -1 = i^2 = \sqrt{-1}\sqrt{-1} = \sqrt{(-1)^2} = \sqrt{1} = 1 \]


The set of the complex numbers is denoted $\C$.

\begin{lemma} \label{boundedThenReal}
Let $z\in\C$. Suppose there is a $C\geq 0$ such that
\[ \forall t\in\R: \quad |z+it|^2\leq C+t^2, \]
then $z\in \R$.
\end{lemma}
\begin{proof}
Write $z = a+bi$ for some $a,b\in \R$. Then
\[ |z+it|^2-t^2 = a^2 + (b+t)^2 - t^2 = a^2+b^2+2bt. \]
The left side is bounded by $C$ for all $t\in\R$. If $b>0$, the right side is unbounded for $t\to +\infty$. If $b<0$, the right side is unbounded for $t\to -\infty$. So we need $b=0$ and thus $z=a\in\R$.
\end{proof}

\section{How to represent complex numbers}
In the previous section we saw that every complex number can be written as $a + bi (a,b \in \R)$. Conversely for every $a,b $ in $\R$ there is a unique complex number $a + bi$. Thus we can see that every complex number can be constructed using two real numbers. We give those real numbers special names: For a complex number $z = a + bi$, we call $a$ the real part (denoted $\Re(z)$) and $b$ the complex part (denoted $\Im(z)$).

TODO complex plane

modulus argument
Euler formula??
Conversions

\section{Practical calculations}
The following methods give a practical way to perform calculations with complex numbers. Assume we have two complex numbers $z_1$ and $z_2$. 
\subsection{Addition} is usually easiest if the complex numbers are in the form $a+bi$. Then we have
\begin{align*}
z_1 + z_2 &= (a_1 + b_1i) + (a_2 + b_2i) \\
&= (a_1+a_2) + (b_1+b_2)i
\end{align*} 
\subsection{Multiplication} is usually easiest if the complex numbers are in the form $re^{i\phi}$. Then we have
\begin{align*}
z_1 \cdot z_2 &= r_1e^{i\phi_1}\cdot r_2e^{i\phi_2} \\
&= (r_1\cdot r_2)e^{i(\phi_1+\phi_2)}
\end{align*}
So we multiply the moduli and add the arguments.
\subsection{Exponentiation} with an integer (or real) exponent is again usually easiest if the complex number is in the form $re^{i\phi}$.
\begin{align*}
z^n &= (re^{i\phi})^n \\
&= r^n e^{in\phi}
\end{align*}
\section{Trigonometry revisited}
\subsection{Waves and complex numbers}

Cayley-Dickinson