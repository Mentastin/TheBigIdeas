TODO MacLaurin, propto, other expansions (multipolar, binomial etc) 
Taylor polynomials, big O, possible big O types

\section{Sequences}
\subsection{Convergence}
Applying the definition of convergent sequences to sequences in $\R$ gives the so-called $\varepsilon-n_0$ criterion for convergence:
\begin{proposition}
Let $(x_n)$ be a real sequence. Then $(x_n)$ converges to $L$ \textup{if and only if}
\[ \forall \varepsilon> 0: \exists n_0\in\N: \forall n\in\N: n\geq n_0 \implies |x_n-L|<\varepsilon. \]
\end{proposition}
The definition for divergence to $\pm\infty$ is identical.

TODO Bolzano-Weierstrass

\subsubsection{Examples of sequences}
\begin{proposition}
Let $p\in\R$. Then
\[ \lim_{n\to\infty} n^p = \begin{cases}
+\infty & (p>0) \\
1 & (p=0) \\
0 & (p<0)
\end{cases}. \]
\end{proposition}

\begin{proposition}
Let $r\in\R$. Then
\[ \lim_{n\to\infty} r^n = \begin{cases}
+\infty & (r>1) \\
1 & (r=1) \\
0 & (-1<r<1) \\
\text{does not exist} & (r\leq -1)
\end{cases}. \]
\end{proposition}

\section{Series}
\begin{definition}
Let $\seq{a_n}_{n\in \N}$ be a sequence. The \udef{series} generated by this sequence is the sequence
\[ \seq{\sum_{i=0}^n a_i}_{n\in \N}. \]
The $n^\text{th}$ element of this sequence, $\sum_{i=0}^n a_n$, is called the \udef{$n^\text{th}$ partial sum}.

If the sequence of partial sums converges, we call the series \udef{convergent} otherwise it is \udef{divergent}.

The expression
\[ \sum_{i=0}^\infty a_n \]
may be used to denote the limit of the series or the series itself.
\begin{itemize}
\item If the series $\sum_{i=0}^\infty a_n$ converges to $+\infty$ as a sequence in $\overline{\R}$, we say it \udef{diverges to $+\infty$} and write $\sum_{i=0}^\infty a_n = +\infty$.
\item If it converges to $-\infty$ in $\overline{\R}$, we say it \udef{diverges to $-\infty$} and write $\sum_{i=0}^\infty a_n = -\infty$.
\end{itemize}
\end{definition}

\subsection{Difference calculus}

\begin{proposition}[Summation by parts]
Let $\seq{x_k}$ and $\seq{y_k}$ be sequences in some field and $m,n\in \N$. Then
\begin{align*}
\sum_{k=m}^n x_k\Delta^+y_k &= (x_ny_{n+1} - x_{m-1}y_m)-\sum_{k=m}^ny_k\Delta^-x_k \\
&= (x_ny_{n+1} - x_my_m) - \sum_{k=m}^{n-1}y_{k+1}\Delta^+x_k.
\end{align*}
\end{proposition}

\subsection{Difference calculus on sequences}
TODO move!!

TODO $\Delta^+_\epsilon, \Delta^-_\epsilon$ for real functions! Also central difference $\Delta$.
\subsubsection{Difference operators}
\begin{definition}
Let $\seq{x_n}$ be a sequence. We define
\begin{itemize}
\item the \udef{forward difference operator} $\Delta^+: \R^\N \to \R^\N$ by $\Delta^+ x_n \defeq (\Delta^+x)_n \defeq x_{n+1} - x_n$; and
\item the \udef{backward difference operator} $\Delta^-: \R^\N \to \R^\N$ by $\Delta^- x_n \defeq (\Delta^-x)_n \defeq x_{n} - x_{n-1}$.
\end{itemize}
\end{definition}

\subsection{Series of positive real numbers}
\url{https://en.wikipedia.org/wiki/Convergence_tests}
\subsubsection{Ratio test hierarchy}
\url{https://en.wikipedia.org/wiki/Ratio_test}
\subsubsection{Root test hierarchy}
\url{https://en.wikipedia.org/wiki/Root_test#Root_tests_hierarchy}

\begin{proposition}[Cauchy's criterion] \label{rootTest}
Let $\sum_{n=0}^\infty a_n$ be a positive series. Define the number
\[ C = \limsup_{n\to \infty}\sqrt[n]{a_n}. \]
Then
\begin{itemize}
\item if $C<1$, the series converges;
\item if $C>1$, the series diverges;
\item if $C=1$, the test is inconclusive.
\end{itemize}
\end{proposition}


\subsection{Series in normed abelian groups}
\begin{definition}
Let $\sum_{i=0}^\infty a_i$ be a convergent series. We call the series
\begin{itemize}
\item \udef{absolutely convergent} if the series $\sum_{i=0}^\infty \norm{a_i}$ is convergent;
\item \udef{unconditionally convergent} if for all bijections $\sigma: \N\to \N$, the series $\sum_{i=0}^\infty a_{\sigma(i)}$ is convergent and \udef{conditionally convergent} otherwise.
\end{itemize}
\end{definition}

\begin{proposition}
Let $\sum_{i=0}^\infty a_i$ be a series in a complete abelian normed group. If $\sum_{i=0}^\infty a_i$ is absolutely convergent, it is convergent.
\end{proposition}
\begin{proof}
Reverse triangle inequality TODO!
\end{proof}

\begin{proposition}
\url{https://en.wikipedia.org/wiki/L%C3%A9vy%E2%80%93Steinitz_theorem}
\end{proposition}
\begin{corollary}[Riemann series theorem]
Let $\sum_{i=0}^\infty a_i$ be a convergent series that does not converge absolutely and $L\in \overline{\R}$. There exists a bijection $\sigma:\N\to\N$ such that
\[ \sum_{i=0}^\infty a_{\sigma(i)} = L. \]
There also exists a bijection $\tau:\N\to\N$ such that $\sum_{i=0}^\infty a_{\tau(i)}$ fails to approach any limit, finite or infinite.
\end{corollary}
\begin{proof}
\url{https://en.wikipedia.org/wiki/Riemann_series_theorem}
\end{proof}
\begin{corollary}
A convergent series is unconditionally convergent \textup{if and only if} it is absolutely convergent.
\end{corollary}



\subsection{Convergence}
\begin{theorem}[Tannery's theorem] \label{tannery}
Let $s_n = \sum_{k=0}^\infty a_{n,k}$ be the limit of a convergent series for each $n$ such that $\lim_{n\to \infty}a_{n,k}$ converges to $a_k$ for all $n$. If there exists a sequence $M_k$ such that $|a_{n,k}|\leq M_k$ for all $n,k$ and $\sum_{k=0}^\infty M_k<\infty$, then
\[ \lim_{n\to\infty} s_n = \lim_{n\to\infty}\sum_{k=0}^\infty a_{n,k} = \sum_{k=0}^\infty\lim_{n\to\infty} a_{n,k} = \sum_{k=0}^\infty a_k.   \]
\end{theorem}
\begin{proof}
Choose an arbitrary $\varepsilon>0$. For any $n,N\in\N$ we can write
\[ \left|s_n - \sum_{k=0}^\infty a_k\right| \leq \sum_{k=0}^\infty\left|a_{n,k} -  a_k\right|\leq \sum_{k=0}^N\left|a_{n,k} -  a_k\right| + 2\sum_{k>N}M_k \leq N\max_{k<N}\left|a_{n,k} -  a_k\right| + 2\sum_{k>N}M_k. \]
So we aim to find some $N_0$ such that $2\sum_{k>N}M_k \leq \varepsilon/2$ for all $N\geq N_0$, which of course we can. Then we choose an $n_0$, in function of this $N_0$ and $\varepsilon$, such that $\max_{k<N_0}\left|a_{n,k} -  a_k\right|\leq \varepsilon/(2N_0)$ for all $n\geq n_0$. It is clear we can do so for each $k$ separately, but there are only finitely many $k$s so we take the largest $n_0$. Then
\[ \left|s_n - \sum_{k=0}^\infty a_k\right| \leq \varepsilon \qquad \text{for all $n\geq n_0$, implying the limit is zero.} \]
\end{proof}
\url{https://www.coloradomesa.edu/math-stat/documents/JohnGillresearchnoteTanneryTheorem.pdf}



\subsection{Examples of series}
\subsubsection{Geometric series}
\subsubsection{Harmonic series}

\section{Infinite sums}
\begin{definition}
Let $I$ be an index set and $a: I\to \R^+$ a positive function. Then we define the \udef{infinite sum}
\[ \sum_{i\in I}a_i \defeq \sup\setbuilder{\sum_{i\in F}a_i}{\text{$F\subseteq I$ finite}}. \]
\end{definition}
If $I = \N$, then the notation conflicts with the notation for the limit of an infinite series. These notions coincide:

\begin{lemma}
Let $a: \N\to \R^+$ be a positive function. Then
\[ \lim_{n\to\infty}\sum_{i=0}^n = \sup\setbuilder{\sum_{i\in F}a_i}{\text{$F\subseteq \N$ finite}}. \]
\end{lemma}
\begin{proof}
The initial segments of $\N$ are in particular finite subsets and the limit of an increasing sequence is its supremum, so the inequality $\leq$ follows.

Now for any finite $F\subseteq \N$, there exists $m\in \N$ such that $F\subseteq \interval{0,m}$ and so
\[ \sum_{i\in F}a_i \leq \sum_{i=0}^m. \]
This proves the other inequality.
\end{proof}


\begin{proposition} \label{finiteSumsAreCountable}
Let $I$ be an index set and $a: I\to \R^+$ a positive function. If $\sum_{i\in I}a_i < \infty$, then $a_i\neq 0$ for at most countably many $i\in I$.
\end{proposition}
\begin{proof}
Define $A_n = \setbuilder{i\in I}{a_i \geq n^{-1}}$ and $A = \bigcup_{n\in \N}A_n = \setbuilder{i\in I}{a_i > 0}$. Suppose, towards a contradiction, that $A$ is uncountable. Then, by at least one of the $A_n$ must be uncountable (if not, $A$ would be countable by \ref{repeatedAdditionMultiplicationCardinals} and \ref{stringsInCountableAlphabetCountable}). We have
\begin{align*}
\sum_{i\in I}a_i &= \sup\setbuilder{\sum_{i\in F}a_i}{\text{$F\subseteq I$ finite}} \\
&\geq \sup\setbuilder{\sum_{i\in F}a_i}{\text{$F\subseteq A_n$ finite}} \\
&\geq \sup\setbuilder{\sum_{i\in F}n^{-1}}{\text{$F\subseteq A_n$ finite}} \\
&= \sum_{j\in \N}n^{-1} = \infty.
\end{align*}
\end{proof}

\section{Functions defined by series}
\subsection{Power series}
\begin{definition}
A \udef{power series} is a partial function of the form
\[ f: \C\not\to \C: z\mapsto \sum_{i=0}^\infty a_i(z-z_0)^i, \]
where $z_0\in \C$ and $\seq{a_n}$ is a sequence of complex numbers. 
\end{definition}
TODO more general algebras.

\begin{proposition}[Cauchy-Hadamard] \label{CauchyHadamard}
Let $f: \C\not\to \C: z\mapsto \sum_{i=0}^\infty a_i(z-z_0)^i$ be a power series. Define the real number $R$ by
\[ \frac{1}{R} \defeq \limsup_{n\to\infty}\left(|a_n|^{1/n}\right). \]
For all $z\in \C$ such that $|z-z_0| < R$, the value $f(z)$ is well-defined.
\end{proposition}
If $\limsup_{n\to\infty}\left(|a_n|^{1/n}\right) \to \infty$, we consider $R$ to be zero.
\begin{proof}
\ref{rootTest} TODO.
\end{proof} 

\begin{definition}
The $R$ defined in \ref{CauchyHadamard} is called the \udef{radius of convergence} of the power series.
\end{definition}

\subsubsection{Taylor and MacLaurin}
\subsection{Laurent series}
\begin{definition}
A \udef{Laurent series} is a partial function of the form
\[ f: \C\not\to \C: z\mapsto \sum_{i=-\infty}^\infty a_i(z-z_0)^i \defeq \sum_{i=0}^\infty a_i(z-z_0)^i + a_{-i}(z-z_0)^{-i}, \]
where $z_0\in \C$ and $\seq{a_n}, \seq{a_{-n}}$ are sequences of complex numbers.

The series $\sum_{i=1}^\infty a_{-i}(z-z_0)^{-i}$ is called the \udef{principal part} of the Laurent series.
\end{definition}

\begin{proposition} \label{LaurentSeriesConvergence}
Let $f: \C\not\to \C: z\mapsto \sum_{i=-\infty}^\infty a_i(z-z_0)^i$ be a Laurent series. Define the real numbers $r$ and $R$ by
\begin{align*}
r &\defeq \limsup_{n\to\infty}\left(|a_{-n}|^{1/n}\right); \\
\frac{1}{R} &\defeq \limsup_{n\to\infty}\left(|a_n|^{1/n}\right).
\end{align*}
For all $z\in \C$ such that $r < |z-z_0| < R$, the value $f(z)$ is well-defined.
\end{proposition}
So the domain of convergence of a Laurent series is an annulus around $z_0$.

\subsection{Puiseux series}
\url{https://en.wikipedia.org/wiki/Puiseux_series}

\section{Sequences and series in normed structures}

\section{Matrix exponential}
The matrix exponential is quite simply defined as the MacLaurin series of the normal exponential applied to square matrices.

\begin{definition}
Let $X$ be an $n\times n$ matrix. The exponential of $X$ is given by the power series
\[ e^X = \sum^\infty_{m=0} \frac{X^m}{m!}. \]
\end{definition}

It's nice to know that for any $n \times n$ real or complex matrix $X$, this series does actually converge. The matrix exponential is also a \ueig{continuous} function of $X$.

Here are also some elementary properties of the matrix exponential, that may be useful for somebody somewhere.
\begin{eigenschap}
Let $X$ be an arbitrary $n\times n$ matrix. Let $C$ be invertible.
\begin{enumerate}
\item $e^0 = \mathbb{1}_n$.
\item $\left(e^X\right)^\dagger = e^{X^\dagger}$.
\item $e^X$ is invertible and $\left(e^X\right)^{-1} = e^{-X}$.
\item $\left(e^X\right)^* = e^{A^*}$
\item $\left(e^X\right)^\dagger = e^{A^\dagger}$
\item $\left(e^X\right)^\intercal = e^{A^\intercal}$
\item $e^{CXC^{-1}} = Ce^XC^{-1}$
\end{enumerate}
\end{eigenschap}
It is in general \textbf{not} true that $e^{X+Y} = e^Xe^Y$; this is only true if $X$ and $Y$ commute.

Here are a number of properties of the matrix exponential that will be useful later. We will assume $X, Y$ are complex $n\times n$ matrix.
\begin{eigenschap}
The map $\R \to \C^{n\times n}: t \mapsto e^{tX}$ is a smooth curve in $\C^{n\times n}$ and
\[ \od{}{t}e^{tX} = Xe^{tX} = e^{tX}X. \]
In particular,
\[ \left.\od{}{t}e^{tX}\right|_{t=0} = X. \]
\end{eigenschap}

\begin{eigenschap}
\ueig{Lie product formula} 
\[ e^{X+Y} = \lim_{m\to\infty} \left(e^{\frac{X}{m}}e^{\frac{Y}{m}}\right)^m \]
\end{eigenschap}
Finally for the determinant we have:
\begin{eigenschap}
\[\det \left(e^X\right) = e^{\Tr(X)}\]
\end{eigenschap}

\section{Binomial theorem and binomial series}