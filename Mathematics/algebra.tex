\chapter{Universal algebra}
TODO: forgetful functors.

\section{Algebras and terms}
\begin{definition}
A \udef{signature} or \udef{operational type} or \udef{operator domain} is a pair $(\Omega, \alpha)$ where $\Omega$ is a set whose elements are called \udef{operator symbols} or just operators and $\alpha: \Omega \to \N$ is a function. We call $\alpha(\omega)$ the \udef{arity} of the operator $\omega\in\Omega$. If the arity of $\omega\in\Omega$ is $n$, then we say $\omega$ is an \udef{$n$-ary} operator.
\end{definition}
We also say \udef{unary} instead of $1$-ary, \udef{binary} instead of $2$-ary and \udef{ternary} instead of $3$-ary.

\begin{definition}
A \udef{structure} of type $(\Omega,\alpha)$, also called an \udef{$\Omega$-structure} or \udef{$\Omega$-algebra}, is a set $A$, called the \udef{carrier}, equipped with a function
\[ \omega_A: A^{\alpha(\omega)}\to A \]
for each $\omega\in\Omega$. We call $\omega_A$ the \udef{interpretation} of $\omega$ in $A$.

If $\alpha(\omega) = 0$ we take $\omega_A$ to be a constant.
\end{definition}

\begin{definition}
Let $A$ be an $\Omega$-algebra. An \udef{$\Omega$-subalgebra} of $A$ is a subset that is closed under the operations of $\Omega$.
\end{definition}

\begin{lemma} \label{lemma:intersectionSubalgebra}
Let $A$ be an $\Omega$-algebra. Let $\mathcal{E}$ be a family of subalgebras. Then $\bigcap \mathcal{E}$ is also a subalgebra. 
\end{lemma}

\begin{definition}
Let $A$ be an $\Omega$-algebra and $X$ a subset of $A$. The subalgebra of $A$ \udef{generated} by $X$ is the intersection of all subalgebras containing $X$. We call $X$ the \udef{generating set} of this subalgebra.
\end{definition}
Every algebra has a (non-unique) generating set. For example the algebra itself.

\begin{definition}
The \udef{trivial} $\Omega$-algebra is the algebra generated by $\emptyset$.
\end{definition}

\subsection{Homomorphisms}
\begin{definition}
Let $A,B$ be $\Omega$-algebras. A \udef{homomorphism} of $\Omega$-algebras is a function $f:A\to B$ such that
\[ \forall \omega\in\Omega, \forall a_1,\ldots,a_{\alpha(\omega)}\in A: \quad f(\omega_A(a_1,\ldots, a_{\alpha(\omega)})) = \omega_B(f(a_1), \ldots, f(a_{\alpha(\omega)})). \]
\end{definition}

\begin{proposition}
Let $f,g:A\to B$ be two homomorphisms between $\Omega$-algebras $A,B$. If $f,g$ agree on a generating set of $A$, then they are equal.
\end{proposition}
\begin{proof}
The set $\setbuilder{x\in A}{f(x) = g(x)}$ is a subalgebra of $A$. By hypothesis it contains a generating set of $A$ and thus it is all of $A$.
\end{proof}

\begin{proposition}
Let $f:A\to B$ be a homomorphism. Then $\im f$ is a subalgebra of $B$.
\end{proposition}

\begin{proposition}
Let $(\Omega,\alpha)$ be a signature. Then the $\Omega$-algebras form a category with homomorphisms as arrows.
\end{proposition}
Consequently we have the concepts of isomorphism, endomorphism and automorphism.

\begin{proposition} \label{prop:bijectiveHomomorphism}
Let $f:A\to B$ be a bijective homomorphism. Then $f^{-1}$ is also a homomorphism and thus $f$ is an isomorphism.
\end{proposition}
\begin{proof}
Take arbitrary $\omega\in\Omega$ and $b_1,\ldots, b_{\alpha(\omega)}\in B$. We calculate
\begin{align*}
f^{-1}(\omega_B(b_1,\ldots, b_{\alpha(\omega)})) &= f^{-1}(\omega_B(ff^{-1}b_1,\ldots, ff^{-1}b_{\alpha(\omega)})) \\
&= f^{-1}f(\omega_A(f^{-1}b_1,\ldots, f^{-1}b_{\alpha(\omega)})) = \omega_A(f^{-1}b_1,\ldots, f^{-1}b_{\alpha(\omega)}).
\end{align*}
\end{proof}

\subsection{Congruences}
\subsubsection{Direct product}
\begin{definition}
Let $\{A_i\}_{i\in I}$ be a family of $\Omega$-algebras. The \udef{direct product} $\prod_{i\in I} A_i$ is the $\Omega$-algebra whose carrier is the Cartesian product of $\{A_i\}_{i\in I}$ and where operations are carried out componentwise and relations are verified pointwise.

Similarly a \udef{direct power} of $A$ is a Cartesian power of $A$ with operations and relations defined componentwise.
\end{definition}

The direct product of $\{A,B\}$ is simply written $A\times B$.

\subsubsection{Relations as algebras}
Let $A,B$ be $\Omega$-algebras. Then $A\times B$ is also an $\Omega$-algebra and subalgebras are too. Such subalgebras can be seen as a binary relation on $A,B$.

\begin{lemma}
Let $A,B,C$ be $\Omega$-algebras and $\Gamma, \Delta$ subalgebras of $A\times B$ and $B\times C$, respectively. Then
\begin{enumerate}
\item $\Gamma^{\transp} \subset B\times A$ is an $\Omega$-algebra;
\item $\Gamma;\Delta \subset A\times C$ is an $\Omega$-algebra;
\item for any subalgebra $A'$ of $A$, $A'\Gamma \subset B$ is an $\Omega$-algebra;
\item for any subalgebra $B'$ of $B$, $\Gamma B' \subset A$ is an $\Omega$-algebra.
\end{enumerate}
\end{lemma}

\subsubsection{Congruences}
\begin{definition}
Let $A$ be an $\Omega$-algebra. A \udef{congruence} $\mathfrak{q}$ on $A$ is an equivalence relation on $A$ that is a subalgebra of $A^2$.
\end{definition}

\begin{example}
Any algebra has the \udef{trivial congruences} $I_A$ and $A^2$.
\end{example}

An algebra is \udef{simple} if there are no congruences on it other than the trivial ones. We assume a simple algebra is non-trivial.

\begin{lemma} \label{lemma:basicCongruenceLemma}
Let $\mathfrak{q}$ be a congruence on an $\Omega$-algebra $A$ and $B$ an $\Omega$-subalgebra of $A$. Then
\begin{enumerate}
\item $\mathfrak{q}$ can be extended pointwise to a congruence on $A^2$, that is
\[ (a,b)\mathfrak{q}(c,d) \iff a\mathfrak{q}c \land b\mathfrak{q}d; \]
\item 
\item $\mathfrak{q}|_B^B$ is a congruence on $B$.
\end{enumerate}
\end{lemma}
\begin{proof}
(1) The extension of $\mathfrak{q}$ is an equivalence relation by \ref{lemma:relationPropertiesDirectProduct}.

We then need to prove that this extension is a subalgebra of $(A^2)^2$. This is easily verified using pointwise operations. (TODO develop easier machinery for this?)

(2) The restriction is an equivalence relation by \ref{lemma:relationPropertiesRestriction} and an $\Omega$-algebra by \ref{lemma:intersectionSubalgebra}.
\end{proof}
For simplicity we may write $B/\mathfrak{q}$ instead of $B/(\mathfrak{q}|_B^B)$.

\begin{proposition}
Let $f:A\to B$ be a homomorphism. Then $\ker f$ is a congruence on $A$.
\end{proposition}

\subsubsection{Quotient algebras}
\begin{proposition} \label{prop:quotientAlgebra}
Let $A$ be an $\Omega$-algebra and $\mathfrak{q}$ an equivalence relation. Then there exists an interpretation of $A/\mathfrak{q}$ such that the function
\[ A \to A/\mathfrak{q}: a\mapsto [a]_\mathfrak{q} \]
is a homomorphism if and only if $\mathfrak{q}$ is a congruence.

Explicitly, this interpretation is unique and given by
\[ \omega_{A/\mathfrak{q}}([a_1]_{\mathfrak{q}},\ldots,[a_{\alpha(\omega)}]_{\mathfrak{q}}) = [\omega_A(a_1,\ldots, a_{\alpha(\omega)})]_{\mathfrak{q}} \qquad \forall \omega\in\Omega. \]\end{proposition}
\begin{proof}
The requirement that $[\cdot]_\mathfrak{q}$ be a homomorphism forces the interpretation $\omega_{A/\mathfrak{q}}$ of $\omega$ to be the one given.

We just need to show that $\omega_{A/\mathfrak{q}}$ is well-defined if and only if $\mathfrak{q}$ is a congruence. To that end, choose arbitrary $a_1, \ldots, a_{\alpha(\omega)}$ and $a'_1, \ldots, a'_{\alpha(\omega)}$ such that $a_1'\in[a_1]_\mathfrak{q}, \ldots, a'_{\alpha(\omega)}\in [a_{\alpha(\omega)}]_\mathfrak{q}$. This is equivalent to choosing $(a_1,a'_1),\ldots, (a_{\alpha(\omega)},a'_{\alpha(\omega)}) \in \mathfrak{q}$. Then
\begin{align*}
[\omega_A(a_1,\ldots, a_{\alpha(\omega)})]_{\mathfrak{q}} = [\omega_A(a'_1,\ldots, a'_{\alpha(\omega)})]_{\mathfrak{q}} &\iff (\omega_A(a_1,\ldots, a_{\alpha(\omega)}),\omega_A(a'_1,\ldots, a'_{\alpha(\omega)})) \in \mathfrak{q} \\
&\iff \omega_{A^2}((a_1,a'_1),\ldots, (a_{\alpha(\omega)},a'_{\alpha(\omega)})) \in \mathfrak{q}
\end{align*}
where the first statement is the requirement of being well-defined and the last is the requirement for being a subalgebra of $A^2$.
\end{proof}
\begin{definition}
The $\Omega$-algebra $A/\mathfrak{q}$ is called the \udef{quotient algebra} of $A$ by $\mathfrak{q}$. The function $A \to A/\mathfrak{q}: a\mapsto [a]_\mathfrak{q}$ is known as the quotient map.
\end{definition}

\begin{proposition}[Factor theorem] \label{prop:factorTheorem}
Let $f:A\to B$ be a homomorphism of $\Omega$-algebras and $\mathfrak{q}$ a congruence on $A$ such that $\mathfrak{q}\subseteq \ker f$. Then
\[ f': A/\mathfrak{q} \to B: [a]_\mathfrak{q} \mapsto f'([a]_\mathfrak{q}) = f(a) \]
is a well-defined homomorphism. Further, $f'$ is injective \textup{if and only if} $\mathfrak{q} = \ker f$.
\end{proposition}
\begin{proof}
To show the function is well defined, take $a,a'\in A$ such that $[a]_\mathfrak{q} = [a']_\mathfrak{q}$, i.e. $(a,a')\in \mathfrak{q}$. This implies $(a,a')\in\ker f$, so $f(a) = f(a')$ and $f'$ is well-defined.

We see that $f'$ is a homomorphism by the calculation
\begin{align*}
f'(\omega_{A/\mathfrak{q}}([a_1], \ldots, [a_{\alpha(\omega)}])) &= f'([\omega_{A}(a_1, \ldots, a_{\alpha(\omega)})]) = f(\omega_{A}(a_1, \ldots, a_{\alpha(\omega)})) \\
&= \omega_B(f(a_1), \ldots f(a_{\alpha(\omega)})) = \omega_B(f'([a_1]), \ldots, f'([a_{\alpha(\omega)}])).
\end{align*}
Finally $f'$ is injective iff no two distinct $\mathfrak{q}$-classes are identified by $f'$, which is exactly the condition $\mathfrak{q} = \ker f$.
\end{proof}

\subsubsection{Isomorphism theorems}
\begin{theorem}[First isomorphism theorem] \label{theorem:firstIsomorphism}
Let $f:A\to B$ be a homomorphism of $\Omega$-algebras. Then we have the isomorphism
\[ A/\ker f \cong \im f. \]
\end{theorem}
\begin{proof}
From the factor theorem \ref{prop:factorTheorem} we get an injective homomorphism $f': A/\ker f \to B$ which is made surjective by restricting the codomain to $\im f$. By \ref{prop:bijectiveHomomorphism} this is an isomorphism.
\end{proof}

\begin{theorem}[Second isomorphism theorem]
Let $A$ be an $\Omega$-algebra, $B$ an $\Omega$-subalgebra of $A$ and $\mathfrak{q}$ a congruence on $A$. Then we have the isomorphism
\[ (\mathfrak{q}B)/\mathfrak{q} \cong B/(\mathfrak{q}\cap B^2). \]
\end{theorem}
Note that $\mathfrak{q}B = B\mathfrak{q}$ because $\mathfrak{q}$ is a congruence. Also $\mathfrak{q}\cap B^2 = \mathfrak{q}|_B^B$, so the quotient is well-defined by \ref{lemma:basicCongruenceLemma}. Further, $\mathfrak{q}$ should really be restricted in $(\mathfrak{q}B)/\mathfrak{q}$, as in \ref{lemma:basicCongruenceLemma}.
\begin{proof}
Take the homomorphism $[\cdot]_\mathfrak{q}:A \to A/\mathfrak{q}$ as defined in \ref{prop:quotientAlgebra} and restrict it to $B$. Applying the first isomorphism theorem \ref{theorem:firstIsomorphism} yields the required result.
\end{proof}

\begin{theorem}[Third isomorphism theorem]
Let $A$ be an $\Omega$-algebra and $\mathfrak{q},\mathfrak{r}$ congruences on $A$ such that $\mathfrak{q} \subseteq \mathfrak{r}$. Then we have the isomorphism
\[ (A/\mathfrak{q})/(\mathfrak{r}/\mathfrak{q}) \cong A/\mathfrak{r}. \]
\end{theorem}
In order to properly interpret $\mathfrak{r}/\mathfrak{q}$, we need to first extend $\mathfrak{q}$ to $A^2$ and then restrict to $\mathfrak{r}$, both as in \ref{lemma:basicCongruenceLemma}.
\begin{proof}
Applying the factor theorem \ref{prop:factorTheorem} to the homomorphism $A\to A/\mathfrak{r}$ from \ref{prop:quotientAlgebra}. We get a homomorphism $f: A/\mathfrak{q} \to A/\mathfrak{r}$. We apply the first isomorphism theorem \ref{theorem:firstIsomorphism} to this homomorphism. Then we just need to show that $\ker f = \mathfrak{r}/\mathfrak{q}$. TODO!
\end{proof}
In particular, we see that $A/\mathfrak{q}$ is simple if and only if $\mathfrak{q}$ is a maximal proper congruence on $A$.

\section{Free algebras and varieties}

\section{Algebraic theories}
\subsection{Properties of a single binary operator}
\begin{definition}
Let $(\Omega, \alpha)$ be a signature, $A$ an $\Omega$-structure and $\omega$ a binary operator. We call an interpretation $\omega_A$
\begin{itemize}
\item \udef{associative} if $\forall x,y,z\in A: \omega_A(\omega_A(x,y),z) = \omega_A(x,\omega_A(y,z))$;
\item \udef{commutative} if $\forall x,y\in A: \omega_A(x,y) = \omega_A(y,x)$;
\item \udef{idempotent} if $\forall x\in A: \omega_A(x,x) = x$.
\end{itemize}
We say
\begin{itemize}
\item $A$ has a \udef{left-identity} $e_L$ for $\omega_A$ if $\forall x\in A: \omega_A(e_L, x) = x$;
\item $A$ has a \udef{right-identity} $e_R$ for $\omega_A$ if $\forall x\in A: \omega_A(x, e_R) = x$;
\item $A$ has an \udef{identity} $e$ if $e$ is both a left- and a right-identity.
\end{itemize}
Let $A$ have an identity $e$ for $\omega_A$, then we say an element $x\in A$
\begin{itemize}
\item has a \udef{left-inverse} $y$ if $\omega_A(y,x) = e$;
\item has a \udef{right-inverse} $y$ if $\omega_A(x,y) = e$;
\item has an \udef{(two-sided) inverse} $y$ if $\omega_A(x,y) = e = \omega_A(y,x)$.
\end{itemize}
\end{definition}

\begin{lemma}
Let $(\Omega, \alpha)$ be a signature, $A$ an $\Omega$-structure and $\omega$ a binary operator. If $A$ has both a left-identity $e_L$ and a right-identity $e_R$, then $A$ has an identity $e$ and
\[ e= e_L = e_R. \]
\end{lemma}
\begin{proof}
Assume $A$ has a left- and a right-identity. Then $e_L = \omega_A(e_L, e_R) = e_R$.
\end{proof}
\begin{corollary}
A structure may have multiple left-identities or multiple right-identities, but if it has both, then the identity is unique.
\end{corollary}

\subsection{Properties of two binary operators}
\begin{definition}
Let $(\Omega, \alpha)$ be a signature, $A$ an $\Omega$-structure and $\omega, \chi$ binary operators. We say
\begin{itemize}
\item $\omega_A$ is \udef{left-distributive} over $\chi_A$ if
\[ \forall x,y,z\in A: \; \omega_A(x,\chi_A(y,z)) = \chi_A(\omega_A(x,y),\omega_A(x,z)); \]
\item $\omega_A$ is \udef{right-distributive} over $\chi_A$ if
\[ \forall x,y,z\in A: \; \omega_A(\chi_A(x,y), z) = \chi_A(\omega_A(x,z),\omega_A(y,z)); \]
\item $\omega_A$ is \udef{distributive} over $\chi_A$ if it is left- and right-distributive;
\item $\omega_A$ is \udef{self-distributive} if it is distributive over itself.
\end{itemize}
We say
\begin{itemize}
\item $\omega_A, \chi_A$ are linked by the \udef{absorption law} if
\[ \forall x,y\in A:\; \omega_A(x,\chi_A(x,y)) = x = \chi_A(x,\omega_A(x,y)) \]
\end{itemize}
\end{definition}

\begin{lemma} \label{lemma:absorptionIdempotency}
Let $(\Omega, \alpha)$ be a signature, $A$ an $\Omega$-structure and $\omega, \chi$ binary operators. If $\omega_A, \chi_A$ are linked by the absorption law, then they are both idempotent.
\end{lemma}
\begin{proof}
For all $x\in A$ we have $\omega_A(x,x) = \omega_A(x,\chi_A(x,\omega_A(x,x))) = x$.
\end{proof}

\subsection{Notation for binary operators}
Prefix, infix, postfix, Polish, necessity of brackets.


\chapter{Magmas}
\section{Semigroups}

\subsection{Bands}

\section{Monoids}

\begin{lemma}
A locally small category with a single object is a monoid.
\end{lemma}

\begin{proposition} \label{prop:leftRightInverseMonoid}
Let $(M,\cdot,1)$ be a monoid and $a\in M$. If $a$ has both a left inverse $l$ and a right inverse $r$, then $l=r$.
\end{proposition}
\begin{proof}
We calculate
\[ l = l\cdot 1 = l\cdot(a\cdot r) = (l\cdot a)\cdot r= 1\cdot r = r. \]
\end{proof}

TODO: \udef{delooping} $\cat{B}M$.

\subsection{Ordered monoids}
\begin{definition}
An \udef{ordered monoid} is a monoid $(M, \cdot, 0)$ on which a partial order $\preceq$ is defined that is compatible, i.e. $\forall x,y,z\in M$
\[ x\preceq y \implies x\cdot z \preceq y \cdot z \land z\cdot x \preceq z \cdot y. \]

Positive: $x > 0$.
\end{definition}

\subsection{The Archimedean property}
\begin{definition}
Let $(M,+,\leq)$ be a totally ordered monoid and $x,y\in M$ positive. Then
\begin{itemize}
\item $x$ is \udef{infinitesimal w.r.t.} $y$ or $y$ is \udef{infinite w.r.t.} $x$ if $nx<y$ for all $n\in\N$;
\item $M$ is \udef{Archimedean} if there is no pair $(x,y)$ such that $x$ is infinitesimal w.r.t. $y$.
\end{itemize}
\end{definition}
every submonoid is Archimedean.

abelian??

\section{Divisibility}
$m|n$ order relation.

$\sup\{n,m\} = kgv(n,m)$ and $\inf\{n,m\} = ggd(n,m)$




\chapter{Lattices}

\section{Semilattice}
\begin{definition}
A \udef{semilattice} is an algebraic structure $\seq{S,\vee}$ where $\vee$ is a binary operation on the set $S$ satisfying
\begin{itemize}[leftmargin=2.5cm]
\item[\textbf{Associativity}] $x\vee (y\vee z) = (x\vee y) \vee z$;
\item[\textbf{Commutativity}] $x \vee y = y \vee x$;
\item[\textbf{Idempotency}] $x\vee x = x$.
\end{itemize}
We call a semilattice \udef{bounded} if it contains an identity.
\end{definition}
In other words a semilattice is a commutative band.

\begin{proposition}
Let $(P,\leq)$ be a poset and let $\{x,y,z\}\subseteq P$ be such that each subset has a supremum. Then
\[ \sup\{\sup\{x,y\},z\} = \sup\{x,y,z\} = \sup\{x,\sup\{y,z\}\} \]
and, dually,
\[ \inf\{\inf\{x,y\},z\} = \inf\{x,y,z\} = \inf\{x,\inf\{y,z\}\}. \]
\end{proposition}
\begin{corollary}
Let $(P,\leq)$ be a poset.
\begin{enumerate}
\item If $\sup\{x,y\}$ exists for all $x,y\in P$, then $(P,\sup)$ is a semilattice.
\item If $\inf\{x,y\}$ exists for all $x,y\in P$, then $(P,\inf)$ is a semilattice.
\end{enumerate}
\end{corollary}
The converse to this corollary also holds:
\begin{proposition} \label{prop:orderSemilattice}
Let $\seq{L,\vee}$ be a semilattice. Define the relation $\leq$ on $L$ by
\[ \forall x,y\in L:\; x\leq y \qquad \iff \qquad x\vee y = y. \]
Then $\seq{L,\leq}$ is a poset such that $\forall x,y\in L: x\vee y = \sup\{x,y\}$.
\end{proposition}
\begin{proof}
First we prove $\seq{L,\leq}$ is a poset:
\begin{itemize}
\item Reflexivity follows from idempotency.
\item Antisymmetry follows from commutativity.
\item For transitivity: assume $x\leq y$ and $y\leq z$. This implies $x\vee y = y$ and $y\vee z = z$, and so $z = (x\vee y)\vee z =x\vee (y\vee z) =x\vee z$. This implies $x\leq z$.
\end{itemize}
If $x,y$ are comparable, the $x \vee y$ is clearly $\sup\{x,y\}$. Because $x\vee(x\vee y) = (x\vee x)\vee y = x\vee y$, we have $x\leq (x\vee y)$. So $x\vee y$ is an upper bound of $\{x,y\}$. Let $u$ be an upper bound of $\{x,y\}$. Then $u = x\vee u = x\vee (y\vee u) = (x\vee y)\vee u$. So $x\vee y \leq u$, meaning $x\vee y$ is the supremum.
\end{proof}
Dually we also have:
\begin{corollary}
Let $\seq{L,\wedge}$ be a semilattice. Define the relation $\leq$ on $L$ by
\[ x\leq y \qquad \iff \qquad x\wedge y = x. \]
Then $\seq{L,\leq}$ is a poset such that $\forall x,y\in L: x\wedge y = \inf\{x,y\}$.
\end{corollary}

\section{Lattices}
\url{file:///C:/Users/user/Downloads/Gr%C3%A4tzer,%20George%20-%20General%20lattice%20theory-Birkh%C3%A4user%20(2007).pdf}
\url{file:///C:/Users/user/Downloads/R.%20Padmanabhan,%20S.%20Rudeanu%20-%20Axioms%20for%20lattices%20and%20Boolean%20algebras-World%20Scientific%20(2008).pdf}
\begin{definition}
A \udef{lattice} is an algebraic structure $\seq{L,\vee, \wedge}$, where $\vee, \wedge$ are binary operations on the set $L$ such that $\seq{L,\vee}$ and $\seq{L,\wedge}$ are semilattices and $\vee,\wedge$ are linked by the absorption law:
\[ \forall a,b\in L: \; a \vee (a \wedge b) = a = a \wedge (a\vee b). \]
We call
\begin{itemize}
\item $\seq{L,\vee}$ the \udef{join-semilattice} and $a\vee b$ the \udef{join} of $a$ and $b$;
\item $\seq{L,\wedge}$ the \udef{meet-semilattice} and $a\wedge b$ the \udef{meet} of $a$ and $b$.
\end{itemize}
We call a lattice \udef{bounded} if both the join- and the meet-semilattice are bounded. We denote
\begin{itemize}
\item the identity of the join-semilattice by $\top$;
\item the identity of the meet-semilattice by $\bot$.
\end{itemize}
\end{definition}
By \ref{lemma:absorptionIdempotency} the absortion law renders the axiom of  idempotency of the semilattices redundant. So we just need that $\seq{L,\vee}$ and $\seq{L,\wedge}$ are commutative semigroups that are linked by the absorption law.

\begin{proposition}
Observations from universal algebra:
\begin{enumerate}
\item Subset closed under $\vee,\wedge$ is sublattice.
\item Product lattices
\item Inverse of homomorphism is homomorphism
\end{enumerate}
\end{proposition}


\subsection{Lattices and order}
As for semilattices, we can equivalently characterise lattices as posets with certain conditions.
\begin{proposition}
Let $L$ be a set.
\begin{enumerate}
\item If $\seq{L,\vee, \wedge}$ is a lattice, then $\seq{L,\leq}$ is a poset such that every two element set has a supremum and an infimum, where
\[ \forall x,y\in L:\; x\leq y \qquad \iff \qquad x\vee y = y \]
or, equivalently,
\[ \forall x,y\in L:\; x\leq y \qquad \iff \qquad x\wedge y = x. \]
\item If $\seq{L,\leq}$ is a poset such that every two element set has a supremum and an infimum, then $\seq{L,\vee, \wedge}$ is a lattice, where
\[ \forall x,y\in L: \; x\vee y = \sup\{x,y\} \quad \text{and} \quad x\wedge y = \inf\{x,y\}. \]
\end{enumerate}
The order can also be defined by
\[ \forall x,y\in L:\; x\leq y \qquad \iff \qquad x\wedge y = x. \]
\end{proposition}
\begin{proof}
Mostly this follows from \ref{prop:orderSemilattice}. We just need to show the two definitions of order are equivalent. This follows from the absorption law:
\begin{align*}
x\vee y &= y \implies x\wedge y = x\wedge (x\vee y) = x \\
x\wedge y &= x \implies x\vee y = (x\wedge y) \vee y = y.
\end{align*}
\end{proof}

\begin{lemma} \label{lemma:orderLattice}
Let $L$ be a lattice and $a,b,c,d\in L$. If $a\leq b$ and $c\leq d$, then
\[ a\vee c\leq b\vee d \qquad\text{and}\qquad a\wedge c \leq b\wedge d. \]
\end{lemma}
\begin{proof}
If $a\leq b$ and $c\leq d$, then we have
\[ \begin{cases}
a = a\wedge b \\ c = c\wedge d
\end{cases} \quad \text{and} \quad \begin{cases}
b = a\vee b \\ d = c\vee d.
\end{cases}\]
We calculate
\[ (a\vee c)\wedge (b\vee d) = (a\vee c)\wedge (a\vee b\vee c\vee d) = (a\vee c)\wedge ((a\vee c)\vee (b\vee d)) = a\vee c, \]
so $a\vee c\leq b\vee d$.
Similarly
\[ (a\wedge c)\vee (b\wedge d) = (a\wedge b\wedge c\wedge d)\vee (b\wedge d) = ((a\wedge c)\wedge (b\wedge d))\vee (b\wedge d) = b\wedge d, \]
so $a\wedge c \leq b\wedge d$.
\end{proof}
\begin{corollary} \label{corollary:orderLattice}
Let $L$ be a lattice and $a,b,c,d\in L$. Then
\begin{enumerate}
\item if $a\leq b$ then $a\vee c \leq b \vee c$ and $a\wedge c \leq b\wedge c$;
\item if $a\leq c$ and $b\leq c$, then $a\vee b \leq c$;
\item if $a\leq b$ and $a\leq c$, then $a\leq b\wedge c$.
\end{enumerate}
\end{corollary}

\begin{lemma}
Let $X$ be a set. Then $\mathcal{P}(X)$ ordered by inclusion is a lattice and $\cup, \cap$ are the corresponding join and meet operations.

In particular, for all $A,B\in \mathcal{P}(X)$:
\begin{enumerate}
\item $\inf\{A,B\} = A\cap B$;
\item $\sup\{A,B\} = A\cup B$.
\end{enumerate}
\end{lemma}

\begin{lemma}
The natural numbers forms a lattice if ordered by division. The meet and join are given by
\[ m\vee n = \lcm\{m,n\} \qquad \text{and}\qquad m\wedge n = \gcd\{m,n\}. \]
\end{lemma}

\begin{lemma}
Let $L,K$ be lattices and $f:L\to K$ a function. The following are equivalent:
\begin{enumerate}
\item $f$ is order-preserving;
\item $\forall x,y\in L:\; f(a\vee b)\geq f(a)\vee f(b)$;
\item $\forall x,y\in L:\; f(a\wedge b)\leq f(a)\wedge f(b)$.
\end{enumerate}
\end{lemma}

\begin{proposition}[Mini-max theorem]
Let $L$ be a lattice and let $\seq{a_{i,j}}\subset L$ be indexed by $i,j\in \N$. Then
\[ \bigvee_{j=1}^n \left(\bigwedge_{i=1}^m a_{i,j}\right) \leq \bigwedge_{i=1}^m \left(\bigvee_{j=1}^n a_{i,j}\right). \]
\end{proposition}
\begin{proof}
For all $k,l$ we have $a_{k,l}\leq \bigvee_{j=1}^n a_{k,j}$. This implies $\bigwedge_{i=1}^m a_{i,l} \leq \bigwedge_{i=1}^m \left(\bigvee_{j=1}^n a_{i,j}\right)$ for all $l$. Taking the supremum over $l$ gives the result.
\end{proof}
\begin{corollary}[Distributive inequalities] \label{corollary:distributiveInequality}
Let $L$ be a lattice and $a,b,c \in L$, then
\begin{align*}
(a\wedge b)\vee (a\wedge c) &\leq a\wedge (b \vee c); \\
(a\vee b)\wedge (a\vee c) &\geq a\vee (b \wedge c).
\end{align*}
In particular this also means
\[ c \leq a \implies (a\wedge b)\vee c \leq a\wedge (b\vee c).  \]
\end{corollary}
Use $a_{i,j} = \begin{pmatrix}
a & a \\ b & c 
\end{pmatrix}$ and $a_{i,j} = \begin{pmatrix}
a & b \\ a & c 
\end{pmatrix}$. The particular cases follow because in this case $a\wedge c = c$. This statement is self-dual.
\begin{corollary}[Median inequality]
Let $L$ be a lattice and $a,b,c \in L$, then
\[ (a\wedge b) \vee (b\wedge c) \vee (c\wedge a) \leq (a\vee b)\wedge (b\vee c) \wedge (c\vee a). \]
\end{corollary}
\begin{proof}
Use $a_{i,j} = \begin{pmatrix}
a & b & a \\ b & b & c \\ a & c & c
\end{pmatrix}$.
\end{proof}


\subsection{Filters and ideals}
\begin{definition}
Let $L$ be a lattice and $J\subset L$ a non-empty subset. We call $J$ 
\begin{itemize}
\item an \udef{ideal} if it is a down-set closed under join;
\item a \udef{filter} if it is an up-set closed under meet.
\end{itemize}
A filter of ideal is called \udef{proper} if it is not $L$.

The set of ideals of $L$ is denoted $\ideals(L)$ and the set of filters of $L$ is denoted $\filters(L)$.
\end{definition}
In other word $J$ is an ideal if
\begin{itemize}
\item $a,b\in J \implies a\vee b\in J$;
\item $x\in L, b\in J$ and $x\leq b$ implies $x\in J$;
\end{itemize}
and a filter if
\begin{itemize}
\item $a,b\in J \implies a\wedge b\in J$;
\item $x\in L, b\in J$ and $x\geq b$ implies $x\in J$.
\end{itemize}

\begin{lemma}
Let $L$ be a lattice and $J$ a subset. Then
\begin{enumerate}
\item $J$ is an ideal \textup{if and only if}
\begin{enumerate}
\item $\forall a,b\in J: \qquad a\vee b\in J$;
\item $\forall x\in L, \forall b\in J: \quad x\wedge b \in J$.
\end{enumerate}
\item $J$ is a filter \textup{if and only if}
\begin{enumerate}
\item $\forall a,b\in J: \qquad a\wedge b\in J$;
\item $\forall x\in L, \forall b\in J: \quad x\vee b \in J$.
\end{enumerate}
\end{enumerate}
\end{lemma}
\begin{proof}
By the equivalence $x\leq b \iff x\wedge b = x$.
\end{proof}

\begin{lemma}
Let $L$ be a lattice. Then
\begin{enumerate}
\item if $L$ contains $\top$, then an ideal in $L$ is proper \textup{if and only if} is does not contain $\top$;
\item if $L$ contains $\bot$, then a filter in $L$ is proper \textup{if and only if} is does not contain $\bot$.
\end{enumerate}
\end{lemma}

\begin{lemma}
Let $L$ be a lattice. For all $x\in L$,
\begin{enumerate}
\item $\downset x$ is an ideal;
\item $\upset x$ is a filter.
\end{enumerate}
These ideals and filters are called \udef{principle ideals} and \udef{principle filters}.
\end{lemma}

\subsection{Complete lattices}
\begin{lemma}
Let $L$ be a lattice. For every finite set $S\subset L$, $\sup(S)$ and $\inf(S)$ exist.
\end{lemma}
\begin{definition}
Let $L$ be a lattice. We call $L$ a \udef{complete lattice} if each subset $S\subseteq L$ has both a supremum and an infimum. We write
\[ \sup(S) = \bigvee S \qquad \inf(S) = \bigwedge S. \]
If we want to emphasise that the supremum/infimum of $S$ is taken as a subset of $L$, we write $\bigvee_L S$ and $\bigwedge_L S$.
\end{definition}
Clearly every finite lattice is complete.

\begin{lemma} \label{lemma:supInfFiniteSubsetsLattice}
Let $L$ be a lattice and $F\subseteq L$ a finite subset. Then $\bigvee F$ and $\bigwedge F$ exist. 
\end{lemma}

\begin{example}
For any set $X$, $\powerset(X)$ is a complete lattice.
\end{example}

\begin{lemma}
Let $P$ be an ordered set such that all relevant suprema and infima exist and $S,T\subseteq P$. Then
\begin{enumerate}
\item $\bigvee S \leq \bigwedge T$ if and only if $s\leq t$ for all $s\in S,t\in T$;
\item if $S\subseteq T$, then $\bigvee S \leq \bigvee T$ and $\bigwedge S \geq \bigwedge T$;
\item $\bigvee(S\cup T) = \left(\bigvee S\right)\vee \left(\bigvee T\right)$ and $\bigwedge(S\cup T) = \left(\bigwedge S\right)\wedge \left(\bigwedge T\right)$.
\end{enumerate}
\end{lemma}

\begin{proposition} \label{prop:completeLatticeBasic}
Let $P$ be a non-empty ordered set. Then the following are equivalent:
\begin{enumerate}
\item $P$ is a complete lattice;
\item $\bigvee S$ exists for all subsets $S\subseteq P$;
\item $\bigwedge S$ exists for all subsets $S\subseteq P$;
\item $P$ has a bottom element $\bot$ and $\bigvee S$ exists for all non-empty $S\subseteq P$;
\item $P$ has a top element $\top$ and $\bigwedge S$ exists for all non-empty $S\subseteq P$;
\item for all $x\in P$ both $\upset x$ and $\downset x$ are complete lattices.
\end{enumerate}
\end{proposition}
\begin{proof}
The only difficult implication is $(5)\Rightarrow (1)$. All infima exist because $\inf(\emptyset) = \top \in P$. New each non-empty set $S$ in $P$ has an upper bound, $\top$, so $\bigvee S$ exists in $P$ by \ref{prop:existenceSupremaInfima}. Finally $\bigvee \emptyset = \bot = \bigwedge P \in P$.
\end{proof}

\begin{theorem}[Knaster-Tarski fixed-point theorem]
Let $L$ be a complete lattice and $f:L\to L$ an order-preserving map. Then 
\[ \bigvee \setbuilder{x\in L}{x\leq f(x)} \qquad\text{and}\qquad \bigwedge \setbuilder{x\in L}{x\geq f(x)} \]
are, resp., the greatest and the least fixed point of $f$.
\end{theorem}
\begin{proof}
Let $P$ be the set of fixed points, set $H = \setbuilder{x\in L}{x\leq f(x)}$ and $\alpha = \bigvee H$. It is clear that $\alpha$ is an upper bound of $P$ because $P \subset  \setbuilder{x\in L}{x\leq f(x)}$. So we just need to show that $\alpha$ is a fixed point.

Now $f(\alpha)$ is an upper bound of $H$ due to $f$ being order preserving: $x \leq f(x) \leq  f(\alpha)$ for all $x\in P$. So $\alpha \leq f(\alpha)$.
Conversely, $f(\alpha) \leq f(f(\alpha))$ because $f$ is order preserving. This means $f(\alpha)\in H$, so $f(\alpha)\leq \alpha$. We have thus shown that $\alpha = f(\alpha)$.

The proof that $\bigwedge \setbuilder{x\in L}{x\geq f(x)}$ is the least fixed point is completely dual.
\end{proof}
\begin{corollary}
The set of fixed points of $f$ forms a complete lattice.
\end{corollary}
\begin{proof}
Let $P$ be the set of fixed points. To show $P$ is a complete lattice, take any subset $S\subset P$.
Set $w = \bigvee S$, where $S$ is considered as a subset of $L$. For all $x\in W$: $x\leq w$, which implies $x=f(x)\leq f(w)$. As $w$ is the least upper bound, we have $w\leq f(w)$. This implies $f[\upset w]\subseteq \upset w$, meaning we can view $f$ as a function on the complete lattice $\upset w$. In particular $f|_{\upset w}$ has a least fixed point by the theorem, so $S$ has a supremum in $P$. The existence of the infimum is dual.
\end{proof}
\begin{corollary}[Banach decomposition theorem]
Let $X,Y$ be sets and $f:X\to Y$ and $g:Y\to X$ functions. There exist partitions $X_1,X2$ and $Y_1,Y_2$ of $X$ and $Y$ such that
\[ f[X_1] = Y_1 \qquad\text{and}\qquad g[Y_2] = X_2. \]
\end{corollary}
\begin{proof}
Consider the map $F: \powerset(X)\to\powerset(X): S\mapsto X\setminus g[Y\setminus f[S]]$. By the theorem this map has a fixed point, which we call $X_1$. We then need to set $Y_1 = f[X_1], X_2 = X\setminus X_1$ and $Y_2 = Y\setminus Y_1$. The fact $X_1$ is a fixed point means that $X_1 = X\setminus g[Y\setminus f[X_1]] = g[Y\setminus Y_1] = g[Y_2]$.
\end{proof}
\begin{corollary}[Schröder-Bernstein]
Let $X,Y$ be sets and $f:X\rightarrowtail Y$ and $g:Y\rightarrowtail X$ injective functions. Then there exists a bijective function $h: X\twoheadrightarrowtail Y$.
\end{corollary}
\begin{proof}
Use the Banach decomposition theorem to obtain partitions $X_1,X_2$ and $Y_1,Y_2$. Then $f|_{X_1}: X_1\twoheadrightarrowtail Y_1$ and $g|_{Y_2}: Y_2 \twoheadrightarrowtail X_2$ are bijective, so we can construct
\[ h: X\twoheadrightarrowtail Y: x \mapsto \begin{cases}
f(x) & x\in X_1 \\ (g|_{Y_2})^{-1}(x) & x\in X_2
\end{cases}. \]
\end{proof}
The Schröder-Bernstein theorem was already proven in \ref{theorem:SchroederBernstein}.

\subsubsection{Chain conditions}
The following requires dependent choice:
\begin{proposition} \label{prop:ascendingDescendingChainLattice}
Let $L$ be a lattice. Then
\begin{enumerate}
\item if $L$ satisfies the ascending chain condition, then for all non-empty subsets $S\subset L$ there exists a finite set $F\subset L$ such that $\bigvee S = \bigvee F$;
\item if $L$ satisfies the descending chain condition, then for all non-empty subsets $S\subset L$ there exists a finite set $F\subset L$ such that $\bigwedge S = \bigwedge F$;
\item if $L$ has a bottom element and satisfies the ascending chain condition, then $L$ is complete;
\item if $L$ has a top element and satisfies the descending chain condition, then $L$ is complete;
\item if $L$ has no infinite chains, then $L$ is complete.
\end{enumerate}
\end{proposition}
\begin{proof}
(1) Assume $L$ satisfies the ascending chain condition and let $S\subset L$ be non-empty. Define
\[ B = \setbuilder{\bigvee G}{\text{$G$ is a finite, non-empty subset of $S$}}. \]
This is well-defined by \ref{lemma:supInfFiniteSubsetsLattice}. Then $B$ has a maximal element $m = \bigvee F$ for some finite $F$ by \ref{prop:welfoundedACC}.

Now $m$ is an upper bound of $S$. Indeed, let $x\in S$. Then $m= \bigvee F \leq \bigvee (F\cup\{x\})$ because $F\subseteq (F\cup \{x\})$. Since $m$ is maximal in $B$, we have $m = \bigvee (F\cup\{x\}) \geq x$. It is clearly also the least upper bound, otherwise it was not the least upper bound of $F$.

(2) Dual of 1.

(3) This follows from 1. and \ref{prop:completeLatticeBasic}.

(4) Dual of 3.

(5) A lattice with no infinite chains satisfies the ascending chain condition. Also a lattice
with no infinite chains has a bottom element. (TODO: need dependent/countable choice?)
\end{proof}

\subsection{Join- and meet-irreducible elements}
\begin{definition}
Let $L$ be a lattice. We call $x\in L$ \udef{join-irreducible} if
\begin{itemize}
\item $x$ is not a least element of $L$,
\item for all $a,b\in L$: $x= a\vee b$ implies $x=a$ or $x=b$.
\end{itemize}
The definition of \udef{meet-irreducible} is dual.

We denote the set of join-irreducible elements in $L$ as $\joinIr(L)$ and the set of meet-irreducible elements in $L$ as $\meetIr(L)$.
\end{definition}

\begin{lemma}
Let $L$ be a lattice and $x\in L$ not a least element. Then the following are equivalent:
\begin{enumerate}
\item $x$ is join-irreducible;
\item for all $a,b\in L$: $x= a\vee b$ implies $x \leq a$ or $x \leq b$;
\item for all $a,b\in L$: $x > a$ and $x > b$ implies $x > a\vee b$;
\item for all finite $F\subseteq L$: $x = \bigvee F$ implies $x\in F$.
\end{enumerate}
\end{lemma}

\begin{lemma}
In a finite lattice $L$, an element is join-irreducible if and only if it
has exactly one lower cover.
\end{lemma}

\begin{example}
Consider the lattice $\seq{\N,\lcm, \gcd}$. A non-zero element of $\N$ is join irreducible if and only if it is of the form $p^r$ for some prime $p$ and $r\in\N$.
\end{example}

\begin{proposition} \label{prop:joinIrreducibilityDescendingChainLattice}
Let $L$ be a lattice satisfing the descending chain condition. Then
\begin{enumerate}
\item $\forall a,b\in L:\; a\nleq b \implies \exists x\in \joinIr(L): \; x\leq a$ and $x\nleq b$;
\item $\forall a\in L:\; a = \bigvee\setbuilder{x\in\joinIr(L)}{x\leq a}$.
\end{enumerate}
\end{proposition}
\begin{proof}
(1) Set $S = \setbuilder{x\in L}{\text{$x\leq a$ and $x\nleq b$}}$, which is non-empty and thus contains a minimal element $m$ by \ref{prop:welfoundedACC}. We claim $m$ is join-irreducible. Assume, towards a contradiction, $x = c\vee d$ and $c < x > d$. By minimality of $x$, $c,d\notin S$. As $c,d< x \leq a$, we must have $c,d\leq b$. But this means $x\leq b$, so $x\notin S$ which is a contradiction.

(2) Set $T = \setbuilder{x\in\joinIr(L)}{x\leq a}$. Clearly $a$ is an upper bound of $T$. To see that it is the least upper bound, take a different upper bound $c$. Assume, towards a contradiction, that $a\nleq c$. Then $a\nleq a\wedge c$. By point 1. there exists an $x\in\joinIr(L)$ such that $x\leq a$ (meaning $x\in T$) and $x\nleq a\wedge c$. But if $c$ were an upper bound of $T$, then $x\leq a\wedge c$, which is a contradiction.
\end{proof}


\subsubsection{Join- and meet-dense subsets}
\begin{definition}
Let $P$ be a poset and let $Q\subset P$ be a subset. Then $Q$ is called \udef{join-dense} in $P$ if for every $x\in P$, there exists a subset $S\subset Q$ such that $x= \bigvee_P S$.

The dual of join-dense is \udef{meet-dense}.
\end{definition}

\begin{proposition}
Let $L$ be a lattice.
\begin{enumerate}
\item If $L$ satisfies the descending chain condition, then any subset $Q\supseteq \joinIr(L)$ is join-dense in $L$.
\item If $L$ satisfies the ascending chain condition and $Q$ is join-dense in $L$, then for all $a\in L$ there exists a finite subset $F$ of $Q$ such that $a = \bigvee F$.
\end{enumerate}
\end{proposition}
\begin{proof}
(1) is a corollary of \ref{prop:joinIrreducibilityDescendingChainLattice}. (2) is a corollary of \ref{prop:ascendingDescendingChainLattice}.
\end{proof}
\begin{corollary}
Let $L$ be a lattice with no infinite chains. Then
\begin{enumerate}
\item for each $a \in L$, there exists a finite subset $F$ of $\joinIr(L)$ such that $a = F$.
\item $Q\subseteq L$ is join-dense in $L$ if and only if $Q \supseteq \joinIr(L)$.
\end{enumerate}
\end{corollary}
\begin{proof}
If $L$ has no finite chains, then it satisfies the ascending and descending chain conditions.

Only the $\Rightarrow$ direction of (2) is not immediately obvious. Assume $Q$ is join-dense and let $x\in \joinIr(L)$. By the proposition there exists a finite $F\subseteq Q$ such that $x = \bigvee F$. Since $x$ is join-irreducible, we have $x \in F$ and hence $x \in Q$. Thus, $\joinIr(L) \subseteq Q$.
\end{proof}

\subsection{Distributive lattices}
For all lattices $L$ the distributive inequalities, \ref{corollary:distributiveInequality}, hold: $\forall a,b,c \in L$:
\begin{align*}
a \vee (b\wedge c) &\leq (a\vee b) \wedge (a\vee c); \\
a\wedge (b \vee c) &\geq (a\wedge b)\vee (a\wedge c).
\end{align*}

The two corresponding equalities are equivalent:
\begin{proposition} \label{lemma:equivalenceDistributiveLaws}
Let $L$ be a lattice. Then the following are equivalent:
\begin{enumerate}
\item $\forall a,b,c \in L: \; a \vee (b\wedge c) = (a\vee b) \wedge (a\vee c)$;
\item $\forall a,b,c \in L: \; a\wedge (b \vee c) = (a\wedge b)\vee (a\wedge c)$.
\end{enumerate}
\end{proposition}
These equivalent equalities are known as the \udef{distributive laws}.
\begin{proof}
We show $(1)\Rightarrow (2)$. Then other implication follows by duality.

Assume (1). Then, for all $a,b,c \in L$:
\begin{align*}
(a\wedge b)\vee (a\wedge c) &= ((a\wedge b)\vee a) \wedge ((a\wedge b)\vee c) & \text{by (1)}\\
&= (a\wedge (c\vee (a\wedge b))  & \text{by the absorption law}\\
&= (a \wedge ((c\vee b) \wedge (c\vee a))  & \text{by (1)}\\
&= a\wedge (b\vee c) & \text{by the absorption law.}
\end{align*}
\end{proof}
Note that it is \emph{not} true that
\[ \forall a,b,c \in L: \; a \vee (b\wedge c) = (a\vee b) \wedge (a\vee c) \iff a\wedge (b \vee c) = (a\wedge b)\vee (a\wedge c). \]

\begin{definition}
A lattice $L$ is called \udef{distributive} if it satisfies the distributive laws.
\end{definition}

\begin{lemma}
A lattice is distributive \textup{if and only if} its dual is distributive.
\end{lemma}

\begin{lemma}
Let $L$ be a lattice. The following are equivalent:
\begin{enumerate}
\item $L$ is distributive;
\item for all $x,y,z,w\in L: \quad x\wedge y \leq w \;\text{and}\; x\wedge z\leq w \implies x\wedge (y\vee z) \leq w;$
\item for all $x,y,z,w\in L: \quad x\vee y \geq w \;\text{and}\; x\vee z\geq w \implies x\vee (y\wedge z) \geq w$.
\end{enumerate}
\end{lemma}
\begin{proof}
Point (1) is self-dual and points (2) and (3) are dual, so it is enough to show that (1) and (2) are equivalent.

Assume $L$ distributive. By \ref{corollary:orderLattice}, $ x\vee y \geq w$ and $x\vee z\geq w$ imply $(x\wedge y)\vee (x\wedge z) \leq w$. By distributivity, we get (2).

Conversely, we can take $w = (x\wedge y)\vee (x\wedge z)$. Then (2) gives $x\wedge (y\vee z) \leq (x\wedge y)\vee (x\wedge z)$ and the distributive inequality \ref{corollary:distributiveInequality} gives the other inequality. 
\end{proof}

\subsection{Modular lattices}
For all lattices $L$ the modular inequality holds: $\forall a,b,c \in L$:
\[ a\leq c \implies a \vee (b\wedge c) \leq (a\vee b) \wedge c. \]
See \ref{corollary:distributiveInequality}.

\begin{proposition} \label{prop:modularEquivalences}
Let $L$ be a lattice. Then the following are equivalent:
\begin{enumerate}
\item $\forall a,b,c\in L$: $a \vee (b\wedge c) = (a\vee b) \wedge c$ if $a\leq c$;
\item $\forall a,b,c\in L$: $a \vee (b\wedge c) = (a\vee b) \wedge (a\vee c)$ if $a\leq b$ or $a\leq c$; the dual of (2);
\item $\forall a,b,c\in L$: $a\vee (b\wedge (a\vee c)) = (a\vee b)\wedge (a\vee c)$; the dual of (3);
\item \textup{Shearing identity}: $\forall a,b,c\in L$: $(a\vee b) \wedge c = (a\vee (b\wedge (a\vee c)))\wedge c$; the dual of the shearing identity.
\end{enumerate}
\end{proposition}
The first of these is referred to as the \udef{modular law}. Notice that it is self-dual.
\begin{proof}
(1) is equivalent to its dual by replacing $a\leftrightarrow c$. This will imply all statements are equivalent to their duals once the equivalence with (1) has been established. 

$\boxed{(1)\Rightarrow (2)}$ Assume (1). Assume $a\leq b$ or $a\leq c$. By relabelling we can assume $a\leq c$.  Then $a\vee c = c$ and (2) clearly follows from (1).


$\boxed{(2)\Rightarrow (3)}$ Apply (2) to $a\leq a\vee c$.

$\boxed{(3)\Rightarrow (1)}$ Assume $a\leq c$. Then $a\vee c = c$ and $a\vee (b\wedge (a\vee c)) = (a\vee b)\wedge (a\vee c)$ reduces to $a \vee (b\wedge c) = (a\vee b) \wedge c$.

$\boxed{(3)\Rightarrow (4)}$ We calculate, using (3),
\[ (a\vee (b\wedge (a\vee c)))\wedge c = ((a\vee b)\wedge (a\vee c))\wedge c = (a\vee b) \wedge (a\vee c) \wedge c = (a\vee b) \wedge c. \]

$\boxed{(4)\Rightarrow (3)}$ TODO????
\end{proof}

\begin{definition}
A lattice $L$ is called \udef{modular} if it satisfies the modular law.
\end{definition}

\begin{lemma}
If a lattice is distributive, it is also modular.
\end{lemma}
\begin{proof}
If the lattice is distributive, point (2) of \ref{prop:modularEquivalences} holds unconditionally, and so in particular also conditionally.
\end{proof}

\section{Complementation}
\subsection{Disjoint elements and disjoint complement}
\begin{definition}
Let $L$ be a lattice with a bottom $\bot$. We say $x,y\in L$ are \udef{disjoint} if $x\wedge y = \bot$.

Let $S\subset L$. Then the set
\[ S^d \defeq \setbuilder{x\in L}{\forall y\in S: x\wedge y = \bot} \]
is called the \udef{disjoint complement} of $Y$.
\end{definition}

\begin{lemma}
Let $L$ be a distributive lattice with a bottom $\bot$ and $S\subset L$. Then $S^d$ is an ideal.
\end{lemma}
\begin{proof}
Let $a,b\in S^d$. Then for all $x\in S$, we have
\[ (a\vee b)\wedge x = (a\wedge x)\vee (b\wedge x) = \bot\vee \bot = \bot, \]
so $a\vee b\in S^d$.

It is also clear $S^d$ must be a down set by \ref{corollary:orderLattice}.
\end{proof}

\subsection{Complementation}
\begin{definition}
Let $L$ be a bounded lattice and $x\in L$. We call $y$ a \udef{complement} of $x$ if
\[ x \vee y = \top \qquad \text{and} \qquad x\wedge y = \bot. \]
\end{definition}

\begin{proposition} \label{prop:distributiveComplementUnique}
Let $L$ be a bounded lattice. If $L$ is distributive, then any $x\in L$ has at most one complement.
\end{proposition}
\begin{proof}
Let $y,y'$ be complements of $x$. Then
\[ y = y\vee \bot = y\vee (x\wedge y') = (y\vee x)\wedge (y\vee y') = \top \wedge (y\vee y') = y\vee y'. \]
Similarly $y' = y'\vee y$, so $y= y'$.
\end{proof}

\subsubsection{Complemented lattices}
\begin{definition}
A \udef{complemented lattice} is a bounded lattice with a function $c:L\to L$, called the \udef{complementation}, such that $c(x)$ is a complement of $x$ for all $x\in L$.

A lattice in which every element has exactly one complement is called a \udef{uniquely complemented lattice}.

A lattice with the property that every interval (viewed as a sublattice) is complemented is called a \udef{relatively complemented lattice}.
\end{definition}
A distributive complemented lattice is uniquely complemented.

\begin{lemma}
Let $L$ be a complemented lattice with complementation $c:L\to L$. Then $c(\top) = \bot$ and $c(\bot) = \top$.
\end{lemma}
\begin{proof}
For $c(\top)$ to be a complement of $\top$, we need $\top \wedge c(\top) = \bot$. But for all $x\in L$ we have $x\leq \top$, so $\top\wedge x = x$. This means we have $c(\top) = \bot$.
\end{proof}

\begin{lemma} \label{lemma:uniqueComplementInvolution}
Let $L$ be a uniquely complemented lattice. Then the complementation $c:L\to L$ is an involution.
\end{lemma}
\begin{proof}
Clearly if $c(x)$ is a complement of $x$, then $x$ is a complement of $c(x)$. By uniqueness $c(c(x)) = x$.
\end{proof}

\subsubsection{Orthocomplemented lattices}
\begin{definition}
Let $L$ be a bounded lattice. An \udef{orthocomplementation} is a function $\perp: L \to L$ that maps each element $x\in L$ to an orthocomplement $x^\perp$ such that
\begin{itemize}
\item $x$ and $x^\perp$ are complements;
\item $\perp$ is an involution: $x^{\perp\perp} = x$;
\item $\perp$ is order-reversing: $x\leq y \implies y^\perp \leq x^\perp$.
\end{itemize}
An \udef{orthocomplemented lattice} or \udef{ortholattice} is a bounded lattice equipped with an orthocomplementation.
\end{definition}
An orthocomplemented lattice is not necessarily uniquely complemented.

\begin{lemma}
The dual lattice of an ortholattice is an ortholattice. We can view an orthocomplementation as an isomorphism between an ortholattice and its dual.
\end{lemma}

\begin{proposition}
The cardinality of any finite ortholattice is either even or $1$.
\end{proposition}
\begin{proof}
Let $L$ be a finite ortholattice.
The orthocomplementation pairs elements $x,y$ such that $x^\perp = y$ and $y^\perp = x$. If for all such pairs we have $x\neq y$, then the cardinality of $L$ is even. Now assume there exists an $x\in L$ such that $x^\perp = x$. Then 
\[ \bot = x\wedge x^\perp = x\wedge x = x = x\vee x = x\vee x^\perp = \top. \]
So $\bot = \top$, which is only possible if $L = \{\bot\}$.
\end{proof}

\begin{example}
\begin{itemize}
\item The lattice $\mathbf{M}_2 = \begin{tikzcd}[column sep={2em,between origins},row sep={2em,between origins}]
&\circ \ar[ld, dash] \ar[rd, dash] & \\ \circ \ar[rd, dash] && \circ \ar[ld, dash] \\ &\circ &
\end{tikzcd}$ admits a unique orthocomplementation.
\item The lattice $\mathbf{M}_3 = \begin{tikzcd}[column sep={2em,between origins},row sep={2em,between origins}]
&\circ \ar[ld, dash] \ar[d,dash] \ar[rd, dash] &\\ \circ \ar[rd, dash]& \circ \ar[d, dash] & \circ \ar[ld, dash] \\ &\circ &
\end{tikzcd}$ admits no orthocomplementations.
\item The lattice $\mathbf{M}_4 = \begin{tikzcd}[column sep={1em,between origins},row sep={2em,between origins}]
&&&\circ \ar[llld, dash] \ar[ld,dash] \ar[rd, dash] \ar[rrrd, dash] &&&\\ \circ \ar[rrrd, dash]&& \circ \ar[rd, dash] && \circ \ar[ld, dash] && \circ \ar[llld, dash] \\ &&&\circ &&&
\end{tikzcd}$ admits three orthocomplementations.
\item The hexagon lattice $\begin{tikzcd}[column sep={1.5em,between origins},row sep={1.7em,between origins}]
& \circ \ar[ld, dash] \ar[rd, dash] & \\
a \ar[d, dash] & & b \ar[d, dash] \\
c \ar[rd, dash] & & d \ar[ld, dash] \\
& \circ &
\end{tikzcd}$ admits a unique orthocomplementation, but it is not uniquely complemented. Indeed both of the functions $f: \begin{tikzcd}[column sep={1.5em,between origins},row sep={1.7em,between origins}]
& \circ \ar[ld, dash] \ar[rd, dash] & \\
a \ar[d, dash] \ar[rr, leftrightarrow] & & b \ar[d, dash] \\
c \ar[rd, dash] \ar[rr, leftrightarrow] & & d \ar[ld, dash] \\
& \circ &
\end{tikzcd}$ and $g: \begin{tikzcd}[column sep={1.5em,between origins},row sep={1.7em,between origins}]
& \circ \ar[ld, dash] \ar[rd, dash] & \\
a \ar[d, dash] \ar[rrd, leftrightarrow] & & b \ar[d, dash] \\
c \ar[rd, dash] \ar[rru, leftrightarrow] & & d \ar[ld, dash] \\
& \circ &
\end{tikzcd}$ are complementations. Only $g$ is an orthocomplementation, because $f$ does not reverse order: $a \geq c$  and $f(a) = b \geq d = f(c)$.
\end{itemize}
\end{example}

\begin{theorem}[De Morgan's laws]
Let $L$ be an ortholattice, then for all $x,y\in L$
\begin{enumerate}
\item $(x\vee y)^\perp = x^\perp \wedge y^\perp$;
\item $(x\wedge y)^\perp = x^\perp \vee y^\perp$.
\end{enumerate}
\end{theorem}
\begin{proof}
From $x\leq x\vee y$ and $y\leq x\vee b$, we have $(x\vee b)^\perp \leq x^\perp$ and $(x\vee b)^\perp \leq y^\perp$. By \ref{corollary:orderLattice} we have $(x\vee y)^\perp\leq x^\perp \wedge y^\perp$. For the other inequality we start with $x^\perp \geq x^\perp \wedge y^\perp$ and $y^\perp \geq x^\perp \wedge y^\perp$ to obtain $x\vee y \leq (x^\perp \wedge y^\perp)^\perp$, which implies $x^\perp \wedge y^\perp \leq (x\vee y)^\perp$.
\end{proof}

\begin{proposition}
Let $L$ be a complemented lattice.

If the complement is an involution and satisfies either of the de Morgan laws, then $L$ is an ortholattice.
\end{proposition}
\begin{proof}
Assume the de Morgan law $(x\vee y)^\perp = x^\perp \wedge y^\perp$ holds. Assume $x\leq y$. Then $x\vee y = y$, so
\[ y^\perp = (x\vee y)^\perp = x^\perp \wedge y^\perp \]
meaning $y^\perp \leq x^\perp$.
\end{proof}

The requirement that the complement be an involution is important. There are lattices in which the de Morgan laws hold that are not ortholattices.

\begin{example}
The lattice $\mathbf{M}_3 = \begin{tikzcd}[column sep={2em,between origins},row sep={2em,between origins}]
&\top \ar[ld, dash] \ar[d,dash] \ar[rd, dash] &\\ a \ar[rd, dash]& b \ar[d, dash] & c \ar[ld, dash] \\ &\bot &
\end{tikzcd}$ admits a complementation $'$ such that $a' = b$, $b' = c$ and $c' = a$ that is clearly not an orthocomplementation, but does satisfy the de Morgan laws.
\end{example}

TODO Ockham algebras, De Morgan algebras, Kleene algebras, Stone algebras.

\subsection{Boolean lattices}
\begin{definition}
A distributive complemented lattice is called a \udef{Boolean lattice} or \udef{Boolean algebra}.
\end{definition}

\begin{proposition} \label{prop:BooleanComplementLargestDisjoint}
Let $L$ be a Boolean lattice. For all $x\in L$ the complement $x^\perp$ is the largest element disjoint from $x$.
\end{proposition}
\begin{proof}
Let $x\in L$. We need to show that $y\leq x^\perp$ for all $y\in \{x\}^d$, or, equivalently, $x^\perp \vee y = x^\perp$. By the uniqueness of the complement, \ref{prop:distributiveComplementUnique}, it is enough to prove that $x^\perp \vee y$ is a complement. Indeed
\begin{align*}
x \vee (x^\perp \vee y) &= (x\vee x^\perp) \vee y = \top \vee y = \top \\
x \wedge (x^\perp \vee y) &= (x\wedge x^\perp)\vee (x\wedge y) = \bot \vee \bot = \bot.
\end{align*}
\end{proof}

\begin{proposition}
Every Boolean lattice is an ortholattice.
\end{proposition}
\begin{proof}
By \ref{prop:distributiveComplementUnique} we know that a Boolean lattice is uniquely complemented, so its complement is an involution by \ref{lemma:uniqueComplementInvolution}. We just need to check the complementation reverses order.

Let $x\leq y$. Then $y^\perp \wedge x \leq y^\perp \wedge y = \bot$, so $y^\perp \wedge x = \bot$ and thus $y^\perp$ is disjoint from $x$. Then $y^\perp \leq x^\perp$ follows from \ref{prop:BooleanComplementLargestDisjoint}.
\end{proof}
\begin{corollary}
The laws of de Morgan hold in Boolean lattices.
\end{corollary}

\begin{proposition}
Let $L$ be a Boolean lattice with complement $\perp$ and $a,b\in L$. Then $[a,b]$ is a Boolean lattice with complement $': x\mapsto (x^\perp \wedge b)\vee a$. 
\end{proposition}
\begin{proof}
Clearly $[a,b]$ inherits distributivity from $L$. All we need to show is that for all $x\in [a,b]$ the complement of $x$ in $[a,b]$ is $x'$. We calculate
\begin{align*}
x \wedge x' &= x\wedge ((x^\perp \wedge b)\vee a) = (x\wedge(x^\perp \wedge b))\vee (x\wedge a) = (( x\wedge x^\perp) \wedge b)\vee (x\wedge a) = (\bot \wedge b)\vee a = \bot \vee a = a \\
x \vee x' &= x\vee ((x^\perp \wedge b)\vee a) = ((x\vee x^\perp) \wedge (x\vee b))\vee a = (\top \wedge b) \vee a = b\vee a = b.
\end{align*}
\end{proof}

\begin{proposition}
An algebra of sets is a Boolean algebra with as top the unit $\Omega$, as bottom the empty set $\emptyset$ and as complement $A\mapsto A^c = \Omega\setminus A$.
\end{proposition}

\section{Completions}
TODO: Dedekind-MacNeille completion.

\section{Lattices of subgroups}

\section{Formal concept analysis}
\url{file:///C:/Users/user/Downloads/978-3-540-31881-1.pdf}
\url{file:///C:/Users/user/Downloads/978-3-662-49291-8.pdf}

\begin{definition}
A \udef{context} is a triple $\seq{G,M,I}$ where $G$ is a set of \udef{objects}, $M$ is a set of \udef{attributes} and $I\subseteq G\times M$ is a binary relation.

For $g\in G, m\in M$ we interpret $gIm$ as ``the object $g$ has the attribute $m$''.



A \udef{concept} is a pair $\seq{A,B}$ where $A\subset G$ is a set of objects and $B\subset M$ is a set of attributes such that
\begin{itemize}
\item $A = \setbuilder{g\in G}{\forall m\in B: gIm}$;
\item $B = \setbuilder{m\in M}{\forall g\in A: gIm}$
\end{itemize}
\end{definition}
The letters $G$ and $M$ come from the German: Gegenstände and Merkmale. The $I$ is for ``incidence relation'' (I think).





\chapter{Groups}
\url{https://www.maths.ed.ac.uk/~tl/gt/gt.pdf}

\section{Basic definitions}

\begin{definition}
A \udef{group} is a structured set $(G,\boldsymbol{\cdot}, e)$ where $e\in G$ and $\boldsymbol{\cdot}$ is a binary operation on $G$
\[\boldsymbol{\cdot}: G\times G \to G: (g,h)\mapsto g\cdot h\]
such that
\begin{enumerate}
\item $\boldsymbol{\cdot}$ is associative:
\[ \forall g_1,g_2,g_3 \in G: \quad g_1\cdot (g_2\cdot g_3) = (g_1\cdot g_2)\cdot g_3 \]
\item $e$ is an \udef{identity} for $\boldsymbol{\cdot}$:
\[ \forall g\in G: \quad g\cdot e = e\cdot g = g \]
\item every element has an \udef{inverse}:
\[ \forall g\in G: \exists h \in G: \quad  gh = hg = e \]
We write the inverse $h$ as $g^{-1}$. 
\end{enumerate}
If $\boldsymbol{\cdot}$ is satisfies
\[ \forall g_1, g_2 \in G: \quad g_1 \cdot g_2 = g_2\cdot g_1 \]
then the group is called \udef{commutative} or \udef{abelian}.

The cardinality of $G$ is the \udef{order} of the group, denoted $|G|$.
\end{definition}

\begin{example}
The \udef{Klein 4-group} has carrier $A = \{e,a,b,c\}$ and is defined by the Cayley table
\[ \begin{array}{l|llll}
\boldsymbol{\cdot} & e & a & b & c \\ \hline
e & e & a & b & c \\
a & a & e & c & b \\
b & b & c & e & a \\
c & c & b & a & e
\end{array}. \]
It is commutative (which can be seen by noting that the Cayley table equals its transpose).
It is also
\begin{itemize}
\item the unique commutative $4$-element group with $a^2 = b^2 = c^2 = e$;
\item the unique commutative $4$-element semigroup with identity $e = a^2 = b^2 = c^2$, and $ab = c$.
\end{itemize}
\end{example}

\begin{lemma}
Let $S$ be a semigroup. Then $S$ is a group \textup{if and only if}
\[ \forall x\in S: \; xS = S = Sx. \]
\end{lemma}
\begin{proof}
TODO? Need identity??
\end{proof}

The inverse of any element of the group is unique by \ref{inverseUniqueness}. Thus $(\cdot)^{-1}$ is a well-defined function.

\begin{proposition}
A group is a structure of type $(e, (\cdot)^{-1}, \boldsymbol{\cdot})$ with arity defined by
\[ \alpha(e) = 0, \qquad \alpha((\cdot)^{-1}) = 1, \qquad \alpha(\boldsymbol{\cdot}) = 2. \]
Conversely, a $(e, (\cdot)^{-1}, \boldsymbol{\cdot})$-algebra is a group if
\begin{itemize}
\item $\boldsymbol{\cdot}$ is associative;
\item $\forall g\in G: g\cdot e = e\cdot g = g$;
\item $\forall g\in G: g\cdot g^{-1} = g^{-1}\cdot g = e$.
\end{itemize}
\end{proposition}
In particular the concepts of homomorphism and isomorphism apply.

\subsection{Group homomorphisms}
\begin{definition}
Let $G,H$ be groups. A function $f: G\to H$ is called a \udef{group homomorphism} if it is a $\{\cdot\}$-homomorphism, i.e. for all $x,y\in G$
\[ f(x\cdot y) = f(x)\cdot f(y). \]
\end{definition}

\begin{lemma} \label{groupHomomorphismPreservesSignature}
Let $G,H$ be groups. A function $f: G\to H$ is a group homomorphism iff it is a $\{\cdot, {}^{-1}, e\}$-homomorphism.
\end{lemma}
\begin{proof}
The direction $\Leftarrow$ is clear.

Now assume $f$ is a $\{\cdot\}$-homomorphism and take arbitrary $x\in G$. We first prove that it also preserves the identity: we have $f(e) = f(e\cdot e) = f(e)\cdot f(e)$. Multiplting both sides by $f(e)^{-1}$ gives $f(e) = e$.

In order to prove that $f(x^{-1}) = f(x)^{-1}$, it is enough, by \ref{inverseUniqueness}, to prove that $f(x^{-1})$ is an inverse of $f(x)$. We calculate
\[ f(x^{-1})f(x) = f(x^{-1}x) = f(e) = e \qquad\text{and}\qquad f(x)f(x^{-1}) = f(xx^{-1}) = f(e) = e. \]
\end{proof}


\subsection{Notations}
We can use whatever symbols we want to denote the group operation, but there are two main conventions:
\begin{enumerate}
\item In \udef{multiplicative notation} the group operation is denoted by $\boldsymbol{\cdot}$, $*$ or just by concatenation (i.e.\ we write $gh$ instead of $g\cdot h$). In this case the inverse of $g$ is written $g^{-1}$, the neutral element $e$ is denoted $1$ and we can define
\[ g^n \defeq \underbrace{gg\ldots g}_{\text{$n$ factors}}\]
which is unambiguous due to associativity. Also
\[ g^{-n} \defeq (g^{-1})^n = (g^{n})^{-1}. \]
\item \udef{Additive notation} is mainly used for abelian groups. Conversion between multiplicative and additive notation is as follows:
\[ \begin{tikzcd}
g\cdot h \arrow[r, leftrightarrow]& g+h \\
1 \arrow[r, leftrightarrow]& 0 \\
g^{-1} \arrow[r, leftrightarrow]& -g \\
g^n \arrow[r, leftrightarrow] & ng.
\end{tikzcd} \]
\end{enumerate}

\begin{lemma} \label{calculusRepeatedGroupOperation}
Let $G$ be a group, $g\in G$ and $m,n\in \Z$. Then
\begin{enumerate}
\item in multiplicative notation we have
\[ g^mg^n = g^{m+n} \qquad (g^m)^n = g^{mn}; \]
\item in additive notation we have
\[ mg+ng = (m+n)g \qquad n(mg) = (mn)g. \]
\end{enumerate}
These statements are equivalent.
\end{lemma}

\subsection{Translation invariance}
\label{sec:translationInvariance}
\begin{definition}
Let $G$ be a group, $X$ a set and $f:G\times G \to X$ a binary function. Then $f$ is called
\begin{itemize}
\item \udef{left translation invariant} if $\forall x,y,z\in G:\; f(x, y) = f(zx,zy)$;
\item \udef{right translation invariant} if $\forall x,y,z\in G:\; f(x, y) = f(xz,yz)$;
\item \udef{translation invariant} if $f$ is left and right translation invariant.
\end{itemize}
\end{definition}

TODO: just for relations??
\begin{proposition}[Universal property translation invariance]
Let $G$ be a group. Define
\[ \Delta_r:G\times G\to G: (x,y) \to xy^{-1} \qquad \Delta_l:G\times G\to G: (x,y) \to x^{-1}y. \]
\begin{enumerate}
\item For any set $X$ and right translation invariant function $f:G\times G\to X$, there exists a unique $\widetilde{f}: G\to X$, such that
\[ \begin{tikzcd}
G\times G \rar{\Delta_r} \ar[dr, swap, "{f}"] & G \dar[dashed]{\widetilde{f}} \\
 & X
\end{tikzcd} \qquad\text{commutes.} \]
\item For any set $X$ and left translation invariant function $f:G\times G\to X$, there exists a unique $\widetilde{f}: G\to X$, such that
\[ \begin{tikzcd}
G\times G \rar{\Delta_l} \ar[dr, swap, "{f}"] & G \dar[dashed]{\widetilde{f}} \\
 & X
\end{tikzcd} \qquad\text{commutes.} \]
\end{enumerate}
\end{proposition}
\begin{proof}
TODO

(1) $\widetilde{f} = f(-, e)$;

(2) $\widetilde{f} = f(e, -)$;
\end{proof}

TODO which do we conventionally choose?

\begin{lemma} \label{leftToRightTranslationInvarianceLemma}
Let $\sSet{G, \cdot, 1}$ be a group. Then
\begin{enumerate}
\item $(-)^{-1} = \Delta_r\circ(\underline{1}, \id_G) = \Delta_l\circ (\id_G, \underline{1})$;
\item $\boldsymbol{\cdot} = \Delta_r\circ \big(\proj_1, \Delta_r\circ(\underline{1}, \proj_2)\big) = \Delta_l\circ \big(\Delta_l\circ(\proj_1, \underline{1}), \proj_2\big)$;
\item $\Delta_r = \Delta_l\circ \big(\Delta_l\circ (\proj_1, \underline{1}), \Delta_l\circ (\proj_2, \underline{1}) \big)$;
\item $\Delta_l = \Delta_r\circ \big(\Delta_l\circ (\underline{1},\proj_1), \Delta_l\circ (\underline{1}, \proj_2) \big)$.
\end{enumerate}
\end{lemma}

\begin{definition}
Let $G$ be a group and $X\subseteq G^2$ a set. Then $X$ is called
\begin{itemize}
\item \udef{left translation invariant} if the indicator function $\chi_X$ is left translation invariant;
\item \udef{right translation invariant} if the indicator function $\chi_X$ is right translation invariant;
\item \udef{translation invariant} if the indicator function $\chi_X$ is translation invariant.
\end{itemize}
\end{definition}

\begin{lemma}
Let $G$ be a group and $X\subseteq G^2$ a set. Then
\begin{enumerate}
\item $X$ is left translation invariant \textup{if and only if} $\forall g\in G: \forall (x,y)\in X: \; (gx, gy)\in X$;
\item $X$ is right translation invariant \textup{if and only if} $\forall g\in G: \forall (x,y)\in X: \; (xg, yg)\in X$.
\end{enumerate}
\end{lemma}

\begin{lemma}
A binary relation is
\begin{enumerate}
\item left translation invariant \textup{if and only if} it is left compatible;
\item right translation invariant \textup{if and only if} it is right compatible.
\end{enumerate}
\end{lemma}
\begin{proof}
TODO
\end{proof}

For a left compatible binary relation $R$, we have
\[ xRy \qquad\iff\qquad x^{-1}y \in \widetilde{R}. \]

\begin{proposition} \label{congruenceTranslationInvariant}
Let $G$ be a group and $\mathfrak{q}$ a congruence on $G$. Then $\mathfrak{q}$ is translation invariant. 
\end{proposition}
\begin{proof}
Take arbitrary $(x,y)\in\mathfrak{q}$ and $z\in G$. Then $(z,z)\in\mathfrak{q}$ by reflexivity and thus $(z\cdot x, z\cdot y) = (z,z)\cdot (x,y)\in\mathfrak{q}$. Similarly $(x\cdot z, y\cdot z) = (x,y)\cdot (z,z)\in\mathfrak{q}$.
\end{proof}

In particular, as the kernel of a homomorphism is a congruence, it is translation invariant.
When dealing with groups, we will redefine $\ker$ to mean $\widetilde{\ker}$.

\subsection{Subgroups}
\begin{definition}
Let $(G,\boldsymbol{\cdot})$ be a group. We call $(H,*)$ a \udef{subgroup} if it is a group and $H\subseteq G$ and $* = \boldsymbol{\cdot}|_H$.
\end{definition}

\begin{lemma}[Subgroup criterion] \label{subgroupCriterion}
Let $(G,\boldsymbol{\cdot})$ be a group and $H$ a non-empty subset of $G$. The following are equivalent:
\begin{enumerate}
\item $(H,\boldsymbol{\cdot}|_H)$ is a subgroup;
\item for all $a,b\in H$:
\begin{itemize}
\item $a\cdot b \in H$,
\item $a^{-1}\in H$;
\end{itemize}
\item for all $a,b\in H: a\cdot b^{-1} \in H$;
\item $H$ is a sub-$\{\cdot, {}^{-1}, e\}$-algebra of $G$;
\item $H$ is a non-empty sub-$\{\cdot, {}^{-1}\}$-algebra of $G$;
\end{enumerate}
\end{lemma}

\begin{lemma}
Let $G$ be a group and $H_1, H_2$ be subgroups. Then $H_1\cap H_2$ is again a subgroup of $G$.
\end{lemma}
\begin{lemma}
Let $f:G\to H$ be a group homomorphism. Then $\ker(f)$ is a subgroup.
\end{lemma}

\subsubsection{Cosets}
\begin{definition}
Let $G$ be a group and $H\subseteq G$ a subgroup. We call a subset of the form
\begin{itemize}
\item $g\cdot H$ for some $g\in G$ a \udef{left coset};
\item $H\cdot g$ for some $g\in G$ a \udef{right coset}.
\end{itemize}
A \udef{coset} is a subset that is either a left coset or a right coset.
\end{definition}

\begin{lemma} \label{differentCosetsDisjoint}
Let $G$ be a group and $H\subseteq G$ a subgroup. Any two left (resp. right) cosets are either identical or disjoint.
\end{lemma}
\begin{proof}
Take $g,h\in G$. Assume $x\in gH\cap hH$. Then there exist $x_1,x_2\in H$ such that $gx_1 = x = hx_2$. Thus $g = hx_2x_1^{-1}$ and $h = gx_1x_2^{-1}$, meaning $gH = hx_2x_1^{-1}H = hH$ by \ref{groupCriterion}.
\end{proof}

\subsubsection{Lagrange's theorem}
\begin{theorem}
Let $G$ be a group and $H$ a subgroup of $G$. Then
\[ |G| = [G:H]\cdot |H|. \]
\end{theorem}
If $G$ is finite, $|G|$ and $|H|$ are natural numbers. If $G$ is infinite, the theorem still holds, but the orders and index are cardinals.

\subsubsection{Normal subgroups}
\begin{definition}
Let $G$ be a group. A subgroup $N\subseteq G$ is called \udef{normal} or \udef{self-conjugate} if $gNg^{-1} \subseteq N$ for all $g\in G$.

We write $N \lhd G$.
\end{definition}

\begin{proposition} \label{congruenceNormalSubgroup}
Let $G$ be a group.
A translation invariant binary relation $\mathfrak{q}$ on $G$ is a $\{\cdot, {}^{-1}, e\}$-congruence \textup{if and only if} $\widetilde{\mathfrak{q}}$ is a normal subgroup.
\end{proposition}
Note that all congruences are translation invariant by \ref{congruenceTranslationInvariant}. The hypothesis is necessary for $\widetilde{\mathfrak{q}}$ to be well-defined.
\begin{proof}
First assume $\mathfrak{q}$ is a congruence and take $z\in\widetilde{\mathfrak{q}}$ and arbitrary $g\in G$. We need to show that $gzg^{-1}\in \widetilde{\mathfrak{q}}$. We can find $(x,y)\in \mathfrak{q}$ such that $z = xy^{-1}$. Because $\mathfrak{q}$ is reflexive, we have $(g,g)\in\mathfrak{q}$. Because is it is a subalgebra of $G^2$, we have $(gx,gy) = (g,g)\cdot (x,y)\in \mathfrak{q}$. So $gzg^{-1} = g(xy^{-1})g^{-1} = gx(gy)^{-1} \in \widetilde{\mathfrak{q}}$.

Now assume $\widetilde{\mathfrak{q}}$ is a normal subgroup. We first check that $\mathfrak{q}$ is an equivalence relation:
\begin{itemize}
\item \emph{reflexivity}: $e = gg^{-1}\in \widetilde{\mathfrak{q}}$, so $(g,g)\in\mathfrak{q}$ for all $g\in G$;
\item \emph{symmetry}: if $xy^{-1}\in \widetilde{\mathfrak{q}}$, then $yx^{-1} = (xy^{-1})^{-1}\in \widetilde{\mathfrak{q}}$;
\item \emph{transitivity}: if $xy^{-1}, yz^{-1}\in \widetilde{\mathfrak{q}}$, then $xy^{-1}yz^{-1} = xz^{-1}\in \widetilde{\mathfrak{q}}$.
\end{itemize}
Now we need to show that $\mathfrak{q}$ is a subalgebra. To show $\mathfrak{q}$ is closed under taking the inverse, take $(x,y) \in \mathfrak{q}$. Then $xy^{-1}\in \widetilde{\mathfrak{q}}$ and $y^{-1}(xy^{-1})^{-1}y = x^{-1}y = x^{-1}(y^{-1})^{-1} \in \widetilde{\mathfrak{q}}$ by normality. So $(x^{-1}, y^{-1}) \in \mathfrak{q}$.

To show $\mathfrak{q}$ is closed under the binary group operation, take $(x,y),(u,v)\in\mathfrak{q}$. Then $uv^{-1}, xy^{-1}$ and $y^{-1}x = y^{-1}(xy^{-1})y$ are elements of $\widetilde{\mathfrak{q}}$. Then
\[ xu(yv)^{-1} = xuv^{-1}y^{-1} = x(uv^{-1})(y^{-1}x)x^{-1} \in \widetilde{\mathfrak{q}}. \]

Finally it is enough to note that $\mathfrak{q}$ is not empty, by \ref{subgroupCriterion}.
\end{proof}
Note that the left factorisation $\widetilde{\mathfrak{q}}$ is equal to the right factorisation $\widetilde{\mathfrak{q}}$.
\begin{corollary} \label{kernelNormalSubgroup}
Let $f: G\to H$ be a group homomorphism. Then $\ker f$ is a normal subgroup.
\end{corollary}

If $N\lhd G$ is a normal subgroup, we define the quotient algebra
\[ G/N \defeq G/(\setbuilder{(x,y)\in G^2}{xy^{-1}\in N}). \]
This is a group because homomorphisms preserve associativity, inverses and identity (TODO ref). We call such a group a \udef{quotient group}.

We denote the equivalence classes by $[x]_N$.

\subsection{Conjugation}
\begin{definition}
Let $G$ be a group and $g\in G$ an element. Then the mapping
\[ \Ad_g: G\to G: x\mapsto g^{-1}xg \]
is called \udef{conjugation by $g$}.
We also write $h^g \defeq \Ad_g(h) = g^{-1}hg$.
\end{definition}
Thus a subgroup is normal if and only if $\Ad_g[N] \subseteq N$ for all $g\in G$.


\subsubsection{Conjugacy}
\begin{definition}
Let $G$ be a group. Elements $g,h\in G$ are called \udef{conjugate} if $\exists x: \; \Ad_x(g) = h$.
\end{definition}

\begin{proposition}
Conjugacy is an equivalence relation.
\end{proposition}
The equivalence classes under conjugation are called \udef{conjugacy classes}.


\subsubsection{Centraliser and normaliser}

\begin{lemma}
Let $G$ be a group and $A\subseteq G$ a subset. Then
\begin{enumerate}
\item $Z_G(A) = \setbuilder{g\in G}{\forall a\in A:\;\Ad_g(a) = a}$;
\item $N_G(A) = \setbuilder{g\in G}{\Ad_g[A] = A}$.
\end{enumerate}
\end{lemma}

\begin{proposition}
Let $G$ be a group and $A\subseteq G$ a subset. Then
\begin{enumerate}
\item $Z_G(A) \lhd N_G(A) \lhd G$;
\item $N_G(A)$ is the largest subgroup of $G$ in which $A$ is normal.
\end{enumerate}
\end{proposition}
\begin{corollary}
$Z_G\lhd G$.
\end{corollary}


\subsubsection{Inner and outer automorphisms}
\begin{lemma}
Let $G$ be a group and $g\in G$ an element. Then $\Ad_g$ is an automorphism.
\end{lemma}
\begin{definition}
Let $G$ be a group.
Automorphisms of the form $\Ad_g$ for some $g\in G$ are called \udef{inner automorphisms}. Automorphisms that are not of this form are called \udef{outer automorphisms}.

The set of inner automorphisms forms a group, denoted $\Inn(G)$.
\end{definition}

\begin{theorem}[N/C theorem]
Let $G$ be a group and $H\subseteq G$ a subgroup. Then
\[ N_H(H) / Z_G(H) \cong \Inn(H). \]
\end{theorem}
\begin{proof}
TODO \url{https://proofwiki.org/wiki/Centralizer_is_Normal_Subgroup_of_Normalizer}
\end{proof}
\begin{corollary}
Let $G$ be a group. Then $G / Z_G \cong \Inn(G)$.
\end{corollary}


\subsection{Direct product}
\begin{definition}
The \udef{direct product} $G \equiv H\otimes F$ of two groups $H$ and $F$ is defined with the following operation:
\[ (H\otimes F) \times (H\otimes F) \rightarrow (H\otimes F): ((h_1,f_1),(h_2,f_2)) \mapsto (h_1\cdot h_2, f_1\cdot f_2)\]
\end{definition}
The direct product is a group with
\[ \begin{cases}
e_G = (e_H,e_F) \\
g^{-1} = (h^{-1}, f^{-1})\qquad \forall g = (h,f) \in G.
\end{cases} \]
The groups $F$ and $H$ are subgroups of $G$ and can be recovered by considering, respectively the elements of $G$ of the form $(e_H, g)$ and $(g ,e_F)$.

\subsection{Semidirect product}
TODO

\section{Types of groups}
\begin{example}
Examples of Groups:
\begin{enumerate}
\item The \udef{trivial group} $\{e\}$.
\item $\Z_n = 0:(n-1)$ with addition modulo $n$, is a group of order $n$.
\item $\Z_n$, the group of all $n^{\text{th}}$ roots of $1$ with the ordinary product, is of order $n$.
\begin{itemize}
\item $Z_2 = \{1,-1\}$
\item $Z_3 = \{1, e^{i2/3\pi}, e^{i1/3\pi}\}$
\end{itemize}
\item $S_n$, the group of all permutations of $n$ elements, is of order $n!$.
\item Integers with addition.
\item $\mathbb{R}\setminus\{0\}$ with multiplication.
\item The square (i.e.\ $n\times n$) invertible matrices with matrix multiplication form a group.
\end{enumerate}
\end{example}

\begin{proposition}
The groups $Z_n$ and $\Z_n$ are isomorphic.
\end{proposition}
We use $Z_n$ to denote the group if we are using multiplicative notation and $\Z_n$ if we are using additive notation.

In particular we have $\Z_2 = \sSet{\{0,1\},+}$ and $Z_2 = \sSet{\{1,-1\},\cdot}$.

\begin{lemma}
All groups of order $2$ are isomorphic to $\Z_2$.
\end{lemma}
\begin{proof}
Let $G = \sSet{\{e,g\}, \cdot}$ be a groups of order $2$, with $e$ the identity.
We must have $g\cdot g = e$. Indeed, from $g \neq e$, we get $g\cdot g \neq g\cdot e = g$ and $g\cdot g = e$ is the only other option.

Consider the function
\[ f: G \to \Z_2: \begin{cases}
e\mapsto 0 \\ g \mapsto 1
\end{cases}. \]
This functions is clearly bijective. We just need to see that it is a homomorphism. Indeed we have
\[ \begin{cases}
f(e\cdot e) = f(e) = 0 = 0+0 = f(e) + f(e) & f(e\cdot g) = f(g) = 1 = 0+1 = f(e) + f(g) \\
f(g\cdot e) = f(g) = 1 = 1+0 = f(g) + f(e) & f(g\cdot g) = f(e) = 0 = 1+1 = f(g) + f(g). 
\end{cases} \] 
\end{proof}

\subsection{Words, relations and presentations}
\begin{example}
Quaternion group
\[ \mathbb{H} \defeq \group\setbuilder{a,b}{a^4=e, a^2=b^2, b^{-1}ab = a^{-1}} \]
\end{example}

\subsection{Cyclic groups}
\begin{definition}
A group is called \udef{cyclic} if it is generated by a single element.
\end{definition}
\begin{lemma}
\begin{enumerate}
\item The group $(\Z, +)$ is cyclic.
\item Every cyclic group is a an image of $\Z$ by a homomorphism.
\item Every cyclic group is isomorphic to $\Z$ or $\Z/m\Z$.
\end{enumerate}
\end{lemma}
We write $\Z_m$ or $C_m$ for $\Z/m\Z$.
 
\subsection{Torsion groups and orders of elements}
\begin{definition}
Let $G$ be a group. An element $a$ satisfying $a^n = 1$ for some $n$ is said to be of \udef{finite order}. In this case the \udef{order} of the element $a$ is $n$.

A group in which every element is of finite order is called a \udef{torsion group} or a \udef{periodic group}.
\end{definition}
\begin{lemma}
Every finite group is a torsion group. The converse is not true.
\end{lemma}
\begin{proof}
Let $G$ be a finite group. Assume $G$ is not a torsion group. Then we can find an element $g\in G$ that is not of finite order. Consider the mapping $\N \to G: n\mapsto g^n$. The image of this mapping is a subset of $G$ and thus finite, so the mapping is not injective, so we can find $n<m\in\N$ such that $g^n = g^m$. Then $g^{m-n}=1$ with $m-n\in \N$, so $g$ is of finite order, which is a contradiction. This falsity of the converse is shown by the following examples.
\end{proof}

\begin{example}
The set
\[ \setbuilder{z\in \C}{z^n = 1\;\text{for some}\; n\in \Z} \]
together with complex multiplication forms an infinite torsion group.
\end{example}

\begin{lemma}
Let $G$ be an abelian group. Then the set of all elements of finite order forms a subgroup, called the \udef{torsion subgroup}.
\end{lemma}


\subsection{Permutation groups}
\begin{proposition} \mbox{}
\begin{enumerate}
\item Let $X$ be a set. The set of bijections $X\to X$ forms a group;
\item Let $X,Y$ be sets. The groups of bijections on $X$ and $Y$ are isomorphic \textup{if and only if} $X$ and $Y$ are equinumerous.
\end{enumerate}
\end{proposition}

\begin{definition}
We call the group of bijections on a set $X$ the \udef{symmetric group} of $X$, denoted $S(X)$.
\begin{itemize}
\item The \udef{degree} of $S(X)$ is the cardinality of $X$.
\item For any cardinal $n$, we denote the unique permutation group of degree $n$ by $S_n$.
\item Elements of $S_n$ are called \udef{permutations} and subgroups of $S_n$ are called \udef{permutation groups}.
\end{itemize}
\end{definition}

TODO: cfr. Clifford algebra with $V$ containing the transpositions.

\begin{proposition}
For all sets $X$, we have $S(X) = S_{|X|}$.
\end{proposition}
In general we state and prove results for $S_n$, without loss of generality.

\begin{lemma}
For all cardinals $\kappa >_c 2$, the permutation group $S_\kappa$ is non-abelian.
\end{lemma}

\subsubsection{Cycles}
\subsubsection{Transpositions, parity and the alternating group}
\begin{definition}
Let $S_n$ be a permutation group and $x\in S_n$. Consider the set $\Fixedpoints(x)$. If $|\Fixedpoints(x)| = 2$, then $x$ is called a \udef{transposition}.
\end{definition}

\begin{definition}
Let $S_n$ be a finite permutation group. We call the function
\[ p_n: S_n \to Z_2: x\mapsto \begin{cases}
1 & \text{$|\Fixedpoints(x)|$ is even} \\
-1 & \text{$|\Fixedpoints(x)|$ is odd}
\end{cases} \]
the \udef{parity homomorphism}.

The group $A_n \defeq \ker p_n$ is called the \udef{alternating group} of \udef{degree} $n$.
\end{definition}

\begin{proposition}
The parity homomorphism is a homomorphism.
\end{proposition}

\begin{lemma}
The alternating group $A_n$ has order $n!/2$.
\end{lemma}

\subsection{Dihedral groups}
\begin{definition}
Dihedral group of order $2n$.
\[ D_n \defeq \group\setbuilder{ a,b }{a^n=b^2=e, b^{-1}ab = a^{-1} }. \]
\end{definition}

\begin{proposition}
Let $n\in \N$. Then
\[ Z_G = \begin{cases}
\{e, a^{n/2}\} & \text{$n$ is even} \\
\{e\} & \text{$n$ is odd}.
\end{cases} \]
\end{proposition}
\begin{corollary} \label{dihedralDoubleCover}
For all $n\in \N$ there is a short exact sequence
\[ \begin{tikzcd}
1 \rar & \Z_2 \rar & D_{2n} \rar & D_n \rar & 1.
\end{tikzcd} \]
\end{corollary}

\subsubsection{Full dihedral group}
TODO: full dihedral group $D$ of isometries of $\C$ that fix the origin.
\[ \begin{tikzcd}
1 \rar & \T \rar & D\rar & Z_2 \rar & 1
\end{tikzcd} \qquad\text{is short exact.} \]

\section{Short exact sequences}
\subsection{Quotient sequences}
\begin{proposition}
Let $G$ be a group and $N\lhd G$ a normal subgroup. Then
\[ \begin{tikzcd}
1 \rar & N \rar[hookrightarrow]{\subseteq} & G \rar{[\cdot]_N} & G/N \rar & 1
\end{tikzcd} \]
is a short exact sequence.
\end{proposition}
\begin{proof}
Clearly the inclusion $N\hookrightarrow G$ is injective and $[\cdot]_N$ is surjective. Finally note that $x\in \ker[\cdot]_N \iff [x]_N = [e]_N \iff xe^{-1} = x\in N$.
\end{proof}

\begin{proposition}
For any short exact sequence of groups
\[ \begin{tikzcd}
1 \rar & H_1 \rar{\alpha} & G \rar{\beta} & H_2 \rar & 1
\end{tikzcd}, \]
there exist isomorphisms $f,g$ and subgroup $N\lhd G$ such that
\[ \begin{tikzcd}
1 \rar & H_1 \dar{f} \rar{\alpha} & G \dar{\id_G} \rar{\beta} & H_2 \dar{g} \rar & 1 \\
1 \rar & N \rar[hookrightarrow] & G \rar{[\cdot]_N} & G/N \rar & 1
\end{tikzcd} \]
commutes.
\end{proposition}
\begin{proof}
We can take $N = \im(\alpha) = \ker(\beta)$, which is a normal subgroup by \ref{kernelNormalSubgroup}. Because $\alpha$ is injective, $\alpha|^{\im(\alpha)}: H_1 \to N$ is bijective. We take $f$ to be this.

The function $\beta': G/N \to H_2$ defined in the factor theorem \ref{factorTheorem} is bijective because $\beta$ is surjective.

Both constituent squares clearly commute. So the rectangle commutes by \ref{commutingRectangle}.
\end{proof}

So in some sense all short exact sequences of groups are of the form
\[ \begin{tikzcd}
1 \rar & N \rar[hookrightarrow] & G \rar & G/N \rar & 1
\end{tikzcd}. \]
Given $G$ and either $N$ or $G/N$, we can easily find the third group in the sequence. Given $N$ and $G/N$, there may be several inequivalent ways to complete the short exact sequence. Groups that fit in the middle of a short exact sequence are called group extensions.

\subsection{Group extensions}
\begin{definition}
Let $N,Q$ be groups. An \udef{extension} of $Q$ by $N$ is a group $G$ such that
\[
\begin{tikzcd}
1 \ar[r] & N \ar[r, "\iota"] & G \ar[r, "\pi"] & Q \ar[r] & 1
\end{tikzcd}.
\]
is a short exact sequence.
\end{definition}
\begin{lemma}
If $G$ is an extension of $Q$ by $N$, then $G$ is a group (TODO: closure), $\iota(N)$ is a normal subgroup of $G$ and $Q$ is isomorphic to $Q$.
\end{lemma}

\begin{example}
The real numbers $\R$ are an extension of the unit complex numbers by the integers $\Z$:
\[ \begin{tikzcd}
0 \rar & \Z \rar{\subseteq} & \R \ar[rr, "\theta\mapsto e^{2\pi i \theta}"] && \T \rar & 1
\end{tikzcd} \]
\end{example}

\subsubsection{Equivalent group extensions}
\begin{definition}
Two extensions $G,G'$ of $Q$ by $N$ are \udef{equivalent} if there is a homomorphism $T:G\to G'$ making the following diagram commutative:
\[
\begin{tikzcd}
1 \ar[r] & N \ar[r, "\iota"] \ar[equal]{d} & G \ar[r, "\pi"] \ar[d,"T"] & Q \ar[r] \ar[d, equal] & 1 \\
1 \ar[r] & N \ar[r, "\iota"] & G' \ar[r, "\pi"] & Q \ar[r] & 1.
\end{tikzcd}
\]
\end{definition}
\begin{lemma}
If $G,G'$ are equivalent extensions, then they are isomorphic. So equivalence of extension is an equivalence relation.
\end{lemma}
\begin{proof}
The short five lemma (TODO).
\end{proof}
The converse is \emph{not} true! TODO: For instance, there are $8$ inequivalent extensions of the Klein four-group by $\mathbb{Z}/2\mathbb{Z}$, but there are, up to group isomorphism, only four groups of order $8$ containing a normal subgroup of order $2$ with quotient group isomorphic to the Klein four-group.

\subsubsection{Split exact sequences}
\url{https://kconrad.math.uconn.edu/blurbs/grouptheory/splittinggp.pdf}

\subsubsection{Double covers}
\begin{definition}
Let $G_1, G_2$ be groups. If $G_1$ is an extension of $G_2$ by $\Z_2$, i.e.\
\[\begin{tikzcd}
1 \rar & \Z_2 \rar & G_1 \rar & G_2 \rar & 1
\end{tikzcd} \]
is a short exact sequence, then we call $G_1$ a \udef{double cover} of $G_2$.
\end{definition}

\begin{example}
For all $n\in \N$, the dihedral group $D_{2n}$ is a double cover of $D_n$:
\[\begin{tikzcd}
1 \rar & \Z_2 \rar & D_{2n} \rar & D_n \rar & 1
\end{tikzcd} \]
See \ref{dihedralDoubleCover}.
\end{example}
Quaternionic group gives inequivalent extension.


\section{Group action}
\begin{definition}
An \udef{action} of a group $G$ on a set $X$ is a mapping
\[ \cdot: G\times X \to X: (g,x) \mapsto g\cdot x \]
satisfying
\begin{enumerate}
\item $e\cdot x = x$ for all $x\in X$ where $e$ is the neutral element of $G$;
\item $g(h\cdot x) = (gh)\cdot x$ for all $x\in X$ and $g,h\in G$.
\end{enumerate}
We will often just right $gx$ instead of $g\cdot x$.

A set $X$ with a given action of $G$ on it is called a \udef{$G$-set}.
\end{definition}
This definition can be reformulated using the curried form of $\pi$, namely
\[ \rho \defeq \operatorname{curry}(\pi): G \to (X\to X). \]
Then the rest of the definition of group action amounts to the statement that 
\[ \rho: G \to S(X) \quad \text{is a group homomorphism.} \]

We may then specify $G$-sets by the data $(X,\rho)$, where $X$ is a set and $\rho: G \to S(X)$ a group homomorphism.

TODO: opposite action.

\begin{definition}
Given two $G$-sets $X,Y$, a \udef{$G$-equivariant mapping} or \udef{intertwiner} is a map $f:X \to Y$ such that
\[ f(gx) = gf(x) \]
for all $g\in G$ and $x\in X$.
\end{definition}
We can express this by saying $f: (X,\rho_1) \to (Y,\rho_2)$ is a map between $G$-sets such that
\[ f\circ \rho_1(g) = \rho_2(g)\circ f \]
for all $g\in G$.

\begin{lemma}
The $G$-sets form a locally small category with $G$-equivariant maps as morphisms.
\end{lemma}


\subsection{Orbits and stabilisers}
\begin{definition}
Let $G$ be a group acting on a set $X$. We define the \udef{orbit} of $x\in X$ as the set
\[ Gx = G\cdot x \defeq \setbuilder{g\cdot x \in X}{g\in G}  \]
and the \udef{stabiliser} of $x\in X$ as the set
\[ G_x \defeq \setbuilder{g\in G}{g\cdot x = x}. \]
\end{definition}
\begin{proposition}[Orbit-stabiliser theorem]
Let $X$ be a $G$-set and $x\in X$. Then
\begin{enumerate}
\item $|Gx| = [G:G_x]$;
\item $|G| = |Gx|\cdot |G_x|$.
\end{enumerate}
\end{proposition}

\subsection{Actions of groups on themselves}
\subsubsection{Regular actions}
\begin{definition}
A group $G$ has a natural left action on the set $G$:
\[ G\times \operatorname{Field}(G) \to \operatorname{Field}(G): (g,h) \mapsto gh. \]
This action of $G$ is called the \udef{left-regular} group action.

Similarly, the natural right action on the set $G$ is called the \udef{right-regular} group action:
\[ \operatorname{Field}(G) \times G \to \operatorname{Field}(G): (h,g) \mapsto hg. \]
\end{definition}

\subsubsection{Conjugation}
Let $G$ be a group. The conjugation mapping
\[ \Ad: G\times \operatorname{Field}(G) \to \operatorname{Field}(G): (g,h) \mapsto \Ad_g(h) = g^{-1}hg \]
is a group action by $G$ on itself.

The orbits under this action are the conjugacy classes.

The stabiliser of $a\in G$ when $G$ is acting on itself by conjugation, is the \udef{centraliser} of $\{a\}$.





\section{Group action}
We have seen that symmetry transformations naturally form a group. Based on the concrete set of transformations that are symmetries we saw they form this abstract structure which we called a group. The advantage of working with this abstract entity is that it contains exactly the relevant details about the symmetry. We need not worry ourselves about the peculiarities of the particular system and we can easily make use of results others have obtained solving other problems.

Once we have thoroughly studied the symmetries of our system, we will want a way to move back from studying abstract groups to studying transformations of the system we are actually interested in.

Sometimes there is a natural correspondence between the set of group elements and the set of transformations. If this is the case the group can be interpreted as acting on the system in a \udef{canonical} (or natural) way.

\begin{example}
\begin{itemize}
\item Dihedral group $D_4$ acts quite naturally on a blanc, square piece of paper.
\item The symmetric group $\mathcal{S}_n$ of all permutations of a set of $n$ elements acts naturally on a set of $n$ elements.
\item The group of $n\times n$ matrices acts naturally on $n$-dimensional vectors through matrix multiplication.
\end{itemize}
\end{example}

In general the transition back may not be so clear, simple or natural. For instance there may be a subset of the $n$-dimensional vectors with a symmetry group isomorphic to $D_4$. To what transformations do these group elements correspond? We cannot just rotate and flip these vectors. It is for understanding these cases that the concept of a \udef{group action} is useful.

\subsection{Definition}
We start with a group $G$ and a set $X$. The set $X$ is frequently the set of configurations of the system and thus transformations of the system are functions of the type $f:\,X\to X$; to keep things general, we only assume we have set and we are agnostic as to its origins.
A group action quite simply associates a transformation of the set to every element of the group.

We do however require that this association has some fairly natural features, so that the nature and essence of the group is not lost in transition: the group action must respect the identity element and and group operation. This leads us to the following definition:
\begin{definition}
Let $G$ be a group and $X$ a set, then a \udef{(left) group action $\varphi$} of $G$ on $X$ is a function
\[ \varphi: \, G\times X \to X: \, (g,x)\mapsto \varphi(g,x) = g\cdot x \]
with the properties:
\begin{enumerate}
\item For the identity element $e$ and all $x\in X$: 
\[e\cdot x = x \]
\item For all $g,h \in G$ and $x\in X$:
\[ (gh)\cdot x = g\cdot (h\cdot x) \]
\end{enumerate}
Notice that we have introduced the notation $g\cdot x$ meaning apply the transformation attributed to $g$ through the group action to the element $x$.
\end{definition}

The above definition is for a \textit{left} group action. We can analogously define a right group action. The only difference between the two is that in the right group action in the transformation
\[ x \cdot (gh) = (x\cdot g)\cdot h \]
the transformation associated with $g$ gets applied first. Using the formula $(gh)^{-1} = h^{-1}g^{-1}$ we can always construct a left group action from a right one and vice versa, so typically we only consider left group actions.

An important property is immediately apparent from the definition:
\begin{eigenschap}
The transformation associated with $g$ (i.e.\ $x\mapsto g\cdot x$) is always a bijection because the inverse is given by $x \mapsto g^{-1}\cdot x$.
\end{eigenschap}

\subsection{Types of action}
What follows is simply an enumeration of some properties group actions may have. The action of $G$ on $X$ is called
\begin{enumerate}
\item \udef{transitive} if $X$ is non-empty and for each $x,y$ in $X$ there exists a $g \in G$ such that $g\cdot x = y$.
\item \udef{faithful} if for every distinct $g,h$ in $G$ there exists an $x \in X$ such that $g\cdot x \neq h\cdot x$. In other words the mapping of elements of $G$ to transformations of $X$ is 1-to-1 or injective.
\end{enumerate}

\subsection{Orbits and stabilizers}
\begin{definition}
Consider a group $G$ acting on a set $X$. The \udef{orbit} of an element $x$ of $X$ is denoted $G\cdot x$.
\[ G\cdot x = \{ g\cdot x | g\in G \}. \]
\end{definition}

The \udef{stabilizer subgroup} of $G$ with respect to an element $x$ of $X$ is the set of all elements in $G$ that fix $x$ and is denoted $G_x$.
\[G_x = \{ g\in G | g\cdot x = x \} \]

\subsection{Continuous group action}
A continuous group action on a topological space $X$ is a group action of a topological group $G$ that is continuous: i.e.\,
\[G \times X \to X : \;(g, x) \mapsto g \cdot x \]
is a continuous map.

This is the proper type of group action to use with topological groups, if their topologicalness is relevant and to be preserved.

\subsection{Representations}
If the group action is the action of a group on a vector space such that the transformations the group elements are mapped to are linear transformations, we call this group action a \udef{representation}.

\begin{definition}
A \udef{representation} of a group $G$ on an $n$-dim vector space $V$ is a mapping of the elements of $G$ to the set of invertible linear operations acting on $V$:
\[D: G \rightarrow GL(V): g \mapsto D(g)\]
Such that
\begin{itemize}
\item $D(e) = \mathbb{1}_V$
\item $D(g_1\cdot g_2) = D(g_1)D(g_2) = D(g_3)$
\end{itemize}
\end{definition}

\begin{example}
\begin{itemize}
\item Representations of $Z_3 = \{e,\omega, \omega^2\} \qquad (\omega = e^{i2/3\pi})$
\begin{itemize}
\item Trivial representations
\[D(e) = D(\omega) = D(\omega^2) = \mathbb{1}_V\]
\item Representation $\GL(1, \mathbb{C})$
\[ D(e) = 1, \quad D(\omega) = e^{i\frac{2}{3}\pi} , \qquad D(\omega^2) = e^{i\frac{1}{3}\pi} \]
\item Regular representation:
\[D(e) = 
\begin{pmatrix}
1&0&0\\0&1&0\\0&0&1
\end{pmatrix}, \qquad D(\omega) = \begin{pmatrix}
0&0&1\\1&0&0\\0&1&0
\end{pmatrix}, \qquad D(\omega^2) = \begin{pmatrix}
0&1&0\\0&0&1\\1&0&0
\end{pmatrix}
\]
In general a we can define a regular representation for any finite group $G$ as follows: Let $V$ be a vector space with basis $e_t$ indexed by the elements of $G$, $t \in G$. The mapping $D: e_t \mapsto e_{ts}$ defines the \udef{(left) regular representation} of $G$. This notion can be extended to groups of infinite order.
\end{itemize}
\item The standard representation of a subgroups $H$ of $\GL(n,\C)$ on the vector space $\C^n$ is given by the inclusion:
\[ D: H \to \GL(\C^n) = \GL(n,\C): h \mapsto h \]
\end{itemize}
\end{example}


\begin{definition}
Two representations are \udef{equivalent} if there exists a linear operator $S$ such that
\[D(g) \mapsto D'(g) = S^{-1}D(g)S\]
In other words there exists a similarity transformation $S$
\end{definition}

\begin{definition}
A representation is \udef{unitary} if $\forall g \in G$
\[D(g)D^\dagger(g) = D^\dagger(g)D(g) = \mathbb{1}_V\]
\end{definition}

\begin{definition}
Consider a \undline{representation $D$} of a \undline{group $G$} on a \undline{vector space $V$}
\begin{enumerate}
\item A subspace $W$ of $V$ is called \udef{invariant} if $D(g)w$ is in $W$ for all $w \in W$ and all $g \in G$. An invariant subspace $W$ is called nontrivial if $W\neq\{0\}$ and $W \neq V$.
\item We call $D$ \udef{reducible} if there exists a nontrivial subspace $U$ of $V$ that is invariant under $D$.
\item $D$ is \udef{irreducible} if the only subspaces invariant under all elements of the image of $D$ are $\emptyset$ and $V$
\item $D$ is \udef{completely reducible} if we can decompose $V$ into invariant subspaces:
\[V = U_1\oplus U_2 \oplus \ldots \oplus U_n\]
There then exists a similarity transformation such that
\[\forall g: D(g) = \begin{pmatrix}
D_1(g) & 0 & \dots & 0\\
0 & D_2(g)  & \dots & 0\\
\vdots & & \ddots & \vdots\\
0&0&\dots&D_n(g)
\end{pmatrix}\qquad \text{with}\quad D \equiv D_1\oplus D_2 \oplus \ldots \oplus D_n\]
\end{enumerate}
\end{definition}

\begin{example}
The regular representation of $Z_3$ is completely reducible. The linear operators $D(e), D(\omega)$ and $D(\omega^2)$ have eigenvalues $1,\omega, \omega^2$ with eigenvectors 
\[ \begin{pmatrix}
1\\1\\1
\end{pmatrix}\, \qquad \begin{pmatrix}
1\\ \omega^2 \\ \omega
\end{pmatrix} \qquad \text{and} \qquad \begin{pmatrix}
1 \\ \omega \\ \omega^2
\end{pmatrix}. \]
Each eigenvector generates an invariant subspace. We can then apply the following coordinate transformation
\[ S = \frac{1}{\sqrt{3}}\begin{pmatrix}
1&1&1\\
1&\omega^2 & \omega \\
1&\omega& \omega^2
\end{pmatrix} \]
in order to get the following matrices
\[D'(e) = \begin{pmatrix}
1&0&0\\
0&1&0\\
0&0&1
\end{pmatrix}, \qquad D'(\omega) = \begin{pmatrix}
1 & 0 & 0\\
0& \omega & 0 \\
0&0&\omega^2
\end{pmatrix}, \qquad D'(\omega^2) = \begin{pmatrix}
1 & 0 & 0\\
0& \omega^2 & 0 \\
0&0&\omega
\end{pmatrix}\]
\[ D' = D_1\oplus D_2 \oplus D_3 = \diag\{1,1,2\}\oplus \diag\{1,\omega, \omega^2\} \oplus \diag\{1, \omega^2, \omega\} \]
\end{example}

\subsubsection{Projective representations}
Bargmann theorem



\section{Topological groups}
A group is a set with an extra structure layered on top: the group operation that satisfies the group axioms. A topological space is also a set with an extra structure layered on top: the topology, as discussed in a previous part. Now here's a novel idea: let's layer both of these structures on a set at once. This gives no new mathematics because the two structures do not interact in any way; in order for interesting things to occur, we must pose some additional requirements.

\begin{definition}
A \udef{topological group} $G$ is a topological space that is also a a group such that the group operations of
\begin{enumerate}
\item product
\[ G\times G \to G: \, (x,y)\mapsto xy \]
\item and taking inverses
\[ G\to G: \, x\mapsto x^{-1} \]
\end{enumerate}
are \textbf{continuous}.
\end{definition}
TODO also need that points are closed?

\begin{lemma}
The continuity of the product and inverse is equivalent to the continuity of $G\times G \to G: (s,r)\mapsto sr^{-1}$.
\end{lemma}
TODO; reframe as criterion?

\begin{lemma}
Let $G$ be a topological group. The following are homeomorphisms:
\begin{enumerate}
\item $G\to G: s\mapsto s^{-1}$;
\item $G\to G: s\mapsto rs$ for any $r\in G$.
\end{enumerate}
\end{lemma}
An important consequence of this is that the topology of $G$ is determined by the topology near the identity $e$.

Topological groups are also sometimes called continuous groups.



\section{Grothendieck group}
Given a commutative monoid $M$, the Grothendieck group $G(M)$ is the ``most general'' Abelian group that arises from $M$. Intuitively it is formed by adding additive inverses for all elements of $M$.



 
TODO Grothendieck construction for Abelian monoids: $G(M)$.
Universality, functoriality

Cancellation property: simplified construction.

Grothendieck map $M\to G(M)$ is injective \textup{if and only if} $M$ has cancellation.

\subsection{The integers}
\begin{definition}
$\Z$
\end{definition}


\section{Ordered groups}

\begin{lemma}
Let $(G,+,\leq)$ be an ordered group and $x,y\in G$. Then
\[ (\forall \varepsilon > 0: x< y+\varepsilon) \implies x\leq y. \]
\end{lemma}
\begin{proof}
The proof is by contraposition. Assume $x>y$, then we can take $\varepsilon = x-y>0$. This implies $x = y+\varepsilon$ and so $x \geq y+\varepsilon$. 
\end{proof}


\chapter{Rings}
TODO: addition, multiplication and scalar multiplication of functions: pointwise.

TODO: unital homomorphisms; unital subalgebra.

\begin{lemma}
Let $R$ be a ring and $a\in R$.
\begin{enumerate}
\item If $a$ has a left and a right inverse, they are equal. Thus $a$ has an inverse.
\item The inverse of $a$ is unique, if it exists.
\end{enumerate}
\end{lemma}
\begin{proof}
Let $l$ be a left inverse of $a$ and $r$ a right inverse. Then
\[ l = l(ar) = (la)r = r. \]
The unicity of the inverse is an easy consequence.
\end{proof}

\begin{lemma} \label{lemma:productInvertibility}
Let $R$ be a ring and $a,b\in R$. Then $a$ and $b$ are invertible \textup{if and only if} $ab$ and $ba$ are invertible.
\end{lemma}
\begin{proof}
Assume $a,b$ invertible. Then $b^{-1}a^{-1}$ is an inverse for $ab$ and $a^{-1}b^{-1}$ is an inverse for $ba$.

Assume both $ab$ and $ba$ have inverses. Then from
\begin{align*}
a[b(ab)^{-1}] &= \vec{1} & [(ba)^{-1}b]a &= \vec{1} \\
[(ab)^{-1}a]b &= \vec{1} & b[a(ba)^{-1}] &= \vec{1}
\end{align*} 
we see that both $a$ and $b$ have left and right inverses.
\end{proof}

\begin{proposition} \label{prop:everyProperIdealInMaximalIdeal}
Every proper ideal is contained in a maximal ideal.
\end{proposition}

\begin{lemma} \label{lemma:nonInvertibleGeneratedIdeals}
Let $R$ be a unital ring. If $a\in R$ is non-invertible, then the generated ideal $(a)$ is not the whole ring.
\end{lemma}
\begin{proof}
If $(a) = R$, then $1\in (a)$, implying $ab=1$ for some $b\in R$. A contradiction.
\end{proof}

\begin{proposition}
Let $f$ be a ring homomorphism. If $f$ is invertible as a function (i.e. bijective), its inverse $f^{-1}$ is also a ring homomorphism.
\end{proposition}

\begin{proposition} \label{prop:kernelIsIdeal}
Kernel of Ring Homomorphism is Ideal
\end{proposition}

\begin{definition}
A $*$-rng is a structured set $(R,+,\cdot, *)$, where $R$ is a rng and $*:R\to R$ is an involutive anti-automorphism. That is, $\forall x,y\in R$:
\begin{itemize}
\item $(xy)^* = y^*x^*$;
\item $(x+y)^* = x^* + y^*$;
\item $(x^*)^* = x$.
\end{itemize}
This is also known as an \udef{involutive rng} or \udef{rng with involution}.
\end{definition}
\begin{lemma}
If $R$ is a unital ring with involution, then $1^* = 1$.
\end{lemma}
\begin{proof}
From $1^*x = (x^*1)^* = (x^*)^* = x$, we see that $1^*$ is a multiplicative identity, which is unique.
\end{proof}
\begin{definition}
An element of a $*$-rng is \udef{self-adjoint} if $x^* = x$.
\end{definition}

\section{Group rings}
\begin{definition}
Let $G$ be a finite group and $R$ a r(i)ng. The \udef{group ring} $RG$ is the set of functions $(G\to R)$ with pointwise addition and the convolution product
\[ (x\star y)(g) = \sum_h x(h)y(h^{-1}g) = \sum_{g=hk}x(h)y(k) \]
for all $x,y\in RG$ and $g\in G$. 
\end{definition}
The a group ring can be seen as a free module generated by $G$. (TODO: this as definition?)



\chapter{Fields}
\section{Totally ordered fields}
\begin{definition}
Let $K$ be a set with binary operations $+,\cdot$ such that $(K,+,\cdot)$ is a field and a binary relation $\leq$ such that $(F,\leq)$ is a total order. Then the structured set $(K,+,\cdot,\leq)$ is a \udef{totally ordered field} if $\forall a,x,y\in K$:
\begin{enumerate}
\item $x\leq y \implies x+a \leq y+a$;
\item $x\leq y \land a\geq 0 \implies ax \leq ay$;
\item $x\leq y \land a\leq 0 \implies ax \geq ay$.
\end{enumerate}
\end{definition}

\begin{lemma}
Let $(K,+,\cdot,\leq)$ be a totally ordered field and $a,b,c,d\in K$. Then
\begin{enumerate}
\item $a\leq b \land c\leq d \implies (a+c) \leq (b+d)$;
\item $a\leq b \implies -b\leq -a$;
\item $a\geq 0 \iff -a\leq 0$;
\item $a\geq 0 \land b\geq 0 \implies ab \geq 0$;
\item $a\geq 0 \implies a^n \geq 0$ for all $n\in\N$;
\item $a\leq 0 \implies (a^{2n} \geq 0 \land a^{2n+1}\leq 0)$ for all $n\in\N$;
\item $a > 0 \iff a^{-1} > 0$ and $a < 0 \iff a^{-1} < 0$;
\item if $b \geq a>0$, then $b^{-1} \leq a^{-1}$;
\item $0<1$.
\end{enumerate}
\end{lemma}
\begin{proof}
(1) By applying point 1. of the definition twice, we get $(a+c)\leq(b+c)\leq(b+d)$.

(2) This is the result of multiplying by $-1$.

(3) Idem, using $-0=0$.

(4) Special case of point 2. of the definition.

(5) By induction on $n$ and point 2. of the definition.

(6) By induction on $n$ and point 3. of the definition.

(7) By multiplying $a \geq 0$ with $(a^{-1})^{2}$, which is positive, we get $a^{-1}\geq 0$. Also $a\neq 0 \iff a^{-1}\neq 0$.

(8) By point 7. and point 4. of the lemma $a^{-1}b^{-1}$ is positive, so multiplying $b\geq a$ by $a^{-1}b^{-1}$ yields $b^{-1} \leq a^{-1}$.

(9) Assume, towards a contradiction, that this is false, so $1\leq 0$. Then $1$ is negative and multiplying the inequality with $1$ yields $1\cdot 1 \geq 1\cdot 0$, or $1\geq 0$. Taking both inequalities gives $1=0$ by anti-symmetry, which is prohibited for fields.
\end{proof}

\begin{definition}
Let $F\subset K$ be totally ordered fields. Then we call $F$ \udef{dense} in $K$ if
\[ \forall a,b\in K: \exists x\in F: a<x<b. \]
\end{definition}
TODO: topology definition?

\chapter{Valuation theory}
Absolute values on integral domains.