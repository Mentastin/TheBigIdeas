\chapter{Magmas}
\begin{definition}
A \udef{magma} or \udef{groupoid} is an algebra whose signature contains a single ginary operator.

Let $A$ be the carrier of the algebra and $\boldsymbol{\cdot}$ the interpretation of the operator. We denote the magma as $\sSet{A,\boldsymbol{\cdot}}$. We call $\sSet{A,\boldsymbol{\cdot}}$
\begin{itemize}
\item a \udef{semigroup} if $\boldsymbol{\cdot}$ is associative;
\item a \udef{monoid} if $\boldsymbol{\cdot}$ is associative and has an identity;
\item a \udef{group} if $\boldsymbol{\cdot}$ is associative, has an identity and every $a\in A$ has an inverse.
\end{itemize}
We call a magma \udef{commutative} if its operation is commutative.

If $\sSet{A,\boldsymbol{\cdot}}$ is a monoid or group with identity $e$, we denote the monoid/group as $\sSet{A,\boldsymbol{\cdot}, e}$.
\end{definition}
TODO: change of signature for monoid!

Let $x,y\in A$. We usually write $x\boldsymbol{\cdot} y$ instead of $\boldsymbol{\cdot}(x,y)$. (TODO ref infix notation)

\begin{definition}
We write $x^n$ to abbreviate $\underbrace{x\cdot x\cdot \ldots \cdot x}_{\text{$n$ times}}$.

If $x^2 = x$, we say $x$ is an \udef{idempotent}.
\end{definition}

\begin{lemma}
Let $\sSet{M,\cdot}$ be a magma and $a\in M$. Then $a\cdot M \subseteq M$.
\end{lemma}

\begin{proposition} \label{leftRightInverseMonoid}
Let $(M,\cdot,e)$ be a monoid and $a\in M$. If $a$ has both a left inverse $l$ and a right inverse $r$, then $l=r$.
\end{proposition}
\begin{proof}
We calculate
\[ l = l\cdot e = l\cdot(a\cdot r) = (l\cdot a)\cdot r= e\cdot r = r. \]
\end{proof}

\section{Semigroups}
\begin{lemma}
Let $S$ be a semigroup. If $a\in S$ is idempotent, then $\{a\}$ is a subsemigroup.
\end{lemma}
\begin{corollary}
Let $S$ be a semigroup.
\begin{enumerate}
\item If $S$ contains an identity $e$, then $\{e\}$ is a subsemigroup.
\item If $S$ contains an absorbing element $u$, then $\{u\}$ is a subsemigroup.
\end{enumerate}
\end{corollary}

\begin{proposition} \label{groupCriterion}
Let $S$ be a semigroup. Then $S$ is a group \textup{if and only if} for all $a\in S$
\[ aS = S = Sa. \]
\end{proposition}
\begin{proof}
If $S$ is a group, then we have for all $a\in S$
\[ S \supseteq aS \supseteq aa^{-1}S = S. \]
Similarly we also have $S = Sa$.

For the converse: from $aS = S$, we het that there exists an $x\in S$ such that $ax = a$. We claim $x$ is a right-identity for $S$. Indeed, take arbitrary $b\in S$. Then $b = ya$ for some $y$ and so $bx = yax = ya b$. In the same way we can also find a left-identity. So $S$ contains an identity $e$ by \ref{leftRightIdentity}.

Then for all $a$ we can find $u,v\in S$ such that $au = e = va$. This means $a$ has an inverse by \ref{leftRightInverseMonoid}.
\end{proof}

\subsection{Adjoining identities and absorbing elements}
\subsubsection{Adjoining identity}
\begin{definition}
Let $\sSet{S,\boldsymbol{\cdot}}$ be a semigroup. We define
\[ \widetilde{S} \defeq \begin{cases}
S & \text{if $S$ has an identity} \\
S\uplus \{e\} & \text{if $S$ has no identity.}
\end{cases} \]
and also
\[ \widetilde{\boldsymbol{\cdot}} \quad\defeq\quad \widetilde{S}\times \widetilde{S}\to \widetilde{S}: (a,b) \mapsto \begin{cases}
a\boldsymbol{\cdot} b & a,b\in S \\
b & a = e \\
a & b = e.
\end{cases} \]
\end{definition}


\subsubsection{Adjoining an absorbing element}
\begin{definition}
Let $\sSet{S,\cdot}$ be a semigroup. We define
\[ \widehat{S} \defeq \begin{cases}
S & \text{if $S$ has an absorbing element} \\
S\uplus \{u\} & \text{if $S$ has no absorbing element.}
\end{cases} \]
and also
\[ \widehat{\boldsymbol{\cdot}} \quad\defeq\quad \widehat{S}\times \widehat{S}\to \widehat{S}: (a,b) \mapsto \begin{cases}
a\boldsymbol{\cdot} b & a,b\in S \\
u & (a = u \lor b = u.
\end{cases} \]
\end{definition}

TODO: link with partial semigroup.

\subsection{Subsets and subsemigroups}
\subsubsection{Ideals}
\begin{definition}
Let $S$ be a semigroup and $A\subseteq S$ a non-empty subset. Then $A$ is called
\begin{itemize}
\item a \udef{left ideal} if $SA \subseteq A$;
\item a \udef{right ideal} if $AS \subseteq A$;
\item a \udef{(two-sided) ideal} if $A$ is both a left and a right ideal.
\end{itemize}
Clearly $S$ is an ideal. If $u\in S$ is an absorbing element, then $\{u\}$ is an ideal. If an ideal is neither of these two, it is called \udef{proper}.
\end{definition}

\begin{lemma}
Let $\sSet{S,\cdot}$ be a semigroup and $a\in S$. Then
\begin{enumerate}
\item $\widetilde{S}a = Sa \cup \{a\}$ is a left ideal;
\item $a\widetilde{S} = aS \cup \{a\}$ is a right ideal;
\item $\widetilde{S}a\widetilde{S} = SaS \cup Sa \cup aS \cup \{a\}$ is an ideal.
\end{enumerate}
In particular $\widetilde{S}a \subseteq S$, $a\widetilde{S} \subseteq S$ and $\widetilde{S}a\widetilde{S} \subseteq S$.
\end{lemma}

\begin{definition}
We call
\begin{itemize}
\item $\widetilde{S}a$ the principal left ideal generated by $a$; 
\item $a\widetilde{S}$ the principal right ideal generated by $a$; 
\item $\widetilde{S}a\widetilde{S}$ the principal ideal generated by $a$.
\end{itemize}
\end{definition}

\subsubsection{Generated semigroups}
TODO!

\subsubsection{Periodic semigroups}
Period and index.

\subsection{Homomorphism}

\begin{proposition}
Let $S$ be a semigroup. Then the functions
\begin{align*}
&\lambda: S \to (\widetilde{S} \to \widetilde{S}) : a\mapsto (\lambda_a: x\mapsto ax) \\
&\rho: S \to (\widetilde{S} \to \widetilde{S}): a\mapsto (\rho_a: x\mapsto xa)
\end{align*}
are injective homomorphisms.
\end{proposition}
Note that for all $s\in S$ we view $\lambda_a$ as a function $\widetilde{S} \to \widetilde{S}$. This is necessary for injectivity.
\begin{proof}
The functions $\lambda, \rho$ are homomorphisms by associativity.

For injectivity, let $\lambda_a = \lambda_b$. Then $a = a\cdot e = \lambda_a(e) = \lambda_b(e) b\cdot e = b$.
\end{proof}

\begin{definition}
We call $\lambda$ ($\rho$) the \udef{extended left (right) regular representation} of $S$.
\end{definition}

\subsubsection{Congruences}
\begin{definition}
Let $\sSet{S,\boldsymbol{\cdot}}$ be a semigroup and $R$ a relation on $S$. Then $R$ is called
\begin{itemize}
\item \udef{left compatible} if it is $\Omega$-compatible with $\Omega = \setbuilder{\lambda_a}{a\in S}$;
\item \udef{right compatible} if it is $\Omega$-compatible with $\Omega = \setbuilder{\rho_a}{a\in S}$;
\item \udef{compatible} if it is $\Omega$-compatible with $\Omega = \{\boldsymbol{\cdot}\}$.
\end{itemize}
If the relation $R$ is additionally an equivalence relation, then $R$ is called a \udef{( left / right) congruence}.
\end{definition}

In other words:
\begin{itemize}
\item $R$ is left compatible if $\forall x,y,a\in S: \; xRy \implies (ax)R(ay)$;
\item $R$ is right compatible if $\forall x,y,a\in S: \; xRy \implies (xa)R(ya)$;
\item $R$ is compatible if $\forall x,y,v,w\in S: \; xRy \land vRw \implies (xv)R(yw)$.
\end{itemize}

\begin{proposition}
A relation $R$ on a semigroup $S$ is a congruence \textup{if and only if} it is a left and a right congruence.
\end{proposition}
\begin{proof}
($\Rightarrow$) If $R$ is a congruence, then $aRa$ by reflexivity. So $xRy$ implies $(ax)R(ay)$ and $(xa)R(ya)$, meaning that $R$ is a left and a right congruence.

($\Leftarrow$) Assume $R$ a left and a right congruence. Take $x,y,v,w$ such that $xRy$ and $vRw$. Then $(xv)R(yv)$ and $(yv)R(yw)$ by left and right compatibility. We conclude $(xv)R(yw)$ by transitivity.
\end{proof}

\subsection{Green's relations}
\begin{definition}
Let $\sSet{S,\cdot}$ be a semigroup. Let
\begin{itemize}
\item $\mathcal{L}$ be a relation on $S$ defined by $a\mathcal{L}b \defequiv \widetilde{S}a = \widetilde{S}b$;
\item $\mathcal{R}$ be a relation on $S$ defined by $a\mathcal{R}b \defequiv a\widetilde{S} = b\widetilde{S}$;
\item $\mathcal{H} \defeq \mathcal{L}\cap \mathcal{R}$;
\item $\mathcal{D} \defeq \mathcal{L}\vee \mathcal{R}$; TODO: which lattice?
\item $\mathcal{J}$ be a relation on $S$ defined by $a\mathcal{J}b \defequiv \widetilde{S}a\widetilde{S} = \widetilde{S}b\widetilde{S}$.
\end{itemize}
These five relations are known as \udef{Green's relations}.
\end{definition}

Clearly we have $\mathcal{D} \subseteq \mathcal{J}$.

\begin{lemma}
Let $\sSet{S,\cdot}$ be a semigroup and $a,b\in S$. Then
\begin{enumerate}
\item $a\mathcal{L}b$ \textup{if and only if} $\exists x,y\in \widetilde{S}: (xa = b) \land (yb = a)$;
\item $a\mathcal{R}b$ \textup{if and only if} $\exists x,y\in \widetilde{S}: (ax = b) \land (by = a)$;
\item $a\mathcal{J}b$ \textup{if and only if} $\exists x,y, u,v\in \widetilde{S}: (xay = b) \land (ubv = a)$.
\end{enumerate}
\end{lemma}
\begin{proof}
We prove (1), (2) is analogous.

($\Rightarrow$) $a\in \widetilde{S}a = \widetilde{S}b$, so there exists $y\in \widetilde{S}$ such that $a = yb$. The other equation is similar.

($\Leftarrow$) Because $\widetilde{S}x \subseteq \widetilde{S}$, we have $\widetilde{S}a \subseteq \widetilde{S}xa = \widetilde{S}b$. Similarly $\widetilde{S}b \subseteq \widetilde{S}a$.
\end{proof}
\begin{corollary}
$\mathcal{L}$ is a \emph{right} congruence and $\mathcal{R}$ a \emph{left} congruence.
\end{corollary}

\begin{proposition}
The relations $\mathcal{L}$ and $\mathcal{R}$ commute.
\end{proposition}
\begin{proof}
Assume $a(\mathcal{L};\mathcal{R})b$, meaning $\exists c: a\mathcal{L}c$ and $c\mathcal{R}b$. Then there exist $x,y,u,v\in \widetilde{S}$ such that
\[ (xa = c) \land (yc = a) \;\land\; (cu = b) \land (bv = c). \]
Now define $d = ycu$. Then $a\mathcal{R}d$ because
\[ d = (yc)u = au \;\text{and}\; a = yc = ybv = ycuv = dv \]
and $d\mathcal{L}b$ because
\[ d = ycu = yb \;\text{and}\; b = cu = xau = xycu = xd. \]
So $a(\mathcal{R};\mathcal{L})b$. The other inclusion is similar.
\end{proof}
\begin{corollary}
$\mathcal{D} = \mathcal{L};\mathcal{R} = \mathcal{R};\mathcal{L}$.
\end{corollary}
In other words, $a\mathcal{D}b \iff a\mathcal{L} \mesh \mathcal{R}b \iff a\mathcal{R} \mesh \mathcal{L}b$.

\begin{proposition}
Let $S$ be a periodic semigroup. Then $\mathcal{D} = \mathcal{J}$.
\end{proposition}
\begin{proof}
The inclusion $\mathcal{D} \subseteq \mathcal{J}$ is generally true. We want to prove the other inclusion.

Suppose $a\mathcal{J}b$. Then we can find $x,y,u,v\in \widetilde{S}$ such that
\[ xay = b, \quad ubv = a. \]
We see that $a = (ux)a(yu) = (ux)^2a(yu)^2 = \ldots$. By periodicity (TODO ref) we can find an $m\in \N$ such that $(ux)^m$ is idempotent. Then set $c= xa$, so that
\[ a = (ux)^ma(yu)^m = (ux)^m(ux)^ma(yu)^m = (ux)^ma = (ux)^{m-1}uc, \]
which means that $a\mathcal{L}c$.

Similarly we can choose $n\in \N$ such that $(vy)^n$ is idempotent, so from $b = (xu)b(vy) = (xu)^2b(vy)^2 = \ldots$
we get
\begin{align*}
c &= xa = x(ux)^{n+1}a(yu)^{n+1} = (xu)^{n+1}xay(vy)^nv \\
&= (xu)^{n+1}b(vy)^{2n}v = (xu)^{n+1}b(vy)^{n+1}(vy)^{n-1}v \\
&= b(vy)^{n-1}v.
\end{align*}
This along with $cy = xay = b$, gives $c\mathcal{R}b$. Thus $a\mathcal{L};\mathcal{R}b$, i.e. $a\mathcal{D}b$.
\end{proof}
\begin{corollary}
If the semigroup is finite, then $\mathcal{D} = \mathcal{J}$.
\end{corollary}

\begin{proposition}
Let $S$ be a semigroup. If $\sSet{S/\mathcal{L}, \subseteq}$ and $\sSet{S/\mathcal{R}, \subseteq}$ are well-founded, then $\mathcal{D} = \mathcal{J}$.
\end{proposition}
\begin{proof}
TODO
\end{proof}

\subsubsection{Egg-box diagrams}
\begin{definition}
In an \udef{egg-box diagram} a semigroup $S$ is depicted as a grid. Each element is put in this grid such that
\begin{itemize}
\item the rows are $\mathcal{R}$-classes; and
\item the columns are $\mathcal{L}$-classes.
\end{itemize} 
\end{definition}

Thus for $a,b\in S$, $[a]_\mathcal{L} = [b]_\mathcal{L}$ if and only if $a$ and $b$ are in the same column. Similarly, $[a]_\mathcal{R} = [b]_\mathcal{R}$ if and only if $a$ and $b$ are in the same row.

\begin{lemma}
Let $S$ be a semigroup and $a,b\in S$. Then
\begin{enumerate}
\item the cells in the egg-box diagram are $\mathcal{H}$-equivalence classes;
\item $a\mathcal{D}b$ \textup{if and only if} there is an element in the intersection of the row of $a$ and the column of $b$ or vice versa.
\end{enumerate}
\end{lemma}

\begin{example}
Consider the semigroup $(\{1,2,3\} \to \{1,2,3\})$ with composition $;$. We can represent an element $f$ of this semigroup as $(f(1) f(2) f(3))$. We have
\begin{itemize}
\item $f\mathcal{L}g$ if $f$ and $g$ have the same image;
\item $f\mathcal{R}g$ if $f$ and $g$ have the same kernel.
\end{itemize}
An egg-box diagram can be drawn as follows:
\[ \begin{array}{|c|c|c|c|c|c|c|}
\hline
\mathbf{(1 1 1)} & \mathbf{(2 2 2)} & \mathbf{(3 3 3)} &&& &  \\ \hline
&& & \mathbf{(1 2 2)}, & \mathbf{(1 3 3)}, & (2 3 3), &  \\
&& & (2 1 1)  & (3 1 1)  & (3 2 2)  &  \\ \hline
&& & (2 1 2), & (3 1 3), & \mathbf{(3 2 3)},  &  \\
&& & \mathbf{(1 2 1)}  & (1 3 1)  & (2 3 2)  &  \\ \hline
&& & (2 2 1), & (3 3 1), & (3 3 2),  &  \\
&& & (1 1 2)  & \mathbf{(1 1 3)}  & \mathbf{(2 2 3)}  &  \\ \hline
&& &&& & \mathbf{(1 2 3)}, (2 3 1) (3 1 2)  \\
&& &&& & (1 3 2), (2 1 3) (3 2 1)  \\ \hline
\end{array} \]
The bold elements are idempotents.
\end{example}


\begin{proposition}[Green's lemma] \label{GreensLemma}
Let $S$ be a semigroup and $a,b,x \in S$.
\begin{enumerate}
\item If $ax = b$ and $a\mathcal{R}b$, then
\begin{enumerate}
\item $\rho_x|_{[a]_\mathcal{L}}: [a]_\mathcal{L} \to [b]_\mathcal{L}$ is a bijection;
\item if $by = a$ for some $y\in \widetilde{S}$, then $\rho_y|_{[b]_\mathcal{L}}$ is the inverse;
\item $\rho_x|_{[a]_\mathcal{L}}$ preserves $\mathcal{R}$-classes.
\end{enumerate}
\item If $xa = b$ and $a\mathcal{L}b$, then
\begin{enumerate}
\item $\lambda_x|_{[a]_\mathcal{R}}: [a]_\mathcal{R} \to [b]_\mathcal{R}$ is a bijection with inverse $\lambda_y|_{[b]_\mathcal{R}}$;
\item if $yb = a$ for some $y\in \widetilde{S}$, then $\lambda_y|_{[b]_\mathcal{R}}$ is the inverse;
\item $\lambda_x|_{[a]_\mathcal{R}}$ preserves $\mathcal{L}$-classes.
\end{enumerate} 
\end{enumerate}
In particular
\begin{enumerate}
\item if $a\mathcal{R}b$, then there exists $x\in \widetilde{S}$ such that $ax = b$ and thus $|[a]_\mathcal{L}| = |[b]_\mathcal{L}|$;
\item if $a\mathcal{L}b$, then there exists $x\in \widetilde{S}$ such that $xa = b$ and thus $|[a]_\mathcal{R}| = |[b]_\mathcal{R}|$.
\end{enumerate}
\end{proposition}
\begin{proof}
The function $\rho_x$ is well-defined: let $u\in [a]_\mathcal{L}$ so that $u = va$ for some $v\in \widetilde{S}$. Then $\rho_x(u) = vax = vb \in [b]_\mathcal{L}$.

Because $a\mathcal{R}b$, we can always find a $y\in \widetilde{S}$ such that $by = a$. The function $\rho_y|_{[b]_\mathcal{L}}$ is clearly the inverse: $\rho_y(\rho_x(u)) = \rho_y(vb) = vby = va = u$. This shows that $\rho_x$ is bijective.

It is clear that $\rho_x$ preserves $\mathcal{R}$-classes, because it operates on the right.

The arguments for $\lambda_x$ and $\lambda_y$ are completely analogous.
\end{proof}
\begin{corollary}
Let $S$ be a semigroup and $a,b\in S$ such that $a\mathcal{D}b$, then $|[a]_\mathcal{H}| = |[b]_\mathcal{H}|$.
\end{corollary}
\begin{proof}
If $a(\mathcal{L};\mathcal{R})b$, then there exists a $c\in S$ such that $c = sa$ and $b = ct$. Then $\lambda_s\circ \rho_t$ is the required bijection.
\end{proof}

\begin{theorem}[Green's theorem]
Let $S$ be a semigroup and $H$ an $\mathcal{H}$-class in $S$. Then either
\begin{enumerate}
\item $H^2\perp H$; or
\item $H^2 = H$ and $H$ is a subgroup of $S$.
\end{enumerate}
\end{theorem}
\begin{proof}
Suppose $H^2\cap H \neq \emptyset$, then there exist $a,b\in H$ such that $ab = c\in H$. By the Green's lemma \ref{GreensLemma} we have that $\rho_b:H\to H$ and $\lambda_a: H\to H$ are bijections.

Then for all $h\in H$, $\rho_b(h) = hb \in H$. Again by the Green's lemma, this means that $\lambda_h: H\to H$ is a bijection. Similarly $\rho_h: H\to H$ is a bijection for all $h$. So for all $h\in H$ we have $hH = H = Hh$. This means $H^2 = H$ and $H$ is a group by \ref{groupCriterion}.
\end{proof}
\begin{corollary} \label{GreensTheoremCorollary}
Let $S$ be a semigroup and $H$ an $\mathcal{H}$-class in $S$. Then
\begin{enumerate}
\item if $x$ is an idempotent in $H$, then $H$ is a subgroup of $S$;
\item no $\mathcal{H}$-class can contain more than one idempotent. 
\end{enumerate}
\end{corollary}

\subsubsection{Regular elements $\mathcal{D}$-classes}
\begin{definition}
Let $S$ be a semigroup. An element $a\in S$ is called \udef{regular} if there exists $x\in S$ such that $axa = a$.
\end{definition}

\begin{proposition} 
Let $S$ be a semigroup. If $a\in S$ is regular, then every element in $[a]_{\mathcal{D}}$ is regular.
\end{proposition}
So it makes sense to call a $\mathcal{D}$-class \udef{regular} if it consists of regular elements and \udef{irregular} otherwise.
\begin{corollary}
If there is an idempotent $x\in [a]_{\mathcal{D}}$, then $[a]_{\mathcal{D}}$ is regular.
\end{corollary}
\begin{proof}
An idempotent is regular: $x = xx = x(xx) = xxx$.
\end{proof}

\begin{proposition} \label{idempotentsInGreensClasses}
Let $S$ be a semigroup.
\begin{enumerate}
\item If $x\in S$ is an idempotent, then $x$ is a left identity for $[x]_{\mathcal{R}}$ and $x$ is a right identity for $[x]_{\mathcal{L}}$.
\item If $a$ is regular with $axa = a$, then $xa$ and $ax$ are idempotents.
\item If $a$ is regular with $axa = a$, then $xa\mathcal{L}a$ and $ax\mathcal{R}a$.
\item In a regular $\mathcal{D}$-class each $\mathcal{L}$-class and each $\mathcal{R}$-class contains at least one idempotent.
\end{enumerate}
\end{proposition}
\begin{proof}
(1) Let $a\in [x]_\mathcal{R}$. Then there exists a $y\in \widetilde{S}$ such that $a = xy$ and thus $xa = xxy = xy = a$. The claim of right-identity is similar.

(2) We have $(xa)(xa) = x(axa) = xa$ and $(ax)(ax) = (axa)x = ax$.

(3) If $a$ is regular with $axa = a$, then $xa$ is idempotent: $(xa)(xa) = x(axa) = xa$ and $xa \in [a]_{\mathcal{L}}$:
\[ xa = (xa) \qquad a(xa) = a \]
Similarly $ax \in [a]_{\mathcal{R}}$ and is idempotent.

(4) Let $[a]_\mathcal{L}$ be an $\mathcal{L}$-class in a regular $\mathcal{D}$-class. By regularity there exists an $x\in S$ such that $axa = a$. From (2) and (3), we have that $[a]_\mathcal{L}$ contains the idempotent $xa$ and $[a]_\mathcal{R}$ the idempotent $ax$.
\end{proof}

\begin{proposition}
Let $a,b \in S$ such that $a\mathcal{D}b$. Then $ab\in [a]_{\mathcal{R}}\cap [b]_{\mathcal{L}}$ \textup{if and only if} $[b]_{\mathcal{R}}\cap [a]_{\mathcal{L}}$ contains an idempotent.
\end{proposition}
\begin{proof}
Suppose $[b]_{\mathcal{R}}\cap [a]_{\mathcal{L}}$ contains an idempotent $x$. Then $xb = b$ by \ref{idempotentsInGreensClasses}. Then $\rho_b$ maps $[x]_{\mathcal{L}} = [a]_{\mathcal{L}}$ to $[b]_{\mathcal{L}}$ by \ref{GreensLemma}. It preserves $\mathcal{R}$-classes, so in particular it maps $[a]_{\mathcal{H}}$ to $[a]_{\mathcal{R}}\cap [b]_{\mathcal{L}}$. Thus $ab = \rho_b(a) \in [a]_{\mathcal{R}}\cap [b]_{\mathcal{L}}$.

Now assume $ab\in [a]_{\mathcal{R}}\cap [b]_{\mathcal{L}}$. Then there exists a $c\in \widetilde{S}$ such that $abc = a$. By \ref{GreensLemma} this means $\rho_c\in ([ab]_{\mathcal{L}}\to [a]_{\mathcal{L}}) = ([b]_{\mathcal{L}}\to [a]_{\mathcal{L}})$. In particular this restricts to $([b]_{\mathcal{H}}\to [b]_{\mathcal{R}}\cap[a]_{\mathcal{L}})$. Because $ab = (ab)$, $\rho_b$ is the inverse of $\rho_c$ by the same lemma.

So $\rho_c(b) = bc \in [b]_{\mathcal{R}}\cap[a]_{\mathcal{L}}$ and bc is idempotent because
\[ bcbc = (\rho_c\circ\rho_b\circ\rho_c) (b) = \rho_c(b) = bc. \]
\end{proof}

\subsubsection{Generalised inverses}
\begin{definition}
Let $S$ be a semigroup and $a\in S$. We call $a'$ a \udef{(generalised) inverse} of $a$ if
\[ aa'a = a, \qquad a'aa' = a'. \]
\end{definition}

\begin{lemma}
Let $S$ be a semigroup and $a\in S$. Then $a$ has a generalised inverse \textup{if and only if} it is regular.
\end{lemma}
\begin{proof}
Clearly every element with a generalised inverse is regular. Conversely, assume $a$ regular with $axa = a$. Then $a' = xax$ is a generalised inverse of $a$.
\end{proof}

\begin{proposition} \label{inversesIdempotentsDclass}
Let $a$ be an element of a regular $\mathcal{D}$-class $D$ in a semigroup $S$.
\begin{enumerate}
\item If $a'$ is a generalised inverse of $a$, then
\begin{enumerate}
\item $a' \in D$;
\item $[aa']_\mathcal{H} = [a]_\mathcal{R}\cap [a']_\mathcal{L}$;
\item $[a'a]_\mathcal{H} = [a']_\mathcal{R}\cap [a]_\mathcal{L}$.
\end{enumerate}
\item If $b\in S$ is such that $[a]_\mathcal{R}\cap [b]_\mathcal{L}$ contains an idempotent $x$ and $[b]_\mathcal{R}\cap [a]_\mathcal{L}$ contains an idempotent $y$, then $[b]_\mathcal{H}$ contains a generalised inverse $a'$ of $a$ such that $aa' = x$ and $a'a = y$.
\item No $\mathcal{H}$-class contains more than one generalised inverse of $a$.
\end{enumerate}
\end{proposition}
\begin{proof}
(1) Firstly notice that $[a]_\mathcal{R}\cap [a']_\mathcal{L}$ is an $\mathcal{H}$-class. Also $a\mathcal{R}aa'$ and $a'\mathcal{L}aa'$ by \ref{idempotentsInGreensClasses}.

(2) From $a\mathcal{R}x$, we deduce that there exists a $u\in\widetilde{S}$ such that $au = x$. 

Set $a' = yux$. Then, using that $x$ is a left-identity on $[a]_\mathcal{R}$ and $y$ a right-identity on $[a]_\mathcal{L}$ by \ref{idempotentsInGreensClasses},
\begin{align*}
aa'a &= a(yux)a = (ay)u(xa) = aua = (au)a = xa = a \\
a'aa' &= (yux)a(yux) = (yu)(xay)(ux) = (yu)a(ux) = (yu)(au)x = yux^2 = yux = a' \\
aa' &= a(yux) = ((ay)u)x = (au)x = x^2 = x.
\end{align*}
For the last equality, we need that $a\mathcal{L}y$ implies the existence of $v\in \widetilde{S}$ such that $va = y$. Then
\[ a'a = (yux)a = (va)u(xa) = (va)(ua) = v(au)a = v(xa) = va = y. \]
We now just need to show that $a' \in [b]_\mathcal{H}$. By \ref{idempotentsInGreensClasses} we have $a'\in [aa']_{\mathcal{L}}$ and $a'\in [a'a]_{\mathcal{R}}$, so
\[ a' \in [aa']_{\mathcal{L}} \cap [a'a]_{\mathcal{R}} = [x]_{\mathcal{L}} \cap [y]_{\mathcal{R}} = [b]_{\mathcal{L}} \cap [b]_{\mathcal{R}} = [b]_\mathcal{H}. \]

(3) Suppose $[b]_\mathcal{H}$ is an $\mathcal{H}$-class containing two generalised inverses $a_1', a_2'$ of $a$. Then $aa'_1 = aa'_2$ and $a'_1a = a'_2a$ by \ref{GreensTheoremCorollary} and thus
\[ a'_1 = a'_1(aa'_1) = a'_1(aa'_2) = (a'_2a)a'_2 = a'_2. \]
\end{proof}
\begin{corollary} \label{corollaryInversesIdempotentsDclass}
Let $S$ be a semigroup and $x,y\in S$ idempotents. Then $x\mathcal{D}y$ \textup{if and only if} we can write $x = aa'$ and $y = a'a$ for some $a\in [x]_\mathcal{R}\cap [y]_\mathcal{L}$ and $a'\in [y]_\mathcal{R}\cap [x]_\mathcal{L}$ that are generalised inverses of each other.
\end{corollary}
\begin{proof}
We have $x\mathcal{D}y$ iff there exist $a,b\in S$ such that $x\mathcal{R}a$, $a\mathcal{L}y$ and $x\mathcal{L}b$, $b\mathcal{R}y$. This means $x\in [a]_\mathcal{R}\cap [b]_\mathcal{L}$ and $y\in [b]_\mathcal{R}\cap [a]_\mathcal{L}$. So by the proposition we can find an $a'\in [b]_\mathcal{H}$ that satisfies the requirements.
\end{proof}

\begin{proposition}
Let $S$ be a semigroup and let $H,K$ be $\mathcal{H}$-classes that are subgroups and members of the same $\mathcal{D}$-class. Then $H$ and $K$ are isomorphic groups.
\end{proposition}
\begin{proof}
Let $e$ be the idempotent (and thus identity) in $H$ and $f$ the identity in $K$. By \ref{corollaryInversesIdempotentsDclass} we have mutually inverse $a\in [x]_\mathcal{R}\cap [y]_\mathcal{L}$ and $a'\in [y]_\mathcal{R}\cap [x]_\mathcal{L}$ such that $x = aa'$ and $y = a'a$.

We claim $\lambda_{a'}\circ\rho_{a}|_H$ is an isomorphism $H\to K$. Indeed it is bijective by \ref{GreensLemma}. We need to show that it is a homomorphism: take $u,v\in H$
\[ (\lambda_{a'}\circ\rho_{a})(u)(\lambda_{a'}\circ\rho_{a})(v) = a'uaa'va = a'uxva = a'uva = (\lambda_{a'}\circ\rho_{a})(uv), \]
where we have used that $x$ is the identity for $H$.
\end{proof}

\subsubsection{Regular semigroups}
\begin{definition}
A regular semigroup is a semigroup where every element is regular.
\end{definition}

\begin{lemma}
Let $S$ be a regular semigroup. Then for all $a\in S$
\[ \widetilde{S}a = Sa, \qquad  a\widetilde{S} = aS \qquad\text{and}\qquad \widetilde{S}a\widetilde{S} = SaS. \]
\end{lemma}
So we can drop all mention of $\widetilde{S}$ when working with ideal. In particular
\begin{itemize}
\item $a\mathcal{L}b$ if and only if $Sa = Sb$;
\item $a\mathcal{R}b$ if and only if $aS = bS$;
\item $a\mathcal{J}b$ if and only if $SaS = SbS$.
\end{itemize}

\subsection{Inverse semigroups}

\section{Monoids}


\begin{lemma}
A locally small category with a single object is a monoid.
\end{lemma}
TODO: \udef{delooping} $\cat{B}M$.


\subsection{Ordered monoids}
\begin{definition}
An \udef{ordered monoid} is a monoid $(M, \cdot, 0)$ on which a partial order $\preceq$ is defined that is compatible, i.e. $\forall x,y,z\in M$
\[ x\preceq y \implies x\cdot z \preceq y \cdot z \land z\cdot x \preceq z \cdot y. \]

Positive: $x > 0$.
\end{definition}

\subsubsection{The Archimedean property}
\begin{definition}
Let $(M,+,\leq)$ be a totally ordered monoid and $x,y\in M$ positive. Then
\begin{itemize}
\item $x$ is \udef{infinitesimal w.r.t.} $y$ or $y$ is \udef{infinite w.r.t.} $x$ if $nx<y$ for all $n\in\N$;
\item $M$ is \udef{Archimedean} if there is no pair $(x,y)$ such that $x$ is infinitesimal w.r.t. $y$.
\end{itemize}
\end{definition}
every submonoid is Archimedean.

abelian??

\section{Divisibility}
$m|n$ order relation.

$\sup\{n,m\} = kgv(n,m)$ and $\inf\{n,m\} = ggd(n,m)$



\chapter{Relational structures}
\begin{definition}
A \udef{relational structure} is a structured set $\sSet{A, R}$ where $R$ is a homogeneous relation on the set $A$.
\end{definition}


\section{Duality}
\begin{definition}
Let $\sSet{A,R}$ be a relational structure.

The relational structure \udef{dual} to $\sSet{A,R}$ is $\sSet{A, R}^o \defeq \sSet{A^o,R^\transp}\defeq \sSet{A,R^\transp}$.
\end{definition}



\section{Functions on relational structures}

\begin{lemma} \label{functionSubsetRelation}
Let $f: A\to A$ be a function on a relational structure $\sSet{A,R}$. Then
\[ f \subseteq R \iff \id_A \subseteq R;f^\transp. \]
\end{lemma}
\begin{proof}
We have
\[ f \subseteq R \implies \id_A \subseteq f;f^\transp \subseteq R;f^\transp \implies f \subseteq R;f^\transp;f \subseteq R. \]
\end{proof}

\begin{definition}
Let $(A, R)$ and $(B, S)$ be relational structures and $f: A\to B$ a function. We say
\begin{itemize}
\item $f$ is \udef{relation-preserving} if
\[ \forall x,y\in A: xRy \implies f(x)Sf(y); \]
\item $f$ is \udef{relation-reflecting} if
\[ \forall x,y\in A: f(x)Sf(y) \implies xRy; \]
\item $f$ is a \udef{relation embedding} if it is relation-preserving and -reflecting:
\[ \forall x,y\in A: xRy \iff f(x)Sf(y). \]
\end{itemize}
\end{definition}

\begin{proposition} \label{relationPreserving}
Let $\sSet{X, R}$ and $\sSet{Y, S}$ be relational structures and $f: X\to Y$ a function. Then the following are equivalent:
\begin{enumerate}
\item $f$ is relation-preserving;
\item $R \subseteq f;S;f^\transp$;
\item $R;f \subseteq f;S$.
\end{enumerate}
\end{proposition}
\begin{proof}
$(1) \Leftrightarrow (2)$ Is clear from the definition.

$(2) \Rightarrow (3)$ We calculate
\[ R;f \subseteq (f;S;f^\transp);f = f;S;(f^\transp;f) \subseteq f;S;\id = f;S. \]

$(3) \Rightarrow (2)$ We calculate
\[ R \subseteq R;(f;f^\transp) = (R;f);f^\transp \subseteq f;S;f^\transp. \]
\end{proof}
\begin{corollary}
Let $\sSet{X, R}$ and $\sSet{Y, S}$ be relational structures and $f: X\to Y$ a function. Then the following are also equivalent to $f$ being relation-preserving:
\begin{enumerate}
\item $f: X^o \to Y^o$ is relation preserving;
\item $R^\transp \subseteq f;S^\transp;f^\transp$;
\item $R^\transp;f \subseteq f;S^\transp$;
\item $f^\transp;R \subseteq S;f^\transp$.
\end{enumerate}
\end{corollary}
\begin{proof}
(2) Is equivalent to $R \subseteq f;S;f^\transp$ by taking the transpose. The equivalence of (1), (2) and (3) is then given by the proposition.

(4) Is equivalent to (3) by taking the transpose.
\end{proof}
\begin{corollary} \label{functionImagesPreimagesAndPrincipalImages}
Let $\sSet{A,R}$ and $\sSet{B, S}$ be relational structures and $f: A\to B$ a function. Then the following are equivalent:
\begin{enumerate}
\item $f$ is order-preserving;
\item for all $x\in A$: $f[xR] \subseteq f(x)S$;
\item for all $x\in A$: $f[Rx] \subseteq Sf(x)$;
\item for all $y\in B$: $f^{-1}[\{y\}]_R \subseteq f^{-1}[yS]$;
\item for all $y\in B$: $_Rf^{-1}[\{y\}] \subseteq f^{-1}[Sy]$.
\end{enumerate}
\end{corollary}
TODO: this implies $R$-closure for $f^{-1}$: by preservation of union we have $f^{-1}[yS]_R \subseteq f^{-1}[yS^2] = f^{-1}[yS]$.
\begin{proof}
All statements follow from (1) by taking principal (pre)images of $R;f \subseteq f;S$ and $R^\transp;f \subseteq f;S^\transp$. For the converse we note that a relation is completely characterised by its principal (pre)images. See \ref{relationFromPrincipalImages}.

To be more explicit, we can use calculations of the form
\[ y\in xR \iff xRy \implies f(x)Rf(y) \iff f(y) \in f(x)R \]
to prove (2) and (3).

For (4) and (5) we may use calculations of the form
\[ x\in f^{-1}[\{y\}]_R \iff \exists z: f(z) = y \land zRx \implies \exists z: f(z) = y \land f(z)Rf(x) \implies ySf(x) \iff f(x) \in yS \iff x\in f^{-1}[yS]. \]
\end{proof}

\begin{lemma}
Let $(A, R)$ and $(B, S)$ be relational structures and $f: A\to B$ a function. Then
\begin{enumerate}
\item $f$ is relation-reflecting \textup{if and only if} $R \supseteq f;S;f^\transp$;
\item $f$ is a relation embedding \textup{if and only if} $R = f;S;f^\transp$.
\end{enumerate}
\end{lemma}

\begin{definition}
A function $f: A\to A$ on a relational structure $\sSet{A, R}$ is
\begin{itemize}
\item \udef{left-restrictive} if $f;R \subseteq R$;
\item \udef{right-restrictive} if $R;f^\transp \subseteq R$;
\item \udef{left-expansive} if $R \subseteq f;R$;
\item \udef{right-expansive} if $R \subseteq R;f^\transp$.
\end{itemize}
\end{definition}

TODO: expansive, contractive, extensive.

\begin{lemma} \label{transitiveRestrictiveLemma}
Let $f: A\to B$ be a function and $R$ a transitive relation on $B$. Then $f\subseteq R$ implies $f;R\subseteq R$ and $R;f\subseteq R$.
\end{lemma}
\begin{proof}
We have $f;R\subseteq R^2 \subseteq R$ and $R;f \subseteq R^2 \subseteq R$.
\end{proof}

\begin{lemma} \label{expansiveRelationPreserving}
Let $f: \sSet{A,R} \to \sSet{A,R}$ be a relation-preserving function. Then
\[ f;R \subseteq R \implies R \subseteq R;f^\transp \]
\end{lemma}
\begin{proof}
We calculate
\[ R \subseteq f;R;f^\transp \subseteq R;f^\transp. \]
\end{proof}

\subsection{Relation isomorphisms}
\begin{definition}
Let $(A_1, R_1)$ and $(A_2, R_2)$ be relational structures. A bijection $\phi:A_1 \twoheadrightarrowtail A_2$ such that
\[ \forall (s_1,t_1)\in R_1: (\phi(s_1),\phi(t_1))\in R_2 \qquad \text{and} \qquad \forall (s_2,t_2)\in R_2: (\phi^{-1}(s_2),\phi^{-1}(t_2))\in R_1, \]
is called a \udef{(relation) isomorphism}. If there exists a relation isomorphism $A_1 \twoheadrightarrowtail A_2$, then $(A_1, R_1)$ and $(A_2, R_2)$ are \udef{isomorphic}, denoted $A_1 \cong A_2$.
\end{definition}

\begin{lemma}
A map is a relation isomorphism \textup{if and only if} it is a bijective relation embedding.
\end{lemma}
\begin{proof}
Let $f: \sSet{A, R}\to \sSet{B, }$ be a bijective map. Then $f$ is relation-reflecting iff $f^{-1}$ is relation-preserving. Indeed
\[ f;R;f^{-1} \subseteq R \iff R = f^{-1};f;R;f^{-1};f \subseteq f^{-1};R;f. \]
\end{proof}

\begin{lemma} \label{isomorphismEquivalence}
Let $(A_1, R_1)$, $(A_2, R_2)$ and $(A_3, R_3)$ be relational structures and $f:A_1 \twoheadrightarrowtail A_2$, $g:A_2 \twoheadrightarrowtail A_3$ isomorphisms. Then
\begin{enumerate}
\item $I_{A_1}: A_1 \to A_1: a\mapsto a$ is an isomorphism;
\item $f^{-1}: A_2\to A_1$ is an isomorphism;
\item $(g\circ f): A_1\to A_3$ is an isomorphism.
\end{enumerate}
\end{lemma}
Consequently, relation isomorphism would be an equivalence relation on all structured sets, except there is no set of all structured sets, by Russell's paradox. Relation isomorphism can be an equivalence relation on a (restricted) set of structured sets.

\begin{lemma}
Let $(A_1, R_1)$ and $(A_2, R_2)$ be isomorphic relational structures. Then $R_1$ has the same properties as $R_2$.
\end{lemma}
E.g.: reflexivity, symmetry, transitivity, being an equivalence relation etc.


\section{The semigroup of relation-preserving functions}
\begin{lemma}
Let $\{\sSet{A_i, R_i}\}_{i\in I}$ be a set of relational structures. Then the set of relation-preserving functions together with $\emptyset$ forms a semigroup under composition.
\end{lemma}


TODO: Full semigroup, i.e. if relational structure is involved, all morphisms must be present.
\begin{proposition}
Let $M$ be a semigroup of relation-preserving functions and $f: \sSet{A, R} \to \sSet{B,S}$, $g: \sSet{A, R} \to \sSet{C,T}$ relation-preserving functions. Then
\[ f\mathcal{R}g \iff (f;S;f^\transp = g;T;g^\transp)\land (\ker f = \ker g). \]
\end{proposition}
\begin{proof}
First assume $f\mathcal{R}g$, then there exist $x,y\in \tilde{M}$ such that $f = g;x$ and $g = f;y$. Now $x$ and $y$ are relation preserving: $T \subseteq x;S;x^\transp$ and $S \subseteq y;T;y^\transp$. So
\[ f;S;f^\transp = g;x;S;x^\transp;g^\transp \supseteq g;T;g^\transp \quad\text{and}\quad g;T;g^\transp = f;y;T;y^\transp;f^\transp  \supseteq f;S;f^\transp. \]

For the converse, we can find functions $x,y$ such that $f = g;x$ and $g = f;y$ because of the equality of the kernels. We just need to show that $x$ and $y$ are relation-preserving. Because $g^\transp;f = g^\transp;g;x \subseteq x$, we have
\[ T \subseteq g^\transp;g;T;g^\transp;g = g^\transp;f;S;f^\transp;g \subseteq x;S;x^\transp. \]
The argument for $y$ is similar.
\end{proof}
\begin{corollary} \label{relationPreservingGeneralisedInversesEmbeddings}
Let $a,a'$ be generalised inverses in the semigroup of relation-preserving functions. Then
\begin{enumerate}
\item $aa'R a^{\prime\transp} = aSa^\transp a^{\prime\transp}$;
\item $a'R a^{\prime\transp}a^\transp = a'aSa^\transp$;
\item $a|_{\im a'}$ and $a'|_{\im a}$ are relation embeddings.
\end{enumerate}
\end{corollary}
\begin{proof}
Let $a: \sSet{A,R} \to \sSet{B,S}$ and $a': \sSet{B,S} \to \sSet{A,R}$ be generalised inverses. From \ref{idempotentsInGreensClasses} we have $a\mathcal{R}aa'$ and $a'\mathcal{R}a'a$. The proposition then gives us $aSa^\transp = aa'Ra^{\prime\transp} a^\transp$.

(1) Multiplying this equality on the right by $a^{\prime\transp}$ and using $a^{\prime\transp} = (a'aa')^\transp = a^{\prime\transp} a^\transp a^{\prime\transp}$ gives us
\[ aSa^\transp a^{\prime\transp} = aa'R(a^{\prime\transp} a^\transp a^{\prime\transp}) = aa'Ra^{\prime\transp}. \]

(2) Similar to (1), except multiplying on the left by $a'$.

(3) By assumption $a$ and $a'$ are relation-preserving. Also, using $aSa^\transp = aa'Ra^{\prime\transp} a^\transp$, we have
\[ S \supseteq S|_{\im(a)}^{\im(a)} = a^\transp aSa^\transp a = a^\transp (aa'Ra^{\prime\transp} a^\transp) a = (a^\transp a)a'Ra^{\prime\transp} (a^\transp a) = a'|_{\im(a)}R(a'|_{\im(a)})^{\transp}. \]
This means that $a'|_{\im(a)}$ is also relation-reflecting. The argument for $a|_{\im(a')}$ is similar.
\end{proof}


\subsection{Galois connections}
\begin{definition}
Let $a: \sSet{A,R} \to \sSet{B,S}$ and $a': \sSet{B,S} \to \sSet{A,R}$ be relation-preserving generalised inverses. We say $(a,a')$ is a \udef{Galois connection} between $\sSet{A, R}$ and $\sSet{B, S}$ if
\begin{itemize}
\item $aa'$ is left-restrictive for $R$, i.e. $aa'R \subseteq R$; and
\item $a'a$ is left-restrictive for $S^\transp$, i.e. $a'aS^\transp \subseteq S^\transp$.
\end{itemize}
We call $a$ the \udef{lower adjoint} or \udef{residuated map} and $a'$ the \udef{upper adjoint} or \udef{residual}.

Sometimes a Galois connection between $\sSet{A, R}$ and $\sSet{B, S}^o$ is referred to as an \udef{antitone Galois connection} between $\sSet{A, R}$ and $\sSet{B, S}$. In this situation we may refer to the ordinary Galois connection as a \udef{monotone Galois connection} to highlight the difference.
\end{definition}
Some authors exclusively use the term ``Galois connection'' to refer to antitone Galois connections. They my call monotone Galois connections residuated mappings.

\begin{lemma}
Let $\sSet{A, R}$ and $\sSet{B, S}$ be relational structures, and $a: \sSet{A,R} \to \sSet{B,S}$ and $a': \sSet{B,S} \to \sSet{A,R}$ functions. Then $(a,a')$ is an antitone Galois connection \textup{if and only if}
\begin{enumerate}
\item $a, a'$ are relation-reversing;
\item $a, a'$ are generalised inverses;
\item $aa'R \subseteq R$ and $a'aS \subseteq S$.
\end{enumerate}
\end{lemma}

For any relation-preserving involution $f: \sSet{A,R} \to \sSet{A,R}$,  $(f,f)$ is a Galois connection. For any relation-reversing involution $f: A\to A$, $(f,f^o)$ is a Galois connection between $\sSet{A,R}$ and $\sSet{A,R}^o$, where we consider $f$ as a function $f: \sSet{A,R} \to \sSet{A,R}^o$ and $f^o$ as the function $f^o: \sSet{A,R}^o \to \sSet{A,R}$ with the same graph.

\begin{example}
\begin{itemize}
\item Let $\sSet{A, \id_A}$ and $\sSet{B, \id_B}$ be discrete relational structures. Then $f: A\to B$ and $g: B \to A$ form a Galois connection if and only if they are invertible and $f = g^{-1}$.
\item Let $A$ be a set. Then complementation $c: \powerset(A) \to \powerset(A)$ is a relation-reversing involution, so it gives a Galois connection between $\sSet{\powerset(A), \subseteq}$ and $\sSet{\powerset(A), \subseteq}^o$. In particular we have the identity
\[ \forall X, Y\in\powerset(A): \qquad X^c \subseteq Y \iff X \supseteq Y^c, \]
which is the Galois identity \ref{GaloisIdentity}.
\end{itemize}
\end{example}

\begin{lemma} \label{immediateGaloisCorollaries}
If $(a,a')$ is a Galois connection between $\sSet{A,R}$ and $\sSet{B,S}$, then
\begin{enumerate}
\item $R^\transp \subseteq aa'R^\transp$ and $S \subseteq a'aS$;
\item $Raa' \subseteq R$ and $S^\transp a'a \subseteq S^\transp$.
\end{enumerate}
\end{lemma}
\begin{proof}
(1) This is a direct application of \ref{expansiveRelationPreserving}. I can also be obtained as a corollary to \ref{GaloisIdentity} (later).

(2) Follows from the Galois condition and the fact $a'a, aa'$ are relation-preserving (\ref{relationPreserving}):
\[ Raa' \subseteq aa'R \subseteq R \qquad\text{and}\qquad S^\transp a'a \subseteq a'aS^\transp \subseteq S^\transp. \]
\end{proof}

In particular, if the relations are reflexive, we have
\[ \id_A \subseteq R^\transp \subseteq aa'R^\transp \qquad\text{and}\qquad \id_B \subseteq S \subseteq a'aS. \]
This means that for all $x\in A$ we have $ xRa'(a(x))$ and for all $y\in B$ we have $a(a'(y))Sy$.

\subsubsection{The Galois identity}
\begin{proposition} \label{GaloisIdentity}
If $(a,a')$ is a Galois connection between $\sSet{A,R}$ and $\sSet{B,S}$, then
\[ aS = Ra^{\prime \transp}. \]
\end{proposition}
This identity is particularly important because it gives also gives a sufficient condition for a pair of maps between preorders to be a Galois connection, see \ref{preorderGaloisIdentity}.
\begin{proof}
By \ref{relationPreservingGeneralisedInversesEmbeddings} we have $aa'R a^{\prime\transp} = aSa^\transp a^{\prime\transp}$. Because $a,a'$ are relation-preserving, we have
\[ aS \subseteq aa'R a^{\prime\transp} = aSa^\transp a^{\prime\transp} = aS(a'a)^\transp \subseteq aS \quad\text{and}\quad Ra^{\prime\transp} \subseteq aSa^\transp a^{\prime\transp} = aa'R a^{\prime\transp} \subseteq Ra^{\prime\transp}, \]
which implies $aS = aSa^\transp a^{\prime\transp} = aa'R a^{\prime\transp} = Ra^{\prime\transp}$.
\end{proof}
\begin{corollary} \label{reflexiveGaloisCorollary}
Let $(a,a')$ be a Galois connection between $\sSet{A,R}$ and $\sSet{B,S}$.
\begin{enumerate}
\item If $S$ is reflexive, then $aa' \subseteq R$.
\item If $R$ is reflexive, then $a'a \subseteq S^\transp$.
\end{enumerate}
\end{corollary}
\begin{proof}
(1) We calculate $aa' \subseteq aSa' = Ra^{\prime \transp}a' \subseteq R$.

(2) Similarly, $a'a \subseteq a'R^\transp a = S^\transp a^\transp a \subseteq S^\transp$.
\end{proof}

\begin{lemma} \label{preimagesGaloisIdentity}
Let $\sSet{A, R}$ and $\sSet{B, S}$ be relationals structures and $f: A\to B$, $g: B\to A$ functions.
The following are equivalent:
\begin{enumerate}
\item $f;S = R;g^\transp$;
\item for all $x$: $f(x)S = g^{-1}[xR]$;
\item for all $y$: $f^{-1}[Sy] = Rg(y)$.
\end{enumerate}
\end{lemma}
\begin{proof}
Relations are completely characterised by their principal images / preimages. See \ref{relationFromPrincipalImages}.
\end{proof}

\begin{proposition} \label{preorderGaloisCondition}
Let $a: \sSet{A,R} \to \sSet{B,S}$ and $a': \sSet{B,S} \to \sSet{A,R}$ be order-preserving generalised inverses on preorders. Then the following are equivalent:
\begin{enumerate}
\item $(a, a')$ is a Galois connection;
\item $aa' \subseteq R$ and $a'a \subseteq S^\transp$;
\item $\id_A \subseteq aa'R^\transp$ and $\id_B \subseteq a'aS$.
\end{enumerate}
If the orders are partial orders, then we do not need the additional assumption that $a, a'$ are generalised inverses.
\end{proposition}
\begin{proof}
$(1) \Rightarrow (2)$ is given by \ref{reflexiveGaloisCorollary}.

$(2) \Rightarrow (1)$ follows by transitivity, see \ref{transitiveRestrictiveLemma}.

$(2) \Leftrightarrow (3)$ is given by \ref{functionSubsetRelation}.

Now assume the orders are anti-symmetric. We need to show that $(2,3)$ imply that $a, a'$ are generalised inverses. First, by transitivity, we have
\[ R^\transp \subseteq aa'(R^2)^\transp \subseteq aa'R^\transp \qquad\text{and}\qquad S \subseteq a'aS^2 \subseteq a'aS. \]
Then, using the fact that $a$ and $a'$ are relation preserving,
\begin{align*}
\id &\subseteq R \subseteq aSa^\transp \subseteq aa'aSa^\transp \\
\id &\subseteq R^\transp \subseteq aa'R^\transp \subseteq aa'aS^\transp a^{\transp} \\
\id &\subseteq S \subseteq a'aS \subseteq a'aa'Ra^{\prime\transp} \\
\id &\subseteq S^\transp \subseteq a'R^\transp a^{\prime\transp} \subseteq a'aa'R^\transp a^{\prime\transp}.
\end{align*}
By anti-symmetry, we have
\[ \id \subseteq a'aa'(R\cap R^\transp)a^{\prime\transp} \subseteq a'aa'a^{\prime\transp} \qquad\text{and}\qquad \id \subseteq  aa'a(S\cap S^\transp)a^\transp \subseteq aa'aa^\transp. \]
Finally \ref{functionEqualityIdComparison} gives $a = aa'a$ and $a' = a'aa'$.
\end{proof}

\begin{proposition} \label{preorderGaloisIdentity}
Let $\sSet{A,R}$ and $\sSet{B,S}$ be preorders. Let $a: A\to B$ and $a': B\to A$ be functions. If $a,a'$ are generalised inverses and
\[ aS = Ra^{\prime \transp}, \]
then $(a, a')$ is a Galois connection.

If $\sSet{A,R}$ and $\sSet{B,S}$ are partial orders, then the assumption of generalised inverses is superfluous. 
\end{proposition}
So the identity $aS = Ra^{\prime \transp}$ is sufficient for two functions $a,a'$ between posets to form a Galois connection. Indeed this is often taken as the definition of a Galois connection for posets.
\begin{proof}
We need to prove that $a,a'$ are order-preserving, generalised inverses and that $aa', a'a$ are properly restrictive.

We first verify restrictivity. By reflexivity we have
\[ aa' \subseteq aSa' = Ra^{\prime \transp}a' \subseteq R \qquad\text{and}\qquad a'a \subseteq a'R^\transp a = S^\transp a^\transp a \subseteq S^\transp. \]
By \ref{transitiveRestrictiveLemma} $aa'$ and $a'a$ are properly restrictive.

Next we prove order-preservation $aa'\subseteq R$. We first prove the identities of \ref{expansiveRelationPreserving}:
\[ R^\transp \subseteq aa'(aa')^\transp R^\transp \subseteq aa'R^\transp (aa')^\transp R^\transp = aa' (aa'R)^\transp R^\transp \subseteq aa'R^\transp R^\transp = aa'R^\transp \]
\[ S \subseteq a'a(a'a)^\transp S \subseteq a'a S (a'a)^\transp S = a'a(a'aS^\transp)^\transp S \subseteq a'a(S^\transp)^\transp S = a'aS. \]
From this relation-preservation follows easily:
\[ R \subseteq R(aa')^\transp = (Ra^{\prime \transp})a^\transp = aSa^\transp \qquad\text{and}\qquad S \subseteq a'aS = a'(aS) = a'Ra^{\prime \transp}. \]

Finally, assuming $R,S$ are anti-symmetric, then $a, a'$ are generalised inverses by \ref{preorderGaloisCondition}.
\end{proof}

\begin{proposition}
Let $\sSet{A,R}$ and $\sSet{B,S}$ be relational structures.
\begin{enumerate}
\item If $A$ is a poset, then every residuated map $a: A \to B$ has a unique residual.
\item If $B$ is a poset, then every residual map $a': B \to A$ is the residual of a unique residuated map.
\end{enumerate}
\end{proposition}
\begin{proof}
(1) Let $a$ be a residuated map with residuals $a_1'$ and $a_2'$. Then
\[ \id_B \subseteq S \subseteq a_1'Ra_1^{\prime \transp} = a_1'aS = a_1'Ra_2^{\prime \transp}. \]
Similarly $\id_B \subseteq a_2'Ra_1^{\prime \transp}$, which we can transpose to get $\id_B \subseteq a_1'R^\transp a_2^{\prime \transp}$. Together this gives
\[ \id_B \subseteq a_1'(R\cap R^\transp)a_2^{\prime \transp} = a_1'a_2^{\prime \transp}. \]
We conclude using \ref{functionEqualityIdComparison}.

(2) Similar.
\end{proof}


\begin{proposition}
Let $\sSet{A,R}$ and $\sSet{B,S}$ be partial orders. Then every residuated map $a: A \to B$ has a unique residual and every residual map $a': B \to A$ is the residual of a unique residuated map.
\end{proposition}
\begin{proof}
Let $a$ be a residuated map with residuals $a_1'$ and $a_2'$. It is enough to show that $aa_1' = aa_2'$ and $a_1'a = a_2'a$. Indeed \ref{idempotentsInGreensClasses} then shows that $a_1'\mathcal{H}a_2'$ and by \ref{inversesIdempotentsDclass} this means that $a_1' = a_2'$.

To this end we combine $\id_A \subseteq aa_1'R^\transp$, $\id_A \subseteq $
\end{proof}

\subsubsection{Derived Galois connections}
\begin{lemma}
Let $(a, a')$ be a Galois connection between $\sSet{A,R}$ and $\sSet{B,S}$. Then $(a',a)$ is a Galois connection between the dual structures $\sSet{B^o,S^\transp}$ and $\sSet{A^o,R^\transp}$.
\end{lemma}

\begin{lemma}
Let $(a, a')$ be a Galois connection between posets $\sSet{A,R}$ and $\sSet{B,S}$ and $(b, b')$ a Galois connection between posets $\sSet{B,S}$ and $\sSet{A,R}$. Then
\begin{enumerate}
\item $(ab, b'a')$ is a Galois connection between $\sSet{A,R}$ and $\sSet{A,R}$;
\item $(ba, a'b')$ is a Galois connection between $\sSet{B,S}$ and $\sSet{B,S}$.
\end{enumerate}
\end{lemma}
\begin{proof}
We prove (1). The proof of (2) is identical.
By \ref{preorderGaloisIdentity} we just need to verify the Galois identity:
\[ abR = aRb^{\prime\transp} = Ra^{\prime\transp}b^{\prime\transp} = R(b'a')^\transp. \]
\end{proof}

\subsection{Monads and comonads}
\begin{definition}
Let $\sSet{A,R}$ be a relational structure and $f: A\to A$ a function. We call $f$ a
\begin{itemize}
\item \udef{monad} if there exists a relational structure $\sSet{B,S}$ and functions $a: A\to B$ and $a': B\to A$ such that $(a, a')$ is a Galois connection and $f = aa'$;
\item \udef{comonad} if $f = a'a$.
\end{itemize}
\end{definition}

\begin{lemma}
Let $\sSet{A,R}$ be a relational structure and $f: A \to A$ a function. Then
\begin{enumerate}
\item $f$ is a monad \textup{if and only if}
\begin{enumerate}
\item $f$ is relation preserving;
\item $f$ is idempotent: $f^2 = f$;
\item $f$ is left-restrictive: $f;R \subseteq R$.
\end{enumerate}
\item $f$ is a comonad \textup{if and only if}
\begin{enumerate}
\item $f$ is relation preserving;
\item $f$ is idempotent: $f^2 = f$;
\item $f$ is right-restrictive: $R;f^\transp \subseteq R$;
\end{enumerate}
\end{enumerate}
\end{lemma}
\begin{proof}
Sufficiency $\Rightarrow$ is clear. For necessity, we may simply take $a: A \to A^o: x\mapsto f(x)$ and $a': A^o \to A: x\mapsto f(x)$.
\end{proof}

\begin{proposition}
When is a subset $X\subseteq A$ the image of a (co)monad? For posets: enough that $\upset x \cap X$ has bottom.
\end{proposition}

\begin{proposition}
Bijection monads - images of monads.
\end{proposition}

\begin{definition}
Completeness
\end{definition}

\subsection{Inclusion-preserving functions on powersets}
\url{file:///C:/Users/user/Downloads/(Mathematics%20and%20Its%20Applications%20565)%20Marcel%20Ern%C3%A9%20(auth.),%20K.%20Denecke,%20M.%20Ern%C3%A9,%20S.%20L.%20Wismath%20(eds.)%20-%20Galois%20Connections%20and%20Applications-Springer%20Netherlands%20(2004).pdf}

\url{https://sciendo.com/pdf/10.2478/ausm-2014-0019}

\begin{proposition} \label{polarsGaloisConnection}
Let $R$ be a relation on $(A,B)$. Then the polar functions
\[ \powerset(A)^o\to\powerset(B): X\mapsto X^R \qquad\text{and}\qquad \powerset(A)^o\to\powerset(B): X\mapsto {^RX} \]
form a Galois connection. Every antitone Galois connection between powersets is of this form, for some relation.
\end{proposition}
\begin{proof}
As $\sSet{\powerset(A), \subseteq}^o$ and $\sSet{\powerset(B), \subseteq}$ are posets, by \ref{preorderGaloisIdentity} we just need to verify the Galois identity: $\forall X\in \powerset(A), Y\in\powerset(B):$ we have
\[ X^R \subseteq^\transp Y \iff X^R \supseteq Y \;\iff\; \Big[ \forall x\in X: \forall y\in Y: xRy \Big] \;\iff\; X \subseteq {^RY}. \]

We can give the same calculation using the language of \ref{polarsCartesianProduct}:
\[ X^R \subseteq^\transp Y \iff X\times Y \subseteq R \iff Y\times X \subseteq R^\transp \iff X \subseteq Y^R. \]

TODO semilattice morphism
\end{proof}
\begin{corollary}
Let $R$ be a transitive homogeneous relation on $A$. Then the principal image and preimage functions
\[ A \to\powerset(B): X\mapsto X^R \qquad\text{and}\qquad \powerset(A)^o\to\powerset(B): X\mapsto {^RX} \]
\end{corollary}

\begin{proposition} \label{imagePreimageGaloisConnection}
Let $R$ be a homogeneous relation on $A$. Then
\begin{enumerate}
\item the image function is part of a Galois connection
\[ \powerset(A)\to\powerset(A): X\mapsto X_R \qquad\text{and}\qquad \powerset(A)\to\powerset(A): X\mapsto {^{\overline{R}}(X^c)}; \]
\item the preimage function is part of a Galois connection
\[ \powerset(A)\to\powerset(A): X\mapsto {_RX} \qquad\text{and}\qquad \powerset(A)\to\powerset(A): X\mapsto (X^c)^{\overline{R}}. \]
\end{enumerate}
These Galois connections are equivalent by replacing $R \leftrightarrow R^\transp$. Every isotone Galois connection between powersets is of this form, for some relation.
\end{proposition}
\begin{proof}
We compose the polar Galois connection of $\overline{R}$ with the Galois connection of complementation to get
\[ X_R = (X^{\overline{R}})^c \subseteq Y \iff X^{\overline{R}} \supseteq Y^c \iff X \subseteq {^{\overline{R}}(Y^c)}. \]
Similarly we have
\[ {_RX} = ({^{\overline{R}}X})^c \subseteq Y \iff {^{\overline{R}}}X \supseteq Y^c \iff X \subseteq (Y^c)^{\overline{R}}. \]

TODO semilattice morphism
\end{proof}

\begin{proposition}
Let $f: A\to B$ be a function. Then the preimage and image functions
\[ \powerset(B)\to\powerset(A): X\mapsto f^{-1}[X] \qquad\text{and}\qquad \powerset(A)\to\powerset(B): X\mapsto f[X] \]
form a Galois connection.
\end{proposition}
\begin{proof}
We verify the Galois identity for \ref{preorderGaloisIdentity} by writing $f^{-1}[X] = {_fX}$ and $f[X] = X_f$:
\[ {_fX} \subseteq Y \implies X \subseteq {_{f^\transp;f}X} \subseteq {_{f^\transp}Y} \implies {_fX} \subseteq {_{f;f^\transp}Y} \subseteq Y. \]
\end{proof}

\subsubsection{Closure and dual closure}

\begin{corollary} \label{upDownsetUnionIntersection}
Let $\sSet{P,\Yleft}$ be an ordered set and $\mathcal{A}\subseteq \powerset(P)$. Then
\begin{enumerate}
\item $\upset \bigcup \mathcal{A} = \bigcup_{A\in \mathcal{A}} \upset A$ and $\downset \bigcup \mathcal{A} = \bigcup_{A\in \mathcal{A}} \downset A$;
\item $\upset \bigcap \mathcal{A} \subseteq \bigcap_{A\in \mathcal{A}} \upset A$ and $\downset \bigcap \mathcal{A} \subseteq \bigcap_{A\in \mathcal{A}} \downset A$;
\item $\upset \bigcap_{A\in \mathcal{A}} \upset A = \bigcap_{A\in \mathcal{A}} \upset A$ and $\downset \bigcap_{A\in \mathcal{A}} \downset A = \bigcap_{A\in \mathcal{A}} \downset A$.
\end{enumerate}
\end{corollary}
\begin{proof}
(1) We calculate using \ref{unionIntersectionLabelSet}:
\[ \upset \bigcup \mathcal{A} = \bigcup_{x\in \bigcup\mathcal{A}} \upset x = \bigcup_{A\in \mathcal{A}}\bigcup_{x\in A}\upset x = \bigcup_{A\in \mathcal{A}} \upset A. \]

(2) Again we calculate using \ref{unionIntersectionLabelSet}:
\[ \upset \bigcap \mathcal{A} = \bigcup_{x\in \bigcap\mathcal{A}} \upset x \subseteq \bigcap_{A\in \mathcal{A}}\bigcup_{x\in A}\upset x = \bigcap_{A\in \mathcal{A}} \upset A. \]

(3) We calculate using (2) and the fact that closures are idempotent:
\[ \bigcap_{A\in \mathcal{A}} \upset A = \bigcap_{A\in \mathcal{A}} \upset\upset A \supseteq \upset \bigcap_{A\in \mathcal{A}} \upset A \supseteq \bigcap_{A\in \mathcal{A}} \upset A. \]
\end{proof}
Also t
\begin{corollary} \label{unionIntersectionDownUpSets}
Let $P$ be an ordered set and $\{Q_i\}_{i\in I}$ be a set of down sets in $P$. Then
\begin{enumerate}
\item $\bigcup Q_i$ is a down set;
\item $\bigcap Q_i$ is a down set.
\end{enumerate}
The same is true for up sets.
\end{corollary}
\begin{proof}
This follows from $\downset \bigcup Q_i = \bigcup \downset Q_i = \bigcup Q_i$ and $\downset \bigcap Q_i \subseteq \bigcap \downset Q_i = \bigcap Q_i$ by \ref{TODO}.
\end{proof}

\subsubsection{Maps and polars}

\begin{proposition} \label{imagePolars}
Let $\sSet{A, R}$ and $\sSet{B, S}$ be relational structures, $f: A\to B$ a relation-preserving function and $X\subseteq A$ a subset. Then
\begin{enumerate}
\item $f[X^R] \subseteq f[X]^S \cap \im(f)$;
\item $f[{^R}X] \subseteq {^Sf[X]} \cap \im(f)$;
\item $f[\max(X)] \subseteq \max(f[X])$;
\item $f[\min(X)] \subseteq \min(f[X])$.
\end{enumerate}
For relation-reflecting functions we have reversed inclusions. For relation embeddings the inclusions become equalities.
\end{proposition}
\begin{proof}
(1) We calculate, using \ref{boundsFromPrincipalImages} and \ref{functionImagesPreimagesAndPrincipalImages},
\[ f[X^R] = f\left[\bigcap_{x\in X}xR\right] \subseteq \bigcap_{x\in X}f[xR] \subseteq \bigcap_{x\in X}Sf(x) = \bigcap_{y\in f[X]} Sy =  f[S]^S. \]

(b) Dual to (a).

(c) We calculate
\[ f[\max(X)] = f[X\cap X^R] \subseteq f[X]\cap f[X^R] \subseteq f[X]\cap f[X]^S \cap \im(f) = f[X]\cap f[X]^S = \max(f[X]). \]

(d) Dual to (c).
\end{proof}
We cannot say anything about the supremum or infimum for general relation-preserving functions, because $f[X^R] \subseteq f[X]^S$ implies $f[X^R]^{S^\transp} \supseteq (f[X]^S)^{S^\transp}$, so the calculation would be
\[ f[\sup(X)] = f[X^R\cap (X^R)^{R^\transp}] \subseteq f[X]^S \cap f[(X^R)]^{S^\transp} \supseteq f[X]^S \cap (f[X]^S)^{S^\transp} = \sup(f[X]), \]
from which we cannot conclude anything.

\begin{proposition}
Let $\sSet{A, R}$ and $\sSet{B, S}$ be relational structures, $f: A\to B$ a function and $X\subseteq A$ a subset.
\begin{enumerate}
\item If $f$ is a residuated map, then $f[\sup(X)] \subseteq \sup(f[X])$.
\end{enumerate}
\end{proposition}
\begin{proof}
Let $f': B\to A$ be a residual of $f$.

From above we see that it is enough to have $f[(X^R)^{R^\transp}] \subseteq (f[X]^S)^{S^\transp}$.

From $ff'R \subseteq R$, we get $X^{ff'R} = (f'\circ f)[X]^{R} \subseteq X^{R}$. We first take the lower bound and then the image under $f$ to obtain
\begin{align*}
f[(X^{R})^{R^\transp}] &\subseteq f\big[((f'\circ f)[X]^{R})^{R^\transp}\big] \subseteq f\big[(f'\circ f)[X]^{R}\big]^{R^\transp} \subseteq f\big[(f'\circ f)[X]^{R} \cap \im(f')\big]^{R^\transp} \\
&= f|_{\im f'}\big[(f'\circ f)[X]^{R}\big]^{R^\transp} = \Big(f|_{\im f'}\big[(f'\circ f)[X]\big]^{R} \cap \im(f|_{\im f'})\Big)^{R^\transp} \\
&= (f[X]^{R} \cap \im(f|_{\im f'}))^{R^\transp}. 
\end{align*}
Using the fact that $f|_{\im f'}$ is a relation embedding (\ref{relationPreservingGeneralisedInversesEmbeddings})
\end{proof}

\chapter{Groups}
\url{https://www.maths.ed.ac.uk/~tl/gt/gt.pdf}

\section{Basic definitions}

\begin{definition}
A \udef{group} is a structured set $(G,\boldsymbol{\cdot}, e)$ where $e\in G$ and $\boldsymbol{\cdot}$ is a binary operation on $G$
\[\boldsymbol{\cdot}: G\times G \to G: (g,h)\mapsto g\cdot h\]
such that
\begin{enumerate}
\item $\boldsymbol{\cdot}$ is associative:
\[ \forall g_1,g_2,g_3 \in G: \quad g_1\cdot (g_2\cdot g_3) = (g_1\cdot g_2)\cdot g_3 \]
\item $e$ is an \udef{identity} for $\boldsymbol{\cdot}$:
\[ \forall g\in G: \quad g\cdot e = e\cdot g = g \]
\item every element has an \udef{inverse}:
\[ \forall g\in G: \exists h \in G: \quad  gh = hg = e \]
We write the inverse $h$ as $g^{-1}$. 
\end{enumerate}
If $\boldsymbol{\cdot}$ is satisfies
\[ \forall g_1, g_2 \in G: \quad g_1 \cdot g_2 = g_2\cdot g_1 \]
then the group is called \udef{commutative} or \udef{abelian}.

The cardinality of $G$ is the \udef{order} of the group, denoted $|G|$.
\end{definition}

\begin{example}
The \udef{Klein 4-group} has carrier $A = \{e,a,b,c\}$ and is defined by the Cayley table
\[ \begin{array}{l|llll}
\boldsymbol{\cdot} & e & a & b & c \\ \hline
e & e & a & b & c \\
a & a & e & c & b \\
b & b & c & e & a \\
c & c & b & a & e
\end{array}. \]
It is commutative (which can be seen by noting that the Cayley table equals its transpose).
It is also
\begin{itemize}
\item the unique commutative $4$-element group with $a^2 = b^2 = c^2 = e$;
\item the unique commutative $4$-element semigroup with identity $e = a^2 = b^2 = c^2$, and $ab = c$.
\end{itemize}
\end{example}

\begin{lemma}
Let $S$ be a semigroup. Then $S$ is a group \textup{if and only if}
\[ \forall x\in S: \; xS = S = Sx. \]
\end{lemma}
\begin{proof}
TODO? Need identity??
\end{proof}

The inverse of any element of the group is unique by \ref{inverseUniqueness}. Thus $(\cdot)^{-1}$ is a well-defined function.

\begin{proposition}
A group is a structure of type $(e, (\cdot)^{-1}, \boldsymbol{\cdot})$ with arity defined by
\[ \alpha(e) = 0, \qquad \alpha((\cdot)^{-1}) = 1, \qquad \alpha(\boldsymbol{\cdot}) = 2. \]
Conversely, a $(e, (\cdot)^{-1}, \boldsymbol{\cdot})$-algebra is a group if
\begin{itemize}
\item $\boldsymbol{\cdot}$ is associative;
\item $\forall g\in G: g\cdot e = e\cdot g = g$;
\item $\forall g\in G: g\cdot g^{-1} = g^{-1}\cdot g = e$.
\end{itemize}
\end{proposition}
In particular the concepts of homomorphism and isomorphism apply.

\subsection{Group homomorphisms}
\begin{definition}
Let $G,H$ be groups. A function $f: G\to H$ is called a \udef{group homomorphism} if it is a $\{\cdot\}$-homomorphism, i.e. for all $x,y\in G$
\[ f(x\cdot y) = f(x)\cdot f(y). \]
\end{definition}

\begin{lemma} \label{groupHomomorphismPreservesSignature}
Let $G,H$ be groups. A function $f: G\to H$ is a group homomorphism iff it is a $\{\cdot, {}^{-1}, e\}$-homomorphism.
\end{lemma}
\begin{proof}
The direction $\Leftarrow$ is clear.

Now assume $f$ is a $\{\cdot\}$-homomorphism and take arbitrary $x\in G$. We first prove that it also preserves the identity: we have $f(e) = f(e\cdot e) = f(e)\cdot f(e)$. Multiplting both sides by $f(e)^{-1}$ gives $f(e) = e$.

In order to prove that $f(x^{-1}) = f(x)^{-1}$, it is enough, by \ref{inverseUniqueness}, to prove that $f(x^{-1})$ is an inverse of $f(x)$. We calculate
\[ f(x^{-1})f(x) = f(x^{-1}x) = f(e) = e \qquad\text{and}\qquad f(x)f(x^{-1}) = f(xx^{-1}) = f(e) = e. \]
\end{proof}


\subsection{Notations}
We can use whatever symbols we want to denote the group operation, but there are two main conventions:
\begin{enumerate}
\item In \udef{multiplicative notation} the group operation is denoted by $\boldsymbol{\cdot}$, $*$ or just by concatenation (i.e.\ we write $gh$ instead of $g\cdot h$). In this case the inverse of $g$ is written $g^{-1}$, the neutral element $e$ is denoted $1$ and we can define
\[ g^n \defeq \underbrace{gg\ldots g}_{\text{$n$ factors}}\]
which is unambiguous due to associativity. Also
\[ g^{-n} \defeq (g^{-1})^n = (g^{n})^{-1}. \]
\item \udef{Additive notation} is mainly used for abelian groups. Conversion between multiplicative and additive notation is as follows:
\[ \begin{tikzcd}
g\cdot h \arrow[r, leftrightarrow]& g+h \\
1 \arrow[r, leftrightarrow]& 0 \\
g^{-1} \arrow[r, leftrightarrow]& -g \\
g^n \arrow[r, leftrightarrow] & ng.
\end{tikzcd} \]
\end{enumerate}

\begin{lemma} \label{calculusRepeatedGroupOperation}
Let $G$ be a group, $g\in G$ and $m,n\in \Z$. Then
\begin{enumerate}
\item in multiplicative notation we have
\[ g^mg^n = g^{m+n} \qquad (g^m)^n = g^{mn}; \]
\item in additive notation we have
\[ mg+ng = (m+n)g \qquad n(mg) = (mn)g. \]
\end{enumerate}
These statements are equivalent.
\end{lemma}

\subsection{Translation invariance}
\label{sec:translationInvariance}
\begin{definition}
Let $G$ be a group, $X$ a set and $f:G\times G \to X$ a binary function. Then $f$ is called
\begin{itemize}
\item \udef{left translation invariant} if $\forall x,y,z\in G:\; f(x, y) = f(zx,zy)$;
\item \udef{right translation invariant} if $\forall x,y,z\in G:\; f(x, y) = f(xz,yz)$;
\item \udef{translation invariant} if $f$ is left and right translation invariant.
\end{itemize}
\end{definition}

TODO: just for relations??
\begin{proposition}[Universal property translation invariance]
Let $G$ be a group. Define
\[ \Delta_r:G\times G\to G: (x,y) \to xy^{-1} \qquad \Delta_l:G\times G\to G: (x,y) \to x^{-1}y. \]
\begin{enumerate}
\item For any set $X$ and right translation invariant function $f:G\times G\to X$, there exists a unique $\widetilde{f}: G\to X$, such that
\[ \begin{tikzcd}
G\times G \rar{\Delta_r} \ar[dr, swap, "{f}"] & G \dar[dashed]{\widetilde{f}} \\
 & X
\end{tikzcd} \qquad\text{commutes.} \]
\item For any set $X$ and left translation invariant function $f:G\times G\to X$, there exists a unique $\widetilde{f}: G\to X$, such that
\[ \begin{tikzcd}
G\times G \rar{\Delta_l} \ar[dr, swap, "{f}"] & G \dar[dashed]{\widetilde{f}} \\
 & X
\end{tikzcd} \qquad\text{commutes.} \]
\end{enumerate}
\end{proposition}
\begin{proof}
TODO

(1) $\widetilde{f} = f(-, e)$;

(2) $\widetilde{f} = f(e, -)$;
\end{proof}

TODO which do we conventionally choose?

\begin{lemma} \label{leftToRightTranslationInvarianceLemma}
Let $\sSet{G, \cdot, 1}$ be a group. Then
\begin{enumerate}
\item $(-)^{-1} = \Delta_r\circ(\underline{1}, \id_G) = \Delta_l\circ (\id_G, \underline{1})$;
\item $\boldsymbol{\cdot} = \Delta_r\circ \big(\proj_1, \Delta_r\circ(\underline{1}, \proj_2)\big) = \Delta_l\circ \big(\Delta_l\circ(\proj_1, \underline{1}), \proj_2\big)$;
\item $\Delta_r = \Delta_l\circ \big(\Delta_l\circ (\proj_1, \underline{1}), \Delta_l\circ (\proj_2, \underline{1}) \big)$;
\item $\Delta_l = \Delta_r\circ \big(\Delta_l\circ (\underline{1},\proj_1), \Delta_l\circ (\underline{1}, \proj_2) \big)$.
\end{enumerate}
\end{lemma}

\begin{definition}
Let $G$ be a group and $X\subseteq G^2$ a set. Then $X$ is called
\begin{itemize}
\item \udef{left translation invariant} if the indicator function $\chi_X$ is left translation invariant;
\item \udef{right translation invariant} if the indicator function $\chi_X$ is right translation invariant;
\item \udef{translation invariant} if the indicator function $\chi_X$ is translation invariant.
\end{itemize}
\end{definition}

\begin{lemma}
Let $G$ be a group and $X\subseteq G^2$ a set. Then
\begin{enumerate}
\item $X$ is left translation invariant \textup{if and only if} $\forall g\in G: \forall (x,y)\in X: \; (gx, gy)\in X$;
\item $X$ is right translation invariant \textup{if and only if} $\forall g\in G: \forall (x,y)\in X: \; (xg, yg)\in X$.
\end{enumerate}
\end{lemma}

\begin{lemma}
A binary relation is
\begin{enumerate}
\item left translation invariant \textup{if and only if} it is left compatible;
\item right translation invariant \textup{if and only if} it is right compatible.
\end{enumerate}
\end{lemma}
\begin{proof}
TODO
\end{proof}

For a left compatible binary relation $R$, we have
\[ xRy \qquad\iff\qquad x^{-1}y \in \widetilde{R}. \]

\begin{proposition} \label{congruenceTranslationInvariant}
Let $G$ be a group and $\mathfrak{q}$ a congruence on $G$. Then $\mathfrak{q}$ is translation invariant. 
\end{proposition}
\begin{proof}
Take arbitrary $(x,y)\in\mathfrak{q}$ and $z\in G$. Then $(z,z)\in\mathfrak{q}$ by reflexivity and thus $(z\cdot x, z\cdot y) = (z,z)\cdot (x,y)\in\mathfrak{q}$. Similarly $(x\cdot z, y\cdot z) = (x,y)\cdot (z,z)\in\mathfrak{q}$.
\end{proof}

In particular, as the kernel of a homomorphism is a congruence, it is translation invariant.
When dealing with groups, we will redefine $\ker$ to mean $\widetilde{\ker}$.

\subsection{Subgroups}
\begin{definition}
Let $(G,\boldsymbol{\cdot})$ be a group. We call $(H,*)$ a \udef{subgroup} if it is a group and $H\subseteq G$ and $* = \boldsymbol{\cdot}|_H$.
\end{definition}

\begin{lemma}[Subgroup criterion] \label{subgroupCriterion}
Let $(G,\boldsymbol{\cdot})$ be a group and $H$ a non-empty subset of $G$. The following are equivalent:
\begin{enumerate}
\item $(H,\boldsymbol{\cdot}|_H)$ is a subgroup;
\item for all $a,b\in H$:
\begin{itemize}
\item $a\cdot b \in H$,
\item $a^{-1}\in H$;
\end{itemize}
\item for all $a,b\in H: a\cdot b^{-1} \in H$;
\item $H$ is a sub-$\{\cdot, {}^{-1}, e\}$-algebra of $G$;
\item $H$ is a non-empty sub-$\{\cdot, {}^{-1}\}$-algebra of $G$;
\end{enumerate}
\end{lemma}

\begin{lemma}
Let $G$ be a group and $H_1, H_2$ be subgroups. Then $H_1\cap H_2$ is again a subgroup of $G$.
\end{lemma}
\begin{lemma}
Let $f:G\to H$ be a group homomorphism. Then $\ker(f)$ is a subgroup.
\end{lemma}

\subsubsection{Cosets}
\begin{definition}
Let $G$ be a group and $H\subseteq G$ a subgroup. We call a subset of the form
\begin{itemize}
\item $g\cdot H$ for some $g\in G$ a \udef{left coset};
\item $H\cdot g$ for some $g\in G$ a \udef{right coset}.
\end{itemize}
A \udef{coset} is a subset that is either a left coset or a right coset.
\end{definition}

\begin{lemma} \label{differentCosetsDisjoint}
Let $G$ be a group and $H\subseteq G$ a subgroup. Any two left (resp. right) cosets are either identical or disjoint.
\end{lemma}
\begin{proof}
Take $g,h\in G$. Assume $x\in gH\cap hH$. Then there exist $x_1,x_2\in H$ such that $gx_1 = x = hx_2$. Thus $g = hx_2x_1^{-1}$ and $h = gx_1x_2^{-1}$, meaning $gH = hx_2x_1^{-1}H = hH$ by \ref{groupCriterion}.
\end{proof}

\subsubsection{Lagrange's theorem}
\begin{theorem}
Let $G$ be a group and $H$ a subgroup of $G$. Then
\[ |G| = [G:H]\cdot |H|. \]
\end{theorem}
If $G$ is finite, $|G|$ and $|H|$ are natural numbers. If $G$ is infinite, the theorem still holds, but the orders and index are cardinals.

\subsubsection{Normal subgroups}
\begin{definition}
Let $G$ be a group. A subgroup $N\subseteq G$ is called \udef{normal} or \udef{self-conjugate} if $gNg^{-1} \subseteq N$ for all $g\in G$.

We write $N \lhd G$.
\end{definition}

\begin{proposition} \label{congruenceNormalSubgroup}
Let $G$ be a group.
A translation invariant binary relation $\mathfrak{q}$ on $G$ is a $\{\cdot, {}^{-1}, e\}$-congruence \textup{if and only if} $\widetilde{\mathfrak{q}}$ is a normal subgroup.
\end{proposition}
Note that all congruences are translation invariant by \ref{congruenceTranslationInvariant}. The hypothesis is necessary for $\widetilde{\mathfrak{q}}$ to be well-defined.
\begin{proof}
First assume $\mathfrak{q}$ is a congruence and take $z\in\widetilde{\mathfrak{q}}$ and arbitrary $g\in G$. We need to show that $gzg^{-1}\in \widetilde{\mathfrak{q}}$. We can find $(x,y)\in \mathfrak{q}$ such that $z = xy^{-1}$. Because $\mathfrak{q}$ is reflexive, we have $(g,g)\in\mathfrak{q}$. Because is it is a subalgebra of $G^2$, we have $(gx,gy) = (g,g)\cdot (x,y)\in \mathfrak{q}$. So $gzg^{-1} = g(xy^{-1})g^{-1} = gx(gy)^{-1} \in \widetilde{\mathfrak{q}}$.

Now assume $\widetilde{\mathfrak{q}}$ is a normal subgroup. We first check that $\mathfrak{q}$ is an equivalence relation:
\begin{itemize}
\item \emph{reflexivity}: $e = gg^{-1}\in \widetilde{\mathfrak{q}}$, so $(g,g)\in\mathfrak{q}$ for all $g\in G$;
\item \emph{symmetry}: if $xy^{-1}\in \widetilde{\mathfrak{q}}$, then $yx^{-1} = (xy^{-1})^{-1}\in \widetilde{\mathfrak{q}}$;
\item \emph{transitivity}: if $xy^{-1}, yz^{-1}\in \widetilde{\mathfrak{q}}$, then $xy^{-1}yz^{-1} = xz^{-1}\in \widetilde{\mathfrak{q}}$.
\end{itemize}
Now we need to show that $\mathfrak{q}$ is a subalgebra. To show $\mathfrak{q}$ is closed under taking the inverse, take $(x,y) \in \mathfrak{q}$. Then $xy^{-1}\in \widetilde{\mathfrak{q}}$ and $y^{-1}(xy^{-1})^{-1}y = x^{-1}y = x^{-1}(y^{-1})^{-1} \in \widetilde{\mathfrak{q}}$ by normality. So $(x^{-1}, y^{-1}) \in \mathfrak{q}$.

To show $\mathfrak{q}$ is closed under the binary group operation, take $(x,y),(u,v)\in\mathfrak{q}$. Then $uv^{-1}, xy^{-1}$ and $y^{-1}x = y^{-1}(xy^{-1})y$ are elements of $\widetilde{\mathfrak{q}}$. Then
\[ xu(yv)^{-1} = xuv^{-1}y^{-1} = x(uv^{-1})(y^{-1}x)x^{-1} \in \widetilde{\mathfrak{q}}. \]

Finally it is enough to note that $\mathfrak{q}$ is not empty, by \ref{subgroupCriterion}.
\end{proof}
Note that the left factorisation $\widetilde{\mathfrak{q}}$ is equal to the right factorisation $\widetilde{\mathfrak{q}}$.
\begin{corollary} \label{kernelNormalSubgroup}
Let $f: G\to H$ be a group homomorphism. Then $\ker f$ is a normal subgroup.
\end{corollary}

If $N\lhd G$ is a normal subgroup, we define the quotient algebra
\[ G/N \defeq G/(\setbuilder{(x,y)\in G^2}{xy^{-1}\in N}). \]
This is a group because homomorphisms preserve associativity, inverses and identity (TODO ref). We call such a group a \udef{quotient group}.

We denote the equivalence classes by $[x]_N$.

\subsection{Conjugation}
\begin{definition}
Let $G$ be a group and $g\in G$ an element. Then the mapping
\[ \Ad_g: G\to G: x\mapsto g^{-1}xg \]
is called \udef{conjugation by $g$}.
We also write $h^g \defeq \Ad_g(h) = g^{-1}hg$.
\end{definition}
Thus a subgroup is normal if and only if $\Ad_g[N] \subseteq N$ for all $g\in G$.


\subsubsection{Conjugacy}
\begin{definition}
Let $G$ be a group. Elements $g,h\in G$ are called \udef{conjugate} if $\exists x: \; \Ad_x(g) = h$.
\end{definition}

\begin{proposition}
Conjugacy is an equivalence relation.
\end{proposition}
The equivalence classes under conjugation are called \udef{conjugacy classes}.


\subsubsection{Centraliser and normaliser}

\begin{lemma}
Let $G$ be a group and $A\subseteq G$ a subset. Then
\begin{enumerate}
\item $Z_G(A) = \setbuilder{g\in G}{\forall a\in A:\;\Ad_g(a) = a}$;
\item $N_G(A) = \setbuilder{g\in G}{\Ad_g[A] = A}$.
\end{enumerate}
\end{lemma}

\begin{proposition}
Let $G$ be a group and $A\subseteq G$ a subset. Then
\begin{enumerate}
\item $Z_G(A) \lhd N_G(A) \lhd G$;
\item $N_G(A)$ is the largest subgroup of $G$ in which $A$ is normal.
\end{enumerate}
\end{proposition}
\begin{corollary}
$Z_G\lhd G$.
\end{corollary}


\subsubsection{Inner and outer automorphisms}
\begin{lemma}
Let $G$ be a group and $g\in G$ an element. Then $\Ad_g$ is an automorphism.
\end{lemma}
\begin{definition}
Let $G$ be a group.
Automorphisms of the form $\Ad_g$ for some $g\in G$ are called \udef{inner automorphisms}. Automorphisms that are not of this form are called \udef{outer automorphisms}.

The set of inner automorphisms forms a group, denoted $\Inn(G)$.
\end{definition}

\begin{theorem}[N/C theorem]
Let $G$ be a group and $H\subseteq G$ a subgroup. Then
\[ N_H(H) / Z_G(H) \cong \Inn(H). \]
\end{theorem}
\begin{proof}
TODO \url{https://proofwiki.org/wiki/Centralizer_is_Normal_Subgroup_of_Normalizer}
\end{proof}
\begin{corollary}
Let $G$ be a group. Then $G / Z_G \cong \Inn(G)$.
\end{corollary}


\subsection{Direct product}
\begin{definition}
The \udef{direct product} $G \equiv H\otimes F$ of two groups $H$ and $F$ is defined with the following operation:
\[ (H\otimes F) \times (H\otimes F) \rightarrow (H\otimes F): ((h_1,f_1),(h_2,f_2)) \mapsto (h_1\cdot h_2, f_1\cdot f_2)\]
\end{definition}
The direct product is a group with
\[ \begin{cases}
e_G = (e_H,e_F) \\
g^{-1} = (h^{-1}, f^{-1})\qquad \forall g = (h,f) \in G.
\end{cases} \]
The groups $F$ and $H$ are subgroups of $G$ and can be recovered by considering, respectively the elements of $G$ of the form $(e_H, g)$ and $(g ,e_F)$.

\subsection{Semidirect product}
TODO

\section{Types of groups}
\begin{example}
Examples of Groups:
\begin{enumerate}
\item The \udef{trivial group} $\{e\}$.
\item $\Z_n = 0:(n-1)$ with addition modulo $n$, is a group of order $n$.
\item $\Z_n$, the group of all $n^{\text{th}}$ roots of $1$ with the ordinary product, is of order $n$.
\begin{itemize}
\item $Z_2 = \{1,-1\}$
\item $Z_3 = \{1, e^{i2/3\pi}, e^{i1/3\pi}\}$
\end{itemize}
\item $S_n$, the group of all permutations of $n$ elements, is of order $n!$.
\item Integers with addition.
\item $\mathbb{R}\setminus\{0\}$ with multiplication.
\item The square (i.e.\ $n\times n$) invertible matrices with matrix multiplication form a group.
\end{enumerate}
\end{example}

\begin{proposition}
The groups $Z_n$ and $\Z_n$ are isomorphic.
\end{proposition}
We use $Z_n$ to denote the group if we are using multiplicative notation and $\Z_n$ if we are using additive notation.

In particular we have $\Z_2 = \sSet{\{0,1\},+}$ and $Z_2 = \sSet{\{1,-1\},\cdot}$.

\begin{lemma}
All groups of order $2$ are isomorphic to $\Z_2$.
\end{lemma}
\begin{proof}
Let $G = \sSet{\{e,g\}, \cdot}$ be a groups of order $2$, with $e$ the identity.
We must have $g\cdot g = e$. Indeed, from $g \neq e$, we get $g\cdot g \neq g\cdot e = g$ and $g\cdot g = e$ is the only other option.

Consider the function
\[ f: G \to \Z_2: \begin{cases}
e\mapsto 0 \\ g \mapsto 1
\end{cases}. \]
This functions is clearly bijective. We just need to see that it is a homomorphism. Indeed we have
\[ \begin{cases}
f(e\cdot e) = f(e) = 0 = 0+0 = f(e) + f(e) & f(e\cdot g) = f(g) = 1 = 0+1 = f(e) + f(g) \\
f(g\cdot e) = f(g) = 1 = 1+0 = f(g) + f(e) & f(g\cdot g) = f(e) = 0 = 1+1 = f(g) + f(g). 
\end{cases} \] 
\end{proof}

\subsection{Words, relations and presentations}
\begin{example}
Quaternion group
\[ \mathbb{H} \defeq \group\setbuilder{a,b}{a^4=e, a^2=b^2, b^{-1}ab = a^{-1}} \]
\end{example}

\subsection{Cyclic groups}
\begin{definition}
A group is called \udef{cyclic} if it is generated by a single element.
\end{definition}
\begin{lemma}
\begin{enumerate}
\item The group $(\Z, +)$ is cyclic.
\item Every cyclic group is a an image of $\Z$ by a homomorphism.
\item Every cyclic group is isomorphic to $\Z$ or $\Z/m\Z$.
\end{enumerate}
\end{lemma}
We write $\Z_m$ or $C_m$ for $\Z/m\Z$.
 
\subsection{Torsion groups and orders of elements}
\begin{definition}
Let $G$ be a group. An element $a$ satisfying $a^n = 1$ for some $n$ is said to be of \udef{finite order}. In this case the \udef{order} of the element $a$ is $n$.

A group in which every element is of finite order is called a \udef{torsion group} or a \udef{periodic group}.
\end{definition}
\begin{lemma}
Every finite group is a torsion group. The converse is not true.
\end{lemma}
\begin{proof}
Let $G$ be a finite group. Assume $G$ is not a torsion group. Then we can find an element $g\in G$ that is not of finite order. Consider the mapping $\N \to G: n\mapsto g^n$. The image of this mapping is a subset of $G$ and thus finite, so the mapping is not injective, so we can find $n<m\in\N$ such that $g^n = g^m$. Then $g^{m-n}=1$ with $m-n\in \N$, so $g$ is of finite order, which is a contradiction. This falsity of the converse is shown by the following examples.
\end{proof}

\begin{example}
The set
\[ \setbuilder{z\in \C}{z^n = 1\;\text{for some}\; n\in \Z} \]
together with complex multiplication forms an infinite torsion group.
\end{example}

\begin{lemma}
Let $G$ be an abelian group. Then the set of all elements of finite order forms a subgroup, called the \udef{torsion subgroup}.
\end{lemma}


\subsection{Permutation groups}
\begin{proposition} \mbox{}
\begin{enumerate}
\item Let $X$ be a set. The set of bijections $X\to X$ forms a group;
\item Let $X,Y$ be sets. The groups of bijections on $X$ and $Y$ are isomorphic \textup{if and only if} $X$ and $Y$ are equinumerous.
\end{enumerate}
\end{proposition}

\begin{definition}
We call the group of bijections on a set $X$ the \udef{symmetric group} of $X$, denoted $S(X)$.
\begin{itemize}
\item The \udef{degree} of $S(X)$ is the cardinality of $X$.
\item For any cardinal $n$, we denote the unique permutation group of degree $n$ by $S_n$.
\item Elements of $S_n$ are called \udef{permutations} and subgroups of $S_n$ are called \udef{permutation groups}.
\end{itemize}
\end{definition}

TODO: cfr. Clifford algebra with $V$ containing the transpositions.

\begin{proposition}
For all sets $X$, we have $S(X) = S_{|X|}$.
\end{proposition}
In general we state and prove results for $S_n$, without loss of generality.

\begin{lemma}
For all cardinals $\kappa >_c 2$, the permutation group $S_\kappa$ is non-abelian.
\end{lemma}

\subsubsection{Cycles}
\subsubsection{Transpositions, parity and the alternating group}
\begin{definition}
Let $S_n$ be a permutation group and $x\in S_n$. Consider the set $\Fixedpoints(x)$. If $|\Fixedpoints(x)| = 2$, then $x$ is called a \udef{transposition}.
\end{definition}

\begin{definition}
Let $S_n$ be a finite permutation group. We call the function
\[ p_n: S_n \to Z_2: x\mapsto \begin{cases}
1 & \text{$|\Fixedpoints(x)|$ is even} \\
-1 & \text{$|\Fixedpoints(x)|$ is odd}
\end{cases} \]
the \udef{parity homomorphism}.

The group $A_n \defeq \ker p_n$ is called the \udef{alternating group} of \udef{degree} $n$.
\end{definition}

\begin{proposition}
The parity homomorphism is a homomorphism.
\end{proposition}

\begin{lemma}
The alternating group $A_n$ has order $n!/2$.
\end{lemma}

\subsection{Dihedral groups}
\begin{definition}
Dihedral group of order $2n$.
\[ D_n \defeq \group\setbuilder{ a,b }{a^n=b^2=e, b^{-1}ab = a^{-1} }. \]
\end{definition}

\begin{proposition}
Let $n\in \N$. Then
\[ Z_G = \begin{cases}
\{e, a^{n/2}\} & \text{$n$ is even} \\
\{e\} & \text{$n$ is odd}.
\end{cases} \]
\end{proposition}
\begin{corollary} \label{dihedralDoubleCover}
For all $n\in \N$ there is a short exact sequence
\[ \begin{tikzcd}
1 \rar & \Z_2 \rar & D_{2n} \rar & D_n \rar & 1.
\end{tikzcd} \]
\end{corollary}

\subsubsection{Full dihedral group}
TODO: full dihedral group $D$ of isometries of $\C$ that fix the origin.
\[ \begin{tikzcd}
1 \rar & \T \rar & D\rar & Z_2 \rar & 1
\end{tikzcd} \qquad\text{is short exact.} \]

\section{Short exact sequences}
\subsection{Quotient sequences}
\begin{proposition}
Let $G$ be a group and $N\lhd G$ a normal subgroup. Then
\[ \begin{tikzcd}
1 \rar & N \rar[hookrightarrow]{\subseteq} & G \rar{[\cdot]_N} & G/N \rar & 1
\end{tikzcd} \]
is a short exact sequence.
\end{proposition}
\begin{proof}
Clearly the inclusion $N\hookrightarrow G$ is injective and $[\cdot]_N$ is surjective. Finally note that $x\in \ker[\cdot]_N \iff [x]_N = [e]_N \iff xe^{-1} = x\in N$.
\end{proof}

\begin{proposition}
For any short exact sequence of groups
\[ \begin{tikzcd}
1 \rar & H_1 \rar{\alpha} & G \rar{\beta} & H_2 \rar & 1
\end{tikzcd}, \]
there exist isomorphisms $f,g$ and subgroup $N\lhd G$ such that
\[ \begin{tikzcd}
1 \rar & H_1 \dar{f} \rar{\alpha} & G \dar{\id_G} \rar{\beta} & H_2 \dar{g} \rar & 1 \\
1 \rar & N \rar[hookrightarrow] & G \rar{[\cdot]_N} & G/N \rar & 1
\end{tikzcd} \]
commutes.
\end{proposition}
\begin{proof}
We can take $N = \im(\alpha) = \ker(\beta)$, which is a normal subgroup by \ref{kernelNormalSubgroup}. Because $\alpha$ is injective, $\alpha|^{\im(\alpha)}: H_1 \to N$ is bijective. We take $f$ to be this.

The function $\beta': G/N \to H_2$ defined in the factor theorem \ref{factorTheorem} is bijective because $\beta$ is surjective.

Both constituent squares clearly commute. So the rectangle commutes by \ref{commutingRectangle}.
\end{proof}

So in some sense all short exact sequences of groups are of the form
\[ \begin{tikzcd}
1 \rar & N \rar[hookrightarrow] & G \rar & G/N \rar & 1
\end{tikzcd}. \]
Given $G$ and either $N$ or $G/N$, we can easily find the third group in the sequence. Given $N$ and $G/N$, there may be several inequivalent ways to complete the short exact sequence. Groups that fit in the middle of a short exact sequence are called group extensions.

\subsection{Group extensions}
\begin{definition}
Let $N,Q$ be groups. An \udef{extension} of $Q$ by $N$ is a group $G$ such that
\[
\begin{tikzcd}
1 \ar[r] & N \ar[r, "\iota"] & G \ar[r, "\pi"] & Q \ar[r] & 1
\end{tikzcd}.
\]
is a short exact sequence.
\end{definition}
\begin{lemma}
If $G$ is an extension of $Q$ by $N$, then $G$ is a group (TODO: closure), $\iota(N)$ is a normal subgroup of $G$ and $Q$ is isomorphic to $Q$.
\end{lemma}

\begin{example}
The real numbers $\R$ are an extension of the unit complex numbers by the integers $\Z$:
\[ \begin{tikzcd}
0 \rar & \Z \rar{\subseteq} & \R \ar[rr, "\theta\mapsto e^{2\pi i \theta}"] && \T \rar & 1
\end{tikzcd} \]
\end{example}

\subsubsection{Equivalent group extensions}
\begin{definition}
Two extensions $G,G'$ of $Q$ by $N$ are \udef{equivalent} if there is a homomorphism $T:G\to G'$ making the following diagram commutative:
\[
\begin{tikzcd}
1 \ar[r] & N \ar[r, "\iota"] \ar[equal]{d} & G \ar[r, "\pi"] \ar[d,"T"] & Q \ar[r] \ar[d, equal] & 1 \\
1 \ar[r] & N \ar[r, "\iota"] & G' \ar[r, "\pi"] & Q \ar[r] & 1.
\end{tikzcd}
\]
\end{definition}
\begin{lemma}
If $G,G'$ are equivalent extensions, then they are isomorphic. So equivalence of extension is an equivalence relation.
\end{lemma}
\begin{proof}
The short five lemma (TODO).
\end{proof}
The converse is \emph{not} true! TODO: For instance, there are $8$ inequivalent extensions of the Klein four-group by $\mathbb{Z}/2\mathbb{Z}$, but there are, up to group isomorphism, only four groups of order $8$ containing a normal subgroup of order $2$ with quotient group isomorphic to the Klein four-group.

\subsubsection{Split exact sequences}
\url{https://kconrad.math.uconn.edu/blurbs/grouptheory/splittinggp.pdf}

\subsubsection{Double covers}
\begin{definition}
Let $G_1, G_2$ be groups. If $G_1$ is an extension of $G_2$ by $\Z_2$, i.e.\
\[\begin{tikzcd}
1 \rar & \Z_2 \rar & G_1 \rar & G_2 \rar & 1
\end{tikzcd} \]
is a short exact sequence, then we call $G_1$ a \udef{double cover} of $G_2$.
\end{definition}

\begin{example}
For all $n\in \N$, the dihedral group $D_{2n}$ is a double cover of $D_n$:
\[\begin{tikzcd}
1 \rar & \Z_2 \rar & D_{2n} \rar & D_n \rar & 1
\end{tikzcd} \]
See \ref{dihedralDoubleCover}.
\end{example}
Quaternionic group gives inequivalent extension.


\section{Group action}
\begin{definition}
An \udef{action} of a group $G$ on a set $X$ is a mapping
\[ \cdot: G\times X \to X: (g,x) \mapsto g\cdot x \]
satisfying
\begin{enumerate}
\item $e\cdot x = x$ for all $x\in X$ where $e$ is the neutral element of $G$;
\item $g(h\cdot x) = (gh)\cdot x$ for all $x\in X$ and $g,h\in G$.
\end{enumerate}
We will often just right $gx$ instead of $g\cdot x$.

A set $X$ with a given action of $G$ on it is called a \udef{$G$-set}.
\end{definition}
This definition can be reformulated using the curried form of $\pi$, namely
\[ \rho \defeq \operatorname{curry}(\pi): G \to (X\to X). \]
Then the rest of the definition of group action amounts to the statement that 
\[ \rho: G \to S(X) \quad \text{is a group homomorphism.} \]

We may then specify $G$-sets by the data $(X,\rho)$, where $X$ is a set and $\rho: G \to S(X)$ a group homomorphism.

TODO: opposite action.

\begin{definition}
Given two $G$-sets $X,Y$, a \udef{$G$-equivariant mapping} or \udef{intertwiner} is a map $f:X \to Y$ such that
\[ f(gx) = gf(x) \]
for all $g\in G$ and $x\in X$.
\end{definition}
We can express this by saying $f: (X,\rho_1) \to (Y,\rho_2)$ is a map between $G$-sets such that
\[ f\circ \rho_1(g) = \rho_2(g)\circ f \]
for all $g\in G$.

\begin{lemma}
The $G$-sets form a locally small category with $G$-equivariant maps as morphisms.
\end{lemma}


\subsection{Orbits and stabilisers}
\begin{definition}
Let $G$ be a group acting on a set $X$. We define the \udef{orbit} of $x\in X$ as the set
\[ Gx = G\cdot x \defeq \setbuilder{g\cdot x \in X}{g\in G}  \]
and the \udef{stabiliser} of $x\in X$ as the set
\[ G_x \defeq \setbuilder{g\in G}{g\cdot x = x}. \]
\end{definition}
\begin{proposition}[Orbit-stabiliser theorem]
Let $X$ be a $G$-set and $x\in X$. Then
\begin{enumerate}
\item $|Gx| = [G:G_x]$;
\item $|G| = |Gx|\cdot |G_x|$.
\end{enumerate}
\end{proposition}

\subsection{Actions of groups on themselves}
\subsubsection{Regular actions}
\begin{definition}
A group $G$ has a natural left action on the set $G$:
\[ G\times \operatorname{Field}(G) \to \operatorname{Field}(G): (g,h) \mapsto gh. \]
This action of $G$ is called the \udef{left-regular} group action.

Similarly, the natural right action on the set $G$ is called the \udef{right-regular} group action:
\[ \operatorname{Field}(G) \times G \to \operatorname{Field}(G): (h,g) \mapsto hg. \]
\end{definition}

\subsubsection{Conjugation}
Let $G$ be a group. The conjugation mapping
\[ \Ad: G\times \operatorname{Field}(G) \to \operatorname{Field}(G): (g,h) \mapsto \Ad_g(h) = g^{-1}hg \]
is a group action by $G$ on itself.

The orbits under this action are the conjugacy classes.

The stabiliser of $a\in G$ when $G$ is acting on itself by conjugation, is the \udef{centraliser} of $\{a\}$.





\section{Group action}
We have seen that symmetry transformations naturally form a group. Based on the concrete set of transformations that are symmetries we saw they form this abstract structure which we called a group. The advantage of working with this abstract entity is that it contains exactly the relevant details about the symmetry. We need not worry ourselves about the peculiarities of the particular system and we can easily make use of results others have obtained solving other problems.

Once we have thoroughly studied the symmetries of our system, we will want a way to move back from studying abstract groups to studying transformations of the system we are actually interested in.

Sometimes there is a natural correspondence between the set of group elements and the set of transformations. If this is the case the group can be interpreted as acting on the system in a \udef{canonical} (or natural) way.

\begin{example}
\begin{itemize}
\item Dihedral group $D_4$ acts quite naturally on a blanc, square piece of paper.
\item The symmetric group $\mathcal{S}_n$ of all permutations of a set of $n$ elements acts naturally on a set of $n$ elements.
\item The group of $n\times n$ matrices acts naturally on $n$-dimensional vectors through matrix multiplication.
\end{itemize}
\end{example}

In general the transition back may not be so clear, simple or natural. For instance there may be a subset of the $n$-dimensional vectors with a symmetry group isomorphic to $D_4$. To what transformations do these group elements correspond? We cannot just rotate and flip these vectors. It is for understanding these cases that the concept of a \udef{group action} is useful.

\subsection{Definition}
We start with a group $G$ and a set $X$. The set $X$ is frequently the set of configurations of the system and thus transformations of the system are functions of the type $f:\,X\to X$; to keep things general, we only assume we have set and we are agnostic as to its origins.
A group action quite simply associates a transformation of the set to every element of the group.

We do however require that this association has some fairly natural features, so that the nature and essence of the group is not lost in transition: the group action must respect the identity element and and group operation. This leads us to the following definition:
\begin{definition}
Let $G$ be a group and $X$ a set, then a \udef{(left) group action $\varphi$} of $G$ on $X$ is a function
\[ \varphi: \, G\times X \to X: \, (g,x)\mapsto \varphi(g,x) = g\cdot x \]
with the properties:
\begin{enumerate}
\item For the identity element $e$ and all $x\in X$: 
\[e\cdot x = x \]
\item For all $g,h \in G$ and $x\in X$:
\[ (gh)\cdot x = g\cdot (h\cdot x) \]
\end{enumerate}
Notice that we have introduced the notation $g\cdot x$ meaning apply the transformation attributed to $g$ through the group action to the element $x$.
\end{definition}

The above definition is for a \textit{left} group action. We can analogously define a right group action. The only difference between the two is that in the right group action in the transformation
\[ x \cdot (gh) = (x\cdot g)\cdot h \]
the transformation associated with $g$ gets applied first. Using the formula $(gh)^{-1} = h^{-1}g^{-1}$ we can always construct a left group action from a right one and vice versa, so typically we only consider left group actions.

An important property is immediately apparent from the definition:
\begin{eigenschap}
The transformation associated with $g$ (i.e.\ $x\mapsto g\cdot x$) is always a bijection because the inverse is given by $x \mapsto g^{-1}\cdot x$.
\end{eigenschap}

\subsection{Types of action}
What follows is simply an enumeration of some properties group actions may have. The action of $G$ on $X$ is called
\begin{enumerate}
\item \udef{transitive} if $X$ is non-empty and for each $x,y$ in $X$ there exists a $g \in G$ such that $g\cdot x = y$.
\item \udef{faithful} if for every distinct $g,h$ in $G$ there exists an $x \in X$ such that $g\cdot x \neq h\cdot x$. In other words the mapping of elements of $G$ to transformations of $X$ is 1-to-1 or injective.
\end{enumerate}

\subsection{Orbits and stabilizers}
\begin{definition}
Consider a group $G$ acting on a set $X$. The \udef{orbit} of an element $x$ of $X$ is denoted $G\cdot x$.
\[ G\cdot x = \{ g\cdot x | g\in G \}. \]
\end{definition}

The \udef{stabilizer subgroup} of $G$ with respect to an element $x$ of $X$ is the set of all elements in $G$ that fix $x$ and is denoted $G_x$.
\[G_x = \{ g\in G | g\cdot x = x \} \]

\subsection{Continuous group action}
A continuous group action on a topological space $X$ is a group action of a topological group $G$ that is continuous: i.e.\,
\[G \times X \to X : \;(g, x) \mapsto g \cdot x \]
is a continuous map.

This is the proper type of group action to use with topological groups, if their topologicalness is relevant and to be preserved.

\subsection{Representations}
If the group action is the action of a group on a vector space such that the transformations the group elements are mapped to are linear transformations, we call this group action a \udef{representation}.

\begin{definition}
A \udef{representation} of a group $G$ on an $n$-dim vector space $V$ is a mapping of the elements of $G$ to the set of invertible linear operations acting on $V$:
\[D: G \rightarrow GL(V): g \mapsto D(g)\]
Such that
\begin{itemize}
\item $D(e) = \mathbb{1}_V$
\item $D(g_1\cdot g_2) = D(g_1)D(g_2) = D(g_3)$
\end{itemize}
\end{definition}

\begin{example}
\begin{itemize}
\item Representations of $Z_3 = \{e,\omega, \omega^2\} \qquad (\omega = e^{i2/3\pi})$
\begin{itemize}
\item Trivial representations
\[D(e) = D(\omega) = D(\omega^2) = \mathbb{1}_V\]
\item Representation $\GL(1, \mathbb{C})$
\[ D(e) = 1, \quad D(\omega) = e^{i\frac{2}{3}\pi} , \qquad D(\omega^2) = e^{i\frac{1}{3}\pi} \]
\item Regular representation:
\[D(e) = 
\begin{pmatrix}
1&0&0\\0&1&0\\0&0&1
\end{pmatrix}, \qquad D(\omega) = \begin{pmatrix}
0&0&1\\1&0&0\\0&1&0
\end{pmatrix}, \qquad D(\omega^2) = \begin{pmatrix}
0&1&0\\0&0&1\\1&0&0
\end{pmatrix}
\]
In general a we can define a regular representation for any finite group $G$ as follows: Let $V$ be a vector space with basis $e_t$ indexed by the elements of $G$, $t \in G$. The mapping $D: e_t \mapsto e_{ts}$ defines the \udef{(left) regular representation} of $G$. This notion can be extended to groups of infinite order.
\end{itemize}
\item The standard representation of a subgroups $H$ of $\GL(n,\C)$ on the vector space $\C^n$ is given by the inclusion:
\[ D: H \to \GL(\C^n) = \GL(n,\C): h \mapsto h \]
\end{itemize}
\end{example}


\begin{definition}
Two representations are \udef{equivalent} if there exists a linear operator $S$ such that
\[D(g) \mapsto D'(g) = S^{-1}D(g)S\]
In other words there exists a similarity transformation $S$
\end{definition}

\begin{definition}
A representation is \udef{unitary} if $\forall g \in G$
\[D(g)D^\dagger(g) = D^\dagger(g)D(g) = \mathbb{1}_V\]
\end{definition}

\begin{definition}
Consider a \undline{representation $D$} of a \undline{group $G$} on a \undline{vector space $V$}
\begin{enumerate}
\item A subspace $W$ of $V$ is called \udef{invariant} if $D(g)w$ is in $W$ for all $w \in W$ and all $g \in G$. An invariant subspace $W$ is called nontrivial if $W\neq\{0\}$ and $W \neq V$.
\item We call $D$ \udef{reducible} if there exists a nontrivial subspace $U$ of $V$ that is invariant under $D$.
\item $D$ is \udef{irreducible} if the only subspaces invariant under all elements of the image of $D$ are $\emptyset$ and $V$
\item $D$ is \udef{completely reducible} if we can decompose $V$ into invariant subspaces:
\[V = U_1\oplus U_2 \oplus \ldots \oplus U_n\]
There then exists a similarity transformation such that
\[\forall g: D(g) = \begin{pmatrix}
D_1(g) & 0 & \dots & 0\\
0 & D_2(g)  & \dots & 0\\
\vdots & & \ddots & \vdots\\
0&0&\dots&D_n(g)
\end{pmatrix}\qquad \text{with}\quad D \equiv D_1\oplus D_2 \oplus \ldots \oplus D_n\]
\end{enumerate}
\end{definition}

\begin{example}
The regular representation of $Z_3$ is completely reducible. The linear operators $D(e), D(\omega)$ and $D(\omega^2)$ have eigenvalues $1,\omega, \omega^2$ with eigenvectors 
\[ \begin{pmatrix}
1\\1\\1
\end{pmatrix}\, \qquad \begin{pmatrix}
1\\ \omega^2 \\ \omega
\end{pmatrix} \qquad \text{and} \qquad \begin{pmatrix}
1 \\ \omega \\ \omega^2
\end{pmatrix}. \]
Each eigenvector generates an invariant subspace. We can then apply the following coordinate transformation
\[ S = \frac{1}{\sqrt{3}}\begin{pmatrix}
1&1&1\\
1&\omega^2 & \omega \\
1&\omega& \omega^2
\end{pmatrix} \]
in order to get the following matrices
\[D'(e) = \begin{pmatrix}
1&0&0\\
0&1&0\\
0&0&1
\end{pmatrix}, \qquad D'(\omega) = \begin{pmatrix}
1 & 0 & 0\\
0& \omega & 0 \\
0&0&\omega^2
\end{pmatrix}, \qquad D'(\omega^2) = \begin{pmatrix}
1 & 0 & 0\\
0& \omega^2 & 0 \\
0&0&\omega
\end{pmatrix}\]
\[ D' = D_1\oplus D_2 \oplus D_3 = \diag\{1,1,2\}\oplus \diag\{1,\omega, \omega^2\} \oplus \diag\{1, \omega^2, \omega\} \]
\end{example}

\subsubsection{Projective representations}
Bargmann theorem



\section{Topological groups}
A group is a set with an extra structure layered on top: the group operation that satisfies the group axioms. A topological space is also a set with an extra structure layered on top: the topology, as discussed in a previous part. Now here's a novel idea: let's layer both of these structures on a set at once. This gives no new mathematics because the two structures do not interact in any way; in order for interesting things to occur, we must pose some additional requirements.

\begin{definition}
A \udef{topological group} $G$ is a topological space that is also a a group such that the group operations of
\begin{enumerate}
\item product
\[ G\times G \to G: \, (x,y)\mapsto xy \]
\item and taking inverses
\[ G\to G: \, x\mapsto x^{-1} \]
\end{enumerate}
are \textbf{continuous}.
\end{definition}
TODO also need that points are closed?

\begin{lemma}
The continuity of the product and inverse is equivalent to the continuity of $G\times G \to G: (s,r)\mapsto sr^{-1}$.
\end{lemma}
TODO; reframe as criterion?

\begin{lemma}
Let $G$ be a topological group. The following are homeomorphisms:
\begin{enumerate}
\item $G\to G: s\mapsto s^{-1}$;
\item $G\to G: s\mapsto rs$ for any $r\in G$.
\end{enumerate}
\end{lemma}
An important consequence of this is that the topology of $G$ is determined by the topology near the identity $e$.

Topological groups are also sometimes called continuous groups.



\section{Grothendieck group}
Given a commutative monoid $M$, the Grothendieck group $G(M)$ is the ``most general'' Abelian group that arises from $M$. Intuitively it is formed by adding additive inverses for all elements of $M$.



 
TODO Grothendieck construction for Abelian monoids: $G(M)$.
Universality, functoriality

Cancellation property: simplified construction.

Grothendieck map $M\to G(M)$ is injective \textup{if and only if} $M$ has cancellation.

\subsection{The integers}
\begin{definition}
$\Z$
\end{definition}


\section{Ordered groups}

\begin{lemma}
Let $(G,+,\leq)$ be an ordered group and $x,y\in G$. Then
\[ (\forall \varepsilon > 0: x< y+\varepsilon) \implies x\leq y. \]
\end{lemma}
\begin{proof}
The proof is by contraposition. Assume $x>y$, then we can take $\varepsilon = x-y>0$. This implies $x = y+\varepsilon$ and so $x \geq y+\varepsilon$. 
\end{proof}


\chapter{Rings}
TODO: signature $\sSet{R,+,\cdot, 0, 1}$.
TODO: addition, multiplication and scalar multiplication of functions: pointwise.

TODO: unital homomorphisms; unital subalgebra.

\begin{lemma}
Let $R$ be a ring and $a\in R$.
\begin{enumerate}
\item If $a$ has a left and a right inverse, they are equal. Thus $a$ has an inverse.
\item The inverse of $a$ is unique, if it exists.
\end{enumerate}
\end{lemma}
\begin{proof}
Let $l$ be a left inverse of $a$ and $r$ a right inverse. Then
\[ l = l(ar) = (la)r = r. \]
The unicity of the inverse is an easy consequence.
\end{proof}

\begin{lemma} \label{productInvertibility}
Let $R$ be a ring and $a,b\in R$. Then $a$ and $b$ are invertible \textup{if and only if} $ab$ and $ba$ are invertible.
\end{lemma}
\begin{proof}
Assume $a,b$ invertible. Then $b^{-1}a^{-1}$ is an inverse for $ab$ and $a^{-1}b^{-1}$ is an inverse for $ba$.

Assume both $ab$ and $ba$ have inverses. Then from
\begin{align*}
a[b(ab)^{-1}] &= \vec{1} & [(ba)^{-1}b]a &= \vec{1} \\
[(ab)^{-1}a]b &= \vec{1} & b[a(ba)^{-1}] &= \vec{1}
\end{align*} 
we see that both $a$ and $b$ have left and right inverses.
\end{proof}

\begin{proposition} \label{everyProperIdealInMaximalIdeal}
Every proper ideal is contained in a maximal ideal.
\end{proposition}

\begin{lemma} \label{nonInvertibleGeneratedIdeals}
Let $R$ be a unital ring. If $a\in R$ is non-invertible, then the generated ideal $(a)$ is not the whole ring.
\end{lemma}
\begin{proof}
If $(a) = R$, then $1\in (a)$, implying $ab=1$ for some $b\in R$. A contradiction.
\end{proof}

\begin{proposition}
Let $f$ be a ring homomorphism. If $f$ is invertible as a function (i.e. bijective), its inverse $f^{-1}$ is also a ring homomorphism.
\end{proposition}

\begin{proposition} \label{kernelIsIdeal}
Kernel of Ring Homomorphism is Ideal
\end{proposition}

\begin{definition}
A $*$-rng is a structured set $(R,+,\cdot, *)$, where $R$ is a rng and $*:R\to R$ is an involutive anti-automorphism. That is, $\forall x,y\in R$:
\begin{itemize}
\item $(xy)^* = y^*x^*$;
\item $(x+y)^* = x^* + y^*$;
\item $(x^*)^* = x$.
\end{itemize}
This is also known as an \udef{involutive rng} or \udef{rng with involution}.
\end{definition}
\begin{lemma}
If $R$ is a unital ring with involution, then $1^* = 1$.
\end{lemma}
\begin{proof}
From $1^*x = (x^*1)^* = (x^*)^* = x$, we see that $1^*$ is a multiplicative identity, which is unique.
\end{proof}
\begin{definition}
An element of a $*$-rng is \udef{self-adjoint} if $x^* = x$.
\end{definition}

\section{Group rings}
\begin{definition}
Let $G$ be a finite group and $R$ a r(i)ng. The \udef{group ring} $RG$ is the set of functions $(G\to R)$ with pointwise addition and the convolution product
\[ (x\star y)(g) = \sum_h x(h)y(h^{-1}g) = \sum_{g=hk}x(h)y(k) \]
for all $x,y\in RG$ and $g\in G$. 
\end{definition}
The a group ring can be seen as a free module generated by $G$. (TODO: this as definition?)



\chapter{Fields}
\section{Totally ordered fields}
\begin{definition}
Let $K$ be a set with binary operations $+,\cdot$ such that $(K,+,\cdot)$ is a field and a binary relation $\leq$ such that $(F,\leq)$ is a total order. Then the structured set $(K,+,\cdot,\leq)$ is a \udef{totally ordered field} if $\forall a,x,y\in K$:
\begin{enumerate}
\item $x\leq y \implies x+a \leq y+a$;
\item $x\leq y \land a\geq 0 \implies ax \leq ay$;
\item $x\leq y \land a\leq 0 \implies ax \geq ay$.
\end{enumerate}
\end{definition}

\begin{lemma}
Let $(K,+,\cdot,\leq)$ be a totally ordered field and $a,b,c,d\in K$. Then
\begin{enumerate}
\item $a\leq b \land c\leq d \implies (a+c) \leq (b+d)$;
\item $a\leq b \implies -b\leq -a$;
\item $a\geq 0 \iff -a\leq 0$;
\item $a\geq 0 \land b\geq 0 \implies ab \geq 0$;
\item $a\geq 0 \implies a^n \geq 0$ for all $n\in\N$;
\item $a\leq 0 \implies (a^{2n} \geq 0 \land a^{2n+1}\leq 0)$ for all $n\in\N$;
\item $a > 0 \iff a^{-1} > 0$ and $a < 0 \iff a^{-1} < 0$;
\item if $b \geq a>0$, then $b^{-1} \leq a^{-1}$;
\item $0<1$.
\end{enumerate}
\end{lemma}
\begin{proof}
(1) By applying point 1. of the definition twice, we get $(a+c)\leq(b+c)\leq(b+d)$.

(2) This is the result of multiplying by $-1$.

(3) Idem, using $-0=0$.

(4) Special case of point 2. of the definition.

(5) By induction on $n$ and point 2. of the definition.

(6) By induction on $n$ and point 3. of the definition.

(7) By multiplying $a \geq 0$ with $(a^{-1})^{2}$, which is positive, we get $a^{-1}\geq 0$. Also $a\neq 0 \iff a^{-1}\neq 0$.

(8) By point 7. and point 4. of the lemma $a^{-1}b^{-1}$ is positive, so multiplying $b\geq a$ by $a^{-1}b^{-1}$ yields $b^{-1} \leq a^{-1}$.

(9) Assume, towards a contradiction, that this is false, so $1\leq 0$. Then $1$ is negative and multiplying the inequality with $1$ yields $1\cdot 1 \geq 1\cdot 0$, or $1\geq 0$. Taking both inequalities gives $1=0$ by anti-symmetry, which is prohibited for fields.
\end{proof}

\begin{definition}
Let $F\subset K$ be totally ordered fields. Then we call $F$ \udef{dense} in $K$ if
\[ \forall a,b\in K: \exists x\in F: a<x<b. \]
\end{definition}
TODO: topology definition?

\chapter{Valuation theory}
Absolute values on integral domains.