\chapter{Ordinary differential equations}
\url{https://www.mat.univie.ac.at/~gerald/ftp/book-ode/ode.pdf}
\url{file:///C:/Users/user/Downloads/978-3-030-47849-0.pdf}
\url{file:///C:/Users/user/Downloads/Polyanin%20A.,%20D.,%20Zaitsev%20V.%20F.,%20Handbook%20of%20exact%20solutions%20for%20ordinary%20differential%20equations.pdf}

\section{Classification}
\begin{definition}
An \udef{$n^\text{th}$ order (ordinary) differential equation} (or ODE) is an equation of the form
\[ F(t, u, u', u^{\prime\prime}, \ldots, u^{(n)}) \equiv 0. \]
We call a real function $u: ]a,b[ \to \R$ a \udef{solution} of this differential equation on $]a,b[$
if it has at least $n$ continuous derivatives such that $F(t, u(t), u'(t), u^{\prime\prime}(t), \ldots, u^{(n)}(t))$ is zero for all $t\in ]a,b[$. The set of all solutions is called the \udef{general solution}.

We call the differential equation
\begin{enumerate}
\item \udef{linear} if $\md{F}{2}{u^{(i)}}{}{u^{(j)}}{} \equiv 0$ for all $i,j\in [0,n]$;
\item \udef{homogenous} if $F(t, 0,\ldots, 0) = 0$.
\end{enumerate}
Unless explicitly stated, we will always assume that $F$ can be solved for the highest derivative, such that the ODE can be written as
\[ u^{(n)} \equiv f(t,u,u',\ldots, u^{(n-1)}). \]
\end{definition}

A linear $n^\text{th}$ order differential equation can be written in the form
\[ \sum_{i=0}^n a_{i}(t)u^{(i)}(t) - g(t) = 0 \]
where $a_i, g \in (]a,b[\to\R)$.

\begin{definition}
Let $\sum_{i=0}^n a_{i}(t)u^{(i)}(t) \equiv g(t)$ be a linear differential equation. We call the functions $a_i$ the \udef{coefficients} and we say the ODE has \udef{constant coefficients} if all the $a_i$ are constants.
\end{definition}

In the linear case we can introduce the linear operator
\[ L \defeq \sum_{i=0}^n a_{i} \left(\dod{}{t}\right). \]
Then the differential equation can be written as $Lu = g$.

The differential equation is homogenous if and only if $g = 0$.

\subsection{Differential problems}
\subsubsection{Initial value problems}
\begin{definition}
An \udef{initial value problem} (IVP) is an $n^{\text{th}}$ order ODE together with \udef{initial conditions}
\[ u^{(i)}(t_0) = c_i \qquad i \in 0:n-1 \]
where $t_0$ and the $c_i$ are real numbers.
\end{definition}

TODO: replace derivatives with Lipschitz continuity.
\begin{theorem}
Let $u^{(n)} \equiv f(t,u,u',\ldots, u^{(n-1)})$ be an $n^{\text{th}}$ order ODE with initial conditions
\[ u^{(i)}(t_0) = c_i \qquad i \in 0:n-1. \]
Assume
\[ f, \pd{f}{t}, \pd{f}{u}, \pd{f}{u'}, \ldots, \pd{f}{u^{(n-1)}} \]
are defined and continuous on a neighbourhood of $(t_0, c_0, \ldots, c_{n-1})\in \R^{n+1}$.

Then there exists $\epsilon > 0$ such that the IVP has a unique solution on the interval $]t_0-\epsilon, t_0+\epsilon[$.
\end{theorem}
\begin{proof}
TODO
\end{proof}

It is possible for a solution $u$ to be defined outside the interval $]t_0-\epsilon, t_0+\epsilon[$, but not be a solution to the IVP.
\begin{example}
Consider the IVP
\[ u' = u^2 \qquad u(0) = c. \]
The function
\[ u: \R\setminus\{1/c\} \to \R: t\mapsto \frac{c}{1-ct} \]
is the solution only for $t < 1/c$.
\end{example}

\subsubsection{Boundary value problems}
\begin{definition}
An \udef{initial value problem} (IVP) is an $n^{\text{th}}$ order ODE together with \udef{boundary conditions}
\[ u(t_i) = c_i \qquad i \in 0:n-1 \]
where the $t_i$ are real numbers.
\end{definition}

The questions of existence and uniqueness are less clear than for IVPs.
\begin{example}
Consider the ODE
\[ u''(t) + u(t) = 0 \qquad 0<t<\pi. \]
The general solution is $u(t) = c_1\sin t + c_2 \cos t$. All solutions have $u(0) = u(\pi)$.
So the BVP with boundary conditions
\[ u(0) = 0 \qquad u(\pi) = 1 \]
has no solution and the BVP with boundary conditions
\[ u(0) = 0 \qquad u(\pi) = 0 \]
has infinitely many solutions.
\end{example}

\paragraph{}

\subsection{Well- and ill-posed problems.}
\begin{definition}
Let $X,Y$ be spaces of functions and let $T: X\not\to Y$ be a partial function.

A differential problem $T(u) = f$ is said to be \udef{well-posed} if
\begin{itemize}
\item a solution $u$ exists for all $f\in Y$;
\item the solution is unique in $X$;
\item the solution depends continuously on $f$.
\end{itemize}
If the problem is not well-posed, it is \udef{ill-posed}.
\end{definition}

\begin{lemma}
Let $T$ be a linear operator between function spaces and $\lambda \in \C$. Consider the differential problem
\[ \lambda u - Tu = f. \]
\begin{enumerate}
\item If $\lambda\in \rho(T)$, then it is well-posed.
\item If $\lambda\in\sigma_\text{p}(T)$, then uniqueness fails.
\item If $\lambda\in\sigma_\text{c}(T)$ or $\sigma\in \lambda\in\sigma_\text{r}(T)$, then existence fails for some $f$.
\end{enumerate}
\end{lemma}

\subsection{Solutions of ODEs}
Classical, weak and distributional solutions.

\subsection{Systems of differential equations}

\section{Existence and uniqueness}
\subsection{Existence}
\subsubsection{Picard-Lindelöf theorem}
\begin{theorem}[Picard-Lindelöf]
Consider $f: \R\times \R^d \to \R^d$ and the differential problem of finding $y:\R\to \R^d$ such that
\[ y' = f(t,y) \qquad y(t_0) = y_0 \qquad (t_0\in \R, y_0\in\R^d). \]
Let $R$ be the rectangle $[t_0,t_0+a]\times B(y_0, b)$ for some $a,b\in \R$.
If $f$ is continuous on $R$  and uniformly Lipschitz continuous w.r.t. $y$, then the differential problem has a unique solution $y(t)$ on $[t_0,t_0+\alpha]$, where $\alpha = \min(a,b/M)$ and $M$ is a bound for $|f(t,y)|$ on the rectangle $R$.
\end{theorem}
Any norm on $\R^d$ can be used to define $B(y_0, b)$, as they are all equivalent.
\begin{proof}
For any potential solution $y(t)$, the function $t\mapsto f(t,y(t))$ is integrable (TODO ref). So the solution must satisfy \[ y(t) = y_0 + \int_{t_0}^tf(s, y(s))\diff{s}. \]
We use this the construct a sequence of approximate solutions. Set $y_0: t\mapsto y_0$ and
\[ y_{n+1}: [t_0, t_0+a] \to \R^d: t\mapsto y_0 + \int_{t_0}^t f(s,y_n(s))\diff{s}. \]
These integrals can be taken because the graph of each $y_n$ lies in the rectangle $R$:
\[ |y_n(t) - y_0| \leq \int_{t_0}^t|f(s,y_{n-1}(s)|\diff{s} \leq M\alpha \leq b. \]

TODO
\end{proof}

\subsubsection{Peano's existence theorem}
\begin{theorem}[Peano's existence theorem]
Consider $f: \R\times \R^d \to \R^d$ and the differential problem of finding $y:\R\to \R^d$ such that
\[ y' = f(t,y) \qquad y(t_0) = y_0 \qquad (t_0\in \R, y_0\in\R^d). \]
Let $R$ be the rectangle $[t_0,t_0+a]\times B(y_0, b)$ for some $a,b\in \R$.

If $f$ is continuous on $R$ and $|f(t,y)|$ is bounded on $R$ with bound $M$, then the differential problem has at least one solution $y(t)$ on $[t_0,t_0+\alpha]$.
\end{theorem}
Any norm on $\R^d$ can be used to define $B(y_0, b)$, as they are all equivalent.
\begin{proof}
TODO
\end{proof}

\subsection{Differential inequalities}
\subsection{Dependence on initial conditions and parameters}

\section{First order differential equations}
\subsection{Existence and uniqueness}
\subsection{Qualitative properties of solutions}
\subsection{A miscellany of solutions for different types of equations}
\subsubsection{Separable equations}
\paragraph{The logistic equation.}
\subsubsection{Exact equations}
\subsubsection{Linear first order differential equations}
\subsubsection{Homogeneous equations}
\subsubsection{Bernoulli equations}

\section{Systems of equations and higher order equations}
\subsection{Problem statement and notation}
Equivalence systems and higher order
\subsection{Existence and uniqueness}
\subsection{Second order equations}
\subsection{Higher order linear equations}
\subsection{Systems of first order equations}

\section{Qualitative analysis}

\section{Solutions by infinite series and Bessel functions}

\section{Sturm Liouville eigenvalue theory}

\chapter{Partial differential equations}
Transport equation, Laplace's equation, Heat equation, Wave equation
\section{Classification}
\begin{definition}
An \udef{$n^\text{th}$ order partial differential equation} (or PDE) is an equation of the form
\[ F(x, \{D^\alpha u\}_{|\alpha|\leq m}) \equiv 0. \]
We call a  function $u: \Omega \subset \R^N \to \R$ a \udef{solution} of this differential equation on $\Omega$ if $D^\alpha u$ exists and is continuous for $|\alpha|\leq m$ and $F(x, \{D^\alpha u(x)\}_{|\alpha|\leq m})$ is zero for all $x\in \Omega$.

The set of all solutions is called the \udef{general solution}.

We call the differential equation
\begin{enumerate}
\item \udef{linear} if $\md{F}{2}{(D^\alpha u)}{}{(D^\beta u)}{} \equiv 0$ for $|\alpha|,|\beta|\leq m$;
\item \udef{homogenous} if $F(x, 0,\ldots, 0) = 0$.
\end{enumerate}
\end{definition}

\begin{lemma}
A PDE is linear \textup{if and only if} it can be written in the form
\[ Lu(x) = \sum_{|\alpha|\leq m}a_\alpha(x)D^\alpha u(x) = g(x). \]
A linear PDE is homogeneous \textup{if and only if} $g = 0$.
\end{lemma}

\subsection{Elliptic, Hyperbolic and Parabolic PDEs}