\url{https://en.wikipedia.org/wiki/Cauchy_space}

\url{https://en.wikipedia.org/wiki/Cauchy-continuous_function}

\url{https://en.wikipedia.org/wiki/Proximity_space}

\url{https://www.bioinf.uni-leipzig.de/~studla/Publications/PREPRINTS/01-pfs-007-subl1.pdf}

\chapter{Convergence}
\section{Convergence spaces}
Intuition: directed sets and refinement.

\begin{definition}
Let $X$ be a set and let $\xi$ be a relation between non-empty downward directed sets in $\powerset(X)$ and elements of $X$. Then
\begin{itemize}
\item we denote the image function of $\xi$ as $\lim_\xi: \directed(\powerset(X)) \to \powerset(X): F\mapsto F\xi$; we call $\lim_\xi F$ the \udef{$\xi$-limit} of $F$;
\item $\xi$ is called a \udef{preconvergence} on $X$ if $\lim_\xi$ is order-preserving when $\directed(\powerset(X))$ is ordered by refinement:
\[ F \preceq G \implies \lim_\xi F \subseteq \lim_\xi G; \]
\item $\xi$ is called a \udef{convergence} if it is a preconvergence and it is \udef{centered}:
\[ \forall x\in X: \quad x\in \lim_\xi \big\{\{x\}\big\}. \]
\end{itemize}
If $\xi$ is a convergence, then we call $\sSet{X, \xi}$ a \udef{convergence space}.

If $\lim_\xi F \neq \emptyset$, we say the directed set $F$ \udef{converges}.
\end{definition}
We write ${\lim_\xi}^{-1}$ for the preimage function of $\xi$ restricted to directed sets $\nsim \powerset(X)$. Thus ${\lim_\xi}^{-1}(x)$ is the set of all directed sets $D$ such that
\begin{itemize}
\item $D\overset{\xi}{\longrightarrow} x$;
\item $D \nsim \powerset(X)$.
\end{itemize}

\begin{lemma}
Let $X$ be a set and $x\in X$. A preconvergence $\xi$ on $X$ is a convergence \textup{if and only if} $\forall x\in X: \{x\} \in {\lim_\xi}^{-1}(x)$.
\end{lemma}

\begin{lemma}
Let $X$ be a set, $\xi$ a convergence on $X$ and $x\in X$. Then $\lim_\xi^{-1}(x)$ is upwards closed.
\end{lemma}

\subsection{Lattices of preconvergences and convergences}
\begin{definition}
Let $X$ be a set and $\xi,\zeta$ preconvergences on $X$. We say $\xi$ is \udef{finer} (or \udef{stronger}) than $\zeta$, denoted $\xi \leq \zeta$, if $\lim_\xi F \subseteq \lim_\zeta F$ for all $F\in\powerfilters(X)$. We also say $\zeta$ is \udef{coarser} (or \udef{weaker}) than $\xi$.
\end{definition}
We can think of strength as the ability to stop a filter converging. So $\xi$ is strictly stronger that $\zeta$ if there are filters that converge in $\zeta$, but not in $\xi$. 

\begin{lemma}
Let $X$ be a set and $\xi$ a preconvergence on $X$. Then
\begin{enumerate}
\item $E_{\powerfilters(X), X} \leq \xi \leq U_{\powerfilters(X), X}$;
\item $\xi$ is a convergence \textup{if and only if} $\incl_X \leq \xi$.
\end{enumerate}
\end{lemma}

\begin{definition}
Let $X$ be a set.
\begin{itemize}
\item The \udef{empty preconvergence} on $X$ is $E_{\powerfilters(X), X}$, i.e.\ the limit of all filters is the empty set.
\item The \udef{discrete convergence} $\iota_X$ on $X$ is defined by $x \in \lim_\iota F \iff F = \pfilter{x}$ if $F$ is a proper filter. If $F = \powerset(X)$, then $\lim_\iota F = X$.
\item The \udef{chaotic convergence} on $X$ is $U_{\powerfilters(X), X}$, i.e.\ the limit of all filters is $X$.
\end{itemize}
\end{definition}

\begin{proposition} \label{latticeConvergences}
Let $X$ be a set and and let $\Xi$ be a set of (pre)convergences on $X$. The sets of preconvergences and of convergences are complete bounded lattices and for all $F \in \powerfilters(X)$:
\[ \lim_{\bigwedge \Xi} F = \bigcap_{\xi\in\Xi}\lim_\xi F \quad\text{and}\quad \lim_{\bigvee \Xi} F = \bigcup_{\xi\in\Xi}\lim_\xi F. \]
\begin{itemize}
\item The top of both is the chaotic convergence.
\item The bottom of the lattice of preconvergences is the empty preconvergence.
\item The bottom of the lattice of convergences is the discrete convergence.
\end{itemize}
\end{proposition}

\begin{lemma}
A preconvergence is a convergence \textup{if and only if} it is coarser than the discrete convergence.
\end{lemma}


\subsection{Filters and convergence}
\begin{lemma}
Let $X$ be a set and $\xi$ a convergence on $X$. Let $A,B\in \directed(X)$ be downward directed sets. If $A \approx B$, then $\lim_\xi A = \lim_\xi B$.
\end{lemma}
\begin{proof}
We have $A \preceq B$ and $B \preceq A$, so $\lim_\xi A \subseteq \lim_\xi B$ and $\lim_\xi B \subseteq \lim_\xi A$.
\end{proof}
This lemma means we can view $\lim_\xi$ as a function on the quotient $\directed(X) /\approx$, which is order isomorphic to $\filters(X)$ (indeed there is one filter in each equivalence class in $\directed(X) /\approx$ and for filters the refinement relation simplifies to inclusion).

Consequently we will usually think of a convergence on a set $X$ as a relation between filters on $\powerset(X)$ and elements of $X$. The axioms then become
\begin{itemize}
\item $\xi$ is a preconvergence on $X$ if $\lim_\xi$ is order-preserving when $\powerfilters(X)$ is ordered by inclusion:
\[ F \subseteq G \implies \lim_\xi F \subseteq \lim_\xi G; \]
\item $\xi$ is called a convergence if it is a preconvergence and it is centered:
\[ \forall x\in X: \quad x\in \lim_\xi \pfilter{x}. \]
\end{itemize}

Any convergence defined for filters uniquely extends to a convergence defined for all downward directed sets. 

\begin{lemma}
Let $X$ be a set and $\xi$ a preconvergence on $X$. Then
\begin{enumerate}
\item $\forall F,G \in \powerfilters(X): \quad \lim_\xi(F\cap G) \subseteq \lim_\xi F \cap \lim_\xi G$;
\item $\forall F,G \in \powerfilters(X): \quad \lim_\xi(F\cup G) \supseteq \lim_\xi F \cup \lim_\xi G$.
\end{enumerate}
\end{lemma}
\begin{proof}
Reformulation of \ref{orderPreservingFunctionLatticeOperations}, because a preconvergence is order-preserving.
\end{proof}
\begin{corollary} \label{limitDegenerateFilter}
For any convergence $\xi$ on $X$ we have $\lim_\xi \powerset(X) = X$.
\end{corollary}
\begin{proof}
Clearly $\pfilter{x} \subseteq \powerset(X)$ for all $x\in X$, so
\[ \lim_\xi(\powerset(X)) \supseteq \lim_\xi \left(\bigcup_{x\in X}\pfilter{x}\right) \supseteq X. \]
\end{proof}

In the sequel we will usually consider convergence as a property of filters, however sometimes it will be easier to consider downward directed sets (e.g.\ for continuity).

\subsubsection{Limit points}
\begin{definition}
Let $\sSet{X,\xi}$ be a convergence space and $x\in X$. Then
\begin{itemize}
\item $x$ is called a \udef{limit point} if there exists a filter $F\in\powerfilters(X)$ other than $\pfilter{x}$ that converges to $x$;
\item $x$ is called an \udef{isolated point} if it is not a limit point.
\end{itemize}
Let $A\subseteq X$ be a subset. Then $x$ is called a \udef{limit point of $A$} if there exists a filter $F\in\powerfilters(X)$ other than $\pfilter{x}$ that converges to $x$ and has $A\in F$.
\end{definition}

\begin{proposition}
Let $\sSet{X,\xi}$ be a convergence space and $x\in X$. The following are equivalent:
\begin{enumerate}
\item $x$ is a limit point;
\item $\vicinity_\xi(x) \neq \pfilter{x}$;
\item $x\in \adh_\xi\big(X\setminus\{x\}\big)$;
\end{enumerate}
\end{proposition}
\begin{proof}
TODO (??)
\end{proof}

\subsubsection{Approaches}
\begin{definition}
Let $\sSet{X,\xi}$ be a convergence space. An \udef{approach} on $X$ is a function $F: X\to \powerfilters(X)$ such that $F(x) \overset{\xi}{\longrightarrow} x$ for all $x\in X$.
\end{definition}

\begin{lemma}
Let $\sSet{X,\xi}$ be a convergence space. The function $X\to \powerfilters(X): x\mapsto \pfilter{x}$ is an approach.
\end{lemma}



\subsubsection{Convergence of nets}
\begin{definition}
Let $\sSet{X,\xi}$ be a convergence space $\seq{x_i}_{i\in I}$ a net on $X$ and $x\in X$. The net $\seq{x_i}_{i\in I}$ \udef{converges} to $x$ in $\xi$, denoted $\seq{x_i}_{i\in I} \overset{\xi}{\longrightarrow} x$, if the associated filter converges to $x$.
\end{definition}


\subsection{Depth requirements}
\begin{definition}
Let $\sSet{X,\xi}$ be a preconvergence space. Take arbitrary $F,G \in \powerfilters(X)$, $\mathcal{F}\subseteq \powerfilters(X)$ and $x\in X$. The preconvergence space is called
\begin{itemize}
\item a \udef{Kent space} or \udef{localised convergence space} if $F \to x \implies F\cap \pfilter{x} \to x$;
\item \udef{finitely deep}, a \udef{limit space}  or a \udef{limitierung} if $\lim_\xi(F\cap G) = \lim_\xi F \cap \lim_\xi G$;
\item a \udef{Choquet space} or a \udef{pseudotopological space} if
\[ F\overset{\xi}{\longrightarrow} x \qquad\iff\qquad \text{$U\overset{\xi}{\longrightarrow} x$ for all ultrafilters $U$ such that $F\subseteq U$.} \]
\item a \udef{pretopological space} if 
\[ \lim_\xi\Big(\bigcap \mathcal{F}\Big) = \bigcap \setbuilder{\lim_\xi(F')}{F'\in \mathcal{F}}. \]
\end{itemize}
\end{definition}

The point of a Kent space is that if $F\to x$, then ``essentially'' every $A\in F$ contains $x$.

\begin{lemma}
Let $\sSet{X,\xi}$ be a preconvergence space. If $\xi$ is a Kent space and each element of $x$ is a limit point, then it is a convergence space.
\end{lemma}
\begin{proof}
Take arbitrary $x\in X$. As $x$ is a limit point, there exists $F\to x$. Now $F\cap \pfilter{x} \to x$, so $\pfilter{x} \to x$ by monotonicity of the preconvergence.
\end{proof}

\begin{lemma} \label{finiteDepthLemma}
Let $X$ be a set and $\xi$ a preconvergence on $X$. Then the following are equivalent:
\begin{enumerate}
\item $\xi$ is finitely deep;
\item $\forall F,G \in \powerfilters(X): \quad \lim_\xi(F\cap G) \supseteq \lim_\xi F \cap \lim_\xi G$;
\item $\lim_\xi^{-1}(x)\cup \powerset(X)$ is a filter for all $x\in X$.
\end{enumerate}
\end{lemma}
In other words, a preconvergence is finitely deep \textup{if and only if} $\lim_\xi^{-1}(x)$ is \emph{directed} (and thus a filter) for all $x\in X$.

\begin{proposition} \label{depthImplications}
Let $X$ be a set and $\xi$ a preconvergence on $X$. Then each of the following statements implies the next:
\begin{enumerate}
\item $\xi$ is pretopological;
\item $\xi$ is Choquet;
\item $\xi$ is finitely deep.
\end{enumerate}
If $\xi$ is a convergence space, then these also imply
\begin{enumerate} \setcounter{enumi}{3}
\item $\xi$ is Kent.
\end{enumerate}
\end{proposition}
\begin{proof}
$(1) \Rightarrow (2)$ By \ref{filtersCoatomistic}, we have
\[ F = \bigcap_{\substack{U\in \powerultrafilters(X) \\ U\geq F}} U. \]
By the pretopological property this means
\[ \lim_\xi(F) = \lim_\xi\left(\bigcap_{\substack{U\in \powerultrafilters(X) \\ U\geq F}} U\right) = \bigcap_{\substack{U\in \powerultrafilters(X) \\ U\geq F}} \lim_\xi(U). \]
Thus $x\in \lim_\xi(F)$ iff $x\in \lim_\xi(U)$ for all ultrafilters $U$ such that $F\subseteq U$.

$(2) \Rightarrow (3)$ Take $F,G\in\powerfilters(X)$. By \ref{finiteDepthLemma} it is enough to prove $\lim_\xi(F\cap G) \supseteq \lim_\xi F \cap \lim_\xi G$. Take $x\in \lim_\xi F \cap \lim_\xi G$. If $U$ is an ultrafilter such that either $F\subseteq U$ or $G\subseteq U$, then $U\to x$.

Now take an ultrafilter $U' \supseteq F\cap G$. Then either $U'\supseteq F$ or $U'\supseteq G$ by \ref{finiteUltrafilterFactorisation}. By the previous remark $U'\to x$. By the Choquet property we conclude that $x\in \lim_\xi(F\cap G)$.

$(3) \Rightarrow (4)$. This is immediate if $\pfilter(x) \to x$ for all $x\in X$, i.e.\ if $\xi$ is a convergence.
\end{proof}

\begin{example}
There exist Choquet spaces that are not pretopological. Take any infinite set $X$. Then there exists a nonprincipal ultrafilter $U$ on $\powerset(X)$ (take the cofinite filter and extend to ultrafilter by ultrafilter lemma, see \ref{filterFreeIFFfinerThanCofinite}).

Take some $x_0$ and let each principal ultrafilter converge to $x_0$. Let no other ultrafilter converge. Define convergence of all other filters by the Choquet property. This convergence is not pretopological: We have $\bigcap \lim^{-1}(x_0) = \big\{\{X\}\big\}$, but $\{X\} \not\to x_0$ because $\{X\}\subseteq U$ and $U \not\to x_0$.
\end{example}

\begin{lemma} \label{pseudotopologicalConditions}
Let $\sSet{X, \xi}$ be a convergence space. Then the following are equivalent:
\begin{enumerate}
\item $\xi$ is pseudotopological;
\item $\forall U\in \powerultrafilters(X): \big( F\subseteq U \implies U \overset{\xi}{\longrightarrow} x\big) \quad\implies\quad F \overset{\xi}{\longrightarrow} x$;
\item $\displaystyle\lim_\xi F = \bigcap_{U\in \powerultrafilters(X)\land F \subseteq U}\lim_\xi U$ for all filters $F\in \powerfilters(X)$;
\item $\big({\lim}^{-1}(x)\big)^\mesh = \upset\mathfrak{I}^{\imf}\big({\lim}^{-1}(x)^\mesh\big)$ for all $x\in X$.
\end{enumerate}
\end{lemma}
\begin{proof}
$(1) \Leftarrow (2)$ One implication is immediate. The other is implied by monotonicity of the convergence.

$(1)\Leftrightarrow (3)$ Immediate.

$(1)\Leftrightarrow (4)$ By \ref{ChoquetModificationFilterSet}.
\end{proof}

TODO finitely deep = quotient of (pre)topologies; inf of topologies; prime topologies dense in it (?)



\section{The vicinity filter}
\begin{definition}
Let $X$ be a set, $\xi$ a preconvergence on $X$ and $x\in X$. The \udef{vicinity filter} of $\xi$ at $x$ is
\[ \vicinity_\xi(x) \defeq \bigcap \lim_\xi^{-1}(x) = \bigcap \setbuilder{F\in \powerfilters(X)}{x\in \lim_\xi F}. \]
A subset $V\subseteq X$ is called a \udef{vicinity} of $x$ for $\xi$ if $V\in \vicinity_\xi(z)$.

We extend the vicinity filters to be defined for elements of $\powerset(X)$ and $\powerset^2(X)$ by contours.
\end{definition}

\begin{lemma}
Let $X$ be a set, $\xi$ a preconvergence on $X$ and $A\subseteq X$. Then
\begin{align*}
\vicinity_\xi(A) &= \bigcap \setbuilder{F\in \powerfilters(X)}{\exists x\in A: \; x\in \lim_\xi F} \\
&= \bigcap \setbuilder{F\in \powerfilters(X)}{A \mesh \lim_\xi F}.
\end{align*}
\end{lemma}

\begin{lemma}
Let $\sSet{X,\xi}$ be a convergence space. Then $\vicinity_\xi$ is an approach \textup{if and only if} $\xi$ is pretopological.
\end{lemma}

\begin{lemma} \label{vicinityOfSetLemma}
Let $\sSet{X,\xi}$ be a convergence space and $A\subseteq X$. Then $\vicinity_\xi(A) \subseteq \upset\{A\}$.
\end{lemma}
\begin{proof}
We have $\vicinity_\xi(A) \subseteq \bigcap_{x\in A}\pfilter{x} = \upset \{A\}$. 
\end{proof}
\begin{corollary} \label{vicinityOfSetCorollary}
For $A\subseteq X$ and $x\in X$, we have
\begin{enumerate}
\item $A \subseteq \bigcap\vicinity_\xi(A)$;
\item $\vicinity_\xi(x) \subseteq \pfilter{x}$;
\item $\{x\}\in\vicinity_\xi(x)^{\mesh}$.
\end{enumerate}
\end{corollary}

\begin{lemma}
Let $X$ be a set, $\xi$ a preconvergence on $X$ and $x\in X$. Then
\[ \vicinity_\xi(x) = \bigcap_{F\in {\lim_\xi}^{-1}(x)\;\cap\; \ultrafilters(\powerset(X))}F. \]
\end{lemma}
\begin{proof}
TODO
\end{proof}

\begin{lemma} \label{vicinityMapAntitone}
Let $X$ be a set and $\zeta, \xi$ convergences on $X$. If $\zeta \leq \xi$, then $\vicinity_\zeta(x) \supseteq \vicinity_\xi(x)$ for all $x\in X$.
\end{lemma}
\begin{proof}
Assume $\zeta \leq \xi$. For all $F\in \powerfilters(X)$ we have $\lim_\zeta^{-1}(x) \subseteq \lim_\xi^{-1}(x)$, so
\[ \vicinity_\zeta(x) = \bigcap \lim_\zeta^{-1}(x) \supseteq \bigcap \lim_\xi^{-1}(x) = \vicinity_\xi(x). \]
\end{proof}


\subsection{Pretopological convergence}
A convergence space $\sSet{X,\xi}$ is pretopological if it $1$-paved, i.e.\ at each point $x$ there is a filter $G$ that converges to $x$ such that
\[ F\in {\lim}_\xi^{-1}(x) \implies G \leq F. \]

\begin{lemma} \label{pretopologyLemma}
Let $\xi$ be a convergence on a set $X$. Then the following are equivalent:
\begin{enumerate}
\item $\xi$ is a pretopology;
\item $\lim_\xi^{-1}(x)$ has a least element for all $x\in X$;
\item $\vicinity_\xi(x) \in \lim_\xi^{-1}(x)$ for all $x\in X$;
\item $\vicinity_\xi(x) \leq F \implies F\in\lim^{-1}_\xi x$;
\item $x\in\lim_\xi F \iff \vicinity_\xi(x) \leq F$.
\end{enumerate}
\end{lemma}
\begin{proof}
$(1) \Leftrightarrow (2)$. Because $G$ converging to $x$ is equivalent to $G\in \lim_\xi^{-1}(x)$, we see that $G$ is a least element of $\lim_\xi^{-1}(x)$.

$(2) \Leftrightarrow (3)$. The vicinity filter $\vicinity_\xi(x)$ is the infimum of $\lim_\xi^{-1}(x)$, and a set contains a least element iff it contains an infimum.

$(3) \Leftrightarrow (4)$. The set $\lim^{-1}_\xi x$ is upwards closed.

$(4) \Leftrightarrow (5)$. The opposite implication to (4) is immediate because $\vicinity_\xi(x)$ is the infimum of $\lim^{-1}_\xi$.
\end{proof}
Thus in a pretopological convergence space we can determine whether $x$ is in the limit of a filter $F$ by simply comparing $F$ to the vicinity filter of $x$.

\begin{lemma} \label{pretopologicalNetConvergence}
Let $\sSet{X,\xi}$ be a pretopological convergence space, $x\in X$ and $\seq{x_i}_{i\in I}$ a net. Then $\TailsFilter\seq{x_i} \overset{\xi}{\longrightarrow} x$ \textup{if and only if}
\[ \forall A\in \vicinity_\xi(x): \exists i_0\in I: \forall i \geq i_0: \quad x_i\in A. \]
\end{lemma}
\begin{proof}
By \ref{pretopologyLemma}, $\TailsFilter\seq{x_i} \overset{\xi}{\longrightarrow} x$ is equivalent to $\vicinity_\xi(x) \subseteq \TailsFilter\seq{x_i} = \upset \Tails\seq{x_i}$. Unpacking definitions gives the statement.
\end{proof}

\begin{lemma} \label{filterFunctionToPretopology}
Let $X$ be a set and $\mathcal{F}:X\to \powerfilters(X)$ a filter-valued function. If $\mathcal{F}(x) \subseteq \pfilter{x}$ for all $x\in X$, then $\mathcal{F}$ is the vicinity filter of a pretopological convergence.
\end{lemma}
The converse is given by \ref{vicinityOfSetCorollary}.

\begin{proposition}
Let $\xi$ be a convergence on a set $X$. Then $\xi$ is a pretopological convergence \textup{if and only if} $\lim_\xi: \powerfilters(X)\to \powerset(X)$ is a complete meet-semilattice homomorphism, i.e.\
for all families of filters $\mathcal{F} \subseteq \powerset(\powerfilters(X))$,
\[ \lim_\xi \bigwedge_{F\in \mathcal{F}}F = \bigcap_{F\in\mathcal{F}}\lim_\xi F. \] 
\end{proposition}
\begin{proof}
We have
\begin{align*}
x\in \lim_\xi \bigwedge_{F\in \mathcal{F}}F &\iff \bigwedge_{F\in \mathcal{F}}F \geq \vicinity_\xi(x) \iff \forall F\in\mathcal{F}: F\geq \vicinity_\xi(x) \\
&\iff \forall F\in \mathcal{F}: x\in\lim_\xi F \iff x\in \bigcap_{F\in\mathcal{F}}\lim_\xi F.
\end{align*}
\end{proof}
\begin{corollary}
The set of pretopological convergences on a set $X$ forms a complete meet-subsemilattice of the lattice of convergences.
\end{corollary}
\begin{proof}
Complete meet-semilattice homomorphism form a complete meet-subsemilattice, \ref{semilatticeOfSemilatticeHomomorphisms}.
\end{proof}
\begin{corollary}
There exists a closure operator $S_0$ that maps a convergence $\xi$ to the finest pretopological convergence coarser than $\xi$.
\end{corollary}

\begin{proposition} \label{pretopologicalConvergenceDeterminedBySuperSuperFilters}
Let $\sSet{X,\xi}$ be a pretopological convergence space, $F\in\powerfilters(X)$ a filter and $x\in X$. Then the following are equivalent:
\begin{enumerate}
\item $F\to x$;
\item every proper filter greater than $F$ is contained in a proper filter that converges to $x$;
\item $F|_A$ is contained in a proper filter that converges to $x$ for all $A\in F^\mesh$.
\end{enumerate}
\end{proposition}
\begin{proof}
$(1)\Rightarrow (2)$ If $F\to x$, then every greater filter converges to $x$.

$(2) \Rightarrow (3)$ If $A\in F^\mesh$, then $F|_A$ is a proper filter greater than $F$.

$(3)\Rightarrow (1)$ Suppose, towards a contradiction, that $F$ does not converge to $x$, so $\vicinity_\xi(x)\nsubseteq F$ and there exists $A\in \vicinity_\xi(x) \setminus F$. Since $A\notin F$, we have $A^c\in F^\mesh$ by \ref{complementInIsotoneGrill}. By assumption $F|_{A^c} \subseteq G$, where $G$ is a proper filter such that $G\to x$. Then $A\in \vicinity_\xi(x) \subseteq G$ and $A^c \in F|_{A^c} \subseteq G$, so $G$ is not proper. This is a contradiction.
\end{proof}
\begin{corollary} \label{pretopologicalConvergenceDeterminedBySubsubsequence}
Let $\sSet{X,\xi}$ be a pretopological convergence space, $\seq{x_n}_{n\in \N}$ a sequence and $x\in X$. Then $x_n\to x$ \textup{if and only if} every subsequence of $\seq{x_n}$ has a subsequence that converges to $x$.
\end{corollary}
\begin{proof}
The direction $\Rightarrow$ is immediate.

For the converse, take $A\subseteq X$. Then $\TailsFilter\seq{x_n}|_A$ is sequential by \ref{traceOfSequentialFilterSequential} an thus equals $\TailsFilter\seq{y_n}$ for some sequence $\seq{y_n}$. This sequence is an AA subnet of $seq{x_n}$ by construction and thus, by \ref{sequenceAAsubnetEquivalentToSubsequence}, equivalent to some sequence $\seq{z_n}$ that is a subsequence of $\seq{x_n}$. By assumption, $\seq{z_n}$ has a subsequence $\seq{z'_n}$ that converges to $x$. Now $\TailsFilter\seq{x_n}|_A = \TailsFilter\seq{z_n} \subseteq \TailsFilter\seq{z'_n}$ by \ref{subnetImplications} and we can conclude by the proposition.
\end{proof}


\section{Adherence and inherence}
\begin{definition}
Let $\sSet{X,\xi}$ be a preconvergence space and $\mathcal{A}\subseteq \powerset(X)$ a family of subsets. We define
\begin{itemize}
\item the \udef{adherence} of $\mathcal{A}$ as
\begin{align*}
\adh_\xi(\mathcal{A}) &= \setbuilder{x\in X}{\exists F\to x: \; F\amesh\mathcal{A}} \\
&= \bigcup_{\mathcal{A}\amesh F}\lim_\xi F;
\end{align*}
\item the \udef{inherence} of $\mathcal{A}$ as
\begin{align*}
\inh_\xi(\mathcal{A}) &= \setbuilder{x\in X}{\forall F\to x: \; F\mesh\mathcal{A}} \\
&= \bigcap_{\mathcal{A}\mesh F}\lim_\xi F.
\end{align*}
Note that it is $F\mesh\mathcal{A}$ (i.e.\ $F$ and $\mathcal{A}$ have a set in common), not $F\amesh\mathcal{A}$.
\end{itemize}
Let $A\subseteq X$ be a subset. Then we define $\adh_\xi(A) \defeq \adh_\xi(\{A\})$ and $\inh_\xi(A) \defeq \inh_\xi(\{A\})$.
\end{definition}

\begin{lemma} \label{adherenceInherenceOrderClosure}
Let $\sSet{X,\xi}$ be a preconvergence space and $\mathcal{A}\subseteq \powerset(X)$ a family of subsets. Then
\begin{enumerate}
\item $\adh_\xi(\mathcal{A}) = \adh_\xi(\upset\mathcal{A})$;
\item $\inh_\xi(\mathcal{A}) = \inh_\xi(\downset\mathcal{A})$.
\end{enumerate}
\end{lemma}
\begin{proof}
(1) By \ref{ameshUpwardClosure}, we have $\mathcal{A}\amesh F \iff (\upset\mathcal{A})\amesh F$.

(2) By \ref{GaloisConnectionOrderClosure}, we have $F\mesh\mathcal{A} \iff (\upset F)\mesh\mathcal{A} \iff F\mesh(\downset \mathcal{A})$.
\end{proof}

\begin{lemma} \label{inherenceAdherenceMonotoneInConvergence}
Let $X$ be a set, $\zeta, \xi$ preconvergences on $X$ such that $\zeta \leq \xi$ and $\mathcal{A}\subseteq \powerset(X)$. Then
\begin{enumerate}
\item $\adh_\zeta(\mathcal{A}) \subseteq \adh_\xi(\mathcal{A})$;
\item $\inh_\zeta(\mathcal{A}) \supseteq \inh_\xi(\mathcal{A})$.
\end{enumerate}
\end{lemma}

\begin{proposition} \label{differentUnionsForAdherence}
Let $\sSet{X,\xi}$ be a preconvergence space and $G\in\powerfilters(X)$ a filter. Then
\[ \adh_\xi(G) = \displaystyle\bigcup_{\substack{F\in\powerfilters(X) \\ G\lhd F}}\lim_\xi F = \displaystyle\bigcup_{\substack{U\in\powerultrafilters(X) \\ G\subseteq U}}\lim_\xi U. \]
\end{proposition}
\begin{proof}
For the first equality, take $x\in \adh_\xi(G)$. Then there exists a filter $F$ such that $G\amesh F$ and $F\to x$.
Now $F\vee G$ is a proper filter by \ref{joinProperFilter} and $G\subseteq F\vee G$, so $G\lhd F\vee G$. Also $F\vee G\to x$ by monotonicity of the limit. So $x\in \bigcup_{\substack{F\in\powerfilters(X) \\ G\lhd F}}\lim_\xi F$.

The other inclusion is immediate, because $G\lhd F$ implies $F^\mesh \subseteq G^\mesh$. Also $F\subseteq F^\mesh$ by \ref{properSubsemilatticeLemma}, so $F\subseteq G^\mesh$. This is equivalent to $G\amesh F$ by \ref{polarsSetRelation}.

For the second equality, the inclusion $\supseteq$ is immediate. For the other inclusion, take $x\in \bigcup_{\substack{F\in\powerfilters(X) \\ G\lhd F}}\lim_\xi F$. Then there exists $F\in\powerfilters(X)$ such that $G\lhd F$ and $F\to x$. Now $F$ is contained in an ultrafilter $U$ by the ultrafilter lemma \ref{ultrafilterLemma}. By monotonicity of the limit, $U\to x$. Thus $x\in \bigcup_{\substack{U\in\powerultrafilters(X) \\ G \lhd U}}\lim_\xi U$.
\end{proof}

\begin{example}
For a counterexample of \ref{differentUnionsForAdherence} if $G$ is not a filter, take the convergence space $\R$ and $G = \{\interval[o]{-\infty, 0}, \interval[co]{0,+\infty}\}$. Then the tailfilter $F$ of $\seq{\frac{(-1)^n}{n+1}}_{n\in\N}$ meshes with $G$. Also $F\to 0$, so $0\in \adh_\R(G)$.

But there exist no proper filters that contain $G$, so
\[ \bigcup_{\substack{F\in\powerfilters(X) \\ G\lhd F}}\lim_\xi F = \emptyset \neq \adh_\R(G). \]
\end{example}

\begin{lemma} \label{adherenceLimitFilter}
Let $\sSet{X,\xi}$ be a preconvergence space and $F\in \powerfilters(X)$ a proper filter. Then
\begin{enumerate}
\item $\lim_\xi(F) \subseteq \adh_\xi(F)$;
\item if $A\in F$, then $\lim_\xi(F)\subseteq \adh_\xi(A)$;
\item if $F$ is an ultrafilter, then $\lim_\xi(F) = \adh_\xi(F)$.
\end{enumerate}
\end{lemma}
\begin{proof}
(1) If $F$ is proper, then $F\lhd F$.

(2) If $A\in F$, then $\{A\}\lhd F$.

(3) If $F$ is an ultrafilter, then the only filter $G$ such that $F\lhd G$ is $F$.
\end{proof}
\begin{corollary} \label{singletonAdherence}
Let $\sSet{X,\xi}$ be a preconvergence space and $x\in X$. Then $\adh_\xi(\{x\}) = \lim_\xi \pfilter{x}$.
\end{corollary}
\begin{proof}
We have $\adh_\xi(\{x\}) = \adh_\xi(\pfilter{x})$ by definition and $\pfilter{x}$ is an ultrafilter.
\end{proof}
If $\xi$ is a convergence, then $\adh_\xi(\{x\}) = \lim_\xi \pfilter{x}$ is not empty.

\begin{proposition} \label{inherenceComplementAdherence}
Let $\sSet{X,\xi}$ be a preconvergence space and $\mathcal{A}\subseteq \powerset(X)$ a family of subsets. Then
\[ \inh_\xi(\mathcal{A})^c = \adh_\xi\Big((-^c)^{\imf}(\mathcal{A})\Big). \]
\end{proposition}
\begin{proof}
We have
\begin{align*}
\inh_\xi(\mathcal{A})^c &= \setbuilder{x\in X}{\exists F\to x: \; F\perp\mathcal{A}} \\
&= \setbuilder{x\in X}{\exists F\to x: \;\forall A\in\mathcal{A}: A\notin F} \\
&= \setbuilder{x\in X}{\exists F\to x: \; \forall A\in\mathcal{A}: A^c\subseteq F^\mesh} \\
&= \setbuilder{x\in X}{\exists F\to x: \; (-^c)^{\imf}(\mathcal{A}) \amesh F} \\
&= \adh_\xi\big((-^c)^{\imf}(\mathcal{A})\big),
\end{align*}
where we have used \ref{complementInIsotoneGrill}.
\end{proof}
\begin{corollary} \label{principalInherenceComplementAdherence}
Let $X$ be a set, $\xi$ a preconvergence on $X$ and $A \subseteq X$. Then
\[ (\inh_\xi A)^c = \adh_\xi (A^c). \]
\end{corollary}

\begin{proposition} \label{principalAdherenceInherence}
Let $\sSet{X,\xi}$ be a preconvergence space and $A\subseteq X$ a subset. Then
\begin{enumerate}
\item the following are equivalent:
\begin{enumerate}
\item $x\in \adh_\xi(A)$;
\item $\displaystyle A \in \bigcup_{\substack{\text{$F$ proper filter} \\F\to x}} F$;
\item there exists $F\in\powerfilters(X)$ such that $F\to x$ and $A\in F$;
\item $A\in \vicinity_\xi(x)^{\mesh}$;
\item $\displaystyle A \in \bigcup_{F\to x} F^{\mesh}$;
\end{enumerate}
\item $x\in \inh_\xi(A)$ \textup{if and only if} $A\in \vicinity_\xi(x)$.
\end{enumerate}
\end{proposition}
\begin{proof}
(1) $(a) \Rightarrow (b)$ Assume $x\in \adh_\xi(A)$. Then, by \ref{differentUnionsForAdherence}, there exists a proper filter $F$ such that $\upset\{A\}\subseteq F$ and $F\to x$. This is equivalent to (b).

$(b) \Leftrightarrow (c)$ Immediate translation.

$(b) \Rightarrow (d)$ We have that there exists a proper filter $F$ such that $\upset\{A\}\subseteq F$ and $F\to x$. Also $\vicinity_\xi(x)\subseteq \bigcup_{F\to x} F$, so $\left(\bigcup_{F\to x} F\right)^\mesh \subseteq \vicinity_\xi(x)^\mesh$ and $F\subseteq F^\mesh$ by \ref{properSubsemilatticeLemma}. Then
\[ \{A\} \subseteq F\subseteq F^\mesh\subseteq \vicinity_\xi(x)^\mesh, \]
so $A\in \vicinity_\xi(x)^\mesh$.

$(d) \Rightarrow (e)$ We have
\[ \vicinity_\xi(x)^\mesh = \left(\bigcap_{F\to x} F\right)^\mesh = \bigcup_{F\to x}F^\mesh, \]
by \ref{grillUpsetOrderSimilarity} (and the fact that joins and meets in the lattice of upwards closed sets are just given by the union and intersection, see \ref{meetJoinRClosedSets}).

$(e) \Rightarrow (a)$ There exists a filter $F\to x$ such that $A\in F^\mesh$. Then $x\in \bigcup_{A\subseteq F^\mesh}\lim_\xi(F) = \adh_\xi(A)$.

(2) We use \ref{principalInherenceComplementAdherence} and \ref{complementInIsotoneGrill} to compute
\[ x\in \inh_\xi(A) \iff x\in \adh_\xi(A^c)^c \iff x\notin \adh_\xi(A^c) \iff A^c\notin \vicinity_\xi(x)^\mesh \iff A\in \vicinity_\xi(x). \]
\end{proof}
Thus principal adherence and inherence is a pretopological notion, while filter adherence and inherence is a pseudotopological notion (see \ref{differentUnionsForAdherence}).
\begin{corollary} \label{setAdherenceInherence}
Let $\sSet{X,\xi}$ be a preconvergence space and $A,B\subseteq X$ subsets. Then
\[ \adh_\xi(A) \mesh B \qquad\iff\qquad \{A\} \amesh \vicinity_\xi(B). \]
\end{corollary}
\begin{proof}
We have
\begin{align*}
\adh_\xi(A) \mesh B &\iff \exists b\in B: b\in \adh_\xi(A) \\
&\iff \exists b\in B: A\in \vicinity_\xi(b)^{\mesh} \\
&\iff A \in \bigcup_{b\in B}\vicinity_\xi(b)^{\mesh} = \left(\bigcap_{b\in B}\vicinity_\xi(b)\right)^{\mesh} = \vicinity_\xi(B)^{\mesh}.
\end{align*}
\end{proof}

\begin{note}
Exercise: prove point (2) of \ref{principalAdherenceInherence} straight from definition.
\end{note}

\begin{proposition}
Let $\sSet{X,\xi}$ be a preconvergence space and $\mathcal{A}\subseteq \powerset(X)$. Then
\begin{enumerate}
\item $\displaystyle \adh_\xi(\mathcal{A}) \subseteq \bigcap_{A\in\mathcal{A}}\adh_\xi(A)$;
\item $\displaystyle \inh_\xi(\mathcal{A}) \supseteq \bigcup_{A\in\mathcal{A}}\inh_\xi(A)$.
\end{enumerate}
If $\xi$ is pretopological, then the two inclusions become equalities.
\end{proposition}
\begin{proof}
(1) We calculate
\begin{align*}
x\in\adh_\xi(\mathcal{A}) \iff& \exists F\to x: F\amesh \mathcal{A} \\
\iff& \exists F\to x: \forall A\in\mathcal{A}: A\in F^\mesh \\
\implies& \forall A\in\mathcal{A}: \exists F\to x: A\in F^\mesh \subseteq \vicinity_\xi(x)^\mesh \\
\implies& \forall A\in\mathcal{A}: A\in \vicinity_\xi(x)^\mesh \\
\iff& \forall A\in\mathcal{A}: x\in \adh_\xi(A) \\
\iff& x\in \bigcap_{A\in\mathcal{A}}\adh_\xi(A).
\end{align*}

Now assume $\xi$ is pretopological. Then $x\in \bigcap_{A\in\mathcal{A}}\adh_\xi(A)$ implies $\forall A\in\mathcal{A}: A\in \vicinity_\xi(x)^\mesh$ by the previous calculation. As $\xi$ is pretopological, $\vicinity_\xi(x) \to x$ and by setting $F = \vicinity_\xi(x)$, we see that $\exists F\to x: \forall A\in\mathcal{A}: A\in F^\mesh$. Again by the previous calculation, this implies $x\in \adh_\xi(\mathcal{A})$.

(2) We calculate using \ref{inherenceComplementAdherence}:
\[ \inh_\xi(\mathcal{A}) = \adh_\xi\big(\setbuilder{A^c}{A\in \mathcal{A}}\big)^c \supseteq \left(\bigcap_{A\in\mathcal{A}}\adh_{\xi}(A^c)\right)^c = \bigcup_{A\in\mathcal{A}}\adh_{\xi}(A^c)^c = \bigcup_{A\in\mathcal{A}}\inh_{\xi}(A). \]
\end{proof}

\begin{lemma}
Let $X$ be a set and $F$ be a filter in $\powerfilters(X)$. Then $\ker F = \adh_{\iota_X} F$.
\end{lemma}
\begin{proof}
Let $x\in X$. Then $\pfilter{x} \amesh F$ iff $\big\{\{x\}\big\} \amesh F$ iff $x\in \ker F$.
\end{proof}
\begin{corollary} \label{kernelAdherence}
Let $\sSet{X,\xi}$ be a \emph{convergence} space, $F\in \powerfilters(X)$ a filter and $I\in\powerideals(X)$ an ideal. Then
\begin{enumerate}
\item $\ker(F) \subseteq \adh_\xi(F)$;
\item $\bigcup I \supseteq \inh_\xi(I)$.
\end{enumerate}
\end{corollary}
\begin{proof}
(1) This follows from the lemma, $\iota_X \leq \xi$ and \ref{inherenceAdherenceMonotoneInConvergence}.

(2) Set $I' \defeq \setbuilder{A^c}{A\in I}$. Then $I'$ is a filter (TODO ref) and, by \ref{inherenceComplementAdherence}, we have
\[ \left(\bigcup I\right)^c = \ker(I') \subseteq  \adh_\xi(I') = \inh_\xi(I)^c. \]
This implies $\bigcup I \supseteq \inh_\xi(I)$.
\end{proof}

\begin{proposition}
Let $X$ be a set, $\xi$ a preconvergence on $X$ and $\mathcal{A},\mathcal{B} \subseteq \powerset(X)$. Then
\begin{enumerate}
\item $\adh_\xi \big(\powerset(X)\big) = \emptyset$;
\item if $\mathcal{A} \subseteq \mathcal{B}$, then $\adh_\xi \mathcal{A} \supseteq \adh_\xi \mathcal{B}$;
\item if $\mathcal{A},\mathcal{B}$ are upwards closed, then $\adh_\xi(\mathcal{A}\cap \mathcal{B}) = \adh_\xi(\mathcal{A}) \cup \adh_\xi(\mathcal{B})$.
\end{enumerate}
Also
\begin{enumerate} \setcounter{enumi}{3}
\item $\inh_\xi\big(\powerset(X)\big) = X$;
\item if $\mathcal{A} \subseteq \mathcal{B}$, then $\inh_\xi \mathcal{A} \subseteq \inh_\xi \mathcal{B}$.
\end{enumerate}
\end{proposition}
\begin{proof}
(1) There exist no non-empty filters $F$ such that $F\amesh \powerset(X)$.

(2) If $\mathcal{A} \subseteq \mathcal{B}$ and $F\amesh \mathcal{B}$, then $F\amesh \mathcal{A}$.

(3) From point (2) and \ref{orderPreservingFunctionLatticeOperations}, we get $\adh_\xi(\mathcal{A}\cap \mathcal{B}) \supseteq \adh_\xi(\mathcal{A}) \cup \adh_\xi(\mathcal{B})$.

Now take $x\in\adh_\xi(\mathcal{A}\cap \mathcal{B})$, then there exists $F\to x$ such that $F\subseteq (\mathcal{A}\cap \mathcal{B})^\mesh = \mathcal{A}^\mesh\cup \mathcal{B}^\mesh$, where the last equality is due to \ref{grillUpsetOrderSimilarity} (and the fact that joins and meets in the lattice of upwards closed sets are just given by the union and intersection, see \ref{meetJoinRClosedSets}).

It is enough to prove that either $F\subseteq \mathcal{A}^\mesh$ or $F\subseteq \mathcal{B}^\mesh$. Assume, towards a contradiction, that this is not the case, i.e.\ there exist $A\in\mathcal{A}$, $B\in \mathcal{B}$ and $C,D\in F$ such that $A\perp C$ and $B\perp D$. Then $C\cap D\perp A$ and $C\cap D \perp B$, so $C\cap D \notin \mathcal{A}^\mesh\cup \mathcal{B}^\mesh$. But $C\cap D \in F$ because $F$ is a filter.

(4) For all non-empty filters $F$, we have $F\mesh \powerset(X)$.

(5) If $\mathcal{A} \subseteq \mathcal{B}$ and $F\mesh \mathcal{A}$, then $F\mesh \mathcal{B}$.
\end{proof}
\begin{corollary} \label{principalInherenceAdherenceProperties}
Let $X$ be a set, $\zeta\leq \xi$ preconvergences on $X$ and $A,B \subseteq X$. Then
\begin{enumerate}
\item $\adh_\xi \emptyset = \emptyset$;
\item if $A \subseteq B$, then $\adh_\xi A \subseteq \adh_\xi B$;
\item $\adh_\xi(A\cup B) = \adh_\xi A \cup \adh_\xi B$;
\end{enumerate}
and
\begin{enumerate} \setcounter{enumi}{4}
\item $\inh_\xi X = X$;
\item if $A \subseteq B$, then $\inh_\xi A \subseteq \inh_\xi B$;
\item $\inh_\xi(A\cap B) = \inh_\xi A \cap \inh_\xi B$.
\end{enumerate}
Also
\[ \inh_\xi(A) \subseteq \inh_\zeta(A) \subseteq A \subseteq \adh_\zeta(A) \subseteq \adh_\xi(A). \]
\end{corollary}
Compare with \ref{discreteTopologyCharacterisation}.
\begin{proof}
Most results are immediate using \ref{adherenceInherenceOrderClosure} and \ref{inherenceAdherenceMonotoneInConvergence}.

The inclusions $A \subseteq \adh_\zeta(A)$ and $\inh_\zeta(A) \subseteq A$ follow from \ref{kernelAdherence}.

For point (7), we calculate,
\begin{align*}
\inh_\xi(A\cap B) &= \adh_\xi\big((A\cap B)^c\big)^c = \adh_\xi(A^c\cup B^c)^c = \big(\adh_\xi(A^c)\cup \adh_\xi(B^c)\big)^c \\
&= \adh_\xi(A^c)^c\cap \adh_\xi(B^c)^c = \inh_\xi(A) \cap \inh_\xi(B).
\end{align*}
\end{proof}
\begin{corollary} \label{interleavedAdherenceInherenceInclusion}
Let $\sSet{X,\xi}$ be a convergence space and $A\subseteq X$ a subset. Then
\begin{enumerate}
\item $\adh\big(\inh(A)\big)\subseteq \adh(A)$;
\item $\inh(A) \subseteq \inh\big(\adh(A)\big)$.
\end{enumerate}
\end{corollary}
\begin{proof}
(1) We have $\inh(A)\subseteq A$, so $\adh\big(\inh(A)\big)\subseteq \adh(A)$.

(2) We have $A\subseteq \adh(A)$, so $\inh(A) \subseteq \inh\big(\adh(A)\big)$.
\end{proof}


\begin{lemma} \label{subsetWithVicinitiesInInherence}
Let $\sSet{X,\xi}$ be a preconvergence space and $A,B\subseteq X$ subsets. If for every $a\in A$ there exists a vicinity of $a$ that is a subset of $B$, then $A\subseteq \inh_\xi(B)$.
\end{lemma}
\begin{proof}
Assume that for every $a\in A$ there exists a $U_a \in \vicinity_\xi(a)$ such that $U_a \subseteq B$. Then $B\in \vicinity_\xi(a)$ for all $a$ in $A$ and thus $a\in \inh_\xi(B)$ for all $a\in A$.
\end{proof}

\subsection{Čech closure and interior operators}
\begin{definition}
Let $X$ be a set and $f: \powerset(X)\to \powerset(X)$ a function. Then $f$ is called
\begin{itemize}
\item a \udef{Čech closure operator} if
\begin{itemize}
\item $f(\emptyset) = \emptyset$;
\item $A\subseteq f(A)$ for all $A\subseteq X$;
\item $f(A \cup B) = f(A)\cup f(B)$ for all $A,B\subseteq X$;
\end{itemize}
\item a \udef{Čech interior operator} if
\begin{itemize}
\item $f(X) = X$;
\item $f(A)\subseteq A$ for all $A\subseteq X$;
\item $f(A \cap B) = f(A)\cap f(B)$ for all $A,B\subseteq X$.
\end{itemize}
\end{itemize}
\end{definition}

\begin{lemma}
Let $X$ be a set and $f: \powerset(X)\to \powerset(X)$ either
\begin{enumerate}
\item a Čech closure operator on $X$; or
\item a Čech interior operator on $X$.
\end{enumerate}
Then $f$ is isotone, i.e.\ $A\subseteq B\subseteq X$ implies $f(A) \subseteq f(B)$.
\end{lemma}
\begin{proof}
(1) If $A\subseteq B$, then 
\[ f(A) \subseteq f(A)\cup f(B) = f(A\cup B) = f(B). \]

(2) If $A\subseteq B$, then 
\[ f(A) = f(A\cap B) = f(A)\cap f(B) \subseteq f(B). \]
\end{proof}

\begin{lemma}
Let $X$ be a set and $f: \powerset(X)\to \powerset(X)$ a function. Then
\begin{enumerate}
\item $f$ is a Čech closure operator \textup{if and only if} $g: \powerset(X)\to \powerset(X): A\mapsto f(A^c)^c$ is a Čech interior operator;
\item $f$ is a Čech interior operator \textup{if and only if} $g: \powerset(X)\to \powerset(X): A\mapsto f(A^c)^c$ is a Čech closure operator.
\end{enumerate}
\end{lemma}
\begin{proof}
(1) Assume $f$ is a Čech closure operator. Then we verify
\begin{itemize}
\item $g(X) = f(X^c)^c = f(\emptyset)^c = \emptyset^c = X$;
\item $g(A) = f(A^c)^c \subseteq (A^c)^c = A$;
\item $g(A\cap B) = f\big((A\cap B)^c\big)^c = f(A^c\cup B^c)^c = \big(f(A^c) \cup f(B^c)\big)^c = f(A^c)^c \cap f(B^c)^c = g(A) \cap g(B)$. 
\end{itemize}
(2) Similar.
\end{proof}

\begin{proposition} \label{CechClosureInteriorPretopology}
Let $X$ be a set. Then
\begin{enumerate}
\item for any convergence $\xi$ on $X$, then the adherence and inherence operators are Čech closure and interior operators;
\item for any Čech interior operator $f$, there exists an unique pretopology $\xi$ on $X$, determined by
\[ \vicinity_\xi(x) = \setbuilder{A\subseteq X}{x\in f(A)} = f^{\preimf}(\pfilter{x}), \]
such that $f = \inh_\xi$;
\item for any Čech closure operator $f$, there exists an unique pretopology $\xi$ on $X$, determined by
\[ \vicinity_\xi(x) = \big(f^\preimf(\pfilter{x})\big)^\mesh, \]
such that $f = \adh_\xi$.
\end{enumerate}
\end{proposition}
\begin{proof}
(1) Immediate from \ref{principalInherenceAdherenceProperties}.

(2) We need to prove that $f^{\preimf}(\pfilter{x})$ is a filter and $f^{\preimf}(\pfilter{x})\subseteq \pfilter{x}$. Then $f = \inh_\xi$ is clear from \ref{principalAdherenceInherence}.

First we show upwards closure of $f^{\preimf}(\pfilter{x})$. Take $A\in f^{\preimf}(\pfilter{x})$ and $A\subseteq B$. Then $x\in f(A)\subseteq f(B)$, so $x\in f(B)$ and thus $B\in f^{\preimf}(\pfilter{x})$.

Now we show downwards directedness. Take $A,B\in f^{\preimf}(\pfilter{x})$. Then $x\in f(A)$ and $x\in f(B)$, so $x\in f(A)\cap f(B) = f(A\cap B)$, so $A\cap B\in f^{\preimf}(\pfilter{x})$.

Finally we show $f^{\preimf}(\pfilter{x})\subseteq \pfilter{x}$. Take $A\in f^{\preimf}(\pfilter{x})$, so $x\in f(A)\subseteq A$, which means that $A\in \pfilter{x}$.

(3) We first show that $f^\preimf(\pfilter{x})$ is a filter-grill using \ref{filterGrillEquivalences}. It upwards closed by the same reasoning as in (2). Now take $A\cup B\in f^\preimf(\pfilter{x})$. Then $x\in f(A\cup B) = f(A) \cup f(B)$, so either $x\in f(A)$ (and thus $A\in f^\preimf(\pfilter{x})$) or $x\in f(B)$ (and thus $B\in f^\preimf(\pfilter{x})$).

Now we show that $f = \adh_\xi$. Take arbitrary $A\subseteq X$ and $x\in X$. Then
\[ x\in f(A) \quad\iff\quad A\in f^\preimf(\pfilter{x}) = \big(f^\preimf(\pfilter{x})\big)^{\mesh\mesh} = \vicinity_\xi(x)^\mesh \quad\iff\quad x\in \adh_\xi(A). \]
\end{proof}

\subsection{Kuratowski closure and interior operators}
\begin{definition}
Let $X$ be a set and $f: \powerset(X)\to \powerset(X)$ a function. Then $f$ is called
\begin{itemize}
\item a \udef{Kuratowski closure operator} if it is an idempotent Čech closure operator;
\item a \udef{Kuratowski interior operator} if it is an idempotent Čech interior operator.
\end{itemize}
\end{definition}

\begin{lemma}
Let $X$ be a set and $f: \powerset(X)\to \powerset(X)$ a function. Then
\begin{enumerate}
\item if $f$ is an idempotent Čech closure operator, then its associated Čech interior operator is also idempotent;
\item if $f$ is an idempotent Čech interior operator, then its associated Čech closure operator is also idempotent.
\end{enumerate}
\end{lemma}
\begin{proof}
(1) Let $g: \powerset(X)\to \powerset(X): A\mapsto f(A^c)^c$ be the associated Čech interior operator. Then, for all $A\subseteq X$,
\[ g^2(A) = f\Big(\big(f(A^c)^c\big)^c\Big)^c = f\Big(f(A^c)\Big)^c = f^2(A^c)^c = f(A^c)^c = g(A). \]

(2) Similar.
\end{proof}

\begin{proposition}
Let $X$ be a set and $f: \powerset(X)\to \powerset(X)$ a function. Then the following are equivalent:
\begin{enumerate}
\item $f$ is a Kuratowski closure operator;
\item $f(\emptyset) = \emptyset$ and $A\cup B\cup f^2(A) \cup f^2(B) = f(A\cup B)$ for all $A,B\subseteq X$;
\item $f(\emptyset) = \emptyset$ and $A\cup f(A) \cup f^2(B) = f(A\cup B)$ for all $A,B\subseteq X$;
\item $A\cup f(A) \cup f^2(B) = f(A\cup B) \setminus f(\emptyset)$ for all $A,B\subseteq X$.
\end{enumerate}
\end{proposition}
TODO: replacing equalities by inclusions gives Moore closure.
\begin{proof}
$(1) \Rightarrow (2)$ We calculate
\[ f(A\cup B) = f(A)\cup f(B) = A\cup B \cup f(A)\cup f(B) = A\cup B \cup f^2(A)\cup f^2(B). \]

$(2) \Rightarrow (3)$ Setting $B=\emptyset$ gives $A\cup f^2(A) = f(A) = f(A)\cup A$. In particular $f$ is expansive, so $f^2(B) \subseteq B\cup f^2(B) = f(B) \subseteq f^2(B)$ and thus $B\cup f^2(B) = f^2(B)$.

Now we calculate
\[ A\cup B\cup f^2(A) \cup f^2(B) = A\cup \big(A\cup f^2(A)\big) \cup \big(B\cup f^2(B)\big) = A\cup f(A) \cup f^2(B). \]

$(3) \Rightarrow (4)$ Immediate, since $f(A\cup B) = f(A\cup B) \setminus \emptyset = f(A\cup B) \setminus f(\emptyset)$.

$(4) \Rightarrow (1)$ We first prove that the empty set is a fixed point: setting $A=\emptyset = B$ give
\[ \emptyset \cup f(\emptyset) \cup f^2(\emptyset) = f(\emptyset)\setminus f(\emptyset) = \emptyset, \]
so $f(\emptyset) = \emptyset = f^2(\emptyset)$.

To see that $f$ is expansive, just set $B = \emptyset$. Then $A\cup f(A) = f(A)$, so $A\subseteq f(A)$.

Next we show idempotency: setting $A=\emptyset$ gives $f^2(B)\subseteq f(B)$ and we have just proved the other inequality.

Finally we show that the operator preserves unions: by expansivity and idempotency, we have $f(A\cup B) = A\cup f(A) \cup f^2(B) = f(A) \cup f^2(B) = f(A) \cup f(B)$.
\end{proof}

\begin{theorem}[Kuratowski closure-complement theorem]
Let $X$ be a set, $f: \powerset(X)\to\powerset(X)$ a Kuratowski closure operator and $A\subseteq X$. Then
\begin{enumerate}
\item at most 14 sets can be obtained from $A$ by taking closures and complements;
\item at most 7 sets can be obtained from $A$ by taking closures and interiors.
\end{enumerate}
The bounds can be reached.
\end{theorem}
\begin{proof}
\url{https://citeseerx.ist.psu.edu/document?repid=rep1&type=pdf&doi=3b4985d19737d5c00abc5a3ac359734e72291657}

And: \url{https://en.wikipedia.org/wiki/Kuratowski%27s_closure-complement_problem}
\end{proof}

\subsection{Dense sets}
\begin{definition}
Let $\sSet{X, \xi}$ be a convergence space and $A$ a subset. The subset $A$ is called \udef{dense} in $X$ if $\adh_\xi(A) = X$.
\end{definition}

\begin{lemma} \label{denseIffMeshesWithAllVicinities}
Let $\sSet{X, \xi}$ be a convergence space and $A$ a subset. Then $A$ is dense in $X$ \textup{if and only if} $A\mesh V$ for all $V\in \vicinity(x)$ for all $x\in X$.
\end{lemma}
In other words, $A$ is dense in $X$ iff each vicinity contains at least one point in $A$.
\begin{proof}
We calculate, using \ref{principalAdherenceInherence},
\begin{align*}
X = \adh_\xi(A) &\iff \forall x\in X: \; x\in \adh_\xi(A) \\
&\iff \forall x\in X: \; A \in \vicinity_\xi(x)^\mesh \\
&\iff \forall x\in X: \forall V\in\vicinity_\xi(x): \; A \mesh V.
\end{align*}
\end{proof}

\begin{lemma} \label{convergentFiltersInDenseSet}
Let $\sSet{X, \xi}$ be a convergence space and $A$ a subset. Then $A$ is dense in $X$ \textup{if and only if} for all $x\in X$, there exists a filter $F\to x$ with $A\in F$.
\end{lemma}
\begin{proof}
We have $A$ dense in $X$ iff $x\in \adh(A)$ for all $x\in X$. This is equivalent to the existence of such a filter by \ref{principalAdherenceInherence}.
\end{proof}

\begin{lemma} \label{openDensityLemma}
Let $\sSet{X, \xi}$ be a convergence space and $A$ a subset.
If $A^c$ contains a non-empty open set, then $A$ is not dense,
\end{lemma}
\begin{proof}
If $A$ is dense, then $\adh(A) = \inh(A^c)^c = X$, so $\inh(A^c) = \emptyset$. If $A^c$ contains an open set $B$, then $\inh(A^c) \supseteq \inh(B) = B$. This means that $\inh(A^c) \neq \emptyset$ and thus that $A$ is not dense.
\end{proof}

\subsubsection{Strict density}
\begin{definition}
Let $\sSet{X,\xi}$ be a convergence space and $A\subseteq X$ a subset. Then $A$ is called \udef{strictly dense} in $X$ if for all $F\to x\in X$, there exists $G\in \powerfilters(X)$ such that
\begin{itemize}
\item $G\to x$
\item $A\in G$;
\item $\adh_\xi^\imf(G)\subseteq F$.
\end{itemize} 
\end{definition}

\begin{lemma}
Let $\sSet{X,\xi}$ be a convergence space and $A\subseteq X$ a subset. If $A$ is strictly dense in $X$, then it is dense in $X$.
\end{lemma}
\begin{proof}
We need to show that $\adh_\xi(A) = X$. It is enough to show that $x\in \adh_\xi(A)$ for all $x\in X$.

Take arbitrary $x\in X$. Then $\pfilter{x}\to x$ and take a $G_x$ such that $A\in G_x$. Since $\adh_\xi^\imf(G_x)\subseteq \pfilter{x}$, we have $\adh_\xi(A) \in \pfilter{x}$. This means that $x\in \adh_\xi(A)$. 
\end{proof}


\subsection{Accumulation points}
\begin{definition}
Let $\sSet{X,\xi}$ be a convergence space and $\mathcal{A}\subseteq \powerset(X)$ a family of subsets. A point $x\in X$ is called an \udef{accumulation point} of $\mathcal{A}$ if $\mathcal{A}\amesh \vicinity_\xi(x)$.
\end{definition}

TODO $x\in \adh_\xi(A\setminus\{x\})$.

\begin{proposition} \label{subfilterToAccumulationPoint}
Let $\sSet{X,\xi}$ be a pretopological convergence space, $F\in\powerfilters(X)$ and $x\in X$. If $x$ is an accumulation point of $F$, then there exists a proper filter $G \geq F$ such that $G\overset{\xi}{\longrightarrow} x$. 
\end{proposition}
\begin{proof}
By \ref{joinProperFilter}, we have that $F\cup \vicinity_\xi(x)$ is a proper filter. We can take this filter to be $G$.
\end{proof}

\subsection{Cover}
\begin{definition}
Let $X$ be a set, $\xi$ a convergence on $X$, $A\subseteq X$ and $\mathcal{A}\subseteq \powerset(X)$.
We say $\mathcal{A}$ is an \udef{$\xi$-cover} (or simply \udef{cover}) of $A$ if every filter converging to a point in $A$ contains an element of $\mathcal{A}$. We write $\mathcal{A} \succ_\xi A$.
\end{definition}
So we have
\[ \mathcal{A} \succ_\xi A \;\iff\; \forall F\in \powerfilters(X): \Big( \lim_\xi F \amesh A \implies F \mesh \mathcal{A} \Big). \]

\begin{proposition}
Let $X$ be a set, $\xi$ a convergence on $X$, $A\subseteq X$ and $\mathcal{A}\subseteq \powerset(X)$. Then
\[ \mathcal{A} \succ_\xi A \quad\iff\quad \adh_\xi [\mathcal{A}]^c \perp A \]
\end{proposition}


TODO: move to compactness + inherence $\inh(\mathcal{A})$ largest set for which $\mathcal{A}$ is a cover.


\section{Base, pavement and cover}
\subsection{Pavement}
\begin{definition}
Let $\sSet{X,\xi}$ be a convergence space.
\begin{itemize}
\item A \udef{pavement} of $\xi$ at a point $x\in X$ is a family of filters $\mathcal{H}$ such that
\[ {\lim_\xi}^{-1}(x) = \upset\mathcal{H}. \]
\item The \udef{paving number} of $\xi$ at $x$ is the least cardinality $\kappa$ such that there is a pavement of $\xi$ of cardinality $\kappa$ at $x$.
\item The \udef{paving number} $\lambda$ of $\xi$ is
\[ \lambda = \sup\setbuilder{\kappa}{\text{$\kappa$ is the paving number of $\xi$ at $x$ for some $x\in X$}}. \]
We say $\xi$ is \udef{$\kappa$-paved} if $\lambda \leq \kappa$.
\item We call the convergence $\xi$ \udef{pretopological} if it is $1$-paved.
\item Let $\mathcal{F} \subseteq \powerfilters(X)$ be a set of filters. The convergence $\xi$ is said to be \udef{paved in} $\mathcal{F}$ if, at each point $x\in X$, $\xi$ has a pavement that is a subset of $\mathcal{F}$.
\end{itemize}
\end{definition}
Note that
\begin{itemize}
\item A convergence is pretopological iff it is $1$-paved.
\item A finitely deep convergence that is finitely paved is pretopological.
\end{itemize}

\begin{lemma} \label{vicinityMeetOfPavement}
Let $\sSet{X,\xi}$ be a convergence space, $x\in X$ and $\mathcal{H}$ a pavement of $\xi$ at $x$. Then $\vicinity_\xi(x) = \bigcap\mathcal{H}$.
\end{lemma}
\begin{proof}
We have $\vicinity_\xi(x) = \bigcap\lim^{-1}_{\xi}(x) = \bigcap\upset\mathcal{H} = \bigcap\mathcal{H}$.
\end{proof}

\begin{lemma}
Let $\sSet{X,\xi}$ be a pretopological convergence space. Then $\mathcal{H}\in\powerset^2(X)$ is a pavement of $\xi$ \textup{if and only if} $\setbuilder{\vicinity_\xi(x)}{x\in X} \subseteq \mathcal{H}$.
\end{lemma}

\subsection{Bases of a convergence}
\begin{definition}
Let $\sSet{X,\xi}$ be a convergence space and $x\in X$. A set $\mathcal{Z}\subseteq \powerset(X)$ is called
\begin{itemize}
\item a \udef{local base} at $x$ of the converence $\xi$ if there exists a pavement $\mathcal{H}$ of $\xi$ at $x$, such that each $F\in\mathcal{H}$ is based in $\mathcal{Z}$;
\item a \udef{base} of the convergence if it is a local base at each point $x\in X$;
\item if $\mathrm{P}$ is a property of a subset of a convergence space, then $X$ is said to be \udef{locally $\mathrm{P}$} if the set of $\mathrm{P}$-sets in $X$ is a base of the convergence. 
\end{itemize}
If $\mathcal{Z}$ is a base of $\xi$, we also say $\xi$ is \udef{based in} $\mathcal{Z}$.
\end{definition}
For example, we have local compactness, local convexity, etc.

Note that sometimes ``locally $\mathrm{P}$'' means ``locally open and $\mathrm{P}$'' (see e.g. locally pathconnected).  

\subsubsection{Bases of pretopological convergences}
\begin{definition}
Let $\sSet{X,\xi}$ be a convergence space and $\mathcal{Z}$ a base of $\xi$. The \udef{effective portion} of $\mathcal{Z}$ is
\[ \setbuilder{Z\in \mathcal{Z}}{\exists x\in X: \; Z\in \vicinity_\xi(x)} = \mathcal{Z}\cap \bigcup\setbuilder{\vicinity_\xi(x)}{x\in X}. \]
\end{definition}

\begin{lemma}
Let $\sSet{X,\xi}$ be a pretopological convergence space and $\mathcal{Y}, \mathcal{Z}$ bases of $\xi$. Then
\begin{enumerate}
\item the effective portion of $\mathcal{Z}$ is a base of $\xi$;
\item the effective portions of $\mathcal{Y}$ and $\mathcal{Z}$ are equally fine.
\end{enumerate}
\end{lemma}
\begin{proof}
(1) Immediate because any pavement of $\xi$ contains $\setbuilder{\vicinity_\xi(x)}{x\in X}$.

(2) Let $\mathcal{Y}'$ and $\mathcal{Z}'$ be the respective effective portions. Take $Y\in \mathcal{Y}'$; we need to show that there exists $Z\in \mathcal{Z}'$ such that $Z\subseteq Y$. Take $x\in X$ such that $Y\in \vicinity(x)$. Then there exist $Z \in \mathcal{Z}'$ such that $Z\subseteq Y$ because $\vicinity(x)$ is based in $\mathcal{Z}'$. 
\end{proof}
\begin{corollary} \label{cardinalityPretopologicalBase}
Let $\sSet{X,\xi}$ be a pretopological convergence space with bases $\mathcal{X},\mathcal{Y}$. Then there exists a base $\mathcal{Z} \subseteq \mathcal{X}$ with cardinality smaller than or equal to $\mathcal{Y}$.
\end{corollary}
\begin{proof}
Take the effective portion $\mathcal{X}'$ of $\mathcal{X}$. Because the effective portion $\mathcal{Y}'$ of $\mathcal{Y}$ refines $\mathcal{X}'$, we have
\[ \forall Y\in \mathcal{Y}': \exists X_Y \in \mathcal{X}': \; X_Y\subseteq Y. \]
The set $\mathcal{Z} \defeq \setbuilder{X_Y}{Y\in \mathcal{Y}'} \subseteq \mathcal{X}$ has cardinality smaller than $\mathcal{Y}'$, which has cardinality smaller than $\mathcal{Y}$.
\end{proof}

\subsection{Countability properties}
\subsubsection{$C1$ or first countable}
\begin{definition}
A convergence space $\sSet{X, \xi}$ is called \udef{first countable} or \udef{$C1$} if at each point there is a pavement of countably based filters.
\end{definition}

\subsubsection{Strongly $C1$ or strongly first countable}
\begin{definition}
A convergence space $\sSet{X, \xi}$ is called \udef{strongly first countable} or \udef{strongly $C1$} if there exists a countable local base of the convergence at each $x\in X$.
\end{definition}

\subsubsection{$C2$ or second countable}
\begin{definition}
A convergence space $\sSet{X, \xi}$ is called \udef{second countable} or \udef{$C2$} if $\xi$ has a countable base.
\end{definition}

\begin{lemma}
Second countable implies first countable.
\end{lemma}

\begin{lemma} \label{C2openBase}
Let $\sSet{X,\xi}$ be a $C_2$ topological convergence space. Then $\xi$ has a countable base of open sets.
\end{lemma}
\begin{proof}
See \ref{cardinalityPretopologicalBase}.
\end{proof}

\begin{lemma} \label{AnySetCountableIntersectionOfOpenSets}
Let $\sSet{X,\xi}$ be a $C_2$ and $T_1$ topological convergence space. Then every set $A\subseteq X$ can be written as both
\begin{enumerate}
\item a countable intersection of open sets; and
\item a countable union of closed sets.
\end{enumerate}
\end{lemma}
\begin{proof}
(1) By \ref{C2openBase}, $\xi$ has a countable base of open sets. By \ref{setKernelVicinityFilter} we have $A = \bigcap \neighbourhood(A) = \bigcap \upset \mathcal{F} = \bigcap \mathcal{F}$, for some subset $\mathcal{F}$ of the countable base of open sets.

(2) We can write $A^c$ as a countable intersection of open sets by (1). Taking the complement yields the result.
\end{proof}


\begin{proposition} \label{countableRegularityImpliesNormality}
Let $\sSet{X,\xi}$ be a topological space. If $\xi$ is regular and second countable, then $\xi$ is normal.
\end{proposition}
\begin{proof}
TOOD eg \url{https://www.math.auckland.ac.nz/~gauld/750-05/section3.pdf}
\end{proof}


\subsection{Covers}
\begin{definition}
Let $\sSet{X,\xi}$ be a convergence space and $\mathcal{C}\subseteq \powerset(X)$ a family of sets. We call $\mathcal{C}$ 
\begin{itemize}
\item a \udef{convergence cover} if $\mathcal{C}\mesh F$ for each convergent filter $F\in\powerfilters(X)$;
\item a \udef{local convergence cover} at $x\in X$ if $\mathcal{C}\mesh F$ for each filter $F\in\powerfilters(X)$ that converges to $x$;
\item a \udef{cover of vicinities} if $\mathcal{C}\mesh \vicinity_\xi(x)$ for all $x\in X$;
\item a \udef{cover of neighbourhoods} if $\mathcal{C}\mesh \neighbourhood_\xi(x)$ for all $x\in X$;
\item an \udef{open cover} if $\mathcal{C} \subseteq \topology_\xi$ and $\bigcup \mathcal{C} = X$;
\item a \udef{cover} of $X$ if $\bigcup \mathcal{C} = X$.
\end{itemize}
\end{definition}

\begin{lemma}
Let $\sSet{X,\xi}$ be a convergence space and $x\in X$. Then
\begin{enumerate}
\item the set of local convergence covers at $x$ is given by $\setbuilder{F}{F\overset{\xi}{\longrightarrow} x}^\mesh$;
\item the set of convergence covers is given by $\setbuilder{F}{\exists y\in X: F\overset{\xi}{\longrightarrow} y}^\mesh$.
\end{enumerate}
\end{lemma}

\begin{lemma}
Let $\sSet{X,\xi}$ be a convergence space.
\begin{enumerate}
\item Every cover of vicinities is a convergence cover.
\item If $\xi$ is pretopological, then every convergence cover is a cover of vicinities.
\end{enumerate}
\end{lemma}

\begin{lemma}
Let $\sSet{X,\xi}$ be a convergence space and $\mathcal{C}\subseteq \powerset(X)$ a family of sets. Then $\mathcal{C}$ is a cover of neighbourhoods \textup{if and only if} $\interior^\imf[\mathcal{C}]$ is an open cover.
\end{lemma}
\begin{proof}
First assume $\mathcal{C}$ is a cover of neighbourhoods. Clearly $\mathcal{C} \subseteq \topology_\xi$. For all $x\in X$ there exists $A\in \mathcal{C}$ such that $A\in \neighbourhood(x)$ and thus $\interior(A) \in \neighbourhood(x)$ by \ref{interiorModificationNeighbourhoods}. This means that $x\in \interior(A)$ and thus $X\subseteq \bigcup\interior^\imf[\mathcal{C}]$.

Next assume $\interior^\imf[\mathcal{C}]$ is an open cover. Then for all $x\in X$ there exists $A\in \mathcal{C}$ such that $x\in \interior(A)$, which implies $A\in \neighbourhood(x)$ by \ref{interiorClosureMembership}.
\end{proof}




\section{Topology}
\subsection{Open and closed sets}
\begin{definition}
Let $X$ be a set and $\xi$ a convergence on $X$.
\begin{itemize}
    \item A subset $O \subseteq X$ is called \udef{open} if $\inh_\xi(O) = O$.
    \item A subset $C \subseteq X$ is called \udef{closed} if $\adh_\xi(C) = C$.
    \item A subset is called \udef{clopen} if it is both open and closed.
\end{itemize}
The set of all open sets in $\sSet{X,\xi}$ is called the \udef{topology} of $\sSet{X,\xi}$ and is denoted $\topology_\xi$.
\end{definition}

\begin{lemma} \label{openClosedComplement}
Let $X$ be a set, $\xi$ a convergence on $X$ and $A\subseteq X$ a subset. Then $A$ is open \textup{if and only if} $A^c$ is closed.
\end{lemma}
\begin{proof}
Assume $\inh_\xi(A) = A$. Then $A^c = \inh_\xi(A)^c = \adh_\xi(A^c)$.
\end{proof}

\begin{proposition} \label{propertiesTopology}
Let $X$ be a set and $\xi$ a convergence on $X$. Then
\begin{enumerate}
\item the topology $\topology_\xi$ has the following properties:
\begin{enumerate}
\item $\emptyset, X\in \topology_\xi$;
\item $\topology_\xi$ is closed under arbitrary unions;
\item $\topology_\xi$ is closed under finite intersections;
\end{enumerate}
\item the set of closed sets in $\sSet{X,\xi}$ has the following properties:
\begin{enumerate}
\item it contains $\emptyset$ and $X$;
\item it is closed under arbitrary intersections;
\item it is closed under finite unions.
\end{enumerate}
\end{enumerate}
\end{proposition}
\begin{proof}
(1a) By \ref{principalInherenceAdherenceProperties} we have $\emptyset \subseteq \inh(\emptyset) \subseteq \emptyset$, so $\emptyset = \inh(\emptyset)$ and thus $\emptyset$ is open. That $X$ is open follows straight from \ref{principalInherenceAdherenceProperties}.

(1b) Let $\{A_i\}_{i\in I}$ be a set of open sets. From \ref{principalInherenceAdherenceProperties} and \ref{orderPreservingFunctionLatticeOperations} we get $\inh\left(\bigcup_{i\in I}A_i\right) \subseteq \bigcup_{i\in I}A_i$ and $\bigcup_{i\in I}\inh(A_i) \subseteq \inh\left(\bigcup_{i\in I} A_i\right)$. Thus
\[ \bigcup_{i\in I}A_i  = \bigcup_{i\in I}\inh(A_i) \subseteq \inh\left(\bigcup_{i\in I} A_i\right) \subseteq \bigcup_{i\in I}A_i. \]

(1c) Let $A, B\subseteq X$ be open sets. Then, by \ref{principalInherenceAdherenceProperties},
\[ A\cap B = \inh(A)\cap \inh(B) = \inh(A\cap B), \]
so $A\cap B$ is open.

(2) Follows from (1) using \ref{openClosedComplement}.
\end{proof}
\begin{corollary}
The topology $\topology_\xi$ is a complete sublattice of $\powerset(X)$.
\end{corollary}



\begin{lemma} \label{openClosedCriteria}
Let $\sSet{X,\xi}$ be a convergence space and $O,C\subseteq X$ subsets. The following are equivalent:
\begin{enumerate}
\item $O$ is open;
\item $O\subseteq \inh(O)$;
\item $O\in \vicinity(O)$;
\item for all $x\in O$ there exists $U_x\in \vicinity(x)$ such that $U_x\subseteq O$;
\item $O\in \vicinity(x)$ for all $x\in O$;
\item for all $x\in O$: $F\to x \implies O\in F$;
\end{enumerate}
as are the following:
\begin{enumerate}
\item $C$ is closed;
\item $\adh(C)\subseteq C$;
\item $\closure(C)\subseteq C$;
\item for all $x\in X$: $C\in \vicinity_\xi(x)^\mesh \implies x\in C$;
\item for all $x\in X$: $C\in \vicinity_\xi(x)^\mesh \iff x\in C$.
\end{enumerate}
\end{lemma}
Note ``for all $x\in O$: $F\to x \implies O\in F$'' seems to be the most important.
\begin{proof}
TODO \ref{subsetWithVicinitiesInInherence}
\end{proof}

\begin{lemma} \label{topologyMonotoneInConvergence}
Let $X$ be a set and $\zeta \leq \xi$, then $\topology_{\xi} \subseteq \topology_{\zeta}$.
\end{lemma}
\begin{proof}
Take $O\in \topology_{\xi}$. Then $O\subseteq \inh_\xi(O) \subseteq \inh_\zeta(O)$, so $O\in \topology_\zeta$ by \ref{openClosedCriteria}.
\end{proof}

\begin{lemma} \label{openClosedConvergenceInclusions}
Let $X$ be a set, $\zeta, \xi$ convergences on $X$ such that $\zeta\leq\xi$ and $A\subseteq X$ a subset.
\begin{enumerate}
\item if $A$ is open in $\xi$, then it is also open in $\zeta$;
\item if $A$ is closed in $\xi$, then it is also closed in $\zeta$.
\end{enumerate}
\end{lemma}
\begin{proof}
Assume $A$ is open in $\xi$. Then $\inh_\xi(A) = A$. But $\inh_\xi(A) \subseteq \inh_\zeta(A) \subseteq A$ by \ref{principalInherenceAdherenceProperties}, so $\inh_\xi(A) = \inh_\zeta(A) = A$. The argument for closedness is similar.
\end{proof}

\begin{proposition} \label{discreteTopologyCharacterisation}
Let $\sSet{X,\xi}$ be a convergence space. The following are equivalent:
\begin{enumerate}
\item $\xi$ is discrete;
\item all subsets of $X$ are open;
\item all singletons are open;
\item all subsets of $X$ are closed;
\item $\adh_\xi(A\cap B) = \adh_\xi A \cap \adh_\xi B$ for all $A,B\subseteq X$.
\end{enumerate}
\end{proposition}
In particular, the discrete convergence is topological because, for all sets $A\subseteq X$ we have $A = \interior(A) \subseteq \inh(A) \subseteq A$.
\begin{proof}
$(1) \Rightarrow (2)$ By \ref{principalAdherenceInherence}, we have, for arbitrary $A\subseteq X$
\[ x\in \inh(A) \iff A\in \vicinity(x) = \pfilter{x} \iff x\in A, \]
so $A$ is open.
$(2) \Rightarrow (3)$ Immediate.
$(3) \Rightarrow (4)$ Take arbitrary $A\subseteq X$. Then $A = \cap_{x\in A^c}\{x\}^c$, which is an intersection of closed sets and thus closed, by \ref{propertiesTopology}.
$(4) \Rightarrow (1)$ Assume, towards a contradiction, that $\xi$ is not discrete. Then there exists $F\to x$ such that $F \neq \pfilter{x}$. Take $A\in F$ and set $B = A\setminus\{x\}$. Now $A\in F^\mesh$, so for all $C\in F$, either $B\mesh C$ or $A\cap C = \{x\}$. In the second case, $\{x\}\in F$, so $F = \pfilter{x}$, which was excluded. Thus $B\in F^\mesh$. This means that $x\in \adh_\xi(B)$, but $x\notin B$, so $B$ is not closed. 

$(4) \Leftrightarrow (5)$ If all sets are closed, then
\[ \adh_\xi(A\cap B) = A\cap B = \adh_\xi A \cap \adh_\xi B. \]

Now assume (5) and, towards a contradiction, that not all subsets are closed. We can then take a non-closed $A\subseteq X$, so that $x\in \adh_\xi(A)\setminus A$. Then
\[ x\in \adh_\xi A \cap \adh_\xi \{x\} = \adh_\xi(A\cap\{x\}) = \adh_\xi(\emptyset) = \emptyset, \]
which is a contradiction.
\end{proof}

\begin{lemma} \label{openClosedSetLemma}
Let $\sSet{X,\xi}$ be a convergence space, $A\subseteq X$ a subset and $O\subseteq X$ an open subset. Then
\begin{enumerate}
\item $\adh_\xi(A)\cap O = \adh_\xi(A\cap O)\cap O$;
\end{enumerate}
\end{lemma}
\begin{proof}
(1) We have $\adh_\xi(A) = \adh_\xi\big((A\cap O) \cup (A\cap O^c)\big) = \adh_\xi(A\cap O) \cup \adh_\xi(A\cap O^c)$ by \ref{principalAdherenceInherence}. Now, since $O^c$ is closed (\ref{openClosedComplement}), we have $\adh_\xi(A\cap O^c) \subseteq \adh_\xi(O^c) = O^c$. So
\[ \adh_\xi(A)\cap O = \big(\adh_\xi(A\cap O)\cap O\big) \cup \cancel{\big(\adh_\xi(A\cap O^c)\cap O\big)} = \adh_\xi(A\cap O)\cap O. \]
\end{proof}

\subsubsection{Interior, closure and boundary}
\begin{definition}
Let $\sSet{X,\xi}$ be a convergence space.
\begin{itemize}
\item The dual closure mapping of $A\in \powerset(X)$ into $\topology_\xi$ is called the \udef{interior} of $A$, denoted $\interior_\xi(A)$ or $A^\circ$. 
\item The closure mapping of $A\in \powerset(X)$ into the set of closed sets in $X$ is called the \udef{closure} of $A$, denoted $\closure_\xi(A)$ or $\overline{A}^\xi$. 
\end{itemize}
The \udef{boundary} of $A\in\powerset(X)$ is $\partial A \defeq \overline{A}\setminus A^{\circ}$
\end{definition}

In particular, we have $\interior^2 = \interior$ and $\closure^2 = \closure$.

\begin{lemma} \label{interiorInherenceInclusion}
Let $\sSet{X,\xi}$ be a convergence space and $A\subseteq X$. Then
\begin{enumerate}
\item $\interior_\xi(A) \subseteq \inh_\xi(A)$;
\item $\adh_\xi(A) \subseteq \closure_\xi(A)$.
\end{enumerate}
\end{lemma}
\begin{proof}
(1) By definition $\interior_\xi(A) \subseteq A$. Then $\inh_\xi\big(\interior_\xi(A)\big) \subseteq \inh_\xi(A)$ by \ref{principalInherenceAdherenceProperties} and the result follows because $\interior_\xi(A)$ is open and thus $\interior_\xi(A) = \inh_\xi\big(\interior_\xi(A)\big)$.

(2) By definition $A \subseteq \closure_\xi(A)$. Then $\adh_\xi(A) \subseteq \adh_\xi\big(\closure_\xi(A)\big)$ by \ref{principalInherenceAdherenceProperties} and the result follows because $\closure_\xi(A)$ is closed and thus $\closure_\xi(A) = \adh_\xi\big(\closure_\xi(A)\big)$.
\end{proof}
\begin{corollary} \label{interiorInherenceClosureAdherenceComposition}
We have
\begin{enumerate}
\item $\inh \circ \interior = \interior = \interior \circ \inh$;
\item $\adh\circ \closure = \closure = \closure \circ \adh$.
\end{enumerate}
\end{corollary}
\begin{proof}
(1) Take arbitrary $A\subseteq X$. The first equality is just the statement that $\interior(A)$ is open (i.e.\ $\interior(A) = \inh\big(\interior(A)\big)$).

For the second equality, the lemma gives $\interior(A)\subseteq \inh(A)$. By monotonicity, we have $\interior(A) = \interior^2(A) \subseteq \interior\big(\inh(A)\big)$.

Then $\interior\big(\inh(A)\big) \subseteq \interior(A)$ follows from \ref{principalInherenceAdherenceProperties} and the monotonicity of the dual closure operator.

(2) Similar.
\end{proof}

\begin{lemma} \label{interiorClosureMonotoneInConvergence}
Let $X$ be a set, $\zeta, \xi$ convergences on $X$ such that $\zeta\leq \xi$ and $A\subseteq X$ a subset. Then
\begin{enumerate}
\item $\closure_\zeta(A) \subseteq \closure_
\xi(A)$;
\item $\interior_\zeta(A) \supseteq \interior_
\xi(A)$.
\end{enumerate}
\end{lemma}
\begin{proof}
(1) Every $\xi$-closed set is $\zeta$-closed by \ref{openClosedConvergenceInclusions}, so the least $\zeta$-closed superset of $A$ is smaller that the least $\xi$-closed superset of $A$.

(2) Every $\xi$-open set is $\zeta$-open by \ref{openClosedConvergenceInclusions}, so the greatest $\zeta$-open subset of $A$ is bigger than the greatest $\xi$-open subset of $A$.
\end{proof}

\begin{proposition}
Let $\sSet{X,\xi}$ be a convergence space and $A\subseteq X$ a subset. We have
\[ \interior_\xi(A) = \setbuilder{x\in X}{\exists U\in \vicinity_\xi(x): U\subseteq A}. \]
\end{proposition}
\begin{proof}
By \ref{subsetWithVicinitiesInInherence}, we have $\interior_\xi(A) \supseteq \setbuilder{x\in X}{\exists U\in \vicinity_\xi(x) U\subseteq A}$. The opposite inclusion is given by noting that $\interior_\xi(A)$ is open and using \ref{openClosedCriteria}.
\end{proof}

\begin{lemma} \label{interiorClosureComplement}
Let $\sSet{X,\xi}$ be a convergence space and $A\subseteq X$ a subset. Then $\closure_\xi(A) = \interior_\xi(A^c)^c$.
\end{lemma}
\begin{proof}
First note that $\interior_\xi(A^c)^c$ is closed by \ref{openClosedComplement}.

Now take arbitrary closed $C\subseteq X$. Then we have the following equivalences (using the fact that $C^c$ is open, by \ref{openClosedComplement}):
\[ A\subseteq C \iff C^c \subseteq A^c \iff C^c \subseteq \interior_\xi(A^c) \iff \interior_\xi(A^c)^c\subseteq C. \]
Thus $\interior_\xi(A^c)^c$ is the smallest closed set containing $A$.
\end{proof}

\subsubsection{Neighbourhoods}
\begin{definition}
Let $\sSet{X,\xi}$ be a convergence space and $x\in X$. We call a subset $A\subseteq X$ a \udef{neighbourhood} of $x$ is there exists an open set $O$ such that $x\in O \subseteq A$.

The set of all neighbourhoods of $x$ is denoted $\neighbourhood_\xi(x)$.
\end{definition}
\begin{lemma}
Let $\sSet{X,\xi}$ be a convergence space and $x\in X$. Then $\neighbourhood_\xi(x)$ is a filter.
\end{lemma}
\begin{proof}
By construction it is clearly upwards closed: if a set $A$ contains an open set $O$ that contains $x$, then any superset of $A$ also contains $O$.

Now take $A,B\in \neighbourhood_xi(x)$. Then there exist $O_A, O_B \in \topology_\xi$ such that $x\in O_A$ and $x\in O_B$. Now $x\in O_A\cap O_B$ and $O_A\cap O_B \in \topology_\xi$ by \ref{propertiesTopology}. Since $O_A\cap O_B\subseteq A\cap B$, we have $A\cap B\in \neighbourhood_\xi(x)$.
\end{proof}

\begin{proposition} \label{interiorClosureMembership}
Let $\sSet{X,\xi}$ be a convergence space, $A\subseteq X$ and $x\in X$. Then
\begin{enumerate}
\item $x\in \interior_\xi(A) \iff A \in \neighbourhood_\xi(x)$;
\item $x\in \closure_\xi(A) \iff A \in \neighbourhood_\xi(x)^{\mesh}$.
\end{enumerate}
\end{proposition}
\begin{proof}
(1) First assume $x\in \interior_\xi(A)$. Then $x$ is an element of an open subset of $A$ and thus $A\in \neighbourhood_\xi(x)$ by definition.

Now assume $x$ is an element of some open subset $O$ of $A$. Then $O \cup \interior_\xi(A)$ is a subset of $A$ that is open by \ref{propertiesTopology}, so $O \cup \interior_\xi(A) \subseteq \interior_\xi(A)$ by definition of $\interior_\xi(A)$ and thus $x\in O\subseteq \interior_\xi(A)$.

(2) We calculate, using \ref{interiorClosureComplement}, point (1) and \ref{complementInIsotoneGrill}:
\begin{align*}
x\in \closure_\xi(A) &\iff x\in \interior_\xi(A^c)^c \\
&\iff x\notin \interior_\xi(A^c) \\
&\iff A^c \notin \neighbourhood_\xi(x) \\
&\iff A \in \neighbourhood_\xi(x)^\mesh.
\end{align*}
\end{proof}

\begin{lemma}
Let $\sSet{X,\xi}$ be a convergence space and $x\in X$. Then $\neighbourhood_\xi(x) \subseteq \vicinity_\xi(x)$.
\end{lemma}
\begin{proof}
We have, by \ref{interiorClosureMembership}, \ref{interiorInherenceInclusion} and \ref{principalAdherenceInherence},
\[ A \in \neighbourhood_\xi(x) \iff x\in \interior_\xi(A) \implies x\in \inh_\xi(A) \iff A\in\vicinity_\xi(x). \]
\end{proof}

\begin{lemma} \label{interiorModificationNeighbourhoods}
Let $\sSet{X,\xi}$ be a convergence space, $A\subseteq X$ and $x\in X$. Then
\begin{enumerate}
\item $A\in\neighbourhood(x)$ \textup{if and only if} $\interior(A)\in\neighbourhood(x)$;
\item if $\mathcal{A}\in\powerset^2(X)$ is a base for $\neighbourhood(x)$, then $\interior^{\imf}(\mathcal{A})$ is also a base for $\neighbourhood(x)$.
\end{enumerate}
\end{lemma}
\begin{proof}
(1) We have $A\in\neighbourhood(x) \iff x\in \interior(A) = \interior^2(A) \iff \interior(A) \in \neighbourhood(x)$.

(2) First $\interior^{\imf}(\mathcal{A})$ is a filter base, because it is closed under finite intersections: for all $\interior(A),\interior(B)\in \interior^{\imf}(\mathcal{A})$ we have $\interior(A)\cap\interior(B) = \interior(A\cap B) \in \interior^{\imf}(\mathcal{A})$ because $A\cap B \in \mathcal{A}$.

We clearly have $\mathcal{A} \preceq \interior^{\imf}(\mathcal{A})$ because $\interior(A) \subseteq A$. For the opposite inequality, we need to show that for all $A\in \mathcal{A}$, $\interior(A)$ is a neighbourhood of $x$. This is point (1).
\end{proof}

\subsubsection{Locally closed sets}
\begin{definition}
Let $\sSet{X,\xi}$ be a convergence space and $A\subseteq X$ a subset. Then $A$ is called \udef{locally closed} if for all $x\in A$ there exists a vicinity $V_x\in\vicinity_\xi(x)$ such that $A\cap V_x$ is closed in $V_x$.
\end{definition}

\begin{proposition} \label{locallyClosedEquivalents}
Let $\sSet{X,\xi}$ be a convergence space and $A\subseteq X$ a subset. Then the following are equivalent:
\begin{enumerate}
\item $A$ is the intersection of an open set and a closed set;
\item $A$ is the difference between two open sets;
\item $A$ is the difference between two closed sets;
\item $\closure(A)\setminus A$ is closed;
\item $A$ is open in $\closure(A)$;
\item $A \subseteq \inh_\xi\big(A\cup \closure(A)^c\big)$.
\end{enumerate}
Any of these statements implies that $A$ is locally closed.

If $X$ is topological, then local closure implies these statements.
\end{proposition}
TODO show either that locally closed is always equivalent, or give counterexample.
\begin{proof}
$(1) \Leftrightarrow (2) \Leftrightarrow (3)$ If $A = O\cap C$ for some open set $O$ and closed set $C$, so $A= O\cap C^{cc} = O\setminus C^c$ is a difference of open sets by \ref{openClosedComplement}. Similarly $A = C\cap O^{cc} = C\setminus O^c$ is a difference of closed sets.

If $A = O_1\setminus O_2$ is a difference of open sets, then $A = O_1 \cap O_2^c$ is the intersection of an open and a closed set by \ref{openClosedComplement}.

If $A = C_1\setminus C_2$ is a difference of closed sets, then $A = C_1 \cap C_2^c$ is the intersection of an open and a closed set by \ref{openClosedComplement}.

$(3) \Rightarrow (4)$ Suppose $A = C_1 \setminus C_2$ for some closed sets $C_1,C_2$. Then $A \subseteq C_1$, so $\closure(A) \subseteq \closure(C_1) = C_1$ and so
\begin{align*}
\closure(A)\setminus A &= \closure(A)\setminus (C_1\setminus C_2) \\
&= \closure(A)\cap \big(C_1\cap C_2^c\big)^c \\
&= \cancel{\big(\closure(A)\cap C_1^c\big)} \cup \big(\closure(A)\cap C_2\big) \\
&= \closure(A) \cap C_2,
\end{align*}
which is closed by \ref{propertiesTopology}.

$(4) \Rightarrow (3)$ We have
\begin{align*}
\closure(A)\setminus \big(\closure(A)\setminus A\big) &= \closure(A)\cap \big(\closure(A)\cap A^c\big)^c \\
&= \big(\closure(A)\cap \closure(A)^c\big) \cup \big(\closure(A)\cap A\big) \\
&= A,
\end{align*}
so $A$ is the difference between two closed sets.

$(4) \Rightarrow (5)$ Since $\adh_\xi\big(\closure(A)\setminus A\big) = \closure(A)\setminus A$, we have, by \ref{subspaceAdherence},
\begin{align*}
\inh_{\xi|_{\closure(A)}}(A) &= \closure(A)\setminus \adh_\xi\big(\closure(A)\setminus A\big) \\
&= \closure(A)\setminus \big(\closure(A)\setminus A\big) \\
&= \closure(A)\cap \big(\closure(A)\cap A^c\big)^c \\
&= \cancel{\big(\closure(A)\cap \closure(A)^c\big)} \cup \big(\closure(A)\cap A\big) \\
&= \closure(A)\cap A = A,
\end{align*}
so $A$ is open in $\closure(A)$.

$(5) \Rightarrow (6)$ By \ref{subspaceAdherence} we have $A = \inh_{\xi|_{\closure(A)}}(A) = \inh_\xi\big(A \cup \closure(A)^c\big) \cap \closure(A)$, so, in particular, $A \subseteq \inh_\xi\big(A \cup \closure(A)^c\big)$.

$(6) \Rightarrow (4)$ This implies
\[ \adh_\xi\big(\closure(A)\setminus A\big) = \inh_\xi\big(A \cup \closure(A)^c\big)^c \subseteq A^c. \]
Now $\closure(A)\setminus A \subseteq \closure(A)$, so $\adh_\xi\big(\closure(A)\setminus A\big) \subseteq \adh_\xi\big(\closure(A)\big) = \closure(A)$ and thus
\[ \adh_\xi\big(\closure(A)\setminus A\big) = \adh_\xi\big(\closure(A)\setminus A\big) \cap \closure(A) \subseteq \closure(A) \cap A^c = \closure(A)\setminus A. \]
This implies that $\closure(A) \setminus A$ is closed by \ref{openClosedCriteria}.


$(6) \Rightarrow (\text{Locally closed})$ Take arbitrary $x\in A$. Then $x\in \inh_\xi\big(A \cup \closure(A)^c\big)$, so $A\cup \closure(A)^c \in \vicinity_\xi(x)$. We set $V_x \defeq A\cup \closure(A)^c$. Then, since $A\subseteq \closure(A)$,
\[ A\cap V_x = A\cap \big(A \cup \closure(A)^c\big) = A \cup \big(A \cap \closure(A)^c\big) = A \cup \cancel{\big(A \setminus \closure(A)\big)} = A. \]
Similarly, since $A\subseteq \adh_\xi(A)\subseteq \closure(A)$,
\begin{align*}
\adh_\xi(A)\cap V_x &= \adh_\xi(A)\cap \big(A \cup \closure(A)^c\big) \\
&= \big(\adh_\xi(A) \cap A\big) \cup \big(\adh_\xi(A) \cap \closure(A)^c\big) \\
&= A \cup \cancel{\big(\adh_\xi(A) \setminus \closure(A)\big)} = A.
\end{align*}
So, we have, by \ref{subspaceAdherence},
\[ A\cap V_x = A = \adh_\xi(A)\cap V_x = \adh_\xi\big(A\cap V_x\big)\cap V_x = \adh_{V_x}\big(A\cap V_x\big), \]
which means that $A\cap V_x$ is closed in $V_x$.

$(\text{Locally closed}) \Rightarrow (1)$. Assume $\xi$ topological and $A$ locally closed. For all $x\in A$, take some vicinity $V_x$ such that $A\cap V_x$ is closed in $V_x$. Since $\xi$ is topological, we can find an open $U_x$ such that $x\in U_x\subseteq V_x$. Then $A\cap U_x$ is closed in $U_x$: using \ref{subspaceAdherence}, we have
\begin{align*}
\closure_{U_x}(A\cap U_x) &= \closure(A\cap U_x) \cap U_x \\
&\subseteq \closure(A\cap V_x) \cap U_x \\
&= \closure(A\cap V_x) \cap V_x \cap U_x \\
&= \closure_{V_x}(A\cap V_x) \cap U_x \\
&= A\cap V_x \cap U_x = A \cap U_x.
\end{align*}
Now, using \ref{openClosedSetLemma}, we have $A \cap U_x = \closure(A\cap U_x) \cap U_x = \closure(A) \cap U_x$. Now $A\subseteq \bigcup_{x\in A}U_x$ and $\bigcup_{x\in A}U_x$ is open by \ref{propertiesTopology}. So
\[ A = A\cap \bigcup_{x\in A}U_x = \bigcup_{x\in A}A\cap U_x = \bigcup_{x\in A}\closure(A)\cap U_x = \closure(A) \cap \bigcup_{x\in A}U_x, \]
which is the intersection of an open set and a closed set.
\end{proof}

\subsection{Topological convergence}
\begin{definition}
A pretopological convergence space $\sSet{X,\xi}$ is called \udef{topological} if the topology $\topology_\xi$ is a base of $\xi$.
\end{definition}
In other words, a pretopological convergence space is topological if it is locally open.

\begin{definition}
A pretopological convergence space $\sSet{X,\xi}$ is called \udef{topological} if $\adh_\xi^2 = \adh_\xi$.
\end{definition}
TODO pretopological assumption necessary?

\begin{proposition} \label{pretopologicalSpaceTopological}
Let $\sSet{X,\xi}$ be a pretopological convergence space. The following are equivalent:
\begin{enumerate}
\item $\xi$ is topological;
\item $\adh_\xi^2 = \adh_\xi$;
\item $\inh_\xi^2 = \inh_\xi$;
\item $\adh_\xi = \closure_\xi$;
\item $\inh_\xi = \interior_\xi$;
\item $\vicinity_\xi(x) = \neighbourhood_\xi(x)$ for all $x\in X$;
\item $\forall U\in \vicinity_\xi(x): \exists V\in \vicinity_\xi(x): \forall y\in V: U\in\vicinity_\xi(y)$.
\end{enumerate}
\end{proposition}
\begin{proof}
TODO
\end{proof}
\begin{corollary}
Let $\sSet{X,\xi}$ be a convergence space. Then $\xi$ is topological \textup{if and only if} $\neighbourhood_\xi(x) \overset{\xi}{\longrightarrow} x$ for all $x\in X$.
\end{corollary}

\begin{lemma} \label{topologicalNetConvergence}
Let $\sSet{X,\xi}$ be a pretopological convergence space, $x\in X$ and $\seq{x_i}_{i\in I}$ a net. Then the following are equivalent:
\begin{enumerate}
\item $\TailsFilter\seq{x_i} \overset{\xi}{\longrightarrow} x$;
\item $\forall A\in \neighbourhood_\xi(x): \exists i_0\in I: \forall i \geq i_0: \quad x_i\in A$;
\item $\forall A\in \topology_\xi: x\in A \implies \Big(\exists i_0\in I: \forall i \geq i_0: \; x_i\in A\Big)$.
\end{enumerate}
\end{lemma}
\begin{proof}
$(1) \Leftrightarrow (2)$ By \ref{topologicalNetConvergence} (with $\vicinity_\xi(x) = \neighbourhood_\xi(x)$ by \ref{pretopologicalSpaceTopological}).


$(2) \Leftrightarrow (3)$ The set of all $A\in \topology_\xi$ such that $x\in A$ forms a basis of $\neighbourhood_\xi(x)$.
\end{proof}

TODO: direct characterisation of topological convergence spaces \url{https://en.wikipedia.org/wiki/Axiomatic_foundations_of_topological_spaces}.

\subsubsection{Specifying a topological convergence}

\begin{proposition}
Let $X$ be a set and $\mathcal{F}: X\to \powerfilters(X)$ a filter-valued function. Then $\mathcal{F}$ determines the neighbourhood filter of a unique topological spaces \textup{if and only if}
\begin{itemize}
\item $\mathcal{F}(x)\subseteq \pfilter{x}$ for all $x\in X$;
\item $\forall x\in X: \forall N\in \mathcal{F}(x): \exists M\in \mathcal{F}(x): \forall y\in M: N\in \mathcal{F}(y)$.
\end{itemize}
\end{proposition}
\begin{proof}
A filter-valued function $\mathcal{F}$ that satisfies the first requirement determined the vicinity filter of a pretopological space, \ref{filterFunctionToPretopology}.

This pretopological convergence is topological iff the second point holds, by \ref{pretopologicalSpaceTopological}.
\end{proof}

\begin{proposition} \label{specifyingTopology}
Let $X$ be a set and $\mathcal{T}\subseteq \powerset(X)$ a family of subsets. Then $\mathcal{T}$ is the topology of a topological convergence \textup{if and only if}
\begin{itemize}
\item $\emptyset\in \mathcal{T}$ and $X\in \mathcal{T}$;
\item if $A,B\in \mathcal{T}$, then $A\cap B\in \mathcal{T}$;
\item if $\{A_i\}_{i\in I}\subseteq \mathcal{T}$, then $\bigcup_{i\in I}A_i \in \mathcal{T}$.
\end{itemize}
Then for all $F\in\powerfilters(X)$ and $x\in X$, we have
\[ F\to x \quad\iff\quad \forall U\in \mathcal{T}: \; x\in U \implies U\in F. \]
\end{proposition}
This implies that the topological convergence with a given topology is unique.
\begin{proof}
TODO \ref{propertiesTopology}
\end{proof}
\begin{corollary}
Let $X$ be a set and $\mathcal{C}\subseteq \powerset(X)$ a family of subsets. Then $\mathcal{C}$ is the set of closed sets in a topological convergence \textup{if and only if}
\begin{itemize}
\item $\emptyset\in \mathcal{C}$ and $X\in \mathcal{C}$;
\item if $A,B\in \mathcal{T}$, then $A\cup B\in \mathcal{C}$;
\item if $\{A_i\}_{i\in I}\subseteq \mathcal{C}$, then $\bigcap_{i\in I}A_i \in \mathcal{C}$.
\end{itemize}
The family $\mathcal{C}$ uniquely determines the corresponding topology $\mathcal{T}$:
\[ \mathcal{T} = \setbuilder{O\subset X}{O^c\in\mathcal{C}}. \]
\end{corollary}

\begin{definition}
Let $X$ be a set. Any set $\mathcal{T} \subseteq \powerset(X)$ that satisfies the conditions in \ref{specifyingTopology} (and thus is the topology of a topological convergence) is called a \udef{topology} on $X$.
\end{definition}

\begin{example}
We supply some examples of topologies:
\begin{itemize}
\item Let $X$ be a three-element set, $X = \{a,b,c\}$. There are many possible topologies on $X$, to name a few:
\begin{itemize}
\item $\left\{\emptyset, X\right\}$
\item $\left\{\emptyset, \{a\}, \{a,b\}, X\right\}$
\item $\left\{\emptyset, \{a\}, X\right\}$
\item $\left\{\emptyset, \{a,b\}, X\right\}$
\item $\left\{\emptyset, \{a,b\}, \{a,c\}, \{b\}, X\right\}$
\item $\left\{\emptyset, \{a,b\}, \{c\}, X\right\}$
\item $\left\{\emptyset, \{a\}, \{b\}, \{a,b\}, X\right\}$
\item $\ldots$
\end{itemize}
\item For any set $X$, the collection of all subsets of $X$ is a topology, called the \udef{discrete topology}.
\item For any set $X$, the topology $\mathcal{T} = \left\{\emptyset, X\right\}$ is called the \udef{trivial} topology.
\item In any topology, both $X$ and $\emptyset$ are both open and closed.
\item Let $X$ be a set. Let $\mathcal{T}_f$ be a collection of all subsets $U$ of $X$ such that $X\setminus U$ is finite or $U=\emptyset$. Then $\mathcal{T}_f$ is a topology on $X$ called the \udef{finite complement topology}.
\end{itemize}
\end{example}

\subsection{The basis of a topology}
\begin{definition}
Let $\mathcal{T}$ be a topology on a set $X$. A \udef{basis} of $\mathcal{T}$ is a set $\mathcal{B} \subseteq \mathcal{T}$ such that
\[ \forall O\in \mathcal{T}:\forall x\in O: \exists B_x\in \mathcal{B}: \; (x\in B_x)\land (B_x\subseteq O).  \]
\end{definition}

\begin{lemma}
Let $\mathcal{T}$ be a topology on a set $X$, $\mathcal{B}$ a basis of $\mathcal{T}$ and $A\subseteq X$. Then
\begin{enumerate}
\item $\mathcal{T} = \Closure_{\bigcup}\mathcal{B}$;
\item $A\in \mathcal{T} \iff A = \bigcup\setbuilder{B\in \mathcal{B}}{B\subseteq A}$.
\end{enumerate}
\end{lemma}
Note that we include $\emptyset$ in $\Closure_{\bigcup}\mathcal{B}$.
\begin{proof}
(1) Since $\mathcal{B} \subseteq \mathcal{T}$ and $\mathcal{T}$ is closed under arbitrary unions, we have $\Closure_{\bigcup}\mathcal{B} subseteq \mathcal{T}$. 

Now take $O\in\mathcal{T}$. Then forall $x\in O$, there exists $B_x\in\mathcal{B}$ such that $x\in B_x$ and $B_x \subseteq O$. Then $x\in \bigcup_{x\in O}B_x$ for all $x\in O$, so $O \subseteq \bigcup_{x\in O}B_x$. Conversely, since $B_x\subseteq O$, we have $\bigcup_{x\in O}B_x \subseteq O$. This means $O = \bigcup_{x\in O}B_x$ and thus $O\in \Closure_{\bigcup}\mathcal{B}$.

(2) Immediate from (1).
\end{proof}

\begin{lemma}
Let $X$ be a set and $\mathcal{B}\subseteq \powerset(X)$. Then $\Closure_{\bigcup}\mathcal{B}$ is a topology on $X$ \textup{if and only if}
\begin{itemize}
\item For each $x\in X$, there is at least one basis element $B\in\mathcal{B}$ containing $x$.
\item If $x$ belongs to the intersection of two basis elements $B_1$ and $B_2$, then there is a basis element $B_3\in\mathcal{B}$ containing $x$ such that $B_3\subset B_1 \cap B_2$.
\end{itemize}
\end{lemma}
\begin{proof}
It is straightforward to verify the conditions in \ref{specifyingTopology}. 
\end{proof}

\begin{example}
For any set $X$, the collection of all one-point subsets of $X$ is a basis for the discrete topology.
\end{example}

We can link the basis to the coarseness of the topology.
\begin{lemma} \label{basisCoarseness}
Let $\mathcal{B}$ and $\mathcal{B}'$ be bases for the topologies $\mathcal{T}$ and $\mathcal{T}'$, respectively, on $X$. The following are equivalent:
\begin{enumerate}
\item $\mathcal{T}'$ is finer than $\mathcal{T}$.
\item For each $x\in X$ and each basis element $B\in\mathcal{B}$ containing $x$, there is a basis element $B'\in\mathcal{B'}$ such that $x\in B'\subset B$.
\end{enumerate}
\end{lemma}

Closures of sets can also be described using a basis.
\begin{lemma}
Let $A$ be a subset of $X$ which has a topology generated by a basis $\mathcal{B}$, then $x\in\bar{A}$ if and only if every basis element $B\in\mathcal{B}$ containing $x$ intersects $A$.
\end{lemma}

\subsubsection{Subbasis}
\begin{definition}
If $X$ is a set, a \udef{subbasis} is a subset $\mathcal{S}$ of the powerset of $X$ such that $X = \bigcup \mathcal{S}$.

The \udef{topology $\mathcal{T}$ generated by $\mathcal{S}$} is the collection of all unions of finite intersections of elements of $\mathcal{S}$. 
\end{definition}
The topology $\mathcal{T}$ is exactly the coarsest topology that makes all sets in the subbasis open.


\section{Convergence constructions}
\subsection{Directional convergence}
\begin{definition}
Let $X$ be a set and $D: X\to \powerset(X)$ a function such that $x\in D(x)$ for all $x\in X$.

Then a filter $F\in \powerfilters(X)$ is said to \udef{converge to $x$ in the direction $D$} if $D(x)\in F$. We call this convergence the \udef{directional convergence}.

The directional convergence w.r.t.\ $D^\transp$ is the \udef{dual directional convergence}.
\end{definition}

By \ref{principalImageEquivalence}, specifying a function $D: X\to \powerset(X)$ a function such that $x\in D(x)$ for all $x\in X$ is equivalent to specifying a reflexive relation $R = \setbuilder{(x,y)\in X^2}{x\in D(y)}$.

\begin{lemma} \label{directionalConvergenceLemma}
Let $X$ be a set and $D: X\to \powerset(X)$ a function such that $x\in D(x)$ for all $x\in X$. Then the directional convergence w.r.t.\ $D$ is a pretopological convergence with
\begin{enumerate}
\item $\vicinity_D(x) = \upset \{D(x)\} = \pfilter{x}R_s$ for all $x\in X$;
\item $\adh_D(A) = \setbuilder{x\in X}{A\mesh D(x)} = \bigcup_{y\in A}D^\transp(y)$ for all $A\subseteq X$;
\item $\inh_D(A) = \setbuilder{x\in X}{D(x)\subseteq A}$ for all $A\subseteq X$.
\end{enumerate}
\end{lemma}
\begin{proof}
(1) The form of the vicinity filter is immediate and this implies upwards closure. The convergence is centered, because $D(x)\in \pfilter{x}$ for all $x\in X$.

(2) We have
\begin{align*}
x\in \adh_D(A) &\iff A\in \vicinity_D(x)^\mesh \\
&\iff A\mesh D(x) \\
&\iff \exists y\in A: y \in D(x) \\
&\iff \exists y\in A: x \in D^\transp(y) \\
&\iff x\in \bigcup_{y\in A}D^\transp(y),
\end{align*}
using \ref{principalAdherenceInherence} and \ref{transpositionSetValuedFunctionLemma}.

(3) We have $x\in \inh_D(A) \iff A\in \vicinity_D(x) \iff D(x)\subseteq A$.
\end{proof}

\begin{proposition} \label{directionalConvergenceTopological}
Let $X$ be a set and $D: X\to \powerset(X)$ a function such that $x\in D(x)$ for all $x\in X$. Then the following are equivalent:
\begin{enumerate}
\item the directional convergence is topological;
\item $y\in D^\transp(x) \implies D^\transp(y)\subseteq D^\transp(x)$ for all $x,y\in X$;
\item the relation $R = \setbuilder{(x,y)\in X^2}{x\in D(y)}$ is transitive;
\item $y\in D(x) \implies D(y)\subseteq D(x)$ for all $x,y\in X$.
\end{enumerate}
\end{proposition}
\begin{proof}
$(1) \Rightarrow (2)$ If the convergence is topological, then $\adh_D$ is idempotent, by \ref{pretopologicalSpaceTopological}.
We have
\begin{align*}
y\in D^\transp(x) \implies& \{y\} \subseteq \adh_D(\{x\}) \\ \implies& \adh_D(\{y\}) \subseteq \adh_D^2(\{x\}) = \adh_D(\{x\}) \\
\implies& D^\transp(y)\subseteq D^\transp(x),
\end{align*}
for all $x,y\in X$.

$(2) \Rightarrow (3)$ From \ref{transitivityReflexiveRelationLemma}, we have that $R^\transp$ is transitive. By \ref{transposeTransitive}, $R$ is transitive.

$(3) \Rightarrow (4)$ Immediate from \ref{transitivityReflexiveRelationLemma}.

$(4) \Rightarrow (1)$ We use \ref{pretopologicalSpaceTopological} to prove that the convergence is topological. Take arbitrary $U\in \vicinity_D(x)$. Then we can set $V = D(x)\in \vicinity_D(x)$. Now take arbitrary $y\in D(x)$. By assumption, $D(y) \subseteq D(x) \subseteq U$, so $U\in \vicinity_D(y)$.
\end{proof}

\begin{proposition} \label{topologyDirectionalConvergence}
Let $X$ be a set and $D: X\to \powerset(X)$ a function such that $x\in D(x)$ for all $x\in X$. If the directional convergence w.r.t.\ $D$ is topological, then the topology is given by the closure of $\{D(x)\}_{x\in X}$ under arbitrary unions.
\end{proposition}
\begin{proof}
By \ref{directionalConvergenceLemma}, we have that all open sets are of the form $\bigcup_{D(x)\subseteq A}\{x\}$, for some $A$.

Also $\bigcup_{D(x)\subseteq A}\{x\} = \bigcup_{D(x)\subseteq A}D(x)$: suppose $D(x)\subseteq A$ and $y\in D(x)$, then $D(y)\subseteq D(x)\subseteq A$ by \ref{directionalConvergenceTopological}, so $\{y\}\subseteq \bigcup_{D(x)\subseteq A}\{x\}$.
\end{proof}

\begin{example}
Consider the real numbers $\R$ and set $D(x) = \interval[co]{x,\infty}$. The directional convergence w.r.t.\ $D$ is topological. This implies that the directional convergence w.r.t.\ $D^\transp$ is also topological. We have $D^\transp(x) = \interval[oc]{-\infty, x}$.
\end{example}

\subsubsection{Specified points convergence}
\begin{definition}
Let $X$ be a set and $A\subseteq X$ a subset. Set
\[ D_A: X \to \powerset(X): x\mapsto \begin{cases}
X & (x\in A) \\
\{x\} & (x\notin A)
\end{cases} \]
We call the directional convergence w.r.t.\ $D_A$ the \udef{specified points convergence} determined by $A$.
\end{definition}

\begin{lemma}
Let $X$ be a set and $A\subseteq X$ a subset. Then
\begin{enumerate}
\item the specified points convergence w.r.t.\ $A$ is topological;
\item the topology is given by
\[ \setbuilder{B\subseteq X}{(B\subseteq A) \lor (B = X)}. \]
\end{enumerate}
\end{lemma}
\begin{proof}
(1) We use \ref{directionalConvergenceTopological}. Take $x,y\in X$ such that $y\in D(x)$. First suppose $x\in A$. Then $D(x) = X$, so $D(y)\subseteq D(x)$. Now suppose $x\notin A$. Then $y \in D(x) = \{x\}$, so $y = x$ and thus $D(y) \subseteq D(x)$.

(2) Application of \ref{topologyDirectionalConvergence}.
\end{proof}

\begin{definition}
Let $\sSet{X,\xi}$ be a convergence space and $A\subseteq X$. We denote by $\xi_A$ the meet of $\xi$ and the specified points convergence determined by $A$.
\end{definition}

\begin{lemma} \label{specifiedPointsModificationVicinity}
Let $\sSet{X,\xi}$ be a convergence space and $A\subseteq X$. Then, for all $x\in X$,
\[ \vicinity_{\xi_A}(x) = \begin{cases}
\vicinity_\xi(x) & (x\in A) \\
\pfilter{x} & (x\notin A).
\end{cases} \]
\end{lemma}
\begin{proof}
First suppose $x\in A$. Then
\begin{align*}
\vicinity_{\xi_A}(x) &= \bigcap{F\in\powerfilters(X)}{\big(F\overset{\xi}{\longrightarrow} x\big) \land \big(F\overset{A}{\longrightarrow} x\big)} \\
&= \bigcap{F\in\powerfilters(X)}{\big(F\overset{\xi}{\longrightarrow} x\big) \land \big(X\in F\big)} \\
&= \bigcap{F\in\powerfilters(X)}{F\overset{\xi}{\longrightarrow} x} \\
&= \vicinity_\xi(x).
\end{align*}
Now suppose $x\notin A$. Then
\begin{align*}
\vicinity_{\xi_A}(x) &= \bigcap{F\in\powerfilters(X)}{\big(F\overset{\xi}{\longrightarrow} x\big) \land \big(\pfilter{x} \subseteq F\big)} \\
&= \bigcap{F\in\powerfilters(X)}{\pfilter{x} \subseteq F} \\
&= \pfilter{x}.
\end{align*}
\end{proof}

\subsection{Specified sets convergence}
\begin{definition}
Let $\sSet{X,\xi}$ be a convergence space and $\mathcal{A}\subseteq \powerset(X)$ a cover of $X$ (i.e. $\bigcup \mathcal{A} = X$). Then the \udef{specified sets convergence} $\xi_{\mathcal{A}}$ is defined by
\[ \forall F\in\powerfilters(X), x\in X: \qquad F\overset{\xi_{\mathcal{A}}}{\longrightarrow} x \iff \Big(F \overset{\xi}{\longrightarrow} x\Big) \land (F \mesh \mathcal{A}). \]
\end{definition}

\begin{lemma}
Let $\sSet{X,\xi}$ be a convergence space and $\mathcal{A}\subseteq \powerset(X)$ a cover of $X$. Then
\begin{enumerate}
\item the specified sets convergence $\xi_A$ is a convergence;
\item if $\mathcal{A}$ is closed under finite unions and $\xi$ is of finite depth, then $\xi_A$ is of finite depth.
\end{enumerate}


If $\xi $
\end{lemma}
\begin{proof}
(1) Upwards closure is immediate. To show $\xi_A$ is centered, take $x\in X$. Then $\bigcup \mathcal{A} = X$ implies there exists an $A\in \mathcal{A}$ such that $x\in A$. Then $A\in \pfilter{x}$, so $\pfilter{x} \overset{\xi_A}{\longrightarrow} x$.

(2) TODO
\end{proof}



\section{Continuous functions}
\begin{definition}
Let $\sSet{X,\xi}$ and $\sSet{Y,\zeta}$ be (pre)convergence spaces. A function $f: X\to Y$ is called \udef{continuous} if it preserves limits: for all $D \in \powerdirected(X)$ and $x\in X$:
\[ D \overset{\xi}{\longrightarrow} x \implies f^{\imf\imf}(D) \overset{\zeta}{\longrightarrow} f(x). \]

The set of all continuous functions $\sSet{X,\xi} \to \sSet{Y,\zeta}$ is denoted $\cont(\xi, \zeta)$ or $\cont(X,Y)$, if the convergence is clear. If $X=Y$, we also write $\cont(X)$.

If $\xi,\zeta$ are preconvergences we write $\cont_\text{pre}(\xi, \zeta)$ for the set of continuous functions.
\end{definition}
In other words, a function is continuous if it is relation-preserving as a function
\[ \sSet{X \cup \powerset^2(X), \overset{\xi}{\longrightarrow}} \to \sSet{Y \cup \powerset^2(Y), \overset{\zeta}{\longrightarrow}}. \]

Note that for filters $F\in\powerfilters(X)$, $f[F]$ is in general a directed set (by \ref{imageDirectedSet}), but not necessarily a filter. Taking the upward closure gives a filter (see \ref{imageFilter}).

\begin{lemma}
Let $\sSet{X,\xi}$ and $\sSet{Y,\zeta}$ be (pre)convergence spaces. A function $f: X\to Y$ is continuous \textup{if and only if} for all $D \in \powerdirected(X)$
\[ f^{\imf}\left[\lim_\xi D \right] \subseteq \lim_\zeta f^{\imf\imf}[D]. \]
\end{lemma}
\begin{proof}
Immediate from \ref{relationPreserving}.
\end{proof}

\begin{lemma} \label{continuityComposition}
Let $\sSet{X,\xi}$, $\sSet{Y,\sigma}$ and $\sSet{Z,\zeta}$ be (pre)convergence spaces. If $f: X\to Y$ and $g: Y\to Z$ are continuous, then $g\circ f$ is continuous.
\end{lemma}
\begin{proof}
Let $F\to x\in X$. Then $f[F] \to f(x)$ by continuity of $f$ and $g[f[F]] \to g(f(x))$ by continuity of $g$. So $g\circ f$ is continuous.
\end{proof}

\begin{lemma} \label{finerCoarserContinuity}
Let $\sSet{X,\xi}, \sSet{Y,\zeta}$ be (pre)convergence spaces and $f\in \cont_\text{(pre)}(\xi, \zeta)$.
\begin{enumerate}
\item Let $\sigma$ be a (pre)convergence on $X$ such that $\sigma \leq \xi$, then $f\in \cont_\text{(pre)}(\sigma, \zeta)$.
\item Let $\tau$ be a (pre)convergence on $Y$ such that $\tau \geq \zeta$, then $f\in \cont_\text{(pre)}(\xi, \tau)$.
\end{enumerate}
\end{lemma}
\begin{proof}
(1) Let $F\overset{\sigma}{\longrightarrow} x \in X$, then $F\overset{\xi}{\longrightarrow} x$ (because $\sigma \leq \xi$), so $f[F]\overset{\zeta}{\longrightarrow} f(x)$, meaning $f\in \cont_\text{(pre)}(\sigma, \zeta)$.

(2) Let $F\overset{\xi}{\longrightarrow} x \in X$, then $f[F]\overset{\zeta}{\longrightarrow} f(x)$, so $f[F]\overset{\tau}{\longrightarrow} f(x)$  (because $\zeta \leq \tau$), meaning $f\in \cont_\text{(pre)}(\xi, \tau)$.
\end{proof}

\begin{proposition} \label{continuityVicinityFilter} \label{adherenceInherenceContinuity}
Let $\sSet{X,\xi}$ and $\sSet{Y,\zeta}$ be convergence spaces, $f: X\to Y$ a function, $A\subseteq X$ a subset and $x\in X$. Then the following are equivalent:
\begin{enumerate}
\item $\vicinity_\zeta(f(x)) \subseteq \upset f^{\imf\imf}[\vicinity_\xi(x)]$;
\item for all $U\in \vicinity_\zeta(f(x))$, there exists $V\in \vicinity_\xi(x)$ such that $f^\imf[V] \subseteq U$;
\item for all $U\in \vicinity_\zeta(f(x))$, $f^{\preimf}[U]\in \vicinity_\xi(x)$;
\item $f^{\preimf\imf}\big(\vicinity_\zeta(f(x))\big) \subseteq \vicinity_\xi(x)$;
\end{enumerate}
All these points hold if $f$ is continuous at $x$. The following are also equivalent:
\begin{enumerate} \setcounter{enumi}{4}
\item any of (1)-(4) holds for all $x\in X$;
\item $f^\preimf[\adh_\zeta(A)] \supseteq \adh_\xi(f^\preimf[A])$;
\item $f^\imf[\adh_\xi(A)] \subseteq \adh_\zeta(f^\imf[A])$;
\item $f^\preimf[\inh_\zeta(A)] \subseteq \inh_\xi(f^\preimf[A])$.
\end{enumerate}
All these points hold if $f$ is continuous.
\end{proposition}
\begin{proof}
We first show that continuity of $f$ implies (1): We have
\begin{align*}
\vicinity_\zeta(f(x)) &= \bigcap\setbuilder{G\in\powerfilters(Y)}{G\to f(x)} \\
&\subseteq \bigcap\setbuilder{\upset f^{\imf\imf}(F)\in\powerfilters(Y)}{F\to x} \\
&= \bigcap (\upset\circ f^{\imf\imf})^{\imf}\Big[\setbuilder{F\in\powerfilters(X)}{F\to x}\Big] \\
&= \upset f^{\imf\imf}\Big[\bigcap \setbuilder{F\in\powerfilters(X)}{F\to x}\Big] = \upset f^{\imf\imf}[\vicinity_\xi(x)],
\end{align*}
where we have used \ref{imageUpsetsPreservesIntersection}.

Now take $B\in \bigcap\setbuilder{\upset f^{\imf\imf}(F)\in\powerfilters(Y)}{F\to x}$. This means that for all $F\in\lim^{-1}_\xi(x)$, there exists an $A_F\in F$ such that $f^\imf[A_F]\subseteq B$. By the upward closure of the filters $F$, the set $A = \bigcup_{F\in \lim^{-1}_\xi(x)}A_F$ is an element of each $F$ and
\[ f^\imf[A] = \bigcup_{F\in \lim^{-1}_\xi(x)}f^{\imf}[A_F] \subseteq B.  \]
Now $A\in \vicinity_\xi(x)$ and thus $B\in \upset f^{\imf\imf}[\vicinity_\xi(x)]$.


$(1) \Leftrightarrow (2)$ Reformulation.

$(2) \Leftrightarrow (3)$ We get (3) from (2) by noting that $V\subseteq f^{\preimf}f^\imf(V) \subseteq f^\preimf(U)$ and that $\vicinity_\xi(x)$ is a filter, so this implies $f^\preimf(U)\in \vicinity_\xi(x)$.

Conversely, we may take $V = f^{\preimf}[U]$.

$(3) \Leftrightarrow (4)$ Reformulation.

$(5) \Rightarrow (6)$ Using $(4)$, we calculate
\begin{align*}
x\in \adh_\zeta(f^\preimf[A]) \iff& f^\preimf[A]\in \vicinity_\xi(x)^{\mesh} \subseteq \left(f^{\preimf\imf}\vicinity_\zeta(f(x))\right)^{\mesh} \\
\implies& f^\imf f^\preimf[A] \in f^{\imf\imf}\left(\left(f^{\preimf\imf}\vicinity_\zeta(f(x))\right)^{\mesh}\right) \subseteq \left(f^{\imf\imf} f^{\preimf\imf}\vicinity_\zeta(f(x))\right)^{\mesh} \subseteq \left(\vicinity_\zeta(f(x))\right)^{\mesh} \\
\implies& A \in \left(\vicinity_\zeta(f(x))\right)^{\mesh} \\
\iff& f(x)\in \adh_\zeta(A) \\
\iff& x\in f^{\preimf}(\adh_\zeta(A)).
\end{align*}
We have used that (TODO ref)
\begin{itemize}
\item $F\subseteq G$ implies $G^{\mesh} \subseteq F^{\mesh}$;
\item $f^{\imf\imf}(F^{\mesh}) \subseteq \left(f^{\imf\imf}(F)\right)^{\mesh}$;
\item $F^{\mesh}$ is a filter if $F$ is a directed set;
\item $f^\imf f^\preimf[A] \subseteq A$.
\end{itemize}

$(5) \Rightarrow (7)$ Using $(1)$, we calculate
\begin{align*}
y\in f^\imf(\adh(A)) \iff& \exists x\in \adh(A):\;f(x) = y\\
\iff& \exists x\in X: \; x\in \adh(A) \land f(x) = y\\
\iff& \exists x\in X: \; A\in\vicinity(x)^{\mesh}  \land f(x) = y \\
\implies& \exists x\in X: \; f(x) = y \land f^\imf(A)\in f^{\imf\imf}\left(\vicinity(x)^{\mesh}\right) \subseteq f^{\imf\imf}\left(\vicinity(x)\right)^{\mesh} \subseteq \vicinity(f(x))^{\mesh} \\
\iff& \exists x\in X: \; f(x) = y \land f(x)\in \adh\Big(f^\imf(A)\Big) \\
\implies& y\in \adh\Big(f^\imf(A)\Big).
\end{align*}

$(6) \Rightarrow (8)$ We calculate, using \ref{principalInherenceComplementAdherence},
\begin{align*}
f^\preimf[\inh_\xi(A)] &= f^\preimf[\adh_\xi(A^c)^c] \\
&= \im(f)\setminus f^{\preimf}[\adh_\xi(A^c)] \\
&\subseteq f^{\preimf}[\adh_\xi(A^c)]^c \\
&\subseteq \left(\adh_\zeta(f^{\preimf}[A^c])\right)^c \\
&\subseteq \left(\adh_\zeta(f^{\preimf}[A]^c)\right)^c \\
&= \inh_\zeta(f^{\preimf}[A]).
\end{align*}

$(7) \Rightarrow (8)$ We calculate
\begin{align*}
x\in f^\preimf[\inh_\xi(A)] \iff& f(x) \in \inh_\zeta(A) = \adh_\zeta(A^c)^c \subseteq \adh_\zeta\Big(f^\imf\circ f^{\preimf}(A^c)\Big)^c \\
\iff& f(x) \in \adh_\zeta\Big(f^\imf\big(f^{\preimf}(A)^c\big)\Big)^c \subseteq f^\imf\Big(\adh_\xi\Big(f^{\preimf}(A)^c\big)\Big)^c \\
\implies& x\in \adh_\xi\Big(f^{\preimf}(A)^c\big)^c = \inh_\xi\big(f^{\preimf}(A)\big).
\end{align*}

$(8) \Rightarrow (5)$ We prove (1) for all $x\in X$. We calculate
\begin{align*}
B\in \vicinity_\zeta(f(x)) \iff& f(x) \in \inh_\zeta(B) \\
\implies& x\in f^{\preimf}\Big(\inh_\zeta(B)\Big) \subseteq \inh_\xi\Big(f^\preimf(B)\Big) \\
\implies& f^\preimf(B) \in \vicinity_\xi(x) \\
\implies& f^\imf\big(f^\preimf(B)\big) \in f^{\imf\imf}[\vicinity_\xi(x)] \\
\implies& B \in \upset f^{\imf\imf}[\vicinity_\xi(x)].
\end{align*}
\end{proof}
\begin{proof}[Alternative proof of (2), given continuity of $f$]
We calculate
\begin{align*}
f[\adh_\xi(A)] &= f\left[\bigcup \setbuilder{\lim_\xi F}{A \lhd F} \right] = \bigcup f\left[ \setbuilder{\lim_\xi F}{A \lhd F} \right] = \bigcup \setbuilder{f\left[\lim_\xi F\right]}{A \lhd F} \\
&\subseteq \bigcup \setbuilder{\lim_\zeta f[F]}{A \lhd F} \subseteq \bigcup \setbuilder{\lim_\zeta f[F]}{f[A] \lhd f[F]} \subseteq \bigcup \setbuilder{\lim_\zeta G}{f[A] \lhd G} \\
&= \adh_\zeta(f[A]),
\end{align*}
where we have used that $A \lhd B$ implies $f[A] \lhd f[B]$ (TODO ref).
\end{proof}
\begin{corollary} \label{pretopologicalContinuityVicinities}
If $\zeta$ is pretopological, then any of points (1)-(4) implies that $f$ is continuous at $x$.

Any of points (5)-(8) implies that $f$ is continuous.
\end{corollary}
\begin{proof}
We prove that in this case (1) implies the continuity of $f$. Take $F\to x$. Then $F\supseteq \vicinity_\xi(x)$ and thus $f[F]\supseteq f[\vicinity_\xi(x)]$. By assumption this means $\upset f[F]\supseteq \upset f[\vicinity_\xi(x)]\supseteq \vicinity_\zeta(f(x))$. So $f[F]\to f(x)$.
\end{proof}

If $\xi$ is pretopological, we can also give a simplified proof that the continuity of $f$ implies (1): We have that $\vicinity_\xi(x)$ converges to $x$, so by continuity $f[\vicinity_\xi(x)]$ converges to $f(x)$ and thus $\upset f[\vicinity_\xi(x)] \supseteq \vicinity_\zeta(f(x))$ by the pretopological property.

\begin{corollary} \label{preimageOpenClosed}
Let $\sSet{X,\xi}$ and $\sSet{Y,\zeta}$ be convergence spaces and $f: X\to Y$ a continuous function. Then
\begin{enumerate}
\item if $O\subseteq Y$ is open, then $f^\preimf(O)$ is open;
\item if $C\subseteq Y$ is closed, then $f^\preimf(C)$ is closed.
\end{enumerate}
If $\zeta$ is topological, then either (1) or (2) implies that $f$ is continuous.
\end{corollary}
\begin{proof}
(1) We have
\[ f^\preimf(O) = f^\preimf\big(\inh(O)\big) \subseteq \inh\big(f^\preimf(O)\big) \subseteq f^\preimf(O), \]
which implies $f^\preimf(O) = \inh\big(f^\preimf(O)\big)$ and thus that $f^\preimf(O)$ is open.

(2) We have
\[ f^\preimf(C) \subseteq \adh\big(f^\preimf(C)\big) \subseteq f^\preimf\big(\adh(C)\big) = f^\preimf(C), \]
which implies $f^\preimf(C) = \adh\big(f^\preimf(C)\big)$ and thus that $f^\preimf(C)$ is closed.

(2 bis) Alternatively, we can prove (2) from (1) by using the fact that $C^c$ is open and \ref{imagePreimageUniqueness} to calculate
\[ f^\preimf(C) = f^\preimf(Y\setminus C^c) = f^\preimf(Y)\setminus f^\preimf(C^c) = X\setminus f^\preimf(C^c) = f^\preimf(C^c)^c. \]
This is closed because $f^\preimf(C^c)$ is open by (1).

(If $\zeta$ is topological) then, for all $x\in X$,
\[ f^{\preimf\imf}\Big(\vicinity\big(f(x)\big)\Big) = f^{\preimf\imf}\Big(\neighbourhood\big(f(x)\big)\Big) \subseteq \neighbourhood(x) \subseteq \vicinity(x). \]
This implies that $f$ is continuous by \ref{pretopologicalContinuityVicinities}.
\end{proof}


\begin{lemma} \label{identityContinuity}
Let $\xi$ and $\zeta$ be two convergences on the same set $X$. Then $\id_X: \sSet{X,\xi} \to \sSet{X,\zeta}$ is continuous \textup{if and only if} $\xi \leq \zeta$. I.e.\ $\xi$ is finer than $\zeta$.
\end{lemma}
\begin{proof}
This is essentially a restatement of definitions:
\begin{align*}
\text{$\id_X: \sSet{X,\xi} \to \sSet{X,\zeta}$ is continuous} &\iff \forall F\in\powerfilters(X): \id_X\left[\lim_\xi F \right] \subseteq \lim_\zeta \id_X[F] \\
&\iff \forall F\in\powerfilters(X): \lim_\xi F \subseteq \lim_\zeta F \\
&\iff \xi \leq \zeta.
\end{align*}
\end{proof}

\begin{lemma} \label{continuityConstructions}
Let $\sSet{X,\xi}$ and $\sSet{Y,\zeta}$ be a convergence space.
\begin{enumerate}
\item \textup{(Identity function)} The identity function $\id_X:X\to X$ is continuous.
\item \textup{(Constant function)} For all $y$ in $Y$, the constant function $\underline{y}: X \to Y$ is continuous.
\end{enumerate}
\end{lemma}
\begin{proof}
(1) Let $F\to x \in X$. Then $x\in \lim_\xi(\id_X[F]) = \lim_\xi(F)$.

(2) Let $F\to x \in X$. Then $\underline{y}[F] = \pfilter{y} \to y = \underline{y}(x)$.
\end{proof}

\subsection{Continuity at a point}
\begin{definition}
Let $\sSet{X,\xi}, \sSet{Y,\zeta}$ be convergence spaces and $x\in X$. A function $f:X\to Y$ is called \udef{continuous at $x$} if for all $F\in \powerfilters(X)$,
\[ F\overset{\xi}{\longrightarrow} x \qquad\implies\qquad \upset f^{\imf\imf}(F) \overset{\zeta}{\longrightarrow} f(x). \]
\end{definition}

\begin{lemma} \label{continuityAtPointConvergenceLemma}
Let $\sSet{X,\xi}, \sSet{Y,\zeta}$ be convergence spaces, $x\in X$ and $f: X\to Y$ a function. Then $f$ is continuous at $x$ \textup{if and only if} $f: \sSet{X,\xi_{\{x\}}}\to \sSet{Y,\zeta}$ is continuous.
\end{lemma}

\subsection{Open and closed maps}
\begin{definition}
Let $\sSet{X, \xi}, \sSet{Y, \zeta}$ be convergence spaces and $f: X\to Y$ a function. Then
\begin{itemize}
\item $f$ is called an \udef{open function} if $f^\imf(A)$ is open for all open $A\subseteq X$;
\item $f$ is called a \udef{closed function} if $f^\imf(A)$ is closed for all closed $A\subseteq X$.
\end{itemize}
\end{definition}

TODO: we need compactness higher than this.
\begin{lemma}
Let $\sSet{X,\xi}$ be a compact convergence space, $\sSet{Y,\zeta}$ a Hausdorff space and $f: X\to Y$ a continuous function, then $f$ is closed.
\end{lemma}
\begin{proof}
Let $A\subseteq X$ be a closed set. Suppose $F\in\powerfilters(Y)$ contains $f^\imf(A)$ and converges to $y\in Y$. Then $F\subseteq V$ for some $V\in\powerultrafilters(Y)$ by \ref{ultrafilterLemma} and $V\to y$ by monotonicity. By \ref{mappingUltrafiltersLemma}, there exists $U\in \powerultrafilters(X)$ such that $\upset f^{\imf\imf}(U) = V$. By compactness, $U$ converges to some $x\in X$. Since $A$ is closed, $x\in A$. By continuity, $V = \upset f^{\imf\imf}(U) \to f(x)\in f^\imf(A)$. By Hausdorff, $f(x) = y$, so $y\in f^\imf(A)$. This proves that $f^\imf(A)$ is closed.
\end{proof}

\subsection{Homeomorphisms and embeddings}
\begin{definition}
Let $\sSet{X,\xi}$, $\sSet{Y,\zeta}$ be convergence spaces and $f: X\to Y$ a function. Then
\begin{itemize}
\item $f$ is a \udef{homeomorphism} if
\begin{itemize}
\item $f$ is bijective;
\item both $f$ and $f^{-1}$ are continuous;
\end{itemize}
\item $f$ is an embedding if $f: X\to f^\imf(X)$ is a homeomorphism.
\end{itemize}
\end{definition}

\begin{example}
Consider the function $\interval[co]{0,1} \to S\subseteq\C: x\mapsto e^{2\pi i x}$, where $S$ is the unit circle. This function is bijective and continuous, but not a homeomorphism. Consider a sequence approaching $1$ from the lower half of the plane.
\end{example}

\begin{proposition} \label{homeomorphismPreservation}
If $f$ homeomorphism, then
\begin{enumerate}
\item $f[\lim F] = \lim f[F]$;
\item $f[\vicinity(x)] = \vicinity(f[x])$;
\item $f[\adh(A)] = \adh(f[A])$.
\end{enumerate}
\end{proposition}


\subsection{Directional continuity}
\begin{definition}
Let $f: \sSet{X,\xi} \to \sSet{Z,\zeta}$ be a function between convergence spaces, $x\in X$ and $D\subseteq X$. We called $f$ \udef{directionally continuous} at $x$ in the direction $D$ if for all $F\in\powerfilters(X)$
\[ F \overset{\xi,D}{\longrightarrow} x \qquad \implies \qquad f[F] \overset{\zeta}{\longrightarrow} f(x). \]
\end{definition}

\begin{lemma}
Let $f: \sSet{X,\xi} \to \sSet{Z,\zeta}$ be a function between convergence spaces, $x_0\in X$ and $D$ a vicinity of $x_0$. Then $f$ is directionally continuous at $x_0$ in $D$ \textup{if and only if} $f$ is continuous at $x_0$.
\end{lemma}
\begin{proof}
TODO inherence
\end{proof}

\section{Compactness}
\begin{definition}
Let $\sSet{X,\xi}$ be a convergence space, $F\in\powerfilters(X)$ a filter and $A\subseteq X$ a subset. Then we call the \emph{filter} $F$
\begin{itemize}
\item \udef{compactoid} if every ultrafilter that contains $F$ converges;
\item \udef{$A$-compactoid} if every ultrafilter that contains $F$ converges to a point in $A$.
\end{itemize}
We call a \emph{subset} $B\subseteq X$
\begin{itemize}
\item \udef{compactoid} if $\upset\{B\}$ is compactoid;
\item \udef{$A$-compactoid} if $\upset\{B\}$ is $A$-compactoid;
\item \udef{compact} if $\upset\{B\}$ is $B$-compactoid.
\end{itemize}
\end{definition}
In particular the convergence space $X$ is compact if every ultrafilter converges.

\begin{example}
Any convergence on a finite set $X$ makes it compact as the only ultrafilters are the principal ultrafilters, \ref{finiteFiltersPrincipal}.
\end{example}

\begin{lemma}
Let $\sSet{X,\xi}$ be a convergence space and $F\in\powerfilters(X)$. Then $F$ is compactoid \textup{if and only if} every proper filter larger than $F$ is contained in a convergent filter.
\end{lemma}
\begin{proof}
$\boxed{\Rightarrow}$ Every proper filter larger than $F$ is contained in an ultrafilter larger than $F$, by the ultrafilter lemma \ref{ultrafilterLemma}. This ultrafilter converges by assumption.

$\boxed{\Leftarrow}$ Since every ultrafilter larger than $F$ is only contained in one proper filter, namely itself, it must converge.
\end{proof}

\begin{lemma} \label{allCompactoidLimitsInAdherence}
Let $\sSet{X,\xi}$ be a convergence space. A filter $F\in\powerfilters(X)$ is compactoid \textup{if and only if} it is $\adh_\xi(F)$-compactoid.
\end{lemma}
\begin{proof}
The direction $\Leftarrow$ is immediate.

Now assume $F$ is compactoid. We need to show that all ultrafilters larger than $F$ converge to a point in $\adh_\xi(F)$. Indeed take an ultrafilter $U\supseteq F$. Then $U$ converges to some $x\in X$ and
\[ x\in \lim_\xi U \subseteq \bigcup_{F\lhd G}\lim_\xi G = \adh_\xi(F), \]
by \ref{differentUnionsForAdherence}.
\end{proof}

\begin{proposition} \label{compactFiniteSubcover}
Let $\sSet{X,\xi}$ be a convergence space and $A\subseteq X$. Then the following are equivalent:
\begin{enumerate}
\item $A$ is compact;
\item every convergence cover of $A$ has a finite subset that covers $A$.
\end{enumerate}
\end{proposition}
In other words, every convergence cover has a finite subset whose union contains $A$.
\begin{proof}
Application of \ref{ChoquetModificationFilterSet}. In this case $\mathcal{F}$ is the (upwards closed) set of convergent filters and $F = \upset\{A\}$.
\end{proof}
Unpacking the propositions gives the following more elementary proof: (here still for the special case of $A = X$).
\begin{proof}
$\boxed{\Rightarrow}$ Let $\mathcal{C}$ be a convergence cover and assume, towards a contradiction, that $\mathcal{C}$ does not have a finite subset that covers $X$. Then
\[ \setbuilder{X\setminus(C_1\cup \ldots \cup C_n)}{n\in \N; C_1,\ldots, C_n \in \mathcal{C}} \]
is a filter base and does not contain $\emptyset$, so the filter is generates is proper and we can find an ultrafilter $G$ that contains it by the ultrafilter lemma \ref{ultrafilterLemma}.

Now $G$ is convergent, so $\mathcal{C}\mesh G$, i.e.\ there exists a $C\in \mathcal{C}\cap G$. By construction $X\setminus C\in G$, so $\emptyset = C\cap (X\setminus C)\in G$, which means $G$ cannot be an ultrafilter.

$\boxed{\Leftarrow}$ Assume, towards a contradiction, that $X$ is not compact. Then there exists an ultrafilter $G$ that does not converge. This means that $F \not\subseteq G$ for all convergent $F\in\powerfilters(X)$. Let $\mathcal{C}$ consist of one set from $F\setminus G$ for all convergent $F\in\powerfilters(X)$. Now $\mathcal{C}$ is a cover because $\pfilter{x}$ converges for all $x\in X$. There exists a finite subset $\{C_1, \ldots, C_n\}\subseteq \mathcal{C}$ that covers $X$, so $C_1\cup \ldots \cup C_n = X \in G$. Now $G$ is prime by \ref{booleanMaximalFiltersIdeals} and thus at least one $C_1, \ldots, C_n$ is an element of $G$, which is a contradiction.
\end{proof}
\begin{corollary} \label{topologyCompactnessOpenCover}
Let $\sSet{X,\xi}$ be a topological convergence space. Then the following are equivalent:
\begin{enumerate}
\item $X$ is compact;
\item every open cover has a finite subset that covers $X$;
\item every cover of neighbourhoods has a finite subset that covers $X$.
\end{enumerate}
\end{corollary}
\begin{corollary}
Let $\sSet{X,\xi}$ be a topological convergence space. Then $X$ is compact \textup{if and only if} the set
\[ \mathfrak{F} \defeq \setbuilder{\mathcal{C}\subseteq \powerset(X)}{\forall C\in \mathcal{C}: \;\text{$C$ is closed},\; \bigcap \mathcal{C} \neq \emptyset} \]
if of finite character.
\end{corollary}
\begin{proof}
First assume $X$ compact. Take $\mathcal{C}\in \mathfrak{F}$. Clearly every subset of $\mathcal{C}$ is in $\mathfrak{F}$. Now suppose every finite subset of $\mathcal{C}$ is some set such that every finite subset of $\mathcal{C}$ is in $\mathfrak{F}$. In particular every singleton is a set of closed sets, so every element of $\mathcal{C}$ is closed. Now consider $\Big(\bigcap_{C\in \mathcal{C}} C\Big)^c = \bigcup_{C\in \mathcal{C}}C^c$. We need to show that this is not equal to $X$. Indeed, suppose, towards a contradiction that it is $X$. Then $\setbuilder{C^c}{C\in \mathcal{C}}$ is an open cover of $X$. By compactness, it must have a finite subcover. Taking complements, we have a finite subset of $\mathcal{C}$ whose intersection is empty. This contradicts our assumption.

Now assume $\mathfrak{F}$ is of finite character. Let $\mathcal{O}$ be an open cover of $X$ Then $\bigcap_{O\in\mathcal{O}} O^c = \Big(\bigcup_{O\in\mathcal{O}}O\Big)^c = X^c = \emptyset$, so $\setbuilder{O^c}{O\in \mathcal{O}}$ is not an element of $\mathfrak{F}$. Because of the finite character of $\mathfrak{F}$, there must exist a finite subset of $\setbuilder{O^c}{O\in \mathcal{O}}$ that is not in $\mathfrak{F}$. Taking complements yields a finite open subcover.
\end{proof}
\begin{corollary} \label{intersectionClosedChainCompactSet}
Let $\sSet{X,\xi}$ be a compact topological convergence space and $\mathcal{C}$ a chain of closed non-empty subsets of $X$. Then $\bigcap \mathcal{C} \neq \emptyset$.
\end{corollary}
\begin{proof}
Every finite intersection of elements of a chain of non-empty elements is non-empty. Thus $\mathcal{C}\in \mathfrak{F}$ by its finite character.
\end{proof}

\begin{lemma} \label{discreteCompactIffFinite}
Let $X$ be a set. Then the discrete convergence $\sSet{X,\iota_X}$ is compact \textup{if and only if} $X$ is a finite set.
\end{lemma}
\begin{proof}
First suppose the discrete convergence $\sSet{X,\iota_X}$ is compact. The discrete convergence is topological and the set of all singletons is an open cover, by \ref{discreteTopologyCharacterisation}. The only subcover of the set of singletons is the set itself. Thus it must be finite by \ref{topologyCompactnessOpenCover}.

Now suppose $X$ is finite. Then every cover is finite, so we conclude with \ref{compactFiniteSubcover}.
\end{proof}

\begin{lemma} \label{compactSetCompactSubspace}
Let $\sSet{X,\xi}$ be a convergence space and $A\subseteq X$ a subset. Then $A$ is compact \textup{if and only if} $\sSet{A, \xi|_A}$ is compact.
\end{lemma}
\begin{proof}
First assume $A$ is compact and take an ultrafilter in $U\in \powerultrafilters(A)$. Then $\upset \iota^{\imf\imf}[U]$ is an ultrafilter in $X$ by \ref{imageFilterProperties}, so it converges to some $x\in A$ by compactness. Now $x = \iota(x)$, so $U$ converges to $x\in A$ in the subspace convergence $\xi|_A$ by \ref{initialFinalConvergence}.

Now assume $\sSet{A, \xi|_A}$ is compact and take an ultrafilter $U'\in \powerultrafilters(X)$ such that $A\in U'$. Then $\upset \iota^{\preimf\imf}(U')$ is an ultrafilter by \ref{imageFilterProperties}, which converges to $x\in A$ by compactness. Then $U' = \upset\iota^{\imf\imf}\big(\iota^{\preimf\imf}(U')\big)$ by \ref{mappingUltrafiltersLemma} and thus $U'$ converges to $x\in A$ by continuity of the inclusion.
\end{proof}

\begin{proposition} \label{compactClosedSets}
Let $\sSet{X,\xi}$ be a convergence space and $A\subseteq X$ a subset.
\begin{enumerate}
\item If $X$ is compact and $A$ is closed, then $A$ is compact.
\item If $X$ is a Hausdorff space and $A$ is compact, then $A$ is closed.
\end{enumerate}
\end{proposition}
\begin{proof}
(1) Take some ultrafilter $U\in \powerultrafilters(X)$ that contains $A$. Then $U$ converges to some $x\in X$ by compactness.
By \ref{allCompactoidLimitsInAdherence}, $x\in \adh(A) = A$.

(2) By \ref{openClosedCriteria}, it is enough to show that $\adh(A)\subseteq A$. Take $x\in \adh(A)$. By \ref{principalAdherenceInherence}, this means that there exists a filter $F\in \powerfilters(X)$ that converges to $x$ and is such that $A\in F$. By the ultrafilter lemma, \ref{ultrafilterLemma}, $\upset \iota^{\preimf\imf}(F)$ is contained in an ultrafilter $G\in \powerfilters(A)$. By
\ref{upsetPreimageImageGaloisConnection}, this implies $F\subseteq \upset\iota^{\imf\imf}(G)$ and so $\upset\iota^{\imf\imf}(G) \to x$.

Now $G$ converges to some $a\in A$ by compactness. By continuity of the inclusion, we have that $\upset\iota^{\imf\imf}(G) \overset{\xi}{\longrightarrow} \iota(a) = a$.

As $\upset\iota^{\imf\imf}(G)$ converges to both $x$ and $a$, and $\xi$ is a Hausdorff convergence, we have $x = a$, and thus $x\in A$. This shows indeed that $\adh(A)\subseteq A$.
\end{proof}
\begin{corollary} \label{HausdorffCompactIntersection}
Let $\sSet{X,\xi}$ be a Hausdorff convergence space and $\{A_i\}_{i\in I}$ a set of compact sets. Then $\bigcap_{i\in I}A_i$ is compact.
\end{corollary}
\begin{proof}
As each $A_i$ is closed, $\bigcap_{i\in I}A_i$ is closed by \ref{propertiesTopology} and a subset of a compact set. Thus it is compact.
\end{proof}

\begin{example}
If $\sSet{X,\xi}$ is not Hausdorff, then the intersection of two compact sets can fail to be compact.

Consider the bug-eyed line $\R\sqcup\{0'\}$, where any filter converges to $0'$ iff it converges to $0$. Consider the sets $A = \interval{0,1}$ and $B= (A\setminus\{0\})\cup \{0'\}$. Then both $A$ and $B$ are compact, but $A\cap B = \interval[oc]{0,1}$ is not.
\end{example}


\begin{proposition} \label{compactClosedIntersectionCompact}
If $A$ is compact and $B\subseteq X$ is closed, then $A\cap B$ is compact.
\end{proposition}
\begin{proof}
Let $F\in \powerultrafilters(X)$ be an arbitrary ultrafilter that contains $A\cap B$. As $F$ contains $A$, it converges to some $x\in A$. As $B\in F$, all limits lie in $B$. In particular $x\in B$. Thus $x\in A\cap B$. As $F$ was taken arbitrarily, $A\cap B$ is compact.
\end{proof}

\begin{proposition} \label{compactConstructions}
Let $\sSet{X,\xi}, \sSet{Y,\zeta}$ be convergence spaces. Let $A,B\subseteq X$ be subsets and $f: \sSet{X,\xi}\to \sSet{Y,\zeta}$ a continuous function.
\begin{enumerate}
\item If $A$ is $B$-compactoid, then $f^\imf(A)$ is $f^\imf(B)$-compactoid.
\item If $A$ and $B$ are compact, then $A\cup B$ is compact.
\end{enumerate}
\end{proposition}
In particular, if $A$ is compact, then $f^\imf(A)$ is compact.
\begin{proof}
(1) Take an ultrafilter $G$ such that $f^{\imf}(A)\in G$. Then $\{A\}\amesh f^{\preimf\imf}(G)$ by \ref{meshConnectionSetsOfSets}, so $G' \defeq \upset\{A\}\vee f^{\preimf\imf}(G)$ is proper by \ref{joinProperFilter}. It is contained in an ultrafilter $U$ by \ref{ultrafilterLemma}, which converges to some $x\in B$ by assumption. Then $\upset f^{\imf\imf}(U) \to f(x)\in f^\imf(B)$ and
\[ G \subseteq \upset f^{\imf\imf}\big(f^{\preimf\imf}(G)\big) \subseteq \upset f^{\imf\imf}(U), \]
so $G = \upset f^{\imf\imf}(U)$ because $G$ is ultra and $\upset f^{\imf\imf}(U)$ is proper, so $G\to f(x)\in f^\imf(B)$.

(2) Take any ultrafilter $F$ in $A\cup B$. Then either $A\in F$ or $B$ in $F$: if not, then $A^c\in F$ and $B^c \in F$ by \ref{BooleanUltrafilterDisjunction}, so $A^c\cap B^c\cap (A\cup B) = \emptyset \in F$, which is disallowed.

In both cases $F$ must converge: either by compactness of $A$ or by compactness of $B$.
\end{proof}

\begin{theorem}[Tychonoff] \label{TychonoffsTheorem}
Let $\{\sSet{X_i, \xi_i}\}_{i\in I}$ be a family of compact convergence spaces. Then $\prod_{i\in I}X_i$ is compact.
\end{theorem}
\begin{proof}
Let $F$ be an ultrafilter in $\prod_{i\in I}X_i$. Then, by \ref{imageFilterProperties}, $\upset \proj_i^{\imf\imf}(F)$ is an ultrafilter for all $i\in I$, so it converges to some $x_i\in X_i$.

By \ref{initialFinalConvergence}, $F\to \seq{x_i}_{i\in I}$.
\end{proof}

TODO: as stated we do not need choice, however choice is equivalent when converting to the finite subcover definition of compactness for topological spaces.

\begin{proposition} \label{maximalHausdroffMinimalCompactChoquet}
Let $\sSet{X,\xi}$ be a compact, Hausdorff, Choquet convergence space, then
\begin{enumerate}
\item $\xi$ is maximal Hausdorff;
\item $\xi$ is minimal compact Choquet.
\end{enumerate}
\end{proposition}
\begin{proof}
(1) Assume there exists a convergence $\zeta$ on $X$ such that $\xi \subseteq \zeta$ and $\zeta$ is Hausdorff. We then need to show that $\zeta \subseteq \xi$.

Take arbitrary $F \overset{\zeta}{\longrightarrow} x$. Then all ultrafilters larger than $F$ converge in $\zeta$ to $x$. These ultrafilters must also converge in $\xi$, with limit $x$, by compactness of $\xi$ and Hausdorffness of $\zeta$. Then $F \overset{\xi}{\longrightarrow} x$ because $\xi$ is Choquet.

(2) Assume there exists a convergence $\zeta$ on $X$ such that $\zeta \subseteq \xi$ and $\zeta$ is compact Choquet. We then need to show that $\xi \subseteq \zeta$.

Take arbitrary $F \overset{\xi}{\longrightarrow} x$. Then all ultrafilters larger than $F$ converge in $\xi$ to $x$. These ultrafilters must also converge in $\zeta$, with limit $x$, by compactness of $\zeta$ and Hausdorffness of $\xi$. Then $F \overset{\zeta}{\longrightarrow} x$ because $\zeta$ is Choquet.
\end{proof}

\begin{proposition} \label{compactToHausdorffHomeomorphism}
Let $\sSet{X,\xi}$ be a compact Choquet space and $\sSet{Y, \zeta}$ a Hausdorff convergence space. A bijective, continuous function $f: X\to Y$ is a homeomorphism.
\end{proposition}
\begin{proof}
We just need to show $f^{-1}$ preserves convergence. Take a filter $G\overset{\zeta}{\longrightarrow} y \in Y$. Let $F$ be an arbitrary ultrafilter that contains $f^{\preimf\imf}[G]$. Then $F$ converges to some $x$ by compactness and
\[ G = (f\circ f^{-1})^{\imf\imf}[G] = \upset f^{\imf\imf}\big[f^{\preimf\imf}[G]\big]\upset \subseteq \upset f^{\imf\imf}[F] \to f(x). \]
By Hausdorff, $f(x) = y$ and so $x = f^{-1}(y)$. By Choquet (and the arbitrariness of $F$), we have $f^{\preimf\imf}[G] = (f^{-1})^{\imf\imf}[G] \to f^{-1}(y)$.
\end{proof}

\begin{proposition} \label{T3pseudotopologyTopological}
Every compact, regular, Hausdorff, Choquet convergence space is topological.
\end{proposition}
\begin{proof}
TODO
\end{proof}



\subsection{Relative compactness}
\begin{definition}
Let $\sSet{X,\xi}$ be a convergence space and $A\subseteq X$ a subset. We call $A$ \udef{relatively compact} if $\adh_\xi(A)$ is compact.
\end{definition}

Notice that if $A$ is relatively compact, it is compactoid. The converse does not hold.

\begin{lemma}
Let $\sSet{X,\xi}$ be a convergence space and $A\subseteq X$ a relatively compact subset. Then $A$ is contained in a compact set.
\end{lemma}
This is not generally true for compactoid subsets

\begin{example}
Consider $\cball(0,1)\subseteq \C$ with the usual subspace convergence. Consider the cover $\{\ball(0,1)\}\cup \bigcup_{x\in \sphere(0,1)}\big\{\{x\}\}$ and the specified sets convergence determined by this cover. Then $\ball(0,1)$ is compactoid, but not contained in any compact set.
In particular, it is not relatively compact.
\end{example}


\begin{proposition}
Let $\sSet{X,\xi}$ be a Hausdorff convergence space and $A\subseteq X$.
\begin{enumerate}
\item If $A$ is relatively compact, then it has compact closure.
\item If $\xi$ is also regular and pseudotopological, then the converse also holds.
\end{enumerate}
\end{proposition}
\begin{proof}
(1) If $\adh(A)$ is compact, then it is closed by \ref{compactClosedSets}. This means that (by \ref{interiorInherenceInclusion}) $\closure(A) = \adh(A)$ is compact.

(2) Suppose $\closure(A)$ is compact. Then, as a subspace, it is regular, pseudotopological, Hausdorff and compact (by \ref{regularityInitialConvergence}, \ref{pretopologicalInitialConvergence}, \ref{T2initialConvergence} and \ref{compactSetCompactSubspace}). This means that the subspace $\closure(A)$ is topological, by \ref{T3pseudotopologyTopological}.

Now, by \ref{subspaceAdherence} and the fact that $\xi|_{\closure(A)}$ is topological, we have
\[ \adh_\xi(A) = \adh_{\xi|_{\closure(A)}}(A) = \closure_{\xi|_{\closure(A)}}(A), \]
which is a closed subset of a compact space and thus compact by \ref{compactClosedSets}.
\end{proof}

\subsection{Local compactness}
\begin{definition}
A convergence space $\sSet{X, \xi}$ is called \udef{locally compact} if every point has a compact vicinity.
\end{definition}

\begin{lemma} \label{productLocallyCompact}
Let $\sSet{X,\xi}, \sSet{Y,\zeta}$ be locally compact convergence spaces. Then $X\times Y$ is locally compact.
\end{lemma}
\begin{proof}
Take arbitrary $(x,y)\in X\times Y$. By assumption there exists a compact $K\in \vicinity_{\xi}(x)$ and a compact $L\in \vicinity_\zeta(y)$. Then $K\times L$ is compact by Tychonoff's theorem \ref{TychonoffsTheorem} and $K\times L\in \vicinity_{\xi\otimes \zeta}(x,y) = \vicinity_\xi(x)\otimes \vicinity_\zeta(y)$ by \ref{productVicinity}. 
\end{proof}

\begin{proposition} \label{locallyCompactSubspace}
Let $\sSet{X,\xi}$ be a locally compact space and $A\subseteq X$ a subset. Then
\begin{enumerate}
\item if $A$ is locally closed, then $A$ is locally compact;
\item if $X$ is Hausdorff and $A$ is locally compact, then $A$ is locally closed.
\end{enumerate}
\end{proposition}
\begin{proof}
TODO!! Is this even correct?? Do we need topologicity (in this case show that both open and closed subsets of locally compact space are locally compact and use the a locally closed set is the intersection of these two cases)??
\end{proof}

\subsection{Countable compactness}
\begin{definition}
Let $\sSet{X,\xi}$ be a convergence space. We call $X$ \udef{countably compact} if the tails filter of every sequence in $X$ is compactoid.
\end{definition}

\begin{proposition}
TODO: $A$ countably compact iff every countable convergence cover of $A$ has a finite subset that covers $A$.
\end{proposition}

TODO \url{https://cpb-us-w2.wpmucdn.com/sites.uwm.edu/dist/0/158/files/2016/10/751.F10.IIIB-C-16qokmp.pdf}

\subsection{Sequential compactness}
\begin{definition}
Let $\sSet{X,\xi}$ be a convergence space. We call $X$ \udef{sequentially compact} if every sequence in $X$ has a convergent subsequence.
\end{definition}

\begin{example}
Compact space that is not sequentially compact.

Sequentially compact space that is not compact.
\end{example}



\subsection{(Generalised) compactness}
Lindelöf.

\subsection{Compactification}
\subsubsection{One-point compactification}
\begin{definition}
Let $\sSet{X,\xi}$ be a convergence space and $\infty$ some point not in $X$. Then the \udef{one-point comapctification} or \udef{Alexandroff compactification} of $X$ is the set $X^\dagger \defeq X\uplus \{\infty\}$ with the convergence
\[ F\to x \quad\iff\quad \begin{cases}
\text{$X\in F$ and $F|_X \to x\in X$} \\
\text{$x=\infty$, $F$ is an ultrafilter and $F$ does not converge due to the above}
\end{cases} \]
for all $F\in\powerfilters(X^\dagger)$.

We call $\topMod(X^\dagger)$ the \udef{topological one-point compactification}.
\end{definition}
Note that $X^\dagger$ is compact by construction and $X$ carries the subspace convergence as a subset of $X^\dagger$.

The Alexandroff compactification is the strongest convergence on $X\uplus \{\infty\}$ such that
\begin{itemize}
\item the original convergence on $X$ is given by the subspace convergence;
\item $X\uplus \{\infty\}$ is compact.
\end{itemize}
The weakest such convergence is given by the open extension (TODO ref), which lets all filters converge to $\infty$.

\begin{proposition} \label{topologyOnePointCompactification}
Let $\sSet{X, \xi}$ be a convergence space. Then
\[ \topology_{X^\dagger} = \topology_X \cup \setbuilder{X^\dagger \setminus K}{\text{$K\subseteq X$ closed and compact}}. \]
\end{proposition}
Note that $X^\dagger \setminus K = (X\setminus K)\cup \{\infty\}$. Also note that $X$ is open in $X^\dagger$.
\begin{proof}
We first prove the inclusion $\boxed{\subseteq}$. Take arbitrary $O\in \topology_{X^\dagger}$. First assume $\infty\notin O$, so $O\subseteq X$. Then $O$ is open by \ref{subspaceAdherence}.
Now assume $\infty\in O$. Then we need to show that $X^\dagger \setminus O = X\setminus O \subseteq X$ is closed and compact in $X$. Since $X^\dagger \setminus O$ is closed in $X^\dagger$, it is closed in $X$ by \ref{subspaceAdherence}. Since is compact in $X^\dagger$ by \ref{compactClosedSets}, it is compact in $X$.

Now we prove the inclusion $\boxed{\supseteq}$. First take $O\in \topology_X$. We show that $O$ is open using \ref{openClosedCriteria}. Take $x\in O$ and $F\in \powerfilters(X^\dagger)$. Since $x\neq \infty$, we must have that $X\in F$ and $F|_X\to x$. Since $O$ is open in $X$, we have $O\in F|_X$ and thus $O\in F$. This shows that $O$ is open in $X^\dagger$.

Finally take $K\subseteq X$ closed and compact. We show that $X^\dagger \setminus K$ is open using \ref{openClosedCriteria}. Take arbitrary $x\in X^\dagger \setminus K$ and $F\to x$. First take the case $x = \infty$. Then $F$ must be an ultrafilter and we must have $K\notin F$ (if $K\in F$, then $X\in F$ and $F|_X$ is an ultrafilter by \ref{imageFilterUltraIffUltra}, so $F|_X$ must converge to some element of $K$ and $F$ cannot converge to $\infty$ by definition). Since $F$ is an ultrafilter, this implies $X^\dagger \setminus K\in F$ by \ref{booleanMaximalFiltersIdeals}.

Finally consider the case $x\neq \infty$. Then $X\in F$ and $F|_X \to x$. Since $K$ is closed in $X$, $X\setminus K$ is open by \ref{openClosedComplement} and so $X\setminus K \in F|_X \subseteq F$. By upwards closure $X^\dagger \setminus K\in F$.
\end{proof}

\begin{lemma}
Let $\sSet{X, \xi}$ be a convergence space. Then the following are equivalent:
\begin{enumerate}
\item $X$ is not compact;
\item $X$ is dense in $X^\dagger$;
\item $X$ is dense in $\topMod(X^\dagger)$;
\item $\closure_\xi(X) = X^\dagger$;
\end{enumerate}
and the following are also equivalent:
\begin{enumerate} \setcounter{enumi}{4}
\item $X$ is compact;
\item $\infty$ is an isolated point in $X^\dagger$;
\item $\{\infty\}$ is clopen in $X^\dagger$;
\item $X$ is closed in $X^\dagger$;
\end{enumerate}
\end{lemma}
\begin{proof}
We show $(1) \Rightarrow (2) \Rightarrow (3) \Rightarrow (4)$ and $(5) \Rightarrow (6) \Rightarrow (7) \Rightarrow (8)$. Then $(1) \Leftrightarrow \neg(5)$ and $(4) \Leftrightarrow \neq (8)$, so the claimed equivalences hold.

$(1) \Rightarrow (2)$ Suppose $X$ is not compact, then there exists some ultrafilter in $X$ that does not converge. This ultrafilter converges to $\infty$ in $X^\dagger$, so $\infty\in \adh(X)$ and so $X$ is dense in $X^\dagger$.

$(2) \Rightarrow (3)$ The convergence $\topMod(\xi)$ is weaker that $\xi$, so $X^\dagger = \adh_\xi(X) \subseteq \adh_{\topMod(\xi)}(X) \subseteq X^\dagger$ and thus $\adh_{\topMod(\xi)}(X) = X^\dagger$.

$(3) \Rightarrow (4)$ Using \ref{topologicalModificationPreservation}, we have $\closure_\xi(X) = \closure_{\topMod(\xi)}(X) = \adh_{\topMod(\xi)}(X) = X^\dagger$.

$(5) \Rightarrow (6)$ Suppose $X$ is compact. We show that $\{\infty\}$ is open using \ref{openClosedCriteria}. Let $F \to \infty$ in $X^\dagger$. Then $F$ is an ultrafilter and $X\notin F$. Thus $\{\infty\}\in F$ by \ref{booleanMaximalFiltersIdeals} and so $\{\infty\}$ is open, which makes it an isolated point.

$(6) \Leftrightarrow (7)$ Since $X$ is open in $X^\dagger$ by \ref{topologyOnePointCompactification}, we have that $\{\infty\}$ is both open and closed iff it is open.

$(7) \Rightarrow (8)$ Immediate by \ref{openClosedComplement}.
\end{proof}

\begin{proposition} \label{onePointCompactificationHausdorff}
Let $\sSet{X, \xi}$ be a convergence space. Then
\begin{enumerate}
\item $X^\dagger$ is Hausdorff \textup{if and only if} $X$ is Hausdorff;
\item $\topMod(X^\dagger)$ is Hausdorff \textup{if and only if} $\topMod(X)$ is Hausdorff and locally compact.
\end{enumerate}
\end{proposition}
\begin{proof}
(1) By construction: only filters that do not already converge, converge to $\infty$.

(2) First suppose $\topMod(X^\dagger)$ is Hausdorff. Then $\topMod(X^\dagger)|_X$ is Hausdorff by \ref{HausdorffSubspace}. Since this convergence is weaker than $\topMod(X^\dagger|_X) = \topMod(X)$ by \ref{subspaceTopologicalModificationInclusion}, we have that $\topMod(X)$ is Hausdorff.

Now we show that $\topMod(X)$ is locally compact. Take arbitrary $x\in X$. Since $\topMod(X^\dagger)$ is Hausdorff and $x\neq \infty$, we can find disjoint open neighbourhoods $U$ and $V$ of $x$ and $\infty$ in $X^\dagger$ by \ref{pretopologicalHausdorff}.

By \ref{topologyOnePointCompactification}, $V$ is of the form $X^\dagger \setminus K$ for some $K\subseteq X$ which is closed and compact in $X$. Now we must have $U\subseteq K$, since $U\perp V$. Thus $K$ is a compact neighbourhood of $x$.

Finally, suppose $\topMod(X)$ is Hausdorff and locally compact. We will show that $\topMod(X^\dagger)$ is Hausdorff using \ref{pretopologicalHausdorff}. To that end, take $x\neq y$ in $X^\dagger$. If $x,y\in X$, then there exist disjoint open neighbourhoods of $x$ and $y$ in $\topMod(X)$. By \ref{topologyOnePointCompactification}, these disjoint sets are also disjoint open neighbourhoods in $\topMod(X^\dagger)$.

Now suppose $y = \infty$ and $x\in X$. As $\topMod(X)$ is locally compact, $x$ has a compact neighbourhood $K$. Now $K$ is also compact in $X$, since this convergence is weaker. Since $\topMod(X)$ is Hausdorff, $K$ is closed in $\topMod(X)$ by \ref{compactClosedSets} and so also in $X$.

We have that $K$ and $V = X^\dagger \setminus K$ are disjoint neighbourhoods of $x$ and $\infty$ by \ref{topologyOnePointCompactification}, so $\topMod(X^\dagger)$ is Hausdorff by \ref{pretopologicalHausdorff}. 
\end{proof}
\begin{corollary}
The inclusion $\incl_X: X \hookrightarrow X^\dagger$ is an open embedding.
\end{corollary}
(Embeddings are rarely open)


\begin{proposition}[Universal property of the one-point compactification] \label{universalPropertyAlexandroffCompactification}
Let $\sSet{X, \xi}$ be a convergence space, $\sSet{Y,\topology_\zeta, y_0}$ a pointed topological space and $f: X\to Y$ a continuous function such that $f^\preimf(C)$ is compact for all closed $C\subseteq Y\setminus\{y_0\}$. Then there exists a unique continuous function $f^\dagger$ such that
\[ \begin{tikzcd}
X^\dagger \ar[r, "{f^\dagger}"] & Y \\
X \ar[u, hook] \ar[ur, swap, "f"]
\end{tikzcd} \qquad \text{commutes} \]
and $f^\dagger(\infty) = y_0$.
\end{proposition}
We have $f^\dagger = \left(\begin{smallmatrix}
f \\ \constant{y_0}
\end{smallmatrix}\right): X\uplus \{\infty\} = X^\dagger \to Y$.
\begin{proof}
This function is clearly unique, since the diagram fixes its image on $X$, and $f^\dagger(\infty) = y_0$ fixes its image on $X^\dagger \setminus X$. We just need to show that $f^\dagger$ is continuous, which we show using \ref{preimageOpenClosed}.

Take an arbitrary open $U\subseteq Y$. First suppose $y_0\notin U$. Then $(f^\dagger)^\preimf(U) = f^\preimf(U)$, which is open in $X$ by continuity and \ref{preimageOpenClosed}, and thus open in $X^\dagger$ by \ref{topologyOnePointCompactification}.

Now suppose $y_0\in U$. Then, since $U^c$ is closed, $(f^\dagger)^{\preimf}(U^c) = f^\preimf(U^c)$ is closed and compact by \ref{preimageOpenClosed} and assumption.

By \ref{imagePreimageUniqueness}, we have $(f^\dagger)^\preimf(U) = X^\dagger \setminus (f^\dagger)^\preimf(U^c)$, which is open by \ref{topologyOnePointCompactification}.
\end{proof}
\begin{corollary}
Let $\sSet{X,\topology_\xi}$ be a compact Hausdorff topological space and $x_0\in X$. Then $\incl_{X\setminus\{x_0\}}^\dagger: (X\setminus\{x_0\})^\dagger\to X$ is a homeomorphism.
\end{corollary}
\begin{proof}
We have that $\incl_{X\setminus\{x_0\}}$ is continuous by definition of the subspace convergence. Take some closed $C\subseteq X\setminus\{x_0\}$. Then $\incl_{X\setminus\{x_0\}}^{\preimf}(C) = C$, which is compact in $X\setminus\{x_0\}$ by \ref{compactClosedSets} and thus compact in $X$ by \ref{compactSetCompactSubspace}. Thus $\incl_{X\setminus\{x_0\}}^\dagger$ is continuous.

Since $\incl_{X\setminus\{x_0\}}^\dagger$ is clearly bijective, it is a homeomorphism by \ref{compactToHausdorffHomeomorphism} (it is here that we use that $X$ is Hausdorff).
\end{proof}

\subsubsection{Basepoint extensions of functions}
The definition of $f^\dagger$ makes sense even if $f$ is not continuous.

In $\R$ and $\C$ the basepoint is always $0$.

\begin{lemma} \label{basepointExtensionToRingLemma}
Let $\sSet{X,\xi}$ be a convergence space. Then, for all $f,g\in (X\to \F)$ and $\lambda\in \F$, we have
\begin{enumerate}
\item $(f+\lambda g)^\dagger = f^\dagger + \lambda g^\dagger$;
\item $(f\cdot g)^\dagger = f^\dagger\cdot g^\dagger$;
\item $\overline{f}^\dagger = \overline{f^\dagger}$;
\item $\big(\constant{0}|_X\big)^\dagger = \constant{0}_{X^\dagger}$.
\end{enumerate}
\end{lemma}



\subsubsection{Functions that vanish at $\infty$}
\begin{definition}
Let $\sSet{X,\xi}$ be a convergence space and $\sSet{Y,\zeta, y_0}$ a pointed convergence space. A function $f: X\to Y$ is said to \udef{vanish at $\infty$} if
\[ \forall A\in \vicinity(y_0): \exists\, \text{compact $K\subseteq X$}: \; f^\imf(K^c) \subseteq A. \]
We denote the set of all such functions by $\cont_0(X,Y)$.

By $\cont_0(X)$ we mean $\cont_0(X,\C)$ (and sometimes $\cont_0(X,\R)$), where $0$ is the basepoint of $\C$ (and $\R$). 
\end{definition}

\begin{proposition} \label{vanishesAtInfinityIffBasepointExtensionContinuous}
Let $\sSet{X,\xi}$ be a convergence space, $\sSet{Y,\zeta, y_0}$ a pointed topological space and $f:X\to Y$ a function. Then $f$ vanishes at infinity \textup{if and only if} $f^\dagger: X^\dagger \to Y$ is continuous.
\end{proposition}
\begin{proof}
First assume $f$ vanishes at infinity. Then we use \ref{universalPropertyAlexandroffCompactification} to prove continuity. Suppose $C\subseteq Y\setminus\{y_0\}$ is closed. Then $C^c \in \neighbourhood(y_0)$ and so there exists a compact $K\subseteq X$ such that $f^\imf(K^c)\subseteq C^c$, which implies $K^c \subseteq f^\preimf(C^c) = f^\preimf(C)^c$ and thus also $f^\preimf(C)\subseteq K$. This makes $f^\preimf(C)$ compact by \ref{compactClosedSets}.

Now assume $f^\dagger$ is continuous and let $A$ be an open neighbourhood of $y_0$. Then $(f^\dagger)^\preimf(A)$ is an open neighbourhood of $\infty$ by \ref{preimageOpenClosed}. Thus there exists a closed and compact $K\subseteq X$ such that $X^\dagger\setminus K = (f^\dagger)^\preimf(A)$ by \ref{topologyOnePointCompactification}. We then have
\[ f^\imf(K^c) = (f^\dagger)^\imf(K^c) \subseteq (f^\dagger)^\imf(X^\dagger \setminus K) \subseteq A, \]
which means that $f$ vanishes at $\infty$.
\end{proof}
\begin{corollary} \label{functionVanishingAtInftyIffRestrictionOfContinuousBasepointPreservingFunction}
Let $\sSet{X,\xi}$ be a convergence space and $\sSet{Y,\zeta, y_0}$ a pointed topological space. Then
\[ \setbuilder{f\in \cont(X^\dagger, Y)}{f(\infty) = y_0} \to \cont_0(X,Y): f\mapsto f|_X \]
is a bijection.
\end{corollary}
\begin{proof}
This function is clearly injective. Now take arbitrary $f\in \cont_0(X,Y)$. Then $f = f^\dagger|_X$. Since $f^\dagger\in \setbuilder{f\in \cont(X^\dagger, Y)}{f(\infty) = y_0}$ the function is surjective.
\end{proof}



\section{Convergences on ordered sets}
\subsection{Order convergence}
\subsubsection{Convergence compatible with the order}

\begin{definition}
Let $\sSet{X,\leq}$ be a poset. A convergence $\xi$ in $X$ is called
\begin{itemize}
\item \udef{compatible with the weak order} if $F \leq_w G$, $x\in\lim_\xi F$ and $y\in \lim_\xi G$ imply $x\leq y$;
\item \udef{compatible with the strong order} if $F \leq_s G$, $x\in\lim_\xi F$ and $y\in \lim_\xi G$ imply $x\leq y$.
\end{itemize}
\end{definition}

\begin{lemma} \label{compatibleWeakOrderHausdorff}
Let $\sSet{X,\leq}$ be a poset and $\xi$ a convergence compatible with the weak order. Then $\xi$ is Hausdorff.
\end{lemma}
\begin{proof}
Take $F\in\powerfilters(X)$ and $x,y\in \lim_\xi F$. Since $\leq_w$ is reflexive by \ref{reflexivityFilterRelations}, we have $F\leq_w F$, which implies $x\leq y$ and $y\leq x$ by compatibility with the weak order. Thus $x=y$ by antisymmetry.
\end{proof}

\begin{lemma} \label{limsupLiminfInequality}
Let $\sSet{X,\leq}$ be a complete lattice and $F\in \powerfilters(X)$ a proper filter. Then
\[ \liminf F = \bigvee_{A\in F}\bigwedge A \leq \bigwedge_{A\in F}\bigvee A = \limsup F. \]
\end{lemma}
\begin{proof}
For all $A,B\in F$, we have $A\cap B\neq \emptyset$, so
\[ \bigwedge A \leq \bigwedge A\cap B \leq \bigvee A\cap B \leq \bigvee B. \]
This implies $\bigvee_{A\in F}\bigwedge A \leq \bigwedge_{A\in F}\bigvee A$.
\end{proof}

\begin{lemma} \label{completePosetConvergenceCompatibleWithWeakOrder}
Let $\sSet{X,\leq}$ be a complete lattice and $\xi$ a convergence on $X$. Then
\begin{enumerate}
\item if $\lim_\xi F\subseteq \{\liminf F\}\cap\{\limsup F\}$ for all $F\in\powerfilters(X)$, then $\xi$ is compatible with the weak order;
\item if $\lim_\xi F\subseteq \interval{\liminf F, \limsup F}$ for all $F\in\powerfilters(X)$, then $\xi$ is compatible with the strong order.
\end{enumerate}
\end{lemma}
\begin{proof}
(1) Take $F,G\in \powerfilters(X)$ such that $F\leq_w G$. Take $x\in \lim F$ and $y\in \lim G$. By assumption, $x = \liminf F$ and $y = \limsup G$. By \ref{filterInequalityCriterion} we have $x\leq y$.

(2) Take $F,G\in \powerfilters(X)$ such that $F\leq_s G$. Take $x\in \lim F$ and $y\in \lim G$. By assumption, $x \leq \limsup F$ and $y \geq \liminf G$. By \ref{filterInequalityCriterion} we have $\limsup F\leq \liminf G$, so $x\leq \limsup F\leq \liminf G \leq y$.
\end{proof}


\subsubsection{Convergence potentially compatible with weak order}
\begin{definition}
Let $\sSet{X,\leq}$ be a poset. Consider
\[ F \mapsto \begin{cases}
\Big(\bigcap_{A\in F}\upset A\Big)^l \cap \Big(\bigcap_{A\in F}\downset A\Big)^u & \Big(\bigcap_{A\in F}\upset A \neq \emptyset \neq \bigcap_{A\in F}\downset A\Big) \\
\emptyset & (\text{otherwise}).
\end{cases} \]
for all filters $F\in \powerfilters(X)$.
\end{definition}

\begin{proposition} \label{necessityWeakOrderConvergence}
Let $\sSet{X,\leq}$ be a poset, $\xi$ a convergence on $X$ that is compatible with the weak order and $F\in \powerfilters(X)$. Then
\[ \lim_\xi F \subseteq \Big(\bigcap_{A\in F}\upset A\Big)^\geq \cap \Big(\bigcap_{A\in F}\downset A\Big)^\leq. \]
\end{proposition}
\begin{proof}
Let $\xi$ be an arbitrary convergence that is compatible with the weak order.

Take $y\in \lim_\xi(F)$. Take arbitrary $x\in \bigcap_{A\in F}\upset A$. Then $F \leq_w \pfilter{x}$ by \ref{weakRelationPrincipalFilter}, so $y\leq x$ by compatibility with the weak order. Since this holds for all such $x$, we have $y\in \Big(\bigcap_{A\in F}\upset A\Big)^l$. Similarly $y\in \Big(\bigcap_{A\in F}\downset A\Big)^u$.
\end{proof}


\begin{lemma} \label{potentialWeakOrderConvergence}
Let $\sSet{X,\leq}$ be a poset and $F\in \powerfilters(X)$. Then
\[ \lim(F) \defeq \Big(\bigcap_{A\in F}\upset A\Big)^\geq \cap \Big(\bigcap_{A\in F}\downset A\Big)^\leq \]
defines a convergence.
\end{lemma}
\begin{proof}
It is monotone by construction: if $F\subseteq F'$, then $\bigcap_{A\in F}\upset A \supseteq \bigcap_{A\in F'}\upset A$ and thus $\Big(\bigcap_{A\in F}\upset A\Big)^\geq \subseteq \Big(\bigcap_{A\in F'}\upset A\Big)^\geq$. A similar inclusion holds for $\Big(\bigcap_{A\in F}\downset A\Big)^\leq$.

To show that it is centered, take $x\in X$. Then
\begin{align*}
\lim_\mathfrak{o}\pfilter{x} &= \Big(\bigcap_{A\in \pfilter{x}}\upset A\Big)^\geq \cap \Big(\bigcap_{A\in \pfilter{x}}\downset A\Big)^\leq \\
&= \Big(\upset \{x\}\Big)^\geq \cap \Big(\downset \{x\}\Big)^\leq \\
&= \downset \{x\} \cap \upset \{x\} \\
&= \{x\}.
\end{align*}
\end{proof}

\begin{proposition}
Let $\sSet{X,\leq}$ be a poset and let $\xi$ be the convergence defined in \ref{potentialWeakOrderConvergence}. Then the following are equivalent:
\begin{enumerate}
\item $X$ has a weakest convergence compatible with weak order;
\item $\xi$ is Hausdorff;
\item $\xi$ is compatible with the weak order.
\end{enumerate}
\end{proposition}
\begin{proof}
$(1) \Rightarrow (2)$ We prove this by contraposition. Assume the convergence in \ref{potentialWeakOrderConvergence} is not Hausdorff. Then there exists $F\in \powerfilters(X)$ and $x\neq y \in X$ such that $\{x,y\}\subseteq \lim_\xi(F)$. Consider the convergence $\zeta_x$ and $\zeta_y$ on $X$ defined by
\begin{align*}
G\overset{\zeta_x}{\longrightarrow} a \quad&\defequiv\quad \big(G = \pfilter{a}\big) \lor \Big( (F\subseteq G)\land (a = x) \Big) \\
G\overset{\zeta_y}{\longrightarrow} a \quad&\defequiv\quad \big(G = \pfilter{a}\big) \lor \Big( (F\subseteq G)\land (a = y) \Big).
\end{align*}
It is enough to show that both $\zeta_x$ and $\zeta_y$ are compatible with the weak order, since in this case $\zeta_x \vee \zeta_y$ is not Hausdorff and thus not compatible with the weak order by \ref{compatibleWeakOrderHausdorff}.

For $\zeta_x$, take $G, H\in \powerfilters(X)$ such that $G,H$ are convergent and $G\leq_w H$. If $G,H$ are both principal ultrafilters, then weak compatibility follows from \ref{weakRelationPrincipalFilter}. If neither is a principal ultrafilter, then both converge to $x$ and we weak compatibility.
If $H = \pfilter{a}$ and $G\supseteq F$. Then $F\leq_w \pfilter{a}$ by \ref{upDownClosureWeakStrongRelation}, so $a \in \bigcap_{A\in F}\upset A$ by \ref{weakRelationPrincipalFilter} and $x\leq a$ by assumption. The case $G = \pfilter{a}$ is similar.

The convergence $\zeta_y$ is compatible with weak order by a similar argument.

$(2) \Rightarrow (3)$ TODO!

$(3) \Rightarrow (1)$ Immediate by \ref{necessityWeakOrderConvergence}.
\end{proof}


\begin{example}
Consider $P \defeq \setbuilder{(a,b)\in \R^2}{(a=0)\lor (b=0)}$ with ordering
\[ (a,b) \leq (c,d) \;\defequiv\; (b<d) \lor \big((a,b) = (c,d)\big). \]
Now set $A_\epsilon = \setbuilder{(x, 0)}{-\epsilon \leq x\leq \epsilon} \setminus\{(0,0)\}$. Then
\begin{itemize}
\item $F \defeq \upset \{A_\epsilon\}_{\epsilon > 0}$ is a filter;
\item $\bigcap_{\epsilon > 0}\upset A_\epsilon = \setbuilder{(0,y)}{y > 0}$ and $\bigcap_{\epsilon > 0}\downset A_\epsilon = \setbuilder{(0,y)}{y < 0}$;
\item $\Big(\bigcap_{\epsilon > 0}\upset A_\epsilon\Big)^\geq = \setbuilder{(0,y)}{y < 0}\cup \R\times\{0\}$ and $\Big(\bigcap_{\epsilon > 0}\downset A_\epsilon\Big)^\leq = \setbuilder{(0,y)}{y > 0}\cup \R\times\{0\}$;
\item $\Big(\bigcap_{\epsilon > 0}\upset A_\epsilon\Big)^\geq \cap\Big(\bigcap_{\epsilon > 0}\downset A_\epsilon\Big)^\leq = \R\times\{0\}$.
\end{itemize}
Thus $P$ has no weakest convergence compatible with weak ordering.
\end{example}

 

\begin{example}
Let $X$ be an infinite set. Consider $P = \{\top, \bot\}\uplus X$ with the following order: $x\leq y$ if and only if either $x = \bot$, $y = \top$ or $x = y$.

Consider any free filter $F$ on $X$. Then
\[ \Big(\bigcap_{A\in F}\upset A\Big)^\geq \cap \Big(\bigcap_{A\in F}\downset A\Big)^\leq = X. \]
\end{example}

\begin{lemma} \label{cupCapInversionFilters}
Let $\sSet{X,\leq}$ be a poset and $F\in \powerfilters(X)$ a filter. Then
\[ \bigcup_{A\in F}\bigcap_{a\in A}\upset\{a\} \subseteq \bigcap_{A\in F}\bigcup_{a\in A}\upset\{a\} \qquad\text{and}\qquad \bigcup_{A\in F}\bigcap_{a\in A}\downset\{a\} \subseteq \bigcap_{A\in F}\bigcup_{a\in A}\downset\{a\}. \]
\end{lemma}
\begin{proof}
Take arbitrary $A,B\in F$. Then $A\cap B$ is non-empty and we have
\[ \bigcap_{a\in A}\upset \{a\} \subseteq \bigcap_{a\in A\cap B}\upset \{a\} \subseteq \bigcup_{a\in A\cap B}\upset \{a\} \subseteq \bigcup_{a\in B}\upset \{a\}. \]
Since this inclusion is valid for all $A,B\in F$, it is equivalent to $\bigcup_{A\in F}\bigcap_{a\in A}\upset\{a\} \subseteq \bigcap_{A\in F}\bigcup_{a\in A}\upset\{a\}$.

The other statement is dual.
\end{proof}
\begin{corollary} \label{boundsInFilterInclusion}
Let $\sSet{X,\leq}$ be a poset and $F\in \powerfilters(X)$ a filter. Then
\begin{enumerate}
\item $\bigcup_{A\in F}A^\geq \subseteq \bigcap_{A\in F}A^{\leq\geq}$;
\item $\bigcup_{A\in F}A^\leq \subseteq \bigcap_{A\in F}A^{\geq\leq}$;
\item $\Big(\bigcap_{A\in F}A^{\geq\leq}\Big)^\geq \subseteq \bigcap_{A\in F}A^{\leq\geq}$;
\item $\Big(\bigcap_{A\in F}A^{\leq\geq}\Big)^\leq \subseteq \bigcap_{A\in F}A^{\geq\leq}$;
\end{enumerate}
\end{corollary}
\begin{proof}
(1) Since $(-)^\geq$ is order-reversing, we calculate
\begin{align*}
\bigcup_{A\in F}A^\geq &= \bigcup_{A\in F}(\upset A)^\geq \\
&\subseteq \Big(\bigcap_{A\in F} \upset A\Big)^\geq \\
&= \Big(\bigcap_{A\in F} \bigcup_{a\in A} \upset \{a\}\Big)^\geq \\
&\subseteq \Big(\bigcup_{A\in F} \bigcap_{a\in A} \upset \{a\}\Big)^\geq \\
&= \Big(\bigcup_{A\in F} A^\leq\Big)^\geq \\
&= \bigcap_{A\in F}A^{\leq\geq},
\end{align*}
using \ref{cupCapInversionFilters}, \ref{boundsFromPrincipalImages} and \ref{polarOfUnion}.

(2) Dual.

(3) We calculate
\[ \Big(\bigcap_{A\in F}A^{\geq\leq}\Big)^\geq \subseteq \Big(\bigcup_{A\in F}A^{\leq}\Big)^\geq = \bigcap_{A\in F}A^{\leq\geq}, \]
using \ref{polarOfUnion}.

(4) Dual.
\end{proof}

\subsubsection{Order convergence}
\begin{definition}
Let $\sSet{X,\leq}$ be a poset. The \udef{order convergence} $\mathfrak{o}$ is the initial convergence w.r.t. $m: X \to \DM(X)$, the embedding of $X$ into the Dedekind-MacNeille completion.
\end{definition}

\url{https://arxiv.org/pdf/2012.13752.pdf}

\begin{lemma} \label{orderConvergenceCompatibleWeakOrder}
Let $\sSet{X,\leq}$ be a poset. The order convergence on $X$ is compatible with the weak order.
\end{lemma}
\begin{proof}
Suppose $F\overset{\mathfrak{o}}{\longrightarrow} x$, $G\overset{\mathfrak{o}}{\longrightarrow} y$ and $F\leq_w G$. Then $\upset m^{\imf\imf}(F) \overset{\mathfrak{o}}{\longrightarrow} m(x)$ and $\upset m^{\imf\imf}(G) \overset{\mathfrak{o}}{\longrightarrow} m(y)$ by definition and $\upset m^{\imf\imf}(F) \leq_w \upset m^{\imf\imf}(G)$ by \ref{relationPreservingfunctionToFilterInequality}. So $m(x)\leq m(y)$ by \ref{completePosetConvergenceCompatibleWithWeakOrder} and thus $x\leq y$. 
\end{proof}

\begin{proposition} \label{orderConvergenceEquivalents}
Let $\sSet{X,\leq}$ be a poset and $F\in\powerfilters(X)$. Then
the following are equivalent:
\begin{enumerate}
\item $F\overset{\mathfrak{o}}{\longrightarrow} x$;
\item $\bigwedge_{A\in F}\bigvee_{\DM(P)} m^\imf(A) = \downset\{x\} = \bigvee_{A\in F}\bigwedge_{\DM(P)} m^\imf(A)$;
\item $\bigcap_{A\in F}A^{\leq \geq} = \downset \{x\}$ and $\bigcap_{A\in F}A^{\geq\leq} = \upset\{x\}$;
\item $\bigcap_{A\in F}A^{\leq \geq} \subseteq \downset \{x\}$ and $\bigcap_{A\in F}A^{\geq\leq} \subseteq \upset\{x\}$;
\item $\upset\{x\} \subseteq \Big(\bigcap_{A\in F}A^{\leq \geq}\Big)^\leq$ and $\downset\{x\} \subseteq \Big(\bigcap_{A\in F}A^{\geq \leq}\Big)^\geq$;
\item $x\in \Big(\bigcap_{A\in F}A^{\leq \geq}\Big)^\leq \cap \Big(\bigcap_{A\in F}A^{\geq\leq}\Big)^\geq$.
\end{enumerate}
\end{proposition}
\begin{proof}
The equivalence $(1) \Leftrightarrow (2)$ is a restatement of the definition.

$(2) \Leftrightarrow (3)$ First note that
\[ \bigwedge_{A\in F}\bigvee_{\DM(P)} m^\imf(A) = \bigwedge_{A\in F}A^{\leq\geq} = \bigcap_{A\in F}A^{\leq\geq}, \]
by \ref{joinMeetDMImage}. Also
\[ \bigcap_{A\in F}A^{\geq\leq} = \Big(\bigcup_{A\in F}A^\geq\Big)^\leq. \]
Then, because the polars form a Galois connection and thus are generalised inverses, as well as by \ref{polarOfUnion}, \ref{joinMeetDMImage} and \ref{DedekindMacNeilleCompletionComplete}, we have
\begin{align*}
\upset\{x\} = \Big(\bigcup_{A\in F}A^\geq\Big)^\leq \iff \downset\{x\} &= \Big(\bigcup_{A\in F}A^\geq\Big)^{\leq\geq} \\
&= \Big(\bigcup_{A\in F}\bigwedge_{\DM(P)}m^\imf(A)\Big)^{\leq\geq} \\
&= \bigvee_{A\in F}\bigwedge_{\DM(P)} m^\imf(A).
\end{align*}

$(3) \Leftrightarrow (4)$ The direction $\Rightarrow$ is immediate. For the converse, we first prove $\downset \{x\}\subseteq \bigcap_{A\in F}A^{\leq \geq}$.

From the second part, we calculate
\[ \downset\{x\} = \big(\upset\{x\}\big)^\geq \subseteq \Big(\bigcap_{A\in F}A^{\geq\leq}\Big)^\geq \subseteq \bigcap_{A\in F}A^{\leq\geq}, \]
using \ref{boundsInFilterInclusion}.

$(4) \Rightarrow (5)$ Follows because taking the polar is antitone.

$(5) \Rightarrow (6)$ Immediate.

$(6) \Rightarrow (4)$ Take $y\in \bigcap_{A\in F} A^{\leq \geq}$. Then $y\leq x$ by assumption, so $y\in \downset\{x\}$. The other assertion is similar.
\end{proof}

\begin{proposition}
Let $\sSet{X,\leq}$ be a poset. Then the order convergence on $X$ makes it a Kent space.
\end{proposition}
\begin{proof}
Take arbitrary $F \overset{\mathfrak{o}}{\longrightarrow} x$.
To show that $F\cap \pfilter{x}\overset{\mathfrak{o}}{\longrightarrow} x$, we can, by \ref{orderConvergenceEquivalents}, show that
\[ \bigcap_{A\in F}\big(A\cup \{x\}\big)^{\leq\geq} \subseteq \downset\{x\} \quad\text{and}\quad \bigcap_{A\in F}\big(A\cup \{x\}\big)^{\geq\leq} \subseteq \upset\{x\}. \]
For the first claim, we calculate using \ref{polarOfUnion},
\begin{align*}
\bigcap_{A\in F}\big(A\cup \{x\}\big)^{\leq\geq} &= \bigcap_{A\in F}\big(A^\leq \cap \{x\}^\leq\big)^{\geq} \\
&= \Big(\bigcup_{A\in F}\big(A^\leq \cap \{x\}^\leq\big)\Big)^{\geq} \\
&= \Big(\upset \{x\} \cap \bigcup_{A\in F}A^\leq \Big)^{\geq},
\end{align*}
so the first claim is equivalent to $\upset \{x\} \subseteq \upset \{x\} \cap \bigcup_{A\in F}A^\leq$ by the polar Galois connection \ref{polarsGaloisConnection}. This inclusion holds since $\upset \{x\} \subseteq \bigcup_{A\in F}A^\leq$ by \ref{orderConvergenceEquivalents}.

The second claim is dual.
\end{proof}

TODO: Erné and Weck, Order convergence in lattices

\begin{example}
The order convergence on a poset is in general not of finite depth.
\end{example}


\begin{proposition} \label{orderConvergenceIntervalBase}
Let $\sSet{X,\leq}$ be a poset and $F\subseteq \powerfilters(X)$ a filter. Then $F\overset{\mathfrak{o}}{\longrightarrow} x$ \textup{if and only if}
\begin{itemize}
\item there exist sets $M,N\subseteq X$ such that $\bigvee M = x = \bigwedge N$ (and these extrema exist); and
\item $\interval{a,b}\in F$ for all $a\in M$ and $b\in N$.
\end{itemize}
If $X$ is a lattice, then $M$ may be taken to be upwards directed and $N$ may be taken to be downwards directed.
\end{proposition}
In particular, if $F$ is the tails filter of a net $\seq{x_i}$, then $N\subseteq \upset\im(\seq{x_i})$ and $M\subseteq \downset\im(\seq{x_i})$.
\begin{proof}
First assume $F\overset{\mathfrak{o}}{\longrightarrow} x$. Then set $M \defeq \bigcup_{A\in F}A^\geq$ and $N \defeq \bigcup_{A\in F}A^\leq$. Then, by \ref{orderConvergenceEquivalents} and \ref{polarOfUnion},
\[ M^\leq = \bigcap_{A\in F}A^{\geq\leq} = \upset\{x\}, \]
so $\bigvee M = x$ and
\[ N^\geq = \bigcap_{A\in F}A^{\leq\geq} = \downset\{x\}, \]
so $\bigwedge N = x$. Now take $a\in M$ and $b\in N$. Then there exists $A,B\in F$ such that $a\in A^\geq$ and $b\in B^\leq$. This implies $A\cap B \subseteq \interval{a,b}$. Since $A\cap B\in F$, we have $\interval{a,b}\in F$.

Now assume the second part. Then we have
\begin{align*}
\bigwedge_{A\in F}\bigvee_{\DM(P)} m^\imf(A) &\subseteq \bigwedge_{a\in M, b\in N}\bigvee_{\DM(P)} m^\imf\big(\interval{a,b}\big) \\
&= \bigwedge_{a\in M, b\in N} m\Big(\bigvee_{P}\interval{a,b}\Big) \\
&= \bigwedge_{a\in M, b\in N} m(b) = \bigwedge_{\DM(P)} m^\imf(N) = m\Big(\bigwedge_P N\Big) \\
&= m(x) \\
&= m\Big(\bigvee_P M\Big) = \bigvee_{\DM(P)} m^\imf(M) = \bigvee_{a\in M, b\in N} m(a) \\
&= \bigvee_{a\in M, b\in N} m\Big(\bigwedge_{P}\interval{a,b}\Big) \\
&= \bigvee_{a\in M, b\in N}\bigwedge_{\DM(P)} m^\imf\big(\interval{a,b}\big) \\
&\subseteq \bigvee_{A\in F}\bigwedge_{\DM(P)} m^\imf\big(A\big).
\end{align*}
By \ref{limsupLiminfInequality}, we also have $\bigvee_{A\in F}\bigwedge_{\DM(P)} m^\imf\big(A\big) \subseteq \bigwedge_{A\in F}\bigvee_{\DM(P)} m^\imf(A)$, so the inclusions are in fact equalities. We have $F\overset{\mathfrak{o}}{\longrightarrow} x$ by \ref{orderConvergenceEquivalents}.

Finally the claim about the directedness of $M$ and $N$ follows from \ref{latticeIntervalIntersection}.
\end{proof}

\begin{lemma}
Let $\sSet{X,\leq}$ be a poset, $a,x\in X$ and $\seq{x_i}_{i\in I}$ a net in $X$ that order converges to $x$. If $\upset\{a\}$ is $\seq{x_i}$-frequent, then $a\leq x$.
\end{lemma}
\begin{proof}
Consider the sets $M,N$ from \ref{orderConvergenceIntervalBase}. Take arbitrary $b\in N$. Then there exists $a\in M$ such that $\interval{a,b}$ is $\seq{x_i}$-eventual (i.e.\ an element of the tails filter). Then $\upset\{a\}$ meshes with $\interval{a,b}$, so there exists $c\in \upset\{a\}\cap \interval{a,b}$. This implies $a\leq c\leq b$ so $b\in \upset\{a\}$. Thus we have $N\subseteq \upset\{a\}$. Taking infima gives $x = \bigwedge N \geq a$.
\end{proof}

\begin{lemma} \label{orderLimitMeetJoin}
Let $L$ be a lattice, $x\in X$ and $\seq{x_i}_{i\in I}$ a net in $L$ that order converges to $x$. Then
\begin{enumerate}
\item $x = \bigwedge_{i\in I}x_i \vee x$;
\item $x = \bigvee_{i\in I}x_i \wedge x$.
\end{enumerate}
\end{lemma}
\begin{proof}
(1) By construction $x$ is a lower bound of $x_i \vee x$. By \ref{orderConvergenceIntervalBase}, there exists a set $N\subseteq L$ such that $x = \bigwedge N$.

We have $N\subseteq \upset\setbuilder{x_i \vee x}{i\in I}$. Indeed, take $b\in N$. Then clearly $x\leq b$ since it is a lower bound of $N$. We also have that $x_i \leq b$ for some $i\in I$. Thus $x_i\vee x \leq b$ by \ref{orderLatticeCorollary}.

We conclude that any lower bound of $\upset\setbuilder{x_i \vee x}{i\in I}$ is a lower bound of $N$ and thus less than $x$.

(2) Dual.
\end{proof}

\begin{proposition} \label{singleArgumentMeetJoinOrderContinuous}
Let $L$ be a lattice and $x\in L$. Then
\begin{enumerate}
\item if $L$ is infinitely join-distributive, then the function $x\wedge -: L\to L$ is order-continuous;
\item if $L$ is infinitely meet-distributive, then the function $x\vee -: L\to L$ is order-continuous.
\end{enumerate}
\end{proposition}
\begin{proof}
(1) Suppose $F\overset{\mathfrak{o}}{\longrightarrow} y$. Take $M,N\subseteq L$ as in \ref{orderConvergenceIntervalBase}. Then $\bigvee x\wedge M = x\wedge \bigvee M = x\wedge y$ and $\bigwedge x\wedge N = x\wedge \bigwedge N = x\wedge y$.

Now, take $x\wedge a \in x\wedge M$ and $x\wedge b \in x\wedge N$. Since $\interval{a, b} \in F$, we have $x\wedge \interval{a,b} \subseteq \interval{x\wedge a, x\wedge b} \in x\wedge F$. We conclude by \ref{orderConvergenceIntervalBase}.

(2) Dual to (1).
\end{proof}

\subsection{Squeeze convergence}
\begin{definition}
Let $\sSet{X,\leq}$ be a poset.
\end{definition}



\begin{proposition}[Squeeze theorem] \label{squeezeTheorem}
Let $\sSet{X,\leq}$ be a poset. Let $\seq{x_i}_{i\in I}, \seq{y_i}_{i\in I},\seq{z_i}_{i\in I}$ be nets in $X$ and $y\in X$.
Suppose
\begin{itemize}
\item $\seq{x_i}_{i\in I}$ is monotonically increasing and $\bigvee_{i\in I}x_i = y$;
\item $\seq{z_i}_{i\in I}$ is monotonically decreasing and $\bigwedge_{i\in I}z_i = y$;
\item there exists $i_0\in I$ such that $x_i \leq y_i \leq z_i$ for all $i\geq i_0$.
\end{itemize}
Then $y_i \overset{\mathfrak{o}}{\longrightarrow} y$.
\end{proposition}
\begin{proof}
We use \ref{orderConvergenceIntervalBase}. Set $M = \setbuilder{x_i}{i\in I}$ and $N = \setbuilder{z_i}{i\in I}$. Now take $x_i\in M$ and $z_j\in N$. Then, since $I$ is directed, there exists $k\in I$ such that $i,j\leq k$. Now for all $k'\geq k$, we have
\[ x_i \;\leq\; x_k \;\leq\; x_{k'} \;\leq\; y_{k'} \;\leq\; z_{k'} \;\leq\; z_k \;\leq\; z_j, \]
so $\setbuilder{y_{k'}}{k'\geq k} \subseteq \interval{x_i, z_j}$ and thus $\interval{x_i, z_j}\in\TailsFilter\seq{y_i}_{i\in I}$.
\end{proof}
\begin{corollary}
Let $\sSet{X,\leq}$ be a poset, $x\in X$ and $\seq{x_i}_{i\in I}$ a net in $X$.
\begin{enumerate}
\item If $\seq{x_i}$ is monotonically increasing with $\bigvee_{i\in I}x_i = x$, then $x_i \overset{\mathfrak{o}}{\longrightarrow} x$.
\item If $\seq{x_i}$ is monotonically decreasing with $\bigwedge_{i\in I}x_i = x$, then $x_i \overset{\mathfrak{o}}{\longrightarrow} x$.
\end{enumerate}
\end{corollary}
\begin{proof}
We can apply the squeeze theorem by bounding the net on the other side by the constant net $\constant{x}$.
\end{proof}
TODO: converse of \ref{squeezeTheorem} for sequences in Riesz space??

Even though to converse of the squeeze theorem fails (even for sequences in lattices), we have \ref{orderLimitMeetJoin}.

\begin{example}
The converse of \ref{squeezeTheorem} does not always hold, even for sequences and even in a lattice.

Let $X$ be an uncountable set and let $Y = \setbuilder{A\subseteq X}{\text{$A$ is finite or $X\setminus A$ is finite}}$ be ordered by inclusion. Let $\seq{y_n}_{n\in\N}$ be a sequence of distinct singletons.

Then $y_n \overset{\mathfrak{o}}{\longrightarrow} \emptyset$ by \ref{orderConvergenceIntervalBase}, since the meet of the cofinite sets is the empty set and every cofinite set contains a tail of $\seq{y_n}$.

There is, however, no monotonically decreasing sequence $\seq{x_n}$ such that $\bigwedge_{n\in\N}x_n = \emptyset$. Suppose, towards a contradiction, that there were such a sequence. Then $X\setminus \Big(\bigcap_{n\in \N}x_n\Big) = \bigcup_{n\in \N}X\setminus x_n$ is countable. Since $X$ is uncountable, we can take $x\in \bigcap_{n\in \N}x_n$. Thus $\{x\}\subseteq x_n$ for all $n\in \N$. Now suppose $\bigwedge_{n\in \N}x_n$ exists. Then $\{x\}\subseteq \bigwedge_{n\in \N}x_n \neq \emptyset$.
\end{example}

\begin{example}
If we modify the third point of \ref{squeezeTheorem} to read ``$x_i \leq y_i \leq z_i$ for all $i\in I$'', then it catches even fewer convergent nets.

Consider $I = \{0,1\}\times\N$ with lexicographical ordering. Consider the net $\seq{y_i}_{i\in I}$ determined by
\[ y_{(i,k)} = \begin{cases}
-k & (i=0) \\
0 & (i=1)
\end{cases} \]
Clearly $y_i \to 0$, but there exists no monotonically increasing net less than $y_i$ for all $i\in I$.
\end{example}

\subsubsection{Topologicity of order convergence}
See ``Birkhoff's order-convergence in partially ordered sets'' and ``On Order-Convergence'' by Wolk.



\subsection{Convergence on $\overline{\N}$}
In this section we consider $\overline{\N} = \N \cup \{\infty\}$, the Dedekind-MacNeille completion of $\N$ with the order convergence.

\begin{lemma} \label{extendedNaturalsOrderConvergenceVicinities}
Let $n\in\overline{\N}$. Then the order convergence on $\overline{\N}$ is pretopological, with
\[ \vicinity_{\overline{\N}}(n) = \begin{cases}
\pfilter{n} & (n\in \N) \\
\upset \setbuilder{\interval{k, \infty}}{k\in \N} & (n = \infty)
\end{cases} \]
\end{lemma}
TODO: clean up proof.
\begin{proof}
First we note that that vicinity filter converges everywhere.

Take an arbitrary proper filter $F\in\powerfilters(\overline{\N})$. First suppose $F\overset{\mathfrak{o}}{\longrightarrow} n$ with $n\in \N$. Then $\bigvee_{A\in F}\bigwedge A = n = \bigwedge_{B\in F}\bigvee B$. This implies $n\in \setbuilder{\bigwedge A}{A\in F}$ and $n\in \setbuilder{\bigvee B}{B\in F}$ by (TODO ref!!). Pick $A,B\in F$ such that $\bigwedge A = n = \bigvee B$. Then $A\cap B = \{n\}$, so $F = \pfilter{n}$.

Now suppose $F\overset{\mathfrak{o}}{\longrightarrow} \infty$. We need to show that $\vicinity_{\overline{\N}}(\infty) \subseteq F$. To that end, take arbitrary $k\in \N$. Since $\bigvee_{A\in F} \bigwedge A = \infty > k$, there exists $A\in F$ such that $\bigwedge A \geq k$. Then $A\subseteq \upset\{k\} = \interval{k,\infty}$.
\end{proof}

\begin{lemma} \label{extendedNaturalsTopological}
The function
\[ d_{\overline{\N}}: \overline{\N}\times \overline{\N} \to \R^+: (m,n) \mapsto \begin{cases}
\Big|\frac{1}{m} - \frac{1}{n}\Big| & (m,n\in \N) \\
\Big|\frac{1}{m}\Big| & (m\in \N, n=\infty) \\
\Big|\frac{1}{n}\Big| & (m=\infty, n\in \N) \\
0 & (m= \infty = n)
\end{cases} \]
is a metric and the metric convergence on $\overline{\N}$ coincides with the order convergence. In particular,
\begin{enumerate}
\item the order convergence on $\overline{\N}$ is topological;
\item the set $\setbuilder{\{n\}}{n\in \N} \cup \setbuilder{\interval{k,\infty}}{k\in \N}$ forms a basis of the topology;
\item the convergence space $\overline{\N}$ is compact.
\end{enumerate}
\end{lemma}
\begin{proof}
This function clearly satisfies the requirements of symmetry, definiteness and the triangle inequality.

Follows by comparison of \ref{extendedNaturalsOrderConvergenceVicinities} and \ref{metricConvergenceNeighbourhood}.

(1), (2) From \ref{metricConvergenceNeighbourhood}.

(3) By \ref{topologyCompactnessOpenCover}, we need to show that every open cover of $\overline{\N}$ has a finite subset that covers $\overline{\N}$. Take any open cover $C$ of $\overline{\N}$. Then $C$ must contain a set $A$ that contains $\infty$. By (2) there exists $k\in \N$ such that $\interval{k,\infty}$ is contained in $A$. For each $l\leq k$, there exists a $B_l\in C$ such that $l\in B_l$. Then $\{A\}\cup \setbuilder{B_l}{l\leq k}$ is a finite subcover.
\end{proof}

\begin{lemma} \label{continuitySequenceAsFunction}
Let $\sSet{X,\xi}$ be a convergence space, $\seq{x_n}_{n\in \N}$ a sequence in $X$ and $x\in X$. If the function
\[ \overline{\N} \to X: n \mapsto \begin{cases}
x_n & (n\in \N) \\
x & (n = \infty)
\end{cases} \]
is continuous, then $x_n \overset{\xi}{\longrightarrow} x$.

If $X$ is a Kent space, then the converse also holds.
\end{lemma}
\begin{proof}
Call this function $f$. By \ref{pretopologicalContinuityVicinities}, the continuity of $f$ is equivalent to the convergence of $f^\imf\big(\vicinity_{\overline{\N}}(n)\big)$ to $f(n)$ for all $n\in \overline{\N}$. For all $n\in \N$ this is trivial, since then $f^\imf\big(\vicinity_{\overline{\N}}(n)\big) = \pfilter{f}(n)$.
Thus the continuity of $f$ is equivalent to the convergence of $f^\imf\big(\vicinity_{\overline{\N}}(\infty)\big)$ to $f(\infty) = x$.
Now, by \ref{extendedNaturalsOrderConvergenceVicinities},
\[ \upset f^\imf\big(\vicinity_{\overline{\N}}(\infty)\big) = \upset \setbuilder[\big]{\setbuilder{x_n}{n\geq k}\cup x}{k\in \N} = \TailsFilter\seq{x_n}\cap \pfilter{x}. \]

Thus the continuity of $f$ implies the convergence of $\TailsFilter\seq{x_n}$.

The converse immediately follows from the Kent property.
\end{proof}


\subsection{TODO: remove}
\begin{proposition} \label{leftRightConvergence}
Let $\sSet{P,\leq}$ be a poset equipped with order convergence, $\sSet{X,\xi}$ a convergence space of finite depth and $f: P\to X$ a function. Then $f$ is continuous at $x_0\in P$ \textup{if and only if} it is left and right continuous at $x_0$.
\end{proposition}
\begin{proof}
If $f$ is continuous at $x_0\in P$, then it is left and right continuous at $x_0$ by \ref{continuityRestrictionExpansion}.

Conversely, assume $f$ is left and right continuous and take $F$ bounded by $\seq{l_i}_{i\in I}$ and $\seq{u_i}_{i\in I}$ such that $F\to x_0$. Define $F_0 = \mathfrak{F}\setbuilder{[l_i, u_i]}{i\in I} = \upset \setbuilder{[l_i, u_i]}{i\in I}$ and note that for any $A\in F_0$, there exists an $i\in I$ such that $[l_i,u_i]\subseteq A$.


Then $F_0|_{\downset x_0}$ is bounded by $\seq{l_i}_{i\in I}$ and $\seq{x_0}$ and $F_0|_{\upset x_0}$ is bounded by $\seq{x_0}$ and $\seq{u_i}_{i\in I}$. So $f[F_0|_{\downset x_0}] \to f(x_0)$ and $f[F_0|_{\upset x_0}] \to f(x_0)$, meaning $f[F_0|_{\downset x_0}]\cap f[F_0|_{\upset x_0}] \to f(x_0)$ by finite depth.

We conclude by showing that $f[F_0|_{\downset x_0}]\cap f[F_0|_{\upset x_0}] \preceq f[F]$. To that end, take $A\in f[F_0|_{\downset x_0}]\cap f[F_0|_{\upset x_0}]$. We need to show that there is a subset of $A$ in $f[F]$.

Indeed, let $A_1\in F_0|_{\downset x_0}$ be such that $f[A_1] = A$ and $A_2\in F_0|_{\upset x_0}$ be such that $f[A_2] = A$.
Then there exist $i,j\in I$ such that $[l_i,x_0]\subseteq A_1$ and $[x_0, u_j]\subseteq A_2$. Set $k = \max\{i,j\}$.

Now $f\Big[[l_k,u_k]\Big] = f\Big[[l_k,x_0]\Big] \cup f\Big[[x_0,u_k]\Big]\subseteq f[A_1]\cup f[A_2] = A$ and also $f\Big[[l_k,u_k]\Big]\in F$ by order convergence.
\end{proof}


\subsection{Convergences on powersets}
\begin{definition}
Let $X$ be a set. The \udef{set-theoretic convergence} on $\powerset(X)$ is the order convergence determined by $\subseteq$.
\end{definition}

\begin{lemma} \label{indicatorFunctionHomeomorphism}
Let $X$ be a set. Then the indicator function determines a homeomorphism
\[ \charFunc{-}: \powerset{X} \to (X\to \{0,1\})_p: A\mapsto \charFunc{A}, \]
if $\powerset(X)$ carries the set-theoretic convergence.
\end{lemma}



\chapter{Lattice constructions}

\section{Initial and final convergences}
\begin{proposition} \label{continuityUnderConvergenceLatticeOperations}
Let $\sSet{X,\xi}$ and $\sSet{Y,\zeta}$ be preconvergence spaces. Let $\Xi$ be a set of preconvergences on $X$ and $Z$ a set of convergences on $Y$. Then
\begin{enumerate}
\item $\cont\left(\xi, \bigwedge Z\right) = \bigcap_{\sigma\in Z} \cont(\xi, \sigma)$;
\item $\cont\left(\bigvee \Xi, \zeta\right) = \bigcap_{\sigma\in \Xi} \cont(\sigma, \zeta)$;
\item $\cont\left(\bigwedge \Xi, \zeta\right) \supseteq \bigcup_{\sigma\in Z} \cont(\xi, \sigma)$;
\item $\cont\left(\xi, \bigvee Z\right) \supseteq \bigcup_{\sigma\in \Xi} \cont(\sigma, \zeta)$;
\end{enumerate}
\end{proposition}
\begin{proof}
(1) We calculate, using \ref{latticeConvergences}:
\begin{align*}
f \in \cont\left(\xi, \bigwedge Z\right) &\iff f\left[\lim_\xi F \right] \subseteq \lim_{\bigwedge Z} \upset f[F] = \bigcap_{\sigma\in Z}\lim_\sigma \upset f[F] \\
&\iff \forall \sigma\in Z: \; f\left[\lim_\xi F \right] \subseteq \lim_\sigma \upset f[F] \\
&\iff \forall \sigma\in Z: \; f \in \cont(\xi,\sigma) \\
&\iff f\in \bigcap_{\sigma\in Z}\cont(\xi,\sigma).
\end{align*}

(2) Similarly, we have
\begin{align*}
f \in \cont\left(\bigvee \Xi, \zeta\right) &\iff f\left[\lim_{\bigvee \Xi} F \right] = \bigcup_{\sigma\in\Xi}f\left[\lim_\sigma F \right] \subseteq \lim_{\zeta} \upset f[F] \\
&\iff \forall \sigma\in \Xi: \; f\left[\lim_\sigma F \right] \subseteq \lim_\zeta \upset f[F] \\
&\iff \forall \sigma\in \Xi: \; f \in \cont(\sigma, \zeta) \\
&\iff f\in \bigcap_{\sigma\in \Xi}\cont(\sigma, \zeta).
\end{align*}

(3) Now we have
\begin{align*}
f \in \cont\left(\bigwedge \Xi, \zeta\right) &\iff f\left[\lim_{\bigwedge \Xi} F \right] = \bigcap_{\sigma\in\Xi}f\left[\lim_\sigma F \right] \subseteq \lim_{\zeta} \upset f[F] \\
&\impliedby \exists \sigma\in \Xi: \; f\left[\lim_\sigma F \right] \subseteq \lim_\zeta \upset f[F] \\
&\iff \exists \sigma\in \Xi: \; f \in \cont(\sigma, \zeta) \\
&\iff f\in \bigcup_{\sigma\in \Xi}\cont(\sigma, \zeta).
\end{align*}

(4) Finally we have
\begin{align*}
f \in \cont\left(\xi, \bigvee Z\right) &\iff f\left[\lim_\xi F \right] \subseteq \lim_{\bigvee Z} \upset f[F] = \bigcup_{\sigma\in Z}\lim_\sigma \upset f[F] \\
&\impliedby \exists \sigma\in Z: \; f\left[\lim_\xi F \right] \subseteq \lim_\sigma \upset f[F] \\
&\iff \exists \sigma\in Z: \; f \in \cont(\xi,\sigma) \\
&\iff f\in \bigcup_{\sigma\in Z}\cont(\xi,\sigma).
\end{align*}
\end{proof}
\begin{corollary}
Let $f: X\to Y$ be a function.
\begin{enumerate}
\item Given a preconvergence $\xi$ on $X$, the set of preconvergences on $Y$ that make $f$ continuous is a complete $\wedge$-subsemilattice.
\item Given a preconvergence $\zeta$ on $Y$, the set of preconvergences on $X$ that make $f$ continuous is a complete $\vee$-subsemilattice.
\end{enumerate}
\end{corollary}

\begin{definition}
Let $f: X\to Y$ be a function, $\xi$ a preconvergence on $X$ and $\zeta$ a preconvergence on $Y$.
\begin{itemize}
\item The dual closure of the chaotic preconvergence on $X$ into the lattice of preconvergences that make $f:X\to \sSet{Y,\zeta}$ continuous is called the \udef{initial preconvergence} on $X$ w.r.t.\ $f:X\to \sSet{Y,\zeta}$.
\item The closure of the empty preconvergence on $Y$ into the lattice of preconvergences that make $f:\sSet{X,\xi}\to Y$ continuous is called the \udef{final preconvergence} on $Y$ w.r.t.\ $f:\sSet{X,\xi}\to Y$.
\end{itemize}
\end{definition}

With multiple functions we project into the intersection of all these lattices.

\begin{definition}
Let $Y$ be a set.
\begin{itemize}
\item Given a set of preconvergence spaces $\{\sSet{Z_i,\zeta_i}\}_{i\in I}$ and a set of functions $\{f_i: Y\to Z_i\}_{i\in I}$, we define the \udef{initial preconvergence} $\initialPreConv\{f_i: Y\to Z_i\}_{i\in I}$ on $Y$ w.r.t. $\{f_i: Y\to Z_i\}$ as the coarsest preconvergence on $Y$ that makes all functions in $\{f_i: Y\to Z_i\}$ continuous:
\[ \initialPreConv\{f_i: Y\to Z_i\}_{i\in I} \defeq \bigvee \setbuilder{\sigma}{\forall i\in I: f_i\in  \cont_\text{(pre)}(\sigma, \zeta_i)}. \]
\item Given a set of preconvergence spaces $\{\sSet{X_i,\xi_i}\}_{i\in I}$ and a set of functions $\{g_i: X_i \to Y\}_{i\in I}$, we define the \udef{final preconvergence} $\finalPreConv\{g_i: X_i \to Y\}_{i\in I}$ on $Y$ w.r.t. $\{g_i: X_i\to Y\}$ as the finest preconvergence on $Y$ that makes all functions in $\{g_i: X_i \to Y\}$ continuous:
\[ \finalPreConv\{g_i: X_i \to Y\}_{i\in I} \defeq \bigwedge \setbuilder{\sigma}{\forall i\in I: g_i\in  \cont_\text{(pre)}(\xi_i, \sigma)}. \]
\end{itemize}
\end{definition}

\begin{proposition} \label{initialFinalConvergenceModification}
Let $Y$ be a set.
\begin{enumerate}
\item Let $\{f_i: Y\to \sSet{Z_i, \zeta_i}\}_{i\in I}$ be set of functions to convergence spaces and $\mu$ the initial preconvergence on $Y$ w.r.t. this set. Then $\mu$ is also the initial convergence w.r.t. $\{f_i: Y\to \sSet{Z_i, \zeta_i}\}$.
\item Let $\{g_i: \sSet{X_i, \xi_i} \to Y\}_{i\in I}$ be set of functions from convergence spaces and $\nu$ the final preconvergence on $Y$ w.r.t. this set. TODO
\end{enumerate}
\end{proposition}
\begin{proof}
TODO convergence modification
\end{proof}

\begin{lemma} \label{initialFinalConvergencesSubsetFunctions}
Let $Y$ be a set.
\begin{enumerate}
\item Let $A\subseteq B\subseteq \{f_i: Y\to \sSet{Z_i, \zeta_i}\}_{i\in I}$ be sets of functions to preconvergence spaces. Then
\begin{enumerate}
\item $\initialPreConv(A) \leq \initialPreConv(B)$;
\item $\initialPreConv(A) = \initialPreConv(B)$ \textup{if and only if} all functions in $A$ are $\initialPreConv(B)$-continuous.
\end{enumerate}
\item TODO same for final convergences.
\end{enumerate}
\end{lemma}

\begin{lemma} \label{initialFinalPreconvergenceUnion}
Let $Y$ be a set.
\begin{enumerate}
\item Let $\{f_i: Y\to \sSet{Z_i, \zeta_i}\}_{i\in I}$ be a set of functions to preconvergence spaces. Let $A_j \subseteq \{f_i: Y\to \sSet{Z_i, \zeta_i}\}_{i\in I}$ be a subset for all $j\in J$. Then
\[ \initialPreConv\Big(\bigcup_{j\in J}A_j\Big) = \bigwedge_{j\in J}\initialPreConv(A_j). \]
\item TODO for final preconvergences.
\end{enumerate}
\end{lemma}

\begin{definition}
Let $Y$ be a set, $L$ the lattice of preconvergences on $Y$, $C$ a complete $\wedge$-subsemilattice of $L$ and $C'$ a complete $\vee$-subsemilattice of $L$.
\begin{itemize}
\item Given a set of preconvergence spaces $\{\sSet{X_i,\xi_i}\}_{i\in I}$ and a set of functions $\{g_i: X_i \to Y\}_{i\in I}$, we define the \udef{final $C$-convergence} on $Y$ w.r.t. $\{g_i: X_i\to Y\}$ as the dual closure of the chaotic convergence on $Y$ into
\[ C \cap \setbuilder{\sigma}{\forall i\in I: g_i\in  \cont_\text{(pre)}(\xi_i, \sigma)}. \]
\item Given a set of preconvergence spaces $\{\sSet{Z_i,\zeta_i}\}_{i\in I}$ and a set of functions $\{f_i: Y\to Z_i\}_{i\in I}$, we define the \udef{initial $C'$-preconvergence} on $Y$ w.r.t. $\{f_i: Y\to Z_i\}$ as the closure of the empty convergence into
\[ C'\cap \setbuilder{\sigma}{\forall i\in I: f_i\in  \cont_\text{(pre)}(\sigma, \zeta_i)}. \]
\end{itemize}
\end{definition}

\begin{lemma}
These closures and dual closures are well-defined.
\end{lemma}

\begin{definition}
Let $Y$ be a set. We define
\begin{itemize}
\item the \udef{final convergence}; (TODO $\initialConv$ and $\finalConv$);
\item the \udef{final Kent space};
\item the \udef{final finite depth space};
\item the \udef{final pseudotopological space};
\item the \udef{final pretopological space};
\item the \udef{final topological space}.
\end{itemize}
\end{definition}

\begin{proposition}
Let $Y$ be a set, $L$ the lattice of preconvergences on $Y$, $C$ a complete $\wedge$-subsemilattice of $L$ and $C'$ a complete $\vee$-subsemilattice of $L$. Then
\begin{enumerate}
\item for any final preconvergence $\nu$, the final $C$-preconvergence is given by $\Closure_C(\nu)$;
\item for any initial preconvergence $\mu$, the initial $C'$-preconvergence is given by $\Closure_{C'}(\mu)$.
\end{enumerate}
\end{proposition}
\begin{proof}
TODO
\end{proof}

Then $\nu \vee \iota_Y$ is the final convergence w.r.t. $\{g_i: \sSet{X_i, \xi_i} \to Y\}$.


\begin{proposition} \label{initialFinalConvergence}
Let $Y$ be a set, $F\in \powerfilters(Y)$ and $y\in Y$.
\begin{enumerate}
\item Let $\{f_i: Y\to \sSet{Z_i, \zeta_i}\}_{i\in I}$ be set of functions to \emph{pre}convergence spaces and $\mu$ the initial \emph{pre}convergence on $Y$ w.r.t. this set. Then
\[ F \overset{\mu}{\longrightarrow} y \quad\iff\quad \forall i\in I: \; f_i^{\imf\imf}[F] \overset{\zeta_i}{\longrightarrow} f_i(y). \]
\item Let $\{g_i: \sSet{X_i, \xi_i} \to Y\}_{i\in I}$ be set of functions from \emph{pre}convergence spaces and $\nu$ the final \emph{pre}convergence on $Y$ w.r.t. this set. Then
\[ F \overset{\nu}{\longrightarrow} y \quad\iff\quad \exists i\in I: \exists x\in g_i^\preimf(\{y\}):  \exists G \overset{\xi_i}{\longrightarrow} x: \; g_i^{\imf\imf}[G] \subseteq F. \]
\end{enumerate}
\end{proposition}
Note that point (1) still holds true for convergences, by \ref{initialFinalConvergenceModification}. In the convergence case, point (2) needs to be modified to 
\[ F \overset{\nu}{\longrightarrow} y \quad\iff\quad F = \pfilter{y} \;\;\lor\;\; \exists i\in I: \exists x\in g_i^\preimf(\{y\}):  \exists G \overset{\xi_i}{\longrightarrow} x: \; g_i^{\imf\imf}[G] \subseteq F. \]
\begin{proof}
(1) The direction $\Rightarrow$ is clear: $\mu$ makes all $f_i$ continuous by \ref{continuityUnderConvergenceLatticeOperations}.

For $\Leftarrow$, assume $F$ such that $\forall i\in I: \; f_i^{\imf\imf}[F] \overset{\zeta_i}{\longrightarrow} f_i(y)$, but $F \overset{\mu}{\not\to} y$. Then define the preconvergence $\mu'$ with the same limits as $\mu$, but with the addition of $G\overset{\mu'}{\longrightarrow} y$ for all $G\supseteq F$. Now $\mu'$ makes all $f_i$ continuous (because $G\supseteq F$ implies $f_i^{\imf\imf}[G] \supseteq f_i^{\imf\imf}[F]$), so $\mu'\leq \mu$. Thus $F \overset{\mu}{\longrightarrow} y$, which is a contradiction.

(2) For the direction $\Leftarrow$, note that $\nu$ makes all $g_i$ continuous by \ref{continuityUnderConvergenceLatticeOperations}.

Assume the objects on the right-hand side exist such that $g_i^{\imf\imf}[G] \subseteq F$. Then $g_i^{\imf\imf}[G]$ converges to $y$ by continuity of $g_i$ and thus $F$ converges to $y$ by monotonicity of the convergence.

For $\Rightarrow$, assume, towards a contradiction, that $F \overset{\nu}{\longrightarrow} y$ and $\forall i\in I: \forall x\in g_i^{\preimf}\{y\}: \forall G\overset{\xi_i}{\longrightarrow} x: \; g_i^{\imf\imf}[G] \nsubseteq F$. Define the $\nu'$ from $\nu$ by removing all limits of the form $F' \to y$, where $F' \subseteq F$.

Now $\nu'$ is a preconvergence in which $F$ does not converge. In order to obtain a contradiction, it is enough to show that each $g_i$ is still continuous. To that end, take arbitrary $i\in I$ and $G \overset{\xi_i}{\longrightarrow} x$. Then we need to show that $g_i^{\imf\imf}[G] \in \lim^{-1}_{\nu'}(y)$. Clearly $g_i^{\imf\imf}[G] \in \lim^{-1}_{\nu}(y)$ and it was also not removed in the construction of $\nu'$ as that would have meant that $g_i^{\imf\imf}[G] \subseteq F$, which was excluded by assumption.

Thus $\nu'$ is a strictly stronger convergence than $\nu$ that makes all $g_i$ continuous. This is a contradiction.
\end{proof}
\begin{corollary} \label{finalConvergenceConvergence}
Let $\{g_i: \sSet{X_i, \xi_i} \to Y\}_{i\in I}$ be set of functions from convergence spaces such that $Y = \bigcup_{i\in I}\im(g_i)$. Then the final \emph{pre}convergence on $Y$ w.r.t. this set of functions is also the final convergence w.r.t. this set.
\end{corollary}
\begin{corollary}[Characteristic property of initial and final convergence] \label{characteristicPropertyInitialFinalConvergence}
Let $Y$ be a set, $\sSet{X,\xi}$ and $\sSet{Z, \zeta}$ preconvergence spaces.
\begin{enumerate}
\item Let $\{f_i: Y\to \sSet{Z_i, \zeta_i}\}_{i\in I}$ be set of functions to preconvergence spaces and $\mu$ the initial preconvergence on $Y$ w.r.t. this set. A function $g: \sSet{X, \xi}\to Y$ is continuous \textup{if and only if} $f_i \circ g$ is continuous for all $i\in I$.
\[ \begin{tikzcd}
Y \ar[r, "f_i"] & Z_i \\ X \ar[u, "g"] \ar[ur, swap, "f_i\circ g"]
\end{tikzcd} \]
\item Let $\{g_i: \sSet{X_i, \xi_i} \to Y\}_{i\in I}$ be set of functions from preconvergence spaces and $\nu$ the final preconvergence on $Y$ w.r.t. this set. A function $f: Y\to \sSet{Z,\zeta}$ is continuous \textup{if and only if} $f\circ g_i$ is continuous for all $i\in I$.
\[ \begin{tikzcd}
X_i \ar[r, "g_i"] \ar[dr, swap, "f\circ g_i"] & Y \ar[d, "f"] \\ & Z
\end{tikzcd} \]
\end{enumerate}
\end{corollary}
\begin{proof}
(1) Take arbitrary $F\overset{\xi}{\longrightarrow} x\in X$. Then the continuity of $g$ is equivalent to the convergence $g^{\imf\imf}[F] \overset{\mu}{\longrightarrow} g(x)$. By the proposition this is equivalent to
\[ \forall i \in I: \; f_i^{\imf\imf}\big[g^{\imf\imf}[F]\big] = (f_i\circ g)^{\imf\imf}[F] \overset{\zeta_i}{\longrightarrow} f_i(g(x)) = (f_i\circ g)(x),  \]
which is equivalent to the continuity of $f_i\circ g$ for all $i\in I$.

(2) If $f$ is continuous, then $f\circ g_i$ is continuous by continuity of composition. Now assume $f$ is not continuous, i.e.\ there exists some $G\overset{\nu}{\longrightarrow} y\in Y$ such that $f^{\imf\imf}[G] \not\to f(y)$. By the proposition, there exists some $F\overset{\xi_i}{\longrightarrow} x\in X_i$ such that $g_i^{\imf\imf}[F] \subseteq G$. Now
\[ f^{\imf\imf}\big[g_i^{\imf\imf}[F]\big] = (f\circ g_i)^{\imf\imf}[F] \subseteq f^{\imf\imf}[G] \not\to f(y) = (f\circ g_i)(x). \]
This means that $f\circ g_i$ is not continuous.
\end{proof}

\begin{lemma} \label{initialBijectionHomeomorphism}
Let $Y$ be a set, $\sSet{Z,\zeta}$ a convergence space and $f:Y\to Z$ a bijection. Then $f$ is a homeomorphism.
\end{lemma}
\begin{proof}
We just need to show that $f^{-1}: Z\to Y$ is continuous. Let $F\in \powerfilters(Z)$ converge to $z$. So $F = \upset f^{\imf\imf}\big((f^{-1}){\imf\imf}(F)\big) \overset{\zeta}{\longrightarrow} z = f\big(f^{-1}(z)\big)$ and thus $(f^{-1}){\imf\imf}(F) \to f^{-1}(z)$ by \ref{initialFinalConvergence}.
\end{proof}


TODO: final convergence does not preserve finite depth! (what about Kent space??)

TODO: initial/final not universal property, but product is.

\subsection{Vicinities in initial and final convergence}
\begin{proposition} \label{pretopologicalInitialConvergence}
Let $Y$ be a set and $y\in Y$.
\begin{enumerate}
\item Let $\{f_i: Y\to \sSet{Z_i, \zeta_i}\}_{i\in I}$ be set of functions to \emph{pseudotopological} preconvergence spaces and $\mu$ the initial preconvergence on $Y$ w.r.t. this set. Then $\mu$ is pseudotopological.
\item If $\{\sSet{Z_i, \zeta_i}\}_{i\in I}$ is a set of \emph{pretopological} preconvergence spaces, then $\mu$ is pretopological and
\[ \vicinity_\mu(y) = \mathfrak{F}\bigcup_{i\in I} \upset f_i^{\preimf\imf}\Big[\vicinity_{\zeta_i}\!\big(f_i(x)\big)\Big] = \mathfrak{F}\setbuilder{f_i^\preimf(U)}{i\in I, U\in \vicinity_{\zeta_i}(f_i(y))}. \]
\item If $\{\sSet{Z_i, \zeta_i}\}_{i\in I}$ is a set of \emph{topological} convergence spaces, then $\mu$ is topological and $\topology_\mu$ is generated by $\setbuilder{f_i^\preimf(U)}{i\in I, U\in \topology_{\zeta_i}}$.
\end{enumerate}
\end{proposition}
Note that if $I$ is a singleton, then
\[ \mathfrak{F}\setbuilder{f_i^\preimf(U)}{i\in I, U\in \vicinity_{\zeta_i}(f_i(y))} = \upset\setbuilder{f_i^\preimf(U)}{i\in I, U\in \vicinity_{\zeta_i}(f_i(y))}. \]
\begin{proof}
(1) We use \ref{pseudotopologicalConditions}. Let $F\in\powerfilters(Y)$ be a filter such that $U\overset{\mu}{\longrightarrow} y$ for all $F\subseteq U\in\powerultrafilters(Y)$ 

Now take arbitrary $i\in I$. And let $V\in\powerultrafilters(Z_i)$ be such that $\upset f_i^{\imf\imf}(F) \subseteq V$. Then by \ref{mappingUltrafiltersLemma} we can find $U'\in\powerultrafilters(Y)$ such that $F\subseteq U'$ and $\upset f_i^{\imf\imf}(U') = V$. Now $U\overset{\mu}{\longrightarrow} y$, so $V = \upset f_i^{\imf\imf}(U') \overset{\zeta_i}{\longrightarrow} f_i(y)$. 

Because $V$ was chosen arbitrarily, all ultrafilters larger than $\upset f_i^{\imf\imf}(F)$ converge to $f_i(y)$. As $\zeta_i$ is pseudotopological, this means that $\upset f_i^{\imf\imf}(F) \overset{\zeta_i}{\longrightarrow} f_i(y)$.

As $i\in I$ was chosen arbitrarily, $F\overset{\mu}{\longrightarrow} y$ by \ref{initialFinalConvergence}.

(2) We have, for all $F\in \powerfilters(Y)$, by \ref{initialFinalConvergence} and \ref{upsetPreimageImageGaloisConnection},
\begin{align*}
F \overset{\mu}{\longrightarrow} y &\iff \forall i\in I: f_i^{\imf\imf}[F] \overset{\zeta_i}{\longrightarrow} f_i(y) \\
&\iff \forall i\in I: \vicinity_{\zeta_i}\!\big(f_i(y)\big) \subseteq \upset f_i^{\imf\imf}[F] \\
&\iff \forall i\in I: \upset f_i^{\preimf\imf}\Big[\vicinity_{\zeta_i}\!\big(f_i(y)\big)\Big] \subseteq F \\
&\iff \bigcup_{i\in I} \upset f_i^{\preimf\imf}\Big[\vicinity_{\zeta_i}\!\big(f_i(y)\big)\Big] \subseteq F.
\end{align*}
Thus the initial convergence is pretopological and
\begin{align*}
\vicinity_\mu(y) &= \mathfrak{F}\bigcup_{i\in I} f_i^{\preimf\imf}\Big[\vicinity_{\zeta_i}\!\big(f_i(y)\big)\Big] \\
&= \mathfrak{F}\bigcup_{i\in I}\setbuilder{f_i^\preimf(U)}{U\in \vicinity_{\zeta_i}(f(y))} \\
&= \mathfrak{F}\setbuilder{f_i^\preimf(U)}{i\in I, U\in \vicinity_{\zeta_i}(f(y))}.
\end{align*}

(3) By (2), $\mu$ is pretopological. Each $U\in \vicinity_{\zeta_i}(f(x)) = \neighbourhood_{\zeta_i}(f(x))$ contains an open set, so $f_i^\preimf(U)$ also contains an open set by \ref{preimageOpenClosed}. Thus $\vicinity_\mu(x)$ is based in open sets. This means that the initial convergence is topological.

Finally $\setbuilder{f_i^\preimf(U)}{i\in I, U\in \topology_{\zeta_i}}$ is a set of open sets. TODO: a set of open sets that generates the convergence also generates the topology.
\end{proof}


\begin{proposition} \label{pretopologicalFinalConvergence}
Let $Y$ be a set, $F\in \powerfilters(Y)$ and $y\in Y$.
\begin{enumerate}
\item Let $\{g_i: \sSet{X_i, \xi_i} \to Y\}_{i\in I}$ be set of functions from \emph{pre}convergence spaces and $\nu$ the final \emph{pre}convergence on $Y$ w.r.t. this set. Then 
\[ \vicinity_\nu(y) = \bigcap_{\substack{i\in I \\ x\in g^\preimf_i\{y\}}}\upset g_i^{\imf\imf}\big[\vicinity_{\xi_i}(x)\big]. \]
\item If $\{\sSet{X_i, \xi_i}\}_{i\in I}$ is a set of \emph{convergence} spaces, then
\[ \vicinity_\nu(y) = \pfilter{y} \cap \bigcap_{\substack{i\in I \\ x\in g^\preimf_i\{y\}}}\upset g_i^{\imf\imf}\big[\vicinity_{\xi_i}(x)\big]. \]
\item If $\{\sSet{X_i, \xi_i}\}_{i\in I}$ is a set of \emph{topological} convergence spaces, then
\[ \topology_\nu = \setbuilder{U\in\powerset(Y)}{\forall i\in I: g_i^{\preimf}(U)\in\topology_{\xi_i}}. \]
In general, $\nu$ is \emph{not} topological in this case.
\end{enumerate}
\end{proposition}
TODO not correct?? Check modifications!
\begin{proof}
(1) We have, by \ref{initialFinalConvergence}, for $y\in Y$
\[ {\lim}^{-1}(y) = \setbuilder{G\in\powerfilters(Y)}{\exists i\in I: \exists x\in g_i^\preimf\{y\}: \exists F\overset{\xi_i}{\longrightarrow}x:\; g_i^{\imf\imf}(F)\subseteq G}. \]
Then
\begin{align*}
\vicinity_\nu(y) &= \bigcap {\lim}^{-1}(y) \\
&= \bigcap \setbuilder{G\in\powerfilters(Y)}{\exists i\in I: \exists x\in g_i^\preimf\{y\}: \exists F\overset{\xi_i}{\longrightarrow}x:\; g_i^{\imf\imf}(F)\subseteq G} \\
&= \bigcap_{\substack{i\in I \\ x\in g^\preimf_i\{y\}}} \left\{ \bigcap\setbuilder{G\in\powerfilters(Y)}{\exists F\overset{\xi_i}{\longrightarrow}x:\; g_i^{\imf\imf}(F)\subseteq G} \right\} \\
&= \bigcap_{\substack{i\in I \\ x\in g^\preimf_i\{y\}}} \left\{ \bigcap\setbuilder{\upset g_i^{\imf\imf}(F)}{F\overset{\xi_i}{\longrightarrow}x} \right\} \\
&= \bigcap_{\substack{i\in I \\ x\in g^\preimf_i\{y\}}} \left\{ \upset g_i^{\imf\imf}\left[\bigcap\setbuilder{F}{F\overset{\xi_i}{\longrightarrow}x}\right] \right\} \\
&= \bigcap_{\substack{i\in I \\ x\in g^\preimf_i\{y\}}} \left\{ \upset g_i^{\imf\imf}\left[\vicinity_{\xi_i}(x)\right] \right\}.
\end{align*}

(2) By \ref{vicinitiesConvergenceModification}.

(3) TODO

(Convergence not topological) See \ref{convergenceSpaceQuotientOfTopologicalSpace}, for example.
\end{proof}

\begin{proposition} \label{adherenceInitialFinalConvergence}
Let $Y$ be a set and $A\subseteq Y$.
\begin{enumerate}
\item Let $\{f_i: Y\to \sSet{Z_i, \zeta_i}\}_{i\in I}$ be set of functions to convergence spaces and $\mu$ the initial convergence on $Y$ w.r.t. this set. Then
\[ \adh_\mu(A) \subseteq \bigcap_{i\in I}(f^{\preimf}_i\circ \adh_{\zeta_i}\circ f^{\imf}_i)(A). \]
\item Let $\{g_i: \sSet{X_i, \xi_i} \to Y\}_{i\in I}$ be set of functions from \emph{pre}convergence spaces and $\nu$ the final \emph{pre}convergence on $Y$ w.r.t. this set. Then
\[ \adh_\nu(A) = \bigcup_{i\in I}(g_i^{\imf}\circ\adh_{\xi_i}\circ g^{\preimf})(A). \]
\end{enumerate}
\end{proposition}
TODO: can we improve (1)?
\begin{proof}
(1) By \ref{adherenceInherenceContinuity} we have $f_i^{\imf}\big(\adh_\mu(A)\big) \subseteq \adh_{\zeta_i}\big(f^{\imf}(A)\big)$, so $(f^{\preimf}_i\circ \adh_{\zeta_i}\circ f^{\imf}_i)(A)$ for all $i\in I$ by \ref{functionImagePreimageGaloisConnection}.

(2) We calculate, using \ref{pretopologicalFinalConvergence}, \ref{grillIntersectionUnion} and \ref{meshConnectionSetsOfSets}, \ref{upsetPreimageImageGaloisConnection} (with the fact that $\vicinity_{\xi_i}(x)$ is upwards closed, by \ref{upwardClosureGrill})
\begin{align*}
y\in \adh_\nu(A) &\iff A\in \vicinity_\nu(y)^\mesh \\
&\iff A\in \Big(\bigcap_{\substack{i\in I \\ x\in g^\preimf_i\{y\}}}\upset g_i^{\imf\imf}\big(\vicinity_{\xi_i}(x)\big)\Big)^\mesh \\
&\iff A\in \bigcup_{\substack{i\in I \\ x\in g^\preimf_i\{y\}}}\big(\upset g_i^{\imf\imf}\big(\vicinity_{\xi_i}(x)\big)\big)^\mesh \\
&\iff A\in \bigcup_{\substack{i\in I \\ x\in g^\preimf_i\{y\}}}\upset g_i^{\imf\imf}\big(\vicinity_{\xi_i}(x)^\mesh\big) \\
&\iff \exists i\in I, \exists x\in g^\preimf_i\{y\}: \;  \{A\}\subseteq \upset g_i^{\imf\imf}\big(\vicinity_{\xi_i}(x)^\mesh\big) \\
&\iff \exists i\in I, \exists x\in g^\preimf_i\{y\}: \;  \upset g_i^{\preimf\imf}\{A\}\subseteq \vicinity_{\xi_i}(x)^\mesh \\
&\iff \exists i\in I, \exists x\in g^\preimf_i\{y\}: \;  g_i^{\preimf\imf}\{A\}\in \vicinity_{\xi_i}(x)^\mesh \\
&\iff \exists i\in I, \exists x\in g^\preimf_i\{y\}: \;  x\in \adh_{\xi_i}\big(g_i^{\preimf\imf}\{A\}\big) \\
&\iff \exists i\in I: \;  y\in g^{\imf}\big(\adh_{\xi_i}\big(g_i^{\preimf\imf}\{A\}\big)\big) \\
&\iff y\in \bigcup_{i\in I}(g_i^{\imf}\circ\adh_{\xi_i}\circ g^{\preimf})(A).
\end{align*}
\end{proof}




TODO:
\begin{itemize}
\item Vicinity filter for non-pretopological initial convergence.
\item Topology for non-topological initial convergence.
\end{itemize}


\section{Constructions}
\subsection{Subspace convergence}
\begin{definition}
Let $\sSet{X,\xi}$ be a convergence space and $A\subseteq X$ a subset. The \udef{subspace convergence} $\xi|_A$ on $A$ is the initial convergence w.r.t. $\{\iota: A \hookrightarrow X: a\mapsto a\}$. The convergence space $\sSet{A,\xi|_A}$ is called a \udef{convergence subspace} of $X$.
\end{definition}

\begin{lemma} \label{setTraceFilterLemma}
Let $\sSet{X,\xi}$ be a convergence space, $A\subseteq X$ and $F\in\powerfilters(X)$ a filter. Then
\begin{enumerate}
\item $F|_A = \upset \setbuilder{B\cap A}{B\in F} = (\upset_X\circ \iota^{\preimf\imf})(F)$;
\item $F = (\upset_X\circ \iota^{\preimf\imf})(F)$ \textup{if and only if} $A\in F$.
\end{enumerate}
\end{lemma}
\begin{proof}
(1) We have $\iota^{\preimf}(B) = B\cap A$ for all $B\subseteq X$.

(2) By 1. and \ref{traceFilterLemma}.
\end{proof}

\begin{lemma} \label{subspaceConvergence}
Let $\sSet{X,\xi}$ be a convergence space, $\sSet{A,\xi|_A}$ a subspace, $F\in \powerfilters(A)$, $G\in \powerfilters(X)$ and $a\in A$. Then
\begin{enumerate}
\item $F \overset{\xi|_A}{\longrightarrow} a$ if and only if $\upset_X F \overset{\xi}{\longrightarrow} a$;
\item if $G\overset{\xi}{\longrightarrow} a$, then $\iota^{\preimf\imf}[G]\overset{\xi|_A}{\longrightarrow} a$;
\item if $A\in G$ and $\iota^{\preimf\imf}[G]\overset{\xi|_A}{\longrightarrow} a$, then $G\overset{\xi}{\longrightarrow} a$.
\end{enumerate}
\end{lemma}
\begin{proof}
(1) By \ref{initialFinalConvergence}, since $\upset_X F = \upset \iota^{\imf\imf}(F)$

(2) Assume $G\overset{\xi}{\longrightarrow} a \in A$. Then $G \subseteq \upset\iota^{\imf\imf}\big(\iota^{\preimf\imf}(G)\big) \to a$ by \ref{upsetPreimageImageGaloisConnection} and so $\iota^{\preimf\imf}(G) \to a$ by \ref{initialFinalConvergence}.

(3) We have $\upset\iota^{\imf\imf}\big[\iota^{\preimf\imf}[G]\big] \overset{\xi}{\longrightarrow} a$ by continuity of the inclusion $\iota$. If $A\in G$, then, by \ref{setTraceFilterLemma}, $\upset\iota^{\imf\imf}\big(\iota^{\preimf\imf}(G)\big) = (\upset_X\circ \iota^{\preimf\imf})(G) = G$, so $G \overset{\xi}{\longrightarrow} a$.
\end{proof}

\begin{proposition} \label{subspaceAdherence}
Let $\sSet{X,\xi}$ be a convergence space and $A\subseteq X$ a subset. Let $B\subseteq A$ and $x\in A$. Then
\begin{enumerate}
\item $\adh_{\xi|_A}(B) = \adh_\xi(B) \cap A$;
\item $\inh_{\xi|_A}(B) = A\setminus \adh(A\setminus B) = \inh_\xi(B\cup A^c) \cap A$;
\item if $A$ is closed, then $\adh_{\xi|_A}(B) = \adh_\xi(B)$;
\item if $A$ is open, then $\inh_{\xi|_A}(B) = \inh_\xi(B)$;
\item if $B$ is $\xi$-closed, then it is $\xi|_A$-closed;
\item if $B$ is $\xi$-open, then it is $\xi|_A$-open.
\end{enumerate}
\end{proposition}
\begin{proof}
(1) We have $\iota^\imf(B) = B$ and $\iota^\imf\big(\adh_{\xi|_A}(B)\big)$, so $\adh_{\xi|_A}(B) \subseteq \adh_\xi(B)$ by \ref{adherenceInherenceContinuity}. Then we immediately have $\adh_{\xi|_A}(B) \subseteq \adh_\xi(B) \cap A$.

Now take $x\in \adh_\xi(B) \cap A$. Then, by \ref{principalAdherenceInherence}, there exists a proper filter $F\overset{\xi}{\longrightarrow} x$ such that $B\in F$. Then also $A\in F$ and so $\upset \iota^{\preimf\imf}(F)\overset{\xi|_A}{\longrightarrow} x$ by \ref{subspaceConvergence} (where we have used that $x\in A$). Because $B\in \upset \iota^{\preimf\imf}(F)$, we have $x\in \adh_{\xi|_A}(B)$.

(2) We calculate
\begin{align*}
\inh_{\xi|_A}(B) &= A\setminus \adh_{\xi|_A}(A\setminus B) \\
&= A\setminus \big(\adh(A\setminus B)\cap A\big) \\
&= A\setminus \adh(A\setminus B) \\
&= A\cap \adh(A\cap B^c)^c \\
&= A\cap \inh(A^c\cup B).
\end{align*}

(3) If $A$ is closed, then $\adh_\xi(B)\subseteq \adh_\xi(A) = A$ by \ref{principalInherenceAdherenceProperties} and so $\adh_\xi(B)\cap A = \adh_\xi(B)$.

(4) If $A$ is open, then $A = \inh_\xi(A)$, so
\begin{align*}
\inh_{\xi|_A}(B) &= \inh_\xi(B\cup A^c) \cap A \\
&= \inh_\xi(B\cup A^c) \cap \inh_\xi(A) \\
&= \inh_\xi\big((B\cup A^c)\cap A\big) \\
&= \inh_\xi(B),
\end{align*}
where we have used \ref{principalInherenceAdherenceProperties}.

(5) Assume $B$ is $\xi$-closed. From (1) we have $\adh_{\xi|_A}(B) = \adh_\xi(B) \cap A = B\cap A = B$.

(6) Assume $B$ is $\xi$-open. We have $B \subseteq \inh_\xi(B) \subseteq \inh_\xi(B\cup A^c) \cap A = \inh_{\xi|_A}(B)$, so $B$ is $\xi|_A$-open.
\end{proof}

\begin{lemma}
Let $\sSet{X,\xi}$ be a convergence space, $A\subseteq X$ a subset and $a\in A$. Then
\begin{enumerate}
\item if $\xi$ is pretopological, then
\[ \vicinity_{\xi|_A}(a) = \setbuilder{U\cap A\subseteq A}{U\in \vicinity_{\xi}(a)}; \]
\item if $\xi$ is topological, then
\[ \topology_{\xi|_A} = \setbuilder{O\cap A}{O\in \topology_{\xi}}. \]
\end{enumerate}
Note that $\setbuilder{U\cap A\subseteq A}{U\in \vicinity_{\xi}(a)}$ is a filter and $\setbuilder{O\cap A}{O\in \topology_{\xi}}$ is a topolgy. We do not need to take any closures.
\end{lemma}
\begin{proof}
This follows straight from \ref{pretopologicalInitialConvergence} since for all $Y\subseteq X$, we have $\incl_A^\preimf(Y) = Y\cap A$.
\end{proof}

\begin{proposition} \label{subspaceVicinityFilter}
Let $\sSet{X,\xi}$ be a convergence space and $A\subseteq X$ a subset. If $A$ is open, then
\[ \upset\iota^{\imf\imf}\big(\vicinity_{\xi|_A}(a)\big) = \vicinity_\xi(a) \]
for all $a\in A$.
\end{proposition}
\begin{proof}
We have $\vicinity_\xi(a) \subseteq \upset\iota^{\imf\imf}\big(\vicinity_{\xi|_A}(a)\big)$ by continuity of $\iota$.

For the other inclusion, we have $A\in \vicinity_\xi(a)$ by \ref{openClosedCriteria}, so $A$ is an element of each filter that converges to $a$ in $\xi$. Now take arbitrary $F\overset{\xi}{\longrightarrow} a$. By \ref{subspaceConvergence} we have $\vicinity_{\xi|_A}(a) \subseteq \upset \iota^{\preimf\imf}(F)$, which implies
\[ \upset \iota^{\imf\imf}\big(\vicinity_{\xi|_A}(a)\big) \subseteq \upset (\iota^{\imf\imf}\circ \iota^{\preimf\imf})(F) = F, \]
by \ref{setTraceFilterLemma}. Taking the intersection over all such $F$ gives the result.
\end{proof}

\begin{proposition} \label{continuityRestrictionExpansion}
Let $\sSet{X,\xi}$, $\sSet{Y,\sigma}$ and $\sSet{Z,\zeta}$ be convergence spaces.
\begin{enumerate}
\item \textup{(Restricting the domain)} If $f:X\to Y$ is continuous and $A$ is a subspace of $X$, then the restricted function $f|_{A}:A\to Y$ is continuous.
\item \textup{(Restricting the codomain)} Let $f:X\to Y$ be continuous. If $Z$ is a subspace of $Y$ containing the image set $f[X]$, then $f:X\to Z$ is continuous.
\item \textup{(Expanding the codomain)} Let $f:X\to Y$ be continuous. If $Y$ is a subspace of $Z$, then $f:X\to Z$ is continuous.
\end{enumerate}
\end{proposition}
\begin{proof}
(1) Composition of continuous maps: $f|_{A} = f\circ\iota$.

(2) Composition of continuous maps: $f:X\to Z = \iota \circ (f: X\to Y)$.

(3) Characteristic property \ref{characteristicPropertyInitialFinalConvergence}: if $f:X\to Y = \iota \circ (f:X\to Z)$ is continuous, then $f:X\to Z$ is too.
\end{proof}

\begin{example}
In order to be able to expand the domain, we need extra assumptions. Let $X = \{-1,0\}$, $Y = \R$ and $Z = \R$. Set $f: X\to Y: x\mapsto \begin{cases}
0 & (x=-1) \\ 1 & (x = 0)
\end{cases}$ and $g: Z\to Y: x\mapsto \begin{cases}
0 & (x < 0) \\
1 & (x \geq 0)
\end{cases}$. Then $f$ is continuous, $X \subseteq Z$ carries the subspace convergence and $f = g|_X$. But $g$ is not continuous, not even at the points that lie in $X$ since it is discontinuous at $0$.
\end{example}

\begin{proposition} \label{continuityExpandedDomain}
Let $\sSet{X,\xi}$ be a convergence space, $\sSet{Y,\zeta}$ a pretopological convergence spaces, $A\subseteq X$ an open subset and $a\in A$. Then $f: X\to Y$ is continuous at $a$ \textup{if and only if} $f|_A: A\to Y$ is continuous at $a$.
\end{proposition}
TODO: is pretopologicity of $\zeta$ really necessary? Counterexample?
\begin{proof}
The implication $\Rightarrow$ is given by \ref{continuityRestrictionExpansion}.

For the converse, it is enough, by \ref{pretopologicalContinuityVicinities}, to prove that $\vicinity_Y\big(f(a)\big) \subseteq \upset f^{\imf\imf}\big(\vicinity_{X}(a)\big)$.
By \ref{continuityVicinityFilter} and \ref{subspaceVicinityFilter}, we have
\[ \vicinity_Y\big(f(a)\big) \subseteq \upset (f|_A)^{\imf\imf}\big(\vicinity_{\xi|_A}(a)\big) = \upset (f\circ \iota)^{\imf\imf}\big(\vicinity_{\xi|_A}(a)\big) = \upset f^{\imf\imf}\big(\vicinity_{X}(a)\big). \]
\end{proof}

\begin{lemma} \label{denseSetDeterminesContinuousFunction}
Let $\sSet{X,\xi}$ and $\sSet{Y, \zeta}$ be convergence spaces, $f,g\in \cont(X,Y)$ continuous functions and $A\subseteq X$ a dense subset. If $f|_A = g|_A$, then $f = g$.
\end{lemma}
\begin{proof}
Take arbitrary $x \in X$. By \ref{convergentFiltersInDenseSet}, there exists $F\to x$ such that $A\in F$. We then calculate
\begin{align*}
f(x) &= \lim f^{\imf\imf}(F) \\
&= \lim f^{\imf\imf}(F|_A) \\
&= \lim \upset \setbuilder{f^\imf(A\cap B)}{B\in F} \\
&= \lim \upset \setbuilder{f|_A^\imf(A\cap B)}{B\in F} \\
&= \lim \upset \setbuilder{g|_A^\imf(A\cap B)}{B\in F} \\
&= \lim \upset \setbuilder{g^\imf(A\cap B)}{B\in F} \\
&= \lim g^{\imf\imf}(F|_A) \\
&= \lim g^{\imf\imf}(F) \\
&= g(x).
\end{align*}
\end{proof}

\subsection{Product convergence}
\begin{definition}
Let $\sSet{X_i, \xi_i}$ be a convergence space for all $i\in I$. The \udef{product convergence space} $\prod_{i\in I}X_i$ is the initial convergence on $\bigtimes_{i\in I}X_i$ w.r.t. the set of projections $p_i: \bigtimes_{i\in I}X_i \to X_i$.
\end{definition}

\begin{lemma} \label{continuityFunctionTuple}
Let $\sSet{X,\xi}, \sSet{Y,\sigma}$ and $\sSet{Z,\zeta}$ be convergence spaces, $f: X\to Y$ and $g: X\to Z$. Then $(f,g): X\to Y\times Z$ is continuous \textup{if and only if} $f$ and $g$ are continuous.
\end{lemma}
\begin{proof}
Follows immediately from \ref{characteristicPropertyInitialFinalConvergence},
because $f = \pi_1\circ (f,g)$ and $g = \pi_2\circ (f,g)$.
\end{proof}
\begin{corollary} \label{continuousEmbeddingProduct}
Let $\sSet{X,\xi}$ and $\sSet{Y,\sigma}$ be convergence spaces. Then for all $y\in Y$, the function $X\to X\times Y: x\mapsto (x,y)$ is continuous.
\end{corollary}
\begin{proof}
The functions $\id_X$ and $\underline{y}$ are continuous.
\end{proof}

\begin{lemma} \label{continuityParallelComposition}
Let $\sSet{X,\xi}, \sSet{Y,\zeta}$, $\sSet{U,\sigma}$ and $\sSet{V,\tau}$ be convergence spaces, $f: X\to Y$ and $g: X\to Z$. Then $(f|g): X\times Y \to U\times V$ is continuous \textup{if and only if} $f$ and $g$ are continuous.
\end{lemma}
\begin{proof}
For the direction $\Rightarrow$, we may assume $X$ and $Y$ non-empty (since otherwise $X\times Y$ is empty and any function on it is continuous). Then there exists $x_0\in X$ and $y_0\in Y$. We have $f = \proj_1\circ (f|g) \circ \big(x\mapsto (x,y_0)\big)$ and $g = \proj_2\circ (f|g) \circ \big(y\mapsto (x_0, y)\big)$, which are continuous by assumption and \ref{continuousEmbeddingProduct}.

For the converse, it is enough to note the definition $(f|g) = (f\circ \proj_1, g\circ \proj_2)$ and use \ref{continuityFunctionTuple}.
\end{proof}

\begin{lemma}
Let $\sSet{X,\xi}$ and $\sSet{Y,\sigma}$ be convergence spaces. Then $\proj_X: X\times Y\to X$ and $\proj_Y: X\times Y\to Y$ are open maps.
\end{lemma}
\begin{proof}
TODO (only for topological spaces?)
\end{proof}

\begin{lemma} \label{projectionClosedFunction}
Let $\sSet{X,\xi}$ be a convergence space and $\sSet{Y,\sigma}$ a compact convergence space. Then $\proj_X: X\times Y\to X$ is a closed function.
\end{lemma}
\begin{proof}
Let $A\subseteq X\times Y$ be a closed set. Suppose $F\in \powerfilters(X)$ contains $\proj_X^\imf(A)$ and $F\to x$. Consider $\upset(\proj_X|_A)^{\preimf\imf}(F)$, which is proper by \ref{filterInImageIsImageFilter}. It is contained in an ultrafilter $U\in\powerultrafilters(X\times Y)$ by \ref{ultrafilterLemma}. Now $\upset\proj_Y^{\imf\imf}(U)$ is an ultrafilter by \ref{imageFilterProperties} and thus convergent to some limit $y\in Y$ by compactness of $Y$. By \ref{filterInImageIsImageFilter}, we also have
\[ F = \upset(\proj_X|_A)^{\imf\imf}\circ(\proj_X|_A)^{\preimf\imf}(F) = \upset(\proj_X)^{\imf\imf}\circ(\proj_X|_A)^{\preimf\imf}(F) \subseteq \upset\proj_X^{\imf\imf}(U), \]
so $\upset\proj_X^{\imf\imf}(U)$ converges to $x$. This means $U\to (x,y)$ and so $(x,y)\in A$. Thus $x\in \proj_X^\imf(A)$ and so $\proj_X^\imf(A)$ is closed.
\end{proof}

\begin{lemma} \label{initialConvergenceWrtOneFunction}
Let $Y$ be a set, $F\in \powerfilters(Y)$ and $y\in Y$. Let $\{f_i: Y\to \sSet{Z_i, \zeta_i}\}_{i\in I}$ be set of functions to convergence spaces and $\mu$ the initial convergence on $Y$ w.r.t. this set.

Then $\mu$ is also the initial convergence w.r.t. the function $f: Y\to \prod_{i\in I}Z_i$ where $f$ is defined by $\proj_i(f(y)) = f_i(y)$ for all $y\in Y$.
\end{lemma}
The definition of $f$ uses the universal property of the product. (TODO ref).
\begin{proof}
Let $F\in \powerfilters(Y)$ be a filter and $x\in X$. Then
\begin{align*}
F \overset{\mu}{\longrightarrow} x &\iff \forall i\in I:  \upset f_i^{\imf\imf}(F) \overset{\zeta_i}{\longrightarrow} f_i(x) \\
&\iff \forall i\in I: \upset (\proj_i\circ f)^{\imf\imf}(F) \overset{\zeta_i}{\longrightarrow} (\proj_i\circ f)(x) \\
&\iff  \upset f^{\imf\imf}(F) \overset{\bigotimes_{i\in I}\zeta_i}{\longrightarrow} f(x),
\end{align*}
where we have used \ref{initialFinalConvergence} twice. A third application proves the lemma.
\end{proof}
\begin{corollary} \label{productAndSubspaceConvergencesCommute}
Let $\sSet{X_i,\xi_i}_{i\in I}$ be a set of convergence spaces and $A_i \subseteq X_i$ a subset for all $i\in I$. Then the subspace convergence on $\bigtimes_{i\in I}A_i$ as a subspace of $\bigtimes_{i\in I}X_i$ equals the product convergence where each $A_i$ individually carries the subspace convergence as a subspace of $X_i$.
\end{corollary}
\begin{proof}
If we set $f_i = \iota_{A_i}\circ \proj_{i}$, then $f = \iota_{\bigtimes_{i\in I}A_i}$.
\end{proof}

\subsubsection{Characterising the product convergence}
\begin{proposition} \label{convergenceProductFilter}
Let $\sSet{X_i, \xi_i}$ be a (pre)convergence space for all $i\in I$, $F \in \powerfilters(\prod_{i\in I}X_i)$ and $x\in \prod_{i\in I}X_i$. Then $F\to x$ \textup{if and only if} $\forall i\in I: \exists F_i\in \powerfilters(X_i): 
\; F_i \overset{\xi_i}{\longrightarrow} p_i(x)$ and
\[ F \supseteq \bigotimes_{i\in I}F_i \defeq \upset\setbuilder{\bigtimes_{i\in I}A_i}{\forall i\in I: A_i \in F_i \;\land\; A_i = X_i, \,\text{except for finitely many $A_i$}}. \]
\end{proposition}
The filter $\bigotimes_{i\in I}F_i$ is called the \udef{product filter} of $\{F_i\}_{i\in I}$.
\begin{proof}
First assume $F\to x$, then we have $F_i = \upset p_i^{\imf\imf}(F)$ to be convergent for each $i\in I$, by \ref{initialFinalConvergence}. By \ref{upsetPreimageImageGaloisConnection}, this implies $\upset p_i^{\preimf\imf}(F_i) \subseteq F$.

Now $p_i^{\preimf\imf}(F_i)$ consist of the sets $p_i^\preimf(A)$ for all $A\in F_i$ and $p_i^\preimf(A)$ is of the form $\bigtimes_{k\in I}B_k$, where $B_i = A$ and $B_k = X_k$ if $k \neq i$.

Thus $F$ contains all products of the form $\bigtimes_{k\in I}B_k \in F$, where only one $B_k$ is not equal to $X_k$ and this $B_k$ is equal to some element of $F_k$.

As $F$ is a filter, it also contains the filter generated by all such products. This filter is exactly the product filter.

Conversely, $p_j\Big(\bigotimes_{i\in I}F_i\Big) = F_j \to p_j(x)$, so $\bigotimes_{i\in I}F_i$ converges to $x$ in the product convergence by \ref{initialFinalConvergence} and $F$ converges by upwards closure.
\end{proof}
TODO adjoint of $\coprod_{i\in I}p_i$.
\begin{corollary} \label{productVicinity}
Let $\sSet{X_i, \xi_i}$ be a (pre)convergence space for all $i\in I$ and $x\in \prod_{i\in I}X_i$. Then
\[\vicinity_{\prod \xi_i}(x) = \upset \setbuilder{\bigtimes_{i\in I}A_i}{\forall i\in I: A_i \in \vicinity_{\xi_i}(p_i(x)) \;\land\; A_i = X_i, \,\text{except for finitely many $A_i$}}. \]
In particular $\vicinity_{\xi\otimes \zeta}((x,y)) = \upset \vicinity_\xi(x)\otimes \vicinity_\zeta(y)$.
\end{corollary}

\begin{corollary} \label{productAdherence}
Let $\sSet{X_i, \xi_i}$ be a (pre)convergence space and $A_i\subseteq X_i$ for all $i\in I$. Then
\[ \adh_{\prod_{i\in I}\xi_i}\left(\bigtimes_{i\in I}A_i\right) = \bigtimes_{i\in I}\adh_{\xi_i}(A_i). \]
In particular $\adh_{\xi\otimes \zeta}(A\times B) = \adh_\xi(A) \times \adh_\zeta(B)$.
\end{corollary}
\begin{proof}
We have
\begin{align*}
\seq{x_i}_{i\in I} \in \adh_{\prod_{i\in I}\xi_i}\left(\bigtimes_{i\in I}A_i\right) &\iff \bigtimes_{i\in I}A_i \in \vicinity_{\prod \xi_i}(\seq{x_i}_{i\in I})^{\mesh} \\
&\iff \forall i\in I: \forall B \in \vicinity_{\xi_i}(x_i): A_i\mesh B \\
&\iff \forall i\in I: A_i \in \vicinity_{\xi_i}(x_i)^{\mesh} \\
&\iff \forall i\in I: x_i \in \adh_{\xi_i}(A_i) \\
&\iff \seq{x_i}_{i\in I} \in \bigtimes_{i\in I}\adh_{\xi_i}(A_i).
\end{align*}
\end{proof}
\begin{corollary} \label{productInherence}
Let $\sSet{X_i, \xi_i}$ be a (pre)convergence space and $A_i\subseteq X_i$ for all $i\in I$. Then
\[ \inh_{\prod_{i\in I}\xi_i}\left(\bigtimes_{i\in I}A_i\right) = \bigtimes_{i\in I}\inh_{\xi_i}(A_i). \]
In particular $\inh_{\xi\otimes \zeta}(A\times B) = \inh_\xi(A) \times \inh_\zeta(B)$.
\end{corollary}
\begin{proof}
We have
\begin{align*}
\seq{x_i}_{i\in I} \in \inh_{\prod_{i\in I}\xi_i}\left(\bigtimes_{i\in I}A_i\right) &\iff \bigtimes_{i\in I}A_i \in \vicinity_{\prod \xi_i}(\seq{x_i}_{i\in I}) \\
&\iff \forall i\in I: \exists B_i \in \vicinity_{\xi_i}(x_i): B_i\subseteq A_i \\
&\iff \forall i\in I: A_i \in \vicinity_{\xi_i}(x_i) \\
&\iff \forall i\in I: x_i \in \inh_{\xi_i}(A_i) \\
&\iff \seq{x_i}_{i\in I} \in \bigtimes_{i\in I}\inh_{\xi_i}(A_i).
\end{align*}
\end{proof}
\begin{corollary} \label{productOpenClosed}
Let $\sSet{X_i, \xi_i}$ be a (pre)convergence space and $A_i\subseteq X_i$ for all $i\in I$. Then
\begin{enumerate}
\item $\bigtimes_{i\in I}A_i$ is open \textup{if and only if} $A_i$ is open for all $i\in I$;
\item $\bigtimes_{i\in I}A_i$ is closed \textup{if and only if} $A_i$ is closed for all $i\in I$.
\end{enumerate}
\end{corollary}
\begin{proof}
(1) The openness of $\bigtimes_{i\in I}A_i$ is equivalent to 
\[\inh_{\prod_{i\in I}\xi_i}\left(\bigtimes_{i\in I}A_i\right) = \bigtimes_{i\in I}A_i. \]
The openness of all $A_i$ is equivalent to
\[\bigtimes_{i\in I}\inh_{\xi_i}(A_i) = \bigtimes_{i\in I}A_i. \]
These two equations are equivalent.

(2) Similar, using adherence.
\end{proof}

\begin{lemma} \label{sliceOpenSetOpen}
Let $\sSet{X,\xi}$ and $\sSet{Y,\zeta}$ be convergence spaces, $V\subseteq X\times Y$ and $(x,y)\in X\times Y$. Then
\[ (x,y) \in \inh_{\xi\otimes \zeta}(V) \quad \implies \quad x\in \inh_\xi(Vy). \]
If $V$ is open, then $Vy$ is also open.
\end{lemma}
\begin{proof}
Assume $(x,y) \in \inh_{\xi\otimes \zeta}(V)$. Then $V\in \vicinity_{\xi\otimes \zeta}(x,y) = \vicinity_\xi(x)\otimes \vicinity_\zeta(y)$, by \ref{productVicinity}. So there exists $A\in \vicinity_\xi(x)$ and $B \in \vicinity_\zeta(y)$ such that $A\times B\subseteq V$. Since $\pfilter{y} \to y$, we have $B\in\vicinity_\zeta(y)\subseteq \pfilter{y}$ and so $y\in B$. Then $A = (A\times B)y \subseteq Vy$ and thus $Vy\in \vicinity_\xi(x)$, which means that $x\in \inh_\xi(Vy)$.

Now suppose $V$ open and take $y\in Y$. For all $x\in Vy$, we have $(x,y) \in V = \inh_{\xi\otimes \zeta}(V)$, so $x\in \inh_\xi(Vy)$. Thus $Vy \subseteq \inh_\xi(Vy)$, which means that $Vy$ is open.
\end{proof}

\subsubsection{Properties of product filters}
TODO generalise to upwards closed sets!!

\begin{lemma}
The product filter of proper filters is proper.
\end{lemma}
\begin{proof}
Let $\{X_i\}_{i\in I}$ be a set of sets and $\setbuilder{F_i \in \powerfilters(X_i)}{i\in I}$ a set of proper filters. Assume, towards a contradiction, that $\bigotimes_{i\in I}F_i$ is not proper. Take $\bigtimes_{i\in I}A_i\in \bigotimes_{i\in I}F_i$, then $\left(\bigtimes_{i\in I}A_i\right)^c \in \bigotimes_{i\in I}F_i$ and thus we can find $\bigtimes_{i\in I}B_i\in \bigotimes_{i\in I}F_i$ such that $\bigtimes_{i\in I}B_i \subseteq \left(\bigtimes_{i\in I}A_i\right)^c$. We have
\[ \emptyset = \left(\bigtimes_{i\in I}A_i\right) \cap \left(\bigtimes_{i\in I}B_i\right) = \bigtimes_{i\in I}(A_i\cap B_i). \]
Now the right-hand side is only empty if there exists $i\in I$ such that $A_i\cap B_i = \emptyset$. In this case $F_i$ is not a proper filter.
\end{proof}

\begin{lemma} \label{projectionsOfProductFilter}
Let $X,Y$ be sets, $F\in \powerfilters(X)$ and $G\in \powerfilters(Y)$. Then
\begin{enumerate}
\item $\proj_1^{\imf\imf}[F\otimes G] = F$;
\item $\proj_2^{\imf\imf}[F\otimes G] = G$.
\end{enumerate}
\end{lemma}
\begin{proof}
The elements of $F$ form a basis of $p_1^{\imf\imf}[F\otimes G]$. Similarly the elements of $G$ form a basis of $p_2^{\imf\imf}[F\otimes G]$.
\end{proof}
\begin{lemma} \label{parallelFunctionProductFilter}
Let $X,Y,V,W$ be sets, $F\in \powerfilters(X)$, $G\in \powerfilters(Y)$, $f: X\to U$ and $g: Y\to W$. Then $(f|g)^{\imf\imf}(F\otimes G) = f^{\imf\imf}(F)\otimes g^{\imf\imf}(G)$.
\end{lemma}

\begin{lemma} \label{filterFactorisationInequality}
Let $X,Y$ be sets and $H \in \powerfilters(X\times Y)$. Then $p_1^{\imf\imf}(H)\otimes p_2^{\imf\imf}(H) \subseteq H$.
\end{lemma}
\begin{proof}
Take some $A\in p_1^{\imf\imf}(H)\otimes p_2^{\imf\imf}(H)$. Then there exist $B,C\in H$ such that $A = p_1^\imf(B)\times p_2^\imf(C)$, which means that $B\cap C\subseteq A$. Now $B\cap C \in H$, so $A\in H$.
\end{proof}

\begin{lemma} \label{productPrincipalUltrafilter}
Let $X,Y$ be sets, $x\in X$ and $y\in Y$. Then $\pfilter{(x,y)} = \pfilter{x} \otimes \pfilter{y}$.
\end{lemma}
In particular $\pfilter{f}(x,y) = f^{\imf\imf}(\pfilter{x}\otimes\pfilter{y})$ for all functions $f: X\times Y \to Z$.
\begin{proof}
It is enough to check that $(x,y)\in\pfilter{x}\otimes\pfilter{y}$ and that $\pfilter{x}\otimes\pfilter{y}$ is not trivial. Both these statements are immediately clear.
\end{proof}

\begin{lemma} \label{functionsOfProductFilters}
Let $X,Y, A,B$ be sets, $F\in \powerfilters(X)$ and $G\in \powerfilters(Y)$. Let $f: X\to A$ and $g: Y\to B$ be functions. Then $\upset (f,g)^{\imf\imf}(F\otimes G) = f^{\imf\imf}(F)\otimes g^{\imf\imf}(G)$.
\end{lemma}
TODO: convenient generalisation?
\begin{proof}
Expand in bases.
\end{proof}

\begin{lemma} \label{intersectionProductFilters}
Let $X,Y$ be sets, $F,G\in \powerfilters(X)$ and $H\in\powerfilters(Y)$. Then
\[ (F\cap G)\otimes H = (F\otimes H)\cap (G\otimes H). \]
\end{lemma}
\begin{proof}
$\boxed{\subseteq}$ Take $A\in (F\cap G)$ and $B\in H$. Then $A\times B\in F\otimes H$ and $A\times B\in G\otimes H$.

$\boxed{\supseteq}$ Take $A \in (F\otimes H)\cap (G\otimes H)$. Then we can find $B_1\in F$, $C_1,C_2\in H$ and $B_2\in G$ such that $B_1\times C_1 \subseteq A$ and $B_2\times C_2 \subseteq A$. Now $(B_1\cup B_2)\times (C_1\cap C_2) \subseteq A$. Also $B_1\cup B_2 \in F\cap G$ and $C_1\cap C_2 \in H$, so $(B_1\cup B_2)\times (C_1\cap C_2) \in (F\cap G)\otimes H$. By upwards closure $A\in (F\cap G)\otimes H$.
\end{proof}

\begin{lemma} \label{filterPairingLemma}
Let $X,Y$ be sets, $F\in\powerfilters(X)$, $\Delta: X\to X^2: x\mapsto (x,x)$ and $f: X^2\to Y$ a function. Then
\[ \upset f^{\imf\imf}(F\otimes F) = \upset (f\circ \Delta)^{\imf\imf}(F). \]
\end{lemma}
\begin{proof}
Let $A,B,C\in F$. Any basis set of the form $f^{\imf}(A\times B)$ contains a set of the form $f^\imf(C\times C)$, namely $f^{\imf}\big((A\cap B)\times (A\cap B)\big)$.
\end{proof}

\begin{proposition}
Let $X,Y$ be sets, $F\in\powerfilters(X)$ and $G\in \powerfilters(Y)$. Then
\begin{enumerate}
\item $\ker(F\otimes G) = \ker(F)\times\ker(G)$;
\item $(F\otimes G)_0 = (F_0\otimes G) \cap (F\otimes G_0)$.
\end{enumerate}
\end{proposition}
\begin{proof}
(1) We have
\[ \ker(F\otimes G) = \bigcap_{\substack{A\in F \\ B\in G}}A\times B = \Big(\bigcap_{A\in F}A\Big)\times \Big(\bigcap_{B\in G} B\Big) = \ker(F)\times \ker(G), \]
by TODO ref.

(2) We calculate, using \ref{freePrincipalDecomposition} and point (1),
\begin{align*}
(F\otimes G)_0 &= (F\otimes G) \vee \upset\{\ker(F\otimes G)^c\} \\
&= (F\otimes G) \vee \upset\{\big(\ker(F)\times \ker(G)\big)^c\} \\
&= (F\otimes G) \vee \upset\{\big(\ker(F)^c\times Y\big) \cup \big(X\times \ker(G)^c\big)\} \\
&= (F\otimes G) \vee \big(\upset\{\ker(F)^c\times Y\} \wedge \upset\{X\times \ker(G)^c\}\big) \\
&= \big((F\otimes G)\vee \upset\{\ker(F)^c\times Y\}\big) \wedge \big((F\otimes G)\vee \upset\{X\times \ker(G)^c\}\big) \\
&= \big((F\vee \upset\{\ker(F)^c\})\otimes (G\vee \upset\{Y\})\big) \wedge \big((F\vee \upset\{X\})\otimes (G\vee \ker(G)^c)\big) \\
&= (F_0\otimes G) \wedge (F\otimes G_0).
\end{align*}
TODO ref for manipulations (through filter bases).
\end{proof}


\begin{lemma} \label{convergenceFiniteProductFilter}
Let $\sSet{X,\xi}$ and $\sSet{Y,\zeta}$ be convergence spaces, $F\in \powerfilters(X\times Y)$, $G\in \powerfilters(X)$ and $H\in \powerfilters(Y)$. Then
\begin{enumerate}
\item $F \overset{\xi \otimes \zeta}{\longrightarrow} (x,y)$ \textup{if and only if} $p_1^{\imf\imf}(F) \overset{\xi}{\longrightarrow} x$ and $p_2^{\imf\imf}(F) \overset{\zeta}{\longrightarrow} y$;
\item $G\otimes H \overset{\xi \otimes \zeta}{\longrightarrow} (x,y)$ \textup{if and only if} $G \overset{\xi}{\longrightarrow} x$ and $H \overset{\zeta}{\longrightarrow} y$.
\end{enumerate}
\end{lemma}
\begin{proof}
Point (1) is a restatement of \ref{initialFinalConvergence}. Point (2) follows from point (1) because $p_1^{\imf\imf}(F\otimes G) = F$ and $p_2^{\imf\imf}(F\otimes G) = G$ by \ref{projectionsOfProductFilter}.
\end{proof}

\subsubsection{Functions with closed graphs}
\begin{definition}
Let $\sSet{X,\xi}, \sSet{Y,\zeta}$ be convergence space and $f: X\not\to Y$ a partial function. We say $f$ has \udef{closed graph} if $\graph(f)$ is closed in $X\times Y$.
\end{definition}
This is not the same as a closed map!

\label{secFunctionsClosedGraph}

\begin{proposition} \label{closedGraphEquivalence}
Let $\sSet{X,\xi}, \sSet{Y,\zeta}$ be convergence space and $f: X\not\to Y$ a partial function. Then
the following are equivalent:
\begin{enumerate}
\item $f$ has closed graph;
\item if $F\in \powerfilters\big(\dom(f)\big)$ converges to $x\in X$ and $\upset f^{\imf\imf}(F)$ converges to $y\in Y$, then $x\in\dom(f)$ and $f(x) = y$.
\end{enumerate}
\end{proposition}
\begin{proof}
$(1) \Rightarrow (2)$ Suppose $F\in \powerfilters\big(\dom(f)\big)$ converges to $x\in X$ and $\upset f^{\imf\imf}(F)$ converges to $y\in Y$. Then $F\otimes f^{\imf\imf}(F) \to (x, y)$, so $(x, y) \in \adh\big(\graph(f)\big)$. Since $\graph(f)$ is closed, we have $(x, y) \in \graph(f)$, so $x\in dom(f)$ and $y= f(x)$.

$(2) \Rightarrow (1)$ Take $(x, y) \in \adh\big(\graph(f)\big)$. Then there exists $H\in \powerfilters(X\times Y)$ such that $H\to (x,y)$ and $\graph(f)\in H$. Now $\upset\proj_1^{\imf\imf}(H) \to x$ and $\upset\proj_2^{\imf\imf}(H) \to y$. By construction, $\upset\proj_2^{\imf\imf}(H) = \upset f^{\imf\imf}\big(\proj_1^{\imf\imf}(H)\big)$, so $x\in \dom(f)$ and $f(x) = y$. This means that $(x,y) = \big(x,f(x)\big) \in \graph(f)$.

Thus $\adh\big(\graph(f)\big) \subseteq \graph(f)$, which means that the graph of $f$ is closed.
\end{proof}

\begin{proposition} \label{closedGraphFunctionConstructions}
Let $\sSet{X,\xi}, \sSet{Y,\sigma}, \sSet{Z,\zeta}$ be convergences space, $f: X\to Y$ a continuous function and $g:Y\not\to Z$ a partial function with closed graph. Then
\begin{enumerate}
\item $g\circ f$ has closed graph;
\item if $g$ is injective, then $g^{-1}: \im(g)\subseteq Y\to \dom(g)$ has closed graph.
\end{enumerate}
\end{proposition}
\begin{proof}
(1) Suppose $F\in \powerfilters\big(\dom(g\circ f)\big)$ converges to $x\in X$ and $\upset (g\circ f)^{\imf\imf}(F)$ converges to $z\in Z$. Then $\upset f^{\imf\imf}(F)$ converges to $f(x)$ by continuity and then $f(x)\in\dom(f)$ and $z= (g\circ f)(x)$ by \ref{closedGraphEquivalence}, since $g$ has closed graph. Since this means that $x\in \dom(g\circ f)$, $g\circ f$ has closed graph by \ref{closedGraphEquivalence}.

(2) Suppose $F\in \powerfilters(Y)$, with $\im(f)\in F$, converges to $y\in Y$ and $\upset f^{\preimf\imf}(F)$ converges to $x\in X$. Then $\upset(f^{\imf\imf}\circ f^{\preimf\imf})(F) = F \to y$ by \ref{filterInImageIsImageFilter}. Then $y = f(x)$ by \ref{closedGraphEquivalence}, since $f$ has closed graph, so $y\in \dom(f^{-1})$ and $f^{-1}(y) = x$. We conclude that $f^{-1}$ has closed graph by \ref{closedGraphEquivalence}.
\end{proof}

TODO: example $f\circ g$ not closed.

\begin{proposition} \label{continuousFunctionClosedGraph}
Let $\sSet{X,\xi}, \sSet{Y,\zeta}$ be convergence space with $Y$ Hausdorff and $f: X\not\to Y$ a partial function. Then
\begin{enumerate}
\item if $f$ is continuous and has closed domain, then $f$ has closed graph;
\item if $X,Y$ are topological, $Y$ is compact and $f$ has closed graph, then $f$ is continuous.
\end{enumerate}
\end{proposition}
TODO: can we extend (2) to non-topological settings??
\begin{proof}
(1) We use \ref{closedGraphEquivalence}. Suppose $F\in \powerfilters\big(\dom(f)\big)$ converges to $x\in X$ and $\upset f^{\imf\imf}(F)$ converges to $y\in Y$. By closure of the domain, $x\in \dom(f)$. By continuity $\upset f^{\imf\imf}(F)$ converges to $f(x)$. By Hausdorff, we have $y = f(x)$.

(2) Take $B\subseteq Y$ closed. Then $(X\times B)\cap \im(f)$ is closed and so $f^\preimf(A) = \proj_1^\imf\big((X\times B)\cap \im(f)\big)$ is closed by \ref{projectionClosedFunction}. This implies that $f$ is continuous by (TODO ref).
\end{proof}


\subsection{Disjoint union or coproduct convergence}
\begin{definition}
Let $\sSet{X_i, \xi_i}$ be a convergence space for all $i\in I$. The \udef{disjoint union convergence space} or \udef{coproduct convergence space} $\coprod_{i\in I}X_i$ is the final convergence on the disjoint union $\bigsqcup_{i\in I}X_i$ w.r.t. the set of canonical injections
\[ \varphi_i: X_i \to \bigsqcup_{i\in I}X_i: x\mapsto (i, x). \]
\end{definition}
This convergence on the disjoint union is equal to the final preconvergence w.r.t. to the canonical injections by \ref{finalConvergenceConvergence}.

\begin{proposition}
Let $\sSet{X_i, \xi_i}$ be a convergence space for all $i\in I$ and $F\in\powerfilters\left(\bigsqcup_{i\in I}X_i\right)$. Then $F\overset{\coprod_{i\in I}X_i}{\longrightarrow} x$ \textup{if and only if} $F|_{X_i} \overset{\xi_i}{\longrightarrow} x'$ for some $i\in I, x'\in X_i$ such that $x= (i,x')$.
\end{proposition}
Note that the ``for some $i$'' can hold for at most one $i$, namely the $i$ for which we have $x = (i, x')$.
\begin{proof}
This is an application of \ref{initialFinalConvergence}.
\end{proof}

\begin{proposition} \label{pretopologicalCoproductConvergence}
Let $\sSet{X_i, \xi_i}$ be a topological convergence space for all $i\in I$. Then
\begin{enumerate}
\item if each $\xi_i$ is pretopological, then $\coprod_{i\in I} \xi_i$ is pretopological;
\item if each $\xi_i$ is topological, then $\coprod_{i\in I} \xi_i$ is topological.
\end{enumerate}
\end{proposition}
\begin{proof}
TODO
\end{proof}

\subsection{Quotient convergence}
\begin{definition}
Let $\sSet{X,\xi}$ be a convergence space, $Y$ a set and $q: \sSet{X,\xi}\to Y$ a surjective function. The final convergence on $Y$ w.r.t. $q$ is called the \udef{quotient convergence structure} on $Y$.
\end{definition}

In this case the final convergence is also the final preconvergence by \ref{finalConvergenceConvergence}.

\begin{example}
Let $\sSet{X,\xi}$ be a convergence space and $\sim$ an equivalence relation. Then the function
\[ [\cdot]_\sim: X\to X/{\sim}: x\mapsto [x]_\sim \]
is surjective. The quotient convergence on $X/{\sim}$ w.r.t. $[\cdot]_\sim$ is the canonical convergence on $X/{\sim}$ and the \udef{quotient space} of $X$ w.r.t. $\sim$ is $X/{\sim}$ with its canonical convergence.
\end{example}

\begin{proposition} \label{convergenceSpaceQuotientOfTopologicalSpace}
Each Kent convergence space is the quotient convergence space of a topological space.
\end{proposition}
\begin{proof}
Let $\sSet{X,\xi}$ be a convergence space. For each $F\overset{\xi}{\longrightarrow} x$, let $\xi_F$ be the pretopological convergence defined by
\[ \begin{cases}
\vicinity_{\xi_F}(x) = F \cap \pfilter{x}  \\
\vicinity_{\xi_F}(y) = \pfilter{y} & (y \neq x).
\end{cases} \]
This pretopological convergence is in fact topological: consider point (6) of \ref{pretopologicalSpaceTopological}.

First consider a point $y\neq x$ and take arbitrary $U\in \vicinity_{\xi_F}(y)$. Then we can take $V = \{y\} \in \vicinity_{\xi_F}(y)$ and we have $U\in \vicinity_{\xi_F}(y)$.

Now consider the point $x$ and take arbitrary $U\in \vicinity_{\xi_F}(x)$. Then we can take $V = U \in \vicinity_{\xi_F}(x)$. For all $y'\in U$ we clearly have $U\in \vicinity_{\xi_F}(y') = \pfilter{y}'$.

Now consider the convergence space $\coprod_{\text{$F$ converges in $\xi$}}\sSet{X, \xi_F}$, which is topological by \ref{pretopologicalCoproductConvergence}. Then the convergence $\xi$ is equal to the final convergence on $X$ w.r.t. to the projection $p_2: \coprod_{\text{$F$ converges in $\xi$}}\sSet{X, \xi_F} \to X$.
\end{proof}


TODO: move to compactness section.
\begin{lemma} \label{convergenceCoverQuotientSpace}
Let $\sSet{X,\xi}$ be a convergence space, $Y$ a set and $q: \sSet{X,\xi}\to Y$ a surjective function. If $\mathcal{C}$ is a convergence cover of $X$, then $q^{\imf\imf}(\mathcal{C})$ is a convergence cover of $Y$.
\end{lemma}
\begin{proof}
Let $F\overset{Y}{\longrightarrow} y$ be a convergent filter in $Y$. Then, by \ref{initialFinalConvergence}, there exists $G\overset{\xi}{\longrightarrow} x$ such that $q^{\imf\imf}(G)\subseteq F$. Now there exists $C\in \mathcal{C}\cap G$, so $q^{\imf}(C)\in F$.
\end{proof}
\begin{corollary} \label{quotientConvergenceCompactConvergenceCover}
Let $\sSet{X,\xi}$ be a locally compact convergence space, $Y$ a set and $q: \sSet{X,\xi}\to Y$ a surjective function. Then
\begin{enumerate}
\item $q^{\imf\imf}\big(\setbuilder{K\subseteq X}{\text{$K$ is compact}}\big)$ is a convergence cover;
\item $Y$ is locally compact.
\end{enumerate}
\end{corollary}
\begin{proof}
(1) Local compactness means that $\setbuilder{K\subseteq X}{\text{$K$ is compact}}$ is a convergence cover.

(2) By \ref{compactConstructions} the convergence cover in (1) consists of compact sets.
\end{proof}

\begin{lemma} \label{preimageCompactnessQuotientConvergence}
Let $\sSet{X,\xi}$ be a locally compact convergence space and $Y$ a set with the quotient convergence w.r.t.\ a surjective function $q: \sSet{X,\xi}\to Y$. 
Then for each compact subset $K\subseteq Y$, there exists a compact $L\in X$ such that $K \subseteq q^\imf(L)$. 
\end{lemma}
\begin{proof}
Let $\mathcal{K}$ be the set of compact subsets of $X$. As it is a convergence cover by local compactness, $q^{\imf\imf}(\mathcal{K})$ is a convergence cover of $Y$ by \ref{convergenceCoverQuotientSpace}.

Since $K$ is compact, by \ref{compactFiniteSubcover} there exists a finite subcover $\mathcal{K}'$ such that
\[ K\subseteq \bigcup q^{\imf\imf}(\mathcal{K}') = q^{\imf}\Big(\bigcup\mathcal{K}'\Big) \defeq q^\imf(L), \]
using \ref{imagePreimageGaloisConnection}. Now $L$ is compact by \ref{compactConstructions}.
\end{proof}

\subsection{Projective and injective limits}
\subsubsection{Projective limits}
\begin{proposition}
All projective systems $\sSet{I, \{\sSet{X_i, \xi_i}\}_{i\in I}, \{p_{j,i}\}_{i\preceq j}}$ in the category of convergence spaces have a projective limit.

This limit is given by the limit in the category $\cat{Set}$ equipped with the initial convergence w.r.t. all the leg morphisms.
\end{proposition}

\subsubsection{Inductive limits}
\begin{proposition} \label{convergenceInductiveLimit}
All inductive systems $\sSet{I, \{\sSet{X_i, \xi_i}\}_{i\in I}, \{e_{i,j}\}_{i\preceq j}}$ in the category of convergence spaces have an inductive colimit.

This colimit is given by the colimit in the category $\cat{Set}$ equipped with the final convergence w.r.t. all the leg morphisms.
\end{proposition}

\begin{proposition}
Let $\sSet{X,\xi}$ be a convergence space and $\mathcal{F}$ a $\pi$-system of closed subsets of $X$ such that $\bigcup \mathcal{F} = X$. The inductive colimit of $\mathcal{F}$, where each $A\in \mathcal{F}$ is equipped with the subspace convergence, is the specified sets convergence on $X$ w.r.t. $\mathcal{F}$.
\end{proposition}
In particular this new convergence is stronger than the original convergence on $V$.
\begin{proof}
The inductive colimit of sets is $\bigcup \mathcal{F} = X$, so, by \ref{convergenceInductiveLimit}, the convergence inductive colimit is given by $X$ equipped with the final convergence w.r.t. each inclusion $\iota_A: A\hookrightarrow X$ (where $A\in \mathcal{F}$).

Let $F\in\powerfilters(X)$ and $x\in X$. Then, by \ref{initialFinalConvergence}, $F$ converges to $x$ in $\varinjlim \mathcal{F}$ iff there exists $A\in \mathcal{F}$ and $G \in \powerfilters(A)$ such that $G\overset{A}{\longrightarrow} x$ and $\upset\iota_A^{\imf\imf}(G) \subseteq F$.

As $A\in G$, this implies that $A\in F$ and thus that $F$ converges in the specified sets convergence.

Conversely, if $F$ converges in the specified sets convergence, then there exists $A\in \mathcal{F}$ such that $A \in F$ and $F\overset{\xi}{\longrightarrow} x$. Because $A$ is closed, we must have $x\in A$. Then $\upset\iota_A^{\preimf\imf}(F) \to x$ by \ref{subspaceConvergence} and so $\upset\iota_A^{\imf\imf}\big(\iota_A^{\preimf\imf}(F)\big) = F$ by \ref{setTraceFilterLemma}, which means that $F$ converges to $x$ in $\varinjlim \mathcal{F}$.
\end{proof}

\begin{definition}
Let $\sSet{X,\xi}$ be a convergence space and $\mathcal{F}$ a $\pi$-system of closed subsets of $X$ such that $\bigcup \mathcal{F} = X$. We call the specified sets convergence on $X$ w.r.t. $\mathcal{F}$, the \udef{inductive strengthening of $X$}.
\end{definition}

\section{Sublattices of (pre)convergences}
\subsection{Functoriality}
\begin{proposition} \label{convergenceModificationFunctoriality}
All modiciations are functors: $f:X\to Y$ continuous implies $f: M(X)\to M(Y)$ continuous.
\end{proposition}

TODO universal properties: given $f: X\to M(Y)$, there exists unique extension $M(f)$ such that
\[ \begin{tikzcd}
M(X) \ar[r, "M(f)"] & M(Y) \\
X \ar[u, "\id_X"] \ar[ur, swap, "f"] & {}
\end{tikzcd} \qquad \text{commutes.} \]
+ Dual for dual closures.

\subsection{Depth properties}
\subsubsection{Convergence spaces}
\begin{lemma}
Let $X$ be a set. The set of convergences on $X$ is a subset of the lattice of preconvergences that is closed under meets.
\end{lemma}
\begin{proof}
\ref{completeSubsemilatticeClosure}
\end{proof}

\begin{definition}
Let $\sSet{X,\xi}$ be a preconvergence space. Then the \udef{convergence modification} $\convMod(\xi)$ is the closure of $\xi$ into the lattice of convergence spaces.
\end{definition}

\begin{lemma} \label{convModConstruction}
Let $\sSet{X,\xi}$ be a preconvergence space. Then $\convMod(\xi) = \xi \cup \setbuilder{(\pfilter{x}, x)}{x\in X}$.
\end{lemma}

\begin{proposition} \label{vicinitiesConvergenceModification}
Let $\sSet{X,\xi}$ be a preconvergence space, $x\in X$ and $A\subseteq X$. Then
\begin{enumerate}
\item $\vicinity_{\convMod(\xi)}(x) = \vicinity_{\xi}(x)\cap \pfilter{x}$;
\item $\adh_{\convMod(\xi)}(A) = \adh_\xi(A) \cup A$;
\item $\inh_{\convMod(\xi)}(A) = \inh_\xi(A) \cap A$.
\end{enumerate}
\end{proposition}
\begin{proof}
(1) Immediate by \ref{convModConstruction}.

(2) We have, using (1) and \ref{grillIntersectionUnion},
\begin{align*}
x\in \adh_{\convMod(\xi)}(A) &\iff A\in \vicinity_{\convMod(\xi)}(x)^\mesh \\
&\iff A\in \big(\vicinity_{\xi}(x) \cap \pfilter{x}\big)^\mesh \\
&\iff A\in \vicinity_{\xi}(x)^\mesh \cup \pfilter{x}^\mesh \\
&\iff \big(A\in \vicinity_{\xi}(x)^\mesh\big) \lor \big(A\in \pfilter{x}^\mesh\big) \\
&\iff \big(x\in \adh_\xi(A)\big) \lor \big(x\in A\big) \\
&\iff x \in \adh_\xi(A) \cup A.
\end{align*}

(3) We calculate, using \ref{principalInherenceComplementAdherence},
\[ \inh_{\convMod(\xi)}(A) = \adh_{\convMod(\xi)}(A^c)^c = \big(\adh_\xi(A^c)\cup A^c\big)^c = \adh_\xi(A^c)^c \cap A = \inh_\xi(A) \cap A. \]
\end{proof}

\subsubsection{Kent property}
\begin{lemma}
Let $X$ be a set. The set of Kent convergences on $X$ is a subset of the lattice of preconvergences that is closed under meets.
\end{lemma}
\begin{definition}
Let $\sSet{X,\xi}$ be a convergence space. Then the \udef{Kent modification} $\kentMod(\xi)$ is the closure of $\xi$ into the lattice of Kent spaces.
\end{definition}


\subsubsection{Finite depth spaces}
\begin{lemma}
Let $X$ be a set. The set of finite depth preconvergences on $X$ is a subset of the lattice of preconvergences that is closed under meets.
\end{lemma}
\begin{definition}
Let $\sSet{X,\xi}$ be a convergence space. Then the \udef{finite depth modification} $\finDepthMod(\xi)$ is the closure of $\xi$ into the lattice of convergences of finite depth.
\end{definition}

\begin{proposition}
Let $\sSet{X,\xi}$ be a convergence space. Then the finite depth modification $\finDepthMod(\xi)$ exists and is given by
\[ F \overset{\finDepthMod(\xi)}{\longrightarrow} x \quad\iff\quad \exists n\in\N, G_0,\ldots, G_n \in {\lim}^{-1}(x): \; F = \bigcap_{k=0}^n G_k. \]
\end{proposition}

\subsubsection{Pseudotopological spaces}
\begin{lemma}
Let $X$ be a set. The set of pseudotopological convergences on $X$ is a subset of the lattice of preconvergences that is closed under meets.
\end{lemma}
\begin{definition}
Let $\sSet{X,\xi}$ be a convergence space. Then the \udef{pseudotopological modification} $\pseudotopMod(\xi)$ is the closure of $\xi$ into the lattice of pseudotopological convergences.
\end{definition}

\begin{proposition} \label{pseudotopModConstruction}
Let $\sSet{X,\xi}$ be a convergence space. Then the pseudotopological modification $\pseudotopMod(\xi)$ exists and is given by
\[ F \overset{\pseudotopMod(\xi)}{\longrightarrow} x \quad\iff\quad \forall H\in \powerultrafilters(X): (F\subseteq H) \implies \big(H \overset{\xi}{\longrightarrow} x\big). \]
\end{proposition}

\begin{proposition}
The following are equivalent:
\begin{enumerate}
\item $F \overset{\pseudotopMod(\xi)}{\longrightarrow} x$;
\item $\forall H\in \powerultrafilters(X): (F\subseteq H) \implies \big(H \overset{\xi}{\longrightarrow} x\big)$;
\item $x\in \bigcap_{U\in \powerultrafilters(X)\cap \pfilter{F}}\lim_\xi U$;
\item $x\in \bigcap_{G\mesh F}\adh_\xi(G)$.
\end{enumerate}
\end{proposition}

\begin{proposition} \label{pseudotopModFunctorial}
Let $\sSet{X, \xi}, \sSet{Y,\zeta}$ be convergence spaces and $f: X\to Y$ a function. If $f$ is continuous, then $f: \pseudotopMod(X)\to \pseudotopMod(Y)$ is also continuous.
\end{proposition}
\begin{proof}
Assume $f$ is continuous and let $F$ be a filter that converges to $x$ in $\pseudotopMod(X)$. Take an ultrafilter $U \supseteq f^{\imf\imf}(F)$. Then, by \ref{mappingUltrafiltersLemma}, there exists some ultrafilter $U'\supseteq F$ such that $\upset f^{\imf\imf}(U') = U$. Now $U'$ converges to $x$ in $\xi$ by construction of the pseudotopological modification \ref{pseudotopModConstruction}, so $U$ converges to $f(x)$ by continuity of $f$. We conclude that $f^{\imf\imf}(F)$ converges to $f(x)$ in $\pseudotopMod(Y)$.
\end{proof}

\begin{proposition} \label{pseudotopologiserCommutesWithInitialStructure}
Let $Y$ be a set, $\sSet{Z_i, \zeta_i}$ a convergence space for all $i\in I$ and $\{f_i: Y\to Z_i\}_{i\in i}$. Then the pseudotopological modification of the initial convergence w.r.t. $\{f_i: Y\to Z_i\}_{i\in i}$ is equal to the initial convergence w.r.t. $\{f_i: Y\to \pseudotopMod(Z_i)\}_{i\in i}$.
\end{proposition}
\begin{proof}
Let $\mu$ be the initial convergence w.r.t. $\{f_i: Y\to Z_i\}_{i\in i}$. Then each $f_i: Y\to \pseudotopMod(Z_i)$ is continuous and the initial convergence w.r.t. $\{f_i: Y\to \pseudotopMod(Z_i)\}_{i\in i}$ is pseudotopological by \ref{pretopologicalInitialConvergence}. Thus $\pseudotopMod(\mu)$ is stronger than the initial convergence w.r.t. $\{f_i: Y\to \pseudotopMod(Z_i)\}_{i\in i}$.

For the converse, it is enough to note that for each ultrafilter $U\in \powerultrafilters(Y)$, convergence in the initial convergence w.r.t. $\{f_i: Y\to \pseudotopMod(Z_i)\}_{i\in i}$ implies convergence in $\mu$. Indeed $f_i^{\imf\imf}(U)$ is an ultrafilter (by \ref{imageFilterProperties}) that converges in $\pseudotopMod(Z_i)$ and thus also in $Z_i$.
\end{proof}
In particular $\pseudotopMod(A\times B) = \pseudotopMod(A)\times \pseudotopMod(B)$.

\subsubsection{Pretopological spaces}
\begin{lemma}
Let $X$ be a set. The set of pretopological convergences on $X$ is a subset of the lattice of preconvergences that is closed under meets.
\end{lemma}
\begin{definition}
Let $\sSet{X,\xi}$ be a convergence space. Then the \udef{pretopological modification} $\pretopMod(\xi)$ is the closure of $\xi$ into the lattice of pretopological convergences.
\end{definition}

\begin{proposition} \label{pretopologicalModification}
Let $\sSet{X,\xi}$ be a convergence space. Then the pretopological modification $\pretopMod(\xi)$ exists and is given by
\[ F \overset{\pretopMod(\xi)}{\longrightarrow} x \quad\iff\quad \vicinity_\xi(x) \subseteq F. \]
\end{proposition}

\begin{lemma}
Let $\sSet{X,\xi}$ be a convergence space, $x\in X$ and $A\subseteq X$. Then
\begin{enumerate}
\item $\vicinity_{\xi}(x) = \vicinity_{\pretopMod(\xi)}(x)$;
\item $\adh_\xi(A) = \adh_{\pretopMod(\xi)}(A)$ and $\inh_\xi(A) = \inh_{\pretopMod(\xi)}(A)$;
\item $\neighbourhood_{\xi}(x) = \neighbourhood_{\pretopMod(\xi)}(x)$;
\item $\topology_{\xi} = \topology_{\pretopMod(\xi)}$;
\item $\interior_\xi(A) = \interior_{\pretopMod(\xi)}(A)$ and $\closure_\xi(A) = \closure_{\pretopMod(\xi)}(A)$.
\end{enumerate}
\end{lemma}
\begin{proof}
(1) is immediate from \ref{pretopologicalModification}. The rest follows immediately from (1).
\end{proof}


\subsubsection{Topological spaces}
\begin{lemma}
Let $X$ be a set. The set of topological convergences on $X$ is a subset of the lattice of preconvergences that is closed under meets.
\end{lemma}
\begin{proof}
Let $\{\xi_i\}_{i\in I}$ be a set of topological convergences on $X$. We need to show that $\bigwedge_{i\in I}\xi_i$ is topological, for which we use \ref{pretopologicalSpaceTopological}.

TODO!!
\end{proof}
TODO: easier to verify that closure operator is a closure operator?

\begin{definition}
Let $\sSet{X,\xi}$ be a convergence space. Then the \udef{topological modification} $\topMod(\xi)$ is the closure of $\xi$ into the lattice of topological convergences.
\end{definition}

\begin{proposition} \label{topologicalModificationConstruction}
Let $\sSet{X,\xi}$ be a convergence space. Then the topological modification $\topMod(\xi)$ exists and is given by
\[ F \overset{\topMod(\xi)}{\longrightarrow} x \quad\iff\quad \neighbourhood_\xi(x) \subseteq F. \]
\end{proposition}

\begin{proposition} \label{topologicalModificationPreservation}
Let $\sSet{X,\xi}$ be a convergence space, $A\subseteq X$ and $x\in X$. Then
\begin{enumerate}
\item $A$ is open (closed) in $\topMod(\xi)$ \textup{if and only if} $A$ is open (closed) in $\xi$;
\item $\topology_{\topMod(\xi)} = \topology_\xi$;
\item $\interior_{\topMod(\xi)}(A) = \interior_{\xi}(A)$ and $\closure_{\topMod(\xi)}(A) = \closure_{\xi}(A)$;
\item $\neighbourhood_{\topMod(\xi)}(x) = \neighbourhood_\xi(x)$.
\end{enumerate}
\end{proposition}
\begin{proof}
(1) By \ref{openClosedCriteria}, we have that $A$ is open in $\topMod(\xi)$ iff $A\in \vicinity_{\topMod(\xi)}(x)$ for all $x\in A$. Thus
\begin{align*}
\text{$A$ is $\topMod(\xi)$-open} &\iff \forall x\in A: \; A\in \vicinity_{\topMod(\xi)}(x) \\
&\iff \forall x\in A: \; A\in \neighbourhood_\xi(x) \\
&\iff \forall x\in A: \; x\in \interior_\xi(A) \\
&\iff A\subseteq \interior_\xi(A) \\
&\iff \text{$A$ is $\xi$-open}.
\end{align*}
The statement for closed sets follows straight from \ref{openClosedComplement}.

(2), (3) Immediate.

(4) From \ref{topologicalModificationConstruction}, $\neighbourhood_\xi(x) = \vicinity_{\topMod(\xi)}(x)$. The result follows because $\topMod(\xi)$ is topological.
\end{proof}


\begin{lemma} \label{subspaceTopologicalModificationInclusion}
Let $\sSet{X,\xi}$ be a convergence space and $A\subseteq X$ a subset. Then $\topMod(\xi|_A)\subseteq \topMod(\xi)|_A$.
\end{lemma}
TODO: generalise and strengthen??
\begin{proof}
We have that $\iota_A: \xi|_A \to \xi$ is continuous by definition. Then $\topMod(\iota_A): \topMod(\xi|_A) \to \topMod(\xi)$ is continuous by \ref{convergenceModificationFunctoriality}. This implies that $\topMod(\xi|_A) \subseteq \topMod(\xi)|_A$.
\end{proof}

\subsection{Pavement properties}
\label{pavementModificationConvergence}
\begin{lemma}
Let $X$ be a set and $\mathcal{B}\subseteq \powerfilters(X)$ a set of filters. The set of convergences on $X$ that are paved in $\mathcal{B}$ is closed under joins.
\end{lemma}
\begin{proof}
Let $\Xi$ be a set of convergences on $X$ that are all paved in $\mathcal{B}$. And suppose $F\in\powerfilters(X)$ conververges to $x\in X$ according to $\bigvee \Xi$. By \ref{latticeConvergences}, there exists $\xi\in \Xi$ such that $F\overset{\xi}{\longrightarrow} x$.

By assumption, there exists $G\in\powerfilters(X)$ such that $G\subseteq F$, $G\overset{\xi}{\longrightarrow} x$ and $G$ is based in $\mathcal{B}$.

Then $G\overset{\Xi}{\longrightarrow} x$. As $F\overset{\xi}{\longrightarrow} x$ was chosed arbitrarily, we have that $\Xi$ is based in $\mathcal{B}$.
\end{proof}

\begin{definition}
Let $\sSet{X, \xi}$ be a convergence space and $\mathcal{B}$ a set of filters. The \udef{$\mathcal{B}$-paved modification} $\pavedMod_\mathcal{B}(\xi)$ is the coclosure into the lattice of $\mathcal{B}$-paved convergences.
\end{definition}

\begin{lemma} \label{pavedModificationConstruction}
Let $\sSet{X, \xi}$ be a convergence space and $\mathcal{B}$ a set of filters. Then
\[ \pavedMod_\mathcal{B}(\xi) = \setbuilder{(F,x)\in \powerfilters(X)\times X}{\exists G\in \powerfilters(X): \; G\subseteq F, G\overset{\xi}{\longrightarrow} x, \text{$G$ is based in $\mathcal{B}$}}. \]
\end{lemma}




\chapter{Sequential spaces}

\section{Sequential filters}
A sequential filter a filter that is the tails filter of a sequence.

\begin{proposition} \label{sequenceConstructionsSequentialFilters}
Let $X$ be a set.
\begin{enumerate}
\item If $A\subseteq X$ is a countable subset, then the principal filter $\upset\{A\}$ is sequential.
\item If $A\subseteq X$ is countably infinite, then the cofinite filter
\[ C \defeq \setbuilder{B\in\powerset(A)}{\text{$A\setminus B$ is finite}} \]
is sequential.
\item If $F,G\in\powerfilters(X)$ are sequential filters, then $F\cap G$ is sequential.
\end{enumerate}
\end{proposition}
\begin{proof}
(1) Take some enumeration (i.e.\ surjective function) $f:\N\to A$.

Consider the ``triangle read by rows'' sequence $t_n = g(n,1)$, where $g: \N\times\N\to \N$ is defined recursively by
\[ (n,m) \mapsto \begin{cases}
n & (n < m) \\
g(n-m, m+1) & (\text{else}).
\end{cases} \]
The sequence $t_n$ goes as
\[ 0,0,1,0,1,2,0,1,2,3,0,1,2,3,4 \ldots \]
Now consider the sequence $\seq{f(t_n)}$. It is clear that each tail of this sequence is equal to $A$ and thus the filter of $\seq{f(t_n)}$ is equal to $\upset\{A\}$.

(2) Let $f: \N\to A$ be a bijection. Then the filter generated by $f$ is the cofinite filter.

Indeed take $B\in C$. Each element not in $B$ (of which there are finitely many) appears in the sequence $f$. Let $k$ be the largest index of an element not in $B$. Then the tail $\setbuilder{f_n}{n\geq k+1}$ is contained in $B$, so $B$ is in the filter generated by $f$.

Conversely, let $B = \setbuilder{x_n}{n\geq k}$ be a tail of $f$. By surjectivity $A = \setbuilder{x_n}{n<k}\uplus \setbuilder{x_n}{n\geq k} = B^c \uplus B$. So only finitely many elements of $A$ are not in $B$, which means that $B$ is in the cofinite filter.

(3) Let $F$ be generated by the sequence $\seq{x_n}$ and $G$ be generated by the sequence $\seq{y_n}$. Then we claim $F\cap G$ is generated by the sequence
\[ z_n \defeq \begin{cases}
x_{n/2} & (\text{$n$ is even}) \\
y_{(n-1)/2} & (\text{$n$ is odd}).
\end{cases} \]
Indeed, for $A\subseteq X$, we have $A\in F\cap G$ iff $A$ contains both a tail $\setbuilder{x_n}{n\geq k}$ and a tail $\setbuilder{y_n}{n\geq k'}$, i.e.\ it contains the union. By replacing $k,k'$ by $\max\{k,k'\}$, we may take $k=k'$ WLOG. This is the case iff $A$ contains a tail of $\seq{z_n}$.
\end{proof}

\begin{proposition} \label{sequentialFilterCriteria}
Let $X$ be a set and $F\in\powerfilters(X)$ a proper filter.
Then the following are equivalent:
\begin{enumerate}
\item $F$ is sequential;
\item $F$ contains a countable set and admits a countable base $\mathcal{B}$ such that $B\setminus B'$ is finite for all $B,B'\in \mathcal{B}$;
\item $F$ contains a countable set and its free part is a cofinite filter.
\end{enumerate}
The trivial filter $\powerset(X)$ is never sequential.
\end{proposition}
\begin{proof}
$\boxed{(1) \Rightarrow (2)}$ Assume $F = \TailsFilter(\seq{x_n})$ for some sequence $\seq{x_n}$. Then every set in $\Tails(\seq{x_n})$ is countable and an element of $F$.

The set $\Tails(\seq{x_n})$ is a countable base of $F$ and for any $B = \setbuilder{x_k}{k\geq n}$ and $B' = \setbuilder{x_k}{k\geq n'}$, we have $B\setminus B' = \setbuilder{x_k}{n\leq k < n'}$, which is finite.

$\boxed{(2) \Rightarrow (3)}$ By \ref{cofiniteBaseFreePart}, we have
\[ F_0 = \bigvee_{B\in\mathcal{B}}\big(B\setminus \ker(F)\big)_0. \]
Now for all $B, B'\in \mathcal{B}$, we have that $\big(B\setminus \ker(F)\big)\symdiff \big(B'\setminus \ker(F)\big)$ is finite, so $\big(B\setminus \ker(F)\big)_0 = \big(B'\setminus \ker(F)\big)_0$ by \ref{cofiniteFiltersSameIffDifferenceFinite}.

As all the terms of the join are the same, we have that $F_0 = \big(B\setminus \ker(F)\big)_0$ for any $B\in\mathcal{B}$ and thus that $F_0$ is a cofinite filter.

$\boxed{(3) \Rightarrow (1)}$ The decomposition into free and principal parts is given by \ref{freePrincipalDecomposition}. If $F$ contains a countable set, then its kernel must also be countable and thus its principal part is sequential by \ref{sequenceConstructionsSequentialFilters}.

Let $A$ be a countable set in $F$. Then $A$ is also an element of the free part of $F$. By assumption, the free part of $F$ is the cofinite filter of some set $B$. Thus there exists a finite set $C$ such that $B\setminus C \subseteq A$. This implies that $B$ is countable.

If $B$ is finite, then the cofinite filter is trivial and $F$ equals its principal part, which we already argued was sequential.

If $B$ is infinite, then the free part of $F$ is sequential by \ref{sequenceConstructionsSequentialFilters} and thus $F$, which is the meet of its free and principal parts, is also sequential by \ref{sequenceConstructionsSequentialFilters}.
\end{proof}
\begin{corollary}
An ultrafilter is sequential \textup{if and only if} it is principal.
\end{corollary}
\begin{proof}
First assume the ultrafilter $U$ is sequential. By the proposition in combination with \ref{ultrafilterPrincipalOrFree}, $U$ is either principal or a cofinite filter. The latter case is excluded by \ref{cofiniteFilterNotUltra}.

Now assume $U$ is principal. Then it is of the form $\pfilter{x}$ and thus generated by the constant sequence $\seq{x}_{n\in \N}$.
\end{proof}

\begin{lemma} \label{countablyBasedFilterHasFinerSequential}
Let $X$ be a set and $F\in\powerfilters(X)$ a countably based filter. Then there exists a sequential filter $G$ such that $F\subseteq G$.
\end{lemma}
This lemma supposes the axiom of countable choice.
\begin{proof}
We can construct a decreasing sequence $\seq{B_n}$ of subsets of $X$ that form a base of $F$ by \ref{decreasingCountableBaseFilter}.

If this base is infinite, we can take $\seq{B_n}$ to be strictly decreasing. If the base is finite, then $F$ is the principal filter generated by its least element. Take any element $x$ in this least basis element and we may take $G = \pfilter{x}$.

Now let $\seq{B_n}$ be strictly decreasing and construct a set $A$ by choosing one element from each set $B_n\setminus B_{n+1}$. We claim $F\vee \upset\{A\}$ is sequential. Indeed, it is proper by \ref{joinProperFilter}, contains the countable set $A$ and, by \ref{baseMeetJoinFilters}, we have a base $\{B_n\cap A\}_{n\in \N}$. 
For all $k\geq m\in \N$, we have that $(B_m\cap A)\setminus (B_k\cap A)$ contains $k-m$ elements (i.e.\ a finite amount).
We conclude by \ref{sequentialFilterCriteria}.
\end{proof}

\begin{proposition}
Let $X$ be a set and $F,G\in\powerfilters(X)$ sequential filters. If $F\amesh G$, then $F\vee G$ is sequential.
\end{proposition}
\begin{proof}
By \ref{sequentialFilterCriteria}, $F$ contains a countable set and has a countable base $\mathcal{B}$ such that the difference of any two elements of $\mathcal{B}$ is finite. For $G$ there exists a similar base $\mathcal{C}$. By \ref{baseMeetJoinFilters}, the set $\mathcal{D} \defeq \setbuilder{B\cap C}{B\in\mathcal{B}, C\in\mathcal{C}}$ is a base of $F\vee G$ and it is countable. Take arbitrary $B\cap C, B'\cap C'\in \mathcal{D}$.
By \ref{differenceProperties},
\[ (B\cap C)\setminus(B'\cap C') = \big((B\cap C)\setminus B'\big) \cup \big((B\cap C)\setminus C'\big) \subseteq (B\setminus B') \cup (C\setminus C'), \]
which is finite. Finally, since $F\subseteq F\vee G$, the filter $F\vee G$ contains a countable set. Thus $F\vee G$ is sequential by \ref{sequentialFilterCriteria}.
\end{proof}

\begin{proposition} \label{traceOfSequentialFilterSequential}
Let $X$ be a set, $A\subseteq X$ a subset and $F\in\powerfilters(X)$ a sequential filter. Then $F|_A$ is sequential.
\end{proposition}
\begin{proof}
By \ref{sequentialFilterCriteria}, $F$ contains a countable set $C$. Then $F|_A$ contains $C\cap A$, which is also countable.

Also by \ref{sequentialFilterCriteria} $F$ has a countable base $\mathcal{B}$ such that $B\setminus B'$ is finite for all $B,B'\in \mathcal{B}$. Then $\mathcal{D} \defeq \setbuilder{B\cap A}{B\in\mathcal{B}}$ is a countable base of $F|_A$. Take arbitrary $B\cap A, B'\cap A\in \mathcal{D}$. By \ref{differenceProperties},
\[ (B\cap A)\setminus(B'\cap A) = \big((B\cap A)\setminus B'\big) \cup \big((B\cap A)\setminus A\big) = (B\cap A)\setminus B' \subseteq B\setminus B', \]
which is finite. Thus $F|_A$ is sequential by \ref{sequentialFilterCriteria}.
\end{proof}

\subsection{Fréchet filters}
\begin{definition}
Let $X$ be a set and $F\in\powerfilters(X)$. We call $F$ a \udef{Fréchet filter} if
\[ F = \bigwedge_{G\in \mathcal{G}}G \]
for some set $\mathcal{G}$ of sequential filters.
\end{definition}
This terminology is fairly non-standard. Often ``Fréchet filter'' is used to mean ``cofinite filter''.

\begin{lemma}
Let $X$ be a set and $F\in\powerfilters(X)$. Then $F$ is Fréchet \textup{if and only if}
\[ F = \bigwedge_{\substack{F\subseteq G \\ \text{$G$ is sequential}}} G. \]
\end{lemma}

\begin{proposition}
Let $X$ be a set and $F\in\powerfilters(X)$. Then the following are equivalent:
\begin{enumerate}
\item $F$ is Fréchet;
\item $A\in F^\mesh$ implies that there exists a sequential filter $G$ such that $F\vee \upset\{A\} \subseteq G$;
\item $A\in F^\mesh$ implies that there exists a countably based filter $H$ such that $F\vee \upset\{A\} \subseteq H$.
\end{enumerate}
\end{proposition}
\begin{proof}
$(1) \Rightarrow (2)$ By \ref{grillIntersectionUnion}, we have
\[ A\in F^\mesh = \hspace{0.7em} \Big( \hspace{-1.5em} \bigcap_{\substack{F\subseteq E \\ \text{$E$ is sequential}}} \hspace{-1.7em} E\Big)^\mesh = \bigcup_{\substack{F\subseteq E \\ \text{$E$ is sequential}}} \hspace{-1.5em} E^\mesh, \]
so there exists a sequential $E\in\powerfilters(X)$ such that $A\in E^\mesh$ and $F\subseteq E$. Then we may take $G = E\vee \upset\{A\}$, which is proper and sequential (by \ref{sequentialFilterCriteria} and \ref{baseMeetJoinFilters}).

$(1) \Leftarrow (2)$ We clearly have
\[ F \subseteq \bigwedge_{\substack{F\subseteq G \\ \text{$G$ is sequential}}}G, \]
so we need to prove the opposite inclusion. Suppose, towards a contradiction, that there exists
\[ B \in \bigwedge_{\substack{F\subseteq G \\ \text{$G$ is sequential}}} G \setminus F, \]
so $B\notin F$. By \ref{complementInIsotoneGrill} we have $B^c\in F^\mesh$. By assumption there exists a sequential filter $G'$ such that $F\vee \upset\{B^c\} \subseteq G'$. As $B\notin G'$, we have that $B\notin \bigwedge_{\substack{F\subseteq G \\ \text{$G$ is sequential}}} G$, which is a contradiction.

$(2) \Rightarrow (3)$ All sequential filters are countably based by \ref{sequentialFilterCriteria}.

$(2) \Leftarrow (3)$ Immediate by \ref{countablyBasedFilterHasFinerSequential}.
\end{proof}
\begin{corollary} \label{countablyBasedFiltersFrechet}
All countably based filters are Fréchet filters
\end{corollary}
\begin{proof}
If $F$ is countably based, then $F\vee \upset\{A\}$ is countably based (see \ref{baseMeetJoinFilters}).
\end{proof}

\begin{example}
There exist Fréchet filters that are not countable based. See Dolecki, Mynard p50.
\end{example}


\subsection{Limits and convergence}
\begin{definition}
Let $(X,\mathcal{T})$ be a topological space and $(a_n)_{n\in\N}$ a sequence in $X$. We can view this sequence as a function $\N\subset (\N\cup\{\infty\})\to X$.

We define \udef{limit} of the sequence as a limit of this function at $\infty$ if $\N\cup\{\infty\}$ is equipped with the order topology.
\end{definition}
By \ref{HausdorffUniqueLimit} a sequence has at most one limit $L\in X$ if $X$ is Hausdorff. In this case we call it the limit of the sequence and write
\[ \lim_{n\to \infty}a_n = L. \]
We call a sequence \udef{divergent} if it does not have a limit and
 \udef{convergent} if it does have a limit $L$. In this last case we say the sequence \udef{converges} to $L$.

\begin{proposition} \label{sequenceConvergence}
Let $(X,\mathcal{T})$ be a topological space and $(a_n)_{n\in\N}$ a sequence in $X$. Then the sequence converges to $L\in X$ \textup{if and only if}
\[ \forall \;\text{open neighbourhood}\; V(L): \exists n_0\in \N: \forall n\geq n_0: a_n\in V(L). \]
\end{proposition}
\begin{proof}
Assume the sequence converges to $L$ and take an arbitrary open neighbourhood $V(L)$. Then there exists an open neighbourhood $U(\infty)$ such that $a^{\imf}(U\setminus\{\infty\}) \subseteq V$. Now by definition of the order topology, there exists an interval $\interval[oc]{m, \infty}\subseteq U(\infty)$. Then $a^{\imf}(\interval[co]{m, \infty}) \subseteq V$ and by setting $n_0=m+1$ we get the criterion of the proposition.

Conversely assume the criterion and fix $V(L)$. Then we can take $U(\infty)=\interval[co]{n_0, \infty}$.
\end{proof}

\begin{lemma} \label{subsequencesConverge}
Let $(X,\mathcal{T})$ be a topological space and $(a_n)_{n\in\N}$ a sequence in $X$ that converges to $L$. Then all subsequences converge to $L$.
\end{lemma}

TODO: Bolzano-Weierstrass (sequence version + accumulation point version)

\section{Sequentially paved spaces and the sequential modification}
\begin{definition}
Let $X$ be a set. A convergence $\xi$  on $x$ is called \udef{sequentially paved} if it is paved in the set of sequential filters in $X$.

The \udef{sequentially paved modification} of $\xi$ is denoted $\seqMod(\xi)$.
\end{definition}
See \ref{pavementModificationConvergence} for details about $\seqMod$.

\begin{lemma}
Let $\sSet{X, \xi}$ be a sequentially paved convergence space and $D:X\to \powerset(X)$ a function such that $x\in D(x)$. Then the convergence in the direction $D$ is also sequentially paved.
\end{lemma}
\begin{proof}
Let $F\in \powerfilters(X)$ converge to $x$ in the directional convergence. Then $D(x)\in F$ and there exists a sequence $\seq{x_n}$ such that $\TailsFilter\seq{x_n}\subseteq F$ and $\seq{x_n}$ converges to $x$ in $\xi$.

We claim there exists a subsequence of $\seq{x_n}$ that lies in $D(x)$, which we can obtain by removing every term that does not lie in $D(x)$. It is enough to see that every tail of $\seq{x_n}$ meshes with $D$, which is true because every tail is an element of $F$ and $D\in F$.

This subsequence $\seq{x'_n}$ clearly converges to $x$. We also have $\TailsFilter\seq{x_n'}\subseteq F$.
\end{proof}

\subsection{Sequential properties}
\subsubsection{Sequential continuity}
\begin{definition}
Let $\sSet{X,\xi}, \sSet{Y,\zeta}$ be convergence spaces and $f:X\to Y$ a function. Then $f$ is called \udef{sequentially continuous} if $f: \sSet{X, \seqMod(\xi)} \to \sSet{Y, \seqMod(\zeta)}$ is continuous.
\end{definition}

\begin{proposition}
Let $\sSet{X,\xi}, \sSet{Y,\zeta}$ be convergence spaces and $f:X\to Y$ a function. Then the following are equivalent:
\begin{enumerate}
\item $f$ is sequentially continuous;
\item $f: \sSet{X, \seqMod(\xi)} \to \sSet{Y, \zeta}$ is continuous;
\item the sequence $\seq{f(x_n)}$ converges to $f(x)$ for all convergent sequences $\seq{x_n}\overset{\xi}{\longrightarrow} x$ in $X$.
\end{enumerate}
\end{proposition}
\begin{proof}
$(1) \Rightarrow (2)$ Immediate.

$(2) \Rightarrow (3)$ Let $\seq{x_n}$ be a sequence such that $\TailsFilter\seq{x_n}\overset{\xi}{\longrightarrow} x$. Now $\TailsFilter\seq{x_n}$ also converges to $x$ in $\seqMod(\xi)$, so $\upset f^{\imf\imf}\big(\TailsFilter\seq{x_n}\big)$ converges to $f(x)$ in $\zeta$.

By \ref{imageTailsFilter}, $\upset f^{\imf\imf}\big(\TailsFilter\seq{x_n}\big) = \TailsFilter\seq{f(x_n)}$, so $\seq{f(x_n)}$ converges to $f(x)$.

$(3) \Rightarrow (1)$ Take $F\in \powerfilters(X)$ such that $F\overset{\seqMod(\xi)}{\longrightarrow} x$. Then there exists a sequence $\seq{x_n}$ that converges to $x$ such that $\TailsFilter\seq{x_n}\subseteq F$. Now $\seq{f(x_n)}$ converges to $f(x)$. 

By \ref{imageTailsFilter}, $\TailsFilter\seq{f(x_n)} = \upset f^{\imf\imf}\big(\TailsFilter\seq{x_n}\big) \subseteq \upset f^{\imf\imf}(F)$, so $\upset f^{\imf\imf}(F)$ converges to $f(x)$ in $\seqMod(\zeta)$.
\end{proof}
\begin{corollary} \label{continuityImpliesSequentialContinuity}
Let $\sSet{X,\xi}, \sSet{Y,\zeta}$ be convergence spaces and $f:X\to Y$ a function. If $f$ is continuous, then $f$ is sequentially continuous.
\end{corollary}
\begin{proof}
Immediate from point (2) and the fact that $\seqMod(\xi) \leq \xi$.
\end{proof}

\subsubsection{Sequential inherence and adherence}
\begin{definition}
Let $\sSet{X,\xi}$ be a convergence space and $A\subseteq X$ a subset. Then we call
\begin{itemize}
\item $\inh_{\seqMod(\xi)}$ the \udef{sequential inherence} of $A$;
\item $\adh_{\seqMod(\xi)}$ the \udef{sequential adherence} of $A$.
\end{itemize}
\end{definition}
The sequential inherence is sometimes called the sequential interior and the sequential adherence is sometimes called the sequential closure.

\begin{proposition} \label{sequentialInherenceAdherence}
Let $\sSet{X,\xi}$ be a convergence space and $A\subseteq X$ a subset. Then
\begin{enumerate}
\item $\inh_{\seqMod(\xi)} = \setbuilder{s\in S}{\text{every sequence in $X$ that converges to $s$ has a tail in $S$}}$;
\item $\adh_{\seqMod(\xi)} = \setbuilder{x\in X}{\text{$\exists$ a sequence in $S$ that converges to $x$ in $X$}}$.
\end{enumerate}
\end{proposition}
\begin{proof}
TODO
\end{proof}

\section{Sequential spaces}
\begin{definition}
Let $\sSet{X,\xi}$ be a convergence space. Then
\begin{itemize}
\item $\xi$ is called \udef{sequential} if $\topMod(\xi) = (\topMod\circ \seqMod)(\xi)$;
\item $X$ is called a \udef{sequential space}.
\end{itemize} 
\end{definition}

\begin{lemma} \label{sequentialLemma}
Let $\sSet{X,\xi}$ be a convergence space. The following are equivalent
\begin{enumerate}
\item $\xi$ is sequential;
\item $\xi \leq (\topMod\circ \seqMod)(\xi)$;
\item $\closure_\xi = \closure_{\seqMod(\xi)}$.
\end{enumerate}
\end{lemma}
\begin{proof}
$(1) \Rightarrow (2)$ We have $\xi \leq \topMod(\xi) = (\topMod\circ \seqMod)(\xi)$.

$(2) \Rightarrow (3)$  We have $\closure_\xi \leq \closure_{(\topMod\circ \seqMod)(\xi)} = \closure_{\seqMod(\xi)} \leq \closure_\xi$, using \ref{interiorClosureMonotoneInConvergence} and \ref{topologicalModificationPreservation}.

$(3) \Rightarrow (1)$ Since the closure operator uniquely determines the topology (the topology consists of the complement of the sets in the image of the closure operator), the topology uniquely determines a topological convergence (\ref{specifyingTopology}) and the topological modification of a convergence has the same topology as the convergence (\ref{topologicalModificationPreservation}), we have $\topMod(\xi) = (\topMod\circ \seqMod)(\xi)$.
\end{proof}

\begin{proposition} \label{sequentialSpaceSequentialContinuity}
Let $\sSet{X,\xi}$ be a convergence space. The following are equivalent
\begin{enumerate}
\item $\xi$ is sequential;
\item for any function $f: X\to \sSet{Y,\tau}$ to a topological convergence space $\tau$, $f$ is continuous \textup{if and only if} $f$ is sequentially continuous.
\end{enumerate}
\end{proposition}
\begin{proof}
$(1) \Rightarrow (2)$ By functoriality of $\seqMod$ (TODO ref), it is enough to prove that in this case sequential continuity implies continuity. Assume $f$ is sequentially continuous, so $f: \sSet{X, \seqMod(\xi)} \to \sSet{Y, \seqMod(\tau)}$ is continous. By functoriality of $\topMod$ (TODO ref), this implies that $f: \sSet{X, (\topMod\circ \seqMod)(\xi)} \to \sSet{Y, (\topMod\circ\seqMod)(\tau)}$ is continous.

Then, since $\seqMod(\tau) \leq \tau$, we have $f: \sSet{X, (\topMod\circ \seqMod)(\xi)} \to \sSet{Y, \topMod(\tau)}$ is continuous and thus $f: \sSet{X, \topMod(\xi)} \to \sSet{Y, \tau}$ is continuous because $\tau = \topMod(\tau)$ and $\xi$ is sequential, so $\topMod(\xi) = (\topMod\circ \seqMod)(\xi)$. This implies that $f: \sSet{X, \xi} \to \sSet{Y, \tau}$ is continuous.

$(2) \Rightarrow (1)$ Set $\tau = (\topMod\circ \seqMod)(\xi)$ and $f = \id_X$. Now $\id_X: \xi \to \tau$ is sequentially continuous: from $\seqMod(\xi) \leq (\topMod\circ \seqMod)(\xi)$, we get $\seqMod(\xi) = \seqMod^2(\xi) \leq (\seqMod\circ \topMod\circ \seqMod)(\xi) = \seqMod(\tau)$.

By assumption, this means that $\id_X: \xi \to \tau$ is continuous, so $\xi \leq (\topMod\circ \seqMod)(\xi)$, which implies that $\xi$ is sequential by \ref{sequentialLemma}.
\end{proof}

\begin{theorem}
A convergence is sequential \textup{if and only if} it is the quotient of a metrisable convergence.
\end{theorem}
\begin{proof}
TODO Dolecki / Mynard p. 159
\end{proof}

\subsection{Fréchet-Urysohn spaces}
\begin{definition}
Let $\sSet{X,\xi}$ be a convergence space. Then
\begin{itemize}
\item $\xi$ is called a \udef{Fréchet-Urysohn} convergence if $\pretopMod(\xi) = (\pretopMod\circ \seqMod)(\xi)$;
\item $X$ is called a \udef{Fréchet-Urysohn space}.
\end{itemize} 
\end{definition}

\begin{lemma} \label{FrechetUrysohnLemma}
Let $\sSet{X,\xi}$ be a convergence space. The following are equivalent
\begin{enumerate}
\item $\xi$ is Fréchet-Urysohn;
\item $\xi \leq (\pretopMod\circ \seqMod)(\xi)$;
\item $\adh_\xi = \adh_{\seqMod(\xi)}$.
\end{enumerate}
\end{lemma}
\begin{proof}
$(1) \Rightarrow (2)$ We have $\xi\leq \pretopMod(\xi) = (\pretopMod\circ \seqMod)(\xi)$.

$(2) \Rightarrow (1)$ We have $\pretopMod(\xi) \leq (\pretopMod^2\circ \seqMod)(\xi) = (\pretopMod\circ \seqMod)(\xi)$. Conversely, $\xi \geq \seqMod(\xi)$ implies $\pretopMod(\xi) \geq (\pretopMod\circ \seqMod)(\xi)$.

$(1) \Leftrightarrow (3)$ Immediate, since $\adh_{\pretopMod(\zeta)} = \adh_\zeta$ for all convergences $\zeta$ and the adherence uniquely determines a pretopological space (see \ref{CechClosureInteriorPretopology}). 
\end{proof}

\begin{lemma}
Each Fréchet-Urysohn space is sequential.
\end{lemma}
\begin{proof}
Let $\xi$ be a Fréchet-Urysohn convergence. If $\pretopMod(\xi) = (\pretopMod\circ \seqMod)(\xi)$, then
\[ \topMod(\xi) = (\topMod\circ\pretopMod)(\xi) = (\topMod\circ\pretopMod\circ \seqMod)(\xi) = (\pretopMod\circ \seqMod)(\xi), \]
so $\xi$ is sequential.
\end{proof}

\begin{proposition} \label{FrechetUrysohnSequentialContinuity}
Let $\sSet{X,\xi}$ be a convergence space. The following are equivalent
\begin{enumerate}
\item $\xi$ is Fréchet-Urysohn;
\item for any function $f: X\to \sSet{Y,\zeta}$ to a pretopological convergence space $\zeta$, $f$ is continuous \textup{if and only if} $f$ is sequentially continuous.
\end{enumerate}
\end{proposition}
\begin{proof}
$(1) \Rightarrow (2)$ By functoriality of $\seqMod$ (TODO ref), it is enough to prove that in this case sequential continuity implies continuity. Assume $f$ is sequentially continuous, so $f: \sSet{X, \seqMod(\xi)} \to \sSet{Y, \seqMod(\zeta)}$ is continous. By functoriality of $\pretopMod$ (TODO ref), this implies that $f: \sSet{X, (\pretopMod\circ \seqMod)(\xi)} \to \sSet{Y, (\pretopMod\circ\seqMod)(\zeta)}$ is continous.

Then, since $\seqMod(\zeta) \leq \zeta$, we have that $f: \sSet{X, (\pretopMod\circ \seqMod)(\xi)} \to \sSet{Y, \pretopMod(\zeta)}$ is continuous and thus $f: \sSet{X, \pretopMod(\xi)} \to \sSet{Y, \zeta}$ is continuous because $\pretopMod(\zeta) = \zeta$ and $\xi$ is Fréchet-Urysohn and so $(\pretopMod\circ \seqMod)(\xi) = \pretopMod(\xi)$. This implies that $f: \sSet{X, \xi} \to \sSet{Y, \zeta}$ is continuous.

$(2) \Rightarrow (1)$ Set $\zeta = (\pretopMod\circ \seqMod)(\xi)$ and $f = \id_X$. Now $\id_X: \xi \to \zeta$ is sequentially continuous: from $\seqMod(\xi) \leq (\pretopMod\circ \seqMod)(\xi)$, we get $\seqMod(\xi) = \seqMod^2(\xi) \leq (\seqMod\circ \pretopMod\circ \seqMod)(\xi) = \seqMod(\zeta)$.

By assumption, this means that $\id_X: \xi \to \zeta$ is continuous (since $\zeta$ is pretopological), so $\xi \leq (\topMod\circ \seqMod)(\xi)$, which implies that $\xi$ is Fréchet-Urysohn by \ref{FrechetUrysohnLemma}.
\end{proof}


\begin{proposition}
A convergence is Fréchet-Urysohn \textup{if and only if} every subspace is sequential. (TODO: need additional assumptions?)
\end{proposition}
\begin{proof}
TODO!(?)
\end{proof}

\begin{proposition} \label{FrechetUrysohnVicinityCharacterisation}
Let $\sSet{X,\xi}$ be a convergence space. Then the following are equivalent:
\begin{enumerate}
\item $\xi$ is a Fréchet-Urysohn convergence;
\item $\vicinity_\xi(x) = \vicinity_{\seqMod(\xi)}(x)$ for all $x\in X$.
\end{enumerate}
If $\xi$ is Fréchet-Urysohn, then $\vicinity_\xi(x)$ is a Fréchet filter for all $x\in X$.

If $\xi$ is pretopological, then the converse also holds.
\end{proposition}
\begin{proof}
We have $\pretopMod(\xi) = (\pretopMod\circ \seqMod)(\xi)$ iff $\vicinity_\xi(x) = \vicinity_{\seqMod(\xi)}(x)$ for all $x\in X$.

Now we have
\[ \vicinity_{\seqMod(\xi)}(x) = \bigcap\setbuilder{F\in {\lim_{\xi}}^{-1}(x)}{\text{$F$ is sequential}}, \]
so $\vicinity_{\seqMod(\xi)}(x)$ is a Fréchet filter.

Now assume $\vicinity_\xi(x)$ is a Fréchet filter. Then there exists a set $\mathcal{G}$ of sequential filters such that $\vicinity_\xi(x) = \bigcap \mathcal{G}$. Since $\xi$ is pretopological, each element of $\mathcal{G}$ converges in $\xi$ and thus also in $\seqMod(\xi)$. So
\[ \vicinity_\xi(x) = \bigcap \mathcal{G} \supseteq \bigcap {\lim_{\seqMod(\xi)}}^{-1}(x) = \vicinity_{\seqMod(\xi)}(x) \supseteq \vicinity_{\xi}(x), \]
which means that $\vicinity_{\seqMod(\xi)}(x) = \vicinity_{\xi}(x)$ and $\xi$ is Fréchet-Urysohn.
\end{proof}
\begin{corollary} \label{C1ImpliesFrechetUrysohn}
Each $C1$ pretopological convergence space is Fréchet-Urysohn and thus sequential.
\end{corollary}
\begin{proof}
Each countably based filter is Fréchet by \ref{countablyBasedFiltersFrechet}.
\end{proof}

\begin{lemma} \label{FrechetUrysohnSequentialContinuityAtAPoint}
Let $\sSet{X,\xi}$ be a Fréchet-Urysohn convergence space, $\sSet{Y,\zeta}$ a pretopological convergence space, $x\in X$ and $f: X\to Y$ a function that is sequentially continuous at $x$. Then $f$ is continuous at $x$.
\end{lemma}
\begin{proof}
By \ref{specifiedPointsModificationVicinity} and \ref{FrechetUrysohnVicinityCharacterisation}, we see that $\xi_{\{x\}}$ is Fréchet-Urysohn.

Thus sequential continuity at $x$ implies continuity at $x$ by \ref{FrechetUrysohnSequentialContinuity} and \ref{continuityAtPointConvergenceLemma}.
\end{proof}

\begin{lemma} \label{existenceConvergentSequenceDenseSubset}
Let $\sSet{X,\xi}$ be a Fréchet-Urysohn space and $S\subseteq X$ a subset. Then $S$ is dense in $X$ \textup{if and only if} for all $x\in X$ there exists a sequence in $S$ that converges to $x$.
\end{lemma}
\begin{proof}
We immediately have the implication $\Leftarrow$ by \ref{convergentFiltersInDenseSet}.

Now suppose $S$ is $\xi$-dense in $X$. Then, by \ref{FrechetUrysohnVicinityCharacterisation}, it is $\seqMod(\xi)$-dense in $X$. By \ref{convergentFiltersInDenseSet}, this implies, for all $x\in X$, the existence of a filter $F$ that contains $S$ and $\seqMod(\xi)$-converges to $x$.
By \ref{pavedModificationConstruction}, there exists a sequential filter $G\in \powerfilters(X)$ such that $G\subseteq F$ and $G\to x$. Now, by \ref{properFiltersSelfMesh} and \ref{antitonicityPolars}, $A \in F \subseteq F^\mesh \subseteq G^\mesh$, so $G|_A$ is proper. It is also sequential by \ref{traceOfSequentialFilterSequential} and converges to $x$.
\end{proof}

\subsection{The sequential topology}
\begin{definition}
Let $(X,\mathcal{T})$ be a topological space and $S\subseteq X$ a subset.
\begin{itemize}
\item The \udef{sequential closure} of $S$ in $X$ is the set
\[ \operatorname{SeqCl}(S) \defeq \setbuilder{x\in X}{\text{$\exists$ a sequence in $S$ that converges to $x$ in $X$}}. \]
\item The \udef{sequential interior} of $S$ in $X$ is the set
\[ \operatorname{SeqInt}(S) \defeq \setbuilder{s\in S}{\text{every sequence in $X$ that converges to $s$ has a tail in $S$}}. \]
\end{itemize}
We call $S$
\begin{itemize}
\item \udef{sequentially open} if $S = \operatorname{SeqInt}(S)$;
\item \udef{sequentially closed} if $S = \operatorname{SeqCl}(S)$;
\item a \udef{sequential neighbourhood} of a point $x\in X$ if $x\in \operatorname{SeqInt}(S)$.
\end{itemize}
\end{definition}
A sequentially closed set is a set $S$ such that all limits of sequences in $S$ are also in $S$.

\begin{lemma} \label{sequentialInteriorClosure}
Let $(X,\mathcal{T})$ be a topological space and $R,S\subseteq X$  subsets. Then
\begin{enumerate}
\item $\operatorname{SeqInt}(S) = (\operatorname{SeqCl}(S^c))^c$;
\item $\operatorname{SeqCl}(\emptyset) = \emptyset$ and $\operatorname{SeqCl}(X) = X$;
\item $S\subseteq \operatorname{SeqCl}(S)$;
\item $\operatorname{SeqCl}(R\cup S) = \operatorname{SeqCl}(R)\cup \operatorname{SeqCl}(S)$;
\item $\operatorname{SeqCl}(S) \subseteq \bar{S}$ and $\operatorname{SeqInt}(S) \supseteq S^\circ$.
\end{enumerate}
\end{lemma}
\begin{proof}
(1) Both sides of the equation are equivalent to
\[ \setbuilder{s\in S}{\text{$\nexists$ a sequence in $X\setminus S$ that converges to $s$ in $X$}}. \]

(2) There are no sequences that converge to a point in $\emptyset$ and all points $x\in X$ are the limit of a constant sequence $n\mapsto x$.

(3) The constant sequence $n\mapsto x$ converges to $x$.

(4) We can find a subsequence in $R$ or in $S$. All subsequences converge by \ref{subsequencesConverge}.

(5) Let $x\in \operatorname{SeqCl}(S)$, so there is a sequence $(a_n)$ in $S$ that converges to $x$. Take an arbitrary open neighbourhood $V$ of $x$. Then by convergence there is a subsequence of $(a_n)$ that is a sequence in $V$. In particular $V$ intersects $S$. So $x\in \bar{S}$ by \ref{closure}.
\end{proof}

\begin{proposition} \label{sequentialTopology}
Let $(X,\mathcal{T})$ be a topological space. The set of all sequentially open sets forms a topology $\mathcal{T}_\text{seq}$ on $X$. This topology is finer than the original topology.
\end{proposition}
\begin{proof}
First note that sequentially closed sets are complements of sequentially open sets by point 1. of \ref{sequentialInteriorClosure}.

By point 2. of \ref{sequentialInteriorClosure}, $\emptyset$ and $X$ are both clopen.

We will prove the rest using closed sets. By point 4. of \ref{sequentialInteriorClosure} finite unions of sequentially closed sets are sequentially closed.

Let $\bigcap_{i\in I}K_i$ be an arbitrary intersection of sequentially closed sets $K_i$. We only need to prove
\[ \operatorname{SeqCl}\left(\bigcap_{i\in I}K_i\right) \subseteq \bigcap_{i\in I}K_i \]
because the other inclusion is immediate. Take an $x\in \operatorname{SeqCl}\left(\bigcap_{i\in I}K_i\right)$. Then there is a sequence in $\bigcap_{i\in I}K_i$ that converges to $x$. Because of the intersection this sequence is in each $K_i$ and thus so is $x$.

The fineness of the topology follows from point 5. of \ref{sequentialInteriorClosure}.
\end{proof}
\begin{corollary}
Let $(X,\mathcal{T})$ be a topological space and $S\subseteq X$  a subset. Then
\[ \text{open/closed} \quad\implies\quad \text{sequentially open/closed.} \]
\end{corollary}

\begin{proposition} \label{sequentialTopologySameConvergentSequences}
Let $(X,\mathcal{T})$ be a topological space and $(x_n)$ a sequence in $X$. Then $x_n\to x$ in $(X,\mathcal{T})$ \textup{if and only if} $x_n\to x$ in $(X,\mathcal{T}_\text{seq})$.
\end{proposition}
\begin{proof}
Now $\mathcal{T}_\text{seq}$ is finer than $\mathcal{T}$, so the $\Leftarrow$ direction is evident. For the $\Rightarrow$ direction, assume $x_n\to x$ in the original topology. Let $V(x)$ be an open neighbourhood in the sequential topology. By definition of the sequential topology $(x_n)$ has a tail in $V$. This means $x_n\to x$ in the sequential topology by \ref{sequenceConvergence}.
\end{proof}

\subsection{Transfinite sequential closure}
It is possible that the sequential closure is not idempotent (unlike the normal topological closure), i.e.\
\[ \operatorname{SeqCl}(\operatorname{SeqCl}(S)) \neq \operatorname{SeqCl}(S). \]

\subsection{Sequential continuity}
\begin{definition}
A function $f:(X,\mathcal{T})\to(Y,\mathcal{T}')$ is called \udef{sequentially continuous} if
\[ f:(X,\mathcal{T}_\text{seq})\to(Y,\mathcal{T}'_\text{seq}) \]
is continuous. i.e.\ $f$ is continuous when $X,Y$ are equipped with their sequential topologies.
\end{definition}

\begin{proposition} \label{sequentialContinuity}
A function $f:(X,\mathcal{T})\to(Y,\mathcal{T}')$ is sequentially continuous \textup{if and only if} for every sequence $(x_n)_{n\in\N}$ in $X$ and $x\in X$
\[ x_n \to x \;\;\text{in}\; (X,\mathcal{T}) \quad\implies\quad f(x_n)\to f(x) \;\;\text{in}\; (Y,\mathcal{T}'). \]
\end{proposition}
\begin{proof}
First assume this property holds and we want to prove sequential continuity. Let $S\subset Y$ be sequentially closed. Then we need to prove $f^{-1}[S]$ is also sequentially closed. Indeed take a converging sequence $(x_n)$ in $f^{-1}[S]$ with limit $x$. Then $(f(x_n))$ converges to $f(x)$ and $f(x)\in S$. This implies $x\in f^{-1}[S]$, meaning it is sequentially closed. 

Conversely, assume $f$ is sequentially continuous. Let $(x_n)$ be a sequence in $X$ that converges to $x$. Let $V(f(x))\in \mathcal{T}'$ be an open neighbourhood of $f(x)$; $V$ is also sequentially open. Then by continuity we have a $U(x)\in\mathcal{T}_\text{seq}$ such that $f[U]\subseteq V$. Because $U$ is sequentially open, there is an $n_0\in\N$ such that $\forall n\geq n_0: x_n\in U$.
This implies $\forall n\geq n_0: f(x_n)\in f[U]\subseteq V$ and so $(f(x_n))$ converges to $f(x)$.
\end{proof}
\begin{proposition}
Every continuous function is sequentially continuous.
\end{proposition}
\begin{proof}
We use the characterisation of sequential continuity in \ref{sequentialContinuity}. Let $x_n\to x$. Let $V$ be an open neighbourhood of $f(x)$. Then there exists an open neighbourhood $U(x)$ such that $f[U]\subset V$. By \ref{sequenceConvergence} $U$ contains all but finitely many elements of the sequence $(x_n)$. Thus $V$ contains all but finitely many of the elements of the sequence $(f(x_n))$. Take $n_0$ larger than the indices of all elements of $(f(x_n))$ omitted from $V$. By \ref{sequenceConvergence} $f(x_n)\to f(x)$.
\end{proof}

\subsection{Sequential spaces}
\begin{definition}
A topological space $(X,\mathcal{T})$ is called a \udef{sequential space} if $\mathcal{T} = \mathcal{T}_\text{seq}$.
\end{definition}

\begin{lemma}
A topological space $(X,\mathcal{T})$ is a sequential space if every sequentially open set is open.
\end{lemma}
\begin{proof}
We already know $\mathcal{T} \subseteq \mathcal{T}_\text{seq}$ from \ref{sequentialTopology}. The hypothesis of the lemma is that $\mathcal{T} \supseteq \mathcal{T}_\text{seq}$. Together this gives $\mathcal{T} = \mathcal{T}_\text{seq}$.
\end{proof}

\begin{lemma}
Let $(X,\mathcal{T})$ be a topological space. Then $X$ equipped with its sequential topology is a sequential space.
\end{lemma}
\begin{proof}
It is enough to show that a sequentially closed set in $(X,\mathcal{T}_\text{seq})$ is also sequentially closed in $(X,\mathcal{T})$. (i.e.\ that passing to the finer topology does not introduce even more sequentially open sets). By \ref{sequentialTopologySameConvergentSequences} the definition of $\operatorname{SeqCl}$ is the same in both topologies, yielding the proof.
\end{proof}

\begin{proposition}
Let $(X,\mathcal{T})$ be a topological space. Then the following are equivalent:
\begin{enumerate}
\item $(X,\mathcal{T})$ is a sequential space;
\item for every subset $S\subset X$ that is not closed in $X$, there exists some $x\in \bar{S}\setminus S$ for which there exists a sequence in $S$ that converges to $x$;
\item $(X,\mathcal{T})$ is the quotient of a first countable space;
\item $(X,\mathcal{T})$ is the quotient of a metric space.
\end{enumerate}
\end{proposition}
TODO: relocate observation about metric spaces.

\begin{proposition}[Universal property of sequential spaces]
Let $(X,\mathcal{T})$ be a topological space. Then $X$ is sequential \textup{if and only if} for every topological space $Y$, a function $f:X\to Y$ is continuous $\Leftrightarrow$ $f$ is sequentially continuous.
\end{proposition}


\subsection{$T$-sequential and $N$-sequential spaces}

\subsection{Fréchet-Urysohn spaces}
\begin{definition}
A topological space $(X,\mathcal{T})$ is called a \udef{Fréchet-Urysohn space} if for every subset $S\subseteq X$
\[ \operatorname{SeqCl}(S) = \bar{S}. \]
\end{definition}
Clearly every Fréchet-Urysohn space is a sequential space.

\begin{proposition} \label{FrechetUrysohn}
Let $(X,\mathcal{T})$ be a topological space. Then the following are equivalent:
\begin{enumerate}
\item $(X,\mathcal{T})$ is a Fréchet-Urysohn space;
\item every subspace of $X$ is a sequential space;
\item for every subset $S\subset X$ that is not closed in $X$ and for all $x\in \bar{S}\setminus S$ there exists a sequence in $S$ that converges to $x$.
\end{enumerate}
\end{proposition}

\subsection{Sequences in ordered space}
In this section we will be considering sequences in a totally ordered set $(X,\leq)$ equipped with the order topology.

\begin{lemma} \label{convergentSequenceIsBounded}
A convergent sequence in a totally ordered space has an upper and a lower bound.
\end{lemma}
\begin{proof}
Let $x_n\to x$. Choose a basis element containing $x$. If it is of the form $]a,b_0]$ for some greates element $b_0$, then $b_0$ is the upper bound. If not, it is of the form $[a_0,b[$ or $]a,b[$. Find an $n_0$ corresponding to this basis element. Then an upper bound is given by
\[ \max(x[\;[0,n_0]\;]\cup\{b\}). \]
The lower bound is analogous.
\end{proof}

\begin{proposition} \label{limitPreservesInequality}
Let $(a_n)$ and $(b_n)$ be convergent sequences in a totally ordered space such that $a_n\leq b_n$ for all $n\in\N$. Then
\[ \lim_{n\to \infty}a_n \leq \lim_{n\to \infty}b_n. \]
\end{proposition}
\begin{proof}
Let $a_n\to a$ and $b_n\to b$. If $a=b$ then the proposition is valid. Now assume $a\neq b$. If $a$ or $b$ are either the greatest or the least element, the proposition is valid. Now assume this is not the case.

Assume towards a contradiction that $a>b$. Then we can find open neighbourhoods of $a$ and $b$ of the form $]b,d[$ and $]c,a[$, respectively. Now find $n_0, n_1$ such that $\forall n\geq n_0: a_n \in ]b,d[$ and $\forall n\geq n_1: b_n \in ]c,a[$. 
Then for all $n \geq \max\{n_0,n_1\}$ we have $a_n\in ]b,d[$ and $b_n\in ]c,a[$, implying $a_n > b_n$ which is a contradiction.
\end{proof}
It is easy to show that this does not in general hold for the strict inequality $<$.

\begin{proposition}[Squeeze theorem for sequences]
Let $(a_n)$, $(b_n)$ and $(c_n)$ be sequences in a totally ordered space such that
\[ \forall n\in \N: a_n\leq b_n \leq c_n. \]
If $(a_n)$ and $(c_n)$ are convergent with the same limit $L$, then
\[ \lim_{n\to \infty}b_n = L. \]
\end{proposition}
\begin{proof}
Let $V(L)$ be an open neighbourhood of $L$. By definition of the order topology there is an interval $I = ]x,y[ \subset V$ such that $L\in I$. Then find $n_0$ and $n_1$ such that $\forall n\geq n_0: a_n\in I$ and $\forall n\geq n_1: c_n\in I$. Then set $n_2 = \max\{n_0,n_2\}$ and we have $\forall n\geq n_2:$
\[ x \leq a_n \leq b_n \quad \text{and} \quad  b_n \leq c_n \leq y. \]
By transitivity we have $b_n\in I \subset V$.
\end{proof}

\begin{proposition} \label{sequenceToSupInf}
Let $X$ be an ordered space and $A$ a subspace.  Assume the axiom of dependent choice.
\begin{enumerate}
\item If $A$ has a supremum $a$, then there exists a sequence in $A$ that converges to $a$ in $X$.
\item If $A$ has an infimum $b$, then there exists a sequence in $A$ that converges to $b$ in $X$.
\end{enumerate}
\end{proposition}
\begin{proof}
Assume the supremum $a$ of $A$ exists. If $a\in A$ we can take the constant sequence $(a)_{n\in \N}$.

If $a\notin A$, we can find for each $x_i\in A$ an $x_{i+1}$ satisfying $x_i < x_{i+1} < a$. The sequence thus defined converges by monotone convergence.
\end{proof}
In many cases the axiom of dependent choice is superfluous, if the details of the spaces $X,A$ allow for the construction of $x_{i+1}$ from $x_i$.

\subsection{Divergence to $\pm\infty$}
\begin{definition}
Let $(x_n)$ be a sequence in a totally ordered space $X$. Then
\begin{itemize}
\item $(x_n)$ \udef{diverges to $+\infty$} if $\forall M\in X: \exists n_0\in\N: \forall n\geq n_0: x_0 > M$; and
\item $(x_n)$ \udef{diverges to $-\infty$} if $\forall M\in X: \exists n_0\in\N: \forall n\geq n_0: x_0 < M$.
\end{itemize}
We write $\lim_{n\to\infty}x_n = +\infty$ and $\lim_{n\to\infty}x_n = -\infty$, respectively.
\end{definition}

\begin{lemma}
Let $(x_n)$ be a sequence in a totally ordered space $X$. Then
\begin{enumerate}
\item if $(x_n)$ is increasing, but not bounded above, it diverges to $+\infty$;
\item if $(x_n)$ is decreasing, but not bounded below, it diverges to $-\infty$.
\end{enumerate}
\end{lemma}

\section{Sequences in complete ordered space}
\subsection{Monotone convergence}
\begin{proposition}[Monotone convergence] \label{sequenceMonotoneConvergence}
Let $(X,\leq)$ be a complete totally ordered space and let $(x_n)$ be a sequence in $X$.
\begin{enumerate}
\item If $(x_n)$ is increasing and bounded above, then it is convergent with limit $\sup_n x_n$.
\item If $(x_n)$ is decreasing and bounded below, then it is convergent with limit $\inf_n x_n$.
\end{enumerate}
\end{proposition}
\begin{proof}
We prove the first point. The second is analogous.

Let $V(\sup_n x_n)$ be an open and $]x,y[\subset V$ such that $\sup_n x_n \in ]x,y[$. Now because $x<\sup_n x_n$ it is not an upper bound of the sequence and there exists an $x_{n_0}> x$. Because the sequence is increasing (and $y$ is a a strict upper bound), all $x_n$ where $n\geq n_0$ are in $]x,y[\subset V$.
\end{proof}

\subsection{Limes superior and inferior}
\begin{definition}
Let $(X,\leq)$ be a complete totally ordered space and let $(x_n)$ be a sequence in $X$. We define
\begin{itemize}
\item the \udef{limes superior} or \udef{limit superior} or \udef{limsup} of $(x_n)$ as
\[ \limsup_{n\to\infty} x_n = \lim_{n\to \infty} \sup\setbuilder{x_m}{m\geq n}; \]
\item the \udef{limes inferior} or \udef{limit inferior} or \udef{liminf} of $(x_n)$ as
\[ \liminf_{n\to\infty} x_n = \lim_{n\to \infty} \inf\setbuilder{x_m}{m\geq n}. \]
\end{itemize}
\end{definition}
The liminf and limsup may not exist.
\begin{lemma}
Let $(X,\leq)$ be a complete totally ordered space and let $(x_n)$ be a sequence in $X$. The limsup and liminf exist \textup{if and only if}  $(x_n)$ is bounded above and below.
\end{lemma}
\begin{proof}
The sequences $\sup\setbuilder{x_m}{m\geq n}$ and $\inf\setbuilder{x_m}{m\geq n}$ are bounded if the limsup and liminf exist and bound $(x_n)$.

The converse follows because the sequences $\sup\setbuilder{x_m}{m\geq n}$ and $\inf\setbuilder{x_m}{m\geq n}$ are monotone.
\end{proof}

\begin{proposition} \label{characterisationLimsupLiminf}
Let $(X,\leq)$ be a complete totally ordered space and let $(x_n)$ be a bounded sequence in $X$. Then 
\begin{enumerate}
\item $L_s = \limsup_{n\to\infty} x_n$ \textup{if and only if}
\begin{align*}
&\forall b > L_s: \exists n_0\in \N:\forall n\geq n_0: x_n < b \qquad \text{and} \\
&\forall a < L_s: \forall n_0\in \N:\exists n\geq n_0: a < x_n
\end{align*}
\item $L_i = \liminf_{n\to\infty} x_n$ \textup{if and only if}
\begin{align*}
&\forall a < L_i: \exists n_0\in \N:\forall n\geq n_0: a < x_n \qquad \text{and} \\
&\forall b > L_i: \forall n_0\in \N:\exists n\geq n_0: x_n < b.
\end{align*}
\end{enumerate}
\end{proposition}

\begin{proposition}
Let $(X,\leq)$ be a complete totally ordered space and let $(x_n)$ be a sequence in $X$. Then $(x_n)$ is convergent \textup{if and only if}
\[ \liminf_{n\to \infty} x_n = \limsup_{n\to \infty} x_n. \]
In this case
\[ \lim_{n\to \infty} x_n = \liminf_{n\to \infty} x_n = \limsup_{n\to \infty} x_n. \]
\end{proposition}
\begin{proof}
Assume $(x_n)$ is a sequence with identical liminf and limsup. Now
\[ \inf\setbuilder{x_m}{m\geq n} \leq x_n \leq \sup\setbuilder{x_m}{m\geq n} \]
so we can apply the squeeze theorem for sequences.

For the converse we use \ref{characterisationLimsupLiminf}.
\end{proof}

\begin{lemma} \label{monotonicityLimsupLiminf}
Let $(a_n)$ and $(b_n)$ be bounded sequences in a totally ordered space such that $a_n\leq b_n$ for all $n\in\N$. Then
\[ \limsup_{n\to \infty}a_n \leq \limsup_{n\to \infty}b_n \quad\text{and}\quad \liminf_{n\to \infty}a_n \leq \liminf_{n\to \infty}b_n. \]
\end{lemma}








\begin{lemma} \label{sequencesSupInf}
There exist sequences converging to supremum and infimum.
\end{lemma}










\chapter{Separation axioms and other properties of convergences spaces}

\section{Distinguishability, separation and regularity}
\subsection{Distinguishable points}
\begin{definition}
Let $\sSet{X,\xi}$ be a convergence space and $x,y\in X$. We call $x$ and $y$ \udef{distinguishable} if ${\lim_\xi}^{-1}(x) \neq {\lim_\xi}^{-1}(y)$.
\end{definition}

In orther words, $x,y \in X$ are distinguishable if there exists a filter $F \in \powerfilters(X)$ such that
\[ \Big(x\in \lim_\xi F \land y\notin \lim_\xi F\Big)\;\lor\; \Big(x\notin \lim_\xi F \land y\in \lim_\xi F\Big). \]

We say $F$ \udef{distinguishes} $x$ and $y$.

\begin{proposition} \label{distinguishabilityPrincipalUltrafilters}
Let $\sSet{X,\xi}$ be a Kent convergence space and $x,y\in X$. Then $x$ and $y$ are indistinguishable \textup{if and only if}
\[ \pfilter{x} \to y \qquad \text{and}\qquad \pfilter{y}\to x. \]
\end{proposition}
\begin{proof}
The direction $\Rightarrow$ is clear: from $\pfilter{x} \to x$, we get $\pfilter{x}\to y$ by indistinguishability.

The direction $\Leftarrow$ is proved by contradiction. Assume $x$ and $y$ are distinguishable, so there exists a filter $F$ such that $F\to x$ but $F\not\to y$. Then $F\cap \pfilter{x} \to y$ by the definining property of Kent spaces. Now $F\cap \pfilter{y} \subseteq F$, so $F\to y$. This is a contradiction.
\end{proof}


\subsection{Separation}
\begin{definition}
Let $\sSet{X,\xi}$ be a convergence space and $A,B\subseteq X$. We say $A$ and $B$ are \udef{separated} if $\adh_\xi(A) \perp B$ and $A\perp \adh_\xi(B)$.

We call two points $x,y\in X$ \udef{separated} if $\{x\}$ and $\{y\}$ are separated.
\end{definition}

\begin{proposition} \label{separatednessPrincipalUltrafilters}
Let $X$ be a set, $\xi$ a convergence on $X$ and $x,y\in X$. Then $x$ and $y$ are separated \textup{if and only if}
\[ \pfilter{x} \not\to y \qquad \text{and}\qquad \pfilter{y}\not\to x. \]
\end{proposition}
\begin{proof}
By \ref{singletonAdherence} we have $\adh_\xi(\{x\}) = \lim_\xi\pfilter{x}$ and $\adh_\xi(\{y\}) = \lim_\xi\pfilter{y}$.
\end{proof}
In other words, $x$ and $y$ are not separated iff $\pfilter{x} \to y$ or $\pfilter{y} \to x$.

\begin{lemma} \label{separatedDistinguishable}
Let $\sSet{X,\xi}$ be a convergence space and $x,y\in X$. If $x,y$ are separated, then they are distinguishable.
\end{lemma}
\begin{proof}
Assume $x,y$ separated. WLOG we can assume $\pfilter{x} \not\to y$. So $\pfilter{x}$ distinguishes $x$ and $y$.
\end{proof}

\subsubsection{Separation by convergent filters}
\begin{definition}
Let $\sSet{X,\xi}$ be a convergence space and $A,B\subseteq X$. We say $A$ and $B$ are \udef{separated by convergent filters} if for all approaches $F:X\to \powerfilters(X)$, the contours satisfy $\neg(F(A)\amesh F(B))$.

We call two points $x,y\in X$ separated by convergent filters if $\{x\}$ and $\{y\}$ are separated by convergent filters.
\end{definition}

\begin{lemma} \label{pointsSeparatedConvergentFilters}
Let $\sSet{X,\xi}$ be a convergence space. For $x,y\in X$ the following are equivalent
\begin{enumerate}
\item $x,y$ are separated by convergent filters;
\item $\lim^{-1}(x)\perp \lim^{-1}(y)$.
\end{enumerate}
\end{lemma}
\begin{proof}
$(1) \Rightarrow (2)$ Assume $\lim^{-1}(x)\amesh \lim^{-1}(y)$, i.e.\ there exists some filter $F\in \lim^{-1}(x)\cap \lim^{-1}(y)$. Then the constant approach $x,y\mapsto F$ has $F(x) = F(y)$, meaning $F(x)\amesh F(y)$ and thus $x,y$ are not separated by convergent filters.

$(2) \Rightarrow (1)$ Assume $\lim^{-1}(x)\perp \lim^{-1}(y)$. Pick an arbitrary approach $F$. If $F(x)\amesh F(y)$, then $F(x)\vee F(y)$ is proper by \ref{joinProperFilter} and in $\lim^{-1}(x)\cap \lim^{-1}(y)$ by monotonicity, which is a contradiction.
\end{proof}


\subsubsection{Separation by vicinities}
\begin{definition}
Let $\sSet{X,\xi}$ be a convergence space and $A,B\subseteq X$. We say $A$ and $B$ are \udef{separated by vicinities} if there exist $U\in \vicinity_\xi(A)$ and $V\in\vicinity_\xi(B)$ such that $U\perp V$.

We call two points $x,y\in X$ separated by vicinities if $\{x\}$ and $\{y\}$ are separated by vicinities, i.e.\ there exist vicinities $U,V$ of $x,y$, resp., such that $U\perp V$.
\end{definition}

\begin{lemma}
Let $\sSet{X,\xi}$ be a convergence space and $A,B\subseteq X$. The following are equivalent:
\begin{enumerate}
\item $A$ and $B$ are separated by vicinities;
\item $\neg(\vicinity_\xi(A)\amesh \vicinity_\xi(B))$.
\end{enumerate}
\end{lemma}

\begin{proposition} \label{disjointVicinitiesConvergentFilterSeparation}
Let $\sSet{X,\xi}$ be a convergence space.
\begin{enumerate}
\item Separation by vicinities implies separation by convergent filters.
\item If $\xi$ is pretopological, the converse also holds.
\end{enumerate}
\end{proposition}
\begin{proof}
$(1)$ Assume $A,B\subseteq X$ are separated by vicinities $U$ and $V$. Assume, towards a contradiction, that $F(A)\amesh F(B)$ for some approach $F$. Then for all $a\in A$, $U\in F(a)$, so $U\in F(A)$. Similarly $V\in F(B)$. This means $U\mesh V$, which is a contradiction.

$(2)$ If $\xi$ is pretopological, then $\vicinity_\xi$ is an approach. Thus $\neg(\vicinity_\xi(A)\amesh \vicinity_\xi(B))$, meaning there exist disjoint vicinities of $A,B$.
\end{proof}
Thus in the pretopological case all approaches $F$ satisfy $\neg(F(A)\mesh F(B))$ iff the vicinity filter satisfies $\neg(\vicinity(A)\mesh \vicinity(B))$.


\begin{proposition} \label{separationByVicinitiesEquivalences}
Let $\sSet{X,\xi}$ be a convergence space.
\begin{enumerate}
\item If $A,B \subseteq X$ are separated by vicinities, then .
\item If $\xi$ is pretopological and $\lim^{-1}(x)\perp \lim^{-1}(y)$ for some $x,y\in X$, then $x,y$ are separated by vicinities.
\end{enumerate}
\end{proposition}
\begin{proof}
$(1)$ Assume $A,B$ are separated by vicinities $U$ and $V$. Assume, towards a contradiction, that $\lim^{-1}(A)\amesh \lim^{-1}(B)$, i.e.\ there exists a filter $F$ that converges to $x\in A$ and $y\in B$. Then $U,V\in F$ and thus $U\cap V = \emptyset \in F$, meaning $F$ is not a proper filter.

$(2)$ Assume $x,y$ are not separated by vicinities. Then $\vicinity_\xi(x)\amesh \vicinity_\xi(y)$, meaning $x$ is an accumulation point of $\vicinity_\xi(y)$. By \ref{subfilterToAccumulationPoint}, there exists a filter $G\geq \vicinity_\xi(y)$ such that $G\to x$. Then $G\in \lim^{-1}(x)\cap \lim^{-1}(y)$.
\end{proof}

\subsubsection{Separation by neighbourhoods}
\begin{definition}
Let $\sSet{X,\xi}$ be a convergence space and $A,B\subseteq X$. We say $A$ and $B$ are \udef{separated by neighbourhoods} if there exist disjoint neighbourhoods of $A$ and $B$.

We call two points $x,y\in X$ separated by neighbourhoods if $\{x\}$ and $\{y\}$ are separated by neighbourhoods.
\end{definition}

\begin{lemma} \label{neighbourhoodSeparationLemma}
Let $\sSet{X,\xi}$ be a convergence space and $A,B\subseteq X$. Then $A,B$ are separated by neighbourhoods \textup{if and only if} there exists an open set $U$ such that $A \subseteq U \subseteq \overline{U} \subseteq B^c$.
\end{lemma}
\begin{proof}
WLOG we may take $\overline{U}^c$ to be the neighbourhood of $B$.
\end{proof}

\subsubsection{Separation by closed vicinities}
\subsubsection{Separation by functions}
\url{https://en.wikipedia.org/wiki/Separated_sets}

\subsection{Regularity}
\begin{definition}
Let $\sSet{X,\xi}$ be a convergence space and $\mathcal{Z} \subseteq \powerset(X)$. The convergence $\xi$ is called \udef{$\mathcal{Z}$-regular} if for all $x\in X, F\in\lim_\xi^{-1}(x)$ there exists a filter base $G\subseteq \mathcal{Z}$ such that
\begin{itemize}
\item $G \preceq F$;
\item $G \overset{\xi}{\longrightarrow} x$.
\end{itemize}
\end{definition}

\section{Separation properties}
\subsection{$T_0$ or Kolmogorov}
\begin{definition}
We call a convergence space $\sSet{X, \xi}$ \udef{Kolmogorov} or \udef{$T_0$} if every pair of distinct points in $X$ is distinguishable.
\end{definition}

\subsection{$R_0$ or symmetric}
\begin{definition}
Let $X$ be a set and $\xi$ a convergence on $X$. Then $\xi$ is called \udef{symmetric} or \udef{$R_0$} if all distinguishable pairs of points are separated.
\end{definition}

\begin{proposition} \label{R0conditions}
Let $X$ be a set and $\xi$ a convergence on $X$. Then the following are equivalent:
\begin{enumerate}
\item $\xi$ is an $R_0$ convergence;
\item $x$ and $y$ are indistinguishable \textup{if and only if} they are separated;
\item for all $x,y\in X$, $x$ and $y$ are indistinguishable \textup{if and only if} $\pfilter{x} \to y$;
\item $\adh_\xi(\{x\})$ is the set of points that are indistinguishable from $x$ for all $x\in X$;
\end{enumerate}
The following are consequences of the above. If $\xi$ is a Kent convergence, then they are also equivalent:
\begin{enumerate} \setcounter{enumi}{4}
\item the set $\setbuilder{\adh_\xi(\{x\})}{x\in X}$ is a partition of $X$;
\item for all $x,y\in X$:  $\pfilter{x} \to y$ \textup{if and only if} $\pfilter{y} \to x$;
\end{enumerate}
\end{proposition}
\begin{proof}
$(1) \Rightarrow (2)$ The converse to the $R_0$ condition, that separated points are distinguishable, is automatic (see \ref{separatedDistinguishable}).

$(2) \Rightarrow (3)$ All points that are indistinguishable from $x$ are in $\adh_\xi(\{x\}) = \lim_\xi\{x\}$ by construction. Assume $y$ is distinguishable from $x$. Then $y$ is separated from $x$, so $\pfilter{x}\not\to x$ by \ref{separatednessPrincipalUltrafilters}.

$(3) \Rightarrow (4)$ $\adh_\xi(\{x\}) = \lim_\xi\pfilter{x}$ by \ref{singletonAdherence}.

$(4) \Rightarrow (1)$ Assume $x,y\in X$ are distinguishable. Then $y\notin\adh_\xi(\{x\})$ and $x\notin\adh_\xi(\{y\})$. This means $\pfilter{x} \not\to y$ and $\pfilter{y} \not\to x$ and we conclude by \ref{separatednessPrincipalUltrafilters}.


$(4) \Rightarrow (5)$ Indistinguishability is an equivalence relation.

$(5) \Rightarrow (6)$ Equivalence relations are symmetric.

$(6) \Rightarrow (1)$ Assume $\xi$ of a Kent convergence and $x,y$ distinguishable. Then $\pfilter{x}\not\to y$ or $\pfilter{y}\not\to x$ by \ref{distinguishabilityPrincipalUltrafilters}. Because of (6), the ``or'' becomes an ``and'' and we conclude with \ref{separatednessPrincipalUltrafilters}.
\end{proof}

TODO: characterisation  ``every open set is a union of closed sets'' or
``every closed set is an intersection of open sets''.

?? $\adh_\xi(\{x\})$ is closed for all $x\in X$ ??

\subsection{$T_1$ or Fréchet}
\begin{definition}
Let $X$ be a set and $\xi$ a preconvergence on $X$. Then $\xi$ is called \udef{Fréchet} or \udef{$T_1$} if all distinct points in $X$ are separated.
\end{definition}
\url{https://en.wikipedia.org/wiki/T1_space}

\begin{proposition} \label{FrechetCharacterisation}
Let $X$ be a set and $\xi$ a convergence on $X$. Then the following are equivalent:
\begin{enumerate}
\item $\xi$ is a $T_1$ convergence;
\item $\xi$ is both $T_0$ and $R_0$;
\item all singletons are closed, i.e.\ $\forall x\in X: \; \adh_\xi(\{x\}) = \{x\}$;
\item all finite sets are closed;
\item $\forall x\in X: \; \lim_\xi\pfilter{x} = \{x\}$;
\item $\forall x\in X: \; \lim_\xi \pfilter{x} \subseteq \{x\}$;
\end{enumerate}
\end{proposition}
\begin{proof}
$(1) \Leftrightarrow (2)$ Definitions together with point (2) of \ref{R0conditions}.

$(1) \Leftrightarrow (3)$ From point (4) of \ref{R0conditions}.

$(3) \Leftrightarrow (4)$ From \ref{propertiesTopology}.

$(3) \Leftrightarrow (5)$ From \ref{singletonAdherence}.

$(5) \Leftrightarrow (6)$ Convergences are centered.
\end{proof}
TODO: every singleton in $X$ is closed;  every finite subset of $X$ is closed.
\begin{corollary} \label{finiteConvergenceDiscrete}
Let $X$ be a finite set, then the only $T_1$ convergence on $X$ is the discrete convergence $\iota_X$.
\end{corollary}
\begin{proof}
Let $\xi$ be a $T_1$ convergence on $X$ and $F$ a proper filter in $\powerset(X)$. Now $F$ is principal (TODO ref), so $F = \upset \ker F$. If $\ker F$ is a singleton, then $\lim F = \ker F$ by point (2). Otherwise $\lim F = \emptyset$ by point (4). This is discrete convergence.
\end{proof}

\begin{lemma}
If $\sSet{X,\xi}$ is a $T_1$ convergence on $X$, then $\xi$ is also $T_0$.
\end{lemma}
\begin{proof}
The filters $\pfilter{x}$ and $\pfilter{y}$ distinguish $x,y\in X$.
\end{proof}

\begin{proposition}
Let $\xi$ be a $T_1$ preconvergence on a set $X$ and $F$ a filter in $\powerfilters(X)$. If $x\in \lim F$, then $\ker F \subseteq \{x\}$.
\end{proposition}
\begin{proof}
Suppose $\ker F$ in non-empty and take $y \in \ker F$. Then $F \subseteq \pfilter{y}$ and so
\[ x \in \lim F \subseteq \lim \pfilter{y} \subseteq \{y\}. \]
This means that $y = x$.
\end{proof}
\begin{corollary}
The kernel of a convergent proper filter in a $T_1$ space is empty or a singleton.
\end{corollary}
By constrast, in $T_2$ spaces the \emph{limit} (not kernel) of a convergent proper filter is empty or a singleton.

\begin{proposition} \label{setKernelVicinityFilter}
Let $\sSet{X,\xi}$ be a $T_1$ pretopological convergence space and $A \subseteq X$ a subset. Then $\bigcap\vicinity(A) = A$.
\end{proposition}
\begin{proof}
We have $A\subseteq \bigcap\vicinity(A)$ from \ref{vicinityOfSetCorollary}. For the converse, take $x\notin A$. By $T_1$, we have $\adh\{x\} \perp A$, which implies $\forall y\in A:\; y\notin\adh\{x\}$. By \ref{principalInherenceComplementAdherence} and \ref{principalAdherenceInherence}, we have
\[ y\notin\adh\{x\} \iff y\in \inh\big(\{x\}^c\big) \iff \{x\}^c\in \vicinity(y). \]
This implies that $\{x\}^c \in \bigcap_{y\in A}\vicinity(y) = \vicinity(A)$ and thus $x\notin \bigcap\vicinity(A)$.
\end{proof}
\begin{corollary}
In a topological $T_1$ convergence space, every set is an intersection of open sets.
\end{corollary}
Note that this corollary can also simply be proved by writing
\[ A = \bigcap_{x\in A^c}\{x\}^c. \]

TODO: big question marks:
\begin{proposition}
Let $X$ be a set and $\xi$ a convergence on $X$. If $\xi$ is $T_1$, then $\forall x\neq y\in X: \exists F,G\in \powerfilters(X)$ such that
\[ x\in \lim_\xi F \;\land\; y\in \lim_\xi G \;\land\; x\notin \lim_\xi G \;\land\; y\notin \lim_\xi F. \]
If $\xi$ is topological, the converse also holds.
\end{proposition}
\begin{proof}
$\Rightarrow$ We may take $F = \pfilter{x}$ and $G = \pfilter{y}$.

$\Leftarrow$ If $\xi$ is topological, we may take $F = \vicinity_\xi(x)$ and $G = \vicinity_\xi(y)$. It is enough to show that $y\notin \adh_\xi(\{x\})$.

Assume, towards a contradiction, that $y\in \adh_\xi(\{x\})$. Then $\{x\} \in \vicinity_\xi(y)^{\mesh}$
\end{proof}

\subsection{$R_1$ or reciprocal}
\begin{definition}
Let $\sSet{X,\xi}$ be a convergence space. Then $\xi$ is called \udef{$R_1$}, \udef{reciprocal} or \udef{preregular} if all distinguishable points are separated by convergent filters.
\end{definition}
TODO review definition.

\url{https://gdz.sub.uni-goettingen.de/id/PPN235181684_0187?tify={%22pages%22:[191],%22panX%22:0.893,%22panY%22:0.579,%22view%22:%22info%22,%22zoom%22:0.894}}


\begin{proposition} \label{R1Conditions}
Let $X$ be a set and $\xi$ a convergence on $X$. Then the following are equivalent:
\begin{enumerate}
\item $\xi$ is an $R_1$ convergence;
\item if $x$ and $y$ are distinguishable, then $\lim_\xi^{-1}(x)\perp \lim_\xi^{-1}(y)$;
\item $x$ and $y$ are distinguishable \textup{if and only if} $\lim_\xi^{-1}(x)\perp \lim_\xi^{-1}(y)$;
\item for all $x,y\in X$, either $\lim_\xi^{-1}(x) = \lim_\xi^{-1}(y)$ or $\lim_\xi^{-1}(x)\perp \lim_\xi^{-1}(y)$;
\item for all $x,y\in X$: $\lim_\xi^{-1}(x)\mesh \lim_\xi^{-1}(y)$ implies $\lim_\xi^{-1}(x) = \lim_\xi^{-1}(y)$;
\item the set $\setbuilder{\lim^{-1}_\xi(x)}{x\in X}$ is a partition of the set on convergent filters on $X$;
\item if there exists a filter $F$ such that $F \to x$ and $F \to y$, then $x$ and $y$ are indistinguishable.
\end{enumerate}
\end{proposition}
\begin{proof}
$(1) \Leftrightarrow (2)$ By \ref{pointsSeparatedConvergentFilters}.

$(2) \Rightarrow (3)$ If $\lim_\xi^{-1}(x)\perp \lim_\xi^{-1}(y)$, then $x$ and $y$ are definitely distinguishable, e.g.\ by $\pfilter{x}$.

$(3) \Rightarrow (4) \Rightarrow (5) \Rightarrow (6)$ Clear.

$(6) \Rightarrow (7)$ From $F\to x$ and $F\to y$, we get $F\in \lim_\xi^{-1}(x)$ and $F\in \lim_\xi^{-1}(y)$, so $\lim_\xi^{-1}(x)\amesh \lim_\xi^{-1}(y)$. Using (3) this implies $\lim_\xi^{-1}(x) = \lim_\xi^{-1}(y)$.

$(7) \Rightarrow (2)$ By contraposition.
\end{proof}
\begin{corollary}
Any $R_1$ convergence space is also $R_0$.
\end{corollary}
\begin{proof}
Compare point (2) with point (3) of \ref{R0conditions} and note that $\pfilter{x}\to y$ implies $\lim_\xi^{-1}(x)\amesh \lim_\xi^{-1}(y)$. 
\end{proof}


\subsection{$T_2$ or Hausdorff}
\begin{definition}
Let $X$ be a set and $\xi$ a preconvergence on $X$. Then $\xi$ is called \udef{Hausdorff} or \udef{$T_2$} if every proper $\xi$-limit contains at most one point.
\end{definition}
By ``proper $\xi$-limit'' we mean we exclude from this condition the degenerate filter $\powerset(X)$. Otherwise there would be no $T_2$ convergences by \ref{limitDegenerateFilter}.

\begin{proposition}
Let $X$ be a set and $\xi$ a convergence on $X$. Then the following are equivalent:
\begin{enumerate}
\item $\xi$ is a $T_2$ convergence;
\item if $x \neq y$, then ${\lim_\xi}^{-1}(x)\perp {\lim_\xi}^{-1}(y)$;
\item $\xi$ is $T_0$ and $R_1$;
\item $\xi$ is $T_1$ and $R_1$;
\end{enumerate}
\end{proposition}
\begin{proof}
$(1) \Leftrightarrow (2)$ If $F \in \lim_\xi^{-1}(x)\cap \lim_\xi^{-1}(y)$, then $F \to x$ and $F\to y$, so $\xi$ would not be $T_2$.

$(2) \Leftrightarrow (3)$ Clear.

$(3) \Leftrightarrow (4)$ $R_1$ implies $R_0$ and $R_0+T_0$ is equivalent with $T_1$.
\end{proof}

\begin{proposition} \label{pretopologicalHausdorff}
Let $\sSet{X,\xi}$ be a pretopological convergence space. Then the following are equivalent:
\begin{enumerate}
\item $X$ is Hausdorff;
\item $\neg\big(\vicinity(x) \amesh \vicinity(y)\big)$ for all $x\neq y$ in $X$;
\item there exist disjoint $A\in \vicinity(x)$ and $B\in \vicinity(y)$ for all $x\neq y$ in $X$.
\end{enumerate}
\end{proposition}
\begin{proof}
$(1) \Leftrightarrow (2)$ We have $F\in {\lim_\xi}^{-1}(x)\cap {\lim_\xi}^{-1}(y)$ iff $F\in \upset \{\vicinity(x)\} \cap \upset \{\vicinity(y)\}$ iff $F \supseteq \vicinity(x)\vee \vicinity(y)$. We can find such a proper filter $F$ iff $\vicinity(x) \amesh \vicinity(y)$, by \ref{joinProperFilter}.

$(2) \Leftrightarrow (3)$ Immediate.
\end{proof}

\begin{proposition}
Every $T_2$ convergence is also $T_1$. If the space is finite, the converse also holds.
\end{proposition}
\begin{proof}
Let $\sSet{X, \xi}$ be a $T_2$ convergence space and $x\in X$. By $T_2$, $\lim_\xi\pfilter{x}$ is a singleton. Definition of convergence this singleton is $\{x\}$.

Now let $X$ be a finite set and let $F$ be a proper filter in $\powerfilters(X)$. Then $F$ is principal by \ref{finiteFiltersPrincipal} and not free because it is proper. So we can take $x\in \ker F$ and $F \subseteq \pfilter{x}$. Thus $\lim F \subseteq \lim \pfilter{x} \subseteq \{x\}$, meaning the convergence is $T_2$.
\end{proof}
\begin{corollary}
Let $X$ be a finite set, then the only $T_2$ convergence on $X$ is the discrete convergence $\iota_X$.
\end{corollary}
\begin{proof}
By \ref{finiteConvergenceDiscrete}.
\end{proof}

\begin{lemma} \label{T2initialConvergence}
Let $Y$ be a set and $\{f_i: Y\to \sSet{Z_i, \zeta_i}\}_{i\in I}$ a set of function to convergence spaces. If, for all $x \neq y\in Y$, there exist $i\in I$ such that $f_i(x) \neq f_i(y)$ and $Z_i$ is $T_2$, then the initial convergence on $Y$ w.r.t.\ this set is $T_2$.
\end{lemma}
In particular this holds if $f_i$ is an injective function to a Hausdorff space for at least one $i\in I$.
\begin{proof}
Take $x\neq y\in Y$ and suppose, towards a contradiction, that $F\overset{Y}{\longrightarrow} x,y$. Take $i\in I$ such that $f_i(x) \neq f_i(y)$ and $Z_i$ is $T_2$. Then, by continuity, $\upset f_i^{\imf\imf} \overset{\zeta_i}{\longrightarrow} f_i(x), f_i(y)$, which contradicts the fact that $Z_i$ is $T_2$.
\end{proof}
\begin{corollary} \label{HausdorffSubspace}
Any subspace of a Hausdorff space is Hausdorff.
\end{corollary}


\subsection{$R_2$ or regular}
\begin{definition}
Let $\sSet{X,\xi}$ be a convergence space. Then $\xi$ is called \udef{regular} or \udef{$R_2$} if it is based in $\adh_\xi^{\imf}(\powerset^2(X))$.
\end{definition}

\url{https://en.wikipedia.org/wiki/Regular_space}

\begin{proposition}
Let $\sSet{X,\xi}$ be a convergence space. Then the following are equivalent:
\begin{enumerate}
\item $\xi$ is an $R_2$ convergence;
\item for all $F\in\powerfilters(X)$, $F\overset{\xi}{\longrightarrow} x$ implies $\adh_\xi[F] = \setbuilder{\adh_\xi(A)}{A\in F} \overset{\xi}{\longrightarrow} x$.
\item for all $F\in\powerfilters(X)$: $F\overset{\xi}{\longrightarrow} x$ \textup{if and only if} $\adh_\xi[F] = \setbuilder{\adh_\xi(A)}{A\in F} \overset{\xi}{\longrightarrow} x$.
\end{enumerate}
\end{proposition}
\begin{proof}
$(1) \Leftrightarrow (2)$ Assume (1), then there exists a filter $G$ based in $\setbuilder{\adh_\xi(A)}{A\in\powerset(X)}$ such that $G\to x$ and $G\subseteq F$. We just need to show that $G\subseteq \adh[F]$. Indeed take $A\in G$, then $A = \adh(B)$ for some $B\subseteq X$. Now $B\in F$ TODO: is this wrong?

Because $\adh_\xi[F]\preceq F$ Then $\adh_\xi(A) \supseteq A$, so $\adh$ 

$(2) \Leftrightarrow (3)$ One direction is given by definition. The other follows by monotonicity because $\adh_\xi[F] \preceq F$.
\end{proof}

\begin{lemma}
Any $R_2$ convergence is also $R_1$.
\end{lemma}

\begin{proposition} \label{regularityBySeparation}
Let $\xi$ be a convergence.
\begin{itemize}
\item If $\xi$ is regular, then $\xi$ separates points from the complements of their vicinities by convergent filters.
\item If $\xi$ is pretopological, then the converse also holds.
\end{itemize}
\end{proposition}
\begin{proof}
(1) Fix an arbitrary approach $F: X\to \powerfilters(X)$ and $V_x\in \vicinity(x)$ for all $x\in X$.

By regularity we have $\vicinity(x)\subseteq \upset\adh[F(x)]$ for all $x\in X$.
This means
\begin{align*}
& \forall x\in X: \forall V\in \vicinity(x): \exists U \in F(x): \quad \adh_\xi(U) \subseteq V \\
\iff& \forall x\in X: \forall V\in \vicinity(x): \exists U \in F(x): \quad (\adh_\xi(U) \perp V^c) \\
\iff& \forall x\in X: \forall V\in \vicinity(x): \exists U \in F(x): \quad \neg(\adh_\xi(U) \mesh V^c) \\
\iff& \forall x\in X: \forall V\in \vicinity(x): \exists U \in F(x): \quad \neg(\{U\} \amesh \vicinity(V^c)) \\
\iff& \forall x\in X: \forall V\in \vicinity(x): \quad \neg(\forall U \in F(x):\{U\} \amesh \vicinity(V^c)) \\
\iff& \forall x\in X: \forall V\in \vicinity(x): \quad \neg(F(x) \amesh \vicinity(V^c)) \\
\implies& \forall x\in X: \forall V\in \vicinity(x): \quad \neg(F(x) \amesh F(V^c)).
\end{align*}
We have used \ref{setAdherenceInherence}.

(2) In the pretopological case, we can take $F = \vicinity$ and we can run the argument in reverse, because regularity is implied by $\vicinity(x)\subseteq \upset\adh[\vicinity(x)]$ for all $x\in X$.
\end{proof}

\begin{proposition} \label{topologicalRegularity}
Let $\sSet{X,\xi}$ be a topological space. Then the following are equivalent:
\begin{enumerate}
\item $\xi$ is regular;
\item for any $x\in X$ and any base $\mathcal{B}$ of $\neighbourhood_\xi(x)$, $\closure^\imf(\mathcal{B})$ is also a base of $\neighbourhood_\xi(x)$;
\item for any closed set $C$ and $x\in C^c$ there exist disjoint open sets $U,V$ such that $x\in U$ and $C\subseteq V$;
\item for any open set $O\subseteq X$ and $x\in O$ there exists an open set $U$ such that $x\subseteq U\subseteq \overline{U} \subseteq O$.
\end{enumerate}
\end{proposition}
\begin{proof}
$(1) \Leftrightarrow (2)$ TODO ref.

$(1) \Leftrightarrow (3)$ By \ref{regularityBySeparation}.

$(2) \Leftrightarrow (4)$ By \ref{neighbourhoodSeparationLemma}.
\end{proof}

\begin{proposition} \label{regularityInitialConvergence}
Let $Y$ be a set, $\{f_i: Y\to \sSet{Z_i, \zeta_i}\}_{i\in I}$ a set of functions to \emph{regular} convergence spaces and $\mu$ the initial preconvergence on $Y$ w.r.t. this set. Then $\sSet{Y,\mu}$ is also regular.
\end{proposition}
\begin{proof}
Take $F\in \powerfilters(Y)$ such that $F\overset{\mu}{\longrightarrow} y$. Then, for all $i\in I$, we have $\upset\adh_{\zeta_i}^\imf\big(\upset f_i^{\imf\imf}(F)\big)\subseteq \upset f_i^{\imf\imf}\big(\adh_{\mu}^{\imf}(F)\big)$ by \ref{adherenceInherenceContinuity}.

Now $\upset\adh_{\zeta_i}^\imf\big(\upset f_i^{\imf\imf}(F)\big)$ converges by continuity of $f_i$ and regularity of $\zeta_i$, so $\upset f_i^{\imf\imf}\big(\adh_{\mu}^{\imf}(F)\big)$ converges by monotonicity and thus $\adh_{\mu}^{\imf}(F)$ converges by \ref{initialFinalConvergence}.
\end{proof}

\subsubsection{$\omega$-regular}
\begin{definition}
Let $\sSet{X,\xi}$ be a convergence space. Consider the set of continuous real-valued functions, $\cont(X)$. Let $\omega_\xi$ be the initial convergence on $X$ w.r.t. $\cont(X)$.

Then $\xi$ is called \udef{$\omega$-regular} if it is based in $\adh_{\omega_\xi}^{\imf}(\powerset^2(X))$.
\end{definition}
TODO: $\mu$-regular similar, except for set of real-valued Cauchy functions.

\begin{proposition}
Let $\sSet{X,\xi}$ be a pretopological convergence space. Then the following are equivalent:
\begin{enumerate}
\item $\xi$ is $\omega$-regular;
\item $\xi = \omega_\xi$;
\item $\forall A\subseteq X: \forall x\in \adh_\xi(A)^c: \exists f\in \cont(X): \; f^{\imf}(A) = \{0\} \land f(x) = 1$.
\end{enumerate}
\end{proposition}
\begin{proof}

\end{proof}


\subsubsection{completely regular}
\begin{definition}
A convergence space $\sSet{X,\xi}$ is called completely regular if $\pseudotopMod(\xi) = \omega_\xi$.
\end{definition}

\begin{proposition}
Let $\sSet{X,\xi}$ be a convergence space. Then the following are equivalent:
\begin{enumerate}
\item $\xi$ is $\omega$-regular;
\item $\pretopMod(\xi) = \omega_\xi$.
\item if $A,B\subseteq X$ are $\xi$-separated, then $A,B$ are $\omega_\xi$-separated;
\item $A,B\subseteq X$ are $\xi$-separated \textup{if and only if} $A,B$ are $\omega_\xi$-separated;
\item 
\end{enumerate}
\end{proposition}

\subsection{$T_3$ or regular Hausdorff}
\begin{definition}
Let $\sSet{X,\xi}$ be a convergence space. Then $\xi$ is called \udef{$T_3$} if it is regular and Hausdorff.
\end{definition}

\begin{proposition}
Let $X$ be a set and $\xi$ a convergence on $X$. Then the following are equivalent:
\begin{enumerate}
\item $\xi$ is a $T_3$ convergence, i.e.\ $R_2$ and $T_2$;
\item $\xi$ is $R_2$ and $T_0$.
\end{enumerate}
\end{proposition}

\begin{lemma} \label{compactHausdorffSpacesRegular}
Let $\sSet{X,\xi}$ be a compact Hausdorff topological space. Then $X$ is regular.
\end{lemma}
This is really a lemma for the fact that compact Hausdorff spaces are normal \ref{compactHausdorffSpacesNormal}.
\begin{proof}
We use \ref{topologicalRegularity}. Take a closed subset $C\subseteq X$ and $x\in C^c$. For all $y\in C$, there exist disjoint open neighbourhoods $U_y \in \neighbourhood(x)$ and $V_y\in \neighbourhood(y)$ by \ref{pretopologicalHausdorff}.

Now $\{V_y\}_{y\in C}$ is an open cover of $C$, which is compact by \ref{compactClosedSets}. So there is a finite subcover $\{V_y\}_{y\in F}$ by \ref{topologyCompactnessOpenCover}. Now set $V \defeq \bigcup_{y\in F}V_y$ and $U \defeq \bigcap_{y\in F}U_y$. These sets are both open by \ref{propertiesTopology}. By construction $C\subseteq V$, $x\in U$ and $U\perp V$.
\end{proof}

\subsection{$R_3$ or normal}
\begin{definition}
Let $\sSet{X,\xi}$ be a convergence space. Then $\xi$ is called \udef{normal} or \udef{$R_3$} if all disjoint closed sets are separated by convergent filters.
\end{definition}

\begin{proposition}
Let $\sSet{X,\xi}$ be a pretopological convergence space and $A,B\subseteq X$. Then $\lim^{-1}(\adh_\xi(A)) \perp \lim^{-1}(\adh_\xi(B))$ \textup{if and only if} there exist disjoint vicinities of $\adh_\xi(A)$ and $\adh_\xi(B)$.
\end{proposition}


\begin{proposition} \label{topologicalNormal}
Let $\sSet{X,\xi}$ be a topological convergence space. Then $X$ is normal \textup{if and only if} for all disjoint closed sets $C_1, C_2$ there exist disjoint open sets $O_1,O_2$ such that $C_1\subseteq O_1$ and $C_2\subseteq O_2$.
\end{proposition}

\begin{proposition}
Any $R_3$ convergence is also $R_2$.
\end{proposition}


\subsection{$T_4$ or normal Hausdorff}
\begin{definition}
Let $\sSet{X,\xi}$ be a convergence space. Then $\xi$ is called \udef{$T_4$} if it is normal and Hausdorff.
\end{definition}

\begin{proposition}
Let $X$ be a set and $\xi$ a convergence on $X$. Then the following are equivalent:
\begin{enumerate}
\item $\xi$ is a $T_4$ convergence, i.e.\ $R_3$ and $T_2$;
\item $\xi$ is $R_3$ and $T_4$;
\item $\xi$ is $R_3$ and $T_0$.
\end{enumerate}
\end{proposition}


\begin{proposition} \label{compactHausdorffSpacesNormal}
Let $\sSet{X,\xi}$ be a compact Hausdorff topological space. Then $X$ is normal.
\end{proposition}
\begin{proof}
We use \ref{topologicalNormal}. Take closed disjoint sets $C_1,C_2$. For all $x\in C_1$, there exist disjoint open sets $U_x, V_x$ such that $x\in U_x$ and $C_2\subseteq V_x$, by regularity \ref{compactHausdorffSpacesRegular}.

Now $\{U_x\}_{x\in C_1}$ is an open cover of $C_1$, which is compact by \ref{compactClosedSets}. So there is a finite subcover $\{U_x\}_{x\in F}$ by \ref{topologyCompactnessOpenCover}. Now set $O_1 \defeq \bigcup_{x\in F} U_x$ and $O_2 \defeq \bigcap_{x\in F} V_x$. These sets are both open by \ref{propertiesTopology}. By construction $C_1\subseteq O_1$, $C_2\subseteq O_2$ and $O_1\perp O_2$.
\end{proof}

\section{Comparison with reals and metrisability}
\subsection{Functional convergence properties}
\subsubsection{Functional closure}
\begin{definition}
Let $\sSet{X,\xi}$ be a convergence space and $A\subseteq X$. We call $A$ \udef{functionally closed} if there exists a continuous function $f: X\to \R$ and a closed set $C\subseteq \R$ such that $A = f^{-1}[C]$.
\end{definition}

\begin{proposition}
Every functionally closed set if closed. In a metric space the converse holds.
\end{proposition}
\begin{proof}
TODO + see \ref{distanceToSetContinuous}
\end{proof}

\begin{lemma} \label{functionallyClosedZeroSet}
Let $\sSet{X,\xi}$ be a convergence space and $A\subseteq X$. Then $A$ is functionally closed \textup{if and only if} there exists a continuous function $g: X\to \R$ such that $A = g^{-1}[\{0\}]$.
\end{lemma}
This means we may take $C = \{0\}$ in the definition of functionally closed.
\begin{proof}
If there exists such a function $g$, then $A$ is clearly functionally closed. For the converse, fix a continuous function $f: X\to \R$ and a closed set $C\subseteq \R$ such that $A = f^{-1}[C]$. Then we use the continuity of $d_C: \R\to\R: x\mapsto \inf_{c\in C}d(x,c)$ (see \ref{distanceToSetContinuous}) and set $g = d_C\circ f$.
\end{proof}

\subsubsection{Functional separation}
\begin{definition}
Let $\sSet{X,\xi}$ be a convergence space and $A,B\subseteq X$. Then $A$ and $B$ are \udef{functionally separated} if there exists a continuous function $f: X\to [0,1]\subseteq \R$ with $f[A] = \{0\}$ and $f[B] = \{1\}$.
\end{definition}

\begin{proposition}
Two sets are functionally separated \textup{if and only if} they are included in disjoint functionally closed sets.
\end{proposition}
\begin{proof}
First assume $A,B$ are functionally separated by $f$. Then $f^{-1}[\;\{0\}\;]$ and $f^{-1}[\;\{1\}\;]$ are disjoint functionally closed sets containing $A$, resp. $B$.

Conversely, assume $A\subseteq f^{-1}[\{0\}]$ and $B\subseteq g^{-1}[\{0\}]$ (we can take $C = \{0\}$ by \ref{functionallyClosedZeroSet}) with $f^{-1}[\{0\}]\perp g^{-1}[\{0\}]$. Then
\[ h: X\to \R: x\mapsto \frac{|f(x)|}{|f(x)|+|g(x)|} \]
is well-defined everywhere because $f^{-1}[\{0\}]\perp g^{-1}[\{0\}]$ and functionally separates $A$ and $B$.
\end{proof}

\begin{lemma} \label{UrysohnsLemmaLemma}
Let $\sSet{X,\xi}$ be a topological space and $D\subseteq \R$ a dense subset of the reals. Suppose $\sSet{U_d}{d\in D}$ is a set of open subsets of $X$ satisfying
\begin{enumerate}
\item if $d < t$, then $\overline{U_d}\subseteq U_t$;
\item $\bigcup_{d\in D}U_d = X$;
\item $\bigcap_{d\in D}U_d = \emptyset$.
\end{enumerate}
Then $f: X\to \R: x\mapsto \inf\setbuilder{d\in D}{x\in U_d}$ is continuous.
\end{lemma}
\begin{proof}
The function $f$ is well-defined, because $\setbuilder{d\in D}{x\in U_d}$ is never empty.

We will prove continuity at $x\in X$ using \ref{pretopologicalContinuityVicinities}. To that end, take some $U\in \vicinity_\R(f(x))$. By density, we can find some $d,t\in D$ such that $[d,t]\subseteq U$ and $d < f(x) < t$.

Now set $V = U_t\setminus\overline{U_d}$. It is then enough to prove that $V\in \vicinity_\xi(x)$ and $f[V]\subseteq [d,t]$. For the first point, note that $V = U_t\cap\overline{U_d}^c$, so $V$ is open.

Next we show $x\in V$. Indeed, from $f(x) < t$, we get that there exists $f(x) \leq s < t$ such that $x\in U_s$. Thus $x\in U_t$. From $d < f(x)$, we get that there exists $s>d$ such that $x\notin U_s$. By assumption ($d < t \implies \overline{U_d}\subseteq U_t$) this means that $x\notin \overline{U_d}$.

Finally note that if $y\in V \subseteq U_t$, then $f(y) \leq t$ and if $y\notin \overline{U_d}$, then $f(y) > s$ for all $s< d$. Thus $d \leq f(y)$ and $f[V]\subseteq [d,t]$.
\end{proof}

\begin{theorem}[Urysohn's lemma] \label{UrysohnLemma}
Let $\sSet{X,\xi}$ be a normal topological space. If $C_1,C_2$ are disjoint closed sets, then they are functionally separated.
\end{theorem}
\begin{proof}
We will construct a family $\setbuilder{U_r}{r\in Q}$ of open sets satisfying the conditions of \ref{UrysohnsLemmaLemma}.

First set $U_r = \emptyset$ for all $r<0$ and $U_r = X$ for all $r>1$.

Now set $U_1 = C_2^c$. By normality (TODO ref) there exists an open set $U_0$ such that $C_1 \subseteq U_0 \subseteq \overline{U_0} \subseteq U_1$.

We can enumerate the rationals in a sequence $\seq{r_n}_{n\in \N}$ with $r_0 = 0$ and $r_1 = 1$. We now recursively define $U_{r_n}$ with recursion invariant $d < t \implies \overline{U_d}\subseteq U_t$ as follows: Set
\[ V =  U_i \qquad \text{where $i = \max_{\substack{k < n \\ r_k < r_n}}k$} \qquad\text{and}\qquad W = U_j \qquad \text{where $j = \min_{\substack{k < n \\ r_k > r_n}}k$}. \]
By the recursion invariant, we have $\overline{U_i}\subseteq U_j$, so by normality we can define a $U_{r_k}$ such that $\overline{U_i} \subseteq U_{r_k} \subseteq \overline{U_{r_k}} \subseteq U_j$. This assignment satisfies the recursion invariant.

Then by \ref{UrysohnsLemmaLemma} the function $f: X\to \R: x\mapsto \inf\setbuilder{r\in \Q}{x\in U_r}$ is continuous. It functionally separates $C_1$ and $C_2$.
\end{proof}

\begin{corollary}[Urysohn's metrisation theorem] \label{UrysohnMetrisationTheorem}
Every regular and second countable topological space is pseudometrisable.
\end{corollary}
\begin{proof}
Let $\sSet{X,\xi}$ be such a topological space. By \ref{countableRegularityImpliesNormality}, $\xi$ is automatically normal. TODO embedding into $[0,1]^F$.
\end{proof}
TODO converse??


\subsubsection{Functional regularity}
\begin{definition}
Let $\sSet{X, \xi}$ and $\sSet{Y,\zeta}$ be convergence spaces. We call $\xi$ \udef{$\zeta$-functionally regular} if $\xi$ is the initial convergence on $X$ w.r.t. $\cont(\xi,\zeta)$.

In particular we call $\xi$ \udef{completely regular} if it is $\R$-functionally regular.
\end{definition}

\subsubsection{Tietze extension theorem}
TODO!

\url{https://math.stackexchange.com/questions/47360/urysohns-function-on-a-metric-space#comment276726_47360}

\begin{theorem}[Tietze-Urysohn-Brouwer extension theorem] \label{TietzeExtension}
Let $\sSet{X,\xi}$ be a normal space, $A\subseteq X$ a closed subset and $f:A\to \R$ a continuous function. Then there exists a continuous extension $\hat{f}: X\to \R$ of $f$ such that
\[ \sup_{x\in A}|f(x)| = \sup_{x\in X}|\hat{f}(x)|. \]
\end{theorem}
\begin{proof}
TODO
\end{proof}



\chapter{Function spaces}
\section{Function spaces}
\subsection{Pointwise convergence}
\begin{definition}
Let $X$ be a set and $\sSet{Y,\zeta}$ a convergence space. The \udef{pointwise convergence} on $(X\to Y)$ is the initial convergence w.r.t. the evaluation maps $\{\evalMap_x\}_{x\in X}$.

The set $(X\to Y)$ equipped with the pointwise convergence is denoted $(X\to Y)_p$.
\end{definition}
Thus a filter of functions converges in the pointwise convergence if and only if it converges at each point (see \ref{initialFinalConvergence}).

\begin{proposition} \label{pointwiseConvergenceProductSpace}
Let $X$ be a set and $\sSet{Y,\zeta}$ a convergence space then
\[ (X\to Y)_p \cong \prod_{x\in X}Y, \]
with the homeomorphism given by
\[ \prod_{x\in X}Y \to (X\to Y)_p: f\mapsto \setbuilder{\big(x,p_x(f)\big)}{x\in X}. \]
\end{proposition}
\begin{proof}
The map $(X\to Y)_p \to \prod_{x\in X}Y$ is exactly the one constructed in \ref{initialConvergenceWrtOneFunction}. The initial convergence w.r.t. a bijection makes the bijection a homeomorphism, \ref{initialBijectionHomeomorphism}.
\end{proof}


\begin{lemma} \label{interchangeLimitsWithContinuousLimitFunction}
Let $\sSet{X,\xi}$ and $\sSet{Y,\zeta}$ be convergence spaces. Let $\seq{x_n}_{n\in \N}$ be a sequence that converges to $x\in X$ and $\seq{f_k}_{k\in \N}$ a sequence of continuous functions that converges pointwise to some continuous function $f$. Then
\[ \lim_{n\to \infty}\lim_{k\to\infty}f_n(x_k) = f(x) = \lim_{k\to \infty}\lim_{n\to\infty}f_n(x_k) \]
and all limits exist.
\end{lemma}
\begin{proof}
We first calculate, using the fact that each $f_n$ is continuous,
\begin{align*}
f(x) &= \lim_{n\to \infty}f_n(x) \\
&= \lim_{n\to \infty}f_n\big(\lim_{k\to\infty} x_k\big) \\
&= \lim_{n\to \infty}\lim_{k\to\infty}f_n(x_k).
\end{align*}
Next we calculate, using the fact that $f$ is continuous,
\begin{align*}
f(x) &= f\big(\lim_{k\to\infty} x_k\big) \\
&= \lim_{k\to\infty} f(x_k) \\
&= \lim_{k\to\infty} \lim_{n\to \infty} f_n(x_k).
\end{align*}
\end{proof}


\subsection{Continuous convergence}
\begin{definition}
Let $\sSet{X, \xi}$ and $\sSet{Y,\zeta}$ be convergence spaces. 
A filter $H\in \powerfilters(X\to Y)$ converges in the \udef{continuous preconvergence} on $(X\to Y)$ to $f\in (X\to Y)$ if
\[ \forall F\overset{\xi}{\longrightarrow} x: \quad \evalMap^{\imf\imf}(H \otimes F) \overset{\zeta}{\longrightarrow} f(x). \]
The set $(X\to Y)$ equipped with the continuous convergence is denoted $(X\to Y)_c$.

The set $\cont(X, Y)$ of continuous functions in $(X\to Y)$ equipped with the continuous convergence is denoted $\cont_c(X, Y)$.
\end{definition}

\begin{lemma} \label{evalMapContinuous}
Let $\sSet{X,\xi}$ and $\sSet{Y,\zeta}$ be convergence spaces. The continuous preconvergence on $(X\to Y)$ is the weakest preconvergence on $(X\to Y)$ such that the evaluation map $\evalMap|_{(X\to Y)\times X}: (X\to Y)\times X \to Y$ is continuous.

In particular, the convergence $(X\to Y)_c\otimes X$ is stronger than the initial convergence w.r.t. the evaluation map.
\end{lemma}
\begin{proof}
We first show that the continuous convergence makes the evaluation map continuous. A filter $G\in\powerfilters\big((X\to Y)_c\times X\big)$ converges iff $p_1[G]\to f$ and $p_2[G]\to x$ converge. By definition of continuous convergence $\evalMap^{\imf\imf}(p_1[G] \otimes p_2[G]) \to f(x)$ and thus, by \ref{filterFactorisationInequality}, $\evalMap^{\imf\imf}(p_1[G] \otimes p_2[G]) \subseteq \evalMap^{\imf\imf}(G) \to f(x)$.

Now assume there is some other preconvergence $\sigma$ on $(X\to Y)$ that makes the evaluation map continuous. Take $H \overset{\sigma}{\longrightarrow} f$. Then for all $F\overset{\xi}{\longrightarrow}x$ we have $H\otimes F\overset{\sigma\otimes \xi}{\longrightarrow} (f,x)$, so $\evalMap^{\imf\imf}(H\otimes F) \overset{\zeta}{\longrightarrow} f(x)$ by continuity of the evaluation map. Thus $H\overset{(X,Y)_c}{\longrightarrow} f$.
\end{proof}

\begin{example}
It is in general not true that $(X\to Y)_c\otimes X$ is equal to the initial convergence w.r.t. the evaluation map.

Indeed let $X=\R=Y$ and consider any discontinuous function $f:\R\to \R$. For example
\[ f: \R\to \R: x\mapsto \begin{cases}
0 & (x\leq 0) \\
1 & (x > 0).
\end{cases} \]
Then $\evalMap^{\imf\imf}(\pfilter{f}\otimes \pfilter{0}) = \pfilter{0} \to 0$, so $\pfilter{f}\otimes \pfilter{0}$ converges in the initial convergence w.r.t. the evaluation map. But $\pfilter{f}$ clearly does not converge in $(X\to Y)_c$, so $\pfilter{f}\otimes \pfilter{0}$ does not converge in $(X\to Y)_c\otimes X$.
\end{example}

\begin{lemma} \label{continuousConvergenceIffContinuousPrincipalUltrafilter}
Let $\sSet{X,\xi}$ and $\sSet{Y,\zeta}$ be convergence spaces and $f\in (X\to Y)$. Then $\pfilter{f}$ converges to $f$ in continuous convergence \textup{if and only if} $f$ is continuous.
\end{lemma}
\begin{proof}
We have $\upset\evalMap^{\imf\imf}(\pfilter{f}, F) = \upset f^{\imf\imf}(F)$ for all $F\in \powerfilters(X)$, so the condition of continuous convergence is equivalent to the condition of continuity.
\end{proof}

\begin{lemma}
Let $\sSet{X,\xi}$ and $\sSet{Y,\zeta}$ be convergence spaces. Then
\begin{enumerate}
\item $(X\to Y)_c$ is a preconvergence space;
\item $\cont_c(X,Y)$ is a convergence space.
\end{enumerate}
\end{lemma}
\begin{proof}
It is clear that the continuous preconvergence is monotonic. We just need to show that it is centered, when restricted to $\cont(X,Y)$. This is given by \ref{continuousConvergenceIffContinuousPrincipalUltrafilter}.
\end{proof}


\begin{lemma}
Let $\sSet{X,\xi}$ and $\sSet{Y,\zeta}$ be convergence spaces and $H\in \powerfilters(X\to Y)$. If $\xi$ is pretopological, then $H\overset{(X\to Y)_c}{\longrightarrow} f$ \textup{if and only if}
\[ H\overset{(X\to Y)_c}{\longrightarrow} f\qquad \text{if and only if} \qquad \evalMap^{\imf\imf}\big(H\otimes \vicinity(x)\big) \overset{\zeta}{\longrightarrow} f(x). \]
\end{lemma}
\begin{proof}
TODO
\end{proof}

\begin{example}
There exists a sequence of continuous functions that converges pointwise to a continuous function, but does not converge in continuous convergence.

Consider $\seq{f_n} \subseteq (\R\to\R)$ defined by
\[ f_n: \R\to\R: x\mapsto \begin{cases}
n\cdot x & (x\in \interval{0,n^{-1}}) \\
2 - nx & (x\in \interval{n^{-1}, 2n^{-1}}) \\
0 & (\text{otherwise}).
\end{cases} \]
TODO image.
\end{example}

\begin{lemma} \label{strengthContinuousPointwiseConvergence}
Let $\sSet{X,\xi}$ and $\sSet{Y,\zeta}$ be convergence spaces, then the continuous convergence is stronger than pointwise convergence on $(X\to Y)$.
\end{lemma}
\begin{proof}
Assume $H\to f$ in continuous convergence. Then
\[ \forall x\in X: \; \evalMap^{\imf\imf}(H\otimes \pfilter{x}) \to f(x), \]
because $\pfilter{x}\to x$. This is exactly the requirement for $H\to f$ in pointwise convergence.
\end{proof}


\begin{proposition} \label{continuousLimitsContinuous}
Let $\sSet{X,\xi}$ and $\sSet{Y,\zeta}$ be convergence spaces. Either of the following is sufficient to make all limit functions in $(X\to Y)_c$ continuous:
\begin{enumerate}
\item $(X\to Y)_c$ is a Kent space;
\item $\sSet{Y,\zeta}$ is regular. 
\end{enumerate}
\end{proposition}
TODO: $\sSet{Y,\zeta}$ is regular iff for all convergence spaces $X$ all limit functions are continuous. (Dolecki, Mynard p 421)
\begin{proof}
(1) Take $H\in\powerfilters\big((X\to Y)\big)$ such that $\begin{tikzcd}H \arrow[r, "{(X\to Y)_c}"] & f\end{tikzcd}$. We have, for all $F\to x\in X$,
\begin{align*}
\evalMap^{\imf\imf}\big((H\cap \pfilter{f})\otimes F\big) &= \evalMap^{\imf\imf}\big((H \otimes F) \cap (\pfilter{f} \otimes F)\big) \\
&= \evalMap^{\imf\imf}(H \otimes F) \cap \evalMap^{\imf\imf}(\pfilter{f} \otimes F),
\end{align*} 
using \ref{intersectionProductFilters} and \ref{imageUpsetsPreservesIntersection}. By the Kent property, $\evalMap^{\imf\imf}\big((H\cap \pfilter{f})\otimes F\big)$ converges to $f(x)$. And thus by monotonicity of the convergence,
\[ \evalMap^{\imf\imf}(\pfilter{f} \otimes F) = f^{\imf\imf}[F] \overset{\zeta}{\longrightarrow} f(x). \]
Thus $f$ is continuous.

(2) Take $H\in\powerfilters\big((X\to Y)\big)$ such that $\begin{tikzcd}H \arrow[r, "{(X\to Y)_c}"] & f\end{tikzcd}$ and $F\in\powerfilters(X)$ such that $F\overset{\xi}{\longrightarrow} x$. Take $A\in F$ and $B\in H$.

Now take $a\in A$. By continuous convergence, we have $\evalMap^{\imf\imf}(H\otimes\dot{a}) \overset{\zeta}{\longrightarrow} f(a)$. We have $f(a)\in\adh_\zeta\big(\evalMap^\imf(B\times\{a\})\big) \subseteq \adh_\zeta\big(\evalMap^\imf(B\times A)\big)$ by \ref{adherenceLimitFilter}. As this is true for all $a\in A$, we have $f^\imf(A) \subseteq \adh_\zeta\big(\evalMap^\imf(B\times A)\big)$.

From this last inclusion we can conclude that $\adh_\zeta^\imf\big(\evalMap^{\imf\imf}(H\otimes F)\big) \subseteq f^{\imf\imf}(F)$. Now $\evalMap^{\imf\imf}(H\otimes F)\overset{\zeta}{\longrightarrow} f(x)$ by continuous convergence, so $\adh_\zeta^\imf\big(\evalMap^{\imf\imf}(H\otimes F)\big)\overset{\zeta}{\longrightarrow} f(x)$ by regularity and thus $f^{\imf\imf}(F)\overset{\zeta}{\longrightarrow} f(x)$ by monotonicity.

We conclude that $f$ is continuous.
\end{proof}
\begin{corollary} \label{setOfContinuousFunctionsContinuouslyClosed}
In either of these cases, $\cont_c(X,Y)$ is closed subset of $(X\to Y)_c$.
\end{corollary}

\begin{example}
There exist non-continuous limit points in spaces of continuous convergence.

Set $X = \{a,b\}$ and $Y = \{1,2\}$.

Let $X$ have the convergences $\begin{cases}
\pfilter{a} \to a \\
\pfilter{b} \to a,b
\end{cases}$ and let $Y$ have the convergences $\begin{cases}
\pfilter{1} \to 1,2 \\
\pfilter{2} \to 2
\end{cases}$.
Consider the functions $f: X\to Y: \begin{cases}
a\mapsto 1 \\ b\mapsto 2
\end{cases}$ and $g: X\to Y: \begin{cases}
a\mapsto 1 \\ b\mapsto 1
\end{cases}$.

Then it is straightforward to verify that $\pfilter{g} \overset{(X\to Y)_c}{\longrightarrow} f$, but $f$ is not continuous. Indeed $\pfilter{b} \to a$, but $f^{\imf\imf}[\pfilter{b}] = \pfilter{2} \not\to f(a) = 1$.
\end{example}

\begin{proposition}
Let $\sSet{X,\xi}$ be a convergence space and $\sSet{Y,\zeta}$ a regular convergence space. Let $H\in\powerfilters(\cont(X,Y))$ be a filter that converges to $f$ in the continuous convergence. Let $F\in \powerfilters(X)$ be a filter that converges to $x\in X$. Then the following limits exist and are equal to each other:
\begin{align*}
f(x) &= \lim_\zeta \evalMap^{\imf\imf}\big(H\otimes F\big) \\
&= \lim_\zeta \evalMap^{\imf\imf}(\pfilter{f}\otimes F) \\
&= \lim_\zeta \evalMap_x^{\imf\imf}(H).
\end{align*}
\end{proposition}
\begin{proof}
We have $\lim_\zeta \evalMap^{\imf\imf}\big(H\otimes F\big)$ exists and is equal to $f(x)$ by definition of the continuous convergence.

By \ref{continuousLimitsContinuous}, $f$ is continuous. So $\pfilter{f}$ converges to $f$ in the continuous convergence by \ref{continuousConvergenceIffContinuousPrincipalUltrafilter}. The definition of the continuous convergence again gives the existence of the limit $\lim_\zeta \evalMap^{\imf\imf}(\pfilter{f}\otimes F)$.

Finally we have $\evalMap_x^{\imf\imf}(H) = \evalMap^{\imf\imf}\big(H\otimes \pfilter{x}\big)$. The existence of the limit follows as before.
\end{proof}

\begin{lemma} \label{exchangeSequenceLimitsContinuousConvergence}
Let $\sSet{X,\xi}$ be a convergence space and $\sSet{Y,\zeta}$ a regular convergence space. Suppose that $\seq{x_n}_{n\in \N}$ a sequence that converges to $x\in X$ and $\seq{f_k}_{k\in \N}$ a sequence of continuous functions that converges to $f$ in the continuous convergence. Then
\[ \lim_{n\to \infty}\lim_{k\to\infty}f_n(x_k) = f(x) = \lim_{k\to \infty}\lim_{n\to\infty}f_n(x_k) \]
and all limits exist.
\end{lemma}
\begin{proof}
We can use \ref{interchangeLimitsWithContinuousLimitFunction}, since $f$ is continuous, by \ref{continuousLimitsContinuous}, and the continuous convergence is stronger than the pointwise convergence, \ref{strengthContinuousPointwiseConvergence}.
\end{proof}

\subsubsection{Universal property of the continuous convergence}
\begin{proposition}[Universal property of the continuous convergence] \label{universalPropertyContinuousConvergence}
Let $\sSet{X, \xi}, \sSet{Y,\sigma}$ and $\sSet{Z,\zeta}$ be convergence spaces. Then $h: X\to (Y \to Z)_c$ is continuous \textup{if and only if} $\curry_1^{-1}(h): X\times Y \to Z$ is continuous.

In this case $\im(h) \subseteq \cont_c(Y,Z)$, so we can write $h: X\to \cont(Y,Z)_c$.
\end{proposition}
\begin{proof}
First assume $h$ is continuous. Then $(h\circ \pi_1, \id_Y\circ \pi_2)$ is continuous by \ref{continuityFunctionTuple}, $\curry_1^{-1}(h) = \evalMap \circ (h\circ \pi_1, \id_Y\circ \pi_2)$ by \ref{inverseCurryLemma} and thus $\curry_1^{-1}(h)$ is continuous.

Now assume $\curry_1^{-1}(h)$ is continuous and let $F\to x$ converge in $X$. Then for all convergent $G\to y\in Y$, we have, again using \ref{inverseCurryLemma}.
\begin{align*}
\evalMap^{\imf\imf}\big(h^{\imf\imf}(F)\otimes G\big) &= \evalMap^{\imf\imf}\big((h\circ \pi_1, \id_Y\circ \pi_2)^{\imf\imf}(F\otimes G)\big) \\
&= \curry_1^{-1}(h)^{\imf\imf}(F\otimes G) \to \curry_1^{-1}(h)(x,y) = h(x)(y).
\end{align*}
Thus $h^{\imf\imf}(F) \to h(x)$ in the continuous convergence, which means that $h: X\to (Y \to Z)_c$ is continuous.

Finally we show that if $h$ is continuous, then $h(x)$ is continuous for all $x\in X$. Take $G\to y$ in $Y$ and then
\[ h(x)^{\imf\imf}(G) = \evalMap^{\imf\imf}(\pfilter{h}(x)\otimes G) = \curry_1^{-1}(h)^{\imf\imf}(\pfilter{x}\otimes G) \to \curry_1^{-1}(h)(x,y) = h(x)(y). \]
\end{proof}

\begin{proposition}
Let $\sSet{X, \xi}, \sSet{Y,\sigma}$ and $\sSet{Z,\zeta}$ be convergence spaces. Then
\[ \curry_1|_{\cont_c(X\times Y, Z)}^{\cont_c\big(X, \cont_c(Y,Z)\big)}: \cont_c(X\times Y, Z) \to \cont_c\big(X, \cont_c(Y,Z)\big) \]
is well-defined and a homeomorphism.
\end{proposition}
\begin{proof}
That this is well-defined, is given by \ref{universalPropertyContinuousConvergence}.

Set $f \defeq \curry_1|_{\cont_c(X\times Y, Z)}^{\cont_c\big(X, \cont_c(Y,Z)\big)}$ and consider the operations
\[ \begin{tikzcd}
\cont_c(X\times Y, Z) \to \cont_c\big(X, \cont_c(Y,Z)\big) \ar[d] \\
\cont_c(X\times Y, Z) \to \big(X \to \cont_c(Y,Z)\big)_c \ar[d, "{\curry_1^{-1}}"] \\
\cont_c(X\times Y, Z)\times X \to \cont_c(Y,Z) \ar[d] \\
\cont_c(X\times Y, Z)\times X \to (Y \to Z)_c \ar[d, "{\curry_1^{-1}}"] \\
\curry_1^{-1}\big(\curry_1^{-1}(\curry_1|_{\cont_c(X\times Y, Z)})\big): \big(\cont_c(X\times Y, Z)\times X\big)\times Y \to Z
\end{tikzcd} \]
As all these steps reflect continuity (by \ref{continuityRestrictionExpansion} and \ref{universalPropertyContinuousConvergence}), so it is enough to show the continuity of the result of these operations applied to $f$. Now it is straightforward to calculate that this is equal to the function
\[ \evalMap|_{\cont_c(X\times Y, Z)\times (X\times Y)}\circ \alpha: \big(\cont_c(X\times Y, Z)\times X\big)\times Y \to Z, \]
where $\alpha: \big(\cont_c(X\times Y, Z)\times X\big)\times Y \to \cont_c(X\times Y, Z)\times (X\times Y)$ is the associator homeomorphism constructed in \ref{associatorTernaryProducts}. This function is continuous by \ref{evalMapContinuous} and \ref{continuityRestrictionExpansion}.
\end{proof}
\begin{corollary}
The category of convergence spaces is cartesian closed.
\end{corollary}

\begin{proposition} \label{compositionContinuouslyContinuous}
Let $\sSet{X, \xi}, \sSet{Y,\sigma}$ and $\sSet{Z,\zeta}$ be convergences spaces. The composition operation
\[ \circ: \cont_c(Y,Z) \times \cont_c(X,Y) \to \cont_c(X,Z) \]
is continuous.
\end{proposition}
\begin{proof}
By \ref{universalPropertyContinuousConvergence}, this is equivalent to the continuity of $\curry_1^{-1}(\circ): (Y\to Z)_c \times (X\to Y)_c \times X \to Z$, which follows from the commutativity of the following diagram:
\[\begin{tikzcd}
(Y\to Z)_c \times (X\to Y)_c \times X \ar[d, swap, "\id_{(Y\to Z)}\times \evalMap"] \ar[dr, "\curry_1^{-1}(\circ)"] & \\
(Y\to Z)_c\times Y \ar[r, "\evalMap"] & Z.
\end{tikzcd} \]
\end{proof}

\begin{proposition} \label{curriedEvalMapContinuous}
The map $X\to \cont_c\big(\cont_c(X,Y),Y\big): x\mapsto \evalMap_x$ is continuous.
\end{proposition}
\begin{proof}
By \ref{universalPropertyContinuousConvergence}, the continuity of this map is equivalent to the continuity of the uncurried form $\evalMap: X\times \cont_c(X,Y) \to Y$. This is continuous by \ref{evalMapContinuous}.
\end{proof}


\subsubsection{Convergence properties of the continuous convergence}
\begin{lemma} \label{embeddingInContinuousConvergence}
Let $\sSet{X,\xi}, \sSet{Y,\zeta}$ be non-empty convergence spaces. Then $k: Y \to \cont_c(X,Y): y\mapsto \underline{y}$ is an embedding.
\end{lemma}
\begin{proof}
Clearly $k$ is injective, so $k: Y\to \im(k)$ is bijective.

First we prove that $k$ is continuous. To that end, take arbitrary $G\to y$ in $Y$. Then for arbitrary $F\to x$ in $X$, we have
\[ \evalMap\big(k^{\imf\imf}(G)\otimes F\big) = G \overset{\zeta}{\longrightarrow} y = \underline{y}(x), \]
so $k^{\imf\imf}(G)$ converges to $k(y) = \underline{y}$ in the continuous convergence.

Now suppose $G\in\powerfilters(Y)$ is such that $\upset k^{\imf\imf}(Y)$ converges to $f\in \cont_c(X,Y)$. Then for any $x\in X$, we have
\[ G = \evalMap\big(k^{\imf\imf}(G)\otimes F\big) \overset{\zeta}{\longrightarrow} k(x) = \underline{y}(x) = y. \]
\end{proof}

\begin{lemma}
Let $\sSet{X,\xi}$ and $\sSet{Y,\zeta}$ be convergence spaces. If $Y$ is of finite depth, then so is $(X\to Y)_c$.
\end{lemma}
\begin{proof}
Let $G, H\overset{(X\to Y)_c}{\longrightarrow} f$. Then for all $F\to x\in X$, we have
\begin{align*}
\evalMap^{\imf\imf}\big((G\cap H)\otimes F\big) &= \evalMap^{\imf\imf}\big((G \otimes F) \cap (H \otimes F)\big) \\
&= \evalMap^{\imf\imf}(G \otimes F) \cap \evalMap^{\imf\imf}(H \otimes F) \; \overset{\zeta}{\longrightarrow} f(x),
\end{align*}
using \ref{intersectionProductFilters}, \ref{imageUpsetsPreservesIntersection} and the finite depth of $Y$.
\end{proof}
Note that this does not imply the Kent property, because $(X\to Y)_c$ is only a preconvergence space. The space $\cont_c(X,Y)$ does have the Kent property if $Y$ is of finite depth.

\begin{proposition} \label{continuousConvergencePropertiesFromCodomain}
Let $\sSet{X,\xi}, \sSet{Y,\zeta}$ be non-empty convergence spaces. Then
\begin{enumerate}
\item $\cont_c(X,Y)$ is pseudotopological \textup{if and only if} $Y$ is pseudotopological;
\item $\cont_c(X,Y)$ is Hausdorff \textup{if and only if} $Y$ is Hausdorff;
\item $\cont_c(X,Y)$ is regular \textup{if and only if} $Y$ is regular.
\end{enumerate}
\end{proposition}
\begin{proof}
(1) First assume $\cont_c(X,Y)$ is pseudotopological. Then $Y$ is homeomorphic to a subspace of $\cont_c(X,Y)$ by \ref{embeddingInContinuousConvergence} and thus pseudotopological by \ref{pretopologicalInitialConvergence}.

Now assume $Y$ pseudotopological. By \ref{evalMapContinuous}, it is enough to prove that $\pseudotopMod\big(\cont_c(X,Y)\big)\otimes X$ makes the evaluation map $\evalMap: \big(\cont(X,Y)\times X\big) \to Y$ continuous, as $\pseudotopMod\big(\cont_c(X,Y)\big)$ is weaker than $\cont_c(X,Y)$ by definition.

Let $\mu$ be the initial convergence on $\cont(X,Y)\times X$ w.r.t. the evaluation map. We have that $\cont_c(\xi,\zeta)\otimes \xi \subseteq \mu$ by \ref{evalMapContinuous}. Now
\[ \pseudotopMod\big(\cont_c(\xi,\zeta)\big)\otimes \xi \subseteq \pseudotopMod\big(\cont_c(\xi,\zeta)\big)\otimes \pseudotopMod(\xi) = \pseudotopMod\big(\cont_c(\xi,\zeta)\otimes \xi\big) \subseteq \pseudotopMod(\mu) = \mu, \]
where the equality is given by \ref{pseudotopologiserCommutesWithInitialStructure}. Thus $\pseudotopMod\big(\cont_c(\xi,\zeta)\big)$ makes the evaluation map continuous by \ref{continuityUnderConvergenceLatticeOperations}.

(2) First assume $\cont_c(X,Y)$ is Hausdorff. Then $Y$ is homeomorphic to a subspace of $\cont_c(X,Y)$ by \ref{embeddingInContinuousConvergence} and thus Hausdorff by \ref{HausdorffSubspace}.

Now assume $Y$ Hausdorff and suppose $H\in \powerfilters\big(\cont_c(X,Y)\big)$ converges to both $f$ and $g$. Then, for all $x\in X$, $\evalMap^{\imf\imf}(H\otimes \pfilter{x})$ converges to both $f(x)$ and $g(x)$ in $Y$. As $Y$ is Hausdorff, $f(x) = g(x)$. Thus $f=g$.

(3) The direction $\Rightarrow$ is given by \ref{embeddingInContinuousConvergence}: $Y$ is homeomorphic to a subspace of $\cont_c(X,Y)$ and must therefore be regular by \ref{regularityInitialConvergence}.

For the other direction, assume $\sSet{Y,\zeta}$ is regular and let $H \overset{\cont_c(X,Y)}{\longrightarrow} f$.
Take an arbitrary $F\overset{\xi}{\longrightarrow} x\in X$. We claim $\upset\adh_\zeta^{\imf}\big(\evalMap^{\imf\imf}(H\otimes F)\big)\subseteq \upset \evalMap^{\imf\imf}\big(\adh_c(H)\otimes F\big)$. Indeed, for all $A\subseteq \cont(X,Y)$ and $B\subseteq X$, we have
\[ \evalMap^{\imf}(\adh_{c}(A)\times B) \subseteq \evalMap^{\imf}(\adh_{c}(A)\times \adh_\xi(B)) = \evalMap^{\imf}\big(\adh_{c\otimes \xi}(A\times B)\big) \subseteq \adh_{\zeta}\big(\evalMap^\imf(A\times B)\big), \]
by \ref{principalInherenceAdherenceProperties}, \ref{productAdherence} and \ref{adherenceInherenceContinuity}.

Now $\upset\adh_\zeta^{\imf}\big(\evalMap^{\imf\imf}(H\otimes F)\big)$ converges by the product covergence \ref{convergenceFiniteProductFilter}, continuity of the evaluation map \ref{evalMapContinuous}, and regularity of $\zeta$. Then $\upset \evalMap^{\imf\imf}\big(\adh_c(H)\otimes F\big)$ converges by monotonicity and $\adh_c(H)$ converges in the continuous convergence.
\end{proof}

\subsubsection{$c$-embedded spaces}

\subsubsection{Compact-open topology}
\begin{definition}
Let $\sSet{X,\xi}$ and $\sSet{Y,\zeta}$ be convergence spaces. The set
\[ \setbuilder{\big.\setbuilder{f\in \cont(X,Y)}{f^\imf(K)\subseteq U}}{\text{$K\subseteq X$ compact, $U\subseteq Y$ open}} \]
is a subbasis for the \udef{compact-open topology} on $\cont(X,Y)$. The associated convergence is the \udef{compact-open convergence}.
\end{definition}

\begin{proposition} \label{continuousConvergenceCompactOpenComparison}
Let $\sSet{X,\xi}$ and $\sSet{Y,\zeta}$ be convergence spaces. The continuous convergence on $\cont(X,Y)$ is stronger than the compact-open convergence.
\end{proposition}
Because the compact-open convergence is topological by definition, this is equivalent to $\topMod\big(\cont_c(X,Y)\big)$ being stronger than the compact-open convergence.
\begin{proof}
By \ref{topologicalModificationPreservation} and \ref{topologyMonotoneInConvergence}, this is equivalent to $\topology_{co}\subseteq \topology_{\cont_c(X,Y)}$.
In fact it is enough to show that the set
\[ \setbuilder[\big]{\setbuilder{f\in \cont(X,Y)}{f^\imf(K)\subseteq U}}{\text{$K\subseteq X$ compact, $U\subseteq Y$ open}} \]
is contained in $\topology_{\cont_c(X,Y)}$.

Take arbitrary $K\subseteq X$ compact and $U\subseteq Y$ open. Set $A \defeq \setbuilder{f\in \cont(X,Y)}{f^\imf(K)\subseteq U}$. We show that $A$ is open by \ref{openClosedCriteria}. Suppose $H\in \powerfilters\big(\cont(X,Y)\big)$ converges in continuous convergence to $f\in A$.

Now take arbitrary $F\overset{\xi}{\longrightarrow} x\in K$. By assumption, $f(x)\in O$, so $O\in \upset\evalMap^{\imf\imf}(H\otimes F)$. Thus there exist $B_{F,x}\in H$ and $C_{F,x}\in F$ such that $\evalMap^\imf(B_{F,x}\times C_{F,x})\subseteq O$

Consider the set $\setbuilder{C_{F,x}}{F\to x}$, which is a convergence cover. Then by compactness of $K$ and \ref{compactFiniteSubcover}, we can find a finite subset $C_{x_0, F_0},\ldots C_{x_n, F_n}$ such that $C_{x_0, F_0} \cup \ldots \cup C_{x_n, F_n} \supseteq K$. Now 
\[ \evalMap^\imf\big((B_{x_0, F_0} \cap \ldots \cap B_{x_n, F_n})\times K\big) \subseteq \evalMap^\imf\big((B_{x_0, F_0} \cap \ldots \cap B_{x_n, F_n})\times (C_{x_0, F_0} \cup \ldots \cup C_{x_n, F_n})\big)\subseteq O, \]
so $B_{x_0, F_0} \cap \ldots \cap B_{x_n, F_n}\in H$ is a subset of $A$ and thus $A\in H$. This allows us to conclude that $A$ is open by \ref{openClosedCriteria}.
\end{proof}

\begin{proposition} \label{continuousConvergenceCompactOpen}
Let $\sSet{X,\xi}$ be a convergence space and $\sSet{Y,\zeta}$ a regular topological space. Then each of the following statements implies the next:
\begin{enumerate}
\item $\sSet{X,\xi}$ is locally compact;
\item the continuous convergence on $\cont(X,Y)$ is equal to the compact-open convergence;
\item $\cont_c(X,Y)$ is topological.
\end{enumerate}
If $X$ is $c$-embedded, then these statements are equivalent.
\end{proposition}
TODO generalise???
\begin{proof}
$(1) \Rightarrow (2)$ We show that $\neighbourhood_{co}(f)$ converges to $f$ in continuous convergence for all $f\in \cont(X,Y)$. To that end, take an arbitrary $F\to x\in X$. We need to show that any open set $U$ in $\neighbourhood_\zeta\big(f(x)\big)$ contains an element of $\evalMap^{\imf\imf}(\neighbourhood_{co}(f)\otimes F)$.

By assumption, $F$ contains a compact set $K$. By regularity, there is a closed neighbourhood $V$ of $f(x)$ contained in $U$. Then set $K' \defeq K \cap f^{\preimf}(V)$; because $f^{\preimf}(V)$ is closed by \ref{preimageOpenClosed}, $K'$ is compact by \ref{compactClosedIntersectionCompact}.

Now consider $A \defeq \setbuilder{f\in \cont(X,Y)}{f^\imf(K')\subseteq U}$, which is in $\neighbourhood_{co}(f)$ because $f^\imf(K')\subseteq f^{\imf}\big(f^{\preimf}(V)\big) \subseteq V \subseteq U$, so $f\in A$.

We have $K'\in F$ because $f^{\preimf}(V)$ is a neighbourhood of $x$ by \ref{continuityVicinityFilter}. Thus $U$ contains $\evalMap^\imf(A\times K') \in \evalMap^{\imf\imf}(\neighbourhood_{co}(f)\otimes F)$, which is what we needed to prove.

$(2) \Rightarrow (3)$ Immediate.

$(3) \Rightarrow (1)$ TODO
\end{proof}

\subsection{Even continuity}
\begin{definition}
Let $\sSet{X,\xi},\sSet{Y, \zeta}$ be convergence spaces, $x\in X$ and $H\subseteq (X\to Y)$. Then $H$ is
\begin{itemize}
\item \udef{evenly continuous at $x$} if pointwise and continuous convergence at $x$ coincide for all filters that contain $H$;
\item \udef{evenly continuous} if pointwise and continuous convergence coincide in for all filters that contain $H$.
\end{itemize}
\end{definition}


\begin{lemma} \label{evenlyContinuousLemma}
Let $X,Y$ be convergence spaces, $x\in X$ and $H\subseteq (X\to Y)$. Then $H$ is evenly continuous at $x$ \textup{if and only if} for all filters $F\in \powerfilters(X\to Y)$ that contain $H$, all $G\in \powerfilters(X)$ and all $y\in Y$, we have
\[ \left.\begin{aligned}
\upset\evalMap^{\imf\imf}(F\otimes \pfilter{x}) &\to y \\
G&\to x
\end{aligned}\right\}\qquad\implies\qquad \upset\evalMap^{\imf\imf}(F\otimes G)\to y. \]
\end{lemma}

\begin{lemma} \label{evenlyContinuousSubset}
All subsets of evenly continuous sets are evenly continuous.
\end{lemma}

\begin{lemma}
Let $X,Y$ be convergence spaces, $x\in X$ and $H\subseteq (X\to Y)$. Then
\begin{enumerate}
\item if $H$ is evenly continuous at $x$, then each $f\in H$ is continuous at $x$;
\item if $H$ is evenly continuous, then $H \subseteq \cont(X,Y)$
\end{enumerate}
\end{lemma}
\begin{proof}
Point (2) follows immediately from point (1).

Assume $H$ is evenly continuous at $x$ and take arbitrary $f\in H$. Now take arbitrary $G\overset{\xi}{\longrightarrow} x$. We have $H\in \pfilter{f}$, so
\[ \upset\evalMap^{\imf\imf}(\pfilter{f}\otimes \pfilter{x}) = \pfilter{f}(x) \to f(x) \]
implies $\upset f^{\imf\imf}(G) = \upset\evalMap^{\imf\imf}(\pfilter{f}\otimes G) \to f(x)$, which means that $f$ is continuous at $x$.
\end{proof}

\subsubsection{The Arzelà-Ascoli theorem}
\begin{proposition}
Let $\sSet{X,\xi},\sSet{Y, \zeta}$ be convergence spaces, $H\subseteq \cont(X, Y)$ and $\{A_x\}_{x\in X}\subseteq \powerset(Y)$.

If $H$ is evenly continuous and $\evalMap_x^\imf(H)$ is $A_x$-compactoid in $Y$ for all $x\in X$, then $H$ is $\prod A_x$-compactoid in $(X\to Y)_c$.
\end{proposition}
\begin{proof}
Let $U\in\powerultrafilters\big((X\to Y)\big)$ be an ultrafilter that contains $H$. Take arbitrary $x\in X$. Then $\evalMap_x^\imf(H) \in \upset\evalMap_x^{\imf\imf}(U)$, which is an ultrafilter by \ref{imageFilterProperties}. Because $\evalMap_x^\imf(H)$ was assumed $A_x$-compactoid, $\upset\evalMap_x^{\imf\imf}(U)$ converges to some $y\in A_x$.

Now let $f: X\to Y$ be a function that maps each $x$ to such a $y$. By construction, $U$ converges pointwise to $f$. As $H$ is evenly continuous and $U$ contains $H$, we have that $U$ converges in continuous convergence to $f$. Also $f\in \prod A_x$ because $f(x) \in A_x$ for all $x\in X$.
\end{proof}
\begin{corollary}
Let $\sSet{X,\xi}$, $\sSet{Y, \zeta}$ be a convergence spaces, $H\subseteq \cont(X, Y)$ and $A\subseteq(X\to Y)_c$ a closed subset.
If $H$ is evenly continuous, $H\subseteq A$ and $\evalMap_x^\imf(H)$ is $\evalMap_x^\imf(A)$-compactoid for all $x\in X$, then  $H$ is $A$-compactoid.
\end{corollary}
\begin{proof}
We need to show that the limit function $f$ constructed in the proof of the proposition is an element of $A$.

This is immediate because $A\in U$ and $A$ is continuously closed.
\end{proof}
\begin{corollary}
Let $\sSet{X,\xi}$ be a convergence space, $\sSet{Y, \zeta}$ a regular convergence spaces and $H\subseteq \cont(X, Y)$.

If $H$ is evenly continuous and $\evalMap_x^\imf(H)$ is compactoid in $Y$ for all $x\in X$, then $H$ is compactoid in $\cont(X, Y)_c$.
\end{corollary}
\begin{proof}
We set $A = \cont(X,Y)$ in the previous corollary. This is continuously closed if $Y$ is regular, by \ref{setOfContinuousFunctionsContinuouslyClosed}.
\end{proof}
\begin{corollary} \label{evenContinuityRelativeCompactness}
Let $\sSet{X,\xi}$ be a convergence space, $\sSet{Y, \zeta}$ a Hausdorff, regular, pseudotopological convergence space and $H\subseteq \cont(X, Y)$.

If $H$ is evenly continuous and $\evalMap_x^\imf(H)$ is relatively compact in $Y$ for all $x\in X$, then $H$ is relatively compact in $\cont(X, Y)_c$.
\end{corollary}
\begin{proof}
By assumption, $\adh_\zeta\big(\evalMap_x^\imf(H)\big)$ is compact, and by the proposition $H$ is $\prod_{x\in X}\adh_\zeta\big(\evalMap_x^\imf(H)\big)$-compactoid. Now $\prod_{x\in X}\adh_\zeta\big(\evalMap_x^\imf(H)\big)$ is compact by Tychonoff's theorem \ref{TychonoffsTheorem} and thus topological by \ref{T3pseudotopologyTopological}. This means that $\adh(H)$ is closed in this subspace. It is compact by \ref{compactClosedSets}.
\end{proof}


\begin{proposition} \label{compactoidImpliesEvenContinuity}
Let $\sSet{X,\xi}$ be a convergence space, $\sSet{Y, \zeta}$ a Hausdorff pseudotopological convergence space and $H, A\subseteq (X\to Y)$. If $H$ is $A$-compactoid in $(X\to Y)_c$, then
\begin{enumerate}
\item $\evalMap_x^\imf(H)$ is $\evalMap_x^\imf(A)$-compactoid for all $x\in X$;
\item $H$ is evenly continuous.
\end{enumerate}
\end{proposition}
\begin{proof}
(1) As $\evalMap_x$ is continuous, this follows from \ref{compactConstructions}.

(2) We use \ref{evenlyContinuousLemma}. Take arbitrary $x\in X$, $y\in Y$, $G\in \powerfilters(X)$ that converges to $x$ and $F\in \powerfilters(X\to Y)$ such $H\in F$. Assume $\upset\evalMap^{\imf\imf}(F\otimes \pfilter{x})\to y$. We need to show that $\upset\evalMap^{\imf\imf}(F\otimes G) \to y$. Because $Y$ is a pseudotopology, it is enough to show that $U\to y$ for all ultrafilters $U\supseteq \upset\evalMap^{\imf\imf}(F\otimes G)$.

By \ref{properFiltersSelfMesh}, we have $\upset\evalMap^{\imf\imf}(F\otimes G) \subseteq U^\mesh$, which implies
\[ \upset\evalMap^{\imf\imf}(F\otimes G) \amesh U \iff F\otimes G \amesh \evalMap^{\preimf\imf}(U) \iff \begin{cases}
F \amesh \pi_1^{\imf\imf}\big(\evalMap^{\preimf\imf}(U)\big) \\
G \amesh \pi_2^{\imf\imf}\big(\evalMap^{\preimf\imf}(U)\big)
\end{cases} \]
by \ref{meshConnectionSetsOfSets} and \ref{meshProductIsotoneSets}. Now $F \vee \pi_1^{\imf\imf}\big(\evalMap^{\preimf\imf}(U)\big)$ is proper by \ref{joinProperFilter} and thus it is contained in an ultrafilter $F'$ by \ref{ultrafilterLemma}. Now $H\in F\subseteq F'$, so $F'$ converges continuously to some $f\in (X\to Y)$.

Also $\pi_1^{\imf\imf}\big(\evalMap^{\preimf\imf}(U)\big) \subseteq F' \subseteq F^{\prime\mesh}$ by \ref{properFiltersSelfMesh}, so
\begin{multline*}
\left.\begin{aligned}F' \amesh \pi_1^{\imf\imf}\big(\evalMap^{\preimf\imf}(U)\big) \\
G \amesh \pi_2^{\imf\imf}\big(\evalMap^{\preimf\imf}(U)\big)\end{aligned}\right\} \iff  F'\otimes G \amesh \evalMap^{\preimf\imf}(U) \iff \upset\evalMap^{\imf\imf}(F'\otimes G) \amesh U \\
\iff \upset\evalMap^{\imf\imf}(F'\otimes G) \subseteq U^\mesh = U,
\end{multline*}
again by \ref{meshProductIsotoneSets}, \ref{meshConnectionSetsOfSets} and \ref{ultrafilterCriteria}.

By continuous convergence of $F'$, we have $\upset\evalMap^{\imf\imf}(F'\otimes G) \to f(x)$ and thus also $U\to f(x)$.

Finally we need to show that $f(x) = y$. By upwards closure of the convergence, we have that $\evalMap(F'\otimes \pfilter{x}) \to y$. By continuous convergence, we also have $\evalMap(F'\otimes \pfilter{x}) \to f(x)$. As $Y$ is Hausdorff, we must have $f(x) = y$.
\end{proof}

\begin{corollary}[Arzelà-Ascoli theorem]
Let $\sSet{X,\xi}$ be a convergence space, $\sSet{Y, \zeta}$ a regular Hausdorff pseudotopological convergence space and $H\subseteq \cont(X, Y)$. Then $H$ is compactoid in $\cont_c(X, Y)$ \textup{if and only if}
\begin{enumerate}
\item $\evalMap_x^\imf(H)$ compactoid for all $x\in X$;
\item $H$ is evenly continuous.
\end{enumerate}
\end{corollary}


\chapter{Related types of spaces}
\section{Merotopic spaces}
\begin{definition}
Let $X$ be a set and $\mathfrak{A}\subseteq \powerset^2(X)$. We call $\mathfrak{A}$ a \udef{merotopy} on $X$ if
\begin{itemize}
\item $\pfilter{x} \in \mathfrak{A}$ for all $x\in X$;
\item $\mathfrak{F}$ is upwards closed w.r.t. the refinement ordering, i.e.\ $\mathcal{A}\in \mathfrak{A}$ and $\mathcal{A}\subseteq \upset \mathcal{B}$ implies $\mathcal{B}\in \mathfrak{A}$.
\end{itemize}
We call $\sSet{X, \mathfrak{A}}$ a \udef{merotopic space}. Elements of $\mathfrak{A}$ are called \udef{micromeric collections}.
\begin{itemize}
\item A set $\mathcal{C}\in\powerset^2(X)$ is called \udef{near} if $\mathcal{C}^\mesh$ is micromeric.
\item For all $A,B\subseteq X$, we abbreviate $\{A,B\}\in \mathfrak{A}$ by $A\mathfrak{A}B$
\item We define the \udef{adherence} function
\[ \adh_\mathfrak{A}: \powerset(X)\to\powerset(X): A\mapsto \setbuilder{x\in X}{\{x\}\mathfrak{A}A}. \]
\item We call $\mathfrak{A}$ a \udef{nearness space} if $\mathcal{C}\in \mathfrak{A}$ implies $\adh_{\mathfrak{A}}^\imf(\mathcal{C}) \in \mathfrak{A}$.
\end{itemize}
\end{definition}
Note that the elements of $\mathfrak{A}$ are not required to be directed! (If they are, then the merotopic space is essentially a filtermerotopic space).

\begin{proposition}
Let $X$ be a set. Then set of merotopies on $X$ ordered by inclusion is a complete $\wedge$-subsemilattice.
\end{proposition}

\subsection{Nearness spaces}
$\adh, \inh$ idempotent.

\section{Filtermerotopic and Cauchy spaces}
\begin{definition}
Let $X$ be a set and $\mathfrak{F}\subseteq \powerfilters(X)$ a family of filters such that
\begin{itemize}
\item $\pfilter{x} \in \mathfrak{F}$ for all $x\in X$;
\item $\mathfrak{F}$ is upwards closed.
\end{itemize}
We call $\sSet{X, \mathfrak{F}}$ a \udef{filtermerotopic space}or simply a \udef{filter space}.

\begin{itemize}
\item If $F\cap G\in \mathfrak{F}$ for all $F,G\in \mathfrak{F}$ such that $F\mathrel{(\subseteq|_{\mathfrak{C}} \cup \supseteq|_{\mathfrak{C}}^\transp)^*} G$, then we say the filter space is \udef{reciprocal}.
\item If $F\cap G\in \mathfrak{F}$ for all $F,G\in \mathfrak{F}$ such that $F\amesh G$, then we call the filter space a \udef{Cauchy space}.
\end{itemize}
\end{definition}

\begin{lemma}
If $\sSet{X, \mathfrak{C}}$ is a Cauchy space, then $X$ is reciprocal.
\end{lemma}
\begin{proof}
Suppose $F,G\in \mathfrak{C}$ are such that $F\mathrel{(\subseteq|_{\mathfrak{C}} \cup \supseteq|_{\mathfrak{C}}^\transp)^*} G$. Then there exists a finite sequence $F \subseteq F_1 \supseteq F_2 \subseteq \ldots \supseteq F_n \subseteq G$, where $F_1, \ldots, F_n\in \mathfrak{C}$. Note that $F_k \subseteq F_{k+1} \supseteq F_{k+2}$ implies $F_k \vee F_{k+2} \leq F_{k+1}$, so $F_k \amesh F_{k+2}$ by \ref{joinProperFilter}.

By finite depth, we have $\bigcap_{k=1}^n F_k\in \mathfrak{C}$. Since $F\cap G \supseteq \bigcap_{k=1}^n F_k$, we have $F\cap G\in \mathfrak{C}$.
\end{proof}

\subsection{Filter-continuous maps}
\begin{definition}
Let $\sSet{X, \mathfrak{C}}, \sSet{Y,\mathfrak{D}}$ be filter spaces and $f: X\to Y$ a function. Then $f$ is called \udef{filter continuous} if $f^{\imf\imf\imf}(\mathfrak{C}) \subseteq \mathfrak{D}$.

If $X,Y$ are Cauchy spaces and $f$ satisfies this property, then $f$ is called \udef{Cauchy continuous}.
\end{definition}

\subsubsection{Filter embeddings}
\begin{definition}
Let $\sSet{X, \mathfrak{F}}, \sSet{Y,\mathfrak{G}}$ be filter spaces and $f: X\to Y$ a function. Then $f$ is an embedding if
\begin{itemize}
\item $f$ is injective;
\item $\upset f^{\imf\imf}(F) \in \mathfrak{D}$ \textup{if and only if} $F\in \mathfrak{C}$ for all $F\in \powerfilters(X)$.
\end{itemize}
\end{definition}
TODO make sure all definitions of embedding line up + general definition. Lines up with convergence definition because $\upset f^{\imf\imf}(F\cap \pfilter{x}) = \upset f^{\imf\imf}(F) \cap \pfilter{f}(x)$.

\subsection{Equivalence and convergence}
TODO finite depth?
\begin{definition}
Let $\sSet{X, \mathfrak{F}}$ be a filter space. Define the relation $\sim$ on $(\mathfrak{F},\mathfrak{F})$ by
\[ F \sim G \qquad \defequiv \qquad F\cap G \in \mathfrak{F}. \]
We call the filters $F,G$ equivalent.

We define the \udef{filter convergence} on $X$ by
\[ F \to x \qquad \defequiv \qquad F \sim \pfilter{x}. \]
\end{definition}

\begin{lemma}
The filter convergence is a convergence. In particular it is a Kent space.
\end{lemma}
\begin{proof}
Clearly $\pfilter{x}\cap\pfilter{x} = \pfilter{x} \in \mathfrak{F}$, so $\pfilter{x} \to x$.

Let $F\to x$ and $F\subseteq G$. Then $G\in \mathfrak{F}$ and $G\cap \pfilter{x} \supseteq F \cap\pfilter{x} \in \mathfrak{F}$, so $G \to x$.

The Kent property is immediate.
\end{proof}

\begin{lemma}
Every convergent filter in the filter convergence is a merotopic filter.
\end{lemma}
\begin{proof}
If $F\in \powerfilters(X)$ converges to some $x\in X$, then $F\cap \pfilter{x}$ is in the Cauchy structure and thus so is $F$ by upwards closure.
\end{proof}

\begin{lemma}
Let $\sSet{X, \mathfrak{F}}$ be a filter space and $F,G\in \mathfrak{F}$. Then $F\sim G$ \textup{if and only if} $\{F, G\}$ has a lower bound in $\mathfrak{F}$.
\end{lemma}

\begin{lemma}
Let $\sSet{X,\xi}$ be a convergence space. The filter convergence on the filter space $\sSet{X,\mathfrak{F}}$ is the dual Kent closure of $\xi$.
\end{lemma}

\begin{lemma} \label{CauchyEquivalence}
Let $\sSet{X, \mathfrak{F}}$ be a filter space. Then the following are equivalent
\begin{enumerate}
\item $X$ is a Cauchy space;
\item $F\amesh G$ implies $F\sim G$ for all $F,G\in \mathfrak{F}$;
\item the relation $\sim$ is transitive, when restricted to proper filters;
\item the relation $\sim$ is an equivalence relation, when restricted to proper filters.
\end{enumerate}
\end{lemma}
\begin{proof}
$(1) \Leftrightarrow (2)$ Rephrasing of the definition in terms of $\sim$.

$(2) \Rightarrow (3)$ Suppose $F\sim G$ and $G\sim H$. Then $F\cap G, G\cap H\in\mathfrak{F}$. Since $(F\cap G)\vee (G\cap H) \subseteq G$, we have $(F\cap G)\amesh (G\cap H)$ by \ref{joinProperFilter}. By assumption, this means $F\cap G\cap H \in \mathfrak{F}$, so $F\cap H\in \mathfrak{F}$ by upwards closure and thus $F\sim H$.

$(3) \Rightarrow (2)$ Take $F,G\in\mathfrak{F}$ and suppose $F\amesh G$. By \ref{joinProperFilter} $F\vee G$ is proper. We have $F = F\cap (F\vee G)$ and $G = G\cap (F\vee G)$, so $F \sim (F\vee G)$ and $G\sim (F\vee G)$. By transitivity we have that $F\sim (F\vee G)\sim G$ implies $F\sim G$.

$(3) \Leftrightarrow (4)$ The relation $\sim$ is automatically symmetric and reflexive.
\end{proof}

\begin{lemma}
Let $\sSet{X, \mathfrak{F}}$ be a filter space.
\begin{enumerate}
\item If $X$ is a Cauchy space, then the filter convergence is $R_1$ and of finite depth.
\item If $X$ is complete, then the converse also holds.
\end{enumerate}
\end{lemma}
\begin{proof}
(1) We will use the fact that $\sim$ is transitive in Cauchy spaces, \ref{CauchyEquivalence}.

We first show finite depth. Suppose $F,G\to x$. Then $F\sim \pfilter{x}$ and $\pfilter{x} \sim G$, so $F\sim G$, or $F\cap G\in\mathfrak{F}$. Since $F\cap G = F\cap (F\cap G)$, we have $F\cap G\sim F\sim \pfilter{x}$, so $(F\cap G)\to x$.

To show $R_1$, take $x,y\in X$ and suppose $\lim^{-1}(x)\mesh \lim^{-1}(y)$, so there exists $G\in \lim^{-1}(x)\cap \lim^{-1}(y)$. Then for all $F\in \powerfilters(X)$, we have $F\sim F\cap G \sim G\sim \pfilter{x}$ and, similarly, $F\sim G\sim \pfilter{y}$, so $F\in \lim^{-1}(x)\cap \lim^{-1}(y)$, which implies $R_1$ by \ref{R1Conditions}.

(2) Take $F,G \in\mathfrak{F}$ such that $F\amesh G$. By completeness, there exist $x,y\in X$ such that $F\to x$ and $G\to y$. By upwards closure $F\vee G \to x$ and $F\vee G\to y$. By \ref{R1Conditions}, this means that $\lim^{-1}(x) = \lim^{-1}(y)$, so $G\to x$. By finite depth, we have $F\cap G \to x$, so $F\cap G\in \mathfrak{F}$.
\end{proof}

\begin{proposition}
Let $\sSet{X,\mathfrak{F}}$ and $\sSet{Y,\mathfrak{G}}$ be filter spaces and $f: X\to Y$ a function. If $f$ is filter continuous, then $f$ is continuous when $X$ and $Y$ are equipped with their filter convergences.
\end{proposition}
\begin{proof}
Assume $f$ is filter continuous and suppose $F\to x\in X$, i.e.\ $F\cap \pfilter{x}\in \mathfrak{F}$. Then by \ref{imageUpsetsPreservesIntersection} and filter continuity, $\upset f^{\imf\imf}(F\cap \pfilter{x}) = \upset f^{\imf\imf}(F) \cap \pfilter{f}(x) \in \mathfrak{G}$, so $\upset f^{\imf\imf}(F) \to f(x)$.
\end{proof}

\subsection{Initial and final filter structures}
\subsubsection{Product filter structures}
\begin{lemma}
Let $\sSet{X, \mathfrak{F}}$ and $\sSet{Y, \mathfrak{G}}$ be filter spaces. Then the product filter structure on $X\times Y$ is given by
\[ \upset \setbuilder{F\otimes G}{F\in \mathfrak{F}, G\in \mathfrak{G}}. \]
\end{lemma}

\subsection{Completeness}
\begin{definition}
Let $\sSet{X, \mathfrak{C}}$ be a Cauchy space. We call $X$ \udef{Cauchy complete} (or just \udef{complete}) if every $F\in \mathfrak{C}$ converges in the Cauchy convergence.
\end{definition}

\begin{proposition}
\begin{enumerate}
\item Each closed subspace of a complete Cauchy space is complete.
\item A subspace of a complete Hausdorff Cauchy space is complete \textup{if and only if} it is closed.
\item The product of complete Cauchy spaces is complete (TODO products!).
\item Each compact uniform convergence is complete.
\end{enumerate}
\end{proposition}
\begin{proof}
TODO
\end{proof}

\subsubsection{Completion}
\begin{definition}
Let $\sSet{X, \mathfrak{C}}$ be a Cauchy space. A \udef{completion} of $\sSet{X, \mathfrak{C}}$ is a complete Cauchy space $\sSet{Y, \mathfrak{D}}$ and a function $k: X\to Y$ such that
\begin{itemize}
\item $k: X\to Y$ is a Cauchy embedding;
\item $\im(k)$ is dense in $Y$.
\end{itemize}
The completion is called \udef{strict} if $\im(k)$ is strictly dense in $Y$.
\end{definition}

Let $\hat{X}$ be the set of equivalence classes of proper Cauchy filters (which exist since the Cauchy structure if of finite depth, \ref{CauchyEquivalence}) and $k: X\to \hat{X}: x\mapsto [\pfilter{x}]$. Since $X$ is Hausdorff, $k$ is injective.
Define
\[ \Lambda_\mathfrak{C} \defeq \setbuilder{\lambda\in \big(\hat{X} \to \powerfilters(X)\big)}{\big(\forall p\in \hat{X}: \exists F\in p: \lambda(p) \subseteq F \in p\big) \land \big((\lambda\circ k)(x) = \pfilter{x}\big)}. \]

For all $\lambda \in \Lambda_\mathfrak{C}$ and $A\subseteq X$, define
\[ A^\lambda \defeq \setbuilder{p\in \hat{X}}{A\in \lambda(p)}. \]
For all $F\in \powerfilters(X)$, define $F^\lambda \defeq \upset \big((-)^\lambda\big)^\imf(F)$. This is a filter by \ref{imageFilter} (using the fact that $(-)^\lambda$ is order preserving).

Now define a Cauchy structure $C_\lambda$ on $\hat{X}$ as follows
\[ C_\lambda \defeq \upset\setbuilder{F^\lambda \cap \pfilter{p}}{F\in p}. \]

\begin{lemma} \label{hausdorffCauchyCompletionLemma}
Let $\sSet[\big]{X, \mathfrak{C}}$ be a Hausdorff Cauchy space, $A,B\subseteq X$, $F, G\in \powerfilters(X)$ and $\lambda \in \Lambda_\mathfrak{C}$. Then
\begin{enumerate}
\item $A \subseteq B \iff k^\imf(A)\subseteq B^\lambda$;
\item $F\subseteq G \iff F^\lambda \subseteq \upset k^{\imf\imf}(G)$;
\item $(A \cap B)^\lambda = A^\lambda \cap B^{\lambda}$;
\item $(F\vee G)^\lambda = F^\lambda \vee G^\lambda$;
\item $A^\lambda = \emptyset \iff A = \emptyset$;
\item $F^\lambda\amesh G^\lambda \iff F\amesh G$;
\item $F$ is proper \textup{if and only if} $F^\lambda$ is proper.
\end{enumerate}
\end{lemma}
By point (3), we can understand $C_\lambda$ as the set of filters $F^\lambda$, where $F$ is a Cauchy filter that is comparable with some $\lambda(p)$.
\begin{proof}
(1) We calculate
\begin{align*}
A \subseteq B &\iff \forall x\in A: \; x\in B \\
&\iff \forall x\in A: \; B\in \pfilter{x} = \lambda\big([\pfilter{x}]\big) \\
&\iff \forall x\in A: \; k(x) = [\pfilter{x}] \in B^\lambda \\
&\iff k^\imf(A) \subseteq B^\lambda.
\end{align*}

(2) Immediate from (1).

(3) We calculate
\begin{align*}
p \in (A\cap B)^\lambda &\iff A\cap B \in \lambda(p) \\
&\iff \big(A \in \lambda(p)\big) \land \big(B \in \lambda(p)\big) \\
&\iff \big(p \in A^\lambda\big) \land \big(p\in B^\lambda\big) \\
&\iff p \in A^\lambda \cap B^\lambda.
\end{align*}

(4) Immediate from (3) and \ref{baseMeetJoinFilters}.

(5) We prove the contrapositive. First suppose $A \neq \emptyset$, so there exists some $x\in A$. Then $A\in \pfilter{x} = \lambda\big([\pfilter{x}]\big)$, so $[\pfilter{x}] \in A^\lambda$, which means that $A^\lambda \neq \emptyset$.

Now suppose $A^\lambda \neq \emptyset$, so there exists some $p\in A^\lambda$. We have $A\in \lambda(p)$. Since $p$ only contains proper filters, we have $\lambda(p) \neq \powerset(X)$, so $\emptyset \notin \lambda(p)$. This implies $A \neq \emptyset$.

(6) Immediate from (4) and \ref{joinProperFilter}.

(7) Immediate from (6) and \ref{properFiltersSelfMesh}.
\end{proof}

\begin{lemma} \label{ReedMapToCauchyFilter}
Let $\sSet[\big]{X, \mathfrak{C}}$ be a Hausdorff Cauchy space, $F \in \powerfilters(X)$, $p\in \hat{X}$ and $\lambda \in \Lambda_\mathfrak{C}$. Then
\begin{enumerate}
\item $\lambda(p)^\lambda \subseteq \pfilter{p}$;
\item if $\lambda(p)\in \mathfrak{C}$ for all $p$, then $C_\lambda = \upset \setbuilder{F^\lambda}{F\in \mathfrak{C}}$.
\end{enumerate}
\end{lemma}
\begin{proof}
(1) For all $A\in \lambda(p)$, we have $p\in A^\lambda$.

(2) It is clear that $F^\lambda\in C_\lambda$ for all $F\in \mathfrak{C}$. For the converse, take $G^\lambda \cap \pfilter{p}$, where $G\in p$. Then, by \ref{hausdorffCauchyCompletionLemma} and (1),
\[ \big(G \cap \lambda(p)\big)^\lambda = G^\lambda \cap \lambda(p)^\lambda \subseteq G^\lambda \cap \pfilter{p}. \]
Since $G \cap \lambda(p)$ is a Cauchy filter, we have $G^\lambda \cap \pfilter{p} \in \upset \setbuilder{F^\lambda}{F\in \mathfrak{C}}$.
\end{proof}

\begin{lemma} \label{mappingFiltersLemmaCauchyCompletion}
Let $\sSet[\big]{X, \mathfrak{C}}$ be a Hausdorff Cauchy space, $F \in \powerfilters(X)$, $p\in \hat{X}$ and $\lambda \in \Lambda_\mathfrak{C}$. Then
\begin{enumerate}
\item $F\in p$ implies $\upset k^{\imf\imf}(F) \cap \pfilter{p} \in C_\lambda$;
\item $\upset k^{\imf\imf}(F) \in C_\lambda$ implies $F\in\mathfrak{C}$.
\end{enumerate}
\end{lemma}
\begin{proof}
(1) Suppose $F\in p$. By construction, $F^\lambda\cap\pfilter{p}\in C_\lambda$. By \ref{hausdorffCauchyCompletionLemma}, $F\subseteq F$ implies $F^\lambda \subseteq \upset k^{\imf\imf}(F)$, so $F^\lambda\cap\pfilter{p} \subseteq \upset k^{\imf\imf}(F)\cap \pfilter{p}$ and thus $\upset k^{\imf\imf}(F)\cap\pfilter{p} \in C_\lambda$ by upwards closure.

(2) Suppose $\upset k^{\imf\imf}(F)\in C_\lambda$. Then there exists $q\in \hat{X}$ and $G\in q$ such that $G^\lambda \cap \pfilter{q} \subseteq \upset k^{\imf\imf}(F)$.

First suppose $q\notin \im(k)$. Then for all $B \in G$, we have $B^\lambda \in \upset k^{\imf\imf}(F)$ iff $B^\lambda \cup \{q\}\in \upset k^{\imf\imf}(F)$, so $G^\lambda \subseteq \upset k^{\imf\imf}(F)$ and thus $G\subseteq F$ by \ref{hausdorffCauchyCompletionLemma}. This means that $F\in \mathfrak{C}$ by upwards closure.

Next suppose $q = k(x) = [\pfilter{x}]$ for some $x\in X$. Then, by monotonicity of $(-)^\lambda$ and \ref{hausdorffCauchyCompletionLemma}, we have
\[ (G\cap \pfilter{x})^\lambda \subseteq G^\lambda \cap \pfilter{x}^\lambda \subseteq G^\lambda \cap \pfilter{k}(x) = G^\lambda \cap \pfilter{q} \subseteq \upset k^{\imf\imf}(F). \]
Again by \ref{hausdorffCauchyCompletionLemma}, this implies $G\cap \pfilter{x} \subseteq F$. Since $G\in q$ and $\pfilter{x}\in q$, we have $G\sim \pfilter{x}$, so $G\cap \pfilter{x}\in \mathfrak{C}$ and thus $F\in \mathfrak{C}$ by upwards closure.
\end{proof}

\begin{proposition}
Let $\sSet[\big]{X, \mathfrak{C}}$ be a Hausdorff Cauchy space and $\lambda \in \Lambda_\mathfrak{C}$. Then $\sSet[\big]{\hat{X}, C_\lambda}$ is a complete Hausdorff Cauchy space.
\end{proposition}
\begin{proof}
Since no equivalence class $p\in \hat{X}$ is empty, there exists some $F\in p$ for all $p\in \hat{X}$. Since $F^\lambda\cap \pfilter{p}\in C_\lambda$, this implies $\pfilter{p}\in C_\lambda$ by upwards closure. Upwards closure is immediate by definition.

Completeness also follows from the existence of some $F$ such that $F^\lambda\cap \pfilter{p}\in C_\lambda$.

For the Cauchy property, suppose $F \in p$ and $G\in q$ are such that $\big(F^\lambda \cap \pfilter{p}\big) \amesh \big(G^\lambda \cap \pfilter{q}\big)$, then, by \ref{subsetsFilterGrillIntersectionPrime}, at least one of the following holds: $\pfilter{p}\amesh \pfilter{q}$, $F^\lambda \amesh G^\lambda$, $F^\lambda \amesh \pfilter{q}$ or $\pfilter{p} \amesh G^\lambda$. We will show that each of these implies $F\sim G$ (or, equivalently, $p=q$), so $F\cap G\in p$ and thus
\[ \big(F^\lambda \cap \pfilter{p}\big) \cap \big(G^\lambda \cap \pfilter{q}\big) = \big(F^\lambda \cap G^\lambda\big) \cap \pfilter{p} \supseteq (F\cap G)^\lambda \cap \pfilter{p} \in C_\lambda, \]
where the inclusion follows by monotonicity of $(-)^\lambda$.
\begin{itemize}
\item It is clear that $\pfilter{p}\amesh \pfilter{q}$ in particular implies $\{p\}\mesh \{q\}$, so $p=q$. 
\item $F^\lambda \amesh G^\lambda$ implies that $F^\lambda \vee G^\lambda = (F\vee G)^\lambda$ is proper by \ref{joinProperFilter} and \ref{hausdorffCauchyCompletionLemma}, so $F\vee G$ is proper and thus $F\amesh G$ by the same results. By finite depth of $X$ we have $F\cap G\in \mathfrak{C}$ and thus $F\sim G$.
\item $F^\lambda \amesh \pfilter{q}$ implies that $q\in A^\lambda$ for all $A\in F$, so $A\in \lambda(q)$. Thus $F\subseteq \lambda(q)$. Also $\lambda(q)\subseteq G'$ for some $G'\in q$, so $F\sim \lambda(q) \sim G' \sim G$.
\item If $\pfilter{p} \amesh G^\lambda$, then a similar argument for $F\sim G$ holds.
\end{itemize}
Finally, for Hausdorffness, let $G\in \powerfilters(X)$ be a convergent filter that converges to $q\in \hat{X}$, so $G\cap \pfilter{q} \in C_\lambda$. Then there exists $p\in \hat{X}$ and $F\in p$ such that $F^\lambda \cap \pfilter{p} \subseteq G\cap \pfilter{q} \subseteq \pfilter{q}$. Thus $(F^\lambda \cap \pfilter{p})\amesh \pfilter{q}$. By \ref{subsetsFilterGrillIntersectionPrime}, we have either $F^\lambda \amesh \pfilter{q}$ or $\pfilter{p}\amesh \pfilter{q}$. We have already argued that $p = q$ in both cases. Thus $G$ can only have one limit, namely $p$.
\end{proof}

\begin{lemma} \label{ReedCompletionAdherenceLemma}
Let $\sSet[\big]{X, \mathfrak{C}}$ be a Hausdorff Cauchy space, $A\subseteq X$ and $\lambda \in \Lambda_\mathfrak{C}$. Then $\adh_{C_\lambda}\big(A^\lambda\big) = \adh_{C_\lambda}\big(k^{\imf}(A)\big)$.
\end{lemma}
\begin{proof}
We prove both inclusions. First take $p\in \adh_{C_\lambda}\big(A^\lambda\big)$. Then there exists $H \to p$ that contains $A^\lambda$, so there exists $G\in p$ such that $G^\lambda \cap \pfilter{p}\subseteq H$. By \ref{properFiltersSelfMesh}, \ref{antitonicityPolars} and \ref{grillIntersectionUnion}, we have
\[ A^\lambda \in H^\mesh \subseteq (G^\lambda \cap \pfilter{p})^\mesh = (G^\lambda)^\mesh \cup \pfilter{p}^\mesh. \]
So either $A^\lambda \in \pfilter{p}^\mesh$ or $A^\lambda \in (G^\lambda)^\mesh$. In the first case, we have $p\in A^\lambda$, so $A\in\lambda(p)\subseteq F$, where $F$ is some Cauchy filter in $p$. By \ref{mappingFiltersLemmaCauchyCompletion}, we have $k^{\imf}(A)\in \upset k^{\imf\imf}(F) \to p$, so $p\in \adh_{C_\lambda}\big(k^{\imf}(A)\big)$.

In the second case, since $\upset \{A^\lambda\} = \upset \{A\}^\lambda$, we have $\{A\}^\lambda \amesh G^\lambda$ and thus $\{A\}\amesh G$ by \ref{hausdorffCauchyCompletionLemma}. Then $G|_A$ is proper and $G|_A \sim G$. This implies that $\upset k^{\imf\imf}(G|_A) \to p$, by \ref{mappingFiltersLemmaCauchyCompletion} and, since $k^\imf(A)\in \upset k^{\imf\imf}(G|_A)$, we have $p\in \adh_{C_\lambda}\big(k^{\imf}(A)\big)$.

For the other inclusion, it is enough to note that $k^\imf(A) \subseteq A^\lambda$ by \ref{hausdorffCauchyCompletionLemma}, so $\adh_{C_\lambda}\big(A^\lambda\big) \subseteq \adh_{C_\lambda}\big(k^{\imf}(A)\big)$ by \ref{principalInherenceAdherenceProperties}.
\end{proof}

\begin{proposition}
Let $\sSet[\big]{X, \mathfrak{C}}$ be a Hausdorff Cauchy spacedepth and $\lambda \in \Lambda_\mathfrak{C}$. Then
\begin{enumerate}
\item $\sSet[\big]{\sSet[\big]{\hat{X}, C_\lambda}, k}$ is a completion of $X$;
\item if $\lambda(p)$ is a Cauchy filter for all $p\in\hat{X}$, then the completion is a strict completion.
\end{enumerate}
\end{proposition}
TODO always strict completion??????????
\begin{proof}
(1) We first show that $k$ is an embedding. We have already remarked that $k$ is injective. Take $F\in \mathfrak{F}$. There exists $p\in \hat{X}$ such that $F\in p$. By \ref{mappingFiltersLemmaCauchyCompletion}, $\upset k^{\imf\imf}(F)\cap\pfilter{p}\in C_\lambda$ and so $\upset k^{\imf\imf}(F)\in C_\lambda$.

If $\upset k^{\imf\imf}(F)\in C_\lambda$, then $F\in\mathfrak{C}$ by \ref{mappingFiltersLemmaCauchyCompletion}.

Finally we show that $\im(k)$ is dense in $\hat{X}$. Take $p\in \hat{X}$. Then we can find $F\in p$, so $\upset k^{\imf\imf}(F)\cap\pfilter{p}\in C_\lambda$ by \ref{mappingFiltersLemmaCauchyCompletion}. Thus $\upset k^{\imf\imf}(F) \overset{C_\lambda}{\longrightarrow} p$.

(2) We need to show that $\im(k)$ is strictly dense in $\hat{X}$. Take arbitrary $H\to p\in\hat{X}$. By \ref{ReedMapToCauchyFilter}, there $F\in \mathfrak{C}$ such that $F^\lambda \subseteq H$. Now $\upset k^{\imf\imf}(F)$ witnesses the strict density of $\im(k)$. Indeed, $\im(k) \in \upset k^{\imf\imf}(F)$, $\upset k^{\imf\imf}(F) \to p$ and, by \ref{ReedCompletionAdherenceLemma},
\[ \adh_{C_\lambda}^\imf\big(\upset k^{\imf\imf}(F)\big) = \adh_{C_\lambda}^\imf(F^\lambda) \subseteq F^\lambda \subseteq H. \]
\end{proof}


\begin{proposition} \label{extensionCauchyContinuousMapToCompletion}
Let $\sSet{X, \mathfrak{C}}$ be a Hausdorff Cauchy space, $\lambda \in \Lambda_\mathfrak{C}$ and $\sSet{Y,\mathfrak{D}}$ a complete regular Hausdorff Cauchy space. Let $f: X\to Y$ be a Cauchy continuous map. Then $f$ can be extended to a unique Cauchy continuous map $\hat{f}: \hat{X}\to Y$.
\end{proposition}
By ``$\hat{f}$ extends $f$'' we mean that $f = \hat{f}\circ k$.

We need regularity to show that $\hat{f}$ is Cauchy continuous. Note that we really do only need regularity, not uniform regularity.
\begin{proof}
For all $[F]\in \hat{X}$, we define $\hat{f}\big([F]\big) = \lim \upset f^{\imf\imf}(F)$. This is well-defined: if $F\sim G\in \powerfilters(X)$, then $\upset f^{\imf\imf}(F) \sim \upset f^{\imf\imf}(G)$ since $f$ is uniform relation preserving by \ref{CauchyContinuousUniformRelationPreserving}. Thus $\upset f^{\imf\imf}(F)$ and $\upset f^{\imf\imf}(G)$ have the same (unique, by Hausdorff) limit.

The function $\hat{f}$ extends $f$: since, by definition $\hat{f}\big([\pfilter{x}]\big) = \lim \upset f^{\imf\imf}(\pfilter{x})$, we have
\[ (\hat{f}\circ k)(x) = \hat{f}\big([\pfilter{x}]\big) = \lim \upset f^{\imf\imf}(\pfilter{x}) = \lim \pfilter{f}(x) = f(x). \]

To show unicity, suppose $g: \hat{X}\to Y$ is some other Cauchy continuous map that extends $f$. Take arbitrary $p\in \hat{X}$ and take $F\in p$. Then
\[ g(p) = \lim \upset g^{\imf\imf}(\pfilter{p}) = \lim \upset g^{\imf\imf}\big(\upset k^{\imf\imf}(F)\big) = \lim\upset (g\circ k)^{\imf\imf}(F) = \lim\upset f^{\imf\imf}(F) = \hat{f}(p). \]

Finally we show that $\hat{f}$ is Cauchy continuous. For this we need regularity of $Y$.
It is enough to show that $\upset \hat{f}^{\imf\imf}(F^\lambda \cap \pfilter{p}) \in \mathfrak{D}$ for all $p\in \hat{X}$ and $F\in p$. First note that, by definition, $\upset f^{\imf\imf}(F) \to \hat{f}(p)$. By regularity, we have $\adh^\imf\big(\upset f^{\imf\imf}(F)\big) \to \hat{f}(p)$.

Now we show that $\adh^{\imf}(\upset f^{\imf\imf}(F)) \subseteq \upset\hat{f}^{\imf\imf}(F^\lambda)$, for which it is enough to show $\hat{f}^\imf(A^\lambda) \subseteq \adh(f^\imf(A))$ for all $A\in F$. Indeed, take $p\in A^\lambda$ (so $A \in \lambda(p)$ and $\lambda(p) \subseteq G$ for some $G\in p$), then $\hat{f}(p) = \lim \upset f^{\imf\imf}(G)$. Since $f^\imf(A) \in f^{\imf\imf}(G)$, we have $\hat{f}(p) = \lim \upset f^{\imf\imf}(G) \in \adh\big(f^\imf(A)\big)$.

By monotonicity of the convergence, we have $\upset\hat{f}^{\imf\imf}(F^\lambda) \to \hat{f}(p)$. Thus
\[ \mathfrak{D} \quad\ni\quad \upset\hat{f}^{\imf\imf}(F^\lambda) \cap \pfilter{\hat{f}}(p) \; = \; \upset\hat{f}^{\imf\imf}(F^\lambda) \cap \upset \hat{f}^{\imf\imf}(\pfilter{p}) \; = \; \upset\hat{f}^{\imf\imf}(F^\lambda \cap \pfilter{p}), \]
using \ref{imageUpsetsPreservesIntersection}.
\end{proof}


\subsubsection{Wyler completion}
\begin{lemma}
Let $\sSet{X, \mathfrak{C}}$ be a Hausdorff Cauchy space. Consider the function
\[ \gamma: \hat{X} \to \powerfilters{X}: p \mapsto \begin{cases}
\pfilter{x} & (\pfilter{x}\in p) \\
\{X\} & (\text{otherwise}).
\end{cases} \]
Then $\gamma \in \Lambda_\mathfrak{C}$.
\end{lemma}
\begin{definition}
The completion of $\sSet{X, \mathfrak{C}}$ determined by $\gamma$ is called the \udef{Wyler completion} of $X$.
\end{definition}
\begin{lemma}
Let $\sSet{X, \mathfrak{C}}$ be a Hausdorff Cauchy space. The Cauchy structure $\mathfrak{C}_W$ of the Wyler completion is given by
\[ \mathfrak{C}_W = \upset \setbuilder{\upset k^{\imf\imf}(F) \cap \pfilter{p}}{F\in p}.  \]
\end{lemma}
\begin{proof}
We show that $F^\gamma = \upset k^{\imf\imf}(F)$ for all $F\in \powerfilters(X)\setminus \{X\}$. The equality in particular holds for all Cauchy filters, except if $\{X\}$ is a Cauchy filter. In this last case $X$ must be a singleton $\{x\}$ in order for the Cauchy structure to be Hausdorff. Thus the equality still holds for all Cauchy filters, since
\[ \big\{\{x\}\big\}^\gamma = \Big\{\big\{\{x\}\big\}\Big\} = k^{\imf\imf}\Big(\big\{\{x\}\big\}\Big). \]

For the other cases, it is enough to show $A^\gamma = k^\imf(A)$ for all $A\subsetneq X$. This is equivalent to $A\in \gamma(p) \iff p\in k^{\imf}(A)$ for all $p\in \hat{X}$. To show the equivalence, first support $\pfilter{x} \in p$ for some $x\in X$. Then
\[ A\in \gamma(p) \iff A\in \pfilter{x} \iff x\in A \iff p = k(x) = [\pfilter{x}] \in k^\imf(A), \]
where the last equivalence follows from the injectivity of $k$.

Now suppose there is no $x\in X$ such that $\pfilter{x} \in p$. Then $p\notin \im(k)$, so $p\notin k^\imf(A)$. By definition of $\gamma$, $A\in \gamma(p) = \{X\}$ is true iff $A = X$, which was excluded. Thus $A\in \gamma(p) \iff p\in k^{\imf}(A)$, since both statements are false.
\end{proof}

\begin{lemma}
Let $\sSet{X, \mathfrak{C}}$ be a Hausdorff Cauchy space. The only non-trivial Cauchy filters in $\mathfrak{C}_W$ that contain $\hat{X}\setminus \im(k)$ are the principal ultrafilters.
\end{lemma}
\begin{proof}
Take non-trivial $G\in \mathfrak{C}_W$ and suppose $\hat{X}\setminus \im(k)\in G$. Then there exists $F\in\mathfrak{C}$ and $p\in \hat{X}$ such that $F\in p$ and $\upset k^{\imf\imf}(F)\cap \pfilter{p}\subseteq G$. Then $X\in F$, so $\im(k)\cup\{p\} \in G$. By assupmtion, $\{p\} = \big(\hat{X}\setminus \im(k)\big)\cap \im(k)\cup\{p\} \in G$, so $G = \pfilter{p}$ because $G$ was assumed non-trivial.
\end{proof}

\begin{proposition}[Universal propery of the Wyler completion]
Let $\sSet{X, \mathfrak{C}}$ be a Hausdorff Cauchy space and let $\sSet[\big]{\sSet{Y,\mathfrak{D}}, k': X\to Y}$ be a completion of $X$. Then there exist a unique Cauchy continuous function $g: \hat{X} \to Y$ such that
\[ \begin{tikzcd}[column sep=2.4em, row sep=1.8em]
X \ar[r, "k"] \ar[dr, swap, "{k'}"] & \hat{X} \ar[d, dashed, "g"] \\
{} & Y
\end{tikzcd} \qquad \text{commutes.} \]
\end{proposition}
\begin{proof}
TODO (?)
\end{proof}

\begin{proposition}
Let $\sSet{X, \mathfrak{C}}$ be a Hausdorff Cauchy space and $\sSet{Y,\mathfrak{D}}$ a complete Hausdorff Cauchy of finite depth. Let $f: X\to Y$ be a Cauchy continuous map. Then $f$ can be extended to a unique Cauchy continuous map $\hat{f}: \sSet{\hat{X}, \mathfrak{C}_W} \to Y$.
\end{proposition}
Note the similarty with \ref{extensionCauchyContinuousMapToCompletion}. In this case we do not need to assume that $Y$ is regular.
\begin{proof}
We construct $\hat{f}$ in the same way as \ref{extensionCauchyContinuousMapToCompletion}. We just need to show Cauchy continuity.

Take $G\in \mathfrak{C}_W$. Then there exists $F\in \mathfrak{C}$ and $p\in\hat{X}$ such that $F\in p$ and $\upset k^{\imf\imf}(F)\cap \pfilter{p} \subseteq G$.
Then
\[ \upset \hat{f}^{\imf\imf}(G) \supseteq \upset (\hat{f}\circ k)^{\imf\imf}(F) \cap \hat{f}^{\imf\imf}(\pfilter{p}) = \upset f^{\imf\imf}(F) \cap \pfilter{\hat{f}}(p). \]
Now $\upset f^{\imf\imf}(F)$ is a Cauchy filter because $f$ is Cauchy continuous and it converges to $\hat{f}(p)$ by definition of $\hat{f}$. This implies that $\upset \hat{f}^{\imf\imf}(G)$ is a Cauchy filter and thus that $\hat{f}$ is Cauchy continuous.
\end{proof}

\begin{lemma}
Let $\sSet{X, \mathfrak{C}}, \sSet{A, \mathfrak{D}}$ be Hausdorff Cauchy spaces of finite depth and $f: A\to X$ a Cauchy embedding. Then $\widehat{k\circ f}: \hat{A}\to \hat{X}$ is injective.
\end{lemma}
\begin{proof}
Take $p,q\in \hat{A}$. Suppose $\widehat{k\circ f}(p) = \widehat{k\circ f}(q)$. This means that for $F\in p, G\in q$, we have
\[ \upset (k\circ f)^{\imf\imf}(F\cap G) = \upset (k\circ f)^{\imf\imf}(F)\cap \upset (k\circ f)^{\imf\imf}(G) \in \mathfrak{C}_W. \]
Since both $k$ and $F$ are Cauchy embeddings, this implies that $F\cap G\in \mathfrak{D}$, so $F\sim G$ and thus $p=q$.
\end{proof}

\begin{proposition}
The Wyler completion preserves the pseudotopological, pretopological or topological property of the given Cauchy space.
\end{proposition}

\subsubsection{Kowalsky completion}
\begin{definition}
Let $\sSet{X, \mathfrak{C}}$ be a Hausdorff Cauchy space. Define
\[ \Sigma = \setbuilder{\lambda \in \Lambda_\mathfrak{C}}{\forall p\in \hat{X}: \; \lambda(p) \in p} \]
Consider the Cauchy structure $\mathfrak{C}_K \defeq\bigwedge_{\lambda \in \Sigma}C_\lambda$. Then the completion $\sSet[\big]{\sSet[]{\hat{X}, \mathfrak{C}_K}, k}$ is called the \udef{Kowalsky completion} of $X$.
\end{definition}

For $A\subseteq X$, set
\[ A^\Sigma \defeq \bigcup_{\lambda\in \Sigma}A^\lambda \]
and $F^\Sigma \defeq \upset\setbuilder{A^\Sigma}{A\in F}$.

\begin{proposition}
The Kowalsky completion preserves uniformizability and the pseudotopological, pretopological or topological property of the given Cauchy space.
\end{proposition}

For $A,B\subseteq X$, we define
\[ A <^\lambda B \quad \defequiv \quad \forall p\in \hat{X}: \; A\in\bigcup p \implies B\in\lambda(p). \]
For $F\in\powerfilters{X}$, we define
\[ s_\lambda(F) \defeq \setbuilder{B\subseteq X}{A <^\lambda B}. \]

\begin{proposition}
Let $\sSet{X, \mathfrak{C}}$ be a Hausdorff Cauchy space. Then the Kowalsky completion is regular \textup{if and only if} $F\in \mathfrak{C} \implies s_\lambda(F) \in \mathfrak{C}$.
\end{proposition}
\begin{proof}
TODO: ON THE REGULARITY OF THE KOWALSKY
COMPLETION
\end{proof}

\section{Closure spaces}
\url{https://www.researchgate.net/profile/Peter-Stadler-2/publication/239066337_Higher_Separation_Axioms_in_Generalized_Closure_Spaces/links/53d1cf440cf2a7fbb2e95303/Higher-Separation-Axioms-in-Generalized-Closure-Spaces.pdf?origin=publication_detail}

\section{Merotopic and nearness spaces}
\url{https://en.wikipedia.org/wiki/Proximity_space}

\section{Bornological spaces}
\begin{definition}
Let $X$ be a set. A set $\mathcal{B}\subseteq \powerset(X)$ of subsets of $X$ is called a \udef{bornology} if
\begin{itemize}
\item it is an ideal;
\item it covers $V$.
\end{itemize}
A pair $\sSet{X,\mathcal{B}}$, where $X$ is a set and $\mathcal{B}$ is a bornology on $X$ is called a \udef{bornological space}.

The set $c\mathcal{B}\defeq \setbuilder{A^c}{A\in\mathcal{B}}$ is called the \udef{cobornology}.
\end{definition}

\begin{lemma} \label{bornologyCoveringLemma}
Let $X$ be a set and $\mathcal{B}\subseteq \powerset(X)$ an ideal of subsets of $X$. Then $\mathcal{B}$ covers $X$ \textup{if and only if} $\{x\}\in \mathcal{B}$ for all $x\in X$.
\end{lemma}

\begin{example}
Let $\sSet{X,\xi}$ be a convergence space. The set of compactoid subsets forms a bornology.
\end{example}

\begin{lemma}
Let $X$ be a set. A set $c\mathcal{B}\subseteq \powerset(X)$ of subsets of $X$ is a cobornology \textup{if and only if}
\begin{itemize}
\item $c\mathcal{B}$ is a filter;
\item $(X)_0 \subseteq c\mathcal{B}$.
\end{itemize}
\end{lemma}
\begin{proof}
The set $c\mathcal{B}$ is a filter iff $\mathcal{B}\defeq \setbuilder{A^c}{A\in c\mathcal{B}}$ is an ideal by \ref{filterIdealDuality}.

TODO link with cofinite filter using \ref{bornologyCoveringLemma}.
\end{proof}


\subsection{Bounded maps}
\begin{definition}
Let $\sSet{X, \mathcal{B}}$, $\sSet{Y,\mathcal{C}}$ be bornological spaces. A function $f: X\to Y$ is called \udef{bounded} if $f^{\imf\imf}(\mathcal{B}) \subseteq \mathcal{C}$.
\end{definition}


\chapter{Uniform convergence}
\section{Uniformities}
\subsection{Operations on filters}
\begin{definition}
Let $X$ be a set, $x\in X$ and $F,G\in\powerfilters(X^2)$. We define
\begin{enumerate}
\item $F^{\transp} \defeq \setbuilder{A^\transp}{A\in F}$;
\item $Fx \defeq \setbuilder{Ax}{A\in F}$;
\item $F;G \defeq \upset \setbuilder{A;B}{A\in F, B\in G}$.
\end{enumerate}
We call a filter $H\in\powerfilters(X^2)$ \udef{diagonal} if $H\subseteq \upset \{\id_X\}$.
\end{definition}

We have $F^{\transp} = t^{\imf\imf}[F]$ where $t: X^2 \to X^2: (x,y)\mapsto (y,x)$.

TODO Galois connection $\powerfilters(X^2) \leftrightarrow \powerfilters(X)^2$.

\begin{lemma} \label{filterOperationsOnRelationFilters}
Let $X$ be a set, $x\in X$ and $F,G\in\powerfilters(X^2)$. Then
\begin{enumerate}
\item $F^\transp$, $Fx$ and $F;G$ are filters;
\item $F^\transp$ is proper \textup{if and only if} $F^\transp$ is proper;
\item $F;G$ is proper \textup{if and only if} $p_2^{\imf\imf}[F]\amesh p_1^{\imf\imf}[G]$;
\item $Fx$ is proper \textup{if and only if} $x\in\ker\big(p_2^{\imf\imf}(F)\big)$.
\end{enumerate}
\end{lemma}
\begin{proof}
(1) We have that $(-)^\transp:\powerset(X^2)\to\powerset(X^2)$ is an order similarity. 

Next we show that $F;G$ is closed under finite intersections: take $A,B\in F$ and $C,D\in G$. Then we have $A;C \supseteq (A\cap B);(C\cap D)$ and $B;D \supseteq (A\cap B);(C\cap D)$, so $A;C \cap B;D \supseteq (A\cap B);(C\cap D) \in F;G$. Thus $A;C \cap B;D\in F;G$ by upwards closure.

First, it is clear that $Fx$ is upwards closed: take $Ax\in Fx$. Then for all $B\supseteq Ax$, we have $B = \big((B\times\{x\})\cup A\big)x$, so $B\in Fx$.

Finally, to show closure under finite intersections, take $Ax, Bx\in F$. Then $Ax\cap Bx \supseteq (A\cap B)x \in Fx$ by \ref{orderPreservingFunctionLatticeOperations}.

(2) Immediate from the order similarity $(-)^\transp:\powerset(X^2)\to\powerset(X^2)$.

(3) We have $F;G = \powerset(X^2)$ iff there exist $A\in F,B\in G$ such that $A;B = \emptyset$. This is equivalent to $p_2^{\imf}[A]\perp p_1^{\imf}[B]$.

Thus $F;G \neq \powerset(X^2)$ is equivalent to $p_2^{\imf}[A]\mesh p_1^{\imf}[B]$ for all $A\in F$ and $B\in G$, i.e.\ $p_2^{\imf\imf}[F]\amesh p_1^{\imf\imf}[G]$.

(4) We have, for all $A\in F$, that $Ax = \emptyset$ iff $x\notin p_2^\imf(A)$. Thus $Fx$ is proper iff $x\in p_2^\imf(A)$ for all $A\in F$ iff $x\in \ker\big(p_2^{\imf\imf}(F)\big)$.
\end{proof}

\begin{proposition}
Let $X$ be a set and $F\in\powerfilters(X)$ a filter. Then the function $F\otimes -$ preserves joins.
\end{proposition}
\begin{proof}
We have $F\otimes G = \bigvee_{A\in F}\{A\}\times^{\imf}G$. The result follows from \ref{joinResiduatedMaps} and \ref{filterResiduatedImageGaloisConnection}.
\end{proof}

\begin{proposition} \label{filterCompositionResidual}
Let $X$ be a set and $H\in\powerfilters(X^2)$ a filter. Then the function $H;-: \powerfilters(X^2) \to \powerfilters(X^2)$ preserves finite intersections.
\end{proposition}
\begin{proof}
Let $\mathcal{F}\subseteq \powerfilters(X^2)$ be a finite set of filters. Then we calculate, using \ref{baseMeetJoinFilters},
\begin{align*}
\Big(\bigcap \mathcal{F}\Big); H &= \upset\setbuilder{A;B}{A\in \bigcap \mathcal{F}, B\in H} \\
&= \upset\setbuilder{\Big(\bigcup_{F\in \mathcal{F}}A_F\Big);B}{\forall F\in\mathcal{F}: A_F\in F, B\in H} \\
&= \upset\setbuilder{\bigcup_{F\in \mathcal{F}}A_F;B}{\forall F\in\mathcal{F}: A_F\in F, B\in H} \\
&= \bigcap_{F\in \mathcal{F}} \upset\setbuilder{A_F; B}{A_F\in F, B\in H} \\
&= \bigcap_{F\in \mathcal{F}} F;H.
\end{align*}
\end{proof}

\begin{lemma} \label{principalImageProductFilter}
Let $X$ be a set and $F,G\in\powerfilters(X)$. Then
\[ (F\otimes G)x = \begin{cases}
F & \big(x\in\ker(G)\big) \\
\powerset(X) & (\text{otherwise}).
\end{cases} \]
\end{lemma}
\begin{proof}
If $x\notin\ker(F)$, then there exists $A\in G$ such that $(X\times A)x = \emptyset$. Thus $(F\otimes G)x = \powerset(X)$.

Now suppose $x\in\ker(G)$. Take $B\in F\otimes G$, so there exist $C\in F, D\in G$ such that $C\times D \subseteq B$. Now $x\in D$, so $(C\times D)x = C$. Thus $(F\otimes G)x \subseteq F$. For the other inclusion, note that $A\times X\in F\otimes G$ and $(A\times X)x = A$ for all $A\in F$.
\end{proof}

\begin{lemma} \label{compositionProductFilters}
Let $X$ be a set and $F,G, G', H\in\powerfilters(X)$. Then
\[ (F\otimes G);(G'\otimes H) = \begin{cases}
F\otimes H & (G\amesh G') \\
\powerset(X ^2) & (\text{otherwise}).
\end{cases}  \]
\end{lemma}
\begin{proof}
TODO
\end{proof}

Note that for all proper filters $G$ we have $G\amesh G$. However $\powerset(X)\cancel\amesh\powerset(X)$, so $\big(F\otimes \powerset(X)\big);\big(\powerset(X)\otimes G\big) = \powerset(X^2)$, independent of $F$ and $G$.

\begin{lemma} \label{componentInclusionsFilterComposition}
Let $X$ be a set and $F,G\in\powerfilters(X^2)$. Then
\begin{enumerate}
\item $p_1^{\imf\imf}[F]\subseteq p_1^{\imf\imf}[F;G]$;
\item $p_2^{\imf\imf}[G]\subseteq p_2^{\imf\imf}[F;G]$.
\end{enumerate}
\end{lemma}
\begin{proof}
(1) This follows immediately from $p_1^{\imf}[A;B] = \setbuilder{p_1((x,y))}{(x,y)\in A \land y\in p_1^{\imf}(B)} \subseteq p_1^{\imf}[A]$.

(2) Similar.
\end{proof}

\begin{lemma} \label{filterCompositionFactorisationLemma}
Let $X$ be a set, $H\in \powerfilters(X^2)\setminus \{\powerset(X^2)\}$ and $F,G\in\powerfilters(X)$. Then
\[ F\otimes G = F\otimes p_1^{\imf\imf}(H); H; p_2^{\imf\imf}(H)\otimes G. \]
\end{lemma}
\begin{proof}
This follows from the fact that for all $A\in F$, $B\in G$ and $C,D,E\in H$,
\[ A\times B = \big(A \times p_1^{\imf}(C)\big);D;\big(p_2^\imf(E)\times B\big). \]
Indeed we have
\begin{align*}
(a,b)\in \big(A \times p_1^{\imf}(C)\big);D;\big(p_2^\imf(E)\times B\big) &\iff \exists c,d: \begin{cases}
(a,c)\in A\times p_1^\imf(C),\\ (c,d)\in D,\\ (d,b)\in p_2^\imf(E)\times B
\end{cases} \\
&\iff \begin{cases}
a\in A, b\in B, \\
\exists c,d: \; (c,d)\in D, c\in p_1^\imf(C), d\in p_2^\imf(E)
\end{cases} \\
&\iff a\in A, b\in B \\
&\iff (a,b)\in A\times B.
\end{align*}
The statement $\exists c,d: \; (c,d)\in D, c\in p_1^\imf(C), d\in p_2^\imf(E)$ is true because we may take $(c,d)\in C\cap D\cap E$, which is not empty because $H$ is proper.
\end{proof}
Set $I\defeq F\otimes p_1^{\imf\imf}(H); H; p_2^{\imf\imf}(H)\otimes G$. The inclusion $\subseteq$ can also be calculated using \ref{componentInclusionsFilterComposition} and \ref{filterFactorisationInequality}:
\[ F\otimes G = p_1^{\imf\imf}\big[F\otimes p_1^{\imf\imf}(H)\big] \otimes p_2^{\imf\imf}\big[p_2^{\imf\imf}(H)\otimes G\big] \subseteq p_1^{\imf\imf}[I]\otimes p_2^{\imf\imf}[I] \subseteq I. \]

\begin{lemma} \label{principalImageOfInProductFilter}
Let $X$ be a set, $F\in \powerfilters(X)$ a filter and $A\subseteq X^2$. Then
\begin{enumerate}
\item if $A\in F\otimes F$, then there exists $x\in X$ such that $Ax\in F$;
\item if $Ax, xA\in F$, then $A;A \in F\otimes F$.
\end{enumerate}
\end{lemma}
\begin{proof}
(1) Take $A\in F\otimes F$, then there exists $B\in F$ such that $B\times B\subseteq A$. If $B = \emptyset$, then $F$ is trivial, so $Ax\in F$ for all $x\in X$.

If $B \neq \emptyset$, then we can take $x\in B$ and we have
\[ B = (B\times B)x \subseteq Ax. \]
By upwards closure, $Ax\in F$.

(2) Suppose $Ax, xA\in F$. Then $Ax\times xA \in F\otimes F$ and, because $A;A = \bigcup_{y\in X}Ay\times yA \supseteq Ax\times xA$, we have $A;A\in F\otimes F$ by upwards closure.
\end{proof}

\begin{proposition} \label{imageFilterComposition}
Let $X, Y$ be sets, $A,B\subseteq X^2$, $C,D\subseteq Y^2$, $F,G\in\powerfilters(X^2)$ and $H,I\in \powerfilters(Y^2)$. Let $f: X\to Y$ be a function. Then
\begin{enumerate}
\item $(f|f)^{\imf}(A;B) \subseteq (f|f)^{\imf}(A);(f|f)^\imf(B)$;
\item $(f|f)^{\preimf}(C);(f|f)^\preimf(D) \subseteq (f|f)^{\preimf}(C;D)$;
\item $\upset(f|f)^{\imf\imf}(F;G) \supseteq \big(\upset(f|f)^{\imf\imf}(F)\big);\big(\upset(f|f)^{\imf\imf}(G)\big)$.
\item $\big(\upset(f|f)^{\preimf\imf}(H)\big);\big(\upset(f|f)^{\preimf\imf}(I)\big) \supseteq \upset(f|f)^{\preimf\imf}(H;I)$.
\end{enumerate}
\end{proposition}
\begin{proof}
(1) Take $(x,y)\in A;B$. Then there exists $z\in X$ such that $xAz$ and $zBy$ and thus $f(x)\big((f|f)^{\imf}(A)\big)f(z)$ and $f(z)\big((f|f)^{\imf}(B)\big)f(y)$, so $\big(f(x), f(y)\big)\in (f|f)^{\imf}(A);(f|f)^\imf(B)$.

(2) We calculate, using (1),
\begin{align*}
(f|f)^{\preimf}(C);(f|f)^\preimf(D) &\subseteq \big((f|f)^{\preimf}\circ (f|f)^\imf\big)\big((f|f)^{\preimf}(C);(f|f)^{\preimf}(D)\big) \\
&\subseteq (f|f)^{\preimf}\Big(\big((f|f)^{\imf}\circ(f|f)^{\preimf}\big)(C);\big((f|f)^{\imf}\circ(f|f)^{\preimf}\big)(D)\Big) \\
&\subseteq (f|f)^{\preimf}(C;D).
\end{align*}

(3) Immediate from (1).

(4) Immediate from (2).
\end{proof}

\begin{lemma} \label{principalImageUnderRelationMapping}
Let $X, Y$ be sets, $A\subseteq Y^2$, $x\in X$ and $f: X\to Y$ a function. Then $(f|f)^\preimf(A)x = f^\preimf\big(Af(x)\big)$.
\end{lemma}
\begin{proof}
Take arbitrary $y \in X$. Then
\begin{align*}
y \in (f|f)^\preimf(A)x \iff& (y,x)\in (f|f)^\preimf(A) \\
\iff& (f|f)(y,x) = \big(f(y), f(x)\big) \in A \\
\iff& f(y) \in Af(x) \\
\iff& y \in f^\preimf\big(Af(x)\big).
\end{align*}
\end{proof}

\subsection{Uniformities}
\begin{definition}
Let $X$ be a set. Let $\mathcal{U}$ be a set of filters in $\powerfilters(X^2)$. Suppose
\begin{itemize}
\item $\pfilter{x}\otimes \pfilter{x} \in \mathcal{U}$ for all $x\in X$;
\item $\mathcal{U}$ is upwards closed;
\item if $F\in \mathcal{U}$, then $F^\transp\in\mathcal{U}$;
\item if $F, G\in \mathcal{U}$, then $F;G\in\mathcal{U}$.
\end{itemize}
Then we call $\mathcal{U}$ a \udef{uniformity} on $X$ and $\sSet{X,\mathcal{U}}$ a uniform space. We call the elements of $\mathcal{U}$ \udef{uniform filters}.

If we drop the first condition, we get a \udef{preuniformity} and a \udef{preuniform space}.

A uniformity is called 
\begin{itemize}
\item \udef{factorisable} if for all $H\in\mathcal{U}$, there exist $F,G\in \powerfilters(X^2)$ such that $F\otimes G \in\mathcal{U}$ and $F\otimes G\subseteq H$;
\item \udef{symmetric} if each filter in $\mathcal{U}$ contains a symmetric filter in $\mathcal{U}$;
\item of \udef{finite depth} if it is closed under finite intersections;
\item a \udef{Kent} uniformity if $U\cap (\pfilter{x}\otimes \pfilter{x})\in \mathcal{U}$ for all $U\in \mathcal{U}$ and $x\in X$.
\end{itemize}
\end{definition}

\begin{lemma}
Let $\mathcal{U}$ be a uniformity. If $\mathcal{U}$ is of finite depth, then $\mathcal{U}$ is symmetric.
\end{lemma}
\begin{proof}
Assume $\mathcal{U}$ is of finite depth. For all $F\in\mathcal{U}$, we have $F\supseteq F\cap F^\transp \in \mathcal{U}$.
\end{proof}

\begin{lemma}
Let $\sSet{X,\mathcal{U}}$ be a uniform space. Then $\mathcal{U}$ is factorisable \textup{if and only if} $p_1^{\imf\imf}(H)\otimes p_2^{\imf\imf}(H) \in \mathcal{U}$ for all $H\in\mathcal{U}$.
\end{lemma}
\begin{proof}
Assume $\mathcal{U} \ni F\otimes G \subseteq H$. Then
\[ F\otimes G = p_1^{\imf\imf}(F\otimes G)\otimes p_2^{\imf\imf}(F\otimes G) \subseteq p_1^{\imf\imf}(H)\otimes p_2^{\imf\imf}(H) \in \mathcal{U}, \]
where we have used \ref{projectionsOfProductFilter}.
\end{proof}

\subsubsection{Ordering uniformities}
\begin{proposition} \label{latticeOfUniformities}
Let $X$ be a set. Then the set of uniformities on $X$ is a complete sublattice of $\powerset\big(\powerfilters(X^2)\big)$. Let $\mathfrak{U}$ be a set of uniformities on $X$. Then
\[ \bigwedge\mathfrak{U} = \bigcap\mathfrak{U} \qquad\text{and}\qquad \bigvee\mathfrak{U} = \Closure_{;}\Big(\bigcup\mathfrak{U}\Big). \]
\end{proposition}
\begin{proof}
TODO
\end{proof}

\subsubsection{Diagonality}
\begin{lemma}
Let $X$ be a set, $\mathcal{F}$ an upwards closed set of filters in $\powerfilters(X)$ and $\mathcal{C}$ a cover of $X$. If
\[ \forall C\in\mathcal{C}: \; \upset\{\Delta_C\} \in \mathcal{F}, \]
where $\Delta_C = \setbuilder{(c,c)}{c\in C}$, then $\pfilter{x}\otimes\pfilter{x}\in\mathcal{F}$ for all $x\in X$.
\end{lemma}
\begin{proof}
Because $\{(x,x)\}\subseteq \Delta_C$ for some $C\in\mathcal{C}$ and $\pfilter{x}\otimes\pfilter{x} = \upset \big\{\{(x,x)\}\big\} \supseteq \upset\{\Delta_C\} \in\mathcal{F}$.
\end{proof}

\begin{definition}
Let $X$ be a set and $\mathcal{S}$ be a set of subsets of $X$. A uniformity $\mathcal{U}$ on $X$ is called \udef{$\mathcal{S}$-diagonal} if $\forall S\in\mathcal{S}: \; \upset\{\Delta_S\} \in \mathcal{U}$.

The uniformity is simply called \udef{diagonal} if $\mathcal{S} = \{X\}$.
\end{definition}
We may call the first requirement in the definition of uniform space ``pointwise diagonality''.

\begin{lemma}
Let $\sSet{X,\mathcal{U}}$ be a uniform space and $\mathcal{S}_1, \mathcal{S}_2$ sets of subsets of $X$. If $\mathcal{S}_1 \subseteq \downset \mathcal{S}_2$, then $\mathcal{S}_2$-diagonality implies $\mathcal{S}_1$-diagonality.
\end{lemma}
\begin{proof}
Assume $\mathcal{S}_1 \subseteq \downset \mathcal{S}_2$ and that $\mathcal{U}$ is $\mathcal{S}_2$-diagonal.

Take $S\in \mathcal{S}_1$. Then there exists an $S'\in \mathcal{S}_2$ such that $S\subseteq S'$. Thus $\upset\{\Delta_{S'}\} \subseteq \upset\{\Delta_{S}\}$ and $\upset\{\Delta_{S'}\}\in \mathcal{U}$. By upwards closure, $\upset\{\Delta_{S}\} \in \mathcal{U}$.
\end{proof}

\subsubsection{Entourages}
\begin{definition}
Let $\sSet{X, \mathcal{U}}$ be a uniform space. Then $\entourage_\mathcal{U} \defeq \bigcap \mathcal{U}$ is called the \udef{entourage filter} of $\mathcal{U}$ and the elements of $\entourage_\mathcal{U}$ are called \udef{entourages}.

We call the uniform space \udef{topological} if $\mathcal{U} = \upset\{\entourage_\mathcal{U}\}$.
\end{definition}

\begin{lemma} \label{entourageLemma}
Let $\sSet{X,\mathcal{U}}$ be a uniform space with entourage filter $\entourage$. Then
\begin{enumerate}
\item $\entourage\subseteq \upset\{\id_X\}$;
\item $\entourage^\transp = \entourage$;
\item $\entourage;\entourage \subseteq \entourage$.
\end{enumerate}
If $\sSet{X,\mathcal{U}}$ is a topological uniform space, then
\begin{enumerate}[1'.] \setcounter{enumi}{2}
\item $\entourage;\entourage = \entourage$.
\end{enumerate}
Any filter in $\powerfilters(X\times X)$ satisfying properties 1., 2. and $\entourage \subseteq \entourage;\entourage$ is the entourage filter of a topological uniformity.
\end{lemma}
TODO: can we improve 3??
\begin{proof}
(1) We have
\[ \entourage = \bigcap \mathcal{U} \subseteq \bigcap \setbuilder{\pfilter{x}\otimes \pfilter{x}}{x\in X} = \upset\big\{\setbuilder{(x,x)}{x\in X}\big\} = \upset\{\id_X\}. \]

(2) We have
\[ \entourage^\transp = \bigcap \setbuilder{H^\transp}{H\in\mathcal{U}} = \bigcap \setbuilder{H}{H\in\mathcal{U}} = \entourage, \]
because the transpose is bijective and thus its image function is preserved under intersection.

(3) First take $A;B \in\entourage;\entourage$. We claim $A\subseteq A;B$. Indeed take $(a,b)\in A$. By (1), we have $(b,b)\in B$ and so $(a,b)\in A;B$. Thus $\entourage;\entourage \subseteq \entourage$.

(3') In the topological case, we have that $\entourage\in\mathcal{U}$ and thus $\entourage;\entourage\in \mathcal{U}$, so $\entourage \subseteq \entourage;\entourage$.

To show any such filter is an entourage filter, we check the four requirements
\begin{itemize}
\item From (1), we have for all $x\in X$
\[ \entourage \subseteq \upset\{\Delta\} \subseteq \upset\{(x,x)\} = \pfilter{x}\otimes\pfilter{x}. \]
\item Upwards closure is by construction.
\item If $\entourage \subseteq H$, then $\entourage = \entourage^\transp \subseteq H^\transp$.
\item If $\entourage\subseteq G,H$, then $\entourage \subseteq \entourage;\entourage \subseteq G;H$.
\end{itemize}
\end{proof}
\begin{corollary}
A topological uniform space is diagonal and symmetric.
\end{corollary}

\begin{proposition}
If a uniform space is topological and factorisable, then it is trivial.
\end{proposition}
\begin{proof}
Let $\sSet{X,\mathcal{U}}$ be a topological and factorisable uniform space with entourage filter $\entourage_\mathcal{U}$. Then $\entourage_\mathcal{U} \subseteq p_1^{\imf\imf}[\entourage_\mathcal{U}]\otimes p_2^{\imf\imf}[\entourage_\mathcal{U}]$. Now each $A\in \entourage_\mathcal{U}$ contains $\Delta$, so $p_1^\imf[A] = X = p_2^\imf[A]$. Thus $p_1^{\imf\imf}[\entourage_\mathcal{U}]\otimes p_2^{\imf\imf}[\entourage_\mathcal{U}] = \{X^2\}$.

Now $\entourage_\mathcal{U} = \upset\big\{\{X^2\}\big\}$, which is trivial.
\end{proof}

\subsection{Uniform relations}
\begin{definition}
Let $X$ be a sets. A \udef{uniform relation} on $X$ is a relation $R_\mathcal{U}$ on $\powerfilters(X)^2$ such that
\begin{itemize}
\item $\pfilter{x}\mathrel{R_\mathcal{U}}\pfilter{x}$ for all $x\in X$;
\item $F\,R_\mathcal{U}$ is upwards closed for all $F\in \powerfilters(X)$;
\item $R_\mathcal{U}$ is symmetric;
\item $R_\mathcal{U}$ is transitive when restricted to $\powerfilters(X)\setminus\{\powerset(X)\}$.
\end{itemize}
We call the structured set $\sSet{X,R_\mathcal{U}}$ a \udef{uniform relation space}.

We say the uniform relation
\begin{itemize}
\item  is of \udef{finite depth} if $F\,R_\mathcal{U}$ is a filter in $\filters\big(\powerfilters(X)\big)$ for all $F\in \powerfilters(X)$.
\end{itemize}
\end{definition}
Finite depth add the requirement that $F\,R_\mathcal{U}$ is closed under finite intersections.

\begin{lemma} \label{uniformRelationRelatedElementLemma}
Let $R_\mathcal{U}$ be a uniform relation on $X$ and $F,G\in\powerfilters(X)$. Assume there exist proper filters $F',G'\in\powerfilters(X)$ such that $F\mathrel{R_\mathcal{U}} F'$ and $G\mathrel{R_\mathcal{U}}G'$. Then
\begin{enumerate}
\item $F\mathrel{R_\mathcal{U}} F$;
\item if $F\amesh G$, then $F\mathrel{R_\mathcal{U}} G$.
\end{enumerate}
\end{lemma}
\begin{proof}
(1) By symmetry we have $F'\mathrel{R_\mathcal{U}} F$ and by transitivity $F\mathrel{R_\mathcal{U}} F$.

(2) If $F\amesh G$, then $F\vee G \neq \powerset(X)$ by \ref{joinProperFilter}. From (1) we have $F\mathrel{R_\mathcal{U}} F$ and $G\mathrel{R_\mathcal{U}} G$. By upwards closure, $F\mathrel{R_\mathcal{U}} (F\vee G)$ and $G\mathrel{R_\mathcal{U}} (F\vee G)$. By transitivity and symmetry, $F\mathrel{R_\mathcal{U}} G$.
\end{proof}

\begin{lemma} \label{uniformRelationUpwardsClosure}
Let $R$ be a uniform relation on $X$ and $F,G, F', G'\in\powerfilters(X)$.
\begin{enumerate}
\item If $F \subseteq F'$, $G\subseteq G'$ and $F\mathrel{R} G$, then $F'\mathrel{R} G'$.
\item If $F \mathrel{R} F'$, $G\mathrel{R} G'$ and $F\subseteq G$, then $F'\mathrel{R} G'$.
\end{enumerate}
\end{lemma}
\begin{proof}
(1) By upwards closure, we have $F\mathrel{R} G'$ and  (using symmetry) $F'\mathrel{R} G$. Thus, again using symmetry,
\[ F'\mathrel{R} G \;\text{and}\; G \mathrel{R} F \;\text{and}\; F \mathrel{R} G', \]
so $F'\mathrel{R} G'$ by transitivity.

(2) By symmetry and upwards closure, we have $F'RG$. By transitivity, $F'RG'$.
\end{proof}

\begin{proposition} \label{uniformRelationGaloisConnection}
Let $X$ be a set; $\mathcal{U}$ a uniformity on $X$ and $R$ a uniform relation on $X$. We define a uniformity $\Theta(R)$ and a uniform relation $\Xi(\mathcal{U})$ by
\begin{align*}
\forall H\in \powerfilters(X^2):\qquad H\in\Theta(R) \quad&\defequiv\quad p_1^{\imf\imf}[H]\mathrel{R}p_2^{\imf\imf}[H]; \\
\forall F,G\in\powerfilters(X)\setminus\{\powerset(X)\}:\qquad F\mathrel{\Xi(\mathcal{U})}G \quad&\defequiv\quad F\otimes G\in \mathcal{U}. 
\end{align*}
Then the functions
\begin{align*}
&\Theta: \{\text{uniform relations on $X$}\} \to \{\text{uniformities on $X$}\} \\
&\Xi: \{\text{uniformities on $X$}\} \to \{\text{uniform relations on $X$}\}
\end{align*}
form a Galois connection $(\Theta, \Xi)$. Additionally,
\begin{enumerate}
\item $\im(\Theta)$ is the set of factorisable uniformities on $X$;
\item $\im(\Xi)$ is the set of uniform relations on $X$, i.e.\ $\Xi$ is surjective.
\end{enumerate}
\end{proposition}
\begin{proof}
The prove $\Theta(R)$ is a uniformity, we verify the conditions:
\begin{itemize}
\item From $\pfilter{x}\mathrel{R}\pfilter{x}$, we get $\pfilter{x}\otimes \pfilter{x}\in\Theta(R)$.
\item If $H\in\Theta(R)$ and $H\subseteq H'$, then $p_1^{\imf\imf}[H]\mathrel{R} p_2^{\imf\imf}[H]$, $p_1^{\imf\imf}[H]\subseteq p_1^{\imf\imf}[H']$ and $p_2^{\imf\imf}[H]\subseteq p_2^{\imf\imf}[H']$. Thus, by \ref{uniformRelationUpwardsClosure}, we have $p_1^{\imf\imf}[H']\mathrel{R} p_2^{\imf\imf}[H']$ and so $H'\in\Theta(R)$.
\item Take $H\in \Theta(R)$. Then
\[ p_1^{\imf\imf}[H]\mathrel{R} p_2^{\imf\imf}[H] \iff p_2^{\imf\imf}[H^\transp]\mathrel{R} p_1^{\imf\imf}[H^\transp] \iff p_1^{\imf\imf}[H^\transp]\mathrel{R} p_2^{\imf\imf}[H^\transp] \iff H^\transp\in \Theta(R). \]
\item Take $H_1, H_2\in \Theta(R)$. If $H_1;H_2 = \powerset(X^2)$, then $H_1;H_2\in\Theta(R)$ by upwards closure. If $H_1;H_2 \neq \powerset(X^2)$, then $p_2^{\imf\imf}[H_1]\amesh p_1^{\imf\imf}[H_2]$ by \ref{filterOperationsOnRelationFilters} and thus $p_2^{\imf\imf}[H_1]\mathrel{R} p_1^{\imf\imf}[H_2]$ by \ref{uniformRelationRelatedElementLemma} (we have that $p_1^{\imf\imf}[H_1]\mathrel{R}p_2^{\imf\imf}[H_1]$, so $p_1^{\imf\imf}[H_1]\mathrel{R}\neq \emptyset$. Similarly $p_1^{\imf\imf}[H_2]\mathrel{R}p_2^{\imf\imf}[H_2]$ and $p_1^{\imf\imf}[H_2]\mathrel{R}\neq \emptyset$). So we have
\[ p_1^{\imf\imf}[H_1]\;\mathrel{R} \;p_2^{\imf\imf}[H_1] \;\mathrel{R} \;p_1^{\imf\imf}[H_2]\; \mathrel{R} \;p_2^{\imf\imf}[H_2]. \]
By \ref{componentInclusionsFilterComposition} and upward closure, we get $p_1^{\imf\imf}[H_1;H_2]\mathrel{R}p_2^{\imf\imf}[H_1;H_2]$, which means $H_1;H_2\in \Theta(R)$.
\end{itemize}

The prove $\Xi(\mathcal{U})$ is a uniform relation, we verify the conditions:
\begin{itemize}
\item From $\pfilter{x}\otimes \pfilter{x}\in\mathcal{U}$, we get $\pfilter{x}\mathrel{\Xi(\mathcal{U})}\pfilter{x}$.
\item Assume $F\mathrel{\Xi(\mathcal{U})}G$ and $G\subseteq G'$. Then $F\otimes G\in \mathcal{U}$ and $F\otimes G\subseteq F\otimes G'$. By upwards closure, $F\otimes G'\in\mathcal{U}$.
\item Symmetry is immediate from $(F\otimes G)^\transp = G\otimes F$.
\item For transitivity, assume $F\mathrel{\Xi(\mathcal{U})}G$, $G\mathrel{\Xi(\mathcal{U})}H$ and $G \neq \powerset(X)$. Then $F\otimes G, G\otimes H\in\mathcal{U}$ and $G \amesh G$ (this would not hold if $G = \powerset(X)$), so $(F\otimes G);(G\otimes H) = F\otimes H$ by \ref{compositionProductFilters}. Thus $F\mathrel{\Xi(\mathcal{U})}H$.
\end{itemize}

To show $(\Theta,\Xi)$ is a Galois connection, we need to prove that $\Theta(R) \subseteq \mathcal{U}$ \textup{if and only if} $R \subseteq \Xi(\mathcal{U})$.

First assume $\Theta(R) \subseteq \mathcal{U}$ and take $F,G\in\powerfilters(X)$ such that $F\mathrel{R} G$. Then $F\otimes G\in \Theta(R)\subseteq\mathcal{U}$ which means that $F\mathrel{\Xi(\mathcal{U})}G$.

Now assume $R \subseteq \Xi(\mathcal{U})$ and take $H\in \Theta(R)$. Then $p_1^{\imf\imf}[H]\mathrel{R}p_2^{\imf\imf}[H]$, which implies $p_1^{\imf\imf}[H]\mathrel{\Xi(\mathcal{U})}p_2^{\imf\imf}[H]$. Thus $p_1^{\imf\imf}[H]\otimes p_2^{\imf\imf}[H]\in \mathcal{U}$. By upwards closure and \ref{filterFactorisationInequality} we have $H\in \mathcal{U}$.

(1) It is clear that $\im(\Theta)$ consists of factorisable uniformities. For the other inclusion, let $\mathcal{U}$ be a factorisable uniformity. It is enough to show that $\mathcal{U} \subseteq \Theta(\Xi(\mathcal{U}))$. Take $H\in \mathcal{U}$. By factorisability $p_1^{\imf\imf}[H]\otimes p_2^{\imf\imf}[H]\in \mathcal{U}$. Then $p_1^{\imf\imf}[H]\mathrel{\Xi(\mathcal{U})} p_2^{\imf\imf}[H]$ and thus $H\in \Theta(\Xi(\mathcal{F}))$.

(2) It is enough to prove that for all uniform relations $R$ we have $\Xi(\Theta(R)) \subseteq R$. Take $F,G\in\powerfilters(X)$. We have
\[ F\mathrel{\Xi(\Theta(R))} G \implies F\otimes G \in\Theta(R) \implies F\mathrel{R}G. \]
\end{proof}


\subsubsection{Uniform convergence}
\begin{definition}
Let $\sSet{X,R}$ be a uniform relation space. Then the \udef{uniform convergence} $\Gamma(R)$ on $X$ is defined by
\[ F \overset{\Gamma(R)}{\longrightarrow} x \qquad\defequiv\qquad F\mathrel{R}\pfilter{x}. \]
We also denote the uniform convergence by $F\overset{u}{\longrightarrow} x$.
\end{definition}
If $\mathcal{U}$ is a uniformity, we write $\Gamma(\mathcal{U})$ to mean $\Gamma(\Xi(\mathcal{U}))$. We have
\[ F \overset{\Gamma(\mathcal{U})}{\longrightarrow} x \qquad\iff\qquad F\otimes \pfilter{x}\in\mathcal{U}. \]

\begin{lemma} \label{associatedUniformConvergence}
A uniform convergence is a convergence. It is also reciprocal ($R_1$).
\end{lemma}
\begin{proof}
Let $\sSet{X,R}$ be a uniform relation space. We have that $\Gamma(R)$ is centered, i.e.\ $\pfilter{x} \overset{\Gamma(R)}{\longrightarrow} x$, because $\pfilter{x}\mathrel{R}\pfilter{x}$.

We have that $\Gamma(R)$ is monotonic by upwards closure of $R$.

We prove reciprocity of $\Gamma(R)$ using point (5). of \ref{R1Conditions}. Assume $F \overset{\Gamma(R)}{\longrightarrow} x$ and $F \overset{\Gamma(R)}{\longrightarrow} y$. Then $F\mathrel{R}\pfilter{x}$ and $F\mathrel{R}\pfilter{y}$, so $\pfilter{x}\mathrel{R}\pfilter{y}$ by symmetry and transitivity. This implies
\[ G \in {\lim}_{\Gamma(R)}^{-1}(x) \iff G\mathrel{R}\pfilter{x}\iff G\mathrel{R}\pfilter{y} \iff G\in {\lim}_{\Gamma(R)}^{-1}(y), \]
and so $\lim_{\Gamma(R)}^{-1}(x) = \lim_{\Gamma(R)}^{-1}(y)$.
\end{proof}

\begin{proposition} \label{topologicalInducedUniformConvergence}
Let $\sSet{X,\mathcal{U}}$ be a uniform space and $x\in X$. Then
\begin{enumerate}
\item $\entourage_\mathcal{U}x \subseteq \vicinity_{\Gamma(\mathcal{U})}(x)$;
\item if $\mathcal{U}$ is topological, then $\Gamma(\mathcal{U})$ is topological and
\[ \neighbourhood_{\Gamma(\mathcal{U})}(x) = \upset \setbuilder{V x}{V\in \entourage_\mathcal{U}}. \]
\end{enumerate}
\end{proposition}
\begin{proof}
(1) We show that if $F\overset{\Gamma(\mathcal{U})}{\longrightarrow} x$, then $\entourage_\mathcal{U}x = \upset \setbuilder{V x}{V\in \entourage_\mathcal{U}}\subseteq F$. Indeed we have
\begin{align*}
F\otimes \pfilter{x}\in \mathcal{U} \implies& \entourage_\mathcal{U}\subseteq F\otimes \pfilter{x} \\
\implies& \forall V\in \entourage_\mathcal{U}: \exists A\in F: \; A\times \{x\} \subseteq V \\
\implies& \forall  V\in \entourage_\mathcal{U}: \exists A\in F: \; A \subseteq V x \\
\implies& \upset \setbuilder{V x}{V\in \entourage_\mathcal{U}}\subseteq F.
\end{align*}

(2) We first show that $\Gamma(\mathcal{U})$ is pretopological with vicinity filter $\entourage_\mathcal{U}x = \upset \setbuilder{V x}{V\in \entourage_\mathcal{U}}$. By (1), it is enough to prove that $\entourage_\mathcal{U}x \to x$.
Because $\entourage_\mathcal{U}$ is a topological entourage filter, we have
\begin{align*}
\entourage_\mathcal{U} &= \entourage_\mathcal{U};\entourage_\mathcal{U} \\
&\subseteq \entourage_\mathcal{U}; \pfilter{x}\otimes \pfilter{x} \\
&= \upset \setbuilder{V; \{(x,x)\}}{V\in \entourage_\mathcal{U}} \\
&= \upset \setbuilder{V x\times\{x\}}{V\in \entourage_\mathcal{U}} \\
&= \upset \setbuilder{V x}{V\in \entourage_\mathcal{U}}\otimes \pfilter{x}.
\end{align*}
Thus $\entourage_\mathcal{U}x \overset{\Gamma(\mathcal{U})}{\longrightarrow} x$.

Finally to show that $\Gamma(\mathcal{U})$ is topological, we use \ref{pretopologicalSpaceTopological}: Take $Vx\in \upset \setbuilder{V x}{V\in \entourage_\mathcal{U}}$. Then because $\entourage_\mathcal{U} = \entourage_\mathcal{U};\entourage_\mathcal{U}$, we can find $U,U'\in \entourage_\mathcal{U}$ such that $V = U;U'$. Consider $U'x$. For all $y\in U'x$, we have that $zUy \implies zU;U'x \iff zVx$, so $Uy \subseteq Vx$. Thus $Vx \in \upset \setbuilder{V y}{V\in \entourage_\mathcal{U}}$.
\end{proof}

TODO: is there a way to have a notion of ``pretopological uniform space''?

\begin{proposition}
Let $\sSet{X,\mathcal{U}}$ be a topological uniform space and $V\in \entourage_\mathcal{U}$.

Then $\interior_{\Gamma(\mathcal{U})\otimes \Gamma(\mathcal{U})}(V) \in \entourage_\mathcal{U}$.
\end{proposition}
\begin{proof}
By \ref{entourageLemma}, we have $\entourage_\mathcal{U} = \entourage_\mathcal{U};\entourage_\mathcal{U};\entourage_\mathcal{U}$, so can find $U_1,U_2,U_3$ such that $U_1;U_2;U_3 \subseteq V$. Let $U = U_1\cap U_1^\transp \cap U_2 \cap U_2^\transp \cap U_3\cap U_3^\transp$, so $U$ is symmetric and $U;U;U\subseteq V$.
Now
\[ U;U;U = \bigcup_{(x,y)\in U}U;\{(x, y)\};U = \bigcup_{(x,y)\in U}Ux \times Uy, \]
by \ref{relationCompositionResiduated}. For all $x,y\in U$, $Ux$ is a neighbourhood of $x$ in $\Gamma(\mathcal{U})$ and $Uy$ is a neighbourhood of $y$ in $\Gamma(\mathcal{U})$, by \ref{topologicalInducedUniformConvergence}.  Thus $Ux \times Uy$ is a neighbourhood of $(x,y)$ in $\Gamma(\mathcal{U})\otimes \Gamma(\mathcal{U})$, as this is a vicinity by \ref{productVicinity} and the product convergence is topological by \ref{pretopologicalInitialConvergence}.

By \ref{subsetWithVicinitiesInInherence}, we have that $U\subseteq \interior_{\Gamma(\mathcal{U})\otimes \Gamma(\mathcal{U})}(V)$ and thus the proposition follows by upwards closure.
\end{proof}
\begin{corollary}
Let $\sSet{X,\mathcal{U}}$ be a topological uniform space, then $\entourage_\mathcal{U}$ has a base of sets open in $\Gamma(\mathcal{U})\otimes \Gamma(\mathcal{U})$.
\end{corollary}

\begin{proposition}
Let $\sSet{X,\mathcal{U}}$ be a topological uniform space, $A\subseteq X$ and $M\subseteq X^2$. Then
\begin{enumerate}
\item $\closure_{\Gamma(\mathcal{U})}(A) = \bigcap\setbuilder{U_A}{U\in \entourage_\mathcal{U}}$;
\item $\closure_{\Gamma(\mathcal{U})\otimes \Gamma(\mathcal{U})}(M) = \bigcap\setbuilder{V;M;V}{V\in\entourage_\mathcal{U}}$.
\end{enumerate}
\end{proposition}
\begin{proof}
(1) By \ref{principalAdherenceInherence} and \ref{topologicalInducedUniformConvergence}, we have
\begin{align*}
x\in \closure_{\Gamma(\mathcal{U})}(A) &\iff \forall U\in \entourage_\mathcal{U}: xU \mesh A \\
&\iff \forall U\in \entourage_\mathcal{U}: \exists y\in A: xUy \\
&\iff \forall U\in \entourage_\mathcal{U}: x\in {_A}U = U^\transp_A \\
&\iff x\in\bigcap\setbuilder{U^\transp_A}{U\in \entourage_\mathcal{U}} \\
&\iff x\in\bigcap\setbuilder{U_A}{U\in \entourage_\mathcal{U}}.
\end{align*}
For the last step, we have used the symmetry of $\entourage_\mathcal{U}$, \ref{entourageLemma}.

(2) By \ref{principalAdherenceInherence}, \ref{topologicalInducedUniformConvergence} and \ref{productVicinity}, we have
\begin{align*}
(x,y) \in \closure_{\Gamma(\mathcal{U})\otimes \Gamma(\mathcal{U})}(M) &\iff \forall U,V\in \entourage_\mathcal{U}: xU \times yV \mesh M \\
&\iff \forall U,V\in \entourage_\mathcal{U}:\exists (a,b)\in M: xUa \land yVb \\
&\iff \forall U,V\in \entourage_\mathcal{U}:(x,y) \in U;M;V^\transp \\
&\iff (x,y) \in \bigcap\setbuilder{U;M;V^\transp}{U,V\in\entourage_\mathcal{U}} \\
&\iff (x,y) \in \bigcap\setbuilder{V;M;V}{V\in\entourage_\mathcal{U}}.
\end{align*}
For the last step, we have used the symmetry of $\entourage_\mathcal{U}$, \ref{entourageLemma}, and the fact that it is a filter. We have also used \ref{relationCompositionResiduated}.
\end{proof}
\begin{corollary}
Let $\sSet{X,\mathcal{U}}$ be a topological uniform space, then $\entourage_\mathcal{U}$ has a base of sets closed in $\Gamma(\mathcal{U})\otimes \Gamma(\mathcal{U})$.
\end{corollary}
\begin{proof}
Let $U\in \entourage_\mathcal{U}$. Then by \ref{entourageLemma}, we can find a $V\in\entourage_\mathcal{U}$ such that $V;V;V\subseteq U$. By the proposition, $V;V;V$ contains the closure of $V$. Thus each entourage contains a closed set.
\end{proof}

\begin{definition}
For any property $\mathbf{P}$ that a convergence space may have, we say a uniform relation space $\sSet{X,R}$ has property $\mathbf{P}$ if $\Gamma(R)$ has property $\mathbf{P}$.
\end{definition}

\subsubsection{Induced uniform relation}
\begin{definition}
Let $\sSet{X,\xi}$ be a reciprocal ($R_1$) convergence space. Let $\Phi(\xi)$ be a relation on $\powerfilters(X)$ defined by
\[ F\mathrel{\Phi(\xi)}G \qquad\defequiv\qquad \exists: x\in X: \; \big(F\overset{\xi}{\longrightarrow} x\big) \land \big(G\overset{\xi}{\longrightarrow} x\big) \]
for $F,G\in\powerfilters(X)$.
Then $\Phi(\xi)$ is the \udef{uniform relation associated to} $\xi$.
\end{definition}

\begin{lemma} \label{uniformRelationAssociatedToR1Convergence}
The uniform relation associated to a reciprocal convergence is a uniform relation.
\end{lemma}
\begin{proof}
\begin{itemize}
\item We have $\pfilter{x}\overset{\xi}{\longrightarrow} x$, so $\pfilter{x}\mathrel{\Phi(\xi)}\pfilter{x}$.
\item The set $F\mathrel{\Phi(\xi)}$ is upwards closed by monotonicity of the convergence $\xi$.
\item Symmetry is clear by construction.
\item For transitivity, take proper filters $F,G,H$ such that $F\mathrel{\Phi(\xi)}G$ and $G\mathrel{\Phi(\xi)}H$. Then there exist $x,y\in X$ such that
\[ \big(F\overset{\xi}{\longrightarrow} x\big) \land \big(G\overset{\xi}{\longrightarrow} x\big) \land \big(G\overset{\xi}{\longrightarrow} y\big) \land \big(H\overset{\xi}{\longrightarrow} y\big). \]
Thus $G\in \lim^{-1}_\xi(x) \cap\lim^{-1}_\xi(y)$, so $\lim^{-1}_\xi(x) \mesh\lim^{-1}_\xi(y)$. Using reciprocity, we apply \ref{R1Conditions} to get $\lim^{-1}_\xi(x) = \lim^{-1}_\xi(y)$. Thus $H \overset{\xi}{\longrightarrow} x$, which means that $F\mathrel{\Phi(\xi)}H$.
\end{itemize}
\end{proof}

\begin{proposition} \label{uniformConvergenceGaloisConnection}
Let $X$ be a set. The functions
\begin{align*}
&\Phi: \{\text{$R_1$ convergences on $X$}\} \to \{\text{uniform relations on $X$}\} \\
&\Gamma: \{\text{uniform relations on $X$}\} \to \{\text{$R_1$ convergences on $X$}\}
\end{align*}
form a Galois connection $(\Phi, \Gamma)$. Additionally,
\begin{enumerate}
\item $\im(\Phi)$ is the set of complete uniform relations on $X$;
\item $\im(\Gamma)$ is the set of all reciprocal convergences on $X$; i.e\ $\Gamma$ is surjective.
\end{enumerate}
\end{proposition}
\begin{proof}
The functions $\Phi$ and $\Gamma$ are well-defined by \ref{associatedUniformConvergence} and \ref{uniformRelationAssociatedToR1Convergence}.

To prove the Galois connection, let $R$ be a uniform relation on $X$ and $\xi$ a convergence on $X$. Then we need to prove $\Phi(\xi) \subseteq R$ \textup{if and only if} $\xi \subseteq \Gamma(R)$.

First assume $\Phi(\xi) \subseteq R$. Take $F\overset{\xi}{\longrightarrow} x$. Because also $\pfilter{x}\overset{\xi}{\longrightarrow} x$, we have $F\mathrel{\Phi(\xi)}\pfilter{x}$. By assumption $F\mathrel{R}\pfilter{x}$ and by definition $F\overset{\Gamma(R)}{\longrightarrow} x$.

Now assume $\xi \subseteq \Gamma(R)$. Take $F,G\in \powerfilters(X)$ such that $F\mathrel{\Phi(\xi)}G$. Then $\exists x\in X$ such that $F\overset{\xi}{\longrightarrow} x$ and $G\overset{\xi}{\longrightarrow} x$. By assumption $F\overset{\Gamma(R)}{\longrightarrow} x$ and $G\overset{\Gamma(R)}{\longrightarrow} x$, so by definition $F\mathrel{R}\pfilter{x}$ and $G\mathrel{R}\pfilter{x}$. By symmetry and transitivity $F\mathrel{R}G$.

\begin{enumerate}
\item For the inclusion $\subseteq$: assume $F\mathrel{\Phi(\xi)} F$. Then $F\to x$ for some $x\in X$, so $F\mathrel{\Phi(\xi)} \pfilter{x}$. Thus $F$ converges uniformly to some $x$.

For the other inclusion, $\supseteq$, take a complete uniform relation $R$. It is enough to show that $R \subseteq \Phi(\Gamma(R))$. Take $F,G\in \powerfilters(X)$ such that $F\mathrel{R} G$. By completeness, $F\mathrel{R} \pfilter{x}$ for some $x\in X$. By symmetry and transitivity, $G\mathrel{R} \pfilter{x}$ as well. Thus $F\overset{\Gamma(R)}{\longrightarrow} x$ and $G\overset{\Gamma(R)}{\longrightarrow} x$, so $F\mathrel{\Phi(\Gamma(R))} G$.

\item It is enough to prove $\Gamma(\Phi(\xi)) \subseteq \xi$ for any reciprocal convergence $\xi$. Take $F\overset{\Gamma(\Phi(\xi))}{\longrightarrow} x$. Then $F\mathrel{\Phi(\xi)} \pfilter{x}$ and so $\exists y\in X$ such that $F\overset{\xi}{\longrightarrow} y$ and $\pfilter{x}\overset{\xi}{\longrightarrow} y$. By reciprocity (and because $\pfilter{x}\subseteq \lim_{\xi}^{-1}(x) \cap \lim_{\xi}^{-1}(y)$), we have
\[ F \in {\lim}_{\xi}^{-1}(y) = {\lim}_{\xi}^{-1}(x). \]
So $F\overset{\xi}{\longrightarrow} x$.
\end{enumerate}
\end{proof}

\begin{corollary} \label{completeUniformGaloisConnection}
Let $X$ be a set. The functions
\begin{align*}
&\Theta \circ \Phi: \{\text{$R_1$ convergences on $X$}\} \to \{\text{uniformities on $X$}\} \\
&\Gamma\circ \Xi: \{\text{uniformities on $X$}\} \to \{\text{$R_1$ convergences on $X$}\}
\end{align*}
form a Galois connection $(\Theta \circ \Phi, \Gamma\circ \Xi)$. Additionally,
\begin{enumerate}
\item $\im(\Theta \circ \Phi)$ is the set of complete, factorisable uniformities on $X$;
\item $\im(\Gamma\circ \Xi)$ is the set of all reciprocal convergences on $X$.
\end{enumerate}
\end{corollary}
\begin{proof}
The Galois connection follows from \ref{uniformRelationGaloisConnection} and \ref{uniformConvergenceGaloisConnection}.

(1) The inclusion $\subseteq$ is immediate, because a uniformity is called complete if and only if its associated uniform relation is complete.

For the inclusion $\supseteq$, take some complete, factorisable uniformity $\mathcal{U}$. Then $\mathcal{U} = \Theta(R)$ for some uniform relation $R$, by \ref{uniformRelationGaloisConnection}. Now it is enough to note that $R$ is also complete, so there exists a reciprocal convergence $\xi$ such that $\mathcal{U} = \Theta(\Phi(\xi))$ by \ref{uniformConvergenceGaloisConnection}.

(2) Immediate because both $\Gamma$ and $\Xi$ are surjective.
\end{proof}

We can summarise the mappings between uniformlities $\mathcal{U}$, uniform relations $R$ and uniform convergences $\xi$ as follows
\[ \begin{tikzcd}[labels = {font=\large}, column sep=large]
\mathcal{U} \arrow[bend left, rrr, "\substack{F\mathrel{\Xi(\mathcal{U})}G \Leftrightarrow F\otimes G\in \mathcal{U} \vspace{0.1em} \\ \vspace{0.1em} \Xi}"] &&& \arrow[lll, bend left, "\substack{\Theta \vspace{0.1em} \\ H\in\Theta(R) \Leftrightarrow p_1^{\imf\imf}[H]\mathrel{R}p_2^{\imf\imf}[H]}"] R \arrow[rrr, bend left, "\substack{F\overset{\Gamma(R)}{\longrightarrow} x \Leftrightarrow F\mathrel{R}\pfilter{x} \vspace{0.1em} \\ \vspace{0.1em} \Gamma}"] &&& \arrow[lll, bend left, "\substack{\Phi \vspace{0.1em} \\ F\mathrel{\Phi(\xi)}G \Leftrightarrow \exists x: (F\to x)\land (G\to x)}"] \xi
\end{tikzcd} \]


\section{Uniform continuity}
\begin{definition}
Let $\sSet{X,\mathcal{U}}$ and $\sSet{Y,\mathcal{V}}$ be uniform spaces. A function $f: X\to Y$ is called \udef{uniformly continuous} if
\[ H\in \mathcal{U} \quad\implies\quad \upset (f|f)^{\imf\imf}[H]\in \mathcal{V}. \]
The set of uniform functions from $\sSet{X,\mathcal{U}}$ to $\sSet{Y,\mathcal{V}}$ is denoted $\ucont(\mathcal{U}, \mathcal{V})$ or $\ucont(X,Y)$.
\end{definition}

\begin{lemma} \label{compositionUniformlyContinuousFunctions}
Let $\sSet{X,\mathcal{U}}$, $\sSet{Y,\mathcal{V}}$, $\sSet{Z, \mathcal{W}}$ be uniform spaces and $f:X\to Y$, $g:Y\to Z$ uniformly continuous functions. Then $g\circ f$ is uniformly continuous.
\end{lemma}
\begin{proof}
Take arbitrary $H \in \mathcal{U}$. Then $\upset (f|f)^{\imf\imf}(H)\in \mathcal{V}$ and $\upset(g|g)^{\imf\imf}\big(\upset (f|f)^{\imf\imf}(H)\big) \in \mathcal{W}$. Finally we note
\[ \upset(g|g)^{\imf\imf}\big(\upset (f|f)^{\imf\imf}(H)\big) = \upset\big((g|g)^{\imf\imf}\circ (f|f)^{\imf\imf}\big)(H) = \upset \big((g|g)\circ (f|f)\big)^{\imf\imf}(H) = \upset (g\circ f|g\circ f)^{\imf\imf}(H) \]
by \ref{monotonicityOrderClosure}, \ref{functorialityImageFunction} and \ref{parallelCompositionOfMorphisms}.
\end{proof}

\begin{proposition} \label{uniformContinuityEntourages}
Let $\sSet{X,\mathcal{U}}$, $\sSet{Y,\mathcal{V}}$ be uniform spaces and $f: X\to Y$ a function.
\begin{enumerate}
\item If $f$ is uniformly continuous, then $\entourage_\mathcal{V} \subseteq \upset (f|f)^{\imf\imf}[\entourage_\mathcal{U}]$;
\item If $\mathcal{V}$ is topological, then opposite implication also holds.
\end{enumerate}
\end{proposition}
\begin{proof}
(1) By uniform continuity we have
\[ \entourage_\mathcal{V} \subseteq \bigcap\setbuilder{\upset (f|f)^{\imf\imf}[H]}{H\in \mathcal{U}} = \upset (f|f)^{\imf\imf}\left[\bigcap \mathcal{U}\right] = \upset (f|f)^{\imf\imf}\left[\entourage_\mathcal{U}\right]. \]
The first equality follows from \ref{imageFiltersPreservesIntersection}.

(2) Assume $\entourage_\mathcal{V} \subseteq \upset (f|f)^{\imf\imf}[\entourage_\mathcal{U}]$ and take $H\in \mathcal{U}$. Then $\entourage_\mathcal{U}\subseteq H$, so
\[ \entourage_\mathcal{V} \subseteq \upset (f|f)^{\imf\imf}[\entourage_\mathcal{U}] \subseteq \upset (f|f)^{\imf\imf}[H]. \]
Thus $\upset (f|f)^{\imf\imf}[H] \in\mathcal{V}$.
\end{proof}

\begin{proposition} \label{preservationUniformStructure}
Let $X,Y$ be sets, $\sSet{X,\mathcal{U}}, \sSet{Y,\mathcal{V}}$ uniform spaces, $\sSet{X,R}, \sSet{Y,S}$ uniform relation spaces and $\sSet{X,\xi}, \sSet{Y,\zeta}$ reciprocal convergence spaces. Let $f: X\to Y$ be a function.
\begin{enumerate}
\item if $f: \sSet{X,\mathcal{U}} \to \sSet{Y,\mathcal{V}}$ is uniformly continuous, then $\big(f: \sSet{X,\Xi(\mathcal{U})} \to \sSet{Y,\Xi(\mathcal{V})}\big)^{\imf\imf}$ is relation preserving;
\item if $\big(f: \sSet{X,R} \to \sSet{Y,S}\big)^{\imf\imf}$ is relation preserving, then $f: \sSet{X,\Gamma(R)} \to \sSet{Y,\Gamma(S)}$ is continuous;
\item if $f: \sSet{X,\xi} \to \sSet{Y,\zeta}$ is continuous, then $\big(f: \sSet{X,\Phi(\xi)} \to \sSet{Y,\Phi(\zeta)}\big)^{\imf\imf}$ is relation preserving;
\item if $\big(f: \sSet{X,R} \to \sSet{Y,S}\big)^{\imf\imf}$ is relation preserving, then $f: \sSet{X,\Theta(R)} \to \sSet{Y,\Theta(S)}$ is uniformly continuous.
\end{enumerate}

\end{proposition}
\begin{proof}
(1) Assume $f$ uniformly continuous and $F\mathrel{\Xi(\mathcal{U})} G$. Then $F\otimes G\in \mathcal{U}$. By uniform continuity $\upset (f|f)^{\imf\imf}[F\otimes G] = f^{\imf\imf}[F] \otimes f^{\imf\imf}[G] \in \mathcal{V}$, so $f^{\imf\imf}[F] \mathrel{\Xi(\mathcal{V})} f^{\imf\imf}[G]$.

(2) Assume $f^{\imf\imf}$ relation preserving and take $F \overset{\Gamma(R)}{\longrightarrow} x$. Then $F\mathrel{R}\pfilter{x}$ and, by relation preservation, $f^{\imf\imf}[F]\mathrel{S}f^{\imf\imf}[\pfilter{x}]$. Now $f^{\imf\imf}[\pfilter{x}] = \pfilter{f}(x)$, so $f^{\imf\imf}[F] \overset{\Gamma(S)}{\longrightarrow} f(x)$.

(3) Assume $f$ is continuous and $F\mathrel{\Phi(\xi)} G$. Then there exists $x\in X$ such that $F\overset{\xi}{\longrightarrow} x$ and $G\overset{\xi}{\longrightarrow} x$. By continuity $f^{\imf\imf}[F]\overset{\zeta}{\longrightarrow} f(x)$ and $f^{\imf\imf}[G]\overset{\zeta}{\longrightarrow} f(x)$, so $f^{\imf\imf}[F] \mathrel{\Phi(\zeta)} f^{\imf\imf}[G]$.

(4) Assume $f^{\imf\imf}$ is relation preserving and take $H\in \Theta(R)$. Then $p_1^{\imf\imf}[H]\mathrel{R}p_2^{\imf\imf}[H]$, so $f^{\imf\imf}\big[p_1^{\imf\imf}[H]\big]\mathrel{S}f^{\imf\imf}\big[p_2^{\imf\imf}[H]\big]$. Now
\[ f^{\imf\imf}\big[p_1^{\imf\imf}[H]\big] = (f\circ p_1)^{\imf\imf}[H] = \big(p_1\circ(f|f) \big)^{\imf\imf}[H] = p_1^{\imf\imf}\big[(f|f)^{\imf\imf}[H]\big]. \]
Similarly $f^{\imf\imf}\big[p_2^{\imf\imf}[H]\big] = p_2^{\imf\imf}\big[(f|f)^{\imf\imf}[H]\big]$. Thus $p_1^{\imf\imf}\big[(f|f)^{\imf\imf}[H]\big]\mathrel{S}p_2^{\imf\imf}\big[(f|f)^{\imf\imf}[H]\big]$, which means that $(f|f)^{\imf\imf}[H]\in \Theta(S)$.
\end{proof}
\begin{corollary}
Let $X,Y$ be sets, $\sSet{X,\mathcal{U}}, \sSet{Y,\mathcal{V}}$ uniform spaces, $\sSet{X,R}, \sSet{Y,S}$ uniform relation spaces and $\sSet{X,\xi}, \sSet{Y,\zeta}$ reciprocal convergence spaces. Let $f: X\to Y$ be a function.
\begin{enumerate}
\item if $\mathcal{U}$ is factorisable and $\big(f: \sSet{X,\Xi(\mathcal{U})} \to \sSet{Y,\Xi(\mathcal{V})}\big)^{\imf\imf}$ is relation preserving, then $f: \sSet{X,\mathcal{U}} \to \sSet{Y,\mathcal{V}}$ is uniformly continuous;
\item if $R$ is complete and $f: \sSet{X,\Gamma(R)} \to \sSet{Y,\Gamma(S)}$ is continuous, then $\big(f: \sSet{X,R} \to \sSet{Y,S}\big)^{\imf\imf}$ is relation preserving.
\end{enumerate}
\end{corollary}
\begin{proof}
(1) From the proposition, we have that $f: \sSet{X,\Theta(\Xi(\mathcal{U}))} \to \sSet{Y,\Theta(\Xi(\mathcal{V}))}$ is uniformly continuous. Because $\mathcal{U}$ is factorisable, $\Theta(\Xi(\mathcal{U})) = \mathcal{U}$. Also $\Theta(\Xi(\mathcal{V})) \subseteq \mathcal{V}$.

(2) From the proposition, we have that $\big(f: \sSet{X,\Phi(\Gamma(R))} \to \sSet{Y,\Phi(\Gamma(S))}\big)^{\imf\imf}$ is relation preserving. Because $R$ is complete, we have $R = \Phi(\Gamma(R))$. Also $\Phi(\Gamma(S)) \subseteq S$.
\end{proof}

\begin{proposition}
Let $X,Y$ be sets, $\sSet{X,\mathcal{U}}, \sSet{Y,\mathcal{V}}$ uniform spaces and $\sSet{X,\xi}, \sSet{Y,\zeta}$ reciprocal convergence spaces. Let $f: X\to Y$ be a function.
\begin{enumerate}
\item if $\mathcal{U}$ is compact, $\mathcal{V}$ is uniformly Choquet and $f: \sSet{X,\Gamma(\Xi(\mathcal{U}))} \to \sSet{Y,\Gamma(\Xi(\mathcal{V}))}$ is continuous, then $f: \sSet{X,\mathcal{U}} \to \sSet{Y,\mathcal{V}}$ is uniformly continuous.
\end{enumerate}
\end{proposition}
\begin{proof}
Take $H\in\mathcal{U}$. Take arbitrary ultrafilter $I\in\powerfilters(Y^2)$ such that $(f|f)^{\imf\imf}[H]\subseteq I$. Then there exists an ultrafilter $J\in \powerfilters(X^2)$ such that $\upset (f|f)^{\imf\imf}[J] = I$ by \ref{preimageFilter}.

By \ref{compactUltrafilterFactorisation}, there exists $x\in X$ such that $p_1^{\imf\imf}[J]\overset{\Gamma(\Xi(\mathcal{U}))}{\longrightarrow} x$ and $p_2^{\imf\imf}[J]\overset{\Gamma(\Xi(\mathcal{U}))}{\longrightarrow} x$. By continuity, symmetry and transitivity,
\[ \mathcal{V} \ni (f\circ p_1)^{\imf\imf}[J]\otimes (f\circ p_2)^{\imf\imf}[J] = p_1^{\imf\imf}\big[(f|f)^{\imf\imf}[J]\big]\otimes p_2^{\imf\imf}\big[(f|f)^{\imf\imf}[J]\big] \subseteq (f|f)^{\imf\imf}[J] = I. \]
As this is true for arbitrary ultrafilter $I$, we have $(f|f)^{\imf\imf}[H]\in\mathcal{V}$ because $\mathcal{V}$ is uniformly Choquet.
\end{proof}

\subsection{Initial and final uniform spaces}
\begin{definition}
Let $Y$ be a set.
\begin{itemize}
\item Given a set of uniform spaces $\{\sSet{Z_i, \mathcal{W}_i}\}_{i\in I}$ and a set of functions $\{f_i: Y\to Z_i\}_{i\in I}$, we define the \udef{initial uniformity} $\mathcal{X}$ on $Y$ as
\[ \mathcal{X} = \bigvee \setbuilder{\text{$\mathcal{V}$ a uniformity on $Y$}}{\forall i\in I:\; f_i\in \ucont(\mathcal{V}, \mathcal{W}_i)}. \]
\item Given a set of uniform spaces $\{\sSet{X_i, \mathcal{U}_i}\}_{i\in I}$ and a set of functions $\{g_i: X_i\to Y\}_{i\in I}$, we define the \udef{final uniformity} $\mathcal{Y}$ on $Y$ as
\[ \mathcal{Y} = \bigwedge \setbuilder{\text{$\mathcal{V}$ a uniformity on $Y$}}{\forall i\in I:\; g_i\in \ucont(\mathcal{U}_i, \mathcal{V})}. \]
\end{itemize}
\end{definition}

\begin{proposition} \label{initialFinalUniformity}
Let $Y$ be a set and $H\in\powerfilters(Y^2)$.
\begin{enumerate}
\item Let $\{f_i: Y\to \sSet{Z_i, \mathcal{W}_i}\}_{i\in I}$ be a set of functions to uniform spaces and $\mathcal{X}$ the initial uniformity on $Y$ w.r.t.\ this set. Then
\[ H\in \mathcal{X} \quad\iff\quad \forall i\in I: \; \upset(f_i|f_i)^{\imf\imf}(H)\in \mathcal{W}_i. \]
\item Let $\{g_i: \sSet{X_i, \mathcal{U}_i}\}_{i\in I}$ be a set of functions to uniform spaces and $\mathcal{Y}$ the final uniformity on $Y$ w.r.t.\ this set. Then
\[ H\in \mathcal{Y} \quad\iff\quad \exists \{i_0,\ldots i_n\}\subseteq I: \forall k\leq n: \exists U_{i_k}\in \mathcal{U}_{i_k}:\; (g_{i_0}|g_{i_0})^{\imf\imf}(U_0);\ldots;(g_{i_n}|g_{i_n})^{\imf\imf}(U_n)\subseteq H. \]
\end{enumerate}
\end{proposition}
In particular, all $f_i$s are uniformly continuous w.r.t. the initial uniformity and all $g_i$s are uniformly continuous w.r.t. the final uniformity.
\begin{proof}
(1) For the direction $\Rightarrow$, we need to show that the $f_i$s are continuous. Take $H\in \mathcal{X}$. Then, by \ref{latticeOfUniformities}, $H = H_0;\ldots;H_n$ where each $H_k\in \mathcal{V}_k \in \setbuilder{\text{$\mathcal{V}$ a uniformity on $Y$}}{\forall i\in I:\; f_i\in \ucont(\mathcal{V}, \mathcal{W}_i)}$.

By assumption, each $f_i$ maps each $H_k$ to an element of $\mathcal{W}_i$. By closure under composition, $(f_{i}|f_{i})^{\imf\imf}(H_0);\ldots;(f_{i}|f_{i})^{\imf\imf}(H_k)\in\mathcal{W}_i$. Now
\[ (f_i|f_i)^{\imf\imf}(H) = (f_i|f_i)^{\imf\imf}(H_0;\ldots; H_n) \supseteq (f_{i}|f_{i})^{\imf\imf}(H_0);\ldots;(f_{i}|f_{i})^{\imf\imf}(H_k) \]
by \ref{imageFilterComposition}, so $\upset (f_i|f_i)^{\imf\imf}(H) \in \mathcal{W}_i$ by upwards closure.

For $\Leftarrow$, assume, towards a contradiction, that $H$ is such that $\forall i\in I: \; \upset(f_i|f_i)^{\imf\imf}(H) \in \mathcal{W}_i$, but $H \notin \mathcal{X}$. Then let $\mathcal{X}'$ be the least uniformity that contains $\mathcal{X}$ and $H$, which exists by \ref{latticeOfUniformities}.

If we can show that $\mathcal{X}'$ makes all the $f_i$s uniformly continuous, then $\mathcal{X}'$ must be smaller than $\mathcal{X}$ by construction. This is a contradiction.

For all $i\in I$, we have $\upset(f_i|f_i)^{\imf\imf}(H^\transp) \in \mathcal{W}_i$ and $\upset(f_i|f_i)^{\imf\imf}(H') \in \mathcal{W}_i$ for all $H'\supseteq H$.

We just need to show that $\upset(f_i|f_i)^{\imf\imf}(H;H) \in \mathcal{W}_i$ and $\upset(f_i|f_i)^{\imf\imf}(H;H') \in \mathcal{W}_i$ for all $H'\in \mathcal{X}$. These facts follow from \ref{imageFilterComposition}, by upwards closure of $\mathcal{W}_i$.

(2) For the direction $\Leftarrow$, we need to show that the $g_i$s are continuous. Take arbitrary $i\in I$ and $U_i\in \mathcal{U}_i$. Then $\upset (g_i|g_i)^{\imf\imf}(U_i)\in \mathcal{V}$ for each uniformity on $Y$ that makes all $g_i$ uniformly continuous. Thus $\upset (g_i|g_i)^{\imf\imf}(U_i)$ is in the intersection of all such uniformities, which means it is in $\mathcal{Y}$ by \ref{latticeOfUniformities}.

For $\Rightarrow$, it is enough to prove that the set of $H$s that satisfy the right-hand side forms a uniformity such that all $g_i$ are uniformly continuous.

These points are easy to verify.
\end{proof}


\begin{proposition}[Characteristic property of initial and final uniformity] \label{characteristicPropertyInitialFinalUniformity}
Let $Y$ be a set, $\sSet{X,\mathcal{U}}$ and $\sSet{Z, \mathcal{W}}$ uniform spaces.
\begin{enumerate}
\item Let $\{f_i: Y\to \sSet{Z_i, \mathcal{W}_i}\}_{i\in I}$ be set of functions to uniform spaces and $\mathcal{X}$ the initial uniformity on $Y$ w.r.t. this set. A function $g: \sSet{X, \mathcal{U}}\to Y$ is uniformly continuous \textup{if and only if} $f_i \circ g$ is uniformly continuous for all $i\in I$.
\[ \begin{tikzcd}
Y \ar[r, "f_i"] & Z_i \\ X \ar[u, "g"] \ar[ur, swap, "f_i\circ g"]
\end{tikzcd} \]
\item Let $\{g_i: \sSet{X_i, \mathcal{U}_i} \to Y\}_{i\in I}$ be set of functions from uniform spaces and $\mathcal{Y}$ the final preconvergence on $Y$ w.r.t. this set. A function $f: Y\to \sSet{Z,\mathcal{W}}$ is uniformly continuous \textup{if and only if} $f\circ g_i$ is uniformly continuous for all $i\in I$.
\[ \begin{tikzcd}
X_i \ar[r, "g_i"] \ar[dr, swap, "f\circ g_i"] & Y \ar[d, "f"] \\ & Z
\end{tikzcd} \]
\end{enumerate}
\end{proposition}
\begin{proof}
(1) The uniform continuity of $g$ is equivalent to $\upset(g|g)^{\imf\imf}(H)\in \mathcal{X}$ for all $H\in \mathcal{U}$. By the proposition this is equivalent to
\[ \forall i \in I: \; \upset(f_i|f_i)^{\imf\imf}\big(\upset(g|g)^{\imf\imf}(H)\big) = \upset\big((f_i|f_i)\circ (g|g)\big)^{\imf\imf}(H) = \upset(f_i\circ g|f_i\circ g)^{\imf\imf}(H) \in \mathcal{W}_i,  \]
using \ref{monotonicityOrderClosure}, \ref{functorialityImageFunction} and \ref{parallelCompositionOfMorphisms}.
This is equivalent to the continuity of $f_i\circ g$ for all $i\in I$.

(2) If $f$ is uniformly continuous, then so is $f\circ g_i$, by uniform continuity of $g_i$ (\ref{initialFinalUniformity}) and uniform continuity of the composition (\ref{compositionUniformlyContinuousFunctions}).

Now suppose that $f$ is not uniformly continuous. Then there exists $i\in I$ and filter $H\in \mathcal{U}_i$ such that $\upset (g_i|g_i)^{\imf\imf}(H)$ is not an element of the final uniformity $\mathcal{Y}$. Then, by \ref{initialFinalUniformity}, does not contain a composition of images of elements of uniform filters. In particular, $H$ is not a uniform filter. This is a contradiction.
\end{proof}

\begin{proposition}
Let $Y$ be a set. Let $\{f_i: Y\to \sSet{Z_i, \mathcal{W}_i}\}_{i\in I}$ be set of functions to uniform spaces and $\mathcal{X}$ the initial uniformity on $Y$ w.r.t. this set. Then the convergence $\Gamma(\mathcal{X})$ is the initial convergence on $Y$ w.r.t.\ the functions $\{f_i: Y\to \sSet{Z_i, \Gamma(\mathcal{W}_i)}\}_{i\in I}$.
\end{proposition}
\begin{proof}
Take arbitrary $F\in\powerfilters(Y)$ and $y\in Y$. Then we have, by \ref{initialFinalUniformity},
\begin{align*}
F \overset{\Gamma(\mathcal{X})}{\longrightarrow} y &\iff F\otimes \pfilter{y} \in \mathcal{X} \\
&\iff \forall i\in I:\; \upset (f_i|f_i)^{\imf\imf}(F\otimes \pfilter{y}) \in \mathcal{W}_i \\
&\iff \forall i\in I:\; \upset (f_i)^{\imf\imf}(F)\otimes \pfilter{f}_i(y) \in \mathcal{W}_i \\
&\iff \forall i\in I:\; \upset (f_i)^{\imf\imf}(F)\overset{\Gamma(\mathcal{W}_i)} f_i(y).
\end{align*}
By \ref{initialFinalConvergence}, this last statement is equivalent to the convergence of $F$ to $y$ in the initial convergence w.r.t.\ the functions $\{f_i: Y\to \sSet{Z_i, \Gamma(\mathcal{W}_i)}\}_{i\in I}$.
\end{proof}

\begin{example}
The final convergence w.r.t.\ functions from uniform spaces is in general not induced by the final uniformity.

Consider the line $\interval{0,1}$ with the usual uniformity and convergence and consider the function
\[ f: \interval{0,1}\to \{0,1\}: x\mapsto \begin{cases}
0 & (x\leq 1/2) \\
1 & (x > 1/2).
\end{cases} \]
Now let $\{0,1\}$ have the final convergence w.r.t.\ $f$. The non-trivial filters on $\{0,1\}$ are $\pfilter{0}, \pfilter{1}$ and $\big\{\{0,1\}\big\}$.

We claim that $\big\{\{0,1\}\big\}$ converges to $0$, but not $1$. Indeed consider the filter with base $\setbuilder{\ball(1/2, \epsilon)}{\epsilon>0}$, which converges to $1/2\in \interval{0,1}$. Then $f^{\imf\imf}\big(\setbuilder{\ball(1/2, \epsilon)}{\epsilon>0}\big) = \big\{\{0,1\}\big\}$, which converges to $f(1/2) = 0$ because the final convergence makes $f$ continuous.

To show that $\big\{\{0,1\}\big\}$ does not converge to $1$, it is enough to show that $\pfilter{0}$ does not converge to $1$. Indeed, suppose there was a filter $F\in \powerfilters\big(\interval{0,1}\big)$ that converges to some $x\in \interval[oc]{1/2,1}$, then $\interval[oc]{1/2,1}\in F$ because it is open in $\interval{0,1}$. Thus $\{1\}\in f^{\imf\imf}(F) \neq \pfilter{0}$.

In summary, we have
\[ \big\{\{0,1\}\big\} \to 1, \qquad \pfilter{0} \to 0 \qquad\text{and}\qquad \pfilter{1}\to 0,1. \]
The convergence of $\pfilter{1}$ to $0$ is clear by upwards closure.

Now this convergence is not symmetric ($R_0$), and in particular not reciprocal ($R_1$), so it cannot be induced by any uniformity by \ref{associatedUniformConvergence}. In particular it is not induced by the final uniformity, which in this case contains all filters except the trivial one.
\end{example}

\begin{proposition} \label{topologicalInitialUniformity}
Let $Y$ be a set. Let $\{f_i: Y\to \sSet{Z_i, \mathcal{W}_i}\}_{i\in I}$ be set of functions to \emph{topological} uniform spaces and $\mathcal{X}$ the initial uniformity on $Y$ w.r.t. this set. Then $\mathcal{X}$ is topological and
\[ \entourage_\mathcal{X} = \mathfrak{F}\bigcup_{i\in I} (f_i|f_i)^{\preimf\imf}\big(\entourage_{\mathcal{W}_i}\big) = \mathfrak{F}\setbuilder{(f_i|f_i)^\preimf(U)}{i\in I, U\in \entourage_{\mathcal{W}_i}}. \]
\end{proposition}
\begin{proof}
(1) We have, by \ref{initialFinalUniformity} and \ref{filterPreimageImageGaloisConnection},
\begin{align*}
H \in \mathcal{X} &\iff \forall i\in I: \upset(f_i|f_i)^{\imf\imf}(H) \in \mathcal{W}_i \\
&\iff \forall i\in I: \entourage_{\mathcal{W}_i} \subseteq \upset(f_i|f_i)^{\imf\imf}(H) \\
&\iff \forall i\in I: \upset (f_i|f_i)^{\preimf\imf}\big(\entourage_{\mathcal{W}_i}\big) \subseteq H \\
&\iff \bigcup_{i\in I} \upset (f_i|f_i)^{\preimf\imf}\big(\entourage_{\mathcal{W}_i}\big) \subseteq H.
\end{align*}
Thus the initial uniformity is topological and
\begin{align*}
\entourage_\mathcal{X} &= \mathfrak{F}\bigcup_{i\in I} \upset (f_i|f_i)^{\preimf\imf}\big(\entourage_{\mathcal{W}_i}\big) \\
&= \mathfrak{F}\bigcup_{i\in I}\setbuilder{(f_i|f_i)^\preimf(U)}{U\in \entourage_{\mathcal{W}_i}} \\
&= \mathfrak{F}\setbuilder{(f_i|f_i)^\preimf(U)}{i\in I, U\in \entourage_{\mathcal{W}_i}}.
\end{align*}
\end{proof}

\subsubsection{Product uniformity}
\begin{definition}
Let $\sSet{X_i, \mathcal{U}_i}$ be a uniform space for all $i\in I$. The \udef{product uniformity} $\bigotimes_{i\in I}\mathcal{U}_i$ is the initial uniformity on $\bigtimes_{i\in I}X_i$ w.r.t. the set of projections $p_i: \bigtimes_{i\in I}X_i \to X_i$.

The product uniform space is denoted $\prod_{i\in I}X_i$.
\end{definition}

\subsubsection{Subspace uniformity}
\begin{definition}
Let $\sSet{X, \mathcal{U}}$ be a uniform space and $A\subseteq X$ a subset. The \udef{subspace uniformity} $\mathcal{U}|_A$ is the initial uniformity w.r.t. the inclusion $A\hookrightarrow X$.
\end{definition}

\begin{proposition} \label{subspaceUniformConvergence}
Let $\sSet{X, \mathcal{U}}$ be a uniform space and $A\subseteq X$ a subset. Then $\Gamma(\mathcal{U}|_A) = \Gamma(\mathcal{U})|_A$. 
\end{proposition}
\begin{proof}
Take $F\in \powerfilters{A}$ and $a\in A$. Let $\iota: A\hookrightarrow X$ be the inclusion function. Then we have, by \ref{initialFinalConvergence} and \ref{initialFinalUniformity}
\begin{align*}
F \overset{\Gamma(\mathcal{U})|_A}{\longrightarrow} a &\iff \upset\iota^{\imf\imf}(F) \overset{\Gamma(\mathcal{U})}{\longrightarrow} a \\
&\iff \upset\iota^{\imf\imf}(F)\otimes \pfilter{a} \in \mathcal{U} \\
&\iff \upset(\iota|\iota)^{\imf\imf}\big(F\otimes \pfilter{a}\big) \in \mathcal{U} \\
&\iff F\otimes \pfilter{a} \in \mathcal{U}|_A \\
&\iff F \overset{\Gamma(\mathcal{U}|_A)}{\longrightarrow} a.
\end{align*}
\end{proof}

\section{Properties of uniform spaces}
\subsection{(Total) boundedness}
TODO ref: BOUNDEDNESS IN UNIFORM SPACES, TOPOLOGICAL GROUPS, AND HOMOGENEOUS SPACES - Atkin

J. Hejcman, Boundedness in uniform space and topological group, J. Czechosolvak
math.,Vol. 9, No. 4 (1959), 544-563.

\begin{definition}
Let $\sSet{X,\mathcal{U}}$ be a uniform space. We call $\mathcal{D} \subseteq \powerset(X^2)$ a \udef{uniform cover} of $\sSet{X,\mathcal{U}}$ if
\[ \forall H\in \mathcal{U}:\quad \mathcal{D}\mesh H. \]
\begin{itemize}
\item We say $B\subseteq X$ is \udef{totally bounded} if for all uniform covers $\mathcal{D}$, there exists a \emph{finite} $\mathcal{C} \subseteq \setbuilder{Ax}{A\in \mathcal{D}, x\in X}$ such that $B\subseteq \bigcup \mathcal{C}$.
\item We say $B\subseteq X$ is \udef{bounded} if for all uniform covers $\mathcal{D}$, there exists a \emph{finite} $\mathcal{C} \subseteq \setbuilder{A^nx}{A\in \mathcal{D}, n\in \N, x\in X}$ such that $B\subseteq \bigcup \mathcal{C}$.
\end{itemize}
\end{definition}

\begin{lemma}
Total boundedness implies boundedness.
\end{lemma}

\begin{lemma} \label{topologicalBoundednessLemma}
Let $\sSet{X,\mathcal{U}}$ be a uniform space and $B\subseteq X$ a subset. Then
\begin{enumerate}
\item if $B$ is totally bounded, then for all $A\in \entourage_\mathcal{U}$ there exists a finite $S\subseteq X$ such that $B\subseteq \bigcup_{x\in S}Ax$;
\item if $B$ is bounded, then for all $A\in \entourage_\mathcal{U}$ there exists $n\in \N$ and a finite $S\subseteq X$ such that $B\subseteq \bigcup_{x\in S}A^nx$.
\end{enumerate}
The converses hold in a topological uniform space.
\end{lemma}
\begin{proof}
We may simply observe that $\{V\}$ is a uniform cover for all $V\in \entourage_\mathcal{U}$. This proves (1) and (2).

Now assume $\mathcal{U}$ topological. Then $\entourage_\mathcal{U} \in \mathcal{U}$, so each uniform cover contains an entourage. 
\end{proof}

\begin{lemma} \label{singletonsTotallyBounded}
Let $\sSet{X,\mathcal{U}}$ be a uniform space and $x\in X$. Then $\{x\}$ is totally bounded.
\end{lemma}
\begin{proof}
Let $\mathcal{D}$ be a uniform cover. As $\pfilter{x}\otimes \pfilter{x}\in \mathcal{U}$, there exists $A\in \mathcal{D} \cap \pfilter{x}\otimes \pfilter{x}$. Now $(x,x)\in A$, so $\mathcal{C} = \{Ax\}$ is finite and such that $\{x\} \subseteq \bigcup \mathcal{C} = Ax$.
\end{proof}

\begin{lemma} \label{boundedSetsIdeal}
Let $\sSet{X,\mathcal{U}}$ be a uniform space. The set of all totally bounded subsets
\begin{enumerate}
\item is an ideal in $\powerset(X)$;
\item covers $X$.
\end{enumerate}
The same holds for the set of all bounded subsets.
\end{lemma}
\begin{proof}
(1) First we show that a subset $C$ of a bounded set $B$ is bounded. Take an arbitrary uniform cover $\mathcal{D}$ and finite $\mathcal{C}\subseteq \setbuilder{Ax}{A\in \mathcal{D}, x\in X}$ such that $B\subseteq \bigcup \mathcal{C}$. Then $C\subseteq \bigcup \mathcal{C}$, so $C$ is bounded.

Now let both $B$ and $C$ be bounded sets and let $\mathcal{D}$ be a uniform cover. We can find finite $\mathcal{C}, \mathcal{C}'\subseteq \setbuilder{Ax}{A\in \mathcal{D}, x\in X}$ such that $B\subseteq \bigcup \mathcal{C}$ and $C\subseteq \bigcup \mathcal{C}'$. Then $B\cup C \subseteq \bigcup (\mathcal{C}\cup \mathcal{C}')$ and $\mathcal{C}\cup \mathcal{C}'$ is finite.

(2) Every $x\in X$ is an element of a totally bounded set as $x\in \{x\}$, which is bounded by \ref{singletonsTotallyBounded}.

(Bounded) We can repeat the arguments using $\setbuilder{A^nx}{A\in \mathcal{D}, n\in\N, x\in X}$ instead of $\setbuilder{Ax}{A\in \mathcal{D}, x\in X}$.
\end{proof}
\begin{corollary}
Every finite subset of a uniform space is totally bounded.
\end{corollary}

TODO: closure of bounded set is bounded.
\begin{proposition} \label{adherenceBoundedSet}
Let $\sSet{X, \mathcal{U}}$ be a topological uniform space and $B\subseteq X$ a bounded set. Then $\adh_{\Gamma(\mathcal{U})}(B)$ is also bounded.
\end{proposition}
TODO: also for nontopological uniform spaces?
\begin{proof}
Take an arbitrary symmetric $A\in \entourage_\mathcal{U}$. We can find a finite set $F\subseteq X$ and $n\in \N$ such that $\bigcup_{x\in F}A^nx \supseteq B$. Now $\entourage_\mathcal{U}x \subseteq \vicinity_{\Gamma(\mathcal{U})}$ by \ref{uniformContinuityEntourages}, so we have (using \ref{filterGrillIsPrime})
\begin{align*}
x\in \adh_{\Gamma(\mathcal{U})}(B) \iff& B\in \vicinity(x)^\mesh \\
\implies& B \in (\entourage x)^\mesh \\
\implies& \bigcup_{y\in F}A^ny \in (\entourage x)^\mesh \\
\implies& \exists y\in F: A^ny \in (\entourage x)^\mesh \\
\implies& \exists y\in F: A^ny \mesh Ax \\
\implies& \exists y\in F: x\in A^{n+1}y.
\end{align*}
Thus $\bigcup_{x\in F}A^{n+1}x \supseteq \adh_{\Gamma(\mathcal{U})}(B)$.
\end{proof}

\begin{proposition} \label{imageBoundedSet}
Let $f: \sSet{X,\mathcal{U}}\to \sSet{Y, \mathcal{V}}$ be a unifromly continuous function between uniform spaces and $B\subseteq X$ a subset.
\begin{enumerate}
\item If $\mathcal{D}$ is a uniform cover of $\mathcal{V}$, then $(f|f)^{\preimf\imf}(\mathcal{D})$ is a uniform cover of $\mathcal{U}$.
\item If $B$ is (totally) bounded, then $f^\imf(B)$ is also (totally) bounded.
\end{enumerate}
\end{proposition}
\begin{proof}
(1) Assume $\mathcal{D}$ is a uniform cover of $\mathcal{V}$ and take arbitrary $H\in \mathcal{U}$. Then $\upset (f|f)^{\imf\imf}(H)\in \mathcal{V}$, so there exists $A\in \mathcal{D} \cap \upset (f|f)^{\imf\imf}(H)$.
Then there exists $(f|f)^\imf(A')\in (f|f)^{\imf\imf}(H)$ such that $(f|f)^\imf(A') \subseteq A$, or, equivalently, $A'\subseteq (f|f)^\preimf(A)$. Thus $(f|f)^\preimf(A) \in H$.

(2) We prove the statement for boundedness. For total boundedness the proof is similar, fixing $n=1$.

Now take an arbitrary uniform cover $\mathcal{D}'$ of $\mathcal{V}$. As $(f|f)^{\preimf\imf}(\mathcal{D})$ is a uniform cover of $\mathcal{U}$, by point (1), we can find some finite $\mathcal{C} \subseteq \setbuilder{(f|f)^\preimf(A)^nx}{A\in \mathcal{D}', n\in\N, x\in X}$ such that $B\subseteq \bigcup \mathcal{C}$.

Now by \ref{imageFilterComposition} and \ref{principalImageUnderRelationMapping}, we have
\[ (f|f)^\preimf(A)^nx \subseteq (f|f)^\preimf(A^n)x \subseteq f^\preimf\big(A^nf(x)\big). \]
Set $\mathcal{C}' \defeq \setbuilder{A^nf(x)}{(f|f)^\preimf(A^n)x \in \mathcal{C}}$. Thus
\[ B \subseteq \bigcup_{(f|f)^\preimf(A)^nx \in \mathcal{C}}(f|f)^\preimf(A)^nx \subseteq \bigcup_{A^nf(x) \in \mathcal{C}'}f^\preimf\big(A^nf(x)\big) = f^\preimf\Big(\bigcup_{A^nf(x) \in \mathcal{C}'}A^nf(x)\Big), \]
and so also
\[ f^\imf(B) \subseteq \bigcup_{A^nf(x) \in \mathcal{C}'}A^nf(x) = \bigcup \mathcal{C}'. \]
\end{proof}

\begin{proposition} \label{boundednessUniformityInclusion}
Let $X$ be a set, $\mathcal{U} \subseteq \mathcal{V}$ two uniformities on $X$ such that $\mathcal{U} \subseteq \mathcal{V}$ and $B\subseteq X$. Then
\begin{enumerate}
\item if $B$ is $\mathcal{U}$-bounded, then $B$ is $\mathcal{V}$-bounded;
\item if $B$ is $\mathcal{U}$-totally bounded, then $B$ is $\mathcal{V}$-totally bounded.
\end{enumerate}
\end{proposition}
\begin{proof}
Any $\mathcal{V}$-uniform cover is a $\mathcal{U}$-uniform cover.
\end{proof}

\subsection{Compactness}
\begin{proposition} \label{compactUltrafilterFactorisation}
Let $\sSet{X,\mathcal{U}}$ be a compact uniform space. If $H$ is an ultrafilter in $\mathcal{U}$, then 
\begin{enumerate}
\item $p_1^{\imf\imf}[H]\otimes p_2^{\imf\imf}[H] \in\mathcal{U}$;
\item $p_1^{\imf\imf}[H]$ and $p_2^{\imf\imf}[H]$ converge in $\Gamma(\Xi(\mathcal{U}))$;
\item $\lim\big(p_1^{\imf\imf}[H]\big) = \lim\big(p_2^{\imf\imf}[H]\big)$.
\end{enumerate}
\end{proposition}
\begin{proof}
Let $H$ be an ultrafilter in $\mathcal{U}$. Then $p_1^{\imf\imf}[H]$ and $p_2^{\imf\imf}[H]$ are ultrafilters by \ref{projectionsOfUltrafilterAreUltra}. By compactness there exist $x,y\in X$ such that $p_1^{\imf\imf}[H] \overset{\Gamma(\Xi(\mathcal{U}))}{\longrightarrow} x$ and $p_2^{\imf\imf}[H] \overset{\Gamma(\Xi(\mathcal{U}))}{\longrightarrow} y$. Thus $p_1^{\imf\imf}[H]\otimes \pfilter{x}\in\mathcal{U}$ and $p_2^{\imf\imf}[H]\otimes \pfilter{y}\in\mathcal{U}$. 

By \ref{filterCompositionFactorisationLemma} we have
\[ \pfilter{x}\otimes \pfilter{y} = \pfilter{x}\otimes p_1^{\imf\imf}[H]; H; p_2^{\imf\imf}[H] \otimes \pfilter{y} \in \mathcal{U}. \]

Finally
\[ p_1^{\imf\imf}[H] \mathrel{\Xi(\mathcal{U})} \pfilter{x}, \quad \pfilter{x} \mathrel{\Xi(\mathcal{U})} \pfilter{y} \quad\text{and}\quad \pfilter{y}\mathrel{\Xi(\mathcal{U})} p_2^{\imf\imf}[H], \]
so $p_1^{\imf\imf}[H] \mathrel{\Xi(\mathcal{U})} p_2^{\imf\imf}[H]$ by transitivity. Thus $p_1^{\imf\imf}[H]\otimes p_2^{\imf\imf}[H] \in \mathcal{U}$.
\end{proof}

\subsection{Depth properties}
\subsubsection{Choquet uniform spaces}
\begin{definition}
Let $\sSet{X,\mathcal{U}}$ be a uniform space. We say $\mathcal{U}$ is \udef{uniformly Choquet} if for all $H\in \powerfilters(X^2)$,
\[ H\in\mathcal{U} \qquad\iff\qquad \text{$I \in \mathcal{U}$ for all ultrafilters $I$ such that $H\subseteq I$.} \]
\end{definition}

\begin{lemma}
Let $\sSet{X,\mathcal{U}}$ be a uniform space. If $\mathcal{U}$ is uniformly Choquet, then $\Gamma(\Xi(\mathcal{U}))$ is Choquet.
\end{lemma}
\begin{proof}
????????????????
\end{proof}


\section{Cauchy structure}
\begin{definition}
Let $\sSet{X,R}$ be a uniform relation space and let $\mathcal{F}\subseteq \powerfilters(X)$ be defined by
\[ F\in \mathcal{F} \qquad\defequiv\qquad F\mathrel{R} F. \]
Then $\mathcal{F}$ is called the \udef{induced Cauchy structure} and $\sSet{X,\mathcal{F}}$ is called the \udef{induced Cauchy space}.
\end{definition}
\begin{lemma}
The induced Cauchy space $\sSet{X, \mathcal{F}}$ of a uniform relation space $\sSet{X,R}$ is a Cauchy space.

If the uniform relation is of finite depth, then the induced Cauchy space is too.
\end{lemma}
\begin{proof}
We immediately have $\pfilter{x}\in\mathcal{F}$ for all $x\in X$ because $\pfilter{x}\mathrel{R} \pfilter{x}$.

We need to show upwards closure. Let $F\in \mathcal{F}$, meaning $F\mathrel{R} F$, and $F\subseteq G$. By upwards closure of $F\mathrel{R}$, we have $F\mathrel{R} G$. By symmetry we have $G\mathrel{R} F$ and by transitivity $G\mathrel{R} G$, so $G\in\mathcal{F}$.

Finally, assume the uniform relation is of finite depth. Take $F,G\in \mathcal{F}$ such that $F\amesh G$. Then $F\mathrel{F}G$ by \ref{uniformRelationRelatedElementLemma}.
From $F\mathrel{R}F$ and $F\mathrel{R}G$, we conclude $F\mathrel{R}F\cap G$ by finite depth, so $F\cap G$ is Cauchy by \ref{uniformRelationRelatedElementLemma}.
\end{proof}

\begin{lemma}
Let $\sSet{X, R}$ be a uniform relation space, $F$ a Cauchy filter and $F\mathrel{R} G$. Then $G$ is a Cauchy filter.
\end{lemma}
\begin{proof}
The relationships $F\mathrel{R} F$ and $F\mathrel{R} G$ imply $G\mathrel{R} G$ by transitivity and symmetry.
\end{proof}

\begin{lemma}
Let $\sSet{X,R}$ be a uniform relation space with induced Cauchy structure $\mathcal{F}$, $F\in \powerfilters(X)$ and $x\in X$. Then
\begin{enumerate}
\item if $F\cap\pfilter{x} \in \mathcal{F}$, then $F\overset{\Gamma(R)}{\longrightarrow} x$;
\item the converse holds if $\Gamma(R)$ is a Kent space.
\end{enumerate}
\end{lemma}
\begin{proof}
(1) By upwards closure, we have that $\big(F\cap \pfilter{x}\big)\mathrel{R}\big(F\cap \pfilter{x}\big)$ implies $F\mathrel{R}\pfilter{x}$, so $F\overset{\Gamma(R)}{\longrightarrow} x$.

(2) If $\Gamma(R)$ is a Kent space, then $F\overset{\Gamma(R)}{\longrightarrow} x$ implies $(F\cap\pfilter{x})\overset{\Gamma(R)}{\longrightarrow} x$. So $(F\cap\pfilter{x})\mathrel{R}\pfilter{x}$ and thus $\big(F\cap \pfilter{x}\big)\mathrel{R}\big(F\cap \pfilter{x}\big)$ by \ref{uniformRelationRelatedElementLemma}.
\end{proof}
TODO: when is $\Gamma(R)$ a Kent space?

\begin{lemma} \label{uniformlyConvergentImpliesCauchy}
Let $\sSet{X,R}$ be a uniform relation space and $F\in\powerfilters(X)$. If $F$ converges, then $F$ is a Cauchy filter.
\end{lemma}
\begin{proof}
If $F$ converges uniformly to $x$, then $F\mathrel{R}\pfilter{x}$ and by symmetry also $\pfilter{x}\mathrel{R}F$. By transitivity $F\mathrel{R}F$. 
\end{proof}
If the converse to this lemma holds, then the uniform space is called complete.

\begin{lemma} \label{continuousImageOfCauchy}
Let $\sSet{X,\mathcal{U}}$, $\sSet{Y,\mathcal{V}}$ be uniform spaces and $f: X\to Y$ a uniformly continuous function. If $F\in\powerfilters(X)$ is Cauchy, then $\upset f^{\imf\imf}(F)$ is also Cauchy.
\end{lemma}
\begin{proof}
TODO
\end{proof}

\subsection{Completeness}

\begin{definition}
A uniform relation space $\sSet{X,R}$ is called \udef{complete} if all Cauchy filters converge.
\end{definition}

A uniform space is complete iff for all $F\in\powerfilters(X)$,
\[ F\mathrel{R}F \quad\iff\quad \exists x\in X:\; F\mathrel{R}\pfilter{x}. \]

\begin{proposition} \label{compactImpliesComplete}
Let $\sSet{X,R}$ be a uniform relation space. If $\sSet{X,\Gamma(R)}$ is compact, then $\sSet{X,R}$ is complete.
\end{proposition}
\begin{proof}
Assume $\sSet{X,\Gamma(R)}$ is compact and take $F\in\powerfilters(X)$ such that $F\mathrel{R}F$. Then we can find an ultrafilter $F'\supseteq F$ by the ultrafilter lemma \ref{ultrafilterLemma}. By compactness $F'$ converges uniformly, so $F'\mathrel{R}\pfilter{x}$ for some $x\in X$. Then $F \mathrel{R}\pfilter{x}$ by \ref{uniformRelationUpwardsClosure}, so $F$ is uniformly convergent. Because $F$ was chosen arbitrarily this makes $\sSet{X,R}$ complete.
\end{proof}

\begin{proposition} \label{closedComplete}
Let $\sSet{X,\mathcal{U}}$ be a uniform space and $A\subseteq X$ a subset. Then
\begin{enumerate}
\item if $X$ is complete and $A$ is closed, then $A$ is complete;
\item if $X$ is Hausdorff and $A$ is complete, then $A$ is closed.
\end{enumerate}
\end{proposition}
\begin{proof}
(1) Let $F\in\powerfilters(A)$ be a Cauchy filter in $A$ and $\iota: A\hookrightarrow X$ the inclusion function. Then $\upset \iota^{\imf\imf}(F)\in\powerfilters(X)$ is a Cauchy filter in $X$. So $\upset \iota^{\imf\imf}(F)$ converges to $x$ in $X$ by completeness of $X$. Now, by \ref{properSubsemilatticeLemma},
\[ A \in \upset \iota^{\imf\imf}(F) \subseteq \big(\upset \iota^{\imf\imf}(F)\big)^{\mesh} \subseteq \vicinity_{\Gamma(\mathcal{U})}(x)^{\mesh}, \]
so $x\in \adh_{\Gamma(\mathcal{U})}(A) = A$. Thus $F$ converges to $x$ in $\Gamma(\mathcal{U})|_A$ by \ref{initialFinalConvergence} and by \ref{subspaceUniformConvergence} this is the same as uniform convergence in $A$ (i.e.\ in $\Gamma(\mathcal{U}|_A)$).

(2) We need to show that $\adh(A)\subseteq A$. Take $x\in \adh(A)$. By \ref{principalAdherenceInherence}, this means that there exists a filter $F\in \powerfilters(X)$ that converges to $x$ and is such that $A\in F$. By \ref{uniformRelationRelatedElementLemma}, we have that $F$ is Cauchy. By \ref{setTraceFilterLemma}, we have $F = \upset(\iota^{\imf\imf}\circ \iota^{\preimf\imf})(F)$. Then we have, using \ref{functionsOfProductFilters} and \ref{initialFinalUniformity},
\begin{align*}
F\otimes F \in \mathcal{U} &\iff \big(\upset(\iota^{\imf\imf}\circ \iota^{\preimf\imf})(F)\big)\otimes \big(\upset(\iota^{\imf\imf}\circ \iota^{\preimf\imf})(F)\big) \in \mathcal{U} \\
&\iff \upset(\iota|\iota)^{\imf\imf}\big((\iota|\iota)^{\preimf\imf}(F)\otimes (\iota|\iota)^{\preimf\imf}(F)\big) \in \mathcal{U} \\
&\iff \upset (\iota|\iota)^{\preimf\imf}(F)\otimes (\iota|\iota)^{\preimf\imf}(F) \in \mathcal{U}|_A.
\end{align*}
Thus $\upset (\iota|\iota)^{\preimf\imf}(F)$ is a Cauchy filter in $A$, so it converges to some $a\in A$. By continuity of the inclusion, we have that $\upset(\iota^{\imf\imf}\circ \iota^{\preimf\imf})(F) = F$ also converges to $a$. As $\Gamma(\mathcal{U})$ is Hausdorff, we have $x = a$, so $x\in A$. This shows indeed that $\adh(A)\subseteq A$.
\end{proof}

\subsection{Precompactness}
\begin{definition}
Let $\sSet{X,R}$ be a uniform relation space, $F\in\powerfilters(X)$ a filter and $A\subseteq X$.
Then 
\begin{itemize}
\item we call $F$ \udef{precompactoid} if every ultrafilter that contains $F$ is a Cauchy filter;
\item we call $X$ \udef{precompact} if $\{X\}$ is precompactoid.
\end{itemize}
\end{definition}

\begin{proposition} \label{compactPrecompactComplete}
A uniform space is compact \textup{if and only if} it is precompact and complete.
\end{proposition}
\begin{proof}
Th direction $\Rightarrow$ is given by \ref{uniformlyConvergentImpliesCauchy} and \ref{compactImpliesComplete}.

For the other direction: By precompactness, every ultrafilter is an Cauchy filter. By completeness, every Cauchy filter (and thus every ultrafilter) converges.
\end{proof}

\begin{proposition} \label{precompactTotallyBounded}
Let $\sSet{X,\mathcal{U}}$ be a symmetric uniform space. Then $X$ is precompact \textup{if and only if} $X$ is totally bounded.
\end{proposition}
\begin{proof}
$\boxed{\Rightarrow}$ Assume $X$ is precompact and, towards a contradiction, that there exists a uniform cover $\mathcal{D}$ such that there is no finite subcover of $\setbuilder{Ax}{A\in \mathcal{D}, x\in X}$. Then
\[ \setbuilder{(A_0x_0\cup \ldots \cup A_nx_n)^c}{n\in \N; A_0,\ldots, A_n \in \mathcal{D}; x_0, \ldots, x_n\in X} \]
is a filter base and does not contain $\emptyset$, so the filter is generates is proper and we can find an ultrafilter $F$ that contains it by the ultrafilter lemma \ref{ultrafilterLemma}.

Now $F$ is Cauchy, so $F\otimes F\in \mathcal{U}$ and thus $\mathcal{D}\mesh F$, i.e.\ there exists a $A\in \mathcal{D}\cap F\otimes F$. Then $Ax\in F$ for some $x\in X$ by \ref{principalImageOfInProductFilter}. By construction of $F$, we also have $(Ax)^c\in F$. Thus $\emptyset =  Ax \cap (Ax)^c \in F$, so $F$ is trivial and not an ultrafilter. This is a contradiction.

$\boxed{\Leftarrow}$ Assume, towards a contradiction, that $X$ is not precompact. Then there exists an ultrafilter $F$ that is not Cauchy. This means that $H \not\subseteq F\otimes F$ for all symmetric $H\in\mathcal{U}$ and, in particular $H;H \not\subseteq F\otimes F$. As $H;H$ is generated by sets of the form $A;A$, where $A\in H$ is symmetric, we can find a set of this form that is not an element of $F\otimes F$. Let $\mathcal{D}$ consist of a set $A\in H$ such that $A;A\notin F\otimes F$ for all for all symmetric $H\in \mathcal{U}$. As $\mathcal{U}$ is symmetric, each $H'\in \mathcal{U}$ contains a symmetric $H$ and the corresponding $A$ is an element of $H'$. This means that $\mathcal{D}$ is a uniform cover.

By assumption, there exists a finite subset $\setbuilder{A_0x_0, \ldots, A_nx_n}{\seq{A_k}_{k=0}^n\in \mathcal{D}^\N; \seq{x_k}_{k=0}^n \in X^\N}$ that covers $X$, so $A_0x_0 \cup \ldots \cup A_nx_n = X \in F$. Now $F$ is prime by \ref{booleanMaximalFiltersIdeals} and thus at least one $A_0x_0, \ldots, A_nx_n$ is an element of $F$. Call this $Ax$.

In other words, there exists $A\in \mathcal{D}$ and $x\in X$ such that $Ax \in F$. By symmetry, $xA\in F$. Thus $A;A\in F\otimes F$ by \ref{principalImageOfInProductFilter}. This is a contradiction.
\end{proof}



\section{Uniformities from binary functions}
\begin{definition}
Let $X,Y,Z$ be sets and $f: X\times Y\to Z$ a binary function. Let $\mathcal{W}$ be a uniformity on $Z$. The \udef{associated relation} on $\big(\powerfilters(X^2),\powerfilters(Y^2)\big)$ is defined by
\[ U\mathrel{\mathrm{R}_{f}} V \quad\defequiv\quad \upset\big((f|f)\circ t \big)^{\imf\imf}(U\otimes V) \in \mathcal{W}, \]
where $t \defeq \big((\pi_1\circ \pi_1, \pi_1\circ \pi_2),(\pi_2\circ \pi_1, \pi_2\circ \pi_2)\big)$.
\end{definition}

\begin{lemma} \label{polarUniformityLemma}
Let $f: f: X\times Y\to \sSet{Z, \mathcal{W}}$ be a binary function and $U\in \powerfilters(X^2), V,V'\in \powerfilters(Y^2)$. Then 
\begin{enumerate}
\item $U\mathrel{\mathrm{R}_{f}}$ is upwards closed;
\item $U\mathrel{\mathrm{R}_{f}} V \iff U^\transp\mathrel{\mathrm{R}_{f}} V^\transp$;
\item $\left.\begin{aligned}U&\mathrel{\mathrm{R}_{f}} V \\ U^\transp;U&\mathrel{\mathrm{R}_{f}} V'\end{aligned}\right\} \implies U\mathrel{\mathrm{R}_{f}} V;V'$;
\item $\left.\begin{aligned}
U &\subseteq \upset\{\id_X\} \\
U&\mathrel{\mathrm{R}_{f}} V \\
U&\mathrel{\mathrm{R}_{f}} V'
\end{aligned}\right\} \implies U\mathrel{\mathrm{R}_{f}} V;V'$;
\end{enumerate}
\end{lemma}
\begin{proof}
(1) Let $W,W'\powerfilters(Y^2)$ be such that $U\mathrel{\mathrm{R}_{f}}W$ and $W\subseteq W'$. We then need to show $U\mathrel{\mathrm{R}_{f}}W'$. We have 
\[ \mathcal{W} \ni \upset\big((f|f)\circ t \big)^{\imf\imf}(U\otimes W) \subseteq \upset\big((f|f)\circ t \big)^{\imf\imf}(U\otimes W'), \]
so $\upset\big((f|f)\circ t \big)^{\imf\imf}(U\otimes W') \in \mathcal{W}$ by upwards closure of $\mathcal{W}$ and thus $U\mathrel{\mathrm{R}_{f}}W'$.

(2) We have $\upset\big((f|f)\circ t \big)^{\imf\imf}(U\otimes V)^\transp = \upset\big((f|f)\circ t \big)^{\imf\imf}(U^\transp\otimes V^\transp)$. Indeed, for all $A$ in a base of $U$, $B$ in a base of $V$, $(a_1, a_2)\in A$ and $(b_1, b_2)\in B$, we have
\begin{align*}
\big(f(a_2, b_2), f(a_1, b_1)\big) \in \big((f|f)\circ t \big)^{\imf}(A\times B)^\transp &\iff \big(f(a_1, b_1), f(a_2, b_2)\big) \in \big((f|f)\circ t \big)^{\imf}(A\times B) \\
&\iff \big(f(a_2, b_2), f(a_1, b_1)\big) \in \big((f|f)\circ t \big)^{\imf}(A^\transp\times B^\transp).
\end{align*}
Then
\begin{align*}
\upset\big((f|f)\circ t \big)^{\imf\imf}(U\otimes V) \in \mathcal{W} &\iff \upset\big((f|f)\circ t \big)^{\imf\imf}(U\otimes V)^\transp \in \mathcal{W} \\
&\iff \upset\big((f|f)\circ t \big)^{\imf\imf}(U^\transp\otimes V^\transp) \in \mathcal{W}.
\end{align*}

(3) It is enough to prove
\[ \upset\big((f|f)\circ t \big)^{\imf\imf}(U\otimes V);\upset\big((f|f)\circ t \big)^{\imf\imf}(U^\transp;U\otimes V') \subseteq \upset\big((f|f)\circ t \big)^{\imf\imf}(U\otimes V;V'), \]
because then $U\mathrel{\mathrm{R}_{f}} V;V'$ follows by upwards closure and closure under composition of $\mathcal{W}$.

Any subset of $\upset\big((f|f)\circ t \big)^{\imf\imf}(U\otimes V);\upset\big((f|f)\circ t \big)^{\imf\imf}(U^\transp;U\otimes V')$ contains a set of the form $\big((f|f)\circ t \big)^{\imf}(A\otimes B);\big((f|f)\circ t \big)^{\imf}(A^\transp;A\otimes C)$ for some $A\in U, B\in V$ and $C\in V'$. It is then enough to show that this set contains an element of $\upset\big((f|f)\circ t \big)^{\imf\imf}(U\otimes V;V')$. Indeed we claim
\[ \big((f|f)\circ t \big)^{\imf}(A\otimes B;C) \subseteq \big((f|f)\circ t \big)^{\imf}(A\otimes B);\big((f|f)\circ t \big)^{\imf}(A^\transp;A\otimes C). \]
To show this, take arbitrary $\big(f(a_1, b), f(a_2, c)\big) \in \big((f|f)\circ t \big)^{\imf}(A\otimes B;C)$, which means that $(a_1, a_2)\in A$ and there exists $y\in Y$ such that $(b,y)\in B$ and $(y,c)\in C$. Then
\begin{align*}
\big(f(a_1, b), f(a_2, c)\big) &\in \Big\{\big(f(a_1, b),f(a_2, c) \big)\Big\} \\
&= \Big\{\big(f(a_1, b),f(a_2, y) \big)\Big\}; \Big\{\big(f(a_2, y), f(a_2, c)\big)\Big\} \\
&\subseteq \big((f|f)\circ t \big)^{\imf}(A\otimes B);\big((f|f)\circ t \big)^{\imf}(A^\transp;A\otimes C),
\end{align*}
which is straightforwards, noting that $(a_2,a_2)\in A^\transp; A$.

(4) We can repeat the proof of (3), substituting $U$ for $U^\transp;U$ and $A$ for $A^\transp;A$, because in this case we have $(a_2, a_2)\in A$.
\end{proof}

\subsection{Polars generated by binary functions}
\begin{lemma}
Let $f: X\times Y\to \sSet{Z, \mathcal{W}}$ be a binary function, $\mathcal{W}$ a uniformity and $\mathcal{A}\subseteq \powerfilters(X^2)$ a set. Then
\begin{enumerate}
\item $\mathcal{A}^{\mathrm{R}_{f}}$ is upwards closed;
\item $\mathcal{A}^{\mathrm{R}_{f}} = (\upset \mathcal{A})^{\mathrm{R}_{f}}$;
\item $\mathcal{A}^{\mathrm{R}_{f}} = (\mathcal{A}^\transp)^{\mathrm{R}_{f}}$.
\end{enumerate}
\end{lemma}
\begin{proof}
(1) By \ref{polarUniformityLemma}.

(2) From $\mathcal{A} \subseteq \upset \mathcal{A}$, we have $\mathcal{A}^{\mathrm{R}_{f}} \supseteq (\upset \mathcal{A})^{\mathrm{R}_{f}}$. Now take 

(3)
\end{proof}

\begin{proposition}
Let $f: X\times Y\to \sSet{Z, \mathcal{W}}$ be a binary function, $\mathcal{W}$ a uniformity and $\mathcal{A}\subseteq \powerfilters(X^2)$ a set.
Then the following conditions are sufficient for $\mathcal{A}^{\mathrm{R}_{f}}$ to be a preuniformity on $Y$:
\begin{enumerate}
\item $\mathcal{A}$ is a uniformity;
\item $\mathcal{A}$ is closed under transposition and composition;
\item $\mathcal{A}$ is closed under transposition and each $U\in \mathcal{A}$ is a subset of $\upset\{\id_X\}$.
\end{enumerate}
\end{proposition}
\begin{proof}
Straightforward application of \ref{polarUniformityLemma}.
\end{proof}


\subsection{Function spaces}
In this section we study relations associated to $\evalMap: (X\to Y) \times X \to Y$ and $\evalMap: \cont(X, Y) \times X \to Y$.

\begin{definition}
Consider the evaluation map $\evalMap: (X\to Y)\times X \to \sSet{Y,\mathcal{V}}$, where $X, Y$ are sets and $\mathcal{V}$ a uniformity on $Y$. Then
\begin{itemize}
\item the \udef{uniformity of pointwise convergence} $\mathcal{U}_p$ is given by $\setbuilder{\pfilter{x}\otimes\pfilter{x}}{x\in X}^{\uRelEv^\transp}$;
\item the \udef{uniformity of uniform convergence} $\mathcal{U}_u$ is given by $\{\id_X\}^{\uRelEv^\transp}$.
\end{itemize}
Now suppose $\xi$ is a convergence on $X$. Then
\begin{itemize}
\item the \udef{uniformity of continuous convergence} is given by \\ $\setbuilder{F\otimes F}{\text{$F$ converges in $\xi$}}^{\uRelEv^\transp}$;
\item a set $H\subseteq (X\to Y)$ is \udef{equicontinuous} at some $x\in X$ if
\[ \upset\{\id_H\} \in \setbuilder{F\otimes \pfilter{x}}{F\overset{\xi}{\longrightarrow}x}^{\uRelEv^\transp}. \]
\item a set $H\subseteq (X\to Y)$ is \udef{equicontinuous} if it is equicontinuous at all $x\in X$.
\end{itemize}
\end{definition}

\begin{lemma}
Consider the function $\evalMap: (X\to Y)\times X \to \sSet{Y,\mathcal{V}}$.
A set $H\subseteq (X\to Y)$ is equicontinuous \textup{if and only if}
\[ \upset\{\id_H\} \in \setbuilder{F\otimes \pfilter{x}}{F\overset{\xi}{\longrightarrow}x}^{\uRelEv^\transp}. \]
\end{lemma}
\begin{proof}
We have that $H$ is equicontinuous iff it is equicontinuous at all $x\in X$, i.e.\
\begin{align*}
\upset\{\id_H\}\in \bigcap_{x\in X}\setbuilder{F\otimes \pfilter{x}}{F\overset{\xi}{\longrightarrow}x}^{\uRelEv^\transp} &= \Big(\bigcup_{x\in X}\setbuilder{F\otimes \pfilter{x}}{F\overset{\xi}{\longrightarrow}x}\Big)^{\uRelEv^\transp} \\
&= \setbuilder{F\otimes \pfilter{x}}{F\overset{\xi}{\longrightarrow}x}^{\uRelEv^\transp}.
\end{align*}
\end{proof}


\begin{lemma}
Let $X$ be a set and $\sSet{Y, \mathcal{V}}$ a uniform space. Then the uniformity of pointwise convergence induces pointwise convergence on $(X\to Y)$, when $Y$ is equipped with the induced convergence $\Gamma(\mathcal{V})$.
\end{lemma}
In particular, the uniformity of pointwise convergence is a uniformity, not just a preuniformity.
\begin{proof}
Let $H$ be a filter in $\powerfilters(X\to Y)$ and $f\in (X\to Y)$. Then, by definition of pointwise convergence and \ref{initialFinalConvergence},
\begin{align*}
H\overset{\text{pt-wise}}{\longrightarrow} f &\iff \forall x\in X: \; \evalMap^{\imf\imf}(H\otimes\pfilter{x}) \overset{\Gamma(\mathcal{V})}{\longrightarrow} f(x) \\
&\iff \forall x\in X: \; \evalMap^{\imf\imf}(H\otimes\pfilter{x}) \otimes \pfilter{f}(x) \in\mathcal{V} \\
&\iff \forall x\in X: \; \evalMap^{\imf\imf}(H\otimes\pfilter{x}) \otimes \evalMap^{\imf\imf}(\pfilter{f}\otimes \pfilter{x}) \in\mathcal{V} \\
&\iff \forall x\in X: \; \upset(\evalMap|\evalMap)^{\imf\imf}\big((H\otimes \pfilter{x}) \otimes (\pfilter{f}\otimes \pfilter{x})\big) \in\mathcal{V} \\
&\iff \forall x\in X: \; \upset\big((\evalMap|\evalMap)\circ t\big)^{\imf\imf}\big((H\otimes \pfilter{f}) \otimes (\pfilter{x}\otimes \pfilter{x})\big) \in\mathcal{V} \\
&\iff H\otimes \pfilter{f} \in \setbuilder{\pfilter{x}\otimes\pfilter{x}}{x\in X}^{\uRelEv^\transp} \\
&\iff H \overset{\mathcal{U}_p}{\longrightarrow} f.
\end{align*}
\end{proof}

\begin{lemma}
Let $\sSet{X, \xi}$ be a convergence space and $\sSet{Y, \mathcal{U}}$ a uniform space. Then the uniformity of continuous convergence induces continous convergence on $\cont(X,Y)$.
\end{lemma}
\begin{proof}
Let $H$ be a filter in $\powerfilters\big(\cont(X,Y)\big)$ and $f\in \cont(X, Y)$. Then
\begin{align*}
H\overset{(X\to Y)_c}{\longrightarrow} f &\iff \forall F\overset{\xi}{\longrightarrow}x: \; \evalMap^{\imf\imf}(H\otimes F) \overset{\mathcal{U}}{\longrightarrow} f(x) \\
&\iff \forall F\overset{\xi}{\longrightarrow}x: \; \evalMap^{\imf\imf}(H\otimes F)\otimes \pfilter{f}(x) \in\mathcal{U} \\
&\iff \forall F\overset{\xi}{\longrightarrow}x: \; \evalMap^{\imf\imf}(H\otimes F)\otimes f^{\imf\imf}(F) \in\mathcal{U} \\
&\iff \forall F\overset{\xi}{\longrightarrow}x: \; \evalMap^{\imf\imf}(H\otimes F)\otimes \evalMap^{\imf\imf}(\pfilter{f}\otimes F) \in\mathcal{U} \\
&\iff \forall F\overset{\xi}{\longrightarrow}x: \; \upset(\evalMap\circ \pi_1, \evalMap\circ\pi_2)^{\imf\imf}\big((H\otimes F)\otimes(\pfilter{f}\otimes F)\big) \in\mathcal{U} \\
&\iff \forall F\overset{\xi}{\longrightarrow}x: \; \upset\Big((\evalMap\circ \pi_1, \evalMap\circ\pi_2)\circ \big((\pi_1\circ\pi_1, \pi_1\circ\pi_2),(\pi_2\circ\pi_1, \pi_2\circ\pi_2)\big)\Big)^{\imf\imf}\big((H\otimes \pfilter{f})\otimes(F\otimes F)\big) \in\mathcal{U} \\
&\iff \forall F\overset{\xi}{\longrightarrow}x: \; \upset\Big(\big(\evalMap\circ (\pi_1\circ\pi_1, \pi_1\circ\pi_2), \evalMap\circ(\pi_2\circ\pi_1, \pi_2\circ\pi_2)\big)\Big)^{\imf\imf}\big((H\otimes \pfilter{f})\otimes(F\otimes F)\big) \in\mathcal{U} \\
&\iff \forall F\overset{\xi}{\longrightarrow}x: \; \upset e^{\imf\imf}\big((H\otimes \pfilter{f})\otimes(F\otimes F)\big) \in\mathcal{U} \\
&\iff H\otimes \pfilter{f}\in \mathcal{U}_c \\
&\iff H\overset{\mathcal{U}_c}{\longrightarrow} f.
\end{align*}
\end{proof}

\subsubsection{The Arzelà-Ascoli theorem}
\begin{proposition}
An equicontinuous set is evenly continuous.
\end{proposition}


\subsection{Continuous convergence}

\subsubsection{Continuous uniform convergence}
\begin{definition}
Let $\sSet{X, \xi}$ be a convergence space and $\sSet{Y, \mathcal{U}}$ a uniform space. First define
\[ e \defeq \big(\evalMap\circ (\pi_1\circ \pi_1, \pi_1\circ \pi_2), \evalMap\circ (\pi_2\circ \pi_1, \pi_2\circ \pi_2)\big), \]
then the \udef{uniformity of continuous convergence} is defined by
\[ \mathcal{U}_c \defeq \setbuilder{H\in \powerfilters\big((X\to Y)^2\big)}{\forall F\overset{\xi}{\longrightarrow} x:\; e^{\imf\imf}\Big(H\otimes (F\otimes F)\Big) \in \mathcal{U}}. \]
\end{definition}

\begin{lemma}
Let $\sSet{X, \xi}$ be a convergence space and $\sSet{Y, \mathcal{U}}$ a uniform space. Then the uniformity of continuous convergence
\begin{enumerate}
\item is a preuniformity on $(X\to Y)$;
\item a uniformity on $\cont(X,Y)$.
\end{enumerate}
\end{lemma}
\begin{proof}
Let $\sSet{X,\xi}$ be a convergence space and $\sSet{Y,\mathcal{U}}$ a uniform space. We verify the three conditions
\begin{itemize}
\item Take $f\in \cont(X,Y)$. We need to show that $f^{\imf\imf}[F]\mathrel{\mathcal{U}}f^{\imf\imf}[F]$ for all $F\overset{\xi}{\longrightarrow}x$. Because $f$ is continuous, we have that $f^{\imf\imf}[F]$ converges and the statement follows from \ref{uniformlyConvergentImpliesCauchy}.
\item Take $H,K_1,K_2$ in $\powerfilters(\cont(X,Y))$. Assume $H\mathrel{\mathcal{U}_c}K_1$ and $K_1\subseteq K_2$. Now $\evalMap^{\imf\imf}(K_1\otimes F) \subseteq \evalMap^{\imf\imf}(K_2\otimes F)$ for all $F\overset{\xi}{\longrightarrow}x$. Then upwards closure of $\mathcal{U}_c$ follows from the upwards closure of $\mathcal{U}$.
\item The symmetry and transitivity of $\mathcal{U}_c$ follow from the symmetry and transitivity of $\mathcal{U}$.
\end{itemize}
\end{proof}

\begin{lemma}
Let $\sSet{X, \xi}$ be a convergence space and $\sSet{Y, \mathcal{U}}$ a uniform space. If $\mathcal{U}$ is factorisable, then
\[ \mathcal{U}_c = (\Theta\circ\Phi)\Big(\big(\sSet{X,\xi} \to \sSet{Y, (\Xi\circ\Gamma)(\mathcal{U})}\big)_c\Big). \]
\end{lemma}

\begin{lemma}
Let $\sSet{X, \xi}$ be a convergence space and $\sSet{Y, \mathcal{U}}$ a uniform space. Then the uniformity of continuous convergence induces continous convergence on $\cont(X,Y)$.
\end{lemma}
\begin{proof}
Let $H$ be a filter in $\powerfilters\big(\cont(X,Y)\big)$ and $f\in \cont(X, Y)$. Then
\begin{align*}
H\overset{(X\to Y)_c}{\longrightarrow} f &\iff \forall F\overset{\xi}{\longrightarrow}x: \; \evalMap^{\imf\imf}(H\otimes F) \overset{\mathcal{U}}{\longrightarrow} f(x) \\
&\iff \forall F\overset{\xi}{\longrightarrow}x: \; \evalMap^{\imf\imf}(H\otimes F)\otimes \pfilter{f}(x) \in\mathcal{U} \\
&\iff \forall F\overset{\xi}{\longrightarrow}x: \; \evalMap^{\imf\imf}(H\otimes F)\otimes f^{\imf\imf}(F) \in\mathcal{U} \\
&\iff \forall F\overset{\xi}{\longrightarrow}x: \; \evalMap^{\imf\imf}(H\otimes F)\otimes \evalMap^{\imf\imf}(\pfilter{f}\otimes F) \in\mathcal{U} \\
&\iff \forall F\overset{\xi}{\longrightarrow}x: \; \upset(\evalMap\circ \pi_1, \evalMap\circ\pi_2)^{\imf\imf}\big((H\otimes F)\otimes(\pfilter{f}\otimes F)\big) \in\mathcal{U} \\
&\iff \forall F\overset{\xi}{\longrightarrow}x: \; \upset\Big((\evalMap\circ \pi_1, \evalMap\circ\pi_2)\circ \big((\pi_1\circ\pi_1, \pi_1\circ\pi_2),(\pi_2\circ\pi_1, \pi_2\circ\pi_2)\big)\Big)^{\imf\imf}\big((H\otimes \pfilter{f})\otimes(F\otimes F)\big) \in\mathcal{U} \\
&\iff \forall F\overset{\xi}{\longrightarrow}x: \; \upset\Big(\big(\evalMap\circ (\pi_1\circ\pi_1, \pi_1\circ\pi_2), \evalMap\circ(\pi_2\circ\pi_1, \pi_2\circ\pi_2)\big)\Big)^{\imf\imf}\big((H\otimes \pfilter{f})\otimes(F\otimes F)\big) \in\mathcal{U} \\
&\iff \forall F\overset{\xi}{\longrightarrow}x: \; \upset e^{\imf\imf}\big((H\otimes \pfilter{f})\otimes(F\otimes F)\big) \in\mathcal{U} \\
&\iff H\otimes \pfilter{f}\in \mathcal{U}_c \\
&\iff H\overset{\mathcal{U}_c}{\longrightarrow} f.
\end{align*}
\end{proof}


\section{Metric and pseudometric spaces}
\begin{definition}
Let $X$ be a set. A \udef{pseudometric} on $X$ is a function $d: X\times X\to \R$ satisfying, for all $x,y,z\in X$,
\begin{itemize}
\item \emph{positivity}: $d(x,y) \geq 0$;
\item \emph{symmetry}: $d(x,y) = d(y,x)$;
\item the \emph{triangle inequality}: $d(x,z) \leq d(x,y) + d(y,z)$.
\end{itemize}
The structured set $\sSet{X,d}$ is called a \udef{pseudometric space}.

If $d$ also satisfies
\begin{itemize}
\item \emph{definiteness}: $d(x,y) = 0$ \textup{if and only if} $x=y$;
\end{itemize}
then it is called a \udef{metric} and the structured set $\sSet{X,d}$ is called a \udef{metric space}.
\end{definition}

\begin{lemma}[Reverse triangle inequality] \label{metricReverseTriangleInequality}
Let $X$ be a set and $d: X^2\to\R^+$ a pseudometric on $X$. Then for all $x,y,a\in X$, we have
\[ \big|d(x,a) - d(a,y)\big| \;\leq\; d(x,y) \;\leq\; d(x,a) + d(a,y). \]
\end{lemma}
\begin{proof}
The second inequality is just the triangle inequality. For the first inequality, note that the triangle inequalities $d(x,a) \leq d(a,y) + d(x,y)$ and $d(a,y)\leq d(x,a) + d(x,y)$ hold. Thus
\[ d(x,a) - d(a,y) \leq d(x,y) \qquad\text{and}\qquad d(a,y) = d(x,a) \leq d(x,y). \]
Combinding these results gives the inequality.
\end{proof}

\subsection{Subsets of pseudometric spaces}
\begin{definition}
Let $\sSet{X,d}$ be a pseudometric space, $\epsilon >0$ and $x\in X$,
\begin{itemize}
\item The \udef{$\epsilon$-entourage} as the set
\[ V_\epsilon \defeq \setbuilder{(y,z)\in X^2}{d(y,z)< \epsilon} = d^{\preimf}\big(\interval{0,\epsilon}\big). \]
\item The \udef{$\epsilon$-ball centered at $x$} as the set
\[ \ball_d(x,\epsilon) \defeq \setbuilder{y\in X}{d(x,y)< \epsilon} \]
of all points $y$ whose distance to $x$ is less than $\epsilon$.
\item The \udef{closed $\epsilon$-ball centered at $x$} as the set
\[ \cball_d(x,\epsilon) \defeq \setbuilder{y\in X}{d(x,y)\leq \epsilon} \]
of all points $y$ whose distance to $x$ is less than or equal to $\epsilon$.
\item The \udef{$\epsilon$-sphere centered at $x$} as the set
\[ \sphere_d(x,\epsilon) \defeq \setbuilder{y\in X}{d(x,y) = \epsilon}. \]
\end{itemize}
Let $A\subseteq X$ be a subset. The \udef{diameter} of $A$ is defined by
\[ \diam(A) \defeq \sup \setbuilder{d(x,y)}{x,y\in A}. \]
\end{definition}

\begin{lemma} \label{diameterBoundedByTwiceRadius}
Let $sSet{X,d}$ be a pseudometric space, $c\in X$ and $A\subseteq X$. Then
\[ \diam(A) \leq 2\sup_{x\in A}d(c,x). \]
\end{lemma}
\begin{proof}
We need to prove
\[ \forall x,y\in A: \exists z\in A: \; d(x,y) \leq 2d(c,z). \]
Take arbitrary $x,y\in A$. Then let $z$ be $x$ if $d(c,x)\geq d(c,y)$ and $y$ otherwise. Now, by the triangle inequality, we have
\[ d(x,y) \leq d(x,c) + d(c,y) \leq d(z,c) + d(c,z) = 2d(c,z). \]
\end{proof}

TODO $\sup_{x\in A}{d(c,x)} \leq \diam(A)$ for all $c\in X$ definition of metric convexity??

\subsection{Uniform structure on a pseudometric space}
\begin{definition}
Let $\sSet{X,d}$ be a pseudometric space. The \udef{pseudometric uniformity} $\mathcal{U}_d$ is the uniformity given by
\[ \mathcal{U}_d \quad\defeq\quad \setbuilder{H\in \powerfilters(X^2)}{d^{\imf\imf}(H) \overset{\R}{\longrightarrow} 0}.  \]
\end{definition}
TODO the convergence on $\R$ used is the Scott convergence??

\begin{lemma} \label{pseudometricUniformityTopological}
Let $\sSet{X,d}$ be a pseudometric space. Then the pseudometric uniformity is topological and
\[ \entourage_{\mathcal{U}_d} = \upset d^{\preimf\imf}\big(\neighbourhood_\R(0)\big). \]
Also $\neighbourhood_\R(0) \subseteq \upset d^{\imf\imf}(\entourage_{\mathcal{U}_d})$.
\end{lemma}
\begin{proof}
We calculate, using \ref{filterPreimageImageGaloisConnection},
\begin{align*}
H \in \mathcal{U}_d &\iff \upset d^{\imf\imf}(H)\overset{\R}{\longrightarrow} 0 \\
&\iff \upset d^{\imf\imf}(H)\supseteq \neighbourhood_\R(0) \\
&\iff H \supseteq \upset d^{\preimf\imf}\big(\neighbourhood_\R(0)\big).
\end{align*}
\end{proof}

\begin{lemma}
Let $\sSet{X,d}$ be a pseudometric space. Then
\begin{enumerate}
\item $\cball_d(x,\epsilon) = V_\epsilon x = xV_\epsilon$;
\item $\sphere_d(x,\epsilon) = \cball_d(x,\epsilon)\setminus \ball_d(x,\epsilon)$.
\end{enumerate}
\end{lemma}

\begin{lemma}
Let $\sSet{X,d}$ be a pseudometric space. Then
\begin{enumerate}
\item the pseudometric uniformity is a uniformity;
\item the pseudometric uniformity is a topological uniformity;
\item $\{V_\epsilon\}_{\epsilon>0}$ is a base of the entourage filter.
\end{enumerate}
\end{lemma}
\begin{proof}
We need to show that
\begin{itemize}
\item $\{V_\epsilon\}_{\epsilon>0}$ is a filter base;
\item that $\upset\{V_\epsilon\}_{\epsilon>0}$ is an element of $\mathcal{U}_d$;
\item that $\{V_\epsilon\}_{\epsilon>0}\subseteq H$ for all $H\in \mathcal{U}_d$;
\item the metric uniformity is indeed a uniformity, which we do using \ref{entourageLemma}. We need to prove
\begin{itemize}
\item $\{V_\epsilon\}_{\epsilon>0}\subseteq\upset\{\Delta\}$;
\item $\{V_\epsilon\}_{\epsilon>0}^\transp = \{V_\epsilon\}_{\epsilon>0}$;
\item $\{V_\epsilon\}_{\epsilon>0} \subseteq \{V_\epsilon\}_{\epsilon>0}; \{V_\epsilon\}_{\epsilon>0}$.
\end{itemize}
\end{itemize}
Indeed:
\begin{itemize}
\item That $\{V_\epsilon\}_{\epsilon>0}$ is a filter base is immediate from $V_\epsilon \cap V_\delta = V_{\min(\epsilon, \delta)}$.
\item By definition, $d^{\imf}(V_\epsilon) \subseteq \interval{0,\epsilon}$, so $\setbuilder{\interval{0,\epsilon}}{\epsilon>0} \subseteq d^{\imf\imf}\big(\{V_\epsilon\}_{\epsilon>0}\big)$. Because $\setbuilder{\interval{0,\epsilon}}{\epsilon>0} \overset{\R}{\longrightarrow} 0$, we have $d^{\imf\imf}\big(\{V_\epsilon\}_{\epsilon>0}\big) \overset{\R}{\longrightarrow} 0$ and so $\upset\{V_\epsilon\}_{\epsilon>0} \in \mathcal{U}_d$.
\item Take $H\in \mathcal{U}_d$. 
Now we need to prove that for all $\epsilon >0$ there exists an $A\in H$ such that $V_\epsilon \subseteq A$. Fix an $\epsilon > 0$. By assumption (and definition of Scott convergence), we have
\[ \bigwedge \setbuilder{\bigvee d^\imf(A)}{A\in H} = 0. \]
So there exists $A\in H$ such that $\bigvee d^\imf(A) \leq \epsilon$. This implies $d^\imf(A) \subseteq \interval{0,\epsilon}$ and thus
\[ A \subseteq d^\preimf\big(d^{\imf}(A)\big) \subseteq d^\preimf\big(\interval{0,\epsilon}\big) = V_\epsilon. \]
\item Finally,
\begin{itemize}
\item $d^{\imf}(\Delta) = 0$, so $\Delta \subseteq V_\epsilon$ for all $\epsilon > 0$;
\item $\{V_\epsilon\}_{\epsilon>0}^\transp = \{V_\epsilon\}_{\epsilon>0}$ by symmetry of the metric;
\item take $\epsilon>0$, then $V_{\epsilon/2};V_{\epsilon/2} \subseteq V_\epsilon$ by the triangle inequality, so $V_\epsilon \in \{V_\epsilon\}_{\epsilon>0}; \{V_\epsilon\}_{\epsilon>0}$.
\end{itemize}
\end{itemize}
\end{proof}
\begin{corollary}
Let $\sSet{X, d_X}$ and $\sSet{Y, d_Y}$ be metric spaces and $f:X\to Y$ a function. Then $f$ is uniformly continuous \textup{if and only if}
\[ \forall \epsilon >0: \exists \delta >0: \forall x,y\in X: \quad d_X(x,y) \leq \delta \implies d_Y(f(x), f(y)) \leq \epsilon. \]
\end{corollary}
\begin{proof}
By \ref{uniformContinuityEntourages} we have that $f$ is uniformly continuous \textup{if and only if}
\begin{align*}
\entourage_Y \subseteq \upset(f\times f)^{\imf\imf}[\entourage_X] &\iff \forall V_\epsilon \in \entourage_Y: \exists V_\delta \in \entourage_X: \; (f\times f)^{\imf\imf}[V_\delta] \subseteq V_\epsilon \\
&\iff \forall \epsilon>0: \exists \delta>0: \forall (y,y')\in (f\times f)^{\imf\imf}[V_\delta]: \; (y,y')\in V_\epsilon \\
&\iff \forall \epsilon>0: \exists \delta>0: \forall (x,x')\in V_\delta:\; (f(x),f(x'))\in V_\epsilon \\
&\iff \forall \epsilon>0: \exists \delta>0: \forall x,x'\in X: \; d_X(x,x')\leq \delta \implies d_Y(f(x),f(x'))\leq \epsilon.
\end{align*}
\end{proof}

\begin{lemma}
Let $\sSet{X,d}$ be a metric space and $\seq{x_n}$ a sequence in $X$. Then $\seq{x_n}$ is a Cauchy sequence \textup{if and only if}
\[ \forall \epsilon >0: \exists N\in\N: \forall m,n \geq N: \quad d(x_m, x_n)\leq \epsilon. \]
\end{lemma}
\begin{proof}
We have that $\TailsFilter\seq{x_n}$ is a Cauchy filter \textup{if and only if} $\entourage_d \subseteq \TailsFilter\seq{x_n}\otimes \TailsFilter\seq{x_n}$. This is true iff $\forall \epsilon>0$:
\begin{align*}
V_\epsilon \in \TailsFilter\seq{x_n}\otimes \TailsFilter\seq{x_n} &\iff \exists M,N\in\N: \setbuilder{x_n}{n\geq N}\times\setbuilder{x_n}{n\geq M} \subseteq V_\epsilon \\
&\iff \exists N\in\N: \setbuilder{x_n}{n\geq N}\times\setbuilder{x_n}{n\geq N} \subseteq V_\epsilon \\
&\iff \exists N\in\N: \forall m,n \geq N: d(x_m, x_n)\leq \epsilon.
\end{align*}
\end{proof}

\begin{definition}
Let $\sSet{X,d}$ be a metric space. The convergence $\Gamma(\mathcal{U}_d)$ induced by the metric uniformity $\mathcal{U}_d$ is called the \udef{metric convergence}.

If $F\in\powerfilters(X)$ converges to $x\in X$ in the metric convergence, we write $F\overset{d}{\longrightarrow} x$.
\end{definition}

\begin{lemma}
Let $\sSet{X,d}$ be a metric space. The metric convergence is topological and
\[ \neighbourhood_d(x) = \upset\setbuilder{\cball_d(x,\epsilon)}{\epsilon >0}. \]
\end{lemma}
\begin{proof}
The metric convergence is topological by \ref{topologicalInducedUniformConvergence}, which also gives the form of the neighbourhood filter. 
\end{proof}
\begin{corollary}
Let $\sSet{X,d}$ be a metric space and $\seq{x_n}$ a sequence in $X$. Then $\seq{x_n}$ converges to $x\in X$ \textup{if and only if}
\[ \forall \epsilon>0: \exists N\in\N: \forall n\geq N: \quad d(x_n,x)\leq \epsilon. \]
\end{corollary}

\subsubsection{Metric convergence in $\R$}
\begin{lemma}
The function
\[ d_\R: \R\times \R\to\R^+: (x,y) \mapsto |x-y| \]
is a metric on $\R$. The metric topology induced by this metric is equal to the Scott topology on $\R$.
\end{lemma}
\begin{proof}
TODO
\end{proof}

\begin{proposition} \label{boundedSubsetsRealNumbers}
A set $A\subseteq \R$ is bounded \textup{if and only if} $\sup(A) < \infty$.
\end{proposition}
\begin{proof}
TODO
\end{proof}

\begin{lemma} \label{realBoundedIffTotallyBounded}
A subset of $\R$ is bounded \textup{if and only if} it is totally bounded.
\end{lemma}

\subsubsection{Uniform convergence}
\begin{proposition}
Let $X$ be a set, $\sSet{Y,d}$ a metric space and $\seq{f_n}$ a sequence in $(X\to Y)$. The following are equivalent:
\begin{enumerate}
\item  $\seq{f_n}$ converges uniformly to $f: X\to Y$;
\item $\forall \epsilon > 0: \exists N\in \N: \forall n \geq N: \forall x\in X:  d(f_n(x), f(x)) \leq \epsilon$;
\item $\forall \epsilon > 0: \exists N\in \N: \forall n \geq N: \sup_{x\in X} d(f_n(x), f(x)) \leq \epsilon$.
\end{enumerate}
\end{proposition}
\begin{proof}
The equivalence of (2) and (3) is clear. We prove $(1) \Leftrightarrow (2)$.
\begin{align*}
\seq{f_n} \overset{\text{unif.}}{\longrightarrow} f &\iff \TailsFilter\seq{f_n} \otimes \pfilter{f} \in \mathcal{U}_{(X\to Y)} \\
&\iff (\evalMap, \evalMap)^{\imf\imf}\big((\TailsFilter\seq{f_n} \otimes \pfilter{f})\otimes \{\Delta_X\}\big)^\ttransp \in \mathcal{U}_d \\
&\iff \{V_\epsilon\}_\epsilon \subseteq (\evalMap, \evalMap)^{\imf\imf}\big((\TailsFilter\seq{f_n} \otimes \pfilter{f})\otimes \{\Delta_X\}\big)^\ttransp \\
&\iff \forall \epsilon>0: \exists N\in \N: \; (\evalMap, \evalMap)^{\imf}\big((\setbuilder{f_n}{n\geq N} \times \{f\})\times \Delta_X\big)^\ttransp \subseteq V_\epsilon \\
&\iff \forall \epsilon>0: \exists N\in \N: \; \bigcup_{n\geq N}\bigcup_{x\in X}(\evalMap, \evalMap)^{\imf}\big((\{f_n\} \times \{f\})\times \{(x,x)\}\big)^\ttransp \subseteq V_\epsilon \\
&\iff \forall \epsilon>0: \exists N\in \N: \forall n\geq N: \forall x\in X: \;(\evalMap, \evalMap)^{\imf}\big((\{f_n\} \times \{f\})\times \{(x,x)\}\big)^\ttransp \in V_\epsilon \\
&\iff \forall \epsilon>0: \exists N\in \N: \forall n\geq N: \forall x\in X: \; (\evalMap, \evalMap)^{\imf}\big(\{((f_n, f), (x,x))^\ttransp\}\big) \in V_\epsilon \\
&\iff \forall \epsilon>0: \exists N\in \N: \forall n\geq N: \forall x\in X: \; (\evalMap, \evalMap)\big((f_n,x),(f,x)\big) \in V_\epsilon \\
&\iff \forall \epsilon>0: \exists N\in \N: \forall n\geq N: \forall x\in X: \; (f_n(x), f(x)) \in V_\epsilon \\
&\iff \forall \epsilon>0: \exists N\in \N: \forall n\geq N: \forall x\in X: \; d(f_n(x), f(x)) \leq \epsilon.
\end{align*}
\end{proof}



\subsubsection{(Pseudo)metrisation of uniform spaces}
If Hausdorff, we may take metrics WLOG. If translation invariant,we may take pseudometrics translation invariant WLOG (i.e. Birkhoff-Kakutani).

\begin{lemma} \label{metrisationLemma}
Let $X$ be a set and $\seq{U_n}_{n\in \N}\in (X^2)^\N$ a sequence of subsets of $X^2$ such that
\begin{itemize}
\item $U_0 = X^2$;
\item $\id_X \subseteq U_n$ for all $n\in\N$;
\item $U_{n+1};U_{n+1};U_{n+1} \subseteq U_n$ for all $n\in\N$.
\end{itemize}
Then there exists a function $d: X^2 \to \R^+$ such that
\begin{itemize}
\item the triangle inequality holds: $d(x,z) \leq d(x,y)+d(y,z)$ for all $x,y,z\in X$;
\item $U_{n+1} \subseteq \setbuilder{(x,y)\in X^2}{d(x,y)< 2^{-n}} \subseteq U_{n}$ for all $n\in\N$.
\end{itemize}
Also
\begin{enumerate}
\item If each $U_n$ is symmetric, then $d$ can be taken to be a pseudometric.
\item If $\{U_n\}_{n\in\N} \subseteq \entourage_\mathcal{U}$ for some uniformity $\mathcal{U}$ on $X$, then $\mathcal{U} \subseteq \mathcal{U}_d$.
\end{enumerate}
\end{lemma}
\begin{proof}
First notice that $\seq{U_n}_{n\in \N}$ determines a chain in $X^2$: for all $n\in\N$ we have
\[ U_n \supseteq U_{n+1};U_{n+1};U_{n+1} \supseteq \id_X;U_{n+1};\id_X = U_{n+1}. \]
Now define a function
\[ f: X^2 \to \interval{0,1}: (x,y) \mapsto \begin{cases}
2^{-n} & \big((x,y)\in U_{n}\setminus U_{n+1}\big) \\
0 & \big((x,y)\in \bigcap_{k\in\N}U_k\big).
\end{cases} \]
Then $(x,y)\in U_{n+1}$ iff $f(x,y) < 2^{-n}$ iff $f(x,y) \leq 2^{-(n+1)}$.
We use this function to construct the desired function $d$, which we define by
\[ d(x,y) \defeq \inf\setbuilder{\sum_{i=0}^k f(u_i, u_{i+1})}{\seq{u_i}_{i=0}^k\in X^*, u_0 = x, u_k = y}. \]
It is clear that $d$ satisfies the triangle inequality.

For all $x,y\in X$, we have $d(x,y) \leq f(x,y)$. If we can prove that $f(x,y) \leq 2 d(x,y)$, then we have
\begin{align*}
U_{n+1} &= \setbuilder{(x,y)\in X^2}{f(x,y)< 2^{-n}} \\
&\subseteq \setbuilder{(x,y)\in X^2}{d(x,y)< 2^{-n}} \\
&\subseteq \setbuilder{(x,y)\in X^2}{f(x,y)< 2^{-n+1}} = U_{n}.
\end{align*}

We prove $f(x,y) \leq 2 d(x,y)$ by proving that, for all $x,y\in X$, $f(x,y) \leq 2 \sum_{k\leq\len(u)}f(u_k, u_{k+1})$ for all $u\in X^*$ such that $u_0 = x$ and $u_{\len(u)} = y$. The proof is by induction on $\len(u)$.

Suppose that the induction hypothesis holds, i.e.\ for all $x,y\in X$ we have that $f(x,y) \leq 2 \sum_{k<\len(u)}f(u_k, u_{k+1})$ for all $u\in X^*$ such that $u_0 = x$, $u_{\len(u)} = y$ and $\len(u)\leq n$. We now need to prove that the inequality holds for all sequences of length $n+1$. Take such a sequence $v \in X^{(n+1)}$ with $v_0 = x$, $v_{n+1} = y$. Set $a\defeq \sum_{k=0}^{n}f(v_k,v_{k+1})$. If $a\geq 1$, then $f(x,y)\leq a$ because $f(x,y)\leq 1$ by definition, so $f(x,y) \leq 2a$. We may restrict to the case $a<1$. Now let $m$ be the largest natural number such that $\sum_{k=0}^{m-1}f(v_k,v_{k+1}) \leq a/2$, then
\[ a = \Big(\sum_{k=0}^{m-1}f(v_k,v_{k+1})\Big) + f(v_m, v_{m+1}) + \Big(\sum_{k=m+1}^{n}f(v_k,v_{k+1})\Big) \eqdef b + f(v_m, v_{m+1}) + c, \]
so $c \leq a/2$ (as otherwise we could have absorbed $f(v_m, v_{m+1})$ into the first part). By the induction hypothesis, we have $f(x, v_m) \leq 2b \leq a$ and $f(v_{m+1}, y) \leq 2c \leq a$. Clearly $f(v_{m}, v_{m+1})\leq a$. Now let $p\in \N$ be the 
natural number such that $2^{-p-1}\leq a< 2^{-p}$ (this exists because we already restricted to the case $a< 1$).
Then $f(x,v_m), f(v_m, v_{m+1}), f(v_{m+1}, y) \leq a < 2^{-p}$, so $(x,v_m), (v_m, v_{m+1}), (v_{m+1}, y)\in U_{p+1}$ and so
\[ (x,y) \in \{(x,v_m)\}; \{(v_m, v_{m+1})\};\{(v_{m+1}, y)\} \subseteq U_{p+1};U_{p+1};U_{p+1} \subseteq U_p. \]
Thus $f(x,y) \leq 2^{-p} = 2\cdot 2^{-p-1} \leq 2a$. This concludes the induction step.

(1) Positivity holds by construction. If each $U_n$ is symmetric, then $f$ is symmetric and thus also $d$. The triangle inequality has already been proven.

(2) Take $H\in \mathcal{U}$. In order to prove $H\in \mathcal{U}_d$, we need to prove that $d^{\imf\imf}(H)\overset{\R}{\longrightarrow} 0$. Indeed we have
\[ \neighbourhood_\R(0) \subseteq \upset \setbuilder{\interval[co]{0, n^{-1}}}{n\in \N} \subseteq \upset d^{\imf\imf}\big(\{U_n\}_{n\in\N}\big) \subseteq d^{\imf\imf}(H). \]
\end{proof}

\begin{theorem}[Urysohn metrisation theorem]
A topological uniform space is pseudometrisable \textup{if and only if} its entourage filter has a countable base.
\end{theorem}
\begin{proof}
Let $\sSet{X,\mathcal{U}}$ be a topological uniform space. First suppose $\mathcal{U}$ is generated by a pseudometric $d$. Then each $\epsilon$-entourage is contained in a $n^{-1}$-entourage, for some $n\in \N$. This shows that the entourage filter is countably based.

Now suppose $\entourage_\mathcal{U}$ has a countable basis $\seq{B_n}_{n\in \N}$. WLOG we may take $B_0 = X^2$. Then define $\seq{C_n}$ recursively by setting $C_0 = X^2$ and, given $C_{n-1}$, finding some $C'_{n}\in \entourage_\mathcal{U}$ such that $C'_{n};C'_{n};C'_{n} \subseteq C_{n-1}$ and setting $C_n \defeq C'_n \cap B_n$. We have $C_n\subseteq B_n$ for all $n\in \N$ and each $C_n\in \entourage_\mathcal{U}$. Thus $\seq{C_n}$ forms a basis of the intourage filter and thus the entourage filter is generated by a pseudometric by \ref{metrisationLemma}. 
\end{proof}

\begin{proposition} \label{topologicalUniformityGeneratedByPseudometrics}
Let $X$ be a set and $\mathcal{U}$ a topological uniformity on $X$. Then 
\[ \mathcal{U} =  \bigwedge \setbuilder{\mathcal{U}_d}{\text{$d$ is a pseudometric on $X$ and $\mathcal{U}\subseteq \mathcal{U}_d$}}. \]
\end{proposition}
\begin{proof}
We clearly have
\[ \mathcal{U} \subseteq \bigwedge \setbuilder{\mathcal{U}_d}{\text{$d$ is a pseudometric on $X$ and $\mathcal{U}\subseteq \mathcal{U}_d$}}. \]

For the other inclusion, it is enough to prove $\entourage_\mathcal{U} \subseteq \entourage_{\bigwedge \mathcal{U}_d}$. Take $V\in \entourage_\mathcal{U}$ and define $\seq{B_n}$ recursively by $B_0 = X^2$, $B_1 = V$ and $B_{n+1}$ is any symmetric set in $\entourage_\mathcal{U}$ such that $B_{n+1};B_{n+1};B_{n+1} \subseteq B_n$. Construct a pseudometric $d'$ as in \ref{metrisationLemma}. Then $d^{\prime\preimf}(\interval[co]{0,1/2})\subseteq V$, so, by \ref{pseudometricUniformityTopological},
\[ V \in \entourage_{d'} \subseteq \entourage_{\bigwedge \mathcal{U}_d}. \]
\end{proof}

\begin{definition}
Let $\sSet{X, \mathcal{U}}$ be a uniform space and $d: X\times X\to \R^+$ a pseudometric on $X$. Then we call $d$ a \udef{continuous pseudometric} on $X$ if $\mathcal{U}\subseteq \mathcal{U}_d$.
\end{definition}

\begin{proposition} \label{continuousPseudometricFunctionLemma}
Let $\sSet{X, \mathcal{U}}$ be a uniform space, $a\in X$, $d: X^2\to \R^+$ a pseudometric and $f: X\to \R$ a function.
\begin{enumerate}
\item if $d$ is a continuous pseudometric, then $d(a, -): X\to \R$ is uniformly continuous;
\item if $f$ is uniformly continuous, then $d_\R\circ (f|f): X^2\to \R^+: (x,y)\mapsto |f(x)- f(y)|$ is a continuous pseudometric.
\end{enumerate}
\end{proposition}
\begin{proof}
(1) By \ref{uniformContinuityEntourages}, it is enough to prove that $\entourage_\R \subseteq \upset \big(d(a, -), d(a,-)\big)^{\imf\imf}(\entourage_\mathcal{U})$ and by continuity of the pseudometric, we have $\entourage_d \subseteq \entourage_{\mathcal{U}}$, so we need to prove that
\[ \entourage_\R \subseteq \upset \big(d(a, -), d(a,-)\big)^{\imf\imf}(\entourage_d). \]
Take arbitrary $\epsilon>0$. Then it is enough to prove thave there exists $A\in \entourage_d$ such that $\big(d(a, -), d(a,-)\big)^{\imf}(A) \subseteq V_\epsilon$. Now, by \ref{pseudometricUniformityTopological}, there exists $A\in \entourage_d$ such that $d^{\imf}(A)\subseteq \cball(0,\epsilon)$. We claim this is the set we want, indeed
\begin{align*}
d^{\imf}(A)\subseteq \cball(0,\epsilon) \iff& \forall (x,y)\in A: \; d(x,y) \leq \epsilon \\
\implies& \forall (x,y)\in A: \; |d(x,a), d(a,y)| \leq \epsilon \\
\iff& \forall (x,y)\in A: \; \big(d(-,a), d(a,-)\big)(x,y) \in V_\epsilon \\
\iff& \big(d(-,a), d(a,-)\big)^\imf(A) = \big(d(a,-), d(a,-)\big)^\imf(A) \subseteq V_\epsilon,
\end{align*}
where we have used the reverse triangle inequality, \ref{metricReverseTriangleInequality}.

(2) Set $d_f \defeq d_\R\circ(f|f)$. It is enough to prove that $\entourage_{d_f}\subseteq \entourage_\mathcal{U}$. Using \ref{pseudometricUniformityTopological} for the first and third equality, we indeed have
\[ \entourage_{d_f} = \upset \big(d_\R\circ (f|f)\big)^{\preimf\imf}\big(\neighbourhood_\R(0)\big) = \upset (f|f)^{\preimf\imf}\Big(d_\R^{\preimf\imf}\big(\neighbourhood_\R(0)\big)\Big) = \upset (f|f)^{\preimf\imf}(\entourage_\R) \subseteq \entourage_\mathcal{U}, \]
where the last inclusion follows by \ref{filterPreimageImageGaloisConnection} from $\entourage_\R \subseteq \upset (f|f)^{\imf\imf}(\entourage_\mathcal{U})$, which in turn is the expression of uniform continuity of $f$, by \ref{uniformContinuityEntourages}.
\end{proof}

\subsubsection{Bounded subsets}

\begin{proposition} \label{metricBoundedness}
Let $\sSet{X, \mathcal{U}}$ be a topological uniform space and $B\subseteq X$. Then the following are equivalent:
\begin{enumerate}
\item $B$ is bounded;
\item every real-valued uniformly continuous function on $X$ is bounded on $B$;
\item $B$ is of finite diameter w.r.t. every continuous pseudometric on $X$.
\end{enumerate}
\end{proposition}
\begin{proof}
$(1) \Rightarrow (2)$ Follows from \ref{imageBoundedSet}.

$(2) \Rightarrow (3)$ Let $d$ be a continuous pseudometric on $X$ and take $a\in X$. Then $f_a \defeq d(a,-)$ is uniformly continuous by \ref{continuousPseudometricFunctionLemma}. Now $\diam(B)\leq 2\sup_{x\in B}f_a(x)$ by \ref{diameterBoundedByTwiceRadius} and $f_a^\imf(B)$ is bounded by assumption, so $\sup_{x\in B}f_a(x) < \infty $ by \ref{boundedSubsetsRealNumbers}. Thus $\diam(B)$ is also finite.

$(3) \Rightarrow (1)$ We use \ref{topologicalBoundednessLemma}. Take an arbitrary symmetric $V\in \entourage_\mathcal{U}$ and let $W$ be the transitive closure of $V$, which is an equivalence relation. Let $\seq{A_n}$ be a sequence of distinct $W$-equivalence classes and define
\[ f: X\to \N: x\mapsto \begin{cases}
n & (x\in A_n) \\
0 & (\text{otherwise}).
\end{cases} \]
Now take $(x,y)\in V$. Then $x$ and $y$ must be in the same equivalence class, so $\big(f(x), f(y)\big) = (n,n)$ for some $n\in\N$. Thus $(f|f)^\imf(V)\subseteq \id_\R \subseteq V_\epsilon$ for all $\epsilon>0$. By \ref{uniformContinuityEntourages}, we have that $f$ is uniformly continuous. Now consider the pseudometric $d_f(x,y) = |f(x)-f(y)|$, which is continuous by \ref{continuousPseudometricFunctionLemma}. By assumption, $B$ is of finite radius w.r.t. $d_f$ and thus $B$ meshes with only finitely many $W$-equivalence classes. Let the finite set $F$ consist of one point from the intersection of $B$ with each $W$-equivalence class that meshes with $B$.

Now take some continuous pseudometric $d$ constructed in \ref{topologicalUniformityGeneratedByPseudometrics} and rescaled such that $d^\preimf(\interval[co]{0,1}) \subseteq V$.

Now consider the function $d': X^2\to \R^+$ defined by
\[ d'(x,y) \defeq \begin{cases}
\setbuilder{\sum_{i=0}^kd(u_i, u_{i+1})}{\seq{u_i}_{i=0}^k\in X^*, u_0 = x, u_k = y, (u_{j},u_{j+1})} & \big((x,y)\in W\big) \\
2 & (\text{otherwise})
\end{cases} \]
It is a continuous pseudometric because $d = d'\wedge \underline{2}$ (TODO ref).

Take $x,y\in B$ such that $(x,y)\in W$. If $d'(x,y)\leq m\in\N$, then $(x,y)\in V^m$. As $d'$ is bounded on $B$, we can find some $m\in \N$ such that each $x,y\in B$ that are in the same $W$-equivalence class are $V^m$-related.
\end{proof}

\begin{example}
It is important in \ref{metricBoundedness} to check that $B$ has finite diameter w.r.t. every continuous pseudometric on $X$.

For example, consider $d_\R$ and $d' \defeq d_\R \wedge \underline{1}$. Then both metrics define the same uniformity, but $\sup d^{\prime\imf}(A) \leq 1$ for all subsets $A\subseteq \R$ and thus all subsets of $\R$ have finite diameter w.r.t. $d'$. 
\end{example}

\subsection{Continuous convergence structure}
\begin{proposition}
Let $\sSet{X,\xi}$ be a compact convergence space and $\sSet{Y,d}$ a metric space. The continuous convergence on $\cont(X,Y)$ is given by
\[ \forall H\in\powerset(\cont(X,Y)_c): \qquad H\to f \iff \sup_{x\in X}d(H(x), f(x)) \to 0. \]
\end{proposition}
\begin{proof}
First assume $\sup_{x\in X}d(H(x), f(x)) \to 0$. 

Now assume $\sup_{x\in X}d(H(x), f(x)) \not\to 0$. Then there exists $A\in \neighbourhood(0)$ such that $A \notin \sup_{x\in X}d(H(x), f(x))$ we can construct the set
\[ \setbuilder{\setbuilder{x\in X}{d(h(x), f(x)) \notin A \forall h\in S}}{S\in H}. \]
We claim this is a proper filter in $X$. It is contained in an ultrafilter by the ultrafilter lemma \ref{ultrafilterLemma} and this ultrafilter $F$ converges by compactness. Thus $d(H[F], f[F]) \not\to 0$ and so $H\not\to f$.
\end{proof}


\subsection{Covering theorems}
\begin{proposition}[Finite covering lemma \textit{or} 3-fold covering lemma]
Let $\sSet{X,d}$ be a metric space and $\ball(x_0, r_0), \ldots \ball(x_n, r_n)$ a finite set of balls in $X$. Then there exists a subset $\ball(x_{i_0}, r_{i_0}), \ldots \ball(x_{i_k}, r_{i_k})$ of pairwise disjoint balls such that
\[ \bigcup_{j=0}^n\ball(x_j, r_j) \subseteq \bigcup_{j=0}^k \ball(x_{i_j}, 3r_{i_j}). \]
\end{proposition}
\begin{proof}
We pick the set of balls recursively: we let $i_{l+1}$ be the index of the ball with the largest radius that is disjoint with $\bigcup_{j=0}^l \ball(x_j, r_j)$. If there are multiple, choose one arbitrarily. If there are none, terminate. (TODO ref recursion).

We now show that the inclusion holds. Take some $\ball(x_j, r_j)$. It must mesh with some element $\ball(x_{i_m}, r_{i_m})$ (if not it would have been added and in that case it would mesh with itself). Take $y\in \ball(x_j, r_j) \cap \ball(x_{i_m}, r_{i_m})$. By construction we also have $r_{i_m} \geq r_j$. Now take any $z\in \ball(x_j, r_j)$. By the triangle inequality, we have
\begin{align*}
d(x_{i_m}, z) &\leq d(x_{i_m}, y) + d(y, x_j) + d(x_j, z) \\
&\leq r_{i_m} + r_j + r_j \leq 3r_{i_m}.
\end{align*}
So $z\in \ball(x_{i_m}, 3r_{i_m})$ and thus $\ball(x_j, r_j)\subseteq\ball(x_{i_m}, 3r_{i_m})$. This implies the inclusion of unions.
\end{proof}

\begin{proposition}[Infinite covering lemma \textit{or} 5-fold covering lemma]
Let $\sSet{X,d}$ be a metric space and $\{\ball(x_i, r_i)\}_{i\in I}$ an arbitrary set of balls in $X$ such that $\sup_{i\in I}r_i < \infty$. Then there exists a subset $\{\ball(x_j, r_j)\}_{j\in J}$ of pairwise disjoint balls such that
\[ \bigcup_{i\in I}\ball(x_i, r_i) \subseteq \bigcup_{j\in J} \ball(x_j, 5\, r_j). \]
\end{proposition}
\begin{proof}
Set $R \defeq \sup_{i\in I}r_i < \infty$ and consider the partition $\{F_n\}_{n\in \N}$ of $\{\ball(x_i, r_i)\}_{i\in I}$ defined by
\[ F_n \defeq \setbuilder{\ball(x_i, r_i)}{\frac{R}{2^{n+1}} < r_i \leq \frac{R}{2^n}}. \]
Define two sequences $\seq{G_n}, \seq{H_n}$ of subsets of $\{\ball(x_i, r_i)\}_{i\in I}$ as follows: $H_0 = F_0$ and $G_0$ is a maximal disjoint subcollection of $H_0$ (which exists by \ref{maximalSubsetOfDisjointSets}). We define the sequences recursively by
\[ H_{n+1} = F_{n+1} \cap \left(\bigcup_{k=0}^nG_k\right)^\perp \qquad\text{and $G_{n+1}$ is a maximal disjoint subcollection of $H_{n+1}$.} \]
Now we claim $G \defeq \bigcup_{n\in\N}G_n$ is the subset we are looking for. First note that $G$ is pairwise disjoint: take $A,B\in G$, then $A\in G_k$ and $B\in G_l$. If $k = l$, then $A\perp B$ by construction. Now suppose $k\neq l$, WLOG we may take $l\leq k$. Then $A\in H_k \subseteq G_l^\perp$, by construction. Thus $A\perp B$.

Now take some $\ball(x_m, r_m) \in \bigcup_{i\in I}\ball(x_i, r_i)$. Then $r_m \leq R$, so we can find $n\in \N$ such that $\frac{R}{2^{n+1}} < r_i \leq \frac{R}{2^n}$ and thus $\ball(x_m, r_m)\in F_n$. By construction, either $\ball(x_m, r_m)\in H_n$ or $\ball(x_m, r_m)\notin \Big(\bigcup_{k=0}^{n-1}G_k\Big)^\perp$. In both cases $\ball(x_m, r_m)$ must intersect some $\ball(x_j, r_j)$ with $j\in J$ such that $r_j \geq \frac{R}{2^{n+1}}$. We have $r_m \leq \frac{R}{2^n} = 2 \frac{R}{2^{n+1}} \leq 2 r_j$ and we can find $y\in \ball(x_m, r_m) \cap \ball(x_j, r_j)$.

It is now enough to show that $\ball(x_m, r_m) \subseteq \ball(x_j, r_j)$. Take an arbitrary $z\in \ball(x_m, r_m)$. Then we have
\begin{align*}
d(x_j, z) &\leq d(x_j, y) + d(y, x_m) + d(x_m, z) \\
&\leq r_j + r_m + r_m \\
&\leq r_j + 2r_j + 2r_j = 5r_j.
\end{align*}
So $z\in \ball(x_j, 5r_j)$ and thus $\ball(x_m, r_m)\subseteq\ball(x_j, 5r_j)$. This implies the inclusion of unions.
\end{proof}
\begin{corollary}
The constant $5$ can be improved to $3+\epsilon$ for all $\epsilon >0$.
\end{corollary}
\begin{proof}
In the proof we can replace $2$ by $1+\epsilon/2$. Then the bound becomes $1+(1+\epsilon/2) + (1+\epsilon/2) = 3+\epsilon$.
\end{proof}
\begin{corollary}
If $\sSet{X,d}$ is separable, then the subset is countable.
\end{corollary}
\begin{proof}
TODO ref: every set of disjoint neighbourhoods is countable.
\end{proof}

\begin{example}
The hypothesis that the radius be bounded is necessary: consider the set $\{\ball(x, 1/n)\}_{n\in \N}$. Then no two two balls are disjoint, so a disjoint subset contains at most one ball. The one ball cannot be scaled up to cover everything.
\end{example}

\section{TODO Uniform covers}
\subsection{Stars of filters}
\begin{definition}
Let $X$ be a set, $\mathcal{F} \subseteq \powerfilters(X)$ a set of filters on $X$ and $F\in\powerfilters(X)$ a filter. The \udef{star} of $F$ w.r.t. $\mathcal{F}$, denoted $\starFilter(F,\mathcal{F})$, is defined as
\[ \starFilter(F,\mathcal{F}) \defeq \bigcap \setbuilder{G\in \mathcal{F}}{G \amesh F}. \]
The \udef{star} of $\mathcal{F}$, denoted $\mathcal{F}^*$, is defined as
\[ \mathcal{F}^* \defeq \setbuilder{\starFilter(F,\mathcal{F})}{F\in \mathcal{F}}. \]
\end{definition}

\begin{lemma} \label{starRefinementLemma}
Let $X$ be a set, $\mathcal{F} \subseteq \powerfilters(X)$ a set of filters on $X$ and $F\in \mathcal{F}$. Then
\begin{enumerate}
\item $\starFilter(F,\mathcal{F}) \subseteq F$;
\item $\starFilter(\pfilter{x}, \mathcal{F}) = \Big(\bigcap\setbuilder{F\otimes F}{F\in\mathcal{F}}\Big)x$;
\item $\mathcal{F} \subseteq \upset\mathcal{F}^*$.
\end{enumerate}
\end{lemma}
\begin{proof}
(1) We have $F\in \setbuilder{G\in \mathcal{F}}{G \amesh F}$ by \ref{properFiltersSelfMesh}, so
\[ F \;\supseteq\; \bigcap \setbuilder{G\in \mathcal{F}}{G \amesh F} \;=\; \starFilter(F,\mathcal{F}). \]

(2) By \ref{filterResiduatedImageGaloisConnection} and \ref{principalImageProductFilter}, we have
\begin{align*}
\Big(\bigcap\setbuilder{F\otimes F}{F\in\mathcal{F}}\Big)x &= \bigcap_{F\in\mathcal{F}}(F\otimes F)x \\
&= \bigcap \setbuilder{F\in\mathcal{F}}{x\in\ker(F)} \\
&= \bigcap \setbuilder{F\in\mathcal{F}}{F\amesh\pfilter{x}} = \starFilter(\pfilter{x}, \mathcal{F}).
\end{align*}

(3) Immediate from (1).
\end{proof}

\begin{lemma} \label{starFilterOrderingLemma}
Let $X$ be a set, $\mathcal{F} \subseteq \powerfilters(X)$ a set of filters on $X$ and $G,H\in \powerfilters(X)$ filters. Then
\begin{enumerate}
\item if $G\subseteq H$, then $\starFilter(G,\mathcal{F}) \subseteq \starFilter(H,\mathcal{F})$;
\item if $\mathcal{F}$ is a cover of $X$, then $\starFilter(\pfilter{x}, \mathcal{F}) \in \upset \mathcal{F}^*$ for all $x\in X$.
\end{enumerate}
\end{lemma}
\begin{proof}
(1) If $G\subseteq H$, then $F\amesh H$ implies $F\amesh G$ for all filters $F\in\powerfilters(X)$. Thus $\setbuilder{F\in\mathcal{F}}{F\amesh H} \subseteq \setbuilder{F\in\mathcal{F}}{F\amesh G}$, which implies
\[ \starFilter(G,\mathcal{F}) = \bigcap\setbuilder{F\in\mathcal{F}}{F\amesh G} \subseteq \bigcap\setbuilder{F\in\mathcal{F}}{F\amesh H} = \starFilter(H,\mathcal{F}). \]

(2) By \ref{principalUltrafilterInCover} there exists $F\in \mathcal{F}$ such that $F\subseteq \pfilter{x}$. Then $\starFilter(F, \mathcal{F}) \subseteq \starFilter(\pfilter{x}, \mathcal{F})$. As $\starFilter(F, \mathcal{F}) \in \mathcal{F}^*$, this implies the result.
\end{proof}

\begin{lemma}
Let $X$ be a set, $F\in\powerfilters(X)$ and $\mathcal{F}\subseteq \powerfilters(X)$. Then
\begin{enumerate}
\item $(F\otimes F);\Big(\bigcap_{G\in \mathcal{F}}G\otimes G\Big) \supseteq F\otimes \starFilter(F, \mathcal{F})$;
\item if $H\in\powerfilters(X^2)$ is symmetric, then $\starFilter(\pfilter{x}, \setbuilder{Hy}{y\in X}) = (H;H)x$.
\end{enumerate}
\end{lemma}
\begin{proof}
(1) We calculate, using that $(F\otimes F);-$ preserves intersections by \ref{filterCompositionResidual}, as well as \ref{orderPreservingFunctionLatticeOperations},
\begin{align*}
(F\otimes F);\Big(\bigcap_{G\in \mathcal{F}}G\otimes G\Big) &= \bigcap \setbuilder{(F\otimes F); (G\otimes G)}{G\in \mathcal{F}} \\
&= \bigcap \setbuilder{F\otimes G}{G\in \mathcal{F}, G\amesh F} \\
&\supseteq F\otimes \bigcap \setbuilder{G \in \mathcal{F}}{G\amesh F} \\
&= F\otimes \starFilter(F, \mathcal{F}).
\end{align*}

(2) ????
\end{proof}

\subsubsection{Barycentric refinement}
\begin{definition}
Let $X$ be a set and $\mathcal{F},\mathcal{G} \subseteq \powerfilters(X)$ sets of filters on $X$.
Then we define the \udef{ultrastar} $\mathcal{F}^U$ of $\mathcal{F}$ as
\[ \mathcal{F}^U \defeq \setbuilder{\starFilter(U, \mathcal{F})}{U\in \powerultrafilters(X) \cap \upset\mathcal{F}} \]
We call $\mathcal{F}$ an \udef{ultrastar refinement} or a \udef{barycentric refinement} of $\mathcal{G}$ if $\mathcal{F}^U \subseteq \upset \mathcal{G}$.
We write $\mathcal{F} <^B \mathcal{G}$.
\end{definition}

\subsubsection{Star-refinement}
\begin{definition}
Let $X$ be a set and $\mathcal{F},\mathcal{G} \subseteq \powerfilters(X)$ sets of filters on $X$. Then we call $\mathcal{F}$ a \udef{star-refinement} of $\mathcal{G}$ if $\mathcal{F}^* \subseteq \upset \mathcal{G}$. We write $\mathcal{F} <^* \mathcal{G}$.
\end{definition}

\begin{lemma}
Let $X$ be a set and $\mathcal{F}\subseteq \powerfilters(X)$ a cover of $X$. Then $\mathcal{F} <^* \big\{\{X\}\big\}$.
\end{lemma}
\begin{proof}
Every set of filters is a subset of $\upset \big\{\{X\}\big\}$.
\end{proof}

\begin{proposition}
Let $X$ be a set and $\mathcal{F},\mathcal{G}, \mathcal{H}\subseteq \powerfilters(X)$ covers of $X$. Then
\begin{enumerate}
\item if $\mathcal{F} <^* \mathcal{G}$, then $\mathcal{F} <^B \mathcal{G}$;
\item if $\mathcal{F}$ contains no free filters, then $\mathcal{F} <^B \mathcal{G}$ and $\mathcal{G} <^B \mathcal{H}$ imply $\mathcal{F} <^* \mathcal{H}$.
\end{enumerate}
\end{proposition}
\begin{proof}
(1) For all $U\in \powerultrafilters(X) \cap \upset\mathcal{F}$, there exists $F\in \mathcal{F}$ such that $F\subseteq U$. By \ref{starFilterOrderingLemma}, we have that $\starFilter(U,\mathcal{F}) \supseteq \starFilter(F, \mathcal{F})$. Thus $\mathcal{F}^U \subseteq \upset\mathcal{F}^*$. By assumption $\upset \mathcal{F}^* \subseteq \upset\mathcal{G}$. The result follows by transitivity.

(2) Assume $\mathcal{F} <^B \mathcal{G}$ and $\mathcal{G} <^B \mathcal{H}$. Take $F\in\mathcal{F}$. We need to show that there exists $H\in \mathcal{H}$ such that $H \subseteq \starFilter(F,\mathcal{F})$.

First consider the case $F = \powerset(X)$. Then any $H\in \mathcal{H}$ works because $\starFilter\big(\powerset(X), \mathcal{F}\big) = \powerset(X)$.

Now let $F$ be proper. By the ultrafilter lemma \ref{ultrafilterLemma}, we can find an ultrafilter $U\supseteq F$. By assumption, there exists $H\in \mathcal{H}$ such that $H \subseteq \starFilter(\pfilter{x}, \mathcal{G})$. We claim this $H$ satisfies the requirement. To show
\[ H \subseteq \starFilter(F,\mathcal{F}) = \bigcap\setbuilder{F'\in \mathcal{F}}{F'\amesh F}, \]
we show that $H\subseteq F'$ for all $F'\in \mathcal{F}$ such that $F'\amesh F$.
\end{proof}

\begin{proposition}
Let $X$ be a set. The relation $<^*$ is a transitive relation on $\powerset\big(\powerfilters(X)\big)$.
\end{proposition}
\begin{proof}
Let $\mathcal{F}, \mathcal{G}, \mathcal{H}\subseteq \powerfilters(X)$ be sets of filters on $X$.

Suppose $\mathcal{F}<^* \mathcal{G}$ and $\mathcal{G} <^* \mathcal{H}$.
Then $\mathcal{F}^*\subseteq \upset \mathcal{G}$ and $\mathcal{G}^* \subseteq \upset \mathcal{H}$. By \ref{starRefinementLemma}, we have $\mathcal{G}\subseteq \upset \mathcal{G}^*$.

These inclusions imply $\upset \mathcal{G}^* \subseteq \upset \mathcal{H}$ and $\upset \mathcal{G} \subseteq \upset \mathcal{G}^*$. Combining gives
\[ \mathcal{F}^* \subseteq \upset \mathcal{G} \subseteq \upset \mathcal{G}^* \subseteq \upset \mathcal{H}, \]
so $\mathcal{F} <^* \mathcal{H}$.
\end{proof}
TODO also from previous proposition?

\subsection{Uniform covers}
\begin{definition}
Let $X$ be a set. A set $\mathbf{U}\subseteq \powerset\big(\powerfilters(X)\big)$ is called a \udef{uniform cover set} if it is a filter w.r.t. star-refinement.
\end{definition}

TODO: non-finite-depth: not any two, but any one element of $\mathcal{U}$ has a star-lower bound?

\begin{lemma}
A uniform cover set is upwards closed w.r.t. inclusion.
\end{lemma}
\begin{proof}
Let $\mathbf{U}$ be a uniform cover on a set $X$. 
\end{proof}

\begin{proposition}
Let $X$ be a set. Then the function
\[ \{\text{uniformities on $X$}\} \to \{\text{uniform cover sets of $X$}\}: \mathcal{U} \mapsto \setbuilder{\setbuilder{Hx}{x\in X}\big.}{H\in \mathcal{U}} \]
is a bijection with inverse
\[ \{\text{uniform cover sets of $X$}\} \to \{\text{uniformities on $X$}\}: \mathbf{U} \mapsto \upset\setbuilder{\bigcap \setbuilder{F \otimes F}{F\in \mathcal{F}}}{\mathcal{F}\in \mathbf{U}}. \]
\end{proposition}
\begin{proof}


Now we show that $\upset\setbuilder{\bigcap \setbuilder{F \otimes F}{F\in \mathcal{F}}}{\mathcal{F}\in \mathbf{U}}$ is a uniformity for all uniform covers $\mathbf{U}$. It is clear that it is closed under transposition and upwards closure is given by construction.

Take arbitrary $x\in X$. For $\pfilter{x}\otimes\pfilter{x}$ to be in the uniformity, it is enough to show that $\pfilter{x}\otimes\pfilter{x} \supseteq \bigcap\setbuilder{F\otimes F}{F\in\mathcal{F}}$ any $\mathcal{F}\in\mathbf{U}$. This follows straight from \ref{principalUltrafilterInCover}.
\end{proof}


\section{TODO Star refinement}
\begin{definition}
Let $X$ be a set and $U,V\subseteq \powerset(X)$ covers of $X$.
\begin{itemize}
\item The \udef{star} of $A\subseteq X$ w.r.t. $U$ is the set defined by
\[ \operatorname{star}_U(A) \defeq \bigcup\setbuilder{B\in U}{A\mesh B}. \]
\item The \udef{star} of $U$ is
\[ U^* \defeq \setbuilder{\operatorname{star}_U(B)}{B\in U}. \]
\item The cover $U$ \udef{star refines} the cover $V$ if $U^*\subseteq \downset V$. This is denoted $U <^* V$.
\end{itemize}
\end{definition}
Note the different direction compared with the definition of refinement.

\begin{lemma}
Star refinement is a transitive relation on $\powerset^2(X)$.
\end{lemma}
\begin{proof}
Assume $U <^* V$ and $V<^* W$, i.e.\ $U^* \subseteq \downset V$ and $V^* \subseteq \downset W$. The last inclusion implies $\downset V^* \subseteq \downset W$. Now $V\subseteq \downset V^*$ because $A\subseteq \operatorname{star}_V(A)$ for all $A\in V$, which implies $\downset V\subseteq \downset V^*$. Then we have
\[ U^* \subseteq \downset V \subseteq \downset V^* \subseteq \downset W, \]
and, by transitivity of inclusion, $U <^* W$.
\end{proof}

\begin{example}
Star refinement is not reflexive in general. 

Consider the set $X = \{0,1,2\}$ and the cover $U = \big\{\{0,1\}, \{1,2\}\big\}$. Then $U^* = \big\{\{0,1,2\}\big\}$, but $\{0,1,2\}\notin \downset U$.
\end{example}

\begin{lemma}
Let $X$ be a set and $C,D\subseteq \powerset(X)$. If $C\subseteq D$, them $C^*\subseteq \downset D^*$.
\end{lemma}
\begin{proof}
Take $A \in C^*$. Then $A = \operatorname{star}_C(B) = \bigcup \setbuilder{B'\in C}{B'\mesh B}$
\end{proof}

\subsubsection{Tolerance cover}
\begin{definition}
Let $V$ be a tolerance relation on a set $X$. The \udef{(tolerance) cover} of $X$ associated to $V$ is defined as
\[ C_V \defeq \setbuilder{xV}{x\in X}. \]
\end{definition}

\begin{lemma}
Let $V, W$ be tolerance relations on $X$ and $C_V, C_W$ their associated covers.
\end{lemma}

\begin{lemma} \label{toleranceCoverStarRefinement}
Let $V$ be a tolerance relation on $X$ and $C_V$ its associated cover. Then
\begin{enumerate}
\item $\operatorname{star}_{C_V}(xV) = x(V;V;V)$;
\item $C_V <^* C_{V;V;V}$.
\end{enumerate}
\end{lemma}
\begin{proof}
(1) We calculate
\begin{align*}
\operatorname{star}_{C_V}(xV) &= \bigcup \setbuilder{yV}{xV\mesh yV} \\
&= \bigcup \setbuilder{yV}{x(V;V)y} \\
&= x(V;V;V).
\end{align*}

(2) By (1), each $\operatorname{star}_{C_V}(xV)\in C_V^*$ is an element of $C_{V;V;V}$.
\end{proof}

\subsubsection{Uniform covers}
\begin{definition}
Let $X$ be a set. A \udef{uniform cover set} of $X$ is a filter of covers in $\sSet{\powerset^2(X), <^*}$. Members of a uniform cover set are called \udef{uniform covers}.
\end{definition}

\begin{proposition}
Let $X$ be a set.
\begin{enumerate}
\item Let $\entourage$ be a topological entourage filter. Then
\[ \setbuilder{C}{\exists V\in\entourage: C_V\subseteq \downset C} = \setbuilder{C_V}{V\in\entourage}_{<^*} \]
is a uniform cover set.
\item Let $\mathcal{C}$ be a uniform cover set. Then
\[ \upset \setbuilder{\bigcup\setbuilder{A\times A}{A\in C}}{C\in\mathcal{C}} \]
is a unform filter set.
\end{enumerate}
\end{proposition}
\begin{proof}
(1) We first verify the equality. Take $V\in \entourage$. Then there exists $W\in \entourage$ such that $W;W;W\subseteq V$.


upwards closure: Let $C$ be a uniform cover and $C <^* D$. Then 
\end{proof}



\begin{definition}
Let $X$ be a set. A \udef{uniform cover set} of $X$ is a filter of covers in $\sSet{\powerset^2(X), <^*}$.
\end{definition}






\chapter{Connectedness}
\section{Connectedness}
\begin{definition}
Let $\sSet{X, \xi}$ be a convergence space. A \udef{separation} of $X$ is a pair $(U,V)$ of disjoint nonempty closed subsets of $X$ whose union is $X$.

The space $X$ is said to be \udef{connected} if there does not exist a separation of $X$.
\end{definition}

\begin{lemma} \label{disconnectionLemma}
Let $\sSet{X, \xi}$ be a convergence space and $U,V\subseteq X$. The following are equivalent:
\begin{enumerate}
\item $(U, V)$ is a separation;
\item $U$ is clopen and not equal to $\emptyset$ or $X$ (then $V= U^c$);
\item $(U^c, V^c)$ is a separation.
\end{enumerate}
\end{lemma}
\begin{proof}
$(1) \Rightarrow (2)$ From $U\perp V$, we get $U\subseteq V^c$ by \ref{setPerpInequality}. From $U\cup V = U\cup V^{cc} = X$, we get $V^c \subseteq U$ by \ref{inclusionCriteria}. Now $U$ and $V$ are closed. So $U = V^c$ is also open.

$(2) \Rightarrow (3)$ We have $U^c = V^{cc}$, so $U^c \subseteq V^{cc}$ and $V^{cc}\subseteq U^c$. As before, we get $U^c\perp V^c$ and $U^c \cup V^c = X$ by \ref{setPerpInequality} and \ref{inclusionCriteria}. Clearly $U^c$ and $V^c = U$ are closed.

$(3) \Rightarrow (1)$ This follows by applying the implication $(1) \Rightarrow (3)$ to the separation $(U^c, V^c)$: we get that $(U^{cc}, V^{cc}) = (U,V)$ is a separation.
\end{proof}
\begin{corollary} \label{connectedCriteria}
Let $\sSet{X, \xi}$ be a convergence space. The following are equivalent:
\begin{enumerate}
\item $X$ is connected;
\item $X$ does not contain a pair of disjoint nonempty open subsets, whose union is $X$;
\item the only clopen subsets of $X$ are $X$ and $\emptyset$.
\end{enumerate}
\end{corollary}

\begin{proposition} \label{continuousFunctionsConnectedToDiscrete}
Let $\sSet{X, \xi}$ be a convergence space and $Y$ a discrete convergence space with at least two elements. The following are equivalent:
\begin{enumerate}
\item $X$ is connected;
\item all continuous functions from $X$ to $Y$ are constant.
\end{enumerate}
\end{proposition}
\begin{proof}
By \ref{connectedCriteria}, (1) is equivalent to $X,\emptyset$ being the only clopen sets in $X$.

$(1) \Rightarrow (2)$ Take arbitrary continuous $f\in (X\to Y)$ and take $y\in \im(f)$. Then $\{y\}$ is clopen by \ref{discreteTopologyCharacterisation}. Now $f^\preimf\{y\}$ is clopen by \ref{preimageOpenClosed} and $f^\preimf\{y\}\neq \emptyset$ (because $y\in\im(f)$). So $f^\preimf\{y\} = X$, which means that $f$ is constant.

$(2) \Rightarrow (1)$ Take some clopen $A\subseteq X$. Take distinct $y_1,y_2\in Y$. And set
\[ f: X\to Y: x\mapsto \begin{cases}
y_1 & (x\in A) \\
y_2 & (x\notin A)
\end{cases}. \]
This function is continuous. Indeed, take arbitrary $F\to x$. If $x\in A$, then $A\in F$ because $A$ is open, by \ref{openClosedCriteria}. So $f^{\imf\imf}[F] \supseteq \pfilter{y_1} \to y_1$. Similarly $f^{\imf\imf}[F] \supseteq \pfilter{y_2} \to y_2$ if $x\in A^c$.

Then, by assumption, $f$ must be constant, so either $f = \constant{y_1}$ or $f = \constant{y_2}$. In the first case $A = X$ and in the second $A = \emptyset$.
\end{proof}

TODO: also equivalent in topological spaces: $X$ cannot be written as the union of two non-empty separated sets.

The following lemma characterises separations in the subspace topology.
\begin{lemma}
Let $Y$ be a subspace of a topological space $X$. A pair of disjoint nonempty sets $A,B$ constitute a separation of the subspace $Y$ \textup{if and only if} neither set contains a limit point of the other in $X$.
\end{lemma}
\begin{proof}
TODO ??
\end{proof}

\begin{proposition} \label{connectednessImage}
Let $\sSet{X,\xi}$ and $\sSet{Y,\zeta}$ be convergence spaces and $f:X\to Y$ a continuous function. If $X$ is connected, then $\im(f)$ is a connected subspace of $Y$.
\end{proposition}
\begin{proof}
Suppose, towards a contradiction, that $\im(f)$ is not connected, so there exists a separation $(U,V)$ of $\im(f)$. As $f: X\to \im(f)$ is continuous, we have that $f^\preimf(U)$ and $f^{\preimf}(V)$ are closed by \ref{preimageOpenClosed}. They are clearly not empty. We also have that $f^\preimf(U)$ and $f^{\preimf}(V)$ are disjoint because $\emptyset = f^\preimf(\emptyset) = f^\preimf(U\cap V) = f^\preimf(U)\cap f^\preimf(V)$, by \ref{imagePreimageGaloisConnection}. Thus $\big(f^\preimf(U), f^\preimf(V)\big)$ is a separation of $X$. This is a contradiction because $X$ was assumed connected.
\end{proof}

\begin{proposition} \label{generalisedIntermediateValueTheorem}
Let $\sSet{X,\xi}$ be a connected convergence space, $\sSet{Y, \leq}$ a totally ordered set with the order convergence and $f: X\to Y$ a continuous function. For all $a,b\in X$ and $y\in Y$ such that $f(a) \leq y \leq f(b)$, there exists $c\in X$ such that $f(c) = y$.
\end{proposition}
\begin{proof}
Assume, towards a contradiction, that $y\notin \im(f)$. Then $Y\cap \downset \{y\}$ and $Y\cap \upset \{y\}$ constitutes a separation of $\im(f)$. This contradicts \ref{connectednessImage}.
\end{proof}

\subsection{Chains}
\begin{definition}
Let $\sSet{X,\xi}$ be a convergence space and $\mathcal{U}$ an open cover of $X$. A \udef{chain} in $\mathcal{U}$ between two points $x,y\in X$ is a finite sequence $\seq{U_i}_{i=0}^n$ of sets in $\mathcal{U}$ such that
\begin{itemize}
\item $x\in U_0$ and $y\in U_n$;
\item $U_{i-1}\mesh U_i$ for all $i\in 1:n$.
\end{itemize}
\end{definition}

\begin{proposition} \label{connectedChains}
Let $\sSet{X,\xi}$ be a convergence space. Then $X$ is connected \textup{if and only if} for each open cover $\mathcal{U}$ and each pair of points $x,y$ there exists a chain in $\mathcal{U}$ between $x$ and $y$.
\end{proposition}
\begin{proof}
First assume $X$ not connected. Then we can an open cover $\{U,V\}$, where $U$ and $V$ are disjoint. For any $x\in U$ and $y\in V$, there clearly does not exist a chain in $\{U,V\}$ between $x$ and $y$.

Now assume $X$ connected. Take arbitrary $x\in X$ and open cover $\mathcal{U}$. Define
\[ O \defeq \setbuilder{y\in X}{\text{there exists a chain in $\mathcal{U}$ between $x$ and $y$}}. \]
Then $O$ is a union of open sets in $\mathcal{U}$ and thus open.

We also show that $O^c$ is also open. Take $z\in O^c$. Since $\mathcal{U}$ covers $X$, there exists $U\in\mathcal{U}$ such that $z\in U$. Then $U\in \vicinity(z)$ by \ref{openClosedCriteria}.

Now $U\perp O$, so $U\subseteq O^c$. Indeed, assume not, i.e.\ there exists $y\in U\cap O$. Then there exists a chain $\seq{U_{i}}_{i=0}^n$ between $x$ and $y$, so $\seq{U_{i}}_{i=0}^n\star\seq{U}$ is a chain between $x$ and $z$. This contradicts $z\in O^c$.

Since $U\subseteq O^c$, we conclude that $O^c$ is open by \ref{openClosedCriteria}. So $O$ is non-empty, open and closed. By \ref{connectedCriteria} $O = X$.
\end{proof}


\section{Path connectedness}
\begin{definition}
Let $\sSet{X,\xi}$ be a convergence space. Then $\xi$ is called \udef{path connected} if, for all $x,y\in X$ there exists a continuous function $f:\interval{0,1}\to X$ such that $f(0) = x$ and $f(1) = y$.
\end{definition}

\begin{lemma}
Let $\sSet{X,\xi}$ be a convergence space. If $\xi$ is path connected, then it is connected.
\end{lemma}
TODO: we can also argue that $X/\{U,V\}$ is discrete and use \ref{continuousFunctionsConnectedToDiscrete}.
\begin{proof}
Suppose, towards a contradiction, that $\xi$ is path connected, but not connected. Then there exists a separation $(U,V)$ of $X$. There exist $x\in U, y\in V$ and a continuous function $f: \interval{0,1}\to X$ such that $f(0)=x$ and $f(1) = y$.

Now $f^\preimf(U)$ and $f^\preimf(V)$ are non-empty, disjoint and closed sets. Now $\interval{0,1} = f^\preimf(U\cup V) = f^\preimf(U)\cup f^\preimf(V)$, so $\big(f^\preimf(U), f^\preimf(V)\big)$ is a separation of $\interval{0,1}$. This contradicts the connectedness of $\interval{0,1}$ by \ref{connectedSubsetReals}.
\end{proof}

\begin{example}
Not the same as connectedness: topologist's sine curve. 
\end{example}

\subsection{Local path connectedness}
\begin{definition}
Let $\sSet{X,\xi}$ be a convergence space. Then $X$ is called \udef{locally path connected} if it is locally open and path connected.
\end{definition}

\begin{proposition}
Let $\sSet{X,\xi}$ be a topological convergence space. If $X$ is locally path connected and connected, then it is path connected.
\end{proposition}
\begin{proof}
By assumption, the open path-connected sets form an open cover. Since $X$ is connected, there exists a chain of open path-connected sets between any two points by \ref{connectedChains}. This can be easily turned into a continuous function from the unit interval by concatenation. 
\end{proof}

\begin{proposition}
Let $\sSet{X,\xi}$ be a locally path connected space and $A\subseteq X$ an open subset. Then the subspace $A$ is locally path connected.
\end{proposition}
\begin{proof}
TODO
\end{proof}


\chapter{TODO}

\url{https://en.wikipedia.org/wiki/List_of_topologies}

\url{http://www.dynamics-approx.jku.at/lena/Cooper/riesz.pdf} For order convergence!!!


\url{https://en.wikipedia.org/wiki/Characterizations_of_the_category_of_topological_spaces} 
\url{https://mathoverflow.net/questions/19152/why-is-a-topology-made-up-of-open-sets/19173#19173}




\subsection{Closure and interior of a set}
\url{https://en.wikipedia.org/wiki/Kuratowski_closure_axioms}

\begin{definition}
Given any subset $A$ of a topological space $X$,
\begin{itemize}
\item The \udef{interior} of $A$, denoted $A^\circ$, is the union of all open sets contained in $A$;
\item The \udef{closure} of $A$, denoted $\bar{A}$, is the intersection of all closed sets containing $A$. 
\end{itemize}
The \udef{boundary} of $A$ is $\partial A \defeq \bar{A}\setminus A^{\circ}$.
\end{definition}
We immediately have the inclusions
\[ A^\circ \subset A \subset \bar{A} \]

\begin{lemma}
The interior and closure are dual in the sense that
\[ A^\circ = X\setminus\overline{(X\setminus A)} = \overline{(A^c)}^c \qquad \bar{A} = X\setminus(X\setminus A)^\circ = ((A^c)^\circ)^c \]
where $X$ is a topological space and $A$ is a subset.
\end{lemma}
\begin{proposition}\label{closure}
Let $A$ be a subset of the topological space $X$, then
\[ x\in \bar{A} \qquad \text{\textup{if and only if}}\qquad \text{every open set $U$ containing $x$ intersects $A$}.\]
\end{proposition}
\begin{proof}
We prove the contrapositive.
\[ x\notin \bar{A} \iff \text{there exists an open set $U$ containing $x$ that does not intersect $A$.} \]
\begin{itemize}
\item[$\boxed{\Rightarrow}$] The set $U = X\setminus \bar{A}$ is an open set containing $x$ that does not intersect $A$.
\item[$\boxed{\Leftarrow}$] If there exists such a $U$, then $X\setminus U$ is a closed set containing $A$, so $X\setminus U \supset \bar{A}$. Therefore $x$ cannot be in $\bar{A}$.
\end{itemize}
\end{proof}
\begin{proposition}\label{interior}
Let $A$ be a subset of the topological space $X$, then
\[ x\in A^\circ \qquad \text{\textup{if and only if}}\qquad \text{there exists an open set $U$ such that $x\in U \subset A$}.\]
\end{proposition}
\begin{proof}
The interior is the union of all open sets $U\subset A$. Thus if $x\in A^\circ$, then $x$ is in such a $U$.
\end{proof}
\begin{lemma}
Given any subset $A$ of $X$,
\begin{itemize}
\item $\bar{A}$ is the smallest closed set containing $A$;
\item $A^\circ$ is the largest open set contained in $A$.
\end{itemize}
Consequently the closure and interior are idempotent:
\[ \overline{\bar{A}} = \bar{A} \qquad \text{and} \qquad (A^\circ)^\circ = A^\circ. \]
\end{lemma}

\begin{lemma}
Let $A,B$ be subsets of a topological space $X$. Then
\begin{enumerate}
\item $\overline{A\cup B} = \overline{A}\cup \overline{B}$;
\item $(A\cap B)^\circ = A^\circ \cap B^\circ$.
\end{enumerate}
These properties do not hold for arbitrary unions and intersections.
\end{lemma}
\begin{proof}
TODO
\end{proof}
\begin{lemma}
TODO: Intersection of Interiors contains Interior of Intersection and Closure of Union contains Union of Closure and Closure of Intersection is Subset of Intersection of Closures and Union of Interiors is Subset of Interior of Union
\end{lemma}

\begin{lemma} \label{closureInteriorSubsets}
Let $A\subseteq B$ be sets in a topological space $X$. Then
\begin{enumerate}
\item $\overline{A} \subseteq \overline{B}$;
\item $A^\circ \subseteq B^\circ$.
\end{enumerate}
\end{lemma}

\subsection{Boundaries}
\url{https://math.stackexchange.com/questions/2254363/definitions-of-a-topological-space-reference}
\url{https://math.stackexchange.com/questions/4398247/axiomatizations-of-the-boundary-operator}
\url{https://mathoverflow.net/questions/175800/which-sets-occur-as-boundaries-of-other-sets-in-topological-spaces}

\begin{lemma}
The boundary $\partial A$ is closed.
\end{lemma}

\subsection{Metrics}
Quantales and continuity spaces: \url{https://link.springer.com/content/pdf/10.1007/s000120050018.pdf}
\url{https://arxiv.org/abs/1311.4940}
All Topologies Come From Generalized Metrics - Kopperman

\subsection{Limit points}
\begin{definition}
If $A$ is a subset of the topological space $X$ and if $x$ is a point of $X$ (not necessarily of $A$), we say $x$ is a \udef{limit point} (also sometimes called \udef{cluster point} or \udef{point of accumulation}) of $A$ if every (open) neighbourhood of $x$ intersects $A$ in some point other than $x$ itself.

The set $A'$ of all limit points of $A$ is called the \udef{derived set} of $A$.

An \udef{isolated point} of $A$ is a point $x\in A$ that is not an accumulation point for $A$.
\end{definition}
So $x$ is a limit point of $A$ if it belongs to the closure of $A\setminus \{x\}$.
\begin{example}
Consider $\R$. If $A= ]0,1]$, then the point $0$ is a limit point of $A$. In fact every point in $[0,1]$ is a limit point and no other points of $\R$ are limit points.
\end{example}
This motivates the following assertion:
\begin{proposition}
Let $A$ be a subset of a topological space $X$. Then
\[ \bar{A} = A \cup A' = A^\circ \cup A' \]
where $A'$ is the derived set of $A$.
\end{proposition}
\begin{corollary}
A topological space is closed if and only if it contains all its limit points.
\end{corollary}

\begin{definition}
Let $(X,\mathcal{T})$ be a topological space. A subset $A\subset X$ is \udef{perfect} in $X$ if it is closed and every point of $A$ is an accumulation point of $A$.
\end{definition}
\begin{lemma}
If $A$ has no isolated points, then $\overline{A}$ is perfect in $X$.
\end{lemma}

\subsection{Special subsets}
\begin{definition}
Let $(X,\mathcal{T})$ be a topological space. A set $A\subset X$ is called
\begin{itemize}
\item a \udef{$\mathcal{G}_\delta$-set} if it is a countable intersection of open sets;
\item an \udef{$\mathcal{F}_\sigma$-set} if it is a countable union of closed sets.
\end{itemize}
\end{definition}

\section{Topologies}

\subsection{The subspace topology}
\begin{definition}
Let $X$ be a topological space with topology $\mathcal{T}$. Let $Y$ be a subspace of $X$. The collection
\[ \mathcal{T}_Y = \{ Y\cap U\;|\; U\in \mathcal{T} \} \]
is a topology on $Y$ called the \udef{subspace topology}. With this topology, $Y$ is called a \udef{subspace} of $X$.
\end{definition}
\begin{lemma}
Let $Y$ be a subspace of $X$. A set $A$ is closed in $Y$ \textup{if and only if} it equals the intersection of a closed set of $X$ with $Y$.
\end{lemma}

\begin{lemma}
If $\mathcal{B}$ is a basis for the topology of $X$, then
\[\mathcal{B}_Y = \{ B\cap Y \;|\; B\in \mathcal{B} \}\]
is a basis for the subspace topology on $Y$.
\end{lemma}

\begin{lemma}
Let $Y$ be a subspace of $X$.
\begin{enumerate}
\item If $A$ is open in $Y$ and $Y$ is open in $X$, then $A$ is open in $X$.
\item If $A$ is closed in $Y$ and $Y$ is closed in $X$, then $A$ is closed in $X$.
\end{enumerate}
\end{lemma}
 
We reserve the notation $\overline{A}$ to stand for the closure of $A$ in $X$, not $Y$.
\begin{lemma} \label{subspaceClosure}
Let $X$ be a topological space and $Y\subset X$ a subspace. Let $A$ be a subset of $Y$, then the closure of $A$ in $Y$ is
\[ \Closure_Y(A) = \overline{A}\cap Y.  \]
\end{lemma}

\begin{lemma} \label{notLimitPointSingletonOpen}
Let $A\subseteq X$ be a subspace of $X$. Then $a\in A\setminus A'$, then $\{a\}$ is open in $A$.
\end{lemma}
\begin{proof}
Assume such an $a$. Then there exists an open neighbourhood $U$ of $a$ in $X$ that does not intersect $A$ in any other point. By definition of the subspace topology $\{a\}$ is open.
\end{proof}

\subsection{Topology and order}


\url{https://planetmath.org/orderedspace}
\url{https://www.jstor.org/stable/2032122?seq=2#metadata_info_tab_contents}
\url{http://www.math.wm.edu/~lutzer/drafts/PragueSurveyFinal.pdf}
\url{https://ncatlab.org/nlab/show/pospace}
\url{file:///C:/Users/user/Downloads/order-topological-lattices.pdf}

\subsubsection{Specialisation preorder}
\begin{definition}
Let $(X,\mathcal{T})$ be a topological space and $x,y\in X$. We say $x$
\end{definition}

\paragraph{Alexandrov topology}
\url{https://planetmath.org/inducedalexandrofftopologyonaposet}
\url{https://arxiv.org/pdf/0708.2136.pdf}
\url{https://ncatlab.org/nlab/show/specialization+topology}
\url{http://math.uchicago.edu/~may/REU2018/REUPapers/Asness.pdf}

\subsubsection{Order topology on totally ordered sets}
\begin{definition}
Let $(X,\leq)$ be a linearly ordered set. Let $\mathcal{B}$ be the collection of all sets of the following type:
\begin{enumerate}
\item All open intervals $]a,b[$ in $X$;
\item All intervals of the form $[a_0, b[$, where $a_0$ is the smallest element (if any) of $X$;
\item All intervals of the form $]a, b_0]$, where $b_0$ is the largest element (if any) of $X$;
\end{enumerate}
The collection $\mathcal{B}$ is a basis for a topology, called the \udef{order topology}.
\end{definition}


\section{Separation axioms}

\subsection{$T_1$}
\begin{proposition}
Let $X$ be a topological space satisfying $T_1$; let $A$ be a subset of $X$.
Then the point $x$ is a limit point of $A$ \textup{if and only if} every neighbourhood of $x$ contains infinitely many points of $A$.
\end{proposition}
\subsection{Hausdorff spaces}
\begin{definition}
A topological space $X$ is called a \udef{Hausdorff space} if for each pair $x_1, x_2$ of distinct points in $X$, their exist neighbourhoods $U(x_1)$, $U(x_2)$ that are disjoint.
\end{definition}
In Hausdorff spaces distinct points can be told apart topologically, hence Hausdorff spaces are also called \udef{separated spaces}. In particular the Hausdorff condition implies the uniqueness of limits, which is not otherwise guaranteed.

\begin{proposition}
Every finite point set in a Hausdorff space is closed. TODO: T1
\end{proposition}
\begin{proof}
It suffices to show that every one-point set $\{x_0\}$ is closed. Indeed if $\{x_0\}$ was not closed, the closure of $\{x_0\}$ would contain another point. This other point has a disjoint neighbourhood by Hausdorff, so this fails by proposition ?.
\end{proof}
\begin{proposition}
Limits are unique (sequences, filters, nets)
\end{proposition}
\begin{lemma}
\begin{enumerate}
\item A subspace of a Hausdorff space is Hausdorff.
\item Every totally ordered set is Hausdorff in the order topology.
\item Every metric topology is Hausdorff.
\end{enumerate}
\end{lemma}



\section{Functions on topological spaces}
\subsection{Continuity and continuous functions}
Intuitively, a continuous map is a map between topological spaces that does not make jumps. In particular let $f: X\to Y$ be a potentially continuous function. Say we want to stay in a neighbourhood $V(f(x_0))$, then we want there to be a neighbourhood $U(x_0)$ such that points inside $U(x_0)$ map to points in $V$, i.e.\
\[ x\in U(x_0) \implies f(x) \in V(f(x_0)). \]
That is $f(U(x_0)) \subset V$, or $U(x_0)\subset f^{-1}(V)$. So we conclude that for any point $x_0$ and neighbourhood $V(f(x_0))$ in $Y$, $f^{-1}(V)$ must contain a neighbourhood of $x_0$. Thus $f^{-1}(V)$ can be written as a union of open sets, $\bigcup_{x\in f^{-1(V)}}U(x)$, and therefore must be open. This motivates the definition:
\begin{definition}
Let $X,Y$ be topological spaces.
\begin{itemize}
\item A function $f:X\to Y$ is \udef{continuous} if for each open set $V$ of $Y$, $f^{-1}[V]$ is an open subset of $X$.
\item The function $f$ is \udef{continuous at $x_0$} if for each open neighbourhood $V$ of $f(x_0)$, their is an open neighbourhood $U$ of $x_0$ such that $f[U]\subset V$.
\end{itemize}
If a function is not continuous, it is \udef{discontinuous}.

The set of all continuous functions $X\to Y$ is denoted $\cont(X,Y)$. If $X=Y$, we also write $\cont(X)$.
\end{definition}
\begin{lemma} \label{globalContinuityFromAllPoints}
A function $f:X\to Y$ is continuous \textup{if and only if} it is continuous at every point.
\end{lemma}

\begin{lemma} \label{continuityAtIsolatedPoint}
Let $f:X\to Y$ be a function between topological spaces. If $\{x_0\}\subset X$ is open, then $f$ is continuous at $x_0$.
\end{lemma}

\begin{lemma}
\begin{enumerate}
\item If the topology of $Y$ is given by a basis $\mathcal{B}$, then to prove continuity of $f$ it suffices to show that the inverse image of every basis element is open.
\item If the topology of $Y$ is given by a subbasis $\mathcal{S}$, then to prove continuity of $f$ it suffices to show that the inverse image of every subbasis element is open.
\end{enumerate}
\end{lemma}
\begin{proposition}\label{continuity}
Let $X, Y$ be topological spaces; $f:X\to Y$. The following are equivalent:
\begin{enumerate}
\item $f$ is continuous;
\item $f[\bar{A}]\subset \overline{f[A]}$;
\item for every closed set $B$ of $Y$, the set $f^{-1}[B]$ is closed in $X$. TODO $f$ closed.
\end{enumerate}
\end{proposition}
\begin{proof}
We proceed round-robin-style.
\begin{itemize}[leftmargin=2cm]
\item[$\boxed{(1) \Rightarrow (2)}$] Let $x\in \bar{A}$ and $V$ a neighbourhood of $f(x)$. Then $f^{-1}[V]$ is an open set containing $x$, so it must intersect $A$ in some point $y$ by proposition \ref{closure}. Then $V$ intersects $f[A]$ in $f(y)$, so $f(x) \in \overline{f[A]}$ as desired.
\item[$\boxed{(2) \Rightarrow (3)}$] Let $B$ be closed in $Y$. We observe that $f[f^{-1}[B]]\subset B$. Choose some $x\in \overline{f^{-1}[B]}$, then
\[ f(x) \in f\left[\overline{f^{-1}[B]}\right] \subset \overline{f[f^{-1}[B]]} \subset \bar{B} = B, \]
so that $x\in f^{-1}[B]$. Thus $\overline{f^{-1}[B]}\subset f^{-1}[B]$, meaning $f^{-1}[B]$ is closed.
\item[$\boxed{(3) \Rightarrow (1)}$] Let $V$ be an open set in $Y$. Set $B = Y\setminus V$. Then $V = Y\setminus B$ and
\[ f^{-1}[V] = f^{-1}[Y\setminus B] = f^{-1}[Y]\setminus f^{-1}[B] = X \setminus f^{-1}[B]\]
using lemma \ref{imagePreimageUniqueness}. Thus $f^{-1}[V]$ is open.
\end{itemize}
\end{proof}

\subsubsection{Homeomorphisms \textit{or} topological isomorphisms}
\begin{definition}
Let $X,Y$ be topological spaces and $f:X\to Y$ a bijection. Then $f$ is a \udef{homeomorphism} if both $f$ and $f^{-1}$ are continuous.
\end{definition}
\begin{lemma}
A homeomorphism is a bijection $f$ such that $f(U)$ is open \textup{if and only if} $U$ is open.
\end{lemma}
\subsubsection{Constructing continuous functions}
\begin{proposition} \label{continuousConstructions}
Let $X,Y$ and $Z$ be topological spaces.
\begin{enumerate}
\item \textup{(Identity function)} The identity function $I:X\to X$ is continuous.
\item \textup{(Constant function)} If $f:X\to Y$ maps all of $X$ into a single $y_0$ of $Y$, then $f$ is continuous.
\item \textup{(Inclusion)} Let $A$ be a subspace of $X$, then the inclusion $A\hookrightarrow X$ is continuous.
\item \textup{(Composites)} If $f:X\to Y$ and $g:Y\to Z$ are continuous, then $g\circ f: X\to Z$ is continuous.
\item \textup{(Restricting the domain)} If $f:X\to Y$ is continuous and $A$ is a subspace of $X$, then the restricted function $f|_{A}:A\to Y$ is continuous.
\item \textup{(Restricting the range)} Let $f:X\to Y$ be continuous. If $Z$ is a subspace of $Y$ containing the image set $f[X]$, then $f:X\to Z$ is continuous.
\item \textup{(Expanding the range)} Let $f:X\to Y$ be continuous. If $Y$ is a subspace of $Z$, then $f:X\to Z$ is continuous.
\item \textup{(Local formulation of continuity)} The map $f:X\to Y$ is continuous is $X$ can be written as the union of open sets $U_\alpha$ such that $f|_{U_\alpha}$ is continuous for each $\alpha$.
\end{enumerate}
\end{proposition}
\begin{proposition}[The pasting lemma]
Let $X=A\cup B$ where $A,B$ are closed in $X$. Let $f:A\to Y$ and $g:B\to Y$ be continuous such that $f(x)=g(x)$ for all $x\in A\cap B$. Then the function defined by
\[ h: X\to Y: x\mapsto h(x) = \begin{cases}
f(x) & (x\in A) \\ g(x) & (x\in B)
\end{cases} \]
is continuous.
\end{proposition}
\begin{proof}
Let $C$ be a closed subset of $Y$, then $f^{-1}[C]$ and $g^{-1}[C]$ are both closed. So
\[ h^{-1}[C] = f^{-1}[C]\cup g^{-1}[C] \]
is closed, meaning $h$ is continuous, all by proposition \ref{continuity}.
\end{proof}

TODO: complex conjugation continuous.

\subsection{Limits of functions}
\begin{definition}
Let $X,Y$ be topological spaces. Let $p$ be a limit point of $A\subseteq X$ and $f: A\to Y$. We say $L\in Y$ is a \udef{limit} of $f(x)$ as $x$ approaches $p$ if
\[ \forall\;\text{open neighbourhood}\; V(L):\exists \;\text{open neighbourhood}\; U(p):\; f[(U\cap A)\setminus \{p\}] \subseteq V. \]
We write $f(x)\to L$ as $x\to p$ or
\[ \lim_{x\to p}f(x) = L. \]
\end{definition}
Note that the value of $f$ at $p$ is irrelevant to the definition of the limit. The domain of $f$ does not even need to contain $p$.

\begin{proposition}
Let $f:X\to Y$ be a functions between topological spaces. Then $f$ is continuous at $p\in X$ \textup{if and only if} $\lim_{x\to p}f(x) = f(p)$.
\end{proposition}

Limits may or may not exist and may or may not be unique, but uniqueness is guaranteed if $Y$ is Hausdorff.
\begin{proposition} \label{HausdorffUniqueLimit}
Let $f: A\subseteq X\to Y$ be a function and $X,Y$ be topological spaces. If $Y$ is Hausdorff, then there is a most one limit of $f$ at any point $p\in X$.
\end{proposition}
\begin{proof}
Assume $L_1$ and $L_2$ are two distinct limits of $f(x)$ as $x\to p$. Because $Y$ is Hausdorff there are two disjoint open neighbourhoods $V_1, V_2$ of $L_1,L_2$. Let $U_1,U_2$ be the corresponding open neighbourhoods of $p$. Then $U_1\cap U_2$ must be an open neighbourhood of $p$, so that $U_1\cap U_2\cap A$ contains a point other than $p$, by virtue of $p$ being a limit point. This however means that $f[(U_1\cap A)\setminus \{p\}]$ and $f[(U_2\cap A)\setminus \{p\}]$ are not disjoint, so neither are $V_1,V_2$: a contradiction.
\end{proof}

\subsection{Sets of functions}
\begin{definition}
Let $X, Y$ be topological spaces.
\begin{itemize}
\item The set of continuous functions in $(X\to Y)$ is denoted $\cont(X, Y)$.
\item The set of continuous functions in $(X\to Y)$ which vanish at infinity is denoted $\cont_0(X, Y)$.
\item The set of continuous functions in $(X\to Y)$ with compact support is denoted $\cont_c(X, Y)$.
\item The set of bounded continuous functions in $(X\to Y)$ is denoted $\cont_b(X, Y)$.
\end{itemize}
If we omit $Y$, we generally mean $Y = \R$.
\end{definition}

TODO: define these notions in general!
TODO: ideals and multiplier algebras.


\section{The product topology}
\subsection{Finite Cartesian products}
\begin{definition}
The \udef{product topology} on $X\times Y$ is the topology having as basis the collection $\mathcal{B}$ of all sets of the form $U\times V$, where $U$ is an open subset of $X$ and $V$ is an open subset of $Y$.
\end{definition}
\begin{lemma} \label{basisFiniteProductTopology}
If $\mathcal{B}$ is a basis for the topology of $X$ and $\mathcal{C}$ a basis for the topology of $Y$, then
\[ \mathcal{D} = \{ B\times C\;|\; B\in \mathcal{B}\;\text{and}\; C\in \mathcal{C} \} \]
is a basis for the topology of $X\times Y$.
\end{lemma}
\begin{proposition}
Let $A$ be a subspace of $X$ and $B$ a subspace of $Y$. The product topology on $A\times B$ is the same as the subspace topology on $A\times B$, when viewed as a subset of $X\times Y$.
\end{proposition}

\begin{definition}
Let $X,Y$ be topological spaces. The maps
\begin{align*}
&\pi_1: X\times Y\to X: (x,y)\mapsto x
&\pi_2: X\times Y\to Y: (x,y)\mapsto y
\end{align*}
are called the \udef{projections} of $X\times Y$ onto its first and second factors, respectively.
\end{definition}
\begin{proposition}
The collection
\[ \mathcal{S} = \{ \pi_1^{-1}(U)\;|\; U\;\text{open in}\; X \}\cup \{ \pi_2^{-1}(V)\;|\; V\;\text{open in}\;Y  \} \]
is a subbasis for the product topology on $X\times Y$.
\end{proposition}
\begin{proof}
Let $\mathcal{T}$ denote the product topology on $X\times Y$; let $\mathcal{T'}$ be the topology generated by $\mathcal{S}$.
\begin{itemize}[leftmargin=2cm]
\item[$\boxed{\mathcal{T}'\subset\mathcal{T}}$] We need to prove all elements of $\mathcal{S}$ are open. Indeed $\pi_1^{-1}(U) = U\times Y$ is open and $\pi_2^{-1}(V) = X\times V$ is also open.
\item[$\boxed{\mathcal{T}\subset\mathcal{T}'}$] Let $B\times C$ be an element of the basis, in other words $B\subset X$ and $C\subset Y$ are open. Then $B\times C = \pi_1^{-1}(B)\cap \pi_2^{-1}(C)$.
\end{itemize}
\end{proof}
In particular $\pi_1$ and $\pi_2$ are continuous.
\begin{proposition}\label{continuityCompositeFunctions}
Let $A,X,Y$ be topological spaces and let
\[ f:A\to X\times Y: a\mapsto f(a) = (f_1(a),f_2(a)). \]
Then $f$ is continuous \textup{if and only if} the functions $f_1$ and $f_2$ are continuous.
\end{proposition}
There is no useful criterion for the continuity of a map $f:A\times B \to X$.

\begin{proposition}
Let $X,Y$ be metrisable topological spaces with metrics $d_X$ and $d_Y$. Then $X\times Y$ is metrisable. Possible, equivalent, metrics include
\[ d_\text{max} = \max\circ \{d_X, d_Y\} \]
and
\[ d_\text{graph} = d_X \circ \pi_1 + d_Y \circ \pi_2. \]
\end{proposition}
\begin{proof}
We first prove that the product topology and the metric topology generated by $d_\text{max}$ are the same using \ref{basisCoarseness}.

First take an element of a basis for the product topology, which by \ref{basisFiniteProductTopology} can be taken of the form
\[ B = B_{d_X}(x, \epsilon_1)\times B_{d_Y}(y, \epsilon_2) \qquad \text{for some $x\in X, y\in Y, \epsilon_1,\epsilon_2 >0$.} \]
Then we can find a basiselement $B_{d_\text{max}}((x,y), \min\{\epsilon_1,\epsilon_2\})$ of the metric topology generated by $d_\text{max}$ that is a subset.

Conversely, take $B_{d_\text{max}}((x,y), \epsilon)$. Then $B_{d_X}(x, \epsilon)\times B_{d_Y}(y, \epsilon)$ is a subset.

The equivalence of the two metrics can then be seen by applying \ref{ballsCoarseness} twice:
\[ B_{d_\text{max}}((x,y), \epsilon) \subset B_{d_\text{graph}}((x,y), \epsilon) \qquad B_{d_\text{graph}}((x,y), \epsilon/2) \subset B_{d_\text{max}}((x,y), \epsilon). \]
\end{proof}
\begin{corollary} \label{convergenceFiniteProductTopology}
A sequence $(x_n,y_n)_n$ converges to $(x,y)$ in the product topology \textup{if and only if} $(x_n)_n$ converges to $x$ and $(y_n)_n$ converges to $y$.
\end{corollary}
TODO:also nets?

\subsection{Arbitrary Cartesian products}
\begin{definition}
Let $X = \prod_{\alpha\in J}X_\alpha$ and define
\[ \mathcal{S}_\beta = \{\pi_\beta^{-1}(U_\beta)\;|\; U_\beta\;\text{open in}\;X_\beta\} \qquad \text{and}\qquad \mathcal{S} = \bigcup_{\beta\in J}.\]
Then the topology on $X$ generated by the subbasis $\mathcal{S}$ is the \udef{product topology} and then $X$ is called a \udef{product space}.
\end{definition}
\begin{lemma}
\begin{itemize}
\item The product topology on $\prod X_\alpha$ has as a basis all sets of the form $\prod_\alpha U_\alpha$, where $U_\alpha$ is open in $X_\alpha$ for all $\alpha$ and $U_\alpha = X_\alpha$ except for finitely many values of $\alpha$.
\item If each $X_\alpha$ has a basis $\mathcal{B}_\alpha$, a basis for the product topology is given by all the sets of the form $\prod_{\alpha\in J}B_\alpha$ where $B_\alpha\in\mathcal{B}_\alpha$ for finitely many values of $\alpha$ and $B_\alpha = X_\alpha$ for the rest.
\end{itemize}
\end{lemma}
If we remove the condition that $U_\alpha = X_\alpha$ except for finitely many values of $\alpha$, we get the box topology.

Some results that held for finite Cartesian product also hold for arbitrary products:
\begin{lemma}
Let the topology on $\prod X_\alpha$ be the product topology.
\begin{itemize}
\item If each space $X_\alpha$ is Hausdorff, then $\prod X_\alpha$ is Hausdorff.
\item Let $A_\alpha$ be subsets of $X_\alpha$, then
\[ \prod \bar{A}_\alpha = \overline{\prod A_\alpha}. \]
\item Let $A_\alpha$ be subspaces of $X_\alpha$, for each $\alpha\in J$. Then $\prod A_\alpha$ is a subspace of $\prod X_\alpha$ if both products are given the product topology.
\end{itemize}
\end{lemma}
\begin{proposition}
Let $\prod X_\alpha$ have the product topology. Let $f:A\to \prod X_\alpha$ be given by
\[ f(a)=(f_\alpha(a))_{\alpha\in J} \qquad \text{where $f_\alpha = \pi_\alpha\circ f:A\to X_\alpha$ for each $\alpha \in J$}.\]
Then $f$ is continuous \textup{if and only if} each function $f_\alpha$ is continuous.
\end{proposition}
This does not hold for the box topology.
TODO: Universal mapping property; generalise to initial topologies.
\begin{corollary} \label{productInclusionsContinuous}
Assume we have some points $c_i \in X_i$ for all $i\in J$. Then the functions
\[ i_\alpha: X_\alpha \to X: p \mapsto \left(\begin{cases}
c_i & (i\neq \alpha) \\ p & (i = \alpha)
\end{cases}\right)_{i\in J} \]
are continuous.
\end{corollary}
\begin{proof}
Consider $i_\alpha$. Then for $i = \alpha$, the function $\pi_i \circ i_\alpha: X_\alpha \to X_i$ is the identity in $X_\alpha$ and thus continuous, \ref{continuousConstructions}. For $i \neq \alpha$, the function $\pi_i \circ i_\alpha: X_\alpha \to X_i$ is a constant function $p \mapsto c_i$ and thus continuous, \ref{continuousConstructions}.
\end{proof}

\begin{lemma}
Given points $\vec{x}=(x_i)_{i\in \N}$ and $\vec{y}=(y_i)_{i\in \N}$ of $\R^\N$, define the metric
\[ D(\vec{x}, \vec{y}) = \sup\left\{\frac{\bar{d}(x_i,y_i)}{i}\;|\; i\in \N\right\}, \]
where $\bar{d}$ is the standard bounded metric on $\R$. Then $D$ induces the product topology on $\R^\N$.
\end{lemma}
\begin{proof}
Let $\mathcal{T}$ denote the product topology on $\R^\N$ and $\mathcal{T}_D$ the topology induced by $D$. We prove two inclusions using lemma \ref{basisCoarseness}.
\begin{itemize}[leftmargin=2cm]
\item[$\boxed{\mathcal{T}_D\subset\mathcal{T}}$] Choose arbitrary basis element $B_D(\vec{x},\epsilon)$. Then choose an $N\in\N$ such that $1/N<\epsilon$. Take the basis element
\[ V = ]x_1-\epsilon,x_1+\epsilon[\;\times\;]x_1-\epsilon,x_1+\epsilon[\;\times \ldots\times\; ]x_N-\epsilon, x_N+\epsilon[\;\times \R\times\R\times \ldots \]
for the product topology. We assert that $V\subset B_D(\vec{x},\epsilon)$. Indeed, for all $\vec{y}\in\R^\N$,
\[ \frac{\bar{d}(x_i,y_i)}{i} \leq \frac{1}{i} \leq \frac{1}{N} \qquad \text{if $i\geq N$}. \]
Therefore,
\[ D(\vec{x},\vec{y}) \leq \max\left\{ \frac{\bar{d}(x_1,y_1)}{1},\frac{\bar{d}(x_2,y_2)}{2},\ldots, \frac{\bar{d}(x_N,y_N)}{N}, \frac{1}{N} \right\}. \]
So if $\vec{y}\in V$, then $D(\vec{x},\vec{y})< \epsilon$ and $V\subset B_D(\vec{x},\epsilon)$.
\item[$\boxed{\mathcal{T}\subset\mathcal{T}_D}$] Choose an arbitrary basis element $U = \prod U_i$. Let $U_i=\R$ if $i\notin \{\alpha_1,\ldots, \alpha_n\}$. For each $i\in \{\alpha_1,\ldots, \alpha_n\}$ choose an interval $]x_i-\epsilon_i,x_i+\epsilon_i[\subset U_i$ and define
\[ \epsilon = \min\{\epsilon_i/i\;|\;i=\alpha_1,\ldots, \alpha_n\}. \]
We can easily see that $B_D(\vec{x},\epsilon) \subset U$.
\end{itemize}
\begin{corollary}
Countable products of metrisable spaces are metrisable.
\end{corollary}
\begin{corollary}
Countable products of Hausdorff spaces are Hausdorff.
\end{corollary}
\end{proof}
\begin{lemma}
The product $\R^J$, with $J$ an uncountable index set, is not metrisable.
\end{lemma}
\begin{proof}
In a metrisable space, by TODO ref, we have that if $x\in \bar{A}$, then there exists a sequence of points in $A$ converging to $x$. We construct a counterexample. Let $A$ be the subset of $\R^J$ containing all points $(x_i)_{i\in J}$ such that $x_i=1$ for all but finitely many $i$. Now the point $(0)_{i\in J}$ is in the closure of $A$, but has no sequence in $A$ converging to it. To see that it is in the closure, let $\prod U_\alpha$ be a basis element containing $(0)_{i\in J}$. The intersection $A\cap \prod U_\alpha$ is never empty. Indeed for only finitely many $\alpha$, $U_\alpha\neq \R$. Set $x_\alpha = 0$ for these $\alpha$ and $x_i = 1$ for the rest.
\end{proof}
\subsection{Box topology}
\begin{definition}
Let $X = \prod_{\alpha\in J}X_\alpha$ and take as a basis for a topology the collection of all sets of the form
\[ \prod_{\alpha\in J}U_\alpha \qquad \text{($U_\alpha$ open in $X_\alpha$)}. \]
The topology generated by this basis is the \udef{box topology}.
\end{definition}
The following properties hold, like in the product topology:
\begin{lemma}
Let the topology on $\prod X_\alpha$ be the box topology.
\begin{itemize}
\item If each space $X_\alpha$ is Hausdorff, then $\prod X_\alpha$ is Hausdorff.
\item Let $A_\alpha$ be subsets of $X_\alpha$, then
\[ \prod \bar{A}_\alpha = \overline{\prod A_\alpha}. \]
\item Let each $X_\alpha$ have a basis $\mathcal{B}_\alpha$. The collection of all the sets of the form 
\[ \prod_{\alpha\in J}B_\alpha \qquad B_\alpha\in\mathcal{B}_\alpha \]
serves as a basis for the box topology.
\item Let $A_\alpha$ be subspaces of $X_\alpha$, for each $\alpha\in J$. Then $\prod A_\alpha$ is a subspace of $\prod X_\alpha$ if both products are given the box topology.
\end{itemize}
\end{lemma}
\subsubsection{Failure of metrisability}
\begin{lemma}
$\R^\omega$ is not metrisable in the box topology.
\end{lemma}
\subsubsection{Failure of continuity}
\subsubsection{Failure of compactness}

\section{The quotient topology}
Let $X$ be a topological space. A quotient set can always be defined by a surjective function $f:X\to A$ to a set $A$. Then $A$ can be identified with a partition $X^*$ of $X$. Now we would like to define a topology on the partition. We can think of the quotient as shrinking each partition to a single point. Thus it is natural to call a subset of $A$ open if the union of the corresponding partitions is open:
\[ \text{$V$ is open in $A$}\quad \Leftrightarrow_{\text{def}}\quad \text{$p^{-1}(V)$ is open in $X$}. \]
This gives us the following definition:
\begin{definition}
Let $X,Y$ be topological spaces and $p:X\to Y$ a surjective map. The map $p$ is a \udef{quotient map} if
\[ \text{$V$ is open in $Y$} \iff \text{$p^{-1}(V)$ is open in $X$} \]
\end{definition}
This condition is stronger than continuity.
\begin{definition}
Let $X$ be a topological space.
\begin{itemize}
\item Let a be $A$ a subset, and $p:X\to A$ a surjective map. There exists exactly one topology on $A$ relative to which $p$ is a quotient map; it is called the \udef{quotient topology} induced by $p$.
\item Let $X^*$ be a partition of $X$ and $p:X\to X^*$ the surjective map that carries each point of $X$ to its partition. In the quotient topology induced by $p$, the space $X^*$ is called a \udef{quotient space} of $X$.
\end{itemize}
\end{definition}
We can also characterise the notion of quotient map in another way, starting from the following definition:
\begin{definition}
A subset $C$ of a topological space $X$ is \udef{saturated} with respect to a surjective map $p:X\to Y$ if $C$ is the complete inverse image of a subset of $Y$, i.e.\ it contains every set $p^{-1}(\{y\})$ that it intersects.
\end{definition}
\begin{lemma}
A surjective map $p$ is a quotient map \textup{if and only if} $p$ is continuous and maps saturated open sets of $X$ to open sets of $Y$.
\end{lemma}
\begin{corollary}
\begin{itemize}
\item Surjective continuous open maps are quotient maps.
\item Surjective continuous closed maps are quotient maps.
\end{itemize}
\end{corollary}
There are quotient maps that are neither open or closed.
\begin{proposition}
Let $p:X\to Y$ be a quotient map; let $A$ be a subset of $X$ that is saturated w.r.t. $p$; let $q:A\to p(A) = p|_{A}$.
\begin{enumerate}
\item If $A$ is either open or closed in $X$, then $q$ is a quotient map.
\item If $p$ is either an open map or a closed map, then $q$ is a quotient map.
\end{enumerate}
\end{proposition}
\begin{lemma}
Let $p,q$ be quotient maps.
\begin{enumerate}
\item A composite $q\circ p$ of quotient maps is a quotient map.
\item The product $p\times q$ is not necessarily a quotient map.
\item A quotient space of a Hausdorff space is not necessarily Hausdorff.
\end{enumerate}
\end{lemma}
\begin{proof}
Point (1) follows from
\[ p^{-1}(q^{-1}(U)) = (q\circ p)^{-1}(U). \]
\end{proof}
TODO: theorem 22.2 + corollary



\section{Density}
\begin{definition}
Let $(X,\mathcal{T})$ be a topological space and let $A$ be a subset of $X$. Then $A$ is called
\begin{enumerate}
\item \udef{dense} in $X$ if the closure of $A$ is the whole of $X$: $\overline{A} = X$;
\item \udef{rare} or \udef{nowhere dense} if its closure has empty interior: $(\overline{A})^\circ = \emptyset$;
\item \udef{meagre} (or a \udef{set of first category}) if it is a countable union of rare subsets of $X$;
\item \udef{nonmeagre} (or a \udef{set of second category}) if it is not meagre;
\item \udef{comeagre} if its complement $X\setminus A$ is meagre in $X$.
\end{enumerate}
\end{definition}
\begin{lemma} \label{densityEquivalences}
Let $A$ be a subset of a topological space $X$. $A$ being is dense in $X$ is equivalent to any of the following:
\begin{enumerate}
\item every element of $X$ either lies in $A$ or is a limit point of $A$;
\item $A^c$ has empty interior.
\end{enumerate}
\end{lemma}
\begin{proof}
For the first point: $\overline{A} = A\cup A' = X$.

For the second point:
\[ X = \overline{A} \iff X = ((A^c)^\circ)^c \iff (A^c)^\circ = X^c = \emptyset. \]
\end{proof}

\begin{lemma} \label{nowhereDensityEquivalence}
Let $X$ be a topological space and $A\subset X$ a subset. Then $A$ is nowhere dense \textup{if and only if} $\overline{A}^c$ is dense.
\end{lemma}
\begin{proof}
We calculate:
\[ (\overline{A})^\circ = \emptyset \iff \overline{(\overline{A}^c)}^c = \emptyset \iff \overline{(\overline{A}^c)} = X. \]
\end{proof}

\begin{lemma} \label{meagreSubset}
Any subset of a meagre set is meagre.
\end{lemma}
\begin{proof}
Let $A = \bigcup_k R_k$ be meagre and $B \subseteq A$. Then
\[ B = B\cap A = B\cap \left( \bigcup_k R_k \right) = \bigcup_k B\cap R_k. \]
Now for each $k$, $B\cap R_k \subset R_k$. So $\overline{B \cap R_k}^\circ \subseteq \overline{R_k}^\circ = \emptyset$, using lemma \ref{closureInteriorSubsets}, and thus $B\cap R_k$ is nowhere dense. 
\end{proof}

\begin{lemma} \label{denseSubsetOfDenseSubspaceIsDense}
Let $Y$ be a dense subspace of a topological space $X$. Let $S$ be a dense subset of $Y$. Then $S$ is dense in $X$.
\end{lemma}
\begin{proof}
Let $\overline{S}$ be the closure of $S$ in $X$. Then by \ref{subspaceClosure} we have $Y = Y\cap \overline{S}$, so $\overline{S} \supseteq Y \supseteq S$, which, taking the closure, implies $\overline{S} \supseteq \overline{Y} \supseteq \overline{S}$. Thus $\overline{S} = \overline{Y} = X$.
\end{proof}

\begin{definition}
A topological space $Y$ has the \udef{unique extension property} if for any topological space $X$, any continuous functions $f,g:X\to Y$ and any dense subset $E\subset X$ we have
\[ \forall x\in E: f(x)=g(x) \quad\implies\quad f = g. \]
\end{definition}

TODO:
\begin{proposition}
\begin{enumerate}
\item Assume $X$ Hausdorff and quotient map open, then $X/\sim$ is Hausdorff iff $\sim$ is closed in $X\times X$.
\item $X$ Hausdorff iff diagonal is closed
\item Let $f,g: A\to B$ be continuous functions. If $B$ is Hausdorff, then $\setbuilder{x\in A}{f(x) = g(x)}$ is closed. (Pre-image of diagonal set)
\end{enumerate}
\end{proposition}

\begin{proposition} \label{uniqueExtensionHausdorff}
A topological space $Y$ has the unique extension property \textup{if and only if} $Y$ is Hausdorff.
\end{proposition}
\begin{proof}
First assume $Y$ Hausdorff. Take functions $f,g: E\subset X \to Y$ that agree on $E$. They must agree on a closed set (TODO ref), thus at least on $\overline{E} = X$.

Now suppose $Y$ is not Hausdorff. TODO \url{https://www.jstor.org/stable/2315068?seq=1#metadata_info_tab_contents}
\end{proof}



\begin{lemma}
Let $X$ be a topological space and $E\subset X$ a .
\begin{enumerate}
\item If for any dense subspace $E\subset X$ the only continuous extension of $\id_E$ to $X$ is $\id_X$, then $X$ is $T_0$.
\item If $X$ is $T_2$, then for any dense subspace $E\subset X$ the only continuous extension of $\id_E$ to $X$ is $\id_X$. 
\end{enumerate}
$T_1$ is neither necessary nor sufficient.
\end{lemma}
\begin{proof}
TODO \url{https://math.stackexchange.com/questions/1592144/does-the-identity-map-on-a-dense-subset-of-a-space-extend-uniquely/1592169}
\end{proof}


\subsection{The Baire property}
\begin{definition}
A topological space $X$ has the \udef{Baire property} if it satisfies either of the following equivalent conditions:
\begin{enumerate}
\item every countable union of closed nowhere dense sets has empty interior;
\item every countable intersection of open dense sets is dense.
\end{enumerate}
These properties are equivalent because a subset has empty interior if and only if its complement is dense, see lemma \ref{densityEquivalences}.
\end{definition}


\begin{lemma} \label{BaireEquivalents}
A topological space $X$ is Baire \textup{if and only if} either of the following equivalent conditions:
\begin{enumerate}
\item every meagre subset of $X$ is either empty or not open;
\item every non-empty open subset of $X$ is a nonmeagre subset of $X$;
\item every comeagre subset of $X$ is dense in $X$.
\end{enumerate}
\end{lemma}
\begin{proof}
We prove the characterisation of spaces with the Baire property using countable unions implies the first point, the last point implies the countable intersection Baire condition.
\begin{itemize}[leftmargin=3cm]
\item[$\boxed{\text{Baire}\Rightarrow (1)}$] Every meagre set $A = \bigcup_k R_k$ (where all $R_k$ are nowhere dense) is a subset of $\bigcup_k \overline{R_k}$ where $\overline{R_k}$ are closed nowhere dense sets. Thus if the Baire property holds, $\bigcup_k \overline{R_k}$ has empty interior, meaning $A$ has empty interior. So either $A$ is empty or not open.
\item[$\boxed{(1) \Leftrightarrow (2)}$] By contraposition.
\item[$\boxed{(1) \Rightarrow (3)}$] Suppose $A$ is a meagre set. Then $A^\circ$ must also be meagre, by \ref{meagreSubset}. Now $A^\circ$ is certainly open, so by $(1)$ it must be empty. Thus $A^c$ is dense, by lemma \ref{densityEquivalences}.
\item[$\boxed{(3) \Rightarrow \text{Baire}}$] Let $A = \bigcap_k O_k$ where all $O_k$ are open dense sets. Then $A^c = \bigcup_k O_k^c$. Now for each $k$, $O_k^c$ is nowhere dense by lemma \ref{nowhereDensityEquivalence}, because $\overline{O_k^c}^c = O_k^\circ$ is still dense. Thus $A^c$ is meagre and $A$ is comeagre, so $A$ is dense in $X$.
\end{itemize}
\end{proof}

A topological space has the Baire property if and only if it has the property locally, in the following sense:
\begin{lemma}
A topological space $X$ has the Baire property \textup{if and only if} every point in $X$ has a neighbourhood with the Baire property.
\end{lemma}
\begin{proof}
If $X$ is Baire, the neighbourhood can simply be taken to be $X$.

Assume every point in $X$ has a neighbourhood with the Baire property.
We will prove point (2) in lemma \ref{BaireEquivalents} holds.
Take a non-empty open subset $A$ of $X$.
As $A$ is non-empty, we can take a point $x\in A$ and find a neighbourhood $U$ of $x$ with the Baire property.
Then $A\cap U$ is a non-empty open subset of $U$ and thus must not be meagre in $U$.
By contraposition of lemma \ref{meagreSubset}, we see that $A$ must be non-meagre in $X$, proving the Baireness of $X$.
\end{proof}

\begin{theorem}[Baire category theorem] \label{BaireCategory} \hspace{1em}
\begin{enumerate}
\item Every complete pseudometric space has the Baire property.
\item Every locally compact Hausdorff space has the Baire property.
\end{enumerate}
\end{theorem}
\begin{proof}
TODO + relocate
\end{proof}

\chapter{Convergence groups}
\section{Convergence}
\begin{definition}
Let $G$ be a set and
\begin{itemize}
\item $\boldsymbol{\cdot}: G\times G \to G$ a binary operation such that $\sSet{G, \boldsymbol{\cdot}}$ is a group;
\item $\xi$ a relation on $(\powerset(\powerset(G)), G)$ such that $\sSet{G, \xi}$ is a convergence space;
\end{itemize}
such that
\begin{itemize}
\item $\boldsymbol{\cdot}: G\times G \to G$ is continuous; and
\item $^{-1}: G\to G: x\mapsto x^{-1}$ is continuous.
\end{itemize}
Then we call $\xi$ a group convergence and $\sSet{G, \boldsymbol{\cdot}, \xi}$ a \udef{convergence group}.
\end{definition}

\begin{lemma} \label{convergenceGroupCriterion}
A group $G$ with a convergence structure is a convergence group \textup{if and only if}
\[ \Delta: G\times G \to G: (x,y) \mapsto xy^{-1} \]
is continuous.
\end{lemma}
\begin{proof}
If $G$ is a convergence group, the function $(x,y) \mapsto xy^{-1}$ is the composition of two continuous functions and thus continuous.

Conversely, assume $(x,y) \mapsto xy^{-1}$ continuous. Then $y \mapsto 1y^{-1} = y^{-1}$ is continuous by \ref{continuousEmbeddingProduct}. Then $\boldsymbol{\cdot}: (x,y) \mapsto xy = x(y^{-1})^{-1}$ is a composition of continuous functions.
\end{proof}

\begin{lemma} \label{groupConvergenceIffDeltasContinuous}
Let $\sSet{G,\cdot, 1}$ be a group and $\xi$ a convergence on $G$. Then the following are equivalent:
\begin{enumerate}
\item $\xi$ is a group convergence;
\item $\Delta_r: G\times G \to G: (x,y) \mapsto xy^{-1}$ is continuous;
\item $\Delta_l: G\times G \to G: (x,y) \mapsto x^{-1}y$ is continuous;
\end{enumerate}
\end{lemma}
\begin{proof}
Immediate from the constructions in \ref{leftToRightTranslationInvarianceLemma}.
\end{proof}

\begin{lemma} \label{closureGroupOperation}
Let $\sSet{G, \boldsymbol{\cdot}, \xi}$ be a convergence group and $A,B$ subsets of $G$. Then
\[ \adh_\xi(A)\cdot \adh_\xi(B) \subseteq \adh_\xi(A\cdot B). \]
\end{lemma}
\begin{proof}
From the continuity of $\cdot$ and \ref{adherenceInherenceContinuity} together with $\adh_{\xi\otimes\xi}(A\times B) = \adh_\xi(A)\times \adh_\xi(B)$ (\ref{productAdherence}).
\end{proof}

\begin{lemma} \label{shiftHomeomorphism}
Let $\sSet{G,\cdot, 1, \xi}$ be a convergence group.
\begin{enumerate}
\item For all $a\in G$, both
\[ \lambda_a: G\to G: x\mapsto ax \qquad\text{and}\qquad \rho_a: G\to G: x\mapsto xa \]
are homeomorphisms.
\item $F \to x$ \textup{if and only if} $F\cdot x^{-1} \to 1$ \textup{if and only if} $x^{-1}\cdot F\to 1$.
\item $\vicinity_\xi(x) = \vicinity_\xi(1) \cdot x = x\cdot \vicinity_\xi(1)$.
\item Let $f: \sSet{G,\cdot, 1, \xi} \to \sSet{H,\cdot, 1, \zeta}$ be a group homomorphism. Then $f$ is continuous \textup{if and only if} it is continuous at $1$.
\end{enumerate}
\end{lemma}
\begin{proof}
(1) The functions are clearly bijective. For continuity, take arbitrary $F\to x$. Then $\lambda_a^{\imf\imf}(F) = \cdot^{\imf\imf}(\pfilter{a}\otimes F)$ converges to $ax$ by continuity of the multiplication. The proof for $\rho_a$ is similar.

(2) Assume $F\to x$, then by continuity of $\rho_{x^{-1}}$, we get $F\cdot x^{-1} \to 1$. The converse follows from continuity of $\rho_x$. The second equivalence follows from the continuity of $\lambda_{x}$ and $\lambda_{x^{-1}}$.

(3) We calculate
\[ \vicinity_\xi(x) = \vicinity_\xi(\rho_x(1)) = \rho_x^{\imf\imf}\big(\vicinity_\xi(1)\big) = \vicinity_\xi(1) \cdot x, \]
using \ref{homeomorphismPreservation} and the fact $\rho_x$ is a homeomorphism. The proof of the other equality is similar, using $\lambda_x$.

(4) If $f$ is continuous, it is automatically continuous at $1$. Now assume $f$ continuous at $1$ and let $x\in G$. Then
\[ F\to x \iff F\cdot x^{-1} \to 1 \implies f[F\cdot x^{-1}] = f[F]\cdot f(x)^{-1} \to f(1) = 1 \iff f[F] \to f(x). \]
\end{proof}

The convergence structure of a convergence group is completely determined by $\lim^{-1}(1)$. Thus the following lemma gives a way to generate convergence groups.

\begin{proposition} \label{groupConvergenceConstruction}
Let $\sSet{G, +, 0}$ be a commutative group. And $\mathcal{G} \subseteq \powerfilters(G)$ a family of filters. There exists a convergence $\xi$ on $G$ such that $\mathcal{G} = \lim^{-1}_\xi(0)$ \textup{if and only if}
\begin{enumerate}
\item $\pfilter{0} \in \mathcal{G}$;
\item if $F \in \mathcal{G}$ and $G\supseteq F$, then $G\in \mathcal{G}$;
\item if $F,G\in \mathcal{G}$, then $F - G\in \mathcal{G}$.
\end{enumerate}
\end{proposition}
The group convergence is completely determined by $\lim^{-1}_\xi(0)$ due to the translation homeomorphism \ref{shiftHomeomorphism}.
\begin{proof}
It is clear that $F \to x$ iff $F - x \in \mathcal{G}$ determines a convergence. We just need to show that $u: (x,y) \mapsto x - y$ is continuous.

Let $F \to (x,y) \in G\times G$, so by \ref{convergenceProductFilter} there exist $F_1\to x$ and $F_2 \to y$ such that $F_1\otimes F_2 \leq F$. Then $F_1 - x \in \mathcal{G}$ and $F_2 - y \in \mathcal{G}$. From point (3) (and commutativity) we get $F_1 - F_2 - (x - y) \in \mathcal{G}$, so $F_1 - F_2 \to x-y$. Now $F_1 - F_2 = u[F_1\otimes F_2] \leq u[F]$ by \ref{filterFactorisationInequality}, so $u[F] \to x-y$ and thus $u$ is continuous.
\end{proof}

\begin{proposition} \label{HausdorffCriterionConvergenceGroup}
Let $\sSet{G,\cdot, 1, \xi}$ be a convergence group. Then $G$ is Hausdorff \textup{if and only if} $\{1\}$ is closed.
\end{proposition}
\begin{proof}
The direction $\Rightarrow$ is clear, since every Hausdorff convergence is $T_1$ and in a $T_1$ convergence all singletons are closed.

Conversely, assume $F\to x,y$ in $G$. Then $FF^{-1} \to xy^{-1}$. Now $FF^{-1} \subseteq \pfilter{1}$, so $\pfilter{1} \to xy^{-1}$ and thus $xy^{-1}\in \adh_\xi(\pfilter{1}) = \adh_\xi(\{1\}) = \{1\}$, meaning $x = y$.
\end{proof}
\begin{corollary} \label{kernelClosed}
Let $G$ be a convergence group, $H$ a Hausdorff convergence group and $f: G\to H$ be a continuous group homomorphism. Then $\ker(f)$ is closed.
\end{corollary}
\begin{proof}
Since $\ker(f) = f^{\preimf}\big(\{1_H\}\big)$ and $\{1_H\}$ is closed by the proposition, we have that $\ker(f)$ is closed by \ref{preimageOpenClosed}.
\end{proof}

\begin{lemma} \label{symmetricBase}
Let $\sSet{G,\cdot, 1, \xi}$ be a convergence group, then $\vicinity_\xi(1)$ is based in the symmetric subsets.
\end{lemma}
Symmetric subsets are subsets $U$ such that $U^{-1} = U$.
\begin{proof}
Because $^{-1}$ is a group homeomorphism, we have $\vicinity_\xi(1) = \vicinity_\xi(1^{-1}) = (\vicinity_\xi(1))^{-1}$. So $U$ is a vicinity of $1$ iff $U^{-1}$ is a vicinity of $1$. Thus $U\cap U^{-1}\subseteq U$ is a vicinity of $1$ which is a symmetric set.
\end{proof}

\begin{lemma} \label{vicinityFactorisation}
Let $\sSet{G,\cdot, 1, \xi}$ be a pretopological convergence group and $x,y\in G$. If $U\in \vicinity_\xi(xy)$, then there exist $V\in \vicinity_\xi(x)$ and $W\in \vicinity_\xi(y)$ such that $V\cdot W\subseteq U$.

If $x=y$, then we can take $V = W$.
\end{lemma}
\begin{proof}
Consider the function $f: G\times G \to G: (x,y)\mapsto xy$. Then by \ref{pretopologicalContinuityVicinities} and \ref{productVicinity} we have
\[ \vicinity_\xi(xy) \subseteq \upset f[\vicinity_{\xi\otimes\xi}((x,y))] = \upset f[\upset \vicinity_\xi(x)\otimes \vicinity_\xi(y)] = \upset f[\vicinity_\xi(x)\otimes \vicinity_\xi(y)] = \upset (\vicinity_\xi(x)\cdot \vicinity_\xi(y)) \]
This implies the first result.

Pick $V,W$ as before and consider $V\cap W$, which is a neighbourhood of $x=y$. Also $(V\cap W)\cdot(V\cap W) \subseteq V\cdot W \subseteq U$.
\end{proof}

\begin{proposition} \label{pretopologicalGroupConvergence}
Each pretopological convergence group is topological.
\end{proposition}
\begin{proof}
Let $\sSet{G,\cdot, 1, \xi}$ be a pretopological convergence group. To prove the convergence it topological, it is enough to prove that $\inh_\xi(A) \subseteq \inh^2_\xi(A)$ for all $A\subseteq X$. Fix such an $A$ and take an arbitrary $x\in \inh_\xi(A)$. Then $A\in \vicinity_\xi(x)$ by \ref{principalAdherenceInherence}.

By \ref{vicinityFactorisation} there exist $V\in \vicinity_\xi(x)$ and $W\in\vicinity_\xi(1)$ such that $V\cdot W \subseteq A$.

Now for all $y\in V$, $y\cdot W$ is a vicinity of $y$ by \ref{shiftHomeomorphism}, so $V \subseteq \inh_\xi(A)$ by \ref{subsetWithVicinitiesInInherence}. By the upward closure of the vicinity filter, $V\in \vicinity_\xi(x)$ implies $\inh_\xi(A) \in \vicinity_\xi(x)$. Thus $x\in \inh_\xi^2(A)$ by \ref{principalAdherenceInherence}.
\end{proof}

\begin{proposition} \label{topologicalGroupsRegular}
Every topological group is regular.
\end{proposition}
This follows from \ref{topologicalUniformSpaceRegular}, \ref{groupUniformStructure} and \ref{convergenceGroupTopologicalIffUniformlyTopological}.
In fact, we have that every topological group is completely regular by \ref{topologicallyUniformisableEquivalents}.

Below is an alternate, more elementary, proof.
\begin{proof}
By \ref{topologicalRegularity} we check that for all open $U$ and $x\in U$ there exists an open set $V$ such that $x\in V\subseteq \overline{V}\subseteq U$. In fact it is enough to check this for $e = 1$.\

Because $1\cdot 1 = 1$, we can find $W\in\neighbourhood(1)$ such that $W\cdot W \subseteq U$ by \ref{vicinityFactorisation}. We claim $V= W\cap W^{-1}$ works. Indeed it is an open neighbourhood of $1$ and clearly $V\subseteq U$. We just need to show that $\overline{V}\subseteq U$. Take $y\in \overline{V}$. Then $yV\in \vicinity_\xi(y)$ by \ref{shiftHomeomorphism} and $V\in \vicinity_\xi(y)^{\mesh}$ by \ref{principalAdherenceInherence}. So $yV\mesh V$ and we can find $v_1,v_2\in V$ such that $v_1 = yv_2$. Thus
\[ y = v_1v_2^{-1} \in V\cdot V^{-1} = V\cdot V \subseteq U. \]
\end{proof}
TODO: in fact completely regular.

\begin{proposition}
Let $G$ be a convergence group. Then the pseudotopological modification $\pseudotopMod(G)$ is a convergence
group.
\end{proposition}
(This is in general not true for the pretopological modification).
\begin{proof}
If the group operations $\cdot: G\times G\to G$ and $-^{-1}: G\to G$ are continuous, then $\cdot: \pseudotopMod(G\times G)\to \pseudotopMod(G)$ and $-^{-1}: \pseudotopMod(G)\to \pseudotopMod(G)$ are continuous by \ref{pseudotopModFunctorial}. We conclude by noting that $\pseudotopMod(G\times G) = \pseudotopMod(G)\times \pseudotopMod(G)$ by \ref{pseudotopologiserCommutesWithInitialStructure} and thus $\cdot: \pseudotopMod(G)\times \pseudotopMod(G)\to \pseudotopMod(G)$ is continuous.
\end{proof}

\begin{proposition}
Let $G$ be a convergence group. Any subgroup $H\in\vicinity(1)$ is closed.
\end{proposition}
\begin{proof}
Take $g\in \adh(H)$, so $H\in \vicinity(g)^{\mesh}$ by \ref{principalAdherenceInherence}. Now, by \ref{homeomorphismPreservation}, $gH\in \vicinity(g)$ because $\lambda_g$ is a homeomorphism by \ref{shiftHomeomorphism}. Thus $gH \mesh H = 1\cdot H$. As cosets are either the same or disjoint (by \ref{differentCosetsDisjoint}), we have $gH = H$ and in particular $g = g\cdot 1 \in gH = H$. So $\adh(H) = H$.
\end{proof}

\subsection{Initial and final convergence groups}
\begin{proposition} \label{initialConvergenceGroup}
Let $G$ be a group, $\{G_i\}_{i\in I}$ a set of convergence groups and $\{f_i: G \to G_i\}_{i\in I}$ a set of group homomorphisms. Then the initial convergence on $G$ w.r.t. $\{f_i: G \to G_i\}_{i\in I}$ makes $G$ a convergence group.
\end{proposition}
\begin{proof}
By \ref{convergenceGroupCriterion} we just need to verify that $\Delta: G\times G \to G: (x,y)\mapsto x\cdot y^{-1}$ is continuous. Using \ref{characteristicPropertyInitialFinalConvergence}, we need to verify that $f_i\circ \Delta$ is continuous for all $i\in I$. Because the $f_i$ are group homomorphisms, we have
\[ f_i(x\cdot y^{-1}) = f_i(x)\cdot f_i(y)^{-1} \]
for all $x, y \in G$. This means that $f_i\circ \Delta = \Delta_i \circ (f_i\circ \pi_1, f_i\circ \pi_2)$, where $\Delta_i: G_i\times G_i \to G_i: (x,y)\mapsto x\cdot y^{-1}$.

By \ref{continuityFunctionTuple} this is a composition of continuous functions and thus continuous.
\end{proof}
\begin{corollary}
Subgroups, products and projective limits of convergence groups are convergence groups.
\end{corollary}

\subsubsection{Quotient groups}
\begin{proposition} \label{quotientConvergenceGroup}
Let $\sSet{G, \xi}$ be a convergence group, $H$ a group and $q: \sSet{G, \xi} \to H$ a surjective group homomorphism. Then the quotient convergence on $H$ w.r.t. $q$ is a group convergence.
\end{proposition}
\begin{proof}
By \ref{convergenceGroupCriterion}, it is enough to show that $\Delta: H\times H \to H: (x,y) \mapsto xy^{-1}$ is continuous. Let $G_1 \to x$ and $G_2 \to y$ in $H$. By \ref{initialFinalConvergence} there exist $x'\in q^{\preimf}\{x\}$, $y'\in q^{\preimf}\{y\}$, $F_1 \overset{\xi}{\longrightarrow} x'$ and $F_2 \overset{\xi}{\longrightarrow} y'$ such that $q^{\imf\imf}[F_1] \subseteq G_1$ and $q^{\imf\imf}[F_2] \subseteq G_2$. Then
\[ G_1\cdot G_2^{-1} \supseteq q^{\imf\imf}[F_1]\cdot q^{\imf\imf}[F_2]^{-1} = q^{\imf\imf}[F_1\cdot F_2^{-1}] \to q(x'\cdot {y'}^{-1}) = x\cdot y^{-1}, \]
which shows that $\Delta$ is continuous.
\end{proof}
\begin{corollary}
Let $G$ be a convergence group and $N\lhd G$ a normal subgroup. Then $G/N$ is a convergence group.
\end{corollary}
\begin{proof}
The function $[\cdot]_N: G\to G/N$ is a group homomorphism by \ref{congruenceNormalSubgroup} and \ref{quotientAlgebra}. It is clearly surjective.
\end{proof}

\begin{proposition} \label{groupHomomorphismQuotient}
Let $\sSet{G, \xi}$ be a convergence group, $H$ a group and $q: \sSet{G, \xi} \to H$ a surjective group homomorphism. Equip $H$ with the quotient convergence w.r.t. $q$.
\begin{enumerate}
\item If $G$ is of finite depth, then $H$ is of finite depth.
\item If $G$ is topological, then $H$ is topological.
\item For all $x\in G$:
\[ \vicinity_H\big(q(x)\big) = q^{\imf\imf}\big(\vicinity_\xi(x)\big). \]
\item The map $q$ is open.
\end{enumerate}
\end{proposition}
\begin{proof}
(1) Let $F_1, F_2 \to x \in H$. By \ref{initialFinalConvergence} there exist $x_1, x_2\in q^{\preimf}\{x\}$, $F_1' \overset{\xi}{\longrightarrow} x_1$ and $F_2' \overset{\xi}{\longrightarrow} x_2$ such that $q^{\imf\imf}[F'_1] \subseteq F_1$ and $q^{\imf\imf}[F'_2] \subseteq F_2$. By \ref{shiftHomeomorphism}, $F_2'\cdot x_2^{-1}\cdot x_1 \overset{\xi}{\longrightarrow} x_1$. By finite depth of $G$, $F_1'\cap (F_2'\cdot x_2^{-1}\cdot x_1) \overset{\xi}{\longrightarrow} x_1$. Now, using \ref{imageUpsetsPreservesIntersection} and the fact that $q(x_2^{-1}\cdot x_1) = q(x_2)^{-1}\cdot q(x_1) = x^{-1}\cdot x = 1$,
\[ q^{\imf\imf}\big(F_1'\cap (F_2'\cdot x_2^{-1}\cdot x_1)\big) = q^{\imf\imf}(F_1')\cap q^{\imf\imf}(F_2'\cdot x_2^{-1}\cdot x_1) = q^{\imf\imf}(F_1')\cap q^{\imf\imf}(F_2') \to x, \]
so $F_1\cap F_2 \to x$.

(2) The convergence group $H$ is pretopological by \ref{pretopologicalFinalConvergence} and thus topological by \ref{pretopologicalGroupConvergence}.

(3) We have, by \ref{pretopologicalFinalConvergence},
\[ \vicinity_H\big(q(x)\big) = \bigcap_{y\in q^{\preimf}\{q(x)\}}q^{\imf\imf}\big(\vicinity_\xi(y)\big). \]
It is then enough to show that $q^{\imf\imf}\big(\vicinity_\xi(y)\big) = q^{\imf\imf}\big(\vicinity_\xi(x)\big)$, for all $y\in q^{\preimf}\{q(x)\}$. We have $q(x^{-1}y) = 1$, so we can use \ref{shiftHomeomorphism} to calculate
\[ q^{\imf\imf}\big(\vicinity_\xi(y)\big) = q^{\imf\imf}\big(\vicinity_\xi(x)\cdot x^{-1}y\big) = q^{\imf\imf}\big(\vicinity_\xi(x)\big)\cdot q(x^{-1}y) = q^{\imf\imf}\big(\vicinity_\xi(x)\big) \]

(4) Let $O\subseteq G$ be open. To show $q^\imf(O)$ is open, we use point (6) of \ref{openClosedCriteria}. Take $y\in q^{\imf}(O)$, so there exists $x\in O$ such that $q(x) = y$.

Because $O$ is open, we can use point (6) of \ref{openClosedCriteria} to find a $U_x\in \vicinity_\xi(x)$ such that $U_x \subseteq O$. Then $q^{\imf}(U_x) \subseteq q^\imf(O)$ and $q^{\imf}(U_x)\in \vicinity\big(q(x)\big)$ by point (3). We conclude that $q^{\imf}(O)$ is open.
\end{proof}

\begin{lemma}
Let $\sSet{G, \xi}$ and $\sSet{H, \zeta}$ be convergence groups. Let $f: A\to B$ be a continuous group homomorphism and $N\lhd G$ a normal subgroup such that $N\subseteq \ker f$. Then there exists a unique continuous homomorphism $f': G/N \to H$ such that
\[ \begin{tikzcd}
G \arrow[r, "{[\cdot]_N}"] \arrow[dr, swap, "f"] & G/N \arrow[d, dashed, "{f'}"] \\
& H
\end{tikzcd} \qquad\text{commutes.} \]
Further, $f'$ is injective \textup{if and only if} $N = \ker f$.
\end{lemma}
\begin{proof}
The group homomorphism $f'$ is the one from \ref{factorTheorem}. It is continuous by \ref{characteristicPropertyInitialFinalConvergence}.
\end{proof}

\begin{proposition} \label{quotientConvergenceGroupProperties}
Let $\sSet{G, \xi}$ be a convergence group and $N \lhd G$ a normal subgroup. Then 
\begin{enumerate}
\item $G/N$ is Hausdorff \textup{if and only if} $N$ is closed;
\item $G/N$ is discrete \textup{if and only if} $N$ is open.
\end{enumerate}
\end{proposition}
\begin{proof}
(1) Assume $G/N$ Hausdorff. Then $N = \ker\big([\cdot]_N\big)$ is closed by \ref{kernelClosed}.

Now assume $N$ closed. Then $[\cdot]^{\imf}_N(G\setminus N)$ is open because $[\cdot]_N$ is open by \ref{groupHomomorphismQuotient}. As $[\cdot]^{\imf}_N(G\setminus N) = \{[1]_N\}^c$, we have that $[1]_N$ is closed.

(2) We have that $N$ is open iff $\{[1]_N\}$ is open by \ref{groupHomomorphismQuotient}. This is equivalent to all singletons being open, by \ref{shiftHomeomorphism}, which is equivalent to the $G/N$ being discrete by \ref{discreteTopologyCharacterisation}.
\end{proof}

\subsection{Continuous convergence}
\begin{proposition} \label{continuousConvergenceGroup}
Let $\sSet{X,\xi}$ be a convergence space and $\sSet{G, \cdot, 1, \zeta}$ a convergence group. Then
\begin{enumerate}
\item $(X\to G)_c$ is a preconvergence group;
\item $\cont_c(X,G)$ is a convergence group.
\end{enumerate}
\end{proposition}
\begin{proof}
(1) We first show that pointwise multiplication, $\cdot_\text{pt}: (X\to G)_c^2 \to (X\to G)_c$ is continuous. By \ref{universalPropertyContinuousConvergence}, this is equivalent to the continuity of $\curry_1^{-1}(\cdot_\text{pt}): (X\to G)_c^2\times X \to G$, which follows from the commutativity of the following diagram:
\[ \begin{tikzcd}[column sep=large]
(X\to G)_c^2\times X \to G \arrow[r, "{\curry_1^{-1}(\cdot_\text{pt})}"] \arrow[d, "h"] & G \\
(X\to G)_c\times X \times (X\to G)_c \times X \arrow[r, "(\evalMap|\evalMap)"] & G\times G \arrow[u, "{\boldsymbol{\cdot}}"]
\end{tikzcd} \]
where the map $h: (f,g,x) \mapsto (f,x,g,x)$ is continuous by the continuity of the associator, swap and diagonal maps (\ref{swapMorphism}, \ref{associatorTernaryProducts}) and $\evalMap$ is continuous by \ref{evalMapContinuous}.

Now we show that the pointwise inversion, $(-)^{-1}_\text{pt}: (X\to G)_c \to (X\to G)_c$ is continuous. By \ref{universalPropertyContinuousConvergence}, this is equivalent to the continuity of $\curry_1^{-1}((-)^{-1}_\text{pt}): (X\to G)_c\times X \to G$, which follows from the commutativity of the following diagram:
\[ \begin{tikzcd}
(X\to G)_c\times X \to G \arrow[rr, "{\curry_1^{-1}(\cdot_\text{pt})}"] \arrow[dr, swap, "\evalMap"] & {} & G \\
{} & G \arrow[ur, swap, "{(-)^{-1}}"] & {}
\end{tikzcd}. \]

(2) We just need to show that the pointwise multiplication of two continuous functions is continuous. This follows immediately from the continuity of the pointwise multiplication, because $f\cdot_\text{pt} g = \cdot_\text{pt}\circ (f,g)$.
\end{proof}

\section{Uniform structure}
This section makes use of the function $\Delta = \Delta_r$ from section \ref{sec:translationInvariance}. \label{sec:groupUniformStructure}

TODO: redo with $\Delta_l$??

If $G$ does not commute, then we can define two different uniform structures.

\begin{definition}
Let $\sSet{G, \cdot, 1, \xi}$ be a convergence group. Consider the function $\Delta_r: G\times G\to G: (x,y) \mapsto x\cdot y^{-1}$. Then the \udef{uniform structure} of the group is given by
\[ \mathcal{U}_G \defeq \setbuilder{H\in\powerfilters(G^2)}{\Delta^{\imf\imf}(H) \to 1}. \]
\end{definition}

\begin{lemma} \label{groupUniformityCompositionLemma}
Let $\sSet{G, \cdot, 1, \xi}$ be a convergence group, $A,B\in \powerset(G^2)$ and $H,H'\in \powerfilters(G^2)$. Then
\begin{enumerate}
\item $\Delta^{\imf}(A;B) \subseteq \Delta^\imf(A)\cdot^\imf\Delta^\imf(B)$;
\item $\Delta^{\imf\imf}(H;H') \supseteq \Delta^{\imf\imf}(H) \cdot^{\imf\imf} \Delta^{\imf\imf}(H')$.
\end{enumerate}
\end{lemma}
\begin{proof}
(1) Any element of $\Delta^{\imf}(A;B)$ can be written as $ab^{-1}$, where there exists $x\in G$ such that $(a,x)\in A$ and $(x,b)\in B$. Then $ax^{-1}\in \Delta^\imf(A)$ and $xb^{-1}\in \Delta^\imf(B)$, so $ax^{-1} = ax^{-1}xb^{-1} \in \Delta^\imf(A)\cdot^\imf\Delta^\imf(B)$.

(2) For any set in $\Delta^{\imf\imf}(H) \cdot^{\imf\imf} \Delta^{\imf\imf}(H')$, we can find a subset in $\Delta^{\imf\imf}(H) \cdot^{\imf\imf} \Delta^{\imf\imf}(H')$ of the form $\Delta^\imf(A)\cdot^\imf\Delta^\imf(B)$, where $A\in H$ and $B\in H'$.

Then $\Delta^{\imf}(A;B) \subseteq \Delta^\imf(A)\cdot^\imf\Delta^\imf(B)$ and $\Delta^{\imf}(A;B) \in \Delta^{\imf\imf}(H;H')$, so the original set is in $\Delta^{\imf\imf}(H;H')$ by upwards closure.
\end{proof}

\begin{proposition} \label{groupUniformStructure}
Let $\sSet{G, \cdot, 1, \xi}$ be a convergence group. The associated uniform structure $\mathcal{U}_G$ is a uniform structure and $\xi = (\Gamma\circ\Xi)(\mathcal{U}_G)$.
\end{proposition}
\begin{proof}
We verify the definition.
\begin{enumerate}
\item We have $\Delta^{\imf\imf}(\pfilter{x}\otimes\pfilter{x}) = \pfilter{\Delta}(x,x) = \pfilter{1} \to 1$, by \ref{productPrincipalUltrafilter}.
\item Upwards closure is given by the definition of convergence.
\item For all $H\in \powerfilters(G^2)$, we have $\Delta^{\imf\imf}(H^\transp) = \Delta^{\imf\imf}(H)^{-1}$. By continuity of the inverse, $\Delta^{\imf\imf}(H^\transp)\to 1$ iff $\Delta^{\imf\imf}(H)\to 1$.
\item Assume $H,H'\in \mathcal{U}_G$, i.e.\ $H,H'\in \powerfilters(G^2)$ such that $\Delta^{\imf\imf}(H) \to 1$ and $\Delta^{\imf\imf}(H') \to 1$. Then $\Delta^{\imf\imf}(H) \cdot^{\imf\imf} \Delta^{\imf\imf}(H')\to 1$ by continuity of the group operation and thus $\Delta^{\imf\imf}(H;H') \to 1$ by \ref{groupUniformityCompositionLemma} and monotonicity of the convergence. This means that $H;H'\in \mathcal{U}_G$.
\end{enumerate}
Finally, for all $F\in \powerfilters(G)$ and $x\in G$, we have $F\cdot x^{-1} = F\cdot^{\imf\imf}\{x^{-1}\} = \Delta^{\imf\imf}(F\otimes \pfilter{x})$, so
\begin{multline*} F\overset{\xi}{\longrightarrow} x \iff F\cdot x^{-1} \overset{\xi}{\longrightarrow} 1 \iff \Delta^{\imf\imf}(F\otimes \pfilter{x}) \overset{\xi}{\longrightarrow} 1 \\ \iff F\otimes \pfilter{x}\in \mathcal{U}_G \iff F\mathrel{\Xi(\mathcal{U}_G)}\pfilter{x} \iff F\overset{(\Gamma\circ\Xi)(\mathcal{U}_G)}{\longrightarrow} x, \end{multline*}
where we have used \ref{shiftHomeomorphism}. We conclude that $\xi = (\Gamma\circ\Xi)(\mathcal{U}_G)$.
\end{proof}


\begin{proposition} \label{entourageConvergenceGroup}
Let $\sSet{G,\cdot, 1, \xi}$ be a convergence group. Then
\begin{enumerate}
\item $\Delta^{\preimf\imf}\big(\vicinity_\xi(1)\big) \subseteq \entourage_{G}$;
\item if $\xi$ is topological, then the uniform structure $\mathcal{U}_G$ is topological and $\entourage_{G} = \Delta^{\preimf\imf}\big(\neighbourhood_\xi(1)\big)$.
\end{enumerate}
\end{proposition}
\begin{proof}
(1) We calculate, using \ref{upsetPreimageImageGaloisConnection}:
\begin{align*}
\entourage_G &= \bigcap \mathcal{U}_G \\
&= \bigcap \setbuilder{H\in \powerfilters(G^2)}{\Delta^{\imf\imf}(H) \to 1} \\
&\supseteq \bigcap \setbuilder{H\in \powerfilters(G^2)}{\vicinity(1) \subseteq \Delta^{\imf\imf}(H)} \\
&= \bigcap \setbuilder{H\in \powerfilters(G^2)}{\Delta^{\preimf\imf}\big(\vicinity(1)\big) \subseteq H} \\
&= \Delta^{\preimf\imf}\big(\neighbourhood(1)\big).
\end{align*}

(2) We calculate, using the fact that $\xi$ is topological and \ref{upsetPreimageImageGaloisConnection}:
\begin{align*}
\mathcal{U}_G &= \setbuilder{H\in \powerfilters(G^2)}{\Delta^{\imf\imf}(H) \to 1} \\
&= \setbuilder{H\in \powerfilters(G^2)}{\neighbourhood(1) \subseteq \Delta^{\imf\imf}(H)} \\
&= \setbuilder{H\in \powerfilters(G^2)}{\Delta^{\preimf\imf}\big(\neighbourhood(1)\big) \subseteq H} \\
&= \upset\Big\{\Delta^{\preimf\imf}\big(\neighbourhood(1)\big)\Big\}.
\end{align*}
\end{proof}
\begin{corollary} \label{convergenceGroupTopologicalIffUniformlyTopological}
Let $\sSet{G, \cdot, 1, \xi}$ be a convergence group. Then $\uniformity_G$ is topological \textup{if and only if} $\xi$ is topological.
\end{corollary}
\begin{proof}
One direction is given by the proposition, the other by \ref{groupUniformStructure} and \ref{vicinityUniformConvergence}.
\end{proof}
TODO: can (1) be improved? Cfr. \ref{vicinityUniformConvergence}?

\begin{lemma} \label{deltaPreimageLemma}
Let $G$ be a group, $A\subseteq G$ and $x\in G$. Then $\Delta^{\preimf}(A)x = A\cdot x$.
\end{lemma}
\begin{proof}
Take $y\in G$. Then
\[ y\in \Delta^{\preimf}(A)x \iff (y,x)\in \Delta^{\preimf}(A) \iff yx^{-1}\in A \iff y\in A\cdot x. \]
\end{proof}

\begin{proposition} \label{uniformContinuityGroupHomomorphism}
Let $\sSet{G, \cdot, 1, \xi}$, $\sSet{H, \cdot, 1, \zeta}$ be convergence groups, $f: G\to H$ a group homomorphism and $g\in G$. Then the following are equivalent:
\begin{enumerate}
\item $f$ is continuous;
\item $f$ is continuous at $1$;
\item $f$ is uniformly continuous.
\end{enumerate}
\end{proposition}
\begin{proof}
$(1)\Rightarrow (2)$ Immediate.

$(2)\Rightarrow (3)$ Take $F\in \uniformity_G$. Then $\Delta^{\imf\imf}(F)\overset{\xi}{\longrightarrow} 1$. Beacuse $f$ is a homomorphism, we have $\Delta^{\imf\imf}\big((f,f)^{\imf\imf}(F)\big) = (f,f)^{\imf\imf}\big(\Delta^{\imf\imf}(F)\big)$ and by continuity at $1$, we have that this converges to $1$ in $\zeta$. Thus $(f,f)^{\imf\imf}(F)\in \uniformity_H$.

$(3)\Rightarrow (1)$ Follows from \ref{preservationUniformStructure}.
\end{proof}

\begin{proposition} \label{equicontinuityGroupHomomorphisms}
Let $G,H$ be convergence groups, $K\subseteq \Hom(G,H)$ a set of group homomorphisms and $x\in G$. Then the following are equivalent:
\begin{enumerate}
\item $K$ is equicontinuous;
\item $K$ is equicontinuous at $x$;
\item we have $\upset\evalMap^{\imf\imf}(\{K\}\otimes F) \overset{H}{\longrightarrow} 1$ for all convergent filters $F\overset{G}{\longrightarrow} 1$.
\end{enumerate}
\end{proposition}
\begin{proof}
By \ref{equicontinuityUnionLemma}, the equicontinuity of $K$ at $x$ is equivalent to $\upset k^{\imf\imf}(\{K\}\otimes F') \in \uniformity_H$ for all filters $F'\overset{G}{\longrightarrow} x\in G$, i.e. $\upset (\Delta\circ k)^{\imf\imf}(\{K\}\otimes F')\overset{H}{\longrightarrow} 1$.

Now for all group homomorphism $h: G\to H$ and $y\in G$, we have
\[ (\Delta\circ k)(h,y) = \Delta\big(h(y), h(x)\big) = h(y)\cdot h(x)^{-1} = h(yx^{-1}) = \evalMap(h, yx^{-1}). \]
$(1) \Rightarrow (2)$ Immediate.

$(2) \Rightarrow (3)$ Take an arbitrary $F\overset{G}{\longrightarrow} 1$. We have $F\cdot x\overset{G}{\longrightarrow} x$ by \ref{shiftHomeomorphism} and so, by the calculation above,
\[ \upset\evalMap^{\imf\imf}(\{K\}\otimes F) = \upset\evalMap^{\imf\imf}(\{K\}\otimes F\cdot x\cdot x^{-1}) = \upset (\Delta\circ k)^{\imf\imf}\big(\{K\}\otimes F\cdot x \big) \overset{H}{\longrightarrow} 1. \]

$(3) \Rightarrow (2)$ Now take arbitrary $x'\in G$ and arbitrary $F' \overset{G}{\longrightarrow} x'$. Then
\[ \upset (\Delta\circ k)^{\imf\imf}(\{K\}\otimes F') = \upset\evalMap^{\imf\imf}(\{K\}\otimes F\cdot x^{\prime -1}) \overset{H}{\longrightarrow} 1, \]
because $F'\cdot x^{\prime -1} \to 1$.
\end{proof}
TODO compare with boundedness.

\begin{lemma}
Let $G,H,K$ be convergence groups, $A\subseteq \Hom(G,H)$ and $B\subseteq \Hom(H,K)$. If $A$ and $B$ are equicontinuous, then $B\circ^\imf A$ is equicontinuous.
\end{lemma}
TODO true for general uniform spaces?
\begin{proof}
For all convergent filters $F\overset{G}{\longrightarrow} 1$, we have, by \ref{equicontinuityGroupHomomorphisms},
\[ \upset\evalMap^{\imf\imf}(\{B\circ^\imf A\}\otimes F) = \upset\evalMap^{\imf\imf}\Big(\{B\}\otimes \evalMap^{\imf\imf}\big(\{A\}\otimes F\big)\Big) \overset{K}{\longrightarrow} 1, \]
because $B$ is equicontinuous and $\evalMap^{\imf\imf}(\{A\}\otimes F)\overset{H}{\longrightarrow} 1$.
\end{proof}


\subsection{Cauchy structure}
\begin{lemma}
Let $\sSet{G, \cdot, 1, \xi}$ be a convergence group. The Cauchy structure associated to the uniform structure $\mathcal{U}_G$ is given by
\[ \mathcal{F} \defeq \setbuilder{F\in \powerfilters(G)}{F\cdot F^{-1} \overset{\xi}{\longrightarrow} 1}. \]
\end{lemma}

\begin{lemma} \label{vicinitiesOfUnitCauchyFilters}
Let $\sSet{G, \cdot, 1, \xi}$ be a convergence group and $F\in\powerfilters(G)$ a Cauchy filter. Then for all $U\in \vicinity_\xi(1)$, there exists $x\in G$ such that $U\cdot x\in F$.
\end{lemma}
\begin{proof}
We have $F\cdot F^{-1} \to 1$, so $U\in F\cdot F^{-1}$.
This means that there exist $A, B \in F$ such that $A\cdot B^{-1} \subseteq U$. Take $x \in B$, then $A\cdot x^{-1} \subseteq A\cdot B^{-1} \subseteq U$ and so $A \subseteq U\cdot x$. 
By upwards closure of $F$, this means that $U\cdot x\in F$.
\end{proof}

\subsubsection{Precompact subsets}

\begin{proposition}
Let $\sSet{G, \cdot, 1, \xi}$ be a convergence group and $A\subseteq G$. Then $A$ is totally bounded \textup{if and only if} for all $U\in \vicinity(1)$ there exists a finite set $S\subseteq G$ such that $A \subseteq \bigcup_{s\in S}s\cdot U$.
\end{proposition}
\begin{proof}
TODO
\end{proof}

\section{Compactness in convergence groups}
\begin{proposition}
Let $G$ be a convergence group and $A,B\subseteq G$ subsets.
\begin{enumerate} \label{compactSubsetsConvergenceGroups}
\item If $A$, $B$ are compact, then $A\cdot B$ is compact.
\item If $A$ is compact, then $A^{-1}$ is compact.
\item If $A$ is open, then $A\cdot B$ is open.
\item If $A$ is closed and $B$ is compact, then $A\cdot B$ is closed.
\end{enumerate}
\end{proposition}
\begin{proof}
(1) The set $A\times B$ is compact in the product convergence by Tychonoff's theorem \ref{TychonoffsTheorem}. A continuous image of compact a compact set is compact by \ref{compactConstructions}.

(2) A continuous image of compact a compact set is compact by \ref{compactConstructions}.

(3) The set $A\cdot B = \bigcup_{b\in B} A\cdot b$ is a union of open sets (by \ref{shiftHomeomorphism}) and thus open by \ref{propertiesTopology}.

(4) We need to show that $\adh(A\cdot B) \subseteq A\cdot B$. To that end, take $x\in \adh(A\cdot B)$. By \ref{differentUnionsForAdherence} there exist an ultrafilter $U\in \powerultrafilters(G)$ that converges to $x$ and contains $A\cdot B$, i.e.\ $\upset\cdot^{\imf\imf}\{A\times B\} \subseteq U$. By \ref{mappingUltrafiltersLemma}, there exists $U'\in \powerultrafilters(G)$ such that $\upset\cdot^{\imf\imf}(U') = U$ and $A\times B \in U'$. By \ref{imageFilterProperties}, $\upset \pi_2^{\imf\imf}(U')$ is an ultrafilter and it contains $B$ by construction.
By compactness, $\upset \pi_2^{\imf\imf}(U')$ converges to some $b\in B$.

Finally consider $U\cdot \pi_2^{\imf\imf}(U')^{-1}$, which converges to $xb^{-1}$. By \ref{filterPairingLemma}, we have
\[ U\cdot \pi_2^{\imf\imf}(U')^{-1} = \upset\cdot^{\imf\imf}(U')\cdot \pi_2^{\imf\imf}(U')^{-1} = \upset \big((x,y)\mapsto x\cdot y \cdot y^{-1}\big)^{\imf\imf}(U') = \upset \pi_1^{\imf\imf}(U') \ni A. \]
By \ref{principalAdherenceInherence}, this implies that $xb^{-1}\in \adh(A) = A$, so $x\in A\cdot B$.
\end{proof}


\subsection{Locally compact groups}
Locally compact means that the set of compact subsets is a convergence cover.

\begin{lemma}
A topological convergence group is locally compact \textup{if and only if} $\vicinity(1)$ has a base of compact sets.
\end{lemma}
\begin{proof}
TODO 

It is regular, so is based in closed sets. There is a compact vicinity of $1$. All closed subsets of this vicinity are compact.
\end{proof}

\begin{proposition}
Let $\sSet{G, \cdot, 1, \xi}$ be a locally compact convergence group. Then $G$ is complete.
\end{proposition}
Compare with \ref{compactImpliesComplete}.
\begin{proof}
Let $F$ be a proper Cauchy filter. Then $F\cdot F^{-1} \to 1$, so there exists a compact set $K\in F\cdot F^{-1}$. This means that there exist $A, B \in F$ such that $A\cdot B^{-1} \subseteq K$. Take $x_0 \in B$, then $A\cdot x_0 \subseteq A\cdot B^{-1} \subseteq K$ and so $A \subseteq K\cdot x_0^{-1}$. Now $K\cdot x_0^{-1}\in F$ and $K\cdot x_0^{-1}$ is compact by \ref{shiftHomeomorphism}. Now by the ultrafilter lemma \ref{ultrafilterLemma} we can take an ultrafilter $H \supseteq F$. Then $K\cdot x_0^{-1} \in H$, so $H$ converges to some $y$.

So we have $G\mathrel{\Xi(\mathcal{U}_G)} \pfilter{y}$, as well as $F\mathrel{\Xi(\mathcal{U}_G)} F$ by assumption. Then $F\mathrel{\Xi(\mathcal{U}_G)} \pfilter{y}$ by \ref{uniformRelationUpwardsClosure}. Thus we have $F \overset{(\Gamma\circ \Xi)(\mathcal{U}_G)}{\longrightarrow} y$ and so $F\overset{\xi}{\longrightarrow} y$ by \ref{groupUniformStructure}.
\end{proof}

\begin{lemma} \label{compactSubsetsQuotientGroupPretopological}
Let $\sSet{G, \cdot, 1, \xi}$ be a locally compact convergence group and $N$ a closed normal subgroup. If each compact subset of $G$ is pretopological, then each compact subset of $G/N$ is pretopological.
\end{lemma}
\begin{proof}
Let $K\subseteq G/N$ be a compact set. By \ref{preimageCompactnessQuotientConvergence} there exists a compact set $L\subseteq G$ such that $K\subseteq [L]_N^\imf$. As any subspace of a pretopological space is pretopological (\ref{pretopologicalInitialConvergence}), it is enough to show that $[L]_N^\imf$ is pretopological.

Take arbitrary $x\in L$. We need to show that $\vicinity_{[L]_N^\imf}\big([x]_N\big)$ converges. We claim that
\[ \vicinity_{[L]_N^\imf}\big([x]_N\big) = \upset (\iota^{\preimf}_{[L]_N^\imf} \circ [\cdot]_N^\imf\circ \iota_{L\cdot L^{-1}\cdot L}^\imf)^{\imf}\Big(\vicinity_{L\cdot L^{-1}\cdot L}(x)\Big), \]
which converges because $L\cdot L^{-1}\cdot L$ is compact by \ref{compactSubsetsConvergenceGroups} and thus pretopological by assumption, so $\vicinity_{L\cdot L^{-1}\cdot L}(x)$ converges, $\upset ([\cdot]_N^\imf\circ \iota_{L\cdot L^{-1}\cdot L}^\imf)^{\imf}\Big(\vicinity_{L\cdot L^{-1}\cdot L}(x)\Big)$ converges by continuity of $[\cdot]_N$ and $\iota_{L\cdot L^{-1}\cdot L}$ and $\upset (\iota^{\preimf}_{[L]_N^\imf} \circ [\cdot]_N^\imf\circ \iota_{L\cdot L^{-1}\cdot L}^\imf)^{\imf}\Big(\vicinity_{L\cdot L^{-1}\cdot L}(x)\Big)$ converges by \ref{subspaceConvergence}.

We now just need to prove the claim. Since we know the right-hand side converges, we clearly have the inclusion $\subseteq$. For the other inclusion, we show that any ultrafilter in $[L]_N^\imf$ that converges to $[x]_N$ includes the right-hand side. Let $F\overset{[L]_N^\imf}{\longrightarrow} [x]_N$ be a convergent ultrafilter. 
Then, by \ref{imageFilterProperties},  $\upset\iota_{[L]_N^\imf}^{\imf\imf}(F)$ is an ultrafilter that contains $[L]_N^\imf$.
By \ref{mappingUltrafiltersLemma}, there exists an ultrafilter $F'\in \powerultrafilters(G)$ such that $L\in F'$ and $\upset[\cdot]_N^{\imf\imf}(F') = \upset\iota_{[L]_N^\imf}^{\imf\imf}(F)$.

Since $L$ is compact, $F'$ converges to a point $y\in L$. Since $G/N$ is Hausdorff by \ref{quotientConvergenceGroupProperties}, we have $[x]_N = [y]_N$. Now $F'\cdot y^{-1}\cdot x$ converges to $x$ and contains $L\cdot L^{-1}\cdot L$, so $\upset\iota_{L\cdot L^{-1}\cdot L}^{\preimf\imf}\big(F'\cdot y^{-1}\cdot x\big) \overset{L\cdot L^{-1}\cdot L}{\longrightarrow} x$ by \ref{subspaceConvergence}.

Thus $\vicinity_{L\cdot L^{-1}\cdot L}(x) \subseteq \upset\iota_{L\cdot L^{-1}\cdot L}^{\preimf\imf}\big(F'\cdot y^{-1}\cdot x\big)$ and so, by \ref{setTraceFilterLemma},
\begin{align*}
\upset \iota_{L\cdot L^{-1}\cdot L}^{\imf\imf}\big(\vicinity_{L\cdot L^{-1}\cdot L}(x)\big) &\subseteq \upset(\iota_{L\cdot L^{-1}\cdot L}^{\imf}\circ \iota_{L\cdot L^{-1}\cdot L}^{\preimf})^\imf\big(F'\cdot y^{-1}\cdot x\big) \\
&= F'\cdot y^{-1}\cdot x|_{L\cdot L^{-1}\cdot L} = F'\cdot y^{-1}\cdot x.
\end{align*}
We apply $[\cdot]_N$ to obtain
\begin{align*}
([\cdot]_N\circ \iota_{L\cdot L^{-1}\cdot L})^{\imf\imf}\Big(\vicinity_{L\cdot L^{-1}\cdot L}(x)\Big) &\subseteq \upset[\cdot]_N^{\imf\imf}(F'\cdot y^{-1}\cdot x) \\
&= \upset[\cdot]_N^{\imf\imf}(F)\cdot [y]_N^{-1} \cdot [x]_N \\
&= \upset\iota^{\imf\imf}_{[L]_N^{\imf}}(F).
\end{align*}
By \ref{upsetPreimageImageGaloisConnection}, this implies
\[ \upset (\iota^{\preimf}_{[L]_N^\imf} \circ [\cdot]_N^\imf\circ \iota_{L\cdot L^{-1}\cdot L}^\imf)^{\imf}\Big(\vicinity_{L\cdot L^{-1}\cdot L}(x)\Big) \subseteq F. \]
\end{proof}

\begin{proposition}
Let $\sSet{G, \cdot, 1, \xi}$ be a locally compact, regular, pseudotopological convergence group and $N$ a closed normal subgroup. Then $G/N$ is a locally compact pseudotopological convergence group.
\end{proposition}
\begin{proof}
Let $F\in \powerfilters(G/N)$ be a filter that converges to $[x]_N$ in $\pseudotopMod(G/N)$. We need to show that it converges to $[x]_N$ in $G/N$.

Take an arbitrary compact $K\subseteq G$. As a subspace, $K$ is regular, pseudotopological, Hausdorff and compact (by \ref{regularityInitialConvergence}, \ref{pretopologicalInitialConvergence}, \ref{T2initialConvergence} and \ref{compactSetCompactSubspace}). This means that the subspace $K$ is topological, by \ref{T3pseudotopologyTopological}. In particular $K$ is pretopological. As this holds for all compact subsets of $G$, all compact subsets of $G/N$ are pretopological by \ref{compactSubsetsQuotientGroupPretopological}.

Local compactness is given by \ref{quotientConvergenceCompactConvergenceCover}, which also tells us that $\setbuilder{[K]^\imf_N}{\text{$K\subseteq G$ compact}}$ is a convergence cover of compact sets.

Thus there exists a compact $K\subseteq G$ such that $[K]^\imf_N\in F$. Then $[x]_N\in [K]^\imf_N$ because $[K]^\imf_N$ is closed by \ref{compactClosedSets} and \ref{quotientConvergenceGroupProperties}.

So $\iota_{[K]^\imf_N}^\preimf(F) \overset{\pseudotopMod(G/N)|_{[K]^\imf_N}}{\longrightarrow} [x]_N$ by \ref{subspaceConvergence}. Since $[K]^\imf_N$ is pretopological, we have $\iota_{[K]^\imf_N}^\preimf(F) \overset{G/N|_{[K]^\imf_N}}{\longrightarrow} [x]_N$ and so $F \overset{G/N}{\longrightarrow} [x]_N$, again by \ref{subspaceConvergence}.
\end{proof}

\section{Metrisability and norm}

\begin{theorem}[Birkhoff-Kakutani]
Let $\sSet{G,\cdot, 1, \xi}$ be a convergence group. Then the following are equivalent:
\begin{enumerate}
\item $G$ is pseudometrisable;
\item the topology on $G$ is induced by a left translation invariant pseudometric;
\item the topology on $G$ is induced by a right translation invariant pseudometric;
\item $G$ is first countable and topological.
\end{enumerate}
\end{theorem}


\subsection{Group norms and seminorms}
\begin{definition}
Let $\sSet{G, \cdot, 1}$ be a group. We call a function $\norm{\cdot}: G\to \R^+$ a \udef{group seminorm} on $G$ if $\forall x,y\in G$ the following hold
\begin{enumerate}
\item triangle inequality / subadditivity: $\norm{xy} \leq \norm{x} + \norm{y}$;
\item inversion $\norm{x^{-1}} = \norm{x}$.
\end{enumerate}
We call the group seminorm
\begin{itemize}
\item a \udef{group norm} if $\norm{x} = 0 \implies x = 1$;
\item \udef{cyclically permutable} if $\forall x,y\in G$: $\norm{xy} = \norm{yx}$
\end{itemize}

If $\norm{\cdot}$ is a group (semi)norm for $G$, then we call $\sSet{G, \cdot, 1, \norm{\cdot}}$ a \udef{(semi)normed group}.
\end{definition}

\begin{proposition} \label{groupSeminormConvergence}
Let $\sSet{G, \cdot, 1, \norm{\cdot}}$ be a cyclically permutable seminormed group. Then 
\begin{enumerate}
\item $\norm{\cdot}\circ \Delta_r = \norm{\cdot}\circ \Delta_l$ is a pseudometric;
\item the pseudometric convergence is a group convergence;
\item the pseudometric uniformity is equal to the group uniformity.
\end{enumerate} 
\end{proposition}
\begin{proof}
(1) We first calculate the equality:
\[ (\norm{\cdot}\circ \Delta_r)(x,y) = \norm{xy^{-1}} = \norm{(yx^{-1})^{-1}} = \norm{yx^{-1}} = \norm{x^{-1}y} = (\norm{\cdot}\circ \Delta_l)(x,y). \]

We verify it is a pseudometric. Positivity is immediate. Symmetry follows from the previous calculation. For the triangle inequality, we have
\[ \norm{xy^{-1}} = \norm{xz^{-1}zy^{-1}} \leq \norm{xz^{-1}} + \norm{zy^{-1}}. \]

(2) By \ref{groupConvergenceIffDeltasContinuous} it is enough to show that $\Delta_r$ is continuous. Let $F\to x$ and $F'\to y\in G$. By \ref{imagePseudometricConvergesToZero}, this is equivalent to $\norm{F\cdot x^{-1}} \overset{\R}{\longrightarrow} 0$ and $\norm{F'\cdot y^{-1}} \overset{\R}{\longrightarrow} 0$. Now
\begin{align*}
\norm{\Delta_r\big(\Delta_r(F\otimes F')\otimes \pfilter\Delta_r(x,y)^{-1}\big)} &= \norm{F\cdot (F')^{-1}\cdot y\cdot x^{-1}} \\
&= \norm{x^{-1}\cdot F\cdot (F')^{-1}\cdot y} \\
&\leq \norm{x^{-1}\cdot F}+ \norm{(F')^{-1}\cdot y} \\
&= \norm{F\cdot x^{-1}} + \norm{F'\cdot y^{-1}} \overset{\R}{\longrightarrow} 0,
\end{align*}
so $\Delta_r(F\otimes F')\to \Delta_r(x,y)$, which means that $\Delta_r$ is continuous.

(3) Take $H\in\powerfilters(G^2)$. Then $H$ is an element of the pseudometric uniformity if $\norm{\Delta_r(H)}\overset{\R}{\longrightarrow} 0$ and $H$ is an element of the group uniformity if $\Delta_r(H)\overset{G}{\longrightarrow} 1$, which, by definition of the group convergence is equivalent to $\norm{\Delta_r\big(\Delta_r(H)\otimes\pfilter{1}\big)}\overset{\R}{\longrightarrow} 0$.

Since $\Delta_r\big(\Delta_r(H)\otimes\pfilter{1}\big) = \Delta_r(H)$, these conditions are the same.
\end{proof}
\begin{corollary} \label{normUniformlyContinuous}
Let $\sSet{G, \cdot, 1, \norm{\cdot}}$ be a cyclically permutable seminormed group. Then the norm is uniformly continuous.
\end{corollary}
\begin{proof}
Immediate from \ref{partialApplicationMetricUniformlyContinuous}.
\end{proof}

\begin{lemma} \label{groupUniformNorm}
Let $\sSet{X,d}$ be a metric space and $\sSet{G,\cdot, 1,\norm{\cdot}}$ a seminormed group with cyclically permutable seminorm. Then
\begin{enumerate}
\item the uniform convergence on $(X\to G)$ is generated by the cyclically permutable group seminorm $\norm{f}_{u} \defeq \sup_{x\in X}\norm{f(x)}$;
\item $\norm{\cdot}_u$ is a norm \textup{if and only if} $\norm{\cdot}$ is a norm.
\end{enumerate}
\end{lemma}
\begin{proof}
(1) Comparing \ref{groupSeminormConvergence} and \ref{metricUniformConvergence}, we see that if $\norm{\cdot}_u$ is a cyclically permutable group seminorm, then the convergence it generates is the uniform convergence. So we just need to check that it is indeed a cyclically permutable group seminorm.

We first prove the triangle inequality:
\begin{align*}
\norm{fg}_u &= \sup_{x\in X}\norm{f(x)g(x)} \\
&\leq \sup_{x,y\in X}\norm{f(x)g(y)} \\
&\leq \sup_{x,y\in X}\norm{f(x)} + \norm{g(y)} \\
&= sup_{x\in X}\norm{f(x)} + \sup_{y\in X}\norm{g(y)} \\
&= \norm{f}_u + \norm{g}_u.
\end{align*}
Inversion follows from inversion at every point:
\[ \norm{f^{-1}}_u = \sup_{x\in X}\norm{f(x)^{-1}} = \sup_{x\in X}\norm{f(x)} = \norm{f}_u. \]
The proof of cyclic permutability is similar:
\[ \norm{fg}_u = \sup_{x\in X}\norm{f(x)g(x)} = \sup_{x\in X}\norm{g(x)f(x)} = \norm{gf}_u. \]

(2) First suppose $\norm{\cdot}_u$ is a norm and take $g\in G$ such that $\norm{g}= 0$. Then $\norm{\underline{g}}_u = 0$, so $\underline{g} = \underline{1}$. This implies $g = 1$.

Now suppose $\norm{\cdot}$ is a norm. If $\norm{f}_u = 0$, i.e.\ $\sup_{x\in X}\norm{f(x)}$, then $\norm{f(x)} = 0$ for all $x\in X$ and thus $f(x) = 1$ for all $x\in X$. Thus $f = \underline{1}$.
\end{proof}

\begin{definition}
Let $\sSet{X,d}$ be a metric space and $\sSet{G,\cdot, 1,\norm{\cdot}}$ a seminormed group with cyclically permutable seminorm. Then
\[ \norm{\cdot}_u: (X\to G) \to \overline{\R^+}: f\mapsto \norm{f}_u \defeq \sup_{x\in X}\norm{f(x)} \]
is called the \udef{uniform norm}. It is also called the \udef{supremum norm}, the \udef{Chebyshev norm} or the \udef{infinity norm}.
\end{definition}

\section{Infinite sums}
\begin{definition}
Let $\sSet{G,\xi}$ be an abelian group and $\seq{a_i}_{i\in I}$ a net. Then $\sum_{j\in J}a_j$ forms a net, indexed by all finite subsets $J$ of $I$. The terms of this net are called \udef{partial sums}.

The limit set of this net is denoted $\sum_{i\in I} a_i$.
\end{definition}

\begin{proposition} \label{finiteSumsAreCountable}
Let $I$ be an index set and $a: I\to \R^+$ a positive function. If $\sum_{i\in I}a_i < \infty$, then $a_i\neq 0$ for at most countably many $i\in I$.
\end{proposition}
\begin{proof}
Define $A_n = \setbuilder{i\in I}{a_i \geq n^{-1}}$ and $A = \bigcup_{n\in \N}A_n = \setbuilder{i\in I}{a_i > 0}$. Suppose, towards a contradiction, that $A$ is uncountable. Then, by at least one of the $A_n$ must be uncountable (if not, $A$ would be countable by \ref{repeatedAdditionMultiplicationCardinals} and \ref{stringsInCountableAlphabetCountable}). We have
\begin{align*}
\sum_{i\in I}a_i &= \sup\setbuilder{\sum_{i\in F}a_i}{\text{$F\subseteq I$ finite}} \\
&\geq \sup\setbuilder{\sum_{i\in F}a_i}{\text{$F\subseteq A_n$ finite}} \\
&\geq \sup\setbuilder{\sum_{i\in F}n^{-1}}{\text{$F\subseteq A_n$ finite}} \\
&= \sum_{j\in \N}n^{-1} = \infty.
\end{align*}
\end{proof}

\subsection{Series in normed abelian groups}
\begin{definition}
Let $\seq{a_n}_{n\in \N}$ be a sequence in a normed abelian group. We write $\sum_{n=0}^\infty a_n$ to mean $\lim_{k\to \infty} \sum_{n=0}^k a_n$. If this limit exists, we call this construct a \udef{convergent series}.

We call the series
\begin{itemize}
\item \udef{absolutely convergent} if the series $\sum_{i=0}^\infty \norm{a_i}$ is convergent;
\item \udef{unconditionally convergent} if for all bijections $\sigma: \N\to \N$, the series $\sum_{i=0}^\infty a_{\sigma(i)}$ is convergent and \udef{conditionally convergent} otherwise.
\end{itemize}
\end{definition}

\begin{lemma}
TODO: $\sum_{n\in\N}a_n$ converges iff $\sum_{n=0}^\infty a_n$ converges unconditionally.
\end{lemma}

\begin{proposition} \label{absoluteConvergenceImpliesConvergence}
Let $\sum_{i=0}^\infty a_i$ be a series in a complete abelian normed group. If $\sum_{i=0}^\infty a_i$ is absolutely convergent, it is unconditionally convergent.
\end{proposition}
\begin{proof}
Assume absolute convergence, so $\sum_i\norm{x_i}<\infty$. Then (for $m< n$)
\[ \norm{\sum_{i=1}^n x_i - \sum_{i=1}^m x_i} = \norm{\sum_{i=m+1}^n x_i} \leq \sum_{i=m+1}^n\norm{x_i} = \sum_{i=1}^n \norm{x_i} - \sum_{i=1}^m \norm{x_i}, \]
and because $\sum_i\norm{x_i}$ converges, it is a Cauchy sequence and by the inequality so is $\sum_i x_i$ (TODO ref). By completeness this sequence is convergent.

By (TODO ref) $\sum_i\norm{x_{\sigma(i)}}$ converges for any permutation $\sigma$ of $\N$. We can then repeat the argument to show $\sum x_{\sigma(i)}$ is also convergent and thus unconditionally convergent.
\end{proof}

\begin{proposition}
\url{https://en.wikipedia.org/wiki/L%C3%A9vy%E2%80%93Steinitz_theorem}
\end{proposition}
\begin{corollary}[Riemann series theorem]
Let $\sum_{i=0}^\infty a_i$ be a convergent series that does not converge absolutely and $L\in \overline{\R}$. There exists a bijection $\sigma:\N\to\N$ such that
\[ \sum_{i=0}^\infty a_{\sigma(i)} = L. \]
There also exists a bijection $\tau:\N\to\N$ such that $\sum_{i=0}^\infty a_{\tau(i)}$ fails to approach any limit, finite or infinite.
\end{corollary}
\begin{proof}
\url{https://en.wikipedia.org/wiki/Riemann_series_theorem}
\end{proof}
\begin{corollary}
A convergent series is unconditionally convergent \textup{if and only if} it is absolutely convergent.
\end{corollary}



\subsection{Uniform convergence}

\begin{lemma}
Let $G$ be a complete normed abelian group, $X$ a set and $\seq{f_n: X\to G}_{n\in\N}$ a sequence of functions. If $\sum_{n=0}^\infty \norm{f_n}$ converges uniformly, then $\sum_{n=0}^\infty f_n$ converges uniformly.
\end{lemma}
Here $\norm{f_n}$ is the pointwise application of the norm! I.e.\ $\norm{f_n} = x\mapsto \norm{f_n(x)}$.
\begin{proof}
Assume $\sum_{n=0}^\infty \norm{f_n}$ converges uniformly. Then it converges pointwise and $\sum_{n=0}^\infty f_n$ converges pointwise by \ref{absoluteConvergenceImpliesConvergence}, so the function $\sum_{n=0}^\infty f_n$ is well-defined.

Then we calculate
\begin{align*}
\sup_{x\in X}\norm{\sum_{n=0}^\infty f_n(x) - \sum_{n=0}^k f_n(x)} &= \sup_{x\in X}\norm{\sum_{n=k+1}^\infty f_n(x)} \\
&\leq \sup_{x\in X}\sum_{n=k+1}^\infty \norm{f_n(x)} \\
&= \sup_{x\in X}\sum_{n=0}^\infty \norm{f_n(x)} - \sum_{n=0}^k \norm{f_n(x)} \\
&= \sup_{x\in X}\Big|\sum_{n=0}^\infty \norm{f_n(x)} - \sum_{n=0}^k \norm{f_n(x)}\Big| \overset{k\to \infty}{\longrightarrow} 0.
\end{align*}
where we can apply the triangle inequality to $\norm{\sum_{n=k+1}^\infty f_n(x)}$ since $\sum_{n=k+1}^\infty \norm{f_n(x)}$ converges and we have used \ref{metricUniformConvergence} with the uniform convergence of $\sum_{n=0}^\infty \norm{f_n}$.

Then the squeeze theorem \ref{squeezeTheorem} and \ref{metricUniformConvergence} imply the uniform convergence of $\sum_{n=0}^\infty f_n$.
\end{proof}
\begin{corollary}[Weierstrass $M$-test] \label{WeierstrassMTest}
Let $G$ be a complete normed abelian group, $X$ a set and $\seq{f_n: X\to G}_{n\in\N}$ a sequence of functions. Suppose there exists a sequence $\seq{M_n}_{n\in\N}$ of positive real numbers such that
\begin{itemize}
\item $\sup_{x\in X}\norm{f_n(x)} \leq M_n$;
\item $\sum_{n=0}^\infty M_n$ converges;
\end{itemize}
then $\sum_{n=0}^\infty \norm{f_n}$ and $\sum_{n=0}^\infty f_n$ converge uniformly.
\end{corollary}
\begin{proof}
By the proposition, it is enough to show that $\sum_{n=0}^\infty \norm{f_n}$ converges uniformly. We calculate
\begin{align*}
0\leq \sup_{x\in X}\Big|\sum_{n=0}^\infty \norm{f_n(x)} - \sum_{n=0}^k \norm{f_n(x)}\Big| &= \sup_{x\in X}\sum_{n=k+1}^\infty \norm{f_n(x)} \\
&\leq \sum_{n=k+1}^\infty \sup_{x\in X}\norm{f_n(x)} \\
&\leq \sum_{n=k+1}^\infty M_n \\
&= \sum_{n=0}^\infty M_n - \sum_{n=0}^k M_n \overset{k\to \infty}{\longrightarrow} 0. 
\end{align*}
Then the squeeze theorem \ref{squeezeTheorem} and \ref{metricUniformConvergence} imply the uniform convergence of $\sum_{n=0}^\infty \norm{f_n}$.
\end{proof}
\begin{corollary}[Tannery's theorem] \label{tannery}
Let $G$ be a complete normed abelian group and $a_{n,k}$ an element of $G$ for all $n,k\in \N$. Suppose
\begin{itemize}
\item the series $s(n) = \sum_{k=0}^\infty a_{n,k}$ is convergent for all $k\in \N$;
\item the limit $a_k = \lim_{n\to\infty}a_{n,k}$ exists for all $k\in \N$;
\item there exists a real sequence $\seq{M_k}$ such that $\norm{a_{n,k}} \leq M_k$ for all $n,k\in \N$ and $\sum_{k=0}^\infty M_k$ converges, then
\end{itemize}
\[ \lim_{n\to\infty} s_n = \lim_{n\to\infty}\sum_{k=0}^\infty a_{n,k} = \sum_{k=0}^\infty\lim_{n\to\infty} a_{n,k} = \sum_{k=0}^\infty a_k.   \]
\end{corollary}
\begin{proof}
For each $k \in \N$, we can extend the function $\N\to G: n\mapsto a_{n,k}$ continuously to $\overline{\N}$ by setting $a_{\infty, k}  \defeq a_k$, as in \ref{continuitySequenceAsFunction}, since $G$ is a Kent space.

We now show that the functions $s_K: \overline{\N} \to G: n\mapsto \sum_{k=0}^K a_{n,k}$ converge uniformly to $s$ using the Weierstrass M-test \ref{WeierstrassMTest}. By assumption, $\sup_{n\in \overline{\N}\setminus\{\infty\}}\norm{s_K(n)} \leq M_K$ for all $K\in \N$. To apply the test, we just need to observe that $\norm{a_{n,\infty}} = \norm{\lim_{k\to \infty}a_{n,k}} = \lim_{k\to \infty}\norm{a_{n,k}} \leq M_k$, by the continuity of the norm, \ref{normUniformlyContinuous}, and the compatibility of the convergence of the reals with the order.

Since $\overline{\N}$ is compact, by \ref{extendedNaturalsTopological}, the series $s$ converges in the continuous convergence by (TODO ref, \ref{uniformImpliesContinuousConvergence} is \emph{not} enough!!).

Since $G$ is regular, by \ref{topologicalUniformSpaceRegular}, we can exchange the limits by \ref{exchangeSequenceLimitsContinuousConvergence}.
\end{proof}
TODO: recheck the following is correct:
\begin{proof}[Alternate, elementary, proof]
Choose an arbitrary $\varepsilon>0$. For any $n,N\in\N$ we can write
\[ \norm{s(n) - \sum_{k=0}^\infty a_k} \leq \sum_{k=0}^\infty\norm{a_{n,k} -  a_k}\leq \sum_{k=0}^N\norm{a_{n,k} -  a_k} + 2\sum_{k>N}M_k \leq N\max_{k<N}\norm{a_{n,k} -  a_k} + 2\sum_{k>N}M_k. \]
So we aim to find some $N_0$ such that $2\sum_{k>N}M_k \leq \varepsilon/2$ for all $N\geq N_0$, which of course we can. Then we choose an $n_0$, in function of this $N_0$ and $\varepsilon$, such that $\max_{k<N_0}\norm{a_{n,k} -  a_k}\leq \varepsilon/(2N_0)$ for all $n\geq n_0$. It is clear we can do so for each $k$ separately, but there are only finitely many $k$s so we take the largest $n_0$. Then
\[ \norm{s(n) - \sum_{k=0}^\infty a_k} \leq \varepsilon \qquad \text{for all $n\geq n_0$, implying the limit is zero.} \]
\end{proof}
\url{https://www.coloradomesa.edu/math-stat/documents/JohnGillresearchnoteTanneryTheorem.pdf}

\subsection{Difference calculus}

\begin{proposition}[Summation by parts]
Let $\seq{x_k}$ and $\seq{y_k}$ be sequences in some field and $m,n\in \N$. Then
\begin{align*}
\sum_{k=m}^n x_k\Delta^+y_k &= (x_ny_{n+1} - x_{m-1}y_m)-\sum_{k=m}^ny_k\Delta^-x_k \\
&= (x_ny_{n+1} - x_my_m) - \sum_{k=m}^{n-1}y_{k+1}\Delta^+x_k.
\end{align*}
\end{proposition}

\subsection{Difference calculus on sequences}
TODO move!!

TODO $\Delta^+_\epsilon, \Delta^-_\epsilon$ for real functions! Also central difference $\Delta$.
\subsubsection{Difference operators}
\begin{definition}
Let $\seq{x_n}$ be a sequence. We define
\begin{itemize}
\item the \udef{forward difference operator} $\Delta^+: \R^\N \to \R^\N$ by $\Delta^+ x_n \defeq (\Delta^+x)_n \defeq x_{n+1} - x_n$; and
\item the \udef{backward difference operator} $\Delta^-: \R^\N \to \R^\N$ by $\Delta^- x_n \defeq (\Delta^-x)_n \defeq x_{n} - x_{n-1}$.
\end{itemize}
\end{definition}


\chapter{Convergence rings}
\begin{definition}
A \udef{convergence rng} is a rng $\sSet{R, +, \cdot, 0}$ with convergence $\xi$ such that
\begin{itemize}
\item $\sSet{R,+,0}$ is a convergence group;
\item the multiplication $\cdot: R\times R\to R$ is continuous.
\end{itemize}
If the rng is a (unital) ring, then we call it a convergence ring.
\end{definition}

\begin{lemma} \label{ringVicinityFactorisation}
Let $\sSet{R, +,\cdot, 0, \xi}$ be a topological convergence rng and $x,y\in R$. If $U\in \neighbourhood_\xi(xy)$, then there exist $V\in \neighbourhood_\xi(x)$ and $W\in \neighbourhood_\xi(y)$ such that $V\cdot W\subseteq U$.

If $x=y$, then we can take $V = W$.
\end{lemma}
\begin{proof}
Consider the function $f: R\times R \to R: (x,y)\mapsto xy$. Then by \ref{pretopologicalContinuityVicinities} and \ref{productVicinity} we have
\[ \neighbourhood_\xi(xy) \subseteq \upset f^{\imf\imf}\big[\neighbourhood_{\xi\otimes\xi}((x,y))\big] = \upset f^{\imf\imf}\big[\upset \vicinity_\xi(x)\otimes \vicinity_\xi(y)\big] = \upset f^{\imf\imf}\big[\vicinity_\xi(x)\otimes \vicinity_\xi(y)\big] = \upset \big(\vicinity_\xi(x)\cdot \vicinity_\xi(y)\big) \]
This implies the first result.

If $x=y$, then pick $V,W$ as before. Consider $V\cap W$, which is still a neighbourhood of $x=y$ and $(V\cap W)\cdot(V\cap W) \subseteq V\cdot W \subseteq U$.
\end{proof}

\section{Convergence fields}
\begin{definition}
A \udef{convergence field} is a field $\sSet{F, +, \cdot, 0,1}$ such that
\begin{itemize}
\item $\sSet{F,+,0}$ is a convergence group;
\item $\sSet{F,\cdot, 1}$ is a convergence group.
\end{itemize}
\end{definition}

\begin{proposition} \label{compactT2TopologicalFieldFinite}
Every compact Hausdorff topological convergence field $\sSet{F,+,\cdot, 0,1}$ is finite.
\end{proposition}
We don't need continuity of ${}^{-1}$.
\begin{proof}
Since $F$ is Hausdorff, we have $\neighbourhood(1)\not\to 0$ and so $\neighbourhood(0) \nsubseteq \neighbourhood(1)$. We can then pick an open $V\in \neighbourhood(0) \setminus \neighbourhood(1)$.

For all $x\in F$, we can find $U_x\in \neighbourhood(0)$ and $W_x\in \neighbourhood(x)$ such that $U_x\cdot W_x \subseteq V$ by \ref{ringVicinityFactorisation}.

Now $\setbuilder{W_x}{x\in F}$ is a neighbourhood cover of $F$. By compactness, we can find a finite subcover. Thus there exists a finite set $S\subseteq F$ such that $\bigcup_{x\in S}W_x = F$.

Consider $U = \bigcap_{x\in S}U_x$, which is still a neighbourhood off $0$. We calculate
\begin{align*}
U\cdot F = U\cdot \Big(\bigcup_{x\in S}W_x\Big) &= \bigcup_{x\in S} U\cdot W_x \\
&\subseteq \bigcup_{x\in S} U_x\cdot W_x \subseteq V.
\end{align*}

We now show, by contradiction, that the convergence is discrete. By \ref{discreteCompactIffFinite} we can then conclude that the field is finite.

Assuming that the convergence is not discrete, we have $U \neq \{0\}$, so we can take $x\in U\setminus\{0\}$. Now
\[ 1 = v\cdot v^{-1} \in U\cdot F \subseteq V. \]
Since $V$ was taken to be open, we have $V\in \neighbourhood(1)$, which is a contradiction.
\end{proof}

\subsection{Locally compact fields}

\url{http://alpha.math.uga.edu/~pete/8410Chapter5.pdf}.