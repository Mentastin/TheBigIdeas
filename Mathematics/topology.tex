\url{https://en.wikipedia.org/wiki/Cauchy_space}

\url{https://en.wikipedia.org/wiki/Cauchy-continuous_function}

\url{https://en.wikipedia.org/wiki/Proximity_space}

\url{https://www.bioinf.uni-leipzig.de/~studla/Publications/PREPRINTS/01-pfs-007-subl1.pdf}

\chapter{Convergence}
\section{Convergence spaces}
Intuition: directed sets and refinement.

\begin{definition}
Let $X$ be a set and let $\xi$ be a relation between downward directed sets in $\powerset(X)$ and elements of $X$. Then
\begin{itemize}
\item we denote the image function of $\xi$ as $\lim_\xi: \directed(\powerset(X)) \to \powerset(X): F\mapsto F\xi$; we call $\lim_\xi F$ the \udef{$\xi$-limit} of $F$;
\item $\xi$ is called a \udef{preconvergence} on $X$ if $\lim_\xi$ is order-preserving when $\directed(\powerset(X))$ is ordered by refinement:
\[ F \preceq G \implies \lim_\xi F \subseteq \lim_\xi G; \]
\item $\xi$ is called a \udef{convergence} if it is a preconvergence and it is \udef{centered}:
\[ \forall x\in X: \quad x\in \lim_\xi \big\{\{x\}\big\}. \]
\end{itemize}
If $\xi$ is a convergence, then we call $\sSet{X, \xi}$ a \udef{convergence space}.

If $\lim_\xi F \neq \emptyset$, we say the directed set $F$ \udef{converges}.
\end{definition}
We write ${\lim_\xi}^{-1}$ for the preimage function of $\xi$ restricted to directed sets $\nsim \powerset(X)$. Thus ${\lim_\xi}^{-1}(x)$ is the set of all directed sets $D$ such that
\begin{itemize}
\item $D\overset{\xi}{\longrightarrow} x$;
\item $D \nsim \powerset(X)$.
\end{itemize}

\begin{lemma}
Let $X$ be a set and $x\in X$. A preconvergence $\xi$ on $X$ is a convergence \textup{if and only if} $\forall x\in X: \{x\} \in {\lim_\xi}^{-1}(x)$.
\end{lemma}

\begin{lemma}
Let $X$ be a set, $\xi$ a convergence on $X$ and $x\in X$. Then $\lim_\xi^{-1}(x)$ is upwards closed.
\end{lemma}

\subsection{Filters and convergence}
\begin{lemma}
Let $X$ be a set and $\xi$ a convergence on $X$. Let $A,B\in \directed(X)$ be downward directed sets. If $A \approx B$, then $\lim_\xi A = \lim_\xi B$.
\end{lemma}
\begin{proof}
We have $A \preceq B$ and $B \preceq A$, so $\lim_\xi A \subseteq \lim_\xi B$ and $\lim_\xi B \subseteq \lim_\xi A$.
\end{proof}
This lemma means we can view $\lim_\xi$ as a function on the quotient $\directed(X) /\approx$, which is order isomorphic to $\filters(X)$ (indeed there is one filter in each equivalence class in $\directed(X) /\approx$ and for filters the refinement relation simplifies to inclusion).

Consequently we will usually think of a convergence on a set $X$ as a relation between filters on $\powerset(X)$ and elements of $X$. The axioms then become
\begin{itemize}
\item $\xi$ is a preconvergence on $X$ if $\lim_\xi$ is order-preserving when $\powerfilters(X)$ is ordered by inclusion:
\[ F \subseteq G \implies \lim_\xi F \subseteq \lim_\xi G; \]
\item $\xi$ is called a convergence if it is a preconvergence and it is centered:
\[ \forall x\in X: \quad x\in \lim_\xi \pfilter{x}. \]
\end{itemize}

Any convergence defined for filters uniquely extends to a convergence defined for all downward directed sets. 

\begin{lemma}
Let $X$ be a set and $\xi$ a preconvergence on $X$. Then
\begin{enumerate}
\item $\forall F,G \in \powerfilters(X): \quad \lim_\xi(F\cap G) \subseteq \lim_\xi F \cap \lim_\xi G$;
\item $\forall F,G \in \powerfilters(X): \quad \lim_\xi(F\cup G) \supseteq \lim_\xi F \cup \lim_\xi G$.
\end{enumerate}
\end{lemma}
\begin{proof}
Reformulation of \ref{orderPreservingFunctionLatticeOperations}, because a preconvergence is order-preserving.
\end{proof}
\begin{corollary} \label{limitDegenerateFilter}
For any convergence $\xi$ on $X$ we have $\lim_\xi \powerset(X) = X$.
\end{corollary}
\begin{proof}
Clearly $\pfilter{x} \subseteq \powerset(X)$ for all $x\in X$, so
\[ \lim_\xi(\powerset(X)) \supseteq \lim_\xi \left(\bigcup_{x\in X}\pfilter{x}\right) \supseteq X. \]
\end{proof}

In the sequel we will usually consider convergence as a property of filters, however sometimes it will be easier to consider downward directed sets (e.g.\ for continuity).

\subsubsection{Limit points}
\begin{definition}
Let $\sSet{X,\xi}$ be a convergence space and $x\in X$. Then
\begin{itemize}
\item $x$ is called a \udef{limit point} if there exists a filter $F\in\powerfilters(X)$ other than $\pfilter{x}$ that converges to $x$;
\item $x$ is called an \udef{isolated point} if it is not a limit point.
\end{itemize}
Let $A\subseteq X$ be a subset. Then $x$ is called a \udef{limit point of $A$} if there exists a filter $F\in\powerfilters(X)$ other than $\pfilter{x}$ that converges to $x$ and has $A\in F$.
\end{definition}

\begin{proposition}
Let $\sSet{X,\xi}$ be a convergence space and $x\in X$. The following are equivalent:
\begin{enumerate}
\item $x$ is a limit point;
\item $\vicinity_\xi(x) \neq \pfilter{x}$;
\item $x\in \adh_\xi\big(X\setminus\{x\}\big)$;
\end{enumerate}
\end{proposition}
\begin{proof}
TODO (??)
\end{proof}

\subsubsection{Approaches}
\begin{definition}
Let $\sSet{X,\xi}$ be a convergence space. An \udef{approach} on $X$ is a function $F: X\to \powerfilters(X)$ such that $F(x) \overset{\xi}{\longrightarrow} x$ for all $x\in X$.
\end{definition}

\begin{lemma}
Let $\sSet{X,\xi}$ be a convergence space. The function $X\to \powerfilters(X): x\mapsto \pfilter{x}$ is an approach.
\end{lemma}

\subsection{Depth requirements}
\begin{definition}
Let $\sSet{X,\xi}$ be a convergence space. Take arbitrary $F,G \in \powerfilters(X)$, $\mathcal{F}\subseteq \powerfilters(X)$ and $x\in X$. The convergence space is called
\begin{itemize}
\item a \udef{Kent space} if $F \to x \implies F\cap \pfilter{x} \to x$;
\item \udef{finitely deep} or a \udef{limit space} if $\lim_\xi(F\cap G) = \lim_\xi F \cap \lim_\xi G$;
\item a \udef{Choquet space} or a \udef{pseudotopological space} if
\[ F\overset{\xi}{\longrightarrow} x \qquad\iff\qquad \text{$U\overset{\xi}{\longrightarrow} x$ for all ultrafilters $U$ such that $F\subseteq U$.} \]
\item a \udef{pretopological space} if 
\[ \lim_\xi\Big(\bigcap \mathcal{F}\Big) = \bigcap \setbuilder{\lim_\xi(F')}{F'\in \mathcal{F}}. \]
\end{itemize}
\end{definition}

The point of a Kent space is that if $F\to x$, then ``essentially'' every $A\in F$ contains $x$.

\begin{lemma} \label{finiteDepthLemma}
Let $X$ be a set and $\xi$ a preconvergence on $X$. Then the following are equivalent:
\begin{enumerate}
\item $\xi$ is finitely deep;
\item $\forall F,G \in \powerfilters(X): \quad \lim_\xi(F\cap G) \supseteq \lim_\xi F \cap \lim_\xi G$;
\item $\lim_\xi^{-1}(x)\cup \powerset(X)$ is a filter for all $x\in X$.
\end{enumerate}
\end{lemma}
In other words, a preconvergence is finitely deep \textup{if and only if} $\lim_\xi^{-1}(x)$ is \emph{directed} (and thus a filter) for all $x\in X$.

\begin{proposition}
Let $X$ be a set and $\xi$ a preconvergence on $X$. Then each of the following statements implies the next:
\begin{enumerate}
\item $\xi$ is pretopological;
\item $\xi$ is Choquet;
\item $\xi$ is finitely deep.
\end{enumerate}
If $\xi$ is a convergence space, then these also imply
\begin{enumerate} \setcounter{enumi}{3}
\item $\xi$ is Kent.
\end{enumerate}
\end{proposition}
\begin{proof}
$(1) \Rightarrow (2)$ By \ref{filtersCoatomistic}, we have
\[ F = \bigcap_{\substack{U\in \powerultrafilters(X) \\ U\geq F}} U. \]
By the pretopological property this means
\[ \lim_\xi(F) = \lim_\xi\left(\bigcap_{\substack{U\in \powerultrafilters(X) \\ U\geq F}} U\right) = \bigcap_{\substack{U\in \powerultrafilters(X) \\ U\geq F}} \lim_\xi(U). \]
Thus $x\in \lim_\xi(F)$ iff $x\in \lim_\xi(U)$ for all ultrafilters $U$ such that $F\subseteq U$.

$(2) \Rightarrow (3)$ Take $F,G\in\powerfilters(X)$. By \ref{finiteDepthLemma} it is enough to prove $\lim_\xi(F\cap G) \supseteq \lim_\xi F \cap \lim_\xi G$. Take $x\in \lim_\xi F \cap \lim_\xi G$. If $U$ is an ultrafilter such that either $F\subseteq U$ or $G\subseteq U$, then $U\to x$.

Now take an ultrafilter $U' \supseteq F\cap G$. Then either $U'\supseteq F$ or $U'\supseteq G$ by \ref{finiteUltrafilterFactorisation}. By the previous remark $U'\to x$. By the Choquet property we conclude that $x\in \lim_\xi(F\cap G)$.

$(3) \Rightarrow (4)$. This is immediate if $\pfilter(x) \to x$ for all $x\in X$, i.e.\ if $\xi$ is a convergence.
\end{proof}

\begin{example}
There exist Choquet spaces that are not pretopological. Take any infinite set $X$. Then there exists a nonprincipal ultrafilter $U$ on $\powerset(X)$ (take the cofinite filter and extend to ultrafilter by ultrafilter lemma, see \ref{filterFreeIFFfinerThanCofinite}).

Take some $x_0$ and let each principal ultrafilter converge to $x_0$. Let no other ultrafilter converge. Define convergence of all other filters by the Choquet property. This convergence is not pretopological: We have $\bigcap \lim^{-1}(x_0) = \big\{\{X\}\big\}$, but $\{X\} \not\to x_0$ because $\{X\}\subseteq U$ and $U \not\to x_0$.
\end{example}

TODO finitely deep = quotient of (pre)topologies; inf of topologies; prime topologies dense in it (?)

\subsection{The lattices of preconvergences and convergences}
\begin{definition}
Let $X$ be a set and $\xi,\zeta$ preconvergences on $X$. We say $\xi$ is \udef{finer} (or \udef{stronger}) than $\zeta$, denoted $\xi \leq \zeta$, if $\lim_\xi F \subseteq \lim_\zeta F$ for all $F\in\powerfilters(X)$. We also say $\zeta$ is \udef{coarser} (or \udef{weaker}) than $\xi$.
\end{definition}
We can think of strength as the ability to stop a filter converging. So $\xi$ is strictly stronger that $\zeta$ if there are filters that converge in $\zeta$, but not in $\xi$. 

\begin{lemma}
Let $X$ be a set and $\xi$ a preconvergence on $X$. Then
\begin{enumerate}
\item $E_{\powerfilters(X), X} \leq \xi \leq U_{\powerfilters(X), X}$;
\item $\xi$ is a convergence \textup{if and only if} $\iota_X \leq \xi$.
\end{enumerate}
\end{lemma}

\begin{definition}
Let $X$ be a set.
\begin{itemize}
\item The \udef{empty preconvergence} on $X$ is $E_{\powerfilters(X), X}$, i.e.\ the limit of all filters is the empty set.
\item The \udef{discrete convergence} $\iota_X$ on $X$ is defined by $x \in \lim_\iota F \iff F = \pfilter{x}$ if $F$ is a proper filter. If $F = \powerset(X)$, then $\lim_\iota F = X$.
\item The \udef{chaotic convergence} on $X$ is $U_{\powerfilters(X), X}$, i.e.\ the limit of all filters is $X$.
\end{itemize}
\end{definition}

\begin{proposition} \label{latticeConvergences}
Let $X$ be a set and and let $\Xi$ be a set of (pre)convergences on $X$. The sets of preconvergences and of convergences are complete bounded lattices and for all $F \in \powerfilters(X)$:
\[ \lim_{\bigwedge \Xi} F = \bigcap_{\xi\in\Xi}\lim_\xi F \quad\text{and}\quad \lim_{\bigvee \Xi} F = \bigcup_{\xi\in\Xi}\lim_\xi F. \]
\begin{itemize}
\item The top of both is the chaotic convergence.
\item The bottom of the lattice of preconvergences is the empty preconvergence.
\item The bottom of the lattice of convergences is the discrete convergence.
\end{itemize}
\end{proposition}

\begin{lemma}
A preconvergence is a convergence \textup{if and only if} it is coarser than the discrete convergence.
\end{lemma}

\subsection{Directional convergence}
\begin{definition}
Let $\sSet{X,\xi}$ be a convergence space, $D\subseteq X$ and $x\in X$. Then a filter $F\in \powerfilters(X)$ is said to \udef{converge to $x$ in the direction $D$} if $F\overset{\xi}{\longrightarrow} x$ and $D\in F$.

We write $F\overset{\xi, D}{\longrightarrow} x$.
\end{definition}

\begin{lemma}
Let $\sSet{X,\xi}$ be a convergence space, $D\subseteq X$ and $x\in X$. Then $F$ converges to $x$ in the direction $D$ \textup{if and only if} there exists a net $\seq{x_i}_{i\in I}\subseteq D$ that converges to $x$ such that $F = \TailsFilter(\seq{x_i}_{i\in I})$.
\end{lemma}
\begin{proof}
If there exists such a net, then $F$ converges to $x$ by definition of net convergence. Every tail of $\seq{x_i}$ is a subset of $D$. So $D\in F$ by upward closure.

Now assume $F$ converges to $x$ in the direction $D$. Consider the index set $I_F = \setbuilder{(A,x)\in F\times X}{x\in A}$ as in \ref{filterIndex}. Now define $I_F^D = I_F\setminus\setbuilder{(A,x)\in F\times X}{x\notin D}$. We claim this set if directed and the associated filter of $I_F^D \to X: (A,x) \mapsto x$ is $F$.

Directedness follows because for all $(A,x), (B,y)\in I_F^D$, the set $A\cap B\cap D$ is not empty as all three sets are elements of the filter $F$ and the filter is not trivial.

Finally we show that $F$ is generated by the tails of $I_F^D \to X: (A,x) \mapsto x$. Take $A\in F$. Then $A\cap D\in F$ and by \ref{tailsFilterIndex}
\[ A\cap D = \setbuilder{y\in X}{(B,y) \geq (A\cap D,x)}. \]
Then $y\in B\subseteq A\cap D \subseteq D$, so this is also a tail of $I_F^D \to X: (A,x) \mapsto x$.
\end{proof}


\section{The vicinity filter}
\begin{definition}
Let $X$ be a set, $\xi$ a preconvergence on $X$ and $x\in X$. The \udef{vicinity filter} of $\xi$ at $x$ is
\[ \vicinity_\xi(x) \defeq \bigcap \lim_\xi^{-1}(x) = \bigcap \setbuilder{F\in \powerfilters(X)}{x\in \lim_\xi F}. \]
A subset $V\subseteq X$ is called a \udef{vicinity} of $x$ for $\xi$ if $V\in \vicinity_\xi(z)$.

We extend the vicinity filters to be defined for elements of $\powerset(X)$ and $\powerset^2(X)$ by contours.
\end{definition}

\begin{lemma}
Let $X$ be a set, $\xi$ a preconvergence on $X$ and $A\subseteq X$. Then
\begin{align*}
\vicinity_\xi(A) &= \bigcap \setbuilder{F\in \powerfilters(X)}{\exists x\in A: \; x\in \lim_\xi F} \\
&= \bigcap \setbuilder{F\in \powerfilters(X)}{A \mesh \lim_\xi F}.
\end{align*}
\end{lemma}

\begin{lemma}
Let $\sSet{X,\xi}$ be a convergence space. Then $\vicinity_\xi$ is an approach \textup{if and only if} $\xi$ is pretopological.
\end{lemma}

\begin{lemma} \label{vicinityOfSetLemma}
Let $\sSet{X,\xi}$ be a convergence space and $A\subseteq X$. Then $\vicinity_\xi(A) \subseteq \upset\{A\}$.
\end{lemma}
\begin{proof}
We have $\vicinity_\xi(A) \subseteq \bigcap_{x\in A}\pfilter{x} = \upset \{A\}$. 
\end{proof}
\begin{corollary} \label{vicinityOfSetCorollary}
For $A\subseteq X$ and $x\in X$, we have
\begin{enumerate}
\item $A \subseteq \bigcap\vicinity_\xi(A)$;
\item $\vicinity_\xi(x) \subseteq \pfilter{x}$;
\item $\{x\}\in\vicinity_\xi(x)^{\mesh}$.
\end{enumerate}
\end{corollary}

\begin{lemma}
Let $X$ be a set, $\xi$ a preconvergence on $X$ and $x\in X$. Then
\[ \vicinity_\xi(x) = \bigcap_{F\in {\lim_\xi}^{-1}(x)\;\cap\; \ultrafilters(\powerset(X))}F. \]
\end{lemma}
\begin{proof}
TODO
\end{proof}

\begin{lemma}
Let $\sSet{X,\xi}$ be a convergence space and $x\in X$. Then
\begin{enumerate}
\item if $F\to x$, then $\vicinity_\xi(x)\lhd F$
\end{enumerate}
\end{lemma}
\begin{proof}
(1) Simple application of \ref{containmentInclusionProperFilter}.
\end{proof}

\begin{lemma} \label{vicinityMapAntitone}
Let $X$ be a set and $\zeta, \xi$ convergences on $X$. If $\zeta \leq \xi$, then $\vicinity_\zeta(x) \supseteq \vicinity_\xi(x)$ for all $x\in X$.
\end{lemma}
\begin{proof}
Assume $\zeta \leq \xi$. For all $F\in \powerfilters(X)$ we have $\lim_\zeta^{-1}(x) \subseteq \lim_\xi^{-1}(x)$, so
\[ \vicinity_\zeta(x) = \bigcap \lim_\zeta^{-1}(x) \supseteq \bigcap \lim_\xi^{-1}(x) = \vicinity_\xi(x). \]
\end{proof}

\subsection{Pretopological convergence}


\section{Base and pavement}
\subsection{Pavement}
\begin{definition}
Let $\sSet{X,\xi}$ be a convergence space.
\begin{itemize}
\item A \udef{pavement} of $\xi$ at a point $x\in X$ is a family of filters $\mathcal{H}$ such that
\[ {\lim_\xi}^{-1}(x) = \upset\mathcal{H}. \]
\item The \udef{paving number} of $\xi$ at $x$ is the least cardinality $\kappa$ such that there is a pavement of $\xi$ of cardinality $\kappa$ at $x$.
\item The \udef{paving number} $\lambda$ of $\xi$ is
\[ \lambda = \sup\setbuilder{\kappa}{\text{$\kappa$ is the paving number of $\xi$ at $x$ for some $x\in X$}}. \]
We say $\xi$ is \udef{$\kappa$-paved} if $\lambda \leq \kappa$.
\item We call the convergence $\xi$ \udef{pretopological} if it is $1$-paved.
\end{itemize}
\end{definition}
Being $\kappa$-paved is a pointwise property.

\begin{proposition}
A finitely deep convergence that is finitely paved is pretopological.
\end{proposition}

\begin{lemma}
Let $\sSet{X,\xi}$ be a pretopological convergence space. Then $\mathcal{H}\in\powerset^2(X)$ is a pavement of $\xi$ \textup{if and only if} $\setbuilder{\vicinity_\xi(x)}{x\in X} \subseteq \mathcal{H}$.
\end{lemma}

\subsection{A base of a convergence}
\begin{definition}
Let $\sSet{X,\xi}$ be a convergence space. A set $\mathcal{Z}\subseteq \powerset(X)$ is called a \udef{base} of the converence $\xi$ if for each point $x\in X$ there exists a pavement $\mathcal{H}$ of $\xi$ at $x$, such that each $F\in\mathcal{H}$ is based in $\mathcal{Z}$.

We also say $\xi$ is \udef{based in} $\mathcal{Z}$.
\end{definition}
TODO filter based in set.



\subsubsection{Bases of pretopological convergences}
\begin{definition}
Let $\sSet{X,\xi}$ be a convergence space and $\mathcal{Z}$ a base of $\xi$. The \udef{effective portion} of $\mathcal{Z}$ is
\[ \setbuilder{Z\in \mathcal{Z}}{\exists x\in X: \; Z\in \vicinity_\xi(x)} = \mathcal{Z}\cap \bigcup\setbuilder{\vicinity_\xi(x)}{x\in X}. \]
\end{definition}

\begin{lemma}
Let $\sSet{X,\xi}$ be a pretopological convergence space and $\mathcal{Y}, \mathcal{Z}$ bases of $\xi$. Then
\begin{enumerate}
\item the effective portion of $\mathcal{Z}$ is a base of $\xi$;
\item the effective portions of $\mathcal{Y}$ and $\mathcal{Z}$ are equally fine.
\end{enumerate}
\end{lemma}
\begin{proof}
(1) Immediate because any pavement of $\xi$ contains $\setbuilder{\vicinity_\xi(x)}{x\in X}$.

(2) Let $\mathcal{Y}'$ and $\mathcal{Z}'$ be the respective effective portions. Take $Y\in \mathcal{Y}'$; we need to show that there exists $Z\in \mathcal{Z}'$ such that $Z\subseteq Y$. Take $x\in X$ such that $Y\in \vicinity(x)$. Then there exist $Z \in \mathcal{Z}'$ such that $Z\subseteq Y$ because $\vicinity(x)$ is based in $\mathcal{Z}'$. 
\end{proof}
\begin{corollary} \label{cardinalityPretopologicalBase}
Let $\sSet{X,\xi}$ be a pretopological convergence space with bases $\mathcal{X},\mathcal{Y}$. Then there exists a base $\mathcal{Z} \subseteq \mathcal{X}$ with cardinality smaller than or equal to $\mathcal{Y}$.
\end{corollary}
\begin{proof}
Take the effective portion $\mathcal{X}'$ of $\mathcal{X}$. Because the effective portion $\mathcal{Y}'$ of $\mathcal{Y}$ refines $\mathcal{X}'$, we have
\[ \forall Y\in \mathcal{Y}': \exists X_Y \in \mathcal{X}': \; X_Y\subseteq Y. \]
The set $\mathcal{Z} \defeq \setbuilder{X_Y}{Y\in \mathcal{Y}'} \subseteq \mathcal{X}$ has cardinality smaller than $\mathcal{Y}'$, which has cardinality smaller than $\mathcal{Y}$.
\end{proof}

\section{Adherence and inherence}
\begin{definition}
Let $\sSet{X,\xi}$ be a convergence space and $\mathcal{A}\subseteq \powerset(X)$ a family of subsets. We define
\begin{itemize}
\item the \udef{adherence} of $\mathcal{A}$ as
\[ \adh_\xi(\mathcal{A}) = \bigcup_{\mathcal{A}\lhd F}\lim_\xi F; \]
\item the \udef{inherence} of $\mathcal{A}$ as
\[ \inh_\xi(\mathcal{A}) = \setbuilder{x\in X}{x\in \lim_\xi F \implies F\amesh\mathcal{A}}. \]
\end{itemize}
\end{definition}

\begin{lemma}
\begin{enumerate}
\item If $F\to x$, then $x\in \adh_\xi(F)$;
\item If $F = \mathfrak{F}(\mathcal{A}), then \adh_\xi(F) = \adh_xi(\mathcal{A})$.
\end{enumerate}
\end{lemma}

\begin{lemma} \label{singletonAdherence}
Let $\sSet{X,\xi}$ be a convergence space and $x\in X$. Then $\adh_\xi(\{x\}) = \lim_\xi \pfilter{x}$.
\end{lemma}
\begin{proof}
If $F$ is a filter such that $\{x\} \lhd F$, then $F = \pfilter{x}$ by \ref{ultrafilterContainment}. So $\adh_\xi \{x\} = \lim_\xi \pfilter{x}$.
\end{proof}

\begin{proposition}
Let $\sSet{X,\xi}$ be a convergence space and $F\subseteq \powerfilters(X)$ a filter. Then $\ker(F) \subseteq \adh_\xi(F)$
\end{proposition}
\begin{proof}
Take $x\in \ker(F)$. Then $\pfilter{x}\amesh F$, so $F\vee \pfilter{x}$ is a proper filter that converges to $x$. Now $F\lhd F\vee \pfilter{x}$ by \ref{containmentInclusionProperFilter}, so $x\in \lim_\xi F\vee \pfilter{x} \subseteq \adh_\xi(F)$.
\end{proof}

\begin{definition}
Let $X$ be a set, $\xi$ a convergence on $X$ and $A\subseteq X$ a subset. We define
\begin{itemize}
\item the \udef{adherence} $\adh_\xi(A)$ of $A$ by
\[ x\in \adh_\xi(A) \defequiv A\in \vicinity_\xi(x)^{\mesh}; \]
\item the \udef{inherence} $\inh_\xi(A)$ of $A$ by
\[ x\in \inh_\xi(A) \defequiv A\in \vicinity_\xi(x). \]
\end{itemize}
\end{definition}

\begin{proposition} \label{principalAdherenceInherence}
Let $\sSet{X,\xi}$ be a convergence space and $A\subseteq X$ a subset. Then
\begin{enumerate}
\item $x\in \adh_\xi(A)$ \textup{if and only if} $A\in \vicinity_\xi(x)^{\mesh}$;
\item $x\in \inh_\xi(A)$ \textup{if and only if} $A\in \vicinity_\xi(x)$.
\end{enumerate}
\end{proposition}
\begin{proof}
TODO
\end{proof}
\begin{corollary} \label{setAdherenceInherence}
Let $\sSet{X,\xi}$ be a convergence space and $A,B\subseteq X$ subsets. Then
\[ \adh_\xi(A) \mesh B \qquad\iff\qquad \{A\} \amesh \vicinity_\xi(B). \]
\end{corollary}
\begin{proof}
We have
\begin{align*}
\adh_\xi(A) \mesh B &\iff \exists b\in B: b\in \adh_\xi(A) \\
&\iff \exists b\in B: A\in \vicinity_\xi(b)^{\mesh} \\
&\iff A \in \bigcup_{b\in B}\vicinity_\xi(b)^{\mesh} = \left(\bigcap_{b\in B}\vicinity_\xi(b)\right)^{\mesh} = \vicinity_\xi(B)^{\mesh}.
\end{align*}
\end{proof}

\begin{proposition} \label{adherenceInherenceCharacterisation}
Let $X$ be a set, $\xi$ a convergence on $X$ and $A \subseteq X$ a set. Then
\begin{enumerate}
\item $\displaystyle x\in \adh_\xi(A) \iff A \in \bigcup_{F\in {\lim_\xi}^{-1}(x)} F^{\mesh} \iff A \in \bigcup_{F\in {\lim_\xi}^{-1}(x)} F$;
\item $x\in \adh_\xi(A) \iff \exists U\in \vicinity_\xi(x): \; U\setminus A \notin \vicinity_\xi(x)$;
\item $\displaystyle \adh_\xi(A) = \bigcup_{A \in F^{\mesh}}\lim_\xi F = \bigcup_{\substack{\text{$G$ proper filter} \\ A\in G}} \lim_\xi G$;
\item $\displaystyle x\in \inh_\xi(A) \iff A \in \bigcap_{F\in {\lim_\xi}^{-1}(x)} F$;
\item $\displaystyle \inh_\xi(A) = \bigcap_{A \in F}\lim_\xi F$.
\end{enumerate}
\end{proposition}
\begin{proof}
TODO
\end{proof}
\begin{corollary} \label{inherenceComplementAdherence}
Let $X$ be a set, $\xi$ a convergence on $X$ and $A \subseteq X$. Then
\[ (\inh_\xi A)^c = (\adh_\xi A^c). \]
\end{corollary}
\begin{corollary} \label{inherenceAdherenceProperties}
Let $X$ be a set, $\xi$ a convergence on $X$ and $A,B \subseteq X$. Then
\begin{enumerate}
\item $\adh_\xi \emptyset = \emptyset$;
\item if $A \subseteq B$, then $\adh_\xi A \subseteq \adh_\xi B$;
\item $A\subseteq \adh_\xi A$;
\item $\adh_\xi(A\cup B) = \adh_\xi A \cup \adh_\xi B$.
\end{enumerate}
and
\begin{enumerate} \setcounter{enumi}{4}
\item $\inh_\xi X = X$;
\item if $A \subseteq B$, then $\inh_\xi A \subseteq \inh_\xi B$;
\item $\inh_\xi A \subseteq A$;
\item $\inh_\xi(A\cap B) = \inh_\xi A \cap \inh_\xi B$.
\end{enumerate}
\end{corollary}
Compare with \ref{discreteIntersectionAdherence}.

\begin{lemma}
Let $X$ be a set, $\xi$ a convergence on $X$ and $A \subseteq X$ a set. Then
\[ \adh_\xi A = \bigcup_{A \in F^{\mesh}}\lim_\xi F = \bigcup_{\substack{\text{$G$ proper filter} \\ A\in G}}\lim_\xi G. \]
\end{lemma}
\begin{proof}
The first equality is obvious. For the second equality: Let $F$ be a filter such that $A\in F^{\mesh}$ then $F \vee \upset A$ contains $A$ and is a proper filter by \ref{joinProperFilter}, so $\lim_\xi F \subseteq \lim_x F \vee \upset A \subseteq \bigcup_{A\in G}\lim_\xi G$.

For the other inclusion: $A\in G$ iff $\upset\{A\} \subseteq G$, so $G = G \vee \upset A$. But $G$ is a proper filter, so it must mesh with $\upset A$. In particular it must mesh with $A$.
\end{proof}



\begin{proposition}
Let $X$ be a set, $\xi$ a convergence on $X$ and $A \subseteq X$. Then the following are equivalent:
\begin{enumerate}
\item $x\in \adh_\xi A$;
\item $A\in \bigcup_{F\in {\lim_\xi}^{-1}(x)} F^{\mesh}$;
\item $A \mesh \vicinity_\xi(x)$.
\end{enumerate}
\end{proposition}
\begin{proof}
$(1) \Leftrightarrow (2)$ We have
\[ x\in \bigcup_{A\in F^{\mesh}}\lim_\xi F \iff \exists F: A \in F^{\mesh} \land x\in \lim_\xi F \iff A\in  \bigcup_{F\in {\lim_\xi}^{-1}(x)} F^{\mesh}. \]

$(2) \Leftrightarrow (3)$ We have, using \ref{orderReversingSimilarityLatticeOperations},
\[ A \mesh \vicinity_\xi(x) \iff A\in \left(\vicinity_\xi(x)\right)^{\mesh} = \left(\bigcap_{F\in {\lim_\xi}^{-1}(x)}F\right)^{\mesh} = \bigcup_{F\in {\lim_\xi}^{-1}(x)} F^{\mesh}. \]
\end{proof}

\begin{lemma} \label{inherenceAdherenceInclusion}
Let $X$ be a set, $\zeta, \xi$ convergences on $X$ such that $\zeta \leq \xi$ and $A\subseteq X$ a subset. Then
\[ \inh_\xi(A) \subseteq \inh_\zeta(A) \subseteq A \subseteq \adh_\zeta(A) \subseteq \adh_\xi(A). \]
\end{lemma}
\begin{proof}
We only need to show the first and last inclusion.

From \ref{vicinityMapAntitone} we have $\vicinity_\xi(x) \subseteq \vicinity_\zeta(x)$ and $\vicinity_\zeta(x)^{\mesh} \subseteq \vicinity_\xi(x)^{\mesh}$ for all $x\in X$.

So we have the implications
\[ x\in \inh_\xi(A) \implies A\in \vicinity_\xi(x) \implies A\in \vicinity_\zeta(x) \implies x\in \inh_\zeta(A) \]
and
\[ x\in \adh_\zeta(A) \implies A\in \vicinity_\zeta(x)^{\mesh} \implies A\in \vicinity_\xi(x)^{\mesh} \implies x\in \adh_\xi(A). \]
\end{proof}

\subsection{General adherence (TODO)}
\begin{definition}
Let $X$ be a set, $\xi$ a convergence on $X$ and $\mathcal{A} \subseteq \powerset(X)$. The \udef{adherence} of $\mathcal{A}$ is defined as
\[ \adh_\xi\mathcal{A} \defeq \bigcup_{\substack{F \in \powerfilters(X) \\ F \mesh \mathcal{A}}} \lim_\xi F. \]
The \udef{(principal) adherence} of a set $A\subseteq X$ is the adherence of $\{A\}$.
\end{definition}
Clearly if $\xi,\zeta$ are convergences on $X$, then $\xi \leq \zeta$ implies $\adh_\xi\mathcal{A} \subseteq \adh_\zeta\mathcal{A}$.

\begin{lemma}
Let $X$ be a set, $\xi$ a convergence on $X$ and $\{A_i\}_{i\in I} \subseteq \powerset(X)$. Then
\[ \adh_\xi \{A_i\}_{i\in I} \subseteq \bigcap_{i\in I}\adh_\xi A_i. \]
\end{lemma}
\begin{proof}
We have $x\in \adh_\xi \{A_i\}_{i\in I}$ iff $\exists F: x\in \lim F \land F\amesh \{A_i\}_{i\in I}$ iff $\exists F: \forall i\in I: x\in \lim F \land F\amesh A_i$. This implies $\forall i\in I: \exists F: x\in \lim F \land F\amesh A_i$ iff $\forall i\in I: x\in \adh_\xi A_i$.
\end{proof}



\begin{lemma}
Let $X$ be a set and $F$ be a filter in $\powerfilters(X)$. Then $\ker F = \adh_{\iota_X} F$.
\end{lemma}
\begin{proof}
Let $x\in X$. Then $\pfilter{x} \amesh F$ iff $\Big\{\{x\}\Big\} \amesh F$ iff $x\in \ker F$.
\end{proof}
\begin{corollary}
Let $\xi$ be a convergence on a set $X$ and $F$ be a filter in $\powerfilters(X)$. Then $\ker F \subseteq \adh_\xi F$.
\end{corollary}
\begin{proof}
This follows from $\iota_X \leq \xi$.
\end{proof}

\subsection{Inherence}
\begin{definition}
Let $X$ be a set, $\xi$ a convergence on $X$ and $A \subseteq X$. The \udef{(principal) inherence} of $A$, denoted $\inh_\xi A$, is defined by
\[ x\in \inh_\xi A \defequiv A\in \vicinity_\xi(x). \]
\end{definition}

\begin{lemma}
Let $X$ be a set, $\xi$ a convergence on $X$ and $A \subseteq X$. Then
\begin{enumerate}
\item $\inh_\xi A = (\adh_\xi A^c)^c$;
\item $\adh_\xi A = (\inh_\xi A^c)^c$;
\item $x\in \inh_\xi A \iff (x\in \lim_\xi F \implies A\in F)$.
\end{enumerate}
\end{lemma}

\begin{proposition}
Let $X$ be a set, $\xi$ a convergence on $X$ and $A,B \subseteq X$. Then
\begin{enumerate}
\item $\inh_\xi X = X$;
\item if $A \subseteq B$, then $\inh_\xi A \subseteq \inh_\xi B$;
\item $\inh_\xi A\subseteq A$;
\item $\inh_\xi(A\cap B) = \inh_\xi A \cap \inh_\xi B$.
\end{enumerate}
\end{proposition}

\begin{lemma} \label{subsetWithVicinitiesInInherence}
Let $\sSet{X,\xi}$ be a convergence space and $A,B\subseteq X$ subsets. If for every $a\in A$ there exists a vicinity of $a$ that is a subset of $B$, then $A\subseteq \inh_\xi(B)$.
\end{lemma}
\begin{proof}
Assume that for every $a\in A$ there exists a $U_a \in \vicinity_\xi(a)$ such that $U_a \subseteq B$. Then $B\in \vicinity_\xi(a)$ for all $a$ in $A$ and thus $a\in \inh_\xi(B)$ for all $a\in A$.
\end{proof}

\subsection{Topology}
\subsubsection{Open and closed sets}
\begin{definition}
Let $X$ be a set and $\xi$ a convergence on $X$.
\begin{itemize}
    \item A subset $O \subseteq X$ is called \udef{open} if $\inh_\xi(O) = O$.
    \item A subset $C \subseteq X$ is called \udef{closed} if $\adh_\xi(C) = C$.
\end{itemize}
The set of all open sets in $\sSet{X,\xi}$ is called the \udef{topology} of $\sSet{X,\xi}$ and is denoted $\topology_\xi$.
\end{definition}

\begin{lemma}
Let $X$ be a set, $\xi$ a convergence on $X$ and $A\subseteq X$ a subset. Then $A$ is open \textup{if and only if} $A^c$ is closed.
\end{lemma}
\begin{proof}
Assume $\inh_\xi(A) = A$. Then $A^c = \inh_\xi(A)^c = \adh_\xi(A^c)$.
\end{proof}

\begin{lemma} \label{completeClosureTopology}
Let $X$ be a set and $\xi$ a convergence on $X$. Then
\begin{enumerate}
\item the topology $\topology_\xi$ is closed under arbitrary unions;
\item the set of closed sets in $\sSet{X,\xi}$ is closed under arbitrary intersections.
\end{enumerate}
\end{lemma}
\begin{proof}
(1) Let $\{A_i\}_{i\in I}$ be a set of open sets. From \ref{inherenceAdherenceProperties} and \ref{orderPreservingFunctionLatticeOperations} we get $\inh\left(\bigcup_{i\in I}A_i\right) \subseteq \bigcup_{i\in I}A_i$ and $\bigcup_{i\in I}\inh(A_i) \subseteq \inh\left(\bigcup_{i\in I} A_i\right)$. Thus
\[ \bigcup_{i\in I}A_i  = \bigcup_{i\in I}\inh_(A_i) \subseteq \inh\left(\bigcup_{i\in I} A_i\right) \subseteq \bigcup_{i\in I}A_i. \]

(2) Let $\{A_i\}_{i\in I}$ be a set of closed sets. From \ref{inherenceAdherenceProperties} and \ref{orderPreservingFunctionLatticeOperations} we get $\bigcap_{i\in I}A_i \subseteq \adh\left(\bigcap_{i\in I}A_i\right)$ and $\adh\left(\bigcap_{i\in I} A_i\right)\subseteq \bigcap_{i\in I}\adh(A_i)$. Thus
\[ \bigcap_{i\in I}A_i \subseteq \adh\left(\bigcap_{i\in I}A_i\right) \subseteq \bigcap_{i\in I}\adh(A_i) = \bigcap_{i\in I}A_i. \]
\end{proof}
\begin{corollary}
The topology $\topology_\xi$ is a complete sublattice of $\powerset(X)$.
\end{corollary}

\begin{lemma} \label{openClosedCriteria}
Let $\sSet{X,\xi}$ be a convergence space and $O,C\subseteq X$ subsets. The following are equivalent:
\begin{enumerate}
\item $O$ is open;
\item $O\subseteq \inh(O)$;
\item $O\subseteq \interior(O)$;
\item $O\in \vicinity(O)$;
\item $O\in \neighbourhood(O)$;
\item for all $x\in O$ there exists $U_x\in \vicinity(x)$ such that $U_x\subseteq O$;
\end{enumerate}
as are the following:
\begin{enumerate}
\item $C$ is closed;
\item $\adh(C)\subseteq C$;
\item $\closure(C)\subseteq C$.
\end{enumerate}
\end{lemma}
\begin{proof}
TODO \ref{subsetWithVicinitiesInInherence}
\end{proof}

\begin{lemma}
Let $X$ be a set, $\zeta, \xi$ convergences on $X$ such that $\zeta\leq\xi$ and $A\subseteq X$ a subset.
\begin{enumerate}
\item if $A$ is open in $\xi$, then it is also open in $\zeta$;
\item if $A$ is closed in $\xi$, then it is also closed in $\zeta$.
\end{enumerate}
\end{lemma}
\begin{proof}
Assume $A$ is open in $\xi$. Then $\inh_\xi(A) = A$. But $\inh_\xi(A) \subseteq \inh_\zeta(A) \subseteq A$ by \ref{inherenceAdherenceInclusion}, so $\inh_\xi(A) = \inh_\zeta(A) = A$. The argument for closedness is similar.
\end{proof}

\begin{lemma}
Let $\sSet{X,\xi}$ be a convergence space. All subsets of $X$ are closed \textup{if and only if} $\xi$ is discrete.
\end{lemma}
\begin{proof}
First assume not all subsets are closed. Take $A\subseteq X$ such that $A \subsetneq \adh_\xi(A)$ and take $x\in \adh_\xi(A)\setminus A$. Then there exists $F\to x$ such that $A\in F^\mesh$ by \ref{adherenceInherenceCharacterisation}. Now $A$ does not mesh with $\{x\}$, so $\{x\} \notin F$, which means that $F \neq \pfilter{x}$ and thus that $\xi$ is not discrete.

Now assume $\xi$ is not discrete. Then there exists $F\to x$ such that $F \neq \pfilter{x}$. Take $A\in F$ and set $B = A\setminus\{x\}$. Now $A\in F^\mesh$, so for all $C\in F$, either $B\mesh C$ or $A\cap C = \{x\}$. In the second case, $\{x\}\in F$, so $F = \pfilter{x}$, which was excluded. Thus $B\in F^\mesh$. This means that $x\in \adh_\xi(B)$, but $x\notin B$, so $B$ is not closed. 
\end{proof}
\begin{corollary} \label{discreteIntersectionAdherence}
Let $\sSet{X,\xi}$ be a convergence space. The following are equivalent:
\begin{itemize}
\item $\xi$ is discrete;
\item $\adh_\xi(A\cap B) = \adh_\xi A \cap \adh_\xi B$ for all $A,B\subseteq X$.
\end{itemize}
\end{corollary}
\begin{proof}
If $\xi$ is discrete, then all sets are closed, so
\[ \adh_\xi(A\cap B) = A\cap B = \adh_\xi A \cap \adh_\xi B. \]
Now assume (2). If all subsets are closed, then $\xi$ is discrete and we are done. If not, we can take a non-closed $A\subseteq X$, so that $x\in \adh_\xi(A)\setminus A$. Then
\[ x\in \adh_\xi A \cap \adh_\xi \{x\} = \adh_\xi(A\cap\{x\}) = \adh_\xi(\emptyset) = \emptyset, \]
which is a contradiction.
\end{proof}

\subsubsection{Interior, closure and boundary}
\begin{definition}
Let $\sSet{X,\xi}$ be a convergence space.
\begin{itemize}
\item The dual closure mapping of $A\in \powerset(X)$ into $\topology_\xi$ is called the \udef{interior} of $A$, denoted $\interior_\xi(A)$ or $A^\circ$. 
\item The closure mapping of $A\in \powerset(X)$ into the set of closed sets in $X$ is called the \udef{closure} of $A$, denoted $\closure_\xi(A)$ or $\overline{A}^\xi$. 
\end{itemize}
The \udef{boundary} of $A\in\powerset(X)$ is $\partial A \defeq \overline{A}\setminus A^{\circ}$
\end{definition}

\begin{lemma}
$\interior^2 = \interior$ and $\closure^2 = \closure$.
\end{lemma}

\begin{proposition}
???

$\interior(\closure(A))$ and $\closure(\interior(A))$?
\end{proposition}

\subsubsection{Neighbourhoods}
\begin{definition}
Let $\sSet{X,\xi}$ be a convergence space and $x\in X$. We call a subset $A\subseteq X$ a \udef{neighbourhood} of $x$ is there exists an open set $O$ such that $x\in O \subseteq A$.

The set of all neighbourhoods of $x$ is denoted $\neighbourhood_\xi(x)$.
\end{definition}

\begin{proposition} \label{interiorClosureMembership}
Let $\sSet{X,\xi}$ be a convergence space, $A\subseteq X$ and $x\in X$. Then
\begin{enumerate}
\item $x\in \interior_\xi(A) \iff A \in \neighbourhood_\xi(x)$;
\item $x\in \closure_\xi(A) \iff A \in \neighbourhood_\xi(x)^{\mesh}$.
\end{enumerate}
\end{proposition}

\begin{lemma} \label{interiorModificationNeighbourhoods}
Let $\sSet{X,\xi}$ be a convergence space, $A\subseteq X$ and $x\in X$. Then
\begin{enumerate}
\item $A\in\neighbourhood(x)$ \textup{if and only if} $\interior(A)\in\neighbourhood(x)$;
\item if $\mathcal{A}\in\powerset^2(X)$ is a base for $\neighbourhood(x)$, then $\interior^{\imf}(\mathcal{A})$ is also a base for $\neighbourhood(x)$.
\end{enumerate}
\end{lemma}
\begin{proof}
(1) We have $A\in\neighbourhood(x) \iff x\in \interior(A) = \interior^2(A) \iff \interior(A) \in \neighbourhood(x)$.

(2) First $\interior^{\imf}(\mathcal{A})$ is a filter base, because it is closed under finite intersections: for all $\interior(A),\interior(B)\in \interior^{\imf}(\mathcal{A})$ we have $\interior(A)\cap\interior(B) = \interior(A\cap B) \in \interior^{\imf}(\mathcal{A})$ because $A\cap B \in \mathcal{A}$.

We clearly have $\mathcal{A} \preceq \interior^{\imf}(\mathcal{A})$ because $\interior(A) \subseteq A$. For the opposite inequality, we need to show that for all $A\in \mathcal{A}$, $\interior(A)$ is a neighbourhood of $x$. This is point (1).
\end{proof}

\subsubsection{Topological convergence}
\begin{definition}
A pretopological convergence space $\sSet{X,\xi}$ is called \udef{topological} if the topology $\topology_\xi$ is a base of $\xi$.
\end{definition}
\begin{definition}
A pretopological convergence space $\sSet{X,\xi}$ is called \udef{topological} if $\adh_\xi^2 = \adh_\xi$.
\end{definition}
TODO pretopological assumption necessary?

\begin{proposition} \label{pretopologicalSpaceTopological}
Let $\sSet{X,\xi}$ be a pretopological convergence space. The following are equivalent:
\begin{enumerate}
\item $\xi$ is topological;
\item $\inh_\xi^2 = \inh_\xi$;
\item $\adh_\xi = \closure_\xi$;
\item $\inh_\xi = \interior_\xi$;
\item $\vicinity_\xi(x) = \neighbourhood_\xi(x)$ for all $x\in X$;
\item $\forall U\in \vicinity_\xi(x): \exists V\in \vicinity_\xi(x): \forall y\in V: U\in\vicinity_\xi(y)$.
\end{enumerate}
\end{proposition}
\begin{proof}
TODO
\end{proof}

\begin{proposition}
Let $\sSet{X,\xi}$ be a topological convergence space. Then $\xi$ is topological \textup{if and only if} $\neighbourhood_\xi(x) \overset{\xi}{\longrightarrow} x$ for all $x\in X$.
\end{proposition}

\subsection{Dense sets}
\begin{definition}
Let $\sSet{X, \xi}$ be a convergence space and $A$ a subset. The subset $A$ is called \udef{dense} in $X$ if $\adh_\xi(A) = X$.
\end{definition}

\subsubsection{Strict density}
TODO strict density (Colebunders)

\begin{lemma} \label{openDensityLemma}
Let $\sSet{X, \xi}$ be a convergence space and $A$ a subset.
If $A^c$ contains a non-empty open set, then $A$ is not dense,
\end{lemma}
\begin{proof}
If $A$ is dense, then $\adh(A) = \inh(A^c)^c = X$, so $\inh(A^c) = \emptyset$. If $A^c$ contains an open set $B$, then $\inh(A^c) \supseteq \inh(B) = B$. This means that $\inh(A^c) \neq \emptyset$ and thus that $A$ is not dense.
\end{proof}

\subsection{Accumulation points}
\begin{definition}
Let $\sSet{X,\xi}$ be a convergence space and $\mathcal{A}\subseteq \powerset(X)$ a family of subsets. A point $x\in X$ is called an \udef{accumulation point} of $\mathcal{A}$ if $\mathcal{A}\amesh \vicinity_\xi(x)$.
\end{definition}

TODO $x\in \adh_\xi(A\setminus\{x\})$.

\begin{proposition} \label{subfilterToAccumulationPoint}
Let $\sSet{X,\xi}$ be a pretopological convergence space, $F\in\powerfilters(X)$ and $x\in X$. If $x$ is an accumulation point of $F$, then there exists a proper filter $G \geq F$ such that $G\overset{\xi}{\longrightarrow} x$. 
\end{proposition}
\begin{proof}
By \ref{joinProperFilter}, we have that $F\cup \vicinity_\xi(x)$ is a proper filter. We can take this filter to be $G$.
\end{proof}

\subsection{Cover}
\begin{definition}
Let $X$ be a set, $\xi$ a convergence on $X$, $A\subseteq X$ and $\mathcal{A}\subseteq \powerset(X)$.
We say $\mathcal{A}$ is an \udef{$\xi$-cover} (or simply \udef{cover}) of $A$ if every filter converging to a point in $A$ contains an element of $\mathcal{A}$. We write $\mathcal{A} \succ_\xi A$.
\end{definition}
So we have
\[ \mathcal{A} \succ_\xi A \;\iff\; \forall F\in \powerfilters(X): \Big( \lim_\xi F \amesh A \implies F \mesh \mathcal{A} \Big). \]

\begin{proposition}
Let $X$ be a set, $\xi$ a convergence on $X$, $A\subseteq X$ and $\mathcal{A}\subseteq \powerset(X)$. Then
\[ \mathcal{A} \succ_\xi A \quad\iff\quad \adh_\xi [\mathcal{A}]^c \perp A \]
\end{proposition}


\section{Pointwise properties}
\begin{definition}
A property $\mathbf{P}$ of preconvergences is called \udef{pointwise} if there exists a property $\mathbf{Q}$ of classes of filters such that for all preconvergences $\xi$ on $X$,
\[ \xi \in \mathbf{P} \iff \forall x\in X: {\lim_\xi}^{-1}(x) \in \mathbf{Q}. \]
\end{definition}

\subsection{Isolation and primeness}
\begin{definition}
Let $\sSet{X,\xi}$ be a convergence space.
\begin{itemize}
\item A point $x\in X$ is called \udef{isolated} if ${\lim_\xi}^{-1}(x) = \Big\{ \pfilter{x} \Big\}$.
\item The convergence space $\sSet{X,\xi}$ is called \udef{prime} if it contains at most one non-isolated point. Such a point is called \udef{distinguished}.
\end{itemize}
\end{definition}

A convergence space is discete if and only if each point is isolated. A discrete convergence space is prime.


\section{Examples of (pre)convergences}
\subsection{Preconvergences on two-point sets}
\subsection{Preconvergences on thee-point sets}

\subsection{Convergences on ordered sets}
\subsubsection{Ordering of filters}
\begin{definition}
Let $\sSet{X,\leq}$ be a poset and $F,G\in\powerfilters(X)$. We define
\begin{align*}
F \leq_w G \qquad&\defequiv\qquad \liminf F \leq \limsup G; \\
F \leq_s G \qquad&\defequiv\qquad \limsup F \leq \liminf G.
\end{align*}
We call $\leq_w$ the \udef{weak ordering} of filters and $\leq_s$ the \udef{strong ordering} of filters.
\end{definition}

\begin{lemma} \label{filterInequalityCriterion}
Let $\sSet{X,\leq}$ be a poset and $F,G\in\powerfilters(X)$. Then
\begin{align*}
F \leq_w G \qquad &\iff\qquad \forall A\in F, B\in G: \exists x\in A, y\in B: \; x \leq y; \\
F \leq_s G \qquad &\iff\qquad \exists A\in F, B\in G: \forall x\in A, y\in B: \; x \leq y.
\end{align*}
\end{lemma}
\begin{proof}
We have the equivalences
\begin{align*}
F \leq_w G &\iff \forall A\in F, B\in G: \exists x\in A, y\in B: \; x \leq y \\
&\iff \forall A\in F, B\in G: \bigwedge A \leq \bigvee B \\
&\iff \bigvee_{A\in F}\bigwedge A \leq \bigwedge_{B\in G}\bigvee B \\
&\iff \liminf F \leq \limsup G.
\end{align*}
The other equivalence is proved similarly.
\end{proof}

\begin{lemma}
Let $\sSet{X,\leq}$ be a poset and $F,G\in\powerfilters(X)$. Then
\begin{align*}
F \leq_w G \qquad &\iff\qquad \forall C\in F\otimes G: \exists (x,y) \in A: \; x \leq y; \\
F \leq_s G \qquad &\iff\qquad \exists C\in F\otimes G: \forall (x,y) \in A: \; x \leq y.
\end{align*}
\end{lemma}
\begin{proof}
First assume $F\leq_w G$. Take $C\in F\otimes G$. Then there exist $A\in F, B\in G$ such that $A\times B \subseteq C$. By assumption, we can find $x\in A$ and $y\in B$ such that $x\leq y$. In this case $(x,y)\in C$.

For the converse, take $A\in F$ and $B\in G$. Then $A\times B\in F\otimes G$ and by assumption there exists $(x,y)\in A\times B$ such that $x\leq y$.

TODO $\leq_s$
\end{proof}
TODO contours for propositions.


\subsubsection{Convergence compatible with the order}
\begin{definition}
Let $\sSet{X,\leq}$ be a poset. A convergence $\xi$ in $X$ is called
\begin{itemize}
\item \udef{compatible with the weak order} if $F \leq_w G$, $x\in\lim_\xi F$ and $y\in \lim_\xi G$ imply $x\leq y$;
\item \udef{compatible with the strong order} if $F \leq_s G$, $x\in\lim_\xi F$ and $y\in \lim_\xi G$ imply $x\leq y$.
\end{itemize}
\end{definition}


\begin{lemma}
Let $\sSet{X,\leq}$ be a poset and $\xi$ a convergence on $X$. Then
\begin{enumerate}
\item $\xi$ is compatible with the weak order \textup{if and only if} $\lim_\xi F\subseteq \{\liminf F\}\cap\{\limsup F\}$ for all $F\in\powerfilters(X)$;
\item $\xi$ is compatible with the strong order \textup{if and only if} $\lim_\xi F\subseteq \interval{\liminf F, \limsup F}$ for all $F\in\powerfilters(X)$.
\end{enumerate}
\end{lemma}
\begin{proof}
We have $F \leq \pfilter{x} \iff \liminf F \leq x$ by \ref{filterInequalityCriterion}. This implies 
$F \leq \pfilter{\liminf F}$ and thus for all $y\in\lim_\xi F$ we must have $y\leq \liminf F$ by compatability of the order. By a similar argument, we have $\limsup F \leq y$.
\end{proof}
\begin{corollary} \mbox{}
The weakest convergence compatible with the weak order is given by $\lim F = \liminf F =\limsup F$, if they coincide, for all $F\in \powerfilters(X)$.
\end{corollary}
TODO: there is no weakest convergence compatible with the strong order??

\subsubsection{Order closure and order regularity}
\begin{definition}
Let $\sSet{X, \leq}$ be a poset and $A\subseteq X$ a subset. The \udef{order closure} of $A$ is the set
\[ \closure_o(A) \defeq \setbuilder{x\in X}{\exists a,b\in A: a\leq x\leq b}. \]
We call a convergence $\xi$ on $X$ \udef{order-regular} if for all $F\in \powerfilters(X)$, the filter
\[ \setbuilder{\closure_o(A)}{A\in F} \]
converges to the same points as $F$.
\end{definition}
Order closure is closure into the lattice of (closed) intervals TODO. $\interval{a,b}\vee \interval{c,d} = \interval{(a\wedge c), (b\vee d)}$ and $\interval{a,b}\wedge \interval{c,d} = \interval{(a\vee c), (b\wedge d)}$.


\begin{lemma}
A convergence is order-regular \textup{if and only if} it is based in the closed intervals.
\end{lemma}

\begin{proposition}
Let $\sSet{X,\leq}$ be a poset with order-regular convergence $\xi$ and $F,G,H\in \powerfilters(X)$. If $F,G\overset{\xi}{\longrightarrow} x\in X$ and
\[ \forall A\in F,B\in G: \exists C\in H: \;\exists a\in A, b\in B:\; C\subseteq \interval{a,b},  \]
then $H\overset{\xi}{\longrightarrow} x$.
\end{proposition}
\begin{proof}
TODO
\end{proof}
\begin{corollary}[Squeeze theorem]
Let $\seq{x_n}, \seq{y_n}$ and $\seq{z_n}$ be sequences in $X$ such that $\seq{x_n}, \seq{y_n}\overset{\xi}{\longrightarrow} x\in X$ and
\[ \forall n\in \N: \quad x_n \leq z_n \leq y_n, \]
then $\seq{y_n} \overset{\xi}{\longrightarrow} x$.
\end{corollary}

\subsubsection{Order convergence}
\url{https://core.ac.uk/download/pdf/82382859.pdf}
TODO move down!

\begin{definition}
Let $\sSet{P,\leq}$ be a poset and $F\in\powerfilters(P)$. Then $F$ converges to $x$ in the \udef{order convergence} on $P$ if there exist nets $\seq{l_i}_{i\in I}$ and $\seq{u_i}_{i\in I}$ in $P$ such that
\begin{itemize}
    \item $\seq{l_i}_{i\in I}$ is increasing and $\sup_{i\in I} l_i = x$;
    \item $\seq{u_i}_{i\in I}$ is decreasing and $\inf_{i\in I} u_i = x$;
    \item $\setbuilder{[l_i, u_i]}{i\in I} \subseteq F$.
\end{itemize}
We say $F$ is \udef{bounded by} $\seq{l_i}_{i\in I}$ and $\seq{u_i}_{i\in I}$.
\end{definition}

TODO: initial or final??

\begin{example}
Order convergence not necessarily topological.
\end{example}


\begin{definition}
Let $\sSet{P,\leq}$ be a poset equipped with order convergence, $\sSet{X,\xi}$ a convergence space, $x_0\in P$ and $f: P\to X$ a function. Then
\begin{itemize}
\item $f$ is called \udef{left continuous} at $x_0$ if $f|_{\downset x_0}$ is continous at $x_0$;
\item $f$ is called \udef{right continuous} at $x_0$ if $f|_{\upset x_0}$ is continous at $x_0$.
\end{itemize}
\end{definition}

\begin{proposition} \label{leftRightConvergence}
Let $\sSet{P,\leq}$ be a poset equipped with order convergence, $\sSet{X,\xi}$ a convergence space of finite depth and $f: P\to X$ a function. Then $f$ is continuous at $x_0\in P$ \textup{if and only if} it is left and right continuous at $x_0$.
\end{proposition}
\begin{proof}
If $f$ is continuous at $x_0\in P$, then it is left and right continuous at $x_0$ by \ref{continuityRestrictionExpansion}.

Conversely, assume $f$ is left and right continuous and take $F$ bounded by $\seq{l_i}_{i\in I}$ and $\seq{u_i}_{i\in I}$ such that $F\to x_0$. Define $F_0 = \mathfrak{F}\setbuilder{[l_i, u_i]}{i\in I} = \upset \setbuilder{[l_i, u_i]}{i\in I}$ and note that for any $A\in F_0$, there exists an $i\in I$ such that $[l_i,u_i]\subseteq A$.


Then $F_0|_{\downset x_0}$ is bounded by $\seq{l_i}_{i\in I}$ and $\seq{x_0}$ and $F_0|_{\upset x_0}$ is bounded by $\seq{x_0}$ and $\seq{u_i}_{i\in I}$. So $f[F_0|_{\downset x_0}] \to f(x_0)$ and $f[F_0|_{\upset x_0}] \to f(x_0)$, meaning $f[F_0|_{\downset x_0}]\cap f[F_0|_{\upset x_0}] \to f(x_0)$ by finite depth.

We conclude by showing that $f[F_0|_{\downset x_0}]\cap f[F_0|_{\upset x_0}] \preceq f[F]$. To that end, take $A\in f[F_0|_{\downset x_0}]\cap f[F_0|_{\upset x_0}]$. We need to show that there is a subset of $A$ in $f[F]$.

Indeed, let $A_1\in F_0|_{\downset x_0}$ be such that $f[A_1] = A$ and $A_2\in F_0|_{\upset x_0}$ be such that $f[A_2] = A$.
Then there exist $i,j\in I$ such that $[l_i,x_0]\subseteq A_1$ and $[x_0, u_j]\subseteq A_2$. Set $k = \max\{i,j\}$.

Now $f\Big[[l_k,u_k]\Big] = f\Big[[l_k,x_0]\Big] \cup f\Big[[x_0,u_k]\Big]\subseteq f[A_1]\cup f[A_2] = A$ and also $f\Big[[l_k,u_k]\Big]\in F$ by order convergence.
\end{proof}

\subsubsection{Scott convergence}

\begin{definition}
Let $L$ be a complete lattice.
\begin{itemize}
\item The \udef{Scott convergence}, or \udef{lower convergence}, $S_*$ on $L$ is defined by
\[ x \in \lim_{S_*} F \iff x \leq \bigvee_{A\in F}\bigwedge_{a\in A}a = \liminf F. \]

\item The \udef{upper convergence}, $S^*$ on $L$ is the Scott convergence on the dual $L^o$. This means it is defined by
\[ x \in \lim_{S^*} F \iff x \geq \bigwedge_{A\in F}\bigvee_{a\in A}a = \limsup F. \]
\item The convergence $S_* \wedge S^*$ is called \udef{convergence in order}.
\end{itemize}
\end{definition}
Note that $F$ is a filter in $\powerfilters(L)$, not a filter in $\filters(L)$.

\begin{proposition}
Convergence in order is Hausdorff.
\end{proposition}
\begin{proof}
This is due to the distributive inequality TODO ref.
\end{proof}

\chapter{Continuity}

\section{Mapping filters}

\begin{proposition} \label{preimageFilter}
Let $X,Y$ be sets, $F\in\powerfilters(X)$, $G\in\powerfilters(Y)$ and $f: X\to Y$ a function.
\begin{enumerate}
\item if $\im(f)\in G$, then there exists a $H\in\powerfilters(X)$ such that $G = \upset f^{\imf\imf}[H]$;
\item if $f^{\imf\imf}[F]\subseteq G$ , then there exists a $H\in\powerfilters(X)$ such that $G = \upset f^{\imf\imf}[H]$;
\item if $G$ is an ultrafilter, then the $H$ in the previous points may be taken to be an ultrafilter.
\end{enumerate}
\end{proposition}
\begin{proof}
TODO \ref{baseTraceFilter}
\end{proof}

\section{Continuous functions}
\begin{definition}
Let $\sSet{X,\xi}$ and $\sSet{Y,\zeta}$ be (pre)convergence spaces. A function $f: X\to Y$ is called \udef{continuous} if it preserves limits: for all $D \in \powerdirected(X)$ and $x\in X$:
\[ D \overset{\xi}{\longrightarrow} x \implies f^{\imf\imf}(D) \overset{\zeta}{\longrightarrow} f(x). \]

The set of all continuous functions $\sSet{X,\xi} \to \sSet{Y,\zeta}$ is denoted $\cont(\xi, \zeta)$ or $\cont(X,Y)$, if the convergence is clear. If $X=Y$, we also write $\cont(X)$.

If $\xi,\zeta$ are preconvergences we write $\cont_\text{pre}(\xi, \zeta)$ for the set of continuous functions.
\end{definition}
In other words, a function is continuous if it is relation-preserving as a function
\[ \sSet{X \cup \powerset^2(X), \overset{\xi}{\longrightarrow}} \to \sSet{Y \cup \powerset^2(Y), \overset{\zeta}{\longrightarrow}}. \]

Note that for filters $F\in\powerfilters(X)$, $f[F]$ is in general a directed set, but not necessarily a filter.

\begin{lemma}
Let $\sSet{X,\xi}$ and $\sSet{Y,\zeta}$ be (pre)convergence spaces. A function $f: X\to Y$ is continuous \textup{if and only if} for all $D \in \powerdirected(X)$
\[ f^{\imf}\left[\lim_\xi D \right] \subseteq \lim_\zeta f^{\imf\imf}[D]. \]
\end{lemma}
\begin{proof}
Immediate from \ref{relationPreserving}.
\end{proof}

\begin{lemma} \label{continuityComposition}
Let $\sSet{X,\xi}$, $\sSet{Y,\sigma}$ and $\sSet{Z,\zeta}$ be (pre)convergence spaces. If $f: X\to Y$ and $g: Y\to Z$ are continuous, then $g\circ f$ is continuous.
\end{lemma}
\begin{proof}
Let $F\to x\in X$. Then $f[F] \to f(x)$ by continuity of $f$ and $g[f[F]] \to g(f(x))$ by continuity of $g$. So $g\circ f$ is continuous.
\end{proof}

\begin{lemma} \label{finerCoarserContinuity}
Let $\sSet{X,\xi}, \sSet{Y,\zeta}$ be (pre)convergence spaces and $f\in \cont_\text{(pre)}(\xi, \zeta)$.
\begin{enumerate}
\item Let $\sigma$ be a (pre)convergence on $X$ such that $\sigma \leq \xi$, then $f\in \cont_\text{(pre)}(\sigma, \zeta)$.
\item Let $\tau$ be a (pre)convergence on $Y$ such that $\tau \geq \zeta$, then $f\in \cont_\text{(pre)}(\xi, \tau)$.
\end{enumerate}
\end{lemma}
\begin{proof}
(1) Let $F\overset{\sigma}{\longrightarrow} x \in X$, then $F\overset{\xi}{\longrightarrow} x$ (because $\sigma \leq \xi$), so $f[F]\overset{\zeta}{\longrightarrow} f(x)$, meaning $f\in \cont_\text{(pre)}(\sigma, \zeta)$.

(2) Let $F\overset{\xi}{\longrightarrow} x \in X$, then $f[F]\overset{\zeta}{\longrightarrow} f(x)$, so $f[F]\overset{\tau}{\longrightarrow} f(x)$  (because $\zeta \leq \tau$), meaning $f\in \cont_\text{(pre)}(\xi, \tau)$.
\end{proof}

\begin{proposition} \label{continuityVicinityFilter} \label{adherenceInherenceContinuity}
Let $\sSet{X,\xi}$ and $\sSet{Y,\zeta}$ be convergence spaces, $f: X\to Y$ a function, $A\subseteq X$ a subset and $x\in X$. Then the following are equivalent:
\begin{enumerate}
\item $\vicinity_\zeta(f(x)) \subseteq \upset f^{\imf\imf}[\vicinity_\xi(x)]$;
\item for all $U\in \vicinity_\zeta(f(x))$, there exists $V\in \vicinity_\xi(x)$ such that $f^\imf[V] \subseteq U$;
\item for all $U\in \vicinity_\zeta(f(x))$, $f^{\preimf}[U]\in \vicinity_\xi(x)$;
\item $f^{\preimf\imf}\vicinity_\zeta(f(x)) \subseteq \vicinity_\xi(x)$;
\item $f^\preimf[\adh_\zeta(A)] \supseteq \adh_\xi(f^\preimf[A])$;
\item $f^\imf[\adh_\xi(A)] \subseteq \adh_\zeta(f^\imf[A])$;
\item $f^\preimf[\inh_\zeta(A)] \subseteq \inh_\xi(f^\preimf[A])$.
\end{enumerate}
All these points hold if $f$ is continuous.
\end{proposition}
\begin{proof}
We first show that continuity of $f$ implies (1): We have
\begin{align*}
\vicinity_\zeta(f(x)) &= \bigcap\setbuilder{G\in\powerfilters(Y)}{G\to f(x)} \\
&\subseteq \bigcap\setbuilder{\upset f^{\imf\imf}(F)\in\powerfilters(Y)}{F\to x} \\
&= \bigcap (\upset\circ f^{\imf\imf})^{\imf}\Big[\setbuilder{F\in\powerfilters(X)}{F\to x}\Big] \\
&= \upset f^{\imf\imf}\Big[\bigcap \setbuilder{F\in\powerfilters(X)}{F\to x}\Big] = \upset f^{\imf\imf}[\vicinity_\xi(x)],
\end{align*}
where we have used \ref{imageFiltersPreservesIntersection}.

Now take $B\in \bigcap\setbuilder{\upset f^{\imf\imf}(F)\in\powerfilters(Y)}{F\to x}$. This means that for all $F\in\lim^{-1}_\xi(x)$, there exists an $A_F\in F$ such that $f^\imf[A_F]\subseteq B$. By the upward closure of the filters $F$, the set $A = \bigcup_{F\in \lim^{-1}_\xi(x)}A_F$ is an element of each $F$ and
\[ f^\imf[A] = \bigcup_{F\in \lim^{-1}_\xi(x)}f^{\imf}[A_F] \subseteq B.  \]
Now $A\in \vicinity_\xi(x)$ and thus $B\in \upset f^{\imf\imf}[\vicinity_\xi(x)]$.


$(1) \Leftrightarrow (2)$ Reformulation.

$(2) \Leftrightarrow (3)$ We get (3) from (2) by noting that $V\subseteq f^{\preimf}f^\imf(V) \subseteq f^\preimf(U)$ and that $\vicinity_\xi(x)$ is a filter, so this implies $f^\preimf(U)\in \vicinity_\xi(x)$.

Conversely, we may take $V = f^{\preimf}[U]$.

$(3) \Leftrightarrow (4)$ Reformulation.

$(4) \Rightarrow (5)$ We calculate
\begin{align*}
x\in \adh_\zeta(f^\preimf[A]) \iff& f^\preimf[A]\in \vicinity_\xi(x)^{\mesh} \subseteq \left(f^{\preimf\imf}\vicinity_\zeta(f(x))\right)^{\mesh} \\
\implies& f^\imf f^\preimf[A] \in f^{\imf\imf}\left(\left(f^{\preimf\imf}\vicinity_\zeta(f(x))\right)^{\mesh}\right) \subseteq \left(f^{\imf\imf} f^{\preimf\imf}\vicinity_\zeta(f(x))\right)^{\mesh} \subseteq \left(\vicinity_\zeta(f(x))\right)^{\mesh} \\
\implies& A \in \left(\vicinity_\zeta(f(x))\right)^{\mesh} \\
\iff& f(x)\in \adh_\zeta(A) \\
\iff& x\in f^{\preimf}(\adh_\zeta(A)).
\end{align*}
We have used that (TODO ref)
\begin{itemize}
\item $F\subseteq G$ implies $G^{\mesh} \subseteq F^{\mesh}$;
\item $f^{\imf\imf}(F^{\mesh}) \subseteq \left(f^{\imf\imf}(F)\right)^{\mesh}$;
\item $F^{\mesh}$ is a filter if $F$ is a directed set;
\item $f^\imf f^\preimf[A] \subseteq A$.
\end{itemize}

$(1) \Rightarrow (6)$ We calculate
\begin{align*}
y\in f^\imf(\adh(A)) \iff& \exists x\in \adh(A):\;f(x) = y\\
\iff& \exists x\in X: \; x\in \adh(A) \land f(x) = y\\
\iff& \exists x\in X: \; A\in\vicinity(x)^{\mesh}  \land f(x) = y \\
\implies& \exists x\in X: \; f(x) = y \land f^\imf(A)\in f^{\imf\imf}\left(\vicinity(x)^{\mesh}\right) \subseteq f^{\imf\imf}\left(\vicinity(x)\right)^{\mesh} \subseteq \vicinity(f(x))^{\mesh} \\
\iff& \exists x\in X: \; f(x) = y \land f(x)\in \adh\Big(f^\imf(A)\Big) \\
\implies& y\in \adh\Big(f^\imf(A)\Big).
\end{align*}

$(5) \Rightarrow (7)$ We calculate, using \ref{inherenceComplementAdherence},
\begin{align*}
f^\preimf[\inh_\xi(A)] &= f^\preimf[\adh_\xi(A^c)^c] \\
&= \im(f)\setminus f^{\preimf}[\adh_\xi(A^c)] \\
&\subseteq f^{\preimf}[\adh_\xi(A^c)]^c \\
&\subseteq \left(\adh_\zeta(f^{\preimf}[A^c])\right)^c \\
&\subseteq \left(\adh_\zeta(f^{\preimf}[A]^c)\right)^c \\
&= \inh_\zeta(f^{\preimf}[A]).
\end{align*}

$(6) \Rightarrow (7)$ We calculate
\begin{align*}
x\in f^\preimf[\inh_\xi(A)] \iff& f(x) \in \inh_\zeta(A) = \adh_\zeta(A^c)^c \subseteq \adh_\zeta\Big(f^\imf\circ f^{\preimf}(A^c)\Big)^c \\
\iff& f(x) \in \adh_\zeta\Big(f^\imf\big(f^{\preimf}(A)^c\big)\Big)^c \subseteq f^\imf\Big(\adh_\xi\Big(f^{\preimf}(A)^c\big)\Big)^c \\
\implies& x\in \adh_\xi\Big(f^{\preimf}(A)^c\big)^c = \inh_\xi\big(f^{\preimf}(A)\big).
\end{align*}

$(7) \Rightarrow (1)$ We calculate
\begin{align*}
B\in \vicinity_\zeta(f(x)) \iff& f(x) \in \inh_\zeta(B) \\
\implies& x\in f^{\preimf}\Big(\inh_\zeta(B)\Big) \subseteq \inh_\xi\Big(f^\preimf(B)\Big) \\
\implies& f^\preimf(B) \in \vicinity_\xi(x) \\
\implies& f^\imf\big(f^\preimf(B)\big) \in f^{\imf\imf}[\vicinity_\xi(x)] \\
\implies& B \in \upset f^{\imf\imf}[\vicinity_\xi(x)].
\end{align*}
\end{proof}
\begin{proof}[Alternative proof of (2), given continuity of $f$]
We calculate
\begin{align*}
f[\adh_\xi(A)] &= f\left[\bigcup \setbuilder{\lim_\xi F}{A \lhd F} \right] = \bigcup f\left[ \setbuilder{\lim_\xi F}{A \lhd F} \right] = \bigcup \setbuilder{f\left[\lim_\xi F\right]}{A \lhd F} \\
&\subseteq \bigcup \setbuilder{\lim_\zeta f[F]}{A \lhd F} \subseteq \bigcup \setbuilder{\lim_\zeta f[F]}{f[A] \lhd f[F]} \subseteq \bigcup \setbuilder{\lim_\zeta G}{f[A] \lhd G} \\
&= \adh_\zeta(f[A]),
\end{align*}
where we have used that $A \lhd B$ implies $f[A] \lhd f[B]$ (TODO ref).
\end{proof}
\begin{corollary} \label{pretopologicalContinuityVicinities}
If $\zeta$ is pretopological, then any of these points implies that $f$ is continuous at $x$.
\end{corollary}
\begin{proof}
We prove that in this case (1) implies the continuity of $f$. Take $F\to x$. Then $F\supseteq \vicinity_\xi(x)$ and thus $f[F]\supseteq f[\vicinity_\xi(x)]$. By assumption this means $\upset f[F]\supseteq \upset f[\vicinity_\xi(x)]\supseteq \vicinity_\zeta(f(x))$. So $f[F]\to f(x)$.
\end{proof}

If $\xi$ is pretopological, we can also give a simplified proof that the continuity of $f$ implies (1): We have that $\vicinity_\xi(x)$ converges to $x$, so by continuity $f[\vicinity_\xi(x)]$ converges to $f(x)$ and thus $\upset f[\vicinity_\xi(x)] \supseteq \vicinity_\zeta(f(x))$ by the pretopological property.
\begin{corollary} \label{preimageOpenClosed}
Let $\sSet{X,\xi}$ and $\sSet{Y,\zeta}$ be convergence spaces and $f: X\to Y$ a continuous function. Then
\begin{enumerate}
\item if $O\subseteq Y$ is open, then $f^\preimf(O)$ is open;
\item if $C\subseteq Y$ is closed, then $f^\preimf(C)$ is closed.
\end{enumerate}
\end{corollary}
\begin{proof}
(1) We use \ref{openClosedCriteria}, so take $x\in f^\preimf(O)$. Then $f(x)\in O$ and by \ref{openClosedCriteria} there exists a $U_{f(x)}\in \vicinity_\zeta\big(f(x)\big)$ such that $U_{f(x)} \subseteq O$. Then $f^\preimf(U_{f(x)}) \subseteq f^\preimf(O)$ and $f^\preimf(U_{f(x)})\in \vicinity_\xi(x)$.

(2) We use the fact that $C^c$ is open and \ref{imagePreimageUniqueness} to calculate
\[ f^\preimf(C) = f^\preimf(Y\setminus C^c) = f^\preimf(Y)\setminus f^\preimf(C^c) = X\setminus f^\preimf(C^c) = f^\preimf(C^c)^c. \]
This is closed because $f^\preimf(C^c)$ is open by (1).
\end{proof}

\begin{lemma} \label{identityContinuity}
Let $\xi$ and $\zeta$ be two convergences on the same set $X$. Then $\id_X: \sSet{X,\xi} \to \sSet{X,\zeta}$ is continuous \textup{if and only if} $\xi \leq \zeta$. I.e.\ $\xi$ is finer than $\zeta$.
\end{lemma}
\begin{proof}
This is essentially a restatement of definitions:
\begin{align*}
\text{$\id_X: \sSet{X,\xi} \to \sSet{X,\zeta}$ is continuous} &\iff \forall F\in\powerfilters(X): \id_X\left[\lim_\xi F \right] \subseteq \lim_\zeta \id_X[F] \\
&\iff \forall F\in\powerfilters(X): \lim_\xi F \subseteq \lim_\zeta F \\
&\iff \xi \leq \zeta.
\end{align*}
\end{proof}

\begin{lemma} \label{continuityConstructions}
Let $\sSet{X,\xi}$ and $\sSet{Y,\zeta}$ be a convergence space.
\begin{enumerate}
\item \textup{(Identity function)} The identity function $\id_X:X\to X$ is continuous.
\item \textup{(Constant function)} For all $y$ in $Y$, the constant function $\underline{y}: X \to Y$ is continuous.
\end{enumerate}
\end{lemma}
\begin{proof}
(1) Let $F\to x \in X$. Then $x\in \lim_\xi(\id_X[F]) = \lim_\xi(F)$.

(2) Let $F\to x \in X$. Then $\underline{y}[F] = \pfilter{y} \to y = \underline{y}(x)$.
\end{proof}

\begin{lemma} \label{inverseImageContinuity}
Let $\sSet{X,\xi}$ and $\sSet{Y,\zeta}$ be preconvergence spaces and $f: X\to Y$ a function. Then $f$ is continuous \textup{if and only if} $\forall F\in \powerfilters(Y)$ and $y\in Y$
\[ \exists x\in f^{\preimf}[y]: \; f^{\preimf\imf}[F] \overset{\xi}{\longrightarrow} x \quad\implies\quad F \overset{\zeta}{\longrightarrow} y. \]
\end{lemma}
\begin{proof}
For $\Rightarrow$, assume $f$ continuous and that there exists $x\in f^{\preimf}[y]$ such that $f^{\preimf\imf}[F] \to x$. By continuity we have $f^{\imf\imf}[f^{\preimf\imf}[F]] \to y$. Now $f^{\imf\imf}[f^{\preimf\imf}[F]] \leq F$ by \ref{relationTimesTransposeSubsetIdentity}, so $F \overset{\zeta}{\longrightarrow} y$.

For $\Leftarrow$, take arbitrary $G \overset{\xi}{\longrightarrow} x \in X$. Then $f^{\preimf\imf}[f^{\imf\imf}[G]] \geq G$ by \ref{totalityEquivalences}, so $f^{\preimf\imf}[f^{\imf\imf}[G]] \overset{\xi}{\longrightarrow} x$ and $x \in f^{\preimf}[\{f(x)\}]$. This means $f^{\imf\imf}[G] \overset{\zeta}{\longrightarrow} f(x)$ and thus $f$ is continuous.
\end{proof}

\begin{proposition} \label{continuityUnderConvergenceLatticeOperations}
Let $\sSet{X,\xi}$ and $\sSet{Y,\zeta}$ be convergence spaces. Let $\Xi$ be a set of convergences on $X$ and $Z$ a set of convergences on $Y$. Then
\begin{enumerate}
\item $\cont\left(\xi, \bigwedge Z\right) = \bigcap_{\sigma\in Z} \cont(\xi, \sigma)$;
\item $\cont\left(\bigvee \Xi, \zeta\right) = \bigcap_{\sigma\in \Xi} \cont(\sigma, \zeta)$;
\item $\cont\left(\bigwedge \Xi, \zeta\right) \supseteq \bigcup_{\sigma\in Z} \cont(\xi, \sigma)$;
\item $\cont\left(\xi, \bigvee Z\right) \supseteq \bigcup_{\sigma\in \Xi} \cont(\sigma, \zeta)$;
\end{enumerate}
\end{proposition}
\begin{proof}
(1) We calculate, using \ref{latticeConvergences}:
\begin{align*}
f \in \cont\left(\xi, \bigwedge Z\right) &\iff f\left[\lim_\xi F \right] \subseteq \lim_{\bigwedge Z} \upset f[F] = \bigcap_{\sigma\in Z}\lim_\sigma \upset f[F] \\
&\iff \forall \sigma\in Z: \; f\left[\lim_\xi F \right] \subseteq \lim_\sigma \upset f[F] \\
&\iff \forall \sigma\in Z: \; f \in \cont(\xi,\sigma) \\
&\iff f\in \bigcap_{\sigma\in Z}\cont(\xi,\sigma).
\end{align*}

(2) Similarly, we have
\begin{align*}
f \in \cont\left(\bigvee \Xi, \zeta\right) &\iff f\left[\lim_{\bigvee \Xi} F \right] = \bigcup_{\sigma\in\Xi}f\left[\lim_\sigma F \right] \subseteq \lim_{\zeta} \upset f[F] \\
&\iff \forall \sigma\in \Xi: \; f\left[\lim_\sigma F \right] \subseteq \lim_\zeta \upset f[F] \\
&\iff \forall \sigma\in \Xi: \; f \in \cont(\sigma, \zeta) \\
&\iff f\in \bigcap_{\sigma\in \Xi}\cont(\sigma, \zeta).
\end{align*}

(3) Now we have
\begin{align*}
f \in \cont\left(\bigwedge \Xi, \zeta\right) &\iff f\left[\lim_{\bigwedge \Xi} F \right] = \bigcap_{\sigma\in\Xi}f\left[\lim_\sigma F \right] \subseteq \lim_{\zeta} \upset f[F] \\
&\impliedby \exists \sigma\in \Xi: \; f\left[\lim_\sigma F \right] \subseteq \lim_\zeta \upset f[F] \\
&\iff \exists \sigma\in \Xi: \; f \in \cont(\sigma, \zeta) \\
&\iff f\in \bigcup_{\sigma\in \Xi}\cont(\sigma, \zeta).
\end{align*}

(4) Finally we have
\begin{align*}
f \in \cont\left(\xi, \bigvee Z\right) &\iff f\left[\lim_\xi F \right] \subseteq \lim_{\bigvee Z} \upset f[F] = \bigcup_{\sigma\in Z}\lim_\sigma \upset f[F] \\
&\impliedby \exists \sigma\in Z: \; f\left[\lim_\xi F \right] \subseteq \lim_\sigma \upset f[F] \\
&\iff \exists \sigma\in Z: \; f \in \cont(\xi,\sigma) \\
&\iff f\in \bigcup_{\sigma\in Z}\cont(\xi,\sigma).
\end{align*}
\end{proof}

\subsubsection{Homeomorphisms}
\begin{definition}
Let $\sSet{X,\xi}$ and $\sSet{Y,\zeta}$ be convergence spaces. A function $f: X\to Y$ is called a \udef{homeomorphism} if
\begin{itemize}
\item $f$ is bijective;
\item both $f$ and $f^{-1}$ are continuous.
\end{itemize}
\end{definition}

\begin{proposition} \label{homeomorphismPreservation}
If $f$ homeomorphism, then
\begin{enumerate}
\item $f[\lim F] = \lim f[F]$;
\item $f[\vicinity(x)] = \vicinity(f[x])$;
\item $f[\adh(A)] = \adh(f[A])$.
\end{enumerate}
\end{proposition}


\subsection{Directional continuity}
\begin{definition}
Let $f: \sSet{X,\xi} \to \sSet{Z,\zeta}$ be a function between convergence spaces, $x\in X$ and $D\subseteq X$. We called $f$ \udef{directionally continuous} at $x$ in the direction $D$ if for all $F\in\powerfilters(X)$
\[ F \overset{\xi,D}{\longrightarrow} x \qquad \implies \qquad f[F] \overset{\zeta}{\longrightarrow} f(x). \]
\end{definition}

\begin{lemma}
Let $f: \sSet{X,\xi} \to \sSet{Z,\zeta}$ be a function between convergence spaces, $x_0\in X$ and $D$ a vicinity of $x_0$. Then $f$ is directionally continuous at $x_0$ in $D$ \textup{if and only if} $f$ is continuous at $x_0$.
\end{lemma}
\begin{proof}
TODO inherence
\end{proof}

\section{Initial and final convergences}
\begin{definition}
Let $Y$ be a set.
\begin{itemize}
\item Given a set of (pre)convergence spaces $\{\sSet{Z_i,\zeta_i}\}_{i\in I}$ and a set of functions $\{f_i: Y\to Z_i\}_{i\in I}$, we define the \udef{initial (pre)convergence} $\mu$ on $Y$ w.r.t. $\{f_i: Y\to Z_i\}$ as the coarsest (pre)convergence on $Y$ that makes all functions in $\{f_i: Y\to Z_i\}$ continuous:
\[ \mu = \bigvee \setbuilder{\sigma}{\forall i\in I: f_i\in  \cont_\text{(pre)}(\sigma, \zeta_i)}. \]
\item Given a set of convergence spaces $\{\sSet{X_i,\xi_i}\}_{i\in I}$ and a set of functions $\{g_i: X_i \to Y\}_{i\in I}$, we define the \udef{final convergence} $\nu$ on $Y$ w.r.t. $\{g_i: X_i\to Y\}$ as the finest convergence on $Y$ that makes all functions in $\{g_i: X_i \to Y\}$ continuous:
\[ \nu = \bigwedge \setbuilder{\sigma}{\forall i\in I: g_i\in  \cont_\text{(pre)}(\xi_i, \sigma)}. \]
\end{itemize}
\end{definition}

\begin{proposition} \label{initialFinalConvergenceModification}
Let $Y$ be a set.
\begin{enumerate}
\item Let $\{f_i: Y\to \sSet{Z_i, \zeta_i}\}_{i\in I}$ be set of functions to convergence spaces and $\mu$ the initial \emph{pre}convergence on $Y$ w.r.t. this set. Then $\mu$ is also the initial convergence w.r.t. $\{f_i: Y\to \sSet{Z_i, \zeta_i}\}$.
\item Let $\{g_i: \sSet{X_i, \xi_i} \to Y\}_{i\in I}$ be set of functions from convergence spaces and $\nu$ the final \emph{pre}convergence on $Y$ w.r.t. this set. Then $\nu \vee \iota_Y$ is the final convergence w.r.t. $\{g_i: \sSet{X_i, \xi_i} \to Y\}$.
\end{enumerate}
\end{proposition}
\begin{proof}
TODO convergence modification
\end{proof}

\begin{proposition} \label{initialFinalConvergence}
Let $Y$ be a set, $F\in \powerfilters(Y)$ and $y\in Y$.
\begin{enumerate}
\item Let $\{f_i: Y\to \sSet{Z_i, \zeta_i}\}_{i\in I}$ be set of functions to \emph{pre}convergence spaces and $\mu$ the initial \emph{pre}convergence on $Y$ w.r.t. this set. Then
\[ F \overset{\mu}{\longrightarrow} y \quad\iff\quad \forall i\in I: \; f_i[F] \overset{\zeta_i}{\longrightarrow} f_i(y). \]
\item Let $\{g_i: \sSet{X_i, \xi_i} \to Y\}_{i\in I}$ be set of functions from \emph{pre}convergence spaces and $\nu$ the final \emph{pre}convergence on $Y$ w.r.t. this set. Then
\[ F \overset{\nu}{\longrightarrow} y \quad\iff\quad \exists i\in I: \exists x\in g_i^{\preimf}\{y\}: \; g_i^{\preimf\imf}[F] \overset{\xi_i}{\longrightarrow} x. \]
\end{enumerate}
\end{proposition}
Note that point (1) still holds true for convergences, by \ref{initialFinalConvergenceModification}. In the convergence case, point (2) needs to be modified to 
\[ F \overset{\nu}{\longrightarrow} y \quad\iff\quad \exists i\in I: \exists x\in g_i^{\preimf}\{y\}: \; g_i^{\preimf\imf}[F] \overset{\xi_i}{\longrightarrow} x \;\;\lor\;\; F = \pfilter{y}. \]
\begin{proof}
(1) The direction $\Rightarrow$ is clear: $\mu$ makes all $f_i$ continuous by \ref{finerCoarserContinuity}.

For $\Leftarrow$, assume $F$ such that $\forall i\in I: \; f_i[F] \overset{\zeta_i}{\longrightarrow} f_i(y)$, but $F \overset{\mu}{\not\to} y$. Then define the preconvergence $\mu'$ with the same limits as $\mu$, but with the addition of $G\overset{\mu'}{\longrightarrow} y$ for all $G\geq F$. Now $\mu'$ makes all $f_i$ continuous (because $G\geq F$ implies $f_i[G] \geq f_i[F]$), so $\mu'\leq \mu$. Thus $F \overset{\mu}{\longrightarrow} y$, which is a contradiction.

(2) By \ref{finerCoarserContinuity}, $g_i$ is continuous for all $i\in I$. Then the direction $\Leftarrow$ follows from \ref{inverseImageContinuity}.

For $\Rightarrow$, assume $F \overset{\nu}{\longrightarrow} y$ and $\forall i\in I: \forall x\in g_i^{-1}[y]: \; g_i^{-1}[F] \not\to x$. Define the $\nu'$ from $\nu$ by removing all limits of the form $G \to y$ for $G \leq F$. Now we have $g_i^{-1}[G] \leq g_i^{-1}[F]$, so $\forall x\in g_i^{-1}[y]: \; g_i^{-1}[G] \not\to x$. By \ref{inverseImageContinuity}, $\nu'$ still makes all $g_i$ continuous, meaning $\nu'\geq \nu$. Thus $F \overset{\nu'}{\longrightarrow} y$, which is a contradiction.
\end{proof}
\begin{corollary}[Characteristic property of initial and final convergence] \label{characteristicPropertyInitialFinalConvergence}
Let $Y$ be a set, $\sSet{X,\xi}$ and $\sSet{Z, \zeta}$ (pre)convergence spaces.
\begin{enumerate}
\item Let $\{f_i: Y\to \sSet{Z_i, \zeta_i}\}_{i\in I}$ be set of functions to (pre)convergence spaces and $\mu$ the initial (pre)convergence on $Y$ w.r.t. this set. A function $g: \sSet{X, \xi}\to Y$ is continuous \textup{if and only if} $f_i \circ g$ is continuous for all $i\in I$.
\[ \begin{tikzcd}
Y \ar[r, "f_i"] & Z_i \\ X \ar[u, "g"] \ar[ur, swap, "f_i\circ g"]
\end{tikzcd} \]
\item Let $\{g_i: \sSet{X_i, \xi_i} \to Y\}_{i\in I}$ be set of functions from (pre)convergence spaces and $\nu$ the final (pre)convergence on $Y$ w.r.t. this set. A function $f: Y\to \sSet{Z,\zeta}$ is continuous \textup{if and only if} $f\circ g_i$ is continuous for all $i\in I$.
\[ \begin{tikzcd}
X_i \ar[r, "g_i"] \ar[dr, swap, "f\circ g_i"] & Y \ar[d, "f"] \\ & Z
\end{tikzcd} \]
\end{enumerate}
\end{corollary}
\begin{proof}
(1) Take arbitrary $F\overset{\xi}{\longrightarrow} x\in X$. Then the continuity of $g$ is equivalent to the convergence $g[F] \overset{\mu}{\longrightarrow} g(x)$. By the proposition this is equivalent to
\[ \forall i \in I: \; f_i[g[F]] \overset{\zeta_i}{\longrightarrow} f_i(g(x)),  \]
which is equivalent to the continuity of $f_i\circ g$ for all $i\in I$.

(2) Take arbitrary $F \in \powerfilters(Z)$ and $z\in Z$ such that $F \overset{\zeta}{\not\to} z$. Then, by \ref{inverseImageContinuity}, the continuity of $f$ is equivalent to
\[ \forall y\in f^{-1}[z]: \; f^{-1}[F] \overset{\nu}{\not\to} y \]
By the proposition this is equivalent to
\[ \forall y\in f^{-1}[z]: \forall i\in I: \forall x\in g_{i}^{-1}[y]: \; g_i^{-1}[f^{-1}[F]] \overset{\xi_i}{\not\to} x. \]
Using the equality $g^{-1}_i[f^{-1}[F]] = (f\circ g_i)^{-1}[F]$, we can rewrite this as
\[ \forall i\in I: \forall x\in (f\circ g_{i})^{-1}[z]: \; (f\circ g_i)^{-1}[F] \overset{\xi_i}{\not\to} x. \]
By \ref{inverseImageContinuity}, this is equivalent to the  continuity of all $f\circ g_i$.
\end{proof}

TODO: final convergence does not preserve finite depth! (what about Kent space??)

TODO: initial/final not universal property, but product is.

TODO $\prod, \coprod$

\subsection{Initial pretopological convergence}
TODO

\begin{proposition} \label{pretopologicalInitialFinalConvergence}
Let $X$ be a set and $\xi$ the initial convergence on $X$ w.r.t. a set of functions $\{f_i: X\to \sSet{Y_i, \zeta_i}\}_{i\in I}$ from $X$ to pretopological spaces $Y_i$. Then $\xi$ is pretopological and
\[ \vicinity_\xi(x) = \mathfrak{F}\setbuilder{f_i^\preimf(U)}{i\in I, U\in \vicinity_{\zeta_i}(f(x))}. \]
\end{proposition}
\begin{proof}
We have, by \ref{initialFinalConvergence}
\begin{align*}
F \overset{\xi}{\longrightarrow} x &\iff \forall i\in I: f_i^{\imf\imf}[F] \overset{\zeta_i}{\longrightarrow} f_i(x) \\
&\iff \forall i\in I: \vicinity_{\zeta_i}\!\big(f_i(x)\big) \subseteq \upset f_i^{\imf\imf}[F] \\
&\iff \forall i\in I: \forall U\in \vicinity_{\zeta_i}\!\big(f_i(x)\big): \exists M\in f_i^{\imf\imf}[F]: \; M \subseteq U \\
&\iff \forall i\in I: \forall U\in \vicinity_{\zeta_i}\!\big(f_i(x)\big): \exists N\in F: \; f_i^{\imf}[N] \subseteq U \\
&\iff \forall i\in I: \forall U\in \vicinity_{\zeta_i}\!\big(f_i(x)\big): \exists N\in F: \; N \subseteq f_i^{\preimf}[U] \\
&\iff \forall i\in I: \forall U\in \vicinity_{\zeta_i}\!\big(f_i(x)\big): f_i^{\preimf}[U] \in F \\
&\iff \forall i\in I: f^{\preimf\imf}\Big[\vicinity_{\zeta_i}\!\big(f_i(x)\big)\Big] \subseteq F \\
&\iff \bigcup_{i\in I} f^{\preimf\imf}\Big[\vicinity_{\zeta_i}\!\big(f_i(x)\big)\Big] \subseteq F
\end{align*}
Thus the initial convergence is pretopological and
\begin{align*}
\vicinity_\xi(x) &= \mathfrak{F}\bigcup_{i\in I} f_i^{\preimf\imf}\Big[\vicinity_{\zeta_i}\!\big(f_i(x)\big)\Big] \\
&= \mathfrak{F}\bigcup_{i\in I}\setbuilder{f_i^\preimf(U)}{U\in \vicinity_{\zeta_i}(f(x))} \\
&= \mathfrak{F}\setbuilder{f_i^\preimf(U)}{i\in I, U\in \vicinity_{\zeta_i}(f(x))}.
\end{align*}
\end{proof}

\begin{proposition} \label{topologicalInitialFinalConvergence}
Let $X$ be a set and $\xi$ the initial convergence on $X$ w.r.t. a set of functions $\{f_i: X\to \sSet{Y_i, \zeta_i}\}_{i\in I}$ from $X$ to topological spaces $Y_i$. Then $\xi$ is topological and $\topology_\xi$ is generated by $\setbuilder{f_i^\preimf(U)}{i\in I, U\in \topology_{\zeta_i}}$.
\end{proposition}
\begin{proof}
By \ref{pretopologicalInitialFinalConvergence}, $\xi$ is pretopological. Each $U\in \vicinity_{\zeta_i}(f(x)) = \neighbourhood_{\zeta_i}(f(x))$ contains an open set, so $f_i^\preimf(U)$ also contains an open set by \ref{preimageOpenClosed}. Thus $\vicinity_\xi(x)$ is based in open sets. This means that the initial convergence is topological.

Finally $\setbuilder{f_i^\preimf(U)}{i\in I, U\in \topology_{\zeta_i}}$ is a set of open sets. TODO: a set of open sets that generates the convergence also generates the topology. 
\end{proof}

\subsection{Constructions}
\subsubsection{Product convergence}
\begin{definition}
Let $\sSet{X_i, \xi_i}$ be a convergence space for all $i\in I$. The \udef{product convergence space} $\prod_{i\in I}X_i$ is the initial convergence on $\bigtimes_{i\in I}X_i$ w.r.t. the set of projections $p_i: \bigtimes_{i\in I}X_i \to X_i$.
\end{definition}

\begin{proposition} \label{convergenceProductFilter}
Let $\sSet{X_i, \xi_i}$ be a (pre)convergence space for all $i\in I$, $F \in \powerfilters(\prod_{i\in I}X_i)$ and $x\in \prod_{i\in I}X_i$. Then $F\to x$ \textup{if and only if} $\forall i\in I: \exists F_i\in \powerfilters(X_i): 
\; F_i \overset{\xi_i}{\longrightarrow} p_i(x)$ and
\[ F \geq \bigotimes_{i\in I}F_i \defeq \setbuilder{\bigtimes_{i\in I}A_i}{\forall i\in I: A_i \in F_i \;\land\; A_i = X_i, \,\text{except for finitely many $A_i$}}. \]
\end{proposition}
\begin{proof}
The direction $\Leftarrow$ follows straight form \ref{initialFinalConvergence} because $p_i(F) = F_i \to p_i(x)$.

TODO adjoint of $\coprod_{i\in I}p_i$.
\end{proof}
\begin{corollary} \label{productVicinity}
Let $\sSet{X_i, \xi_i}$ be a (pre)convergence space for all $i\in I$ and $x\in \prod_{i\in I}X_i$. Then
\[\vicinity_{\prod \xi_i}(x) = \upset \setbuilder{\bigtimes_{i\in I}A_i}{\forall i\in I: A_i \in \vicinity_{\xi_i}(p_i(x)) \;\land\; A_i = X_i, \,\text{except for finitely many $A_i$}}. \]
In particular $\vicinity_{\xi\otimes \zeta}((x,y)) = \upset \vicinity_\xi(x)\otimes \vicinity_\zeta(y)$.
\end{corollary}

\begin{corollary} \label{productAdherence}
Let $\sSet{X_i, \xi_i}$ be a (pre)convergence space and $A_i\subseteq X_i$ for all $i\in I$. Then
\[ \adh_{\prod_{i\in I}\xi_i}\left(\bigtimes_{i\in I}A_i\right) = \bigtimes_{i\in I}\adh_{\xi_i}(A_i). \]
In particular $\adh_{\xi\otimes \zeta}(A\times B) = \adh_\xi(A) \times \adh_\zeta(B)$.
\end{corollary}
\begin{proof}
We have
\begin{align*}
\seq{x_i}_{i\in I} \in \adh_{\prod_{i\in I}\xi_i}\left(\bigtimes_{i\in I}A_i\right) &\iff \bigtimes_{i\in I}A_i \in \vicinity_{\prod \xi_i}(\seq{x_i}_{i\in I})^{\mesh} \\
&\iff \forall i\in I: \forall B \in \vicinity_{\xi_i}(x_i): A_i\mesh B \\
&\iff \forall i\in I: A_i \in \vicinity_{\xi_i}(x_i)^{\mesh} \\
&\iff \forall i\in I: x_i \in \adh_{\xi_i}(A_i) \\
&\iff \seq{x_i}_{i\in I} \in \bigtimes_{i\in I}\adh_{\xi_i}(A_i).
\end{align*}
\end{proof}
\begin{corollary} \label{productInherence}
Let $\sSet{X_i, \xi_i}$ be a (pre)convergence space and $A_i\subseteq X_i$ for all $i\in I$. Then
\[ \inh_{\prod_{i\in I}\xi_i}\left(\bigtimes_{i\in I}A_i\right) = \bigtimes_{i\in I}\inh_{\xi_i}(A_i). \]
In particular $\inh_{\xi\otimes \zeta}(A\times B) = \inh_\xi(A) \times \inh_\zeta(B)$.
\end{corollary}
\begin{proof}
We have
\begin{align*}
\seq{x_i}_{i\in I} \in \inh_{\prod_{i\in I}\xi_i}\left(\bigtimes_{i\in I}A_i\right) &\iff \bigtimes_{i\in I}A_i \in \vicinity_{\prod \xi_i}(\seq{x_i}_{i\in I}) \\
&\iff \forall i\in I: \exists B_i \in \vicinity_{\xi_i}(x_i): B_i\subseteq A_i \\
&\iff \forall i\in I: A_i \in \vicinity_{\xi_i}(x_i) \\
&\iff \forall i\in I: x_i \in \inh_{\xi_i}(A_i) \\
&\iff \seq{x_i}_{i\in I} \in \bigtimes_{i\in I}\inh_{\xi_i}(A_i).
\end{align*}
\end{proof}

\begin{definition}
Let $\{X_i\}_{i\in I}$ be a set of sets and $\setbuilder{F_i \in \powerfilters(X_i)}{i\in I}$ a set of filters. Then the \udef{product filter} is defined by
\[ \bigotimes_{i\in I}F_i \defeq \setbuilder{\bigtimes_{i\in I}A_i}{\forall i\in I: A_i \in F_i \;\land\; A_i = X_i, \,\text{except for finitely many $A_i$}}.  \]
\end{definition}

\begin{lemma}
The product filter of proper filters is proper.
\end{lemma}
\begin{proof}
Let $\{X_i\}_{i\in I}$ be a set of sets and $\setbuilder{F_i \in \powerfilters(X_i)}{i\in I}$ a set of proper filters. Assume, towards a contradiction, that $\bigotimes_{i\in I}F_i$ is not proper. Take $\bigtimes_{i\in I}A_i\in \bigotimes_{i\in I}F_i$, then $\left(\bigtimes_{i\in I}A_i\right)^c \in \bigotimes_{i\in I}F_i$ and thus we can find $\bigtimes_{i\in I}B_i\in \bigotimes_{i\in I}F_i$ such that $\bigtimes_{i\in I}B_i \subseteq \left(\bigtimes_{i\in I}A_i\right)^c$. We have
\[ \emptyset = \left(\bigtimes_{i\in I}A_i\right) \cap \left(\bigtimes_{i\in I}B_i\right) = \bigtimes_{i\in I}(A_i\cap B_i). \]
Now the right-hand side is only empty if there exists $i\in I$ such that $A_i\cap B_i = \emptyset$. In this case $F_i$ is not a proper filter.
\end{proof}

\begin{lemma} \label{projectionsOfProductFilter}
Let $X,Y$ be sets, $F\in \powerfilters(X)$ and $G\in \powerfilters(Y)$. Then
\begin{enumerate}
\item $p_1^{\imf\imf}[F\otimes G] = F$;
\item $p_2^{\imf\imf}[F\otimes G] = G$.
\end{enumerate}
\end{lemma}
\begin{proof}
The elements of $F$ form a basis of $p_1^{\imf\imf}[F\otimes G]$. Similarly the elements of $G$ form a basis of $p_2^{\imf\imf}[F\otimes G]$.
\end{proof}

\begin{lemma} \label{filterFactorisationInequality}
Let $X,Y$ be sets and $H \in \powerfilters(X\times Y)$. Then $p_1^{\imf\imf}(H)\otimes p_2^{\imf\imf}(H) \subseteq H$.
\end{lemma}
\begin{proof}
Take some $A\in p_1^{\imf\imf}(H)\otimes p_2^{\imf\imf}(H)$. Then there exist $B,C\in H$ such that $A = p_1^\imf(B)\times p_2^\imf(C)$, which means that $B\cap C\subseteq A$. Now $B\cap C \in H$, so $A\in H$.
\end{proof}

\begin{lemma} \label{projectionsOfUltrafilterAreUltra}
Let $X,Y$ be sets and $H \in \powerfilters(X\times Y)$. If $H$ is an ultrafilter, then $p_1^{\imf\imf}[H]$ and $p_2^{\imf\imf}[H]$ are also ultrafilters.
\end{lemma}
\begin{proof}
Assume, towards a contradiction, that $p_1^{\imf\imf}[H]$ is not an ultrafilter. In this case we can find a proper filter $F\supsetneq p_1^{\imf\imf}[H]$.

Now we show that $F\otimes p_2^{\imf\imf}[H] \amesh H$.
Take $A\in F$ and $B, C\in H$. We need to show that $A\times p_2^{\imf}(B) \mesh C$. Because $B\cap C\in H$ and $p_1^{\imf\imf}[H]\subseteq F$, we have $p_1^\imf(B\cap C)\mesh A$. Take $a\in p_1^\imf(B\cap C)\cap A$. There exists a $b$ such that $(a,b)\in B\cap C$. Also $(a,b) \in A\times p_2^\imf(B\cap C)$. Thus $(B\cap C) \mesh \big(A\times p_2^\imf(B\cap C)\big)$. This implies $C \mesh A\times p_2^{\imf}(B)$.

Then we have that $\big(F\otimes p_2^{\imf\imf}[H]\big)\vee H$ is proper. Because $H$ is an ultrafilter, we must have $\big(F\otimes p_2^{\imf\imf}[H]\big)\vee H = H$, which means that $F\otimes p_2^{\imf\imf}[H] \subseteq H$.

Finally we apply $p_1^{\imf\imf}$ to both sides of the inclusion to get $F \subseteq p_1^{\imf\imf}[H]$. This is a contradiction.
\end{proof}

\begin{lemma} \label{productPrincipalUltrafilter}
Let $X,Y$ be sets, $x\in X$ and $y\in Y$. Then $\pfilter{(x,y)} = \pfilter{x} \otimes \pfilter{y}$.
\end{lemma}
\begin{proof}
It is enough to check that $(x,y)\in\pfilter{x}\otimes\pfilter{y}$ and that $\pfilter{x}\otimes\pfilter{y}$ is not trivial. Both these statements are immediately clear.
\end{proof}

\begin{lemma} \label{intersectionProductFilters}
Let $X,Y$ be sets, $F,G\in \powerfilters(X)$ and $H\in\powerfilters(Y)$. Then
\[ (F\cap G)\otimes H = (F\otimes H)\cap (G\otimes H). \]
\end{lemma}
\begin{proof}
$\boxed{\subseteq}$ Take $A\in (F\cap G)$ and $B\in H$. Then $A\times B\in F\otimes H$ and $A\times B\in G\otimes H$.

$\boxed{\supseteq}$ Take $A \in (F\otimes H)\cap (G\otimes H)$. Then we can find $B_1\in F$, $C_1,C_2\in H$ and $B_2\in G$ such that $B_1\times C_1 \subseteq A$ and $B_2\times C_2 \subseteq A$. Now $(B_1\cup B_2)\times (C_1\cap C_2) \subseteq A$. Also $B_1\cup B_2 \in F\cap G$ and $C_1\cap C_2 \in H$, so $(B_1\cup B_2)\times (C_1\cap C_2) \in (F\cap G)\otimes H$. By upwards closure $A\in (F\cap G)\otimes H$.
\end{proof}

\begin{lemma} \label{convergenceFiniteProductFilter}
Let $\sSet{X,\xi}$ and $\sSet{Y,\zeta}$ be convergence spaces, $F\in \powerfilters(X\times Y)$, $G\in \powerfilters(X)$ and $H\in \powerfilters(Y)$. Then
\begin{enumerate}
\item $F \overset{\xi \otimes \zeta}{\longrightarrow} (x,y)$ \textup{if and only if} $p_1^{\imf\imf}(F) \overset{\xi}{\longrightarrow} x$ and $p_2^{\imf\imf}(F) \overset{\zeta}{\longrightarrow} y$;
\item $G\otimes H \overset{\xi \otimes \zeta}{\longrightarrow} (x,y)$ \textup{if and only if} $G \overset{\xi}{\longrightarrow} x$ and $H \overset{\zeta}{\longrightarrow} y$.
\end{enumerate}
\end{lemma}
\begin{proof}
Point (1) is a restatement of \ref{initialFinalConvergence}. Point (2) follows from point (1) because $p_1^{\imf\imf}(F\otimes G) = F$ and $p_2^{\imf\imf}(F\otimes G) = G$ by \ref{projectionsOfProductFilter}.
\end{proof}


\begin{lemma}
Let $\sSet{X,\xi}, \sSet{Y,\sigma}$ and $\sSet{Z,\zeta}$ be convergence spaces and $f: X\to Y\times Z$ be a function of the form
\[ f: X\to Y\times Z: x\mapsto f(x) = (f_Y(x), f_Z(x)). \]
The $f$ is continuous \textup{if and only if} $f_Y$ and $f_Z$ are continuous.
\end{lemma}
\begin{proof}
Follows immediately from \ref{characteristicPropertyInitialFinalConvergence},
because $f_Y = p_1\circ f$ and $f_Z = p_2\circ f$.
\end{proof}
\begin{corollary} \label{continuousEmbeddingProduct}
Let $\sSet{X,\xi}$ and $\sSet{Y,\sigma}$ be convergence spaces. Then for all $y\in Y$, the function $X\to X\times Y: x\mapsto (x,y)$ is continuous.
\end{corollary}
\begin{proof}
The functions $\id_X$ and $\underline{y}$ are continuous.
\end{proof}
\begin{corollary} \label{productContinuousFunctions}
Let $\sSet{X,\xi}$ and $\sSet{Y,\sigma}$ be convergence spaces and $f_X: X\to X, f_Y: Y\to Y$ be continuous functions. Then
\[ f_X\times f_Y: X\times Y \to X\times Y: (x,y)\mapsto (f_X(x), f_Y(y))  \]
is continuous.
\end{corollary}
\begin{proof}
The functions $p_1\circ f_X: (x,y) \mapsto f_X(x)$ and $p_2\circ f_Y: (x,y) \mapsto f_Y(y)$ are continuous.
\end{proof}

\subsubsection{Subspace convergence}
\begin{definition}
Let $\sSet{X,\xi}$ be a convergence space and $A\subseteq X$ a subset. The \udef{subspace convergence} $\xi|_A$ on $A$ is the initial convergence w.r.t. $\{\iota: A \hookrightarrow X: a\mapsto a\}$. The convergence space $\sSet{A,\xi|_A}$ is called a \udef{convergence subspace} of $X$.
\end{definition}

\begin{lemma}
Let $\sSet{X,\xi}$ be a convergence space and $\sSet{A,\xi|_A}$ a subspace.
\begin{enumerate}
\item If $F\in \powerfilters(A)$ and $F\overset{\xi|_A}{\longrightarrow} a$, then $F\overset{\xi}{\longrightarrow} a$.
\item If $F\in \powerfilters(X)$, $a\in A$ and $F\overset{\xi}{\longrightarrow} a$, then $F|_A\overset{\xi|_A}{\longrightarrow} a$.
\end{enumerate}
\end{lemma}
\begin{proof}
(1) Assume $F\overset{\xi|_A}{\longrightarrow} a$, then $\iota^{\imf\imf}[F] = F$ and $\iota(a) = a$. Then $F\overset{\xi}{\longrightarrow} a$ by continuity.

(2) Assume $F\overset{\xi}{\longrightarrow} a\in A$. Then $\iota^{\preimf\imf}[F] = F|_A$ and $\iota^\preimf\{a\} = \{a\}$, so $F|_A\overset{\xi|_A}{\longrightarrow} a$ by \ref{initialFinalConvergence}.
\end{proof}

\begin{lemma}
Restrictions on continuous functions are continuous (domain + codomain).
\end{lemma}

\begin{proposition} \label{continuityRestrictionExpansion}
Let $\sSet{X,\xi}$, $\sSet{Y,\sigma}$ and $\sSet{Z,\zeta}$ be convergence spaces.
\begin{enumerate}
\item \textup{(Restricting the domain)} If $f:X\to Y$ is continuous and $A$ is a subspace of $X$, then the restricted function $f|_{A}:A\to Y$ is continuous.
\item \textup{(Restricting the range)} Let $f:X\to Y$ be continuous. If $Z$ is a subspace of $Y$ containing the image set $f[X]$, then $f:X\to Z$ is continuous.
\item \textup{(Expanding the range)} Let $f:X\to Y$ be continuous. If $Y$ is a subspace of $Z$, then $f:X\to Z$ is continuous.
\end{enumerate}
\end{proposition}
\begin{proof}
(1) Composition of continuous maps: $f|_{A} = f\circ\iota$.

(2) Composition of continuous maps: $f:X\to Z = \iota \circ (f: X\to Y)$.

(3) Characteristic property \ref{characteristicPropertyInitialFinalConvergence}: if $f:X\to Y = \iota \circ (f:X\to Z)$ is continuous, then $f:X\to Z$ is too.
\end{proof}



\subsubsection{Quotient convergence}
\begin{proposition}
Each convergence space is the quotient convergence space of a topological space.
\end{proposition}

\subsection{Projective and injective limits}
\begin{definition}
Let $\sSet{I, \prec}$ be an upwards directed set.
\begin{itemize}
\item Let $\{sSet{X_i, \xi_i}\}_{i\in I}$ be a set of convergence spaces and for each $j \succ i$, let $p_{j,i}: X_j \to X_i$ be a continuous mapping. Then the structure $\sSet{I, \{\sSet{X_i, \xi_i}\}_{i\in I}, \{p_{j,i}\}_{j\succ i}}$ is called a \udef{projective system} (or \udef{inverse system}) if for all $k \succ j \succ i \in I$ the diagram
\[ \begin{tikzcd}
X_k \ar[rr, "p_{k,i}"] \ar[rd, "p_{k,j}"] & & X_i \\
& X_j \ar[ur, "p_{j,i}"]
\end{tikzcd} \qquad \text{commutes.} \]
Let $\sSet{X, \xi}$ be a convergence space and $p_i: X\to X_i$ a continuous mapping for all $i\in I$. 
\end{itemize}
\end{definition}

\chapter{Separation axioms and other properties of convergences spaces}

\section{Distinguishability, separation and regularity}
\subsection{Distinguishable points}
\begin{definition}
Let $\sSet{X,\xi}$ be a convergence space and $x,y\in X$. We call $x$ and $y$ \udef{distinguishable} if ${\lim_\xi}^{-1}(x) \neq {\lim_\xi}^{-1}(y)$.
\end{definition}

In orther words, $x,y \in X$ are distinguishable if there exists a filter $F \in \powerfilters(X)$ such that
\[ \Big(x\in \lim_\xi F \land y\notin \lim_\xi F\Big)\;\lor\; \Big(x\notin \lim_\xi F \land y\in \lim_\xi F\Big). \]

We say $F$ \udef{distinguishes} $x$ and $y$.

\begin{proposition} \label{distinguishabilityPrincipalUltrafilters}
Let $\sSet{X,\xi}$ be a Kent convergence space and $x,y\in X$. Then $x$ and $y$ are indistinguishable \textup{if and only if}
\[ \pfilter{x} \to y \qquad \text{and}\qquad \pfilter{y}\to x. \]
\end{proposition}
\begin{proof}
The direction $\Rightarrow$ is clear: from $\pfilter{x} \to x$, we get $\pfilter{x}\to y$ by indistinguishability.

The direction $\Leftarrow$ is proved by contradiction. Assume $x$ and $y$ are distinguishable, so there exists a filter $F$ such that $F\to x$ but $F\not\to y$. Then $F\cap \pfilter{x} \to y$ by the definining property of Kent spaces. Now $F\cap \pfilter{y} \subseteq F$, so $F\to y$. This is a contradiction.
\end{proof}


\subsection{Separation}
\begin{definition}
Let $\sSet{X,\xi}$ be a convergence space and $A,B\subseteq X$. We say $A$ and $B$ are \udef{separated} if $\adh_\xi(A) \perp B$ and $A\perp \adh_\xi(B)$.

We call two points $x,y\in X$ \udef{separated} if $\{x\}$ and $\{y\}$ are separated.
\end{definition}

\begin{proposition} \label{separatednessPrincipalUltrafilters}
Let $X$ be a set, $\xi$ a convergence on $X$ and $x,y\in X$. Then $x$ and $y$ are separated \textup{if and only if}
\[ \pfilter{x} \not\to y \qquad \text{and}\qquad \pfilter{y}\not\to x. \]
\end{proposition}
\begin{proof}
By \ref{singletonAdherence} we have $\adh_\xi(\{x\}) = \lim_\xi\pfilter{x}$ and $\adh_\xi(\{y\}) = \lim_\xi\pfilter{y}$.
\end{proof}
In other words, $x$ and $y$ are not separated iff $\pfilter{x} \to y$ or $\pfilter{y} \to x$.

\begin{lemma} \label{separatedDistinguishable}
Let $\sSet{X,\xi}$ be a convergence space and $x,y\in X$. If $x,y$ are separated, then they are distinguishable.
\end{lemma}
\begin{proof}
Assume $x,y$ separated. WLOG we can assume $\pfilter{x} \not\to y$. So $\pfilter{x}$ distinguishes $x$ and $y$.
\end{proof}

\subsubsection{Separation by convergent filters}
\begin{definition}
Let $\sSet{X,\xi}$ be a convergence space and $A,B\subseteq X$. We say $A$ and $B$ are \udef{separated by convergent filters} if for all approaches $F:X\to \powerfilters(X)$, the contours satisfy $\neg(F(A)\amesh F(B))$.

We call two points $x,y\in X$ separated by convergent filters if $\{x\}$ and $\{y\}$ are separated by convergent filters.
\end{definition}

\begin{lemma} \label{pointsSeparatedConvergentFilters}
Let $\sSet{X,\xi}$ be a convergence space. For $x,y\in X$ the following are equivalent
\begin{enumerate}
\item $x,y$ are separated by convergent filters;
\item $\lim^{-1}(x)\perp \lim^{-1}(y)$.
\end{enumerate}
\end{lemma}
\begin{proof}
$(1) \Rightarrow (2)$ Assume $\lim^{-1}(x)\amesh \lim^{-1}(y)$, i.e.\ there exists some filter $F\in \lim^{-1}(x)\cap \lim^{-1}(y)$. Then the constant approach $x,y\mapsto F$ has $F(x) = F(y)$, meaning $F(x)\amesh F(y)$ and thus $x,y$ are not separated by convergent filters.

$(2) \Rightarrow (1)$ Assume $\lim^{-1}(x)\perp \lim^{-1}(y)$. Pick an arbitrary approach $F$. If $F(x)\amesh F(y)$, then $F(x)\vee F(y)$ is proper by \ref{joinProperFilter} and in $\lim^{-1}(x)\cap \lim^{-1}(y)$ by monotonicity, which is a contradiction.
\end{proof}


\subsubsection{Separation by vicinities}
\begin{definition}
Let $\sSet{X,\xi}$ be a convergence space and $A,B\subseteq X$. We say $A$ and $B$ are \udef{separated by vicinities} if there exist $U\in \vicinity_\xi(A)$ and $V\in\vicinity_\xi(B)$ such that $U\perp V$.

We call two points $x,y\in X$ separated by vicinities if $\{x\}$ and $\{y\}$ are separated by vicinities, i.e.\ there exist vicinities $U,V$ of $x,y$, resp., such that $U\perp V$.
\end{definition}

\begin{lemma}
Let $\sSet{X,\xi}$ be a convergence space and $A,B\subseteq X$. The following are equivalent:
\begin{enumerate}
\item $A$ and $B$ are separated by vicinities;
\item $\neg(\vicinity_\xi(A)\amesh \vicinity_\xi(B))$.
\end{enumerate}
\end{lemma}

\begin{proposition} \label{disjointVicinitiesConvergentFilterSeparation}
Let $\sSet{X,\xi}$ be a convergence space.
\begin{enumerate}
\item Separation by vicinities implies separation by convergent filters.
\item If $\xi$ is pretopological, the converse also holds.
\end{enumerate}
\end{proposition}
\begin{proof}
$(1)$ Assume $A,B\subseteq X$ are separated by vicinities $U$ and $V$. Assume, towards a contradiction, that $F(A)\amesh F(B)$ for some approach $F$. Then for all $a\in A$, $U\in F(a)$, so $U\in F(A)$. Similarly $V\in F(B)$. This means $U\mesh V$, which is a contradiction.

$(2)$ If $\xi$ is pretopological, then $\vicinity_\xi$ is an approach. Thus $\neg(\vicinity_\xi(A)\amesh \vicinity_\xi(B))$, meaning there exist disjoint vicinities of $A,B$.
\end{proof}
Thus in the pretopological case all approaches $F$ satisfy $\neg(F(A)\mesh F(B))$ iff the vicinity filter satisfies $\neg(\vicinity(A)\mesh \vicinity(B))$.


\begin{proposition} \label{separationByVicinitiesEquivalences}
Let $\sSet{X,\xi}$ be a convergence space.
\begin{enumerate}
\item If $A,B \subseteq X$ are separated by vicinities, then .
\item If $\xi$ is pretopological and $\lim^{-1}(x)\perp \lim^{-1}(y)$ for some $x,y\in X$, then $x,y$ are separated by vicinities.
\end{enumerate}
\end{proposition}
\begin{proof}
$(1)$ Assume $A,B$ are separated by vicinities $U$ and $V$. Assume, towards a contradiction, that $\lim^{-1}(A)\amesh \lim^{-1}(B)$, i.e.\ there exists a filter $F$ that converges to $x\in A$ and $y\in B$. Then $U,V\in F$ and thus $U\cap V = \emptyset \in F$, meaning $F$ is not a proper filter.

$(2)$ Assume $x,y$ are not separated by vicinities. Then $\vicinity_\xi(x)\amesh \vicinity_\xi(y)$, meaning $x$ is an accumulation point of $\vicinity_\xi(y)$. By \ref{subfilterToAccumulationPoint}, there exists a filter $G\geq \vicinity_\xi(y)$ such that $G\to x$. Then $G\in \lim^{-1}(x)\cap \lim^{-1}(y)$.
\end{proof}

\subsubsection{Separation by neighbourhoods}
\begin{definition}
Let $\sSet{X,\xi}$ be a convergence space and $A,B\subseteq X$. We say $A$ and $B$ are \udef{separated by neighbourhoods} if there exist disjoint neighbourhoods of $A$ and $B$.

We call two points $x,y\in X$ separated by neighbourhoods if $\{x\}$ and $\{y\}$ are separated by neighbourhoods.
\end{definition}

\begin{lemma} \label{neighbourhoodSeparationLemma}
Let $\sSet{X,\xi}$ be a convergence space and $A,B\subseteq X$. Then $A,B$ are separated by neighbourhoods \textup{if and only if} there exists an open set $U$ such that $A \subseteq U \subseteq \overline{U} \subseteq B^c$.
\end{lemma}
\begin{proof}
WLOG we may take $\overline{U}^c$ to be the neighbourhood of $B$.
\end{proof}

\subsubsection{Separation by closed vicinities}
\subsubsection{Separation by functions}
\url{https://en.wikipedia.org/wiki/Separated_sets}

\subsection{Regularity}
\begin{definition}
Let $\sSet{X,\xi}$ be a convergence space and $\mathcal{Z} \subseteq \powerset(X)$. The convergence $\xi$ is called \udef{$\mathcal{Z}$-regular} if for all $x\in X, F\in\lim_\xi^{-1}(x)$ there exists a filter base $G\subseteq \mathcal{Z}$ such that
\begin{itemize}
\item $G \preceq F$;
\item $G \overset{\xi}{\longrightarrow} x$.
\end{itemize}
\end{definition}

\section{Separation properties}
\subsection{$T_0$ or Kolmogorov}
\begin{definition}
We call a convergence space $\sSet{X, \xi}$ \udef{Kolmogorov} or \udef{$T_0$} if every pair of distinct points in $X$ is distinguishable.
\end{definition}

\subsection{$R_0$ or symmetric}
\begin{definition}
Let $X$ be a set and $\xi$ a convergence on $X$. Then $\xi$ is called \udef{symmetric} or \udef{$R_0$} if all distinguishable pairs of points are separated.
\end{definition}

\begin{proposition} \label{R0conditions}
Let $X$ be a set and $\xi$ a convergence on $X$. Then the following are equivalent:
\begin{enumerate}
\item $\xi$ is an $R_0$ convergence;
\item $x$ and $y$ are indistinguishable \textup{if and only if} they are separated;
\item for all $x,y\in X$, $x$ and $y$ are indistinguishable \textup{if and only if} $\pfilter{x} \to y$;
\item $\adh_\xi(\{x\})$ is the set of points that are indistinguishable from $x$ for all $x\in X$;
\end{enumerate}
The following are consequences of the above. If $\xi$ is a Kent convergence, then they are also equivalent:
\begin{enumerate} \setcounter{enumi}{4}
\item the set $\setbuilder{\adh_\xi(\{x\})}{x\in X}$ is a partition of $X$;
\item for all $x,y\in X$:  $\pfilter{x} \to y$ \textup{if and only if} $\pfilter{y} \to x$;
\end{enumerate}
\end{proposition}
\begin{proof}
$(1) \Rightarrow (2)$ The converse to the $R_0$ condition, that separated points are distinguishable, is automatic (see \ref{separatedDistinguishable}).

$(2) \Rightarrow (3)$ All points that are indistinguishable from $x$ are in $\adh_\xi(\{x\}) = \lim_\xi\{x\}$ by construction. Assume $y$ is distinguishable from $x$. Then $y$ is separated from $x$, so $\pfilter{x}\not\to x$ by \ref{separatednessPrincipalUltrafilters}.

$(3) \Rightarrow (4)$ $\adh_\xi(\{x\}) = \lim_\xi\pfilter{x}$ by \ref{singletonAdherence}.

$(4) \Rightarrow (1)$ Assume $x,y\in X$ are distinguishable. Then $y\notin\adh_\xi(\{x\})$ and $x\notin\adh_\xi(\{y\})$. This means $\pfilter{x} \not\to y$ and $\pfilter{y} \not\to x$ and we conclude by \ref{separatednessPrincipalUltrafilters}.


$(4) \Rightarrow (5)$ Indistinguishability is an equivalence relation.

$(5) \Rightarrow (6)$ Equivalence relations are symmetric.

$(6) \Rightarrow (1)$ Assume $\xi$ of a Kent convergence and $x,y$ distinguishable. Then $\pfilter{x}\not\to y$ or $\pfilter{y}\not\to x$ by \ref{distinguishabilityPrincipalUltrafilters}. Because of (6), the ``or'' becomes an ``and'' and we conclude with \ref{separatednessPrincipalUltrafilters}.
\end{proof}

TODO: characterisation  ``every open set is a union of closed sets'' or
``every closed set is an intersection of open sets''.

?? $\adh_\xi(\{x\})$ is closed for all $x\in X$ ??

\subsection{$T_1$ or Fréchet}
\begin{definition}
Let $X$ be a set and $\xi$ a preconvergence on $X$. Then $\xi$ is called \udef{Fréchet} or \udef{$T_1$} if all distinct points in $X$ are separated.
\end{definition}
\url{https://en.wikipedia.org/wiki/T1_space}

\begin{proposition} \label{FrechetCharacterisation}
Let $X$ be a set and $\xi$ a convergence on $X$. Then the following are equivalent:
\begin{enumerate}
\item $\xi$ is a $T_1$ convergence;
\item $\xi$ is both $T_0$ and $R_0$;
\item all singletons are closed, i.e.\ $\forall x\in X: \; \adh_\xi(\{x\}) = \{x\}$;
\item $\forall x\in X: \; \lim_\xi\pfilter{x} = \{x\}$;
\item $\forall x\in X: \; \lim_\xi \pfilter{x} \subseteq \{x\}$;
\end{enumerate}
\end{proposition}
\begin{proof}
$(1) \Leftrightarrow (2)$ Definitions together with point (2) of \ref{R0conditions}.

$(1) \Leftrightarrow (3)$ From point (4) of \ref{R0conditions}.

$(3) \Leftrightarrow (4)$ From \ref{singletonAdherence}.

$(4) \Leftrightarrow (5)$ Convergences are centered.
\end{proof}
TODO: every singleton in $X$ is closed;  every finite subset of $X$ is closed.
\begin{corollary} \label{finiteConvergenceDiscrete}
Let $X$ be a finite set, then the only $T_1$ convergence on $X$ is the discrete convergence $\iota_X$.
\end{corollary}
\begin{proof}
Let $\xi$ be a $T_1$ convergence on $X$ and $F$ a proper filter in $\powerset(X)$. Now $F$ is principal (TODO ref), so $F = \upset \ker F$. If $\ker F$ is a singleton, then $\lim F = \ker F$ by point (2). Otherwise $\lim F = \emptyset$ by point (4). This is discrete convergence.
\end{proof}

\begin{lemma}
If $\sSet{X,\xi}$ is a $T_1$ convergence on $X$, then $\xi$ is also $T_0$.
\end{lemma}
\begin{proof}
The filters $\pfilter{x}$ and $\pfilter{y}$ distinguish $x,y\in X$.
\end{proof}

\begin{proposition}
Let $\xi$ be a $T_1$ preconvergence on a set $X$ and $F$ a filter in $\powerfilters(X)$. If $x\in \lim F$, then $\ker F \subseteq \{x\}$.
\end{proposition}
\begin{proof}
Suppose $\ker F$ in non-empty and take $y \in \ker F$. Then $F \subseteq \pfilter{y}$ and so
\[ x \in \lim F \subseteq \lim \pfilter{y} \subseteq \{y\}. \]
This means that $y = x$.
\end{proof}
\begin{corollary}
The kernel of a convergent proper filter in a $T_1$ space is empty or a singleton.
\end{corollary}
By constrast, in $T_2$ spaces the \emph{limit} (not kernel) of a convergent proper filter is empty or a singleton.

\begin{proposition} \label{setKernelVicinityFilter}
Let $\sSet{X,\xi}$ be a $T_1$ pretopological convergence space and $A \subseteq X$ a subset. Then $\bigcap\vicinity(A) = A$.
\end{proposition}
\begin{proof}
We have $A\subseteq \bigcap\vicinity(A)$ from \ref{vicinityOfSetCorollary}. For the converse, take $x\notin A$. By $T_1$, we have $\adh\{x\} \perp A$, which implies $\forall y\in A:\; y\notin\adh\{x\}$. By \ref{adherenceInherenceCharacterisation}, this is equivalent to
\[ \forall y\in A: \forall U\in \vicinity(y):\; U\setminus \{x\} \in \vicinity(y). \]
This implies
\[ \forall V\in \vicinity(A): \; V\setminus\{x\} \in \vicinity(A), \]
which means that $x\notin \bigcap\vicinity(A)$.
\end{proof}
\begin{corollary}
In a topological $T_1$ convergence space, every set is an intersection of open sets.
\end{corollary}
Note that this corollary can also simply be proved by writing
\[ A = \bigcap_{x\in A^c}\{x\}^c. \]

TODO: big question marks:
\begin{proposition}
Let $X$ be a set and $\xi$ a convergence on $X$. If $\xi$ is $T_1$, then $\forall x\neq y\in X: \exists F,G\in \powerfilters(X)$ such that
\[ x\in \lim_\xi F \;\land\; y\in \lim_\xi G \;\land\; x\notin \lim_\xi G \;\land\; y\notin \lim_\xi F. \]
If $\xi$ is topological, the converse also holds.
\end{proposition}
\begin{proof}
$\Rightarrow$ We may take $F = \pfilter{x}$ and $G = \pfilter{y}$.

$\Leftarrow$ If $\xi$ is topological, we may take $F = \vicinity_\xi(x)$ and $G = \vicinity_\xi(y)$. It is enough to show that $y\notin \adh_\xi(\{x\})$.

Assume, towards a contradiction, that $y\in \adh_\xi(\{x\})$. Then $\{x\} \in \vicinity_\xi(y)^{\mesh}$
\end{proof}

\subsection{$R_1$ or reciprocal}
\begin{definition}
Let $\sSet{X,\xi}$ be a convergence space. Then $\xi$ is called \udef{$R_1$}, \udef{reciprocal} or \udef{preregular} if all distinguishable points are separated by convergent filters.
\end{definition}
TODO review definition.

\url{https://gdz.sub.uni-goettingen.de/id/PPN235181684_0187?tify={%22pages%22:[191],%22panX%22:0.893,%22panY%22:0.579,%22view%22:%22info%22,%22zoom%22:0.894}}


\begin{proposition} \label{R1Conditions}
Let $X$ be a set and $\xi$ a convergence on $X$. Then the following are equivalent:
\begin{enumerate}
\item $\xi$ is an $R_1$ convergence;
\item if $x$ and $y$ are distinguishable, then $\lim_\xi^{-1}(x)\perp \lim_\xi^{-1}(y)$;
\item $x$ and $y$ are distinguishable \textup{if and only if} $\lim_\xi^{-1}(x)\perp \lim_\xi^{-1}(y)$;
\item for all $x,y\in X$, either $\lim_\xi^{-1}(x) = \lim_\xi^{-1}(y)$ or $\lim_\xi^{-1}(x)\perp \lim_\xi^{-1}(y)$;
\item for all $x,y\in X$: $\lim_\xi^{-1}(x)\mesh \lim_\xi^{-1}(y)$ implies $\lim_\xi^{-1}(x) = \lim_\xi^{-1}(y)$;
\item the set $\setbuilder{\lim^{-1}_\xi(x)}{x\in X}$ is a partition of the set on convergent filters on $X$;
\item if there exists a filter $F$ such that $F \to x$ and $F \to y$, then $x$ and $y$ are indistinguishable.
\end{enumerate}
\end{proposition}
\begin{proof}
$(1) \Leftrightarrow (2)$ By \ref{pointsSeparatedConvergentFilters}.

$(2) \Rightarrow (3)$ If $\lim_\xi^{-1}(x)\perp \lim_\xi^{-1}(y)$, then $x$ and $y$ are definitely distinguishable, e.g.\ by $\pfilter{x}$.

$(3) \Rightarrow (4) \Rightarrow (5) \Rightarrow (6)$ Clear.

$(6) \Rightarrow (7)$ From $F\to x$ and $F\to y$, we get $F\in \lim_\xi^{-1}(x)$ and $F\in \lim_\xi^{-1}(y)$, so $\lim_\xi^{-1}(x)\amesh \lim_\xi^{-1}(y)$. Using (3) this implies $\lim_\xi^{-1}(x) = \lim_\xi^{-1}(y)$.

$(7) \Rightarrow (2)$ By contraposition.
\end{proof}
\begin{corollary}
Any $R_1$ convergence space is also $R_0$.
\end{corollary}
\begin{proof}
Compare point (2) with point (3) of \ref{R0conditions} and note that $\pfilter{x}\to y$ implies $\lim_\xi^{-1}(x)\amesh \lim_\xi^{-1}(y)$. 
\end{proof}


\subsection{$T_2$ or Hausdorff}
\begin{definition}
Let $X$ be a set and $\xi$ a preconvergence on $X$. Then $\xi$ is called \udef{Hausdorff} or \udef{$T_2$} if every proper $\xi$-limit contains at most one point.
\end{definition}
By ``proper $\xi$-limit'' we mean we exclude from this condition the degenerate filter $\powerset(X)$. Otherwise there would be no $T_2$ convergences by \ref{limitDegenerateFilter}.

\begin{proposition}
Let $X$ be a set and $\xi$ a convergence on $X$. Then the following are equivalent:
\begin{enumerate}
\item $\xi$ is a $T_2$ convergence;
\item if $x \neq y$, then ${\lim_\xi}^{-1}(x)\perp {\lim_\xi}^{-1}(y)$;
\item $\xi$ is $T_0$ and $R_1$;
\item $\xi$ is $T_1$ and $R_1$;
\end{enumerate}
\end{proposition}
\begin{proof}
$(1) \Leftrightarrow (2)$ If $F \in \lim_\xi^{-1}(x)\cap \lim_\xi^{-1}(y)$, then $F \to x$ and $F\to y$, so $\xi$ would not be $T_2$.

$(2) \Leftrightarrow (3)$ Clear.

$(3) \Leftrightarrow (4)$ $R_1$ implies $R_0$ and $R_0+T_0$ is equivalent with $T_1$.
\end{proof}

\begin{proposition}
Every $T_2$ convergence is also $T_1$. If the space is finite, the converse also holds.
\end{proposition}
\begin{proof}
Let $\sSet{X, \xi}$ be a $T_2$ convergence space and $x\in X$. By $T_2$, $\lim_\xi\pfilter{x}$ is a singleton. Definition of convergence this singleton is $\{x\}$.

Now let $X$ be a finite set and let $F$ be a proper filter in $\powerfilters(X)$. Then $F$ is principal by \ref{finiteFiltersPrincipal} and not free because it is proper. So we can take $x\in \ker F$ and $F \subseteq \pfilter{x}$. Thus $\lim F \subseteq \lim \pfilter{x} \subseteq \{x\}$, meaning the convergence is $T_2$.
\end{proof}
\begin{corollary}
Let $X$ be a finite set, then the only $T_2$ convergence on $X$ is the discrete convergence $\iota_X$.
\end{corollary}
\begin{proof}
By \ref{finiteConvergenceDiscrete}.
\end{proof}

\begin{proposition}
TODO: move: topological Hausdorff implies regular. This does not hold for non-topological convergence in general.
\end{proposition}

\subsection{$R_2$ or regular}
\begin{definition}
Let $\sSet{X,\xi}$ be a convergence space. Then $\xi$ is called \udef{regular} or \udef{$R_2$} if it is based in $\adh_\xi^{\imf}(\powerset^2(X))$.
\end{definition}

\url{https://en.wikipedia.org/wiki/Regular_space}

\begin{proposition}
Let $\sSet{X,\xi}$ be a convergence space. Then the following are equivalent:
\begin{enumerate}
\item $\xi$ is an $R_2$ convergence;
\item for all $F\in\powerfilters(X)$, $F\overset{\xi}{\longrightarrow} x$ implies $\adh_\xi[F] = \setbuilder{\adh_\xi(A)}{A\in F} \overset{\xi}{\longrightarrow} x$.
\item for all $F\in\powerfilters(X)$: $F\overset{\xi}{\longrightarrow} x$ \textup{if and only if} $\adh_\xi[F] = \setbuilder{\adh_\xi(A)}{A\in F} \overset{\xi}{\longrightarrow} x$.
\end{enumerate}
\end{proposition}
\begin{proof}
$(1) \Leftrightarrow (2)$ Assume (1), then there exists a filter $G$ based in $\setbuilder{\adh_\xi(A)}{A\in\powerset(X)}$ such that $G\to x$ and $G\subseteq F$. We just need to show that $G\subseteq \adh[F]$. Indeed take $A\in G$, then $A = \adh(B)$ for some $B\subseteq X$. Now $B\in F$ TODO: is this wrong?

Because $\adh_\xi[F]\preceq F$ Then $\adh_\xi(A) \supseteq A$, so $\adh$ 

$(2) \Leftrightarrow (3)$ One direction is given by definition. The other follows by monotonicity because $\adh_\xi[F] \preceq F$.
\end{proof}

\begin{lemma}
Any $R_2$ convergence is also $R_1$.
\end{lemma}

\begin{proposition} \label{regularityBySeparation}
Let $\xi$ be a convergence.
\begin{itemize}
\item If $\xi$ is regular, then $\xi$ separates points from the complements of their vicinities by convergent filters.
\item If $\xi$ is pretopological, then the converse also holds.
\end{itemize}
\end{proposition}
\begin{proof}
(1) Fix an arbitrary approach $F: X\to \powerfilters(X)$ and $V_x\in \vicinity(x)$ for all $x\in X$.

By regularity we have $\vicinity(x)\subseteq \upset\adh[F(x)]$ for all $x\in X$.
This means
\begin{align*}
& \forall x\in X: \forall V\in \vicinity(x): \exists U \in F(x): \quad \adh_\xi(U) \subseteq V \\
\iff& \forall x\in X: \forall V\in \vicinity(x): \exists U \in F(x): \quad (\adh_\xi(U) \perp V^c) \\
\iff& \forall x\in X: \forall V\in \vicinity(x): \exists U \in F(x): \quad \neg(\adh_\xi(U) \mesh V^c) \\
\iff& \forall x\in X: \forall V\in \vicinity(x): \exists U \in F(x): \quad \neg(\{U\} \amesh \vicinity(V^c)) \\
\iff& \forall x\in X: \forall V\in \vicinity(x): \quad \neg(\forall U \in F(x):\{U\} \amesh \vicinity(V^c)) \\
\iff& \forall x\in X: \forall V\in \vicinity(x): \quad \neg(F(x) \amesh \vicinity(V^c)) \\
\implies& \forall x\in X: \forall V\in \vicinity(x): \quad \neg(F(x) \amesh F(V^c)).
\end{align*}
We have used \ref{setAdherenceInherence}.

(2) In the pretopological case, we can take $F = \vicinity$ and we can run the argument in reverse, because regularity is implied by $\vicinity(x)\subseteq \upset\adh[\vicinity(x)]$ for all $x\in X$.
\end{proof}

\begin{proposition} \label{topologicalRegularity}
Let $\sSet{X,\xi}$ be a topological space. Then the following are equivalent:
\begin{enumerate}
\item $\xi$ is regular;
\item for any $x\in X$ and any base $\mathcal{B}$ of $\neighbourhood_\xi(x)$, $\closure^\imf(\mathcal{B})$ is also a base of $\neighbourhood_\xi(x)$;
\item for any closed set $C$ there exist disjoint open sets $U,V$ such that $x\in U$ and $C\subseteq V$;
\item for any open set $O\subseteq X$ and $x\in O$ there exists an open set $U$ such that $x\subseteq U\subseteq \overline{U} \subseteq O$.
\end{enumerate}
\end{proposition}
\begin{proof}
$(1) \Leftrightarrow (2)$ TODO ref.

$(1) \Leftrightarrow (3)$ By \ref{regularityBySeparation}.

$(2) \Leftrightarrow (4)$ By \ref{neighbourhoodSeparationLemma}.
\end{proof}

\subsection{$T_3$ or regular Hausdorff}
\begin{definition}
Let $\sSet{X,\xi}$ be a convergence space. Then $\xi$ is called \udef{$T_3$} if it is regular and Hausdorff.
\end{definition}

\begin{proposition}
Let $X$ be a set and $\xi$ a convergence on $X$. Then the following are equivalent:
\begin{enumerate}
\item $\xi$ is a $T_3$ convergence, i.e.\ $R_2$ and $T_2$;
\item $\xi$ is $R_2$ and $T_0$.
\end{enumerate}
\end{proposition}

\subsection{$R_3$ or normal}
\begin{definition}
Let $\sSet{X,\xi}$ be a convergence space. Then $\xi$ is called \udef{normal} or \udef{$R_3$} if all disjoint closed sets are separated by convergent filters.
\end{definition}

\begin{proposition}
Let $\sSet{X,\xi}$ be a pretopological convergence space and $A,B\subseteq X$. Then $\lim^{-1}(\adh_\xi(A)) \perp \lim^{-1}(\adh_\xi(B))$ \textup{if and only if} there exist disjoint vicinities of $\adh_\xi(A)$ and $\adh_\xi(B)$.
\end{proposition}

\begin{proposition}
Any $R_3$ convergence is also $R_2$.
\end{proposition}


\subsection{$T_4$ or normal Hausdorff}
\begin{definition}
Let $\sSet{X,\xi}$ be a convergence space. Then $\xi$ is called \udef{$T_4$} if it is normal and Hausdorff.
\end{definition}

\begin{proposition}
Let $X$ be a set and $\xi$ a convergence on $X$. Then the following are equivalent:
\begin{enumerate}
\item $\xi$ is a $T_4$ convergence, i.e.\ $R_3$ and $T_2$;
\item $\xi$ is $R_3$ and $T_4$;
\item $\xi$ is $R_3$ and $T_0$.
\end{enumerate}
\end{proposition}


\section{Countability properties}
\subsection{$C1$ or first countable}
\begin{definition}
A convergence space $\sSet{X, \xi}$ is called \udef{first countable} or \udef{$C_1$} if for all $x\in X$ the vicinity filter $\vicinity_\xi(x)$ has a countable base.
\end{definition}

\subsubsection{Strongly first countable}

\subsection{$C2$ or second countable}
\begin{definition}
A convergence space $\sSet{X, \xi}$ is called \udef{second countable} or \udef{$C_2$} if $\xi$ has a countable base.
\end{definition}

\begin{lemma}
Second countable implies first countable.
\end{lemma}

\begin{lemma} \label{C2openBase}
Let $\sSet{X,\xi}$ be a $C_2$ topological convergence space. Then $\xi$ has a countable base of open sets.
\end{lemma}
\begin{proof}
See \ref{cardinalityPretopologicalBase}.
\end{proof}

\begin{lemma} \label{AnySetCountableIntersectionOfOpenSets}
Let $\sSet{X,\xi}$ be a $C_2$ and $T_1$ topological convergence space. Then every set $A\subseteq X$ can be written as both
\begin{enumerate}
\item a countable intersection of open sets; and
\item a countable union of closed sets.
\end{enumerate}
\end{lemma}
\begin{proof}
(1) By \ref{C2openBase}, $\xi$ has a countable base of open sets. By \ref{setKernelVicinityFilter} we have $A = \bigcap \neighbourhood(A) = \bigcap \upset \mathcal{F} = \bigcap \mathcal{F}$, for some subset $\mathcal{F}$ of the countable base of open sets.

(2) We can write $A^c$ as a countable intersection of open sets by (1). Taking the complement yields the result.
\end{proof}


\begin{proposition} \label{countableRegularityImpliesNormality}
Let $\sSet{X,\xi}$ be a topological space. If $\xi$ is regular and second countable, then $\xi$ is normal.
\end{proposition}
\begin{proof}
TOOD eg \url{https://www.math.auckland.ac.nz/~gauld/750-05/section3.pdf}
\end{proof}

\section{Comparison with reals and metrisability}
\subsection{Functional convergence properties}
\subsubsection{Functional closure}
\begin{definition}
Let $\sSet{X,\xi}$ be a convergence space and $A\subseteq X$. We call $A$ \udef{functionally closed} if there exists a continuous function $f: X\to \R$ and a closed set $C\subseteq \R$ such that $A = f^{-1}[C]$.
\end{definition}

\begin{proposition}
Every functionally closed set if closed. In a metric space the converse holds.
\end{proposition}
\begin{proof}
TODO + see \ref{distanceToSetContinuous}
\end{proof}

\begin{lemma} \label{functionallyClosedZeroSet}
Let $\sSet{X,\xi}$ be a convergence space and $A\subseteq X$. Then $A$ is functionally closed \textup{if and only if} there exists a continuous function $g: X\to \R$ such that $A = g^{-1}[\{0\}]$.
\end{lemma}
This means we may take $C = \{0\}$ in the definition of functionally closed.
\begin{proof}
If there exists such a function $g$, then $A$ is clearly functionally closed. For the converse, fix a continuous function $f: X\to \R$ and a closed set $C\subseteq \R$ such that $A = f^{-1}[C]$. Then we use the continuity of $d_C: \R\to\R: x\mapsto \inf_{c\in C}d(x,c)$ (see \ref{distanceToSetContinuous}) and set $g = d_C\circ f$.
\end{proof}

\subsubsection{Functional separation}
\begin{definition}
Let $\sSet{X,\xi}$ be a convergence space and $A,B\subseteq X$. Then $A$ and $B$ are \udef{functionally separated} if there exists a continuous function $f: X\to [0,1]\subseteq \R$ with $f[A] = \{0\}$ and $f[B] = \{1\}$.
\end{definition}

\begin{proposition}
Two sets are functionally separated \textup{if and only if} they are included in disjoint functionally closed sets.
\end{proposition}
\begin{proof}
First assume $A,B$ are functionally separated by $f$. Then $f^{-1}[\;\{0\}\;]$ and $f^{-1}[\;\{1\}\;]$ are disjoint functionally closed sets containing $A$, resp. $B$.

Conversely, assume $A\subseteq f^{-1}[\{0\}]$ and $B\subseteq g^{-1}[\{0\}]$ (we can take $C = \{0\}$ by \ref{functionallyClosedZeroSet}) with $f^{-1}[\{0\}]\perp g^{-1}[\{0\}]$. Then
\[ h: X\to \R: x\mapsto \frac{|f(x)|}{|f(x)|+|g(x)|} \]
is well-defined everywhere because $f^{-1}[\{0\}]\perp g^{-1}[\{0\}]$ and functionally separates $A$ and $B$.
\end{proof}

\begin{lemma} \label{UrysohnsLemmaLemma}
Let $\sSet{X,\xi}$ be a topological space and $D\subseteq \R$ a dense subset of the reals. Suppose $\sSet{U_d}{d\in D}$ is a set of open subsets of $X$ satisfying
\begin{enumerate}
\item if $d < t$, then $\overline{U_d}\subseteq U_t$;
\item $\bigcup_{d\in D}U_d = X$;
\item $\bigcap_{d\in D}U_d = \emptyset$.
\end{enumerate}
Then $f: X\to \R: x\mapsto \inf\setbuilder{d\in D}{x\in U_d}$ is continuous.
\end{lemma}
\begin{proof}
The function $f$ is well-defined, because $\setbuilder{d\in D}{x\in U_d}$ is never empty.

We will prove continuity at $x\in X$ using \ref{pretopologicalContinuityVicinities}. To that end, take some $U\in \vicinity_\R(f(x))$. By density, we can find some $d,t\in D$ such that $[d,t]\subseteq U$ and $d < f(x) < t$.

Now set $V = U_t\setminus\overline{U_d}$. It is then enough to prove that $V\in \vicinity_\xi(x)$ and $f[V]\subseteq [d,t]$. For the first point, note that $V = U_t\cap\overline{U_d}^c$, so $V$ is open.

Next we show $x\in V$. Indeed, from $f(x) < t$, we get that there exists $f(x) \leq s < t$ such that $x\in U_s$. Thus $x\in U_t$. From $d < f(x)$, we get that there exists $s>d$ such that $x\notin U_s$. By assumption ($d < t \implies \overline{U_d}\subseteq U_t$) this means that $x\notin \overline{U_d}$.

Finally note that if $y\in V \subseteq U_t$, then $f(y) \leq t$ and if $y\notin \overline{U_d}$, then $f(y) > s$ for all $s< d$. Thus $d \leq f(y)$ and $f[V]\subseteq [d,t]$.
\end{proof}

\begin{theorem}[Urysohn's lemma]
Let $\sSet{X,\xi}$ be a normal topological space. If $C_1,C_2$ are disjoint closed sets, then they are functionally separated.
\end{theorem}
\begin{proof}
We will construct a family $\setbuilder{U_r}{r\in Q}$ of open sets satisfying the conditions of \ref{UrysohnsLemmaLemma}.

First set $U_r = \emptyset$ for all $r<0$ and $U_r = X$ for all $r>1$.

Now set $U_1 = C_2^c$. By normality (TODO ref) there exists an open set $U_0$ such that $C_1 \subseteq U_0 \subseteq \overline{U_0} \subseteq U_1$.

We can enumerate the rationals in a sequence $\seq{r_n}_{n\in \N}$ with $r_0 = 0$ and $r_1 = 1$. We now recursively define $U_{r_n}$ with recursion invariant $d < t \implies \overline{U_d}\subseteq U_t$ as follows: Set
\[ V =  U_i \qquad \text{where $i = \max_{\substack{k < n \\ r_k < r_n}}k$} \qquad\text{and}\qquad W = U_j \qquad \text{where $j = \min_{\substack{k < n \\ r_k > r_n}}k$}. \]
By the recursion invariant, we have $\overline{U_i}\subseteq U_j$, so by normality we can define a $U_{r_k}$ such that $\overline{U_i} \subseteq U_{r_k} \subseteq \overline{U_{r_k}} \subseteq U_j$. This assignment satisfies the recursion invariant.

Then by \ref{UrysohnsLemmaLemma} the function $f: X\to \R: x\mapsto \inf\setbuilder{r\in \Q}{x\in U_r}$ is continuous. It functionally separates $C_1$ and $C_2$.
\end{proof}

\begin{corollary}[Urysohn's metrisation theorem]
Every regular and second countable topological space is pseudometrisable.
\end{corollary}
\begin{proof}
Let $\sSet{X,\xi}$ be such a topological space. By \ref{countableRegularityImpliesNormality}, $\xi$ is automatically normal. TODO embedding into $[0,1]^F$.
\end{proof}
TODO converse??


\subsubsection{Functional regularity}
\begin{definition}
Let $\sSet{X, \xi}$ and $\sSet{Y,\zeta}$ be convergence spaces. We call $\xi$ \udef{$\zeta$-functionally regular} if $\xi$ is the initial convergence on $X$ w.r.t. $\cont(\xi,\zeta)$.

In particular we call $\xi$ \udef{completely regular} if it is $\R$-functionally regular.
\end{definition}

\chapter{Compactness}
\begin{definition}
Let $\sSet{X,\xi}$ be a convergence space.
\begin{itemize}
\item The space $\sSet{X,\xi}$ is called \udef{compact} if every ultrafilter converges.
\item A subset $A\subseteq X$ is called \udef{compact} if the subspace $\sSet{A, \xi|_A}$ is compact.
\end{itemize}
\end{definition}

\begin{proposition}
Let $\sSet{X,\xi}$ be a convergence space and $A\subseteq X$ a subset.
\begin{enumerate}
\item If $X$ is compact and $A$ is closed, then $A$ is compact.
\item If $X$ is a Hausdorff space and $A$ is compact, then $A$ is closed.
\end{enumerate}
\end{proposition}
\begin{proof}
(1) Take some ultrafilter $F$ in $A$. Then $\upset F$ is an ultrafilter in $X$ by \ref{traceUltrafilters}, so it converges to some $x\in X$ by compactness. Now $A \in (\upset F)^{\mesh} \subseteq \vicinity(x)^{\mesh}$, so $x\in \adh(A) = A$. Thus $F \to x$ in $A$.

(2) We need to show that $\adh(A)\subseteq A$. Take $x\in \adh(A)$, meaning there exists a filter $F\to x$ such that $A\in F$ by \ref{adherenceInherenceCharacterisation}. By the ultrafilter lemma, \ref{ultrafilterLemma}, $F$ is contained in an ultrafilter $G$ and $G\to x$. Now $G\vee \upset\{A\}$ is proper (and thus an ultrafilter), so $G|_A$ is an ultrafilter in $A$ by \ref{traceUltrafilters}. Then $G|_A$ converges to a point in $A$, and $G$ converges to the same point in $X$. By Hausdorff this point must be $x$, so $x\in A$.
\end{proof}
\begin{corollary}
If $A$ is compact and $B\subseteq X$ is closed, then $A\cap B$ is compact.
\end{corollary}

\begin{proposition} \label{compactConstructions}
Let $f: \sSet{X,\xi}\to \sSet{Y,\zeta}$ be a continuous function.
\begin{enumerate}
\item If $X$ is compact, then $\im(f)$ is compact.
\item 
\end{enumerate}
\end{proposition}
\begin{proof}
It is enough to show that for any ultrafilter $G$ on $\im(f)$, there exists an ultrafilter $F$ on $X$ such that $f^{\imf\imf}[F] = G$.
\end{proof}

\begin{theorem}[Tychonoff]
Let $\{\sSet{X_i, \xi_i}\}_{i\in I}$ be a family of compact convergence spaces. Then $\prod_{i\in I}X_i$ is compact.
\end{theorem}

\section{Covers}
\begin{definition}
Let $\sSet{X,\xi}$ be a convergence space and $\mathcal{C}\subseteq \powerset(X)$ a family of sets. We call $\mathcal{C}$ 
\begin{itemize}
\item a \udef{cover of convergence filters} if $\mathcal{C}\mesh F$ for each convergent filter $F\in\powerfilters(X)$;
\item a \udef{cover of vicinities} if $\mathcal{C}\mesh \vicinity_\xi(x)$ for all $x\in X$;
\item a \udef{cover of neighbourhoods} if $\mathcal{C}\mesh \neighbourhood_\xi(x)$ for all $x\in X$;
\item an \udef{open cover} if $\mathcal{C} \subseteq \topology_\xi$ and $\bigcup \mathcal{C} = X$;
\item a \udef{cover} of $X$ if $\bigcup \mathcal{C} = X$.
\end{itemize}
\end{definition}

\begin{lemma}
Let $\sSet{X,\xi}$ be a convergence space.
\begin{enumerate}
\item Every cover of vicinities is a cover of convergent filters.
\item If $\xi$ is pretopological, then every cover of convergent filters is a cover of vicinities.
\end{enumerate}
\end{lemma}

\begin{lemma}
Let $\sSet{X,\xi}$ be a convergence space and $\mathcal{C}\subseteq \powerset(X)$ a family of sets. Then $\mathcal{C}$ is a cover of neighbourhoods \textup{if and only if} $\interior^\imf[\mathcal{C}]$ is an open cover.
\end{lemma}
\begin{proof}
First assume $\mathcal{C}$ is a cover of neighbourhoods. Clearly $\mathcal{C} \subseteq \topology_\xi$. For all $x\in X$ there exists $A\in \mathcal{C}$ such that $A\in \neighbourhood(x)$ and thus $\interior(A) \in \neighbourhood(x)$ by \ref{interiorModificationNeighbourhoods}. This means that $x\in \interior(A)$ and thus $X\subseteq \bigcup\interior^\imf[\mathcal{C}]$.

Next assume $\interior^\imf[\mathcal{C}]$ is an open cover. Then for all $x\in X$ there exists $A\in \mathcal{C}$ such that $x\in \interior(A)$, which implies $A\in \neighbourhood(x)$ by \ref{interiorClosureMembership}.
\end{proof}


\begin{proposition}
Let $\sSet{X,\xi}$ be a convergence space. Then the following are equivalent:
\begin{enumerate}
\item $X$ is compact;
\item every cover of convergent filters has a finite subset that covers $X$;
\item every cover of vicinities has a finite subset that covers $X$;
\end{enumerate}
\end{proposition}
\begin{proof}
$(1) \Rightarrow (2)$ Let $\mathcal{C}$ be a cover of convergent filters and assume, towards a contradiction, that $\mathcal{C}$ does not have a finite subset that covers $X$. Then
\[ \setbuilder{X\setminus(C_1\cup \ldots \cup C_n)}{n\in \N; C_1,\ldots, C_n \in \mathcal{C}} \]
is a filter base and does not contain $\emptyset$, so the filter is generates is proper and we can find an ultrafilter $G$ that contains it by the ultrafilter lemma \ref{ultrafilterLemma}.

Now $G$ is convergent, so $\mathcal{C}\mesh G$, i.e.\ there exists a $C\in \mathcal{C}\cap G$. By construction $X\setminus C\in G$, so $C\cap (X\setminus C)\in G$, which means $G$ cannot be an ultrafilter.

$(2) \Rightarrow (3)$ Every cover of vicinities is a cover of convergent filters and thus has a finite subset that covers $X$.

$(3) \Rightarrow (1)$ Assume, towards a contradiction, that $X$ is not compact. Then there exists an ultrafilter $G$ that does not converge. This means that $\vicinity(x) \not\subseteq G$ for all $x\in X$. Let $\mathcal{C}$ consist of one vicinity from $\vicinity(x)\setminus G$ for all $x\in X$. There exists a finite subset $\{C_1, \ldots, C_n\}\subseteq \mathcal{C}$ that covers $X$, so $C_1\cap \ldots \cap C_n = X \in G$. Now $G$ is prime by \ref{booleanMaximalFiltersIdeals} and thus at least one $C_1, \ldots, C_n$ is an element of $G$, which is a contradiction.
\end{proof}
\begin{corollary} \label{topologyCompactnessOpenCover}
Let $\sSet{X,\xi}$ be a topological convergence space. Then $X$ is compact \textup{if and only if} every open cover has a finite subset that covers $X$.
\end{corollary}

\section{Relative compactness}
\begin{definition}
Let $\sSet{X,\xi}$ be a convergence space and $A\subseteq X$ a subset. We call $A$ \udef{relatively compact} if $\adh_\xi$ is compact.
\end{definition}

\section{Local compactness}
\begin{definition}
A convergence space $\sSet{X, \xi}$ is called \udef{locally compact}
if every point has a compact vicinity.
\end{definition}


\chapter{Pretopological and Choquet convergence}
\section{Pretopological convergence}
A convergence space $\sSet{X,\xi}$ is pretopological if it $1$-paved, i.e.\ at each point $x$ there is a filter $G$ that converges to $x$ such that
\[ F\in {\lim}_\xi^{-1}(x) \implies G \leq F. \]

\begin{lemma}
Let $\xi$ be a convergence on a set $X$. Then the following are equivalent:
\begin{enumerate}
\item $\xi$ is a pretopology;
\item $\lim_\xi^{-1}(x)$ has a least element for all $x\in X$;
\item $\vicinity_\xi(x) \in \lim_\xi^{-1}(x)$ for all $x\in X$;
\item $\vicinity_\xi(x) \leq F \implies F\in\lim^{-1}_\xi x$;
\item $x\in\lim_\xi F \iff \vicinity_\xi(x) \leq F$.
\end{enumerate}
\end{lemma}
\begin{proof}
$(1) \Leftrightarrow (2)$. Because $G$ converging to $x$ is equivalent to $G\in \lim_\xi^{-1}(x)$, we see that $G$ is a least element of $\lim_\xi^{-1}(x)$.

$(2) \Leftrightarrow (3)$. The vicinity filter $\vicinity_\xi(x)$ is the infimum of $\lim_\xi^{-1}(x)$, and a set contains a least element iff it contains an infimum.

$(3) \Leftrightarrow (4)$. The set $\lim^{-1}_\xi x$ is upwards closed.

$(4) \Leftrightarrow (5)$. The opposite implication to (4) is immediate because $\vicinity_\xi(x)$ is the infimum of $\lim^{-1}_\xi$.
\end{proof}
Thus in a pretopological convergence space we can determine whether $x$ is in the limit of a filter $F$ by simply comparing $F$ to the vicinity filter of $x$.

\begin{proposition}
Let $\xi$ be a convergence on a set $X$. Then $\xi$ is a pretopological convergence \textup{if and only if} $\lim_\xi: \powerfilters(X)\to \powerset(X)$ is a complete meet-semilattice homomorphism, i.e.\
for all families of filters $\mathcal{F} \subseteq \powerset(\powerfilters(X))$,
\[ \lim_\xi \bigwedge_{F\in \mathcal{F}}F = \bigcap_{F\in\mathcal{F}}\lim_\xi F. \] 
\end{proposition}
\begin{proof}
We have
\begin{align*}
x\in \lim_\xi \bigwedge_{F\in \mathcal{F}}F &\iff \bigwedge_{F\in \mathcal{F}}F \geq \vicinity_\xi(x) \iff \forall F\in\mathcal{F}: F\geq \vicinity_\xi(x) \\
&\iff \forall F\in \mathcal{F}: x\in\lim_\xi F \iff x\in \bigcap_{F\in\mathcal{F}}\lim_\xi F.
\end{align*}
\end{proof}
\begin{corollary}
The set of pretopological convergences on a set $X$ forms a complete meet-subsemilattice of the lattice of convergences.
\end{corollary}
\begin{proof}
Complete meet-semilattice homomorphism form a complete meet-subsemilattice, \ref{semilatticeOfSemilatticeHomomorphisms}.
\end{proof}
\begin{corollary}
There exists a closure operator $S_0$ that maps a convergence $\xi$ to the finest pretopological convergence coarser than $\xi$.
\end{corollary}

\subsection{Pretopological modification}
\begin{definition}
Let $X$ be a set. The closure operator $S_0$ on the lattice of convergences on $X$ is called the \udef{pretopologiser}. For any convergence $\xi$ on $X$, the pretopological convergence $S_0\xi$ is called the \udef{pretopological modification} of $\xi$.
\end{definition}

\begin{proposition}
Let $X$ be a set and $\xi$ a convergence on $X$. Then $S_0\xi$ is defined by
\[ x\in \lim_{S_0\xi} F \iff F \geq \vicinity_\xi(x). \]
\end{proposition}

\begin{proposition}
Let $\sSet{X,\xi}$ and $\sSet{Y,\zeta}$ be two pretopological convergence spaces. Then the following are equivalent for a function $f: X\to Y$:
\begin{enumerate}
\item $f\in \cont(\xi, \zeta)$;
\item for all $x\in X$: $f[\vicinity_\xi(x)] \geq \vicinity_\zeta(f(x))$;
\item for all $A\subseteq X$: $f(\adh_\xi) \subseteq \adh_\zeta f[A]$;
\item for all $B\subseteq Y$: $f^{-1}[\inh_\zeta B] \subseteq \inh_\xi(f^{-1}[B])$.
\end{enumerate}
\end{proposition}

\section{Choquet convergence spaces}
\begin{definition}
A convergence space $\sSet{X, \xi}$ is called a \udef{Choquet space} if
\[ x\in \lim_\xi F \quad\iff\quad \forall U\in \ultrafilters(\powerset(X)): U \geq F \implies x\in\lim_\xi U. \]
\end{definition}

\[ \lim_\xi F = \bigcap_{U\in \ultrafilters(\powerset(X))\land F \subseteq U}\lim_\xi U. \]

A pretopological convergence is completely determined by the vicinity filters. A Choquet convergence is completely determined by the convergence of ultrafilters.

\begin{lemma}
Every pretopological convergence space is a Choquet space.
\end{lemma}

\subsection{Compact Choquet spaces}

\chapter{Related types of spaces}
\section{Cauchy spaces}
\begin{definition}
Let $\sSet{X,\xi}$ be a convergence space. The set
\[ \mathcal{F} \defeq \setbuilder{F\in\powerfilters(X)}{\text{$F$ converges}}. \]
is called the \udef{Cauchy structure} of $\xi$.
\end{definition}
A Cauchy space is a set $X$ together with a set of directed sets that could be the set of convergent directed sets in some convergence space.
\begin{definition}
Let $X$ be a set and $\mathcal{F}$ a family of filters in $\directed(\powerset(X))$ such that
\begin{itemize}
\item $\pfilter{x} \in \mathcal{F}$ for all $x\in X$;
\item $\mathcal{F}$ is upwards closed.
\end{itemize}
We call $\sSet{X, \mathcal{F}}$ a \udef{Cauchy space} and $\mathcal{F}$ a \udef{Cauchy structure}.

If only the second property holds, we call $\mathcal{F}$ a \udef{pre-Cauchy structure}.
\end{definition}
As with convergence spaces we may impose additional axioms.

\subsection{Equivalence and convergence}
TODO finite depth?
\begin{definition}
Let $\sSet{X, \mathcal{F}}$ be a Cauchy space. Define the relation $\sim$ on $(\mathcal{F},\mathcal{F})$ by
\[ F \sim G \qquad \defequiv \qquad F\cap G \in \mathcal{F}. \]
We call the filters $F,G$ equivalent.

We define the \udef{Cauchy convergence} on $X$ by
\[ F \to x \qquad \defequiv \qquad F \sim \pfilter{x}. \]
\end{definition}

\begin{lemma}
The Cauchy convergence is a convergence. In particular it is a Kent space.
\end{lemma}
\begin{proof}
Clearly $\pfilter{x}\cap\pfilter{x} = \pfilter{x} \in \mathcal{F}$, so $\pfilter{x} \to x$.

Let $F\to x$ and $F\subseteq G$. Then $G\in \mathcal{F}$ and $G\cap \pfilter{x} \supseteq F \cap\pfilter{x} \in \mathcal{F}$, so $G \to x$.

The Kent property is immediate.
\end{proof}

\begin{lemma}
Let $\sSet{X,\xi}$ be a convergence space. The Cauchy convergence on the Cauchy space $\sSet{X,\mathcal{F}}$ is the dual Kent closure of $\xi$.
\end{lemma}

\begin{proposition}
Let $\sSet{X, \mathcal{F}}$ be a Cauchy space of finite depth. Then
\begin{enumerate}
\item the relation $\sim$ is an equivalence relation;
\item the Cauchy convergence is of finite depth;
\item the Cauchy convergence is $R_1$.
\end{enumerate}
\end{proposition}

\subsubsection{Cauchy continuity}
TODO: equivalent to continuity of Cauchy convergence?

\subsection{Completeness}
\begin{definition}
Let $\sSet{X, \mathcal{F}}$ be a Cauchy space. We call $X$ \udef{Cauchy complete} (or just \udef{complete}) if every $F\in \mathcal{F}$ converges in the Cauchy convergence.
\end{definition}

\begin{proposition}
\begin{enumerate}
\item Each closed subspace of a complete Cauchy space is complete.
\item A subspace of a complete Hausdorff Cauchy space is complete \textup{if and only if} it is closed.
\item The product of complete Cauchy spaces is complete (TODO products!).
\item Each compact uniform convergence is complete.
\end{enumerate}
\end{proposition}
\begin{proof}
TODO
\end{proof}

\subsubsection{Completion}
\begin{definition}
Let $\sSet{X, \mathcal{F}}$ be a Cauchy space. A \udef{completion} of $\sSet{X, \mathcal{F}}$ is another Cauchy space $\sSet{Y, \mathcal{G}}$ and a function $k: X\to Y$ such that
\begin{itemize}
\item $k: X\to Y$ is a Cauchy embedding;
\item $k[X]$ is dense in $Y$.
\end{itemize}
\end{definition}

\section{Closure spaces}
\url{https://www.researchgate.net/profile/Peter-Stadler-2/publication/239066337_Higher_Separation_Axioms_in_Generalized_Closure_Spaces/links/53d1cf440cf2a7fbb2e95303/Higher-Separation-Axioms-in-Generalized-Closure-Spaces.pdf?origin=publication_detail}

\section{Merotopic and nearness spaces}
\url{https://en.wikipedia.org/wiki/Proximity_space}

\chapter{Uniform convergence}
\section{Uniformities}
\subsection{Operations on filters}
\begin{definition}
Let $X$ be a set and $F,G\in\powerfilters(X^2)$. We define
\[ F^{\transp} \defeq \setbuilder{A^\transp}{A\in F}  \]
and
\[ F;G \defeq \upset \setbuilder{A;B}{A\in F, B\in G}. \]
\end{definition}

We have $F^{\transp} = t^{\imf\imf}[F]$ where $t: X^2 \to X^2: (x,y)\mapsto (y,x)$.

TODO Galois connection $\powerfilters(X^2) \leftrightarrow \powerfilters(X)^2$.

\begin{lemma} \label{meshingFilterComposition}
Let $X$ be a set and $F,G\in\powerfilters(X^2)$. If $F;G \neq \powerset(X^2)$, then $p_2^{\imf\imf}[F]\amesh p_1^{\imf\imf}[G]$.
\end{lemma}
\begin{proof}
Assume $F;G \neq \powerset(X^2)$ and, towards a contradiction, that there exist $A\in F$ and $B\in G$ such that $p_2^{\imf}[A]\perp p_1^{\imf}[B]$. Then $\emptyset = A;B$, so $\emptyset \in F;G$. This means that $F;G = \powerset(X^2)$, which is a contradiction.
\end{proof}
TODO with Galois connection??

\begin{lemma} \label{compositionProductFilters}
Let $X$ be a set and $F,G, G', H\in\powerfilters(X)$. Then
\[ (F\otimes G);(G'\otimes H) = \begin{cases}
F\otimes H & (G\amesh G') \\
\powerset(X ^2) & (\text{otherwise}).
\end{cases}  \]
\end{lemma}
\begin{proof}
TODO
\end{proof}

Note that for all proper filters $G$ we have $G\amesh G$. However $\powerset(X)\cancel\amesh\powerset(X)$, so $\big(F\otimes \powerset(X)\big);\big(\powerset(X)\otimes G\big) = \powerset(X^2)$, independent of $F$ and $G$.

\begin{lemma} \label{componentInclusionsFilterComposition}
Let $X$ be a set and $F,G\in\powerfilters(X^2)$. Then
\begin{enumerate}
\item $p_1^{\imf\imf}[F]\subseteq p_1^{\imf\imf}[F;G]$;
\item $p_2^{\imf\imf}[G]\subseteq p_2^{\imf\imf}[F;G]$.
\end{enumerate}
\end{lemma}
\begin{proof}
(1) This follows immediately from $p_1^{\imf}[A;B] = \setbuilder{p_1((x,y))}{(x,y)\in A \land y\in p_1^{\imf}(B)} \subseteq p_1^{\imf}[A]$.

(2) Similar.
\end{proof}

\begin{lemma} \label{filterCompositionFactorisationLemma}
Let $X$ be a set, $H\in \powerfilters(X^2)\setminus \{\powerset(X^2)\}$ and $F,G\in\powerfilters(X)$. Then
\[ F\otimes G = F\otimes p_1^{\imf\imf}(H); H; p_2^{\imf\imf}(H)\otimes G. \]
\end{lemma}
\begin{proof}
This follows from the fact that for all $A\in F$, $B\in G$ and $C,D,E\in H$,
\[ A\times B = \big(A \times p_1^{\imf}(C)\big);D;\big(p_2^\imf(E)\times B\big). \]
Indeed we have
\begin{align*}
(a,b)\in \big(A \times p_1^{\imf}(C)\big);D;\big(p_2^\imf(E)\times B\big) &\iff \exists c,d: \begin{cases}
(a,c)\in A\times p_1^\imf(C),\\ (c,d)\in D,\\ (d,b)\in p_2^\imf(E)\times B
\end{cases} \\
&\iff \begin{cases}
a\in A, b\in B, \\
\exists c,d: \; (c,d)\in D, c\in p_1^\imf(C), d\in p_2^\imf(E)
\end{cases} \\
&\iff a\in A, b\in B \\
&\iff (a,b)\in A\times B.
\end{align*}
The statement $\exists c,d: \; (c,d)\in D, c\in p_1^\imf(C), d\in p_2^\imf(E)$ is true because we may take $(c,d)\in C\cap D\cap E$, which is not empty because $H$ is proper.
\end{proof}
Set $I\defeq F\otimes p_1^{\imf\imf}(H); H; p_2^{\imf\imf}(H)\otimes G$. The inclusion $\subseteq$ can also be calculated using \ref{componentInclusionsFilterComposition} and \ref{filterFactorisationInequality}:
\[ F\otimes G = p_1^{\imf\imf}\big[F\otimes p_1^{\imf\imf}(H)\big] \otimes p_2^{\imf\imf}\big[p_2^{\imf\imf}(H)\otimes G\big] \subseteq p_1^{\imf\imf}[I]\otimes p_2^{\imf\imf}[I] \subseteq I. \]

\subsection{Uniformities}
\begin{definition}
Let $X$ be a set. Let $\mathcal{U}$ be a set of filters in $\powerfilters(X^2)$. Suppose
\begin{itemize}
\item $\pfilter{x}\otimes \pfilter{x} \in \mathcal{U}$ for all $x\in X$;
\item $\mathcal{U}$ is upwards closed;
\item if $F\in \mathcal{U}$, then $F^\transp\in\mathcal{U}$;
\item if $F, G\in \mathcal{U}$, then $F;G\in\mathcal{U}$.
\end{itemize}
Then we call $\mathcal{U}$ a \udef{uniformity} on $X$ and $\sSet{X,\mathcal{U}}$ a uniform space.

If we drop the first condition, we get a \udef{preuniformity} and a \udef{preuniform space}.

A uniformity is called \udef{factorisable} if for all $H\in\mathcal{U}$, there exist $F,G\in \powerfilters(X^2)$ such that $F\otimes G \in\mathcal{U}$ and $F\otimes G\subseteq H$.
\end{definition}

\begin{lemma}
Let $\sSet{X,\mathcal{U}}$ be a uniform space. Then $\mathcal{U}$ is factorisable \textup{if and only if} $p_1^{\imf\imf}(H)\otimes p_2^{\imf\imf}(H) \in \mathcal{U}$ for all $H\in\mathcal{U}$.
\end{lemma}
\begin{proof}
Assume $\mathcal{U} \ni F\otimes G \subseteq H$. Then
\[ F\otimes G = p_1^{\imf\imf}(F\otimes G)\otimes p_2^{\imf\imf}(F\otimes G) \subseteq p_1^{\imf\imf}(H)\otimes p_2^{\imf\imf}(H) \in \mathcal{U}, \]
where we have used \ref{projectionsOfProductFilter}.
\end{proof}

\subsubsection{Diagonality}
\begin{lemma}
Let $X$ be a set, $\mathcal{F}$ an upwards closed set of filters in $\powerfilters(X)$ and $\mathcal{C}$ a cover of $X$. If
\[ \forall C\in\mathcal{C}: \; \upset\{\Delta_C\} \in \mathcal{F}, \]
where $\Delta_C = \setbuilder{(c,c)}{c\in C}$, then $\pfilter{x}\otimes\pfilter{x}\in\mathcal{F}$ for all $x\in X$.
\end{lemma}
\begin{proof}
Because $\{(x,x)\}\subseteq \Delta_C$ for some $C\in\mathcal{C}$ and $\pfilter{x}\otimes\pfilter{x} = \upset \big\{\{(x,x)\}\big\} \supseteq \upset\{\Delta_C\} \in\mathcal{F}$.
\end{proof}

\begin{definition}
Let $X$ be a set and $\mathcal{S}$ be a set of subsets of $X$. A uniformity $\mathcal{U}$ on $X$ is called \udef{$\mathcal{S}$-diagonal} if $\forall S\in\mathcal{S}: \; \upset\{\Delta_S\} \in \mathcal{U}$.

The uniformity is simply called \udef{diagonal} if $\mathcal{S} = \{X\}$.
\end{definition}
We may call the first requirement in the definition of uniform space ``pointwise diagonality''.

\begin{lemma}
Let $\sSet{X,\mathcal{U}}$ be a uniform space and $\mathcal{S}_1, \mathcal{S}_2$ sets of subsets of $X$. If $\mathcal{S}_1 \subseteq \downset \mathcal{S}_2$, then $\mathcal{S}_2$-diagonality implies $\mathcal{S}_1$-diagonality.
\end{lemma}
\begin{proof}
Assume $\mathcal{S}_1 \subseteq \downset \mathcal{S}_2$ and that $\mathcal{U}$ is $\mathcal{S}_2$-diagonal.

Take $S\in \mathcal{S}_1$. Then there exists an $S'\in \mathcal{S}_2$ such that $S\subseteq S'$. Thus $\upset\{\Delta_{S'}\} \subseteq \upset\{\Delta_{S}\}$ and $\upset\{\Delta_{S'}\}\in \mathcal{U}$. By upwards closure, $\upset\{\Delta_{S}\} \in \mathcal{U}$.
\end{proof}

\subsubsection{Entourages}
\begin{definition}
Let $\sSet{X, \mathcal{U}}$ be a uniform space. Then $\entourage_\mathcal{U} \defeq \bigcap \mathcal{U}$ is called the \udef{entourage filter} of $\mathcal{U}$ and the elements of $\entourage_\mathcal{U}$ are called \udef{entourages}.

We call the uniform space \udef{topological} if $\mathcal{U} = \upset\{\entourage_\mathcal{U}\}$.
\end{definition}

\begin{lemma}
Let $\sSet{X,\mathcal{U}}$ be a uniform space with entourage filter $\entourage$. Then
\begin{enumerate}
\item $\entourage\subseteq \upset\{\Delta\}$, where $\Delta = \setbuilder{(x,x)}{x\in X}$;
\item $\entourage^\transp = \entourage$;
\item $\entourage;\entourage \subseteq \entourage$.
\end{enumerate}
If $\sSet{X,\mathcal{U}}$ is a topological uniform space, then
\begin{enumerate}[1'.] \setcounter{enumi}{2}
\item $\entourage;\entourage = \entourage$.
\end{enumerate}
Any filter in $\powerfilters(X\times X)$ satisfying properties 1., 2. and 3'. is the entourage filter of a topological uniformity.
\end{lemma}
TODO: can we improve 3??
\begin{proof}
(1) We have
\[ \entourage = \bigcap \mathcal{U} \subseteq \bigcap \setbuilder{\pfilter{x}\otimes \pfilter{x}}{x\in X} = \upset\big\{\setbuilder{(x,x)}{x\in X}\big\}. \]

(2) We have
\[ \entourage^\transp = \bigcap \setbuilder{H^\transp}{H\in\mathcal{U}} = \bigcap \setbuilder{H}{H\in\mathcal{U}} = \entourage, \]
because the transpose is bijective and thus its image function is preserved under intersection.

(3) First take $A;B \in\entourage;\entourage$. We claim $A\subseteq A;B$. Indeed take $(a,b)\in A$. By (1), we have $(b,b)\in B$ and so $(a,b)\in A;B$. Thus $\entourage;\entourage \subseteq \entourage$.

(3') In the topological case, we have that $\entourage\in\mathcal{U}$ and thus $\entourage;\entourage\in \mathcal{U}$, so $\entourage \subseteq \entourage;\entourage$.

To show any such filter is an entourage filter, we check the four requirements
\begin{itemize}
\item From (1), we have for all $x\in X$
\[ \entourage \subseteq \upset\{\Delta\} \subseteq \upset\{(x,x)\} = \pfilter{x}\otimes\pfilter{x}. \]
\item Upwards closure is by construction.
\item If $\entourage \subseteq H$, then $\entourage = \entourage^\transp \subseteq H^\transp$.
\item If $\entourage\subseteq G,H$, then $\entourage = \entourage;\entourage \subseteq G;H$.
\end{itemize}
\end{proof}
\begin{corollary}
A topological uniform space is diagonal.
\end{corollary}

\begin{proposition}
If a uniform space is topological and factorisable, then it is trivial.
\end{proposition}
\begin{proof}
Let $\sSet{X,\mathcal{U}}$ be a topological and factorisable uniform space with entourage filter $\entourage_\mathcal{U}$. Then $\entourage_\mathcal{U} \subseteq p_1^{\imf\imf}[\entourage_\mathcal{U}]\otimes p_2^{\imf\imf}[\entourage_\mathcal{U}]$. Now each $A\in \entourage_\mathcal{U}$ contains $\Delta$, so $p_1^\imf[A] = X = p_2^\imf[A]$. Thus $p_1^{\imf\imf}[\entourage_\mathcal{U}]\otimes p_2^{\imf\imf}[\entourage_\mathcal{U}] = \{X^2\}$.

Now $\entourage_\mathcal{U} = \upset\big\{\{X^2\}\big\}$, which is trivial.
\end{proof}

\subsection{Uniform relations}
\begin{definition}
Let $X$ be a sets. A \udef{uniform relation} on $X$ is a relation $R_\mathcal{U}$ on $\powerfilters(X)^2$ such that
\begin{itemize}
\item $\pfilter{x}\mathrel{R_\mathcal{U}}\pfilter{x}$ for all $x\in X$;
\item $F\,R_\mathcal{U}$ is upwards closed for all $F\in \powerfilters(X)$;
\item $R_\mathcal{U}$ is symmetric;
\item $R_\mathcal{U}$ is transitive when restricted to $\powerfilters(X)\setminus\{\powerset(X)\}$.
\end{itemize}
We call the structured set $\sSet{X,R_\mathcal{U}}$ a \udef{uniform relation space}.
\end{definition}

\begin{lemma} \label{uniformRelationRelatedElementLemma}
Let $R_\mathcal{U}$ be a uniform relation on $X$ and $F,G\in\powerfilters(X)$. Assume $F\mathrel{R_\mathcal{U}}\setminus\{\powerset(X)\} \neq \emptyset \neq G\mathrel{R_\mathcal{U}}\setminus\{\powerset(X)\}$. Then
\begin{enumerate}
\item $F\mathrel{R_\mathcal{U}} F$;
\item if $F\amesh G$, then $F\mathrel{R_\mathcal{U}} G$.
\end{enumerate}
\end{lemma}
\begin{proof}
(1) There exists $H\in\powerfilters(X)\setminus\{\powerset(X)\}$ such that $F\mathrel{R_\mathcal{U}} H$. By symmetry we have $H\mathrel{R_\mathcal{U}} F$ and by transitivity $F\mathrel{R_\mathcal{U}} F$.

(2) If $F\amesh G$, then $F\vee G \neq \powerset(X)$ by \ref{joinProperFilter}. From (1) we have $F\mathrel{R_\mathcal{U}} F$ and $G\mathrel{R_\mathcal{U}} G$. By upwards closure, $F\mathrel{R_\mathcal{U}} (F\vee G)$ and $G\mathrel{R_\mathcal{U}} (F\vee G)$. By transitivity and symmetry, $F\mathrel{R_\mathcal{U}} G$.
\end{proof}

\begin{lemma} \label{uniformRelationUpwardsClosure}
Let $R_\mathcal{U}$ be a uniform relation on $X$ and $F,G, F', G'\in\powerfilters(X)$. If $F \subseteq F'$, $G\subseteq G'$ and $F\mathrel{R_\mathcal{U}} G$, then $F'\mathrel{R_\mathcal{U}} G'$.
\end{lemma}
\begin{proof}
By upwards closure, we have $F\mathrel{R_\mathcal{U}} G'$ and  (using symmetry) $F'\mathrel{R_\mathcal{U}} G$. Thus, again using symmetry,
\[ F'\mathrel{R_\mathcal{U}} G \;\text{and}\; G \mathrel{R_\mathcal{U}} F \;\text{and}\; F \mathrel{R_\mathcal{U}} G', \]
so $F'\mathrel{R_\mathcal{U}} G'$ by transitivity.
\end{proof}

\begin{proposition} \label{uniformRelationGaloisConnection}
Let $X$ be a set; $\mathcal{U}$ a uniformity on $X$ and $R$ a uniform relation on $X$. We define a uniformity $\Theta(R)$ and a uniform relation $\Xi(\mathcal{U})$ by
\begin{align*}
\forall H\in \powerfilters(X^2):\qquad H\in\Theta(R) \quad&\defequiv\quad p_1^{\imf\imf}[H]\mathrel{R}p_2^{\imf\imf}[H]; \\
\forall F,G\in\powerfilters(X)\setminus\{\powerset(X)\}:\qquad F\mathrel{\Xi(\mathcal{U})}G \quad&\defequiv\quad F\otimes G\in \mathcal{U}. 
\end{align*}
Then the functions
\begin{align*}
&\Theta: \{\text{uniform relations on $X$}\} \to \{\text{uniformities on $X$}\} \\
&\Xi: \{\text{uniformities on $X$}\} \to \{\text{uniform relations on $X$}\}
\end{align*}
form a Galois connection $(\Theta, \Xi)$. Additionally,
\begin{enumerate}
\item $\im(\Theta)$ is the set of factorisable uniformities on $X$;
\item $\im(\Xi)$ is the set of uniform relations on $X$, i.e.\ $\Xi$ is surjective.
\end{enumerate}
\end{proposition}
\begin{proof}
The prove $\Theta(R)$ is a uniformity, we verify the conditions:
\begin{itemize}
\item From $\pfilter{x}\mathrel{R}\pfilter{x}$, we get $\pfilter{x}\otimes \pfilter{x}\in\Theta(R)$.
\item If $H\in\Theta(R)$ and $H\subseteq H'$, then $p_1^{\imf\imf}[H]\mathrel{R} p_2^{\imf\imf}[H]$, $p_1^{\imf\imf}[H]\subseteq p_1^{\imf\imf}[H']$ and $p_2^{\imf\imf}[H]\subseteq p_2^{\imf\imf}[H']$. Thus, by \ref{uniformRelationUpwardsClosure}, we have $p_1^{\imf\imf}[H']\mathrel{R} p_2^{\imf\imf}[H']$ and so $H'\in\Theta(R)$.
\item Take $H\in \Theta(R)$. Then
\[ p_1^{\imf\imf}[H]\mathrel{R} p_2^{\imf\imf}[H] \iff p_2^{\imf\imf}[H^\transp]\mathrel{R} p_1^{\imf\imf}[H^\transp] \iff p_1^{\imf\imf}[H^\transp]\mathrel{R} p_2^{\imf\imf}[H^\transp] \iff H^\transp\in \Theta(R). \]
\item Take $H_1, H_2\in \Theta(R)$. If $H_1;H_2 = \powerset(X^2)$, then $H_1;H_2\in\Theta(R)$ by upwards closure. If $H_1;H_2 \neq \powerset(X^2)$, then $p_2^{\imf\imf}[H_1]\amesh p_1^{\imf\imf}[H_2]$ by \ref{meshingFilterComposition} and thus $p_2^{\imf\imf}[H_1]\mathrel{R} p_1^{\imf\imf}[H_2]$ by \ref{uniformRelationRelatedElementLemma} (we have that $p_1^{\imf\imf}[H_1]\mathrel{R}p_2^{\imf\imf}[H_1]$, so $p_1^{\imf\imf}[H_1]\mathrel{R}\neq \emptyset$. Similarly $p_1^{\imf\imf}[H_2]\mathrel{R}p_2^{\imf\imf}[H_2]$ and $p_1^{\imf\imf}[H_2]\mathrel{R}\neq \emptyset$). So we have
\[ p_1^{\imf\imf}[H_1]\;\mathrel{R} \;p_2^{\imf\imf}[H_1] \;\mathrel{R} \;p_1^{\imf\imf}[H_2]\; \mathrel{R} \;p_2^{\imf\imf}[H_2]. \]
By \ref{componentInclusionsFilterComposition} and upward closure, we get $p_1^{\imf\imf}[H_1;H_2]\mathrel{R}p_2^{\imf\imf}[H_1;H_2]$, which means $H_1;H_2\in \Theta(R)$.
\end{itemize}

The prove $\Xi(\mathcal{U})$ is a uniform relation, we verify the conditions:
\begin{itemize}
\item From $\pfilter{x}\otimes \pfilter{x}\in\mathcal{U}$, we get $\pfilter{x}\mathrel{\Xi(\mathcal{U})}\pfilter{x}$.
\item Assume $F\mathrel{\Xi(\mathcal{U})}G$ and $G\subseteq G'$. Then $F\otimes G\in \mathcal{U}$ and $F\otimes G\subseteq F\otimes G'$. By upwards closure, $F\otimes G'\in\mathcal{U}$.
\item Symmetry is immediate from $(F\otimes G)^\transp = G\otimes F$.
\item For transitivity, assume $F\mathrel{\Xi(\mathcal{U})}G$, $G\mathrel{\Xi(\mathcal{U})}H$ and $G \neq \powerset(X)$. Then $F\otimes G, G\otimes H\in\mathcal{U}$ and $G \amesh G$ (this would not hold if $G = \powerset(X)$), so $(F\otimes G);(G\otimes H) = F\otimes H$ by \ref{compositionProductFilters}. Thus $F\mathrel{\Xi(\mathcal{U})}H$.
\end{itemize}

To show $(\Theta,\Xi)$ is a Galois connection, we need to prove that $\Theta(R) \subseteq \mathcal{U}$ \textup{if and only if} $R \subseteq \Xi(\mathcal{U})$.

First assume $\Theta(R) \subseteq \mathcal{U}$ and take $F,G\in\powerfilters(X)$ such that $F\mathrel{R} G$. Then $F\otimes G\in \Theta(R)\subseteq\mathcal{U}$ which means that $F\mathrel{\Xi(\mathcal{U})}G$.

Now assume $R \subseteq \Xi(\mathcal{U})$ and take $H\in \Theta(R)$. Then $p_1^{\imf\imf}[H]\mathrel{R}p_2^{\imf\imf}[H]$, which implies $p_1^{\imf\imf}[H]\mathrel{\Xi(\mathcal{U})}p_2^{\imf\imf}[H]$. Thus $p_1^{\imf\imf}[H]\otimes p_2^{\imf\imf}[H]\in \mathcal{U}$. By upwards closure and \ref{filterFactorisationInequality} we have $H\in \mathcal{U}$.

(1) It is clear that $\im(\Theta)$ consists of factorisable uniformities. For the other inclusion, let $\mathcal{U}$ be a factorisable uniformity. It is enough to show that $\mathcal{U} \subseteq \Theta(\Xi(\mathcal{U}))$. Take $H\in \mathcal{U}$. By factorisability $p_1^{\imf\imf}[H]\otimes p_2^{\imf\imf}[H]\in \mathcal{U}$. Then $p_1^{\imf\imf}[H]\mathrel{\Xi(\mathcal{U})} p_2^{\imf\imf}[H]$ and thus $H\in \Theta(\Xi(\mathcal{F}))$.

(2) It is enough to prove that for all uniform relations $R$ we have $\Xi(\Theta(R)) \subseteq R$. Take $F,G\in\powerfilters(X)$. We have
\[ F\mathrel{\Xi(\Theta(R))} G \implies F\otimes G \in\Theta(R) \implies F\mathrel{R}G. \]
\end{proof}


\subsubsection{Uniform convergence}
\begin{definition}
Let $\sSet{X,R}$ be a uniform relation space. Then the \udef{uniform convergence} $\Gamma(R)$ on $X$ is defined by
\[ F \overset{\Gamma(R)}{\longrightarrow} x \qquad\defequiv\qquad F\mathrel{R}\pfilter{x}. \]
We also denote the uniform convergence by $F\overset{u}{\longrightarrow} x$.
\end{definition}
If $\mathcal{U}$ is a uniformity, we write $\Gamma(\mathcal{U})$ to mean $\Gamma(\Xi(\mathcal{U}))$. We have
\[ F \overset{\Gamma(\mathcal{U})}{\longrightarrow} x \qquad\iff\qquad F\otimes \pfilter{x}\in\mathcal{U}. \]

\begin{lemma} \label{associatedUniformConvergence}
A uniform convergence is a convergence. It is also reciprocal ($R_1$).
\end{lemma}
\begin{proof}
Let $\sSet{X,R}$ be a uniform relation space. We have that $\Gamma(R)$ is centered, i.e.\ $\pfilter{x} \overset{\Gamma(R)}{\longrightarrow} x$, because $\pfilter{x}\mathrel{R}\pfilter{x}$.

We have that $\Gamma(R)$ is monotonic by upwards closure of $R$.

We prove reciprocity of $\Gamma(R)$ using point (5). of \ref{R1Conditions}. Assume $F \overset{\Gamma(R)}{\longrightarrow} x$ and $F \overset{\Gamma(R)}{\longrightarrow} y$. Then $F\mathrel{R}\pfilter{x}$ and $F\mathrel{R}\pfilter{y}$, so $\pfilter{x}\mathrel{R}\pfilter{y}$ by symmetry and transitivity. This implies
\[ G \in {\lim}_{\Gamma(R)}^{-1}(x) \iff G\mathrel{R}\pfilter{x}\iff G\mathrel{R}\pfilter{y} \iff G\in {\lim}_{\Gamma(R)}^{-1}(y), \]
and so $\lim_{\Gamma(R)}^{-1}(x) = \lim_{\Gamma(R)}^{-1}(y)$.
\end{proof}

\begin{proposition} \label{topologicalInducedUniformConvergence}
Let $\sSet{X,\mathcal{U}}$ be a topological uniform space. Then the associated convergence $\Gamma(\mathcal{U})$ is topological and
\[ \neighbourhood_{\Gamma(\mathcal{U})}(x) = \upset \setbuilder{V x}{V\in \entourage_\mathcal{U}}. \]
\end{proposition}
\begin{proof}
We first show that $\Gamma(\mathcal{U})$ is pretopological with vicinity filter $\upset \setbuilder{V x}{V\in \entourage_\mathcal{U}}$:

Because $\entourage_\mathcal{U}$ is a topological entourage filter, we have
\begin{align*}
\entourage_\mathcal{U} &= \entourage_\mathcal{U};\entourage_\mathcal{U} \\
&\subseteq \entourage_\mathcal{U}; \pfilter{x}\otimes \pfilter{x} \\
&= \upset \setbuilder{V; \{(x,x)\}}{V\in \entourage_\mathcal{U}} \\
&= \upset \setbuilder{V x\times\{x\}}{V\in \entourage_\mathcal{U}} \\
&= \upset \setbuilder{V x}{V\in \entourage_\mathcal{U}}\otimes \pfilter{x}.
\end{align*}
Thus $\upset \setbuilder{V x}{V\in \entourage_\mathcal{U}} \overset{\Gamma(\mathcal{U})}{\longrightarrow} x$.

Now we show that if $F\overset{\Gamma(\mathcal{U})}{\longrightarrow} x$, then $\upset \setbuilder{V x}{V\in \entourage_\mathcal{U}}\subseteq F$. Indeed we have
\begin{align*}
F\otimes \pfilter{x}\in \mathcal{U} \implies& \entourage_\mathcal{U}\subseteq F\otimes \pfilter{x} \\
\implies& \forall V\in \entourage_\mathcal{U}: \exists A\in F: \; A\times \{x\} \subseteq V \\
\implies& \forall  V\in \entourage_\mathcal{U}: \exists A\in F: \; A \subseteq V x \\
\implies& \upset \setbuilder{V x}{V\in \entourage_\mathcal{U}}\subseteq F.
\end{align*}

Finally to show that $\Gamma(\mathcal{U})$ is topological, we use \ref{pretopologicalSpaceTopological}: Take $Vx\in \upset \setbuilder{V x}{V\in \entourage_\mathcal{U}}$. Then because $\entourage_\mathcal{U} = \entourage_\mathcal{U};\entourage_\mathcal{U}$, we can find $U,U'\in \entourage_\mathcal{U}$ such that $V = U;U'$. Consider $U'x$. For all $y\in U'x$, we have that $zUy \implies zU;U'x \iff zVx$, so $Uy \subseteq Vx$. Thus $Vx \in \upset \setbuilder{V y}{V\in \entourage_\mathcal{U}}$.
\end{proof}


\subsubsection{Uniform Cauchy structure}
\begin{definition}
Let $\sSet{X,R}$ be a uniform relation space and let $\mathcal{F}\subseteq \powerfilters(X)$ be defined by
\[ F\in \mathcal{F} \qquad\defequiv\qquad F\mathrel{R} F. \]
Then $\mathcal{F}$ is called the \udef{induced Cauchy structure} and $\sSet{X,\mathcal{F}}$ is called the \udef{induced Cauchy space}.
\end{definition}
\begin{lemma}
The induced Cauchy space $\sSet{X, \mathcal{F}}$ of a uniform relation space $\sSet{X,R}$ is a Cauchy space.
\end{lemma}
\begin{proof}
We immediately have $\pfilter{x}\in\mathcal{F}$ for all $x\in X$ because $\pfilter{x}\mathrel{R} \pfilter{x}$.

We need to show upwards closure. Let $F\in \mathcal{F}$, meaning $F\mathrel{R} F$, and $F\subseteq G$. By upwards closure of $F\mathrel{R}$, we have $F\mathrel{R} G$. By symmetry we have $G\mathrel{R} F$ and by transitivity $G\mathrel{R} G$, so $G\in\mathcal{F}$.
\end{proof}

\begin{lemma}
Let $\sSet{X, R}$ be a uniform relation space, $F$ a Cauchy filter and $F\mathrel{R} G$. Then $G$ is a Cauchy filter.
\end{lemma}
\begin{proof}
The relationships $F\mathrel{R} F$ and $F\mathrel{R} G$ imply $G\mathrel{R} G$ by transitivity and symmetry.
\end{proof}


\begin{lemma} \label{uniformlyConvergentImpliesCauchy}
Let $\sSet{X,R}$ be a uniform relation space and $F\in\powerfilters(X)$. If $F$ is uniformly convergent (i.e.\ $F\mathrel{R}\pfilter{x}$ for some $x\in X$), then $F$ is a Cauchy filter.
\end{lemma}
\begin{proof}
If $F$ converges uniformly to $x$, then $F\mathrel{R}\pfilter{x}$ and by symmetry also $\pfilter{x}\mathrel{R}F$. By transitivity $F\mathrel{R}F$. 
\end{proof}
If the converse to this lemma holds, then the uniform space is called complete.

\paragraph{Completeness}

\begin{definition}
A uniform relation space $\sSet{X,R}$ is called \udef{complete} if all Cauchy filters converge.
\end{definition}

A uniform space is complete iff for all $F\in\powerfilters(X)$,
\[ F\mathrel{R}F \quad\iff\quad \exists x\in X:\; F\mathrel{R}\pfilter{x}. \]


\begin{lemma}
Let $\sSet{X,R}$ be a uniform relation space. If $\sSet{X,\Gamma(R)}$ is compact, then $\sSet{X,R}$ is complete.
\end{lemma}
\begin{proof}
Assume $\sSet{X,\Gamma(R)}$ is compact and take $F\in\powerfilters(X)$ such that $F\mathrel{R}F$. Then we can find an ultrafilter $F'\supseteq F$ by the ultrafilter lemma \ref{ultrafilterLemma}. By compactness $F'$ converges uniformly, so $F'\mathrel{R}\pfilter{x}$ for some $x\in X$. By upwards closure of $R$, we have $F\mathrel{R} F'$ and thus $F \mathrel{R}\pfilter{x}$. So $F$ is uniformly convergent. Because $F$ was chosen arbitrarily this makes $\sSet{X,R}$ complete.
\end{proof}

\begin{proposition}
Let $\sSet{X,R}$ be a complete uniform space and $A\subseteq X$ a closed subset. Then $A$ is complete.
\end{proposition}
\begin{proof}
Let $F$ be a Cauchy filter and let $F$ be in $A$, i.e.\ such that $A \in F$. Then $F\to x$ for some $x\in X$ by completeness. Thus $A \in F^\mesh \subseteq \vicinity(x)^\mesh$, so $x\in \adh(A) = A$ and $F$ converges in $A$.
\end{proof}

\begin{proposition}
Hausdorff: complete iff closed??
\end{proposition}
\begin{proof}
TODO
\end{proof}


\subsubsection{Induced uniform relation}
\begin{definition}
Let $\sSet{X,\xi}$ be a reciprocal ($R_1$) convergence space. Let $\Phi(\xi)$ be a relation on $\powerfilters(X)$ defined by
\[ F\mathrel{\Phi(\xi)}G \qquad\defequiv\qquad \exists: x\in X: \; \big(F\overset{\xi}{\longrightarrow} x\big) \land \big(G\overset{\xi}{\longrightarrow} x\big) \]
for $F,G\in\powerfilters(X)$.
Then $\Phi(\xi)$ is the \udef{uniform relation associated to} $\xi$.
\end{definition}

\begin{lemma} \label{uniformRelationAssociatedToR1Convergence}
The uniform relation associated to a reciprocal convergence is a uniform relation.
\end{lemma}
\begin{proof}
\begin{itemize}
\item We have $\pfilter{x}\overset{\xi}{\longrightarrow} x$, so $\pfilter{x}\mathrel{\Phi(\xi)}\pfilter{x}$.
\item The set $F\mathrel{\Phi(\xi)}$ is upwards closed by monotonicity of the convergence $\xi$.
\item Symmetry is clear by construction.
\item For transitivity, take proper filters $F,G,H$ such that $F\mathrel{\Phi(\xi)}G$ and $G\mathrel{\Phi(\xi)}H$. Then there exist $x,y\in X$ such that
\[ \big(F\overset{\xi}{\longrightarrow} x\big) \land \big(G\overset{\xi}{\longrightarrow} x\big) \land \big(G\overset{\xi}{\longrightarrow} y\big) \land \big(H\overset{\xi}{\longrightarrow} y\big). \]
Thus $G\in \lim^{-1}_\xi(x) \cap\lim^{-1}_\xi(y)$, so $\lim^{-1}_\xi(x) \mesh\lim^{-1}_\xi(y)$. Using reciprocity, we apply \ref{R1Conditions} to get $\lim^{-1}_\xi(x) = \lim^{-1}_\xi(y)$. Thus $H \overset{\xi}{\longrightarrow} x$, which means that $F\mathrel{\Phi(\xi)}H$.
\end{itemize}
\end{proof}

\begin{proposition} \label{uniformConvergenceGaloisConnection}
Let $X$ be a set. The functions
\begin{align*}
&\Phi: \{\text{$R_1$ convergences on $X$}\} \to \{\text{uniform relations on $X$}\} \\
&\Gamma: \{\text{uniform relations on $X$}\} \to \{\text{$R_1$ convergences on $X$}\}
\end{align*}
form a Galois connection $(\Phi, \Gamma)$. Additionally,
\begin{enumerate}
\item $\im(\Phi)$ is the set of complete uniform relations on $X$;
\item $\im(\Gamma)$ is the set of all reciprocal convergences on $X$; i.e\ $\Gamma$ is surjective.
\end{enumerate}
\end{proposition}
\begin{proof}
The functions $\Phi$ and $\Gamma$ are well-defined by \ref{associatedUniformConvergence} and \ref{uniformRelationAssociatedToR1Convergence}.

To prove the Galois connection, let $R$ be a uniform relation on $X$ and $\xi$ a convergence on $X$. Then we need to prove $\Phi(\xi) \subseteq R$ \textup{if and only if} $\xi \subseteq \Gamma(R)$.

First assume $\Phi(\xi) \subseteq R$. Take $F\overset{\xi}{\longrightarrow} x$. Because also $\pfilter{x}\overset{\xi}{\longrightarrow} x$, we have $F\mathrel{\Phi(\xi)}\pfilter{x}$. By assumption $F\mathrel{R}\pfilter{x}$ and by definition $F\overset{\Gamma(R)}{\longrightarrow} x$.

Now assume $\xi \subseteq \Gamma(R)$. Take $F,G\in \powerfilters(X)$ such that $F\mathrel{\Phi(\xi)}G$. Then $\exists x\in X$ such that $F\overset{\xi}{\longrightarrow} x$ and $G\overset{\xi}{\longrightarrow} x$. By assumption $F\overset{\Gamma(R)}{\longrightarrow} x$ and $G\overset{\Gamma(R)}{\longrightarrow} x$, so by definition $F\mathrel{R}\pfilter{x}$ and $G\mathrel{R}\pfilter{x}$. By symmetry and transitivity $F\mathrel{R}G$.

\begin{enumerate}
\item For the inclusion $\subseteq$: assume $F\mathrel{\Phi(\xi)} F$. Then $F\to x$ for some $x\in X$, so $F\mathrel{\Phi(\xi)} \pfilter{x}$. Thus $F$ converges uniformly to some $x$.

For the other inclusion, $\supseteq$, take a complete uniform relation $R$. It is enough to show that $R \subseteq \Phi(\Gamma(R))$. Take $F,G\in \powerfilters(X)$ such that $F\mathrel{R} G$. By completeness, $F\mathrel{R} \pfilter{x}$ for some $x\in X$. By symmetry and transitivity, $G\mathrel{R} \pfilter{x}$ as well. Thus $F\overset{\Gamma(R)}{\longrightarrow} x$ and $G\overset{\Gamma(R)}{\longrightarrow} x$, so $F\mathrel{\Phi(\Gamma(R))} G$.

\item It is enough to prove $\Gamma(\Phi(\xi)) \subseteq \xi$ for any reciprocal convergence $\xi$. Take $F\overset{\Gamma(\Phi(\xi))}{\longrightarrow} x$. Then $F\mathrel{\Phi(\xi)} \pfilter{x}$ and so $\exists y\in X$ such that $F\overset{\xi}{\longrightarrow} y$ and $\pfilter{x}\overset{\xi}{\longrightarrow} y$. By reciprocity (and because $\pfilter{x}\subseteq \lim_{\xi}^{-1}(x) \cap \lim_{\xi}^{-1}(y)$), we have
\[ F \in {\lim}_{\xi}^{-1}(y) = {\lim}_{\xi}^{-1}(x). \]
So $F\overset{\xi}{\longrightarrow} x$.
\end{enumerate}
\end{proof}

\begin{corollary}
Let $X$ be a set. The functions
\begin{align*}
&\Theta \circ \Phi: \{\text{$R_1$ convergences on $X$}\} \to \{\text{uniformities on $X$}\} \\
&\Gamma\circ \Xi: \{\text{uniformities on $X$}\} \to \{\text{$R_1$ convergences on $X$}\}
\end{align*}
form a Galois connection $(\Theta \circ \Phi, \Gamma\circ \Xi)$. Additionally,
\begin{enumerate}
\item $\im(\Theta \circ \Phi)$ is the set of complete, factorisable uniformities on $X$;
\item $\im(\Gamma\circ \Xi)$ is the set of all reciprocal convergences on $X$.
\end{enumerate}
\end{corollary}
\begin{proof}
The Galois connection follows from \ref{uniformRelationGaloisConnection} and \ref{uniformConvergenceGaloisConnection}.

(1) The inclusion $\subseteq$ is immediate, because a uniformity is called complete if and only if its associated uniform relation is complete.

For the inclusion $\supseteq$, take some complete, factorisable uniformity $\mathcal{U}$. Then $\mathcal{U} = \Theta(R)$ for some uniform relation $R$, by \ref{uniformRelationGaloisConnection}. Now it is enough to note that $R$ is also complete, so there exists a reciprocal convergence $\xi$ such that $\mathcal{U} = \Theta(\Phi(\xi))$ by \ref{uniformConvergenceGaloisConnection}.

(2) Immediate because both $\Gamma$ and $\Xi$ are surjective.
\end{proof}

We can summarise the mappings between uniformlities $\mathcal{U}$, uniform relations $R$ and uniform convergences $\xi$ as follows
\[ \begin{tikzcd}[labels = {font=\large}]
\mathcal{U} \arrow[bend left, rrr, "\substack{F\mathrel{\Xi(\mathcal{U})}G \Leftrightarrow F\otimes G\in \mathcal{U} \vspace{0.1em} \\ \vspace{0.1em} \Xi}"] &&& \arrow[lll, bend left, "\substack{\Theta \vspace{0.8em} \\ H\in\Theta(R) \Leftrightarrow p_1^{\imf\imf}[H]\mathrel{R}p_2^{\imf\imf}[H]}"] R \arrow[rrr, bend left, "\substack{F\overset{\Gamma(R)}{\longrightarrow} x \Leftrightarrow F\mathrel{R}\pfilter{x} \vspace{0.1em} \\ \vspace{0.1em} \Gamma}"] &&& \arrow[lll, bend left, "\substack{\Phi \vspace{0.1em} \\ F\mathrel{\Phi(\xi)}G \Leftrightarrow \exists x: (F\to x)\land (G\to x)}"] \xi
\end{tikzcd} \]

\subsection{Properties of uniform spaces}
\begin{definition}
For any property $\mathbf{P}$ that a convergence space may have, we say a uniform space $\sSet{X,\mathcal{U}}$ has property $\mathbf{P}$ if $\Gamma(\Xi(\mathcal{U}))$ has property $\mathbf{P}$.
\end{definition}

\subsubsection{Compactness}
\begin{proposition} \label{compactUltrafilterFactorisation}
Let $\sSet{X,\mathcal{U}}$ be a compact uniform space. If $H$ is an ultrafilter in $\mathcal{U}$, then 
\begin{enumerate}
\item $p_1^{\imf\imf}[H]\otimes p_2^{\imf\imf}[H] \in\mathcal{U}$;
\item $p_1^{\imf\imf}[H]$ and $p_2^{\imf\imf}[H]$ converge in $\Gamma(\Xi(\mathcal{U}))$;
\item $\lim\big(p_1^{\imf\imf}[H]\big) = \lim\big(p_2^{\imf\imf}[H]\big)$.
\end{enumerate}
\end{proposition}
\begin{proof}
Let $H$ be an ultrafilter in $\mathcal{U}$. Then $p_1^{\imf\imf}[H]$ and $p_2^{\imf\imf}[H]$ are ultrafilters by \ref{projectionsOfUltrafilterAreUltra}. By compactness there exist $x,y\in X$ such that $p_1^{\imf\imf}[H] \overset{\Gamma(\Xi(\mathcal{U}))}{\longrightarrow} x$ and $p_2^{\imf\imf}[H] \overset{\Gamma(\Xi(\mathcal{U}))}{\longrightarrow} y$. Thus $p_1^{\imf\imf}[H]\otimes \pfilter{x}\in\mathcal{U}$ and $p_2^{\imf\imf}[H]\otimes \pfilter{y}\in\mathcal{U}$. 

By \ref{filterCompositionFactorisationLemma} we have
\[ \pfilter{x}\otimes \pfilter{y} = \pfilter{x}\otimes p_1^{\imf\imf}[H]; H; p_2^{\imf\imf}[H] \otimes \pfilter{y} \in \mathcal{U}. \]

Finally
\[ p_1^{\imf\imf}[H] \mathrel{\Xi(\mathcal{U})} \pfilter{x}, \quad \pfilter{x} \mathrel{\Xi(\mathcal{U})} \pfilter{y} \quad\text{and}\quad \pfilter{y}\mathrel{\Xi(\mathcal{U})} p_2^{\imf\imf}[H], \]
so $p_1^{\imf\imf}[H] \mathrel{\Xi(\mathcal{U})} p_2^{\imf\imf}[H]$ by transitivity. Thus $p_1^{\imf\imf}[H]\otimes p_2^{\imf\imf}[H] \in \mathcal{U}$.
\end{proof}

\subsubsection{Depth properties}
\begin{definition}
Let $\sSet{X,\mathcal{U}}$ be a uniform space. We say $\mathcal{U}$ is \udef{uniformly Choquet} if for all $H\in \powerfilters(X^2)$,
\[ H\in\mathcal{U} \qquad\iff\qquad \text{$I \in \mathcal{U}$ for all ultrafilters $I$ such that $H\subseteq I$.} \]
\end{definition}

\begin{lemma}
Let $\sSet{X,\mathcal{U}}$ be a uniform space. If $\mathcal{U}$ is uniformly Choquet, then $\Gamma(\Xi(\mathcal{U}))$ is Choquet.
\end{lemma}
\begin{proof}
????????????????
\end{proof}

\section{Uniform continuity}
\begin{definition}
Let $\sSet{X,\mathcal{U}}$ and $\sSet{Y,\mathcal{V}}$ be uniform spaces. A function $f: X\to Y$ is called \udef{uniformly continuous} if
\[ H\in \mathcal{U} \quad\implies\quad \upset (f, f)^{\imf\imf}[H]\in \mathcal{V}. \]
\end{definition}


\begin{proposition} \label{uniformContinuityEntourages}
Let $\sSet{X,\mathcal{U}}$, $\sSet{Y,\mathcal{V}}$ be uniform spaces and $f: X\to Y$ a function.
\begin{enumerate}
\item If $f$ is uniformly continuous, then $\entourage_\mathcal{V} \subseteq \upset (f, f)^{\imf\imf}[\entourage_\mathcal{U}]$;
\item If $\mathcal{V}$ is topological, then opposite implication also holds.
\end{enumerate}
\end{proposition}
\begin{proof}
(1) By uniform continuity we have
\[ \entourage_\mathcal{V} \subseteq \bigcap\setbuilder{\upset (f, f)^{\imf\imf}[H]}{H\in \mathcal{U}} = \upset (f, f)^{\imf\imf}\left[\bigcap \mathcal{U}\right] = \upset (f, f)^{\imf\imf}\left[\entourage_\mathcal{U}\right]. \]
The first equality follows from \ref{imageFiltersPreservesIntersection}.

(2) Assume $\entourage_\mathcal{V} \subseteq \upset (f, f)^{\imf\imf}[\entourage_\mathcal{U}]$ and take $H\in \mathcal{U}$. Then $\entourage_\mathcal{U}\subseteq H$, so
\[ \entourage_\mathcal{V} \subseteq \upset (f, f)^{\imf\imf}[\entourage_\mathcal{U}] \subseteq \upset (f, f)^{\imf\imf}[H]. \]
Thus $\upset (f, f)^{\imf\imf}[H] \in\mathcal{V}$.
\end{proof}

\begin{proposition}
Let $X,Y$ be sets, $\sSet{X,\mathcal{U}}, \sSet{Y,\mathcal{V}}$ uniform spaces, $\sSet{X,R}, \sSet{Y,S}$ uniform relation spaces and $\sSet{X,\xi}, \sSet{Y,\zeta}$ reciprocal convergence spaces. Let $f: X\to Y$ be a function.
\begin{enumerate}
\item if $f: \sSet{X,\mathcal{U}} \to \sSet{Y,\mathcal{V}}$ is uniformly continuous, then $\big(f: \sSet{X,\Xi(\mathcal{U})} \to \sSet{Y,\Xi(\mathcal{V})}\big)^{\imf\imf}$ is relation preserving;
\item if $\big(f: \sSet{X,R} \to \sSet{Y,S}\big)^{\imf\imf}$ is relation preserving, then $f: \sSet{X,\Gamma(R)} \to \sSet{Y,\Gamma(S)}$ is continuous;
\item if $f: \sSet{X,\xi} \to \sSet{Y,\zeta}$ is continuous, then $\big(f: \sSet{X,\Phi(\xi)} \to \sSet{Y,\Phi(\zeta)}\big)^{\imf\imf}$ is relation preserving;
\item if $\big(f: \sSet{X,R} \to \sSet{Y,S}\big)^{\imf\imf}$ is relation preserving, then $f: \sSet{X,\Theta(R)} \to \sSet{Y,\Theta(S)}$ is uniformly continuous.
\end{enumerate}
\end{proposition}
\begin{proof}
(1) Assume $f$ uniformly continuous and $F\mathrel{\Xi(\mathcal{U})} G$. Then $F\otimes G\in \mathcal{U}$. By uniform continuity $\upset (f, f)^{\imf\imf}[F\otimes G] = f^{\imf\imf}[F] \otimes f^{\imf\imf}[G] \in \mathcal{V}$, so $f^{\imf\imf}[F] \mathrel{\Xi(\mathcal{V})} f^{\imf\imf}[G]$.

(2) Assume $f^{\imf\imf}$ relation preserving and take $F \overset{\Gamma(R)}{\longrightarrow} x$. Then $F\mathrel{R}\pfilter{x}$ and, by relation preservation, $f^{\imf\imf}[F]\mathrel{S}f^{\imf\imf}[\pfilter{x}]$. Now $f^{\imf\imf}[\pfilter{x}] = \pfilter{f}(x)$, so $f^{\imf\imf}[F] \overset{\Gamma(S)}{\longrightarrow} f(x)$.

(3) Assume $f$ is continuous and $F\mathrel{\Phi(\xi)} G$. Then there exists $x\in X$ such that $F\overset{\xi}{\longrightarrow} x$ and $G\overset{\xi}{\longrightarrow} x$. By continuity $f^{\imf\imf}[F]\overset{\zeta}{\longrightarrow} f(x)$ and $f^{\imf\imf}[G]\overset{\zeta}{\longrightarrow} f(x)$, so $f^{\imf\imf}[F] \mathrel{\Phi(\zeta)} f^{\imf\imf}[G]$.

(4) Assume $f^{\imf\imf}$ is relation preserving and take $H\in \Theta(R)$. Then $p_1^{\imf\imf}[H]\mathrel{R}p_2^{\imf\imf}[H]$, so $f^{\imf\imf}\big[p_1^{\imf\imf}[H]\big]\mathrel{S}f^{\imf\imf}\big[p_2^{\imf\imf}[H]\big]$. Now
\[ f^{\imf\imf}\big[p_1^{\imf\imf}[H]\big] = (f\circ p_1)^{\imf\imf}[H] = \big(p_1\circ(f,f) \big)^{\imf\imf}[H] = p_1^{\imf\imf}\big[(f,f)^{\imf\imf}[H]\big]. \]
Similarly $f^{\imf\imf}\big[p_2^{\imf\imf}[H]\big] = p_2^{\imf\imf}\big[(f,f)^{\imf\imf}[H]\big]$. Thus $p_1^{\imf\imf}\big[(f,f)^{\imf\imf}[H]\big]\mathrel{S}p_2^{\imf\imf}\big[(f,f)^{\imf\imf}[H]\big]$, which means that $(f,f)^{\imf\imf}[H]\in \Theta(S)$.
\end{proof}
\begin{corollary}
Let $X,Y$ be sets, $\sSet{X,\mathcal{U}}, \sSet{Y,\mathcal{V}}$ uniform spaces, $\sSet{X,R}, \sSet{Y,S}$ uniform relation spaces and $\sSet{X,\xi}, \sSet{Y,\zeta}$ reciprocal convergence spaces. Let $f: X\to Y$ be a function.
\begin{enumerate}
\item if $\mathcal{U}$ is factorisable and $\big(f: \sSet{X,\Xi(\mathcal{U})} \to \sSet{Y,\Xi(\mathcal{V})}\big)^{\imf\imf}$ is relation preserving, then $f: \sSet{X,\mathcal{U}} \to \sSet{Y,\mathcal{V}}$ is uniformly continuous;
\item if $R$ is complete and $f: \sSet{X,\Gamma(R)} \to \sSet{Y,\Gamma(S)}$ is continuous, then $\big(f: \sSet{X,R} \to \sSet{Y,S}\big)^{\imf\imf}$ is relation preserving.
\end{enumerate}
\end{corollary}
\begin{proof}
(1) From the proposition, we have that $f: \sSet{X,\Theta(\Xi(\mathcal{U}))} \to \sSet{Y,\Theta(\Xi(\mathcal{V}))}$ is uniformly continuous. Because $\mathcal{U}$ is factorisable, $\Theta(\Xi(\mathcal{U})) = \mathcal{U}$. Also $\Theta(\Xi(\mathcal{V})) \subseteq \mathcal{V}$.

(2) From the proposition, we have that $\big(f: \sSet{X,\Phi(\Gamma(R))} \to \sSet{Y,\Phi(\Gamma(S))}\big)^{\imf\imf}$ is relation preserving. Because $R$ is complete, we have $R = \Phi(\Gamma(R))$. Also $\Phi(\Gamma(S)) \subseteq S$.
\end{proof}

\begin{proposition}
Let $X,Y$ be sets, $\sSet{X,\mathcal{U}}, \sSet{Y,\mathcal{V}}$ uniform spaces and $\sSet{X,\xi}, \sSet{Y,\zeta}$ reciprocal convergence spaces. Let $f: X\to Y$ be a function.
\begin{enumerate}
\item if $\mathcal{U}$ is compact, $\mathcal{V}$ is uniformly Choquet and $f: \sSet{X,\Gamma(\Xi(\mathcal{U}))} \to \sSet{Y,\Gamma(\Xi(\mathcal{V}))}$ is continuous, then $f: \sSet{X,\mathcal{U}} \to \sSet{Y,\mathcal{V}}$ is uniformly continuous.
\end{enumerate}
\end{proposition}
\begin{proof}
Take $H\in\mathcal{U}$. Take arbitrary ultrafilter $I\in\powerfilters(Y^2)$ such that $(f,f)^{\imf\imf}[H]\subseteq I$. Then there exists an ultrafilter $J\in \powerfilters(X^2)$ such that $\upset (f,f)^{\imf\imf}[J] = I$ by \ref{preimageFilter}.

By \ref{compactUltrafilterFactorisation}, there exists $x\in X$ such that $p_1^{\imf\imf}[J]\overset{\Gamma(\Xi(\mathcal{U}))}{\longrightarrow} x$ and $p_2^{\imf\imf}[J]\overset{\Gamma(\Xi(\mathcal{U}))}{\longrightarrow} x$. By continuity, symmetry and transitivity,
\[ \mathcal{V} \ni (f\circ p_1)^{\imf\imf}[J]\otimes (f\circ p_2)^{\imf\imf}[J] = p_1^{\imf\imf}\big[(f,f)^{\imf\imf}[J]\big]\otimes p_2^{\imf\imf}\big[(f,f)^{\imf\imf}[J]\big] \subseteq (f,f)^{\imf\imf}[J] = I. \]
As this is true for arbitrary ultrafilter $I$, we have $(f,f)^{\imf\imf}[H]\in\mathcal{V}$ because $\mathcal{V}$ is uniformly Choquet.
\end{proof}

\section{Star refinement}
\begin{definition}
Let $X$ be a set and $U,V\subseteq \powerset(X)$ covers of $X$.
\begin{itemize}
\item The \udef{star} of $A\subseteq X$ w.r.t. $U$ is the set defined by
\[ \operatorname{star}_U(A) \defeq \bigcup\setbuilder{B\in U}{A\mesh B}. \]
\item The \udef{star} of $U$ is
\[ U^* \defeq \setbuilder{\operatorname{star}_U(B)}{B\in U}. \]
\item The cover $U$ \udef{star refines} the cover $V$ if $U^*\subseteq \downset V$. This is denoted $U <^* V$.
\end{itemize}
\end{definition}
Note the different direction compared with the definition of refinement.

\begin{lemma}
Star refinement is a transitive relation on $\powerset^2(X)$.
\end{lemma}
\begin{proof}
Assume $U <^* V$ and $V<^* W$, i.e.\ $U^* \subseteq \downset V$ and $V^* \subseteq \downset W$. The last inclusion implies $\downset V^* \subseteq \downset W$. Now $V\subseteq \downset V^*$ because $A\subseteq \operatorname{star}_V(A)$ for all $A\in V$, which implies $\downset V\subseteq \downset V^*$. Then we have
\[ U^* \subseteq \downset V \subseteq \downset V^* \subseteq \downset W, \]
and, by transitivity of inclusion, $U <^* W$.
\end{proof}

\begin{example}
Star refinement is not reflexive in general. 

Consider the set $X = \{0,1,2\}$ and the cover $U = \big\{\{0,1\}, \{1,2\}\big\}$. Then $U^* = \big\{\{0,1,2\}\big\}$, but $\{0,1,2\}\notin \downset U$.
\end{example}

\begin{lemma}
Let $X$ be a set and $C,D\subseteq \powerset(X)$. If $C\subseteq D$, them $C^*\subseteq \downset D^*$.
\end{lemma}
\begin{proof}
Take $A \in C^*$. Then $A = \operatorname{star}_C(B) = \bigcup \setbuilder{B'\in C}{B'\mesh B}$
\end{proof}

\subsubsection{Tolerance cover}
\begin{definition}
Let $V$ be a tolerance relation on a set $X$. The \udef{(tolerance) cover} of $X$ associated to $V$ is defined as
\[ C_V \defeq \setbuilder{xV}{x\in X}. \]
\end{definition}

\begin{lemma}
Let $V, W$ be tolerance relations on $X$ and $C_V, C_W$ their associated covers.
\end{lemma}

\begin{lemma} \label{toleranceCoverStarRefinement}
Let $V$ be a tolerance relation on $X$ and $C_V$ its associated cover. Then
\begin{enumerate}
\item $\operatorname{star}_{C_V}(xV) = x(V;V;V)$;
\item $C_V <^* C_{V;V;V}$.
\end{enumerate}
\end{lemma}
\begin{proof}
(1) We calculate
\begin{align*}
\operatorname{star}_{C_V}(xV) &= \bigcup \setbuilder{yV}{xV\mesh yV} \\
&= \bigcup \setbuilder{yV}{x(V;V)y} \\
&= x(V;V;V).
\end{align*}

(2) By (1), each $\operatorname{star}_{C_V}(xV)\in C_V^*$ is an element of $C_{V;V;V}$.
\end{proof}

\subsubsection{Uniform covers}
\begin{definition}
Let $X$ be a set. A \udef{uniform cover set} of $X$ is a filter of covers in $\sSet{\powerset^2(X), <^*}$. Members of a uniform cover set are called \udef{uniform covers}.
\end{definition}

\begin{proposition}
Let $X$ be a set.
\begin{enumerate}
\item Let $\entourage$ be a topological entourage filter. Then
\[ \setbuilder{C}{\exists V\in\entourage: C_V\subseteq \downset C} = \setbuilder{C_V}{V\in\entourage}_{<^*} \]
is a uniform cover set.
\item Let $\mathcal{C}$ be a uniform cover set. Then
\[ \upset \setbuilder{\bigcup\setbuilder{A\times A}{A\in C}}{C\in\mathcal{C}} \]
is a unform filter set.
\end{enumerate}
\end{proposition}
\begin{proof}
(1) We first verify the equality. Take $V\in \entourage$. Then there exists $W\in \entourage$ such that $W;W;W\subseteq V$.


upwards closure: Let $C$ be a uniform cover and $C <^* D$. Then 
\end{proof}



\begin{definition}
Let $X$ be a set. A \udef{uniform cover set} of $X$ is a filter of covers in $\sSet{\powerset^2(X), <^*}$.
\end{definition}

\section{Function spaces}
\subsection{Evaluation uniformities}
\begin{definition}
Let $\sSet{X, \mathcal{U}}$ and $\sSet{Y,\mathcal{V}}$ be preuniform spaces. The \udef{evaluation preuniformity} on $(X\to Y)$ is defined by
\[ \mathcal{U}_E \defeq \bigcap \setbuilder{\left((\evalMap, \evalMap)^{\imf\imf}(-\otimes K)^\ttransp\right)^{\preimf}[\mathcal{V}] }{K\in\mathcal{U}}. \]
\end{definition}

\begin{lemma}
Let $\sSet{X, \mathcal{U}}$ and $\sSet{Y,\mathcal{V}}$ be preuniform spaces. The evaluation preuniformity on $(X\to Y)$ is a preuniformity. If (TODO when), then it is a (diagonal ?) uniformity.
\end{lemma}
\begin{proof}
We verify the conditions:
\begin{itemize}
\item Upwards closure: If $G\in \mathcal{U}_E$ and $G\subseteq H$ for some $G,H\in\powerfilters(X\to Y)$, then for all $K\in \mathcal{U}$
\[ (\evalMap, \evalMap)^{\imf\imf}(G\otimes K)^\ttransp \subseteq (\evalMap, \evalMap)^{\imf\imf}(H\otimes K)^\ttransp \in \mathcal{V}, \]
by the upwards closure of $\mathcal{V}$. We conclude that $H\in \mathcal{U}_E$.
\item Closure under transposition: take $G\in \mathcal{U}_E$ and arbitrary $K\in \mathcal{U}$. We have
\[ (\evalMap, \evalMap)^{\imf\imf}(G^\transp\otimes K)^\ttransp = \Big((\evalMap, \evalMap)^{\imf\imf}(H\otimes K^\transp)^\ttransp\Big)^\transp \in \mathcal{V}, \]
by the closure under transposition of $\mathcal{V}$. We conclude that $G^\transp\in \mathcal{U}_E$.
\item Closure under composition: take $G,H\in\mathcal{U}_E$ and $K\in\mathcal{U}$. It is enough to show that there exists $K'\in\mathcal{U}$ such that
\[ (\evalMap, \evalMap)^{\imf\imf}(G\otimes K')^\ttransp;(\evalMap, \evalMap)^{\imf\imf}(H\otimes K')^\ttransp \subseteq (\evalMap, \evalMap)^{\imf\imf}(G;H\otimes K)^\ttransp. \]


, because they form bases, that for all $A\in G, B\in H$,
\[ (\evalMap, \evalMap)^\imf\big((A;B) \times (C;D)\big)^\ttransp \subseteq (\evalMap, \evalMap)^\imf\big(A \times C\big)^\ttransp ; (\evalMap, \evalMap)^\imf\big(B \times D\big)^\ttransp. \]
We first pick an arbitrary element of the left-hand set. Take $(x,y)\in C;D$ and $(f, g)\in A;B$, so there exist $z\in X$ and $h:X\to Y$ such that $(f,h)\in A$, $(h,g)\in B$, $(x,z)\in C$ and $(z,y)\in D$. Then
\[ (\evalMap, \evalMap)\big((f,g), (x,y)\big)^\ttransp = (f(x), g(y)) \]
and also
\begin{align*}
(f(x), g(y)) &\in \big\{\big(f(x), h(z)\big)\big\};\big\{\big(h(z), g(y)\big)\big\} \\
&= \big\{(\evalMap, \evalMap)\big((f,h), (x,z)\big)^\ttransp\big\};\big\{(\evalMap, \evalMap)\big((h,g), (z,y)\big)^\ttransp\big\} \\
&\subseteq (\evalMap, \evalMap)^\imf\big(A \times C\big)^\ttransp ; (\evalMap, \evalMap)^\imf\big(B \times D\big)^\ttransp.
\end{align*}
\end{itemize}
Now assume $\sSet{Y, \mathcal{U}}$ diagonal. We verify the diagonality of $\mathcal{U}_\mathcal{S}$.

For any $S\in \mathcal{S}$, we have
\begin{align*}
(\evalMap, \evalMap)^{\imf\imf}\big(\{\Delta_{(X\to Y)}\}\otimes \{\Delta_S\}\big)^\ttransp &= \upset \Big\{ (\evalMap, \evalMap)^{\imf}\big(\Delta_{(X\to Y)}\times \Delta_S\big)^\ttransp \Big\} \\
&= \upset \Big\{ \bigcup_{f\in (X\to Y)}\bigcup_{x\in S}(\evalMap, \evalMap)^{\imf}\big(\{(f,f)\}\times \{(x,x)\}\big)^\ttransp \Big\} \\
&= \upset \Big\{ \bigcup_{f\in (X\to Y)}\bigcup_{x\in S}\big\{(\evalMap, \evalMap)\big((f,x),(f,x)\big)\big\} \Big\} \\
&= \upset \Big\{ \bigcup_{f\in (X\to Y)}\bigcup_{x\in S}\big\{\big(f(x), f(x)\big)\big\} \Big\} \\
&= \upset \Big\{ \Delta_{Y} \Big\} \in \mathcal{U}.
\end{align*}
This means that $\upset \{\Delta_{(X\to Y)}\} \in \mathcal{U}_\mathcal{S}$.
\end{proof}


\begin{lemma}
Let $X$ be a set and $\sSet{Y,\mathcal{V}}$ a preuniform space. If $\mathcal{U}_1 \subseteq \mathcal{U}_2$ are preuniformities on $X$ with corresponding evaluation preuniformities $\mathcal{U}_{\mathcal{E}_1}$ and $\mathcal{U}_{\mathcal{E}_2}$, then $\mathcal{U}_{\mathcal{E}_2} \subseteq \mathcal{U}_{\mathcal{E}_1}$.
\end{lemma}
\begin{proof}
TODO
\end{proof}

\subsection{Uniform convergences and equicontinuity}
\subsubsection{Diagonal maps and filters}
\begin{definition}
Let $X$ be a set and $A\subseteq X$ be a subset. The \udef{diagonal map} on $X$ is defined as
\[ \Delta: \powerset(X)\to \powerset(X^2): A\mapsto \setbuilder{(a,a)}{a\in A}. \]
We usually write $\Delta_A$ instead of $\Delta(A)$. Similarly if $\mathcal{S}\in\powerset^2(X^2)$, then we write $\Delta^\imf_\mathcal{S}$ instead of $\Delta^\imf(\mathcal{S})$.
\end{definition}

TODO: covers upwards directed.

\subsubsection{Uniform convergences and equicontinuity}
\begin{definition}
Let $X$ be a set and $\sSet{Y, \mathcal{U}}$ a  uniform space. For any set $S$, let $\Delta_S = \setbuilder{(s,s)\in S^2}{s\in S}$ be the diagonal on $S$.
\begin{itemize}
\item If $\mathcal{S}$ is a set of subsets of $X$. Then
\[ \mathcal{U}_\mathcal{S} \defeq \setbuilder{H\in \powerfilters\big((X\to Y)^2\big)}{\forall S\in \mathcal{S}:\; (\evalMap, \evalMap)^{\imf\imf}(H\otimes \{\Delta_S\})^\ttransp \in \mathcal{U}} \]
is called the \udef{$\mathcal{S}$-uniformity}.
\item If $\mathcal{S}$ is a set of subsets of $X$ and $K\subseteq (X\to Y)$. Then $K$ is called \udef{$\mathcal{S}$-equicontinuous} if
\[ \forall S\in \mathcal{S}:\; \upset\big\{(\evalMap, \evalMap)^\imf(\Delta_K\times \Delta_S)^\ttransp\big\}\in \mathcal{U}. \]
\end{itemize}
In particular:
\begin{itemize}
\item $\setbuilder{\{x\}}{x\in X}$-convergence is called \udef{pointwise convergence};
\item $\{X\}$-uniform convergence is just called \udef{uniform convergence};
\item $\setbuilder{\{x\}}{x\in X}$-equicontinuity is called \udef{(pointwise) equicontinuity};
\item $\{X\}$-equicontinuity is called \udef{uniform equicontinuity}.
\end{itemize}
\end{definition}

\begin{lemma}
Let $X$ be a set, $\sSet{Y, \mathcal{U}}$ a diagonal uniform space and $\mathcal{S} \subseteq \powerset(X)$. Then
\[ \mathcal{U}_\mathcal{S} = \left(H\mapsto (\evalMap, \evalMap)^{\imf\imf\imf}\big((H\otimes -)^\imf(\Delta^\imf_\mathcal{S})\big)^\ttransp\right)^\preimf[\powerset(\mathcal{U})]. \]
\end{lemma}

\begin{lemma}
Let $X$ be a set, $\sSet{Y, \mathcal{U}}$ a preuniform space and $\mathcal{S}$ a set of subsets of $X$. The preuniformity of $\mathcal{S}$-uniform convergence is a preuniformity on $(X\to Y)$.

If $\sSet{Y, \mathcal{U}}$ is diagonal, then the preuniformity of $\mathcal{S}$-uniform convergence is a diagonal uniformity.
\end{lemma}
Note that if $\mathcal{U}$ is not diagonal, then $\mathcal{U}_\mathcal{S}$ is not necessarily a uniformity.
\begin{proof}
We verify the conditions:
\begin{itemize}
\item Upwards closure: If $G\subseteq H$ for some $G,H\in\powerfilters(X\to Y)$, then
\[ (\evalMap, \evalMap)^{\imf\imf}(G\otimes \{\Delta_S\})^\ttransp \subseteq (\evalMap, \evalMap)^{\imf\imf}(H\otimes \{\Delta_S\})^\ttransp \]
and so we conclude by the upwards closure of $\mathcal{U}$.
\item Closure under transposition: we have
\[ (\evalMap, \evalMap)^{\imf\imf}(H^\transp\otimes \{\Delta_S\})^\ttransp = \Big((\evalMap, \evalMap)^{\imf\imf}(H\otimes \{\Delta_S\})^\ttransp\Big)^\transp, \]
and so we conclude by the closure under transposition of $\mathcal{U}$.
\item Closure under composition: take $G,H\in\mathcal{U}_\mathcal{S}$ and $S\in\mathcal{S}$. It is enough to show, because they form bases, that for all $A\in G, B\in H$,
\[ (\evalMap, \evalMap)^\imf\big(A;B \times \Delta_S\big)^\ttransp \subseteq (\evalMap, \evalMap)^\imf\big(A \times \Delta_S\big)^\ttransp ; (\evalMap, \evalMap)^\imf\big(B \times \Delta_S\big)^\ttransp. \]
We first pick an arbitrary element of the left-hand set. Take $(x,x)\in\Delta_S$ and $(f, g)\in A;B$, so there exists $h:X\to Y$ such that $(f,h)\in A$ and $(h,g)\in B$. Then
\[ (\evalMap, \evalMap)\big((f,g), (x,x)\big)^\ttransp = (f(x), g(x)) \]
and also
\begin{align*}
(f(x), g(x)) &\in \big\{\big(f(x), h(x)\big)\big\};\big\{\big(h(x), g(x)\big)\big\} \\
&= \big\{(\evalMap, \evalMap)\big((f,h), (x,x)\big)^\ttransp\big\};\big\{(\evalMap, \evalMap)\big((h,g), (x,x)\big)^\ttransp\big\} \\
&\subseteq (\evalMap, \evalMap)^\imf\big(A \times \Delta_S\big)^\ttransp ; (\evalMap, \evalMap)^\imf\big(B \times \Delta_S\big)^\ttransp.
\end{align*}
\end{itemize}
Now assume $\sSet{Y, \mathcal{U}}$ diagonal. We verify the diagonality of $\mathcal{U}_\mathcal{S}$.

For any $S\in \mathcal{S}$, we have
\begin{align*}
(\evalMap, \evalMap)^{\imf\imf}\big(\{\Delta_{(X\to Y)}\}\otimes \{\Delta_S\}\big)^\ttransp &= \upset \Big\{ (\evalMap, \evalMap)^{\imf}\big(\Delta_{(X\to Y)}\times \Delta_S\big)^\ttransp \Big\} \\
&= \upset \Big\{ \bigcup_{f\in (X\to Y)}\bigcup_{x\in S}(\evalMap, \evalMap)^{\imf}\big(\{(f,f)\}\times \{(x,x)\}\big)^\ttransp \Big\} \\
&= \upset \Big\{ \bigcup_{f\in (X\to Y)}\bigcup_{x\in S}\big\{(\evalMap, \evalMap)\big((f,x),(f,x)\big)\big\} \Big\} \\
&= \upset \Big\{ \bigcup_{f\in (X\to Y)}\bigcup_{x\in S}\big\{\big(f(x), f(x)\big)\big\} \Big\} \\
&= \upset \Big\{ \Delta_{Y} \Big\} \in \mathcal{U}.
\end{align*}
This means that $\upset \{\Delta_{(X\to Y)}\} \in \mathcal{U}_\mathcal{S}$.
\end{proof}

\begin{lemma}
Take $\mathcal{S}_1, \mathcal{S}_2 \in\powerset^2(X)$. If $\mathcal{S}_1 \preceq \mathcal{S}_2$ (i.e.\ $\mathcal{S}_2$ refines $\mathcal{S}_1$), then $\mathcal{U}_{\mathcal{S}_2} \subseteq \mathcal{U}_{\mathcal{S}_1}$.
\end{lemma}
\begin{proof}
TODO
\end{proof}



\begin{lemma}
Let $X$ be a set and $\sSet{Y, \mathcal{U}}$ a uniform space. Then the uniformity of pointwise convergence induces pointwise convergence on $(X\to Y)$.
\end{lemma}
\begin{proof}
Let $H$ be a filter in $\powerfilters(X\to Y)$ and $f\in (X\to Y)$. Set $\mathcal{S} = \setbuilder{\{x\}}{x\in X}$. Then
\begin{align*}
H\overset{\text{pt-wise}}{\longrightarrow} f &\iff \forall x\in X: \; \evalMap^{\imf\imf}(H, \pfilter{x}) \overset{\mathcal{U}}{\longrightarrow} f(x) \\
&\iff \forall x\in X: \; \evalMap^{\imf\imf}(H, \pfilter{x}) \otimes \pfilter{f}(x) \in\mathcal{U} \\
&\iff \forall x\in X: \; \evalMap^{\imf\imf}(H, \pfilter{x}) \otimes \evalMap^{\imf\imf}(\pfilter{f}, \pfilter{x}) \in\mathcal{U} \\
&\iff \forall x\in X: \; (\evalMap, \evalMap)^{\imf\imf}\big((H\otimes \pfilter{x}) \otimes (\pfilter{f}, \pfilter{x})\big) \in\mathcal{U} \\
&\iff \forall x\in X: \; (\evalMap, \evalMap)^{\imf\imf}\big((H\otimes \pfilter{f}) \otimes (\pfilter{x}, \pfilter{x})\big)^\ttransp \in\mathcal{U} \\
&\iff \forall x\in X: \; (\evalMap, \evalMap)^{\imf\imf}\big((H\otimes \pfilter{f}) \otimes \{\Delta_{\{x\}}\}\big)^\ttransp \in\mathcal{U} \\
&\iff H\otimes \pfilter{f} \in \mathcal{U}_\mathcal{S} \\
&\iff H \overset{\mathcal{U}_\mathcal{S}}{\longrightarrow} f.
\end{align*}
\end{proof}

\subsection{Continuous convergence}
\begin{definition}
Let $\sSet{X,\xi}$ and $\sSet{Y,\zeta}$ be convergence spaces. A filter $H\in(X\to Y)$ converges in \udef{continuous preconvergence} to $f\in(X\to Y)$ if
\[ \evalMap^{\imf\imf}(H\otimes F) \overset{\zeta}{\longrightarrow} f(x) \]
for all $F\overset{\xi}{\longrightarrow} x$.

The space $(X\to Y)$ equipped with the continuous convergence is denoted $(X \to Y)_c$.
\end{definition}

The space $\cont(X,Y)$, when viewed as a convergence subspace of $(X\to Y)_c$, is denoted $\cont_c(X,Y)$

\begin{lemma}
Let $\sSet{X,\xi}$ and $\sSet{Y,\zeta}$ be convergence spaces. Then
\begin{enumerate}
\item $(X\to Y)_c$ is a preconvergence space;
\item $\cont_c(X,Y)$ is a convergence space.
\end{enumerate}
\end{lemma}
\begin{proof}
It is clear that the continuous preconvergence is monotonic. We just need to show that it is centered, when restricted to $\cont(X,Y)$. Take $f\in \cont(X,Y)$. The for all $F\to x\in X$ we have $\evalMap^{\imf\imf}(\pfilter{f}\otimes F) = f^{\imf\imf}[F] \to f(x)$ by continuity. Thus $\pfilter{f}\to f$ in $\cont_c(X,Y)$.
\end{proof}

\begin{lemma}
Let $\sSet{X,\xi}$ and $\sSet{Y,\zeta}$ be convergence spaces and $H\in \powerfilters(X\to Y)$. If $\xi$ is pretopological, then $H\overset{(X\to Y)_c}{\longrightarrow} f$ \textup{if and only if}
\[ H\overset{(X\to Y)_c}{\longrightarrow} f\qquad \text{if and only if} \qquad \evalMap^{\imf\imf}\big(H\otimes \vicinity(x)\big) \overset{\zeta}{\longrightarrow} f(x). \]
\end{lemma}
\begin{proof}
TODO
\end{proof}

\begin{example}
There exists a sequence of continuous functions that converges pointwise to a continuous function, but does not converge in continuous convergence.

Consider $\seq{f_n} \subseteq (\R\to\R)$ defined by
\[ f_n: \R\to\R: x\mapsto \begin{cases}
n\cdot x & (x\in \interval{0,n^{-1}}) \\
2 - nx & (x\in \interval{n^{-1}, 2n^{-1}}) \\
0 & (\text{otherwise}).
\end{cases} \]
TODO image.
\end{example}

\begin{lemma}
Let $\sSet{X,\xi}$ and $\sSet{Y,\zeta}$ be convergence spaces. The continuous preconvergence on $(X\to Y)$ is the coarsest preconvergence such that the evaluation map $\evalMap: (X\to Y)\times X \to Y$ is continuous.
\end{lemma}
\begin{proof}
We first show that the continuous convergence makes the evaluation map continuous. A filter $G\in\powerfilters\big((X\to Y)_c\times X\big)$ converges iff $p_1[G]\to f$ and $p_2[G]\to x$ converge. By definition of continuous convergence $\evalMap^{\imf\imf}(p_1[G] \otimes p_2[G]) \to f(x)$ and thus $\evalMap^{\imf\imf}(p_1[G] \otimes p_2[G]) \subseteq \evalMap^{\imf\imf}(G) \to f(x)$.

Now assume there is some other preconvergence $\sigma$ on $(X\to Y)$ that makes the evaluation map continuous. Take $H \overset{\sigma}{\longrightarrow} f$. Then for all $F\overset{\xi}{\longrightarrow}x$ we have $H\otimes F\overset{\sigma\otimes \xi}{\longrightarrow} (f,x)$, so $\evalMap^{\imf\imf}(H\otimes F) \overset{\zeta}{\longrightarrow} f(x)$ by continuity of the evaluation map. Thus $H\overset{(X,Y)_c}{\longrightarrow} f$.
\end{proof}

\begin{lemma}
Let $\sSet{X,\xi}$ and $\sSet{Y,\zeta}$ be convergence spaces, then the continuous convergence is finer than pointwise convergence on $(X\to Y)$.
\end{lemma}
\begin{proof}
Assume $H\to f$ in continuous convergence. Then
\[ \forall x\in X: \; \evalMap^{\imf\imf}(H\otimes \pfilter{x}) \to f(x), \]
because $\pfilter{x}\to x$. This is exactly the requirement for $H\to f$ in pointwise convergence.
\end{proof}


\begin{lemma}
Let $\sSet{X,\xi}$ and $\sSet{Y,\zeta}$ be convergence spaces. If $Y$ is of finite depth, then so is $(X\to Y)_c$.
\end{lemma}
\begin{proof}
Let $G, H\overset{(X\to Y)_c}{\longrightarrow} f$. Then for all $F\to x\in X$, we have
\begin{align*}
\evalMap^{\imf\imf}\big((G\cap H)\otimes F\big) &= \evalMap^{\imf\imf}\big((G \otimes F) \cap (H \otimes F)\big) \\
&= \evalMap^{\imf\imf}(G \otimes F) \cap \evalMap^{\imf\imf}(H \otimes F) \; \overset{\zeta}{\longrightarrow} f(x),
\end{align*}
using \ref{intersectionProductFilters}, \ref{imageFiltersPreservesIntersection} and the finite depth of $Y$.
\end{proof}
Note that this does not imply the Kent property, because $(X\to Y)_c$ is only a preconvergence space. The space $\cont_C(X,Y)$ does have the Kent property if $Y$ is of finite depth.

\begin{lemma}
Let $\sSet{X,\xi}$, $\sSet{Y,\zeta}$ be convergence spaces such that $(X\to Y)_c$ is a Kent space. Then all limit functions are continuous.
\end{lemma}
\begin{proof}
Take $H\in\powerfilters\big((X\to Y)\big)$ such that $H\overset{(X\to Y)_c}{\longrightarrow} f$. We have, for all $F\to x\in X$,
\begin{align*}
\evalMap^{\imf\imf}\big((H\cap \pfilter{f})\otimes F\big) &= \evalMap^{\imf\imf}\big((H \otimes F) \cap (\pfilter{f} \otimes F)\big) \\
&= \evalMap^{\imf\imf}(H \otimes F) \cap \evalMap^{\imf\imf}(\pfilter{f} \otimes F),
\end{align*} 
using \ref{intersectionProductFilters} and \ref{imageFiltersPreservesIntersection}. By the Kent property this converges to $f(x)$. And thus by monotonicity of the convergence,
\[ \evalMap^{\imf\imf}(\pfilter{f} \otimes F) = f^{\imf\imf}[F] \overset{\zeta}{\longrightarrow} f(x). \]
Thus $f$ is continuous.
\end{proof}
The hypothesis of $(X\to Y)_c$ being a Kent space is necessary.
\begin{example}
There exist non-continuous limit points in spaces of continuous convergence.

Set $X = \{a,b\}$ and $Y = \{1,2\}$.

Let $X$ have the convergences $\begin{cases}
\pfilter{a} \to a \\
\pfilter{b} \to b \\
\pfilter{b} \to a
\end{cases}$ and let $Y$ have the convergences $\begin{cases}
\pfilter{1} \to 1 \\
\pfilter{1} \to 2 \\
\pfilter{2} \to 2
\end{cases}$.
Consider the functions $f: X\to Y: \begin{cases}
a\mapsto 1 \\ b\mapsto 2
\end{cases}$ and $g: X\to Y: \begin{cases}
a\mapsto 1 \\ b\mapsto 1
\end{cases}$.

Then it is straightforward to verify that $\pfilter{g} \overset{(X\to Y)_c}{\longrightarrow} f$, but $f$ is not continuous. Indeed $\pfilter{b} \to a$, but $f^{\imf\imf}[\pfilter{b}] = \pfilter{2} \not\to f(a) = 1$.
\end{example}

\begin{proposition}[Universal property of the continuous convergence structure] \label{universalPropertyContinuousConvergence}
Let $\sSet{X, \xi}, \sSet{Y,\sigma}$ and $\sSet{Z,\zeta}$ be convergences spaces. A function $h: X\to (Y \to Z)_c$ if continuous \textup{if and only if} $\curry_1^{-1}(h): (X\times Y)\to Z$ is continuous.
\end{proposition}
\begin{proof}
First assume $h$ is continuous. Then $\curry_1^{-1}(h) = \evalMap \circ (h,\id_Y)$ and thus $\curry_1^{-1}(h)$ is continuous. (TODO ref).

Now assume $\curry_1^{-1}(h): (X\times Y)\to Z$ is continuous and take $F\to x\in X$. Now for all $G\to y\in Y$ we have that
\begin{align*}
\evalMap^{\imf\imf}\Big(h^{\imf\imf}[F]\otimes G\Big) &= \big(\evalMap \circ (h,\id_Y)\big)^{\imf\imf}(F\otimes G) \\
&= \curry_1^{-1}(h)^{\imf\imf}(F\otimes G) \\
&\to \curry_1^{-1}(h)(x,y) = h(x)(y)
\end{align*}
by continuity of $\curry_1^{-1}(h)$. Thus $h^{\imf\imf}[F]$ converges to $h(x)$ by definition of continuous convergence.
\end{proof}

\begin{proposition}
Let $\sSet{X, \xi}, \sSet{Y,\sigma}$ and $\sSet{Z,\zeta}$ be convergences spaces. Then
\[ \curry_1: (X\times Y \to Z)_c \to (X\to (Y\to Z)_c)_c \]
is a homeomorphism.
\end{proposition}
\begin{proof}
TODO
\end{proof}
\begin{corollary}
The category of convergence spaces is cartesian closed.
\end{corollary}

\begin{proposition}
$\cont_c(X,Y)$ is closed subset of $(X\to Y)_c$??
\end{proposition}

\begin{proposition}
Let $\sSet{X, \xi}, \sSet{Y,\sigma}$ and $\sSet{Z,\zeta}$ be convergences spaces. The composition operation
\[ \circ: (Y\to Z)_c \times (X\to Y)_c \to (X\to Z)_c \]
is continuous.
\end{proposition}
\begin{proof}
By \ref{universalPropertyContinuousConvergence}, this is equivalent to the continuity of $\curry_1^{-1}(\circ): (Y\to Z)_c \times (X\to Y)_c \times X \to Z$, which follows from the commutativity of the following diagram:
\[\begin{tikzcd}
(Y\to Z)_c \times (X\to Y)_c \times X \ar[d, swap, "\id_{(Y\to Z)}\times \evalMap"] \ar[dr, "\curry_1^{-1}(\circ)"] & \\
(Y\to Z)_c\times Y \ar[r, "\evalMap"] & Z.
\end{tikzcd} \]
\end{proof}


\subsubsection{Continuous uniform convergence}
\begin{definition}
Let $\sSet{X, \xi}$ be a convergence space and $\sSet{Y, \mathcal{U}}$ a uniform space. The \udef{uniformity of continuous convergence} is defined by
\[ \mathcal{U}_c \defeq \setbuilder{H\in \powerfilters\big((X\to Y)^2\big)}{\forall F\overset{\xi}{\longrightarrow} x:\; (\evalMap, \evalMap)^{\imf\imf}\Big(H\otimes (F\otimes F)\Big)^\ttransp \in \mathcal{U}}. \]
\end{definition}

\begin{lemma}
Let $\sSet{X, \xi}$ be a convergence space and $\sSet{Y, \mathcal{U}}$ a uniform space. Then the uniformity of continuous convergence
\begin{enumerate}
\item is a preuniformity on $(X\to Y)$;
\item a uniformity on $\cont(X,Y)$.
\end{enumerate}
\end{lemma}
\begin{proof}
Let $\sSet{X,\xi}$ be a convergence space and $\sSet{Y,\mathcal{U}}$ a uniform space. We verify the three conditions
\begin{itemize}
\item Take $f\in \cont(X,Y)$. We need to show that $f^{\imf\imf}[F]\mathrel{\mathcal{U}}f^{\imf\imf}[F]$ for all $F\overset{\xi}{\longrightarrow}x$. Because $f$ is continuous, we have that $f^{\imf\imf}[F]$ converges and the statement follows from \ref{uniformlyConvergentImpliesCauchy}.
\item Take $H,K_1,K_2$ in $\powerfilters(\cont(X,Y))$. Assume $H\mathrel{\mathcal{U}_c}K_1$ and $K_1\subseteq K_2$. Now $\evalMap^{\imf\imf}(K_1\otimes F) \subseteq \evalMap^{\imf\imf}(K_2\otimes F)$ for all $F\overset{\xi}{\longrightarrow}x$. Then upwards closure of $\mathcal{U}_c$ follows from the upwards closure of $\mathcal{U}$.
\item The symmetry and transitivity of $\mathcal{U}_c$ follow from the symmetry and transitivity of $\mathcal{U}$.
\end{itemize}
\end{proof}

\begin{lemma}
Let $\sSet{X, \xi}$ be a convergence space and $\sSet{Y, \mathcal{U}}$ a uniform space. If $\mathcal{U}$ is factorisable, then
\[ \mathcal{U}_c = (\Theta\circ\Phi)\Big(\big(\sSet{X,\xi} \to \sSet{Y, (\Xi\circ\Gamma)(\mathcal{U})}\big)_c\Big). \]
\end{lemma}

\begin{lemma}
Let $\sSet{X, \xi}$ be a convergence space and $\sSet{Y, \mathcal{U}}$ a uniform space. Then the uniformity of continuous convergence induces continous convergence on $\cont(X,Y)$.
\end{lemma}
\begin{proof}
Let $H$ be a filter in $\powerfilters\big(\cont(X,Y)\big)$ and $f\in \cont(X, Y)$. Then
\begin{align*}
H\overset{(X\to Y)_c}{\longrightarrow} f &\iff \forall F\overset{\xi}{\longrightarrow}x: \; \evalMap^{\imf\imf}(H\otimes F) \overset{\mathcal{U}}{\longrightarrow} f(x) \\
&\iff \forall F\overset{\xi}{\longrightarrow}x: \; \evalMap^{\imf\imf}(H\otimes F)\otimes \pfilter{f}(x) \in\mathcal{U} \\
&\iff \forall F\overset{\xi}{\longrightarrow}x: \; \evalMap^{\imf\imf}(H\otimes F)\otimes f^{\imf\imf}(F) \in\mathcal{U} \\
&\iff \forall F\overset{\xi}{\longrightarrow}x: \; \evalMap^{\imf\imf}(H\otimes F)\otimes \evalMap^{\imf\imf}(\pfilter{f}\otimes F) \in\mathcal{U} \\
&\iff \forall F\overset{\xi}{\longrightarrow}x: \; (\evalMap, \evalMap)^{\imf\imf}\big((H\otimes F)\otimes (\pfilter{f}\otimes F)\big) \in\mathcal{U} \\
&\iff \forall F\overset{\xi}{\longrightarrow}x: \; (\evalMap, \evalMap)^{\imf\imf}\big((H\otimes \pfilter{f})\otimes (F \otimes F)\big)^\ttransp \in\mathcal{U} \\
&\iff H\otimes \pfilter{f}\in \mathcal{U}_c \\
&\iff H\overset{\mathcal{U}_c}{\longrightarrow} f.
\end{align*}
\end{proof}

\begin{proposition}
Let $\sSet{X, \xi}$ be a convergence space, $\sSet{Y, \mathcal{U}}$ a uniform space and $\mathcal{S}$ a set of compact subsets of $X$. The continuous convergence uniformity is stronger than the $\mathcal{S}$-uniformity on $\cont(X,Y)$.
\end{proposition}
\begin{proof}

\end{proof}

\subsection{The Arzelà-Ascoli theorem}


\section{Metric spaces}
\begin{definition}
Let $X$ be a set. A \udef{metric} on $X$ is a function $d: X\times X\to \R$ satisfying, for all $x,y,z\in X$,
\begin{itemize}
\item \emph{positivity}: $d(x,y) \geq 0$;
\item \emph{definiteness}: $d(x,y) = 0$ \textup{if and only if} $x=y$;
\item \emph{symmetry}: $d(x,y) = d(y,x)$;
\item the \emph{triangle inequality}: $d(x,z) \leq d(x,y) + d(y,z)$.
\end{itemize}
The structured set $\sSet{X,d}$ is called a \udef{metric space}.
\end{definition}

\subsection{Uniform structures on a metric space}
\begin{proposition} \label{metricUniformities}
Let $\sSet{X,d}$ be a metric space and $\sigma$ an order-regular convergence on $\R^{\geq 0}$ such that addition is continuous. Then the relation $\mathcal{U}^\sigma$ on $\powerfilters(X)$ defined by
\[ F\mathrel{\mathcal{U}^\sigma}G \qquad\defequiv\qquad d^{\imf\imf}\big[F\otimes G\big] \overset{\sigma}{\longrightarrow} 0 \]
is a uniformity on $X$.
\end{proposition}
\begin{proof}
We have that $F\mathrel{\mathcal{U}^N}G$ iff $F\otimes G$ converges to a point in $d^\preimf\{0\}$ in the initial convergence $\psi$ of $d$. Now $d^\preimf\{0\} = \setbuilder{(x,x)}{x\in X}$. In order to apply \ref{uniformityFromConvergenceOnSquare} TODO, we just need to verify the two points. 
\begin{itemize}
\item Let $H\in \powerfilters{X\times X}$. Then, by symmetry of the metric,
\[ H\overset{\psi}{\longrightarrow} (x,x) \iff d^{\imf\imf}[H] \to 0 \iff d^{\imf\imf}[H^\transp] \to 0 \iff H^\transp\overset{\psi}{\longrightarrow} (x,x). \]
\item Assume $H_1\overset{\psi}{\longrightarrow} (x,x)$ and $H_2\overset{\psi}{\longrightarrow} (x,x)$. Then $d^{\imf\imf}[H_1] \overset{\sigma}{\longrightarrow} 0$ and $d^{\imf\imf}[H_2] \overset{\sigma}{\longrightarrow} 0$. By continuity of addition, $d^{\imf\imf}[H_1]+d^{\imf\imf}[H_2] \overset{\sigma}{\longrightarrow} 0$. Now $\pfilter{0} \leq d^{\imf\imf}[H_1;H_2] \leq d^{\imf\imf}[H_1]+d^{\imf\imf}[H_2]$, then $H_1;H_2 \overset{\psi}{\longrightarrow} (x, x)$. TODO squeeze theorem, order closure.
\end{itemize}
\end{proof}

\subsubsection{Standard metric uniformity and convergence}
\begin{definition}
Let $\sSet{X,d}$ be a metric space. The uniformity given by \ref{metricUniformities} when $\R$ is equipped with the standard convergence is called the \udef{(standard) metric uniformity} $\mathcal{U}_d$.

We also define, for $\epsilon >0$ and $x\in X$,
\begin{itemize}
\item The \udef{$\epsilon$-entourage} as the set
\[ V_\epsilon \defeq \setbuilder{(y,z)\in X^2}{d(y,z)\leq \epsilon}. \]
\item The \udef{$\epsilon$-ball centered at $x$} as the set
\[ \ball_d(x,\epsilon) \defeq \setbuilder{y\in X}{d(x,y)< \epsilon} \]
of all points $y$ whose distance to $x$ is less than $\epsilon$.
\item The \udef{closed $\epsilon$-ball centered at $x$} as the set
\[ \cball_d(x,\epsilon) \defeq \setbuilder{y\in X}{d(x,y)\leq \epsilon} \]
of all points $y$ whose distance to $x$ is less than or equal to $\epsilon$.
\item The \udef{$\epsilon$-sphere centered at $x$} as the set
\[ \sphere_d(x,\epsilon) \defeq \setbuilder{y\in X}{d(x,y) = \epsilon}. \]
\end{itemize}
\end{definition}

\begin{lemma}
Let $\sSet{X,d}$ be a metric space. Then
\begin{enumerate}
\item $\cball_d(x,\epsilon) = V_\epsilon x = xV_\epsilon$;
\item $\sphere_d(x,\epsilon) = \cball_d(x,\epsilon)\setminus \ball_d(x,\epsilon)$.
\end{enumerate}
\end{lemma}

\begin{lemma}
Let $\sSet{X,d}$ be a metric space. The metric uniformity is topological and a base of the entourage filter is given by $\{V_\epsilon\}_{\epsilon>0}$.
\end{lemma}
\begin{corollary}
Let $\sSet{X, d_X}$ and $\sSet{Y, d_Y}$ be metric spaces and $f:X\to Y$ a function. Then $f$ is uniformly continuous \textup{if and only if}
\[ \forall \epsilon >0: \exists \delta >0: \forall x,y\in X: \quad d_X(x,y) \leq \delta \implies d_Y(f(x), f(y)) \leq \epsilon. \]
\end{corollary}
\begin{proof}
By \ref{uniformContinuityEntourages} we have that $f$ is uniformly continuous \textup{if and only if}
\begin{align*}
\entourage_Y \subseteq \upset(f\times f)^{\imf\imf}[\entourage_X] &\iff \forall V_\epsilon \in \entourage_Y: \exists V_\delta \in \entourage_X: \; (f\times f)^{\imf\imf}[V_\delta] \subseteq V_\epsilon \\
&\iff \forall \epsilon>0: \exists \delta>0: \forall (y,y')\in (f\times f)^{\imf\imf}[V_\delta]: \; (y,y')\in V_\epsilon \\
&\iff \forall \epsilon>0: \exists \delta>0: \forall (x,x')\in V_\delta:\; (f(x),f(x'))\in V_\epsilon \\
&\iff \forall \epsilon>0: \exists \delta>0: \forall x,x'\in X: \; d_X(x,x')\leq \delta \implies d_Y(f(x),f(x'))\leq \epsilon.
\end{align*}
\end{proof}

\begin{lemma}
Let $\sSet{X,d}$ be a metric space and $\seq{x_n}$ a sequence in $X$. Then $\seq{x_n}$ is a Cauchy sequence \textup{if and only if}
\[ \forall \epsilon >0: \exists N\in\N: \forall m,n \geq N: \quad d(x_m, x_n)\leq \epsilon. \]
\end{lemma}
\begin{proof}
We have that $\TailsFilter\seq{x_n}$ is a Cauchy filter \textup{if and only if} $\entourage_d \subseteq \TailsFilter\seq{x_n}\otimes \TailsFilter\seq{x_n}$. This is true iff $\forall \epsilon>0$:
\begin{align*}
V_\epsilon \in \TailsFilter\seq{x_n}\otimes \TailsFilter\seq{x_n} &\iff \exists M,N\in\N: \setbuilder{x_n}{n\geq N}\times\setbuilder{x_n}{n\geq M} \subseteq V_\epsilon \\
&\iff \exists N\in\N: \setbuilder{x_n}{n\geq N}\times\setbuilder{x_n}{n\geq N} \subseteq V_\epsilon \\
&\iff \exists N\in\N: \forall m,n \geq N: d(x_m, x_n)\leq \epsilon.
\end{align*}
\end{proof}

\begin{definition}
Let $\sSet{X,d}$ be a metric space. The convergence $\Gamma(\mathcal{U}_d)$ induced by the metric uniformity $\mathcal{U}_d$ is called the \udef{metric convergence}.

If $F\in\powerfilters(X)$ converges to $x\in X$ in the metric convergence, we write $F\overset{d}{\longrightarrow} x$.
\end{definition}

\begin{lemma}
Let $\sSet{X,d}$ be a metric space. The metric convergence is topological and
\[ \neighbourhood_d(x) = \upset\setbuilder{\cball_d(x,\epsilon)}{\epsilon >0}. \]
\end{lemma}
\begin{proof}
The metric convergence is topological by \ref{topologicalInducedUniformConvergence}, which also gives the form of the neighbourhood filter. 
\end{proof}
\begin{corollary}
Let $\sSet{X,d}$ be a metric space and $\seq{x_n}$ a sequence in $X$. Then $\seq{x_n}$ converges to $x\in X$ \textup{if and only if}
\[ \forall \epsilon>0: \exists N\in\N: \forall n\geq N: \quad d(x_n,x)\leq \epsilon. \]
\end{corollary}

\subsubsection{Uniform convergence}
\begin{proposition}
Let $X$ be a set, $\sSet{Y,d}$ a metric space and $\seq{f_n}$ a sequence in $(X\to Y)$. The following are equivalent:
\begin{enumerate}
\item  $\seq{f_n}$ converges uniformly to $f: X\to Y$;
\item $\forall \epsilon > 0: \exists N\in \N: \forall n \geq N: \forall x\in X:  d(f_n(x), f(x)) \leq \epsilon$;
\item $\forall \epsilon > 0: \exists N\in \N: \forall n \geq N: \sup_{x\in X} d(f_n(x), f(x)) \leq \epsilon$.
\end{enumerate}
\end{proposition}
\begin{proof}
The equivalence of (2) and (3) is clear. We prove $(1) \Leftrightarrow (2)$.
\begin{align*}
\seq{f_n} \overset{\text{unif.}}{\longrightarrow} f &\iff \TailsFilter\seq{f_n} \otimes \pfilter{f} \in \mathcal{U}_{(X\to Y)} \\
&\iff (\evalMap, \evalMap)^{\imf\imf}\big((\TailsFilter\seq{f_n} \otimes \pfilter{f})\otimes \{\Delta_X\}\big)^\ttransp \in \mathcal{U}_d \\
&\iff \{V_\epsilon\}_\epsilon \subseteq (\evalMap, \evalMap)^{\imf\imf}\big((\TailsFilter\seq{f_n} \otimes \pfilter{f})\otimes \{\Delta_X\}\big)^\ttransp \\
&\iff \forall \epsilon>0: \exists N\in \N: \; (\evalMap, \evalMap)^{\imf}\big((\setbuilder{f_n}{n\geq N} \times \{f\})\times \Delta_X\big)^\ttransp \subseteq V_\epsilon \\
&\iff \forall \epsilon>0: \exists N\in \N: \; \bigcup_{n\geq N}\bigcup_{x\in X}(\evalMap, \evalMap)^{\imf}\big((\{f_n\} \times \{f\})\times \{(x,x)\}\big)^\ttransp \subseteq V_\epsilon \\
&\iff \forall \epsilon>0: \exists N\in \N: \forall n\geq N: \forall x\in X: \;(\evalMap, \evalMap)^{\imf}\big((\{f_n\} \times \{f\})\times \{(x,x)\}\big)^\ttransp \in V_\epsilon \\
&\iff \forall \epsilon>0: \exists N\in \N: \forall n\geq N: \forall x\in X: \; (\evalMap, \evalMap)^{\imf}\big(\{((f_n, f), (x,x))^\ttransp\}\big) \in V_\epsilon \\
&\iff \forall \epsilon>0: \exists N\in \N: \forall n\geq N: \forall x\in X: \; (\evalMap, \evalMap)\big((f_n,x),(f,x)\big) \in V_\epsilon \\
&\iff \forall \epsilon>0: \exists N\in \N: \forall n\geq N: \forall x\in X: \; (f_n(x), f(x)) \in V_\epsilon \\
&\iff \forall \epsilon>0: \exists N\in \N: \forall n\geq N: \forall x\in X: \; d(f_n(x), f(x)) \leq \epsilon.
\end{align*}
\end{proof}
\begin{corollary}

\end{corollary}

\subsubsection{Lipschitz and Hölder diagonal spaces}
\begin{definition}
Let $\sSet{X,d}$ be a metric space. The \udef{$M$-Lipschitz diagonal}
\end{definition}
$\vicinity_\sigma(0) = \upset\{\interval{0,1}\}$


\subsection{Continuous convergence structure}
\begin{proposition}
Let $\sSet{X,\xi}$ be a compact convergence space and $\sSet{Y,d}$ a metric space. The continuous convergence on $\cont(X,Y)$ is given by
\[ \forall H\in\powerset(\cont(X,Y)_c): \qquad H\to f \iff \sup_{x\in X}d(H(x), f(x)) \to 0. \]
\end{proposition}
\begin{proof}
First assume $\sup_{x\in X}d(H(x), f(x)) \to 0$. 

Now assume $\sup_{x\in X}d(H(x), f(x)) \not\to 0$. Then there exists $A\in \neighbourhood(0)$ such that $A \notin \sup_{x\in X}d(H(x), f(x))$ we can construct the set
\[ \setbuilder{\setbuilder{x\in X}{d(h(x), f(x)) \notin A \forall h\in S}}{S\in H}. \]
We claim this is a proper filter in $X$. It is contained in an ultrafilter by the ultrafilter lemma \ref{ultrafilterLemma} and this ultrafilter $F$ converges by compactness. Thus $d(H[F], f[F]) \not\to 0$ and so $H\not\to f$.
\end{proof}

\chapter{Topological convergence and topological spaces}
A topology specifies which points are close to each other and which are not. This is useful for determining continuity and the existence of holes for example.

Obviously one way to get an idea of which points are close to other points is by explicitly supplying a notion of distance. In fact this a particular type of topological space called a metric space. This way of describing the topology will turn out to be too restrictive, however.

Another way of describing topology is saying that it is concerned with the properties of a geometric object that are preserved under continuous deformations, such as stretching, twisting, crumpling and bending, but not tearing or gluing. Those continuous deformations do not change \textit{which} points are close to each other, just exactly how close they are. Tearing separates points that were close and gluing makes points that were not in each others neighbourhoods suddenly neigbours.

A famous example of such a continuous deformation is the deformation between a doughnut and a coffee cup. Because a topologist is only interested in properties that are preserved under such a transformation, the joke goes that for him as doughnut and a mug is the same thing.

All this can be achieved by defining which subsets of the space are \textit{neighbourhoods} of each point. A neighbourhood of a point is an open set around that set and can be thought of as a sort of generalisation of the open intervals on the real line. TODO motivate definition.

\url{https://en.wikipedia.org/wiki/List_of_topologies}

\url{http://www.dynamics-approx.jku.at/lena/Cooper/riesz.pdf} For order convergence!!!

\section{Axiomatisations and basic concepts}
TODO Motivation.
\subsection{Building blocks}
The basic building blocks of topology are neighbourhoods, open sets, closed sets, interior, closure, boundaries, limit points, convergence and nearness. Each of these concepts can be axiomatised and given any one, the others are uniquely fixed.

We first describe how these concepts are related, and then give each axiomatisation and show they are equivalent.

\url{https://en.wikipedia.org/wiki/Characterizations_of_the_category_of_topological_spaces} 
\url{https://mathoverflow.net/questions/19152/why-is-a-topology-made-up-of-open-sets/19173#19173}

\subsection{Neighbourhoods, open sets, closed sets}
TODO: function spaces: closed sets point-wise topology >< uniform topology (bounded by functions v bounded by horizontals)

Let $X$ be a set.

Neighbourhoods: sets that ``completely surround'' $x$.

For every $x\in X$ we specify which subsets of $X$ are neighbourhoods of $x$. This family of sets is denoted $\neighbourhood(x)$. We would like the following to hold:
\begin{enumerate}
\item Every neighbourhood of $x$ must contain $x$.
\item Every set that contains a neighbourhood of $x$ is a neighbourhood of $x$.
\item The intersection of two neighbourhoods is again a neighbourhood.
\item The point $x$ must in some sense be in the interior of each of its neighbourhoods. TODO: \url{https://math.stackexchange.com/questions/2692678/why-does-the-definition-of-a-topology-via-neighborhoods-include-this-axiom}
\end{enumerate}

\begin{proposition}
Let $(X,\mathcal{T})$ be a topological space and $x\in X$. Then
\begin{enumerate}
\item if $N\in\neighbourhood(x)$, then $x\in N$;
\item if $M\subset X$ and there exists $N\in\neighbourhood(x)$ such that $N\subset M$, then $M\in\neighbourhood(x)$;
\item if $M,N\in\neighbourhood(x)$, then $M\cap N\in\neighbourhood(x)$;
\item $\forall N\in\neighbourhood(x): \exists M\in\neighbourhood(x): \forall y\in M: N\in\neighbourhood(y)$.
\end{enumerate}
Conversely, every function $X\to \mathcal{P}(X)$ that satisfies these properties is the neighbourhood topology for some topology on $X$.
\end{proposition}
The fourth point is the least obvious. It essentially says that if $y$ is sufficiently close to $x$ (i.e.\ $y\in M(x)$), then $x$ is also close to $y$.
\begin{proof}
TODO
\end{proof}

\begin{definition}
A \udef{topology} on a set $X$ is a collection $\mathcal{T}\in \mathcal{P}(X)$ of subsets of $X$ having the following properties:
\begin{enumerate}
\item Both $\emptyset$ and $X$ are in $\mathcal{T}$.
\item The union of the elements of any subcollection of $\mathcal{T}$ is in $\mathcal{T}$.
\item The intersection of the elements on any finite subcollection of $\mathcal{T}$ is in $\mathcal{T}$.
\end{enumerate}
A set $X$ for which a topology $\mathcal{T}$ has been specified is called a \udef{topological space}.
\end{definition}
These axioms formalise an idea of open subset.
\begin{definition}
We call a subset $U$ of $T$ an \udef{open set} if $U$ is in $\mathcal{T}$. A \udef{closed set} is any set that can be constructed as $X \setminus U$ for some open set $U$. In a given topology a set may be open, closed, both or neither. If a set is both open and closed, it is called \udef{clopen}.

We say $U$ is an \udef{open neighbourhood} of $x$ if $U$ is an open set containing $x$. This is sometimes denoted $U(x)$.

A \udef{neighbourhood} of $x$ is a set containing an open neighbourhood of $x$. We denote by $\neighbourhood(x)$ the set of neighbourhoods of $x$. The function $\neighbourhood$ is called the \udef{neighbourhood topology}.
\end{definition}

\begin{lemma}
Let $X$ be a topological space. Every open set $O$ can be written as $X\setminus K$ for some closed $K$.
\end{lemma}
\begin{proof}
Lemma \ref{BooleanConsequences}.
\end{proof}

\begin{example}
\begin{itemize}
\item Let $X$ be a three-element set, $X = \{a,b,c\}$. There are many possible topologies on $X$, to name a few:
\begin{itemize}
\item $\left\{\emptyset, X\right\}$
\item $\left\{\emptyset, \{a\}, \{a,b\}, X\right\}$
\item $\left\{\emptyset, \{a\}, X\right\}$
\item $\left\{\emptyset, \{a,b\}, X\right\}$
\item $\left\{\emptyset, \{a,b\}, \{a,c\}, \{b\}, X\right\}$
\item $\left\{\emptyset, \{a,b\}, \{c\}, X\right\}$
\item $\left\{\emptyset, \{a\}, \{b\}, \{a,b\}, X\right\}$
\item $\ldots$
\end{itemize}
\item For any set $X$, the collection of all subsets of $X$ is a topology, called the \udef{discrete topology}.
\item For any set $X$, the topology $\mathcal{T} = \left\{\emptyset, X\right\}$ is called the \udef{trivial} topology.
\item In any topology, both $X$ and $\emptyset$ are both open and closed.
\item Let $X$ be a set. Let $\mathcal{T}_f$ be a collection of all subsets $U$ of $X$ such that $X\setminus U$ is finite or $U=\emptyset$. Then $\mathcal{T}_f$ is a topology on $X$ called the \udef{finite complement topology}.
\end{itemize}
\end{example}

We can also characterise the topology with closed sets or with neighbourhoods:
\begin{proposition}
Let $(X,\mathcal{T})$ be a topological space. Let $\mathcal{T}_c$ be the family of closed subsets of $X$. Then
\begin{enumerate}
\item Both $\emptyset$ and $X$ are in $\mathcal{T}_c$.
\item Let $\mathcal{E}$ be a subset of $\mathcal{T}_c$. Then $\bigcap\mathcal{E}\in\mathcal{T}_c$.
\item If $A,B\in \mathcal{T}_c$, then $A\cup B\in \mathcal{T}_c$.
\end{enumerate}
Conversely, given any set $X$ and any family $\mathcal{T}_c\subset\mathcal{P}(X)$ that satisfies these propserties, the family
\[ \mathcal{T} = \setbuilder{O\subset X}{X\setminus O\in\mathcal{T}_c} \]
is a topology on $X$. 
\end{proposition}


Obviously a set can have different topologies.
\begin{definition}
Sometimes we can compare them. Two topologies $\mathcal{T}$ and $\mathcal{T}'$ are \udef{comparable} if either $\mathcal{T} \subseteq \mathcal{T}'$ or $\mathcal{T} \supseteq \mathcal{T}'$.
\begin{itemize}[leftmargin=2cm]
\item[\boxed{\mathcal{T} \subseteq \mathcal{T}'}] In topology $\mathcal{T}'$ there are more open sets. This allows more granular specification of neighbourhoods. We say $\mathcal{T}'$ is \udef{finer} than $\mathcal{T}$. If $\mathcal{T}'$ is a proper superset, we say it is \udef{strictly finer} than $\mathcal{T}$.
\item[\boxed{\mathcal{T} \supseteq \mathcal{T}'}] In this case $\mathcal{T}'$ is \udef{(strictly) coarser} than $\mathcal{T}$.
\end{itemize}
\end{definition}

\subsection{Closure and interior of a set}
\url{https://en.wikipedia.org/wiki/Kuratowski_closure_axioms}

\begin{definition}
Given any subset $A$ of a topological space $X$,
\begin{itemize}
\item The \udef{interior} of $A$, denoted $A^\circ$, is the union of all open sets contained in $A$;
\item The \udef{closure} of $A$, denoted $\bar{A}$, is the intersection of all closed sets containing $A$. 
\end{itemize}
The \udef{boundary} of $A$ is $\partial A \defeq \bar{A}\setminus A^{\circ}$.
\end{definition}
We immediately have the inclusions
\[ A^\circ \subset A \subset \bar{A} \]

\begin{lemma}
The interior and closure are dual in the sense that
\[ A^\circ = X\setminus\overline{(X\setminus A)} = \overline{(A^c)}^c \qquad \bar{A} = X\setminus(X\setminus A)^\circ = ((A^c)^\circ)^c \]
where $X$ is a topological space and $A$ is a subset.
\end{lemma}
\begin{proposition}\label{closure}
Let $A$ be a subset of the topological space $X$, then
\[ x\in \bar{A} \qquad \text{\textup{if and only if}}\qquad \text{every open set $U$ containing $x$ intersects $A$}.\]
\end{proposition}
\begin{proof}
We prove the contrapositive.
\[ x\notin \bar{A} \iff \text{there exists an open set $U$ containing $x$ that does not intersect $A$.} \]
\begin{itemize}
\item[$\boxed{\Rightarrow}$] The set $U = X\setminus \bar{A}$ is an open set containing $x$ that does not intersect $A$.
\item[$\boxed{\Leftarrow}$] If there exists such a $U$, then $X\setminus U$ is a closed set containing $A$, so $X\setminus U \supset \bar{A}$. Therefore $x$ cannot be in $\bar{A}$.
\end{itemize}
\end{proof}
\begin{proposition}\label{interior}
Let $A$ be a subset of the topological space $X$, then
\[ x\in A^\circ \qquad \text{\textup{if and only if}}\qquad \text{there exists an open set $U$ such that $x\in U \subset A$}.\]
\end{proposition}
\begin{proof}
The interior is the union of all open sets $U\subset A$. Thus if $x\in A^\circ$, then $x$ is in such a $U$.
\end{proof}
\begin{lemma}
Given any subset $A$ of $X$,
\begin{itemize}
\item $\bar{A}$ is the smallest closed set containing $A$;
\item $A^\circ$ is the largest open set contained in $A$.
\end{itemize}
Consequently the closure and interior are idempotent:
\[ \overline{\bar{A}} = \bar{A} \qquad \text{and} \qquad (A^\circ)^\circ = A^\circ. \]
\end{lemma}

\begin{lemma}
Let $A,B$ be subsets of a topological space $X$. Then
\begin{enumerate}
\item $\overline{A\cup B} = \overline{A}\cup \overline{B}$;
\item $(A\cap B)^\circ = A^\circ \cap B^\circ$.
\end{enumerate}
These properties do not hold for arbitrary unions and intersections.
\end{lemma}
\begin{proof}
TODO
\end{proof}
\begin{lemma}
TODO: Intersection of Interiors contains Interior of Intersection and Closure of Union contains Union of Closure and Closure of Intersection is Subset of Intersection of Closures and Union of Interiors is Subset of Interior of Union
\end{lemma}

\begin{lemma} \label{closureInteriorSubsets}
Let $A\subseteq B$ be sets in a topological space $X$. Then
\begin{enumerate}
\item $\overline{A} \subseteq \overline{B}$;
\item $A^\circ \subseteq B^\circ$.
\end{enumerate}
\end{lemma}

\subsection{Boundaries}
\url{https://math.stackexchange.com/questions/2254363/definitions-of-a-topological-space-reference}
\url{https://math.stackexchange.com/questions/4398247/axiomatizations-of-the-boundary-operator}
\url{https://mathoverflow.net/questions/175800/which-sets-occur-as-boundaries-of-other-sets-in-topological-spaces}

\subsection{Metrics}
Quantales and continuity spaces: \url{https://link.springer.com/content/pdf/10.1007/s000120050018.pdf}
\url{https://arxiv.org/abs/1311.4940}
All Topologies Come From Generalized Metrics - Kopperman

\subsection{Limit points}
\begin{definition}
If $A$ is a subset of the topological space $X$ and if $x$ is a point of $X$ (not necessarily of $A$), we say $x$ is a \udef{limit point} (also sometimes called \udef{cluster point} or \udef{point of accumulation}) of $A$ if every (open) neighbourhood of $x$ intersects $A$ in some point other than $x$ itself.

The set $A'$ of all limit points of $A$ is called the \udef{derived set} of $A$.

An \udef{isolated point} of $A$ is a point $x\in A$ that is not an accumulation point for $A$.
\end{definition}
So $x$ is a limit point of $A$ if it belongs to the closure of $A\setminus \{x\}$.
\begin{example}
Consider $\R$. If $A= ]0,1]$, then the point $0$ is a limit point of $A$. In fact every point in $[0,1]$ is a limit point and no other points of $\R$ are limit points.
\end{example}
This motivates the following assertion:
\begin{proposition}
Let $A$ be a subset of a topological space $X$. Then
\[ \bar{A} = A \cup A' = A^\circ \cup A' \]
where $A'$ is the derived set of $A$.
\end{proposition}
\begin{corollary}
A topological space is closed if and only if it contains all its limit points.
\end{corollary}

\begin{definition}
Let $(X,\mathcal{T})$ be a topological space. A subset $A\subset X$ is \udef{perfect} in $X$ if it is closed and every point of $A$ is an accumulation point of $A$.
\end{definition}
\begin{lemma}
If $A$ has no isolated points, then $\overline{A}$ is perfect in $X$.
\end{lemma}

\subsection{Special subsets}
\begin{definition}
Let $(X,\mathcal{T})$ be a topological space. A set $A\subset X$ is called
\begin{itemize}
\item a \udef{$\mathcal{G}_\delta$-set} if it is a countable intersection of open sets;
\item an \udef{$\mathcal{F}_\sigma$-set} if it is a countable union of closed sets.
\end{itemize}
\end{definition}

\section{Topologies}
\subsection{The basis of a topology}
For many topologies specifying \textit{all} the open sets can be challenging. In this section we give a way to specify a smaller collection of subsets of $X$, called a basis, and generate the topology in terms of that.

\begin{definition}
If $X$ is a set, a \udef{basis} is a subset $\mathcal{B}$ of the powerset of $X$ such that
\begin{enumerate}
\item For each $x\in X$, there is at least one basis element $B\in\mathcal{B}$ containing $x$.
\item If $x$ belongs to the intersection of two basis elements $B_1$ and $B_2$, then there is a basis element $B_3\in\mathcal{B}$ containing $x$ such that $B_3\subset B_1 \cap B_2$.
\end{enumerate}
The \udef{topology $\mathcal{T}$ generated by $\mathcal{B}$} is defined as follows: A subset $U$ of $X$ is said to be open in $X$ if, for each $x\in U$, there is a basis element $B\in\mathcal{B}$ such that $x\in B$ and $B\subset U$. 
\end{definition}
Each basis element is itself an open set. It is not too difficult to check that $\mathcal{T}$ is indeed a topology.

\begin{example}
\begin{itemize}
\item For any set $X$, the collection of all one-point subsets of $X$ is a basis for the discrete topology.
\item The collection of all open intervals on the real line is a basis for the \udef{standard topology} on the real line $\R$.
\end{itemize}
\end{example}

There is an easier way to obtain the topology $\mathcal{T}$ from a basis $\mathcal{B}$:
\begin{lemma}
The topology $\mathcal{T}$ equals the collection of all unions of elements in $\mathcal{B}$.
\end{lemma}

Here is a way to obtain a basis from a topology on $X$.
\begin{lemma}
Suppose that $\mathcal{C}$ is a collection of open sets of $X$ such that for each open set $U$ of $X$ and each $x \in U$, there is an element $C$ of $\mathcal{C}$ such that $x\in C \subset U$. Then $\mathcal{C}$ is a basis for the topology of $X$.
\end{lemma}

We can link the basis to the coarseness of the topology.
\begin{lemma} \label{basisCoarseness}
Let $\mathcal{B}$ and $\mathcal{B}'$ be bases for the topologies $\mathcal{T}$ and $\mathcal{T}'$, respectively, on $X$. The following are equivalent:
\begin{enumerate}
\item $\mathcal{T}'$ is finer than $\mathcal{T}$.
\item For each $x\in X$ and each basis element $B\in\mathcal{B}$ containing $x$, there is a basis element $B'\in\mathcal{B'}$ such that $x\in B'\subset B$.
\end{enumerate}
\end{lemma}

Closures of sets can also be described using a basis.
\begin{lemma}
Let $A$ be a subset of $X$ which has a topology generated by a basis $\mathcal{B}$, then $x\in\bar{A}$ if and only if every basis element $B\in\mathcal{B}$ containing $x$ intersects $A$.
\end{lemma}

\subsubsection{Subbasis}
\begin{definition}
If $X$ is a set, a \udef{subbasis} is a subset $\mathcal{S}$ of the powerset of $X$ such that $X = \bigcup \mathcal{S}$.

The \udef{topology $\mathcal{T}$ generated by $\mathcal{S}$} is the collection of all unions of finite intersections of elements of $\mathcal{S}$. 
\end{definition}
The topology $\mathcal{T}$ is exactly the coarsest topology that makes all sets in the subbasis open.

\subsection{The subspace topology}
\begin{definition}
Let $X$ be a topological space with topology $\mathcal{T}$. Let $Y$ be a subspace of $X$. The collection
\[ \mathcal{T}_Y = \{ Y\cap U\;|\; U\in \mathcal{T} \} \]
is a topology on $Y$ called the \udef{subspace topology}. With this topology, $Y$ is called a \udef{subspace} of $X$.
\end{definition}
\begin{lemma}
Let $Y$ be a subspace of $X$. A set $A$ is closed in $Y$ \textup{if and only if} it equals the intersection of a closed set of $X$ with $Y$.
\end{lemma}

\begin{lemma}
If $\mathcal{B}$ is a basis for the topology of $X$, then
\[\mathcal{B}_Y = \{ B\cap Y \;|\; B\in \mathcal{B} \}\]
is a basis for the subspace topology on $Y$.
\end{lemma}

\begin{lemma}
Let $Y$ be a subspace of $X$.
\begin{enumerate}
\item If $A$ is open in $Y$ and $Y$ is open in $X$, then $A$ is open in $X$.
\item If $A$ is closed in $Y$ and $Y$ is closed in $X$, then $A$ is closed in $X$.
\end{enumerate}
\end{lemma}
 
We reserve the notation $\overline{A}$ to stand for the closure of $A$ in $X$, not $Y$.
\begin{lemma} \label{subspaceClosure}
Let $X$ be a topological space and $Y\subset X$ a subspace. Let $A$ be a subset of $Y$, then the closure of $A$ in $Y$ is
\[ \Closure_Y(A) = \overline{A}\cap Y.  \]
\end{lemma}

\begin{lemma} \label{notLimitPointSingletonOpen}
Let $A\subseteq X$ be a subspace of $X$. Then $a\in A\setminus A'$, then $\{a\}$ is open in $A$.
\end{lemma}
\begin{proof}
Assume such an $a$. Then there exists an open neighbourhood $U$ of $a$ in $X$ that does not intersect $A$ in any other point. By definition of the subspace topology $\{a\}$ is open.
\end{proof}

\subsection{Topology and order}


\url{https://planetmath.org/orderedspace}
\url{https://www.jstor.org/stable/2032122?seq=2#metadata_info_tab_contents}
\url{http://www.math.wm.edu/~lutzer/drafts/PragueSurveyFinal.pdf}
\url{https://ncatlab.org/nlab/show/pospace}
\url{file:///C:/Users/user/Downloads/order-topological-lattices.pdf}

\subsubsection{Specialisation preorder}
\begin{definition}
Let $(X,\mathcal{T})$ be a topological space and $x,y\in X$. We say $x$
\end{definition}

\paragraph{Alexandrov topology}
\url{https://planetmath.org/inducedalexandrofftopologyonaposet}
\url{https://arxiv.org/pdf/0708.2136.pdf}
\url{https://ncatlab.org/nlab/show/specialization+topology}
\url{http://math.uchicago.edu/~may/REU2018/REUPapers/Asness.pdf}

\subsubsection{Order topology on totally ordered sets}
\begin{definition}
Let $(X,\leq)$ be a linearly ordered set. Let $\mathcal{B}$ be the collection of all sets of the following type:
\begin{enumerate}
\item All open intervals $]a,b[$ in $X$;
\item All intervals of the form $[a_0, b[$, where $a_0$ is the smallest element (if any) of $X$;
\item All intervals of the form $]a, b_0]$, where $b_0$ is the largest element (if any) of $X$;
\end{enumerate}
The collection $\mathcal{B}$ is a basis for a topology, called the \udef{order topology}.
\end{definition}

\paragraph{Product of linearly ordered topology}


\section{Separation axioms}
\subsection{$T_0$}
\subsubsection{Kolmogorov quotient}

\subsection{$T_1$}
\begin{proposition}
Let $X$ be a topological space satisfying $T_1$; let $A$ be a subset of $X$.
Then the point $x$ is a limit point of $A$ \textup{if and only if} every neighbourhood of $x$ contains infinitely many points of $A$.
\end{proposition}
\subsection{Hausdorff spaces}
\begin{definition}
A topological space $X$ is called a \udef{Hausdorff space} if for each pair $x_1, x_2$ of distinct points in $X$, their exist neighbourhoods $U(x_1)$, $U(x_2)$ that are disjoint.
\end{definition}
In Hausdorff spaces distinct points can be told apart topologically, hence Hausdorff spaces are also called \udef{separated spaces}. In particular the Hausdorff condition implies the uniqueness of limits, which is not otherwise guaranteed.

\begin{proposition}
Every finite point set in a Hausdorff space is closed. TODO: T1
\end{proposition}
\begin{proof}
It suffices to show that every one-point set $\{x_0\}$ is closed. Indeed if $\{x_0\}$ was not closed, the closure of $\{x_0\}$ would contain another point. This other point has a disjoint neighbourhood by Hausdorff, so this fails by proposition \ref{closure}.
\end{proof}
\begin{proposition}
Limits are unique (sequences, filters, nets)
\end{proposition}
\begin{lemma}
\begin{enumerate}
\item A subspace of a Hausdorff space is Hausdorff.
\item Every totally ordered set is Hausdorff in the order topology.
\item Every metric topology is Hausdorff.
\end{enumerate}
\end{lemma}



\section{Functions on topological spaces}
\subsection{Open and closed maps}
TODO
\subsection{Continuity and continuous functions}
Intuitively, a continuous map is a map between topological spaces that does not make jumps. In particular let $f: X\to Y$ be a potentially continuous function. Say we want to stay in a neighbourhood $V(f(x_0))$, then we want there to be a neighbourhood $U(x_0)$ such that points inside $U(x_0)$ map to points in $V$, i.e.\
\[ x\in U(x_0) \implies f(x) \in V(f(x_0)). \]
That is $f(U(x_0)) \subset V$, or $U(x_0)\subset f^{-1}(V)$. So we conclude that for any point $x_0$ and neighbourhood $V(f(x_0))$ in $Y$, $f^{-1}(V)$ must contain a neighbourhood of $x_0$. Thus $f^{-1}(V)$ can be written as a union of open sets, $\bigcup_{x\in f^{-1(V)}}U(x)$, and therefore must be open. This motivates the definition:
\begin{definition}
Let $X,Y$ be topological spaces.
\begin{itemize}
\item A function $f:X\to Y$ is \udef{continuous} if for each open set $V$ of $Y$, $f^{-1}[V]$ is an open subset of $X$.
\item The function $f$ is \udef{continuous at $x_0$} if for each open neighbourhood $V$ of $f(x_0)$, their is an open neighbourhood $U$ of $x_0$ such that $f[U]\subset V$.
\end{itemize}
If a function is not continuous, it is \udef{discontinuous}.

The set of all continuous functions $X\to Y$ is denoted $\cont(X,Y)$. If $X=Y$, we also write $\cont(X)$.
\end{definition}
\begin{lemma} \label{globalContinuityFromAllPoints}
A function $f:X\to Y$ is continuous \textup{if and only if} it is continuous at every point.
\end{lemma}

\begin{lemma} \label{continuityAtIsolatedPoint}
Let $f:X\to Y$ be a function between topological spaces. If $\{x_0\}\subset X$ is open, then $f$ is continuous at $x_0$.
\end{lemma}

\begin{lemma}
\begin{enumerate}
\item If the topology of $Y$ is given by a basis $\mathcal{B}$, then to prove continuity of $f$ it suffices to show that the inverse image of every basis element is open.
\item If the topology of $Y$ is given by a subbasis $\mathcal{S}$, then to prove continuity of $f$ it suffices to show that the inverse image of every subbasis element is open.
\end{enumerate}
\end{lemma}
\begin{proposition}\label{continuity}
Let $X, Y$ be topological spaces; $f:X\to Y$. The following are equivalent:
\begin{enumerate}
\item $f$ is continuous;
\item $f[\bar{A}]\subset \overline{f[A]}$;
\item for every closed set $B$ of $Y$, the set $f^{-1}[B]$ is closed in $X$. TODO $f$ closed.
\end{enumerate}
\end{proposition}
\begin{proof}
We proceed round-robin-style.
\begin{itemize}[leftmargin=2cm]
\item[$\boxed{(1) \Rightarrow (2)}$] Let $x\in \bar{A}$ and $V$ a neighbourhood of $f(x)$. Then $f^{-1}[V]$ is an open set containing $x$, so it must intersect $A$ in some point $y$ by proposition \ref{closure}. Then $V$ intersects $f[A]$ in $f(y)$, so $f(x) \in \overline{f[A]}$ as desired.
\item[$\boxed{(2) \Rightarrow (3)}$] Let $B$ be closed in $Y$. We observe that $f[f^{-1}[B]]\subset B$. Choose some $x\in \overline{f^{-1}[B]}$, then
\[ f(x) \in f\left[\overline{f^{-1}[B]}\right] \subset \overline{f[f^{-1}[B]]} \subset \bar{B} = B, \]
so that $x\in f^{-1}[B]$. Thus $\overline{f^{-1}[B]}\subset f^{-1}[B]$, meaning $f^{-1}[B]$ is closed.
\item[$\boxed{(3) \Rightarrow (1)}$] Let $V$ be an open set in $Y$. Set $B = Y\setminus V$. Then $V = Y\setminus B$ and
\[ f^{-1}[V] = f^{-1}[Y\setminus B] = f^{-1}[Y]\setminus f^{-1}[B] = X \setminus f^{-1}[B]\]
using lemma \ref{imagePreimageUniqueness}. Thus $f^{-1}[V]$ is open.
\end{itemize}
\end{proof}

\subsubsection{Homeomorphisms \textit{or} topological isomorphisms}
\begin{definition}
Let $X,Y$ be topological spaces and $f:X\to Y$ a bijection. Then $f$ is a \udef{homeomorphism} if both $f$ and $f^{-1}$ are continuous.
\end{definition}
\begin{lemma}
A homeomorphism is a bijection $f$ such that $f(U)$ is open \textup{if and only if} $U$ is open.
\end{lemma}
\subsubsection{Constructing continuous functions}
\begin{proposition} \label{continuousConstructions}
Let $X,Y$ and $Z$ be topological spaces.
\begin{enumerate}
\item \textup{(Identity function)} The identity function $I:X\to X$ is continuous.
\item \textup{(Constant function)} If $f:X\to Y$ maps all of $X$ into a single $y_0$ of $Y$, then $f$ is continuous.
\item \textup{(Inclusion)} Let $A$ be a subspace of $X$, then the inclusion $A\hookrightarrow X$ is continuous.
\item \textup{(Composites)} If $f:X\to Y$ and $g:Y\to Z$ are continuous, then $g\circ f: X\to Z$ is continuous.
\item \textup{(Restricting the domain)} If $f:X\to Y$ is continuous and $A$ is a subspace of $X$, then the restricted function $f|_{A}:A\to Y$ is continuous.
\item \textup{(Restricting the range)} Let $f:X\to Y$ be continuous. If $Z$ is a subspace of $Y$ containing the image set $f[X]$, then $f:X\to Z$ is continuous.
\item \textup{(Expanding the range)} Let $f:X\to Y$ be continuous. If $Y$ is a subspace of $Z$, then $f:X\to Z$ is continuous.
\item \textup{(Local formulation of continuity)} The map $f:X\to Y$ is continuous is $X$ can be written as the union of open sets $U_\alpha$ such that $f|_{U_\alpha}$ is continuous for each $\alpha$.
\end{enumerate}
\end{proposition}
\begin{proposition}[The pasting lemma]
Let $X=A\cup B$ where $A,B$ are closed in $X$. Let $f:A\to Y$ and $g:B\to Y$ be continuous such that $f(x)=g(x)$ for all $x\in A\cap B$. Then the function defined by
\[ h: X\to Y: x\mapsto h(x) = \begin{cases}
f(x) & (x\in A) \\ g(x) & (x\in B)
\end{cases} \]
is continuous.
\end{proposition}
\begin{proof}
Let $C$ be a closed subset of $Y$, then $f^{-1}[C]$ and $g^{-1}[C]$ are both closed. So
\[ h^{-1}[C] = f^{-1}[C]\cup g^{-1}[C] \]
is closed, meaning $h$ is continuous, all by proposition \ref{continuity}.
\end{proof}

TODO: complex conjugation continuous.

\subsection{Limits of functions}
\begin{definition}
Let $X,Y$ be topological spaces. Let $p$ be a limit point of $A\subseteq X$ and $f: A\to Y$. We say $L\in Y$ is a \udef{limit} of $f(x)$ as $x$ approaches $p$ if
\[ \forall\;\text{open neighbourhood}\; V(L):\exists \;\text{open neighbourhood}\; U(p):\; f[(U\cap A)\setminus \{p\}] \subseteq V. \]
We write $f(x)\to L$ as $x\to p$ or
\[ \lim_{x\to p}f(x) = L. \]
\end{definition}
Note that the value of $f$ at $p$ is irrelevant to the definition of the limit. The domain of $f$ does not even need to contain $p$.

\begin{proposition}
Let $f:X\to Y$ be a functions between topological spaces. Then $f$ is continuous at $p\in X$ \textup{if and only if} $\lim_{x\to p}f(x) = f(p)$.
\end{proposition}

Limits may or may not exist and may or may not be unique, but uniqueness is guaranteed if $Y$ is Hausdorff.
\begin{proposition} \label{HausdorffUniqueLimit}
Let $f: A\subseteq X\to Y$ be a function and $X,Y$ be topological spaces. If $Y$ is Hausdorff, then there is a most one limit of $f$ at any point $p\in X$.
\end{proposition}
\begin{proof}
Assume $L_1$ and $L_2$ are two distinct limits of $f(x)$ as $x\to p$. Because $Y$ is Hausdorff there are two disjoint open neighbourhoods $V_1, V_2$ of $L_1,L_2$. Let $U_1,U_2$ be the corresponding open neighbourhoods of $p$. Then $U_1\cap U_2$ must be an open neighbourhood of $p$, so that $U_1\cap U_2\cap A$ contains a point other than $p$, by virtue of $p$ being a limit point. This however means that $f[(U_1\cap A)\setminus \{p\}]$ and $f[(U_2\cap A)\setminus \{p\}]$ are not disjoint, so neither are $V_1,V_2$: a contradiction.
\end{proof}

\subsection{Initial and final topologies}

\subsection{Sets of functions}
\begin{definition}
Let $X, Y$ be topological spaces.
\begin{itemize}
\item The set of continuous functions in $(X\to Y)$ is denoted $\cont(X, Y)$.
\item The set of continuous functions in $(X\to Y)$ which vanish at infinity is denoted $\cont_0(X, Y)$.
\item The set of continuous functions in $(X\to Y)$ with compact support is denoted $\cont_c(X, Y)$.
\item The set of bounded continuous functions in $(X\to Y)$ is denoted $\cont_b(X, Y)$.
\end{itemize}
If we omit $Y$, we generally mean $Y = \R$.
\end{definition}

TODO: define these notions in general!
TODO: ideals and multiplier algebras.


\section{The product topology}
\subsection{Finite Cartesian products}
\begin{definition}
The \udef{product topology} on $X\times Y$ is the topology having as basis the collection $\mathcal{B}$ of all sets of the form $U\times V$, where $U$ is an open subset of $X$ and $V$ is an open subset of $Y$.
\end{definition}
\begin{lemma} \label{basisFiniteProductTopology}
If $\mathcal{B}$ is a basis for the topology of $X$ and $\mathcal{C}$ a basis for the topology of $Y$, then
\[ \mathcal{D} = \{ B\times C\;|\; B\in \mathcal{B}\;\text{and}\; C\in \mathcal{C} \} \]
is a basis for the topology of $X\times Y$.
\end{lemma}
\begin{proposition}
Let $A$ be a subspace of $X$ and $B$ a subspace of $Y$. The product topology on $A\times B$ is the same as the subspace topology on $A\times B$, when viewed as a subset of $X\times Y$.
\end{proposition}

\begin{definition}
Let $X,Y$ be topological spaces. The maps
\begin{align*}
&\pi_1: X\times Y\to X: (x,y)\mapsto x
&\pi_2: X\times Y\to Y: (x,y)\mapsto y
\end{align*}
are called the \udef{projections} of $X\times Y$ onto its first and second factors, respectively.
\end{definition}
\begin{proposition}
The collection
\[ \mathcal{S} = \{ \pi_1^{-1}(U)\;|\; U\;\text{open in}\; X \}\cup \{ \pi_2^{-1}(V)\;|\; V\;\text{open in}\;Y  \} \]
is a subbasis for the product topology on $X\times Y$.
\end{proposition}
\begin{proof}
Let $\mathcal{T}$ denote the product topology on $X\times Y$; let $\mathcal{T'}$ be the topology generated by $\mathcal{S}$.
\begin{itemize}[leftmargin=2cm]
\item[$\boxed{\mathcal{T}'\subset\mathcal{T}}$] We need to prove all elements of $\mathcal{S}$ are open. Indeed $\pi_1^{-1}(U) = U\times Y$ is open and $\pi_2^{-1}(V) = X\times V$ is also open.
\item[$\boxed{\mathcal{T}\subset\mathcal{T}'}$] Let $B\times C$ be an element of the basis, in other words $B\subset X$ and $C\subset Y$ are open. Then $B\times C = \pi_1^{-1}(B)\cap \pi_2^{-1}(C)$.
\end{itemize}
\end{proof}
In particular $\pi_1$ and $\pi_2$ are continuous.
\begin{proposition}\label{continuityCompositeFunctions}
Let $A,X,Y$ be topological spaces and let
\[ f:A\to X\times Y: a\mapsto f(a) = (f_1(a),f_2(a)). \]
Then $f$ is continuous \textup{if and only if} the functions $f_1$ and $f_2$ are continuous.
\end{proposition}
There is no useful criterion for the continuity of a map $f:A\times B \to X$.

\begin{proposition}
Let $X,Y$ be metrisable topological spaces with metrics $d_X$ and $d_Y$. Then $X\times Y$ is metrisable. Possible, equivalent, metrics include
\[ d_\text{max} = \max\circ \{d_X, d_Y\} \]
and
\[ d_\text{graph} = d_X \circ \pi_1 + d_Y \circ \pi_2. \]
\end{proposition}
\begin{proof}
We first prove that the product topology and the metric topology generated by $d_\text{max}$ are the same using \ref{basisCoarseness}.

First take an element of a basis for the product topology, which by \ref{basisFiniteProductTopology} can be taken of the form
\[ B = B_{d_X}(x, \epsilon_1)\times B_{d_Y}(y, \epsilon_2) \qquad \text{for some $x\in X, y\in Y, \epsilon_1,\epsilon_2 >0$.} \]
Then we can find a basiselement $B_{d_\text{max}}((x,y), \min\{\epsilon_1,\epsilon_2\})$ of the metric topology generated by $d_\text{max}$ that is a subset.

Conversely, take $B_{d_\text{max}}((x,y), \epsilon)$. Then $B_{d_X}(x, \epsilon)\times B_{d_Y}(y, \epsilon)$ is a subset.

The equivalence of the two metrics can then be seen by applying \ref{ballsCoarseness} twice:
\[ B_{d_\text{max}}((x,y), \epsilon) \subset B_{d_\text{graph}}((x,y), \epsilon) \qquad B_{d_\text{graph}}((x,y), \epsilon/2) \subset B_{d_\text{max}}((x,y), \epsilon). \]
\end{proof}
\begin{corollary} \label{convergenceFiniteProductTopology}
A sequence $(x_n,y_n)_n$ converges to $(x,y)$ in the product topology \textup{if and only if} $(x_n)_n$ converges to $x$ and $(y_n)_n$ converges to $y$.
\end{corollary}
TODO:also nets?

\subsection{Arbitrary Cartesian products}
\begin{definition}
Let $X = \prod_{\alpha\in J}X_\alpha$ and define
\[ \mathcal{S}_\beta = \{\pi_\beta^{-1}(U_\beta)\;|\; U_\beta\;\text{open in}\;X_\beta\} \qquad \text{and}\qquad \mathcal{S} = \bigcup_{\beta\in J}.\]
Then the topology on $X$ generated by the subbasis $\mathcal{S}$ is the \udef{product topology} and then $X$ is called a \udef{product space}.
\end{definition}
\begin{lemma}
\begin{itemize}
\item The product topology on $\prod X_\alpha$ has as a basis all sets of the form $\prod_\alpha U_\alpha$, where $U_\alpha$ is open in $X_\alpha$ for all $\alpha$ and $U_\alpha = X_\alpha$ except for finitely many values of $\alpha$.
\item If each $X_\alpha$ has a basis $\mathcal{B}_\alpha$, a basis for the product topology is given by all the sets of the form $\prod_{\alpha\in J}B_\alpha$ where $B_\alpha\in\mathcal{B}_\alpha$ for finitely many values of $\alpha$ and $B_\alpha = X_\alpha$ for the rest.
\end{itemize}
\end{lemma}
If we remove the condition that $U_\alpha = X_\alpha$ except for finitely many values of $\alpha$, we get the box topology.

Some results that held for finite Cartesian product also hold for arbitrary products:
\begin{lemma}
Let the topology on $\prod X_\alpha$ be the product topology.
\begin{itemize}
\item If each space $X_\alpha$ is Hausdorff, then $\prod X_\alpha$ is Hausdorff.
\item Let $A_\alpha$ be subsets of $X_\alpha$, then
\[ \prod \bar{A}_\alpha = \overline{\prod A_\alpha}. \]
\item Let $A_\alpha$ be subspaces of $X_\alpha$, for each $\alpha\in J$. Then $\prod A_\alpha$ is a subspace of $\prod X_\alpha$ if both products are given the product topology.
\end{itemize}
\end{lemma}
\begin{proposition}
Let $\prod X_\alpha$ have the product topology. Let $f:A\to \prod X_\alpha$ be given by
\[ f(a)=(f_\alpha(a))_{\alpha\in J} \qquad \text{where $f_\alpha = \pi_\alpha\circ f:A\to X_\alpha$ for each $\alpha \in J$}.\]
Then $f$ is continuous \textup{if and only if} each function $f_\alpha$ is continuous.
\end{proposition}
This does not hold for the box topology.
TODO: Universal mapping property; generalise to initial topologies.
\begin{corollary} \label{productInclusionsContinuous}
Assume we have some points $c_i \in X_i$ for all $i\in J$. Then the functions
\[ i_\alpha: X_\alpha \to X: p \mapsto \left(\begin{cases}
c_i & (i\neq \alpha) \\ p & (i = \alpha)
\end{cases}\right)_{i\in J} \]
are continuous.
\end{corollary}
\begin{proof}
Consider $i_\alpha$. Then for $i = \alpha$, the function $\pi_i \circ i_\alpha: X_\alpha \to X_i$ is the identity in $X_\alpha$ and thus continuous, \ref{continuousConstructions}. For $i \neq \alpha$, the function $\pi_i \circ i_\alpha: X_\alpha \to X_i$ is a constant function $p \mapsto c_i$ and thus continuous, \ref{continuousConstructions}.
\end{proof}

\begin{lemma}
Given points $\vec{x}=(x_i)_{i\in \N}$ and $\vec{y}=(y_i)_{i\in \N}$ of $\R^\N$, define the metric
\[ D(\vec{x}, \vec{y}) = \sup\left\{\frac{\bar{d}(x_i,y_i)}{i}\;|\; i\in \N\right\}, \]
where $\bar{d}$ is the standard bounded metric on $\R$. Then $D$ induces the product topology on $\R^\N$.
\end{lemma}
\begin{proof}
Let $\mathcal{T}$ denote the product topology on $\R^\N$ and $\mathcal{T}_D$ the topology induced by $D$. We prove two inclusions using lemma \ref{basisCoarseness}.
\begin{itemize}[leftmargin=2cm]
\item[$\boxed{\mathcal{T}_D\subset\mathcal{T}}$] Choose arbitrary basis element $B_D(\vec{x},\epsilon)$. Then choose an $N\in\N$ such that $1/N<\epsilon$. Take the basis element
\[ V = ]x_1-\epsilon,x_1+\epsilon[\;\times\;]x_1-\epsilon,x_1+\epsilon[\;\times \ldots\times\; ]x_N-\epsilon, x_N+\epsilon[\;\times \R\times\R\times \ldots \]
for the product topology. We assert that $V\subset B_D(\vec{x},\epsilon)$. Indeed, for all $\vec{y}\in\R^\N$,
\[ \frac{\bar{d}(x_i,y_i)}{i} \leq \frac{1}{i} \leq \frac{1}{N} \qquad \text{if $i\geq N$}. \]
Therefore,
\[ D(\vec{x},\vec{y}) \leq \max\left\{ \frac{\bar{d}(x_1,y_1)}{1},\frac{\bar{d}(x_2,y_2)}{2},\ldots, \frac{\bar{d}(x_N,y_N)}{N}, \frac{1}{N} \right\}. \]
So if $\vec{y}\in V$, then $D(\vec{x},\vec{y})< \epsilon$ and $V\subset B_D(\vec{x},\epsilon)$.
\item[$\boxed{\mathcal{T}\subset\mathcal{T}_D}$] Choose an arbitrary basis element $U = \prod U_i$. Let $U_i=\R$ if $i\notin \{\alpha_1,\ldots, \alpha_n\}$. For each $i\in \{\alpha_1,\ldots, \alpha_n\}$ choose an interval $]x_i-\epsilon_i,x_i+\epsilon_i[\subset U_i$ and define
\[ \epsilon = \min\{\epsilon_i/i\;|\;i=\alpha_1,\ldots, \alpha_n\}. \]
We can easily see that $B_D(\vec{x},\epsilon) \subset U$.
\end{itemize}
\begin{corollary}
Countable products of metrisable spaces are metrisable.
\end{corollary}
\begin{corollary}
Countable products of Hausdorff spaces are Hausdorff.
\end{corollary}
\end{proof}
\begin{lemma}
The product $\R^J$, with $J$ an uncountable index set, is not metrisable.
\end{lemma}
\begin{proof}
In a metrisable space, by TODO ref, we have that if $x\in \bar{A}$, then there exists a sequence of points in $A$ converging to $x$. We construct a counterexample. Let $A$ be the subset of $\R^J$ containing all points $(x_i)_{i\in J}$ such that $x_i=1$ for all but finitely many $i$. Now the point $(0)_{i\in J}$ is in the closure of $A$, but has no sequence in $A$ converging to it. To see that it is in the closure, let $\prod U_\alpha$ be a basis element containing $(0)_{i\in J}$. The intersection $A\cap \prod U_\alpha$ is never empty. Indeed for only finitely many $\alpha$, $U_\alpha\neq \R$. Set $x_\alpha = 0$ for these $\alpha$ and $x_i = 1$ for the rest.
\end{proof}
\subsection{Box topology}
\begin{definition}
Let $X = \prod_{\alpha\in J}X_\alpha$ and take as a basis for a topology the collection of all sets of the form
\[ \prod_{\alpha\in J}U_\alpha \qquad \text{($U_\alpha$ open in $X_\alpha$)}. \]
The topology generated by this basis is the \udef{box topology}.
\end{definition}
The following properties hold, like in the product topology:
\begin{lemma}
Let the topology on $\prod X_\alpha$ be the box topology.
\begin{itemize}
\item If each space $X_\alpha$ is Hausdorff, then $\prod X_\alpha$ is Hausdorff.
\item Let $A_\alpha$ be subsets of $X_\alpha$, then
\[ \prod \bar{A}_\alpha = \overline{\prod A_\alpha}. \]
\item Let each $X_\alpha$ have a basis $\mathcal{B}_\alpha$. The collection of all the sets of the form 
\[ \prod_{\alpha\in J}B_\alpha \qquad B_\alpha\in\mathcal{B}_\alpha \]
serves as a basis for the box topology.
\item Let $A_\alpha$ be subspaces of $X_\alpha$, for each $\alpha\in J$. Then $\prod A_\alpha$ is a subspace of $\prod X_\alpha$ if both products are given the box topology.
\end{itemize}
\end{lemma}
\subsubsection{Failure of metrisability}
\begin{lemma}
$\R^\omega$ is not metrisable in the box topology.
\end{lemma}
\subsubsection{Failure of continuity}
\subsubsection{Failure of compactness}

\section{The quotient topology}
Let $X$ be a topological space. A quotient set can always be defined by a surjective function $f:X\to A$ to a set $A$. Then $A$ can be identified with a partition $X^*$ of $X$. Now we would like to define a topology on the partition. We can think of the quotient as shrinking each partition to a single point. Thus it is natural to call a subset of $A$ open if the union of the corresponding partitions is open:
\[ \text{$V$ is open in $A$}\quad \Leftrightarrow_{\text{def}}\quad \text{$p^{-1}(V)$ is open in $X$}. \]
This gives us the following definition:
\begin{definition}
Let $X,Y$ be topological spaces and $p:X\to Y$ a surjective map. The map $p$ is a \udef{quotient map} if
\[ \text{$V$ is open in $Y$} \iff \text{$p^{-1}(V)$ is open in $X$} \]
\end{definition}
This condition is stronger than continuity.
\begin{definition}
Let $X$ be a topological space.
\begin{itemize}
\item Let a be $A$ a subset, and $p:X\to A$ a surjective map. There exists exactly one topology on $A$ relative to which $p$ is a quotient map; it is called the \udef{quotient topology} induced by $p$.
\item Let $X^*$ be a partition of $X$ and $p:X\to X^*$ the surjective map that carries each point of $X$ to its partition. In the quotient topology induced by $p$, the space $X^*$ is called a \udef{quotient space} of $X$.
\end{itemize}
\end{definition}
We can also characterise the notion of quotient map in another way, starting from the following definition:
\begin{definition}
A subset $C$ of a topological space $X$ is \udef{saturated} with respect to a surjective map $p:X\to Y$ if $C$ is the complete inverse image of a subset of $Y$, i.e.\ it contains every set $p^{-1}(\{y\})$ that it intersects.
\end{definition}
\begin{lemma}
A surjective map $p$ is a quotient map \textup{if and only if} $p$ is continuous and maps saturated open sets of $X$ to open sets of $Y$.
\end{lemma}
\begin{corollary}
\begin{itemize}
\item Surjective continuous open maps are quotient maps.
\item Surjective continuous closed maps are quotient maps.
\end{itemize}
\end{corollary}
There are quotient maps that are neither open or closed.
\begin{proposition}
Let $p:X\to Y$ be a quotient map; let $A$ be a subset of $X$ that is saturated w.r.t. $p$; let $q:A\to p(A) = p|_{A}$.
\begin{enumerate}
\item If $A$ is either open or closed in $X$, then $q$ is a quotient map.
\item If $p$ is either an open map or a closed map, then $q$ is a quotient map.
\end{enumerate}
\end{proposition}
\begin{lemma}
Let $p,q$ be quotient maps.
\begin{enumerate}
\item A composite $q\circ p$ of quotient maps is a quotient map.
\item The product $p\times q$ is not necessarily a quotient map.
\item A quotient space of a Hausdorff space is not necessarily Hausdorff.
\end{enumerate}
\end{lemma}
\begin{proof}
Point (1) follows from
\[ p^{-1}(q^{-1}(U)) = (q\circ p)^{-1}(U). \]
\end{proof}
TODO: theorem 22.2 + corollary



\section{Density}
\begin{definition}
Let $(X,\mathcal{T})$ be a topological space and let $A$ be a subset of $X$. Then $A$ is called
\begin{enumerate}
\item \udef{dense} in $X$ if the closure of $A$ is the whole of $X$: $\overline{A} = X$;
\item \udef{rare} or \udef{nowhere dense} if its closure has empty interior: $(\overline{A})^\circ = \emptyset$;
\item \udef{meagre} (or a \udef{set of first category}) if it is a countable union of rare subsets of $X$;
\item \udef{nonmeagre} (or a \udef{set of second category}) if it is not meagre;
\item \udef{comeagre} if its complement $X\setminus A$ is meagre in $X$.
\end{enumerate}
\end{definition}
\begin{lemma} \label{densityEquivalences}
Let $A$ be a subset of a topological space $X$. $A$ being is dense in $X$ is equivalent to any of the following:
\begin{enumerate}
\item every element of $X$ either lies in $A$ or is a limit point of $A$;
\item $A^c$ has empty interior.
\end{enumerate}
\end{lemma}
\begin{proof}
For the first point: $\overline{A} = A\cup A' = X$.

For the second point:
\[ X = \overline{A} \iff X = ((A^c)^\circ)^c \iff (A^c)^\circ = X^c = \emptyset. \]
\end{proof}

\begin{lemma} \label{nowhereDensityEquivalence}
Let $X$ be a topological space and $A\subset X$ a subset. Then $A$ is nowhere dense \textup{if and only if} $\overline{A}^c$ is dense.
\end{lemma}
\begin{proof}
We calculate:
\[ (\overline{A})^\circ = \emptyset \iff \overline{(\overline{A}^c)}^c = \emptyset \iff \overline{(\overline{A}^c)} = X. \]
\end{proof}

\begin{lemma} \label{meagreSubset}
Any subset of a meagre set is meagre.
\end{lemma}
\begin{proof}
Let $A = \bigcup_k R_k$ be meagre and $B \subseteq A$. Then
\[ B = B\cap A = B\cap \left( \bigcup_k R_k \right) = \bigcup_k B\cap R_k. \]
Now for each $k$, $B\cap R_k \subset R_k$. So $\overline{B \cap R_k}^\circ \subseteq \overline{R_k}^\circ = \emptyset$, using lemma \ref{closureInteriorSubsets}, and thus $B\cap R_k$ is nowhere dense. 
\end{proof}

\begin{lemma} \label{denseSubsetOfDenseSubspaceIsDense}
Let $Y$ be a dense subspace of a topological space $X$. Let $S$ be a dense subset of $Y$. Then $S$ is dense in $X$.
\end{lemma}
\begin{proof}
Let $\overline{S}$ be the closure of $S$ in $X$. Then by \ref{subspaceClosure} we have $Y = Y\cap \overline{S}$, so $\overline{S} \supseteq Y \supseteq S$, which, taking the closure, implies $\overline{S} \supseteq \overline{Y} \supseteq \overline{S}$. Thus $\overline{S} = \overline{Y} = X$.
\end{proof}

\begin{definition}
A topological space $Y$ has the \udef{unique extension property} if for any topological space $X$, any continuous functions $f,g:X\to Y$ and any dense subset $E\subset X$ we have
\[ \forall x\in E: f(x)=g(x) \quad\implies\quad f = g. \]
\end{definition}

TODO:
\begin{proposition}
\begin{enumerate}
\item Assume $X$ Hausdorff and quotient map open, then $X/\sim$ is Hausdorff iff $\sim$ is closed in $X\times X$.
\item $X$ Hausdorff iff diagonal is closed
\item Let $f,g: A\to B$ be continuous functions. If $B$ is Hausdorff, then $\setbuilder{x\in A}{f(x) = g(x)}$ is closed. (Pre-image of diagonal set)
\end{enumerate}
\end{proposition}

\begin{proposition} \label{uniqueExtensionHausdorff}
A topological space $Y$ has the unique extension property \textup{if and only if} $Y$ is Hausdorff.
\end{proposition}
\begin{proof}
First assume $Y$ Hausdorff. Take functions $f,g: E\subset X \to Y$ that agree on $E$. They must agree on a closed set (TODO ref), thus at least on $\overline{E} = X$.

Now suppose $Y$ is not Hausdorff. TODO \url{https://www.jstor.org/stable/2315068?seq=1#metadata_info_tab_contents}
\end{proof}



\begin{lemma}
Let $X$ be a topological space and $E\subset X$ a .
\begin{enumerate}
\item If for any dense subspace $E\subset X$ the only continuous extension of $\id_E$ to $X$ is $\id_X$, then $X$ is $T_0$.
\item If $X$ is $T_2$, then for any dense subspace $E\subset X$ the only continuous extension of $\id_E$ to $X$ is $\id_X$. 
\end{enumerate}
$T_1$ is neither necessary nor sufficient.
\end{lemma}
\begin{proof}
TODO \url{https://math.stackexchange.com/questions/1592144/does-the-identity-map-on-a-dense-subset-of-a-space-extend-uniquely/1592169}
\end{proof}


\subsection{The Baire property}
\begin{definition}
A topological space $X$ has the \udef{Baire property} if it satisfies either of the following equivalent conditions:
\begin{enumerate}
\item every countable union of closed nowhere dense sets has empty interior;
\item every countable intersection of open dense sets is dense.
\end{enumerate}
These properties are equivalent because a subset has empty interior if and only if its complement is dense, see lemma \ref{densityEquivalences}.
\end{definition}


\begin{lemma} \label{BaireEquivalents}
A topological space $X$ is Baire \textup{if and only if} either of the following equivalent conditions:
\begin{enumerate}
\item every meagre subset of $X$ is either empty or not open;
\item every non-empty open subset of $X$ is a nonmeagre subset of $X$;
\item every comeagre subset of $X$ is dense in $X$.
\end{enumerate}
\end{lemma}
\begin{proof}
We prove the characterisation of spaces with the Baire property using countable unions implies the first point, the last point implies the countable intersection Baire condition.
\begin{itemize}[leftmargin=3cm]
\item[$\boxed{\text{Baire}\Rightarrow (1)}$] Every meagre set $A = \bigcup_k R_k$ (where all $R_k$ are nowhere dense) is a subset of $\bigcup_k \overline{R_k}$ where $\overline{R_k}$ are closed nowhere dense sets. Thus if the Baire property holds, $\bigcup_k \overline{R_k}$ has empty interior, meaning $A$ has empty interior. So either $A$ is empty or not open.
\item[$\boxed{(1) \Leftrightarrow (2)}$] By contraposition.
\item[$\boxed{(1) \Rightarrow (3)}$] Suppose $A$ is a meagre set. Then $A^\circ$ must also be meagre, by \ref{meagreSubset}. Now $A^\circ$ is certainly open, so by $(1)$ it must be empty. Thus $A^c$ is dense, by lemma \ref{densityEquivalences}.
\item[$\boxed{(3) \Rightarrow \text{Baire}}$] Let $A = \bigcap_k O_k$ where all $O_k$ are open dense sets. Then $A^c = \bigcup_k O_k^c$. Now for each $k$, $O_k^c$ is nowhere dense by lemma \ref{nowhereDensityEquivalence}, because $\overline{O_k^c}^c = O_k^\circ$ is still dense. Thus $A^c$ is meagre and $A$ is comeagre, so $A$ is dense in $X$.
\end{itemize}
\end{proof}

A topological space has the Baire property if and only if it has the property locally, in the following sense:
\begin{lemma}
A topological space $X$ has the Baire property \textup{if and only if} every point in $X$ has a neighbourhood with the Baire property.
\end{lemma}
\begin{proof}
If $X$ is Baire, the neighbourhood can simply be taken to be $X$.

Assume every point in $X$ has a neighbourhood with the Baire property.
We will prove point (2) in lemma \ref{BaireEquivalents} holds.
Take a non-empty open subset $A$ of $X$.
As $A$ is non-empty, we can take a point $x\in A$ and find a neighbourhood $U$ of $x$ with the Baire property.
Then $A\cap U$ is a non-empty open subset of $U$ and thus must not be meagre in $U$.
By contraposition of lemma \ref{meagreSubset}, we see that $A$ must be non-meagre in $X$, proving the Baireness of $X$.
\end{proof}

\begin{theorem}[Baire category theorem] \label{BaireCategory} \hspace{1em}
\begin{enumerate}
\item Every complete pseudometric space has the Baire property.
\item Every locally compact Hausdorff space has the Baire property.
\end{enumerate}
\end{theorem}
\begin{proof}
TODO + relocate
\end{proof}

\section{Connectedness}
\begin{definition}
Let $X$ be a topological space. A \udef{separation} of $X$ is a pair $U,V$ of disjoint nonempty open subsets of $X$ whose union is $X$.

The space $X$ is said to be \udef{connected} if there does not exist a separation of $X$.
\end{definition}
\begin{lemma}
A space $X$ is connected \textup{if and only if} the only subsets of $X$ that are both open and closed in $X$ are $\emptyset$ and $X$.
\end{lemma}
\begin{proof}
We prove the contrapositive of both implications.
\begin{itemize}
\item[$\boxed{\Rightarrow}$] Let $A$ be a nonempty proper subset of $X$ that is both open and closed in $X$. The sets $A$ and $X\setminus A$ form a separation.
\item[$\boxed{\Leftarrow}$] Let $U,V$ be a separation. Then $U$ is open. It is also closed, because its complement in $X$ is $V$, which is open.
\end{itemize}
\end{proof}
The following lemma characterises separations in the subspace topology.
\begin{lemma}
Let $Y$ be a subspace of a topological space $X$. A pair of disjoint nonempty sets $A,B$ constitute a separation of the subspace $Y$ \textup{if and only if} neither set contains a limit point of the other in $X$.
\end{lemma}
\begin{proof}

\end{proof}

\subsection{Path connectedness}

\section{Compactness}
TODO relatively compact.

TODO every compact set in (pseudo??)metric space is closed and bounded. Converse is not automatic (Heine-Borel property).

\begin{proposition} \label{imageCompactIsCompact}
The continuous image of a compact set is compact.
\end{proposition}
\begin{corollary} \label{imageCompactIsClosedBounded}
If $f:X\to Y$ is a continuous function from a topological space to a metric space and $K\subset X$ a compact set, then $f[K]$ is closed and bounded.
\end{corollary}
\begin{corollary}
Let $f:X\to V$ be a function from a compact topological space to a TVS with Heine Borel property. Then $\im(f)$ has a maximum and minimum.
\end{corollary}

\begin{proposition}
A compact set in a Hausdorff space is closed.
\end{proposition}

\begin{proposition} \label{compactToHausdorffHomeomorphism}
Let $f:X\to Y$ be a continuous bijection between topological spaces. If $X$ is compact and $Y$ is Hausdorff, then $f$ is a homeomorphism.
\end{proposition}
\begin{proof}
We just need to prove $f^{-1}$ is continuous. This is equivalent to $f$ being closed by \ref{continuity}. TODO
\end{proof}

\begin{proposition}
Let $X$ be a topological space. Then $X$ is compact \textup{if and only if} every ultrafilter converges.
\end{proposition}
\begin{proof}
We prove the contrapositive. Assume $U$ is an ultrafilter that does not converge. Then for all $x\in X$ we have $\neighbourhood(x) \nleq U$, so we can find some $V_x \in \neighbourhood(x) \setminus U$ and thus also an open set $U_x \setminus V_x \in \neighbourhood(x) \setminus U$. (If $U_x$ were in $U$, then also $V_x\in U$, which is not possible.) The set $\{U_x\}_{x\in X}$ is an open cover. By compactness, we can find a finite subcover $\{U_{x_i}\}_{i=0}^n$. Then $U_{x_0}\cup \ldots \cup U_{x_n} = X \in U$. Because every ultrafilter on $\powerset(X)$ is a prime filter \ref, we can find a $U_{x_i}$ such that $U_{x_i}\in U$, which is a contradiction.

Now assume $X$ is not compact. Then there exists an open cover $\{U_i\}_{i\in I}$ that does not have a finite subcover. Consider the family $\{U_i^c\}_{i\in I}$. Every finite intersection of sets in $\{U_i^c\}_{i\in I}$ is nonempty. (If there was such an empty intersection, then taking the complement would give a finite union equaling $X$.) This means the $\cap$-closure of $\{U_i^c\}_{i\in I}$ does not contain $\emptyset$, meaning $F = \mathfrak{F}\{U_i^c\}_{i\in I}$ is a proper filter. Assume, towards a contradiction, that $F$ converges to $a\in X$. We can find a $U_a$ in the cover $\{U_i\}_{i\in I}$ such that $a\in U_a$. Because $F \geq \neighbourhood(x)$, we have $U_a \in F$. But $U_a^c$ is an element of the subbasis of $F$, so $U_a^c\in F$. Thus $U_a\cap U_a^c = \emptyset \in F$, meaning $F$ is not proper.
\end{proof}

\url{https://fa.ewi.tudelft.nl/~hart/37/publications/the_papers/betaR.pdf}

\subsection{Limit point compactness}
\subsection{Local compactness}






\chapter{Sequences, nets and filters}
TODO: move before metric spaces

Sequences, nets and filters can be seen as probes of the topology.
\section{Sequences}
\subsection{Sequential filters}

\begin{lemma}
Let $X$ be a set and $\seq{x_n}$ a sequence in $X$. Then
\[ \Tails(\seq{x_n}) \defeq \setbuilder{\setbuilder{x_n}{n \geq k}}{k\in \N} = \setbuilder{\seq{x_n}[k:\infty]}{k\in \N} \]
is a downward directed subset of $\powerset(X)$.
\end{lemma}
\begin{proof}
Take $A,B\in \Tails(\seq{x_n})$. If $A = \setbuilder{x_n}{n \geq a}$ and $B = \setbuilder{x_n}{n \geq b}$, then $A$ is a lower bound of $\{A,B\}$ if $a\geq b$. Otherwise $B$ is a lower bound.
\end{proof}

\begin{definition}
Let $X$ be a set and $\seq{x_n}$ a sequence in $X$. We call
\begin{itemize}
\item $\Tails(\seq{x_n})$ the \udef{sequential filter base} of $\seq{x_n}$;
\item the filter generated by $\Tails(\seq{x_n})$ in $\powerset(X)$ the \udef{(sequential) filter} of $\seq{x_n}$, which we denote $\TailsFilter(\seq{x_n})$;
\item any filter that is generated by a sequence a \udef{sequential filter};
\item the \udef{kernel} of the sequence the kernel of the associated sequential filter.
\end{itemize}
We have 
\[ \TailsFilter(\seq{x_n}) = \upset \Tails(\seq{x_n}). \]
\end{definition}

\begin{proposition}
Let $X$ be a set. A filter $F\in \powerfilters(X)$ is sequential \textup{if and only if} $F$ contains a countable set and admits a countable base consisting of almost equal sets.
\end{proposition}
\begin{proof}
$\Rightarrow$ Assume $F = \TailsFilter(\seq{x_n})$ for some sequence $\seq{x_n}$. Then every set in $\Tails(\seq{x_n})$ is countable and an element of $F$.

$\Leftarrow$ Assume $F$ contains a countable set. Consider the intersection of all countable sets in $F$.
\end{proof}

\subsection{Limits and convergence}
\begin{definition}
Let $(X,\mathcal{T})$ be a topological space and $(a_n)_{n\in\N}$ a sequence in $X$. We can view this sequence as a function $\N\subset (\N\cup\{\infty\})\to X$.

We define \udef{limit} of the sequence as a limit of this function at $\infty$ if $\N\cup\{\infty\}$ is equipped with the order topology.
\end{definition}
By \ref{HausdorffUniqueLimit} a sequence has at most one limit $L\in X$ if $X$ is Hausdorff. In this case we call it the limit of the sequence and write
\[ \lim_{n\to \infty}a_n = L. \]
We call a sequence \udef{divergent} if it does not have a limit and
 \udef{convergent} if it does have a limit $L$. In this last case we say the sequence \udef{converges} to $L$.

\begin{proposition} \label{sequenceConvergence}
Let $(X,\mathcal{T})$ be a topological space and $(a_n)_{n\in\N}$ a sequence in $X$. Then the sequence converges to $L\in X$ \textup{if and only if}
\[ \forall \;\text{open neighbourhood}\; V(L): \exists n_0\in \N: \forall n\geq n_0: a_n\in V(L). \]
\end{proposition}
\begin{proof}
Assume the sequence converges to $L$ and take an arbitrary open neighbourhood $V(L)$. Then there exists an open neighbourhood $U(\infty)$ such that $a[U\setminus\{\infty\}] \subseteq V$. Now by definition of the order topology, there exists an interval $]m, \infty]\subseteq U(\infty)$. Then $a[\;m, \infty[\;] \subseteq V$ and by setting $n_0=m+1$ we get the criterion of the proposition.

Conversely assume the criterion and fix $V(L)$. Then we can take $U(\infty)=]n_0, \infty]$.
\end{proof}

\begin{lemma} \label{subsequencesConverge}
Let $(X,\mathcal{T})$ be a topological space and $(a_n)_{n\in\N}$ a sequence in $X$ that converges to $L$. Then all subsequences converge to $L$.
\end{lemma}

TODO: Bolzano-Weierstrass (sequence version + accumulation point version)

\subsection{Sequential spaces}
\subsubsection{The sequential topology}
\begin{definition}
Let $(X,\mathcal{T})$ be a topological space and $S\subseteq X$ a subset.
\begin{itemize}
\item The \udef{sequential closure} of $S$ in $X$ is the set
\[ \operatorname{SeqCl}(S) \defeq \setbuilder{x\in X}{\text{$\exists$ a sequence in $S$ that converges to $x$ in $X$}}. \]
\item The \udef{sequential interior} of $S$ in $X$ is the set
\[ \operatorname{SeqInt}(S) \defeq \setbuilder{s\in S}{\text{every sequence in $X$ that converges to $s$ has a tail in $S$}}. \]
\end{itemize}
We call $S$
\begin{itemize}
\item \udef{sequentially open} if $S = \operatorname{SeqInt}(S)$;
\item \udef{sequentially closed} if $S = \operatorname{SeqCl}(S)$;
\item a \udef{sequential neighbourhood} of a point $x\in X$ if $x\in \operatorname{SeqInt}(S)$.
\end{itemize}
\end{definition}
A sequentially closed set is a set $S$ such that all limits of sequences in $S$ are also in $S$.

\begin{lemma} \label{sequentialInteriorClosure}
Let $(X,\mathcal{T})$ be a topological space and $R,S\subseteq X$  subsets. Then
\begin{enumerate}
\item $\operatorname{SeqInt}(S) = (\operatorname{SeqCl}(S^c))^c$;
\item $\operatorname{SeqCl}(\emptyset) = \emptyset$ and $\operatorname{SeqCl}(X) = X$;
\item $S\subseteq \operatorname{SeqCl}(S)$;
\item $\operatorname{SeqCl}(R\cup S) = \operatorname{SeqCl}(R)\cup \operatorname{SeqCl}(S)$;
\item $\operatorname{SeqCl}(S) \subseteq \bar{S}$ and $\operatorname{SeqInt}(S) \supseteq S^\circ$.
\end{enumerate}
\end{lemma}
\begin{proof}
(1) Both sides of the equation are equivalent to
\[ \setbuilder{s\in S}{\text{$\nexists$ a sequence in $X\setminus S$ that converges to $s$ in $X$}}. \]

(2) There are no sequences that converge to a point in $\emptyset$ and all points $x\in X$ are the limit of a constant sequence $n\mapsto x$.

(3) The constant sequence $n\mapsto x$ converges to $x$.

(4) We can find a subsequence in $R$ or in $S$. All subsequences converge by \ref{subsequencesConverge}.

(5) Let $x\in \operatorname{SeqCl}(S)$, so there is a sequence $(a_n)$ in $S$ that converges to $x$. Take an arbitrary open neighbourhood $V$ of $x$. Then by convergence there is a subsequence of $(a_n)$ that is a sequence in $V$. In particular $V$ intersects $S$. So $x\in \bar{S}$ by \ref{closure}.
\end{proof}

\begin{proposition} \label{sequentialTopology}
Let $(X,\mathcal{T})$ be a topological space. The set of all sequentially open sets forms a topology $\mathcal{T}_\text{seq}$ on $X$. This topology is finer than the original topology.
\end{proposition}
\begin{proof}
First note that sequentially closed sets are complements of sequentially open sets by point 1. of \ref{sequentialInteriorClosure}.

By point 2. of \ref{sequentialInteriorClosure}, $\emptyset$ and $X$ are both clopen.

We will prove the rest using closed sets. By point 4. of \ref{sequentialInteriorClosure} finite unions of sequentially closed sets are sequentially closed.

Let $\bigcap_{i\in I}K_i$ be an arbitrary intersection of sequentially closed sets $K_i$. We only need to prove
\[ \operatorname{SeqCl}\left(\bigcap_{i\in I}K_i\right) \subseteq \bigcap_{i\in I}K_i \]
because the other inclusion is immediate. Take an $x\in \operatorname{SeqCl}\left(\bigcap_{i\in I}K_i\right)$. Then there is a sequence in $\bigcap_{i\in I}K_i$ that converges to $x$. Because of the intersection this sequence is in each $K_i$ and thus so is $x$.

The fineness of the topology follows from point 5. of \ref{sequentialInteriorClosure}.
\end{proof}
\begin{corollary}
Let $(X,\mathcal{T})$ be a topological space and $S\subseteq X$  a subset. Then
\[ \text{open/closed} \quad\implies\quad \text{sequentially open/closed.} \]
\end{corollary}

\begin{proposition} \label{sequentialTopologySameConvergentSequences}
Let $(X,\mathcal{T})$ be a topological space and $(x_n)$ a sequence in $X$. Then $x_n\to x$ in $(X,\mathcal{T})$ \textup{if and only if} $x_n\to x$ in $(X,\mathcal{T}_\text{seq})$.
\end{proposition}
\begin{proof}
Now $\mathcal{T}_\text{seq}$ is finer than $\mathcal{T}$, so the $\Leftarrow$ direction is evident. For the $\Rightarrow$ direction, assume $x_n\to x$ in the original topology. Let $V(x)$ be an open neighbourhood in the sequential topology. By definition of the sequential topology $(x_n)$ has a tail in $V$. This means $x_n\to x$ in the sequential topology by \ref{sequenceConvergence}.
\end{proof}

\subsubsection{Transfinite sequential closure}
It is possible that the sequential closure is not idempotent (unlike the normal topological closure), i.e.\
\[ \operatorname{SeqCl}(\operatorname{SeqCl}(S)) \neq \operatorname{SeqCl}(S). \]

\subsubsection{Sequential continuity}
\begin{definition}
A function $f:(X,\mathcal{T})\to(Y,\mathcal{T}')$ is called \udef{sequentially continuous} if
\[ f:(X,\mathcal{T}_\text{seq})\to(Y,\mathcal{T}'_\text{seq}) \]
is continuous. i.e.\ $f$ is continuous when $X,Y$ are equipped with their sequential topologies.
\end{definition}

\begin{proposition} \label{sequentialContinuity}
A function $f:(X,\mathcal{T})\to(Y,\mathcal{T}')$ is sequentially continuous \textup{if and only if} for every sequence $(x_n)_{n\in\N}$ in $X$ and $x\in X$
\[ x_n \to x \;\;\text{in}\; (X,\mathcal{T}) \quad\implies\quad f(x_n)\to f(x) \;\;\text{in}\; (Y,\mathcal{T}'). \]
\end{proposition}
\begin{proof}
First assume this property holds and we want to prove sequential continuity. Let $S\subset Y$ be sequentially closed. Then we need to prove $f^{-1}[S]$ is also sequentially closed. Indeed take a converging sequence $(x_n)$ in $f^{-1}[S]$ with limit $x$. Then $(f(x_n))$ converges to $f(x)$ and $f(x)\in S$. This implies $x\in f^{-1}[S]$, meaning it is sequentially closed. 

Conversely, assume $f$ is sequentially continuous. Let $(x_n)$ be a sequence in $X$ that converges to $x$. Let $V(f(x))\in \mathcal{T}'$ be an open neighbourhood of $f(x)$; $V$ is also sequentially open. Then by continuity we have a $U(x)\in\mathcal{T}_\text{seq}$ such that $f[U]\subseteq V$. Because $U$ is sequentially open, there is an $n_0\in\N$ such that $\forall n\geq n_0: x_n\in U$.
This implies $\forall n\geq n_0: f(x_n)\in f[U]\subseteq V$ and so $(f(x_n))$ converges to $f(x)$.
\end{proof}
\begin{proposition}
Every continuous function is sequentially continuous.
\end{proposition}
\begin{proof}
We use the characterisation of sequential continuity in \ref{sequentialContinuity}. Let $x_n\to x$. Let $V$ be an open neighbourhood of $f(x)$. Then there exists an open neighbourhood $U(x)$ such that $f[U]\subset V$. By \ref{sequenceConvergence} $U$ contains all but finitely many elements of the sequence $(x_n)$. Thus $V$ contains all but finitely many of the elements of the sequence $(f(x_n))$. Take $n_0$ larger than the indices of all elements of $(f(x_n))$ omitted from $V$. By \ref{sequenceConvergence} $f(x_n)\to f(x)$.
\end{proof}

\subsubsection{Sequential spaces}
\begin{definition}
A topological space $(X,\mathcal{T})$ is called a \udef{sequential space} if $\mathcal{T} = \mathcal{T}_\text{seq}$.
\end{definition}

\begin{lemma}
A topological space $(X,\mathcal{T})$ is a sequential space if every sequentially open set is open.
\end{lemma}
\begin{proof}
We already know $\mathcal{T} \subseteq \mathcal{T}_\text{seq}$ from \ref{sequentialTopology}. The hypothesis of the lemma is that $\mathcal{T} \supseteq \mathcal{T}_\text{seq}$. Together this gives $\mathcal{T} = \mathcal{T}_\text{seq}$.
\end{proof}

\begin{lemma}
Let $(X,\mathcal{T})$ be a topological space. Then $X$ equipped with its sequential topology is a sequential space.
\end{lemma}
\begin{proof}
It is enough to show that a sequentially closed set in $(X,\mathcal{T}_\text{seq})$ is also sequentially closed in $(X,\mathcal{T})$. (i.e.\ that passing to the finer topology does not introduce even more sequentially open sets). By \ref{sequentialTopologySameConvergentSequences} the definition of $\operatorname{SeqCl}$ is the same in both topologies, yielding the proof.
\end{proof}

\begin{proposition}
Let $(X,\mathcal{T})$ be a topological space. Then the following are equivalent:
\begin{enumerate}
\item $(X,\mathcal{T})$ is a sequential space;
\item for every subset $S\subset X$ that is not closed in $X$, there exists some $x\in \bar{S}\setminus S$ for which there exists a sequence in $S$ that converges to $x$;
\item $(X,\mathcal{T})$ is the quotient of a first countable space;
\item $(X,\mathcal{T})$ is the quotient of a metric space.
\end{enumerate}
\end{proposition}
TODO: relocate observation about metric spaces.

\begin{proposition}[Universal property of sequential spaces]
Let $(X,\mathcal{T})$ be a topological space. Then $X$ is sequential \textup{if and only if} for every topological space $Y$, a function $f:X\to Y$ is continuous $\Leftrightarrow$ $f$ is sequentially continuous.
\end{proposition}


\subsubsection{$T$-sequential and $N$-sequential spaces}

\subsubsection{Fréchet–Urysohn spaces}
\begin{definition}
A topological space $(X,\mathcal{T})$ is called a \udef{Fréchet–Urysohn space} if for every subset $S\subseteq X$
\[ \operatorname{SeqCl}(S) = \bar{S}. \]
\end{definition}
Clearly every Fréchet–Urysohn space is a sequential space.

\begin{proposition} \label{FrechetUrysohn}
Let $(X,\mathcal{T})$ be a topological space. Then the following are equivalent:
\begin{enumerate}
\item $(X,\mathcal{T})$ is a Fréchet–Urysohn space;
\item every subspace of $X$ is a sequential space;
\item for every subset $S\subset X$ that is not closed in $X$ and for all $x\in \bar{S}\setminus S$ there exists a sequence in $S$ that converges to $x$.
\end{enumerate}
\end{proposition}

\subsection{Sequences in ordered space}
In this section we will be considering sequences in a totally ordered set $(X,\leq)$ equipped with the order topology.

\begin{lemma} \label{convergentSequenceIsBounded}
A convergent sequence in a totally ordered space has an upper and a lower bound.
\end{lemma}
\begin{proof}
Let $x_n\to x$. Choose a basis element containing $x$. If it is of the form $]a,b_0]$ for some greates element $b_0$, then $b_0$ is the upper bound. If not, it is of the form $[a_0,b[$ or $]a,b[$. Find an $n_0$ corresponding to this basis element. Then an upper bound is given by
\[ \max(x[\;[0,n_0]\;]\cup\{b\}). \]
The lower bound is analogous.
\end{proof}

\begin{proposition} \label{limitPreservesInequality}
Let $(a_n)$ and $(b_n)$ be convergent sequences in a totally ordered space such that $a_n\leq b_n$ for all $n\in\N$. Then
\[ \lim_{n\to \infty}a_n \leq \lim_{n\to \infty}b_n. \]
\end{proposition}
\begin{proof}
Let $a_n\to a$ and $b_n\to b$. If $a=b$ then the proposition is valid. Now assume $a\neq b$. If $a$ or $b$ are either the greatest or the least element, the proposition is valid. Now assume this is not the case.

Assume towards a contradiction that $a>b$. Then we can find open neighbourhoods of $a$ and $b$ of the form $]b,d[$ and $]c,a[$, respectively. Now find $n_0, n_1$ such that $\forall n\geq n_0: a_n \in ]b,d[$ and $\forall n\geq n_1: b_n \in ]c,a[$. 
Then for all $n \geq \max\{n_0,n_1\}$ we have $a_n\in ]b,d[$ and $b_n\in ]c,a[$, implying $a_n > b_n$ which is a contradiction.
\end{proof}
It is easy to show that this does not in general hold for the strict inequality $<$.

\begin{proposition}[Squeeze theorem for sequences]
Let $(a_n)$, $(b_n)$ and $(c_n)$ be sequences in a totally ordered space such that
\[ \forall n\in \N: a_n\leq b_n \leq c_n. \]
If $(a_n)$ and $(c_n)$ are convergent with the same limit $L$, then
\[ \lim_{n\to \infty}b_n = L. \]
\end{proposition}
\begin{proof}
Let $V(L)$ be an open neighbourhood of $L$. By definition of the order topology there is an interval $I = ]x,y[ \subset V$ such that $L\in I$. Then find $n_0$ and $n_1$ such that $\forall n\geq n_0: a_n\in I$ and $\forall n\geq n_1: c_n\in I$. Then set $n_2 = \max\{n_0,n_2\}$ and we have $\forall n\geq n_2:$
\[ x \leq a_n \leq b_n \quad \text{and} \quad  b_n \leq c_n \leq y. \]
By transitivity we have $b_n\in I \subset V$.
\end{proof}

\begin{proposition} \label{sequenceToSupInf}
Let $X$ be an ordered space and $A$ a subspace.  Assume the axiom of dependent choice.
\begin{enumerate}
\item If $A$ has a supremum $a$, then there exists a sequence in $A$ that converges to $a$ in $X$.
\item If $A$ has an infimum $b$, then there exists a sequence in $A$ that converges to $b$ in $X$.
\end{enumerate}
\end{proposition}
\begin{proof}
Assume the supremum $a$ of $A$ exists. If $a\in A$ we can take the constant sequence $(a)_{n\in \N}$.

If $a\notin A$, we can find for each $x_i\in A$ an $x_{i+1}$ satisfying $x_i < x_{i+1} < a$. The sequence thus defined converges by monotone convergence.
\end{proof}
In many cases the axiom of dependent choice is superfluous, if the details of the spaces $X,A$ allow for the construction of $x_{i+1}$ from $x_i$.

\subsubsection{Divergence to $\pm\infty$}
\begin{definition}
Let $(x_n)$ be a sequence in a totally ordered space $X$. Then
\begin{itemize}
\item $(x_n)$ \udef{diverges to $+\infty$} if $\forall M\in X: \exists n_0\in\N: \forall n\geq n_0: x_0 > M$; and
\item $(x_n)$ \udef{diverges to $-\infty$} if $\forall M\in X: \exists n_0\in\N: \forall n\geq n_0: x_0 < M$.
\end{itemize}
We write $\lim_{n\to\infty}x_n = +\infty$ and $\lim_{n\to\infty}x_n = -\infty$, respectively.
\end{definition}

\begin{lemma}
Let $(x_n)$ be a sequence in a totally ordered space $X$. Then
\begin{enumerate}
\item if $(x_n)$ is increasing, but not bounded above, it diverges to $+\infty$;
\item if $(x_n)$ is decreasing, but not bounded below, it diverges to $-\infty$.
\end{enumerate}
\end{lemma}

\subsection{Sequences in complete ordered space}
\subsubsection{Monotone convergence}
\begin{proposition}[Monotone convergence] \label{sequenceMonotoneConvergence}
Let $(X,\leq)$ be a complete totally ordered space and let $(x_n)$ be a sequence in $X$.
\begin{enumerate}
\item If $(x_n)$ is increasing and bounded above, then it is convergent with limit $\sup_n x_n$.
\item If $(x_n)$ is decreasing and bounded below, then it is convergent with limit $\inf_n x_n$.
\end{enumerate}
\end{proposition}
\begin{proof}
We prove the first point. The second is analogous.

Let $V(\sup_n x_n)$ be an open and $]x,y[\subset V$ such that $\sup_n x_n \in ]x,y[$. Now because $x<\sup_n x_n$ it is not an upper bound of the sequence and there exists an $x_{n_0}> x$. Because the sequence is increasing (and $y$ is a a strict upper bound), all $x_n$ where $n\geq n_0$ are in $]x,y[\subset V$.
\end{proof}

\subsubsection{Limes superior and inferior}
\begin{definition}
Let $(X,\leq)$ be a complete totally ordered space and let $(x_n)$ be a sequence in $X$. We define
\begin{itemize}
\item the \udef{limes superior} or \udef{limit superior} or \udef{limsup} of $(x_n)$ as
\[ \limsup_{n\to\infty} x_n = \lim_{n\to \infty} \sup\setbuilder{x_m}{m\geq n}; \]
\item the \udef{limes inferior} or \udef{limit inferior} or \udef{liminf} of $(x_n)$ as
\[ \liminf_{n\to\infty} x_n = \lim_{n\to \infty} \inf\setbuilder{x_m}{m\geq n}. \]
\end{itemize}
\end{definition}
The liminf and limsup may not exist.
\begin{lemma}
Let $(X,\leq)$ be a complete totally ordered space and let $(x_n)$ be a sequence in $X$. The limsup and liminf exist \textup{if and only if}  $(x_n)$ is bounded above and below.
\end{lemma}
\begin{proof}
The sequences $\sup\setbuilder{x_m}{m\geq n}$ and $\inf\setbuilder{x_m}{m\geq n}$ are bounded if the limsup and liminf exist and bound $(x_n)$.

The converse follows because the sequences $\sup\setbuilder{x_m}{m\geq n}$ and $\inf\setbuilder{x_m}{m\geq n}$ are monotone.
\end{proof}

\begin{proposition} \label{characterisationLimsupLiminf}
Let $(X,\leq)$ be a complete totally ordered space and let $(x_n)$ be a bounded sequence in $X$. Then 
\begin{enumerate}
\item $L_s = \limsup_{n\to\infty} x_n$ \textup{if and only if}
\begin{align*}
&\forall b > L_s: \exists n_0\in \N:\forall n\geq n_0: x_n < b \qquad \text{and} \\
&\forall a < L_s: \forall n_0\in \N:\exists n\geq n_0: a < x_n
\end{align*}
\item $L_i = \liminf_{n\to\infty} x_n$ \textup{if and only if}
\begin{align*}
&\forall a < L_i: \exists n_0\in \N:\forall n\geq n_0: a < x_n \qquad \text{and} \\
&\forall b > L_i: \forall n_0\in \N:\exists n\geq n_0: x_n < b.
\end{align*}
\end{enumerate}
\end{proposition}

\begin{proposition}
Let $(X,\leq)$ be a complete totally ordered space and let $(x_n)$ be a sequence in $X$. Then $(x_n)$ is convergent \textup{if and only if}
\[ \liminf_{n\to \infty} x_n = \limsup_{n\to \infty} x_n. \]
In this case
\[ \lim_{n\to \infty} x_n = \liminf_{n\to \infty} x_n = \limsup_{n\to \infty} x_n. \]
\end{proposition}
\begin{proof}
Assume $(x_n)$ is a sequence with identical liminf and limsup. Now
\[ \inf\setbuilder{x_m}{m\geq n} \leq x_n \leq \sup\setbuilder{x_m}{m\geq n} \]
so we can apply the squeeze theorem for sequences.

For the converse we use \ref{characterisationLimsupLiminf}.
\end{proof}

\begin{lemma} \label{monotonicityLimsupLiminf}
Let $(a_n)$ and $(b_n)$ be bounded sequences in a totally ordered space such that $a_n\leq b_n$ for all $n\in\N$. Then
\[ \limsup_{n\to \infty}a_n \leq \limsup_{n\to \infty}b_n \quad\text{and}\quad \liminf_{n\to \infty}a_n \leq \liminf_{n\to \infty}b_n. \]
\end{lemma}








\begin{lemma} \label{sequencesSupInf}
There exist sequences converging to supremum and infimum.
\end{lemma}

\subsection{Completeness}


\section{Nets}
\begin{definition}
Let $(I,\leq)$ be a directed set and $X$ a set. Then a \udef{net} in $X$ is a function $I\to X$. The directed set $I$ is called the \udef{index set}.
\end{definition}
In particular sequences are nets because $(\N,\leq)$ is a directed set.

Note we do \emph{not} require $I$ to be a partial order.

\subsection{Relating filters and nets}
\subsubsection{From nets to filters}
\begin{lemma}
Let $X$ be a set, $I$ a directed index set and $\seq{x_i}_{i\in I}$ a net in $X$. Then
\[ \Tails(\seq{x_i}_{i\in I}) \defeq \setbuilder{\setbuilder{x_i}{i \geq j}}{j\in I} \]
is a downward directed subset of $\powerset(X)$.
\end{lemma}
\begin{proof}
Take $A,B\in \Tails(\seq{x_i}_{i\in I})$. If $A = \setbuilder{x_i}{i \geq a}$ and $B = \setbuilder{x_i}{i \geq b}$, then we can find some $k\in I$ such that $k \geq a$ and $k\geq b$. Then $\setbuilder{x_i}{i \geq k}$ is a subset of both $A$ and $B$.
\end{proof}

\begin{definition}
Let $X$ be a set, $I$ a directed index set and $\seq{x_i}_{i\in I}$ a net in $X$. We call
\begin{itemize}
\item $\Tails(\seq{x_i}_{i\in I})$ the \udef{filter base} of $\seq{x_i}_{i\in I}$;
\item the filter generated by $\Tails(\seq{x_i}_{i\in I})$ in $\powerset(X)$ the \udef{associated filter} of $\seq{x_i}_{i\in I}$, which we denote $\TailsFilter(\seq{x_i}_{i\in I})$;
\item the \udef{kernel} of the net the kernel of the associated filter.
\end{itemize}
We have 
\[ \TailsFilter(\seq{x_i}_{i\in I}) = \upset \Tails(\seq{x_i}_{i\in I}). \]
\end{definition}

\subsubsection{From filters to nets}
\begin{lemma} \label{filterIndex}
Let $X$ be a set and $F\in\powerfilters(X)$ a proper filter. Then
\[ I_F \defeq \setbuilder{(A,x)\in F\times X}{x\in A} \]
is a directed proset when ordered by $(A,x)\leq (B,y) \iff B\subseteq A$.
\end{lemma}
\begin{proof}
Transitivity and reflexivity follow from the properties of $\subseteq$.

Take $(A,x)$ and $(B,y)$ in $I_F$. Then $A\cap B\in F$ and $A\cap B \neq \emptyset$ because $F$ is proper. So we can find $z\in A\cap B$. Then $(A,x) \geq (A\cap B,z)$ and $(B,y) \geq (A\cap B, z)$.
\end{proof}

\begin{definition}
Let $X$ be a set, $F\in\powerfilters(X)\setminus\powerset(X)$ and $I_F$ the directed set of $F$ as in \ref{filterIndex}. Then
\[ I_F \to X: (A,x) \mapsto x \]
is the \udef{associated net} of $F$.
\end{definition}

\begin{lemma} \label{tailsFilterIndex}
Let $X$ be a set, $F\in\powerfilters(X)\setminus\powerset(X)$ and $I_F$ the directed set of $F$ as in \ref{filterIndex}. Then for any $(A,x)\in I_F$,
\[ A = \setbuilder{y}{(B,y) \geq (A,x)}. \]
\end{lemma}
\begin{proof}
$\boxed{\subseteq}$ For all $a\in A$ we have $(A,a)\geq (A,x)$.

$\boxed{\supseteq}$ For any $(B,y) \geq (A,x)$, we have $y\in B\subseteq A$, so $y\in A$.
\end{proof}
\begin{corollary}
Let $X$ be a set and $F\in\powerfilters(X)$. Then $F$ equals the filter associated to its associated net.
\end{corollary}
\begin{proof}
By the lemma we have that $A\subseteq X$ is a tail of the net iff it is an element of the filter.
\end{proof}

\subsection{Convergence}
\begin{definition}
Let $\sSet{X,\xi}$ be a convergence space $\seq{x_i}_{i\in I}$ a net on $X$ and $x\in X$. The net $\seq{x_i}_{i\in I}$ \udef{converges} to $x$ in $\xi$, denoted $\seq{x_i}_{i\in I} \overset{\xi}{\longrightarrow} x$, if the associated filter converges to $x$.
\end{definition}

\begin{proposition}
A topological space is Hausdorff \textup{if and only if} every net converges to at most one point.
\end{proposition}

\subsection{Subnets}

TODO: generalise:
\begin{proposition}
Let $X$ be a topological space and $A\subset X$. If there is a sequence of points of $A$ converging to $x$, then $x\in\bar{A}$. The converse holds if $X$ is metrisable.
\end{proposition}



\chapter{Some topologies}
These are the topologies that these object usually posses. If nothing else is said, these topologies will be assumed.

\section{The metric topology}
\begin{definition}
A \udef{metric} on a set $X$ is a function
\[ d:X\times X \to \R \]
with the properties:
\begin{enumerate}
\item $\forall x,y\in X: d(x,y)\geq 0$ and equality holds \textup{if and only if} $x=y$;
\item $\forall x,y\in X: d(x,y)= d(y,x)$;
\item $\forall x,y,z\in X: d(x,y)+d(y,z)\geq d(x,z)$.
\end{enumerate}
\end{definition}
The number $d(x,y)$ is often called the \udef{distance} between $x$ and $y$ in the metric $d$.
\begin{definition}
\begin{itemize}
\item The \udef{$\epsilon$-ball centered at $x$} is the set
\[ B_d(x,\epsilon) = \{y\;|\; d(x,y)< \epsilon\} \]
of all points $y$ whose distance to $x$ is less than $\epsilon$.
\item The \udef{closed $\epsilon$-ball centered at $x$} is the set
\[ \overline{B}_d(x,\epsilon) = \{y\;|\; d(x,y)\leq \epsilon\} \]
of all points $y$ whose distance to $x$ is less than or equal to $\epsilon$.
\item The \udef{$\epsilon$-sphere centered at $x$} is the set
\[ S_d(x,\epsilon) = \{y\;|\; d(x,y) = \epsilon\}. \]
\end{itemize}
\end{definition}

TODO: tolerance space for give $\epsilon$ !!

\begin{lemma}
Let $X$ be a set and $d$ a metric on $X$. Then the collection of all $\epsilon$-balls forms a basis for a topology on $X$.
\end{lemma}

\begin{definition}
\begin{itemize}
\item A \udef{metric space} $(X,d)$ is a set $X$ together with a metric $d$.
\item The topology generated by the $\epsilon$-balls in $X$ is called the \udef{metric topology} generated by $d$.
\item If $(X,\mathcal{T})$ is a topological space such that there exists a metric $d$ on $X$ such that $\mathcal{T}$ is the metric topology, then $X$ is called \udef{metrisable}.
\end{itemize}
\end{definition}

TODO: metric is continuous + use this to define topology????


Only the local behaviour of the metric is important:
\begin{proposition}
Let $(X,d)$ be a metric space. Define
\[ \bar{d}:X\times X\to \R: (x,y)\mapsto \min\{d(x,y),1\}. \]
Then $\bar{d}$ is a metric that induces the same topology as $d$.
\end{proposition}
The metric $\bar{d}$ is called the \udef{standard bounded metric} corresponding to $d$.

\begin{proposition} \label{ballsCoarseness}
Let $d,d'$ be metrics on the set $X$, inducing $\mathcal{T}$ and $\mathcal{T}'$, respectively. Then $\mathcal{T}'$ is finer than $\mathcal{T}$ \textup{if and only if}
\[ \forall x\in X:\forall \epsilon>0:\exists \delta>0:\quad B_{d'}(x,\delta)\subset B_d(x,\epsilon). \]
\end{proposition}
\begin{proof}
Application of lemma \ref{basisCoarseness}.
\end{proof}

\begin{definition}
Let $(X,d)$ be a metric space and $S\subset X$ a subset. We call $S$ \udef{bounded} if there exists a ball $B_d(x,\epsilon)$ such that $S\subseteq B_d$.
\end{definition}

\subsection{Continuous functions in metric spaces}
For maps between metric spaces, the continuity requirement is equivalent to the $\epsilon-\delta$ formulation:
\begin{proposition}
Let $(X,d_X), (Y,d_Y)$ be metric spaces. The continuity of $f:X\to Y$ is equivalent to the condition that, for all $x\in X$:
\[ \forall \epsilon>0:\exists \delta>0:\; d_X(x,y)<\delta \implies d_Y(f(x),f(y))<\epsilon. \]
\end{proposition}
\begin{definition}
Let $f_n:X\to Y$ be a sequence of functions from the set $X$ to the metric space $(Y,d)$. The sequence \udef{converges uniformly} to the function $f:X\to Y$ if
\[ \forall \epsilon>0:\exists N\in \N:\forall n>N:\forall x\in X: \quad d(f_n(x),f(x))<\epsilon.  \]
\end{definition}
\begin{theorem}[Uniform limit theorem]
Let $f_n:X\to Y$ be a sequence of continuous functions from the topological space $X$ to the metric space $Y$. If $(f_n)$ converges uniformly to $f$, then $f$ is continuous.
\end{theorem}
\begin{proof}
We show $f$ is continuous at every point, so choose a point $x_0$ and a neighbourhood $V$ of $f(x_0)$. We then need to show that we can find a neighbourhood $U$ of $x_0$ such that $f(U)\subset V$ Now choose an $\epsilon>0$ such that $B(f(x_0), \epsilon)\subset V$. By uniform convergence, we can find an $N\in\N$ such that
\[\forall n>N:\forall x \in X:\quad d(f_n(x),f(x))<\epsilon/3.\]
By continuity of $f_N$, choose a neighbourhood $U$ of $x_0$ such that $f(U)\subset B(f_N(x_0),\epsilon/3)$. We claim this is the $U$ we need: take an arbitrary $x\in U$, then
\begin{align*}
d(f(x),f(x_0))&\leq d(f(x), f_N(x)) + d(f_N(x), f_N(x_0)) + d(f_N(x_0),f(x_0)) \\
&< \epsilon/3 + \epsilon/3 + \epsilon/3 = \epsilon.
\end{align*}
So $f(U)\subset B(f(x_0), \epsilon)\subset V$.
\end{proof}

TODO: convex sets; distance point to set (as special case of distance set to set).

TODO: A subspace Y of a Banach space X is complete if and only if the set Y is closed in X.

\begin{proposition} \label{distanceToSetContinuous}
Let $\sSet{X,d}$ be a metric space and $A\subseteq X$. Then the function
\[ d_A: X\to \R: x\mapsto d(x,A) = \inf\setbuilder{d(x,a)}{a\in A} \]
is continuous.
\end{proposition}
\begin{proof}
TODO
\end{proof}
TODO: We have that $x\in \overline{A}$ iff $d_A(x) = 0$.

\subsection{Maps between metric spaces}
\begin{definition}
Let $(X,d_X)$ and $(Y,d_Y)$ be metric spaces. A map $f:X\to Y$ is an \udef{isometry} or \udef{distance preserving} if
\[ \forall a,b \in X: \quad d_Y(f(a),f(b)) = d_X(a,b). \]
\end{definition}
\begin{lemma} \label{isometryInjective}
An isometry is automatically injective.
\end{lemma}
\begin{proof}
Let $f: X\to Y$. Assume $f(a) = f(b) = y$, then
\[ 0 = d_Y(y,y) = d_Y(f(a),f(b)) = d_X(a,b). \]
By non-degeneracy of the metric we have $a=b$, meaning $f$ is injective.
\end{proof}
\begin{lemma} \label{isometryContinuous}
An isometry is automatically continuous.
\end{lemma}
\begin{proof}
For an isometry $f:X\to Y$, we have
\[ f[B(x,\epsilon)] = B(f(x),\epsilon), \]
so $f$ is continuous at each point $x$. It is then globally continuous by \ref{globalContinuityFromAllPoints}.
\end{proof}
TODO: merge lemmas?

\begin{lemma} \label{isometryClosed}
Let $f:X\to Y$ be an isometry and $X$ a complete metric space. Then $f$ is a closed map.
\end{lemma}
\begin{proof}
Take $K\subset X$ closed and $y\in\overline{f[K]}$. Then there exists a sequence $(f(x_n))$ in $f[K]$ converging to $y$. The sequence $(f(x_n))$ is convergent and thus Cauchy. Because $d(f(x_n), f(x_m)) = d(x_n,x_m)$, the sequence $(x_n)$ must also be Cauchy. It is convergent because $X$ is complete and the limit $x$ lies in $K$ because it is complete (TODO ref). By continuity of $f$, \ref{isometryContinuous}, we have
\[ y = \lim_{n\to\infty} f(x_n) = f(\lim_{n\to\infty}x_n) = f(x) \in f[K]. \]
So $\overline{f[K]} = f[K]$ and $f$ is closed.
\end{proof}

\subsection{Cauchy sequences and completeness}
\begin{definition}
Let $(X,d)$ be a metric space and $(x_n)$ a sequence in $X$. Then $(x_n)$ is called a \udef{Cauchy sequence} if
\[ \forall 0 < \epsilon \in \R: \exists n_0\in \N: \forall m,n \geq n_0: d(x_m,x_n) < \epsilon.  \]
\end{definition}

\begin{lemma}
Let $(X,d)$ be a metric space. Cauchy sequences in $X$ are bounded.
\end{lemma}

\begin{proposition}
Let $(X,d)$ be a metric space. Convergent sequences in $X$ are Cauchy.
\end{proposition}
Spaces in which the converse holds are special.
\begin{definition}
Let $(X,d)$ be a metric space, then $X$ is called \udef{(Cauchy) complete} if all Cauchy sequences converge.
\end{definition}

\begin{proposition} \label{CauchyCriterion}
Let $(X,d)$ be a metric space and $(a_n), (b_n)$ sequences in $X$. If $(b_n)$ is Cauchy and there exists some $A\in\R$ such that
\[ \forall m,n\in\N: d(a_n,a_m) \leq A d(b_n,b_m), \]
then $(a_n)$ is also Cauchy.
\end{proposition}

\begin{proposition}[Completeness criterion] \label{completenessCriterion}
Let $(X,d)$ be a metric space and $S\subset X$ a dense subset. If every Cauchy sequence in $S$ converges in $X$, then $X$ is complete.
\end{proposition}
This proposition depends on the axiom of countable choice.
\begin{proof}
TODO
\end{proof}

\begin{lemma}
The real numbers with the standard topology are complete.
\end{lemma}

\subsubsection{Completion}
\begin{proposition}
Let $(X,d_X)$ be a metric space. There exists a complete metric space $(Y,d_Y)$ and an isometry $\pi:X\hookrightarrow Y$ such that $\pi[X]$ is a dense subspace of $Y$.
\end{proposition}
We view $X$ as a subspace of $Y$ through $\pi$ and call $Y$ the \udef{completion} of $X$.
\begin{proof}
Let $Y'$ be the space of Cauchy sequences in $X$. Introduce the equivalence relation on $\seq{x_i},\seq{y_j}\in Y'$:
\[ \seq{x_i} \sim \seq{y_i} \qquad \iff\qquad \lim_{i\to\infty} d_X(x_i,y_i) = 0. \]
Let $Y$ be the set of equivalence classes in $Y'$ under this equivalence relation. Define
\[ d_Y: Y\times Y\to \R: ([\seq{x_i}],[\seq{y_i}]) \mapsto \lim_{i\to\infty}d_X(x_i,y_i) \qquad\text{and}\qquad \pi: X\to Y: x\mapsto \seq{x}_i. \]
We need to show that $d_Y$ is well-defined, that it is a metric on $Y$, that $\pi[X]$ is dense in $Y$ and that $(Y,d_Y)$ is complete:
\begin{itemize}
\item Let $[\seq{x_i'}] = [\seq{x_i}]$. Then
\begin{align*}
d_Y([\seq{x_i'}],[\seq{y_i}]) &= \lim_{i\to\infty}d_X(x'_i,y_i) = \lim_{i\to\infty}d_X(x'_i,y_i) + \lim_{i\to\infty}d_X(x_i,x'_i) \\
&= \lim_{i\to\infty}d_X(x_i,x_i')+d_X(x'_i,y_i) \\
&\geq \lim_{i\to\infty}d_X(x_i,y_i) = d_Y([\seq{x_i}],[\seq{y_i}]).
\end{align*}
Similarly we can show $d_Y([\seq{x_i'}],[\seq{y_i}])\leq d_Y([\seq{x_i}],[\seq{y_i}])$, so $d_Y([\seq{x'_i}],[\seq{y_i}]) = d_Y([\seq{x_i}],[\seq{y_i}])$.

We must also show that the domain and codomain of $d_Y$ make sense, i.e.\ the limit exists and does not diverge. It is enough to show that $(d_X(x_i,y_i))$ is a Cauchy sequence, due to the completeness of $\R$. To this end, let $\epsilon>0$. As $\seq{x_i}$ and $\seq{y_i}$ are Cauchy, we can find $N_x,N_y\in\N$ such that $d_X(x_m,x_n)< \epsilon/2$ and $d_X(y_m,y_n) < \epsilon/2$ for all $m,n \geq N_x,N_y$. Then $\forall m,n \geq \max\{N_x,N_y\}$:
\begin{align*}
|d_X(x_m,y_m) - d_X(x_n,y_n)| &\leq |d_X(x_m,x_n)+d_X(x_n,y_m) - d_X(x_n,y_n)| \\
&\leq |d_X(x_m,x_n)+d_X(y_m,y_n)+ d_X(y_n,x_n) - d_X(x_n,y_n)| \\
&= |d_X(x_m,x_n)+d_X(y_m,y_n)| \\
&< \epsilon/2 + \epsilon/2= \epsilon.
\end{align*}
So $(d_X(x_i,y_i))$ is Cauchy and thus converges in $\R$.
\item That $d_Y$ is a metric is easy to check.
\item To prove $\pi[X]$ is dense in $Y$, we just need to show that every element $y = [\seq{x_i}]\in Y$ is the limit of a sequence in $\pi[X]$, because all metric spaces are sequential. We claim $\seq{\pi(x_j)}_j$ converges to $y$.

Let $\epsilon>0$. Because $\seq{x_i}$ is Cauchy, we can find an $N\in\N$ such that $\forall m,n>N: d_X(x_m,x_n) < \epsilon/2$. Take $j\geq N$ arbitrary. Then
\[ \forall i\geq N: d_X(x_i,x_j) < \epsilon/2 \quad \implies\quad \lim_{i\to\infty} d_X(x_i,x_j) = d_Y([\seq{x_i}],\seq{\pi(x_j)}_j) \leq \epsilon/2  < \epsilon. \]
\item For completeness, it is enough, by \ref{completenessCriterion}, to show that Cauchy sequences in $\pi[X]$ converge in $Y$.

Let $\seq{\pi(x_j)}_j$ be a Cauchy sequence in $\pi[X]$, then $\seq{x_i}$ is Cauchy in $X$ because $\pi$ is isometric. So $\seq{\pi(x_j)}_j$ converges to $\seq{x_i}$ by the previous point.
Let $\seq{\pi(x_j)}_j$ be a Cauchy sequence in $\pi[X]$, then $\seq{x_i}$ is Cauchy in $X$ because $\pi$ is isometric. So $\seq{\pi(x_j)}_j$ converges to $\seq{x_i}$ by the previous point.
\end{itemize}
\end{proof}

\begin{proposition}
Let $(X,d)$ be a metric space. The completion $(Y,\pi)$ of $X$ is unique in the following sense: for any other such completion $(Y',\pi')$, there exists a unique isometric isomorphism $\theta:Y\to Y'$ satisfying $\theta\circ \pi = \pi'$.
\end{proposition}
\begin{proof}
Since $\pi$ is an isometry, it is injective, so $\pi^{-1}:\pi[X]\to X$ is a surjective isometry and so $\pi'\circ\pi^{-1}:\pi[X]\to\pi'[X]$ is too. Now we must have $\theta|_{\pi[X]} = \pi'\circ\pi^{-1}:\pi[X]\to\pi'[X]$ 

TODO universal property!!
\end{proof}



\subsection{Equicontinuity}
Cfr uniform limit theorem
\begin{definition}
Let $(X,d_X)$ and $(Y,d_Y)$ be metric spaces and let $\mathcal{F}$ be a family of functions in $(X\to Y)$.
We say
\begin{itemize}
\item $\mathcal{F}$ is an \udef{equicontinuous family} at $x'\in X$ if
\[ \forall \epsilon> 0: \forall x\in X: \exists \delta>0: \forall f\in\mathcal{F}:\; d_X(x,x') < \delta \implies d_Y(f(x),f(x')) < \epsilon; \]
\item $\mathcal{F}$ is a \udef{uniform equicontinuous family} at $x'\in X$ if
\[ \forall \epsilon> 0: \exists \delta>0: \forall x\in X: \forall f\in\mathcal{F}:\; d_X(x,x') < \delta \implies d_Y(f(x),f(x')) < \epsilon. \]
\end{itemize}
We say $\mathcal{F}$ is (uniformly) equicontinuous if it is (uniformly) equicontinuous at all $x'\in X$.
\end{definition}

\begin{itemize}
\item For continuity, $\delta$ may depend on $\epsilon,f,x_0$;
\item For uniform continuity, $\delta$ may depend on $\epsilon$ and $f$;
\item For pointwise equicontinuity, $\delta$ may depend on $\epsilon$ and $x_0$;
\item For uniform equicontinuity, $\delta$ may depend only on $\epsilon$.
\end{itemize}

TODO: generalise to $X$ general topological space (esp for TVSs).

\begin{proposition}
Let $(f_n)_{n\in\N}$ be an equicontinuous family between metric spaces. If $f_n\to f$ pointwise, then $f$ is continuous.
\end{proposition}
\begin{proof}
TODO
\end{proof}

TODO Reed / Simon

\subsection{Hölder and Lipschitz continuity}
\begin{definition}
Let $f: X\to Y$ be a map between metric spaces, then $f$ is called \udef{$\alpha$-Hölder continuous}, where $0 < \alpha \leq 1$, if there exists a constant $M$ such that
\[ d_Y(f(x), f(y)) \leq M d_X(x,y)^\alpha \qquad \forall x,y\in X. \]

If $\alpha = 1$, then $f$ is called \udef{Lipschitz continuous}.

The set of $\alpha$-Hölder continuous functions $X\to Y$ is denoted $\cont^{0,\alpha}(X,Y)$.
\end{definition}
The $0$ in $\cont^{0,\alpha}(X,Y)$ appears because we are not considering any derivatives.

\begin{lemma} \label{HolderLipschitzContinuity}
Let $f: X\to Y$ be a map between metric spaces and $0<\alpha \leq \beta \leq 1$.

If $f$ is $\beta$-Hölder continuous, then it is $\alpha$-Hölder continuous.
\end{lemma}
TODO consolidate lemmas + uniform continuity.
\begin{lemma} \label{LipschitzcontinuousContinuous}
A Lipschitz continuous function between metric spaces is continuous.
\end{lemma}
\begin{proof}
    Let $f$ be a Lipschitz continuous function and $\seq{x_n}$ a sequence such that $x_n \to x$. Then
    \begin{align*}
        d\left(\lim_{n\to\infty}f(x_n), f(x)\right) &= \lim_{n\to\infty}d(f(x_n), f(x)) \\
        &\leq \lim_{n\to\infty}M d(x_n, x) = M d\left(\lim_{n\to\infty}x_n, x\right) = M d(x,x) = 0. 
    \end{align*} 
\end{proof}

The converse in not necessarily true. TODO example.

\subsubsection{Contractions}
\begin{definition}
Let $f: X\to Y$ be a map between metric spaces, then $f$ is called a \udef{contraction} if it is Lipschitz continuous with Lipschitz constant $M < 1$.
\end{definition}

\begin{proposition} \label{contractionFixedPoint}
Let $f: X\to X$ be a contraction. If $X$ is a complete metric space, then $f$ has a unique fixed point.
\end{proposition}
\begin{proof}
Uniqueness is easy: assume that $f$ has two fixed points $x_1, x_2$. Then $d(x_1,x_2) = d(fx_1, fx_2) \leq M d(x_1,x_2)$. Since $M < 1$ this is only possible if $d(x_1,x_2) = 0$, meaning $x_1  x_2$.

For existence: take some $x_0\in X$. Then define the sequence $\seq{T^n(x_0)}_n$. This is a Cauchy sequence because, for $m>n>1$
\[ d(x_m,x_n) \leq \sum_{i=n}^{m-1}d(x_{i+1}, x_i) \leq \sum^{m-1}_{i=n}M^{i-1}d(x_2,x_1) = \frac{M^{n-1}(1-M^{m-n+1})}{1-M}d(x_2,x_1) \leq \frac{M^{n-1}}{1-M}. \]
By completeness it has a limit. Now $T$ is continuous by \ref{LipschitzcontinuousContinuous}. Then
\[ T\left(\lim_{n\to \infty} T^n(x_0)\right) = \lim_{n\to \infty}T(T^n(x_0)) = \lim_{n\to \infty}T^{n+1}(x_0) = \lim_{n\to \infty} T^n(x_0),  \]
so the limit is a fixed point.
\end{proof}

The construction of the sequence $\seq{T^n(x_0)}_n$ is called \udef{fixed point iteration}. Following the proof of the proposition it is clear we can obtain the fixed point starting the iteration from any point in $X$.

\begin{corollary}
Let $f: X\to X$ be a function on a complete metric space such that $f^n$ is a contration for some $n\in \N$. Then $f$ has a unique fixed point.
\end{corollary}
\begin{proof}
By the proposition we know that $f^n$ has a unique fixed point: $f^n(x) = x$. Applying $f$ to both sides gives
\[ f(f^n(x)) = f^n(f(x)) = f(x), \]
so $f(x)$ is also a fixed point of $f^n$. By uniqueness $f(x) = x$. This shows $f$ has a fixed point.

For uniqueness it is enough to note that any fixed point of $f$ is a fixed point of $f^n$, \ref{fixedPointsMultipleComposition}.
\end{proof}

\subsection{Pseudometric spaces}
\begin{definition}
Let $X$ be a set.
\begin{itemize}
\item A map $p: X\times X\to \R_{\geq 0}$ is called a \udef{pseudometric} or \udef{semimetric} if
\begin{itemize}
\item $\forall x\in X: p(x,x) = 0$;
\item $\forall x,y\in X: p(x,y) = p(y,x)$;
\item $\forall x,y,z\in X: p(x,z)\leq p(x,y)+p(y,z)$.
\end{itemize}
Unlike a metric space, points in a pseudometric space need not be distinguishable; that is, one may have $d(x,y)=0$ for distinct values $x\neq y$.
\item The pair $(X,p)$ is called a \udef{pseudometric space}.
\item The \udef{pseudometric topology} is the topology generated by the basis of open balls
\[ B_p(x_0, \epsilon) = \setbuilder{x\in X}{p(x_0,x)<\epsilon}. \]
A topological space is said to be a \udef{pseudometrizable space} if it can be given a pseudometric such that the pseudometric topology coincides with the given topology on the space.
\end{itemize}
\end{definition}

\begin{proposition}
The notions of compactness, limit point compactness
and sequential compactness are equivalent in a pseudometric space.
\end{proposition}

\url{https://link.springer.com/article/10.1007/BF01351999}

\section{Topologies of $\R$}
\begin{definition}
The \udef{lower limit topology} on $\R$ is the topology generated by the basis
\[ \{ [a,b[ \;|\; a< b \}. \]
\end{definition}
\begin{definition}
\begin{itemize}
\item Let $K$ denote the set
\[ K = \{1/n \;|\; n\in \mathbb{N}_0\}. \]
\item The \udef{$K$-topology} on $\R$ is the topology generated by the basis
\[ \{ ]a,b[ \;|\; a< b \}\;\cup\;\{ ]a,b[\setminus K \;|\; a< b \}. \]
\end{itemize}
\end{definition}
\begin{lemma}
The lower limit and $K$-topologies are strictly finer than then standard topology on $\R$, but are not comparable with one another.
\end{lemma}
\begin{proposition}
Let $X$ be a topological space and $f,g:X\to \R$ continuous functions, then
\begin{enumerate}
\item $f+g, f-g$ and $f\cdot g$ are continuous.
\item If $g(x)\neq 0$ for all $x\in X$, then $f/g$ is continuous.
\end{enumerate}
\end{proposition}
\begin{proof}
First define the map
\[ h:X\to \R\times\R: x\mapsto (f(x),g(x)) \]
which is continuous by proposition \ref{continuityCompositeFunctions}. The functions $f+g,f-g,f\cdot g, f/g$ are the composition of $h$ and the continuous functions $+,-,\cdot,/$.
\end{proof}

\section{Uniform topology}
\begin{definition}
Given an index set $J$ and points $\vec{x}=(x_i)_{i\in J}$ and $\vec{y}=(y_i)_{i\in J}$ of $\R^J$. Define the metric $\bar{\rho}$ on $\R^J$ by
\[ \bar{\rho}(\vec{x}, \vec{y}) = \sup\{\bar{d}(x_i,y_i)\;|\; i\in J\}, \]
where $\bar{d}$ is the standard bounded metric on $\R$. The metric space $(\R^J, \bar{\rho})$ is the \udef{uniform topology} on $\R^J$ and $\bar{\rho}$ is the \udef{uniform metric}.
\end{definition}
\begin{proposition}
On the set $\R^J$, the box topology is finer than the uniform topology is finer than the product topology. These topologies are all different \textup{if and only if} $J$ is infinite.
\end{proposition}

\section{Set-theoretic topology}
\url{https://en.wikipedia.org/wiki/Set-theoretic_limit}
\url{https://math.stackexchange.com/questions/3384916/topology-of-set-theoretic-limits}
\url{https://math.stackexchange.com/questions/2799181/is-there-a-way-to-express-the-set-theoretic-limit-in-terms-of-topology-filters}
\url{https://math.stackexchange.com/questions/97440/do-limits-of-sequences-of-sets-come-from-a-topology}
\url{https://math.stackexchange.com/questions/1947171/the-topology-of-sets}

\section{Initial and final topologies}

\chapter{More topological constructions}
\section{Fibre bundles}
\section{Cones and suspensions}
\section{Wedge sum and smash product}

\chapter{Convergence and topology on algebraic structures}
\section{Convergence relational structures}
\section{Convergence order structures}
\subsection{Difference operators}
TODO

\chapter{Convergence groups}
\section{Convergence}
\begin{definition}
Let $G$ be a set and
\begin{itemize}
\item $\boldsymbol{\cdot}: G\times G \to G$ a binary operation such that $\sSet{G, \boldsymbol{\cdot}}$ is a group;
\item $\xi$ a relation on $(\powerset(\powerset(G)), G)$ such that $\sSet{G, \xi}$ is a convergence space;
\end{itemize}
such that
\begin{itemize}
\item $\boldsymbol{\cdot}: G\times G \to G$ is continuous; and
\item $^{-1}: G\to G: x\mapsto x^{-1}$ is continuous.
\end{itemize}
Then we call $\sSet{G, \boldsymbol{\cdot}, \xi}$ a \udef{convergence group}.
\end{definition}

\begin{lemma} \label{convergenceGroupCriterion}
A group $G$ with a convergence structure is a convergence group \textup{if and only if}
\[ G\times G \to G: (x,y) \mapsto xy^{-1} \]
is continuous.
\end{lemma}
\begin{proof}
If $G$ is a convergence group, the function $(x,y) \mapsto xy^{-1}$ is the composition of two continuous functions and thus continuous.

Conversely, assume $(x,y) \mapsto xy^{-1}$ continuous. Then $y \mapsto 1y^{-1} = y^{-1}$ is continuous by \ref{continuousEmbeddingProduct}. Then $\boldsymbol{\cdot}: (x,y) \mapsto xy = x(y^{-1})^{-1}$ is a composition of continuous functions.
\end{proof}

\begin{lemma} \label{closureGroupOperation}
Let $\sSet{G, \boldsymbol{\cdot}, \xi}$ be a convergence group and $A,B$ subsets of $G$. Then
\[ \adh_\xi(A)\cdot \adh_\xi(B) \subseteq \adh_\xi{A\cdot B}. \]
\end{lemma}
\begin{proof}
From the continuity of $\cdot$ and \ref{adherenceInherenceContinuity} together with $\adh_{\xi\otimes\xi}(A\times B) = \adh_\xi(A)\times \adh_\xi(B)$ (TODO ref).
\end{proof}

\begin{lemma} \label{shiftHomeomorphism}
Let $\sSet{G,\cdot, 1, \xi}$ be a convergence group.
\begin{enumerate}
\item For all $a\in G$, both
\[ \lambda_a: G\to G: x\mapsto ax \qquad\text{and}\qquad \rho_a: G\to G: x\mapsto xa \]
are homeomorphisms.
\item $F \to x$ \textup{if and only if} $F\cdot x^{-1} \to 1$ \textup{if and only if} $x^{-1}\cdot F\to 1$.
\item $\vicinity_\xi(x) = \vicinity_\xi(1) \cdot x = x\cdot \vicinity_\xi(1)$.
\item Let $f: \sSet{G,\cdot, 1, \xi} \to \sSet{H,\cdot, 1, \zeta}$ be a group homomorphism. Then $f$ is continuous \textup{if and only if} it is continuous at $1$.
\end{enumerate}
\end{lemma}
\begin{proof}
(1) The functions are clearly bijective. They are also continuous because the constant function $\underline{a}$ is continuous as well as the multiplication $(x,y)\mapsto xy$. Thus the composition is continuous.

(2) Assume $F\to x$, then by continuity of $\rho_{x^{-1}}$, we get $F\cdot x^{-1} \to 1$. The converse follows from continuity of $\rho_x$. The second equivalence follows from the continuity of $\lambda^{x}$ and $\lambda_{x^{-1}}$.

(3) We calculate
\begin{align*}
\vicinity_\xi(x) &= \bigcap \setbuilder{F\in \powerfilters(X)}{x\in \lim_\xi F} \\
&= \bigcap \setbuilder{F\in \powerfilters(X)}{1\in \lim_\xi F\cdot x^{-1}} \\
&= \bigcap \setbuilder{G\cdot x\in \powerfilters(X)}{1\in \lim_\xi G} \\
&= \bigcap \setbuilder{G\in \powerfilters(X)}{1\in \lim_\xi G}\cdot x = \vicinity_\xi(1)\cdot x.
\end{align*}
TODO $\rho_x$ homeomorphism.

(4) If $f$ is continuous, it is automatically continuous at $1$. Now assume $f$ continuous at $1$ and let $x\in G$. Then
\[ F\to x \iff F\cdot x^{-1} \to 1 \implies f[F\cdot x^{-1}] = f[F]\cdot f(x)^{-1} \to f(1) = 1 \iff f[F] \to f(x). \]
\end{proof}

The convergence structure of a convergence group is completely determined by $\lim^{-1}(1)$. Thus the following lemma gives a way to generate convergence groups.

\begin{proposition} \label{groupConvergenceConstruction}
Let $\sSet{G, +, 0}$ be a commutative group. And $\mathcal{G} \subseteq \powerfilters(G)$ a family of filters. There exists a convergence $\xi$ on $G$ such that $\mathcal{G} = \lim^{-1}_\xi(0)$ \textup{if and only if}
\begin{enumerate}
\item $\pfilter{0} \in \mathcal{G}$;
\item if $F \in \mathcal{G}$ and $G\supseteq F$, then $G\in \mathcal{G}$;
\item if $F,G\in \mathcal{G}$, then $F - G\in \mathcal{G}$.
\end{enumerate}
\end{proposition}
The group convergence is completely determined by $\lim^{-1}_\xi(0)$ due to the translation homeomorphism \ref{shiftHomeomorphism}.
\begin{proof}
It is clear that $F \to x$ iff $F - x \in \mathcal{G}$ determines a convergence. We just need to show that $u: (x,y) \mapsto x - y$ is continuous.

Let $F \to (x,y) \in G\times G$, so by \ref{convergenceProductFilter} there exist $F_1\to x$ and $F_2 \to y$ such that $F_1\otimes F_2 \leq F$. Then $F_1 - x \in \mathcal{G}$ and $F_2 - y \in \mathcal{G}$. From point (3) (and commutativity) we get $F_1 - F_2 - (x - y) \in \mathcal{G}$, so $F_1 - F_2 \to x-y$. Now $F_1 - F_2 = u[F_1\otimes F_2] \leq u[F]$ by \ref{filterFactorisationInequality}, so $u[F] \to x-y$ and thus $u$ is continuous.
\end{proof}

\begin{proposition} \label{HausdorffCriterionConvergenceGroup}
Let $\sSet{G,\cdot, 1, \xi}$ be a convergence group. Then $G$ is Hausdorff \textup{if and only if} $\{1\}$ is closed.
\end{proposition}
\begin{proof}
The direction $\Rightarrow$ is clear, since every Hausdorff convergence is $T_1$ and in a $T_1$ convergence all singletons are closed.

Conversely, assume $F\to x,y$ in $G$. Then $FF^{-1} \to xy^{-1}$. Now $FF^{-1} \subseteq \pfilter{1}$, so $\pfilter{1} \to xy^{-1}$ and thus $xy^{-1}\in \adh_\xi(\pfilter{1}) = \adh_\xi(\{1\}) = \{1\}$, meaning $x = y$.
\end{proof}

\begin{lemma} \label{vicinityFactorisation}
Let $\sSet{G,\cdot, 1, \xi}$ be a pretopological convergence group and $x,y\in G$. If $U\in \vicinity_\xi(xy)$, then there exist $V\in \vicinity_\xi(x)$ and $W\in \vicinity_\xi(y)$ such that $V\cdot W\subseteq U$.

If $x=y$, then we can take $V = W$.
\end{lemma}
\begin{proof}
Consider the function $f: G\times G \to G: (x,y)\mapsto xy$. Then by \ref{pretopologicalContinuityVicinities} and \ref{productVicinity} we have
\[ \vicinity_\xi(xy) \subseteq \upset f[\vicinity_{\xi\otimes\xi}((x,y))] = \upset f[\upset \vicinity_\xi(x)\otimes \vicinity_\xi(y)] = \upset f[\vicinity_\xi(x)\otimes \vicinity_\xi(y)] = \upset (\vicinity_\xi(x)\cdot \vicinity_\xi(y)) \]
This implies the first result.

If $x=1=y$, then consider $V\cap W$. This is still a neighbourhood of $x=y$ and $(V\cap W)\cdot(V\cap W) \subseteq V\cdot W \subseteq U$.
\end{proof}

\begin{lemma} \label{symmetricBase}
Let $\sSet{G,\cdot, 1, \xi}$ be a pretopological convergence group, then $\vicinity_\xi(1)$ is based in the symmetric subsets.
\end{lemma}
Symmetric subsets are subsets $V$ such that $V^{-1} = V$.
\begin{proof}
Because $^{-1}$ is a group homeomorphism, we have $\vicinity_\xi(1) = \vicinity_\xi(1^{-1}) = (\vicinity_\xi(1))^{-1}$. So $V$ is a vicinity of $1$ iff $V^{-1}$ is a vicinity of $1$. Thus $V\cap V^{-1}\subseteq V$ is a vicinity of $1$ and $V\cap V^{-1}$ is also a symmetric set.
\end{proof}

\begin{proposition} \label{pretopologicalGroupConvergence}
Each pretopological convergence group is topological.
\end{proposition}
\begin{proof}
Let $\sSet{G,\cdot, 1, \xi}$ be a pretopological convergence group. To prove the convergence it topological, it is enough to prove that $\inh_\xi(A) \subseteq \inh^2_\xi(A)$ for all $A\subseteq X$. Fix such an $A$ and take an arbitrary $x\in \inh_\xi(A)$. Then $A\in \vicinity_\xi(x)$ by \ref{principalAdherenceInherence}.

By \ref{vicinityFactorisation} there exist $V\in \vicinity_\xi(x)$ and $W\in\vicinity_\xi(1)$ such that $V\cdot W \subseteq A$.

Now for all $y\in V$, $y\cdot W$ is a vicinity of $y$ by \ref{shiftHomeomorphism}, so $V \subseteq \inh_\xi(A)$ by \ref{subsetWithVicinitiesInInherence}. By the upward closure of the vicinity filter, $V\in \vicinity_\xi(x)$ implies $\inh_\xi(A) \in \vicinity_\xi(x)$. Thus $x\in \inh_\xi^2(A)$ by \ref{principalAdherenceInherence}.
\end{proof}

\begin{proposition}
Every topological group is regular.
\end{proposition}
\begin{proof}
By \ref{topologicalRegularity} we check that for all open $U$ and $x\in U$ there exists an open set $V$ such that $x\in V\subseteq \overline{V}\subseteq U$. In fact it is enough to check this for $e = 1$.\

Because $1\cdot 1 = 1$, we can find $W\in\neighbourhood(1)$ such that $W\cdot W \subseteq U$ by \ref{vicinityFactorisation}. We claim $V= W\cap W^{-1}$ works. Indeed it is an open neighbourhood of $1$ and clearly $V\subseteq U$. We just need to show that $\overline{V}\subseteq U$. Take $y\in \overline{V}$. Then $yV\in \vicinity_\xi(y)$ by \ref{shiftHomeomorphism} and $V\in \vicinity_\xi(y)^{\mesh}$ by \ref{principalAdherenceInherence}. So $yV\mesh V$ and we can find $v_1,v_2\in V$ such that $v_1 = yv_2$. Thus
\[ y = v_1v_2^{-1} \in V\cdot V^{-1} = V\cdot V \subseteq U. \]
\end{proof}
TODO: in fact completely regular.

\begin{proposition}
Let $G$ be a convergence group. Then the pseudotopological modification $\chi(G)$ is a convergence
group.
\end{proposition}
(This is in general not true for the pretopological modification).
\begin{proposition} \label{initialConvergenceGroup}
Let $G$ be a group, $\{G_i\}_{i\in I}$ a set of convergence groups and $\{f_i: G \to G_i\}_{i\in I}$ a set of group homomorphisms. Then the initial convergence on $G$ w.r.t. $\{f_i: G \to G_i\}_{i\in I}$ makes $G$ a convergence group.
\end{proposition}
\begin{proof}
By \ref{convergenceGroupCriterion} we just need to verify that $u: G\times G \to G: (x,y)\mapsto x\cdot y^{-1}$ is continuous. Using \ref{characteristicPropertyInitialFinalConvergence}, we need to verify that $f_i\circ u$ is continuous for all $i\in I$. Because the $f_i$ are group homomorphisms, we have
\[ f_i(x\cdot y^{-1}) = f_i(x)\cdot f_i(y)^{-1} \]
for all $x, y \in G$. This means that $f_i\circ u = u_i \circ (f_i\times f_i)$, where $u_i: G_i\times G_i \to G_i: (x,y)\mapsto x\cdot y^{-1}$.

By \ref{productContinuousFunctions} this is a composition of continuous functions and thus continuous.
\end{proof}

\begin{proposition}
Let $G$ be a convergence group. Any subgroup $H\in\vicinity(1)$ is closed.
\end{proposition}
\begin{proof}
Take $g\in \adh(H)$, so $H\in \vicinity(g)^{\mesh}$ by \ref{principalAdherenceInherence}. Now, by \ref{homeomorphismPreservation}, $gH\in \vicinity(g)$ because $\lambda_g$ is a homeomorphism by \ref{shiftHomeomorphism}. Thus $gH \mesh H = 1\cdot H$. As cosets are either the same or disjoint (by \ref{differentCosetsDisjoint}), we have $gH = H$ and in particular $g = g\cdot 1 \in gH = H$. So $\adh(H) = H$.
\end{proof}

\begin{proposition}
Let $G$ be a convergence group and $A,B\subseteq G$ subsets.
\begin{enumerate}
\item If $A$,$B$ are compact, then $A\cdot B$ is compact.
\item If $A$ is open, then $A\cdot B$ is open.
\item If $A$ is closed and $B$ is compact, then $A\cdot B$ is closed.
\end{enumerate}
\end{proposition}
\begin{proof}
TODO

(1) Continuous image of compact is compact.

(2) $A\cdot B = \bigcup_{b\in B} A\cdot b$ is a union of open sets and thus open.

(3)
\end{proof}

\subsection{Locally compact groups}
\begin{lemma}
A topological convergence group is locally compact \textup{if and only if} $\vicinity(1)$ has a base of compact sets.
\end{lemma}
\begin{proof}
TODO 

It is regular, so is based in closed sets. There is a compact vicinity of $1$. All closed subsets of this vicinity are compact.
\end{proof}

\section{Cauchy structure}
\begin{proposition} \label{groupCauchyStructure}
Let $\sSet{G, \cdot, 1, \xi}$ be a convergence group. Consider the family
\[ \mathcal{F} \defeq \setbuilder{F\in \powerfilters(G)}{F\cdot F^{-1} \overset{\xi}{\longrightarrow} 1}. \]
Then
\begin{enumerate}
\item $\sSet{G, \mathcal{F}}$ is a Cauchy space;
\item if $F\in \powerfilters(G)$ Cauchy converges to $x$, then it converges to $x$;
\item if $F\in \powerfilters(G)$ converges to $x$, then $F\in \mathcal{F}$; if $G$ is a Kent space, then $F$ Cauchy converges to $x$.
\end{enumerate}
\end{proposition}
\begin{proof}
(1) We have $\pfilter{x}\cdot \pfilter{x}^{-1} \supseteq \pfilter{1}$, so $\pfilter{x}\cdot \pfilter{x}^{-1} \to 1$ and thus $\pfilter{x} \in \mathcal{F}$.

Take $F\in \mathcal{F}$ and consider $H\supseteq F$. Clearly $F\cdot F^{-1} \subseteq H\cdot H^{-1}$, so $H\cdot H^{-1} \to 1$ and thus $H\in \mathcal{F}$.

(2) Notice that
\[ (F\cap \pfilter{x})\cdot(F\cap \pfilter{x})^{-1} = F\cdot F^{-1} \cap F\cdot \pfilter{x}^{-1} \cap \pfilter{x}\cdot F^{-1} \cap \pfilter{x}\cdot\pfilter{x}^{-1}. \]
So if $F$ Cauchy converges to $x$, then $F\cap\pfilter{x} \in \mathcal{F}$ and this filter converges to $1$. In particular this means that $F\cdot\pfilter{x}^{-1} \to 1$. Then we use $F\cdot\pfilter{x}^{-1} = F\cdot x^{-1}$ (TODO ref?) to conclude that $F \to x$.

(3) Assume $F\to x$. Then by \ref{convergenceFiniteProductFilter}, $F\otimes F \to (x,x)$. By continuity of $(x,y)\mapsto x\cdot y^{-1}$, we have $F\cdot F^{-1}\to x\cdot x^{-1} = 1$, so $F$ is a Cauchy filter.

Now assume $G$ is a Kent space. Then $F\cap \pfilter{x} \to x$, meaning $F\cap \pfilter{x} \in \mathcal{F}$ by the previous argument and so $F$ Cauchy converges to $x$.
\end{proof}


\begin{definition}
Let $\sSet{G, \cdot, 1, \xi}$ be a convergence group. The family $\mathcal{F}$ of \ref{groupCauchyStructure} is the \udef{associated Cauchy structure} of the convergence group.
\end{definition}

\begin{proposition}
Every locally compact convergence group of finite depth is complete.
\end{proposition}
TODO: finite depth necessary?
\begin{proof}
Let $\sSet{G, \cdot, 1, \xi}$ be a locally compact convergence group and let $F$ be a proper Cauchy filter. Then $F\cdot F^{-1} \to 1$, so there exists a compact set $K\in F\cdot F^{-1}$. This means there exist $A, B \in F$ such that $A\cdot B^{-1} \subseteq K$. Take $x_0 \in B$, then $F_0 \subseteq K\cdot x_0$ and $K\cdot x_0$ is compact by TODO ref. Now by the ultrafilter lemma \ref{ultrafilterLemma} we can take an ultrafilter $G \supseteq F$. Then $K\cdot x_0 \in G$, so $G$ converges to some $y$.

Then $G \cap \pfilter{y}$ is a Cauchy filter. By finite depth, $F\cap G \cap \pfilter{y} = F\cap \pfilter{y}$ is also a Cauchy filter, so $F$ Cauchy converges to $y$.
\end{proof}

\section{Metrisability and norm}

\begin{theorem}[Birkhoff-Kakutani]
Let $\sSet{G,\cdot, 1, \xi}$ be a convergence group. Then the following are equivalent:
\begin{enumerate}
\item $G$ is pseudometrisable;
\item the topology on $G$ is induced by a left translation invariant pseudometric;
\item the topology on $G$ is induced by a right translation invariant pseudometric;
\item $G$ is first countable and topological.
\end{enumerate}
\end{theorem}

\begin{definition}
Let $\sSet{G, \cdot, 1}$ be a group. We call a function $\norm{\cdot}: G\to \R^+$ a \udef{group norm} on $G$ if $\forall x,y\in G$ the following hold
\begin{enumerate}
\item triangle inequality / subadditivity: $\norm{xy} \leq \norm{x} + \norm{y}$;
\item positivity $\norm{x} > 0$ if $x \neq 1$;
\item inversion $\norm{x^{-1}} = \norm{x}$.
\end{enumerate}
We call the norm \udef{cyclically permutable} if $\forall x,y\in G$: $\norm{xy} = \norm{yx}$.

If $\norm{\cdot}$ is a group norm for $G$, then we call $\sSet{G, \cdot, 1, \norm{\cdot}}$ a \udef{normed group}.
\end{definition}
