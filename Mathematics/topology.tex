\chapter{Topological spaces}
A topology specifies which points are close to each other and which are not. This is useful for determining continuity and the existence of holes for example.

Obviously one way to get an idea of which points are close to other points is by explicitly supplying a notion of distance. In fact this a particular type of topological space called a metric space. This way of describing the topology will turn out to be too restrictive, however.

Another way of describing topology is saying that it is concerned with the properties of a geometric object that are preserved under continuous deformations, such as stretching, twisting, crumpling and bending, but not tearing or gluing. Those continuous deformations do not change \textit{which} points are close to each other, just exactly how close they are. Tearing separates points that were close and gluing makes points that were not in each others neighbourhoods suddenly neigbours.

A famous example of such a continuous deformation is the deformation between a doughnut and a coffee cup. Because a topologist is only interested in properties that are preserved under such a transformation, the joke goes that for him as doughnut and a mug is the same thing.

All this can be achieved by defining which subsets of the space are \textit{neighbourhoods} of each point. A neighbourhood of a point is an open set around that set and can be thought of as a sort of generalisation of the open intervals on the real line. TODO motivate definition.

\url{https://en.wikipedia.org/wiki/List_of_topologies}

\section{Axiomatisations and basic concepts}
TODO Motivation.
\subsection{Building blocks}
The basic building blocks of topology are neighbourhoods, open sets, closed sets, interior, closure, boundaries, limit points, convergence and nearness. Each of these concepts can be axiomatised and given any one, the others are uniquely fixed.

We first describe how these concepts are related, and then give each axiomatisation and show they are equivalent.

\url{https://en.wikipedia.org/wiki/Characterizations_of_the_category_of_topological_spaces} 
\url{https://mathoverflow.net/questions/19152/why-is-a-topology-made-up-of-open-sets/19173#19173}

\subsection{Neighbourhoods, open sets, closed sets}
TODO: function spaces: closed sets point-wise topology >< uniform topology (bounded by functions v bounded by horizontals)

Let $X$ be a set.

Neighbourhoods: sets that ``completely surround'' $x$.

For every $x\in X$ we specify which subsets of $X$ are neighbourhoods of $x$. This family of sets is denoted $\mathcal{N}(x)$. We would like the following to hold:
\begin{enumerate}
\item Every neighbourhood of $x$ must contain $x$.
\item Every set that contains a neighbourhood of $x$ is a neighbourhood of $x$.
\item The intersection of two neighbourhoods is again a neighbourhood.
\item The point $x$ must in some sense be in the interior of each of its neighbourhoods. TODO: \url{https://math.stackexchange.com/questions/2692678/why-does-the-definition-of-a-topology-via-neighborhoods-include-this-axiom}
\end{enumerate}

\begin{proposition}
Let $(X,\mathcal{T})$ be a topological space and $x\in X$. Then
\begin{enumerate}
\item if $N\in\mathcal{N}(x)$, then $x\in N$;
\item if $M\subset X$ and there exists $N\in\mathcal{N}(x)$ such that $N\subset M$, then $M\in\mathcal{N}(x)$;
\item if $M,N\in\mathcal{N}(x)$, then $M\cap N\in\mathcal{N}(x)$;
\item $\forall N\in\mathcal{N}(x): \exists M\in\mathcal{N}(x): \forall y\in M: N\in\mathcal{N}(y)$.
\end{enumerate}
Conversely, every function $X\to \mathcal{P}(X)$ that satisfies these properties is the neighbourhood topology for some topology on $X$.
\end{proposition}
The fourth point is the least obvious. It essentially says that if $y$ is sufficiently close to $x$ (i.e. $y\in M(x)$), then $x$ is also close to $y$.
\begin{proof}
TODO
\end{proof}

\begin{definition}
A \udef{topology} on a set $X$ is a collection $\mathcal{T}\in \mathcal{P}(X)$ of subsets of $X$ having the following properties:
\begin{enumerate}
\item Both $\emptyset$ and $X$ are in $\mathcal{T}$.
\item The union of the elements of any subcollection of $\mathcal{T}$ is in $\mathcal{T}$.
\item The intersection of the elements on any finite subcollection of $\mathcal{T}$ is in $\mathcal{T}$.
\end{enumerate}
A set $X$ for which a topology $\mathcal{T}$ has been specified is called a \udef{topological space}.
\end{definition}
These axioms formalise an idea of open subset.
\begin{definition}
We call a subset $U$ of $T$ an \udef{open set} if $U$ is in $\mathcal{T}$. A \udef{closed set} is any set that can be constructed as $X \setminus U$ for some open set $U$. In a given topology a set may be open, closed, both or neither. If a set is both open and closed, it is called \udef{clopen}.

We say $U$ is an \udef{open neighbourhood} of $x$ if $U$ is an open set containing $x$. This is sometimes denoted $U(x)$.

A \udef{neighbourhood} of $x$ is a set containing an open neighbourhood of $x$. We denote by $\mathcal{N}(x)$ the set of neighbourhoods of $x$. The function $\mathcal{N}$ is called the \udef{neighbourhood topology}.
\end{definition}

\begin{lemma}
Let $X$ be a topological space. Every open set $O$ can be written as $X\setminus K$ for some closed $K$.
\end{lemma}
\begin{proof}
Lemma \ref{lemma:BooleanConsequences}.
\end{proof}

\begin{example}
\begin{itemize}
\item Let $X$ be a three-element set, $X = \{a,b,c\}$. There are many possible topologies on $X$, to name a few:
\begin{itemize}
\item $\left\{\emptyset, X\right\}$
\item $\left\{\emptyset, \{a\}, \{a,b\}, X\right\}$
\item $\left\{\emptyset, \{a\}, X\right\}$
\item $\left\{\emptyset, \{a,b\}, X\right\}$
\item $\left\{\emptyset, \{a,b\}, \{a,c\}, \{b\}, X\right\}$
\item $\left\{\emptyset, \{a,b\}, \{c\}, X\right\}$
\item $\left\{\emptyset, \{a\}, \{b\}, \{a,b\}, X\right\}$
\item $\ldots$
\end{itemize}
\item For any set $X$, the collection of all subsets of $X$ is a topology, called the \udef{discrete topology}.
\item For any set $X$, the topology $\mathcal{T} = \left\{\emptyset, X\right\}$ is called the \udef{trivial} topology.
\item In any topology, both $X$ and $\emptyset$ are both open and closed.
\item Let $X$ be a set. Let $\mathcal{T}_f$ be a collection of all subsets $U$ of $X$ such that $X\setminus U$ is finite or $U=\emptyset$. Then $\mathcal{T}_f$ is a topology on $X$ called the \udef{finite complement topology}.
\end{itemize}
\end{example}

We can also characterise the topology with closed sets or with neighbourhoods:
\begin{proposition}
Let $(X,\mathcal{T})$ be a topological space. Let $\mathcal{T}_c$ be the family of closed subsets of $X$. Then
\begin{enumerate}
\item Both $\emptyset$ and $X$ are in $\mathcal{T}_c$.
\item Let $\mathcal{E}$ be a subset of $\mathcal{T}_c$. Then $\bigcap\mathcal{E}\in\mathcal{T}_c$.
\item If $A,B\in \mathcal{T}_c$, then $A\cup B\in \mathcal{T}_c$.
\end{enumerate}
Conversely, given any set $X$ and any family $\mathcal{T}_c\subset\mathcal{P}(X)$ that satisfies these propserties, the family
\[ \mathcal{T} = \setbuilder{O\subset X}{X\setminus O\in\mathcal{T}_c} \]
is a topology on $X$. 
\end{proposition}


Obviously a set can have different topologies.
\begin{definition}
Sometimes we can compare them. Two topologies $\mathcal{T}$ and $\mathcal{T}'$ are \udef{comparable} if either $\mathcal{T} \subseteq \mathcal{T}'$ or $\mathcal{T} \supseteq \mathcal{T}'$.
\begin{itemize}[leftmargin=2cm]
\item[\boxed{\mathcal{T} \subseteq \mathcal{T}'}] In topology $\mathcal{T}'$ there are more open sets. This allows more granular specification of neighbourhoods. We say $\mathcal{T}'$ is \udef{finer} than $\mathcal{T}$. If $\mathcal{T}'$ is a proper superset, we say it is \udef{strictly finer} than $\mathcal{T}$.
\item[\boxed{\mathcal{T} \supseteq \mathcal{T}'}] In this case $\mathcal{T}'$ is \udef{(strictly) coarser} than $\mathcal{T}$.
\end{itemize}
\end{definition}

\subsection{Closure and interior of a set}
\url{https://en.wikipedia.org/wiki/Kuratowski_closure_axioms}

\begin{definition}
Given any subset $A$ of a topological space $X$,
\begin{itemize}
\item The \udef{interior} of $A$, denoted $A^\circ$, is the union of all open sets contained in $A$;
\item The \udef{closure} of $A$, denoted $\bar{A}$, is the intersection of all closed sets containing $A$. 
\end{itemize}
The \udef{boundary} of $A$ is $\partial A \defeq \bar{A}\setminus A^{\circ}$.
\end{definition}
We immediately have the inclusions
\[ A^\circ \subset A \subset \bar{A} \]

\begin{lemma}
The interior and closure are dual in the sense that
\[ A^\circ = X\setminus\overline{(X\setminus A)} = \overline{(A^c)}^c \qquad \bar{A} = X\setminus(X\setminus A)^\circ = ((A^c)^\circ)^c \]
where $X$ is a topological space and $A$ is a subset.
\end{lemma}
\begin{proposition}\label{prop:closure}
Let $A$ be a subset of the topological space $X$, then
\[ x\in \bar{A} \qquad \text{\textup{if and only if}}\qquad \text{every open set $U$ containing $x$ intersects $A$}.\]
\end{proposition}
\begin{proof}
We prove the contrapositive.
\[ x\notin \bar{A} \iff \text{there exists an open set $U$ containing $x$ that does not intersect $A$.} \]
\begin{itemize}
\item[$\boxed{\Rightarrow}$] The set $U = X\setminus \bar{A}$ is an open set containing $x$ that does not intersect $A$.
\item[$\boxed{\Leftarrow}$] If there exists such a $U$, then $X\setminus U$ is a closed set containing $A$, so $X\setminus U \supset \bar{A}$. Therefore $x$ cannot be in $\bar{A}$.
\end{itemize}
\end{proof}
\begin{proposition}\label{prop:interior}
Let $A$ be a subset of the topological space $X$, then
\[ x\in A^\circ \qquad \text{\textup{if and only if}}\qquad \text{there exists an open set $U$ such that $x\in U \subset A$}.\]
\end{proposition}
\begin{proof}
The interior is the union of all open sets $U\subset A$. Thus if $x\in A^\circ$, then $x$ is in such a $U$.
\end{proof}
\begin{lemma}
Given any subset $A$ of $X$,
\begin{itemize}
\item $\bar{A}$ is the smallest closed set containing $A$;
\item $A^\circ$ is the largest open set contained in $A$.
\end{itemize}
Consequently the closure and interior are idempotent:
\[ \overline{\bar{A}} = \bar{A} \qquad \text{and} \qquad (A^\circ)^\circ = A^\circ. \]
\end{lemma}

\begin{lemma}
Let $A,B$ be subsets of a topological space $X$. Then
\begin{enumerate}
\item $\overline{A\cup B} = \overline{A}\cup \overline{B}$;
\item $(A\cap B)^\circ = A^\circ \cap B^\circ$.
\end{enumerate}
These properties do not hold for arbitrary unions and intersections.
\end{lemma}
\begin{proof}
TODO
\end{proof}
\begin{lemma}
TODO: Intersection of Interiors contains Interior of Intersection and Closure of Union contains Union of Closure and Closure of Intersection is Subset of Intersection of Closures and Union of Interiors is Subset of Interior of Union
\end{lemma}

\begin{lemma} \label{lemma:closureInteriorSubsets}
Let $A\subseteq B$ be sets in a topological space $X$. Then
\begin{enumerate}
\item $\overline{A} \subseteq \overline{B}$;
\item $A^\circ \subseteq B^\circ$.
\end{enumerate}
\end{lemma}

\subsection{Limit points}
\begin{definition}
If $A$ is a subset of the topological space $X$ and if $x$ is a point of $X$ (not necessarily of $A$), we say $x$ is a \udef{limit point} (also sometimes called \udef{cluster point} or \udef{point of accumulation}) of $A$ if every (open) neighbourhood of $x$ intersects $A$ in some point other than $x$ itself.

The set $A'$ of all limit points of $A$ is called the \udef{derived set} of $A$.

An \udef{isolated point} of $A$ is a point $x\in A$ that is not an accumulation point for $A$.
\end{definition}
So $x$ is a limit point of $A$ if it belongs to the closure of $A\setminus \{x\}$.
\begin{example}
Consider $\R$. If $A= ]0,1]$, then the point $0$ is a limit point of $A$. In fact every point in $[0,1]$ is a limit point and no other points of $\R$ are limit points.
\end{example}
This motivates the following assertion:
\begin{proposition}
Let $A$ be a subset of a topological space $X$. Then
\[ \bar{A} = A \cup A' = A^\circ \cup A' \]
where $A'$ is the derived set of $A$.
\end{proposition}
\begin{corollary}
A topological space is closed if and only if it contains all its limit points.
\end{corollary}

\begin{definition}
Let $(X,\mathcal{T})$ be a topological space. A subset $A\subset X$ is \udef{perfect} in $X$ if it is closed and every point of $A$ is an accumulation point of $A$.
\end{definition}
\begin{lemma}
If $A$ has no isolated points, then $\overline{A}$ is perfect in $X$.
\end{lemma}

\subsection{Special subsets}
\begin{definition}
Let $(X,\mathcal{T})$ be a topological space. A set $A\subset X$ is called
\begin{itemize}
\item a \udef{$\mathcal{G}_\delta$-set} if it is a countable intersection of open sets;
\item an \udef{$\mathcal{F}_\sigma$-set} if it is a countable union of closed sets.
\end{itemize}
\end{definition}

\section{Topologies}
\subsection{The basis of a topology}
For many topologies specifying \textit{all} the open sets can be challenging. In this section we give a way to specify a smaller collection of subsets of $X$, called a basis, and generate the topology in terms of that.

\begin{definition}
If $X$ is a set, a \udef{basis} is a subset $\mathcal{B}$ of the powerset of $X$ such that
\begin{enumerate}
\item For each $x\in X$, there is at least one basis element $B\in\mathcal{B}$ containing $x$.
\item If $x$ belongs to the intersection of two basis elements $B_1$ and $B_2$, then there is a basis element $B_3\in\mathcal{B}$ containing $x$ such that $B_3\subset B_1 \cap B_2$.
\end{enumerate}
The \udef{topology $\mathcal{T}$ generated by $\mathcal{B}$} is defined as follows: A subset $U$ of $X$ is said to be open in $X$ if, for each $x\in U$, there is a basis element $B\in\mathcal{B}$ such that $x\in B$ and $B\subset U$. 
\end{definition}
Each basis element is itself an open set. It is not too difficult to check that $\mathcal{T}$ is indeed a topology.

\begin{example}
\begin{itemize}
\item For any set $X$, the collection of all one-point subsets of $X$ is a basis for the discrete topology.
\item The collection of all open intervals on the real line is a basis for the \udef{standard topology} on the real line $\R$.
\end{itemize}
\end{example}

There is an easier way to obtain the topology $\mathcal{T}$ from a basis $\mathcal{B}$:
\begin{lemma}
The topology $\mathcal{T}$ equals the collection of all unions of elements in $\mathcal{B}$.
\end{lemma}

Here is a way to obtain a basis from a topology on $X$.
\begin{lemma}
Suppose that $\mathcal{C}$ is a collection of open sets of $X$ such that for each open set $U$ of $X$ and each $x \in U$, there is an element $C$ of $\mathcal{C}$ such that $x\in C \subset U$. Then $\mathcal{C}$ is a basis for the topology of $X$.
\end{lemma}

We can link the basis to the coarseness of the topology.
\begin{lemma} \label{lemma:basisCoarseness}
Let $\mathcal{B}$ and $\mathcal{B}'$ be bases for the topologies $\mathcal{T}$ and $\mathcal{T}'$, respectively, on $X$. The following are equivalent:
\begin{enumerate}
\item $\mathcal{T}'$ is finer than $\mathcal{T}$.
\item For each $x\in X$ and each basis element $B\in\mathcal{B}$ containing $x$, there is a basis element $B'\in\mathcal{B'}$ such that $x\in B'\subset B$.
\end{enumerate}
\end{lemma}

Closures of sets can also be described using a basis.
\begin{lemma}
Let $A$ be a subset of $X$ which has a topology generated by a basis $\mathcal{B}$, then $x\in\bar{A}$ if and only if every basis element $B\in\mathcal{B}$ containing $x$ intersects $A$.
\end{lemma}

\subsubsection{Subbasis}
\begin{definition}
If $X$ is a set, a \udef{subbasis} is a subset $\mathcal{S}$ of the powerset of $X$ such that $X = \bigcup \mathcal{S}$.

The \udef{topology $\mathcal{T}$ generated by $\mathcal{S}$} is the collection of all unions of finite intersections of elements of $\mathcal{S}$. 
\end{definition}
The topology $\mathcal{T}$ is exactly the coarsest topology that makes all sets in the subbasis open.

\subsection{The subspace topology}
\begin{definition}
Let $X$ be a topological space with topology $\mathcal{T}$. Let $Y$ be a subspace of $X$. The collection
\[ \mathcal{T}_Y = \{ Y\cap U\;|\; U\in \mathcal{T} \} \]
is a topology on $Y$ called the \udef{subspace topology}. With this topology, $Y$ is called a \udef{subspace} of $X$.
\end{definition}
\begin{lemma}
Let $Y$ be a subspace of $X$. A set $A$ is closed in $Y$ \textup{if and only if} it equals the intersection of a closed set of $X$ with $Y$.
\end{lemma}

\begin{lemma}
If $\mathcal{B}$ is a basis for the topology of $X$, then
\[\mathcal{B}_Y = \{ B\cap Y \;|\; B\in \mathcal{B} \}\]
is a basis for the subspace topology on $Y$.
\end{lemma}

\begin{lemma}
Let $Y$ be a subspace of $X$.
\begin{enumerate}
\item If $A$ is open in $Y$ and $Y$ is open in $X$, then $A$ is open in $X$.
\item If $A$ is closed in $Y$ and $Y$ is closed in $X$, then $A$ is closed in $X$.
\end{enumerate}
\end{lemma}
 
We reserve the notation $\bar{A}$ to stand for the closure of $A$ in $X$, not $Y$.
\begin{lemma}
Let $A$ be a subset of $Y$, then the closure of $A$ in $Y$ equals $\bar{A}\cap Y$.
\end{lemma}

\begin{lemma} \label{lemma:notLimitPointSingletonOpen}
Let $A\subseteq X$ be a subspace of $X$. Then $a\in A\setminus A'$, then $\{a\}$ is open in $A$.
\end{lemma}
\begin{proof}
Assume such an $a$. Then there exists an open neighbourhood $U$ of $a$ in $X$ that does not intersect $A$ in any other point. By definition of the subspace topology $\{a\}$ is open.
\end{proof}

\subsection{Topology and order}
\begin{definition}

\end{definition}

\url{https://planetmath.org/orderedspace}
\url{https://www.jstor.org/stable/2032122?seq=2#metadata_info_tab_contents}
\url{http://www.math.wm.edu/~lutzer/drafts/PragueSurveyFinal.pdf}
\url{https://ncatlab.org/nlab/show/pospace}
\url{file:///C:/Users/user/Downloads/order-topological-lattices.pdf}

\subsubsection{Specialisation preorder}
\begin{definition}
Let $(X,\mathcal{T})$ be a topological space and $x,y\in X$. We say $x$
\end{definition}

\paragraph{Alexandrov topology}
\url{https://planetmath.org/inducedalexandrofftopologyonaposet}
\url{https://arxiv.org/pdf/0708.2136.pdf}
\url{https://ncatlab.org/nlab/show/specialization+topology}
\url{http://math.uchicago.edu/~may/REU2018/REUPapers/Asness.pdf}

\subsubsection{Order topology on totally ordered sets}
\begin{definition}
Let $(X,\leq)$ be a linearly ordered set. Let $\mathcal{B}$ be the collection of all sets of the following type:
\begin{enumerate}
\item All open intervals $]a,b[$ in $X$;
\item All intervals of the form $[a_0, b[$, where $a_0$ is the smallest element (if any) of $X$;
\item All intervals of the form $]a, b_0]$, where $b_0$ is the largest element (if any) of $X$;
\end{enumerate}
The collection $\mathcal{B}$ is a basis for a topology, called the \udef{order topology}.
\end{definition}

\paragraph{Product of linearly ordered topology}



\section{Functions on topological spaces}
\subsection{Open and closed maps}
TODO
\subsection{Continuity and continuous functions}
Intuitively, a continuous map is a map between topological spaces that does not make jumps. In particular let $f: X\to Y$ be a potentially continuous function. Say we want to stay in a neighbourhood $V(f(x_0))$, then we want there to be a neighbourhood $U(x_0)$ such that points inside $U(x_0)$ map to points in $V$, i.e.
\[ x\in U(x_0) \implies f(x) \in V(f(x_0)). \]
That is $f(U(x_0)) \subset V$, or $U(x_0)\subset f^{-1}(V)$. So we conclude that for any point $x_0$ and neighbourhood $V(f(x_0))$ in $Y$, $f^{-1}(V)$ must contain a neighbourhood of $x_0$. Thus $f^{-1}(V)$ can be written as a union of open sets, $\bigcup_{x\in f^{-1(V)}}U(x)$, and therefore must be open. This motivates the definition:
\begin{definition}
Let $X,Y$ be topological spaces.
\begin{itemize}
\item A function $f:X\to Y$ is \udef{continuous} if for each open set $V$ of $Y$, $f^{-1}[V]$ is an open subset of $X$.
\item The function $f$ is \udef{continuous at $x_0$} if for each open neighbourhood $V$ of $f(x_0)$, their is an open neighbourhood $U$ of $x_0$ such that $f[U]\subset V$.
\end{itemize}
If a function is not continuous, it is \udef{discontinuous}.
\end{definition}
\begin{lemma} \label{prop:globalContinuityFromAllPoints}
A function $f:X\to Y$ is continuous \textup{if and only if} it is continuous at every point.
\end{lemma}

\begin{lemma} \label{lemma:continuityAtIsolatedPoint}
Let $f:X\to Y$ be a function between topological spaces. If $\{x_0\}\subset X$ is open, then $f$ is continuous at $x_0$.
\end{lemma}

\begin{lemma}
\begin{enumerate}
\item If the topology of $Y$ is given by a basis $\mathcal{B}$, then to prove continuity of $f$ it suffices to show that the inverse image of every basis element is open.
\item If the topology of $Y$ is given by a subbasis $\mathcal{S}$, then to prove continuity of $f$ it suffices to show that the inverse image of every subbasis element is open.
\end{enumerate}
\end{lemma}
\begin{proposition}\label{prop:continuity}
Let $X, Y$ be topological spaces; $f:X\to Y$. The following are equivalent:
\begin{enumerate}
\item $f$ is continuous;
\item $f[\bar{A}]\subset \overline{f[A]}$;
\item for every closed set $B$ of $Y$, the set $f^{-1}[B]$ is closed in $X$. TODO $f$ closed.
\end{enumerate}
\end{proposition}
\begin{proof}
We proceed round-robin-style.
\begin{itemize}[leftmargin=2cm]
\item[$\boxed{(1) \Rightarrow (2)}$] Let $x\in \bar{A}$ and $V$ a neighbourhood of $f(x)$. Then $f^{-1}[V]$ is an open set containing $x$, so it must intersect $A$ in some point $y$ by proposition \ref{prop:closure}. Then $V$ intersects $f[A]$ in $f(y)$, so $f(x) \in \overline{f[A]}$ as desired.
\item[$\boxed{(2) \Rightarrow (3)}$] Let $B$ be closed in $Y$. We observe that $f[f^{-1}[B]]\subset B$. Choose some $x\in \overline{f^{-1}[B]}$, then
\[ f(x) \in f\left[\overline{f^{-1}[B]}\right] \subset \overline{f[f^{-1}[B]]} \subset \bar{B} = B, \]
so that $x\in f^{-1}[B]$. Thus $\overline{f^{-1}[B]}\subset f^{-1}[B]$, meaning $f^{-1}[B]$ is closed.
\item[$\boxed{(3) \Rightarrow (1)}$] Let $V$ be an open set in $Y$. Set $B = Y\setminus V$. Then $V = Y\setminus B$ and
\[ f^{-1}[V] = f^{-1}[Y\setminus B] = f^{-1}[Y]\setminus f^{-1}[B] = X \setminus f^{-1}[B]\]
using lemma \ref{lemma:preimageProperties}. Thus $f^{-1}[V]$ is open.
\end{itemize}
\end{proof}

\subsubsection{Homeomorphisms \textit{or} topological isomorphisms}
\begin{definition}
Let $X,Y$ be topological spaces and $f:X\to Y$ a bijection. Then $f$ is a \udef{homeomorphism} if both $f$ and $f^{-1}$ are continuous.
\end{definition}
\begin{lemma}
A homeomorphism is a bijection $f$ such that $f(U)$ is open \textup{if and only if} $U$ is open.
\end{lemma}
\subsubsection{Constructing continuous functions}
\begin{proposition} \label{prop:continuousConstructions}
Let $X,Y$ and $Z$ be topological spaces.
\begin{enumerate}
\item \textup{(Identity function)} The identity function $I:X\to X$ is continuous.
\item \textup{(Constant function)} If $f:X\to Y$ maps all of $X$ into a single $y_0$ of $Y$, then $f$ is continuous.
\item \textup{(Inclusion)} Let $A$ be a subspace of $X$, then the inclusion $A\hookrightarrow X$ is continuous.
\item \textup{(Composites)} If $f:X\to Y$ and $g:Y\to Z$ are continuous, then $g\circ f: X\to Z$ is continuous.
\item \textup{(Restricting the domain)} If $f:X\to Y$ is continuous and $A$ is a subspace of $X$, then the restricted function $f|_{A}:A\to Y$ is continuous.
\item \textup{(Restricting the range)} Let $f:X\to Y$ be continuous. If $Z$ is a subspace of $Y$ containing the image set $f[X]$, then $f:X\to Z$ is continuous.
\item \textup{(Expanding the range)} Let $f:X\to Y$ be continuous. If $Y$ is a subspace of $Z$, then $f:X\to Z$ is continuous.
\item \textup{(Local formulation of continuity)} The map $f:X\to Y$ is continuous is $X$ can be written as the union of open sets $U_\alpha$ such that $f|_{U_\alpha}$ is continuous for each $\alpha$.
\end{enumerate}
\end{proposition}
\begin{proposition}[The pasting lemma]
Let $X=A\cup B$ where $A,B$ are closed in $X$. Let $f:A\to Y$ and $g:B\to Y$ be continuous such that $f(x)=g(x)$ for all $x\in A\cap B$. Then the function defined by
\[ h: X\to Y: x\mapsto h(x) = \begin{cases}
f(x) & (x\in A) \\ g(x) & (x\in B)
\end{cases} \]
is continuous.
\end{proposition}
\begin{proof}
Let $C$ be a closed subset of $Y$, then $f^{-1}[C]$ and $g^{-1}[C]$ are both closed. So
\[ h^{-1}[C] = f^{-1}[C]\cup g^{-1}[C] \]
is closed, meaning $h$ is continuous, all by proposition \ref{prop:continuity}.
\end{proof}

TODO: complex conjugation continuous.

\subsection{Limits of functions}
\begin{definition}
Let $X,Y$ be topological spaces. Let $p$ be a limit point of $A\subseteq X$ and $f: A\to Y$. We say $L\in Y$ is a \udef{limit} of $f(x)$ as $x$ approaches $p$ if
\[ \forall\;\text{open neighbourhood}\; V(L):\exists \;\text{open neighbourhood}\; U(p):\; f[(U\cap A)\setminus \{p\}] \subseteq V. \]
We write $f(x)\to L$ as $x\to p$ or
\[ \lim_{x\to p}f(x) = L. \]
\end{definition}
Note that the value of $f$ at $p$ is irrelevant to the definition of the limit. The domain of $f$ does not even need to contain $p$.

\begin{proposition}
Let $f:X\to Y$ be a functions between topological spaces. Then $f$ is continuous at $p\in X$ \textup{if and only if} $\lim_{x\to p}f(x) = f(p)$.
\end{proposition}

Limits may or may not exist and may or may not be unique, but uniqueness is guaranteed if $Y$ is Hausdorff.
\begin{proposition} \label{prop:HausdorffUniqueLimit}
Let $f: A\subseteq X\to Y$ be a function and $X,Y$ be topological spaces. If $Y$ is Hausdorff, then there is a most one limit of $f$ at any point $p\in X$.
\end{proposition}
\begin{proof}
Assume $L_1$ and $L_2$ are two distinct limits of $f(x)$ as $x\to p$. Because $Y$ is Hausdorff there are two disjoint open neighbourhoods $V_1, V_2$ of $L_1,L_2$. Let $U_1,U_2$ be the corresponding open neighbourhoods of $p$. Then $U_1\cap U_2$ must be an open neighbourhood of $p$, so that $U_1\cap U_2\cap A$ contains a point other than $p$, by virtue of $p$ being a limit point. This however means that $f[(U_1\cap A)\setminus \{p\}]$ and $f[(U_2\cap A)\setminus \{p\}]$ are not disjoint, so neither are $V_1,V_2$: a contradiction.
\end{proof}

\subsection{Initial and final topologies}


\section{Density}
\begin{definition}
Let $(X,\mathcal{T})$ be a topological space and let $A$ be a subset of $X$. Then $A$ is called
\begin{enumerate}
\item \udef{dense} in $X$ if the closure of $A$ is the whole of $X$: $\overline{A} = X$;
\item \udef{rare} or \udef{nowhere dense} if its closure has empty interior: $(\overline{A})^\circ = \emptyset$;
\item \udef{meagre} (or a \udef{set of first category}) if it is a countable union of rare subsets of $X$;
\item \udef{nonmeagre} (or a \udef{set of second category}) if it is not meagre;
\item \udef{comeagre} if its complement $X\setminus A$ is meagre in $X$.
\end{enumerate}
\end{definition}
\begin{lemma} \label{lemma:densityEquivalences}
Let $A$ be a subset of a topological space $X$. $A$ being is dense in $X$ is equivalent to any of the following:
\begin{enumerate}
\item every element of $X$ either lies in $A$ or is a limit point of $A$;
\item $A^c$ has empty interior.
\end{enumerate}
\end{lemma}
\begin{proof}
For the first point: $\overline{A} = A\cup A' = X$.

For the second point:
\[ X = \overline{A} \iff X = ((A^c)^\circ)^c \iff (A^c)^\circ = X^c = \emptyset. \]
\end{proof}

\begin{lemma} \label{lemma:nowhereDensityEquivalence}
Let $X$ be a topological space and $A\subset X$ a subset. Then $A$ is nowhere dense \textup{if and only if} $\overline{A}^c$ is dense.
\end{lemma}
\begin{proof}
We calculate:
\[ (\overline{A})^\circ = \emptyset \iff \overline{(\overline{A}^c)}^c = \emptyset \iff \overline{(\overline{A}^c)} = X. \]
\end{proof}

\begin{lemma} \label{lemma:meagreSubset}
Any subset of a meagre set is meagre.
\end{lemma}
\begin{proof}
Let $A = \bigcup_k R_k$ be meagre and $B \subseteq A$. Then
\[ B = B\cap A = B\cap \left( \bigcup_k R_k \right) = \bigcup_k B\cap R_k. \]
Now for each $k$, $B\cap R_k \subset R_k$. So $\overline{B \cap R_k}^\circ \subseteq \overline{R_k}^\circ = \emptyset$, using lemma \ref{lemma:closureInteriorSubsets}, and thus $B\cap R_k$ is nowhere dense. 
\end{proof}

\begin{definition}
A topological space $Y$ has the \udef{unique extension property} if for any topological space $X$, any continuous functions $f,g:X\to Y$ and any dense subset $E\subset X$ we have
\[ \forall x\in E: f(x)=g(x) \quad\implies\quad f = g. \]
\end{definition}

\begin{proposition} \label{prop:uniqueExtensionHausdorff}
A topological space $Y$ has the unique extension property \textup{if and only if} $Y$ is Hausdorff.
\end{proposition}
\begin{proof}
TODO \url{https://www.jstor.org/stable/2315068?seq=1#metadata_info_tab_contents}
\end{proof}

\begin{lemma}
Let $X$ be a topological space and $E\subset X$ a .
\begin{enumerate}
\item If for any dense subspace $E\subset X$ the only continuous extension of $\id_E$ to $X$ is $\id_X$, then $X$ is $T_0$.
\item If $X$ is $T_2$, then for any dense subspace $E\subset X$ the only continuous extension of $\id_E$ to $X$ is $\id_X$. 
\end{enumerate}
$T_1$ is neither necessary nor sufficient.
\end{lemma}
\begin{proof}
TODO \url{https://math.stackexchange.com/questions/1592144/does-the-identity-map-on-a-dense-subset-of-a-space-extend-uniquely/1592169}
\end{proof}


\subsection{The Baire property}
\begin{definition}
A topological space $X$ has the \udef{Baire property} if it satisfies either of the following equivalent conditions:
\begin{enumerate}
\item every countable union of closed nowhere dense sets has empty interior;
\item every countable intersection of open dense sets is dense.
\end{enumerate}
These properties are equivalent because a subset has empty interior if and only if its complement is dense, see lemma \ref{lemma:densityEquivalences}.
\end{definition}


\begin{lemma} \label{lemma:BaireEquivalents}
A topological space $X$ is Baire \textup{if and only if} either of the following equivalent conditions:
\begin{enumerate}
\item every meagre subset of $X$ is either empty or not open;
\item every non-empty open subset of $X$ is a nonmeagre subset of $X$;
\item every comeagre subset of $X$ is dense in $X$.
\end{enumerate}
\end{lemma}
\begin{proof}
We prove the characterisation of spaces with the Baire property using countable unions implies the first point, the last point implies the countable intersection Baire condition.
\begin{itemize}[leftmargin=3cm]
\item[$\boxed{\text{Baire}\Rightarrow (1)}$] Every meagre set $A = \bigcup_k R_k$ (where all $R_k$ are nowhere dense) is a subset of $\bigcup_k \overline{R_k}$ where $\overline{R_k}$ are closed nowhere dense sets. Thus if the Baire property holds, $\bigcup_k \overline{R_k}$ has empty interior, meaning $A$ has empty interior. So either $A$ is empty or not open.
\item[$\boxed{(1) \Leftrightarrow (2)}$] By contraposition.
\item[$\boxed{(1) \Rightarrow (3)}$] Suppose $A$ is a meagre set. Then $A^\circ$ must also be meagre, by \ref{lemma:meagreSubset}. Now $A^\circ$ is certainly open, so by $(1)$ it must be empty. Thus $A^c$ is dense, by lemma \ref{lemma:densityEquivalences}.
\item[$\boxed{(3) \Rightarrow \text{Baire}}$] Let $A = \bigcap_k O_k$ where all $O_k$ are open dense sets. Then $A^c = \bigcup_k O_k^c$. Now for each $k$, $O_k^c$ is nowhere dense by lemma \ref{lemma:nowhereDensityEquivalence}, because $\overline{O_k^c}^c = O_k^\circ$ is still dense. Thus $A^c$ is meagre and $A$ is comeagre, so $A$ is dense in $X$.
\end{itemize}
\end{proof}

A topological space has the Baire property if and only if it has the property locally, in the following sense:
\begin{lemma}
A topological space $X$ has the Baire property \textup{if and only if} every point in $X$ has a neighbourhood with the Baire property.
\end{lemma}
\begin{proof}
If $X$ is Baire, the neighbourhood can simply be taken to be $X$.

Assume every point in $X$ has a neighbourhood with the Baire property.
We will prove point (2) in lemma \ref{lemma:BaireEquivalents} holds.
Take a non-empty open subset $A$ of $X$.
As $A$ is non-empty, we can take a point $x\in A$ and find a neighbourhood $U$ of $x$ with the Baire property.
Then $A\cap U$ is a non-empty open subset of $U$ and thus must not be meagre in $U$.
By contraposition of lemma \ref{lemma:meagreSubset}, we see that $A$ must be non-meagre in $X$, proving the Baireness of $X$.
\end{proof}

\begin{theorem}[Baire category theorem] \label{theorem:BaireCategory} \hspace{1em}
\begin{enumerate}
\item Every complete pseudometric space has the Baire property.
\item Every locally compact Hausdorff space has the Baire property.
\end{enumerate}
\end{theorem}
\begin{proof}
TODO + relocate
\end{proof}

\section{Connectedness}
\begin{definition}
Let $X$ be a topological space. A \udef{separation} of $X$ is a pair $U,V$ of disjoint nonempty open subsets of $X$ whose union is $X$.

The space $X$ is said to be \udef{connected} if there does not exist a separation of $X$.
\end{definition}
\begin{lemma}
A space $X$ is connected \textup{if and only if} the only subsets of $X$ that are both open and closed in $X$ are $\emptyset$ and $X$.
\end{lemma}
\begin{proof}
We prove the contrapositive of both implications.
\begin{itemize}
\item[$\boxed{\Rightarrow}$] Let $A$ be a nonempty proper subset of $X$ that is both open and closed in $X$. The sets $A$ and $X\setminus A$ form a separation.
\item[$\boxed{\Leftarrow}$] Let $U,V$ be a separation. Then $U$ is open. It is also closed, because its complement in $X$ is $V$, which is open.
\end{itemize}
\end{proof}
The following lemma characterises separations in the subspace topology.
\begin{lemma}
Let $Y$ be a subspace of a topological space $X$. A pair of disjoint nonempty sets $A,B$ constitute a separation of the subspace $Y$ \textup{if and only if} neither set contains a limit point of the other in $X$.
\end{lemma}
\begin{proof}

\end{proof}

\subsection{Path connectedness}

\section{Compactness}
TODO relatively compact.

\begin{proposition} \label{prop:imageCompactIsCompact}
The continuous image of a compact set is compact.
\end{proposition}

\begin{proposition}
A compact set in a Hausdorff space is closed.
\end{proposition}

\begin{proposition} \label{prop:compactToHausdorffHomeomorphism}
Let $f:X\to Y$ be a continuous bijection between topological spaces. If $X$ is compact and $Y$ is Hausdorff, then $f$ is a homeomorphism.
\end{proposition}
\begin{proof}
We just need to prove $f^{-1}$ is continuous. This is equivalent to $f$ being closed by \ref{prop:continuity}. TODO
\end{proof}

\subsection{Limit point compactness}
\subsection{Local compactness}




\chapter{Separation axioms}
\section{$T_1$}
\begin{proposition}
Let $X$ be a topological space satisfying $T_1$; let $A$ be a subset of $X$.
Then the point $x$ is a limit point of $A$ \textup{if and only if} every neighbourhood of $x$ contains infinitely many points of $A$.
\end{proposition}
\section{Hausdorff spaces}
\begin{definition}
A topological space $X$ is called a \udef{Hausdorff space} if for each pair $x_1, x_2$ of distinct points in $X$, their exist neighbourhoods $U(x_1)$, $U(x_2)$ that are disjoint.
\end{definition}
In Hausdorff spaces distinct points can be told apart topologically, hence Hausdorff spaces are also called \udef{separated spaces}. In particular the Hausdorff condition implies the uniqueness of limits, which is not otherwise guaranteed.

\begin{proposition}
Every finite point set in a Hausdorff space is closed. TODO: T1
\end{proposition}
\begin{proof}
It suffices to show that every one-point set $\{x_0\}$ is closed. Indeed if $\{x_0\}$ was not closed, the closure of $\{x_0\}$ would contain another point. This other point has a disjoint neighbourhood by Hausdorff, so this fails by proposition \ref{prop:closure}.
\end{proof}
\begin{proposition}
Limits are unique (sequences, filters, nets)
\end{proposition}
\begin{lemma}
\begin{enumerate}
\item A subspace of a Hausdorff space is Hausdorff.
\item Every totally ordered set is Hausdorff in the order topology.
\item Every metric topology is Hausdorff.
\end{enumerate}
\end{lemma}




\chapter{Sequences, nets and filters}
Sequences, nets and filters can be seen as probes of the topology.
\section{Sequences}
\subsection{Limits and convergence}
\begin{definition}
Let $(X,\mathcal{T})$ be a topological space and $(a_n)_{n\in\N}$ a sequence in $X$. We can view this sequence as a function $\N\subset (\N\cup\{\infty\})\to X$.

We define \udef{limit} of the sequence as a limit of this function at $\infty$ if $\N\cup\{\infty\}$ is equipped with the order topology.
\end{definition}
By \ref{prop:HausdorffUniqueLimit} a sequence has at most one limit $L\in X$ if $X$ is Hausdorff. In this case we call it the limit of the sequence and write
\[ \lim_{n\to \infty}a_n = L. \]
We call a sequence \udef{divergent} if it does not have a limit and
 \udef{convergent} if it does have a limit $L$. In this last case we say the sequence \udef{converges} to $L$.

\begin{proposition} \label{prop:sequenceConvergence}
Let $(X,\mathcal{T})$ be a topological space and $(a_n)_{n\in\N}$ a sequence in $X$. Then the sequence converges to $L\in X$ \textup{if and only if}
\[ \forall \;\text{open neighbourhood}\; V(L): \exists n_0\in \N: \forall n\geq n_0: a_n\in V(L). \]
\end{proposition}
\begin{proof}
Assume the sequence converges to $L$ and take an arbitrary open neighbourhood $V(L)$. Then there exists an open neighbourhood $U(\infty)$ such that $a[U\setminus\{\infty\}] \subseteq V$. Now by definition of the order topology, there exists an interval $]m, \infty]\subseteq U(\infty)$. Then $a[\;m, \infty[\;] \subseteq V$ and by setting $n_0=m+1$ we get the criterion of the proposition.

Conversely assume the criterion and fix $V(L)$. Then we can take $U(\infty)=]n_0, \infty]$.
\end{proof}

\begin{lemma} \label{lemma:subsequencesConverge}
Let $(X,\mathcal{T})$ be a topological space and $(a_n)_{n\in\N}$ a sequence in $X$ that converges to $L$. Then all subsequences converge to $L$.
\end{lemma}

TODO: Bolzano-Weierstrass (sequence version + accumulation point version)

\subsection{Sequential spaces}
\subsubsection{The sequential topology}
\begin{definition}
Let $(X,\mathcal{T})$ be a topological space and $S\subseteq X$ a subset.
\begin{itemize}
\item The \udef{sequential closure} of $S$ in $X$ is the set
\[ \operatorname{SeqCl}(S) \defeq \setbuilder{x\in X}{\text{$\exists$ a sequence in $S$ that converges to $x$ in $X$}}. \]
\item The \udef{sequential interior} of $S$ in $X$ is the set
\[ \operatorname{SeqInt}(S) \defeq \setbuilder{s\in S}{\text{every sequence in $X$ that converges to $s$ has a tail in $S$}}. \]
\end{itemize}
We call $S$
\begin{itemize}
\item \udef{sequentially open} if $S = \operatorname{SeqInt}(S)$;
\item \udef{sequentially closed} if $S = \operatorname{SeqCl}(S)$;
\item a \udef{sequential neighbourhood} of a point $x\in X$ if $x\in \operatorname{SeqInt}(S)$.
\end{itemize}
\end{definition}
A sequentially closed set is a set $S$ such that all limits of sequences in $S$ are also in $S$.

\begin{lemma} \label{lemma:sequentialInteriorClosure}
Let $(X,\mathcal{T})$ be a topological space and $R,S\subseteq X$  subsets. Then
\begin{enumerate}
\item $\operatorname{SeqInt}(S) = (\operatorname{SeqCl}(S^c))^c$;
\item $\operatorname{SeqCl}(\emptyset) = \emptyset$ and $\operatorname{SeqCl}(X) = X$;
\item $S\subseteq \operatorname{SeqCl}(S)$;
\item $\operatorname{SeqCl}(R\cup S) = \operatorname{SeqCl}(R)\cup \operatorname{SeqCl}(S)$;
\item $\operatorname{SeqCl}(S) \subseteq \bar{S}$ and $\operatorname{SeqInt}(S) \supseteq S^\circ$.
\end{enumerate}
\end{lemma}
\begin{proof}
(1) Both sides of the equation are equivalent to
\[ \setbuilder{s\in S}{\text{$\nexists$ a sequence in $X\setminus S$ that converges to $s$ in $X$}}. \]

(2) There are no sequences that converge to a point in $\emptyset$ and all points $x\in X$ are the limit of a constant sequence $n\mapsto x$.

(3) The constant sequence $n\mapsto x$ converges to $x$.

(4) We can find a subsequence in $R$ or in $S$. All subsequences converge by \ref{lemma:subsequencesConverge}.

(5) Let $x\in \operatorname{SeqCl}(S)$, so there is a sequence $(a_n)$ in $S$ that converges to $x$. Take an arbitrary open neighbourhood $V$ of $x$. Then by convergence there is a subsequence of $(a_n)$ that is a sequence in $V$. In particular $V$ intersects $S$. So $x\in \bar{S}$ by \ref{prop:closure}.
\end{proof}

\begin{proposition} \label{prop:sequentialTopology}
Let $(X,\mathcal{T})$ be a topological space. The set of all sequentially open sets forms a topology $\mathcal{T}_\text{seq}$ on $X$. This topology is finer than the original topology.
\end{proposition}
\begin{proof}
First note that sequentially closed sets are complements of sequentially open sets by point 1. of \ref{lemma:sequentialInteriorClosure}.

By point 2. of \ref{lemma:sequentialInteriorClosure}, $\emptyset$ and $X$ are both clopen.

We will prove the rest using closed sets. By point 4. of \ref{lemma:sequentialInteriorClosure} finite unions of sequentially closed sets are sequentially closed.

Let $\bigcap_{i\in I}K_i$ be an arbitrary intersection of sequentially closed sets $K_i$. We only need to prove
\[ \operatorname{SeqCl}\left(\bigcap_{i\in I}K_i\right) \subseteq \bigcap_{i\in I}K_i \]
because the other inclusion is immediate. Take an $x\in \operatorname{SeqCl}\left(\bigcap_{i\in I}K_i\right)$. Then there is a sequence in $\bigcap_{i\in I}K_i$ that converges to $x$. Because of the intersection this sequence is in each $K_i$ and thus so is $x$.

The fineness of the topology follows from point 5. of \ref{lemma:sequentialInteriorClosure}.
\end{proof}
\begin{corollary}
Let $(X,\mathcal{T})$ be a topological space and $S\subseteq X$  a subset. Then
\[ \text{open/closed} \quad\implies\quad \text{sequentially open/closed.} \]
\end{corollary}

\begin{proposition} \label{prop:sequentialTopologySameConvergentSequences}
Let $(X,\mathcal{T})$ be a topological space and $(x_n)$ a sequence in $X$. Then $x_n\to x$ in $(X,\mathcal{T})$ \textup{if and only if} $x_n\to x$ in $(X,\mathcal{T}_\text{seq})$.
\end{proposition}
\begin{proof}
Now $\mathcal{T}_\text{seq}$ is finer than $\mathcal{T}$, so the $\Leftarrow$ direction is evident. For the $\Rightarrow$ direction, assume $x_n\to x$ in the original topology. Let $V(x)$ be an open neighbourhood in the sequential topology. By definition of the sequential topology $(x_n)$ has a tail in $V$. This means $x_n\to x$ in the sequential topology by \ref{prop:sequenceConvergence}.
\end{proof}

\subsubsection{Transfinite sequential closure}
It is possible that the sequential closure is not idempotent (unlike the normal topological closure), i.e.
\[ \operatorname{SeqCl}(\operatorname{SeqCl}(S)) \neq \operatorname{SeqCl}(S). \]

\subsubsection{Sequential continuity}
\begin{definition}
A function $f:(X,\mathcal{T})\to(Y,\mathcal{T}')$ is called \udef{sequentially continuous} if
\[ f:(X,\mathcal{T}_\text{seq})\to(Y,\mathcal{T}'_\text{seq}) \]
is continuous. I.e. $f$ is continuous when $X,Y$ are equipped with their sequential topologies.
\end{definition}

\begin{proposition} \label{prop:sequentialContinuity}
A function $f:(X,\mathcal{T})\to(Y,\mathcal{T}')$ is sequentially continuous \textup{if and only if} for every sequence $(x_n)_{n\in\N}$ in $X$ and $x\in X$
\[ x_n \to x \;\;\text{in}\; (X,\mathcal{T}) \quad\implies\quad f(x_n)\to f(x) \;\;\text{in}\; (Y,\mathcal{T}'). \]
\end{proposition}
\begin{proof}
First assume this property holds and we want to prove sequential continuity. Let $S\subset Y$ be sequentially closed. Then we need to prove $f^{-1}[S]$ is also sequentially closed. Indeed take a converging sequence $(x_n)$ in $f^{-1}[S]$ with limit $x$. Then $(f(x_n))$ converges to $f(x)$ and $f(x)\in S$. This implies $x\in f^{-1}[S]$, meaning it is sequentially closed. 

Conversely, assume $f$ is sequentially continuous. Let $(x_n)$ be a sequence in $X$ that converges to $x$. Let $V(f(x))\in \mathcal{T}'$ be an open neighbourhood of $f(x)$; $V$ is also sequentially open. Then by continuity we have a $U(x)\in\mathcal{T}_\text{seq}$ such that $f[U]\subseteq V$. Because $U$ is sequentially open, there is an $n_0\in\N$ such that $\forall n\geq n_0: x_n\in U$.
This implies $\forall n\geq n_0: f(x_n)\in f[U]\subseteq V$ and so $(f(x_n))$ converges to $f(x)$.
\end{proof}
\begin{proposition}
Every continuous function is sequentially continuous.
\end{proposition}
\begin{proof}
We use the characterisation of sequential continuity in \ref{prop:sequentialContinuity}. Let $x_n\to x$. Let $V$ be an open neighbourhood of $f(x)$. Then there exists an open neighbourhood $U(x)$ such that $f[U]\subset V$. By \ref{prop:sequenceConvergence} $U$ contains all but finitely many elements of the sequence $(x_n)$. Thus $V$ contains all but finitely many of the elements of the sequence $(f(x_n))$. Take $n_0$ larger than the indices of all elements of $(f(x_n))$ omitted from $V$. By \ref{prop:sequenceConvergence} $f(x_n)\to f(x)$.
\end{proof}

\subsubsection{Sequential spaces}
\begin{definition}
A topological space $(X,\mathcal{T})$ is called a \udef{sequential space} if every sequentially open set is open.
\end{definition}

\begin{lemma}
Let $(X,\mathcal{T})$ be a topological space. Then $X$ equipped with its sequential topology is a sequential space.
\end{lemma}
\begin{proof}
It is enough to show that a sequentially closed set in $(X,\mathcal{T}_\text{seq})$ is also sequentially closed in $(X,\mathcal{T})$. (I.e. that passing to the finer topology does not introduce even more sequentially open sets). By \ref{prop:sequentialTopologySameConvergentSequences} the definition of $\operatorname{SeqCl}$ is the same in both topologies, yielding the proof.
\end{proof}

\begin{proposition}
Let $(X,\mathcal{T})$ be a topological space. Then the following are equivalent:
\begin{enumerate}
\item $(X,\mathcal{T})$ is a sequential space;
\item for every subset $S\subset X$ that is not closed in $X$, there exists some $x\in \bar{S}\setminus S$ for which there exists a sequence in $S$ that converges to $x$;
\item $(X,\mathcal{T})$ is the quotient of a first countable space;
\item $(X,\mathcal{T})$ is the quotient of a metric space.
\end{enumerate}
\end{proposition}
TODO: relocate observation about metric spaces.

\begin{proposition}[Universal property of sequential spaces]
Let $(X,\mathcal{T})$ be a topological space. Then $X$ is sequential \textup{if and only if} for every topological space $Y$, a function $f:X\to Y$ is continuous $\Leftrightarrow$ $f$ is sequentially continuous.
\end{proposition}


\subsubsection{$T$-sequential and $N$-sequential spaces}

\subsubsection{Fréchet–Urysohn spaces}
\begin{definition}
A topological space $(X,\mathcal{T})$ is called a \udef{Fréchet–Urysohn space} if for every subset $S\subseteq X$
\[ \operatorname{SeqCl}(S) = \bar{S}. \]
\end{definition}
Clearly every Fréchet–Urysohn space is a sequential space.

\begin{proposition} \label{prop:FrechetUrysohn}
Let $(X,\mathcal{T})$ be a topological space. Then the following are equivalent:
\begin{enumerate}
\item $(X,\mathcal{T})$ is a Fréchet–Urysohn space;
\item every subspace of $X$ is a sequential space;
\item for every subset $S\subset X$ that is not closed in $X$ and for all $x\in \bar{S}\setminus S$ there exists a sequence in $S$ that converges to $x$.
\end{enumerate}
\end{proposition}

\subsection{Sequences in ordered space}
In this section we will be considering sequences in a totally ordered set $(X,\leq)$ equipped with the order topology.

\begin{lemma} \label{lemma:convergentSequenceIsBounded}
A convergent sequence in a totally ordered space has an upper and a lower bound.
\end{lemma}
\begin{proof}
Let $x_n\to x$. Choose a basis element containing $x$. If it is of the form $]a,b_0]$ for some greates element $b_0$, then $b_0$ is the upper bound. If not, it is of the form $[a_0,b[$ or $]a,b[$. Find an $n_0$ corresponding to this basis element. Then an upper bound is given by
\[ \max(x[\;[0,n_0]\;]\cup\{b\}). \]
The lower bound is analogous.
\end{proof}

\begin{proposition} \label{prop:limitPreservesInequality}
Let $(a_n)$ and $(b_n)$ be convergent sequences in a totally ordered space such that $a_n\leq b_n$ for all $n\in\N$. Then
\[ \lim_{n\to \infty}a_n \leq \lim_{n\to \infty}b_n. \]
\end{proposition}
\begin{proof}
Let $a_n\to a$ and $b_n\to b$. If $a=b$ then the proposition is valid. Now assume $a\neq b$. If $a$ or $b$ are either the greatest or the least element, the proposition is valid. Now assume this is not the case.

Assume towards a contradiction that $a>b$. Then we can find open neighbourhoods of $a$ and $b$ of the form $]b,d[$ and $]c,a[$, respectively. Now find $n_0, n_1$ such that $\forall n\geq n_0: a_n \in ]b,d[$ and $\forall n\geq n_1: b_n \in ]c,a[$. 
Then for all $n \geq \max\{n_0,n_1\}$ we have $a_n\in ]b,d[$ and $b_n\in ]c,a[$, implying $a_n > b_n$ which is a contradiction.
\end{proof}
It is easy to show that this does not in general hold for the strict inequality $<$.

\begin{proposition}[Squeeze theorem for sequences]
Let $(a_n)$, $(b_n)$ and $(c_n)$ be sequences in a totally ordered space such that
\[ \forall n\in \N: a_n\leq b_n \leq c_n. \]
If $(a_n)$ and $(c_n)$ are convergent with the same limit $L$, then
\[ \lim_{n\to \infty}b_n = L. \]
\end{proposition}
\begin{proof}
Let $V(L)$ be an open neighbourhood of $L$. By definition of the order topology there is an interval $I = ]x,y[ \subset V$ such that $L\in I$. Then find $n_0$ and $n_1$ such that $\forall n\geq n_0: a_n\in I$ and $\forall n\geq n_1: c_n\in I$. Then set $n_2 = \max\{n_0,n_2\}$ and we have $\forall n\geq n_2:$
\[ x \leq a_n \leq b_n \quad \text{and} \quad  b_n \leq c_n \leq y. \]
By transitivity we have $b_n\in I \subset V$.
\end{proof}

\begin{proposition} \label{prop:sequenceToSupInf}
Let $X$ be an ordered space and $A$ a subspace.  Assume the axiom of dependent choice.
\begin{enumerate}
\item If $A$ has a supremum $a$, then there exists a sequence in $A$ that converges to $a$ in $X$.
\item If $A$ has an infimum $b$, then there exists a sequence in $A$ that converges to $b$ in $X$.
\end{enumerate}
\end{proposition}
\begin{proof}
Assume the supremum $a$ of $A$ exists. If $a\in A$ we can take the constant sequence $(a)_{n\in \N}$.

If $a\notin A$, we can find for each $x_i\in A$ an $x_{i+1}$ satisfying $x_i < x_{i+1} < a$. The sequence thus defined converges by monotone convergence.
\end{proof}
In many cases the axiom of dependent choice is superfluous, if the details of the spaces $X,A$ allow for the construction of $x_{i+1}$ from $x_i$.

\subsubsection{Divergence to $\pm\infty$}
\begin{definition}
Let $(x_n)$ be a sequence in a totally ordered space $X$. Then
\begin{itemize}
\item $(x_n)$ \udef{diverges to $+\infty$} if $\forall M\in X: \exists n_0\in\N: \forall n\geq n_0: x_0 > M$; and
\item $(x_n)$ \udef{diverges to $-\infty$} if $\forall M\in X: \exists n_0\in\N: \forall n\geq n_0: x_0 < M$.
\end{itemize}
We write $\lim_{n\to\infty}x_n = +\infty$ and $\lim_{n\to\infty}x_n = -\infty$, respectively.
\end{definition}

\begin{lemma}
Let $(x_n)$ be a sequence in a totally ordered space $X$. Then
\begin{enumerate}
\item if $(x_n)$ is increasing, but not bounded above, it diverges to $+\infty$;
\item if $(x_n)$ is decreasing, but not bounded below, it diverges to $-\infty$.
\end{enumerate}
\end{lemma}

\subsection{Sequences in complete ordered space}
\subsubsection{Monotone convergence}
\begin{proposition}[Monotone convergence] \label{prop:sequenceMonotoneConvergence}
Let $(X,\leq)$ be a complete totally ordered space and let $(x_n)$ be a sequence in $X$.
\begin{enumerate}
\item If $(x_n)$ is increasing and bounded above, then it is convergent with limit $\sup_n x_n$.
\item If $(x_n)$ is decreasing and bounded below, then it is convergent with limit $\inf_n x_n$.
\end{enumerate}
\end{proposition}
\begin{proof}
We prove the first point. The second is analogous.

Let $V(\sup_n x_n)$ be an open and $]x,y[\subset V$ such that $\sup_n x_n \in ]x,y[$. Now because $x<\sup_n x_n$ it is not an upper bound of the sequence and there exists an $x_{n_0}> x$. Because the sequence is increasing (and $y$ is a a strict upper bound), all $x_n$ where $n\geq n_0$ are in $]x,y[\subset V$.
\end{proof}

\subsubsection{Limes superior and inferior}
\begin{definition}
Let $(X,\leq)$ be a complete totally ordered space and let $(x_n)$ be a sequence in $X$. We define
\begin{itemize}
\item the \udef{limes superior} or \udef{limit superior} or \udef{limsup} of $(x_n)$ as
\[ \limsup_{n\to\infty} x_n = \lim_{n\to \infty} \sup\setbuilder{x_m}{m\geq n}; \]
\item the \udef{limes inferior} or \udef{limit inferior} or \udef{liminf} of $(x_n)$ as
\[ \liminf_{n\to\infty} x_n = \lim_{n\to \infty} \inf\setbuilder{x_m}{m\geq n}. \]
\end{itemize}
\end{definition}
The liminf and limsup may not exist.
\begin{lemma}
Let $(X,\leq)$ be a complete totally ordered space and let $(x_n)$ be a sequence in $X$. The limsup and liminf exist \textup{if and only if}  $(x_n)$ is bounded above and below.
\end{lemma}
\begin{proof}
The sequences $\sup\setbuilder{x_m}{m\geq n}$ and $\inf\setbuilder{x_m}{m\geq n}$ are bounded if the limsup and liminf exist and bound $(x_n)$.

The converse follows because the sequences $\sup\setbuilder{x_m}{m\geq n}$ and $\inf\setbuilder{x_m}{m\geq n}$ are monotone.
\end{proof}

\begin{proposition} \label{prop:characterisationLimsupLiminf}
Let $(X,\leq)$ be a complete totally ordered space and let $(x_n)$ be a bounded sequence in $X$. Then 
\begin{enumerate}
\item $L_s = \limsup_{n\to\infty} x_n$ \textup{if and only if}
\begin{align*}
&\forall b > L_s: \exists n_0\in \N:\forall n\geq n_0: x_n < b \qquad \text{and} \\
&\forall a < L_s: \forall n_0\in \N:\exists n\geq n_0: a < x_n
\end{align*}
\item $L_i = \liminf_{n\to\infty} x_n$ \textup{if and only if}
\begin{align*}
&\forall a < L_i: \exists n_0\in \N:\forall n\geq n_0: a < x_n \qquad \text{and} \\
&\forall b > L_i: \forall n_0\in \N:\exists n\geq n_0: x_n < b.
\end{align*}
\end{enumerate}
\end{proposition}

\begin{proposition}
Let $(X,\leq)$ be a complete totally ordered space and let $(x_n)$ be a sequence in $X$. Then $(x_n)$ is convergent \textup{if and only if}
\[ \liminf_{n\to \infty} x_n = \limsup_{n\to \infty} x_n. \]
In this case
\[ \lim_{n\to \infty} x_n = \liminf_{n\to \infty} x_n = \limsup_{n\to \infty} x_n. \]
\end{proposition}
\begin{proof}
Assume $(x_n)$ is a sequence with identical liminf and limsup. Now
\[ \inf\setbuilder{x_m}{m\geq n} \leq x_n \leq \sup\setbuilder{x_m}{m\geq n} \]
so we can apply the squeeze theorem for sequences.

For the converse we use \ref{prop:characterisationLimsupLiminf}.
\end{proof}

\begin{lemma} \label{lemma:monotonicityLimsupLiminf}
Let $(a_n)$ and $(b_n)$ be bounded sequences in a totally ordered space such that $a_n\leq b_n$ for all $n\in\N$. Then
\[ \limsup_{n\to \infty}a_n \leq \limsup_{n\to \infty}b_n \quad\text{and}\quad \liminf_{n\to \infty}a_n \leq \liminf_{n\to \infty}b_n. \]
\end{lemma}








\begin{lemma} \label{lemma:sequencesSupInf}
There exist sequences converging to supremum and infimum.
\end{lemma}

\subsection{Completeness}





\section{Nets}
\begin{definition}
Let $(D,\leq)$ be a directed set and $X$ a set. Then a \udef{net} in $X$ is a function $D\to X$. The directed set $D$ is called the \udef{index set}.
\end{definition}
In particular sequences are nets because $(\N,\leq)$ is a directed set.
\subsection{Convergence}
\begin{definition}

\end{definition}

\begin{proposition}
A topological space is Hausdorff \textup{if and only if} every net converges to at most one point.
\end{proposition}

\subsection{Subnets}

TODO: generalise:
\begin{proposition}
Let $X$ be a topological space and $A\subset X$. If there is a sequence of points of $A$ converging to $x$, then $x\in\bar{A}$. The converse holds if $X$ is metrisable.
\end{proposition}

\section{Filters}


\chapter{Uniform spaces}
\url{file:///C:/Users/user/Downloads/(12)%20(Mathematical%20Surveys%20and%20Monographs)%20J.%20R.%20Isbell%20-%20Uniform%20Spaces-American%20Mathematical%20Society%20(1964).pdf}
\url{file:///C:/Users/user/Downloads/(London%20Mathematical%20Society%20Lecture%20Note%20Series)%20I.%20M.%20James%20-%20Introduction%20to%20Uniform%20Spaces-Cambridge%20University%20Press%20(1990).pdf}




\chapter{Some topologies}
These are the topologies that these object usually posses. If nothing else is said, these topologies will be assumed.

\section{The metric topology}
\begin{definition}
A \udef{metric} on a set $X$ is a function
\[ d:X\times X \to \R \]
with the properties:
\begin{enumerate}
\item $\forall x,y\in X: d(x,y)\geq 0$ and equality holds \textup{if and only if} $x=y$;
\item $\forall x,y\in X: d(x,y)= d(y,x)$;
\item $\forall x,y,z\in X: d(x,y)+d(y,z)\geq d(x,z)$.
\end{enumerate}
\end{definition}
The number $d(x,y)$ is often called the \udef{distance} between $x$ and $y$ in the metric $d$.
\begin{definition}
\begin{itemize}
\item The \udef{$\epsilon$-ball centered at $x$} is the set
\[ B_d(x,\epsilon) = \{y\;|\; d(x,y)< \epsilon\} \]
of all points $y$ whose distance to $x$ is less than $\epsilon$.
\item The \udef{closed $\epsilon$-ball centered at $x$} is the set
\[ \overline{B}_d(x,\epsilon) = \{y\;|\; d(x,y)\leq \epsilon\} \]
of all points $y$ whose distance to $x$ is less than or equal to $\epsilon$.
\item The \udef{$\epsilon$-sphere centered at $x$} is the set
\[ S_d(x,\epsilon) = \{y\;|\; d(x,y) = \epsilon\}. \]
\end{itemize}
\end{definition}

\begin{lemma}
Let $X$ be a set and $d$ a metric on $X$. Then the collection of all $\epsilon$-balls forms a basis for a topology on $X$.
\end{lemma}

\begin{definition}
\begin{itemize}
\item A \udef{metric space} $(X,d)$ is a set $X$ together with a metric $d$.
\item The topology generated by the $\epsilon$-balls in $X$ is called the \udef{metric topology} generated by $d$.
\item If $(X,\mathcal{T})$ is a topological space such that there exists a metric $d$ on $X$ such that $\mathcal{T}$ is the metric topology, then $X$ is called \udef{metrisable}.
\end{itemize}
\end{definition}
Only the local behaviour of the metric is important:
\begin{proposition}
Let $(X,d)$ be a metric space. Define
\[ \bar{d}:X\times X\to \R: (x,y)\mapsto \min\{d(x,y),1\}. \]
Then $\bar{d}$ is a metric that induces the same topology as $d$.
\end{proposition}
The metric $\bar{d}$ is called the \udef{standard bounded metric} corresponding to $d$.

\begin{proposition} \label{lemma:ballsCoarseness}
Let $d,d'$ be metrics on the set $X$, inducing $\mathcal{T}$ and $\mathcal{T}'$, respectively. Then $\mathcal{T}'$ is finer than $\mathcal{T}$ \textup{if and only if}
\[ \forall x\in X:\forall \epsilon>0:\exists \delta>0:\quad B_{d'}(x,\delta)\subset B_d(x,\epsilon). \]
\end{proposition}
\begin{proof}
Application of lemma \ref{lemma:basisCoarseness}.
\end{proof}

\begin{definition}
Let $(X,d)$ be a metric space and $S\subset X$ a subset. We call $S$ \udef{bounded} if there exists a ball $B_d(x,\epsilon)$ such that $S\subseteq B_d$.
\end{definition}

\subsection{Continuous functions in metric spaces}
For maps between metric spaces, the continuity requirement is equivalent to the $\epsilon-\delta$ formulation:
\begin{proposition}
Let $(X,d_X), (Y,d_Y)$ be metric spaces. The continuity of $f:X\to Y$ is equivalent to the condition that, for all $x\in X$:
\[ \forall \epsilon>0:\exists \delta>0:\; d_X(x,y)<\delta \implies d_Y(f(x),f(y))<\epsilon. \]
\end{proposition}
\begin{definition}
Let $f_n:X\to Y$ be a sequence of functions from the set $X$ to the metric space $(Y,d)$. The sequence \udef{converges uniformly} to the function $f:X\to Y$ if
\[ \forall \epsilon>0:\exists N\in \N:\forall n>N:\forall x\in X: \quad d(f_n(x),f(x))<\epsilon.  \]
\end{definition}
\begin{theorem}[Uniform limit theorem]
Let $f_n:X\to Y$ be a sequence of continuous functions from the topological space $X$ to the metric space $Y$. If $(f_n)$ converges uniformly to $f$, then $f$ is continuous.
\end{theorem}
\begin{proof}
We show $f$ is continuous at every point, so choose a point $x_0$ and a neighbourhood $V$ of $f(x_0)$. We then need to show that we can find a neighbourhood $U$ of $x_0$ such that $f(U)\subset V$ Now choose an $\epsilon>0$ such that $B(f(x_0), \epsilon)\subset V$. By uniform convergence, we can find an $N\in\N$ such that
\[\forall n>N:\forall x \in X:\quad d(f_n(x),f(x))<\epsilon/3.\]
By continuity of $f_N$, choose a neighbourhood $U$ of $x_0$ such that $f(U)\subset B(f_N(x_0),\epsilon/3)$. We claim this is the $U$ we need: take an arbitrary $x\in U$, then
\begin{align*}
d(f(x),f(x_0))&\leq d(f(x), f_N(x)) + d(f_N(x), f_N(x_0)) + d(f_N(x_0),f(x_0)) \\
&< \epsilon/3 + \epsilon/3 + \epsilon/3 = \epsilon.
\end{align*}
So $f(U)\subset B(f(x_0), \epsilon)\subset V$.
\end{proof}

TODO: convex sets; distance point to set.

TODO: A subspace Y of a Banach space X is complete if and only if the set Y is closed in X.

\subsection{Maps between metric spaces}
\begin{definition}
Let $(X,d_X)$ and $(Y,d_Y)$ be metric spaces. A map $f:X\to Y$ is an \udef{isometry} or \udef{distance preserving} if
\[ \forall a,b \in X: \quad d_Y(f(a),f(b)) = d_X(a,b). \]
\end{definition}
\begin{lemma} \label{lemma:isometryInjective}
An isometry is automatically injective.
\end{lemma}
\begin{proof}
Let $f: X\to Y$. Assume $f(a) = f(b) = y$, then
\[ 0 = d_Y(y,y) = d_Y(f(a),f(b)) = d_X(a,b). \]
By non-degeneracy of the metric we have $a=b$, meaning $f$ is injective.
\end{proof}
\begin{lemma} \label{lemma:isometryContinuous}
An isometry is automatically continuous.
\end{lemma}
\begin{proof}
For an isometry $f:X\to Y$, we have
\[ f[B(x,\epsilon)] = B(f(x),\epsilon), \]
so $f$ is continuous at each point $x$. It is then globally continuous by \ref{prop:globalContinuityFromAllPoints}.
\end{proof}
TODO: merge lemmas?

\begin{lemma} \label{lemma:isometryClosed}
Let $f:X\to Y$ be an isometry and $X$ a complete metric space. Then $f$ is a closed map.
\end{lemma}
\begin{proof}
Take $K\subset X$ closed and $y\in\overline{f[K]}$. Then there exists a sequence $(f(x_n))$ in $f[K]$ converging to $y$. The sequence $(f(x_n))$ is convergent and thus Cauchy. Because $d(f(x_n), f(x_m)) = d(x_n,x_m)$, the sequence $(x_n)$ must also be Cauchy. It is convergent because $X$ is complete and the limit $x$ lies in $K$ because it is complete (TODO ref). By continuity of $f$, \ref{lemma:isometryContinuous}, we have
\[ y = \lim_{n\to\infty} f(x_n) = f(\lim_{n\to\infty}x_n) = f(x) \in f[K]. \]
So $\overline{f[K]} = f[K]$ and $f$ is closed.
\end{proof}

\subsection{Cauchy sequences and completeness}
\begin{definition}
Let $(X,d)$ be a metric space and $(x_n)$ a sequence in $X$. Then $(x_n)$ is called a \udef{Cauchy sequence} if
\[ \forall 0 < \epsilon \in \R: \exists n_0\in \N: \forall m,n \geq n_0: d(x_m,x_n) < \epsilon.  \]
\end{definition}

\begin{lemma}
Let $(X,d)$ be a metric space. Cauchy sequences in $X$ are bounded.
\end{lemma}

\begin{proposition}
Let $(X,d)$ be a metric space. Convergent sequences in $X$ are Cauchy.
\end{proposition}
Spaces in which the converse holds are special.
\begin{definition}
Let $(X,d)$ be a metric space, then $X$ is called \udef{(Cauchy) complete} if all Cauchy sequences converge.
\end{definition}

\begin{proposition} \label{prop:CauchyCriterion}
Let $(X,d)$ be a metric space and $(a_n), (b_n)$ sequences in $X$. If $(b_n)$ is Cauchy and there exists some $A\in\R$ such that
\[ \forall m,n\in\N: d(a_n,a_m) \leq A d(b_n,b_m), \]
then $(a_n)$ is also Cauchy.
\end{proposition}

\begin{proposition}[Completeness criterion] \label{prop:completenessCriterion}
Let $(X,d)$ be a metric space and $S\subset X$ a dense subset. If every Cauchy sequence in $S$ converges in $X$, then $X$ is complete.
\end{proposition}
This proposition depends on the axiom of countable choice.
\begin{proof}
TODO
\end{proof}

\begin{lemma}
The real numbers with the standard topology are complete.
\end{lemma}

\subsubsection{Completion}
\begin{proposition}
Let $(X,d_X)$ be a metric space. There exists a complete metric space $(Y,d_Y)$ and an isometry $\pi:X\hookrightarrow Y$ such that $\pi[X]$ is a dense subspace of $Y$.
\end{proposition}
We view $X$ as a subspace of $Y$ through $\pi$ and call $Y$ the \udef{completion} of $X$.
\begin{proof}
Let $Y'$ be the space of Cauchy sequences in $X$. Introduce the equivalence relation on $\seq{x_i},\seq{y_j}\in Y'$:
\[ \seq{x_i} \sim \seq{y_i} \qquad \iff\qquad \lim_{i\to\infty} d_X(x_i,y_i) = 0. \]
Let $Y$ be the set of equivalence classes in $Y'$ under this equivalence relation. Define
\[ d_Y: Y\times Y\to \R: ([\seq{x_i}],[\seq{y_i}]) \mapsto \lim_{i\to\infty}d_X(x_i,y_i) \qquad\text{and}\qquad \pi: X\to Y: x\mapsto \seq{x}_i. \]
We need to show that $d_Y$ is well-defined, that it is a metric on $Y$, that $\pi[X]$ is dense in $Y$ and that $(Y,d_Y)$ is complete:
\begin{itemize}
\item Let $[\seq{x_i'}] = [\seq{x_i}]$. Then
\begin{align*}
d_Y([\seq{x_i'}],[\seq{y_i}]) &= \lim_{i\to\infty}d_X(x'_i,y_i) = \lim_{i\to\infty}d_X(x'_i,y_i) + \lim_{i\to\infty}d_X(x_i,x'_i) \\
&= \lim_{i\to\infty}d_X(x_i,x_i')+d_X(x'_i,y_i) \\
&\geq \lim_{i\to\infty}d_X(x_i,y_i) = d_Y([\seq{x_i}],[\seq{y_i}]).
\end{align*}
Similarly we can show $d_Y([\seq{x_i'}],[\seq{y_i}])\leq d_Y([\seq{x_i}],[\seq{y_i}])$, so $d_Y([\seq{x'_i}],[\seq{y_i}]) = d_Y([\seq{x_i}],[\seq{y_i}])$.

We must also show that the domain and codomain of $d_Y$ make sense, i.e. the limit exists and does not diverge. It is enough to show that $(d_X(x_i,y_i))$ is a Cauchy sequence, due to the completeness of $\R$. To this end, let $\epsilon>0$. As $\seq{x_i}$ and $\seq{y_i}$ are Cauchy, we can find $N_x,N_y\in\N$ such that $d_X(x_m,x_n)< \epsilon/2$ and $d_X(y_m,y_n) < \epsilon/2$ for all $m,n \geq N_x,N_y$. Then $\forall m,n \geq \max\{N_x,N_y\}$:
\begin{align*}
|d_X(x_m,y_m) - d_X(x_n,y_n)| &\leq |d_X(x_m,x_n)+d_X(x_n,y_m) - d_X(x_n,y_n)| \\
&\leq |d_X(x_m,x_n)+d_X(y_m,y_n)+ d_X(y_n,x_n) - d_X(x_n,y_n)| \\
&= |d_X(x_m,x_n)+d_X(y_m,y_n)| \\
&< \epsilon/2 + \epsilon/2= \epsilon.
\end{align*}
So $(d_X(x_i,y_i))$ is Cauchy and thus converges in $\R$.
\item That $d_Y$ is a metric is easy to check.
\item To prove $\pi[X]$ is dense in $Y$, we just need to show that every element $y = [\seq{x_i}]\in Y$ is the limit of a sequence in $\pi[X]$, because all metric spaces are sequential. We claim $\seq{\pi(x_j)}_j$ converges to $y$.

Let $\epsilon>0$. Because $\seq{x_i}$ is Cauchy, we can find an $N\in\N$ such that $\forall m,n>N: d_X(x_m,x_n) < \epsilon/2$. Take $j\geq N$ arbitrary. Then
\[ \forall i\geq N: d_X(x_i,x_j) < \epsilon/2 \quad \implies\quad \lim_{i\to\infty} d_X(x_i,x_j) = d_Y([\seq{x_i}],\seq{\pi(x_j)}_j) \leq \epsilon/2  < \epsilon. \]
\item For completeness, it is enough, by \ref{prop:completenessCriterion}, to show that Cauchy sequences in $\pi[X]$ converge in $Y$.

Let $\seq{\pi(x_j)}_j$ be a Cauchy sequence in $\pi[X]$, then $\seq{x_i}$ is Cauchy in $X$ because $\pi$ is isometric. So $\seq{\pi(x_j)}_j$ converges to $\seq{x_i}$ by the previous point.
Let $\seq{\pi(x_j)}_j$ be a Cauchy sequence in $\pi[X]$, then $\seq{x_i}$ is Cauchy in $X$ because $\pi$ is isometric. So $\seq{\pi(x_j)}_j$ converges to $\seq{x_i}$ by the previous point.
\end{itemize}
\end{proof}

\begin{proposition}
Let $(X,d)$ be a metric space. The completion $(Y,\pi)$ of $X$ is unique in the following sense: for any other such completion $(Y',\pi')$, there exists a unique isometric isomorphism $\theta:Y\to Y'$ satisfying $\theta\circ \pi = \pi'$.
\end{proposition}
\begin{proof}
Since $\pi$ is an isometry, it is injective, so $\pi^{-1}:\pi[X]\to X$ is a surjective isometry and so $\pi'\circ\pi^{-1}:\pi[X]\to\pi'[X]$ is too. Now we must have $\theta|_{\pi[X]} = \pi'\circ\pi^{-1}:\pi[X]\to\pi'[X]$ 

TODO universal property!!
\end{proof}

\subsection{Equicontinuity}
Cfr uniform limit theorem
\begin{definition}
Let $(X,d_X)$ and $(Y,d_Y)$ be metric spaces and let $\mathcal{F}$ be a family of functions in $(X\to Y)$.
We say
\begin{itemize}
\item $\mathcal{F}$ is an \udef{equicontinuous family} at $x'\in X$ if
\[ \forall \epsilon> 0: \forall x\in X: \exists \delta>0: \forall f\in\mathcal{F}:\; d_X(x,x') < \delta \implies d_Y(f(x),f(x')) < \epsilon; \]
\item $\mathcal{F}$ is a \udef{uniform equicontinuous family} at $x'\in X$ if
\[ \forall \epsilon> 0: \exists \delta>0: \forall x\in X: \forall f\in\mathcal{F}:\; d_X(x,x') < \delta \implies d_Y(f(x),f(x')) < \epsilon. \]
\end{itemize}
We say $\mathcal{F}$ is (uniformly) equicontinuous if it is (uniformly) equicontinuous at all $x'\in X$.
\end{definition}

\begin{itemize}
\item For continuity, $\delta$ may depend on $\epsilon,f,x_0$;
\item For uniform continuity, $\delta$ may depend on $\epsilon$ and $f$;
\item For pointwise equicontinuity, $\delta$ may depend on $\epsilon$ and $x_0$;
\item For uniform equicontinuity, $\delta$ may depend only on $\epsilon$.
\end{itemize}

TODO: generalise to $X$ general topological space (esp for TVSs).

\begin{proposition}
Let $(f_n)_{n\in\N}$ be an equicontinuous family between metric spaces. If $f_n\to f$ pointwise, then $f$ is continuous.
\end{proposition}
\begin{proof}
TODO
\end{proof}

TODO Reed / Simon

\subsection{Pseudometric spaces}
\begin{definition}
Let $X$ be a set.
\begin{itemize}
\item A map $p: X\times X\to \R_{\geq 0}$ is called a \udef{pseudometric} or \udef{semimetric} if
\begin{itemize}
\item $\forall x\in X: p(x,x) = 0$;
\item $\forall x,y\in X: p(x,y) = p(y,x)$;
\item $\forall x,y,z\in X: p(x,z)\leq p(x,y)+p(y,z)$.
\end{itemize}
Unlike a metric space, points in a pseudometric space need not be distinguishable; that is, one may have $d(x,y)=0$ for distinct values $x\neq y$.
\item The pair $(X,p)$ is called a \udef{pseudometric space}.
\item The \udef{pseudometric topology} is the topology generated by the basis of open balls
\[ B_p(x_0, \epsilon) = \setbuilder{x\in X}{p(x_0,x)<\epsilon}. \]
A topological space is said to be a \udef{pseudometrizable space} if it can be given a pseudometric such that the pseudometric topology coincides with the given topology on the space.
\end{itemize}
\end{definition}

\section{The product topology}
\subsection{Finite Cartesian products}
\begin{definition}
The \udef{product topology} on $X\times Y$ is the topology having as basis the collection $\mathcal{B}$ of all sets of the form $U\times V$, where $U$ is an open subset of $X$ and $V$ is an open subset of $Y$.
\end{definition}
\begin{lemma} \label{lemma:basisFiniteProductTopology}
If $\mathcal{B}$ is a basis for the topology of $X$ and $\mathcal{C}$ a basis for the topology of $Y$, then
\[ \mathcal{D} = \{ B\times C\;|\; B\in \mathcal{B}\;\text{and}\; C\in \mathcal{C} \} \]
is a basis for the topology of $X\times Y$.
\end{lemma}
\begin{proposition}
Let $A$ be a subspace of $X$ and $B$ a subspace of $Y$. The product topology on $A\times B$ is the same as the subspace topology on $A\times B$, when viewed as a subset of $X\times Y$.
\end{proposition}

\begin{definition}
Let $X,Y$ be topological spaces. The maps
\begin{align*}
&\pi_1: X\times Y\to X: (x,y)\mapsto x
&\pi_2: X\times Y\to Y: (x,y)\mapsto y
\end{align*}
are called the \udef{projections} of $X\times Y$ onto its first and second factors, respectively.
\end{definition}
\begin{proposition}
The collection
\[ \mathcal{S} = \{ \pi_1^{-1}(U)\;|\; U\;\text{open in}\; X \}\cup \{ \pi_2^{-1}(V)\;|\; V\;\text{open in}\;Y  \} \]
is a subbasis for the product topology on $X\times Y$.
\end{proposition}
\begin{proof}
Let $\mathcal{T}$ denote the product topology on $X\times Y$; let $\mathcal{T'}$ be the topology generated by $\mathcal{S}$.
\begin{itemize}[leftmargin=2cm]
\item[$\boxed{\mathcal{T}'\subset\mathcal{T}}$] We need to prove all elements of $\mathcal{S}$ are open. Indeed $\pi_1^{-1}(U) = U\times Y$ is open and $\pi_2^{-1}(V) = X\times V$ is also open.
\item[$\boxed{\mathcal{T}\subset\mathcal{T}'}$] Let $B\times C$ be an element of the basis, in other words $B\subset X$ and $C\subset Y$ are open. Then $B\times C = \pi_1^{-1}(B)\cap \pi_2^{-1}(C)$.
\end{itemize}
\end{proof}
In particular $\pi_1$ and $\pi_2$ are continuous.
\begin{proposition}\label{prop:continuityCompositeFunctions}
Let $A,X,Y$ be topological spaces and let
\[ f:A\to X\times Y: a\mapsto f(a) = (f_1(a),f_2(a)). \]
Then $f$ is continuous \textup{if and only if} the functions $f_1$ and $f_2$ are continuous.
\end{proposition}
There is no useful criterion for the continuity of a map $f:A\times B \to X$.

\begin{proposition}
Let $X,Y$ be metrisable topological spaces with metrics $d_X$ and $d_Y$. Then $X\times Y$ is metrisable. Possible, equivalent, metrics include
\[ d_\text{max} = \max\circ \{d_X, d_Y\} \]
and
\[ d_\text{graph} = d_X \circ \pi_1 + d_Y \circ \pi_2. \]
\end{proposition}
\begin{proof}
We first prove that the product topology and the metric topology generated by $d_\text{max}$ are the same using \ref{lemma:basisCoarseness}.

First take an element of a basis for the product topology, which by \ref{lemma:basisFiniteProductTopology} can be taken of the form
\[ B = B_{d_X}(x, \epsilon_1)\times B_{d_Y}(y, \epsilon_2) \qquad \text{for some $x\in X, y\in Y, \epsilon_1,\epsilon_2 >0$.} \]
Then we can find a basiselement $B_{d_\text{max}}((x,y), \min\{\epsilon_1,\epsilon_2\})$ of the metric topology generated by $d_\text{max}$ that is a subset.

Conversely, take $B_{d_\text{max}}((x,y), \epsilon)$. Then $B_{d_X}(x, \epsilon)\times B_{d_Y}(y, \epsilon)$ is a subset.

The equivalence of the two metrics can then be seen by applying \ref{lemma:ballsCoarseness} twice:
\[ B_{d_\text{max}}((x,y), \epsilon) \subset B_{d_\text{graph}}((x,y), \epsilon) \qquad B_{d_\text{graph}}((x,y), \epsilon/2) \subset B_{d_\text{max}}((x,y), \epsilon). \]
\end{proof}
\begin{corollary} \label{corollary:convergenceFiniteProductTopology}
A sequence $(x_n,y_n)_n$ converges to $(x,y)$ in the product topology \textup{if and only if} $(x_n)_n$ converges to $x$ and $(y_n)_n$ converges to $y$.
\end{corollary}
TODO:also nets?

\subsection{Arbitrary Cartesian products}
\begin{definition}
Let $X = \prod_{\alpha\in J}X_\alpha$ and define
\[ \mathcal{S}_\beta = \{\pi_\beta^{-1}(U_\beta)\;|\; U_\beta\;\text{open in}\;X_\beta\} \qquad \text{and}\qquad \mathcal{S} = \bigcup_{\beta\in J}.\]
Then the topology on $X$ generated by the subbasis $\mathcal{S}$ is the \udef{product topology} and then $X$ is called a \udef{product space}.
\end{definition}
\begin{lemma}
\begin{itemize}
\item The product topology on $\prod X_\alpha$ has as a basis all sets of the form $\prod_\alpha U_\alpha$, where $U_\alpha$ is open in $X_\alpha$ for all $\alpha$ and $U_\alpha = X_\alpha$ except for finitely many values of $\alpha$.
\item If each $X_\alpha$ has a basis $\mathcal{B}_\alpha$, a basis for the product topology is given by all the sets of the form $\prod_{\alpha\in J}B_\alpha$ where $B_\alpha\in\mathcal{B}_\alpha$ for finitely many values of $\alpha$ and $B_\alpha = X_\alpha$ for the rest.
\end{itemize}
\end{lemma}
If we remove the condition that $U_\alpha = X_\alpha$ except for finitely many values of $\alpha$, we get the box topology.

Some results that held for finite Cartesian product also hold for arbitrary products:
\begin{lemma}
Let the topology on $\prod X_\alpha$ be the product topology.
\begin{itemize}
\item If each space $X_\alpha$ is Hausdorff, then $\prod X_\alpha$ is Hausdorff.
\item Let $A_\alpha$ be subsets of $X_\alpha$, then
\[ \prod \bar{A}_\alpha = \overline{\prod A_\alpha}. \]
\item Let $A_\alpha$ be subspaces of $X_\alpha$, for each $\alpha\in J$. Then $\prod A_\alpha$ is a subspace of $\prod X_\alpha$ if both products are given the product topology.
\end{itemize}
\end{lemma}
\begin{proposition}
Let $\prod X_\alpha$ have the product topology. Let $f:A\to \prod X_\alpha$ be given by
\[ f(a)=(f_\alpha(a))_{\alpha\in J} \qquad \text{where $f_\alpha = \pi_\alpha\circ f:A\to X_\alpha$ for each $\alpha \in J$}.\]
Then $f$ is continuous \textup{if and only if} each function $f_\alpha$ is continuous.
\end{proposition}
This does not hold for the box topology.
TODO: Universal mapping property; generalise to initial topologies.
\begin{corollary} \label{corollary:productInclusionsContinuous}
Assume we have some points $c_i \in X_i$ for all $i\in J$. Then the functions
\[ i_\alpha: X_\alpha \to X: p \mapsto \left(\begin{cases}
c_i & (i\neq \alpha) \\ p & (i = \alpha)
\end{cases}\right)_{i\in J} \]
are continuous.
\end{corollary}
\begin{proof}
Consider $i_\alpha$. Then for $i = \alpha$, the function $\pi_i \circ i_\alpha: X_\alpha \to X_i$ is the identity in $X_\alpha$ and thus continuous, \ref{prop:continuousConstructions}. For $i \neq \alpha$, the function $\pi_i \circ i_\alpha: X_\alpha \to X_i$ is a constant function $p \mapsto c_i$ and thus continuous, \ref{prop:continuousConstructions}.
\end{proof}

\begin{lemma}
Given points $\vec{x}=(x_i)_{i\in \N}$ and $\vec{y}=(y_i)_{i\in \N}$ of $\R^\N$, define the metric
\[ D(\vec{x}, \vec{y}) = \sup\left\{\frac{\bar{d}(x_i,y_i)}{i}\;|\; i\in \N\right\}, \]
where $\bar{d}$ is the standard bounded metric on $\R$. Then $D$ induces the product topology on $\R^\N$.
\end{lemma}
\begin{proof}
Let $\mathcal{T}$ denote the product topology on $\R^\N$ and $\mathcal{T}_D$ the topology induced by $D$. We prove two inclusions using lemma \ref{lemma:basisCoarseness}.
\begin{itemize}[leftmargin=2cm]
\item[$\boxed{\mathcal{T}_D\subset\mathcal{T}}$] Choose arbitrary basis element $B_D(\vec{x},\epsilon)$. Then choose an $N\in\N$ such that $1/N<\epsilon$. Take the basis element
\[ V = ]x_1-\epsilon,x_1+\epsilon[\;\times\;]x_1-\epsilon,x_1+\epsilon[\;\times \ldots\times\; ]x_N-\epsilon, x_N+\epsilon[\;\times \R\times\R\times \ldots \]
for the product topology. We assert that $V\subset B_D(\vec{x},\epsilon)$. Indeed, for all $\vec{y}\in\R^\N$,
\[ \frac{\bar{d}(x_i,y_i)}{i} \leq \frac{1}{i} \leq \frac{1}{N} \qquad \text{if $i\geq N$}. \]
Therefore,
\[ D(\vec{x},\vec{y}) \leq \max\left\{ \frac{\bar{d}(x_1,y_1)}{1},\frac{\bar{d}(x_2,y_2)}{2},\ldots, \frac{\bar{d}(x_N,y_N)}{N}, \frac{1}{N} \right\}. \]
So if $\vec{y}\in V$, then $D(\vec{,\vec{y}})< \epsilon$ and $V\subset B_D(\vec{x},\epsilon)$.
\item[$\boxed{\mathcal{T}\subset\mathcal{T}_D}$] Choose an arbitrary basis element $U = \prod U_i$. Let $U_i=\R$ if $i\notin \{\alpha_1,\ldots, \alpha_n\}$. For each $i\in \{\alpha_1,\ldots, \alpha_n\}$ choose an interval $]x_i-\epsilon_i,x_i+\epsilon_i[\subset U_i$ and define
\[ \epsilon = \min\{\epsilon_i/i\;|\;i=\alpha_1,\ldots, \alpha_n\}. \]
We can easily see that $B_D(\vec{x},\epsilon) \subset U$.
\end{itemize}
\begin{corollary}
Countable products of metrisable spaces are metrisable.
\end{corollary}
\begin{corollary}
Countable products of Hausdorff spaces are Hausdorff.
\end{corollary}
\end{proof}
\begin{lemma}
The product $\R^J$, with $J$ an uncountable index set, is not metrisable.
\end{lemma}
\begin{proof}
In a metrisable space, by proposition \ref{prop:sequenceLemma}, we have that if $x\in \bar{A}$, then there exists a sequence of points in $A$ converging to $x$. We construct a counterexample. Let $A$ be the subset of $\R^J$ containing all points $(x_i)_{i\in J}$ such that $x_i=1$ for all but finitely many $i$. Now the point $(0)_{i\in J}$ is in the closure of $A$, but has no sequence in $A$ converging to it. To see that it is in the closure, let $\prod U_\alpha$ be a basis element containing $(0)_{i\in J}$. The intersection $A\cap \prod U_\alpha$ is never empty. Indeed for only finitely many $\alpha$, $U_\alpha\neq \R$. Set $x_\alpha = 0$ for these $\alpha$ and $x_i = 1$ for the rest.
\end{proof}
\subsection{Box topology}
\begin{definition}
Let $X = \prod_{\alpha\in J}X_\alpha$ and take as a basis for a topology the collection of all sets of the form
\[ \prod_{\alpha\in J}U_\alpha \qquad \text{($U_\alpha$ open in $X_\alpha$)}. \]
The topology generated by this basis is the \udef{box topology}.
\end{definition}
The following properties hold, like in the product topology:
\begin{lemma}
Let the topology on $\prod X_\alpha$ be the box topology.
\begin{itemize}
\item If each space $X_\alpha$ is Hausdorff, then $\prod X_\alpha$ is Hausdorff.
\item Let $A_\alpha$ be subsets of $X_\alpha$, then
\[ \prod \bar{A}_\alpha = \overline{\prod A_\alpha}. \]
\item Let each $X_\alpha$ have a basis $\mathcal{B}_\alpha$. The collection of all the sets of the form 
\[ \prod_{\alpha\in J}B_\alpha \qquad B_\alpha\in\mathcal{B}_\alpha \]
serves as a basis for the box topology.
\item Let $A_\alpha$ be subspaces of $X_\alpha$, for each $\alpha\in J$. Then $\prod A_\alpha$ is a subspace of $\prod X_\alpha$ if both products are given the box topology.
\end{itemize}
\end{lemma}
\subsubsection{Failure of metrisability}
\begin{lemma}
$\R^\omega$ is not metrisable in the box topology.
\end{lemma}
\subsubsection{Failure of continuity}
\subsubsection{Failure of compactness}

\section{The quotient topology}
Let $X$ be a topological space. A quotient set can always be defined by a surjective function $f:X\to A$ to a set $A$. Then $A$ can be identified with a partition $X^*$ of $X$. Now we would like to define a topology on the partition. We can think of the quotient as shrinking each partition to a single point. Thus it is natural to call a subset of $A$ open if the union of the corresponding partitions is open:
\[ \text{$V$ is open in $A$}\quad \Leftrightarrow_{\text{def}}\quad \text{$p^{-1}(V)$ is open in $X$}. \]
This gives us the following definition:
\begin{definition}
Let $X,Y$ be topological spaces and $p:X\to Y$ a surjective map. The map $p$ is a \udef{quotient map} if
\[ \text{$V$ is open in $Y$} \iff \text{$p^{-1}(V)$ is open in $X$} \]
\end{definition}
This condition is stronger than continuity.
\begin{definition}
Let $X$ be a topological space.
\begin{itemize}
\item Let a be $A$ a subset, and $p:X\to A$ a surjective map. There exists exactly one topology on $A$ relative to which $p$ is a quotient map; it is called the \udef{quotient topology} induced by $p$.
\item Let $X^*$ be a partition of $X$ and $p:X\to X^*$ the surjective map that carries each point of $X$ to its partition. In the quotient topology induced by $p$, the space $X^*$ is called a \udef{quotient space} of $X$.
\end{itemize}
\end{definition}
We can also characterise the notion of quotient map in another way, starting from the following definition:
\begin{definition}
A subset $C$ of a topological space $X$ is \udef{saturated} with respect to a surjective map $p:X\to Y$ if $C$ is the complete inverse image of a subset of $Y$, i.e. it contains every set $p^{-1}(\{y\})$ that it intersects.
\end{definition}
\begin{lemma}
A surjective map $p$ is a quotient map \textup{if and only if} $p$ is continuous and maps saturated open sets of $X$ to open sets of $Y$.
\end{lemma}
\begin{corollary}
\begin{itemize}
\item Surjective continuous open maps are quotient maps.
\item Surjective continuous closed maps are quotient maps.
\end{itemize}
\end{corollary}
There are quotient maps that are neither open or closed.
\begin{proposition}
Let $p:X\to Y$ be a quotient map; let $A$ be a subset of $X$ that is saturated w.r.t. $p$; let $q:A\to p(A) = p|_{A}$.
\begin{enumerate}
\item If $A$ is either open or closed in $X$, then $q$ is a quotient map.
\item If $p$ is either an open map or a closed map, then $q$ is a quotient map.
\end{enumerate}
\end{proposition}
\begin{lemma}
Let $p,q$ be quotient maps.
\begin{enumerate}
\item A composite $q\circ p$ of quotient maps is a quotient map.
\item The product $p\times q$ is not necessarily a quotient map.
\item A quotient space of a Hausdorff space is not necessarily Hausdorff.
\end{enumerate}
\end{lemma}
\begin{proof}
Point (1) follows from
\[ p^{-1}(q^{-1}(U)) = (q\circ p)^{-1}(U). \]
\end{proof}
TODO: theorem 22.2 + corollary
\section{Topologies of $\R$}
\begin{definition}
The \udef{lower limit topology} on $\R$ is the topology generated by the basis
\[ \{ [a,b[ \;|\; a< b \}. \]
\end{definition}
\begin{definition}
\begin{itemize}
\item Let $K$ denote the set
\[ K = \{1/n \;|\; n\in \mathbb{N}_0\}. \]
\item The \udef{$K$-topology} on $\R$ is the topology generated by the basis
\[ \{ ]a,b[ \;|\; a< b \}\;\cup\;\{ ]a,b[\setminus K \;|\; a< b \}. \]
\end{itemize}
\end{definition}
\begin{lemma}
The lower limit and $K$-topologies are strictly finer than then standard topology on $\R$, but are not comparable with one another.
\end{lemma}
\begin{proposition}
Let $X$ be a topological space and $f,g:X\to \R$ continuous functions, then
\begin{enumerate}
\item $f+g, f-g$ and $f\cdot g$ are continuous.
\item If $g(x)\neq 0$ for all $x\in X$, then $f/g$ is continuous.
\end{enumerate}
\end{proposition}
\begin{proof}
First define the map
\[ h:X\to \R\times\R: x\mapsto (f(x),g(x)) \]
which is continuous by proposition \ref{prop:continuityCompositeFunctions}. The functions $f+g,f-g,f\cdot g, f/g$ are the composition of $h$ and the continuous functions $+,-,\cdot,/$.
\end{proof}

\section{Uniform topology}
\begin{definition}
Given an index set $J$ and points $\vec{x}=(x_i)_{i\in J}$ and $\vec{y}=(y_i)_{i\in J}$ of $\R^J$. Define the metric $\bar{\rho}$ on $\R^J$ by
\[ \bar{\rho}(\vec{x}, \vec{y}) = \sup\{\bar{d}(x_i,y_i)\;|\; i\in J\}, \]
where $\bar{d}$ is the standard bounded metric on $\R$. The metric space $(\R^J, \bar{\rho})$ is the \udef{uniform topology} on $\R^J$ and $\bar{\rho}$ is the \udef{uniform metric}.
\end{definition}
\begin{proposition}
On the set $\R^J$, the box topology is finer than the uniform topology is finer than the product topology. These topologies are all different \textup{if and only if} $J$ is infinite.
\end{proposition}

\section{Set-theoretic topology}
\url{https://en.wikipedia.org/wiki/Set-theoretic_limit}
\url{https://math.stackexchange.com/questions/3384916/topology-of-set-theoretic-limits}
\url{https://math.stackexchange.com/questions/2799181/is-there-a-way-to-express-the-set-theoretic-limit-in-terms-of-topology-filters}
\url{https://math.stackexchange.com/questions/97440/do-limits-of-sequences-of-sets-come-from-a-topology}
\url{https://math.stackexchange.com/questions/1947171/the-topology-of-sets}

\section{Initial and final topologies}

\chapter{More topological constructions}
\section{Fibre bundles}
\section{Cones and suspensions}
\section{Wedge sum and smash product}