\chapter{Natural numbers}
The obvious place to start with numbers is to view them as a way of counting things. So we want an abstract idea to keep track of a number of objects, like three sheep or five bagels. We call the set of all such denominations of quantity the natural numbers. We all have an intuitive understanding of how these so-called numbers work, but it can be elucidating to elaborate some of their properties.

First we need to make sure we are talking about. To do that we will enumerate some (intuitively obvious) properties and then make the case that these properties render the natural numbers effectively unique. Technically these properties are known as the \udef{Peano axioms}.

\begin{enumerate}
\item \textbf{$0$ is a natural number}.

This serves basically to give a name to (i.e. assign a glyph to) the special number $0$ which will serve as the first number. One can also decide to start counting at $1$, the axioms remain the same, only the labels differ. The definitions of the addition and multiplication operators would also need to be changed slightly. Also this axiom ensures that the empty set does not satisfy the axioms.

\item \textbf{Every natural number has a successor}.

This axiom states that we can always do an operation to a number and get a new number; it also calls this new number the successor. If we have a number labeled $n$, we write the successor as $S(n)$. We do not know anything much about the operation yet. We have only said it exists. There are many operations that would fit the axioms so far. We need some more axioms to force the successor operation to correspond to adding one to a number.

We can however already form chains of numbers. We know $0$ exists and we know it has a successor due to the second axiom. Then there is also a successor to the successor of $0$ and a successor to the successor to the successor of $0$ and so on, ad infinitum. At least we want it to go on ad infinitum; we have not yet specified the chain cannot make a loop. Maybe the successor of the successor of the ... of $0$ is again $0$. Obviously counting does not work this way and we want to exclude this case, hence axiom 3. It may also be the case that the loop loops back not to the beginning, but to a place further along the chain (e.g. the successor of the successor of the successor of $0$ may equal the successor of $0$). This case is excluded by axiom 4. 

\item \textbf{$0$ is not the successor of any natural number}.

This means a chain cannot loop back to the beginning. It also means a chain always has a beginning and cannot extend infinitely in both directions, as it were.

\item \textbf{If the successor of a number equals the successor of another number, both numbers are equal to each other}.

This means each number has at most one precursor, prohibiting loops from nestling themselves in the chain.  

It may be noted here that we have not really defined equality rigorously. Sometimes four extra axioms are included in the Peano axioms solely to define the equality. The alternative is to assume we are working in \textit{first-order logic with equality}, where equality is already defined.

\item \textbf{There are no natural numbers that are not the successor of the successor of ... of $0$}.

When counting we only want one chain of numbers to follow, namely the one starting at $0$. Without this axiom we might have had multiple chains running in parallel.

This axiom is also called the \emph{Axiom of induction} because it provides the basis of proof by induction. We will not worry about that here; it is a matter for mathematicians.
\end{enumerate}

In order to make it easier to write down, we give each successor of $0$ a name. Thus the successor of $0$ is $1$, the successor of the successor of $0$ is $2$, etc.

Those are the are axioms. We can then give different constructions that satisfy these axioms. There are abstract mathematical structures that satisfy the axioms (like the von Neumann ordinals or the Zermelo ordinals), but we can also take piles of, e.g., beans to represent natural numbers. We start with an empty table (which is $0$) and the successor function adds a bean to the pile. Or we could count on an abacus or using our fingers (assuming we had infinite fingers, beans and beads on our abacus).

These constructions our obviously not the same: a finger is not a bean. We may however reasonably say that they are equivalent. Each construction of natural numbers can be mapped to each other (e.g. one finger equals one abacus bead equals one bean equals ``1'') such that the successor operation is the same in every system. This can intuitively be seen to be true because each construction needs to fit the chain
\[ \begin{tikzcd}0 \rar{} & 1 \rar{} & 2 \rar{} & \ldots\end{tikzcd} \]

This notion of equivalence (that should be very intuitive) has now essentially been defined ad hoc. Luckily it is the application of a more general principle, isomorphism, that we will consider later.

\section{Addition and multiplication of natural numbers}
We can define the addition ($+$) of natural numbers recursively as follows (where $a$ and $b$ are any natural numbers and $S(a)$ is the successor of $a$):
\begin{align}
a + 0 &= a \\
a + S(b) &= S(a+b).
\end{align}
This addition works in the way you would expect. 

We can then define the multiplication ($\cdot$) of natural numbers with the following relations:
\begin{align}
a\cdot 0 &= 0 \\
a\cdot S(b) &= a + (a\cdot b)
\end{align}
This multiplication also works as expected.

Do not worry if these definitions are not clear; some unfamiliar notation has been used. They will not be very useful for us.



\chapter{Constructing more objects we call numbers}
\section{Integers}

The integers are all the whole numbers both positive and negative. Formally they are usually defined using an abstract construction. Working through the details of this is not particularly elucidating for a physicist.

The motivation comes mainly from from trying to provide an operation that is opposite to addition (the opposite of adding three to one is subtracting three from one, giving us minus two, a negative number). This is kind of a hand-wavy explanation, but trying to make it more formal would take too much time, so it will have to suffice.

\section{Rational numbers}
The rational numbers are all possible fractions of integers. Like integers they are often introduced as an abstract construction, the basic idea being to consider all fractions of integers but we need to make sure all equivalent fractions (like $\frac{1}{2}$ and $\frac{2}{4}$) are the same, so equivalence classes are used. Again the details of this construction are not so important for a physicist.

We will however point out that this extension makes it possible to invert the multiplication operation. So with the rational numbers we can invert both the addition and multiplication operations. This will serve as an important component in the justification of real numbers.

\section{Real numbers}
Now we come to the most important type of numbers in physics: the real numbers. There are several (equivalent) constructions: using Dedekind cuts or completing the rational numbers, among others. Here we will define the real numbers axiomatically however, trying to justify why each axiom makes sense. We want real numbers to be the things we use to record measurements.

So as we have observed before, a central part of physics is observing or measuring certain things. To do that we determine ``how big'' the thing is compared to a standard quantity. Thus we need a concept of ``bigness'' or magnitude. Specifically, we only really need a conception of \textit{relative} magnitude, given that we are comparing things to other things. Really we should check that all the base quantities introduced in a later section (such as length, weight, time etc.) work well being modeled by this concept of magnitude being introduced here.  The fact that they can be modeled thusly is part of the hypotheses that we generally tacitly assume. Whether this hypothesis is a good one, is quite a philosophical question.

Now we will list some properties that we would like the real numbers to have if they are to successfully model this concept of magnitude:

\begin{enumerate}
\item \textbf{Addition}.

Firstly we would like to be able to add quantities together. If we run a certain distance and then we run a bit more, we want the total distance to still be expressible using real numbers. We have not yet defined addition for real numbers, but it makes sense to require that any possible definitions have the following properties:
\begin{enumerate}
\item \udef{Associativity}: This may be expressed in symbols in the following way:
\[ a + (b+c) = (a+b)+c \qquad \text{for all real numbers} \;a,b \;\text{and}\; c \]
If we are using real numbers to model the weight of piles of sand, this property asserts that the total weight of three piles does not depend on the order in which we combine the piles.
\item \udef{Commutativity}: Symbolically we write
\[ a + b = b + a \qquad \text{for all real numbers}\; a \; \text{and} \; b. \]
In terms of piles of sand we are asserting that whether we add the first pile to the second one or vice versa the weight does not change.
\item \udef{Identity}: We would like for the real numbers to be able to represent the concept of nothing (i.e. zero). If we add nothing to something, we want that something to remain unchanged:
\[ a + 0 = a \qquad \text{for every real number} \; a \]
\item \udef{Inverse}: Lastly we want a way not only adding sand to the pile, but also taking sand away. We model this by adding a pile of negative weight (represented by a negative number) to the original pile. Since we want to be able to subtract any quantity, we need negative numbers of all possible sizes. More formally we say that every real number, $a$, has an inverse, $-a$, such that
\[ a + (-a) = 0.\]
\end{enumerate}
A quantity being added together with another one is called a \udef{term} of the addition.

\item \textbf{Multiplication}.

We would also like to be able to multiply quantities together. I am struggling to find a simple and intuitive general justification for this. The utility of multiplying real numbers by natural ones I can see: sometimes we add the same thing multiple times. We could then make the case that the inverse of that operation is also useful, forcing us to accept the multiplication of real numbers with rational ones. We could then argue the necessity of completion (as below) in order to justify multiplication of real numbers with real numbers. This is deeply unsatisfying to me. If anybody has a better justification, please let know! It is however a fact that the multiplication of real numbers will prove invaluable to calculate many things. Like with addition we want our notion multiplication (that we have not defined yet) to be \emph{associative}, \emph{commutative}, have an \emph{identity element} (which we call one) and give every element, except zero, a \emph{multiplicative inverse}.

A quantity being multiplied with another one is called a \udef{factor} in the multiplication.

\item \textbf{Distributivity}

We also want the addition and multiplication operations to play nice together. This basically boils down to the \udef{distributivity} of multiplication over addition.
\[ a\cdot (b+c) = (a\cdot b) + (a\cdot c) \qquad \text{for all real numbers} \; a,b \;\text{and}\; c \]
This can be easily justified if we imagine piles of sand. Any set of elements with two binary operations satisfying the properties described above is called a \udef{field}. Thus the whole discussion so far can be summarised by the requirement that the real numbers form a field.

\item \textbf{Order}

Another important property of our notion of magnitude is that it can be ordered. Thus we require that the real numbers have a linear order, which is also known as a total order because part of the definition is the requirement of connexity:
\[ a \leq b \;\text{or}\; b \leq a \qquad \text{for all real numbers}\; a \; \text{and}\; b \]
The rest of the definition does not hold many surprises and may be easily consulted in many other texts. TODO: more in depth.

\item \textbf{Completeness}

Finally we need something else to uniquely define the real numbers. Our axiomatisation so far requires the real numbers to form an ordered field, but there are several of those, including the rational numbers among others. This final piece of the puzzle usually takes the form of some kind of \udef{completeness} property. 

Essentially we want the real numbers to form a continuum; there are several equivalent ways to express this. A detailed analysis of this is can be found in a mathematics book on analysis.

Say for example we have a particular length that we are measuring (and assume that we can measure said length infinitely precisely). We want real numbers to be able to express that length whatever it may be, so there must not be any ``holes'' in the real numbers. The question now is how can we translate that requirement into mathematics as easily as possible? One way to do that is the following: take the set of all real numbers smaller than than the length we want to express (we can do this because we have already required that the real numbers be ordered). We now require that this subset has a least upper bound, also known as a \udef{supremum}. 

Firstly, what is a least upper bound? Well if we take a subset of an ordered set, any element of the superset that is larger than or equal to every element of the subset is an upper bound. The least upper bound or supremum is the smallest upper bound, if it exists. One might initially think that every (bounded, i.e. having an upper bound) subset of an ordered set has a supremum, and for finite sets that is true. Unfortunately for infinite sets the picture is slightly more complicated. The classic counterexample is given by the rational numbers. Take for example the set of rational numbers whose square is less than $2$. We can write that set $S$ as follows
\[ S = \{x\in \mathbb{Q} \;|\; x^2 < 2\}. \]
If we were considering this as a subset of the real numbers, the supremum would be $\sqrt{2}$. This is not a rational number, but this does not mean there cannot be a rational supremum when considering it as a subset of the rational numbers. Like if we were to take the set of the natural numbers whose square is less than two, i.e.
\[ \{ n \in \mathbb{N} \;|\; x^2 < 2 \}. \]
This subset has a supremum in the natural numbers, namely $1$. This is because every natural number whose square is smaller than $2$ is smaller than or equal to $1$ (i.e. $1$ is an upper bound) and all other upper bounds are larger than $1$.

So why does the set $S$ not have a supremum in the rational numbers? The reasoning behind this makes use of the fact that the rationals are dense in the reals. This means that between any two distinct real numbers we can find infinitely many rational ones. So now assume the set $S$ has a supremum $s$ in the rational numbers. This supremum is either smaller than or larger than $\sqrt{2}$, the supremum in the reals. If it were smaller than $\sqrt{2}$, there would be infinitely many rational numbers between $s$ and $\sqrt{2}$ (because the rationals are dense in the reals). The square of any of these numbers (all larger than $s$) is smaller than $2$, so $s$ is not an upper bound. Now if $s$ is larger than $\sqrt{2}$, there are again infinitely many rationals between $\sqrt{2}$ ans $s$. Any of those numbers (who are all smaller than $s$) is an upper bound. This means that $s$ is not the smallest upper bound. So we cannot take the existence of the supremum for granted. If every nonempty bounded subset of a set has a supremum, we say that the set is \udef{complete}. 

Now why does this completeness correspond to the real numbers forming a continuum such that for any length we might want to measure, the real numbers can express such a length? 

TODO

Also non-trivial for eg quantum. Subset good enough??



\end{enumerate}

We can sum up all of these properties in the f

Unique, we automotically get the addition and multiplication operators we know and love. Many different fields with different definitions of operators. One other completion ($p$-adics???), but not ordered??

TODO numberline, subsets, decimal expansion

TODO divide by zero

\section{Complex numbers}
For completeness we also mention the complex numbers exist. We defer a discussion until a later chapter as it will be easier to do with more mathematical tools in out tool belt.

We finish by noting
\[ \mathbb{N} \subset \mathbb{Z} \subset \mathbb{Q} \subset \R \subset \C \]

\chapter{Some mathematical notation to do with numbers and formulae}
Now we have introduced the numbers, we really need to introduce some of the notation used in formulae. We have seen notation for explicitly referencing particular numbers, but sometimes we want to reference a number or a quantity without knowing exactly what it is, like the length of the field in one of our previous examples, or without having to write down the whole decimal expansion. To do that we give that number or quantity a name. In principle the name can be anything, so long as it is well defined in the text. There are however some conventions; once one is familiar with them, they make the formulae much easier to read.

\begin{enumerate}
\item Typically the names are single letters. Usually letters are chosen with some orthographic link to the thing being represented. If we want to give more descriptive names, more letters may be added in subscript. Thus, e.g., we might denote a time something starts at as $t_\text{initial}$, $t_\text{in}$, $t_i$ or $t_0$.

\item If we want a label to refer to a single, fixed number, we call it a \udef{constant} and we typically choose a lowercase letter from the beginning of the alphabet, such as $a, b$ or $c$. Many physical constants have conventional names (sometimes in violation of this rules for mainly historical reasons), such as the speed of light $c$, Newton's gravitational constant $G$ or the mass of an electron $m_e$.

\item Sometimes we want a label to be able to represent different values, usually because we want a general expression we can plug different numbers into depending on the situation, or because we want to consider a range of different values at the same time (such as with the $x$ in the fencing of a field example). In that case we call it a \udef{variable}. The distinction between constants and variables isn't always very clear. Usually lowercase letters from the end of the alphabet, such as $x, y$ and $z$ are used for variables. The letter $t$ is also commonly used as a variable representing time.

\item If we need many variables or the variables have some kind of natural grouping, we can reuse the same letter, indexing with numbers in subscript: $a_1, a_2, \ldots, a_n$, where $n$ is a constant referring to the number of $a$'s. Indexing may also be done with letters or in superscript. If an index is placed in superscript, it is often surrounded by parentheses so it cannot be confused for an exponent (to be introduced later): $a^{(1)}, a^{(2)}, \ldots a^{(n)}$. This notation is also sometimes used to denote higher order derivatives (again introduced later), so some caution is required.

\item We usually explain the meaning of all of the symbols in the text. Sometimes however we will want to perform an assignment using mathematical symbols (such as when we want to label the result of a calculation). In those cases we use the symbol $\equiv$. For example
\[ a \equiv 2.3\]
means from now on $a$ refers to the number $2.3$.

\item For mathematical objects that are not numbers different conventions apply. Often capital letters will be used (such as for sets and matrices for example).
\end{enumerate}

Now we need some symbols to tie the variables and constants together.

\begin{enumerate}
\item Firstly we have addition, as defined above. We denote it using $+$.
\item If we are summing a lot of number together, we use the symbol $\sum$. This is used together with an index to show what we are summing over. For example
\[ \sum_{i=0}^3 (2i) \]
means we are summing together $2i$s for $i$ ranging from $0$ to $3$. Thus this expression equals
\[ \sum_{i=0}^3 (2i) = (2\cdot 0) + (2\cdot 1) + (2\cdot 2) + (2\cdot 3) = 0 + 2 + 4 + 6 = 12. \]
Here the range of $i$s is made explicit. If it is obvious from the context what the range must be, we will sometimes just write
\[ \sum_i 2i. \]
If there is a specific set we are summing over we can write
\[ \sum_{i\in X} 2i.  \]
In this case $X = \{0,1,2,3\}$.
\item Multiplication is denoted by the symbol $\cdot\;$. Often however this symbol will not be written. If there is no other operation between two numbers, multiplication will be assumed:
\[ S_\text{area} = l\cdot w = lw \]
In some contexts the symbol $\times$ will be used as well, in particular when using scientific notation of quantities (as explained below), e.g.
\[ l = 5.6 \times 10^{-9} \si{m} \]
or when multiplying large expressions that might be split across several lines, e.g.
\begin{multline*}
\frac{1}{2}\sum^N_{\substack{k,l=1 \\ k \neq l}} \sum_W \sum_{W'} \int\int \div{x_k}\div{x_l} \psi_{E_k}^\dagger(x_k)\psi_{E_l}^\dagger(x_l) V(x_k,x_l)\psi_W(x_k)\psi_{W'}(x_l) \\ \times C(E_1, \ldots, E_{k-1}, W, E_{k+1}, \ldots, E_{l-1},W',E_{l+1}, \ldots, E_N,t).
\end{multline*}
Do not worry about what this expression means, it is quoted here purely to illustrate the use of $\times$ for multiplication.
\item Repeated multiplication can be written using $\prod$. This is entirely analoguous to $\sum$ for addition.
\item In theory we should really write parentheses around every operation. This makes things less readable, so we conventionally assume that unless told otherwise we first multiply and then add, and remove any parentheses where the order of operations is now unambiguous.
\item Square brackets (i.e. [ and ]) are sometimes used instead of parentheses to add visual clarity, especially when nesting parentheses. These square brackets are also used to denote \udef{intervals}. Because $\R$ is an ordered field, we can meaningfully speak of the \undline{set} of all numbers \textit{between} two numbers, say $a$ and $b$. Now we can choose whether we include the boudary numbers or not. This leads to four types of intervals:
\begin{enumerate}
\item The \textbf{closed interval} $[a,b]$ which contains both $a$ and $b$ as well as all the numbers between;
\item The \textbf{open interval} $]a,b[$ which does not contain $a$ or $b$;
\item The \textbf{half-open interval} $[a,b[$ which contains $a$ but not $b$; and
\item The \textbf{half-open interval} $]a,b]$ which contains $b$ but not $a$.
\end{enumerate}
We can also let the interval extend to infinity (always with an open bracket because infinity itself is not a real number). E.g.
\begin{enumerate}
\item $[a, \infty[$ contains all numbers larger or equal to $a$;
\item $]-\infty, a[$ contains all numbers strictly smaller than $a$;
\item $]-\infty, \infty[$ contains all numbers.
\end{enumerate}
\item We have assumed, axiomatically, that every real number has an additive inverse. We write that using $-\;$. So a number $a$ would have an inverse $-a$. In practice this inverse is often combined with the addition operation, so we will often want to write something like
\[ a + (-b). \]
We will write this more simply as
\[ a-b. \]
\item The multiplicative inverse can be written as $a^{-1}$ or $\frac{1}{a}$. A similar remark applies as with the additive inverse.
\item We will also introduce the equals sign ($=$) here. Mathematically this is a relation (as defined below). Its use is to assert the equality between two expressions. 
\end{enumerate}



\chapter{Some rules of computation}
TODO merkwaardige producten (also complex!), distributivity + inverse