\chapter{The integers  $\Z$}



\chapter{The rational numbers $\Q$}

\begin{proposition}
The ordered set $(\Q,\leq)$ is not complete.
\end{proposition}
\begin{proof}
Consider the set
\[ A = \setbuilder{x\in\Q}{x<0\land x^2<2}. \]
\end{proof}

\begin{proposition}
The rational numbers are embedded in any ordered field $K$ by
\[ \Q \cong \Q\cdot 1 \subset K. \]
Any totally ordered field has characteristic zero.
\end{proposition}

\section{The Archimedean property}
\begin{definition}
A totally ordered field $(K,+,\cdot, \leq)$ is said to have the Archimedean property if $(K,+,\leq)$ is an Archimedean monoid. We say
\begin{itemize}
\item $x$ is \udef{infinitesimal} if $x$ is infinitesimal w.r.t. $1$;
\item $x$ is \udef{infinite} if $x$ is infinite w.r.t. $1$;
\end{itemize}
\end{definition}

\begin{lemma}
Let $(K,+,\cdot, \leq)$ be a totally ordered field. Then
\begin{enumerate}
\item if $x\in K$ is infinitesimal, then $1/x$ is infinite, and vice versa;
\item if $x\in K$ is infinitesimal and $r\in \Q\cdot 1$, then $rx$ is also infinitesimal.
\end{enumerate}
\end{lemma}

\begin{proposition}[Axiom of Archimedes]
Let $(K,+,\cdot, \leq)$ be a totally ordered field. Then the following are equivalent:
\begin{enumerate}
\item $K$ is Archimedean;
\item for all $x\in K$ there exists $n\in\N$ such that $x < n\cdot 1$;
\item for all positive $\varepsilon\in K$ there exists $n\in\N$ such that $(n\cdot 1)^{-1}< \varepsilon$.
\end{enumerate}
\end{proposition}

\begin{lemma}
Let $(K,+,\cdot, \leq)$ be an Archimedean totally ordered field. Then for all $x\in K$ there exists an $n\in\N\cdot 1$ such that
\[ n\cdot 1 \leq x < n\cdot 1 + 1. \]
\end{lemma}



\chapter{The real numbers $\R$}

\begin{proposition}
The field of real numbers is Archimedean.
\end{proposition}
\begin{proof}
Assume, towards a contradiction, that the field of real numbers is Archimedean. Then $\N$ has an upper bound, and by completeness a least upper bound $s = \sup\N$. Then $s-1$ is not an upper bound and thus there exists and $n\in\N$ such that $s-1<n$. But then $s<n+1 \in \N$ so that $s$ is not an upper bound of $\N$, yielding a contradiction.
\end{proof}

\begin{proposition}
There exists a totally ordered field and it is unique up to isomorphism.
\end{proposition}

\begin{proposition}
The rational numbers $\Q$ are dense in $\R$.
\end{proposition}
\begin{proof}
Let $x < y$ be real numbers. By the Archimedean property there exists a natural number $n>(y-x)^{-1}$. TODO: complete.
\end{proof}

\section{Affinely extended real number system}
\begin{definition}
Dedekind–MacNeille completion of the real numbers
\[ \overline{\R} = \R\cup\{-\infty, +\infty\}. \]
\end{definition}


\section{Functions on the real numbers}

\subsection{Functions from reals to integers}
\subsubsection{Rounding}
\subsubsection{Floor and ceiling}
$\floor{x}$ and $\ceil{x}$
\subsubsection{Fractional and integer part}

\section{Irrational numbers}
\subsection{Beatty sequences}
TODO \url{https://en.wikipedia.org/wiki/Lambek%E2%80%93Moser_theorem}
\begin{definition}
Let $r$ be a positive irrational number. The \udef{Beatty sequence} generated by $r$ is the sequence
\[ \mathcal{B}_r: \N \to \N: n\mapsto \floor{nr}  \]
\end{definition}

\begin{lemma}
Let $r$ be a positive irrational number. For a positive integer $m$ the following are equivalent:
\begin{enumerate}
\item $m$ is a term in the Beatty sequence $\mathcal{B}_r$;
\item $1 - \frac{1}{r} < \fractional\left(\frac{m}{r}\right)$;
\item $m = \floor{\left(\floor{\frac{m}{r}}+1\right)}$.
\end{enumerate}
\end{lemma}

\begin{theorem}[Rayleigh]
TODO
\end{theorem}

\begin{theorem}[Uspensky]
If $r_1, \ldots, r_n$ are positive numbers such that the Beatty sequences $\mathcal{B}_{r_1}, \ldots, \mathcal{B}_{r_n}$ partition the positive integers, then $n\leq 2$.
\end{theorem}
This means there is no equivalent to Rayleigh's theorem for more than two sequences.
\begin{proof}
TODO \url{https://mathweb.ucsd.edu/~fan/ron/papers/63_01_uspensky.pdf}
\end{proof}

\subsubsection{Beatty series}
\url{https://math.stackexchange.com/questions/2052179/how-to-find-sum-i-1n-left-lfloor-i-sqrt2-right-rfloor-a001951-a-beatty-s}


\subsection{Games}

\chapter{Complex numbers}
TODO: prove $\C$ cannot be a totally ordered field.
TODO
\[ -1 = i^2 = \sqrt{-1}\sqrt{-1} = \sqrt{(-1)^2} = \sqrt{1} = 1 \]


The set of the complex numbers is denoted $\C$.

\begin{lemma} \label{boundedThenReal}
Let $z\in\C$. Suppose there is a $C\geq 0$ such that
\[ \forall t\in\R: \quad |z+it|^2\leq C+t^2, \]
then $z\in \R$.
\end{lemma}
\begin{proof}
Write $z = a+bi$ for some $a,b\in \R$. Then
\[ |z+it|^2-t^2 = a^2 + (b+t)^2 - t^2 = a^2+b^2+2bt. \]
The left side is bounded by $C$ for all $t\in\R$. If $b>0$, the right side is unbounded for $t\to +\infty$. If $b<0$, the right side is unbounded for $t\to -\infty$. So we need $b=0$ and thus $z=a\in\R$.
\end{proof}

\section{How to represent complex numbers}
In the previous section we saw that every complex number can be written as $a + bi (a,b \in \R)$. Conversely for every $a,b $ in $\R$ there is a unique complex number $a + bi$. Thus we can see that every complex number can be constructed using two real numbers. We give those real numbers special names: For a complex number $z = a + bi$, we call $a$ the real part (denoted $\Re(z)$) and $b$ the complex part (denoted $\Im(z)$).

TODO complex plane

modulus argument
Euler formula??
Conversions

\section{Practical calculations}
The following methods give a practical way to perform calculations with complex numbers. Assume we have two complex numbers $z_1$ and $z_2$. 
\subsection{Addition} is usually easiest if the complex numbers are in the form $a+bi$. Then we have
\begin{align*}
z_1 + z_2 &= (a_1 + b_1i) + (a_2 + b_2i) \\
&= (a_1+a_2) + (b_1+b_2)i
\end{align*} 
\subsection{Multiplication} is usually easiest if the complex numbers are in the form $re^{i\phi}$. Then we have
\begin{align*}
z_1 \cdot z_2 &= r_1e^{i\phi_1}\cdot r_2e^{i\phi_2} \\
&= (r_1\cdot r_2)e^{i(\phi_1+\phi_2)}
\end{align*}
So we multiply the moduli and add the arguments.
\subsection{Exponentiation} with an integer (or real) exponent is again usually easiest if the complex number is in the form $re^{i\phi}$.
\begin{align*}
z^n &= (re^{i\phi})^n \\
&= r^n e^{in\phi}
\end{align*}
\section{Trigonometry revisited}
\subsection{Waves and complex numbers}

Cayley-Dickinson