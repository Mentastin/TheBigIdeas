\chapter{Vector space convergence}
TODO: \url{https://math.stackexchange.com/questions/2001771/existence-of-at-least-one-continuous-coordinate-functional}

\url{https://math.stackexchange.com/questions/60057/does-there-exist-a-linearly-independent-and-dense-subset}

\begin{definition}
Let $\sSet{\F,V,+}$ be a vector space and $\xi$ a convergence on $V$. Then $\sSet{\F,V,+, \xi}$ is a \udef{convergence vector space} (or CVS) if
\begin{itemize}
\item vector addition $+: V\times V \to V$ is continuous;
\item scalar multiplication $\cdot: \F\times V \to V$ is continuous.
\end{itemize}
Let $W$ be another vector convergence space. The set of continuous lineat maps from $V$ to $W$ is denoted $\contLin(V,W)$.
\end{definition}

\begin{lemma}
If $\sSet{\F,V,+, \xi}$ is a convergence vector space, then $\sSet{V,+, 0, \xi}$ is a convergence group.
\end{lemma}
\begin{proof}
We just need to show that $v\mapsto -v$ is continuous, but this scalar multiplication and thus continuous by assumption.
\end{proof}

\begin{lemma} \label{continuityLemmaVectorConvergence}
If $\sSet{\F,V,+, \xi}$ is a convergence vector space, then
\begin{enumerate}
\item the function $V \to V: v \mapsto \lambda\cdot v$ is a homeomorphism for all $\lambda\in \F\setminus\{0\}$;
\item the function $\F \to \Span\{v\}: \lambda \mapsto \lambda\cdot v$ is continuous and invertible for all $v\in V\setminus\{0\}$;
\item the function $\F \to \Span\{v\}: \lambda \mapsto \lambda\cdot v$ is a homeomorphism \textup{if and only if} $\Span\{v\}$ is Hausdorff.
\end{enumerate}
\end{lemma}
TODO picture!
\begin{proof}
The functions $\lambda \mapsto (\lambda, v)$ and $v \mapsto (\lambda, v)$ are continuous by \ref{continuousEmbeddingProduct}. Composition with the continuous scalar product gives the result by continuity of composition (\ref{continuityComposition}).

They are both clearly invertible. (For the second, note that the kernel is $\{0\}$. We can also argue using \ref{scalarMultiplicationBijection}). The inverse of the first is of the same form and thus immediately continuous.

(3) If $u: \F \to \Span\{v\}: \lambda \mapsto \lambda\cdot v$ is a homeomorphism, then $\Span\{v\}$ is Hausdorff, because $\F$ is.

Now assume $\Span\{v\}$ Hausdorff. We need to show that $u^{-1}$ is continuous. It is enough to show continuity at $0$. We use \ref{pretopologicalContinuityVicinities}, so take $\Gamma \in \vicinity_\F(0)$ and we need to show that $\Gamma \in (u^{-1})^{\imf\imf}\big[\vicinity_{\Span\{v\}}(0)\big]$. WLOG we may take $\Gamma = \ball(0, \epsilon)$.

Now consider $\sphere(0,\epsilon)$, which is compact. Then $u^{\imf}\big(\sphere(0,\epsilon)\big) = \sphere(0,\epsilon)\cdot v$ is compact by \ref{compactConstructions} and thus closed by \ref{compactClosedSets} (as $\Span\{v\}$ we assumed Hausdorff). Now $0 \in \big(\sphere(0,\epsilon)\cdot v\big)^c$, so there exists a vicinity $U$ of $0$ disjoint from $\sphere(0,\epsilon)\cdot v$.

For each $F$ that converges to $0$ in $\Span\{v\}$, $\neighbourhood_\F(0)\cdot F$ also converges to $0$, so there exists $\delta_F>0$ and $C_F\in F$ such that $\ball(0, \delta_F)\cdot C_F \subseteq U$. If $\delta_F > 1$, then $C_F \subseteq \ball(0, \delta_F)\cdot C_F \subseteq \ball(0,\epsilon)\cdot v$, so $\ball(0,\epsilon)\cdot v\in F$.

Now assume $\delta_F \leq 1$. Then $\cball(0, \epsilon\cdot\delta_F^{-1})\setminus \ball(0,\epsilon)$ is compact, which, as before, means $\big(\cball(0, \epsilon\cdot\delta_F^{-1})\setminus \ball(0,\epsilon)\big)\cdot v$ is closed and we take a vicinity $U_F$ of $0$ disjoint from it. Now $C_F\cap U_F\in F$. We also claim that $C_F\cap U_F\subseteq \ball(0,\epsilon)\cdot v$: take $x\in C_F\cap U_F$. Then $x = \lambda v$ and $\delta_F |\lambda| \leq \epsilon$ (as $\ball(0, \delta_F)\cdot C_F \subseteq U$), which is equivalent to $|\lambda| \leq \epsilon\cdot\delta_F^{-1}$. Now because $x\in U_F$, this means $|\lambda| < \epsilon$ and thus $x\in \ball(0,\epsilon)\cdot v$.

As $\ball(0,\epsilon)\cdot v \in F$ for all $F$ that converge to $0$, we have $\ball(0,\epsilon)\cdot v\in \vicinity_{\Span\{v\}}$, so $\ball(0,\epsilon) = \Gamma \in (u^{-1})^{\imf\imf}\big[\vicinity_{\Span\{v\}}(0)\big]$, which is what we had to prove.
\end{proof}

TODO: rewrite proof of (3) using balanced neighbourhoods. \url{https://proofwiki.org/wiki/Isomorphism_from_Cartesian_Space_to_Finite-Dimensional_Subspace_of_Hausdorff_Topological_Vector_Space_is_Homeomorphism}.

Alternate proof in Beattie / Butzmann 3.3.19.

\begin{lemma} \label{continuityLinearCombination}
Let $V$ be a convergence vector space over a field $\F$. Let $x_0, \ldots, x_n$ be $n$ vectors in $V$. Then the function
\[ \F^n \to V: (\lambda_0, \ldots, \lambda_{n-1}) \mapsto \sum_{k=0}^{n-1} \lambda_kx_k \]
is continuous.
\end{lemma}
\begin{proof}
Call this function $g$.
For all $k\in \interval{1,n}$, the function $g_k: \F\to V: \lambda \mapsto \lambda x_k$ is continuous by \ref{continuityLemmaVectorConvergence}, so $g = (+)\circ (g_1|\ldots|g_n)$ is continuous by \ref{continuityParallelComposition} and the continuity of the sum.
\end{proof}

\begin{proposition} \label{vectorSpaceConvergenceConstruction}
Let $V$ be a vector space over a field $\F$. And $\mathcal{F} \subseteq \powerfilters(V)$ a family of filters. There exists a vector space convergence $\xi$ on $V$ such that $\mathcal{F} = \lim^{-1}_\xi(0)$ \textup{if and only if}
\begin{enumerate}
\item if $F \in \mathcal{F}$ and $G\supseteq F$, then $G\in \mathcal{F}$;
\item if $F,G \in \mathcal{F}$, then $F + G\in \mathcal{F}$;
\item if $F\in \mathcal{F}$, then $\neighbourhood_\F(0)\cdot F \in \mathcal{F}$;
\item if $v\in V$, then $\neighbourhood_\F(0)\cdot v \in \mathcal{F}$;
\item if $F\in \mathcal{F}$ and $\lambda\in \F$, then $\lambda\cdot F \in \mathcal{F}$.
\end{enumerate}
\end{proposition}
Note the similarity with \ref{groupConvergenceConstruction} for convergence groups. A group convergence is completely determined by $\lim^{-1}_\xi(0)$ due to the translation homeomorphisms \ref{shiftHomeomorphism}.
\begin{proof}
Assume first that $\mathcal{F} = \lim^{-1}_\xi(0)$ for some vector space convergence $\xi$.
\begin{enumerate}
\item This is just the monotonicity of the convergence.
\item If $F,G\to 0$, then $F\otimes G \to (0,0)$ by \ref{convergenceFiniteProductFilter}. By continuity of addition we have $F+G\to 0$.
\item The convergence on the scalar field is pretopological, so $\neighbourhood_\F(0)\to 0$. By \ref{convergenceFiniteProductFilter}, $\neighbourhood_\F(0)\otimes F \to (0,0)$ and by continuity of the scalar multiplication $\neighbourhood_\F(0)\cdot F \to 0$.
\item By \ref{continuityLemmaVectorConvergence}.
\item By \ref{continuityLemmaVectorConvergence}.
\end{enumerate}

Now assume the five points hold. Define the convergence $\xi$ by $F\to v$ iff $F-v \in \mathcal{F}$. We need to show that this is a convergence and that it makes both the vector addition and scalar multiplication continuous.

Monotonicity is guaranteed by (1). To show the convergence is centered, note that $\mathcal{F} \neq \emptyset$ by (4), so long as $V\neq \emptyset$. Then for any $F\in \mathcal{F}$, $\big\{\{0\}\big\} = 0\cdot F \in \mathcal{F}$ by (5).

To show that the vector addition is continuous, take $F\to (v_1, v_2)$. Then $p_1^{\imf\imf}(F) = F_1\to v_1$ and $p_2^{\imf\imf}(F) = F_2 \to v_2$, i.e.\ $F_1-v_1 \in \mathcal{F}$ and $F_2-v_2 \in \mathcal{F}$. By (1), $(F_1-v_1) + (F_2-v_2) = (F_1+F_2) - (v_1 + v_2) \in \mathcal{F}$, so $F_1+F_2 \to v_1 + v_2$. Thus by \ref{filterFactorisationInequality}, $F_1+F_2 = +^{\imf\imf}[F_1\otimes F_2] \subseteq +^{\imf\imf}[F] \to v_1+v_2$ and the addition is continuous.

Let $G \to (\lambda, v)$. Then $G_1 = p_1^{\imf\imf}(G) \to \lambda$ and $G_2 = p_2^{\imf\imf}(G) \to v$, so $G_1 \supseteq \neighbourhood_\F(\lambda)$. We have
\begin{align*}
\cdot^{\imf\imf}[G] - \lambda\cdot v &\supseteq \cdot^{\imf\imf}[G_1\otimes G_2] - \lambda\cdot v = G_1\cdot G_2 - \lambda\cdot v \\
&\supseteq \neighbourhood_\F(\lambda) \cdot G_2 - \lambda\cdot v \\
&= (\neighbourhood_\F(0) + \lambda)\cdot((G_2 - v) + v) - \lambda\cdot v \\
&\supseteq \lambda\cdot (G_2 - v) + \neighbourhood_\F(0)\cdot(G_2-v) + \lambda\cdot v + \neighbourhood_\F(0)\cdot v - \lambda\cdot v \\
&= \lambda\cdot (G_2 - v) + \neighbourhood_\F(0)\cdot(G_2-v) + \neighbourhood_\F(0)\cdot v \in \mathcal{F}.
\end{align*}
So $\cdot^{\imf\imf}[G] \to \lambda\cdot v$, making the scalar multiplication continuous. Note the last inclusion is not an equality because we go from one instance of $G_2$ and $\neighbourhood_\F(0)$ to two!
\end{proof}
\begin{corollary} \label{vicinityFilterAtOrigin}
Let $\sSet{V, \xi}$ be a convergence vector space. Then
\begin{enumerate}
\item for all $A\in \vicinity_\xi(0)$ and $\lambda\in \F\setminus\{0\}$: $\lambda A\in \vicinity_\xi(0)$;
\item each $A \in \vicinity_\xi(0)$ is absorbent;
\item if $\xi$ is topological, then $\neighbourhood_\xi(0)$ has a balanced base;
\item if $\xi$ is topological, then $\neighbourhood_\xi(0)$ has a closed, balanced base.
\end{enumerate}
\end{corollary}
If $\sSet{V,\xi}$ is equable, then $\vicinity_\xi(0)$ has a balanced base, see \ref{equableConvergenceBalancedBase}.

This is not true for all convergence vector spaces. What is true for all convergence vector spaces is that each $A \in \vicinity_\xi(0)$ contains a balanced subset. This is immediate since $\{0\}$ is balanced and $\{0\} \subseteq A$ because $A$ is absorbent.
\begin{proof}
(1) Take arbitrary $\lambda\in\F\setminus\{0\}$. Then $v\mapsto \lambda v$ is a homeomorphism by \ref{continuityLemmaVectorConvergence}, so $\lambda A \in \vicinity_\xi(\lambda 0) = \vicinity_\xi(0)$, by \ref{homeomorphismPreservation}.

(2) For absorbence, take $A\in \vicinity_\xi(0)$ and $v\in V$. As $\neighbourhood_\F(0)\cdot v \to 0$, we must have $A\in \upset\neighbourhood_\F(0)\cdot v$, so there exists $\Gamma \in \neighbourhood_\F(0)$ such that $\Gamma\cdot v \subseteq A$. Now we can find a $r>0$ such that $\ball(0,r)\subseteq \Gamma$, so for all $|c|\geq r^{-1}$ we have $v\in cA$.

(3) By point (3) of of the proposition, $\neighbourhood_\F(0)\cdot \vicinity_\xi(0)$ converges to $0$ and thus $\vicinity_\xi(0) \subseteq \neighbourhood_\F(0)\cdot\vicinity_\xi(0)$. Take $A\in \vicinity_\xi(0)$. Then there exists a $\Gamma\in\neighbourhood_\F(0)$ and $B\in \vicinity_\xi(0)$ such that $\Gamma\cdot B \subseteq \vicinity_\xi(0)$. We can find some ball $\ball(0,\epsilon) \subseteq \Gamma$, so $\ball(0,1)\cdot \epsilon B\subseteq \epsilon B \subseteq A$. Thus $\epsilon B$ is balanced and a neighbourhood by point(1). So every $A\in \vicinity_\xi(0)$ contains a balanced set in $\vicinity_\xi(0)$.

(4) Take arbitrary $A\in \neighbourhood_\xi(0)$. By regularity, \ref{topologicalGroupsRegular}, $A$ contains a closed neighbourhood $B$. By (3), $B$ contains a balanced neighbourhood $C$. Now consider $\closure_\xi(C)$, which is closed and a subset of $B$ (as $\closure(C)\subseteq \closure(B) = B$). It is now enough to note that the closure of a balanced set is balanced: if $x$ is the limit of a filter $F$ in $B$, then $r\cdot x$ is the limit of $r\cdot F$ for all $|r|\leq 1$. Now $r\cdot F$ is a filter in $B$ by balance.
\end{proof}

\begin{proposition} \label{vectorSumInherenceAdherence}
Let $\sSet{V,\xi}$ be a vector space convergence and $A,B\subseteq V$. Then
\begin{enumerate}
\item $\adh(A)+\adh(B) \subseteq \adh(A+B)$;
\item $\inh(A)+\inh(B) \subseteq A+\inh(B) \subseteq \inh(A+B)$;
\item $\interior(A)+\interior(B) \subseteq A+\interior(B) \subseteq \interior(A+B)$.
\end{enumerate}
\end{proposition}
TODO: same for closure?
\begin{proof}
(1) We use \ref{productAdherence} and \ref{adherenceInherenceContinuity} to compute
\[ \adh(A)+\adh(B) = +^\imf[\adh(A)\times\adh(B)] = +^\imf[\adh(A\times B)] \subseteq \adh(+^\imf[A\times B]) = \adh(A+B). \]

(2) The inclusion $\inh(A)+\inh(B) \subseteq A+\inh(B)$ is immediate. Now for all $v\in V$ we have $v+\inh(B) = \inh(v+B)$, so
\[ A+\inh(B) = \bigcup_{v\in A}v+\inh(B) = \bigcup_{v\in A}\inh(v+B) \subseteq \inh\left(\bigcup_{v\in A} v+B\right) = \inh(A+B), \]
where we have used the monotonicity of $\inh$ and \ref{orderPreservingFunctionLatticeOperations}.

(3) Similar to (2).
\end{proof}
Notice that the argument used for (2) and (3) does not work for the adherence because $\adh(A)+\adh(B) \nsubseteq A+\adh(B)$ in general.

\begin{lemma}
Let $V$ be a convergence vector space over a field $K$ and $F\subseteq K$ a subfield. Then the $F$-vector space $V_F$ with the same convergence structure is also a convergence vector space.
\end{lemma}
\begin{proof}
It is enough that the restriction of the scalar multiplication to $F$ remains continuous. Alternatively, we can use \ref{vectorSpaceConvergenceConstruction} and note that passing to the field $F$ simply represents a weakening of condition $(5)$.
\end{proof}

\section{Equable filters and spaces}
\begin{definition}
Let $\sSet{V,\xi}$ be a convergence vector space and $F\in \powerfilters(V)$ a filter.
\begin{itemize}
\item The filter $F$ is called \udef{equable} if $\neighbourhood_\F(0)\cdot F = F$.
\item The space $\sSet{V, \xi}$ is called \udef{equable} if each filter that converges to $0$ contains an equable filter that still converges to $0$
\end{itemize}
\end{definition}

\begin{lemma} \label{TVSEquable}
Let $\sSet{V,\xi}$ be a topological convergence vector space. Then $\neighbourhood_\xi(0)$ is equable and thus $\sSet{V,\xi}$ is equable.
\end{lemma}
\begin{proof}
By \ref{vicinityFilterAtOrigin}, $\neighbourhood_\xi(0)$ has a balanced base and $U\in \neighbourhood_\xi(0)$ iff $\lambda U\in \neighbourhood_\xi(0)$ for all $\lambda\in \F\setminus\{0\}$. Thus we have
\[ \neighbourhood_\xi(0) = \upset\{\cball(0,1)\}\cdot \neighbourhood_\xi(0) = \upset\{\cball(0,\epsilon)\}\cdot \neighbourhood_\xi(0) = \neighbourhood_\F(0)\cdot \neighbourhood_\xi(0). \]
This shows that $\neighbourhood_\xi(0)$ is equable. Now each filter that converges to $0$ contains $\neighbourhood_\xi(0)$, which is equable, so $\sSet{V, \xi}$ is equable.
\end{proof}

\begin{proposition} \label{equableConvergenceBalancedBase}
Let $\sSet{V,\xi}$ be an equable convergence vector space. Then $\vicinity_\xi(0)$ has a balanced base.
\end{proposition}
\begin{proof}
We have
\[ \vicinity_\xi(0) = \bigcap_{F\in \lim^{-1}(0)} F = \bigcap_{F\in \lim^{-1}(0)} \neighbourhood_\F(0)\cdot F. \]
Now take some $A\in \vicinity_\xi(0)$. As each $\neighbourhood_\F(0)\cdot F$ has a balanced base and $A\in\neighbourhood_\F(0)\cdot F$, we can find a balanced $B_F \in \neighbourhood_\F(0)\cdot F$ that is contained in $A$. Now $B = \bigcup_{F\in\lim^{-1}(0)}B_F$ is balanced by \ref{balancedLemma} and a subset of $A$. Since $B\in \neighbourhood_\F(0)\cdot F$ for all $F$ by upwards closure, we have $B\in \vicinity_\xi(0)$.
\end{proof}

\section{Spaces of functions}

\begin{lemma} \label{linearFunctionsClosedSubset}
Let $\sSet{V, \xi}$ be a convergence vector space and $\sSet{W, \zeta}$ a Hausdorff convergence vector space over the same field. Then $\Lin(V,W)$ is a closed subset of $(V\to W)_p$ and $(V\to W)_c$.
\end{lemma}
\begin{proof}
By \ref{openClosedConvergenceInclusions} it is enough to show that $\Lin(V,W)$ is closed in the pointwise convergence.

It is enough to show $\adh_p\big(\Lin(V,W)\big) \subseteq \Lin(V,W)$. Suppose $f\in \adh_p\big(\Lin(V,W)\big)$. By \ref{principalAdherenceInherence}, we can take a convergent filter $H\in (V\to W)_p$ with limit $f$ such that $\Lin(V,W) \in H$. 

Now we have, for all $v,w\in V, \lambda \in \F$, by \ref{filterPairingLemma},
\begin{align*}
\upset\evalMap_{\lambda v+w}^{\imf\imf}(H) &= \upset\evalMap_{\lambda v+w}^{\imf\imf}(H|_{\Lin(V,W)}) \\
&\supseteq \upset\lambda\evalMap_{v}^{\imf\imf}(H|_{\Lin(V,W)}) + \upset\evalMap_{w}^{\imf\imf}(H|_{\Lin(V,W)}) \\
&= \upset\lambda\evalMap_{v}^{\imf\imf}(H) + \upset\evalMap_{w}^{\imf\imf}(H).
\end{align*}
To show that $f$ is linear, we calculate, for all $v,w\in V, \lambda \in \F$
\[ f(\lambda v + w) = \evalMap_{\lambda v+w}(f) \in \lim\upset\evalMap_{\lambda v+w}^{\imf\imf}(H) \supseteq \lim \upset\lambda\evalMap_{v}^{\imf\imf}(H) + \upset\evalMap_{w}^{\imf\imf}(H) \supseteq \{\lambda f(v) + f(w)\}.  \]
Since the convergence on $W$ is Hausdorff, $\lim\upset\evalMap_{\lambda v+w}^{\imf\imf}(H)$ is a singleton and thus $f(\lambda v + w) = \lambda f(v) + f(w)$.
\end{proof}

\begin{proposition} \label{continuousConvergenceVectorSpace}
Let $\sSet{X,\xi}$ be a convergence space and $\sSet{V, \zeta}$ a convergence vector space. Then
\begin{enumerate}
\item $(X\to V)_c$ is a preconvergence vector space;
\item $\cont_c(X,V)$ is a convergence vector space.
\end{enumerate}
\end{proposition}
\begin{proof}
Comparing with \ref{continuousConvergenceGroup}, we just need to show the continuity of pointwise scalar multiplication $\cdot_\text{pt}: \F\times (X\to V)_c \to (X\to V)_c$. By \ref{universalPropertyContinuousConvergence}, this is equivalent to the continuity of $\curry_1^{-1}(\cdot_\text{pt}): \F\times (X\to V)_c\times X \to V$, which follows from the commutativity of the following diagram:

\[ \begin{tikzcd}[column sep=large]
\big(\F\times (X\to V)_c\times X\big) \to V \arrow[r, "{\curry_1^{-1}(\cdot_\text{pt})}"] \arrow[d, "a"] & V \\
\F\times \big((X\to V)_c\times X\big) \arrow[r, "(\id_\F|\evalMap)"] & \F\times V \arrow[u, "{\boldsymbol{\cdot}}"]
\end{tikzcd} \]
where the map $a$ is just the associator and thus continuous by  \ref{associatorTernaryProducts}.
\end{proof}

\begin{proposition} \label{continuousDualComplete}
Let $\sSet{X,\xi}$ be a convergence space. Then $\contLin_c(X,\F)$ is a complete convergence vector space.
\end{proposition}
\begin{proof}
TODO Beattie Butzmann p.84
\end{proof}


\section{Initial and final vector space convergences}
\subsection{Initial vector space convergence}
\begin{proposition} \label{initialVectorSpaceConvergence}
Let $V$ be a vector space, $\{V_i\}_{i\in I}$ a set of convergence vector spaces and $\{L_i: V \to V_i\}_{i\in I}$ a set of linear maps. Then the initial convergence on $V$ w.r.t. $\{L_i: V \to V_i\}_{i\in I}$ makes $V$ a convergence vector space.
\end{proposition}
\begin{proof}
Continuity of vector addition follows from \ref{initialConvergenceGroup}.

We verify continuity of scalar multiplication $m: \F\times V \to V: (\lambda, v) \mapsto \lambda v$. Using \ref{characteristicPropertyInitialFinalConvergence}, we need to verify that $L_i\circ m$ is continuous for all $i\in I$. Because the $L_i$ are linear, we have
\[ L_i(\lambda v) = \lambda L_i(v) \]
for all $\lambda \in \F, v \in V$. This means that $L_i\circ m = m_i \circ (\id_{V_i}, L_i)$, where $m_i$ is scalar multiplication in $V_i$. Now $(\id_{\F}, L_i)$ is continuous by \ref{continuityFunctionTuple}, so $L_i \circ m$ is continuous.
\end{proof}
\begin{corollary}
Let $\{V_i\}_{i\in I}$ be a set of convergence vector spaces. Then the direct product $\prod_{i\in I} V_i$ with the product convergence is a convergence vector space.
\end{corollary}
\begin{corollary} 
Let $\sSet{I, \{\sSet{V_i, \xi_i}\}_{i\in I}, \{p_{j,i}\}_{i\preceq j}}$ be an projective system of convergence vector spaces with linear linking morphisms $p_{j,i}$. Then the projective limit $\varprojlim_{i\in I} V_i$ in the category $\cat{Set}$, equipped with the projective limit preconvergence structure and the projective limit vector space structure is a convergence vector space.
\end{corollary}

\subsection{Quotient spaces}
\begin{proposition} \label{quotientConvergenceVectorSpaceConvergence}
Let $\sSet{V, \xi}$ be a convergence vector space, $W$ a vector space and $q: \sSet{V, \xi} \to W$ a surjective linear function. Then the quotient convergence on $W$ w.r.t. $q$ is a vector space convergence.
\end{proposition}
\begin{proof}
Continuity of vector addition follows from
\ref{quotientConvergenceGroup}.

We verify continuity of scalar multiplication $m: \F\times W \to W: (\lambda, w) \mapsto \lambda w$. Take $F\overset{\F}{\longrightarrow}\lambda$ and $G \overset{W}{\longrightarrow} w$. Then, by \ref{initialFinalConvergence}, there exist $w'\in q^{\preimf}\{w\}$ and $G' \overset{\xi}{\longrightarrow} w'$ such that $q^{\imf\imf}[G'] \subseteq G$. Then
\[ F\cdot G \supseteq F\cdot q^{\imf\imf}[G'] = q^{\imf\imf}[F\cdot G'] \to q(\lambda \cdot w') = \lambda q(w') = \lambda w, \]
which shows that $m$ is continuous.
\end{proof}
\begin{corollary}
Let $\sSet{V,\xi}$ be a convergence vector space and $U\subseteq V$ a subspace. Then $G/U$ is a convergence vector space.
\end{corollary}
\begin{proof}
The function $[\cdot]_U: V\to V/U$ is linear by \ref{quotientAlgebra} and \ref{congruenceSubspace}. It is clearly surjective.
\end{proof}
\begin{corollary}
Let $\sSet{V, \xi}$ and $\sSet{W, \zeta}$ be CVSs. Let $f: V\to W$ be a continuous linear function and $U\subseteq G$ a subspace such that $N\subseteq \ker f$. Then there exists a unique continuous linear function $f': A/N \to B$ such that
\[ \begin{tikzcd}
A \arrow[r, "{[\cdot]_N}"] \arrow[dr, swap, "f"] & A/N \arrow[d, dashed, "{f'}"] \\
& B
\end{tikzcd} \qquad\text{commutes.} \]
Further, $f'$ is injective \textup{if and only if} $N = \ker f$.
\end{corollary}
\begin{proof}
The linear function $f'$ is the one from \ref{factorTheorem}. It is continuous by \ref{characteristicPropertyInitialFinalConvergence}.
\end{proof}

\subsection{Direct sum}
\begin{definition}
Let $\{\sSet{V_i, \xi_i}\}_{i\in I}$ be a set of convergence vector spaces. Then the \udef{direct sum convergence} is the final \emph{vector space} convergence on $\bigoplus_{i\in I}V_i$ w.r.t. the set $\{e_j: V_j \to \bigoplus_{i\in I}V_i\}$ of natural injections.
\end{definition}
The direct sum convergence is the final vector space convergence. It is equal to the final convergence if and only if the direct sum is trivial (TODO why?).

\begin{proposition}
Let $\{\sSet{V_i, \xi_i}\}_{i\in I}$ be a set of convergence vector spaces. Then the direct sum convergence is the final convergence on $\bigoplus_{i\in I}V_i$ w.r.t. the set of natural injections $\setbuilder{e_J: \prod_{j\in J} V_j \to \bigoplus_{i\in I}V_i}{\text{$J\subseteq I$ finite}}$.
\end{proposition}
\begin{proof}

\end{proof}
\begin{corollary} \label{finiteDirectSumIsProduct}
If $I$ is finite, then $\bigoplus_{i\in I}V_i = \prod_{i\in I}V_i$.
\end{corollary}

\begin{proposition}
Let $\{\sSet{V_i, \xi_i}\}_{i\in I}$ be a set of convergence vector spaces. The inclusion map
\[ \bigoplus_{i\in I}V_i \hookrightarrow \prod_{i\in I}V_i \]
is a continuous map into a dense subspace.
\end{proposition}
\begin{proof}

\end{proof}

\begin{proposition}
Let $\{\sSet{V_i, \xi_i}\}_{i\in I}$ be a set of convergence vector spaces. The direct sum $\bigoplus_{i\in I}V_i$ is topological \textup{if and only if} all $V_i$ are topological and only finitely many are non-trivial (i.e. not $\{0\}$).
\end{proposition}

\subsubsection{Finite direct sums}
\begin{lemma} \label{directSumsOpenClosedSets}
Let $\sSet{V,\xi}, \sSet{W,\zeta}$ be convergence vector spaces and $A\subseteq V, B\subseteq W$ subsets. Then
\begin{enumerate}
\item if $A,B$ are open, then $A\oplus B$ is open;
\item if $A,B$ are closed, then $A\oplus B$ is closed.
\end{enumerate}
\end{lemma}
\begin{proof}
(1) We have that $A\oplus W = \proj_1^{\preimf}(A)$ and $V\oplus B = \proj_2^{\preimf}(B)$ are open by \ref{preimageOpenClosed}. Thus $A\oplus B = A\oplus W \cap V\oplus B$ is open by \ref{propertiesTopology}.

(2) We have that $A\oplus W = \proj_1^{\preimf}(A)$ and $V\oplus B = \proj_2^{\preimf}(B)$ are closed by \ref{preimageOpenClosed}. Thus $A\oplus B = A\oplus W \cap V\oplus B$ is closed by \ref{propertiesTopology}.
\end{proof}

\subsubsection{Internal direct sums}
\begin{definition}
Let $\sSet{V,\xi}$ be a convergence vector space and $A,B\subseteq V$ subspaces such that $A\cap B = \{0\}$. We call the internal algebraic direct sum $A\oplus^i B$ a \udef{convergence direct sum} if $+: A\oplus B \to A\oplus^i B$ is a homeomorphism.
\end{definition}

\begin{lemma} \label{sumOnDirectSumsContinuous}
Let $\sSet{V,\xi}$ be a convergence vector space and $A,B\subseteq V$ subspaces such that $A\cap B = \{0\}$. Then
\[ +: A\oplus B \to A\oplus^i B \]
is continuous.
\end{lemma}
The function $+: A\oplus B \to A\oplus^i B$ is bijective, but not necessarily homeomorphic.
\begin{proof}
Since the convergence on $A\oplus B$ is the subspace convergence of $A\oplus B$ in $V\oplus V$, by \ref{productAndSubspaceConvergencesCommute},
we have $+: A\oplus B \to A\oplus^i B = +: V\oplus V|_{A\oplus B} \to A\oplus^i B$.
Then the result follows from the definition of a convergence vector space and \ref{continuityRestrictionExpansion}.
\end{proof}

\begin{proposition} \label{internalConvergenceDirectSumEquivalents}
Let $\sSet{V,\xi}$ be a convergence vector space and $A,B\subseteq V$ subspaces such that $A\cap B = \{0\}$. The following are equivalent:
\begin{enumerate}
\item internal algebraic direct sum $A\oplus^i B$ is a convergence direct sum;
\item both $P_A: A\oplus^i B \to A: x+y \mapsto x$ and $P_B: A\oplus^i B \to B: x+y \mapsto y$ are continuous;
\item one of $P_A$ and $P_B$ is continuous;
\item the function $(P_A, P_B): A\oplus^i B \to A\oplus B$ is continuous.
\end{enumerate}
\end{proposition}
Note that $(P_A, P_B): A\oplus^i B \to A\oplus B$ is the inverse of $+: A\oplus B \to A\oplus^i B$.
\begin{proof}
$(1) \Rightarrow (2)$ We have
\[ (\proj_A: A\oplus B \to A) = (P_A: A\oplus^i B \to A)\circ \big((+): A\oplus B \to A\oplus^i B\big). \]
Since $(+)$ is a homeomorphism, the continuity of $P_A$ is equivalent to the continuity of $\proj_A$. The function $\proj_A$ is continuous by \ref{finiteDirectSumIsProduct}.

The argument for the continuity of $P_B$ is similar.

$(2) \Leftrightarrow (3)$ The function $\id_{A\oplus^i B}$ is continuous and $P_B = \id_{A\oplus^i B} - P_A$.

$(2) \Rightarrow (4)$ This is the defining property of the product and thus follows from \ref{finiteDirectSumIsProduct}.

$(4) \Rightarrow (1)$ From point (4) and \ref{sumOnDirectSumsContinuous} it follows that $+: A\oplus B \to A\oplus^i B$ is a homeomorphism.
\end{proof}

\begin{example}
Consider the space $\ell^\infty(\N)$. Let $e_n$ be the sequence defined by $e_n(k) \defeq \delta_{n,k}$. Let
\begin{itemize}
\item $V$ be the subspace of sequences $s\in \ell^\infty(\N)$ such that $s(2k+1) = 0$ for all $k\in \N$;
\item $W$ be the subspace of sequences $s\in \ell^\infty(\N)$ such that $s(2k+1) = \frac{s(2k)}{k+1}$ for all $k\in \N$.
\end{itemize}
We have $V\cap W = \{0\}$ and $V\oplus W = \ell^\infty(\N)$.
Let $P_V$ be the projector on $V$ along $W$. This projector is not continuous.

Consider $v_n \defeq e_{2n} \in V$ and $w_n \defeq e_{2n}+ \frac{1}{1+n}e_{2n+1}\in W$.
Then $v_n - w_n = -\frac{1}{1+n}e_{2n+1} \to 0$, but $P_V(v_n-w_n) = v_n = e_{2n} \not\to 0$.
\end{example}

\begin{proposition} \label{convergenceDirectSumsClosedSubspaces}
Let $\sSet{V,\xi}$ be a Hausdorff convergence vector space and $A,B\subseteq V$ subspaces such that $A\cap B = \{0\}$.
If $A\oplus^i B$ is a convergence direct sum, then $A$ and $B$ are closed subspaces of $A\oplus^i B$.
\end{proposition}
The converse holds for Banach spaces, see (TODO ref.)
\begin{proof}
The set $\{0\}\subseteq V$ is closed by \ref{HausdorffCriterionConvergenceGroup}, so $A = P_B^{\preimf}\big(\{0\}\big)$ is closed as a subset of $A\oplus^i B$ by \ref{preimageOpenClosed} since $P_A$ is continuous by \ref{internalConvergenceDirectSumEquivalents}.

The argument for $B$ is similar. 
\end{proof}

\begin{proposition}
Let $\sSet{V,\xi}$ be a Hausdorff convergence vector space and $A,B\subseteq V$ subspaces such that $A\cap B = \{0\}$. If $A$ is closed and $B$ is finite-dimensional, then $A\oplus^i B$ is a convergence direct sum.
\end{proposition}
\begin{proof}
By \ref{internalConvergenceDirectSumEquivalents}, it is enough the show that $P_B$ is continuous. We have $\ker(P_B) = A$. By the factor theorem \ref{factorTheorem},
we can decompose
\[ \begin{tikzcd}
A\oplus^i B \ar[rr, "{P_B}"] \ar[dr, "{[\cdot]_A}"] && B \\
{}& (A\oplus^i B)/A \ar[ur, "{P_B'}"] &{}
\end{tikzcd}. \]
Equipping $(A\oplus^i B)/A$ with the quotient convergence makes $[\cdot]_A$ continuous. Since $A$ is closed, $(A\oplus^i B)/A$ is Hausdorff, by \ref{quotientConvergenceGroupProperties}. Then $P_B'$ is continuous by \ref{finiteDimensionalLinearMapContinuous}. Thus $P_B = P_B'\circ [\cdot]_A$ is continuous.
\end{proof}

\subsubsection{Complemented subspaces}
\begin{definition}
Let $\sSet{V,\xi}$ be a convergence vector space and $A\subseteq V$ a subspace. Then $A$ is called \udef{complemented} if there exists another subspace $B\subseteq V$ such that
\begin{itemize}
\item $A\oplus^i B = V$; and
\item $A\oplus^i B$ is a convergence direct sum.
\end{itemize}
\end{definition}

\subsection{Inductive systems}
\begin{proposition} 
Let $\sSet{I, \{\sSet{V_i, \xi_i}\}_{i\in I}, \{e_{i,j}\}_{i\preceq j}}$ be an inductive system of convergence vector spaces with linear linking morphisms $e_{i,j}$. Then the inductive colimit $\varinjlim_{i\in I} V_i$ in the category set, equipped with the inductive colimit preconvergence structure and the inductive colimit vector space structure is a convergence vector space.
\end{proposition}
\begin{proof}
By \ref{elementsOfInductiveLimitImages} and \ref{finalConvergenceConvergence} the inductive preconvergence structure is in fact a convergence structure.

Let $\nu$ be the inductive limit convergence. We first show the continuity of the addition.
Let $F,G\in \powerfilters(\varinjlim_{i\in I} V_i)$ be filters that converge to $v$, resp., $w$. By \ref{initialFinalConvergence} there exist $F'\overset{\xi_i}{\longrightarrow} v'$ and $G'\overset{\xi_j}{\longrightarrow} w'$ such that $\upset e_i^{\imf\imf}(F')\subseteq F$ and $e_j^{\imf\imf}(G')\subseteq G$.

Now take $k\succeq i,j$. Then $e_{i,k}^{\imf\imf}(F') + e_{j,k}^{\imf\imf}(G')$ converges to $e_{i,k}(v')+e_{i,k}(w')$ by continuity of the linking morphisms and addition in $V_k$. By the continuity of $e_k$ and the construction of $v+w$ in \ref{vectorSpaceInductiveLimit}, we have
\begin{align*}
F + G &\supseteq \upset e_i^{\imf\imf}(F') + \upset e_j^{\imf\imf}(G') \\
&= \upset e_k^{\imf\imf}\big(e_{i,k}^{\imf\imf}(F')\big) + \upset e_k^{\imf\imf}\big(e_{j,k}^{\imf\imf}(G')\big) \\
&= \upset e_k^{\imf\imf}\big(e_{i,k}^{\imf\imf}(F') + e_{j,k}^{\imf\imf}(G')\big) \to e_k\big(e_{i,k}(v')+e_{i,k}(w')\big) = v + w.
\end{align*}
To show the continuity of scalar multiplication, take a filter $F\in \powerfilters(\varinjlim_{i\in I} V_i)$ that converges to $v$ and a filter $H\in \powerfilters(\F)$ that converges to $\lambda$. By \ref{initialFinalConvergence} there exist $F'\overset{\xi_i}{\longrightarrow} v'$ such that $\upset e_i^{\imf\imf}(F')\subseteq F$. By linearity, we have
\[ H \cdot F \supseteq H\cdot e_i^{\imf\imf}(F') = e_i^{\imf\imf}(H\cdot F') \to e_i(\lambda v') = \lambda e_i(v') = \lambda v. \]
\end{proof}
TODO generalise to groups and general algebraic structures.

\begin{proposition}
Let $\sSet{I, \{\sSet{V_i, \xi_i}\}_{i\in I}, \{e_{i,j}\}_{i\preceq j}}$ be an inductive system of convergence vector spaces. Let $\sSet{F, \{f_i\}_{i\in I}}$ be a cone under the inductive system such that
\begin{enumerate}
\item all $f_i$ are injective;
\item $F = \bigcup_{i\in I}f_i^\imf(V_i)$;
\item $F$ carries the final convergence structure w.r.t. $\{f_i\}_{i\in I}$.
\end{enumerate}
Then $\sSet{F, \{f_i\}_{i\in I}}$ is the inductive limit.
\end{proposition}
\begin{proof}
Let $\sSet{F', \{f_i'\}_{i\in I}}$ be another cone under the inductive system. We define a function $u: F \to F'$ by mapping $x\in F$ to the unique element of
\[ \bigcup_{i\in I}f_i^{\prime\imf}\Big(f_i^{\preimf}\big(\{x\}\big)\Big). \]
We need to show that this set indeed has exactly one element. For each $i\in I$, the set $f_i^{\prime\imf}\Big(f_i^{\preimf}\big(\{x\}\big)\Big)$ has at most one element by injectivity. There is also at least one $i\in I$ such that this set has an element by the second point. Finally let $y_i$ be the unique element in $f_i^{\prime\imf}\Big(f_i^{\preimf}\big(\{x\}\big)\Big)$ and $y_j$ the unique element in $f_j^{\prime\imf}\Big(f_j^{\preimf}\big(\{x\}\big)\Big)$. Take $k\succeq i,j$. Then, by compatibility, 
\begin{align*}
f_i^{\prime\imf}\circ f_i^{\preimf}\big(\{x\}\big) \cup f_j^{\prime\imf}\circ f_j^{\preimf}\big(\{x\}\big) &= (f_k'\circ e_{i,k})^{\imf}\circ (f_k\circ e_{i,k})^{\preimf}\big(\{x\}\big) \cup (f_k'\circ e_{j,k})^{\imf}\circ (f_k\circ e_{j,k})^{\preimf}\big(\{x\}\big) \\
&= f_k^{\prime\imf}\circ e_{i,k}^{\imf}\circ e_{i,k}^{\preimf} \circ f_k^\preimf \big(\{x\}\big) \cup f_k^{\prime\imf}\circ e_{j,k}^{\imf}\circ e_{j,k}^{\preimf} \circ f_k^\preimf \big(\{x\}\big) \\
&= f_k^{\prime\imf} \circ f_k^\preimf \big(\{x\}\big) \cup f_k^{\prime\imf}\circ f_k^\preimf \big(\{x\}\big) \\
&= f_k^{\prime\imf} \circ f_k^\preimf \big(\{x\}\big),
\end{align*}
where we have used that $e_{i,k}^{\imf}\circ e_{i,k}^{\preimf} \circ f_k^\preimf \big(\{x\}\big) = f_k^\preimf \big(\{x\}\big)$, because $f_k^\preimf \big(\{x\}\big) \subseteq \im(e_{i,k})$.

By construction, $u$ is linear and also the unique function such that $f_i' = u\circ f_i$ for all $i\in I$. Finally $u$ is continuous by \ref{characteristicPropertyInitialFinalConvergence}.
\end{proof}

\begin{proposition}
The colimit of a reduced inductive family of complete convergence vector spaces is complete.
\end{proposition}
\begin{proof}
TODO Beattie / Butzmann p. 101.
\end{proof}
\begin{corollary}
Let $\{\sSet{V_i, \xi_i}\}_{i\in I}$ be a set of complete convergence vector spaces. Then $\bigoplus_{i\in I}V_i$ is complete.
\end{corollary}

\section{Dimension, coordinates and bases}
\subsection{Finite dimensional spaces}

\begin{proposition} \label{finiteDimensionalHausdorffEuclidean}
Let $\sSet{V,\xi}$ be a finite-dimensional Hausdorff convergence vector space over a complete field $\F$. Then
\begin{enumerate}
\item every linear functional $f\in \Lin(V,\F)$ is continuous;
\item for every basis $\{x_1, \ldots, x_n\}$, the function $\F^n \to V: (\lambda_1, \ldots, \lambda_n)\mapsto \sum_{k=1}^n \lambda_k x_k$ is a homeomorphism.
\end{enumerate}
\end{proposition}
In particular, $V$ is homeomorphic to $\F^{\dim(V)}$.
\begin{proof}
We prove (1) and (2) simultaneously by induction on the dimension $n$ of $V$. First we show the base step $n=1$.

Take some linear functional $f\in \Lin(V,\F)$. If $f = \constant{0}$, then it is continuous by \ref{continuityConstructions}. If not, then there exists $v\in V$ such that $f(v) \neq 0$. Now set $w \defeq \frac{v}{f(v)}$, so $f(w) = 1$. For all $\lambda\cdot w\in \Span\{w\} = V$, we have $f(\lambda w) = \lambda f(w) = 1$.
Since $V$ is Hausdorff, $f$ is a homeomorphism by \ref{continuityLemmaVectorConvergence}. This implies that $V$ is homeomorphic to $\F$ and, in particular, that $f$ is continuous.

Now suppose (1) and (2) hold for $n-1$. First take some linear functional $f\in \Lin(V,\F)$. If $f = \constant{0}$, then it is continuous by \ref{continuityConstructions}. If not, then $\dim\big(\ker(f)\big) = n-1$, by the dimension theorem \ref{dimensionLinearMaps}. Since $\ker(f)$ is Hausdorff by \ref{HausdorffSubspace}, it is homeomorphic to $\F^{n-1}$ by the induction hypothesis. Now $\F^{n-1}$ is complete by \ref{productCompleteSpacesComplete}.

Since $V$ is Hausdorff, $\ker(f)$ is a closed subset of $V$ by \ref{closedComplete}. This implies that $f$ is continuous by \ref{continuityLinearFunctionals}.

To show $V \cong \F^n$, take a basis $\{x_1, \ldots, x_n\}$. Then the function $g: \F^n \to V: (\lambda_1, \ldots, \lambda_n)\mapsto \sum_{k=1}^n \lambda_k x_k$ is a bijection by \ref{finiteBasisUniqueDecomposition}. It is continuous by \ref{continuityLinearCombination}.  

For the continuity of $g^{-1}$, note that $\proj_k\circ g^{-1}$ is a functional, and thus continuous, for all $k\in \interval{1,n}$. Thus $g^{-1}$ is continuous by \ref{characteristicPropertyInitialFinalConvergence}.
\end{proof}
\begin{corollary} \label{finiteDimensionalLinearMapContinuous}
Let $\sSet{V,\xi}$ and $\sSet{W,\zeta}$ be convergence vector spaces over a complete field and $f: V\to W$ a linear map. If $V$ is finite-dimensional and Hausdorff, then $f$ is continuous.
\end{corollary}
\begin{proof}
Let $\{x_0, \ldots, x_{n-1}\}$ be a basis of $V$. The function $f$ can be decomposed as
\[ \begin{tikzcd}
\sum_{k=0}^{n-1}\lambda_kx_k \ar[r, mapsto] & (\lambda_0, \ldots, \lambda_{k-1}) \ar[r, mapsto] & \sum_{k=0}^{n-1}\lambda_kf(x_k).
\end{tikzcd} \]
The first part is continuous by \ref{finiteDimensionalHausdorffEuclidean} and the second part by \ref{continuityLinearCombination}.
\end{proof}
\begin{corollary} \label{finiteDimensionalHausdorffLinearBijectionIsHomeomorphism}
Every bijective linear map between finite-dimensional Hausdorff convergence vector spaces over a complete field is a homeomorphism.
\end{corollary}


\begin{example}
\url{https://terrytao.wordpress.com/2011/05/24/locally-compact-topological-vector-spaces/}

discusses some almost counterexamples to \ref{finiteDimensionalHausdorffEuclidean}
\end{example}

\begin{proposition} \label{closedKernelLinearFunctionToFiniteDimSpaceContinuous}
Let $\sSet{V,\xi}$ be a convergence vector space and $\sSet{W,\zeta}$ a finite dimensional convergence vector space, both over a complete field $\F$. Let $L: V\to W$ be a linear function such that $\ker(L)$ is closed. Then $L$ is continuous.
\end{proposition}
If $W$ is Hausdorff, then the converse also holds by \ref{kernelClosed}: if $L$ is continuous, then $\ker(L)$ is closed.
\begin{proof}
By the factor theorem \ref{factorTheorem}, we have an injective linear function $L': V/\ker(f)\to W$ such that
\[ \begin{tikzcd}
V \arrow[r, "{[\cdot]_{\ker(L)}}"] \arrow[dr, swap, "L"] & V/\ker(L) \arrow[d, dashed, "{L'}"] \\
& W
\end{tikzcd} \qquad\text{commutes.} \]

Restricting $L'$ to its image, we have an isomorphism between $V/\ker(L)$ and a subspace of $W$, which means that $V/\ker(L)$ is finite dimensional. We also have that $V/\ker(L)$ is Hausdorff by \ref{quotientConvergenceGroupProperties}. Thus $L': V/\ker(L) \to W$ is continuous by \ref{finiteDimensionalLinearMapContinuous}. Since $[\cdot]_{\ker(L)}$ is continuous by definition of the quotient convergence, we have that $L = L'\circ [\cdot]_{\ker(L)}$ is continuous as a composition of continuous functions.
\end{proof}

\begin{proposition} \label{finiteDimSubspaceClosed}
Let $\sSet{V,\xi}$ be a Hausdorff CVS over a complete field. Then any finite-dimensional subspace $U$ of $V$ is closed.
\end{proposition}
\begin{proof}
Let $W$ be a finite dimensional subspace of $V$. Then $U$ equipped with the subspace convergence is a convergence vector space by \ref{initialVectorSpaceConvergence}. It is complete by \ref{finiteDimensionalHausdorffEuclidean} and \ref{productCompleteSpacesComplete}.

Since $V$ is complete, $U$ is closed by \ref{closedComplete}.
\end{proof}
\begin{corollary}
Let $\sSet{V,\xi}$ be a Hausdorff CVS over a complete field, $U\subseteq V$ a finite-dimensional subspace and $W \subseteq V$ a closed subspace. Then $U+W$ is a closed subspace of $V$.
\end{corollary}
\begin{proof}
Consider the quotient map $[\cdot]_W: V\to V/W$. Then $V/W$ is Hausdorff by \ref{quotientConvergenceGroupProperties}, so $[\cdot]_W^{\imf}(U)$ is closed, because it is a finite-dimensional subspace. This implies that $U+W = \big([\cdot]_W^{\preimf}[\cdot]_W^{\imf}\big)(U)$ is closed, by \ref{preimageOpenClosed}.
\end{proof}

\begin{lemma} \label{compactHCVSZeroDim}
Let $V$ be a compact Hausdorff vector space over $\F$. Then $V = \{0\}$.
\end{lemma}
\begin{proof}
Suppose $v\in V$ and $v\neq 0$. Then $\Span\{v\}$ is a subspace that is not compact, by \ref{continuityLemmaVectorConvergence} and the fact that $\F$ is not compact. 

Since $\Span\{v\}$ is finite-dimensional, it is closed, by \ref{finiteDimSubspaceClosed}. Thus it should be compact by \ref{compactClosedSets}. This is a contradiction.
\end{proof}

\begin{proposition}
Let $\sSet{V,\xi}$ be a Hausdorff convergence vector space over a complete and locally compact field $\F$. Then
\begin{enumerate}
\item if $V$ is finite-dimensional, then $V$ is locally compact;
\item if $V$ is locally compact and topological, then $V$ is finite-dimensional.
\end{enumerate}
\end{proposition}
In particular, this proposition applies if $\F = \R$ or $\F = \C$, by \ref{realsLocallyCompact} and \ref{propertiesRealNumbers}.

TODO generalise to convergence space?

TODO: Aliprantis / Border p.180

TODO: There exist infinite-dimensional locally compact CVS! See Beattie / Butzmann p. 133.
\begin{proof}
(1) If $V$ is finite-dimensional, then $V\cong \F^{\dim(V)}$ by \ref{finiteDimensionalHausdorffEuclidean}. Since $\F$ is locally compact, we have that $\F^{\dim(V)}$ is locally compact by \ref{productLocallyCompact}.

(2) If $V$ is locally compact and topological, then there exists a compact neighbourhood $K$ of $0$. The set $\frac{1}{2}K$ is also a neighbourhood of $0$ by \ref{vicinityFilterAtOrigin}, so $\frac{1}{2}K$ contains an open set $O$ that contains the origin. Now $k+O$ is open for all $k\in K$ by \ref{shiftHomeomorphism} and $K\subseteq \bigcup_{k\in K}k+O$. Thus the sets $k+O$ form an open cover of $K$. By compactness and \ref{topologyCompactnessOpenCover}, there exists a finite set $S\subseteq K$ such that $K\subseteq \bigcup_{k\in S}k+O$.

Set $M \defeq \Span(S)$, which is a finite-dimensional subspace and thus closed by \ref{finiteDimSubspaceClosed}. The quotient space $V/M$ is then Hausdorff by \ref{quotientConvergenceGroupProperties}. We have $K\subseteq \bigcup_{k\in S}k+O \subseteq M+O \subseteq M+\frac{1}{2}K$. Then $[K]_{M}^\imf \subseteq [\frac{1}{2}K]_{M}^\imf$, so $2[K]_M^\imf \subseteq [K]_M^\imf$.

For all $n\in \N$, we have
\[ 2^n[K]_M^\imf = 2^{n-1}\big(2[K]_M^\imf\big) \subseteq 2^{n-1}[K]_M^\imf \subseteq \ldots \subseteq [K]_M^\imf. \]
Since $K$ is absorbent (by \ref{vicinityFilterAtOrigin}), we have $V = \bigcup_{n\in \N} 2^nK$. By \ref{functionImagePreimageGaloisConnection},
\[ [K]_M \subseteq V/M = [V]_M^\imf = \Big[\bigcup_{n\in \N} 2^nK\Big]^\imf_M = \bigcup_{n\in \N}[2^nK]_{M}^\imf = \bigcup_{n\in \N}2^n[K]_{M}^\imf \subseteq [K]_M, \]
so $V/M = [K]_M$, which is compact by \ref{compactConstructions} (and the fact that $[\cdot]_M$ is continuous, i.e. the definition of the quotient convergence).

Since $V/M$ is compact, it is zero-dimensional by \ref{compactHCVSZeroDim}. By the dimension theorem \ref{dimensionTheoremQuotientSpace}, we have $\dim V = \dim M$. So $V$ is finite-dimensional.
\end{proof}

\subsection{Hamel coordinate functions}
\begin{lemma} \label{finiteNonZeroHamelCoordinateFunctions}
Let $V$ be a vector space and $\seq{e_i}_{i\in I}$ a Hamel basis of $V$. Let $\seq{\varphi_i}_{i\in I}$ be the associated coordinate functions. Then for all $v\in V$, at most finitely many $\varphi_i(v)$ are non-zero.
\end{lemma}
\begin{proof}
By definition of a Hamel basis, we have $v = \sum_{i\in I}\lambda_i e_i = \sum_{i\in I}\varphi_i(v) e_i$ and the sum must be finite.
\end{proof}
TODO: compare existence dual basis.

\begin{proposition}
Let $\sSet{V,\norm{\cdot}}$ be a Banach space and $\seq{e_i}_{i\in I}$ a Hamel basis of $V$. Let $\seq{\varphi_i}_{i\in I}$ be the associated coordinate functions. At most finitely many $\varphi_i$ are continuous.
\end{proposition}
\begin{proof}
Suppose there existed a countable sequence $\seq{e_i}_{i\in I}$ of basis elements, each with a continuous coordinate function $\varphi_n$. Consider the vector $v = \sum_{n=0}^\infty 2^{-n}e_i/\norm{e_i}$. Then, by continuity, each $\varphi_n(v)$ is non-zero. This is impossible by \ref{finiteNonZeroHamelCoordinateFunctions}.
\end{proof}



\section{Topological vector spaces}
\begin{definition}
A \udef{topological vector space} (or TVS) is a convergence vector space that is topological.
\end{definition}
As with convergence groups, any pretopological vector space convergence is topological, see \ref{pretopologicalGroupConvergence}.

\subsection{Neighbourhoods and base}
\begin{proposition} \label{TVSconstruction}
Let $V$ be a vector space and $N\in\powerfilters(V)$. Then $N = \neighbourhood_\xi(0)$ for some topological convergence on $V$ \textup{if and only if}
\begin{enumerate}
\item for all $A\in N$ and $\lambda\in \F\setminus\{0\}$: $\lambda A\in N$;
\item each $A \in N$ is absorbent;
\item $N$ has a balanced base;
\item for all $A\in N$, there exists some $B\in N$ such that $B+B\subseteq A$.
\end{enumerate}
\end{proposition}
\begin{proof}
We adapt \ref{vectorSpaceConvergenceConstruction} to the present situation.

First assume $N$ has a balanced and absorbent base. We check the five conditions for $\mathcal{F} = \pfilter{N}$. In this case the convergence will necessarily be topological by \ref{pretopologicalGroupConvergence}.

\begin{enumerate}
\item Immediate because $\mathcal{F} = \pfilter{N}$.
\item Take $F,G\in \pfilter{N}$. We need to show that $\upset (F + G) \supseteq N$, which means that for all $A \in N$ there exist $B\in F$ and $C\in G$ such that $B+C\subseteq A$. We can take $B = C$ equal to the $B$ of point (2).
\item Take $F\in \pfilter{N}$. We need to show that $\upset(\neighbourhood_\F(0)\cdot F) \supseteq N$, which means that for all $A\in N$ there exists a $\Gamma\in \neighbourhood_\F(0)$ and $B\in F$ such that $\Gamma \cdot B\subseteq A$. We can take $B = \balancedCore(A) \in N \subseteq F$ and $\Gamma = B(0,1)$.
\item Take $v\in V$. We need to show that all $A\in N$ contain $\Gamma\cdot v$ for some $\Gamma \in \neighbourhood_\F(0)$. Because $A$ is absorbent, there exists an $r>0$ such that $v\in cA$ for all $|c|\geq r$. Conversely $c^{-1}v \in A$ for all $|c^{-1}| \leq r^{-1}$. So $\ball(0,r^{-1})\cdot v \subseteq A$ and $B(0,r^{-1}) \in \neighbourhood_\F(0)$.
\item Take $F\in \pfilter{N}$ and $\lambda\in \F$. We need to show that for all $A\in N$ there exists a $B\in F$ such that $\lambda\cdot B\subseteq A$. We can take $B = \lambda^{-1}A \in N\subseteq F$.
\end{enumerate}

Now assume $\xi$ is a topological vector space convergence and $N = \neighbourhood_\xi(0)$. The first 3 points follow from \ref{vicinityFilterAtOrigin}. The fourth from \ref{vicinityFactorisation}.
\end{proof}
\begin{corollary} \label{TVSbase}
Let $V$ be a vector space and $\mathcal{B}\subseteq\powerset(V)$. If $\mathcal{B}$ is such that
\begin{enumerate}
\item all $A\in \mathcal{B}$ are balanced and absorbent;
\item for each $A\in\mathcal{B}$, there exists some $B\in \mathcal{B}$ such that $B+B\subseteq A$;
\end{enumerate}
then $\mathfrak{F}(\mathcal{B}) = \neighbourhood_\xi(0)$ for some topological convergence on $V$.
\end{corollary}
\begin{proof}
We verify the 4 points of the proposition:
\begin{enumerate}
\item From point (2) of the hypothesis, we can prove by induction that for all $A\in\mathcal{B}$ and $n\in \N$, there exists $B\in \mathcal{B}$ such that $2^nB \subseteq A$.

Now take arbitrary $A\in\mathfrak{F}(\mathcal{B})$ and $\lambda\in\F\setminus\{0\}$. Then there exists a finite set $\{B_i\}_{i=0}^k\subseteq \mathcal{B}$ such that $B_0\cap \ldots \cap B_k \subseteq A$. Pick $n\in\N$ such that $2^{-n}\leq |\lambda|$ (and thus $|\lambda^{-1}2^{-n}| \leq 1$). Then we can find a finite set $\{C_i\}_{i=0}^k\subseteq \mathcal{B}$ such that $C_i \subseteq 2^{-n}B_i$. As each $B_i$ is absorbent, we have
\[ C_i \subseteq 2^{-n}B_i = \lambda (\lambda^{-1}2^{-n}) \subseteq \lambda B_i. \]
Thus $(C_0\cap \ldots \cap C_k) \subseteq \lambda (C_0\cap \ldots \cap C_k) \subseteq \lambda A$.
\item By \ref{absorbingSetProperties}, finite intersections of absorbent sets are absorbent. Thus each element of $\mathfrak{F}(\mathcal{B})$ contains an absorbent set and, by \ref{absorbingSetProperties}, is also absorbent.
\item Each element of $\mathfrak{F}(\mathcal{B})$ contains a finite intersection of balanced sets, which is also balanced by \ref{balancedLemma}.
\item Take arbitrary $A\in\mathfrak{F}(\mathcal{B})$. Then there exists a finite set $\{B_i\}_{i=0}^k\subseteq \mathcal{B}$ such that $B_0\cap \ldots \cap B_k \subseteq A$. By assumption, we can find a finite set $\{C_i\}_{i=0}^k\subseteq \mathcal{B}$ such that $C_i + C_i \subseteq B_i$. Then, using \ref{orderPreservingFunctionLatticeOperations}, we have
\[ \Big(\cap_{i\leq k}C_i\Big) + \Big(\cap_{j\leq k}C_j\Big) \subseteq \bigcap_{i,j\leq k} C_i + C_j \subseteq \bigcap_{i\leq k}C_i + C_i \subseteq \bigcap_{i\leq k} B_i \subseteq A. \]
\end{enumerate}
\end{proof}



\section{Properties of subsets}
\begin{lemma} \label{sumOpenSetsOpen}
Let $\sSet{V,\xi}$ be a vector space convergence and $A,B\subseteq V$ subsets. If $A$ is open, then $A+B$ is open.
\end{lemma}
\begin{proof}
We have $A + B = \bigcup_{b\in B} A+b$. Now each $A+b$ is open because adding $b$ is a homeomorphism and thus the union is open by \ref{propertiesTopology}.
\end{proof}

\begin{proposition} \label{inherenceAdherenceBalanced}
Let $\sSet{V,\xi}$ be a vector space convergence and $A\subseteq V$. Then
\begin{enumerate}
\item if $A$ is balanced, then $\adh(A)$ is balanced;
\item if $A$ is balanced and $0\in\inh(A)$, then $\inh(A)$ is balanced;
\item if $A$ is open and contains the origin, then $\balanced(A)$ is open.
\end{enumerate}
\end{proposition}
\begin{proof}
(1) We use \ref{productAdherence} and \ref{adherenceInherenceContinuity} to compute
\begin{align*}
\cball(0,1)\cdot \adh_\xi(A) &= \cdot^\imf[\adh_\F(\cball(0,1))\times \adh_\xi(A)] = \cdot^\imf[\adh_{\F\otimes \xi}(\cball(0,1)\times A)] \\
&\subseteq \adh_{\xi}\big(\cdot^\imf[\cball(0,1)\times A]\big) = \adh_{\xi}(\cball(0,1)\cdot A) = \adh_\xi(A).
\end{align*}

(2) Take $0<r<1$. Then multiplying by $r$ is a homeomorphism and thus $r\cdot\inh(A) = \inh(r\cdot A) \subseteq \inh(A)$. Now take $r=0$. If $0\in\inh(A)$, then
\[ r\cdot\inh(A) = \{0\} \subseteq \inh(A). \]

(3) For all $r\neq 0$, $r\cdot A$ is open. Thus
\[ \balanced(A) = \bigcup_{|r|\leq 1}r\cdot A = \{0\}\cup \bigcup_{0< r\leq 1}r\cdot A = \bigcup_{0< r\leq 1}r\cdot A \]
is a union of open sets and thus open by \ref{propertiesTopology}.
\end{proof}
\begin{proof}[Alternative proof of (1)]
Take $|\lambda|\leq 1$ and $v\in \adh_\xi(A)$, then we need to show that $\lambda v \in \adh_\xi(A)$. We have $A\in \vicinity_\xi(v)^{\mesh}$ and $A\subseteq \lambda^{-1}A$ (TODO: why, is this correct??). So for all $B\in \vicinity_\xi(v)$:
\[ A\mesh B \quad\implies\quad \lambda^{-1}A\mesh B \quad\implies\quad A\mesh \lambda B. \]
Thus $A\in \vicinity_\xi(\lambda v)^{\mesh}$, which is what we needed to show by \ref{principalAdherenceInherence}.
\end{proof}

\begin{proposition} \label{inherenceAdherenceCone}
Let $\sSet{V,\xi}$ be a vector space convergence and $K\subseteq V$ a cone. Then
\begin{enumerate}
\item $\adh(K)$ is a cone;
\item $\inh(K)$ is a cone;
\item if $K\neq \emptyset$, then $0\in \adh(K)$;
\item if $K$ is salient, then $0\notin \inh(K)$.
\end{enumerate}
\end{proposition}
\begin{proof}
(1) For all $r\in \interval[o]{0,+\infty}$, multiplying by $r$ is a homeomorphism. Then
\[ \interval[o]{0,+\infty}\cdot \adh(K) = \bigcup_{r> 0}r\cdot \adh(K) = \bigcup_{r>0}\adh(r\cdot K) \subseteq \bigcup_{r>0}\adh(K) = \adh(K). \]

(2) For all $r\in \interval[o]{0,+\infty}$, multiplying by $r$ is a homeomorphism. Then
\[ \interval[o]{0,+\infty}\cdot \inh(K) = \bigcup_{r> 0}r\cdot \inh(K) = \bigcup_{r>0}\inh(r\cdot K) \subseteq \bigcup_{r>0}\inh(K) = \inh(K). \]

(3) Assume $K\neq \emptyset$. Then we can take $v\in K$. Now $\seq{n^{-1}v}$ is a sequence in $K$ that converges to $0$ by \ref{continuityLemmaVectorConvergence}. So $0\in \adh(K)$.

(4) Assume $K$ is salient. Suppose, towards a contradiction, that $0\in \inh(K)$. Then $K\in \vicinity(0)$ and so $K$ is absorbent by \ref{vicinityFilterAtOrigin}.

Take $v\in V$. By absorption, we have $\ball(0,\epsilon)\cdot\{v\} \subseteq K$ for some $\epsilon > 0$. In particular both $\epsilon v\in K$ and $-\epsilon v\in K$. Thus $K$ is flat, not salient.
\end{proof}

\begin{proposition} \label{inherenceAdherenceConvex}
Let $\sSet{V,\xi}$ be a vector space convergence and $A\subseteq V$. Then
\begin{enumerate}
\item if $A$ is convex, then $\adh(A)$, $\inh(A)$ and $\interior(A)$ is convex;
\item if $A$ is open, then $\convex(A)$ is open.
\end{enumerate}
\end{proposition}
TODO same for closure?
\begin{proof}
(1) For all $0<r<1$, we have
\[ r\adh(A) + (1-r)\adh(A) = \adh(rA) + \adh\big((1-r)A\big) \subseteq \adh\big(rA + (1-rA)\big) \subseteq \adh(A) \]
by \ref{vectorSumInherenceAdherence}. 

The argument for $\inh(A)$ and $\interior(A)$ is similar.

(2) We have $\convex(A) = \bigcup_{0\leq r\leq 1} rA + (1-r)A$ by \ref{convexHullLemma}. For all $0\leq r\leq 1$, we have that $rA + (1-r)A$ is open by \ref{sumOpenSetsOpen}, so the union is open by \ref{propertiesTopology}.
\end{proof}

\begin{lemma}
Let $\sSet{V,\xi}$ be a convergence vector space, $C\subseteq V$ a convex subset and $0<r\leq 1$. Then
\[ r\inh_\xi(C) + (1-r)\adh_\xi(C) \subseteq \inh_\xi(C). \]
\end{lemma}
TODO also for algebraic convergence!
\begin{proof}
Take $v\in \inh_\xi(C)$ and $w\in \adh_\xi(C)$. Then, by \ref{principalAdherenceInherence}, $C\in \vicinity_\xi(v)$. Then $(C- v)\in \vicinity_\xi(0)$, so $-\frac{r}{1-r}(C-v)\in \vicinity_\xi(0)$ and $w - \frac{r}{1-r}(C-v) \in \vicinity_\xi(w)$. Again by \ref{principalAdherenceInherence}, we have $C \mesh \big(w - \frac{r}{1-r}(C-v)\big)$, so we can take $u \in C \cap \big(w - \frac{r}{1-r}(C-v)\big)$. Since $u\in w - \frac{r}{1-r}(C-v)$, we can write $u = w - \frac{r}{1-r}(u'-v)$ for some $u'\in C$. Rearranging gives $rv + (1-r)w = ru'+ (1-r)u \in C$.
\end{proof}
\begin{corollary} \label{adherenceInherenceClosureConvexSets}
Let $\sSet{V,\xi}$ be a convergence vector space and $C\subseteq V$ a convex subset with nonempty interior. Then
\begin{enumerate}
\item $\adh_\xi\big(\inh_\xi(C)\big) = \adh_\xi(C)$;
\item $\inh_\xi\big(\adh_\xi(C)\big) = \inh_\xi(C)$.
\end{enumerate}
\end{corollary}
The first point says that $\inh_\xi(C)$ is dense in $\adh_\xi(C)$.
\begin{proof}
(1) We have $\adh_\xi\big(\inh_\xi(C)\big) \subseteq \adh_\xi(C)$ by \ref{principalInherenceAdherenceProperties}.

Now take $v\in \adh_\xi(C)$. To show $v\in \adh_\xi\big(\inh_\xi(C)\big)$, we look for a proper filter that converges to $v$ and contains $\inh_\xi(C)$. As $\inh_\xi(C)$ was assumed nonempty, we can find $w\in \inh_\xi(C)$. Consider the tail filter of $\seq{n^{-1}w + (1-n^{-1})v}_{n\in \N}$. This filter clearly converges to $v$ and each term of the sequence is contained in $\inh_\xi(C)$ by the lemma.

(2) We have $\inh_\xi(C) \subseteq \inh_\xi\big(\adh_\xi(C)\big)$ by \ref{principalInherenceAdherenceProperties}.

Now take $v\in \inh_\xi\big(\adh_\xi(C)\big)\subseteq \adh_\xi(C)$, which implies that $\adh_\xi(C)\in \vicinity_\xi(x)$ and thus $\adh_\xi(C)-x\in \vicinity_\xi(0)$, which is absorbent by \ref{vicinityFilterAtOrigin}. Now take $w\in \inh_\xi(C)$. By arbsorbence, there exists $0<\epsilon \leq 1$ such that $\epsilon(v-w)\in \adh_\xi(C)-v$. Thus $v+ \epsilon(v-w)\in \adh_\xi(C)$. Also
\[ v - \epsilon(v-w) = (1-\epsilon)v - \epsilon w \in \inh_\xi(C). \]
Thus we have
\[ v = \frac{1}{2}\big(v - \epsilon(v-w)\big) + \frac{1}{2}\big(v + \epsilon(v-w)\big)\in \inh_\xi(C). \]
\end{proof}

\begin{proposition}
Let $\sSet{V,\xi}$ be a vector space convergence and $A\subseteq V$ a subspace. Then $\adh(A)$ is a subspace.
\end{proposition}
\begin{proof}
Clearly $\adh_\xi(A)$ is not empty. It is then enough to note that $\adh_\xi(A)+\adh_\xi(A)\subseteq \adh_\xi(A)$, by \ref{vectorSumInherenceAdherence}, and $\F\cdot \adh_\xi(A) \subseteq \adh_\xi(A)$, for which we use \ref{productAdherence} and \ref{adherenceInherenceContinuity} to compute
\begin{align*}
\F\cdot \adh_\xi(A) &= \cdot^\imf[\adh_\F(\F)\times \adh_\xi(A)] = \cdot^\imf[\adh_{\F\otimes \xi}(\F\times A)] \\
&\subseteq \adh_{\xi}\big(\cdot^\imf[\F\times A]\big) = \adh_{\xi}(\F\cdot A) = \adh_\xi(A).
\end{align*}
\end{proof}
\begin{corollary} \label{hyperplaneClosedDense}
A hyperplane in a convergence vector space is either closed or dense, but not both.
\end{corollary}
\begin{proof}
Let $H$ be a hyperplane in a vector space $V$. Then $H \subseteq \adh(H)$ and $\adh(H)$ is a subspace. Because $H$ is a coatom, we have either $\adh(H) = H$ or $\adh(H) = V$. In the first case $H$ is closed, in the second dense.

If the hyperplane were both closed and dense, then $H = \adh(H) = V$, which is a contradiction.
\end{proof}



\chapter{Algebraic convergence}
\begin{definition}
Let $V$ be a vector space over a field $\F$. The \udef{algebraic convergence} $\mathfrak{a}$ on $V$ is defined by
\[ F\overset{\mathfrak{a}}{\longrightarrow} v \quad\defequiv\quad \exists w\in V: \; \neighbourhood_\F(0)\cdot w \subseteq F - v \]
for all $F\in \powerfilters(V)$ and $v\in V$.
\end{definition}

\begin{lemma} \label{algebraicConvergenceStrongerThanVectorConvergence}
Let $V$ be a vector space over a field $\F$ and $\xi$ a vector space convergence on $V$. Then $\mathfrak{a} \subseteq \xi$.
\end{lemma}
\begin{proof}
This is immediate because $\neighbourhood_\F(0)\cdot w \overset{\xi}{\longrightarrow} 0$ and $F-v\overset{\xi}{\longrightarrow}0$ implies $F\overset{\xi}{\longrightarrow} v$.
\end{proof}

\begin{lemma}
Let $V$ be a vector space over a field $\F$. Then the following are equivalent:
\begin{enumerate}
\item $V$ is 1D;
\item $\mathfrak{a}$ is topological;
\item $\mathfrak{a}$ is pretopological;
\item $\mathfrak{a}$ is pseudotopological;
\item $\mathfrak{a}$ is of finite depth;
\item $\mathfrak{a}$ is a vector space convergence.
\end{enumerate}
\end{lemma}
\begin{proof}
If $V$ is 1D, then $V= \Span\{v\}$ for some/any $v\in V$ and $\neighbourhood_F(0)\cdot w = \neighbourhood_F(0)\cdot v$ for all $w\in v$. Thus $\sSet{V, \mathfrak{a}}$ is homeomorphic with $\F$ and $(2),(3),(4),(5),(6)$ follow.

Now $(2)\Rightarrow (3) \Rightarrow (4) \Rightarrow (5)$ is immediate by \ref{depthImplications}, so we prove $(5)\Rightarrow (1)$. Take arbitrary $v,w\in V$. Then $\neighbourhood_\F(0)\cdot v \cap \neighbourhood_\F(0)\cdot w$ converges to $0$ by finite depth, so there exists $u\in V$ such that $\neighbourhood_\F(0)\cdot u \subseteq \neighbourhood_\F(0)\cdot v\cap \neighbourhood_\F(0)\cdot w$. Thus for any $\epsilon > 0$, there exist $\epsilon_1, \epsilon_2 >0$ such that
\[ \cball(0,\epsilon_1)v \cup \cball(0,\epsilon_2)w \subseteq \cball(0,\epsilon)u. \]
In particular, $\epsilon_1v = \lambda_1 u$ and $\epsilon_2w = \lambda_2 u$ for some $\lambda_1,\lambda_2 \in\F$, which implies that $v,w$ are linearly dependent. If any two vectors are linearly dependent, then the vector space is 1D.

Finally we prove $(6)\Rightarrow (1)$. Assume that $V$ is not 1D, so we can find linearly independent $v,w\in V$. Construct the sequence $\seq{\frac{v+nw}{n^2}}_{n\in \N}$, which converges in any convergence vector space. Any set in the tails filter contains a pair of linearly independent vectors, so the tails filter does not contain a filter of the form $\neighbourhood_\F(0)\cdot u$ and thus does not converge in $\mathfrak{a}$, which is a contradiction.
\end{proof}

\begin{lemma} \label{constructionsInAlgebraicConvergence}
Let $V$ be a vector space over a field $\F$, $v\in V$ and $A\subseteq V$ a subset. Then
\begin{enumerate}
\item $w\mapsto w+v$ is a homeomorphism;
\item $\vicinity_\mathfrak{a}(v) = v+\vicinity_\mathfrak{a}(0)$;
\item $\begin{aligned}[t]
\vicinity_\mathfrak{a}(0) &= \bigcap_{v\in V} \upset \neighbourhood_\F(0)\cdot v \\
&= \setbuilder{B\in \powerset(V)}{\forall v\in V: \exists \Gamma_v\in \neighbourhood_\F(0):\; \Gamma_v\cdot v\subseteq B} \\
&= \setbuilder{\bigcup_{v\in V} \Gamma_v\cdot v}{\forall v\in V:\; \Gamma_v \in \neighbourhood_\F(0)};
\end{aligned}$
\item $\inh_\mathfrak{a}(A) = \setbuilder{x\in V}{\forall v\in V:\exists \Gamma_v \in \neighbourhood_\F(0):\; x + \Gamma_v\cdot v \subseteq A}$;
\item $\adh_\mathfrak{a}(A) = \setbuilder{x\in V}{\exists v\in V: \forall \Gamma\in\neighbourhood_\F(0):\; (x+\Gamma\cdot v)\mesh A}$.
\end{enumerate}
\end{lemma}
\begin{proof}
(1, 2, 3) are immediate from the definition and \ref{homeomorphismPreservation}.

(4) We have $\inh_\mathfrak{a}(A) = \setbuilder{x}{A\in \vicinity_\mathfrak{a}(x)} = \setbuilder{x}{A-x\in \vicinity_\mathfrak{a}(0)}$. From (1) we get 
\begin{align*}
\inh_\mathfrak{a}(A) &= \setbuilder{x}{\forall v\in V: \exists \Gamma_v\in \neighbourhood_\F(0): \Gamma_v\cdot v \subseteq A-x} \\
&= \setbuilder{x}{\forall v\in V: \exists \Gamma_v\in \neighbourhood_\F(0): x + \Gamma_v\cdot v \subseteq A}.
\end{align*}

(5) We calculate
\begin{align*}
\adh_\mathfrak{a}(A) &= \big(\inh_\mathfrak{a}(A^c)\big)^c \\
&= \setbuilder{x\in V}{\exists v\in V:\forall \Gamma \in \neighbourhood_\F(0):\; \neg(x + \Gamma\cdot v \subseteq A^c)} \\
&= \setbuilder{x\in V}{\exists v\in V:\forall \Gamma \in \neighbourhood_\F(0):\; (x + \Gamma\cdot v) \mesh A}.
\end{align*}
\end{proof}

\begin{lemma}
Let $V$ be a vector space. Then every subspace $U\subseteq V$ is algebraically closed.
\end{lemma}
\begin{proof}
We need to show that $\adh_\mathfrak{a}(U)\subseteq U$. Take $x\in \adh_\mathfrak{a}(U)$. Then take a $v\in V$ such that $\forall \Gamma\in\neighbourhood_\F(0):\; (x+\Gamma\cdot v)\mesh U$.

Pick some $\Gamma\in\neighbourhood_\F(0)$. Then $x+\lambda v\in U$ for some $\lambda\in \Gamma$. Then take $\ball(0,|\lambda|/2)\in \neighbourhood_\F(0)$, so $x+\mu v\in U$ for some $\mu\in \ball(0,|\lambda|/2)$. In particular $\lambda \neq \mu$. If either $\lambda =0$ or $\mu = 0$, then $x\in U$ and we are done. Suppose $\lambda\neq 0 \neq \mu$. Then
\[ \lambda^{-1}(x+\lambda v) - \mu^{-1}(x+\mu v) = (\lambda^{-1} - \mu^{-1})x \in U. \]
So $x\in U$.
\end{proof}

\begin{lemma}
Let $V$ be a vector space  over a field $\F$, $\sSet{W,\zeta}$ a convergence vector space over $\F$ and $f: V\to W$ a function. If $f$ is homogeneous, then $f: \sSet{V, \mathfrak{a}}\to \sSet{W,\zeta}$ is continuous at $0$.
\end{lemma}
\begin{proof}
Suppose $F\overset{\mathfrak{a}}{\longrightarrow} 0$, so there exists $v\in V$ such that $F\supseteq \neighbourhood_\F(0)\cdot v$. Then
\[ \upset f^{\imf\imf}(F) \supseteq \upset f^{\imf\imf}\big(\neighbourhood_\F(0)\cdot v\big) = \neighbourhood_\F(0)\cdot f(v) \overset{\zeta}{\longrightarrow} 0. \]
\end{proof}

\section{The core}
\begin{definition}
Let $V$ be a vector space and $A\subseteq V$ a subset. Then algebraic inherence $\inh_\mathfrak{a}(A)$ is also called the \udef{core} of $A$.
\end{definition}

\begin{proposition} \label{coreProperties}
Let $V$ be a vector space and $A \subseteq V$ a subset. Then
\begin{enumerate}
\item $A$ is absorbing \textup{if and only if} $0\in \inh_\mathfrak{a}(A)$;
\item if $A$ is convex, then $\inh_\mathfrak{a}(A)$ is convex;
\item if $A$ is balanced, then $\inh_\mathfrak{a}(A)$ is balanced.
\end{enumerate}
\end{proposition}
\begin{proof}
(1) We have that
\begin{align*}
\text{$A$ is absorbing} &\iff \forall v\in V: \exists \epsilon >0: \; \ball(0,\epsilon)\cdot v\subseteq A \\
&\iff \forall v\in V: \exists \Gamma \in \neighbourhood_\F(0): \; \Gamma\cdot v\subseteq A \\
&\iff 0\in \inh_\mathfrak{a}(A).
\end{align*}

(2) Take $x,y \in \inh_\mathfrak{a}(A)$. Take an arbitrary $v\in V$. Then there exist $\Gamma_v,\Gamma_v'\in \neighbourhood_\F(0)$ such that $x+ \Gamma_v\cdot v \subseteq A$ and $y+ \Gamma_v'\cdot v \subseteq A$. Then
\begin{align*}
\lambda x+(1-\lambda)y + (\Gamma_v\cap\Gamma_v')\cdot v &= \lambda x+(1-\lambda)y + (\Gamma_v\cap\Gamma_v')\big(\lambda v+(1-\lambda)v\big) \\
&= \lambda \big(x+ (\Gamma_v\cap\Gamma_v')\cdot v\big) + (1-\lambda)\big(y+ (\Gamma_v\cap\Gamma_v')\cdot v\big) \\
&\subseteq \lambda \big(x+ \Gamma_v\cdot v\big) + (1-\lambda)\big(y+ \Gamma_v'\cdot v\big) \\
&\subseteq A,
\end{align*}
where the last inclusion follows from \ref{convexCriteria}. Since $\Gamma_v\cap\Gamma_v'\in\neighbourhood_\F(0)$, we have that $\lambda x+(1-\lambda)y\in \inh_\mathfrak{a}(A)$.

(3) Take $x\in \inh_\mathfrak{a}(A)$ and $|r|\leq 1$. We need to show that $rx\in \inh_\mathfrak{a}(A)$. To that end, take arbitrary $v\in V$. Then there exists $\Gamma_v\in\neighbourhood_\F(0)$ such that $x+\Gamma_v\cdot v\subseteq A$. Now
\[ rx+ r\Gamma_v\cdot v = r\big(x+\Gamma_v\cdot v\big) \subseteq rA \subseteq A. \]
Since $r\Gamma_v\in \neighbourhood_\F(0)$, this implies $rx\in \inh_\mathfrak{a}(A)$.
\end{proof}

\begin{proposition} \label{algebraicallyOpen}
Let $V$ be a vector space and $A \subseteq V$ an algebraically open subset. Then
\begin{enumerate}
\item $A+U$ is algebraically open for any subspace $U\subseteq V$;
\end{enumerate}
\end{proposition}
\begin{proof}
TODO
\end{proof}

\begin{lemma} \label{rectangleSubsetLemma}
Let $V$ be a real vector space and $U\subseteq V$ an algebraically open convex subset. For all $a\in U$ and $u,v\in V$, there exists $\epsilon > 0$ such that
\[ a + \interval{0,\epsilon}\cdot u + \interval{0,\epsilon}\cdot v \subseteq U. \]
\end{lemma}
\begin{proof}
We have $0\in a-U = \inh_{\mathfrak{a}}(a-U)$, by \ref{constructionsInAlgebraicConvergence}, so $a-U$ is absorbing by \ref{coreProperties}.

Thus there exist $\epsilon_1,\epsilon_2 > 0$ such that
\[ \interval{0,\epsilon_1}\cdot \{u\} \subseteq \cball(0,\epsilon_1)\cdot \{u\} \subseteq U-a \]
and
\[ \interval{0,\epsilon_2}\cdot \{v\} \subseteq \cball(0,\epsilon_2)\cdot \{v\} \subseteq U-a. \]
Now set $\epsilon \defeq \frac{1}{2}\min\{\epsilon_1, \epsilon_2\}$, so $a + 2\interval{0,\epsilon}\cdot u \subseteq U$ and $a + 2\interval{0,\epsilon}\cdot v \subseteq U$. By convexity,
\[ a + \interval{0,\epsilon}\cdot u + \interval{0,\epsilon}\cdot v = \frac{1}{2}\big(a + 2\interval{0,\epsilon}\cdot u\big) + \frac{1}{2}\big(a + 2\interval{0,\epsilon}\cdot v\big) \subseteq \frac{1}{2}U + \frac{1}{2}U \subseteq U. \]
\end{proof}

\subsection{Strict convexity}
\begin{definition}
Let $V$ be a vector space and $X\subseteq V$ a subset. Then $X$ is called \udef{strictly convex} if
\[ \forall x\neq y \in X: t\in\interval[o]{0,1}: \quad tx + (1-t)y\in \inh_\mathfrak{a}(X). \]
\end{definition}

\begin{proposition} \label{strictConvexityEquivalentsConvexSubset}
Let $V$ be a vector space and $X\subseteq V$ a convex subset.
Then the following are equivalent:
\begin{enumerate}
\item $X$ is strictly convex;
\item $X\setminus \ext(X) \subseteq \inh_\mathfrak{a}(X)$;
\item $X\setminus \inh_\mathfrak{a}(X) \subseteq \ext(X)$.
\end{enumerate}
\end{proposition}
\begin{proof}
$(1) \Rightarrow (2)$ Suppose $X$ is strictly convex and take $a\in X\setminus \ext(X)$. By \ref{notExtremePointLemma} there exists $x\neq y\in X$ and $t\in\interval[o]{0,1}$ such that $a  = tx+(1-t)y$. Then, by assumption, $a\in \inh_\mathfrak{a}(X)$.

$(2) \Rightarrow (1)$ Take arbitrary $x\neq y\in X$ and $t\in \interval[o]{0,1}$. Then $tx+(1-t)y\in X$ by convexity, but $tx+(1-t)y\notin \ext(X)$ by \ref{notExtremePointLemma}. So $tx+(1-t)y \in X\setminus \ext(X)$ and thus $tx+(1-t)y\in \inh_\mathfrak{a}(X)$ by assumption.

$(2) \Leftrightarrow (3)$ Immediate.
\end{proof}
\begin{corollary}
Let $\sSet{V,\xi}$ be a convergence vector space and $X\subseteq V$ a convex subset. Then
\begin{enumerate}
\item if $X\setminus \inh_\xi \subseteq \ext(X)$, then $X$ is strictly convex;
\item if $X$ is open, then $X$ is strictly convex.
\end{enumerate}
\end{corollary}
\begin{proof}
(1) Since $\mathfrak{a}\leq \xi$, by \ref{algebraicConvergenceStrongerThanVectorConvergence}, we have $\inh_\xi(X)\subseteq \inh_\mathfrak{a}(X)$, by \ref{principalInherenceAdherenceProperties}. Thus $X\setminus \inh_\xi \subseteq \ext(X)$ implies $X\setminus \inh_\mathfrak{a}\subseteq X\setminus \inh_\xi \subseteq \ext(X)$ and thus that $X$ is strictly convex by the proposition.

(2) If $X$ is $\xi$-open, then it is $\mathfrak{a}$-open by \ref{openClosedConvergenceInclusions} and \ref{algebraicConvergenceStrongerThanVectorConvergence}. Thus $X = \inh_\mathfrak{a}(X)$, so $X\setminus \inh_\mathfrak{a}(X) = \emptyset \subseteq \ext(X)$.
\end{proof}

\section{The algebraic dual}
\begin{definition}
Let $V$ be a vector space. The \udef{algebraic dual} of $V$ is the set $\contLin(\sSet{V,\mathfrak{a}}, \F)$, where $\mathfrak{a}$ is the algebraic convergence. It is denoted $\dual{V}$.
\end{definition}

\begin{proposition} \label{algebraicDual}
Let $V$ be a vector space. Then the algebraic dual of $V$ is the set of all linear functionals: $V^* = \Lin(V,\F)$.

Thus $V^* \supseteq \sSet{V,\xi}^*$ for all vector space convergences $\xi$ on $V$.
\end{proposition}
\begin{proof}
We need to show that all linear functionals are continuous when $V$ is equipped with the algebraic convergence. Assume $F\overset{\mathfrak{a}}{\longrightarrow} x$. Then there exists a $v\in V$ such that $\neighbourhood_\F(0)\cdot v+x \subseteq F$ and so $\neighbourhood_\F(0)\cdot f(v)+f(x) \subseteq f^\imf[F]$, meaning $f^\imf[F] \overset{\F}{\longrightarrow} f(x)$. Thus $f$ is continuous.
\end{proof}


\begin{proposition} \label{dualBasisDimension}
Let $V$ be a vector space. Then $\dim V^* \geq \dim V$ and
\[ \dim V^* = \dim V \iff \text{$V$ is finite-dimensional}. \]
If $V$ is finite-dimensional with a basis $v_1, \ldots, v_n$, then the \udef{dual basis} $\varphi_1, \ldots, \varphi_n$ is the set of linear functionals on $V$ such that
\[ \varphi_j(v_k) = \begin{cases}
1 & (k=j), \\ 0 & (k\neq j)
\end{cases}. \]
This dual basis is indeed a basis of $V^*$.
\end{proposition}
\begin{proof}
We first assume $V$ is finite-dimensional and prove the dual basis is a basis, which proves $\dim V^* = \dim V$. We then assume $V$ is infinite-dimensional and prove $\dim V^* \neq \dim V$.\footnote{Reference: \url{https://mathoverflow.net/questions/13322/slick-proof-a-vector-space-has-the-same-dimension-as-its-dual-if-and-only-if-i}}
\begin{enumerate}
\item Assume $V$ is finite-dimensional. To show the dual basis spans $V^*$, take a linear functional $\varphi$. Now define $a_i = \varphi(v_i)$. It is clear that $\varphi = \sum_{i=1}^n a_i\varphi_i$. To show linear independence, take a combination
\[ b_1\varphi_1 + \ldots +b_n\varphi_n =0. \]
Filling in all basis vectors $v_i$ in turn, gives $b_i=0$ for all $i$.
\item Assume $V$ is infinite-dimensional. At first let us assume $\dim_{\mathbb{F}}V \geq |\mathbb{F}|$. Then we can apply lemma \ref{vsCardinality} to obtain $\dim_{\mathbb{F}}V = |V|$. Let $\beta$ be a basis for $V$. The elements of $V^*$ correspond bijectively to functions from $\beta$ to $\mathbb{F}$. Thus
\[ |V^*| = |\mathbb{F}^\beta| = |\mathbb{F}|^{|\beta|} > |\beta| = |V|. \]
Now we relax the condition $\dim_{\mathbb{F}}V \geq |\mathbb{F}|$. We first note that every field contains a subfield that is at most denumerable. Take such a field $K\subset \mathbb{F}$. We introduce the new vector space $W = \Span_K(\beta)$. Every functional from $W$ to $K$ extends to a functional from $V$ to $\mathbb{F}$. Hence
\[ \dim_\mathbb{F} V = \dim_K W < \dim_K W^* \leq \dim_{\mathbb{F}} V^* \]
using $\dim_{K}W \geq |K| \geq \aleph_0$.
\end{enumerate}
\end{proof}
\begin{corollary}
Let $V$ be a finite-dimensional vector space. Then the algebraic convergence is the unique Hausdorff vector space convergence on $V$.
\end{corollary}
\begin{proof}
Consider a basis $v_1, \ldots, v_n$ of $V$ with dual basis $\varphi_1, \ldots, \varphi_n$. Let $\xi$ be some vector space convergence. By definition we have $\mathfrak{a} \subseteq \xi$. Now take $F \overset{\xi}{\longrightarrow} v$. We have $F = v_1\cdot \varphi_1^{\imf\imf}[F] + \ldots + v_n\cdot \varphi_n^{\imf\imf}[F]$. Now each $\varphi_1^{\imf\imf}[F]$ converges in both $\mathfrak{a}$ and $\xi$ by \ref{algebraicDual} and the proposition, so by continuity of addition and scalar multiplication, $F$ also converges in $\mathfrak{a}$. 
\end{proof}
\begin{corollary}
Let $\sSet{V,\xi}$ be a convergence vector space. If $V$ is finite-dimensional, then $\sSet{V,\xi}^* = \sSet{V,\mathfrak{a}}^*$.
\end{corollary}
\begin{proof}
We have $\sSet{V,\xi}^* \subseteq \sSet{V,\mathfrak{a}}^*$ by \ref{algebraicDual}. Because $V$ is finite-dimensional, we obtain equality equality of space from equality of dimension by \ref{vectorSpaceEquality}.
\end{proof}

\begin{proposition} \label{algebraicDualComplete}
Let $V$ be a vector space over the field $\F$. Then $\Lin(V,\F) = \dual{\sSet{V, \mathfrak{a}}}$ is pointwise-complete.
\end{proposition}
\begin{proof}
Let $F\in\powerfilters\big(\Lin(V,\F)\big)$ be a Cauchy filter. Take a $v\in V$ then $\evalMap_v: \Lin(V,\F)\to \F$ is continuous and linear, which means it is uniformly continuous by \ref{uniformContinuityGroupHomomorphism}. Thus $\upset\evalMap_v^{\imf\imf}(F)$ is Cauchy by \ref{continuousImageOfCauchy}. Now $\upset\evalMap_v^{\imf\imf}(F)$ converges to a unique $y$ in $\F$ by Hausdorffity and completeness of $\F$. Consider the function $f: V\to \F$ that maps each such $v$ to the corresponding $y$.

It is clear that $f$ is linear and thus in $\Lin(V,\F)$. By \ref{initialFinalConvergence}, $F$ converges to this $f$ in the pointwise convergence.
\end{proof}
TODO $\big(\Lin(V,\F),V\big)$ \udef{semi-Montel space}.

\subsection{The bidual space}
TODO!
\begin{definition}
Let $V$ be a convergence vector space. The \udef{bidual space} is the dual of the dual $\abidual{V} = \adual{(\adual{V})}$.
\end{definition}
TODO continuous convergence!!

\begin{definition}
Let $V$ be a vector space over $\mathbb{F}$ and $v\in V$. The \udef{evaluation map} $\evalMap: V\to \abidual{V}: v\mapsto \evalMap_v$ is given by
\[ \evalMap_v: \adual{V} \to \mathbb{F}: l\mapsto l(v). \]
\end{definition}

\begin{lemma}
Let $V$ be a vector space. The evaluation map $\evalMap: V\to \abidual{V}: v\mapsto \evalMap_v$ is linear:
\[ \forall v,w\in V, a\in\mathbb{F}: \quad \evalMap_{av + w} = a\evalMap_v + \evalMap_w. \]
\end{lemma}
\begin{lemma}
Let $V$ be vector space over $\mathbb{F}$. The evaluation map is injective.
\end{lemma}
\begin{proof}
Assume $\evalMap_v = \evalMap_w$ for some $v,w\in V$. Then
\[ 0 = \evalMap_v - \evalMap_w  = \evalMap_{v-w}. \]
So $\forall l\in \adual{V}: \evalMap_{v-w}(l) = l(v-w) = 0$. Now define the sublinear functional by
\[ p(x) = \begin{cases}
\alpha & x = \alpha(v-w) \\
0 & \text{else}.
\end{cases} \]
Then the functional $f$ defined on $\Span\{v-w\}$ by $f(\alpha(v-w)) = \alpha$ is bounded by $p$ and can be extended to a functional on all $V$ by the Hahn-Banach theorem \ref{sublinearHahnBanach} if $v-w\neq 0$. Then $f(v-w) \neq 0$, which contradicts our assumptions. Thus $v=w$.
\end{proof}

\begin{proposition}
The mapping $\evalMap: V\to \abidual{V}: v\mapsto \evalMap_v$ is an isomorphism \textup{if and only if} $V$ is finite-dimensional.
\end{proposition}
\begin{proof}
Assume $V$ finite dimensional. As the evaluation map is injective, it is an isomorphism by \ref{invertibleFiniteDim}.
The other direction is a dimensional argument by proposition \ref{dualBasisDimension}.
\end{proof}




\chapter{Functionals}
\begin{definition}
Let $X$ be a set and $\F$ a field.
\begin{itemize}
\item A \udef{functional} on $X$ is a map $V\to \F$;
\item A \udef{real functional} on $X$ is a map $V\to \R$.
\end{itemize}
If $X$ is a vector space on $\F$, then a \udef{linear functional} is a functional that is a linear function.

If $X$ is a convergence space, then a \udef{continuous functional} is a functional that is a continuous function.
If $\F = \C$, then the set of continuous functionals is denoted $\cont(X)$.
\end{definition}

\begin{proposition}
Let $\sSet{X, \xi}$ be a convergence space and $\F$ a field. Then $\cont_c(X, \F)$ with pointwise operations is a complete convergence vector space.
\end{proposition}
\begin{proof}
TODO Beattie / Butzmann p84.
\end{proof}
\begin{corollary}
Let $\sSet{X, \xi}$ be a convergence space and $\F$ a field. Then $\contLin_c(X, \F)$ with pointwise operations is a complete convergence vector space.
\end{corollary}
\begin{proof}
As $\F$ is Hausdroff, $\contLin_c(X, \F)$ is closed by \ref{linearFunctionsClosedSubset} and thus complete by \ref{closedComplete}.
\end{proof}

\begin{lemma} \label{continuityDominatedFunctional}
Let $V$ be a TVS and $f:V\to \F$ a continuous functional. If $g:V\to \F$ is a functional such that $|g(v)|\leq |f(v)|$ for all $v\in V$, then $g$ is continuous.
\end{lemma}
\begin{proof}
We use \ref{pretopologicalContinuityVicinities} to show continuity. To that end take $K\in \neighbourhood_\F(0)$. Then there exists $\epsilon >0$ such that $\ball(0,\epsilon)\subseteq K$ and so
\[ g^{\preimf}(K) \supseteq g^\preimf[\ball(0,\epsilon)] \supseteq f^\preimf[\ball(0,\epsilon)] \in \neighbourhood_V(0). \]
\end{proof}


\section{Real functionals}
\begin{definition}
Let $V$ be a real or complex vector space. Let $f: V\to \R$ be a real functional. We say
\begin{itemize}
\item $f$ is \udef{subadditive} or satisfies the \udef{triangle inequality} if $\forall x,y\in V: f(x+y) \leq f(x) + f(y)$;
\item $f$ is \udef{quasi-subadditive} if $\exists K>0: \forall x,y\in V: f(x+y) \leq K\big(f(x) + f(y)\big)$;
\item $f$ is \udef{point-separating} if $\forall x\in V: f(x) = 0 \implies x = 0$.
\end{itemize}
We call $f$
\begin{itemize}
\item \udef{sublinear} if it is subadditive and positively homogeneous;
\item a \udef{seminorm} if it is subadditive and absolutely homogeneous;
\item a \udef{quasi-seminorm} if it is quasi-subadditive and absolutely homogeneous;
\item a \udef{norm} is a point-separating seminorm;
\item a \udef{quasi-norm} is a point-separating quasi-seminorm.
\end{itemize}
Let $C\subseteq V$ be a convex subset. Then we call a real functional $g: C \to \R$
\begin{itemize}
\item \udef{convex} if $\forall x,y\in V, \lambda\in[0,1]: g(\lambda x + (1-\lambda)y) \leq \lambda g(x) + (1-\lambda)g(y)$.
\end{itemize}
\end{definition}

TODO general valued fields.

\begin{lemma}
Let $V$ be a real or complex vector space and $f: V\to \R$ be a real functional. Then
\begin{enumerate}
\item absolute homogeneity $\implies$ positive homogeneity;
\item subadditivity $\implies$ quasi-subadditivity;
\item subadditivity+positive homogeneity $\implies$ convexity $\implies$ subadditivity.
\end{enumerate}
\end{lemma}
Thus norms and seminorms are sublinear.

\begin{lemma} \label{seminormPositivity}
Let $f:V\to \R$ be a quasi-seminorm. Then $\im(f)\subseteq \R^+$.
\end{lemma}
Thus (quasi)-seminorms are often considered as function in $V\to \R^+$.
\begin{proof}
For all $v\in V$ we have $0 = f(v-v) \leq K\big(f(v)+f(-v)\big) = 2Kf(v)$, so $f(v) \geq 0$.
\end{proof}

\begin{proposition}[Reverse triangle inequality] \label{reverseTriangleInequality}
Let $V$ be a vector space and $\norm{\cdot}: V\to \R$ a function that satisfies the triangle inequality and has $\norm{-v} = \norm{v}$ for all $v\in V$. Then $\forall v,w\in V$:
\begin{enumerate}
\item $|\norm{v}-\norm{w}|\leq \norm{v-w}$;
\item $|\norm{v}-\norm{w}|\leq \norm{v+w}$.
\end{enumerate}
In particular this holds if $\norm{\cdot}$ is a norm or seminorm.
\end{proposition}
\begin{proof}
We calculate $\norm{v} = \norm{v-w+w} \leq \norm{v-w} + \norm{w}$, so $\norm{v}-\norm{w}\leq \norm{v-w}$. By swapping $v\leftrightarrow w$ we also get $-\norm{v}+\norm{w}\leq \norm{w-v} = \norm{v-w}$ and thus the first inequality is established.

For the second inequality, set $w\to -w$ and use $\norm{-w} = \norm{w}$.
\end{proof}

\subsection{Extended real functionals}
\begin{definition}
Let $V$ be a real or complex vector space. An \udef{extended real functional} is a function $V \to \overline{\R}$.
\end{definition}

\begin{lemma} \label{realPartExtendedRealFunctional}
Let $V$ be a real or complex vector space and $f: V\to \overline{\R}$ an extended real functional. If $f$ is a quasi-seminorm and there exists $v\in V$ such that $f(v)\in\R$, then $f^{\preimf}(\R)$ is a subspace of $V$. 
\end{lemma}
\begin{proof}
We verify the subspace criteria (\ref{subspaceCriterion}). 

By assumption, $f^{\preimf}(\R)$ is not empty.

Take $v,w\in f^{\preimf}(\R)$. As $\im(f) \subseteq \overline{\R}^+$ by \ref{seminormPositivity}, we have $0\leq f(v+w)$. Also $f(v+w)\leq K\big(f(v)+f(w)\big)\in \R$. Thus $f(v+w)\in\R$.

Take $\lambda\in\F$. Then $f(\lambda v) = |\lambda|f(v)\in \R$.
\end{proof}

\subsection{Epigraphs}
\begin{definition}
Let $V$ be a vector space and $f: V\to \R$ a real functional on $V$. Then \udef{epigraph} of $f$ is defined as
\[ \epigraph(f) \defeq \setbuilder{(v,r)\in V\times \R}{f(v)\leq r}. \]
\end{definition}

\begin{lemma} \label{epigraphLemma}
Let $V$ be a vector space and $f: V\to \R$ a real functional on $V$. Then for all $v\in V$:
\[ f(v) = \inf\setbuilder{r}{(v,r)\in \epigraph(f)}. \]
\end{lemma}

\begin{proposition} \label{epigraphProperties}
Let $V$ be a real vector space and $f: V\to \R$ a functional. Then
\begin{enumerate}
\item $f$ is convex \textup{if and only if} $\epigraph(f)$ is a convex subset of $V\oplus \R$;
\item $f$ is positively homogeneous \textup{if and only if} $\epigraph(f)$ is a cone in $V\oplus \R$.
\end{enumerate}
\end{proposition}
\begin{proof}
(1) First assume $f$ convex and pick $(v, s), (w,t)\in \epigraph(f)$ and $\lambda\in [0,1]$. Then we need to show that $(\lambda v + (1-\lambda)w, \lambda s + (1-\lambda)t) \in \epigraph(f)$. This is equivalent to saying $f(\lambda v + (1-\lambda)w) \leq \lambda s + (1-\lambda)t$. Indeed we have $f(\lambda v + (1-\lambda)w) \leq \lambda f(v) + (1-\lambda)f(w) \leq \lambda s + (1-\lambda)t$ by the convexity of $f$.

Conversely, assume $\epigraph(f)$ convex. Then $(v, f(v)), (w,f(w))\in \epigraph(f)$, $(\lambda v + (1-\lambda)w, \lambda f(v) + (1-\lambda)f(w)) \in \epigraph(f)$ for all $\lambda\in [0,1]$. This implies $f(\lambda v + (1-\lambda)w) \leq \lambda f(v) + (1-\lambda)f(w)$.

(2) First assume $f$ is positively homogeneous, take $(v,s)\in \epigraph(f)$ and $r>0$. Then we need to show that $r(v,s) = (rv,rs)\in \epigraph(f)$. This follows because of the implications $f(v)\leq s \implies rf(v) \leq rs \implies f(rv) \leq rs$.

Conversely, assume that $\epigraph(f)$ is a cone. Then $\lambda\cdot \epigraph(f) = \epigraph(f)$ for all $\lambda>0$ by \ref{coneEqualityLemma}. We then calculate using \ref{epigraphLemma}:
\begin{align*}
f(\lambda v) &= \inf\setbuilder{r}{(\lambda v,r)\in \epigraph(f)} \\
&= \inf\setbuilder{r}{(\lambda v,r)\in \lambda\cdot\epigraph(f)} \\
&= \inf\setbuilder{r}{\lambda(v,\lambda^{-1}r)\in \lambda\cdot\epigraph(f)} \\
&= \inf\setbuilder{r}{(v,\lambda^{-1}r)\in \epigraph(f)} \\
&= \inf\setbuilder{\lambda r}{(v,r)\in \epigraph(f)} = \lambda f(v).
\end{align*} 
\end{proof}
\begin{corollary}
A functional on a real vector space is sublinear \textup{if and only if} its epigraph is a convex cone.
\end{corollary}

\subsection{Types of real functionals}
\subsubsection{Convex functionals}
TODO can a convex functional on a convex set be extended to a vector space??

\begin{lemma} \label{preimageConvexSetConvexFunctionalIsConvex}
Let $V$ be a vector space, $C\subseteq V$ a convex subset, $f:C\to \R$ a convex functional and $\epsilon >0$. Then $f^{\preimf}(\interval{0,\epsilon})$ is a convex set.
\end{lemma}
\begin{proof}
Take $x,y\in f^{\preimf}(\interval{0,\epsilon})$ and $0\leq r\leq 1$. Then
\[ f\big(rx + (1-r)y\big) \leq rf(x) + (1-r)f(y) \leq r\epsilon + (1-r)\epsilon = \epsilon, \]
as $f(x), f(y)\leq \epsilon$. So $rx + (1-r)y \in f^{\preimf}(\interval{0,\epsilon})$ and $f^{\preimf}(\interval{0,\epsilon})$ is convex.
\end{proof}

\begin{lemma} \label{convexContinuityLemma}
Let $V$ be a vector space over a field $\F$, $C\subseteq V$ a convex subset and $f:C\to \R$ a convex functional. Let $v,w\in V$ be such that $v-w, v, v+w\in C$ and $\delta\in \interval{0,1}$. Then
\[ |f(v+\delta w) - f(v)| \leq \delta \Big(\max\big\{ f(v+w), f(v-w) \big\} - f(v)\Big). \]
\end{lemma}
\begin{proof}
From $v+\delta w = v - \delta v + \delta v + \delta w = (1-\delta)v+\delta(v+w)$, we get $f(v+\delta w)\leq (1-\delta)f(v) + \delta f(v+w)$ and thus
\[ f(v+\delta w) - f(v)\leq \delta\big(f(v+w)- f(v)\big). \]
Replacing $w$ by $-w$ similarly gives $f(v-\delta w)-f(v)\leq \delta\big(f(v-w)- f(v)\big)$. Now $v = \frac{1}{2}(v+\delta w) + \frac{1}{2}(v-\delta w)$, so $f(v)\leq \frac{1}{2}f(v+\delta w) + \frac{1}{2}f(v-\delta w)$, which we can rewrite as
\[ -f(v+\delta w) \leq f(v-\delta w) - 2f(v). \]
Then we calculate
\begin{align*}
|f(v+\delta w) - f(v)| &= \max\big\{f(v+\delta w) - f(v), f(v) - f(v+\delta w)\big\} \\
&\leq \max\big\{\delta\big(f(v) + f(v+w)\big), f(v) + f(v-\delta w) - 2f(v)\big\} \\
&= \max\big\{\delta\big(f(v) + f(v+w)\big), f(v-\delta w) - f(v)\big\} \\
&\leq \max\big\{\delta\big(f(v) + f(v+w)\big), \delta\big(f(v-w)- f(v)\big)\big\} \\
&= \delta \Big(\max\big\{ f(v+w), f(v-w) \big\} - f(v)\Big).
\end{align*}
\end{proof}

\begin{proposition} \label{convexContinuity}
Let $\sSet{V,\xi}$ be an real convergence vector space, $C\subseteq V$ an open convex set, $f: C\to \R$ a convex functional and $v\in C$. Then each of the following points implies the next:
\begin{enumerate}
\item $f$ is continuous;
\item $f$ is continuous at $v$;
\item $f$ is bounded on some vicinity $U\in\vicinity_\xi(v)$ for some point $v\in C$;
\item for all $w\in C$ there exists a vicinity $U'\in\vicinity_\xi(w)$ such that $f^\imf(U')$ has an upper bound.
\end{enumerate}
If $\sSet{V,\xi}$ is equable, then they are all equivalent.
\end{proposition}
\begin{proof}
The implication $(1) \Rightarrow (2)$ is immediate. 

$(2) \Rightarrow (3)$ 
We have $\vicinity_\R\big(f(v)\big) \subseteq \upset f^{\imf\imf}\big(\vicinity_{\xi|_C}(v)\big)$ by \ref{continuityVicinityFilter}, so there exists $U\in \vicinity_{\xi|_C}(v)$ such that $f^\imf(U) \subseteq \interval{f(v)-1, f(v)+1}$. By \ref{subspaceVicinityFilter}, we may take $U\in \vicinity_\xi(v)$.

$(3) \Rightarrow (4)$ Pick arbitrary $w\in C$. The function $g: \lambda \mapsto w + \lambda(w-v)$ is continuous. By \ref{openClosedCriteria}, we can find an $A\in \vicinity_\xi(w)$ such that $A\subseteq C$. Then $g^\preimf(A)$ is a vicinity of $0$ by \ref{continuityVicinityFilter}. In particular, there exists $\epsilon \in g^\preimf(A)$ for some $\epsilon >0$, so $u = g(\epsilon) = w + \epsilon(w-v) \in A\subseteq C$. Rearranging gives $w = \frac{1}{1+\epsilon}u + \frac{\epsilon}{1+\epsilon}v$.

We may take $U\in\vicinity_\xi(v)$ such that $f^\imf(U)$ is bounded. Let $M$ be an upper bound of $f^\imf(U)$. We claim that $f$ is upper bounded on
\[ U' \defeq w + \frac{\epsilon}{1+\epsilon}(U-v) = \frac{1}{1+\epsilon}u + \frac{\epsilon}{1+\epsilon}U. \]
Indeed, for all $u'\in U$ we have
\[ f\Big(\frac{1}{1+\epsilon}u + \frac{\epsilon}{1+\epsilon}u'\Big) \leq \frac{1}{1+\epsilon}f(u) + \frac{\epsilon}{1+\epsilon}f(u') \leq \frac{1}{1+\epsilon}f(u) + \frac{\epsilon}{1+\epsilon}M. \]

$(4) \Rightarrow (1)$ Take arbitrary $w$. By \ref{pretopologicalContinuityVicinities}, it is enough to show $\vicinity_\R\big(f(w)\big) \subseteq \upset f^{\imf\imf}\big(\vicinity_\xi(w)\big)$.

Any vicinity of $f(w)$ contains a vicinity of the form $\interval{f(w)-\epsilon, f(w)+\epsilon}$ for some $\epsilon >0$. Now, by assumption and \ref{equableConvergenceBalancedBase}, we can find a vicinity $w+A$ of $w$ such that $A$ is balanced and $f^\imf(w+A)$ is bounded above, say by $f(w)+M$ for some $M \geq 0$.

Now take $0 < \delta \leq 1$ such that $\delta M \leq \epsilon$. Then we claim that $f^\imf(w + \delta A) \subseteq \interval{f(w)-\epsilon, f(w)+\epsilon}$. Indeed, take $a\in A$. Then $-a$ is also an element of $A$, so $f(w+a), f(w-a)\leq f(w)+M$. By \ref{convexContinuityLemma}, we have
\begin{align*}
|f(w+\delta A) - f(w)| &\leq \delta \Big(\max\big\{ f(w+a), f(w-a) \big\} - f(w)\Big) \\
&\leq \delta(f(w) + M - f(w)) = \delta M \leq \epsilon.
\end{align*}
\end{proof}

TODO picture Aliprantis / Border p. 189

TODO: also works for algebraic convergence?

\begin{example}
A convex functional does not need to be continuous at the boundary points of its domain of definition. For example, the function
\[ f: \interval{0,1}\to \R: x\mapsto \Iverson{x=0} \]
is convex, but not continuous at $0$.
\end{example}

\begin{proposition}
Let $p: V\to\R$ be convex functional. Then
\[ P: V\to\R: x\mapsto \inf_{t>0} t^{-1}p(tx) \]
is sublinear and $P(x)\leq p(x)$.

Also, if $f:V\to \R$ is a linear functional, then $f\leq p \iff f\leq P$.
\end{proposition}
\begin{proof}
For sublinearity: let $x,y\in V$, then for all $s,t>0$
\[ P(x+y) \leq \frac{s+t}{st}p\left(\frac{st}{s+t}(x+y)\right) = \frac{s+t}{st}p\left(\frac{s}{s+t}(tx)+\frac{t}{s+t}(sy)\right) \leq t^{-1}p(tx) + s^{-1}p(sy). \]
This implies that $P(x+y)\leq P(x)+P(y)$.

For positive homogeneity: let $x\in V,\lambda\geq 0$
\[ P(\lambda x) = \inf_{t>0} t^{-1}p(t\lambda x) = \inf_{t\lambda>0} \lambda (t\lambda)^{-1}p(t\lambda x) = \inf_{t>0} \lambda (t)^{-1}p(tx) = \lambda P(x). \]

Finally we prove that $f\leq p \implies f\leq P$ for linear functionals $f$. For all $t>0$ we have $f(tx) \leq p(tx)$, which implies $f(x) = t^{-1}f(tx) \leq t^{-1}p(tx) \leq P(x)$. So $f\leq P$.
\end{proof}

\subsubsection{Sublinear functionals}

\begin{lemma}
Let $V$ be a \emph{real} vector space and $f: V \to \R$ a sublinear functional. Then $f': V\to \R: x\mapsto \max\{f(x), f(-x)\}$ is a seminorm.
\end{lemma}
We call the seminorm $f'$ the \udef{associated seminorm} of the sublinear functional.
\begin{proof}
Take arbitrary $x,y\in V$ and $a\in \R$

For subadditivity, we have
\begin{align*}
f'(x+y) &= \max\{f(x+y), f(-x-y)\} \\
&\leq \max\{f(x)+f(y), f(-x)+ f(-y)\} \\
&\leq \max\{f(x)+f(y), f(-x)+ f(-y), f(x) + f(-y), f(-x) + f(y)\} \\
&= \max\{f(x), f(-x)\} + \max\{f(y), f(-y)\} \\
&= f'(x) + f'(y).
\end{align*}

For absolute homogeneity, we have,
\begin{align*}
f'(ax) &= \begin{cases}
f'(|a|x) & (a \geq 0) \\ f'(-|a|x) & (a < 0)
\end{cases} \\
&= \begin{cases}
\max\{f(|a|x), f(-|a|x)\} & (a \geq 0) \\ \max\{f(-|a|x), f(|a|x)\} & (a < 0)
\end{cases} \\
&= \max\{|a|f(x), |a|f(-x)\} = |a|f'(x).
\end{align*}
\end{proof}

\begin{lemma} \label{superSubtractiveContinuityEverywhere}
Let $\sSet{V, \xi}$ be a convergence vector space and $f: V\to \R$ a function such $f(0) = 0$ and $|f(v) - f(w)| \leq \max\{|f(v-w)|, |f(w-v)|\}$ for all $v,w\in V$.

Then continuity of $f$ at $0$ implies continuity everywhere.
\end{lemma}
\begin{proof}
Take arbitrary $F\overset{\xi}{\longrightarrow} u$. Then $F-u \to 0$, so $|f^{\imf\imf}(F-u)|\to 0$ by continuity at $0$. Similarly $|f^{\imf\imf}(u-F)|\to 0$. Then, by assymption, $|f^{\imf\imf}(F) - f(u)| \leq \max\{|f^{\imf\imf}(F-u)|, |f^{\imf\imf}(u-F)|\} \to 0$, so $|f^{\imf\imf}(F) - f(u)| \to 0$, by TODO ref squeeze theorem. We conclude that $f^{\imf\imf}(F)\to f(u)$.
\end{proof}

\begin{proposition} \label{sublinearContinuity}
Let $\sSet{V,\xi}$ be a convergence vector space and $f: V\to \R$ a sublinear functional. Then the following are equivalent:
\begin{enumerate}
\item $f$ is continuous;
\item $f$ is continuous at $0$;
\item $f$ is bounded on some $U\in\vicinity_\xi(0)$.
\end{enumerate}
\end{proposition}
\begin{proof}
The implication $(1) \Rightarrow (2)$ is immediate. 

The implication $(2) \Rightarrow (1)$ follows from \ref{superSubtractiveContinuityEverywhere}: 
for arbitrary $v,w\in V$, we have $f(v) = f(v-w+w) \leq f(v-w) + f(w)$, so $f(v) - f(w) \leq f(v-w)$. Similarly $f(w) - f(v) \leq f(w-v)$, so
\begin{align*}
|f(v) - f(w)| &= \max\{f(v) - f(w), f(w) - f(v)\} \\
&\leq \max\{f(v-w), f(w-v)\} \\
&\leq \max\{|f(v-w)|, |f(w-v)|\}.
\end{align*}

The equivalence $(2) \Leftrightarrow (3)$ is given by \ref{continuityToNormedSpace}.
\end{proof}
\begin{corollary} \label{continuityAbsFunctional}
Let $\sSet{V,\xi}$ be a convergence vector space and $f: V\to \R$ a \emph{linear} functional. Then $|f|$ is continuous \textup{if and only if} $f$ is continuous.
\end{corollary}
\begin{proof}
The functional $|f|$ is sublinear and bounded on the same sets as $f$. We can then compare the proposition to \ref{continuityToNormedSpace}.
\end{proof}

\subsubsection{Seminorms}
\begin{lemma} \label{kernelSeminormVectorSpace}
The kernel of a seminorm is a vector space.
\end{lemma}
Note this does not follow from \ref{kernelSubspace} because seminorms are not linear.
\begin{proof}
Let $p:V\to \R$ be a seminorm. We verify the subspace criterion \ref{subspaceCriterion}. First $0\in\ker(p)$ because $p(0) = p(0\cdot 0) = |0|p(0) = 0$.

Now take $v,w\in \ker(p)$ and $\lambda\in \F$. Then $0\leq p(v+\lambda w) \leq p(v)+|\lambda|p(w) = 0$, so $v+\lambda w\in\ker(p)$.
\end{proof}

\begin{proposition} \label{uniformContinuitySeminorms}
Let $\sSet{V,\xi}$ be a convergence vector space and $p: V\to \R^+$ a seminorm. Then the following are equivalent:
\begin{enumerate}
\item $p$ is continuous;
\item $p$ is continuous at $0$;
\item $p$ is uniformly continuous.
\end{enumerate}
\end{proposition}
Cfr. \ref{uniformContinuityGroupHomomorphism}.
\begin{proof}
$(1) \Rightarrow (2)$ Immediate.

$(2) \Rightarrow (3)$ We have that $p\circ \Delta$ is a pseudometric. We first show that $p\circ \Delta$ is a continuous pseudometric by taking $H\in \uniformity_V$. By definition of the uniformity, we have $\Delta^{\imf\imf}(H)\overset{\xi}{\longrightarrow} 0$, so $(p\circ \Delta)^{\imf\imf}(H) \overset{\R}{\longrightarrow} 0$ by continuity of $p$ at $0$. Thus $H\in \uniformity_{p\circ \Delta}$ and $p\circ \Delta$ is continuous.

We conclude that $p = (p\circ \Delta)(0,-)$ is uniformly continuous by \ref{partialApplicationMetricUniformlyContinuous}.

$(3) \Rightarrow (1)$ This follows from \ref{preservationUniformStructure}.
\end{proof}

\begin{proposition} \label{seminormFactorisation}
Let $V$ be a vector space and $p: V\to \R$ a function. Then $p$ is a seminorm \textup{if and only if} there exists a normed space $\sSet{W,\norm{\cdot}}$ and linear function $T: V\to W$ such that $p = \norm{\cdot}\circ T$.
\end{proposition}
\begin{proof}
If $p$ is of the form $\norm{\cdot}\circ T$, then it is clearly a seminorm.

Now assume $p$ is a seminorm, then define $W \defeq V/\ker(p)$, which works because $\ker(p)$ is a vector space by \ref{kernelSeminormVectorSpace}. Then define
\[ \norm{[x]_{\ker(p)}} \defeq p(x). \]
We show that this is well-defined: if $[x]_{\ker(p)} = [y]_{\ker(p)}$, then $x-y\in \ker(p)$, so 
\[ |p(x) - p(y)| \leq p(x-y) = 0, \]
by the reverse triangle inequality \ref{reverseTriangleInequality}. It is easy to verify that $\norm{\cdot}$ is a norm.

Finally set $T \defeq [\cdot]_{\ker(p)}$, which is linear. We have $p = \norm{\cdot}\circ [\cdot]_{\ker(p)} = \norm{\cdot}\circ T$.
\end{proof}

\begin{example}
The kernel of a seminorm is a vector space, but, unlike linear functionals, it is not generally a hyperplane. Equivalently, the space $W$ in \ref{seminormFactorisation} is not generally $1$ dimensional.

For example take any norm on a space that is not $1$ dimensional.
\end{example}

\begin{proposition} \label{seminormOpenMap}
Let $V$ be a convergence vector space and $f:V\to \R^+$ a non-zero seminorm. Then $f$ is an open map.
\end{proposition}
Note that $f: V\to \R^+$ is open as a function with codomain $\R^+$. If we take $\R$ to be the codomain, then it is not open (indeed $f^\imf(V) = \interval[co]{0,\infty}$).

We have a similar result for linear functions, see \ref{linearFunctionalOpen}.
\begin{proof}
It is enough to show $f$ is open when $V$ is equipped with the algebraic convergence.

Let $A$ be an algebraically open set. We use \ref{openClosedCriteria} to show $f^\imf(A)$ is also open. Take some $y\in f^\imf(A)$. Then there exists an $x\in A$ such that $f(x) = y$.

First suppose $f(x) \neq 0$. Because $A$ is open, $x\in \inh_\mathfrak{a}(A)$ and there exists $\Gamma\in \neighbourhood_\F(0)$ such that $x+\Gamma\cdot x \subseteq A$ by \ref{constructionsInAlgebraicConvergence}. There exists $\epsilon > 0$ such that $\ball_\F(0,\epsilon) \subseteq f(x)\cdot \Gamma$ and $\epsilon \leq f(x)$. Now we calculate
\begin{align*}
\interval[o]{f(x)-\epsilon, f(x)+\epsilon} &= \interval[o]{1 - \frac{\epsilon}{f(x)}, 1 + \frac{\epsilon}{f(x)}} \cdot f(x) \\
&= f^\imf\Big(\ball_\F\big(1, \epsilon / f(x)\big)\cdot x\Big) \\
&= f^\imf\Big(x + \ball_\F\big(0, \epsilon / f(x)\big)\cdot x\Big) \\
&\subseteq f^\imf\Big(x + \Gamma\cdot x\Big) \subseteq f^\imf(A).
\end{align*}

Now suppose $f(x) = 0$. Because $f$ is non-zero, there exists a $v\in V$ such that $f(v) \neq 0$.  By the triangle inequality and the reverse triangle inequality \ref{reverseTriangleInequality}, we have, for all $t\geq 0$
\[ tf(v) = |f(tv)| = \big|f(tv) - f(x)\big| \leq f(x+tv) \leq f(x) + f(tv) = tf(v), \]
so $tf(v) = f(x+tv)$
Because $A$ is algebraically open, $x\in \inh_\mathfrak{a}(A)$ and there exists $\Gamma\in \neighbourhood_\F(0)$ such that $x+\Gamma\cdot v \subseteq A$ by \ref{constructionsInAlgebraicConvergence}. Now there exists $\epsilon > 0$ such that $\interval[co]{0,\epsilon} \subseteq \Gamma$, so
\[ \interval[co]{0,\epsilon f(v)} = \interval[co]{0,\epsilon}\cdot f(v) = f^\imf\big(x + \interval[co]{0,\epsilon}\cdot v\big) \subseteq f^\imf(A). \]
\end{proof}


\subsection{Gauges}
\begin{definition}
Let $V$ be a vector space and $A\subseteq V$ an absorbent subset. The function
\[ p_A: V\to \overline{\R^+}: v\mapsto \inf\setbuilder{\lambda\in \R^{\geq 0}}{v\in \lambda A} \]
is called the \udef{gauge} or \udef{Minkowski functional} of $A$.
\end{definition}

\begin{lemma}
Let $V$ be a vector space and $A,B\subseteq V$. Then
\begin{enumerate}
\item if $A$ absorbs $B$, then $p_A^\imf(B)$ is bounded;
\item if $A$ is balanced and $p_A^\imf(B)$ is bounded, then $A$ absorbs $B$.
\end{enumerate}
\end{lemma}
\begin{proof}
(1) Assume $A$ absorbs $B$. Then there exists $r >0$ such that $B\subseteq cA$ for all $|c|\geq r$. In particular $v\in rA$, so $p_A(v) \leq r$, for all $v\in B$. Thus $p_A^\imf(B)$ is bounded by $r$.

(2) Assume $p_A^\imf(B)$ is bounded. Then we can find an upper bound $s>0$. Take $|c|\geq s$ arbitrarily. We need to show that $B\subseteq cA$.

For all $v\in B$, we can find some $\lambda \leq s$ such that $v\in\lambda A = c\left(\frac{\lambda}{c}\right)A \subseteq cA$. The last inclusion follows because $A$ is balanced and $\left|\frac{\lambda}{c}\right|\leq 1$ (as $\lambda \leq s \leq |c|$).
\end{proof}
\begin{corollary} \label{gaugeWellDefined}
Let $V$ be a vector space and $A\subseteq V$. Then
\begin{enumerate}
\item if $A$ is absorbent, then $p_A(v)$ is finite for all $v\in V$;
\item if $A$ is balanced and $p_A(v)$ is finite for all $v\in V$, then $A$ is absorbent.
\end{enumerate}
\end{corollary}

\begin{lemma} \label{gaugeScaling}
Let $V$ be a vector space and $A\subseteq V$ an absorbent subset. For all $v\in V$ and $t\geq 0$:
\[ p_A(tv) = p_{t^{-1}A}(v) = t p_A(v). \]
Thus $p_A$ is positively homogeneous.
\end{lemma}
\begin{proof}
We calculate
\begin{align*}
p_A(tv) &= \inf\setbuilder{\lambda\in \R^{\geq 0}}{tv\in \lambda A} \\
&= \inf\setbuilder{\lambda\in \R^{\geq 0}}{v\in t^{-1}\lambda A} = p_{t^{-1}A}(v) \\
&= \inf\setbuilder{t\lambda\in \R^{\geq 0}}{v\in \lambda A} \\
&= t\inf\setbuilder{\lambda\in \R^{\geq 0}}{v\in \lambda A} = tf(v).
\end{align*}
\end{proof}

\begin{lemma} \label{semibalancedClosureGauge}
Let $V$ be a vector space and $A\subseteq V$ an absorbent subset. Then $p_A = p_{\semibalanced(A)}$.
\end{lemma}
\begin{proof}
We calculate
\begin{align*}
\setbuilder{\lambda\in\R^{\geq 0}}{v\in \lambda \semibalanced(A)} &= \setbuilder{\lambda\in\R^{\geq 0}}{v\in \lambda\cdot\interval{0,1}\cdot A} \\
&= \setbuilder{\lambda\in\R^{\geq 0}}{\exists k\in \interval{0,1}:\; v\in \lambda k\cdot A} \\
&= \setbuilder{\lambda\in\R^{\geq 0}}{\exists \lambda'\leq \lambda:\; v\in \lambda' \cdot A} \\
&= \upset \setbuilder{\lambda'\in\R^{\geq 0}}{v\in \lambda' \cdot A}.
\end{align*}
Thus the infima of both $\setbuilder{\lambda\in\R^{\geq 0}}{v\in \lambda \semibalanced(A)}$ and $\setbuilder{\lambda'\in\R^{\geq 0}}{v\in \lambda' \cdot A}$ are the same.
\end{proof}

\begin{lemma} \label{gaugeLemma}
Let $V$ be a vector space, $A\subseteq V$ an absorbent subset and $\lambda\in \R^{> 0}$. Then
\begin{enumerate}
\item if $A$ is semibalanced, then $p_A(v) < \lambda \implies \lambda^{-1}v\in A$;
\item $\lambda^{-1}v\in A \implies p_A(v) \leq \lambda$;
\item $p_A^{\preimf}[\ball(0,1)] \;\subseteq\; \semibalanced(A) \;\subseteq\; p_A^{\preimf}[\cball(0,1)]$;
\item $\frac{v}{p_A(v)+\epsilon} \in \semibalanced(A)$ for all $v\in V$ and $\epsilon >0$.
\end{enumerate}
\end{lemma}
\begin{proof}
(1) If $p_A(v) < \lambda$, then there exists $\lambda' \leq \lambda$ such that $v\in \lambda' A$. As we can write $\lambda' = k\lambda$ for some $k\in\interval{0,1}$, we have
\[ v\in \lambda k A \subseteq \lambda \cdot \interval{0,1}\cdot A = \lambda A, \]
where we have used that $A = \semibalanced(A) = \interval{0,1}\cdot A$. Thus $\lambda^{-1}v \in A$.

(2) Assume $\lambda^{-1}v \in A$. Then $v\in \lambda A$, so we immediately have $p_A \leq \lambda$.

(3) This follows from (1), the fact that $p_A(v) < p_A(v) + \epsilon$ and the fact that $p_A = p_{\semibalanced(A)}$, \ref{semibalancedClosureGauge}.
\end{proof}

\begin{lemma} \label{gaugeClassificationLemma}
Let $V$ be a vector space, $f: V\to \R^{\geq 0}$ a positively homogeneous function and $A \subseteq V$ a semibalanced subset.
Then the following are equivalent:
\begin{enumerate}
\item $f = p_A$;
\item $f^{\preimf}[\ball(0,1)] \subseteq A \subseteq f^{\preimf}[\cball(0,1)]$.
\end{enumerate}
\end{lemma}
\begin{proof}
$(1) \Rightarrow (2)$ We calculate, using \ref{gaugeLemma},
\begin{align*}
x\in p_{A}^\preimf[\ball(0,1)] \iff& p_{A}(x) < 1 \\
\implies& x\in A \\
\implies& p_{A}(x) \leq 1 \\
\iff& x\in p_{A}^\preimf[\cball(0,1)].
\end{align*}
Reformulating in terms of sets gives the result.

$(2) \Rightarrow (1)$ We calculate, for arbitrary $v\in V$,
\[ \begin{aligned}
p_A(v) &= \inf\setbuilder{\lambda\in \R^{\geq 0}}{v\in \lambda A} \\
&\leq \inf\setbuilder{\lambda\in \R^{\geq 0}}{v\in \lambda f^{\preimf}[\ball(0,1)]} \\
&= \inf\setbuilder{\lambda\in \R^{\geq 0}}{v\in f^{\preimf}[\ball(0,\lambda)]} \\
&= \inf\setbuilder{\lambda\in \R^{\geq 0}}{f(v) < \lambda} = f(v)
\end{aligned} \quad\text{and}\quad \begin{aligned}
p_A(v) &= \inf\setbuilder{\lambda\in \R^{\geq 0}}{v\in \lambda A} \\
&\geq \inf\setbuilder{\lambda\in \R^{\geq 0}}{v\in \lambda f^{\preimf}[\cball(0,1)]} \\
&= \inf\setbuilder{\lambda\in \R^{\geq 0}}{v\in f^{\preimf}[\cball(0,\lambda)]} \\
&= \inf\setbuilder{\lambda\in \R^{\geq 0}}{f(v) \leq \lambda} = f(v).
\end{aligned} \]
We conclude that $f(v) = p_A(v)$.
\end{proof}

\begin{proposition} \label{gaugeClassification}
Let $V$ be a vector space and $f: V\to \R^{\geq 0}$ a function.
Then the following are equivalent:
\begin{enumerate}
\item $f$ is positively homogenous;
\item $f = p_A$ for some semibalanced, absorbent subset $A$.
\end{enumerate}
\end{proposition}
\begin{proof}
Assume $f$ positively homogeneous. Then we can set $A = f^{\preimf}[\ball(0,1)]$ in \ref{gaugeClassificationLemma} because $f^{\preimf}[\ball(0,1)]$ is semibalanced. Indeed
\[ \interval{0,1}\cdot f^{\preimf}[\ball(0,1)] = f^{\preimf}[\interval{0,1}\cdot\ball(0,1)] = f^{\preimf}[\ball(0,1)]. \]

The converse is given by \ref{gaugeScaling}.
\end{proof}

\begin{lemma} \label{gaugeZeroLemma}
Let $V$ be a vector space, $A\subseteq V$ an absorbent subset and $a\in A$. If there exists a subspace $U\subseteq A$ such that $a\in U$, then $p_A(a) = 0$.
\end{lemma}
\begin{proof}
For all $\epsilon > 0$, $\epsilon^{-1}a\in A$, so $a\in \epsilon A$.
\end{proof}

\begin{proposition} \label{gaugeProperties}
Let $V$ be a vector space and $A\subseteq V$ an absorbent subset. Then
\begin{enumerate}
\item $p_A$ is absolutely homogenous if $A$ is balanced;
\item $p_A$ is sublinear if $A$ is convex;
\item $p_A$ is point-separating if $A$ is balanced and contains only the trivial subspace.
\end{enumerate}
\end{proposition}
\begin{proof}
(1) By \ref{balancedLemma} we have $\mu A = |\mu| A$ and thus
\begin{align*}
p_A(\mu\cdot v) &= \inf\setbuilder{\lambda\in \R^{\geq 0}}{\mu\cdot v\in \lambda A} = \inf\setbuilder{\lambda\in \R^{\geq 0}}{v\in \frac{\lambda}{\mu} A} \\
&= \inf\setbuilder{\lambda\in \R^{\geq 0}}{v\in \frac{\lambda}{|\mu|} A} = \inf\setbuilder{|\mu|\lambda\in \R^{\geq 0}}{v\in \lambda A} = |\mu|\cdot p_A(v).
\end{align*}

(2) We just need to show subadditivity. Positive homogeneity is automatic by \ref{gaugeScaling}. Take $v,w\in V$. Now take arbitrary $\epsilon > 0$, so $(p_A(v)+\epsilon)^{-1}v \in A$ and $(p_A(w)+\epsilon)^{-1}w \in A$ by \ref{gaugeLemma} (as $A$ is semibalanced by \ref{convexAbsorbentImpliesSemibalanced}). By convexity of $A$, we have
\[ \frac{v+w}{p_A(v)+p_A(w)+2\epsilon} = \frac{p_A(v)+\epsilon}{p_A(v)+p_A(w)+2\epsilon}(p_A(v)+\epsilon)^{-1}v + \frac{p_A(w)+\epsilon}{p_A(v)+p_A(w)+2\epsilon}(p_A(w)+\epsilon)^{-1}w \in A. \]
By \ref{gaugeLemma} this means $p_A(v)+p_A(w)+2\epsilon \geq p_A(v+w)$ and because $\epsilon$ was arbitrary, we conclude that $p_A(v+w) \leq p_A(v)+p_A(w)$.

(3) Assume $A$ contains only the trivial subspace. Then for all $v\in V$ there exists some $\lambda\in \F$ such that $\lambda\cdot v\notin A$. Now for all $|c|\geq |\lambda|$, $c\cdot v\notin A$ because $A$ is balanced. Then $p_A(2\lambda\cdot v) \neq 0$ and because $p_A$ is absolutely homogeneous we have $p_A(v) = (2\lambda)^{-1}p_A(2\lambda\cdot v) \neq 0$.
\end{proof}
\begin{corollary}
The gauge of an absolutely convex and absorbent subset is a seminorm. If the subset contains only the trivial subspace, then the gauge is a norm.
\end{corollary}

\begin{proposition} \label{continuityConvexGauge}
Let $\sSet{V, \xi}$ be a convergence vector space and $A\subseteq V$ a convex, absorbent subset. Then $p_A: \sSet{V,\xi} \to \F$ is continuous \textup{if and only if} $A\in\vicinity_\xi(0)$.
\end{proposition}
\begin{proof}
First assume $A\in\vicinity_\xi(0)$. By \ref{gaugeProperties}, we have that $p_A$ is sublinear. By \ref{gaugeClassificationLemma}, we have that $A \subseteq p_A^\preimf[\cball(0,1)]$ and thus that $p_A$ is bounded on $A$. Then $p_A$ is continuous by \ref{sublinearContinuity}.

Now assume $p_A$ continuous. Then $\ball(0,1) \in \neighbourhood_\F(0) = \neighbourhood_\F(p_A(0))$, so $p_A^\preimf[\ball(0,1)] \in \vicinity_\xi(0)$ by \ref{continuityVicinityFilter}. By \ref{gaugeClassificationLemma} (using the fact that $A$ is semibalanced, \ref{convexSemibalanced}), $p_A^\preimf[\ball(0,1)] \subseteq A$, so $A\in\vicinity_\xi(0)$.
\end{proof}
\begin{corollary} \label{gaugeContinuousAlgebraicConvergence}
Let $V$ be a vector space. Then $p_A: \sSet{V,\mathfrak{a}} \to \F$ is continuous for any convex, absorbent subset $A \subseteq V$.
\end{corollary}
\begin{proof}
By \ref{coreProperties}, we have $0\in\inh_\mathfrak{a}(A)$. Then $A\in \vicinity_\mathfrak{a}(0)$ by \ref{principalAdherenceInherence}.
\end{proof}

\begin{lemma} \label{absoluteFunctionalGauge}
Let $\sSet{V, \xi}$ be a convergence vector space and $f: V\to \F$ a linear functional. Then $f$ is continuous \textup{if and only if} $|f| = p_K$ for some $K\in\vicinity_\xi(0)$.
\end{lemma}
\begin{proof}
We have that $|f|$ is positively homogeneous. Set $K \defeq |f|^{\preimf}[\ball(0,1)] = |f|^{\preimf}(\interval[o]{0,1})$. Then $|f| = p_{K}$, by \ref{gaugeClassificationLemma}. 

First assume $f$ is continuous. Then $p_K^\preimf\big[\ball(0,1)\big] = K$ is in $\vicinity_\xi(0)$ by \ref{continuityVicinityFilter}.

Now assume $K\in\vicinity_\xi(0)$. As $K \subseteq p_K^\preimf\big[\cball(0,1)\big]$ (\ref{gaugeClassificationLemma}), we have that $|f|$ is bounded on $K$. It is also sublinear and thus continuous by \ref{sublinearContinuity}.
\end{proof}


\begin{proposition} \label{gaugeInherenceAdherence}
Let $V$ be a vector space and $A\subseteq V$ an absorbent semibalanced subset. Then
\begin{enumerate}
\item $\inh_\mathfrak{a}(A) \subseteq p_A^{\preimf}[\ball(0,1)]$;
\item $p_A^{\preimf}[\cball(0,1)] \subseteq \adh_\mathfrak{a}(A)$.
\end{enumerate}
The inclusions are equalities if $A$ is convex.
\end{proposition}
\begin{proof}
(1) Take $x\in \inh_\mathfrak{a}(A)$. Then by \ref{constructionsInAlgebraicConvergence}, there exists $\Gamma\in\neighbourhood_\F(0)$ such that $v+\Gamma\cdot v = (1 + \Gamma)\cdot v \subseteq A$. There exists $\lambda \in 1 + \Gamma$ that is real and strictly less than $1$. Thus $p_A(x) <1$, meaning $x\in p^\preimf_A\big(\ball(0,1)\big)$.

(Convex) For the other inclusion we use that $p_A$ is continuous by \ref{gaugeContinuousAlgebraicConvergence}. By \ref{adherenceInherenceContinuity} and \ref{gaugeClassificationLemma} we have
\[ p_A^{\preimf}[\ball(0,1)] = p_A^{\preimf}\Big[\interior_\F\big(\ball(0,1)\big)] \subseteq \inh_\mathfrak{a}\big(p_A^{\preimf}[\ball(0,1)]\big) \subseteq \inh_\mathfrak{a}(A). \]

(2) Take $x\in p^\preimf_A\big(\cball(0,1)\big)$, which means that $p_A(x) \leq 1$, so $p_A(x) < 1+\epsilon$ for all $\epsilon > 0$. By \ref{gaugeLemma}, $(1+\epsilon)^{-1}x = x - \epsilon(1+\epsilon)^{-1}x \in A$. Because $\frac{\epsilon}{1+\epsilon} < \epsilon$, we have $x - \epsilon(1+\epsilon)^{-1}x \in x + \ball(0,\epsilon)\cdot x$ and so $x + \ball(0,\epsilon)\cdot x \mesh A$.

Now for all $\Gamma\in\neighbourhood_\F(0)$, we have $\ball(0,\epsilon) \subseteq \Gamma$ for some $\epsilon >0$. Thus $x+ \Gamma\cdot x \mesh A$ and $x\in\adh_\mathfrak{a}(A)$ by \ref{constructionsInAlgebraicConvergence}.

(Convex) For the other inclusion we again use that $p_A$ is continuous by \ref{gaugeContinuousAlgebraicConvergence}. From \ref{gaugeClassificationLemma} we have $A \subseteq p^\preimf_A\big(\cball(0,1)\big)$. Thus, by \ref{adherenceInherenceContinuity},
\[ \adh_\mathfrak{a}(A) \subseteq \adh_\mathfrak{a}\Big(p^\preimf_A\big(\cball(0,1)\big)\Big) \subseteq p_A^\preimf\big(\overline{\cball(0,1)}\big) = p^\preimf_A\big(\cball(0,1)\big). \]
\end{proof}

\begin{example}
TODO square with corner missing.
\end{example}



\begin{proposition} \label{gaugeConstructions}
Let $V$ be a vector space and $A,B\subseteq V$ semibalanced subsets. Then
\begin{enumerate}
\item the gauge of $A\cap B$ is $\max\{p_A, p_B\}$;
\item if $A\subseteq B$, then $p_B \leq p_A$;
\item if $p_B \leq p_A$ and $A$ is convex, then $\adh_\mathfrak{a}(A) \subseteq \adh_\mathfrak{a}(B)$.
\end{enumerate}
\end{proposition}
\begin{proof}
(1) Clearly $p_A \leq p_{A\cap B}$:
\begin{align*}
p_A(v) &= \inf\setbuilder{\lambda\in\R^{\geq 0}}{v\in \lambda A} \\
&\leq \setbuilder{\lambda\in\R^{\geq 0}}{v\in \lambda (A\cap B)} \\
&= p_{A\cap B}(v).
\end{align*}
Similarly $p_B \leq p_{A\cap B}$. Thus $\max\{p_A, p_B\} \leq p_{A\cap B}$.

For the other inequality we use \ref{gaugeLemma}: take $v\in V$ and arbitrary $\epsilon > 0$. Then $\max\{p_A(v)+\epsilon, p_B(v)+\epsilon\} > p_A(v)$, so $\max\{p_A(v)+\epsilon, p_B(v)+\epsilon\}^{-1}v\in A$. Similarly $\max\{p_A(v)+\epsilon, p_B(v)+\epsilon\}^{-1}v\in B$. Thus $\max\{p_A(v)+\epsilon, p_B(v)+\epsilon\}^{-1}v \in A\cap B$ and finally $p_{A\cap B}(v) \leq \max\{p_A(v)+\epsilon, p_B(v)+\epsilon\}$. Taking the limit $\epsilon \to 0$, we have $p_{A\cap B} \leq \max\{p_A, p_B\}$.

(2) First assume $A\subseteq B$. Then $\frac{v}{p_A(v)+\epsilon}\in A$ by \ref{gaugeLemma}, for all $\epsilon >0$. Then $\frac{v}{p_A(v)+\epsilon}\in B$, so $p_B(v) \leq p_A(v)+\epsilon$ (again by \ref{gaugeLemma}). Taking the limit $\epsilon\to 0$ yields the result.

(3) Assume $p_B \leq p_A$ and take $v\in \adh_\mathfrak{a}(A)$. Then $p_B(v) \leq p_A(v) \leq 1$ by \ref{gaugeInherenceAdherence}, which also gives $v\in\adh_\mathfrak{a}(B)$.
\end{proof}


\section{Linear functionals}
\begin{lemma} \label{kernelHyperplane}
Let $V$ be a vector space and $U\subseteq V$ a subspace. Then $U$ is a hyperplane \textup{if and only if} it is the kernel of a linear functional.
\end{lemma}

\begin{lemma} \label{functionalBoundedNeighbourhood}
Let $f: V\to \F$ be a linear functional and $x\notin \ker(f)$. Let $A\subseteq V$ be a balanced set. Then $(x+A)\perp \ker(f)$ \textup{if and only if} $A \subseteq f^{\preimf}(\ball(0,|f(x)|))$.
\end{lemma}
\begin{proof}
Suppose $A \subseteq f^{\preimf}(\ball(0,|f(x)|))$. Then for all $a\in A$: $f(x+a) = f(x) + f(a) \neq 0$.

Conversely, suppose $A \not\subseteq f^{\preimf}(\ball(0,|f(x)|))$, i.e.\ there exists $a\in A$ such that $|f(a)| \geq |f(x)|$. Then $v= -\frac{f(x)}{f(a)}a\in A$, because $A$ is balanced and so $f(x+ v) = f(x)-\frac{f(x)}{f(a)}f(a) = 0$ and so $(x+A) \mesh \ker(f)$.
\end{proof}

\begin{proposition} \label{linearFunctionalOpen}
Let $V$ be a convergence vector space and $f:V\to \F$ a non-zero linear functional. Then $f$ is an open map.
\end{proposition}
We have a similar result for seminorms, see \ref{seminormOpenMap}
\begin{proof}
It is enough to show $f$ is open when $V$ is equipped with the algebraic convergence.

Let $A$ be an algebraically open set. We use \ref{openClosedCriteria} to show $f^\imf(A)$ is also open. Take some $y\in f^\imf(A)$. Then there exists an $x\in A$ such that $f(x) = y$.

Because $f$ is non-zero, there exists a $v\in V$ such that $f(v) \neq 0$. Because $A$ is open, $x\in \inh_\mathfrak{a}(A)$ and there exists $\Gamma\in \neighbourhood_\F(0)$ such that $x+\Gamma\cdot v \subseteq A$ by \ref{constructionsInAlgebraicConvergence}.

Now $f^\imf\big(x+ \Gamma\big) = y+\Gamma \cdot f(v) \subseteq f^\imf(A)$ and $y+\Gamma \cdot f(v)$ is a vicinity of $y$, since $f(v) \neq 0$, so we are done.
\end{proof}

\begin{lemma} \label{complexRangeExtensionRealFunctional}
Let $V$ be a complex vector space and $g: V_\R\to \R$ a linear functional. Then there exists a unique linear functional $f: V\to \C$ such that $g = \Re(f)$.
\end{lemma}
\begin{proof}
We can write $f = g + ih$ for some function $h: V\to \R$. Then for all $x\in V$
\[ g(ix)+ih(ix) = f(ix) = if(x) = ig(x) - h(x). \]
Comparing real parts gives $h(x) = - g(ix)$. So $f$ must be given by $f(x) = g(x) - ig(ix)$. Clearly $f$ is real-linear. We just need to verify that this makes $f$ complex-linear. Indeed, take $\lambda = a +ib \in \C = \R+i\R$ arbitrarily. Then for all $v\in V$
\begin{align*}
f(\lambda v) &= f\big((a+ib)v\big) \\
&= af(v) + bf(iv) \\
&= af(v) + b\big(g(iv) - ig(i^2v)\big) \\
&= af(v) + b\big(g(iv) + ig(v)\big) \\
&= af(v) + ib\big(-ig(iv) + g(v)\big) \\
&= af(v) + ibf(v) = (a+ib)f(v) = \lambda f(v).
\end{align*}
\end{proof}

\begin{lemma} \label{linearDependenceLinearFunctionals}
Let $V$ be a vector space and $f_0,\ldots, f_n, f$ linear functionals in $(V\to \F)$. Then the following are equivalent:
\begin{enumerate}
\item $f$ is a linear combination of $f_0,\ldots, f_n$;
\item there exists a $C>0$ such that $f(v) \leq C \max_{0\leq i\leq n}|f_i(v)|$;
\item $\ker(f) \supseteq \bigcap_{0\leq i \leq n}\ker(f_i)$;
\end{enumerate}
\end{lemma}
\begin{proof}
The implications $(1) \Rightarrow (2) \Rightarrow (3)$ are clear.

Now assume $(3)$ holds. Consider the function
\[ \begin{pmatrix}
f_0 \\ \vdots \\ f_n
\end{pmatrix}: V\to \F^{n+1}: v\mapsto \begin{pmatrix}
f_0(v) \\ \vdots \\ f_n(v)
\end{pmatrix}. \]
Due to the assumption, we can find a linear function $F: \F^{n+1}\to \F$ such that $f = F\circ \begin{pmatrix}
f_0 \\ \vdots \\ f_n
\end{pmatrix}$.

This function $F$ can be represented as a matrix by \ref{ellIsomorphism}. Thus $f$ is a linear combination of $f_0,\ldots, f_n$.
\end{proof}
\begin{corollary}
Let $V$ be a vector space and $f_0,\ldots, f_n$ linearly independent linear functionals in $(V\to \F)$. Then there exist $v_0, \ldots, v_n$ such that $f_i(v_j) = \delta_{i,j}$.
\end{corollary}
\begin{proof}
The proof is by induction. The case $n=1$ is clear: if there was no such $a_1$, then $f_1$ would be zero and thus not linearly independent.

Suppose the statement holds for $n-1$ and take $f_0,\ldots, f_n$ linearly independent linear functionals with corresponding $v_0,\ldots, v_{n-1}$. By point (3) of the proposition we can find $v_n \in \bigcap_{0\leq i \leq n}\ker(f_i)\setminus \ker(f_n)$, which after rescaling can be taken to be such that $f_n(v_n) = 1$. By construction $f_i(v_n) = 0$ for $i<n$.

Now replace $v_i$ with $v_i-f_n(v_i)v_n$ and rescale.
\end{proof}

\subsection{The dual space}
\begin{definition}
Let $\sSet{V,\xi}$ be a convergence vector space over a field $\mathbb{F}$.

The \udef{dual} of $V$ is the vector space of all continuous linear functionals on $V$.

The dual is denoted $\sSet{V,\xi}^{*}$ (or just $V^*$ is the convergence is clear from the context).
\end{definition}

\begin{proposition} \label{continuityLinearFunctionals}
Let $\sSet{V, \xi}$ be a CVS and $f:V\to \F$ a linear functional on $V$. Then the following are equivalent:
\begin{enumerate}
\item $f\in V^{*}$, i.e.\ $f$ is continuous;
\item there exists a vicinity $U\in \vicinity_\xi(0)$ such that $f$ is bounded on $U$;
\item $\ker(f)$ is closed;
\item $\ker(f)$ is not dense.
\end{enumerate}
\end{proposition}
\begin{proof}
$(1) \Leftrightarrow (2)$ By \ref{continuityToNormedSpace}.

$(1) \Rightarrow (3)$ Because $\ker(f) = f^{\preimf}(\{0\})$ and $\{0\}$ is closed in $\F$, $\ker(f)$ is closed by \ref{preimageOpenClosed}.

$(3) \Rightarrow (1)$ This follows from \ref{closedKernelLinearFunctionToFiniteDimSpaceContinuous}.

$(3) \Leftrightarrow (4)$ By \ref{kernelHyperplane} $\ker(f)$ is a hyperplane and by \ref{hyperplaneClosedDense} this hyperplane is either closed or dense, but not both.
\end{proof}

\begin{note}
In the topological case, we have the following argument for the implication $\ker(f)$ closed $\Rightarrow$ $f$ bounded on a neighbourhood of $0$.

Assume $\ker(f)$ closed. If $\ker(f) = V$, then $f$ is constant and thus continuous by \ref{continuityConstructions}. If $\ker(f) \neq V$, we can find some some $x\in \ker(f)^c$, which is open. Thus $\ker(f)^c - x$ is a neighbourhood of the origin, meaning we can take a balanced subset $A$ by \ref{vicinityFilterAtOrigin}. Now $(x+A)\perp \ker(f)$ by construction, so $f$ is bounded on $A$ by \ref{functionalBoundedNeighbourhood}.
\end{note}

\section{Locally convex spaces}
\begin{definition}
A \udef{locally convex} convergence vector space is a convergence vector space that is locally convex.
\end{definition}

\begin{lemma} \label{locallyConvexTVSLocallyConvexOpen}
If $\sSet{V,\xi}$ is a locally convex topological vector space, then $V$ is locally convex open.
\end{lemma}
\begin{proof}
Take $v\in V$ and $A\in \neighbourhood(v)$. Since $\xi$ is locally convex, $A$ contains a convex neighbourhood $B$ of $v$. Then $\interior(B)$ is convex by \ref{inherenceAdherenceConvex} and it is a neighbourhood of $v$ since $\xi$ is locally open.
\end{proof}

\begin{lemma} \label{locallyConvexNeighbourhoodsLemma}
Let $\sSet{V,\xi}$ be a TVS. Then the following are equivalent:
\begin{enumerate}
\item $\xi$ is locally convex;
\item $\neighbourhood_\xi(0)$ is based in the convex sets;
\item $\neighbourhood_\xi(0)$ is based in the absolutely convex sets;
\item $\neighbourhood_\xi(0) = \upset\disked^{\imf}\big(\neighbourhood_\xi(0)\big)$.
\end{enumerate}
\end{lemma}
\begin{proof}
$(1) \Leftrightarrow (2)$ One direction is immediate, for the other it is enough to note that if $U$ is convex, then so is the translated set $x+U$ for all $x\in V$.

$(2) \Leftrightarrow (3)$ One direction is immediate, the other follows because the balanced core of a convex set is convex by \ref{balancedCoreConvexSet}.

$(3) \Leftrightarrow (4)$ Because $\disked$ preserves intersections (as it is a closure operator), we have that $\upset\disked^{\imf}\big(\neighbourhood_\xi(0)\big)$ is a filter.

We automatically have $\neighbourhood_\xi(0) \subseteq \upset\disked^{\imf}\big(\neighbourhood_\xi(0)\big)$. If $\neighbourhood_\xi(0)$ is based in the absolutely convex sets, then $\upset\disked^{\imf}\big(\neighbourhood_\xi(0)\big)$ contains a base of $\upset\disked^{\imf}\big(\neighbourhood_\xi(0)\big)$ and so the other inclusion holds.

If $\neighbourhood_\xi(0) = \upset\disked^{\imf}\big(\neighbourhood_\xi(0)\big)$, then $\neighbourhood_\xi(0)$ is based in the absolutely convex sets, because clearly $\upset\disked^{\imf}\big(\neighbourhood_\xi(0)\big)$ is.
\end{proof}

\begin{proposition} \label{LCTVSconstruction}
Let $V$ be a vector space and $N\in\powerfilters(V)$. Then $N = \neighbourhood_\xi(0)$ for some locally convex topological convergence on $V$ \textup{if and only if}
\begin{enumerate}
\item for all $A\in N$ and $\lambda\in \F$: $\lambda A\in N$;
\item each $A \in N$ is absorbent;
\item $N$ has an absolutely convex base.
\end{enumerate}
\end{proposition}
\begin{proof}
This almost completely follows from \ref{TVSconstruction} and \ref{locallyConvexNeighbourhoodsLemma}. We just need to show that for all $A\in N$, there exists some $B\in N$ such that $B+B\subseteq A$. We may take $B = \frac{1}{2}A'$, where $A'$ is a convex subset of $A$, because for all $v,w\in A'$ we have $\frac{1}{2}v + \frac{1}{2}w \in A'$ by convexity.
\end{proof}

\begin{proposition}
Every locally convex TVS is locally path connected.
\end{proposition}
\begin{proof}
TODO
\end{proof}

\subsection{Seminormed spaces}
\begin{definition}
Let $V$ be a vector space and $S$ a set of seminorms on $V$. 
Then the initial vector space convergence w.r.t.\ $S$ is called the \udef{seminorm convergence} on $V$ and $V$ equipped with the seminorm convergence is called a \udef{seminormed space}.
\end{definition}
Note that the seminorm convergence is not in general an initial convergence.
\begin{example}
Let $\sSet{V, \norm{\cdot}}$ be a normed space. Suppose $u,v\in V$ are unit vectors (i.e.\ $\norm{u} = 1 = \norm{v}$). Then $\norm{\pfilter{u}} \to v$ in the initial convergence.
\end{example}

In fact the seminorm convergence is the initial convergence iff $S = \{\underline{0}\}$.

\begin{lemma} \label{seminormPseudometricLemma}
Let $V$ be a vector space and $p:V\to \R$ a seminorm. Then
\begin{enumerate}
\item $p\circ \Delta$ is a pseudometric;
\item the pseudometric convergence is a vector space convergence;
\item the pseudometric convergence is the initial vector space convergence w.r.t. $\{p\}$;
\item the pseudometric uniformity is equal to the vector space (i.e.\ group) uniformity.
\end{enumerate}
\end{lemma}
\begin{proof}
Every seminorm is a cyclically permutable group seminorm, so most of the results follow from \ref{groupSeminormConvergence}. We just need to show that the resulting convergence makes the scalar multiplication continuous and that any other vector space convergence that makes $p$ continuous is contained in this convergence.

We first show continuity of the scalar multiplication. As the convergence in $\F$ is topological, it is enough to consider the convergence of $\neighbourhood_\F(\lambda)\cdot F$ for some $\lambda \in \F$ and $F\to v\in V$. 

Now we have
\begin{align*}
\upset p^{\imf\imf}\big(\neighbourhood_\F(\lambda)\cdot F - \lambda \pfilter{v}\big) &= \upset p^{\imf\imf}\big((\neighbourhood_\F(\lambda)\cdot F - \neighbourhood_\F(\lambda)\pfilter{v}) + (\neighbourhood_\F(\lambda)\pfilter{v} -\lambda \pfilter{v})\big) \\
&\leq_w \upset p^{\imf\imf}\big(\neighbourhood_\F(\lambda)\cdot F - \neighbourhood_\F(\lambda)\pfilter{v}\big) + \upset p^{\imf\imf}\big(\neighbourhood_\F(\lambda)\pfilter{v} -\lambda \pfilter{v}\big) \\
&= \big|\neighbourhood_\F(\lambda)\big|\cdot p^{\imf\imf}(F-\pfilter{v}) + \big(\big|\neighbourhood_\F(\lambda)\big| - \lambda\big)\pfilter{p}(v) \\
&\to 0 + 0 = 0.
\end{align*}
The weak inequality follows from \ref{pointwisefunctionToFilterInequality}. Then $p^{\imf\imf}(F-\pfilter{v})$ converges to $0$ by \ref{metricConvergence}.
Since $\big|\neighbourhood_\F(\lambda)\big|$, $\big(\big|\neighbourhood_\F(\lambda)\big| - \lambda\big)$ and $\pfilter{p}(v)$ converge in $\R$, we obtain the convergence by continuity of the multiplication.

Thus $\upset p^{\imf\imf}\big(\neighbourhood_\F(\lambda)\cdot F - \lambda \pfilter{v}\big)$ converges by the squeeze theorem (TODO ref).

Finally suppose $F$ converges to $v$ in some other vector space convergence that makes $p$ continuous. Then $\upset p^{\imf\imf}(F-v)$ converges to $0$ in this convergence, but this implies that $F$ converges to $v$ in the pseudometric convergence.
\end{proof}

\begin{proposition} \label{initialSeminormConvergence}
Let $V$ be a vector space, $S$ a set of seminorms on $V$, $F\in\powerfilters{V}$ and $v\in V$.
\begin{enumerate}
\item the seminormed convergence w.r.t.\ $S$ is topological and given by the join of the seminorm convergences determined by each $p\in S$;
\item $F\to v$ in the seminormed convergence w.r.t.\ $S$ \textup{if and only if} $p^{\imf}\big(F-v\big)\overset{\F}{\longrightarrow} 0$ for all $p\in S$;
\item the neighbourhood filter of the origin in the seminormed convergence is
\begin{align*}
\neighbourhood_\xi(0) &= \mathfrak{F}\setbuilder{p^\preimf[\ball(0,\epsilon)]}{p\in S, \epsilon > 0} \\
&= \mathfrak{F}\setbuilder{p^\preimf[\cball(0,\epsilon)]}{p\in S, \epsilon > 0}.
\end{align*}
\end{enumerate}
\end{proposition}
Note that $\neighbourhood_\xi(0)$ is the same as the neighbourhood filter at $0$ in the initial (not necessarily vector space) convergence w.r.t. $S$.
\begin{proof}
The seminormed convergence w.r.t.\ $S$ is clearly smaller than the initial convergenc w.r.t.\ $\{\id_V: V\to \sSet{V,p}\}_{p\in S}$. This is a vector space convergence by \ref{initialVectorSpaceConvergence} and thus exactly the seminorm convergence. 

Point (2) then follows from \ref{seminormPseudometricLemma} and \ref{metricConvergence}.

The convergence is topological by \ref{pretopologicalInitialConvergence}, which also gives the form of the neighbourhood filter of the origin.
\end{proof}

\begin{proposition} \label{dualSeminormedConvergence}
Let $V$ be a vector space and $S$ a set of seminorms on $V$. Let $\xi$ be the initial convergence on $V$ w.r.t. $S$. Then $f\in \sSet{V, \xi}^*$ \textup{if and only if}
\begin{itemize}
\item $f\in \sSet{V,\mathfrak{a}}^*$;
\item there exists a finite subset $A\subseteq S$ and $C>0$ such that $|f(v)| \leq C\max_{g\in A} g(v)$ for all $v\in V$.
\end{itemize}
\end{proposition}
\begin{proof}
We have by \ref{continuityLinearFunctionals}
\[ f\in \sSet{V, \xi}^* \iff \exists D>0: \exists U\in \neighbourhood_\xi(0):\; f^\imf[U] \subseteq \cball(0,D). \]
Now $U\in \neighbourhood_\xi(0)$ iff there exists a finite $A = \{p_n^\preimf[\ball(0,\epsilon_n)]\}_{n=0}^N \subseteq \setbuilder{p^\preimf[\ball(0,\epsilon)]}{p\in S, \epsilon > 0}$ such that $\bigcap A\subseteq U$. So WLOG we may take $U$ of this form.
Now 
\begin{align*}
f^\imf\Big[\bigcap A\Big] \subseteq \cball(0,D) &\iff \forall v\in V: \; \Big(\forall n\leq N: p_n(v)\leq \epsilon_n \Big) \implies |f(v)|\leq D \\
&\iff \forall v\in V: \; \max_{n\leq N}\epsilon_n^{-1}p_n(v)\leq 1 \implies |f(v)|\leq D \\
&\iff \forall v\in V: \; \max_{n\leq N}\epsilon_n^{-1}p_n\left(\frac{\max_{n\leq N}\epsilon_n^{-1}p_n(v)}{\max_{n\leq N}\epsilon_n^{-1}p_n(v)} v\right)\leq 1 \implies |f(v)|\leq D \\
&\iff \forall v\in V: \; \left(\max_{n\leq N}\epsilon_n^{-1}p_n(v)\right)^{-1}\max_{n\leq N}\epsilon_n^{-1}p_n(v)\leq 1 \implies \left(\max_{n\leq N}\epsilon_n^{-1}p_n(v)\right)^{-1}|f(v)|\leq D \\
&\iff \forall v\in V: \; 1\leq 1 \implies |f(v)|\leq D\max_{n\leq N}\epsilon_n^{-1}p_n(v) \\
&\iff \forall v\in V: \; |f(v)|\leq D\max_{n\leq N}\epsilon_n^{-1}p_n(v).
\end{align*}
WLOG we may take all $\epsilon_n = \epsilon = \min_{n\leq N}\epsilon_n$. We may then take $C = D/\epsilon$.
\end{proof}


\begin{proposition} \label{locallyConvexSeminormTopology}
Let $V$ be a vector space. A topological convergence on $V$ is locally convex \textup{if and only if} it is the seminorm convergence w.r.t. some set $S$ of seminorms on $V$.
\end{proposition}
\begin{proof}
Since a shifted convex set is still convex (\ref{translationScalingConvexSet}), localy convexity is equivalent to saying that the neighbourhood filter at the origin is based in the convex sets.

If $V$ has the seminorm convergence w.r.t. $S$, then $V$ is locally convex by \ref{initialSeminormConvergence} because $p^\preimf[\ball(0,\epsilon)]$ is convex for all $p\in S$ by \ref{preimageConvexSetConvexFunctionalIsConvex}.

Now let $V$ be a locally convex TVS, so there exists an absolutely convex base $\mathcal{B}$ of $\neighbourhood(0)$ by \ref{locallyConvexNeighbourhoodsLemma}.

Since $\{\inh(B)\}_{B\in\mathcal{B}}$ is a basis of $\neighbourhood(0)$ and $\inh(B) \subseteq \inh_\mathfrak{a}(B)$ by \ref{principalInherenceAdherenceProperties}, we may replace $\mathcal{B}$ by $\{\inh_\mathfrak{a}(B)\}_{B\in\mathcal{B}}$.
This basis consists of absolutely convex, algebraically open sets by \ref{coreProperties}.

Then $S = \setbuilder{p_B}{B\in \mathcal{B}}$ is a set of continuous seminorms, by \ref{gaugeProperties} and \ref{continuityConvexGauge}.

In order to show that the convergence on $V$ is the seminorm convergence w.r.t. $S$, we verify the form of $\neighbourhood(0)$ given in \ref{initialSeminormConvergence}.

Since $\epsilon B\in \neighbourhood(0)$ for each $B\in\mathcal{B}$ and all $\epsilon > 0$, we have that $\setbuilder{\epsilon B}{\epsilon > 0, B\in \mathcal{B}}$ is a basis of $\neighbourhood(0)$.
Then
\begin{align*}
\neighbourhood(0) &= \mathfrak{F}\setbuilder{\epsilon B}{\epsilon > 0, B\in \mathcal{B}} \\
&= \mathfrak{F}\setbuilder{\epsilon p_B^\preimf\big(\ball(0,1)\big)}{\epsilon > 0, B\in \mathcal{B}} \\
&= \mathfrak{F}\setbuilder{p_B^\preimf\big(\ball(0,\epsilon)\big)}{\epsilon > 0, B\in \mathcal{B}},
\end{align*}
where we have used \ref{gaugeInherenceAdherence}. Thus the convergence is indeed the seminorm convergence w.r.t.\ $S = \setbuilder{p_B}{B\in \mathcal{B}}$.
\end{proof}

\begin{proposition}
Let $V$ be a vector space. The functions
\begin{align*}
\powerset\setbuilder{A\subseteq V}{\text{$A$ is convex}} &\to \setbuilder{p: V\to \R}{\text{$p$ is a seminorm}}: &&\mathcal{B}\mapsto \setbuilder{p_K}{K\in \mathcal{B}} \\
\setbuilder{p: V\to \R}{\text{$p$ is a seminorm}} &\to \powerset\setbuilder{A\subseteq V}{\text{$A$ is convex}}: &&S\mapsto \setbuilder{p^\preimf[U]}{p\in S, U\in \neighbourhood_\R(0)}
\end{align*}
form an antitone Galois connection, where we order the seminorms pointwise.
\end{proposition}
\begin{proof}
TODO + previous as corollary
\end{proof}

Note: metrisable is not equivalent to normable!


\subsection{Locally convex modification}
\begin{definition}
Let $\sSet{V, \xi}$ be a convergence vector space. The \udef{locally convex modification} of $V$ is the initial convergence on $V$ w.r.t. the set of continuous seminorms on $V$.

We denote the locally convex modification of $\xi$ by $\lconvMod(\xi)$.
\end{definition}

\begin{lemma} \label{locallyConvexModLemma}
Let $\sSet{V, \xi}$ be a convergence vector space. The locally convex modification $\lconvMod(\xi)$ is the least locally convex topological vector space convergence that contains $\xi$.
\end{lemma}
In particular $\xi \leq \lconvMod(\xi)$.
\begin{proof}
Let $S$ be the set of continuous seminorms on $\sSet{V, \xi}$.

As $\xi$ makes all the functions in $S$ continuous, we must have $\xi \leq \lconvMod(\xi)$.

The convergence $\lconvMod(\xi)$ is locally convex topological by \ref{locallyConvexSeminormTopology}.

Finally, let $\zeta$ be another locally convex topological vector convergence on $V$ such that $\xi \leq \zeta$. Then, by \ref{locallyConvexSeminormTopology}, $\zeta$ is the initial convergence w.r.t. some set $S'$ of seminorms. We must have $S'\subseteq S$, so $\lconvMod(\xi)\leq \zeta$.
\end{proof}

\begin{proposition} \label{neighbourhoodLconvMod}
Let $\sSet{V,\xi}$ be a vector convergence space and $S$ the set of continuous seminorms on $V$. Then
\begin{enumerate}
\item $\neighbourhood_{\lconvMod(\xi)}(0) = \upset \bigcup_{p\in S} p^{\preimf\imf}\big(\neighbourhood_\R(0)\big) = \upset \setbuilder{p^{\preimf}\big(\ball_\R(0, \epsilon)\big)}{p\in S, \epsilon > 0}$;
\item $\neighbourhood_{\lconvMod(\xi)}(0) = \upset\disked^{\imf}\big(\vicinity_\xi(0)\big)$.
\end{enumerate}
\end{proposition}
\begin{proof}
(1) Comparing with \ref{pretopologicalInitialConvergence}, we need to show that for any two continuous seminorms $p_1, p_2$ on $V$ and $\epsilon_1,\epsilon_2 > 0$, there exists a continuous seminorm $q$ on $V$ and $\epsilon >0$ such that
\[ q^\preimf\big(\ball_\R(0,\epsilon)\big) \subseteq p_1^\preimf\big(\ball_\R(0,\epsilon_1)\big) \cap p_2^\preimf\big(\ball_\R(0,\epsilon_2)\big). \]
We may simply take $\epsilon = \min\{\epsilon_1, \epsilon_2\}$ and $q = p_1+p_2$, which is continuous. Indeed, $x\in (p_1+p_2)^\preimf\big(\ball_\R(0,\epsilon)\big)$ is equivalent to $p_1(x) + p_2(x) < \epsilon$, which, by positivity of the seminorm \ref{seminormPositivity}, implies $p_1(x)\leq \epsilon \leq \epsilon_1$ and $p_2(x)\leq \epsilon \leq \epsilon_2$.

(2) By \ref{locallyConvexNeighbourhoodsLemma}, $\neighbourhood_{\lconvMod(\xi)}(0)$ has a base of absolutely convex sets. As $\vicinity_\xi(0)\subseteq\neighbourhood_{\lconvMod(\xi)}(0)$, we have $\upset\disked^{\imf}\big(\vicinity_\xi(0)\big) \subseteq \neighbourhood_{\lconvMod(\xi)}(0)$. The show the converse, we just need to prove that the pretopological convergence with vicinity filter $\upset\disked^{\imf}\big(\vicinity_\xi(0)\big)$ at $0$ is a locally convex topological vector space convergence. We prove this by verifying the three points of \ref{LCTVSconstruction}.

First, take $A\in \vicinity_\xi(0)$. Then for all $\lambda\in\F\setminus\{0\}$, $\lambda A \in \vicinity_\xi(0)$ by \ref{vicinityFilterAtOrigin}. As $\disked(\lambda A) = \lambda\disked(A)$, by \ref{balancedHullHomogeneous}, \ref{convexHullHomogeneous} and \ref{diskedIsCoBal}, we have $\lambda \disked(A)\in \upset\disked^{\imf}\big(\vicinity_\xi(0)\big)$.

Secondly, each $A\in \vicinity_\xi(0)$ is absorbent by \ref{vicinityFilterAtOrigin}. As $A\subseteq \disked(A)$, $\disked(A)$ is absorbent by \ref{absorbingSetProperties} and so each element of $\upset\disked^{\imf}\big(\vicinity_\xi(0)\big)$ is absorbent.

The final point is immediate by construction.
\end{proof}

\begin{proposition} \label{lconvContinuityLinearFunctions}
Let $\sSet{V,\xi}$ and $\sSet{W,\zeta}$ be vector convergence spaces and $T: V\to W$ a linear function. Then
\begin{enumerate}
\item $T: \sSet{V,\xi} \to \sSet{W, \lconvMod(\zeta)}$ is continuous \textup{if and only if} $T: \sSet{V,\lconvMod(\xi)} \to \sSet{W, \lconvMod(\zeta)}$ is continuous;
\item if $T: \sSet{V,\xi} \to \sSet{W, \zeta}$ is continuous, then $T: \sSet{V,\lconvMod(\xi)} \to \sSet{W, \lconvMod(\zeta)}$ is continuous.
\end{enumerate}
\end{proposition}
\begin{proof}
(1) The direction $\Leftarrow$ is immediate by \ref{locallyConvexModLemma}.

Now suppose $T: \sSet{V,\xi} \to \sSet{W, \lconvMod(\zeta)}$ is continuous. Take arbitrary $F\in \powerfilters(V)$ that converges to $0$ in $\lconvMod(\xi)$. We just need to show that $\upset T^{\imf\imf}(F)$ converges to $0$ in $\lconvMod(\zeta)$, which, by \ref{initialFinalConvergence}, is equivalent to $\upset (p\circ T)^{\imf\imf}(F)$ converging to $0$ for all continuous seminorms on $\zeta$. And indeed all these filter do converge because $p\circ T$ are continuous seminorms on $\xi$ and thus $\upset (p\circ T)^{\imf\imf}(F)$ converges to $0$ by \ref{initialFinalConvergence}.

(2) If $T: \sSet{V,\xi} \to \sSet{W, \zeta}$ is continuous, then $T: \sSet{V,\xi} \to \sSet{W, \lconvMod(\zeta)}$ is continuous by \ref{locallyConvexModLemma} and $T: \sSet{V,\lconvMod(\xi)} \to \sSet{W, \lconvMod(\zeta)}$ is continuous by point (1).
\end{proof}

\begin{proposition} \label{equicontinuousSetsLconvMod}
Let $\sSet{V,\xi}$ be a convergence vector space and $\sSet{W,\zeta}$ a locally convex TVS. Then $\contLin(V,W)$ and $\contLin\big(\lconvMod(V), W\big)$ have the same equicontinuous subsets.
\end{proposition}
\begin{proof}
As sets, we have $\contLin(V,W) = \contLin\big(\lconvMod(V), W\big)$ by \ref{lconvContinuityLinearFunctions}.

Since $\xi \leq \lconvMod(\xi)$, we have that each equicontinuous subset of $\contLin\big(\lconvMod(V), W\big)$ is an equicontinuous subset of $\contLin(V,W)$.

Now suppose $H$ is an equicontinuous subset of $\contLin(V,W)$. 
In order to prove equicontinuity of $H$ as a subset of $\contLin\big(\lconvMod(V), W\big)$, it is enough to prove
\[ \upset \evalMap^{\imf\imf}\big(\{H\}\otimes \neighbourhood_{\lconvMod(\xi)}(0)\big) \overset{\zeta}{\longrightarrow} 0 \]
by \ref{equicontinuityGroupHomomorphisms}.

By equicontinuity of $H$ as a subset of $\contLin(V,W)$ and \ref{equicontinuityGroupHomomorphisms}, we have
\[ \upset \evalMap^{\imf\imf}(\{H\}\otimes F) \overset{\zeta}{\longrightarrow} 0 \]
for all $F\overset{\xi}{\longrightarrow} 0$. Because $\zeta$ is pretopological, this implies that
\begin{align*}
\bigcap_{F\overset{\xi}{\longrightarrow} 0}\upset \evalMap^{\imf\imf}(\{H\}\otimes F) &= \upset \evalMap^{\imf\imf}\Big(\bigcap_{F\overset{\xi}{\longrightarrow} 0}\big(\{H\}\otimes F\big)\Big) \\
&= \upset\evalMap^{\imf\imf}\Big(\{H\}\otimes \bigcap_{F\overset{\xi}{\longrightarrow} 0} F\Big) = \upset\evalMap^{\imf\imf}\big(\{H\}\otimes \vicinity_\xi(0)\big)
\end{align*}
converges to $0$ in $\zeta$. The equalities follow from \ref{imageUpsetsPreservesIntersection}, \ref{productGaloisConnections} and \ref{upsetResiduatedImageGaloisConnection}.
Now, we have for all $A\subseteq V$, by \ref{orderPreservingFunctionLatticeOperations} and \ref{linearFunctionsPreserveDiskedHull}, that
\begin{align*}
\disked\big(\evalMap^{\imf}(H\times A)\big) &= \disked\Big(\bigcup_{f\in H}\evalMap^{\imf}\big(\{f\}\times A\big)\Big) \\
&= \disked\Big(\bigcup_{f\in H}f^\imf(A)\Big) \\
&\supseteq \bigcup_{f\in H}\disked\big(f^\imf(A)\big) \\
&= \bigcup_{f\in H}f^\imf\big(\disked(A)\big) \\
&= \bigcup_{f\in H}\evalMap^\imf\big(\{f\}\times \disked(A)\big) \\
&= \evalMap^{\imf}\big(H\times \disked(A)\big),
\end{align*}
so, by \ref{neighbourhoodLconvMod},
\[ \upset\disked^\imf\Big(\evalMap^{\imf\imf}\big(\{H\}\otimes \vicinity_\xi(0)\big)\Big) \subseteq \upset \evalMap^{\imf\imf}\Big(\{H\}\otimes \disked^{\imf}\big(\vicinity_\xi(0)\big)\Big) = \upset \evalMap^{\imf\imf}\big(\{H\}\otimes \neighbourhood_{\lconvMod(\xi)}(0)\big), \]
which converges to $0$ by local convexity of $\zeta$ (\ref{locallyConvexNeighbourhoodsLemma}).
\end{proof}

\subsection{Initial and final locally convex spaces}
\begin{proposition}
The initial convergence w.r.t. a set of linear functions to LCTVSs is a locally convex topological vector convergence.
\end{proposition}
\begin{proof}
Let $V$ be a vector space and $\{u_i: V\to V_i\}$ a set of linear functions to LCTVSs.

The initial convergence on $V$ w.r.t. $\{u_i: V\to V_i\}$ is a topological vector convergence by \ref{pretopologicalInitialConvergence} and \ref{initialVectorSpaceConvergence}.

In order to show that the initial convergence is locally convex, we prove that $\id: \lconvMod(V) \to V$ is continuous. By \ref{characteristicPropertyInitialFinalConvergence}, this follows if each $u_i: \lconvMod(V) \to V_i$ is continuous, which in turn follows from \ref{lconvContinuityLinearFunctions}.
\end{proof}

\begin{proposition}
Let $V$ be a vector space and $\{u_i: \sSet{V_i, \xi_i} \to V\}$ a set of linear functions. Let $\nu$ be the final vector space convergence on $V$ w.r.t. this set and $\nu'$ be the final vector space convergence on $V$ w.r.t. $\{u_i: \sSet{V_i, \lconvMod(\xi_i)} \to V\}$. Then $\lconvMod(\nu) = \lconvMod(\nu')$.
\end{proposition}
\begin{proof}
We clearly have $\nu \leq \nu'$, so $\lconvMod(\nu) \leq \lconvMod(\nu')$. To show the converse, we need to show that $\id: \sSet{V, \lconvMod(\nu')} \to \sSet{V, \lconvMod(\nu)}$ is continuous. By \ref{lconvContinuityLinearFunctions}, this is equivalent to the continuity of $\id: \sSet{V, \nu'} \to \sSet{V, \lconvMod(\nu)}$, which by \ref{characteristicPropertyInitialFinalConvergence} is equivalent to the continuity of each $u_i: \sSet{V_i, \lconvMod(\xi_i)} \to \sSet{V, \lconvMod(\nu)}$, which in turn follows from the continuity of each $u_i: \sSet{V_i, \xi_i}\to \sSet{V, \nu}$ by \ref{lconvContinuityLinearFunctions}.
\end{proof}

\begin{proposition}
Let $\sSet{I, \{\sSet{V_i, \xi_i}\}_{i\in I}, \{e_{i,j}\}_{i\preceq j}}$ be an inductive system of LCTVSs with linear linking morphisms $e_{i,j}$.

Then $\topMod\big(\varinjlim_{i\in I}V_i\big) = \lconvMod\big(\varinjlim_{i\in I}V_i\big)$.
\end{proposition}
\begin{proof}
TODO?? Beattie / Butzmann p. 105
\end{proof}

\section{Hahn-Banach extension theorems}
\begin{theorem}[Hahn-Banach majorised by convex functionals] \label{convexHahnBanach}
Let $V$ be a real vector space, $U\subset V$ a subspace and $p$ a convex functional on $V$. Let $f:U\to\R$ be a linear functional that is bounded by $p$:
\[ \forall u\in U: \quad f(u) \leq p(u). \]
Then $f$ has an extension $\tilde{f}: V\to \R$ such that $\tilde{f}$ is a linear functional on $V$ bounded by $p$:
\[ \forall v\in V: \tilde{f}(v) \leq p(v) \qquad \text{and} \qquad \forall u\in U: \tilde{f}(u) = f(u). \]
\end{theorem}
\begin{proof}
As a first step, we want to extend $f$ to a functional $g$ on a space that is one dimension larger than $U$. This means $g$ is of the form
\[ g: U\oplus\Span\{v_1\}\to\R: v + \alpha v_1 \mapsto f(v) + \alpha c \]
for some $v_1\in V\setminus U$.

If we want $g$ to be majorised by $p$, then we need to find a $c$ such that
\[ \forall v\in U: \forall \alpha\in\R: \; g(\alpha v_1 + v) = \alpha c + f(v) \leq p(\alpha v_1 + v) \]
this means that we need
\[ \forall v\in U: \forall \alpha\in\R:\; \frac{-p(v - |\alpha|v_1) + f(v)}{|\alpha|} \leq c \leq \frac{p(v + |\alpha|v_1) - f(v)}{|\alpha|} \]
and we can find such a $c$ if and only if
\[ \forall v\in U: \forall \alpha\in\R:\; -p(v - |\alpha|v_1) + f(v) \leq p(v + |\alpha|v_1) - f(v), \]
which is equivalent to $2f(v) \leq p(v+|\alpha|v_1)+p(v-|\alpha|v_1)$. This follows from
\begin{align*}
f(v) \leq p(v) &= p(\tfrac{1}{2}(v+|\alpha|v_1) + \tfrac{1}{2}(v-|\alpha|v_1)) \\
&\leq \tfrac{1}{2}p(v+|\alpha|v_1) + \tfrac{1}{2}p(v-|\alpha|v_1).
\end{align*}
So we can extend the domain of $f$ by one dimension such that it is still majorised by $p$.

We can iterate the construction to extend $f$ by multiple dimensions. Each extension can be viewed as a subset of $V\times \R$, by identifying it with its graph.
Consider the family of all such subsets that determine a majorised extension of $f$ (not just those obtained by iteration of the previous construction!). This is a family of finite character. We apply the Teichmüller-Tukey lemma, \ref{ZornEquivalents}, to obtain a maximal element.

This maximal element has domain $V$, because if it did not, it could be extended and was not a maximal element.
\end{proof}
Clearly if $V$ has a well-ordered Hamel basis, we do not need choice as we can just take successive $v$s in the basis and find $c$s constructively.
\begin{corollary}[Hahn-Banach majorised by sublinear functionals] \label{sublinearHahnBanach}
Any majorant $p$ that is sublinear is also convex and can be used in the Hahn-Banach theorem.
\end{corollary}
\begin{corollary}[Hahn-Banach majorised by seminorms] \label{seminormHahnBanach}
Let $(\mathbb{F},V,+)$ be a real or complex vector space, $U\subset V$ a subspace and $p$ a seminorm on $V$. Let $f:U\to\mathbb{F}$ be a linear functional that is bounded by $p$:
\[ \forall u\in U: \quad |f(u)| \leq p(u). \]
Then $f$ has an extension $\tilde{f}: V\to \R$ such that $\tilde{f}$ is a linear functional on $V$ bounded by $p$:
\[ \forall v\in V: |\tilde{f}(v)| \leq p(v) \qquad \text{and} \qquad \forall u\in U: \tilde{f}(u) = f(u). \]
\end{corollary}
\begin{proof}
First assume $V$ is a \emph{real} vector space. Because every seminorm is a sublinear function, we can use \ref{sublinearHahnBanach} to find an extension $\tilde{f}$. We then just need to check it satisfies $\forall v\in V: |\tilde{f}(v)| \leq p(v)$.
From \ref{sublinearHahnBanach} we know $\forall v\in V: \tilde{f}(v) \leq p(v)$.
To prove $-\tilde{f}(v) \leq p(v)$, we calculate
\[ -\tilde{f}(v) = \tilde{f}(-v) \leq p(-v) = |-1|p(v) = p(v). \]

If $V$ is a \emph{complex} vector space, consider the realification $V_\R$ and apply the preceding proof to obtain a linear functional $g: V_\R \to \R$ that extends $f$ and is majorised by $p$. Then by \ref{complexRangeExtensionRealFunctional} we can find a complex-linear functional $\tilde{f}:V \to \C$ such that $\Re(\tilde{f}) = g$.

We just need to show that $f$ is bounded by $p$. Take arbitrary $v\in V$ and write $\tilde{f}(v) = |\tilde{f}(v)|e^{i\theta}$ then
\[ |\tilde{f}(v)| = \Re|\tilde{f}(v)| = \Re\Big(e^{-i\theta}\tilde{f}(v)\Big) = \Re\Big(\tilde{f}(e^{-i\theta}v)\Big) = g(e^{-i\theta}v) \leq p(e^{-i\theta}v) = |e^{-i\theta}|p(v) = p(v). \]
\end{proof}
\begin{corollary}
Let $V$ be a locally convex convergence vector space, $U\subseteq V$ a subspace and $f:U\to \F$ a continuous functional. Then $f$ has a continuous extension to $V$.
\end{corollary}
\begin{proof}
We have that $|f| = p_K$ for some $K\in \vicinity_U(0)$ by \ref{absoluteFunctionalGauge}. 
By continuity of the inclusion map, we can find an $M \in \vicinity_V(0)$ such that $M\cap U = K$. Then $|f|\leq p_M$ and $f$ can be extended by the Hahn-Banach extension theorem to a functional $f'$ defined in the whole of $V$. Then $f'$ is bounded on $M$ and thus continuous by \ref{boundedOnVicinityImpliesContinuous}.
\end{proof}


\subsection{Hahn-Banach separation}

\begin{lemma} \label{gaugeSeparationLemma}
Let $V$ be a real vector space, $A$ an absorbent, semibalanced set and $x_0 \notin A$. Consider the functional $f_{x_0}: \Span\{x_0\}\to \F: tx_0 \mapsto t$. Then $f_{x_0}(x)\leq p_A(x)$ for all $x\in \Span\{x_0\}$.
\end{lemma}
\begin{proof}
Let $x = tx_0$. If $t\leq 0$, then the inequality is immediate. Suppose $t>0$. Because $p_A(x_0) \geq 1$ (by the converse of \ref{gaugeLemma}), we have
\[ f_{x_0}(x) = f_{x_0}(tx_0) = t \leq tp_A(x_0) = p_A(tx_0) = p_A(x)  \]
using positive homogeneity (\ref{gaugeScaling}).
\end{proof}

\begin{theorem}[Mazur] \label{MazurTheorem}
Let $V$ be a real or complex convergence vector space and $A$ an open and convex set. If $U$ is a subspace such that $A\perp U$, then there exists a closed hyperplane $H \supseteq U$ such that $A\perp H$.
\end{theorem}
\begin{proof}
First suppose $V$ is a \emph{real} vector space. Because $A$ is open, it is algebraically open. Take $a\in A$. Then $0\in a-A = \inh_{\mathfrak{a}}(a-A)$, by \ref{constructionsInAlgebraicConvergence}, so $a-A$ is absorbing by \ref{coreProperties}. It is also semibalanced (by \ref{convexAbsorbentImpliesSemibalanced} as it is convex by \ref{translationScalingConvexSet}).

Then we have
\[ U\perp A \iff 0\notin U-A \iff a \notin a-A+U. \]
Consider the functional $f_{a}$ of \ref{gaugeSeparationLemma}, which is majorised by the gauge $p_{a-A+U}$, which is sublinear by \ref{gaugeProperties}. Then $f_a$ can be extended as an $\R$-linear function to all $V$ by the Hahn-Banach extension theorem \ref{sublinearHahnBanach}.

We note that $U\subseteq \ker(f_a)$, because $p_{a-A+U}(u) = 0$ by \ref{gaugeZeroLemma}.

In order to conclude with \ref{functionalBoundedNeighbourhood}, we need to show that $A-a \subseteq f_a^{\preimf}(\ball(0,|f_a(a)|)) = f_a^{\preimf}(\ball(0,1))$.
Indeed $A-a \subseteq U+A-a = \inh_\mathfrak{a}(U+A-a) \subseteq p_{U+A-a}^\preimf[\ball(0,1)] \subseteq f_{a}^\preimf[\ball(0,1)]$ by \ref{algebraicallyOpen} and \ref{gaugeInherenceAdherence} ($U+A-a$ is semibalanced because $A-a$ is).

Note that $\ker(f_a)^c$ contains the open set $A$ and thus $\ker(f_a)$ is not dense by \ref{openDensityLemma}. By \ref{hyperplaneClosedDense} this means that $\ker(f_a)$ is closed.

Now suppose $V$ is a \emph{complex} vector space. We can consider the realification $V_\R$ with the same convergence, which is a real convergence vector space by TODO \ref{}. So we can use the preceding proof to find a real hyperplane $K$ in $V$. Then \ref{realComplexHyperplane} gives that $H = K\cap iK$ is a complex hyperplane in $V$. Now $H$ and $A$ must be disjoint because $K$ and $A$ are disjoint and $H \subseteq K$.

Also $H$ is closed because $K$ and $iK$ are closed (the first by the preceding proof, the second because multiplication by $i$ is a homeomorphism \ref{continuityLemmaVectorConvergence}) and the intersection of two closed sets is closed.
\end{proof}
\begin{corollary} \label{functionalZeroOnClosedSubSpace}
Let $V$ be a locally convex vector space and $M$ a closed subspace. There exists a non-zero bounded linear functional $f$ on $V$ such that $M\subseteq \ker(f)$.
\end{corollary}
\begin{proof}
The set $M^c$ is open and by local convexity it contains a convex set $C$. We may take $C$ open by replacing it with its interior, which is convex by \ref{inherenceAdherenceConvex}. Now $C\perp M$ and we can apply the theorem. Now $\ker(f)$ is closed, so $f$ is bounded by \ref{continuityLinearFunctionals}.
\end{proof}

\begin{theorem}[Hahn-Banach separation theorem] \label{HahnBanachSeparation}
Let $V$ be a convergence vector space. Suppose $A,B$ are disjoint, non-empty, convex sets and that $A$ is open. Then there exists a continuous linear functional $f:V\to \F$ such that $f^\imf[A]$ and $f^\imf[B]$ are disjoint.
\end{theorem}
\begin{proof}
The set $A-B = \bigcup_{b\in B}A-b$ is convex and open by \ref{sumOpenSetsOpen}.
The set $A-B$ and the vector space $\{0\}$ are disjoint, so by \ref{MazurTheorem} we can find a closed hyperplane that is disjoint with $A-B$.

By \ref{kernelHyperplane} and \ref{continuityLinearFunctionals} this is the kernel of a continuous linear functional $f$.
\end{proof}
\begin{corollary} \label{separatingFunctionalOrderedImage}
Let $V$ be a real or complex convergence vector space and $A,B$ as in the proposition. Then there exists a continuous linear functional $f:V\to \F$ and $t\in \R$ such that
\[ \Re f(a) < t \leq \Re f(b) \]
for all $a\in A$ and $b\in B$.
\end{corollary}
This means $A$ and $B$ are separated by a closed affine hyperplane $\ker(f)+v$, where $v \in f^\preimf[\{t\}]$.

We can reverse the inequalities by replacing $f$ by $-f$.
\begin{proof}
Apply the proposition to the realification $V_\R$. This gives us an $\R$-linear functional $g: V\to \R$ such that $g^\imf[A]$ and $g^\imf[B]$ are disjoint convex sets. Additionally $g^{\imf}[A]$ is open in $\R$ by \ref{linearFunctionalOpen}.

Because $g^\imf[A]$ and $g^\imf[B]$ are convex, we either have $g^\imf[A]\leq g^\imf[B]$ or $g^\imf[A]\geq g^\imf[B]$. In the second case we simply replace $g$ by $-g$ to obtain the first case. We may take $t= \sup g^\imf[A]$. This is not in $g^\imf[A]$ because it is open.

If $V$ is a real vector space we take $f=g$ are done. If $V$ is complex, we can find a suitable $f$ by \ref{complexRangeExtensionRealFunctional}.
\end{proof}
\begin{corollary} \label{locallyConvexHahnBanachSeparationClosedSet}
Let $\sSet{V, \xi}$ be a locally convex vector convergence space. Let $B$ be a closed convex set and $v\notin B$, then there exists a continuous linear functional $f:V\to \F$ such that $f(v) \notin \overline{f^\imf[B]}$.
\end{corollary}
\begin{proof}
As $B^c$ is open, it is a neighbourhood of $v$ and we can find an convex neighbourhood $U$ of $v$ in $B^c$. Now replace $U$ by its interior. This is open and still convex by \ref{inherenceAdherenceConvex}, so we can apply the theorem.

Then $f^\imf[U]$ is open by \ref{linearFunctionalOpen} and thus in $\neighbourhood_\F\big(f(v)\big)$. It is however disjoint from $f^\imf[B]$, which means that $f(v) \notin \overline{f^\imf[B]}$.
\end{proof}
\begin{corollary}
Let $V$ be a locally convex TVS. Suppose $A,B$ are disjoint, non-empty, convex sets and that $A$ is compact, $B$ is closed. Then there exists a continuous linear functional $f:V\to \F$ and $s,t\in \R$ such that
\[ \Re f(a) < t < s < \Re f(b) \]
for all $a\in A$ and $b\in B$.
\end{corollary}
\begin{proof}
TODO
\end{proof}
\begin{corollary} \label{locallyConvexDualPair}
Let $V$ be a Hausdorff locally convex convergence vector space and $v\in V$. If $f(v) = 0$ for all $f\in V^*$, then $v = 0$.
\end{corollary}
\begin{proof}
We prove the contrapositive. Assume $v\neq 0$. By \ref{FrechetCharacterisation}, we have that $\{0\}$ is closed, so $\{0\}^c$ is a neighbourhood of $v$. By local convexity, $\{0\}^c$ contains a convex vicinity of $x$. We may take this vicinity to be open by taking the interior (which is still convex by \ref{inherenceAdherenceConvex}). Let this open vicinity be $A$ and set $B = \{0\}$. By Hahn-Banach separation, there exists $f\in \dual{V}$ such that $f^\imf[A]$ and $f^\imf[B] = \{0\}$ are disjoint. Thus $f(v) \neq 0$.
\end{proof}

\subsection{Banach limits}
\begin{proposition}
There exists a linear map $L:l^\infty(\N) \to \C$ satisfying
\begin{enumerate}
\item $\displaystyle L(x) = \lim_{n\to \infty}x_n$ if the limit exists;
\item $L((x_{n+1})_{n\in\N}) = L((x_n)_{n\in\N})$;
\item if $\forall n\in\N:x_n\geq 0$, then $L(x) \geq 0$;
\item $\norm{L} = 1$.
\end{enumerate}
Such a linear map is called a \udef{Banach limit}.
\end{proposition}
\begin{proof}
TODO, after Cesàro means.
\end{proof}

\section{Continuous functionals}

TODO???
\begin{proposition}
Let $V$ and $W$ be TVSs and $f: V\to W$ a linear function.
\begin{enumerate}
\item If $f$ is continuous and $W$ is Hausdorff, then $\ker(f)$ is closed.
\item If $f$ has closed kernel and finite-dimensional image, then $f$ is continuous.
\end{enumerate}
\end{proposition}
\begin{proof}
(1) Because $W$ is Hausdorff, it is also $T_1$ and thus $\{0\}$ is closed by \ref{FrechetCharacterisation}. Then $\ker(f) = f^{\preimf}(\{0\})$ is closed by \ref{continuity}.

(2) 
\end{proof}
??

\chapter{Boundedness and bornology}

\section{Bornologies}
\begin{definition}
Let $V$ be a vector space. A bornology $\mathcal{B}\subseteq \powerset(V)$ is called a \udef{vector bornology} if
\begin{itemize}
\item $A+B\in \mathcal{B}$ for all $A,B\in \mathcal{B}$;
\item $\lambda A \in \mathcal{B}$ for all $\lambda\in \F$ and $A\in \mathcal{B}$;
\item $\balanced(A) \in \mathcal{B}$ for all $A\in \mathcal{B}$.
\end{itemize}
It is called a \udef{convex vector bornology} if, in addition,
\begin{itemize}
\item $\convex(A) \in \mathcal{B}$ for all $A\in \mathcal{B}$.
\end{itemize}
A pair $\sSet{V,\mathcal{B}}$, where $V$ is a vector space and $\mathcal{B}$ is a vector bornology on $V$ is called a \udef{bornological vector space}.
\end{definition}

\subsection{Convergence derived from bornology}
\begin{definition}
Let $\sSet{V, \mathcal{B}}$ be a bornological vector space. The \udef{associated convergence} $\bornConv{\mathcal{B}}$ is defined by
\[ F\overset{\bornConv{\mathcal{B}}}{\longrightarrow} v \qquad\defequiv\qquad \exists B\in \mathcal{B}: \; \neighbourhood_\F(0)\cdot B \subseteq F-v, \]
for all $F\in \powerfilters(V)$ and $v\in V$.
\end{definition}

\begin{lemma}
Let $\sSet{V, \mathcal{B}}$ be a bornological vector space. The associated convergence $\bornConv{\mathcal{B}}$ is a vector space convergence.
\end{lemma}
\begin{proof}
It is straightforward to verify the points in \ref{vectorSpaceConvergenceConstruction}.
\end{proof}

\section{Von Neumann boundedness}
\begin{definition}
Let $\sSet{V,\xi}$ be a convergence vector space on a field $\F$. A subset $A\subseteq V$ is called \udef{von Neumann bounded} or \udef{vN bounded} if $\neighbourhood_\F(0) \cdot^{\imf\imf} \upset\{A\} \overset{\xi}{\longrightarrow} 0$.
\end{definition}
It may be more correct to trace this definition of bounded to Mazur and Orlicz or Kolmogorov.

Clearly subsets of bounded sets are bounded.

\begin{lemma} \label{vonNeumannBoundednessAbsorption}
Let $\sSet{V,\xi}$ be a convergence vector space on a field $\F$ and $A\subseteq V$ a subset. 
\begin{enumerate}
\item If $A$ is von Neumann bounded, then $A$ is absorbed by all vicinities of the origin (i.e.\ by all elements of $\vicinity_{\xi}(0)$).
\item If $\xi$ is topological, the following are equivalent:
\begin{enumerate}
\item $A$ is von Neumann bounded;
\item $A$ is absorbed by all vicinities of the origin;
\item $A$ is absorbed be all elements of a base of $\neighbourhood_\xi(0)$;
\item $\forall U\in \neighbourhood_{\xi}(0): \exists r > 0: \forall c \geq r: \; A \subseteq cU$;
\item $\{\interval{0,\epsilon}\}_{\epsilon >0}\cdot \{A\} \overset{\xi}{\longrightarrow} 0$.
\end{enumerate}
\end{enumerate}
\end{lemma}
\begin{proof}
We just need to show that ``$A$ is absorbed by all vicinities of the origin'' is equivalent to $\vicinity_\xi(0) \subseteq \neighbourhood_\F(0) \cdot \{A\}$. Indeed $A$ is absorbed by $V\in \vicinity_\xi(0)$ iff there exists $\epsilon >0$ such that $\ball(0,\epsilon)\cdot A \subseteq V$ (see \ref{absorbingSetLemma}). Then (1) follows from the fact that $\vicinity_\xi(0)$ is smaller than all convergent filters.

(2) The implication $(a) \Rightarrow (b)$ is given by point (1).

$(b)\Rightarrow (a)$ As above, $(b)$ implies $\neighbourhood_\xi(0) \subseteq \neighbourhood_\F(0) \cdot \{A\}$. By topologicity this implies that $\neighbourhood_\F(0) \cdot \{A\}$ converges and thus that $A$ is von Neumann bounded.

The implication $(b)\Rightarrow (c)$ is immediate and the converse follows by \ref{absorbingSetProperties}.

$(a) \Rightarrow (d)$ is immediate.

$(d) \Rightarrow (c)$ Pick a balanced base of $\neighbourhood_{\xi}(0)$ (which is possible due to \ref{vicinityFilterAtOrigin}). Then the result follows immediately from the observation that $cU = |c|U$ (\ref{balancedLemma}) for all balanced sets.

$(d) \Leftrightarrow (e)$ We have that $\forall c \geq r: A\subseteq cU$ is equivalent to $\interval{0,r^{-1}}\cdot A \subseteq U$. Thus $(d)$ is equivalent to $\neighbourhood_\xi(0)\subseteq \{\interval{0,\epsilon}\}_{\epsilon >0}\cdot \{A\}$, which is equivalent to $(e)$ because $\xi$ is topological.
\end{proof}


\begin{lemma} \label{vonNeumannBoundedSetLemma}
Let $\sSet{V,\xi}$ be a convergence vector space and $A,B\subseteq V$ von Neumann bounded subsets. Then
\begin{enumerate}
\item every $C\subseteq A$ is von Neumann bounded;
\item the von Neumann bounded sets cover $V$;
\item $\lambda A$ is von Neumann bounded for all $\lambda\in\F$;
\item $A+B$ is von Neumann bounded;
\item $\balanced(A)$ is von Neumann bounded.
\end{enumerate}
If $\xi$ is of finite depth, then
\begin{enumerate} \setcounter{enumi}{2}
\item $A\cup B$ is von Neumann bounded.
\end{enumerate}
If $\xi$ is topological, then
\begin{enumerate} \setcounter{enumi}{3}
\item $\closure_\xi(A)$ is von Neumann bounded.
\end{enumerate}
If $\xi$ is topological and locally convex, then
\begin{enumerate}\setcounter{enumi}{6}
\item $\disked(A)$ is von Neumann bounded.
\end{enumerate}
\end{lemma}
\begin{proof}
(1) We have $\neighbourhood_\F(0) \cdot^{\imf\imf} \upset\{A\} \subseteq \neighbourhood_\F(0) \cdot^{\imf\imf} \upset\{C\}$.

(2) It is enough to show that $\{v\}$ is von Neumann bounded for all $v\in V$. We have that $\neighbourhood_\F(0) \cdot^{\imf\imf} \pfilter{v}$ converges for all $v\in V$ by continuity of scalar multiplication and the fact that $\pfilter{v}$ converges. 

(3) We have, by \ref{balancedClosures},
\[ \neighbourhood_\F(0) \cdot \upset\{A\} = \upset \{\cball(0,\epsilon) \cdot A\}_{\epsilon >0} = \upset \{\cball(0,\epsilon)\cdot \cball(0,1) \cdot A\}_{\epsilon >0} = \upset \{\cball(0,\epsilon)\cdot \balanced(A)\}_{\epsilon >0} = \neighbourhood_\F(0) \cdot \upset\{\balanced(A)\} \]

(4,5) Immediate by continuity of scalar multiplication and addition.

(6) We have $\neighbourhood_\F(0)\cdot^{\imf\imf}\upset\{A\cup B\} = \neighbourhood_\F(0)\cdot^{\imf\imf}\upset\{A\} \cap \neighbourhood_\F(0)\cdot^{\imf\imf}\upset\{B\}$.

(7) By \ref{vicinityFilterAtOrigin}, we may take a closed and balanced base of $\vicinity_{\xi}(0)$.  
By \ref{vonNeumannBoundednessAbsorption} it is enough to show absorption by these basis elements. This is immediate, because for all closed and bounded $U$, we have $A\subseteq cU$ iff $\closure_\xi(A) \subseteq cU$ iff $\disked(A) \subseteq cU$.

(8) Similar, now taking convex balanced base, \ref{locallyConvexNeighbourhoodsLemma}.
\end{proof}
\begin{corollary}
Let $\sSet{V, \xi}$ be a convergence vector space. The set of von Neumann bounded subsets of $V$ forms a vector space bornology.
\end{corollary}

\begin{definition}
Let $\sSet{V, \xi}$ be a convergence vector space. The set of von Neumann bounded subsets of $V$ is called the \udef{von Neumann bornology} of $V$ and is denoted $\vNborn{\xi}$.
\end{definition}

\subsection{Von Neumann boundedness and continuity}
\begin{proposition} \label{continuousMappingBoundedSets}
Let $\sSet{V,\xi}, \sSet{W,\zeta}$ be vector convergence spaces, $f:V\to W$ a function and $A\subseteq V$ a von Neumann bounded subset. Then
\begin{enumerate}
\item if $f$ is linear and continuous, then $f^\imf(A)$ is von Neumann bounded;
\item if $\zeta$ is topological and $f$ is positively homogeneous and continuous at $0$, then $f^\imf(A)$ is bounded.
\end{enumerate} 
\end{proposition}
\begin{proof}
(1) As $A$ is von Neumann bounded set, we have $\neighbourhood_\F(0) \cdot \upset\{A\} \overset{\xi}{\longrightarrow} 0$. By continuity and linearity of $f$, we have
\[ \neighbourhood_\F(0) \cdot \upset\big\{f^{\imf}(A)\big\} = \upset f^{\imf\imf}\big(\neighbourhood_\F(0)\cdot \{A\}\big) \overset{\zeta}{\longrightarrow} 0, \]
so $f^\imf(A)$ is von Neumann bounded.

(2) We have $\neighbourhood_\F(0) \cdot \upset\{A\} \subseteq \{\interval{0,\epsilon}\}_{\epsilon >0} \cdot \upset\{A\} \overset{\xi}{\longrightarrow} 0$. By positive homogeneity and continuity at $0$, we have
\[ \{\interval{0,\epsilon}\}_{\epsilon >0} \cdot \upset\big\{f^{\imf}(A)\big\} = \upset f^{\imf\imf}\big(\{\interval{0,\epsilon}\}_{\epsilon >0} \cdot \upset\{A\}\big) \overset{\zeta}{\longrightarrow} f(0) = 0. \]
By \ref{vonNeumannBoundednessAbsorption} this implies that $f^{\imf}(A)$ is von Neumann bounded.
\end{proof}
\begin{corollary}
Let $\sSet{V,\xi}, \sSet{W,\zeta}$ be vector convergence spaces and $f:V\to W$ a linear function. If $f: \sSet{V,\xi}\to \sSet{W,\zeta}$ is continuous, then $f: \sSet{V,\vNborn{\xi}}\to \sSet{W,\vNborn{\zeta}}$ is bounded.
\end{corollary}

\begin{lemma} \label{boundedSetVicinityBase}
Let $\sSet{V,\xi}$ be a topological vector convergence space. If $U\subseteq V$ is a von Neumann bounded neighbourhood of the origin, then
\[ \neighbourhood_\xi(0) = \neighbourhood_\F(0) \cdot\upset\{U\} = \upset\{\epsilon U\}_{\epsilon >0}. \]
\end{lemma}
TODO does this imply topological for equable CVSs?
\begin{proof}
As $U$ is von Neumann bounded, we have $\neighbourhood_\xi(0) \subseteq \neighbourhood_\F(0) \cdot \upset\{U\}$. The converse inclusion is given by the fact that $U\in \neighbourhood_\xi(0)$, so $\upset\{U\}\subseteq \neighbourhood_\xi(0)$ and $\neighbourhood_\F(0) \cdot \upset\{U\} \subseteq \neighbourhood_\F(0) \cdot\neighbourhood_\xi(0) = \neighbourhood_\xi(0)$ by \ref{TVSEquable}.

For the second equality, we have that $U$ contains a balanced neighbourhood $U'$ of the origin by \ref{vicinityFilterAtOrigin}, which is von Neumann bounded by \ref{vonNeumannBoundedSetLemma}. By \ref{balancedClosures}, and the previous argument, we have
\[ \neighbourhood_\xi(0) = \neighbourhood_\F(0) \cdot\upset\{U'\} = \upset \{\cball(0,\epsilon)\cdot U'\}_{\epsilon > 0} = \upset \{\epsilon\cball(0,1)\cdot U'\}_{\epsilon > 0} = \upset \{\epsilon U'\}_{\epsilon > 0}. \]
Then we have
\[ \neighbourhood_\xi(0) = \neighbourhood_\F(0) \cdot\upset\{U\} \subseteq \upset \{\epsilon U\}_{\epsilon > 0} \subseteq \upset \{\epsilon U'\}_{\epsilon > 0} = \neighbourhood_\xi(0). \]
\end{proof}
\begin{corollary} \label{metrisableBoundedNeighbourhood}
Let $\sSet{V,\xi}$ be a topological vector convergence space. If $U\subseteq V$ is a von Neumann bounded neighbourhood of the origin, then
\begin{enumerate}
\item $\sSet{V,\xi}$ is pseudometrisable;
\item if $U$ is convex, then $\sSet{V,\xi}$ is seminormable.
\end{enumerate}
\end{corollary}
\begin{proof}
(1) Clearly $\neighbourhood_\xi(0)$ has a countable base, so the neighbourhood filter at every point has a countabel base by \ref{shiftHomeomorphism}, which means that $\xi$ is second countable. It is also regular by \ref{topologicalGroupsRegular} and this pseudometrisable by Urysohn's metrisation theorem \ref{UrysohnMetrisationTheorem}.

(2) TODO Narici / Beckenstein p. 160.
\end{proof}

\begin{proposition} \label{boundedOnVicinityImpliesContinuous}
Let $\sSet{V,\xi}$ vector convergence space and $\sSet{W,\zeta}$ a topological vector convergence space and $f: V\to W$ a positively homogeneous function. Then
\begin{enumerate}
\item if there exists $U\in \vicinity_\xi(0)$ such that $f^\imf(U)$ is bounded, then $f$ is continuous at the origin;
\item if $f$ is linear, then this implies the continuity of $f$ everywhere.
\end{enumerate}
\end{proposition}
\begin{proof}
(1) By \ref{pretopologicalContinuityVicinities}, it is enough to verify $\neighbourhood_\zeta(0) \subseteq \upset f^{\imf\imf}\big(\vicinity_\xi(0)\big)$. Indeed, by \ref{boundedSetVicinityBase}, we have
\[ \neighbourhood_\zeta(0) = \upset \{\epsilon f^\imf(U)\}_{\epsilon > 0} = \upset f^{\imf\imf}\big(\{\epsilon U\}_{\epsilon > 0}\big) \subseteq \upset f^{\imf\imf}\big(\vicinity_\xi(0)\big), \]
where the last inclusion follows from \ref{vicinityFilterAtOrigin}.

(2) The continuity of $f$ is equivalent to the continuity of $f$ at $0$, by \ref{shiftHomeomorphism}.
\end{proof}
\begin{corollary} \label{continuityToNormedSpace}
Let $\sSet{V, \xi}$ be a convergence vector space, $\sSet{W, \norm{\cdot}}$ a normed space and $f: V\to W$ a positively homogeneous function. Then
\begin{enumerate}
\item $f$ is continuous at $0$ \textup{if and only if} $f$ is bounded on some $U\in \vicinity_\xi(0)$;
\item if $f$ is linear, then this is equivalent to the continuity of $f$ everywhere.
\end{enumerate}
\end{corollary}
\begin{proof}
(1) The direction $\Leftarrow$ is given by the proposition.

For the converse, assume $f$ continuous at $0$, then $\ball\big(0,1\big)\in \neighbourhood_W(0) = \neighbourhood_W\big(f(0)\big) \subseteq f^{\imf\imf}[\vicinity_\xi(0)]$ by \ref{continuityVicinityFilter}. So there exists $U\in \vicinity_\xi(0)$ such that $f^{\imf}[U] \subseteq \ball(0,1)$, which means that $f$ is bounded by $1$ on $U$.

(2) The continuity of $f$ is equivalent to the continuity of $f$ at $0$, by \ref{shiftHomeomorphism}.
\end{proof}

\begin{proposition} \label{vonNeumannBoundednessInitialSpace}
Let $V$ be a vector space and $B\subseteq V$.
\begin{enumerate}
\item Let $L = \{f_i:V\to \sSet{Y_i,\zeta_i}\}_{i\in I}$ be a set of linear functions to topological vector spaces, then $B$ is von Neumann bounded in the initial convergence w.r.t. $L$ iff $f_i^\imf(B)$ is von Neumann bounded for all $f_i\in L$.
\item Let $S$ be a set of seminorms on $V$, then $B$ is von Neumann bounded in the seminorm topology w.r.t. $S$ \textup{if and only if} $p^\imf(B)$ is von Neumann bounded for all $p\in S$.
\end{enumerate}
\end{proposition}
\begin{proof}
(1) We have, by \ref{initialFinalConvergence}, that $\neighbourhood_\F(0)\cdot\upset \{B\}$ converges to $0$ in the initial convergence if and only if $f_i^{\imf\imf}\big(\neighbourhood_\F(0)\cdot\upset \{B\}\big) = \neighbourhood_\F(0)\cdot\upset \big\{f_i^{\imf}(B)\big\} \overset{\xi_i}{\longrightarrow} 0$. I.e\ $f_i^\imf(B)$ is bounded for each $i\in I$.

(2) Since the vector space convergence on $V$ is topological in this case (by \ref{initialSeminormConvergence}), we have that $B$ is von Neumann bounded if and only if $\upset \{\interval{0,\epsilon}\}_{\epsilon > 0}\cdot\upset\{B\}$ converges to $0$, which, by \ref{initialSeminormConvergence}, is equivalent to $\upset p^{\imf\imf}\big(\{\interval{0,\epsilon}\}_{\epsilon > 0}\cdot\upset\{B\}\big) = \upset \{\interval{0,\epsilon}\}_{\epsilon > 0}\cdot\upset\big\{p^{\imf}(B)\big\} \overset{\R}{\longrightarrow} 0$ for all $p\in S$. I.e.\ $p^{\imf}(B)$ is bounded for all $p\in S$.
\end{proof}

\subsection{Uniform and von Neumann boundedness}
\begin{proposition} \label{boundednessTVS}
Let $\sSet{V,\xi}$ be a topological vector space and $B\subseteq V$. Then
\[ \text{$B$ is totally bounded}\quad\implies\quad\text{$B$ is von Neumann bounded}\quad\implies\quad\text{$B$ is bounded}. \]
If $\xi$ is locally convex, then von Neumann boundedness and boundedness are equivalent.
\end{proposition}
In fact all three are equivalent in the case of weak convergences (\ref{weakBoundedness}).
\begin{proof}
First suppose $B$ is totally bounded and let $U$ be a neighbourhood of $0$. By \ref{TVSconstruction}, we can find a balanced neighbourhood $U'$ of $0$ such that $U'+U'\subseteq U$.

Now $\Delta^\preimf(U')\in \entourage_\xi$ by \ref{entourageConvergenceGroup}. So by total boundedness (and \ref{topologicalBoundednessLemma}), there exists a finite set $S\subseteq V$ such that $B\subseteq \bigcup_{v\in S}\Delta^\preimf(U')v$. By \ref{deltaPreimageLemma}, we have
\[ B\subseteq \bigcup_{v\in S}\Delta^\preimf(U')v = \bigcup_{v\in S} v+U' = S+U'. \]
As $S$ is von Neumann bounded by \ref{vonNeumannBoundedSetLemma}, we can find $a\geq 1$ such that $S\subseteq cU'$ for all $|c|\geq a$.

For all $|c|\geq a$, we have $B\subseteq S+U' \subseteq cU' + U' \subseteq cU'+cU' \subseteq cU$. As $U$ was chosen arbitrarily, von Neumann boundedness follows from \ref{vonNeumannBoundednessAbsorption}.

Now assume $B$ is von Neumann bounded. We show boundedness using \ref{topologicalBoundednessLemma}, so take $A\in \entourage_\xi$. Then there exists $U\in \neighbourhood_\xi(0)$ such that $\Delta^\preimf(U)\subseteq A$.

By von Neumann boundedness, we have $B\subseteq mU$ for some $m\in\N$. So, using \ref{deltaPreimageLemma} and \ref{vectorDeltaLemma},
\[ B\subseteq mU = m\Delta^\preimf(U)_{\{0\}} = \Delta^\preimf(mU)_{\{0\}} \subseteq \Delta^\preimf(U)^m_{\{0\}} \subseteq A^m_{\{0\}}. \]
As $A$ was taken arbitrarily, this shows boundedness.

Finally, let $\sSet{V,\xi}$ be a locally convex TVS and assume $B$ bounded. Now $\xi$ is the initial topology w.r.t. some set $S$ of seminorms on $V$, by \ref{locallyConvexSeminormTopology}.

All the seminorms in $S$ are uniformly continuous by \ref{uniformContinuitySeminorms} and thus bounded on $B$ by \ref{metricBoundedness}. This implies von Neumann boundedness by \ref{vonNeumannBoundednessInitialSpace}.
\end{proof}

\begin{proposition} \label{compactSubsetsVonNeumannBounded}
Let $\sSet{V,\xi}$ be a Hausdorff pseudotopological convergence vector space. Then each compact subset of $V$ is von Neumann bounded.
\end{proposition}
TODO: holds in more generality for countably pseudotopological spaces.
\begin{proof}
Let $K\subseteq V$ be a compact set. We need to show that $\neighbourhood_\F(0)\cdot K$ converges to $0$. Since $\xi$ is pseudotopological, it is enough to show that any larger ultrafilter converges to $0$. Take such an ultrafilter $G\supseteq \neighbourhood_\F(0)\cdot K$. Now $\cball(0,\epsilon)\cdot K = \cdot^{\imf}\big(\cball(0,1)\times K\big)$ is compact for all $\epsilon > 0$ by Tychonoff's theorem \ref{TychonoffsTheorem} and \ref{compactConstructions}.

Now $G$ contains $\cball(0,\epsilon)\cdot K$ for all $\epsilon > 0$, so $G$ must converge and its (unique) limit $x$ must lie in each $\cball(0,\epsilon)\cdot K$. Thus $x\in \bigcap_{\epsilon>0}\cball(0,\epsilon)\cdot K \eqdef K_0$.

By \ref{HausdorffCompactIntersection} $K_0$ is compact. Now 
for all $\lambda\in \F$, we have $x\in \cball(0,\epsilon)\cdot K$ iff $\lambda x \in\cball(0,|\lambda|^{-1}\epsilon)\cdot K$. As $\bigcap_{\epsilon>0}\cball(0,\epsilon)\cdot K = \bigcap_{\epsilon>0}\cball(0,|\lambda|^{-1}\epsilon)\cdot K$, we have $\lambda x\in K_0$. This means that $\F\cdot \{x\} \subseteq K_0$. Thus $K_0$ can only be compact if $x=0$.
\end{proof}

\begin{lemma} \label{vonNeumannBoundedImpliesEquicontinuous}
Let $\sSet{V,\xi}, \sSet{W,\zeta}$ be convergence vector spaces and $H\subseteq \contLin_c(V, W)$. If $V$ is equable and $H$ is von Neumann bounded, then $H$ is equicontinuous.
\end{lemma}
\begin{proof}
We want to use \ref{equicontinuityGroupHomomorphisms}, so take arbitrary $F\overset{\xi}{\longrightarrow} 0$. Then there exists an equable filter $G\subseteq F$ which also converges to $0$. Then we have
\[ \upset \evalMap^{\imf\imf}(\{H\}\otimes F) \supseteq \upset \evalMap^{\imf\imf}(\{H\}\otimes G) = \upset \evalMap^{\imf\imf}(\{H\}\otimes \neighbourhood_\F(0)\cdot G) = \upset \evalMap^{\imf\imf}(\neighbourhood_\F(0)\cdot \{H\}\otimes G). \]
Since $\neighbourhood_\F(0)\cdot \{H\}$ continuously converges to $0$ (by the vN boundedness of $H$), we have $\upset \evalMap^{\imf\imf}(\neighbourhood_\F(0)\cdot \{H\}\otimes G)\to 0$.
Thus also $\upset \evalMap^{\imf\imf}(\{H\}\otimes F)$.
\end{proof}

\begin{proposition}
Let $\sSet{V,\xi}$ be a convergence vector space and $H\subseteq \contLin_c(V, \F)$. Then
\begin{enumerate}
\item if $H$ is relatively compact, then $H$ is von Neumann bounded;
\item if $V$ is equable, then the converse also holds.
\end{enumerate}
\end{proposition}
\begin{proof}
(1) The space $\contLin_c(V, \F)$ is Hausdorff and pseudotopological by \ref{continuousConvergencePropertiesFromCodomain}. As $\adh(H)$ is compact, it is also von Neumann bounded, by \ref{compactSubsetsVonNeumannBounded}. Thus $H$ is von Neumann bounded by \ref{vonNeumannBoundedSetLemma}.

(2) Assume $H$ is von Neumann bounded. Then $H$ is equicontinuous by \ref{vonNeumannBoundedImpliesEquicontinuous} and thus evenly continuous by \ref{equicontinuityEvenContinuity}. For all $x\in X$, $\evalMap_x^\imf(H)$ is von Neumann bounded by \ref{continuousMappingBoundedSets}, and thus relatively compact. By \ref{evenContinuityRelativeCompactness} $H$ is relatively compact.
\end{proof}

\begin{proposition}
Let $\sSet{V,\xi}$ be a topological vector space and $U\in\neighbourhood_\xi(0)$.
\begin{enumerate}
\item if $U$ is bounded, then $\xi$ is the initial topology w.r.t. some absolutely homogeneous function $f: V\to \R$;
\item if $U$ is bounded and convex, then $\xi$ is the initial topology w.r.t. some seminorm $f$.
\item if $U$ is bounded and convex and $\xi$ is Hausdorff, then $\xi$ is normable.
\end{enumerate}
\end{proposition}
TODO: (1) implies pseudometrisable? Narici/Beckenstein.
This proposition implies that the balls of a metrisable, non-normable LCTVS are not bounded sets.
\begin{proof}
(1) Suppose $U$ bounded. Then $\balanced(U)$ is bounded by \ref{vonNeumannBoundedSetLemma}, so we may take $U$ balanced WLOG. Then $\{\epsilon U\}_{\epsilon >0}$ forms a base of $\neighbourhood_\xi(0)$ by \ref{boundedSetVicinityBase}. Consider the gauge $p_U$, which is absolutely continuous by \ref{gaugeProperties}. By \ref{gaugeClassificationLemma}, each $\epsilon U$ is contained in a preimage of $p_U$, so the initial topology w.r.t. $p_U$ is stronger than $\xi$. We just need to show that $p_U$ is continuous, which immediately follows from \ref{boundedOnVicinityImpliesContinuous} and \ref{gaugeClassificationLemma}.If $p_U(v) = 0$ for some 

(2) In this case $p_U$ is a seminorm by \ref{gaugeProperties}.

(3) TODO.
\end{proof}

\begin{proposition}
Let $\sSet{V, \xi}$ be a Hausdorff LCTVS. If $V$ has a totally bounded neighbourhood, then $V$ is finite dimensional.
\end{proposition}
\begin{proof}
Robertson/Robertson p. 50.
\end{proof}


\subsection{Local von Neumann boundedness}
\begin{definition}
Let $\sSet{V,\xi}$ be a convergence vector space. Then $V$ is \udef{locally von Neumann bounded} if the set of von Neumann bounded sets forms a convergence cover of $V$.
\end{definition}

\begin{proposition}
Let $\sSet{V,\xi}$ be a locally von Neumann bounded convergence vector space of finite depth and $K\subseteq V$ a compact subset. Then $K$ is von Neumann bounded.
\end{proposition}
\begin{proof}
By \ref{compactFiniteSubcover} the convergence cover of von Neumann bounded subsets has a finite subset that covers $K$. This finite union of von Neumann bounded sets is von Neumann bounded by \ref{vonNeumannBoundedSetLemma} and thus the subset $K$ is von Neumann bounded, also by \ref{vonNeumannBoundedSetLemma}.
\end{proof}

\subsection{The Mackey modification}
\begin{definition}
Let $\sSet{V,\xi}$ be a convergence vector space. The convergence associated to the von Neumann bornology is called the \udef{Mackey modification} $\mackeyMod(\xi)$ of $\xi$.

We call $\sSet{V, \xi}$ a \udef{Mackey space} if $\xi = \mackeyMod(\xi)$.
\end{definition}

\begin{lemma} \label{MackeyModPreservesVNBoudnedSets}
Let $\sSet{V,\xi}$ be a convergence vector space. A subset $B\subseteq V$ is $\xi$-von Neumann bounded \textup{if and only if} $B$ is $\mackeyMod(\xi)$-von Neumann bounded.
\end{lemma}
Thus the von Neumann bornology of the Mackey modification $\mackeyMod(\xi)$ is the same as the von Neumann bornology of $\xi$.
\begin{proof}
If $B$ is $\xi$-von Neumann bounded, then $\neighbourhood_\F(0)\cdot\{B\} \subseteq \neighbourhood_\F(0)\cdot\{B\}$, so
\[ \neighbourhood_\F(0)\cdot\{B\} \overset{\mackeyMod(\xi)}{\longrightarrow} 0, \]
which means that $B$ is $\mackeyMod(\xi)$-von Neumann bounded.

Now assume $B$ is $\mackeyMod(\xi)$-von Neumann bounded. Then $\neighbourhood_\F(0)\cdot\{B'\} \subseteq \neighbourhood_\F(0)\cdot\{B\}$ for some $\xi$-von Neumann bounded $B'\subseteq V$. As $\neighbourhood_\F(0)\cdot\{B'\}$ $\xi$-converges to $0$, $\neighbourhood_\F(0)\cdot\{B\}$ must do so as well, which means that $B$ is $\xi$-von Neumann bounded.
\end{proof}
\begin{corollary} \label{MackeyModLocallyBounded}
Let $\sSet{V,\xi}$ be a convergence vector space. Then $\mackeyMod(\xi)$ is locally bounded.
\end{corollary}

\begin{lemma}
The Mackey modificication $\mackeyMod$ is a dual closure function.
\end{lemma}
\begin{proof}
The larger the convergence, then more von Neumann bounded sets and thus the larger the Mackey modification. Thus $\mackeyMod$ is monotone.

If a filter converges in the Mackey modification, it converges in the original convergence by definition of von Neumann boundedness, so $\mackeyMod(\xi)\subseteq \xi$ for all vector space convergences $\xi$.

The function $\mackeyMod$ is idempotent by \ref{MackeyModPreservesVNBoudnedSets}.
\end{proof}

\begin{proposition}
Let $\sSet{V,\xi}, \sSet{W,\zeta}$ be vector convergence spaces and $f:V\to W$ a linear function. Then the following are equivalent:
\begin{enumerate}
\item $f: \sSet{V,\vNborn{\xi}}\to \sSet{W,\vNborn{\zeta}}$ is bounded;
\item $f: \sSet{V,\mackeyMod(\xi)}\to \sSet{W,\mackeyMod(\zeta)}$ is continuous;
\item $f: \sSet{V,\mackeyMod(\xi)}\to \sSet{W,\zeta}$ is continuous.
\end{enumerate}
\end{proposition}
\begin{proof}
$(1) \Rightarrow (2)$ Let $F\overset{\mackeyMod(\xi)}{\longrightarrow} 0$. Then there exists $B\in \vNborn{\xi}$ such that $\neighbourhood_\F(0)\cdot \{B\} \subseteq F$. We have 
\[ \upset f^{\imf\imf}(F) \supseteq \upset f^{\imf\imf}\big(\neighbourhood_\F(0)\cdot \{B\}\big) = \neighbourhood_\F(0)\cdot \{f^\imf(B)\}. \]
As $f^\imf(B)$ is bounded, $f^{\imf\imf}(F)$ converges in $\mackeyMod(\zeta)$.

$(2) \Rightarrow (3)$ Immediate because $\mackeyMod(\zeta)\subseteq \zeta$.

$(3) \Rightarrow (1)$ Take arbitrary $B\in \vNborn{\xi}$, so $\neighbourhood_\F(0)\cdot \{B\}\overset{\mackeyMod(\xi)}{\longrightarrow} 0$. Then
\[ \neighbourhood_\F(0)\cdot \{f^\imf(B)\} = \upset f^{\imf\imf}\big(\neighbourhood_\F(0)\cdot \{B\}\big) \overset{\zeta}{\longrightarrow} 0, \]
so $f^\imf(B)\in\vNborn{\zeta}$.
\end{proof}

\subsubsection{Mackey spaces}
\begin{proposition}
Let $\sSet{V,\xi}$ be a vector convergence space. Then $V$ is a Mackey space \textup{if and only if} it is locally bounded and equable.
\end{proposition}
\begin{proof}
A Mackey space is clearly equable. It is locally bounded by \ref{MackeyModLocallyBounded}.

Now let $\sSet{V,\xi}$ be a locally bounded and equable vector convergence space. Since $\mackeyMod(\xi)\leq \xi$, it is enough to prove the converse inequality. Assume $F\overset{\xi}{\longrightarrow} 0$. Then we can find an equable $G\subseteq F$ that also converges to $0$. Choose a bounded set $B\in G$, which means that $\upset \{B\}\subseteq G$. Then $\neighbourhood_\F(0)\cdot \{B\} \subseteq \neighbourhood_\F(0)\cdot G = G \subseteq F$, so $F$ converges to $0$ in $\mackeyMod(\xi)$.
\end{proof}


\section{Banach-Steinhaus pairs}
\begin{definition}
Let $\sSet{V,\xi}, \sSet{W,\zeta}$ be convergence vector spaces. The pair $(V,W)$ is called a \udef{Banach-Steinhaus pair} if every von Neumann bounded subset of $\contLin_p(V,W)$ is equicontinuous.
\end{definition}
TODO is von Neumann the right bornology?

\begin{lemma} \label{BanachSteinhausPairDomainInclusion}
Let $\sSet{V,\xi}, \sSet{W,\zeta}$ be convergence vector spaces and $\xi'$ a vector space convergence on $V$ such that $\xi'\leq \xi$. If $\big(\sSet{V,\xi}, \sSet{W,\zeta}\big)$ is a Banach-Steinhaus pair, then so is $\big(\sSet{V,\xi'}, \sSet{W,\zeta}\big)$.
\end{lemma}
\begin{proof}
The pointwise convergence on $\contLin_p(V,W)$ does not depend on the convergence on $V$. Thus the bounded subsets of $\contLin_p(V,W)$ are the same for both convergences. The larger the convergence on $V$, the fewer equicontinuous sets, which implies the lemma.
\end{proof}

\begin{proposition}
Let $\sSet{V,\xi}, \sSet{W,\zeta}$ be convergence vector spaces. Then $\big(\sSet{V,\xi}, \sSet{W,\zeta}\big)$ is a Banach-Steinhaus pair \textup{if and only if} $\big(\sSet{V,\lconvMod(\xi)}, \sSet{W,\zeta}\big)$ is a Banach-Steinhaus pair.
\end{proposition}
\begin{proof}
The pointwise convergence on $\contLin_p(V,W)$ does not depend on the convergence on $V$, so $\contLin_p(V,W)$ and $\contLin_p\big(\lconvMod(V),W\big)$ have the same bouded sets. They also have the same equicontinuous sets by \ref{equicontinuousSetsLconvMod}.
\end{proof}

\subsection{Barrelled spaces}
\begin{definition}
Let $\sSet{V,\xi}$ be a convergence vector space. Then $V$ is called \udef{barrelled} if every bounded subset of $\dual{V}_p$ is equicontinuous.
\end{definition}

\begin{theorem}
BB 7.2.17
\end{theorem}
\begin{corollary}[Banach-Steinhaus]
Let $\sSet{V,\xi}$ be a barrelled convergence vector space and $\sSet{W,\zeta}$ an LCTVS. Then $(V,W)$ is a Banach-Steinhaus pair.
\end{corollary}
\begin{corollary} \label{BanachNormedBanachSteinhauspair}
Let $V$ be a Banach space and $W$ a normed space. Then $(V,W)$ is a Banach-Steinhaus pair.
\end{corollary}

\subsection{Quasi-completeness}
\begin{definition}
Let $\sSet{V,\xi}$ be a convergence vector space. Then $V$ is called \udef{quasi-complete} if every Cauchy filter that contains a vN bounded set is convergent.
\end{definition}

\begin{proposition} \label{quasiCompleteImpliesSequentiallyComplete}
Let $\sSet{V,\xi}$ be a quasi-complete topological convergence vector space. Then $V$ is sequentially complete.
\end{proposition}
\begin{proof}
Let $\seq{v_n}\in V^\N$ be a Cauchy sequence. Then $\{v_n\}_{n\in\N}$ is totally bounded by \ref{imageCauchySequenceTotallyBounded} and thus vN bounded by \ref{boundednessTVS}. Finally $\seq{v_n}$ is convergent by quasi-completeness.
\end{proof}

\begin{proposition} \label{quasiCompletenessFunctionSpaces}
Let $\sSet{V,\xi}, \sSet{W,\zeta}$ be convergence vector spaces. Such that
\begin{itemize}
\item $(V,W)$ is a Banach-Steinhaus pair;
\item $W$ is quasi-complete, regular and Hausdorff.
\end{itemize}
Then $\contLin_p(V,W)$ is quasi-complete.
\end{proposition}
\begin{proof}
Let $F\in \powerfilters\big(\contLin_p(V,W)\big)$ be a Cauchy filter that contains a vN bounded set. For all $v\in V$, $\evalMap_{v}: \contLin_p(V,W) \to W$ is uniformly continuous, by \ref{uniformContinuityGroupHomomorphism}. Thus $\upset \evalMap_v^{\imf\imf}(F)$ is Cauchy and contains a vN bounded set by \ref{continuousMappingBoundedSets}. Since $W$ was assumed quasi-complete, $\upset \evalMap_v^{\imf\imf}(F)$ converges to some $y_x\in W$.

Consider the function $T: x\mapsto y_x$. Then $F\to T$ in $(V\to W)_p$. Since $(V,W)$ is a Banach-Steinhaus pair, $F$ contains an equicontinuous set. This set is evenly continuous by \ref{equicontinuityEvenContinuity} and thus $F\to T$ in $(V\to W)_c$. Then $F\in \contLin(V,W)$ by \ref{linearFunctionsClosedSubset}.
\end{proof}

\chapter{General duality theory}
\section{Paired spaces}
\begin{definition}
A \udef{pairing} is a triple $\sSet{V,W, \pair{\cdot,\cdot}}$ where $V,W$ are vector spaces over $\mathbb{F}$ and $\pair{\cdot,\cdot}: V\times W\to \mathbb{F}$ is a bilinear form. Often we will write the pairing as just $\sSet{V,W}$.

We say $W$
\begin{itemize}
\item \udef{distinguishes} points of $V$; or
\item is \udef{separating} on $V$; or
\item \udef{separates} $V$
\end{itemize}
if 
\[ \forall v\in V\setminus\{0\}: \exists w\in W: \pair{v,w} \neq 0. \]

A \udef{dual system}, \udef{dual pair} or \udef{duality} over a field $\mathbb{F}$ is a pairing $\sSet{V,W,\pair{\cdot,\cdot}}$ such that $V$ distinguishes points of $W$ and $W$ distinguishes points of $V$.
\end{definition}

\begin{lemma} \label{dualSystemInjective}
Let $\sSet{V,W, \pair{\cdot,\cdot}}$ be a pairing. Then $W$ separates $V$ \textup{if and only if} $v\mapsto \pair{v, \cdot}$ is injective.
\end{lemma}
\begin{proof}
Suppose $W$ separates $V$ and take arbitrary $v,v'\in V$ such that $\pair{v, \cdot} = \pair{v', \cdot}$. Then $\pair{v-v',\cdot} = \underline{0}$, which by assumption means $v-v' = 0$, or $v= v'$.

Now suppose $v\mapsto \pair{v, \cdot}$ is injective and take arbitrary $v\in V\setminus\{0\}$. If $\pair{v,w} = 0$ for all $w\in W$, then $\pair{v,\cdot} = \pair{0,\cdot}$ and thus $v=0$ by injectivity. This was disallowed by assumption.
\end{proof}

\begin{lemma}
Let $\sSet{V,W,\pair{\cdot,\cdot}}$ be a dual pair. Then $\sSet{W,V,\pair{\cdot,\cdot}^d}$ is also a dual pair.
\end{lemma}

\begin{example}
Let $\sSet{V, \xi}$ be a convergence vector space. Then $\sSet{\dual{V}, V, \evalMap(\cdot,\cdot)}$ is a pairing.
\end{example}

\begin{proposition} \label{HausdorffLCTVSdualSystem}
Let $\sSet{V,\xi}$ be a Hausdorff locally convex convergence vector space. Then $\sSet{\dual{V}, V, \evalMap(\cdot,\cdot)}$ is a dual pair.
\end{proposition}
\begin{proof}
It is immediate that $V$ separates $\dual{V}$. We have that $\dual{V}$ separates $V$ by \ref{locallyConvexDualPair}
\end{proof}

\section{The weak topology}
\begin{definition}
Let $(X,Y,\pair{\cdot,\cdot})$ be paired vector spaces. 
The initial convergence on $Y$ w.r.t. the set of linear functionals $\setbuilder{\pair{x,\cdot}}{x\in X}$ is called the \udef{weak topology} $\sigma(X,Y)$ on $Y$ for the pair $\sSet{X,Y}$\footnote{What we denote $\sigma(X,Y)$ is usually denoted $\sigma(Y,X)$.}.

The \udef{weak-$*$ topology} $\sigma^*(X,Y)$ on $X$ is the weak topology $\sigma(Y,X)$ (i.e. the weak topology for the pairing $(Y,X,\pair{\cdot,\cdot}^d)$).
\end{definition}

\begin{lemma} \label{weakTopologyLCTVS}
Let $\sSet{X,Y,\pair{\cdot,\cdot}}$ be a pairing. The weak topology $\sigma(X,Y)$ on $Y$ 
\begin{enumerate}
\item is the same as the initial convergence w.r.t. the set of seminorms $\setbuilder{\abspair{x,\cdot}}{x\in X}$;
\item is a locally convex vector space topology;
\item is Hausdorff \textup{if and only if} $X$ separates $Y$;
\item has the neighbourhood filter
\[ \neighbourhood_{\sigma(X,Y)}(0) = \mathfrak{F}\setbuilder{y\in Y}{\exists x\in X: \abspair{x,y} \leq 1}. \]
\end{enumerate}
\end{lemma}
\begin{proof}
(1, 2) Let $\sigma_1$ be the initial convergence w.r.t. $\setbuilder{\pair{x,\cdot}}{x\in X}$ and $\sigma_2$ the initial convergence w.r.t. $\setbuilder{\abspair{x,\cdot}}{x\in X}$.

By \ref{continuityAbsFunctional}, all functionals of the form $\abspair{x,\cdot}: \sSet{V,\sigma_1} \to \R$, for some $x\in X$, are continuous. Thus $\sigma_1 \subseteq \sigma_2$.

By \ref{locallyConvexSeminormTopology}, $\sigma_2$ is a locally convex vector space topology. Thus all functionals of the form $\pair{x,\cdot}: \sSet{V,\sigma_2} \to \R$, for some $x\in X$, are continuous. This means that $\sigma_2 \subseteq \sigma_1$.

(3) If $X$ separates $Y$, then the weak topology is Hausdroff by \ref{T2initialConvergence}. If $X$ does not separate $Y$, then there exists $y\in Y\setminus\{0\}$ such that $\pair{x,y} = 0$ for all $x\in X$. Then $\pair{x,-}^{\imf\imf}(\pfilter{y}) = \pfilter{0} \to 0 = \pair{x,0}$ for all $x\in X$ and so $\pfilter{y}$ converges weakly to both $y$ and $0$. Since these are different point, the convergence is not Hausdorff.

(4) The form of the neighbourhood filter also follows from \ref{locallyConvexSeminormTopology}. It is enough to show that $\setbuilder{\abspair{x,\cdot}^{\preimf}(\,[0,\epsilon]\,)}{x\in X, \epsilon > 0} = \setbuilder{y\in Y}{\exists x\in X: \abspair{x,y} \leq 1}$. Then
\begin{align*}
y \in \setbuilder{\abspair{x,\cdot}^{\preimf}(\,[0,\epsilon]\,)}{x\in X, \epsilon > 0} &\iff \exists x\in X: \exists \epsilon > 0: \; \abspair{x,y} \leq \epsilon \\
&\iff \exists x\in X: \exists \epsilon > 0: \; \epsilon^{-1}\abspair{x,y} \leq 1 \\
&\iff \exists x\in X: \exists \epsilon > 0: \; \abspair{\epsilon^{-1}x,y} \leq 1 \\
&\iff \exists x\in X: \; \abspair{x,y} \leq 1 \\
&\iff y\in \setbuilder{y\in Y}{\exists x\in X: \abspair{x,y} \leq 1}.
\end{align*}
\end{proof}

\begin{lemma}
Let $(X,Y,\pair{\cdot,\cdot})$ be a pairing and $x\in X$. Then $\abspair{x,\cdot} = p_{^\pol\{x\}}$.
\end{lemma}

\begin{proposition} \label{functionalContinuityWeakTopology}
Let $\sSet{X,Y,\pair{\cdot,\cdot}}$ be a pairing and $f: Y\to \F$ a linear functional. Then
\[ \dual{(Y,\sigma(X,Y))} = \setbuilder{\pair{x,\cdot}}{x\in X}. \]
\end{proposition}
\begin{proof}
This inclusion $\supseteq$ is immediate by the definition of initial topology.

Now take $f\in \dual{(Y,\sigma(X,Y))}$. By \ref{dualSeminormedConvergence}, there exists a finite set $\{x_1, \ldots, x_n\}$ such that $|f(v)| \leq C\max_{i}\abspair{x_i, \cdot}$ for all $v\in V$. By \ref{linearDependenceLinearFunctionals}, this means that $f$ is some linear combination of the $\pair{x_i,\cdot}$, say $\sum_{i=0}^n \alpha_i \pair{x_i, \cdot}$. So
\[ f = \sum_{i=0}^n \alpha_i \pair{x_i, \cdot} = \pair{\sum_{i=0}^n \alpha_i x_i, \cdot}. \]
\end{proof}
\begin{corollary} \label{dualSystemBijection}
Let $\sSet{X,Y,\pair{\cdot,\cdot}}$ be a pairing such that $Y$ separates $X$. Then
\begin{enumerate}
\item $X\to \sSet{Y, \sigma}^*: x\mapsto \pair{x,\cdot}$ is a bijection that preserves the pairing;
\item it is a homeomorphism if $X$ has the weak-$*$ topology and $(Y,\sigma)^*$ the pointwise topology.
\end{enumerate} 
\end{corollary}
In particular this holds if $\sSet{X,Y,\pair{\cdot,\cdot}}$ is a dual system. We then have
\[ \sSet{X,\sigma^*(X,Y)} \cong \dual{\sSet{Y,\sigma(X,Y)}}. \]
\begin{proof}
(1) Injectivity is given by \ref{dualSystemInjective}. Surjectivity by the proposition.

(2) Take arbitrary $y\in Y$. We have, for all $y\in Y$,
\[ \pair{\cdot, y}(x) = \pair{x,y} = \evalMap_y\big(\pair{x, \cdot}\big), \]
so $\pair{\cdot, y} = \evalMap_y \circ (x\mapsto \pair{x,\cdot})$ and $\evalMap_y = \pair{\cdot, y} \circ (x\mapsto \pair{x,\cdot})^{-1}$. Thus both $(x\mapsto \pair{x,\cdot})$ and its inverse are continuous by \ref{characteristicPropertyInitialFinalConvergence}.
\end{proof}
\begin{corollary} \label{weakestConvergenceOfTheDualPair}
Let $V$ be a convergence vector space. Then $V^* = \sSet{V, \sigma(V^*, V)}^*$.

The weak topology is the weakest convergence $\tau$ such that $V^* = \sSet{V, \tau}^*$.
\end{corollary}
Thus for any convergence vector space $\sSet{V,\xi}$, the space $\sSet{V, \sigma(V^*, V)}$ is a locally convex TVS with the same continuous functionals.

\subsection{The pairing $\sSet{V^*, V, \evalMap}$}
\subsubsection{Convergences of the dual pair}
\begin{definition}
Let $V$ be a vector space and $V^*$ the continuous dual under some convergence. Any convergence on $V$ such that the continuous dual is still $V^*$ is called a \udef{convergence of the dual pair}.
\end{definition}
Any property that only depends on continuous linear functionals is the same for any convergence of the dual pair.

\begin{proposition}
The weak topology is the weakest convergence of the dual pair.
\end{proposition}
\begin{proof}
Reformulation of \ref{weakestConvergenceOfTheDualPair}.
\end{proof}

\subsubsection{Weak continuity}
\begin{definition}
Let $T: \sSet{V,\xi} \to \sSet{W,\zeta}$ be a linear operator between convergence vector spaces. Then $T$ is called \udef{weakly continuous} if
\[ T: \sSet{V, \sigma(V^*,V)} \to \sSet{W, \sigma(W^*,W)} \]
is continuous.
\end{definition}

\begin{lemma} \label{continuityImpliesWeakContinuity}
If $T: V\to W$ is continuous, then it is weakly continuous.
\end{lemma}
\begin{proof}
By the characteristic property of the initial convergence \ref{characteristicPropertyInitialFinalConvergence}, the weak continuity of $T$ is equivalent to the continuity of $f\circ T$ for all $f\in W^*$. This holds because the composition of continuous functions is continuous, \ref{continuityComposition}.
\end{proof}

\subsubsection{Properties of subsets}

\begin{proposition} \label{weakClosureConvexSubsets}
Let $V$ be a vector space and $A\subseteq V$ a convex subset. Let $\xi$ be a locally convex vector space convergence on $V$. Then $\closure_\xi(A) = \closure_{\sigma(V^*, V)}(A)$.
\end{proposition}
In other words the closure of convex sets is the same for all locally convex convergences of the dual pair.
\begin{proof}
From \ref{principalInherenceAdherenceProperties}, we have $\closure_\xi(A) \subseteq \closure_{\sigma(V^*, V)}(A)$.

Now assume $x\notin \closure_\xi(A)$. Then, by \ref{locallyConvexHahnBanachSeparationClosedSet}, we can find a continuous functional $f\in V^*$ such that $f(x) \notin \overline{f^{\imf}\big(\closure_\xi(A)\big)}$. So not every neighbourhood of $f(x)$ meshes with $f^{\imf}\big(\closure_\xi(A)\big)$; take some neighbourhood $U$ that is disjoint from it. By continuity and \ref{continuityVicinityFilter}, we have that $f^{\preimf}(U)$ is a $\sigma(V^*, V)$-neighbourhood of $x$ that must be disjoint from $\closure_\xi(A)$ (we have used that $\sigma(V^*, V)$ is topological, \ref{weakTopologyLCTVS}). This implies that $A$ is also disjoint from $f^{\preimf}(U)$.

Thus $x\notin \closure_{\sigma(V^*, V)}(A)$ by \ref{interiorClosureMembership}.
\end{proof}

\begin{proposition} \label{weaklyBoundedIffBounded}
Let $\sSet{V,\xi}$ be a LCTVS and $A\subseteq V$. Then $A$ is bounded \textup{if and only if} $A$ is weakly bounded.
\end{proposition}
\begin{proof}
As both $\xi$ and $\sigma(\dual{V}, V)$ are LCTVSs, boundedness is equivalent to von Neumann boundedness, by \ref{boundednessTVS}. Now $\xi$ is the initial convergence w.r.t. all continuous seminorms on $V$ (by \ref{locallyConvexSeminormTopology}) and $\sigma(\dual{V}, V)$ is the initial convergence w.r.t. all continuous linear functionals on $V$.

Now consider \ref{vonNeumannBoundednessInitialSpace}. We need to show that $f^\imf(A)$ is bounded for all continuous linear functionals $f$ iff $p^\imf(A)$ is bounded for all seminorms $p$. The direction $\Leftarrow$ is immediate because $f$ is both continuous and bounded on $A$ iff the seminorm $|f|$ is (see \ref{continuityAbsFunctional}).

Now suppose $f^\imf(A)$ is bounded for all continuous linear functionals $f$. 

TODO use uniform boundedness, Voigt p 45.
\end{proof}


\subsection{Weak boundedness and completeness}
TODO: emphasise that weakly bounded = weakly von Neumann bounded.


\begin{proposition} \label{weakBoundedness}
Let $\sSet{X,Y,\pair{\cdot,\cdot}}$ be a pairing such that $Y$ separates $X$. A subset $B\subseteq Y$ is weakly bounded \textup{if and only if} $B$ is weakly totally bounded.
\end{proposition}
\begin{proof}
The direction $\Leftarrow$ is immediate.

For the other direction, take arbitrary $x\in X$. Then $\pair{x,\cdot}: \sSet{Y,\sigma}\to \F$ is a continuous linear functional and thus uniformly continuous by \ref{uniformContinuityGroupHomomorphism}. 

By \ref{imageBoundedSet}, we have that $\pair{x, B}$ is bounded and thus $\overline{\pair{x,B}}$ is also bounded by \ref{adherenceBoundedSet}. As $\F$ has the Heine-Borel property (TODO ref), $\overline{\pair{x,B}}$ is in fact compact.

Now $\prod_{x\in X}\overline{\pair{x, B}}$ is compact by Tychonoff's theorem \ref{TychonoffsTheorem}, and in particular totally bounded by \ref{precompactTotallyBounded} and \ref{compactPrecompactComplete}.

By \ref{dualSystemBijection} and \ref{pointwiseConvergenceProductSpace}, we have
\[ Y \cong \sSet{X,\sigma^*}^* \hookrightarrow (X\to \F)_p \cong \prod_{x\in X}\F, \]
so we have an embedding $f: Y \hookrightarrow \prod_{x\in X}\F$, with respect to which $Y$ has the initial topology (TODO ref). We restrict $f$ to its range, such that it is a bijection and thus a homeomorphism by \ref{initialBijectionHomeomorphism}. As the function $f$ is linear, it is uniformly continuous by \ref{uniformContinuityGroupHomomorphism}.
We have
\[ B = (f^{-1})^\imf\circ f^\imf(B) \subseteq (f^{-1})^\imf\Big(\prod_{x\in X}\overline{\pair{x, B}}\Big), \]
and thus $B$ is totally bounded by \ref{imageBoundedSet} and \ref{boundedSetsIdeal}.
\end{proof}

\section{Polars}
\subsection{Polar sets}
\begin{definition}
Let $\sSet{X,Y,\pair{\cdot,\cdot}}$ be a pairing and $B\subseteq Y$ a subset. The \udef{polar} of $B$ is the polar w.r.t. the relation $\pol$ on $(Y,X)$ defined by
\[ y\pol x \qquad\iff\qquad \abspair{x, y} \leq 1. \]
Conventionally, we also use $\pol$ to denote $\pol^\transp$. Thus the \udef{bipolar} $B^{\pol\pol}$ of $B$ is $B^{\pol\pol^\transp}$.
\end{definition}

\begin{lemma} \label{polarLemma}
Let $\sSet{X,Y,\pair{\cdot,\cdot}}$ be a pairing and $B\subseteq Y$ a subset. Then
\begin{align*}
B^{\pol} &= \setbuilder{x\in X}{\sup_{y\in B}\abspair{x,y} \leq 1} \\
&= \bigcap_{y\in B}\setbuilder{x\in X}{\abspair{x,y}\leq 1}.
\end{align*}
and for the gauge of $B^\pol$, we have $p_{B^\pol}(x) = \sup_{y\in B}\abspair{x,y}$.
\end{lemma}
\begin{proof}
The expression for the polar is straightforward. For the gauge we calculate
\begin{align*}
p_{B^\pol}(x) &= \inf \setbuilder{\lambda\in\overline{\R^+}}{x\in \lambda B^\pol} \\
&= \inf \setbuilder{\lambda\in\overline{\R^+}}{\lambda^{-1}x\in B^\pol} \\
&= \inf \setbuilder{\lambda\in\overline{\R^+}}{\sup_{y\in B}\abspair{\lambda^{-1}x,y}\leq 1} \\
&= \inf \setbuilder{\lambda\in\overline{\R^+}}{\sup_{y\in B}\abspair{x,y}\leq \lambda} \\
&= \sup_{y\in B}\abspair{x,y}.
\end{align*}
\end{proof}

\begin{lemma} \label{polarPropertiesLemma}
Let $\sSet{X,Y,\pair{\cdot,\cdot}}$ be a pairing and $B\subseteq Y$ a subset. Then
\begin{enumerate}
\item for all $\lambda \neq 0$: $(\lambda B)^{\pol} = \lambda^{-1}B^{\pol}$;
\item $B^{\pol}$ is absolutely convex;
\item $B^{\pol}$ is $\sigma^*(X,Y)$-closed.
\end{enumerate}
\end{lemma}
\begin{proof}
(1) We calculate
\begin{align*}
x\in (\lambda B)^{\pol} &\iff \forall y\in B:\; \abspair{x,\lambda y} \leq 1 \\
&\iff \forall y\in B:\; \abspair{\lambda x,y} \leq 1 \\
&\iff \lambda x\in B^\pol \\
&\iff x\in \lambda^{-1}B^\pol.
\end{align*}

(2) Take $x,y\in B^\pol$ and $r,r'\in \R$ such that $|r| + |r'|\leq 1$. We need to show that $rx + r'y\in B^\pol$. Indeed, for arbitrary $z\in B$ we have
\begin{align*}
\abspair{rx + r'y, z} &= |r\pair{x,z} + r'\pair{y,z}| \\
&\leq |r|\;\abspair{x,z} + |r'|\;\abspair{y,z} \\
&\leq |r| + |r'| \\
&\leq 1,
\end{align*}
where the last inequality follows from $1 = |1 - r + r| \leq |1-r| + |r|$. Since $z\in B$ was chosen arbitrarily, we have $rx + (1-r)y\in B^\pol$.

(3) We have, using \ref{polarOfUnion},
\begin{align*}
B^\pol &= \left(\bigcup_{y\in B}\{y\}\right)^\pol \\
&= \bigcap_{y\in B} \{y\}^\pol \\
&= \bigcap_{y\in B}\setbuilder{x\in X}{\abspair{x,y}\leq 1} \\
&= \bigcap{y\in B}\abspair{\cdot, y}^{\preimf}[\cball(0,1)],
\end{align*}
which is an intersection of closed sets (as each $\abspair{\cdot, y}$ is continuous in the $\sigma(X,Y)$ topology) and thus closed.
\end{proof}

\begin{proposition}[Bipolar theorem] \label{bipolarTheorem}
Let $\sSet{X,Y,\pair{\cdot,\cdot}}$ be a pairing and $B\subseteq Y$ a subset. Then
\[ B^{\pol\pol} = \overline{\disked(B)}^{\sigma(X,Y)}. \]
\end{proposition}
TODO: clean up proof.
\begin{proof}
We have $B\subseteq B^{\pol\pol}$ by \ref{reflexiveGaloisCorollary}. Then $\disked(B) \subseteq B^{\pol\pol}$ because $B^{\pol\pol}$ is absolutely convex by \ref{polarPropertiesLemma}. Similarly $\overline{\disked(B)}^{\sigma(X,Y)} \subseteq B^{\pol\pol}$, because $B^{\pol\pol}$ is $\sigma(X,Y)$-closed.

The other inclusion is proved by contradiction. Assume, to this end, that $y_0\in B^{\pol\pol} \setminus \overline{\disked(B)}^{\sigma(X,Y)}$. Since $\overline{\disked(B)}^{\sigma(X,Y)}$ is closed and convex by \ref{inherenceAdherenceConvex}, we can apply Hahn-Banach separation \ref{locallyConvexHahnBanachSeparationClosedSet} to obtain a continuous functional $f$ such that $f^\imf\left[\overline{\disked(B)}\right]$ is disjoint from some open neighbourhood $U$ of $f(y_0)$.

Now $\overline{\disked(B)}$ is absolutely convex by \ref{inherenceAdherenceBalanced} and \ref{inherenceAdherenceConvex}, so $f^\imf\left[\overline{\disked(B)}\right]$ is also absolutely convex by linearity. Then, because $|U|$ is open, there exists $t\in |U|$ such that $|f(b)| < t < |f(y_0)|$ for all $b\in \overline{\disked(B)}$. Thus $\sup_{b\in \overline{\disked(B)}}|f(b)| < |f(y_0)|$. By rescaling $f$ we can take $\sup_{b\in \overline{\disked(B)}}|f(b)| \leq 1 < |f(y_0)|$.

By \ref{functionalContinuityWeakTopology}, we can find some $x\in X$ such that $f = \pair{x,\cdot}$. Then $\sup_{b\in \overline{\disked(B)}}\abspair{x, b} \leq 1$ implies $x\in \overline{\disked(B)}^\pol \subseteq B^\pol$. Finally $1 < \abspair{x, y_0}$ implies $y_0\notin B^{\pol\pol}$. This is a contradiction.
\end{proof}

\subsection{The pairing $\sSet{\Lin(V,\F), V, \evalMap}$}
\begin{proposition} \label{dualFromPolars}
Let $\sSet{V,\xi}$ be a topological vector space. Then
\[ \dual{\sSet{V, \xi}} = \bigcup_{U\in \neighbourhood_\xi(0)}U^\pol, \]
where the polars are taken in $\sSet{\Lin(V, \F), V, \evalMap}$.
\end{proposition}
\begin{proof}
First take $f\in \dual{\sSet{V, \xi}}$. Then
\[ \{f\}^\pol = \setbuilder{v\in V}{\abspair{f,x} = |f(x)| \leq 1} = f^\preimf\big(\cball_\F(0,1)\big), \]
which is equal to some $U\in \neighbourhood_\xi(0)$ by continuity of $f$ and \ref{continuityVicinityFilter}. By \ref{polarsGaloisConnection} this implies $\{f\} \subseteq U^\pol$ and thus
\[ \dual{\sSet{V, \xi}} = \bigcup \setbuilder[\big]{\{f\}}{f\in \dual{\sSet{V, \xi}}} \subseteq \bigcup_{U\in \neighbourhood_\xi(0)}U^\pol. \]
For the converse, take some $U\in \neighbourhood_\xi(0)$ and $f\in U^\pol$. Then $U\subseteq \{f\}^\pol = f^\preimf\big(\cball_\F(0,1)\big)$, \ref{polarsGaloisConnection} and the calculation above, so $f^\imf(U) \subseteq \cball_\F(0,1)$ by \ref{functionImagePreimageGaloisConnection}. This implies that $f$ is continuous by \ref{boundedOnVicinityImpliesContinuous}, so $f\in \dual{\sSet{V, \xi}}$.
\end{proof}

\subsection{Polar topologies}

\begin{lemma} \label{weaklyBoundedLemma}
Let $\sSet{X,Y,\pair{\cdot,\cdot}}$ be a pairing and $A\subseteq Y$ a subset. The following are equivalent:
\begin{enumerate}
\item $A$ is weakly bounded;
\item $\sup_{y\in A}\abspair{x, y}$ is finite for all $x\in X$;
\item $A^\pol$ is an absorbent subset of $X$.
\end{enumerate}
\end{lemma}
\begin{proof}
$(1) \Leftrightarrow (2)$ Immediate from \ref{vonNeumannBoundednessInitialSpace} and \ref{weakTopologyLCTVS}, noting that
\[ \sup_{y\in A}\abspair{x, y} = \sup\Big(\abspair{x,-}^{\imf}(A)\Big) \]
and that a subset of $\R$ is bounded iff its supremum is finite.

$(2) \Leftrightarrow (3)$ By \ref{polarLemma}, point (2) is equivalent to the finiteness of $p_{A^\pol}(x)$ for all $x\in X$. By \ref{gaugeWellDefined} (and the fact that $A^\pol$ is balanced, \ref{polarPropertiesLemma}) we have that this is equivalent to the absorbence of $A^\pol$.
\end{proof}

\begin{proposition}
Let $\sSet{X,Y,\pair{\cdot,\cdot}}$ be a pairing and $\mathcal{A}\subseteq\powerset(Y)$ a set of weakly bounded sets. Then the filter
\[ N = \mathfrak{F}\setbuilder{\epsilon A^\pol}{\epsilon>0, A\in \mathcal{A}} \]
is the neighbourhood filter of $0$ in a locally convex vector space topology on $X$.
\end{proposition}
\begin{proof}
We use \ref{TVSbase} to verify that this filter is the neighbourhood filter of a topological vector space. Indeed, the sets $\epsilon A^\pol$ are absorbent by \ref{vonNeumannBoundedSetLemma} and \ref{weaklyBoundedLemma}. They are balanced by \ref{polarPropertiesLemma}. Finaly by convexity (\ref{polarPropertiesLemma}), we have $\frac{1}{2}\epsilon A^\pol + \frac{1}{2} \epsilon A^\pol \subseteq \epsilon A^\pol$.

As noted, the basis sets are convex, so the topology is locally convex.
\end{proof}
\begin{corollary}
Let $\sSet{X,Y,\pair{\cdot,\cdot}}$ be a pairing and $\mathcal{A}\subseteq\powerset(Y)$ a set of weakly bounded sets.
If $\mathcal{A}$ is upwards directed and closed under scalar multiplication, then $\upset\{A^\pol\}_{A\in \mathcal{A}}$ is the neighbourhood filter of $0$ in a locally convex vector space topology on $X$.
\end{corollary}
\begin{proof}
For all $A\in \mathcal{A}$ and $\epsilon>0$, we have $\epsilon A^\pol = (\epsilon^{-1}A)^\pol$. Thus $\mathfrak{F}\setbuilder{\epsilon A^\pol}{\epsilon>0, A\in \mathcal{A}} = \mathfrak{F}\setbuilder{A^\pol}{A\in \mathcal{A}}$.

We just need to show that $\upset\setbuilder{A^\pol}{A\in \mathcal{A}}$ is closed under intersections. Take $A,B\in \mathcal{A}$. By the upwards direction, we have $C\in \mathcal{A}$ such that $A\cup B \subseteq C$.
Then, using \ref{polarOfUnion}, $A^\pol\cap B^\pol = (A\cup B)^\pol \supseteq C^\pol$.
\end{proof}

\begin{definition}
Let $\sSet{X,Y,\pair{\cdot,\cdot}}$ be a pairing and $\mathcal{A}\subseteq\powerset(Y)$ a set of weakly bounded sets. The topology on $X$ with neighbourhood filter
\[ \neighbourhood(0) \defeq \mathfrak{F}\setbuilder{\lambda A^\pol}{\lambda\in \F, A\in \mathcal{A}} \]
is called the \udef{polar topology} determined by $\mathcal{A}$. We denote it $\polarTop{\mathcal{A}}$.
\end{definition}

\begin{lemma} \label{setsDeterminingTheSamePolarTopologies}
Let $\sSet{X,Y,\pair{\cdot,\cdot}}$ be a pairing and $\mathcal{A}\subseteq\powerset(Y)$ a set of weakly bounded sets. Then $\polarTop{\mathcal{A}}$ is equal to the polar topology determined by
\begin{enumerate}
\item $\mathfrak{I}(\mathcal{A})$;
\item $(\closure_{\sigma(X,Y)}\circ \disked)^{\imf}(\mathcal{A})$.
\end{enumerate}
\end{lemma}
\begin{proof}
(1) If $B\subseteq A\in \mathcal{A}$, then $B^\pol \supseteq A^\pol$, which means that $B^\pol$ is a neighbourhood of $0$ in the polar topology.

If $A,B\in \mathcal{A}$, then $A^\pol\cap B^\pol = (A\cup B)^\pol$ (by \ref{polarOfUnion}) is a neighbourhood (because the neighbourhoods form a filter).

(2) We have $A^\pol = A^{\pol\pol\pol} = \Big(\overline{\disked(B)}^{\sigma(X,Y)}\Big)^\pol$ by the bipolar theorem \ref{bipolarTheorem}.
\end{proof}

\begin{lemma} \label{polarTopologyOrdering}
Let $\sSet{X,Y,\pair{\cdot,\cdot}}$ be a pairing and $\mathcal{A}, \mathcal{B}\subseteq\powerset(Y)$ sets of weakly bounded sets. Then $\polarTop{\mathcal{A}} \subseteq \polarTop{\mathcal{B}}$ \textup{if and only if} $\mathfrak{I}\big((\closure_{\sigma(X,Y)}\circ \disked)^{\imf}(\mathcal{A})\big) \supseteq \mathfrak{I}\big((\closure_{\sigma(X,Y)}\circ \disked)^{\imf}(\mathcal{B})\big)$.
\end{lemma}
\begin{proof}
TODO
\end{proof}

\begin{proposition}
Let $\sSet{X,Y,\pair{\cdot,\cdot}}$ be a dual system and  $\mathcal{A}\subseteq\powerset(Y)$ a set of weakly bounded sets. Then the polar topology $\polarTop{\mathcal{A}}$ is Hausdorff \textup{if and only if} $\Span\Big(\bigcup \mathcal{A}\Big)$ is $\sigma(X,Y)$-dense in $Y$.
\end{proposition}
\begin{proof}
TODO! 8.5.1 Narici / Beckenstein
\end{proof}

\begin{proposition} \label{weak*topologyPolarTopology}
Let $\sSet{X,Y,\pair{\cdot,\cdot}}$ be a pairing. The polar topology determined by the singletons in $Y$ is $\sigma^*(X,Y)$. Thus
\[ \neighbourhood_{\sigma^*(X,Y)}(0) = \mathfrak{F}\setbuilder{\{y\}^\pol}{y\in Y} = \upset\setbuilder{F^\pol}{\text{$F\subseteq Y$ finite}} \]
\end{proposition}
\begin{proof}
Comparing with \ref{weakTopologyLCTVS}, we have, for all $y\in Y$, that $\setbuilder{x\in X}{\abspair{x,y}\leq 1} = \{y\}^\pol$ by \ref{polarLemma}.
\end{proof}


\begin{lemma} \label{dualPolarTopology}
Let $\sSet{X,Y,\pair{\cdot,\cdot}}$ be a pairing and $\mathcal{A}$ a set of weakly bounded subsets of $Y$. Then
\[ \dual{\sSet{X, \polarTop{\mathcal{A}}}} = \bigcup_{A\in \mathcal{A}, \epsilon>0}\epsilon A^{\pol\pol'}, \]
where $\pol'$ is the polar relation in the pair $\sSet{\Lin(X, \F), X, \evalMap}$.
\end{lemma}
\begin{proof}
From \ref{dualFromPolars} we have
\begin{align*}
\dual{\sSet{X, \polarTop{\mathcal{A}}}} &= \bigcup_{U\in \neighbourhood_{\polarTop{\mathcal{A}}}(0)}U^{\pol'} \\
&= \bigcup \big((-)^{\pol'}\big)^\imf\Big(\mathfrak{F}\setbuilder{(\epsilon A)^\pol}{\epsilon>0, A\in \mathcal{A}}\Big) \\
&= \bigcup \downset\big((-)^{\pol'}\big)^\imf\Big(\mathfrak{F}\setbuilder{(\epsilon A)^\pol}{\epsilon, A\in \mathcal{A}}\Big) \\
&= \bigcup \mathfrak{I}\setbuilder{(\epsilon A)^{\pol\pol'}}{\epsilon>0, A\in \mathcal{A}} \\
&= \bigcup \setbuilder{(\epsilon A)^{\pol\pol'}}{\epsilon>0, A\in \mathcal{A}} \\
&= \bigcup_{A\in \mathcal{A}, \epsilon>0}\epsilon A^{\pol\pol'}.
\end{align*}
\end{proof}

\subsubsection{Strong topology}
\begin{definition}
Let $\sSet{X,Y,\pair{\cdot,\cdot}}$ be a pairing. The polar topology on $X$ determined by the set of all weakly bounded subsets of $Y$ is called the \udef{strong topology} and is denoted $\beta(X,Y)$.
\end{definition}
The set of all weakly bounded subsets is upwards directed and closed under scalar multiplication by \ref{vonNeumannBoundedSetLemma}, so the neighbourhood filter of $0$ in the strong topology on $X$ is given by
\[ \upset\setbuilder{A^\pol}{\text{$A$ is weakly bounded}}. \]

\subsubsection{The Mackey topology}
\begin{definition}
Let $\sSet{X,Y,\pair{\cdot,\cdot}}$ be a pairing. The polar topology determined by weak-compact and absolutely convex subsets of $Y$ is called the \udef{Mackey topology} and is denoted $\tau(X, Y)$.
\end{definition}

TODO: for Banach space Mackey topology determined by weak-compact sets

\begin{proposition}
Let $\sSet{X,Y,\pair{\cdot,\cdot}}$ be a pairing. Then $\dual{\sSet{X, \tau(X,Y)}} = \dual{\sSet{X, \sigma^*(X,Y)}}$.
\end{proposition}
\begin{proof}
Let $\pol$ be the polar relation in the pairing $\sSet{X,Y,\pair{\cdot,\cdot}}$ and $\pol'$ the polar relation in the pairing $\sSet{\dual{\sSet{X, \tau(X,Y)}}, X, \evalMap}$. Let $K$ be the set of weak-compact and absolutely convex subsets of $Y$. As $K$ is closed under scalar multiplication (by \ref{diskedHullHomogeneous} and \ref{continuityLemmaVectorConvergence}), we have, by \ref{dualPolarTopology}
\[ \dual{\sSet{X, \tau(X,Y)}} = \bigcup_{A\in K}A^{\pol\pol'} \]
TODO
\end{proof}


\subsection{Weak-$*$-compactness}
\begin{theorem}[Banach/Bourbaki-Alaoglu] \label{alaogluTheorem}
Let $(X,Y,\pair{\cdot,\cdot})$ be a pairing such that $Y$ separates $X$ and $A\subseteq Y$ a neighbourhood of the origin in the weak topology. Then $A^\pol$ is $\sigma^*(X,Y)$-compact.
\end{theorem}
\begin{proof}
Since $A$ is absorbent, by \ref{TVSconstruction}, we have that $A\subseteq A^{\pol\pol}$ is also absorbent. Then $A^\pol = A^{\pol\pol\pol}$ is weak-$*$ bounded by \ref{weaklyBoundedLemma} and thus weak-$*$ totally bounded by \ref{weakBoundedness}.

By \ref{dualSystemBijection} we have $\sSet{X, \sigma^*(X,Y)} \cong \dual{\sSet{Y, \sigma(X,Y)}} \subseteq \Lin(Y, \F)_p$. Let $\pol'$ be the relation defined on the pairing $\sSet{\dual{\sSet{Y, \sigma(X,Y)}},Y, \evalMap}$ and $\pol^{\prime\prime}$ the relation defined on the pairing $\sSet{\Lin(Y, \F),Y, \evalMap}$. Clearly $A^{\pol'} = A^{\pol^{\prime\prime}} \cap \dual{\sSet{Y, \sigma(X,Y)}}$, so $A^{\pol'} = A^{\pol^{\prime\prime}}$ by \ref{dualFromPolars}.

Now $\sigma^*(\Lin(Y, \F), Y)$-convergence is exactly pointwise convergence and in this convergence, $A^{\pol^{\prime\prime}}$ is totally bounded by \ref{subspaceBoundedness} and closed by \ref{polarPropertiesLemma}, complete by \ref{algebraicDualComplete} and \ref{closedComplete} and thus finally compact by \ref{compactPrecompactComplete}.
\end{proof}

\subsubsection{The Mackey-Arens theorem}


\ref{HausdorffLCTVSdualSystem}

\begin{theorem}[Mackey-Arens]
Let $\sSet{V,\xi}$ be a Hausdorff convergence vector space and consider the pairing $\sSet{\dual{V}, V, \evalMap}$. Let $\zeta$ be a locally convex topological vector space convergence on $V$. Then the following are equivalent:
\begin{enumerate}
\item $\dual{\sSet{V,\xi}} = \dual{\sSet{V,\zeta}}$;
\item $\zeta$ is a polar topology determined by a cover of absolutely convex $\sigma(\dual{V}, V)$-compact subsets of $\dual{V}$;
\item $\tau(\dual{V}, V) \subseteq \zeta \subseteq \sigma(\dual{V}, V)$.
\end{enumerate}
\end{theorem}
\begin{proof}
$(1) \Rightarrow (2)$ By \ref{lconvModPolarTopology}, $\zeta$ is the polar topology determined by $\setbuilder{U^\pol}{U\in \neighbourhood_\zeta(0)}$, which is a set of absolutely convex $\sigma(\dual{V}, V)$-compact (= $\sigma(\dual{\sSet{V, \xi}}, V)$-compact = $\sigma(\dual{\sSet{V, \zeta}}, V)$-compact) subsets of $\dual{V}$, by \ref{polarPropertiesLemma} and \ref{alaogluTheorem}.

$(2) \Rightarrow (3)$ Clearly $\tau(\dual{V}, V)$ is the strongest such convergence. That $\sigma(\dual{V}, V)$ is such a convergence is given by \ref{weak*topologyPolarTopology}. 

Now take any such convergence $\polarTop{\mathcal{A}}$, where $\mathcal{A}$ covers $\dual{V}$. By \ref{setsDeterminingTheSamePolarTopologies}, $\polarTop{\mathcal{A}}$ is also determined by $\mathfrak{I}(\mathcal{A})$ and $\setbuilder[\big]{\{f\}}{f\in \dual{V}} \subseteq \mathfrak{I}(\mathcal{A})$, so $\polarTop{\mathcal{A}} \subseteq \sigma(\dual{V}, V)$ by \ref{polarTopologyOrdering}.

$(3) \Rightarrow (1)$ It is enough to show $\dual{\sSet{V,\sigma(\dual{V}, V)}} = \dual{\sSet{V,\xi}} = \dual{\sSet{V,\tau(\dual{V}, V)}}$. The first equality is given by \ref{functionalContinuityWeakTopology}. 
\end{proof}


\subsection{The pairing $\sSet{V^*, V, \evalMap}$}
\subsubsection{Equicontinuity and polars}
\begin{proposition} \label{equicontinuityTopologicalFunctionals}
Let $\sSet{V, \xi}$ be a topological convergence space. A subset $H\subseteq \dual{V}$ is equicontinuous \textup{if and only if} there exists $U\in \neighbourhood_\xi(0)$ such that $H\subseteq U^\pol$.
\end{proposition}
The polar is taken in either $\sSet{\Lin(V,\F), V,\evalMap}$ or
$\sSet{\dual{V}, V,\evalMap}$.
\begin{proof}
By \ref{equicontinuityGroupHomomorphisms}, the equicontinuity of $H$ is equivalent to (using pretopologicity, it is enough to consider $\neighbourhood_\xi(0)$)
\begin{align*}
\upset\evalMap^{\imf\imf}\big(\{H\}\otimes \neighbourhood_\xi(0)\big) \overset{\F}{\longrightarrow} 0 &\iff \forall \epsilon >0: \exists U\in \neighbourhood_\xi(0): \; \evalMap^\imf(H\times U) \subseteq \cball(0,\epsilon) \\
&\iff \forall \epsilon >0: \exists U\in \neighbourhood_\xi(0): \forall f\in H: \forall x\in U: \; \abspair{f,x}\leq \epsilon \\
&\iff \forall \epsilon >0: \exists U\in \neighbourhood_\xi(0): \forall f\in H: \forall x\in U: \; \abspair{f,\epsilon^{-1}x}\leq 1 \\
&\iff \forall \epsilon >0: \exists U\in \neighbourhood_\xi(0): \;H\subseteq (\epsilon^{-1}U)^\pol = \epsilon U^\pol.
\end{align*}
In particular, setting $\epsilon =1$, there exists $U\in \neighbourhood_\xi(0)$ such that $H\subseteq U^\pol$. 

Conversely, if $H\subseteq U^\pol$, then for all $\epsilon > 0$, we have $H \subseteq \epsilon (\epsilon U)^\pol$ and $\epsilon U\in \neighbourhood_\xi(0)$ by \ref{vicinityFilterAtOrigin}.
\end{proof}
\begin{corollary}
Let $(X,Y,\pair{\cdot,\cdot})$ be a dual system and $A\subseteq X$. If $A$ is weakly equicontinuous, then $\overline{A}^{\sigma(X,Y)}$ is $\sigma(X,Y)$-compact.
\end{corollary}
\begin{proof}
The equicontinuity of $A$ is equivalent to being a subset of $U^\pol$ for some $U\in \neighbourhood_{\sigma(X,Y)}(0)$ by the proposition. Now $U^\pol$ is compact by \ref{alaogluTheorem}, and so $\overline{A}^{\sigma(X,Y)}$ is compact by \ref{compactClosedSets}.
\end{proof}

\begin{proposition} \label{lconvModPolarTopology}
Let $\sSet{V, \xi}$ be a convergence vector space. Then $\lconvMod(\xi)$ is the polar topology w.r.t. the equicontinuous subsets of $\dual{V}$.
\end{proposition}
Note we are implicitly considering the pairing $\sSet{V,\dual{V},\evalMap^d}$.
\begin{proof}
By \ref{vicinityFilterAtOrigin}, we $\neighbourhood_{\lconvMod(\xi)}(0)$ has a base of closed balanced sets. These sets are also weakly closed by \ref{weakClosureConvexSubsets}. For each $U$ in this base, we have $U = U^{\pol\pol}$ by the bipolar theorem \ref{bipolarTheorem} (applied to the pairing $\sSet{V,\dual{V}, \evalMap}$).

By \ref{equicontinuityTopologicalFunctionals} this base consist of polars of equicontinuous subsets of $\dual{\sSet{V, \lconvMod(\xi)}}$, which, by \ref{equicontinuousSetsLconvMod}, are the same as the equicontinuous subsets of $\dual{V}$.

Conversely, every equicontinuous subset of $\dual{V}$ is contained in the upwards closure of this base, by \ref{equicontinuityTopologicalFunctionals}.
\end{proof}



\subsection{Orthogonal complements}
\begin{definition}
Let $\sSet{X,Y,\pair{\cdot,\cdot}}$ be a pairing and $B\subseteq Y$ a subset. The \udef{orthogonal complement} of $B$ is the polar w.r.t. the relation $\perp$ on $(Y,X)$ defined by
\[ y\perp x \qquad\iff\qquad \pair{x, y} = 0. \]
\end{definition}

\begin{proposition} \label{perpAsPolar}
Let $\sSet{X,Y,\pair{\cdot,\cdot}}$ be a pairing and $B\subseteq Y$ a subset. Then $B^\perp = \Span(B)^\pol$.
\end{proposition}
\begin{proof}
We show both inclusions. First assume $x\in B^\perp$. We need to show that $\abspair{x,y} \leq 1$ for all $y\in \Span(B)$. Indeed we can write $y = \sum_{j=1}^n c_jy_j$ for some $c_j\in\F$ and $y_j\in B$. Then
\[ \abspair{x,y} = \abspair{x, \sum_{j=1}^n c_jy_j} \leq \sum_{j=1}^n |c_j|\;\abspair{x, y_j} = 0 \leq 1, \]
as $y_j\perp x$.

Now assume $x\in \Span(B)^\pol$. If $\abspair{x,y} = 0$ for all $y\in B$, then $x\in B^\perp$ and we are done. Now assume, towards a contradiction, that this is not the case. Then there exists $y\in B$ such that $\abspair{x,y} = \epsilon \neq 1$. Then $\abspair{x,2\epsilon^{-1}y} = 2$, but $2\epsilon^{-1}y\in\Span(B)$, so $\abspair{x,2\epsilon^{-1}y} \leq 1$. This is a contradiction.
\end{proof}
\begin{corollary} \label{corollaryPerpAsPolar}
Let $\sSet{X,Y,\pair{\cdot,\cdot}}$ be a pairing and $B\subseteq Y$ a subset. Then
\begin{enumerate}
\item $B^\perp$ is a subspace;
\item $B^\perp = \Span(B)^\perp = \Span(B^\perp)$;
\item $\{0\}^\perp = X$;
\item $Y$ separates $X$ \textup{if and only if} $Y^\perp = \{0\}$;
\item $B^\perp$ is $\sigma^*(X,Y)$-closed.
\item $B^{\perp\perp} = \overline{\Span(B)}^{\sigma(X,Y)}$;
\item if $\Span(B)$ is $\sigma(X,Y)$-dense in $Y$ and $Y$ separates $X$, then $B^\perp = \{0\}$.
\end{enumerate}
\end{corollary}
\begin{proof}
(1) As $\Span(B)^\pol$ is convex by \ref{polarPropertiesLemma}, we just need to show it is closed under multiplication by $2$, \ref{convexSubspace}. Now $2\cdot \Span(B)^\pol = (2^{-1}\cdot\Span(B))^\pol = \Span(B)^\pol$ by \ref{polarPropertiesLemma}.

(2) We have $B^\perp = \Span(B)^\pol = \Span\big(\Span(B)\big)^\pol = \Span(B)^\perp$. The second equality follows straight from (1).

(3) For all $y\in Y$ we have $y\perp 0$.

(4) We have $0\in Y$, as for all $y\in Y$ we have $y\perp 0$. For all $x\in X\setminus\{0\}$, the following are equivalent: $x\notin Y^\perp$ and $\exists y\in Y: \pair{x,y}\neq 0$. Thus $Y$ being separating on $X$ is equivalent to $\forall x\in X\setminus \{0\}: x\notin Y^\perp$, which is equivalent to $Y^\perp \subseteq \{0\}$.

(5) By \ref{polarPropertiesLemma}.

(6) We have 
\[ B^{\perp\perp} = \Span(B^\perp)^\pol = \big(B^\perp\big)^\pol = \Span(B)^{\pol\pol} = \overline{\disked(\Span(B))}^{\sigma(X,Y)} = \overline{\Span(B)}^{\sigma(X,Y)}, \]
using the bipolar theorem \ref{bipolarTheorem}.

(7) In this case we have $B^\perp = B^{\perp\perp\perp} = \left(\overline{\Span(B)}^{\sigma(X,Y)}\right)^\perp = Y^\perp = \{0\}$.
\end{proof}


\begin{proposition}
Let $\sSet{X,Y,\pair{\cdot,\cdot}}$ be a pairing and $W_1,W_2$ subspaces of $Y$. Then
\[ (W_1+W_2)^\perp = W_1^\perp \cap W_2^\perp. \]
\end{proposition}
\begin{proof}
For a vector $v\in X$,
\begin{align*}
v\in (W_1+W_2)^\perp = (W_1 \cup W_2)^\perp &\implies \forall x\in W_1\cup W_2: \inner{v,x} = 0 \\
&\implies v\in W_1^\perp \cap W_2^\perp
\end{align*}
and
\begin{align*}
v\in W_1^\perp \cap W_2^\perp &\implies \forall x\in W_1, y\in W_2: \inner{v,x} = 0 = \inner{v,y} \\
&\implies \forall x\in W_1, y\in W_2:\inner{v, x+y} = 0 \implies v\in (W_1+W_2)^\perp.
\end{align*}
\end{proof}
TODO: dual result for closed subspaces.


\section{Adjoints}
\begin{definition}
Let $\sSet{X,Y,\pair{\cdot,\cdot}}$ and $\sSet{X',Y',\pair{\cdot,\cdot}'}$ be two pairings and $T: Y\to Y'$ a linear function. An \udef{adjoint} or \udef{transpose} is a linear operator $S: X'\to X$ such that
\[ \pair{x, Ty}' = \pair{Sx, y} \qquad \forall x\in X', \; y\in Y. \]
\end{definition}

\subsection{Adjoints for pairings}
\subsubsection{Existence of adjoints}
\begin{proposition} \label{existenceAdjointWeaklyContinuousFunction}
Let $\sSet{X,Y,\pair{\cdot,\cdot}}$ and $\sSet{X',Y',\pair{\cdot,\cdot}'}$ be two pairings, $T: Y\to Y'$ a linear function. Then $T$ has an adjoint $S: X'\to X$ \textup{if and only if} it is weakly continuous, i.e.\ continuous as a function
\[ T: \sSet{Y, \sigma(X,Y)} \to \sSet{Y', \sigma(X',Y')}. \]
The adjoint is unique if $Y$ separates $X$.
\end{proposition}
\begin{proof}
The weak continuity of $T$ is, by the characteristic property of the initial convergence \ref{characteristicPropertyInitialFinalConvergence}, equivalent to the continuity of
\[ \pair{x', \cdot}' \circ T = \pair{x', T(\cdot)}' : \sSet{Y, \sigma(X,Y)} \to \F \]
for all $x'\in X'$. In other words, $\pair{x', T(\cdot)}' \in \dual{\sSet{Y, \sigma(X,Y)}}$ for all $x'\in X'$. By \ref{functionalContinuityWeakTopology}, this is equivalent to the existence of some $x\in X$ such that $\pair{x', T(\cdot)}' = \pair{x,\cdot}$. We set $S(x') = x$.

Thus the weak continuity of $T$ is equivalent to the existence of some adjoint function $S$, without the requirement that it be linear. We now just need to show that $S$ can always be taken to be linear.

Indeed, pick some Hamel basis $\mathcal{X}$ of $X'$ and define $S'$ by setting $S'|_{\mathcal{X}} = S|_{\mathcal{X}}$ and extending linearly to the whole of $X'$. It is clear that $S'$ is an adjoint: set $x' = \sum_{x\in\mathcal{X}}\lambda_x x$ (with only finitely many $\lambda_x$ non-zero) and calculate
\begin{multline*}
\pair{x', T(\cdot)}' = \pair{\sum_{x\in\mathcal{X}}\lambda_x x, T(\cdot)}' = \sum_{x\in\mathcal{X}}\lambda_x \pair{x, T(\cdot)} = \sum_{x\in\mathcal{X}}\lambda_x \pair{S(x), \cdot} = \\ \sum_{x\in\mathcal{X}}\lambda_x \pair{S'(x), \cdot} = \pair{S'\left(\sum_{x\in\mathcal{X}}\lambda_x x\right), \cdot} = \pair{S'(x'), \cdot}.
\end{multline*}

Finally we note that the choice of $S(x')$ is unique if $Y$ separates $X$ by \ref{dualSystemBijection}.
\end{proof}
\begin{corollary} \label{weak*continuityAdjoint}
The adjoint of a weakly continuous operator is weak-$*$ continuous.
\end{corollary}
\begin{proof}
Suppose $S$ is an adjoint of $T$. By symmetry of the definition, $T$ is an adjoint of $S$ when considering the dual pairings. Thus $S$ is weak-$*$ continuous by the proposition.
\end{proof}
\begin{corollary} \label{adjointContinuousFunction}
Let $\sSet{V,\xi}, \sSet{W,\zeta}$ be vector convergence spaces and $T: V\to W$ a continuous linear function. Then $T$ has a unique adjoint
\[ S: \dual{W}\to \dual{V}: f\mapsto f\circ T. \]
\end{corollary}
Thus $S$ is the pre-composition $T^\star$. We can summarise this by $T^* = T^\star$.
\begin{proof}
The function $T$ is weakly continuous by \ref{continuityImpliesWeakContinuity} and $V$ separates $\dual{V}$ in the pairing $\sSet{\dual{V}, V, \evalMap}$. Finally the adjoint relation gives $f\big(T(x)\big) = S(f)(x)$, which determines $S$.
\end{proof}

\subsubsection{Calculating with adjoints}
\begin{proposition}
Let $\sSet{X,Y,\pair{\cdot,\cdot}}$, $\sSet{X',Y',\pair{\cdot,\cdot}'}$ be pairings, $T, T': Y\to Y'$ weakly continuous linear functions with adjoints $S, S': X'\to X$ and $\lambda\in \F$. Then
\begin{enumerate}
\item $S+S'$ is an adjoint of $T+T'$;
\item $\lambda S$ is an adjoint of $\lambda T$;
\item $\id_X$ is an adjoint of $\id_Y$.
\end{enumerate}
\end{proposition}
\begin{proof}
TODO
\end{proof}

\begin{proposition} \label{adjointPreimagePolarLemma}
Let $\sSet{X,Y,\pair{\cdot,\cdot}}$, $\sSet{X',Y',\pair{\cdot,\cdot}'}$ be pairings and $T: Y\to Y'$ a weakly continuous linear function with adjoint $S: X'\to X$. Then for all subsets $A\subseteq Y$, we have
\[ \big(T^{\imf}(A)\big)^\pol = S^{\preimf}(A^\pol). \]
\end{proposition}
\begin{proof}
We calculate, using \ref{polarLemma},
\begin{align*}
S^{\preimf}(A^\pol) &= S^{\preimf}\Big(\bigcap_{y\in A}\setbuilder{x\in X}{\abspair{x,y}\leq 1}\Big) \\
&= \bigcap_{y\in A}S^{\preimf}\Big(\setbuilder{x\in X}{\abspair{x,y}\leq 1}\Big) \\
&= \bigcap_{y\in A}\Big(\setbuilder{x'\in X'}{\abspair{S(x'),y}\leq 1}\Big) \\
&= \bigcap_{y\in A}\Big(\setbuilder{x'\in X'}{\abspair{x',T(y)}\leq 1}\Big) \\
&= \bigcap_{y'\in T^{\imf}(A)}\Big(\setbuilder{x'\in X'}{\abspair{x',y'}\leq 1}\Big) \\
&= \big(T^{\imf}(A)\big)^\pol.
\end{align*}
\end{proof}
\begin{corollary}
Let $\sSet{X,Y,\pair{\cdot,\cdot}}$, $\sSet{X',Y',\pair{\cdot,\cdot}'}$ be pairings and $T: Y\to Y'$ a weakly continuous linear function with adjoint $S: X'\to X$. Then
\begin{enumerate}
\item $\ker(S) = \im(T)^\perp$;
\item if $T$ is surjective and $Y'$ separates $X'$, then $S$ is injective.
\end{enumerate}
\end{corollary}
TODO: can we strengthen (2)?
\begin{proof}
(1) Set $A = Y$. Then $S^\preimf(Y^\pol) = S^\preimf(\{0\}) = \ker(S)$ and $T^\imf(Y) = \im(T)$.

(2) In this case $\ker(S) = (Y')^\perp = \{0\}$ by \ref{corollaryPerpAsPolar}, which means that $S$ is injective by \ref{injectivityKernelTriviality}.
\end{proof}

\begin{proposition}
Let $\sSet{X,Y,\pair{\cdot,\cdot}}$, $\sSet{X',Y',\pair{\cdot,\cdot}'}$ and $\sSet{X^{\prime\prime},Y^{\prime\prime},\pair{\cdot,\cdot}^{\prime\prime}}$ be pairings and $T: Y\to Y'$, $T': Y'\to Y^{\prime\prime}$ linear functions with adjoints $S: X'\to X$ and $S': X^{\prime\prime}\to X'$. Then $S\circ S': X^{\prime\prime}\to X$ is an adjoint of $T'\circ T: Y\to Y^{\prime\prime}$.
\end{proposition}
\begin{proof}
For all $x\in X^{\prime\prime}$ and $y\in Y$ we have
\[ \pair{x, (T'\circ T)(y)}^{\prime\prime} = \pair{S'(x), T(y)}' = \pair{(S\circ S')(x), y}. \]
\end{proof}
\begin{corollary}
Let $\sSet{X,Y,\pair{\cdot,\cdot}}$ and $\sSet{X',Y',\pair{\cdot,\cdot}'}$be a pairing and $T: Y\to Y'$ a weakly continuous linear bijection with adjoint $S: X'\to X$. Then $S$ has a (left/right?) inverse $R$ which is an adjoint of $T^{-1}$.
\end{corollary}
\begin{proof}
TODO
\end{proof}

\subsubsection{Continuity of adjoints}
\begin{proposition}
Let $\sSet{X,Y,\pair{\cdot,\cdot}}$, $\sSet{X',Y',\pair{\cdot,\cdot}'}$ be pairings and $T: Y\to Y'$ a weakly continuous linear function with adjoint $S: X'\to X$. Let $\mathcal{A}\subseteq\powerset(Y)$ be a set of weakly bounded sets. Then $S$ is continuous as a function
\[ S: \sSet{X',\polarTop{T^\imf(\mathcal{A})}} \to \sSet{X, \polarTop{\mathcal{A}}}. \]
\end{proposition}
\begin{proof}
By \ref{pretopologicalContinuityVicinities} and \ref{shiftHomeomorphism}, it is enough to check that $S^\preimf(U)\in \mathfrak{F}\setbuilder{\lambda T^\imf(A)^\pol}{\lambda\in \F, A\in \mathcal{A}}$ for all $U \in \mathfrak{F}\setbuilder{\lambda A^\pol}{\lambda\in \F, A\in \mathcal{A}}$.

Any such $U$ a subset $\lambda_0 A_0^\pol \cap \ldots \cap \lambda_n A_n^\pol$ for some $A_0,\ldots, A_n\in \mathcal{A}$ and $\lambda_0,\ldots, \lambda_n\in \F$. Now, by \ref{adjointPreimagePolarLemma},
\begin{align*}
S^\preimf\big(\lambda_0 A_0^\pol \cap \ldots \cap \lambda_n A_n^\pol\big) &= \lambda_0 S^\preimf(A_0^\pol) \cap \ldots \cap \lambda_n S^\preimf(A_n^\pol) \\
&= \lambda_0 T^\imf(A_0)^\pol \cap \ldots \cap \lambda_n T^\imf(A_n)^\pol \in \mathfrak{F}\setbuilder{\lambda T^\imf(A)^\pol}{\lambda\in \F, A\in \mathcal{A}}.
\end{align*}
\end{proof}
This proposition, together with \ref{weak*topologyPolarTopology}, provides another way to prove the weak-$*$ continuity of the adjoint $S$ (\ref{weak*continuityAdjoint}).
\begin{corollary}
Let $\sSet{V,\norm{\cdot}}$ and $\sSet{W,\norm{\cdot}}$ be normed spaces. Let $\dual{V}$ and $\dual{W}$ have their norm topologies and consider the pairings $\sSet{\dual{V}, V, \evalMap}$ and $\sSet{\dual{W}, W, \evalMap}$. Let $T: V\to W$ be a linear function with adjoint $S: \dual{W} \to \dual{V}$. Then $T$ is continuous \textup{if and only if} $S$ is continuous.
\end{corollary}
\begin{proof}
TODO with \ref{dualNormTopologyStrong}
\end{proof}


\section{The continuous dual}
\begin{definition}
Let $\sSet{V,\xi}$ be a convergence vector space. The \udef{continuous dual} of $V$ is the vector convergence space $\dual{V}_c = \contLin_c(V, \F)$.
\end{definition}
The continuous dual is a convergence vector space by \ref{continuousConvergenceVectorSpace}.

\begin{proposition}
Let $\sSet{V,\xi}$ be a convergence vector space and $H\in \powerfilters(\dual{V})$. Then $H$ converges continuously to $f$ \textup{if and only if} $H$ converges pointwise to $f$ and for all $F \overset{\xi}{\longrightarrow} 0$, there exists $A\in F$ such that $A^\pol\in H$.
\end{proposition}
\begin{proof}
First assume $H\overset{c}{\longrightarrow} f$. Then $H$ converges pointwise to $f$ by \ref{strengthContinuousPointwiseConvergence}.

Take $F \overset{\xi}{\longrightarrow} 0$. Then, by continuous convergence, $\upset\evalMap^{\imf\imf}(H\otimes F) \overset{\F}{\longrightarrow} f(0) = 0$. In particular there exist $B\in H, A\in F$ such that $\evalMap^\imf(B\times A) \subseteq \cball(0,1)$. This means that $B\subseteq A^\pol$ and thus $A^\pol\in H$.

Now consider the direction $\Leftarrow$. Take some $G\overset{\xi}{\longrightarrow} x$. We need to show that $\upset\evalMap^{\imf\imf}(H\otimes G) \overset{\F}{\longrightarrow} f(x)$. Now
\[ \upset\evalMap^{\imf\imf}(H\otimes G) = \upset\evalMap^{\imf\imf}\big(H\otimes (G-\pfilter{x})\big) + \upset\evalMap^{\imf\imf}(H\otimes \pfilter{x}) \]
and $\upset\evalMap^{\imf\imf}(H\otimes \pfilter{x})$ converges to $f(x)$ by pointwise convergence, so we need to show that $\upset\evalMap^{\imf\imf}\big(H\otimes (G-\pfilter{x})\big)$ converges to $0$.

For all $\epsilon > 0$, $\epsilon(G-\pfilter{x})$ converges to $0$ and thus there exists $A\in G-\pfilter{x}$ such that $B\subseteq (\epsilon A)^\pol = \epsilon^{-1}A^\pol$ for some $B\in H$. This implies $\evalMap^{\imf}(B\times A) \subseteq \cball(0,\epsilon)$ and thus $\cball(0,\epsilon)\in \upset\evalMap^{\imf\imf}\big(H\otimes (G-\pfilter{x})\big)$.

As $\epsilon$ was taken arbitrarily, we have $\neighbourhood_\F(0) \subseteq \upset\evalMap^{\imf\imf}\big(H\otimes (G-\pfilter{x})\big)$ and thus $\upset\evalMap^{\imf\imf}\big(H\otimes (G-\pfilter{x})\big) \overset{\F}{\longrightarrow} 0$.
\end{proof}
\begin{corollary} \label{continuousDualTopologicalSpace}
If $\sSet{V, \xi}$ is topological, then
\begin{enumerate}
\item $H$ converges continuously to $f$ \textup{if and only if} $H$ converges pointwise to $f$ and there exists $A\in \neighbourhood_\xi(0)$ such that $A^\pol\in H$;
\item $\dual{V}_c$ carries the specified sets convergence on $\dual{V}_p$ determined by $\setbuilder{A^\pol}{A\in \neighbourhood_\xi(0)}$;
\item $\dual{V}_c$ carries the specified sets convergence on $\dual{V}_p$ determined by the equicontinuous subsets of $\dual{V}$.
\end{enumerate}
\end{corollary}
\begin{corollary}
Let $\sSet{V,\xi}$ be a convergence vector space and $H\in \powerfilters(\dual{V})$. Then $H$ converges continuously to $0$ \textup{if and only if} $H$ contains all polars of finite sets and for all $F \overset{\xi}{\longrightarrow} 0$, there exists $A\in F$ such that $A^\pol\in H$.
\end{corollary}
\begin{proof}
By \ref{weak*topologyPolarTopology}.
\end{proof}

\begin{proposition}
Let $\sSet{V,\xi}$ be a locally convex topological space. If there exists a vector space topology on $\dual{V}$ such that $\evalMap: \dual{V}\times V\to \F$ is continuous, then $\sSet{V,\xi}$ is seminormable.
\end{proposition}
\begin{proof}
Since $\cball(0,1)\subseteq \F$ is a neighbourhood of $0$, $\evalMap^{\preimf}\big(\cball(0,1)\big)$ is a neighbourhood of $(0,0)\in \dual{V}\times V$ by \ref{continuityVicinityFilter}. Thus we can find $A\subseteq \neighbourhood_{\dual{V}}(0)$ and $B\subseteq \neighbourhood_V(0)$ such that $\evalMap^\imf(A\times B)\subseteq \cball(0,1)$, which means that $A\subseteq B^\pol$. By local convexity we may take $B$ to be convex.

Now $A$ is absorbent by \ref{vicinityFilterAtOrigin}, so $B^\pol$ is absorbent by \ref{absorbingSetProperties} and thus $B$ weakly bounded by \ref{weaklyBoundedLemma}. It is bounded by \ref{weaklyBoundedIffBounded}. This implies $\sSet{V,\xi}$ is seminormable by \ref{metrisableBoundedNeighbourhood}.
\end{proof}

\begin{lemma} \label{continuousDualArzelaAscoli}
Let $\sSet{V,\xi}$ be a vector convergence space and $H\subseteq \dual{V}_c$. Then
\begin{enumerate}
\item $H$ is relatively compact \textup{if and only if} it is equicontinuous;
\item $H$ is compact \textup{if and only if} it is equicontinuous and closed.
\end{enumerate}
\end{lemma}
\begin{proof}
(1) If $H$ is relatively compact, then $H$ is evenly continuous by \ref{compactoidImpliesEvenContinuity} and thus equicontinuous by \ref{equicontinuityEvenContinuity}.

For the converse, equicontinuity implies even continuity by \ref{equicontinuityEvenContinuity} and so we can apply \ref{evenContinuityRelativeCompactness}. We just need to show that $\evalMap_x^\imf(H)$ is relatively compact in $\F$ for all $x\in V$.

Since $\neighbourhood_\F(0)\cdot v\to 0$, we have, by equicontinuity \ref{equicontinuityGroupHomomorphisms}, that
\[ \neighbourhood_\F(0)\cdot \{\evalMap_x^\imf(H)\} = \neighbourhood_\F(0)\cdot \evalMap^{\imf\imf}(\{H\}\otimes \pfilter{x}) = \evalMap^{\imf\imf}\big(\{H\}\otimes \neighbourhood_\F(0)\cdot \pfilter{x}\big) \to 0, \]
so $\evalMap_x^\imf(H)\subseteq \F$ is bounded. By Heine-Borel (TODO ref) it is relatively compact.

(2) Follows from (1) by \ref{compactClosedSets} and the fact that $\dual{V}_c$ is Hausdorff, by \ref{continuousConvergencePropertiesFromCodomain}.
\end{proof}

\begin{proposition}
Let $\sSet{V,\xi}$ be a topological vector convergence space and $U\in \neighbourhood_\xi(0)$. Then $U^\pol$ is a compact subset of $\dual{V}_c$.
\end{proposition}
\begin{proof}
We apply \ref{continuousDualArzelaAscoli} by noting that $U^\pol$ is equicontinuous (\ref{equicontinuityTopologicalFunctionals}) and closed (\ref{polarPropertiesLemma}, \ref{openClosedConvergenceInclusions} and \ref{strengthContinuousPointwiseConvergence}).
\end{proof}
\begin{corollary} \label{continuousDualTVSLocallyCompact}
Let $\sSet{V,\xi}$ be a topological vector convergence space. Then $\dual{V}_c$ is locally compact.
\end{corollary}
\begin{proof}
By \ref{continuousDualTopologicalSpace}, each continuously convergence filter in $\powerfilters(\dual{V})$ contains $A^\pol$ for some $A\in\neighbourhood_\xi(0)$.
\end{proof}
TODO if $V$ is infinite dimensional, then no topological vector convergence on $\dual{V}$ makes it locally compact (?)

\subsection{Continuous adjoints}
\begin{proposition}
Let $\sSet{V,\xi}$, $\sSet{W, \zeta}$ be convergence vector spaces and $T: V\to W$ a continuous linear function with adjoint $S: \dual{W}\to \dual{V}: f\mapsto f\circ T$. Then $S$ is continuous as a function $S: \dual{W}_c\to \dual{V}_c$.
\end{proposition}
\begin{proof}
We have that $(\id_{\dual{W}}, \underline{T}): f\mapsto (f, T)$ is continuous by \ref{embeddingInContinuousConvergence} and \ref{continuityFunctionTuple}. The adjoint $S$ is this function composed with the composition operator, which is continuous by \ref{compositionContinuouslyContinuous}.
\end{proof}

\subsection{Reflexivity}
Let $\sSet{V,\xi}$ be a convergence vector space. Then the map $\evalMap_-: V\to \dual{(\dual{V}_c)}_c: v\mapsto \evalMap_v$ is linear and continuous by \ref{curriedEvalMapContinuous}.

\begin{definition}
Let $\sSet{V,\xi}$ be a convergence vector space. It is called \udef{reflexive} if $\evalMap_-$ is an isomorphism.
\end{definition}

\begin{proposition}
Let $\sSet{X,\xi}$ be a convergence space. Then $\cont_c(X)$ is reflexive.
\end{proposition}
\begin{proof}
TODO Beattie / Butzmann p. 125
\end{proof}

\begin{proposition}
Let $\sSet{V,\xi}$ be a reflexive convergence vector space. Then $\lconvMod(\dual{V}_c) = \contLin_\text{co}(V)$.
\end{proposition}
\begin{proof}
We have that $\id: \dual{V}_c \to \contLin_\text{co}(V)$ is continuous by \ref{continuousConvergenceCompactOpenComparison}.

TODO  BB 4.2.19
\end{proof}

\subsection{The continuous dual of particular topological vector spaces}
\subsubsection{The continuous dual of a locally comapct TVS}
\begin{proposition} \label{continuousDualLocallyCompactTVS}
Let $\sSet{V,\xi}$ be a locally compact vector convergence space. Then $\dual{V}_c = \contLin_\text{co}(V)$ and
\[ \neighbourhood_{\dual{V}_c}(0) = \upset\setbuilder{K^\pol}{\text{$K\subseteq V$ compact}}. \]
\end{proposition}
\begin{proof}
The equality of the continuous and compact open convergences follows from \ref{continuousConvergenceCompactOpen}.

Pick compact $K\subseteq V$ and open $U\subseteq \F$ with $0\in U$. Then $\cball(0,\epsilon)\subseteq U$ for some $\epsilon >0$. If $f^\imf\big(K\big) \subseteq \cball(0,\epsilon) = \epsilon \cball(0,1)$, then $f\in (\epsilon^{-1}K)^\pol = \epsilon K^\pol$, which is compact. Thus every basis element of the neighbourhood filter of the compact open topology contains a polar of a compact set.

We further not that the set of polars of compact sets is closed under finite intersections: by \ref{polarOfUnion} $K_1^\pol \cap K_2^\pol = (K_1\cup K_2)^\pol$ and $K_1\cup K_2$ is compact by \ref{compactConstructions}.

Conversely $f\in K^\pol$ implies $f^{\imf}(K) \subseteq \ball(0, 1.1)$, so $K^\pol$ is contained in an open neighbourhood of the origin in the compact open topology.
\end{proof}

\subsubsection{The continuous dual of a locally convex TVS}
\begin{proposition}
Let $\sSet{V,\xi}$ be a locally convex, Hausdorff topological vector space. Then
\begin{enumerate}
\item $\dual{(\dual{V}_c)}_c$ is a complete, locally convex and Hausdorff TVS;
\item $\evalMap_-: V\to \dual{(\dual{V}_c)}_c$ is an embedding into a dense subspace.
\end{enumerate}
\end{proposition}
\begin{proof}
(1) Completeness is given by \ref{continuousDualComplete}; Hausdorffness by \ref{continuousConvergencePropertiesFromCodomain}.

The space $\dual{V}_c$ is locally compact by \ref{continuousDualTVSLocallyCompact} and then \ref{continuousDualLocallyCompactTVS} gives that $\dual{(\dual{V}_c)}_c$ is a locally convex TVS.

(2) TODO BB 4.3.19 
\end{proof}
\begin{corollary}
Let $\sSet{V,\xi}$ be a locally convex, Hausdorff topological vector space. Then $\dual{(\dual{V}_c)}_c$ is the completion of $V$.
\end{corollary}
\begin{corollary}
A Hausdorff locally convex topological vector space is reflexive \textup{if and only if} it is complete.
\end{corollary}

\section{Adjoints TODO}
\begin{definition}
Let $\sSet{X,Y,\pair{\cdot,\cdot}}$ and $\sSet{X',Y',\pair{\cdot,\cdot}'}$ be two pairings and $T: Y\not\to Y'$ a linear operator. An \udef{adjoint} or \udef{transpose} is a linear operator $S: X'\not\to X$ such that
\[ \pair{x, Ty}' = \pair{Sx, y} \qquad \forall x\dom(S), \; y\in\dom(T). \]
\end{definition}

\begin{lemma}
Let $\sSet{X,Y,\pair{\cdot,\cdot}}$ and $\sSet{X',Y',\pair{\cdot,\cdot}'}$ be two pairings and $T: Y\not\to Y'$ a linear operator.

Let $S_1, S_2: X'\not\to X$ be adjoints of $T$ then for all $x\in \dom(S_1)\cap\dom(S_2)$ we have $S_1(x) - S_2(x) \in \dom(T)^\perp$.

Conversely, let $S$ be an adjoint of $T$ and $x\in\dom(S)$. Then for all $v\in \dom(T)^\perp$ there exists an adjoint $S'$ such that $S'(x) = S(x) + v$.
\end{lemma}
\begin{proof}
For all $u\in \dom(T)$ we have
\[ \pair{S_1(x) - S_2(x), u} = \pair{S_1(x), u} - \pair{S_2(x), u} = \pair{x, Tu}' - \pair{x, Tu}' = 0. \]
So $(S_1(x) - S_2(x)) \in \dom(T)^\perp$.

For the converse, pick some $x\in X$ and set $S' = S + \pair{x,\cdot}v$. This is an adjoint: for all $a\in \dom(T), b\in \dom(S') = \dom(S)$ we have
\[  \inner{S'b,a} = \inner{Sb, a} + \pair{x,b}\pair{v,a} = \inner{Sb, a} = \inner{b,Ta}'. \]
\end{proof}
\begin{corollary}
Let $\sSet{X,Y,\pair{\cdot,\cdot}}$ and $\sSet{X',Y',\pair{\cdot,\cdot}'}$ be two pairings, $T: Y\not\to Y'$ a linear operator and $S_1, S_2$ two adjoints of $T$.

If $Y$ separates $X$ and $T$ is $\sigma(X,Y)$-densely defined, then for all $x\in \dom(S_1)\cap\dom(S_2)$ we have $S_1(x) = S_2(x)$.
\end{corollary}
\begin{proof}
We have, by \ref{corollaryPerpAsPolar}, $\dom(T)^\perp = \{0\}$. So $S_1(x) - S_2(x) = 0$.
\end{proof}
\begin{corollary}
Let $\sSet{X,Y,\pair{\cdot,\cdot}}$ and $\sSet{X',Y',\pair{\cdot,\cdot}'}$ be two pairings, such that $Y$ separates $X$.

Let $T: Y\not\to Y'$ be a linear operator. Then
\[ \bigcup\setbuilder{\graph(S)}{\text{$S\in (X'\not\to X)$ is an adjoint of $T$}} \]
is the graph of an operator \textup{if and only if} $T$ is $\sigma(X,Y)$-densely defined.
\end{corollary}

\begin{definition}
Let $\sSet{X,Y,\pair{\cdot,\cdot}}$ and $\sSet{X',Y',\pair{\cdot,\cdot}'}$ be two pairings. Let $T: Y\not\to Y'$ be a linear operator.

We define the adjoint $T^*$ as the \emph{relation} on $(X',X)$ with graph
\[ \graph(T^*) \defeq \bigcup\setbuilder{\graph(S)}{\text{$S\in (X'\not\to X)$ is an adjoint of $T$}}. \]
\end{definition}

\begin{lemma}
Let $\sSet{X,Y,\pair{\cdot,\cdot}}$ and $\sSet{X',Y',\pair{\cdot,\cdot}'}$ be two pairings, such that $Y$ separates $X$. Let $T: Y\not\to Y'$ be a linear operator with $\sigma(X,Y)$-dense domain.

If $S$ is an adjoint of $T$ that is defined everywhere, then $T^* = S$.
\end{lemma}

\begin{lemma} \label{pairAdjointRelationLemma}
Let $\sSet{X,Y,\pair{\cdot,\cdot}}$ and $\sSet{X',Y',\pair{\cdot,\cdot}'}$ be two pairings. Let $T: Y\not\to Y'$ be a linear operator and $(x,y)\in X'\times X$.

Then $(x, y)\in T^*$ \textup{if and only if}
\[ \forall z\in\dom(T): \; \pair{x, T(z)} = \pair{y, z}. \]
\end{lemma}
\begin{proof}
$\boxed{\Rightarrow}$ If $(x, y)\in T^*$, then there exists an adjoint $S: X'\not\to X$ such that $S(x) = y$. For all $z\in \dom(T)$ we have $\pair{x, T(z)}' = \pair{S(x), z} = \pair{y, z}$.

$\boxed{\Leftarrow}$ The function defined by $S(x) = y$ and extended to $\Span\{x\}$ by linearity is an adjoint.
\end{proof}

\begin{proposition} \label{pairAdjointDomain}
Let $\sSet{X,Y,\pair{\cdot,\cdot}}$ and $\sSet{X',Y',\pair{\cdot,\cdot}'}$ be two pairings. Let $T: Y\not\to Y'$ be a linear operator. Then
\[ \dom(T^*) = \setbuilder{x\in X'}{\text{$\dom(T)\to \F: u\mapsto \pair{x, Tu}$ is a $\sigma(X,Y)$-continuous functional}}. \]
\end{proposition}
\begin{proof}
$\boxed{\subseteq}$ If $\omega_x: u\mapsto \inner{x, Tu}$ is $\sigma(X,Y)$-continuous, then it can be extended to a continuous functional on all of $Y$ by 


------ TODO: rework from here!! ------

its domain can be extended by continuity to $\overline{\dom(T)}$, which is a Hilbert space. This extended functional has a Riesz vector $x^*$ such that $\omega_x = u\mapsto \inner{x^*, u}$. The linear operator with domain $\Span\{x\}$ that maps $x$ to $x^*$ is then an adjoint.

$\boxed{\supseteq}$ If $x\in\dom(T^*)$, then, using the Cauchy-Schwarz inequality,
\[ |\inner{x,Tu}| = |\inner{T^*x,u}| \leq \norm{T^*x}\;\norm{u}, \]
so the functional $u\mapsto \inner{x, Tu}$ is bounded.
\end{proof}
\begin{corollary}
The domain $\dom(T^*)$ is a vector space and in particular contains $0$.
\end{corollary}

\begin{proposition} \label{HilbertAdjointGaloisConnection}
Let $H, K$ be Hilbert spaces. Take $T\in (H\not\to K)$ and $S\in (K\not\to H)$. Then
\[ S \subseteq T^* \iff T\subseteq S^*. \]
Thus $(*,*)$ is an antitone Galois connection between $\sSet{(H\not\to K), \subseteq}$ and $\sSet{(K\not\to H), \subseteq}$.
\end{proposition}
\begin{proof}
We have $S \subseteq T^*$ iff $S$ is an adjoint of $T$ iff $T$ is an adjoint of $S$ (by \ref{adjointRequirementSymmetric}) iff $T\subseteq S^*$.
\end{proof}
\begin{corollary} \label{HilbertAdjointAntitone}
Let $S,T: H\not\to K$ be operators between Hilbert spaces such that $S\subseteq T$. Then $T^* \subseteq S^*$.
\end{corollary}
\subsection{Properties of the adjoint relation}

\begin{proposition}
Let $T$ be an operator between Hilbert spaces and $\lambda\in\C$. If $\lambda \neq 0$, then
\[ \begin{pmatrix}
\id & 0 \\ 0 & \overline{\lambda}\id
\end{pmatrix} \graph(T^*) = (\lambda T)^*. \]
\end{proposition}
Note that if $T^*$ is a function (i.e.\ if $T$ is densely defined), then $\begin{pmatrix}
\id & 0 \\ 0 & \overline{\lambda}\id
\end{pmatrix} \graph(T^*) = \overline{\lambda}T^*$. We write the former in the proposition, because we have not made this assumption.

If $\lambda = 0$ and $T: H\not\to K$, then
\[ \begin{pmatrix}
\id & 0 \\ 0 & 0
\end{pmatrix} \graph(T^*) = \big(0: \dom(T^*)\to H\big) \subseteq \big(0: K\to H\big) = (0 T)^*, \]
where the last equality is given by \ref{adjointBoundedEverywhereDefined}.
\begin{proof}
For the inclusion $\subseteq$, take $f$ to be an adjoint of $T$. It is enough to show that $\overline{\lambda}f$ is an adjoint of $\lambda T$. This follows from
\[ \inner{\overline{\lambda}f(w), v} = \lambda\inner{f(w), v} = \lambda\inner{w,Tv} = \inner{w,\lambda Tv} \qquad \forall w\in \dom(f), v\in \dom(T). \]

For the other inclusion, let $f$ be an adjoint of $\lambda T$. It is enough to show that $\overline{\lambda^{-1}}f$ is an adjoint of $T$, because then $f = \overline{\lambda}\cdot\overline{\lambda^{-1}}f \subseteq \begin{pmatrix}
\id & 0 \\ 0 & \overline{\lambda}\id
\end{pmatrix} \graph(T^*)$. Indeed
\[ \inner{\overline{\lambda^{-1}}f(w), v} = \lambda^{-1}\inner{f(w),v} = \inner{w,\lambda^{-1}\lambda Tv} = \inner{w,Tv} \quad \forall w\in \dom(f), v\in \dom(T). \]
\end{proof}

\begin{proposition} \label{adjointGraph}
Let $T: H\not\to K$ be an operator between Hilbert spaces. Then
\begin{align*}
\graph(T^*) &= \left( \begin{pmatrix}
0 & -\id \\ \id & 0
\end{pmatrix}\graph(T) \right)^\perp 
=  \begin{pmatrix}
0 & -\id \\ \id & 0
\end{pmatrix}\graph(T)^\perp.
\end{align*}
If $T$ is densely defined, then $T^*$ is a closed operator.
\end{proposition}
\begin{proof}
We have
\[ \graph(T^*) = \bigcup\setbuilder{\graph(S)}{\text{$S\in (K\not\to H)$ is an adjoint of $T$}}. \]
Take an adjoint $S$ and $(w, Sw)$ in $\graph(S)$. Then for all $v\in\dom(T)$:
\[ 0 = \inner{w, Tv}_K - \inner{Sw, v}_H = \inner{w, Tv}_K + \inner{Sw, -v}_H = \inner{(w, Sw), (Tv,-v)}_{K\oplus H}. \]
So $(Tv,-v) = \begin{pmatrix}
0 & -\id \\ \id & 0
\end{pmatrix} (v,Tv) \in \graph(S)^\perp $.

The final equality follows from \ref{perpUnderIsometry}, using the fact that $\begin{pmatrix}
0 & -\id \\ \id & 0
\end{pmatrix}$ is a surjective isometry.

If $T$ is densely defined, then $T^*$ is a function by \ref{maximalAdjointIsOperator}. It is closed by \ref{orthogonalComplementClosed}.
\end{proof}
\begin{corollary} \label{adjointDenselyDefinedClosable}
Let $T: H\not\to K$ be a densely defined operator between Hilbert spaces.
Then
\begin{enumerate}
\item $\graph(T^{**}) = \overline{\graph(T)}$;
\item $T^*$ is densely defined \textup{if and only if} $T$ is closable;
\item If $T$ is closable, then $\overline{T} = T^{**}$.
\end{enumerate}
\end{corollary}
\begin{proof}
From the proposition we have
\begin{align*}
\graph(T^{**}) &=  \begin{pmatrix}
0 & -\id \\ \id & 0
\end{pmatrix}\graph(T^*)^\perp 
=  \begin{pmatrix}
0 & -\id \\ \id & 0
\end{pmatrix}\left(\begin{pmatrix}
0 & -\id \\ \id & 0
\end{pmatrix}\graph(T)^\perp\right)^\perp \\
&= \begin{pmatrix}
0 & -\id \\ \id & 0
\end{pmatrix}^2\graph(T)^{\perp\perp} = -\graph(T)^{\perp\perp}
= \overline{\graph(T)}.
\end{align*}
The right-hand side is the graph of an operator iff $T$ is closable and the left-hand side is the graph of an operator iff $T^*$ is densely defined, by \ref{maximalAdjointIsOperator}.

For a closable operator, the closure is defined by $\overline{\graph(T)} = \graph(\overline{T})$.
\end{proof}

\begin{proposition} \label{adjointBoundedEverywhereDefined}
Let $T: H\to K$ be a densely defined operator between Hilbert spaces. Then $\dom(T^*) = K$ \textup{if and only if} $T$ is bounded.
\end{proposition}
\begin{proof}
The direction $\Leftarrow$ is given by \ref{adjointDomain}.

For the other direction, note that $T^*$ is closed by \ref{adjointGraph}. Then $T^*$ is bounded by the closed graph theorem \ref{BanachClosedGraphTheorem}. We use the direction $\Leftarrow$ to see that $\dom(T^{**}) = H$. Similarly, $T^{**}$ is closed by \ref{adjointGraph} and bounded by the closed graph theorem \ref{BanachClosedGraphTheorem}. Thus $T\subseteq \overline{T} = T^{**}$ is bounded.
\end{proof}

An important application of this proposition is the Hellinger-Toeplitz theorem \ref{HellingerToeplitz}.

\begin{proposition} \label{adjointAlgebraicProperties}
Let $T,S$ be compatible operators between Hilbert spaces. Then
\begin{enumerate}
\item $S^* + T^* \subseteq (S+T)^*$;
\item $S^*T^* \subseteq (TS)^*$.
\end{enumerate}
\end{proposition}
\begin{proof}
(1) Let $f$ be an adjoint of $S$ and $g$ an adjoint of $T$. It is enough to see that $f+g$ is an adjoint of $S+T$. Indeed $\forall w\in \dom(f + g), v\in \dom(S+T)$
\[ \inner{(f + g)(w), v} = \inner{f(w),v} + \inner{g(w), Tv} = \inner{w,Sv} + \inner{w,Tv} = \inner{w,(S+T)v}. \]

(2) Let $f$ be an adjoint of $T$ and $g$ an adjoint of $S$. It is enough to see that $gf$ is an adjoint of $TS$. Indeed
\[ \inner{g\circ f(w), v} = \inner{f(w), Sv} = \inner{w,TSv} \qquad \forall w\in \dom(g\circ f), v\in \dom(TS). \]
\end{proof}


There exist various conditions that make the inclusions in \ref{adjointAlgebraicProperties} equalities.
\begin{proposition} \label{equalityAlgebraicPropertiesAdjoint}
Let $T,S$ be compatible operators between Hilbert spaces.
\begin{enumerate}
\item if $T$ is densely defined, $\dom(S) \subseteq \dom(T)$ and $\dom\big((S+T)^*\big) \subseteq \dom(T^*)$, then $S^* + T^* = (S+T)^*$;
\item if $T$ is densely defined, $\im(S)\subseteq \dom(T)$ and $\dom\big((TS)^*)\subseteq \dom(T^*)$, then $S^*T^* = (TS)^*$;
\item if $S$ is densely defined and $\im(S)$ has finite codimension, then $S^*T^* = (TS)^*$.
\end{enumerate}
\end{proposition}
\begin{proof}
(1) By \ref{adjointAlgebraicProperties}, we have
\[ (S+T)^* - T^* \subseteq (S+T-T)^* = S^*, \]
where the last equality is due to $\dom(S) \subseteq \dom(T)$. Now take $x,y$ such that $x\in \dom\big((S+T)^*\big)$. Then $T^*(x)$ exists and we have the implications
\begin{align*}
x(S+T)^*y \iff& x\big((S+T)^* - T^* + T^*\big)y \\
\iff& \exists z: \; x\big((S+T)^* - T^*\big)z \land (z+T^*(x) = y) \\
\implies& \exists z: \; x(S^*)z \land (z+T^*(x) = y) \\
\iff& x(S^* + T^*)y.
\end{align*}
Thus $(S+T)^* \subseteq S^* + T^*$.

(2) We need to prove $(TS)^* \subseteq S^*T^*$. Assume $(x,y)\in (TS)^*$. By \ref{adjointRelationLemma}, we have
\[ \forall z\in \dom(TS):\; \inner{x, TS(z)} = \inner{y, z}. \]
Because $\im(S)\subseteq \dom(T)$, we have $\dom(TS) = \dom(S)$. Also, by assumption, $x\in \dom(T^*)$. So we have
\[ \forall z\in \dom(S):\; \inner{x, TS(z)} = \inner{T^*(x), S(z)} = \inner{y, z}, \]
which means that $\big(T^*(x), y\big)\in S^*$, so $(x,y)\in S^*T^*$.

(3)
\end{proof}
\begin{corollary}
If $T$ is bounded and everywhere defined, then
\[ S^* + T^* = (S+T)^* \qquad\text{and}\qquad S^*T^* = (TS)^*. \]
\end{corollary}


\begin{lemma} \label{HilbertAdjointLemma}
Let $S,T\in\Bounded(H,K)$ and $\lambda \in \mathbb{F}$.
\begin{enumerate}
\item $(T^*)^* = T$;
\item $(S+T)^* = S^* + T^*$;
\item $(\lambda T)^* = \bar{\lambda}T^*$;
\item $\id_V^* = \id_V$.
\end{enumerate}
Let $T\in\Bounded(H_1,H_2), S\in\Bounded(H_2,H_3)$
\begin{enumerate}
\setcounter{enumi}{4}
\item $(ST)^* = T^*S^*$.
\end{enumerate}
\end{lemma}

\begin{note}
Useful exercise: The identities of \ref{HilbertAdjointLemma} can also be proven by elementary manipulations. For example, to prove (1), we take arbitrary $v\in H$ and $w\in K$, Then
\[ \inner{w,Tv} = \inner{T^*w,v} = \overline{\inner{v,T^*w}} = \overline{\inner{(T^*)^*v,w}} = \inner{w, (T^*)^*v}. \]
By lemma \ref{elementaryOrthogonality} we have $Tv = (T^*)^*v$ for all $v\in V$. 
\end{note}

\subsection{Adjoints of densely defined operators}
The adjoint of an operator is a function if and only the operator is densely defined.

\begin{proposition} \label{adjointRangeCriterion}
Let $S: K\not\to H$ and $T: H\not\to K$ be linear operators between Hilbert spaces. If
\[ \im(S\cap T^*) = H \qquad\text{and}\qquad \im(T\cap S^*) = K, \]
then $S$ and $T$ are densely defined with $S^* = T$ and $T^* = S$.
\end{proposition}
\begin{proof}
Notice that $S\cap T^*$ and $T\cap S^*$ are linear operators that are adjoints of each other.

We claim that they are densely defined: take $x\in \dom(S\cap T^*)^\perp$. Then there exists some $y\in H$ such that $x = (T\cap S^*)y$ because of surjectivity. Now for all $z\in \dom(S\cap T^*)$
\[ 0 = \inner{z,x} = \inner{z, (T\cap S^*)y} = \inner{(S\cap T^*)z, y}, \]
so $\inner{z',y} = 0$ for all $z'\in H$, by surjectivity. This means, by \ref{elementaryOrthogonality}, that $y=0$ and thus also $x = (T\cap S^*)y = 0$. We conclude that $\dom(S\cap T^*)^\perp = \{0\}$, meaning $(S\cap T^*)$ is densely defined. The argument for $(T\cap S^*)$ is similar.

It follows that $S$ and $T$ must be densely defined. We have, by \ref{kernelImageAdjoint},
\[ \ker(S) = \im(S^*)^\perp \subseteq \im(T\cap S^*)^\perp = \{0\}. \]
Similarly $\ker(T) = \ker(S^*) = \ker(T^*) = \{0\}$.

So we have $\ker(S) = \ker(T^*)$, $\im(S)\subseteq \im(S\cap T^*)$ and $\im(T^*)\subseteq \im(S\cap T^*)$. The equality $S = T^*$ follows from \ref{partialFunctionSubset}. The equality $T = S^*$ is similar.
\end{proof}


\begin{proposition} \label{kernelImageAdjoint}
Let $T: H\not\to K$ be an operator between Hilbert spaces. Then
\[ \forall v\in K: \; (v,0)\in T^* \iff v\in \im(T)^\perp. \]
If $T^*$ is densely defined, this reduces to
\begin{enumerate}
\item $\ker(T^*) = \im(T)^\perp$;
\item $\ker(T) \subseteq \im(T^*)^\perp$;
\item if $T$ is closed, then $\ker(T) = \im(T^*)^\perp$
\end{enumerate}
\end{proposition}
\begin{proof}
(1) Because $\dom(T)$ is dense in $H$, we have $\dom(T)^\perp = \{0\}$ by \ref{orthogonalComplementDenseSpace}. Take $v\in K$. We have the equivalences
\begin{align*}
v\in \im(T)^\perp &\iff \forall x \in\dom(T): \inner{v, T(x)} = 0 \\
&\iff \forall x \in\dom(T): \inner{v, T(x)} = \inner{v, 0} \\
&\iff (v,0)\in T^*,
\end{align*}
using \ref{adjointRelationLemma}.

Point (1) is a direct translation in the case that $T^*$ is a function.

For point (2) note that $T\subseteq T^{**}$ (by \ref{adjointDenselyDefinedClosable}) implies that $(v,0)\in T \implies (v,0)\in T^{**}$.

For point (3): in this case $\ker(T) = \ker(T^{**}) = \im(T^*)^\perp$.
\end{proof}
\begin{corollary}[Closed range theorem for Hilbert spaces]
Let $T$ be a closed, densely defined operator between Hilbert spaces. Then the following are equivalent:
\begin{enumerate}
\item $\im(T)$ is closed;
\item $\im(T^*)$ is closed;
\item $\im(T) = \ker(T^*)^\perp$;
\item $\im(T^*) = \ker(T)^\perp$.
\end{enumerate}
\end{corollary}
\begin{proof}
By the proposition and \ref{orthogonalComplementClosed}, we have $\overline{\im(T)} = \ker(T^*)^\perp$. This shows $(1) \Leftrightarrow (3)$ and $(2) \Leftrightarrow (4)$.

TODO equivalence $(1)\Leftrightarrow (2)$.
\end{proof}
TODO ref closed range theorem for Banach spaces. This is, e.g., the case when $T$ is bounded below, see \ref{boundedBelowClosedRange}.

\begin{proposition}
Let $T: H\not\to K$ be a densely defined operator between Hilbert spaces. Then
\begin{enumerate}
\item $\im(T)$ is dense in $K$ \textup{if and only if} $T^*$ is injective;
\item if $T$ and $T^*$ are injective, then $(T^*)^{-1} = (T^{-1})^*$;
\item if $T$ is closable and $\overline{T}$ is injective, then $\overline{T}^{\,-1} = \overline{T^{-1}}$.
\end{enumerate}
\end{proposition}
\begin{proof}
(1) This is immediate from \ref{kernelImageAdjoint} and \ref{injectivityKernelTriviality}:
\[ \text{$\im(T)$ is dense} \quad\iff\quad \{0\} = \im(T)^\perp = \ker(T^*). \]

(2) We have $\graph(T^{-1}) = \begin{pmatrix}
0 & \id \\ \id & 0
\end{pmatrix}\graph(T)$. Also note that $\begin{pmatrix}
0 & \id \\ \id & 0
\end{pmatrix}$ and $\begin{pmatrix}
0 & -\id \\ \id & 0
\end{pmatrix}$ commute. Then we compute using \ref{adjointGraph}:
\begin{align*}
\graph((T^*)^{-1}) &= \begin{pmatrix}
0 & \id \\ \id & 0
\end{pmatrix}\begin{pmatrix}
0 & -\id \\ \id & 0
\end{pmatrix}\graph(T)^\perp \\
&= \begin{pmatrix}
0 & -\id \\ \id & 0
\end{pmatrix}\begin{pmatrix}
0 & \id \\ \id & 0
\end{pmatrix}\graph(T)^\perp \\
&= \begin{pmatrix}
0 & -\id \\ \id & 0
\end{pmatrix}\left(\begin{pmatrix}
0 & \id \\ \id & 0
\end{pmatrix}\graph(T)\right)^\perp = \graph((T^{-1})^*).
\end{align*}
The penultimate equality follows from \ref{perpUnderIsometry}, using the fact that $\begin{pmatrix}
0 & \id \\ \id & 0
\end{pmatrix}$ is a surjective isometry.
\end{proof}

\subsection{Adjoints of bounded operators}
\begin{proposition}
Let $T: H\to K$ be a densely defined operator between Hilbert spaces. Then
\begin{enumerate}
\item if $T\in\Bounded(H,K)$, then $T^*\in\Bounded(K,H)$;
\item if $T^*\in\Bounded(K,H)$, then $T$ is bounded. If $T$ is closed, then $T$ is defined everywhere.
\end{enumerate}
Assume $T\in\Bounded(H,K)$. Then
\begin{enumerate} \setcounter{enumi}{2}
\item $\norm{T} = \norm{T^*}$;
\item $T^* = C_H^{-1}T^tC_K$, where $C_K$ is the Riesz isometry from \ref{RieszIsometry}.
\end{enumerate}
\end{proposition}
\begin{proof}
(1) Assume $T\in\Bounded(H,K)$. Then $u\mapsto \inner{x,Tu}$ is a bounded functional for all $x\in K$, so $\dom(T^*) = K$ by \ref{adjointDomain}. Also $T^*$ is closed by \ref{adjointGraph}, so it is bounded by the closed graph theorem \ref{BanachClosedGraphTheorem}.

(2) Assume $T^*\in\Bounded(K,H)$. By the previous argument $T \subseteq \overline{T} = T^{**}\in\Bounded(H,K)$.

(3) The function $(x,u)\mapsto \inner{x,Tu}$ is a bounded sesquilinear form. By proposition \ref{sesquilinearRepresentation}, $T^*$ must be the unique $S$ from the proposition, which has norm $\norm{T}$.

(4) Finally we note that $C_H^{-1}T^tC_K$ is an adjoint with domain $K$ and conclude by \ref{everywhereDefinedAdjointLemma}.
\end{proof}

\begin{lemma}
The adjoint defines a map $*:\Bounded(H,K)\to \Bounded(K,H)$ that is antilinear and continuous in the weak and uniform operator topologies. It is continuous in the strong operator topology \textup{if and only if} finite dimensional.
\end{lemma}
\begin{proof}
By the proposition the adjoint map is antilinear. It is also bounded with norm $1$. Then by corollary \ref{boundedAntiLinearMaps} it must be bounded.

TODO
\end{proof}

\begin{proposition}
Let $H,K$ be Hilbert spaces and $T:H\to K$ a bijective bounded linear operator with bounded inverse. Then $(T^*)^{-1}$ exists and
\[ (T^*)^{-1} = (T^{-1})^*. \]
\end{proposition}
\begin{proof}
We prove $(T^{-1})^*$ is both a left- and a right-inverse of $T^*$: $\forall x\in H, y\in K$
\begin{align*}
\inner{T^*(T^{-1})^*x,y} &= \inner{x,T^{-1}Ty} = \inner{x,y} \\
\inner{x,(T^{-1})^*T^*y} &= \inner{TT^{-1}x,y} = \inner{x,y}
\end{align*}
So, by lemma \ref{elementaryOrthogonality}, $T^*(T^{-1})^* = \id_H$ and $(T^{-1})^*T^* = \id_K$.
\end{proof}

\begin{proposition} \label{normOfSquare}
Let $T\in \Bounded(H,K)$ with $H,K$ Hilbert spaces. Then
\[ \norm{T^*T}= \norm{T}^2 = \norm{TT^*}. \]
\end{proposition}
\begin{proof}
For $\norm{T^*T}= \norm{T}^2$ first observe that
\[ \norm{T^*T} \leq \norm{T^*}\cdot\norm{T} = \norm{T}^2. \]
Conversely, $\forall x\in H$:
\[ \norm{T(x)}^2 = \inner{Tx,Tx} = \inner{T^*Tx,x} \leq \norm{T^*Tx}\cdot \norm{x} \leq \norm{T^*T}\cdot\norm{x}^2. \]
The other equality follows by applying the first to $T^*$ and using $\norm{T^*}=\norm{T}$.
\end{proof}

\chapter{Operators on convergence vector spaces}

\section{Continuous operators}

\section{Discontinuous linear operators}
\subsection{Closed operators}
\begin{definition}
Let $T:\dom(T)\subseteq X\to Y$ be an operator. Then $T$ is a \udef{closed operator} if $\graph(T)$ is closed in $X\oplus Y$.
\end{definition}
This is not the same as a closed map in the convergence sense! It is studied in the convergence setting as well, see \ref{secFunctionsClosedGraph}.

\url{https://en.wikipedia.org/wiki/Unbounded_operator#Closed_linear_operators}
\url{https://en.wikipedia.org/wiki/Closed_graph_theorem_(functional_analysis)}

{closedGraphFunctionConstructions}

\begin{lemma} \label{algebraClosedOperators}
Let $T$ be a closed and $S$ a bounded operator, then $S+T$ is closed.
\end{lemma}
\begin{proof}
TODO
\end{proof}

\begin{lemma} \label{closedOperatorKernelClosed}
Let $T$ be a closed operator, then $\ker(T)$ is closed.
\end{lemma}
\begin{proof}
Let $F\in\powerfilters\big(\ker(T)\big)$ be a convergent filter that converges to $x$. Then $T^{\imf\imf}(F) = \pfilter{0}$ and thus converges to $0$. By closedness of $T$, $Tx = 0$ and thus $x\in\ker(T)$. 
\end{proof}
\begin{proof}[Alternate proof.]
Let $T: X\to Y$. Then $\ker(T)\times\{0\} = \graph(T)\cap X\times\{0\}$. As $\graph(T)$ is closed and $X\times\{0\}$ is closed by \ref{productOpenClosed}, we have that $\ker(T)$ is closed by \ref{productOpenClosed}.
\end{proof}
We have already proven this for bounded operators, see \ref{kernelClosed}.

\section{Convergences on spaces of continuous operators}
\subsection{Strong operator topology}
\begin{definition}
Let $\sSet{V,\xi}, \sSet{W,\zeta}$ be a convergence vector spaces. The pointwise convergence on $\contLin(V,W)$ is called the \udef{strong operator convergence} in this case.
\end{definition}

\subsection{Weak operator topology}
\begin{definition}
Let $\sSet{V, \xi}, \sSet{W,\zeta}$ be convergence vector spaces. Consider the pairing $\sSet{\dual{W}\otimes V, \contLin(V,W), \pair{\cdot, \cdot}}$, where $\pair{f\otimes v, T} = f(Tx)$.

Then the \udef{weak operator topology} (or WOT) on $\contLin(V,W)$ is the weak topology $\sigma\big(\dual{W}\otimes V, \contLin(V,W)\big)$.
\end{definition}
\begin{lemma}
Let $\sSet{V, \xi}$ be a convergence vector space and $\sSet{W,\zeta}$ a Hausdorff locally convex convergence vector space. Then $\sSet{\dual{W}\otimes V, \contLin(V,W), \pair{\cdot, \cdot}}$ is a dual system. 
\end{lemma}
\begin{proof}
First take $T\in \contLin(V,W)\setminus \{0\}$. Then there exists $v\in V$ such that $Tv \neq 0$. By \ref{locallyConvexDualPair} there exists $f\in \dual{W}$ such that $f(Tv) \neq 0$, i.e. $\pair{f\otimes v, T} \neq 0$.

Now take $\sum_{i=1}^nf_i\otimes v_i \in \dual{W}\otimes V$. By \ref{tensorProductLinearlyIndependentBasis} we can take both the $f_i$s and the $v_i$s linearly independent. Since $\sum_{i=1}^n f_i \neq 0$, there exists $v\in V$ such that $\sum_{i=1}^n f_i(v) \neq 0$. Since $\{v_i\}_{i=1}^n$ is linearly independent, can construct a linear operator $T$ that maps all $v_i$ to $v$. Then $\pair{\sum_{i=1}^nf_i\otimes v_i, T} \neq 0$.
\end{proof}

\begin{lemma}
Let $\sSet{V, \xi}, \sSet{W,\zeta}$ be convergence vector spaces. Then the weak operator topology on $\contLin(V,W)$ is weaker than the strong topology.
\end{lemma}
\begin{proof}
Let $F\in\powerfilters\big(\contLin(V,W)\big)$ be a filter that converges in then strong topology. Take $f\otimes v\in \dual{W}\otimes V$. Then $\pair{f\otimes v,\cdot}^{\imf\imf}(F) = (f\circ \evalMap_v)^{\imf\imf}(F)$ converges since the strong topology makes $\evalMap_v$ continuous.
\end{proof}

\begin{lemma}
Let $\sSet{V, \xi}, \sSet{W,\zeta}$ be convergence vector spaces. Then the weak operator topology is the topology of pointwise convergence on $\contLin\big(V, \sSet{W, \sigma{\dual{W}, W}}\big)$.
\end{lemma}

\section{Compact operators}
\begin{definition}
A linear operator $T:X\to Y$ between TVSs is \udef{compact} if it maps a neighbourhood of the origin to a precompact set, i.e.\ 
\[ \exists U \in \neighbourhood(0): \;  \text{$\overline{T[U]}$ is compact.} \]
The set of compact linear operators in $(X\to Y)$ is denoted $\Compact(X,Y)$.
\end{definition}

\begin{lemma} \label{compactOperatorEquivalents}
Let $X$ be a normed space and $Y$ a TVS and $T:X\to Y$ a linear operator. Then the following are equivalent:
\begin{enumerate}
\item $T$ is a compact operator;
\item there exists a neighbourhood $U \subset X$ of the origin and a compact set $V\subset Y$ such that $T[U] \subset V$;
\item the image of the unit ball of $X$, $T[B(\vec{0},1)]$, is precompact in $Y$;
\item the image of any bounded set in $X$ is precompact in $Y$.
\end{enumerate}
If $Y$ is a normed space, these are also equivalent to
\begin{enumerate} \setcounter{enumi}{4}
\item for any bounded sequence $(x_{n})_{n\in \mathbb{N}}$ in $X$, the sequence $(Tx_{n})_{n\in \mathbb{N} }$ contains a converging subsequence.
\end{enumerate}
\end{lemma}
\begin{proof}
TODO
\end{proof}


\begin{lemma}
Let $X,Y$ be TVSs.
\begin{enumerate}
\item Then $\Compact(X, Y)$ is a vector space.
\item If $X,Y$ are normed spaces, then $\Compact(X, Y)$ is a subspace of $\Bounded(X, Y)$.
\end{enumerate}
\end{lemma}
\begin{proof}
(1) Let $K,K':X\to Y$ be compact operators. Then, by \ref{closureGroupOperation} (TODO opposite inclusion!),
\[ \overline{K[B(0, 1)]+K'[B(0, 1)]} \subseteq \overline{K[B(0, 1)]}+\overline{K'[B(0, 1)]}, \qquad \overline{K[\lambda B(0, 1)]} = \lambda\overline{K[B(0, 1)]}. \]

(2) Let $K\in\Compact(X, Y)$. Then the image of the unit ball is precompact, meaning it is bounded. So $K$ is bounded by \ref{existenceOperatorNorm}.
\end{proof}

\begin{lemma}
Let $T:V\to W$ be a bounded operator. If $W$ has the Heine-Borel property, then $T$ is compact.
\end{lemma}
\begin{proof}
The set $T[B(\vec{0},1)]$ is bounded because $T$ is. By the Heine-Borel (TODO ref) property of $W$, $\overline{B(\vec{0},1)}$ is compact.
\end{proof}
\begin{corollary}
Bounded operators with as image a finite dimensional normed space are compact.
\end{corollary}
\begin{corollary}
The identity on a normed space $X$ is compact \textup{if and only if} $X$ is finite-dimensional.
\end{corollary}
\begin{proof}
TODO ref. 
\end{proof}

\begin{proposition}
Compact operators map weakly convergent sequences to strongly convergent sequences. TODO! + remove from Hilbert section.
\end{proposition}
\begin{corollary} \label{limitCompactImageOrthonormalSequence}
Let $V$ be an inner product space and $\seq{e_n}$ a sequence of orthonormal vectors in $V$. If $K$ is a compact operator, then $\lim_{n\to\infty}Ke_n = 0$.
\end{corollary}
\begin{proof}
Any sequence of orthonormal vectors $\seq{e_n}$ converges weakly to $0$. Because $K$ is compact, $\seq{Ke_n}$ converges strongly to zero. TODO ref.
\end{proof}
\begin{corollary}
If $V$ is infinite-dimensional and $K$ is invertible, then its inverse is unbounded.
\end{corollary}
\begin{proof}
Due to $\lim_{n\to\infty}Ke_n = 0$ the operator $K$ cannot be bounded below, so $K^{-1}$ is not bounded by \ref{boundedBelow}.
\end{proof}

