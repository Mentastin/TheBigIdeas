\section{Abstract machines}
\begin{definition}
An \udef{abstract machine} is a function $M: I\times T \to S$,where
\begin{itemize}
\item $I$ is a set of possible inputs;
\item $T$ is a set that is supposed to index the progress of the machine; suppose it is a complete lattice;
\item $S$ is a set of possible states of the machine;
\end{itemize}
such that there exists a \udef{transition function}
\[ \mathcal{T}_M: (T\times T)\times S \to S \]
with the property that
\[ \forall i\in I: \forall t \geq t'\in T: \quad M(i,t) = \mathcal{T}_M(t,t'; M(i,t')). \]
We will often work with the partial application
\[ U_M(t,t') \defeq \mathcal{T}_M(t,t'; -), \]
which is also called a \udef{transition function}.

We call the machine \udef{serial} if $T$ is totally ordered.
\end{definition}

\url{https://en.wikipedia.org/wiki/Concurrency_(computer_science)}

\subsection{Halting time and the halting problem}
\begin{definition}
Let $M: I\times T \to S$ be an abstract machine, $A \subseteq S$ a subset of states called \udef{accepted states} and $i\in I$ an input.

A \udef{halting time} is a minimal element of $\setbuilder{t\in T}{M(i,t) \in A}$. We denote the set of such minimal elements by $h(i, A, M)$.
\end{definition}

TODO we want next step to depend only on previous.

\subsection{Complexity}
\subsubsection{Time complexity}
\begin{definition}
Let  $M: I\times T \to S$ be an abstract machine, $A \subseteq S$ a subset of accepted states and $i\in I$ an input.
ra
Then the \udef{time complexity} of $M$ with input $i$ is just the halting time $h(i,A, M)$.
\end{definition}

TODO also a priori halting times.

\subsubsection{Space complexity}
\begin{definition}
Let  $M: I\times T \to S$ be an abstract machine, $A \subseteq S$ a subset of accepted states and $i\in I$ an input.

Let $c_S: S\to \R^+$ be a function called the \udef{(spatial) cost function}.

The \udef{space complexity} of $M$ with input $i$ is defined as
\[ \max_{t\leq h(i,A,M)} c_S(M(i,t)). \]
\end{definition}

\subsubsection{Worst case, average case, best case}
TODO

\subsection{Building machines}
\subsubsection{Serial algorithms}
\begin{definition}
Let $\{M_j: I_j\times T_j \to S_j\}_{j\in 1:N}$ be a set abstract machines and $\{f_j: S_{j-1}\times T_{j-1} \to I_{j}\}_{j\in 2:N}$ a set of functions.

Then we can define a new abstract machine $M: I\times T\to S$ with $I= I_1$, $T = \bigtimes_{j\in 1:N} T_j$ and $S = S_N$ as follows:
\begin{multline*}
M: I\times T \to S: (i, \seq{t_j}_{j\in 1:N}) \mapsto \\
\Big(M_N(-, t_N) \circ f_{N}(-, t_{N-1}) \circ \ldots \circ M_2(-, t_2) \circ f_2(-, t_1) \circ M_1(-, t_1)\Big)(i)
\end{multline*}

We define the partial machines $M_{j\to k}: I_j\times T \to S_k$ by
\begin{multline*}
M_{j\to k}: I_j\times T \to S_n: (i, (t_j)_{j\in 1:N}) \mapsto \\
\Big(M_k(-, t_k) \circ f_{k}(-, t_{k-1}) \circ \ldots \circ M_{j+1}(-, t_{j+1}) \circ f_{j+1}(-, t_j) \circ M_j(-, t_j)\Big)(i)
\end{multline*}

We let $\seq{t_j}_{j\in 1:N}$ be a halting time if
\[ \forall j\in 1:N: \quad \text{$t_j$ is a halting time of $M_j(f_j(M_{1\to j-1}(i, t_{j-1}), t_{j-1}), -)$}. \]

We depict the machine $M$ as
\begin{centeredAlgorithm}
\KwIn{i}
$M_1(i)$\;
$M_2(f_2(s))$\;
\Dots
$M_N(f_N(s))$\;
\end{centeredAlgorithm}

The $s$ in the $j^\text{th}$ row refers to the state $M_{1\to j}(i, t)$, i. e.\ the last state produced by the $(j-1)^\text{th}$ machine. Sometimes we leave the $f_j(s)$ implicit.
\end{definition}


\subsubsection{Parallel and non-deterministic algorithms}

\subsubsection{Subroutines}
TODO: what product structure??
\begin{definition}
Let $M: I\times T\to S$ be a serial abstract machine, where $S$ is a state space that admits a product $\otimes$. 
If $M$ is of the form
\begin{centeredAlgorithm}
\KwIn{i}
$N_1$\;
$s\otimes R$\;
$N_2$\;
\end{centeredAlgorithm}
for some serial abstract machines $N_1, R, N_2$, then we say $R$
is a \udef{subroutine} of $M$ and $M$ \udef{calls} $R$.
\end{definition}

We may also write a subroutine in a machine as follows:

\begin{centeredAlgorithm}
\KwIn{i}
$N_1$\;
\Subroutine{$R$\;}
$N_2$ \; 
\end{centeredAlgorithm}

\begin{algorithm}[H]
%\SetLine
\KwData{this text}
\KwResult{how to write algorithm with \LaTeX2e }
initialization\;
\While{not at end of this document}{
read current\;
\eIf{understand}{
go to next section\;
current section becomes this one\;
}{
go back to the beginning of current section\;
}
}
\caption{How to write algorithms}
\end{algorithm}

\subsubsection{If-then-else blocks}
\subsubsection{While loops}
\subsubsection{For loops}

\subsection{Recursion}
\begin{definition}
An abstract machine $M$ is called \udef{recursive} if it calls itself as a subroutine.
\end{definition}

Defined this way, recursion is a property of machines. We would also like to use it to define machines.

\subsubsection{Recursive definition of machines}


\subsubsection{Space complexity}
\subsubsection{Tail-call optimisation}

\section{Automata}
\begin{definition}
An \udef{automaton} is an abstract machine $M: I\times T \to S$ such that
\begin{itemize}
\item $I$ is the Kleene closure $\Sigma^*$ of some finite set $\Sigma$, called the \udef{input alphabet};
\item $T = \N$;
\item $M$ is of the form $(i,n)\mapsto \left(\prod_{i=0}^{\len(i)}\delta(\cdot, i(n))\right)(s_0)$ for some
\begin{itemize}
\item \udef{transition function} $\delta: S \times \Sigma \to S$;
\item \udef{initial state} $s_0$.
\end{itemize}
\end{itemize}
We also fix a set of accepted states $A\subseteq S$.
\end{definition}
Note that each successive transitition function call receives the next letter from the input word.

\section{Models of computation}
Stack machine (0-operand machine)
Accumulator machine (1-operand machine)
Register machine (2,3,... operand machine)

Expressive power

\url{http://cs.brown.edu/people/jsavage/book/pdfs/ModelsOfComputation.pdf}

\subsection{Sequential models}
\subsubsection{Finite state machines}
\subsubsection{Pushdown automata}
\subsubsection{Turing machine}
\subsection{Register machines}
\subsubsection{Counter machine}
\subsubsection{Pointer machine}
\subsubsection{Random-access machine (RAM)}
\subsubsection{Random-access stored-program machine model (RASP)}
\subsection{Functional models}
\subsubsection{Lambda calculus}
\subsubsection{Recursive calculus}
\subsubsection{Combinatory logic}
\subsubsection{Abstract rewriting systems}
\subsection{Concurrent models}
\subsubsection{Cellular automata}
\subsubsection{Kahn process networks}
\subsubsection{Petri nets}
\subsubsection{Synchronous data flow}
\subsubsection{Interaction nets}
\subsubsection{Actor model}
\section{Computability theory}
\section{Computational complexity}
Depends on model of computation. TODO Cell-probe model