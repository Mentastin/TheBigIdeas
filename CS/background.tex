\chapter{What is a computer?}
What is data? Also binary
Turing machine: undecidability, the halting problem, 
Lambda calculus
Abacus
Von Neumann / computer architecture (+infra)
Church-turing thesis (+ materialism?)
Minimal working computer

\chapter{Organization of computer systems}
A computer system can be divided roughly into four components:
\begin{enumerate}
\item The hardware;
\item The operating system;
\item The application programs;
\item The users.
\end{enumerate}

\section{Hardware}
\subsection{The CPU}
assembly / machine language

address of execution / program counter

\subsection{Storage}
Pointers

\subsubsection{Design parameters}
\begin{itemize}
\item Speed (read / write)
\item Size
\item Volatility
\item Ability to be written to (yes - RW / no / limited - WORM)
\end{itemize}

\subsubsection{Types of storage}
\paragraph{ROM}
Also EEPROM. Firmware.
\paragraph{Registers}
\paragraph{Cache}
\paragraph{Main memory (RAM)}
\paragraph{Secondary memory}
\paragraph{External (tertiary) storage}

\subsection{I/O devices and controllers}

\section{Computer-system architecture}
TODO: what is it? Configuration of pieces of hardware.
\subsection{System design}
\subsubsection{Computing environments}
\paragraph{Traditional computing}
\paragraph{Mobile computing} Includes GPS, accelerometers etc.
\paragraph{Distributed systems} Networks characterised based on distance between nodes:
\begin{itemize}
\item Personal-area network (PAN)
\item Local-area network (LAN)
\item Metropolitan-area network (MAN)
\item Wide-area network (WAN)
\end{itemize}
(Mostly LAN - WAN)
\begin{itemize}
\item Client-server computing
\item Peer-to-peer. Needs discovery protocol.
\end{itemize}
\paragraph{Cloud computing} Public or private: for anyone willing to pay or company internal.
\begin{itemize}
\item Software as a service (SaaS): one or more applications available via the internet.
\item Platform as a service (PaaS): a software stack ready for application use via the internet.
\item Infrastructure as a service (IaaS): servoers or storage available over the internet.
\end{itemize}
\paragraph{Real-time embedded systems}

\subsubsection{Building systems}
\paragraph{Interrupts}
from hardware or software (system call = monitor call)

Interrupts must be handled quickly. Generally, a table of pointers to the interrupt service routines is stored in low memory. This is called the \udef{interrupt vector} and is indexed by a unique device number, given with the interrupt request.

The address of the interrupted instruction must also be saved. Older systems simply saved this at a fixed location. More modern architectures store the return address on the system stack. If the interrupt routine needs to modify the processor state (for example by modifying register values) it must explicitly save the current state.

After the interrupt is serviced, the saved return address is loaded into the program counter and the interrupted computation resumes.
\paragraph{Firmware}
Bootstrap program to get computer started.

\subsubsection{Some architectures}
\paragraph{Single-processor systems}
+ use of special-purpose microprocessors (in disk / keyboard / \ldots)
\paragraph{Multiprocessing}
\udef{Multipocessor systems}, also known as \udef{parallel systems} or \udef{tightly coupled systems}, have two or more processors in close communication, sharing the computer bus and sometimes the clock, memory, and peripheral devices.

Multiprocessor systems have three main advantages:
\begin{enumerate}
\item \textbf{Increased throughput}. More processors means more computational power. The computational power does not scale linearly with the number of processors, though. There is some overhead due to process coordination. Also competition for resources such as memory can occur.
\item \textbf{Economy of scale}. A multiprocessor system is typically cheaper than multiple single-processor systems, because peripherals can be shared.
\item \textbf{Increased reliability}. Failure of a single processor does not need to halt the whole operation. \begin{itemize}
\item The ability to continue providing service proportional to the level of surviving hardware is called \udef{graceful degradation}.
\item If a system has enough redundancy that operation is unaffected by failure, it is called \udef{fault tolerant}.
\end{itemize}
\end{enumerate}

There are two main types of multi-processor systems:
\begin{enumerate}
\item In systems using \udef{asymmetric multiprocessing} there is a master CPU that coordinates the slave CPUs.
\item In \udef{symmetric multiprocessing (SMP)} there is no such hierarchical relationship. All CPUs are equals.
\end{enumerate}

Multiprocessor systems may have \textit{uniform memory access (UMA)} or \textit{non-uniform memory access (NUMA)}. With UMA accessing any RAM from any CPU takes the same amount of time.

Multiprocessing may also be achieved by including multiple computing \textbf{cores} on a single chip. This has the advantage of fast on-chip communication and a minimal increase in power requirements.

\udef{Blade servers} multiple processor, I/O and networking boards in the same chassis.

\paragraph{Clustered systems}
Clustered systems consist of several individual systems, or \udef{nodes}, joined together. The definition of the term \textit{clustered} is not concrete.

Clustering is often used to provide \textit{high-availability} or \textit{high-performance}.

Applications must be specifically written to take advantage of the cluster. This is called \udef{parallelisation}.

Clusters may \textbf{symmetric} or \textbf{asymmetric}. In asymmetric clustering, one machine is in hot-standby mode, ready to jump in if the active server fails.


\begin{example}
Beowulf clusters are clusters built using commodity hardware using a set of open source packages for high-performance computing tasks.
\end{example}
\subsubsection{Direct memory access (DMA)}
\subsection{Instruction set architecture (ISA)}
RISC, CISC
\subsection{Microarchitecture}
Von Neumann?

\section{Operating systems}
\begin{definition}
An \udef{operating system} is a program that manages the computer hardware. It acts as an intermediary between the user and the hardware. The purpose of of an operating system is to provide an environment in which a user can execute programs in a \textit{convenient} and \textit{efficient} manner.
\end{definition}
An operating system may be thought of as a control program and as a \textit{resource allocator}.

There are many different types of computer systems with different design goals. For this reason we have no completely adequate definition of what an operating system really is or exactly which programs are part of the operating system and which are not. A simple approach is to call everything a vendor ships the ``operating system''.

The software a vendor ships can be split into three categories:
\begin{enumerate}
\item The \udef{kernel} which is the one program running at all times on the computer;
\item Systems programs, which are associated with the operating system, but are not part of the kernel;
\item Application programs, which include all programs not associated with the operation of the system.
\end{enumerate}
Often people use the term ``operating system'' to refer specifically to the kernel.

ALSO: Microsoft case for bundling IE.

Now: \udef{middleware}, especially in mobile OSs: software that provides services to software applications beyond those available from the operating system, such as for databases, multimedia and graphics.

Operating systems typically provide a set of services that are helpful to the user:
\begin{itemize}
\item A user interface (UI). This may be graphical (GUI) or text based.
\item Program execution.
\item I/O operations
\item File-system manipulation
\item Communication between processes. E.g. via shared memory or message passing.
\item Error detection. For each type of error, the operating system should take the appropriate action to ensure correct and consistent computing.
\end{itemize}
Another set of functions ensures the efficient operation of the system:
\begin{itemize}
\item Resource allocation.
\item Accounting, i.e. keeping track of which users use which computing resources.
\item Protection and security.
\end{itemize}
Some relevant design parameters for operating systems include
\begin{itemize}
\item Ease of use;
\item Performance;
\item Fair resource allocation;
\item Maximisation of resource utilisation;
\item Real-time responsiveness;
\item Battery life.
\end{itemize}

\subsection{Interfacing with the system}
\subsubsection{Command interpreter}
Some operating systems include the command interpreter in the kernel, but many, like Windows and UNIX, treat the command interpreter as a special program that is running when a user first logs on. On systems with multiple command interpreters to choose from, the interpreters are known as \udef{shells}.

The user operates the command interpreter by typing in commands. Some command interpreters then jump to code execute the command issued, using the appropriate system calls. An alternative approach (used by UNIX among others) is to let the interpreter call a relevant system program. This way new commands can very easily be added to the system.

\subsubsection{Graphical user interfaces}
TODO skeuomorph
\paragraph{WIMP and post-WIMP interfaces}
\paragraph{Window managers}
\begin{itemize}
\item Compositing window managers
\item Stacking window managers
\item Tiling window managers
\item Dynamic window managers
\end{itemize}
\paragraph{Desktop environments}
\paragraph{Mobile interfaces}
gestures

\subsection{CPU allocation}
\subsubsection{Multiprogramming and time sharing}
In general a single program cannot keep the CPU at all times. Also a single user generally is running multiple programs simultaneously. The solution is \udef{multiprogramming}: the system keeps several \textit{jobs} in memory at all times. The operating system picks a job to execute for a while until it needs to wait for some task (such as an I/O operation) to complete. At that point the operating system switches to a different job from the job pool.

A program loaded into memory and executing is called a \udef{process}.

\udef{Time sharing} or \udef{multitasking} operating systems switch jobs often enough that the user can interact with each program while it is running.

A \udef{time-shared} operating system allows many users to share the computer simultaneously.

\subsubsection{Timer}
The operating system must keep control of the CPU in order to be able to provide multitasking. A user program must always give control of the CPU back to the operating system and may not enter an infinite loop or just fail and not return control to the operating system.

To ensure proper operation, a \udef{timer} is set which sends an interrupt to the CPU to hand control back to the operating system. Before giving control to a user program, the operating system makes sure to set the timer.

\subsection{Dual-mode operation}
Clearly the operating system needs to be able to perform operations that the user's programs may never, such as setting the timer or modifying the job pool and scheduling information.

The approach taken by most computer systems is to provide hardware support that allows us to differentiate among various modes of execution. A bit, called the \udef{mode bit} is added to indicate the current mode: \udef{user mode} or \udef{kernel mode} (also called supervisor mode, system mode or privileged mode).

Some of the machine instructions are \udef{privileged instructions} that may only be executed in kernel mode. Examples include
\begin{itemize}
\item I/O control;
\item time management;
\item interrupt management.
\end{itemize}
System calls provide the means for a user program to ask the operating system to perform tasks reserved for the operating system on the user program's behalf.

I/O must be reserved for the operating system because otherwise a user program may try to overwrite the operating system or other programs or two programs may try writing to a device at the same time.

\subsection{System calls}
System calls provide an interface to the services made available by an operating system. These calls are generally available as routines written in C and C++, although certain low-level tasks (for example, tasks where hardware must be accessed directly), may need to be written using assembly-language instructions.

\subsection{System programs}
system processes or system deamons: provide services outide the kernel. run entire time kernel is running?

\subsection{Storage management}
One of the major ways operating systems provide resource management is by abstracting the physical properties of its storage devices to define a logical storage unit, the \udef{file}.

A file is a collection of related information defined by its creator. Files are used to represent both data and programs. Data files may be binary or text-based.

The operating system controls who can access which files in what way.

Caching. Cache management. Cache coherency. TODO OSv9 p28

\chapter{Abstraction and mathematics}