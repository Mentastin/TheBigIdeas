\paragraph{History}
First specified in 1972. SWI-Prolog offers a comprehensive free Prolog environment.
\paragraph{Concepts}
\begin{itemize}
\item A Prolog database consists of known instances of relations, called \udef{facts}.
\item New relations can be constructed from old ones. These are expressed as \udef{rules}.
\item \udef{Queries} are expressions containing one or more variables.
\item \undline{Unification} is used to determine whether a query has a valid substitution consistent with the rules and facts in the database.
\item \udef{Clauses} are sets of statements.
\end{itemize}

\begin{itemize}
\item A subprogram (called a \udef{predicate}) represents a state of the world.
\item A command (called a \udef{goal}) tells Prolog to make that state of the world true, if possible.
\end{itemize}

\paragraph{Basic setup}
Code entered in interactive mode and code in a file is treated differently. Facts and rules should be put in a file.

The interactive prompt is \texttt{?-} and lines end with a period.


\paragraph{Syntactic elements}
Comments:
\begin{lstlisting}[language=prolog, style=snippet]
% This is a comment
\end{lstlisting}
Facts:
\begin{lstlisting}[language=prolog, style=snippet]
magicNumber(7).
magicNumber(9).
magicNumber(42).
\end{lstlisting}
which we can query:
\begin{lstlisting}[language=prolog, style=snippet]
?- magicNumber(7).                   % True
?- magicNumber(8).                   % False
?- magicNumber(9).                   % True
\end{lstlisting}

Multiple operations can be chained using commas.

\paragraph{Unification}
We request unification by passing an undefined variable:
\begin{lstlisting}[language=prolog, style=snippet]
?- magicNumber(Presto).              % Presto = 7 ;
                                     % Presto = 9 ;
                                     % Presto = 42.
\end{lstlisting}
The equals sign represents unification. We have three cases.
\begin{enumerate}
\item If both sides are bound (i.e., defined), Prolog checks equality.
\item If one side is free (i.e., undefined), Prolog tries to assign the variable to match the other side.
\item If both sides are free, the assignment is remembered.
\end{enumerate}
Attempted unification can have three outcomes. It can
\begin{enumerate}
\item Succeed (return True) without changing anything, because an equality-style unification was true;
\item Succeed (return True) and bind one or more variables in the process; or
\item Fail (return False) because an equality-style unification was false (failure can never bind variables).
\end{enumerate}


The equals sign can not do arithmetic (as Prolog cannot solve equations out of the box). The \texttt{is} operator does allow arithmetic, but right side must always be bound.
\begin{lstlisting}[language=prolog, style=snippet]
?- X = 3+2.             % X = 3+2 - unification can't do arithmetic
?- X is 3+2.            % X = 5 - "is" does arithmetic.
?- 5 = X+2.             % This is why = can't do arithmetic -
                        % because Prolog can't solve equations
?- 5 is X+2.            % Error. Unlike =, the right hand side of IS
                        % must always be bound, thus guaranteeing
                        % no attempt to solve an equation.
\end{lstlisting}
We can however reverse addition if we try to unify with the \texttt{plus} predicate.
\begin{lstlisting}[language=prolog, style=snippet]
?- plus(1, 2, 3).                    % True
?- plus(1, 2, X).                    % X = 3 because 1+2 = X.
?- plus(1, X, 3).                    % X = 2 because 1+X = 3.
?- plus(X, 2, 3).                    % X = 1 because X+2 = 3.
?- plus(X, 5, Y).                    % Error - infinite solutions
\end{lstlisting}

TODO: more