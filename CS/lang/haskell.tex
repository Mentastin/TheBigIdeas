\paragraph{History.}

\paragraph{Concepts.}
Haskell has \udef{lazy evaluation}; it only evaluates things when it needs to.

\paragraph{Basic setup.}
Haskell can be run in a REPL (Read-Eval-Print Loop). The REPL can be started with the command \texttt{ghci}.

In the REPL, new values are created with \texttt{let}.
\begin{lstlisting}[language=haskell, style=snippet]
let foo = 5
\end{lstlisting}
Type can be inspected using \texttt{:t}.
\begin{lstlisting}[language=haskell, style=snippet]
> :t foo
foo :: Integer
\end{lstlisting}
Additional information on any identifier is given by \texttt{:i}:
\begin{lstlisting}[language=haskell, style=snippet]
> :i (+)
class Num a where
  (+) :: a -> a -> a
  ...
    -- Defined in 'GHC.Num'
infixl 6 +
\end{lstlisting}


\paragraph{Syntactic elements.}
Comments:
\begin{lstlisting}[language=haskell, style=snippet]
-- Single line comments start with two dashes.
{- Multiline comments can be enclosed
in a block like this.
-}
\end{lstlisting}
Lines end with a newline character.

\paragraph{Primitive data types and operators.}
\subparagraph{Numbers.}
Math is a you expect. Division is floating point by default. Integer division done using \texttt{`div`}.
\begin{lstlisting}[language=haskell, style=snippet]
1 + 1 -- 2
8 - 1 -- 7
10 * 2 -- 20
35 / 4 -- 8.75
35 `div` 4 -- 8
\end{lstlisting}
\subparagraph{Booleans.} The primitives \texttt{True} and \texttt{False} are capitalised. Operations:
\begin{lstlisting}[language=haskell, style=snippet]
not True -- False
not False -- True
1 == 1 -- True
1 /= 1 -- False
1 < 10 -- True
\end{lstlisting}
Here \texttt{not} is actually a function.
\subparagraph{Strings and characters.} Strings are lists of characters.
\begin{lstlisting}[language=haskell, style=snippet]
"This is a string."
'a' -- character
'You cant use single quotes for strings.' -- error!

['H', 'e', 'l', 'l', 'o'] -- "Hello"
"This is a string" !! 0 -- 'T'
\end{lstlisting}

Function identifiers do not need to contain letters.
\begin{lstlisting}[language=haskell, style=snippet]
(//) a b = a `div` b
35 // 4 -- 8
\end{lstlisting}
\paragraph{Lists and tuples.}
\subparagraph{Lists.}
Every element in a list must have the same type. Ranges can be used and are versatile.
\begin{lstlisting}[language=haskell, style=snippet]
[1, 2, 3, 4, 5]
[1..5]              -- [1, 2, 3, 4, 5]
['A'..'F']          -- "ABCDEF"
[0,2..10] -- [0, 2, 4, 6, 8, 10]
[5..1] -- [] (Haskell defaults to incrementing)
[5,4..1] -- [5, 4, 3, 2, 1]
\end{lstlisting}
Indexing is zero-based and done using \texttt{!!}.
\begin{lstlisting}[language=haskell, style=snippet]
[1..10] !! 3 -- 4
\end{lstlisting}
Thanks to lazy evaluation the following is possible:
\begin{lstlisting}[language=haskell, style=snippet]
[1..] -- a list of all the natural numbers
[1..] !! 999 -- 1000
\end{lstlisting}
List operations:
\begin{lstlisting}[language=haskell, style=snippet]
-- joining two lists
[1..5] ++ [6..10]

-- adding to the head of a list
0:[1..5] -- [0, 1, 2, 3, 4, 5]

head [1..5] -- 1
tail [1..5] -- [2, 3, 4, 5]
init [1..5] -- [1, 2, 3, 4]
last [1..5] -- 5
\end{lstlisting}
List comprehension:
\begin{lstlisting}[language=haskell, style=snippet]
[x*2 | x <- [1..5]] -- [2, 4, 6, 8, 10]
[x*2 | x <- [1..5], x*2 > 4] -- [6, 8, 10]
\end{lstlisting}

\subparagraph{Tuples.} Every element in a tuple can be a different type, but a tuple has a fixed length. Tuple literals are written with parentheses.
\begin{lstlisting}[language=haskell, style=snippet]
("haskell", 1)

-- accessing elements of a pair (i.e. a tuple of length 2)
fst ("haskell", 1) -- "haskell"
snd ("haskell", 1) -- 1

-- pair element accessing does not work on n-tuples (i.e. triple, quadruple, etc)
snd ("snd", "can't touch this", "da na na na") -- error! see function below
\end{lstlisting}

\paragraph{Functions.}
Declaring and calling functions.
\begin{lstlisting}[language=haskell, style=snippet]
add a b = a + b
add 1 2 -- 3
\end{lstlisting}
Haskell also supports infix notation.
\begin{lstlisting}[language=haskell, style=snippet]
1 `add` 2 -- 3
\end{lstlisting}

Branching can be achieved with guards.
\begin{lstlisting}[language=haskell, style=snippet]
fib x
  | x < 2 = 1
  | otherwise = fib (x - 1) + fib (x - 2)
\end{lstlisting}

\subparagraph{Overloading.}
\begin{lstlisting}[language=haskell, style=snippet]
fib 1 = 1
fib 2 = 2
fib x = fib (x - 1) + fib (x - 2)
\end{lstlisting}

\subparagraph{Pattern matching.}
\begin{itemize}
\item On tuples:
\begin{lstlisting}[language=haskell, style=snippet]
sndOfTriple (_, y, _) = y
\end{lstlisting}
\item On lists:
\begin{lstlisting}[language=haskell, style=snippet]
myMap func [] = []
-- x is the first element of the list. 
-- xs is the rest of the list. 
myMap func (x:xs) = func x:(myMap func xs)
\end{lstlisting}
\end{itemize}

\subparagraph{Anonymous functions.} Anonymous functions are created with a backslash followed by all the arguments.
\begin{lstlisting}[language=haskell, style=snippet]
myMap (\x -> x + 2) [1..5] -- [3, 4, 5, 6, 7]
\end{lstlisting}

\subparagraph{Partial application.}
\begin{lstlisting}[language=haskell, style=snippet]
add a b = a + b
foo = add 10 -- foo is now a function that takes a number and adds 10 to it
foo 5 -- 15
-- Another way to write the same thing
foo = (10+)
foo 5 -- 15
\end{lstlisting}

\subparagraph{Function composition.} Function composition is achieved with the \texttt{.} operator.
\begin{lstlisting}[language=haskell, style=snippet]
foo = (4*) . (10+)
foo 5 -- 60 because 4*(10+5) = 60
\end{lstlisting}

\subparagraph{Operator precedence.} The \texttt{\$} operator applies a function to a given parameter. It is low priority and is right-associative. The expression on its right is applied as a parameter to the function on its left.


\subparagraph{Built in functions}

Another \texttt{map} example.
\begin{lstlisting}[language=haskell, style=snippet]
map (*2) [1..5] -- [2, 4, 6, 8, 10]
\end{lstlisting}

foldr, foldl

\paragraph{Type signatures and data types.}
Haskell has a very strong type system, and every valid expression has a type.

Some basic types:
\begin{lstlisting}[language=haskell, style=snippet]
5 :: Integer
"hello" :: String
True :: Bool
\end{lstlisting}

When you define a value, it's good practice to write its type above it:
\begin{lstlisting}[language=haskell, style=snippet]
doubleInt :: Integer -> Integer
doubleInt x = x * 2
\end{lstlisting}

Custom types can be defined.
\begin{lstlisting}[language=haskell, style=snippet]
data Color = Red | Blue | Green
\end{lstlisting}
The type is \texttt{Color} and its possible values are \texttt{Red}, \texttt{Blue} and \texttt{Green}.
Data types can have parameters as well.
\begin{lstlisting}[language=haskell, style=snippet]
data Maybe a = Nothing | Just a
-- These are all of type Maybe
Just "hello"    -- of type `Maybe String`
Just 1          -- of type `Maybe Int`
Nothing         -- of type `Maybe a` for any `a`
\end{lstlisting}

\paragraph{Flow control.}
\subparagraph{\texttt{if}-expressions.}
\begin{lstlisting}[language=haskell, style=snippet]
haskell = if 1 == 1 then "awesome" else "awful" -- haskell = "awesome"
\end{lstlisting}
Multiline \texttt{if}. Indentation is important.
\begin{lstlisting}[language=haskell, style=snippet]
haskell = if 1 == 1
            then "awesome"
            else "awful"
\end{lstlisting}
\subparagraph{\texttt{case} expressions.}
\begin{lstlisting}[language=haskell, style=snippet]
case args of
  "help" -> printHelp
  "start" -> startProgram
  _ -> putStrLn "bad args"
\end{lstlisting}
\subparagraph{Recursion.} Haskell does not have any loops, but we can make them using the \texttt{map} function.
\begin{lstlisting}[language=haskell, style=snippet]
for array func = map func array
for [0..5] $ \i -> show i -- Using the for loop.
\end{lstlisting}


\paragraph{Monads.}
TODO

\paragraph{I/O.}
TODO
