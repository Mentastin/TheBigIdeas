\chapter{Hardware}
\section{Screws}
\chapter{Devices}
\section{Power supplies}
\subsection{Bench power supply}
Output: three terminals: positive, negative and ground. Ground should be isolated from other terminals.
\section{Oscilloscopes}
The main function of an oscilloscope is the measure voltage in function of time. Time is on the horizontal axis.

The scope will sweep along the time axis. Triggering resets the horizontal sweep at a particular location each time a particular event, the trigger event, occurs.
Typically a trigger event is when the signal passes a certain threshold and is either rising or falling.
\subsection{Probes}
Usually BNC connector on one end and other end has measuring tip and alligator clip for reference.
\subsubsection{Isolation}
It is important that the circuit being measured is completely isolated from the oscilloscope. If this is not the case and the ground clip is connected to a part of the circuit that is not ground, then this will complete a low resistance path to ground, causing much current to flow.

Usually oscilloscopes probes are connected to the ground of the grid. So the circuit being tested should be isolated from the grid! It is done this way round for a couple of reasons: as soon as the scope is connected to something else, e.g. to a computer via USB, then it is no longer floating. Also floating a scope may leave it with charge that will discharge through the next user.

If the circuit being measured cannot be isolated, then isolated probes can be used. There are also devices for galvanically isolating USB connections. Remember that a circuit connected by USB is not isolated!
\subsubsection{Types}
There exist many types of probes. These include passive probes, active probes, differential probes, current probes, logic probes, high voltage probes, optical probes and isolated. Most common are passive probes.
\paragraph{Passive probes} Passive are distinguished by their attenuation factor: usually 1X, 10X or 100X. Probes may be switchable: it can be both 1X and 10X or 10X and 100X.

The attenuation is achieved by connecting a resistor in series inside the probe, such that the oscilloscope is measuring over a voltage divider.

WHY?

\subsection{Controls}
Most oscilloscopes have the following controls:
\begin{itemize}
\item Moving vertical position up and down.
\item Change scale of time and voltage axis. This is often labelled as seconds / division and volts / division. Some scopes have a button or a sensing device (Tektronix) to know whether the probe is 1X or 10X. All this does is change the label on the voltage axis.
\item Coupling: DC, AC and ground. DC shows the signal as is, AC translates the signal vertically such that the average is zero and ground disconnects the signal.
\item Setting trigger type.
\item XY mode: displays a second channel along the horizontal axis instead of time.
\item Autoset tries to find settings that show something of the signal. Very useful if you don't know why a signal is not showing up.
\end{itemize}
There will also be a square wave output to calibrate probes with compensation.
