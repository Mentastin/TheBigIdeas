\chapter{Introduction}
Condensed matter physics is the branch of physics that studies systems with a large number of constituents and where the constituents are close enough together (i.e. condensed enough) that the forces between them are strong.

Solid state physics studies the properties of rigid condensed matter, aka solids. In particular we will be discussing crystalline solids in this chapter. Amorphous solids and glasses require a very different treatment and will be discussed in the chapter on soft matter physics.
\chapter{Crystal structure}
Crystals are three-dimensional periodic arrays of atoms. This may include atoms of different types. In this section we will develop the language and mathematics necessary to describe these periodic arrays. This is the language of crystallography.

The essential characteristic of crystals is that they are periodic. Hopefully it is somewhat intuitively obvious what that means, but we would also like to also express it in a more rigorous, mathematical way.

\section{Periodicity and crystal structure}

Periodicity means that if we add a fixed quantity an integer number of times, the resulting operation is a symmetry. In other words there is a vector $\vec{a}$ such that the transformation $\vec{r} \to \vec{r'}$ with
\[ \vec{r'} = \vec{r} + n_1\vec{a} \]
is a symmetry for any integer $n_1$. This is obviously a form of translational symmetry. There may also be other symmetries that have the same form. Let us assume for example that there is a different vector $\vec{b}$ that has the same property as the vector $\vec{a}$. Applying the two symmetry operations will still give a symmetry operation
\[ \vec{r} \to \vec{r'} = \vec{r} + n_1 \vec{a} + n_2 \vec{b}. \]
Conversely if we set $n_1$ or $n_2$ to $0$, we get the original symmetry groups.

Now it may be that $\vec{b}$ is a multiple of $\vec{a}$, i.e. $\vec{b} = \alpha \vec{a}$. First we consider the case where $\alpha$ is a rational number. This means we can write $\alpha$ as $\frac{m}{n}$ where $m$ and $n$ are integers that have no common divisors other than one. In this case translating over an integer multiple of $\frac{\vec{a}}{n}$ is also a symmetry (this fact is not entirely trivial and follows from Bézout's identity). In other words, there is a vector $\vec{a'}$ such that
\[ \begin{cases}
\vec{a} = n \vec{a'} \\
\vec{b} = m \vec{a'}
\end{cases} \]
and all translations of the form
\[ \vec{r'} = \vec{r} + n_1\vec{a'} \]
are symmetries.

If $\alpha$ is \textit{not} a rational number, then we can find arbitrarily small translations that are still symmetries (this follows from the fact that $\alpha$ can be approximated by $\frac{m}{n}$ with $n$ arbitrarily large while the greatest common divisor of $m$ and $n$ is still one). This would mean that there is continuous translational symmetry, i.e. the transformation
\[ \vec{r} \to \vec{r'} = \vec{r} +  \lambda\vec{a}\]
is a symmetry for any real $\lambda$. This should not fit with your idea of periodicity and thus we require $\alpha$ to be rational.

So we can assume $\vec{a}$ and $\vec{b}$ are linearly independent (otherwise we can just use $\vec{a'}$ instead). We can now compose this with a third symmetry. Following exactly the same reasoning we get
\[ \vec{r} \to \vec{r'} = \vec{r} + n_1 \vec{a} + n_2 \vec{b} + n_3 \vec{c} \]
with $\vec{a}, \vec{b}$ and $\vec{c}$ linearly independent. In three dimensional space we can only ever find three linearly independent vectors, so we need look no further. Any translational symmetry (that is not continuous) can be written in this form. These operations are called \udef{lattice translation operations}.

The vectors $\vec{a}, \vec{b}$ and $\vec{c}$ are called the \udef{fundamental translation vectors}; they define a \udef{lattice} which is a regular periodic arrangement of points in space. The magnitude of the translation vectors, $a, b$ and $c$ are called the \udef{lattice constants}. The lattice and translation vectors are said to be \udef{primitive} if every translational symmetry can be obtained by choosing the correct $n_1, n_2$ and $n_3$. We have shown that it is alway possible to find primitive translation vectors.

The full crystal structure can now be described by specifying a unit, called a \udef{basis} that is repeated at every lattice point, which is what gives us periodicity. In summary
\remark{The crystal structure is specified by a lattice and a basis.}
TODO image

It may be remarked at this point that a crystal structure may be described by different lattices and bases. In particular if the lattice is not primitive, the basis must necessarily be larger. The basis associated with a primitive lattice may be called a primitive basis. No basis contains fewer atoms than a primitive basis.


\section{Lattice cells}
We can also divide the crystal into functionally identical regions of space called cells. A \udef{unit cell} will fill all space under the action of suitable lattice translation operations. A \udef{primitive cell} is a unit cell of minimum volume. The basis contained in a primitive cell is a primitive basis. There are two important types of primitive cells:
\begin{enumerate}
\item The parallelepiped defined by primitive axes $\vec{a}, \vec{b}$ and $\vec{c}$ is a primitive cell. It has a volume
\[ V_c = |\vec{a}\times \vec{b} \cdot \vec{c}|. \]
\item The \udef{Wigner-Seitz primitive cell} is defined by the procedure shown in figure TODO.
\end{enumerate}

Positions of points in a unit cell are specified in terms of \udef{atomic coordinates} $u, v, w$. The position vector is then given by
\[ \vec{r} = u \vec{a} + v \vec{b} + w \vec{c}. \] 

\section{Fundamental types of lattices}
Lattices may also exhibit other types of symmetry such as rotational and reflectional symmetry, not just translational. A \udef{lattice point group} is the collection of symmetry operations  which, when applied about a lattice point, leave the lattice invariant.

We can sort the lattices into types based on their point group symmetries. We each fundamental type of lattice is called a \udef{Bravais lattice}. It is important to note that these are the symmetries of the \textit{lattice}, not necessarily of the crystal structure. The crystal structure may have fewer symmetries if the basis does not have the symmetries the lattice has.

\subsection{Bravais lattices in two dimensions}
In this case there are two fundamental translation vectors, $\vec{a}$ and $\vec{b}$. Any of the infinite possible lattices are determined by the lengths $a$ and $b$ of the vectors along with the angle $\varphi$ between them.

TODO

\subsection{Bravais lattices in three dimensions}
TODO, conventional unit cell.


\section{Position and orientation of planes in crystals: Miller indies}
The position and orientation of planes in crystals are typically specified in terms of \udef{Miller indices}. The Miller indices $(hkl)$ denote the family of planes that intercept the three points $\vec{a}/h, \vec{b}/k, \vec{c}/l$, or some integer multiple thereof. If one of the indices is zero, it means that the planes are parallel to that axis (the intercept is ``at infinity''). A negative index is indicated by putting a bar above the index.

TODO: indices integers?? Meaning of $(200)$. GCD = 1. Through lattice points.

The set of planes that are equivalent to $(hkl)$ due to the symmetry of the bravais lattice is denote $\{hkl\}$. Again we must be careful to specify whether we are working with the lattice or the crystal structure. In the crystal the family $\{hkl\}$ will be smaller.

Directions can be denoted $[hkl]$. This means the direction of the vector
\[ h \vec{a} + k \vec{b} + l \vec{c}. \]

\section{Examples of simple crystal structures}
TODO

\chapter{Crystal diffraction and the reciprocal lattice}
Crystal diffraction is one of the main ways to investigate crystals. When X-ray diffraction was introduced in 1912 it decisively proved that crystals were comprised of periodically repeating units.
\section{Bragg law}
When an atom is exposed to electromagnetic radiation, the atomic electrons may scatter part or all of the radiation elastically, at the frequency of the incident radiation. At optical wavelengths the superposition of all the waves scattered elastically by the atoms in the crystal result in ordinary optical refraction as you would expect. If the wavelength of the incident radiation is comparable or smaller than the lattice constants, we might find one or more diffracted beams in directions quite different from the incident direction.

Lawrence Bragg presented an explanation of this phenomenon based on a very simple model. The model is so simplified in fact that it is only really credible because its result also follows from Laue's derivation. The model is as follows
\begin{enumerate}
\item Suppose the incident waves are reflected from parallel planes of atoms in the crystal. Any set of parallel planes will do, provided each plane passes through at least three non-collinear lattice points.
\item Suppose the reflection is specular (i.e. mirror-like). So we are briefly assuming geometrical optics for the reflection, but we must switch back to viewing the beam as wave-like immediately after.
\item Each plane only reflects a small fraction of the incoming beam.
\item The diffracted beams are only found when the reflections from parallel planes interfere constructively. If the planes are a distance $d$ apart, the difference in distance traveled by rays reflected off adjacent planes is $2d \sin \theta$. See figure TODO. In order to have constructive interference, this distance must be an integer multiple of the wavelength.
\end{enumerate}
Thus we get the \textbf{Bragg law}
\[ \boxed{2d\sin\theta = n\lambda} \]
where $\lambda$ is the wavelength of the incident beam.

Bragg reflection can only occur for wavelengths $\lambda \leq 2d$, which is why it is not observed with visible light.

\section{Experimental diffraction methods}
The Bragg law gives the locations of the peaks of diffraction in function of the wavelength $\lambda$ and angle $\theta$ of the incoming beam. In order to find those peaks experimentally we must scan a continuous range of either $\lambda$ or $\theta$, usually the latter.

The incident beam may be composed of x-rays, neutrons or, less commonly, electrons.
\subsection{Laue method}
TODO
\subsection{Rotating crystal method}
TODO
\subsection{Powder method}
TODO
\section{Laue derivation}
TODO: make the explanation less terrible

We now give a more careful treatment of crystal diffraction due to Laue. We now take the incident beam to be a plane wave that is not greatly disturbed by the crystal, neither by the refractive index of the crystal nor by the loss of energy through scattering. We assume the crystal structure is made up of identical point scattering centers at every lattice point. Each atom scatters a part of the wave elastically. We then sum all of these contributions to find a condition for when they interfere constructively.


The incident beam is a plane wave and may be expressed as
\[ F(\vec{x}) = F_0 \exp[i(\vec{k}\cdot \vec{x} - \omega t)] \]
with wavevector $\vec{k}$ and angular frequency $\omega$. We assume the origin is within the crystal.
We are calculating the diffraction condition in a stable state, so we may do the calculation at any time; for ease we consider the instant of time $t=0$. Say there is an atom in the crystal at a point $\vec{\rho}$. Then the incident beam has an amplitude
\[ F(\vec{\rho}) = F_0 \exp(i \vec{k}\cdot \vec{\rho}) \]
at location $\vec{\rho}$ and time $t=0$.

Now what does this scattered wave look like at a location $\vec{R}$ outside the crystal? The total phase factor is given by
\[ \exp(i \vec{k}\cdot \vec{\rho}) \exp(ikr) = \exp(i \vec{k}\cdot \vec{\rho} + ikr) \]
where $\vec{r}$ is the displacement from $\vec{\rho}$ to $\vec{R}$ (so $\vec{R} = \vec{\rho} + \vec{r}$). Here we only need the magnitude because the scattered wave expands spherically and thus $\vec{k}$ and $\vec{r}$ point in the same direction (see figure TODO).

The amplitude of the wave is may be taken to be proportional to $\frac{1}{r}$ and to the electron concentration in the volume element. Due to our assumptions about the crystal structure, we can assume the electron concentration to be the same for every atom.

We can write
\[ r^2 = (\vec{R} - \vec{\rho})^2 = R^2 + \rho^2 - 2\rho R \cos(\vec{\rho}, \vec{R}) \]
where $\cos(\vec{\rho}, \vec{R})$ is the cosine of the angle between the vectors $\vec{\rho}$ and $\vec{R}$.
If $R$ is much larger than the dimensions of the crystal, $\rho/R \ll 1$ and we can approximate $r$ as
\[ r \approx R[1-(2\rho / R)\cos(\vec{\rho}, \vec{R})]^{1/2} \approx R - \rho \cos(\vec{\rho}, \vec{R}) \]
Thus the phase factor becomes
\[ \exp[i \vec{k}\cdot \vec{\rho} + ikR - ik\rho\cos (\vec{\rho}, \vec{R})]. \]
The factor $\exp(ikR)$ is constant and thus does not change for waves scattered from different $\vec{\rho}$. The only part of the amplitude that depends on $\vec{\rho}$ is the factor $1/r$. For $R$ large enough we may take $1/r \approx 1/R$. Thus the amplitude is independent of $\vec{\rho}$.

The scattered wave has a wave vector $\vec{k'}$ with the same direction as $\vec{r}$, which is approximately the direction of $\vec{R}$ and because the scattering is elastic, $k' = k$. The phase factor can now be written
\[ \exp[i \vec{k}\cdot \vec{\rho} - ik\rho\cos (\vec{\rho}, \vec{R})] = \exp[i \vec{\rho}\cdot (\vec{k}- \vec{k'})] = \exp[-i \vec{\rho} \cdot \Delta \vec{k}] \]
where we define $\Delta \vec{k} \equiv \vec{k'} - \vec{k}$.

We have assumed the scattering centers to be at the lattice points, so
\[ \vec{\rho} = n_1 \vec{a} + n_2 \vec{b} + n_3 \vec{c}\]
Now the total scattered radiation amplitude seen at $\vec{R}$ is proportional to
\begin{align*}
\mathcal{A} \equiv \sum_{\vec{\rho}}\exp(-i\rho\cdot \vec{k}) = \left(\sum_{n_1}\exp[-in_1(\vec{a}\cdot \Delta\vec{k})]\right)\left(\sum_{n_2}\exp[-in_2(\vec{b}\cdot \Delta\vec{k})]\right)\left(\sum_{n_3}\exp[-in_3(\vec{c}\cdot \Delta\vec{k})]\right)
\end{align*}
The intensity is proportional to the square of the amplitude and thus proportional to
\[ |\mathcal{A}|^2 = \left|\sum_{n_1}\exp[-in_1(\vec{a}\cdot \Delta\vec{k})]\right|^2\left|\sum_{n_2}\exp[-in_2(\vec{b}\cdot \Delta\vec{k})]\right|^{2}\left|\sum_{n_3}\exp[-in_3(\vec{c}\cdot \Delta\vec{k})]\right|^2 \]

We now take a closer look at the first sum in this expression. We assume the crystal has a dimension $M a$ in the direction $\vec{a}$, where $M$ is an integer. Thus the sum becomes
\[ \sum^{M-1}_{n}\exp[-in(\vec{a}\cdot \Delta\vec{k})] \]
Using the series
\[ \sum_{n=0}^{M-1}x^n = \sum_{n=0}^\infty x^n - \sum_{n=M}^\infty x^n = \frac{1}{1-x} - \frac{x^M}{1-x} \]
with $x = \exp[-i(\vec{a}\cdot \Delta\vec{k})]$, we get
\begin{align*}
\sum^{M-1}_{n}\exp[-in(\vec{a}\cdot \Delta\vec{k})] &= \frac{1- \exp[-iM(\vec{a}\cdot \Delta\vec{k})]}{1- \exp[-i(\vec{a}\cdot \Delta\vec{k})]} \\
&= \frac{\exp[- \frac{1}{2} iM(\vec{a}\cdot \Delta\vec{k})]}{\exp[-i \frac{1}{2}(\vec{a}\cdot \Delta\vec{k})]}\cdot \frac{\exp[\frac{1}{2}iM(\vec{a}\cdot \Delta\vec{k})] - \exp[- \frac{1}{2}iM(\vec{a}\cdot \Delta\vec{k})]}{\exp[\frac{1}{2}i(\vec{a}\cdot \Delta\vec{k})] - \exp[- \frac{1}{2}i(\vec{a}\cdot \Delta\vec{k})]}
\end{align*}
Multiplying this by its complex conjugate, we get
\begin{align*}
\left|\sum_{n}\exp[-in(\vec{a}\cdot \Delta\vec{k})]\right|^2 &= \frac{\exp[\frac{1}{2}iM(\vec{a}\cdot \Delta\vec{k})] - \exp[- \frac{1}{2}iM(\vec{a}\cdot \Delta\vec{k})]}{\exp[\frac{1}{2}i(\vec{a}\cdot \Delta\vec{k})] - \exp[- \frac{1}{2}i(\vec{a}\cdot \Delta\vec{k})]}\cdot\frac{\exp[-\frac{1}{2}iM(\vec{a}\cdot \Delta\vec{k})] - \exp[\frac{1}{2}iM(\vec{a}\cdot \Delta\vec{k})]}{\exp[-\frac{1}{2}i(\vec{a}\cdot \Delta\vec{k})] - \exp[\frac{1}{2}i(\vec{a}\cdot \Delta\vec{k})]} \\
&= \frac{-1}{-1}\left(\frac{\cancel{\cos[\frac{1}{2}iM(\vec{a}\cdot \Delta\vec{k})]} + i\sin[\frac{1}{2}iM(\vec{a}\cdot \Delta\vec{k})] - \cancel{\cos[\frac{1}{2}iM(\vec{a}\cdot \Delta\vec{k})]} + i\sin[\frac{1}{2}iM(\vec{a}\cdot \Delta\vec{k})]}{\cancel{\cos[\frac{1}{2}i(\vec{a}\cdot \Delta\vec{k})]} + i\sin[\frac{1}{2}i(\vec{a}\cdot \Delta\vec{k})] - \cancel{\cos[\frac{1}{2}i(\vec{a}\cdot \Delta\vec{k})]} + i\sin[\frac{1}{2}i(\vec{a}\cdot \Delta\vec{k})]}\right)^{2} \\
&= \frac{\sin^2 \frac{1}{2}M(\vec{a}\cdot \Delta \vec{k})}{\sin^2 \frac{1}{2}(\vec{a}\cdot \Delta \vec{k})}.
\end{align*}
This function is steeply peaked with maxima whenever
\[ \vec{a}\Delta \cdot \vec{k} = 2\pi q \]
with $q$ an integer.
\subsection{Diffraction conditions}
Repeating the analysis for the directions $\vec{b}$ and $\vec{c}$, we obtain the \textbf{Laue equations}
\[ \boxed{\vec{a}\Delta \cdot \vec{k} = 2\pi q \qquad
\vec{b}\Delta \cdot \vec{k} = 2\pi r \qquad
\vec{c}\Delta \cdot \vec{k} = 2\pi s} \]
where $q,r,s$ are integers. We will later show that these equations are equivalent to the Bragg law.
\subsection{Width of the maximum}
To get an idea of the width of the peak, we note that the expression
\[ \sin^2 \frac{1}{2}M(2\pi q + x) \]
has a zero at $x = \frac{2\pi}{M}$. This is the closest zero to the peak. We conclude that the width of the peak is proportional to $1/M$.

\section{The reciprocal lattice}
It turns out the possible values of $\Delta \vec{k}$ also form a periodic lattice, namely the \udef{reciprocal lattice}:
\[ \Delta \vec{k} = q \vec{A} + r \vec{B} + s \vec{C} \]
where $q,r,s$ are integers and the fundamental vectors of the reciprocal lattice are given by
\[ \boxed{\vec{A} = 2\pi \frac{\vec{b}\times \vec{c}}{\vec{a}\cdot \vec{b}\times \vec{c}} \qquad
\vec{B} = 2\pi \frac{\vec{c}\times \vec{a}}{\vec{a}\cdot \vec{b}\times \vec{c}} \qquad
\vec{C} = 2\pi \frac{\vec{a}\times \vec{b}}{\vec{a}\cdot \vec{b}\times \vec{c}}} \]
It is obvious that any $\Delta \vec{k}$ of this form is a solution to the Laue equations (just fill it in and see that they are satisfied). The denominators make sense because the triple product is cyclic.

Conversely, any vector $\Delta k$ can be written as a linear combination of $\vec{A}, \vec{B}$ and $\vec{C}$:
\[ \Delta \vec{k} = \alpha \vec{A} + \beta \vec{B} + \gamma \vec{C} \]
we now need to show that $\alpha, \beta$ and $\gamma$ are integers. We illustrate this for $\alpha$. Using the fact that $\vec{a}\cdot \vec{A} = 2\pi$ and $\vec{a}\cdot \vec{ B}= \vec{a}\cdot \vec{C} = 0$, we write
\[ \vec{a} \cdot \Delta \vec{k} = \alpha \vec{a}\cdot \vec{A} = \alpha 2\pi. \]
From the Laue equations we see that $\alpha$ must be equal to the integer $q$ and is thus an integer.
\subsection{Interpretation and properties of the reciprocal lattice}
We write a generic lattice point as
\[ \vec{\rho} = m \vec{a} + n \vec{b} + p \vec{c} \qquad m, n, p \quad \text{integers} \]
and a generic reciprocal lattice point as
\[ \vec{G} = h \vec{A} + k \vec{B} + l \vec{C} \qquad h, k, l \quad \text{integers} \]
We note the following properties of the reciprocal lattice:
\begin{itemize}
\item Vectors $\vec{\rho}$ of the lattice have the dimensions of $[L]$. Vectors $\vec{G}$ of the reciprocal lattice have the dimensions of $[L^{-1}]$.
\item For any lattice points $\vec{\rho}$ and $\vec{G}$ the scalar product is
\begin{align*}
\vec{G}\cdot \vec{\rho} &= (h \vec{A} + k \vec{B} + l \vec{C})\cdot (m \vec{a} + n \vec{b} + p \vec{c}) \\
&= 2\pi (hm + kn  + lp) = 2\pi \times (\text{integer}).
\end{align*}
Hence also
\[ \exp(i \vec{G}\cdot \vec{\rho}) = 1. \]
\item A diffraction pattern is a map of the reciprocal lattice of the crystal, in contrast to a microscope image which is a map of the real crystal structure. (TODO: clarify!!)
\item The reciprocal lattice is a lattice in \textbf{Fourier space}. If we have a quantity (such as for example electron density) that is invariant under all crystal lattice translations, then the only values of $\vec{k}$ that contribute to the Fourier series are the lattice points $\vec{G}$.

To prove this, say we have a periodic quantity $n(\vec{r})$ such that $n(\vec{r}) = n(\vec{r}+\vec{\rho})$ for any crystal lattice vector $\vec{\rho}$. We write the Fourier series
\[ n(\vec{r}) = \sum_{\vec{k}} n_{\vec{k}}e^{i \vec{k}\cdot \vec{r}} = \sum_{\vec{k}} n_{\vec{k}}e^{i \vec{k}\cdot (\vec{r}+\vec{\rho})} = \sum_{\vec{k}} n_{\vec{k}}e^{i \vec{k}\cdot \vec{r}}e^{i\vec{k}\cdot\vec{\rho})} \]
So if we want the contribution associated with a wavevector $\vec{k}$ to be non-vanishing, we need $e^{i\vec{k}\cdot\vec{\rho})} = 1$ for all lattice point vectors $\vec{\rho}$. We have shown this to be the case for reciprocal lattice vectors $\vec{G}$. In fact it is only the case for reciprocal lattice vectors, because if we set $\vec{\rho} = \vec{a}$, $\vec{\rho} = \vec{b}$ and $\vec{\rho} = \vec{c}$ we get the Laue equations. Thus we can write the Fourier series as
\[ n(\vec{r}) = \sum_{\vec{G}} n_{\vec{G}}e^{i \vec{G}\cdot \vec{r}} \]
\item Every reciprocal lattice vector is normal to a lattice plane of the crystal lattice. In particular the vector
\[ \vec{G}(hkl) \equiv h \vec{A} + k \vec{B} + l \vec{C} \]
is normal to the plane with Miller indices $(hkl)$. In fact this fact can be used as an alternative definition of Miller indices. TODO proof.
\item TODO spacing of planes of crystal lattice plus equivalence of Bragg law.
\end{itemize}

\subsection{Ewald construction}
TODO + fig 23 Kittel.
\subsection{Brillouin zones}
The primitive cell formed in the reciprocal lattice according to the Wigner-Seitz method is called the \udef{first Brillouin zone}. This will be essential in the theory of electronic energy bands.
\subsection{Examples}
\subsubsection{Reciprocal lattice to an bcc lattice.} TODO
\subsubsection{Reciprocal lattice to an fcc lattice. } TODO

\section{Geometrical structure factor}
\subsection{Of the bcc lattice}
\subsection{Of the fcc lattice}
\section{Atomic scattering or form factor}
\section{Relating the crystal structure to the diffraction pattern}

\chapter{Crystal binding}
\section{Crystals of inert gasses}
\section{Ionic crystals}
\section{Covalent crystals}
\section{Metal crystals}
\section{Hydrogen-bonded crystals}
\section{Atomic radii}
TODO: nuclear physics?

\chapter{Lattice vibrations}
\section{Elasticity}
\subsection{Analysis of elastic strains}
\subsection{Elastic compliance and stiffness constants}
\section{Elastic waves in cubic crystals}
\subsection{Experimental determination of elastic constants}
\section{Phonons}
\subsection{Quantization of lattice vibrations}
\subsection{Phonon momentum}
\subsection{Inelastic scattering by phonons}
\subsubsection{Scattering of photons}
\subsubsection{Scattering of X-rays}
\subsubsection{Scattering of neutrons}
\subsection{Vibrations of monatomic lattices}
\subsection{Crystal with two atoms per primitive cell}
\subsection{Optical properties in the infrared}
\subsection{Local phonon modes}

\chapter{Thermal properties of insulators}
\section{Lattice heat capacity}
\section{Anharmonic crystal interactions}
\section{Thermal conductivity}

\chapter{Free electron Fermi gas}
Why does this work so well? \url{https://physicstoday.scitation.org/doi/10.1063/1.2995618}
\section{Energy levels and density of states in one dimension}
\section{Effect of temperature on the Fermi-Dirac distribution function}
\section{Free electron gas in three dimensions}
\section{Heat capacity of the electron gas}
\section{Electrical conductivity and Ohm's law}
\section{Thermal conductivity of metals}
\section{Electrical conductivity at high frequencies}
\section{Plasmons}
\section{Motion in magnetic fields}
\section{Cohesive energy and interatomic spacing of an idealized metal}
\section{Thermionic emission}

\chapter{Energy bands}
\section{Ansatz: the nearly free electron model}
We essentially treat the electrons as being confined to the solid and only weakly perturbed by the periodic potential of the ion cores.
\section{Wave equation of electron in a periodic potential}
\section{Approximate solution near a zone boundary}
\section{Number of states in a band}
\section{Construction of Fermi surfaces}
\section{Electrons, holes, and open orbits}
\section{Effective mass of electrons in crystals}
\section{Wavefunctions for zero wavevector}
\section{Orthogonalized plane waves}
\section{Experimental methods in fermi surface studies}

\chapter{Semiconductor crystals}
\section{Intrinsic conductivity}
\section{Band gap}
\section{Law of mass action}
\section{Intrinsic carrier concentration}
\section{Impurity conductivity}
\section{Impurity states}
\section{Thermal ionization of impurities}
\section{Energy bands in silicon and germanium}
\section{Carrier lifetime and recombination}
\section{$p-n$ junction rectification}
\section{Polarons}

\chapter{Superconductivity}

\chapter{Magnetism}
The magnetic moment of a free atom has three principal sources:
\begin{enumerate}
\item The spin of the electrons;
\item Orbital angular momentum of the electrons about the nucleus;
\item Change in orbital moment induced by an applied magnetic field.
\end{enumerate}
Nuclear magnetic moments are of the order of $10^{-3}$ times smaller than the magnetic moment of the electron.

\section{Magnetism in a free atom}
The \udef{magnetisation $M$} is the magnetic moment per unit volume.

The \udef{magentic susceptibility $\chi$} per unit volume is defined as
\[ \chi \equiv \frac{\mu_0 M}{B} \]
where $B$ is the macroscopic magnetic field intensity.
\begin{itemize}
\item Substances with a \textit{negative} magnetic susceptibility are called \udef{diamagnetic}.
\item Substances with a \textit{positive} magnetic susceptibility are called \udef{paramagnetic}.
\end{itemize}
\subsection{Diamagnetism}
\subsubsection{Classical treatment: the Langevin result}
\subsubsection{A quantum treatment in mononuclear systems}
\subsection{Paramagnetism}

\section{Ordered arrays of magnetic moments}
\subsection{Ferromagnetism}
\subsection{Ferrimagnetism}
\subsection{Antiferromagnetism}
\section{Magentic resonance}


\chapter{Applying an electric field}
\section{Dielectric properties}
\section{Ferroelectric crystals}

\chapter{Optical phenomena in insulators}
\chapter{Point defects}
\chapter{Dislocations}