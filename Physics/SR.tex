The physics we have described so far is what is known as classical physics and all the major conceptual ideas had been stated in some form or other by the end of the nineteenth century (Maxwell's equations were first published between 1861 and 1862). A story that is often repeated, including in introductions to quantum mechanics textbooks, is that there was a general idea shared by many eminent physicists of the time that the grand underlying principles of physics had all been discovered. Admittedly based on a very minimal amount of research, I am not at all sure this is true. The following quote is from a lecture by Lord Kelvin at the Royal Institution on April 27, 1900.

\begin{displayquote}
The beauty and clearness of the dynamical theory, which asserts heat and light to be modes of motion, is at present obscured by two clouds. I. The first came into existence with the undulatory theory of light, and was dealt with by Fresnel and Dr. Thomas Young; it involved the question, how could the earth move through an elastic solid, such as essentially is the luminiferous ether? II. The second is the Maxwell–Boltzmann doctrine regarding the partition of energy.
\footnote{From a 1900, April 27, Royal Institution lecture. “Lord Kelvin, Nineteenth Century Clouds over the Dynamical Theory of Heat and Light”, reproduced in Notices of the Proceedings at the Meetings of the Members of the Royal Institution of Great Britain with Abstracts of the Discourses, Volume 16, p. 363–397 and Philosophical Magazine, Sixth Series, 2, 1–40 (1901).}
\end{displayquote}
This is often paraphrased in a way that suggests that Lord Kelvin thought basically all of physics was solved except these two small clouds. In reality he almost certainly realized how big and (in his words) dense these clouds were.

The first one prompted the development of special (and later general) relativity. The second cloud required quantum mechanics to be resolved. Investigations into these new areas of physics produced lots of interesting new physics and the question of how to unite these two areas is still an open one.

We will now have a closer look at the first of the two clouds.

\chapter{Origins and justification}
TODO: review after chapter on EM

Before the theory of relativity, electric and magnetic fields were thought of as strains in an invisible medium called the (luminiferous) \udef{ether}. Light waves were thought to be waves propagating through this medium, like sound waves through air.

This would imply that depending on the speed we are traveling relative to this ether, electromagnetic experiments would give different results. Looking at the laws of electrodynamics, this would seem to be correct: for example, a charge in motion produces a magnetic field, whereas a charge at rest does not; several electrodynamic laws even make explicit reference to \emph{the} velocity of the charge.

Thus we might expect there to be a unique stationary reference frame, with respect to which all velocities are to be measured. In other words we expect the principle of relativity we saw in classical mechanics to not hold true in electrodynamics. Also, it is of utmost importance to \emph{find} this reference frame, otherwise all our calculations become invalid.

One might suppose that finding this unique stationary reference frame would be easy. It is the only reference frame in which Maxwell's equations are valid. Therefor surely we need only conduct a simple electromagnetic experiment; any discrepancy between the experimental findings and predictions based on Maxwell's equations would indicate movement with respect to the ether frame.

Unfortunately this proved to be quite difficult. Consider for example the following simple experiment:

\begin{example}
[TODO: better example]

Suppose we put a wire loop on a train. This train then rides at constant velocity between the poles of a giant magnet.

As the loop passes through the magnetic field, an electromotive force is established according to the flux rule (TODO reference eq.),
\[ \mathcal{E} = - \od{\Phi}{t}. \]
If we now look at the situation on board the train. There would be no magnetic force, because the loop is at rest. As the magnet flies by, it would induce an electric field, according to Faraday's law (TODO reference eq). The resulting electric force would generate the following electromotive force in the loop:
\[ \mathcal{E} = - \od{\Phi}{t}. \]
Because Faraday's law and the flux rule predict exactly the same electromotive force, people in both reference frames will get the same answer, even though the physical interpretation of the process is completely different.
\end{example}

The problem with the example experiment above proved quite general. When changing (inertial) reference frame, the physical interpretation changed, but ``coincidences'' would conspire to make the values of measurable observables the same. So it takes an uncommonly delicate experiment to do the job (TODO?).

\section{Michelson-Morley experiment}
This brings us to the famous Michelson-Morley experiment. Conducted between April and July 1887 by Albert Michelson and Edward Morley.

One of the predictions of classical electrodynamics is that electromagnetic waves travel through the vacuum at a speed
\[ \frac{1}{\sqrt{\epsilon_0\mu_0}} = 3.00 \times 10^8 \si{m/s} \]
relative (presumably) to the ether.

Michelson and Morley used an interferometer to measure the speed of light in different directions. If we were moving through the ether, then we would expect light to be slower in the direction we were traveling in. Unfortunately Michelson and Morley, despite going to great lengths to achieve maximum precision, were not able to detect any difference in speed. This presented quite a problem. Over the years many unsatisfactory theories were devised to explain why our movement with respect to the ether was undetectable by the Michelson-Morley experiment, such as ``ether drag'' and various so-called ``emission'' theories.

It was not until Einstein that anybody took the result of the Michelson-Morley experiment at face value and suggested that the speed of light is a universal constant.

\section{Einstein's postulates}
Inspired by the Michelson-Morley experiment and the fact that applying the laws of electrodynamics in different inertial reference frames yields the same values for experimental observables, Einstein proposed his two famous postulates:
\begin{enumerate}
\item \textbf{The principle of relativity}: The laws of physics apply in all inertial reference systems.
\item \textbf{The universal speed of light}: The speed of light in a vacuum, $c$, is the same for all inertial observers, regardless of the motion of the source.
\end{enumerate}
(TODO: second postulate redundant?)
Thus the principle of relativity, which in classical mechanics is an observation, is elevated to the status of a general law. The second postulate asserts that there is no ether. This was radically new and has some strange consequences.

TODO: name lightspeed misleading, no object, no signal, nothing faster than the speed of light!!!


\section{Gedankenexperimente and phenomena}
If we accept Einstein's postulates, we are forced to accept that some strange phenomena can occur. In this section we explore thought experiments exhibiting some of the most striking new phenomena.
\subsection{Velocity addition}
In classical mechanics, if one walks at \SI{7}{km/h} towards the front of a train going \SI{100}{km/h}, one's speed with respect to the ground \SI{107}{km/h}. This result was so obvious that until Einstein nobody has bothered to give it a name. Einstein called it \udef{Galileo's velocity addition rule}. In general we can write it as follows 
\[ \vec{v}_{AB} = \vec{v}_{AB} + \vec{v}_{BC} \]
where $\vec{v}_{AB}$ represents the velocity of $A$ relative to $B$.

If $A$ is a light signal, this rule cannot hold, as Einstein's second postulate states that
\[ v_{AC} = v_{AB} = c \]
regardless of $v_{BC}$.
Instead we have \udef{Einstein's velocity addition rule} (assuming movement along one axis):
\[ v_{AC} = \frac{v_{AB}+v_{BC}}{1+(v_{AB}v_{BC}/c^2)}.\]
We will derive this properly later as well as give a more complete treatment of relativistic kinematics, but for now we will just make a couple of remarks:
\begin{enumerate}
\item For speeds encountered in everyday life (i.e. $v_{AB} \ll c$ and $v_{BC} \ll c$), the denominator is so close to one that the difference between the two velocity addition formulae is negligible.
\item If $v_{AB} = c$, then we automatically have that $v_{AC} = c$:
\[ v_{AC} = \frac{c+v_{BC}}{1+(cv_{BC}/c^2)} = c. \]
\end{enumerate}

\subsection{The relativity of simultaneity}
If we accept Einstein's postulates we are forced to abandon the idea of objective, universal simultaneity. Two events that happen simultaneously according to one observer, do not necessarily happen at the same time according to a different one. Even more strangely, if according to one observer one event happens before another one, there may be a different observer in a different reference frame that sees the second event happen before the first.

\begin{example}
Imagine a freight train, traveling a constant speed along a smooth, straight track (TODO image). In the centre of the train there hangs a light bulb. At a certain time the light is switched on. The light leaves the light bulb at speed $c$. An observer in the same reference frame as the train will see the light reach both ends of the train at the same time, because the bulb is hanging in the centre. Thus the two events in question: light reaching the front end and light reaching the back end, happen simultaneously.

To an observer on the ground these events do not happen simultaneously. As the light travels backwards (still at lightspeed), the train moves forward and thus the light moving backwards has a shorter distance to travel. The observer on the ground observes the light reach the back \emph{before} it reaches the front.
\end{example}
Conclusion:

\begin{center}
\textbf{Two events that are simultaneous in one inertial system are not, in general, simultaneous in another.}
\end{center}

\subsection{Time dilation}
This effect can be summed up as follows:

\begin{center}
\textbf{Moving clocks run slow.}
\end{center}

\begin{example}
Consider the same setup as before. We now want to know how long it takes for the light from the lamp to reach the floor of the train. For an observer in the train the answer is straightforward:
\[ t_\text{train} = \frac{h}{c} \]
where $t_\text{train}$ is the time it takes the light to reach the floor and $h$ is the height of the train.

According to the ground observer however the light has to travel $\sqrt{h^2+ (v\cdot t_\text{ground})^2}$, see figure TODO. Thus we get
\[ t_\text{ground} = \frac{\sqrt{h^2+ (v\cdot t_\text{ground})^2}}{c} \]
where $v$ is the speed the freight train is traveling at and $t_\text{ground}$ is the time it takes the light to reach the floor according to an observer on the ground.

Solving this equation for $t_\text{ground}$, we get
\[ t_\text{ground} = \frac{h}{c}\frac{1}{\sqrt{1-v^2/c^2}} = t_\text{train}\frac{1}{\sqrt{1-v^2/c^2}} = t_\text{train}\gamma \]
Where we have introduced the factor
\[ \gamma \equiv \frac{1}{\sqrt{1-v^2/c^2}} \]
\end{example}
This result is true in general. Clocks moving at a speed $v$ will run a factor
\[\gamma = \frac{1}{\sqrt{1-v^2/c^2}}\]
slower according to a stationary observer compared to the time measured by a comoving observer.

One might think this leads to a paradox. In the above example the freight train is moving with respect to the ground, so an observer on the ground observes time on the train as moving slower. Conversely the ground is moving with respect to the freight train and thus an observer on the freight train observes time on the ground as moving slower.

This may seem contradictory, but it actually is not. The trick is that the two observers are actually measuring different things. (TODO: explanation?)

\subsubsection{Twin paradox.} Imagine we have twins. One of them becomes an astronaut and flies off in a rocket. Because that twin is flying fast, they will experience time dilation, meaning that from the point of view of twin who stayed at home time moves more slowly on the space ship. Thus when the astronaut returns home, they will be younger than the twin who stayed on earth.

One might be tempted to think that this reasoning can be reversed. From the point of view of the astronaut the earth is moving and so time is moving more slowly on earth, meaning that the earth bound twin would be younger. This line of reasoning is false however: the twin on the ground stays in a (roughly) inertial reference frame, whereas the astronaut has to accelerate and decelerate, thus the traveling twin is not a stationary observer.


\subsection{Length contraction}
This effect, also known as Lorentz contraction, can be summed up as follows:

\begin{center}
\textbf{Moving objects are shortened.}
\end{center}
\begin{example}
Imagine now our lamp is at the back of the freight train and we have set up a mirror at the front. How long does it take light to travel from the back of the train to the mirror in the front and beck to where it started?

To an observer on the train the answer is
\[ t_\text{train} = 2 \frac{l_\text{train}}{c} \]
where $l_\text{train}$ denotes the length of the train as measured by an observer on the train.

To an observer on the ground the process is more complicated.

TODO figure

We call the time, as observed by an observer on the ground, for the light signal to reach the mirror $t_\text{ground,1}$ and the return time $t_\text{ground,2}$.
These quantities obey the following equations:
\[ t_\text{ground,1} = \frac{l_\text{ground} + v t_\text{ground,1}}{c}, \qquad t_\text{ground,2} = \frac{l_\text{ground} - v t_\text{ground,2}}{c}\]
Solving for $t_\text{ground,1}$ and $t_\text{ground,2}$ we get
\[ t_\text{ground,1} = \frac{l_\text{ground}}{c-v}, \qquad t_\text{ground,2} = \frac{l_\text{ground}}{c+v} \]
so the round trip time is
\[ t_\text{ground} = t_\text{ground,1} + t_\text{ground,2} = 2\frac{l_\text{ground}}{c}\frac{1}{1-v^2/c^2} \]
We have already derived the relationship between time intervals:
\[ t_\text{train} = \sqrt{1-v^2/c^2}t_\text{ground} \]
We conclude that
\[l_\text{train} = \frac{1}{\sqrt{1-v^2/c^2}}l_\text{ground}\]
\end{example}
Lorentz contraction works very much like time dilation, except moving rulers are shorter. Many of the same remarks apply.

\subsubsection{Dimensions perpendicular to the velocity} Objects are only shortened along the direction of its motion! We have so far taken this fact for granted. We will now justify it with another thought experiment.

\begin{example}
Imagine there is a wall next to the train tracks with a blue line \SI{5}{m} above the surface, as measured by an observer on the ground. As the train goes by, a passenger paints a red line \SI{5}{m} above the ground, as measured by that passenger on the train. The question now is whether the red line is above or below the blue one.

Were there contraction in the vertical direction, an observer on the ground would say that the red line is lower and the passenger would say that the blue line is lower. They cannot both be right, thus we conclude there cannot be any contraction along the vertical axis.
\end{example}

\subsubsection{Ehrenfest's paradox.} TODO

\chapter{The mathematics of special relativity}
Having discussed some phenomena of special relativity, we now want to derive a more precise mathematical description.

So far we have talked about events, such as light hitting a particular part of the train, without rigorously defining the concept. An \udef{event} is something that takes place at a specific location $(x,y,z)$ and at a specific time $(t)$. The coordinates $(x,y,z,t)$ of an event are only specified relative to a reference frame $\mathcal{S}$.
\section{The Lorentz transformations}
Suppose we know the coordinates $(x,y,z,t)$ of an event $E$ in an inertial reference frame $\mathcal{S}$ and we would like to know what the coordinates $(\bar{x}, \bar{y}, \bar{z}, \bar{t})$ of $E$ are in a different inertial reference frame $\bar{\mathcal{S}}$.

We can orient the axes of the frames so that the spatial axes are parallel and $\bar{\mathcal{S}}$ is moving away from $\mathcal{S}$ along the $x$ axis at a speed $v$.

We start time (i.e. set $t=0$) at the moment the origins of both frames coincide. Then, according to classical theory, we can transform coordinates from $\mathcal{S}$ to $\bar{\mathcal{S}}$ using the \udef{Galilean transformations}:
\[ \begin{cases}
\bar{x} = x - vt \\
\bar{y} = y \\
\bar{z} = z \\
\bar{t}= t
\end{cases} \]
Until Einstein these were considered self-evident. Based on the thought experiments of the previous section, we know enough to replace these transformations with the \udef{Lorentz transformations}:
\[ \begin{cases}
\bar{x} = \gamma(x - vt) \\
\bar{y} = y \\
\bar{z} = z \\
\bar{t}= \gamma\left(t- \frac{v}{c^2}x\right)
\end{cases} \]
The inverse transformations can be found by switching the sign of $v$ (or by solving for $x$ and $t$):
\[ \begin{cases}
x = \gamma(\bar{x} - v\bar{t}) \\
y = \bar{y} \\
z = \bar{z} \\
t= \gamma\left(\bar{t}- \frac{v}{c^2}\bar{x}\right)
\end{cases} \]

The critical information to remember when deriving these equations is
\begin{enumerate}
\item $l_\text{train} > l_\text{ground}$;
\item $t_\text{train} < t_\text{ground}$;
\item Time and space are scaled by a factor $\gamma > 1$.
\end{enumerate}

\subsection{Einstein velocity addition} TODO
\section{The structure of spacetime}
TODO: coordinate with section on mathematics

In this section we introduce some mathematical methods to make it easier to deal with Lorentz transformations. 

\subsection{Four-vectors}
In order to simplify notation we introduce the dimensionless quantity
\[ \beta \equiv \frac{v}{c}. \]
We then define an object with four components (labeled with subscripts) which we call a \udef{four-vector}:
\[ x = (x^0,x^1,x^2, x^3) \equiv (ct, x,y,z) \]

Notice that we rescale the time dimension with $c$. This means we measure $x^0$ in units of length, like the other components, so we can easily compare and combine all components.

The Lorentz transformation considered above becomes:
\[ \begin{cases}
\bar{x}^0 = \gamma(x^0 - \beta x^1) \\
\bar{x}^1 = \gamma(x^1 - \beta x^0) \\
\bar{x}^2 = x^2 \\
\bar{x}^3 = x^3
\end{cases} \quad \text{or, in matrix form}\quad \begin{pmatrix}
\bar{x}^0 \\ \bar{x}^1 \\ \bar{x}^2 \\ \bar{x}^3
\end{pmatrix} = \begin{pmatrix}
\gamma & -\gamma\beta & 0 & 0 \\
-\gamma\beta & \gamma & 0 & 0 \\
0 & 0 & 1 & 0 \\
0 & 0 & 0 & 1
\end{pmatrix} \begin{pmatrix}
x^0 \\ x^1 \\ x^2 \\ x^3
\end{pmatrix} \]
Letting the Greek indices run from  $0$ to $3$, we can now write this more compactly as
\[ \bar{x}^\mu = \sum^3_{\nu=0}\left(\Lambda^\mu_{\;\nu}\right)x^\nu \]

\subsubsection{Consequences of Einstein summation} The formula above for $\bar{x}^\mu$ may be considerably simplified if we use the Einstein summation convention.

TODO: guaranteed and manifestly in-/covariant?

TODO: \udef{4-vector} (spatial part in bold) transforms under Lorentz transformations. Minkowsky product invariant, covariant, contravariant, Einstein summation

TODO: Greek $0$ to $3$ and Latin $1$ to $3$.

TODO: raise / lower indices with $\eta$.

TODO: merge with mathematics.
 
\subsection{The invariant interval} TODO: show that the interval stays invariant under Lorentz transformations. use that to define Lorentz transformations. -> like fourth type of rotation. Add translation Poincaré. Any transformation $x^\mu \to x^{\prime\mu}$ that preserves length given by metric $\eta$ is Poincaré:
\[ x^{\prime\mu} = \tensor{\Lambda}{^\mu_\nu}x^\nu + \epsilon^\mu \] 
\subsection{Space-time diagrams}
The invariant interval between causally related events is timelike, and their temporal ordering is the same for all inertial observers.

TODO Hartle 4.3 world line

\subsubsection{Conformal transformations}
Leave speed of light invariant

\subsection{Light cones}
spacelike, null, timelike separated

\subsection{Differing conventions}
mostly minus and mostly plus

\section{Symmetries}
What are symmetries of Minkowski space? What transformations leave $\diff{s}^2$ invariant?
\begin{itemize}
\item Translations
\item Rotations
\[ \Lambda^\intercal \eta \Lambda = \eta \qquad \eta_{\rho\sigma} \]
Analogous to
\[ R^\intercal R = \mathbb{1} \]
Lorentz group $\O(3,1)$. 
\end{itemize}
Examples of $\Lambda$:
\[ \Lambda = \begin{pmatrix}
1 & 0 & 0 & 0 \\
0 & \cos\theta & \sin\theta & 0 \\
0 & -\sin\theta & \cos\theta & 0 \\
0 & 0 & 0 & 1
\end{pmatrix} \qquad \Lambda = \begin{pmatrix}
\cosh\phi & - \sinh\phi & 0 & 0 \\
-\sinh\phi & \cosh\phi & 0 & 0 \\
0 & 0 & 1 & 0 \\
0 & 0 & 0 & 1 
\end{pmatrix} \]

In general:
\begin{itemize}
\item 4 translations
\item 3 rotations
\item 3 boosts
\end{itemize}
Poincaré: all. Lorentz only rotations and boosts.

\chapter{Relativistic mechanics}
\section{Proper time and proper velocity}
As we have seen there is no absolute notion of time. Each observer has their own clock that all run differently. There is however a clock that all observers can agree on: the clock attached to the object being observed. We call this time the \udef{proper time} $\tau$.

Due to time dilation, we know that the proper time intervals are given by
\[ \diff{\tau} = \sqrt{1- v^2 / c^2} \diff{t} \]

This proper time gives us a very natural way to parametrise the world line of a particle, so we might describe the world line by the equations
\[ x^\alpha = x^\alpha(\tau). \]

We can now define the \udef{proper velocity} $u^\alpha$ as the derivatives of the position along the world line with respect to the proper time:
\[ u^\alpha \equiv \od{x^\alpha}{\tau}. \]
Because $x^\alpha$ is a good four-vector and $\tau$ is invariant (i.e. all observers can agree on it), the four-velocity does in fact transform as a four-vector and the scalar product with itself is invariant.

If we now take the inertial frame centered around $x^\alpha(\tau)$ at a certain proper time $\tau$ moving with the same velocity as the instantaneous velocity of the particle, we see that the scalar product of $u^\alpha$ with itself must be
\[ u^\alpha u_\alpha = \eta_{\alpha\beta}\od{x^\alpha}{\tau}\od{x^\beta}{\tau} = -c^2 \]
TODO: better derivation with proper velocity also if not zero rest mass


TODO:
\[ \tau_{AB} = \int^B_A \diff{\tau} = \int^B_A \left(- \frac{\diff{s}^2}{c^2}\right)^{1/2} \]
\[ \tau = \int \left(-\eta_{\mu\nu}\od{x^\mu}{\lambda}\od{x^\nu}{\lambda}\right)^{1/2}\diff{\lambda} \]

TODO define $\vec{v}$

\[ (\Delta \tau)^2  = - (\Delta s)^2 = -\eta_{\mu\nu}\Delta x^\mu \Delta x^\nu \]

If you do not move:
\[ \Delta \tau = \Delta t \]

\section{Relativistic energy and momentum}
TODO $\vec{v} = \vec{p}/E$

Inspired by the classical definition of momentum, we can now \emph{define} \udef{relativistic momentum} as follows:
\[ \vec{p} \equiv m\vec{u} = \frac{m\vec{v}}{\sqrt{1-v^2/c^2}} \]
We can also define the \udef{relativistic energy}
\[ E \equiv mcu^0 = \frac{mc^2}{\sqrt{1-v^2/c^2}} \]
which we can identify with $cp^0$.
Thus we can define the \udef{energy-momentum 4-vector} (or just momentum 4-vector) as
\[ \boxed{p^\mu \equiv mu^\mu = (E/c, \vec{p})}. \]

As an historical aside: Einstein originally defined the \udef{relativistic mass} 
\[m_\text{rel} \equiv \frac{m}{\sqrt{1-v^2/c^2}}.\]
This definition differs from $E$ by a constant factor $c^2$, and thus is fairly redundant. 

We have not yet justified our definitions. In the classical limit ($u \ll c$), it is clear that $\vec{p} = m\vec{v}$ is just the classical momentum.

The energy is slightly more tricky. A first thing to verify is that it has the dimension of energy ($[ML^2T^{-2}]$), which it does. We can also notice that the relativistic energy is nonzero even when the object is stationary. We call this the \udef{rest energy}:
\[E_\text{rest} \equiv mc^2\]
The remainder, which is attributable to the motion, we call the \udef{kinetic energy}:
\[ E_\text{kin} \equiv E - mc^2 = mc^2 \left(\frac{1}{\sqrt{1-v^2/c^2}}-1\right).\]
In the classical limit ($v \ll c$ or $v/c \ll 1$) we can do a Taylor expansion in powers of $v^2/c^2$:
\[ E_\text{kin} = \frac{1}{2} mv^2 + \frac{3}{8}\frac{mv^4}{c^2} + \ldots \]
The leading term is the classical formula for the kinetic energy.

Importantly we also have the following experimental fact (TODO???):
\remark{In every closed system, the total relativistic energy and momentum are conserved.}

Finally we can take the scalar product of $p^\mu$ with itself. This quantity is not only conserved, but also scalar and thus invariant.
\[ p^\mu p_\mu = -(p^0)^2 + (\vec{p}\cdot\vec{p}) = -m^2c^2 \]
Multiplying this by $c^2$, we get the famous equation
\[ \boxed{E^2 - p^2c^2 = m^2c^4} \]

\section{Relativistic kinematics}
\subsection{Photon momenta}
In classical mechanics there is no such thing as a massless particle: you could not apply a force to it ($\vec{F} = m \vec{a}$) and hence (by Newton's third law) it could not apply a force on anything else. It's kinetic energy ($\frac{1}{2}mv^2$) and momentum ($m\vec{v}$) would be zero as well.

In relativistic mechanics there is a loophole however. A massless particle going at lightspeed would leave the equations
\[ \vec{p} = \frac{m \vec{v}}{\sqrt{1-v^2/c^2}} \quad \text{and} \quad E = \frac{mc^2}{\sqrt{1-v^2/c^2}} \]
indeterminate (zero over zero). It might be conceivable, therefore, that there exist massless particles with energy and momentum (which means forces can act on them, because $\vec{F} = \od{\vec{p}}{t}$).

Incredible as it may sound, this is in fact the case: the photon is massless but carries energy and momentum. Relativity cannot predict what the energy or momentum of a photon is, but it does force light to travel at lightspeed. Later we will see that quantum mechanics allows us to calculate the energy and momentum of photons.

One does have to be slightly more careful when dealing with massless particles though. The proper time interval between any two points is zero (as is the separation), which leads to some complications. Proper time can no longer be used to parametrize the world line of a massless particle.

For massless particles we can still define the tangent vector
\[ u^\alpha \equiv \od{x^\alpha}{\lambda} \]
where $\lambda$ is a parameter.
Different parametrizations will give different tangent vectors, but they are all null vectors. Thus for massless particles we have 
\[ u^\alpha u_\alpha = \eta_{\alpha\beta}\od{x^\alpha}{\lambda}\od{x^\beta}{\lambda} = 0 \]
We may choose any parameter to parametrize the curve, but there is a special class of parameters, called \udef{affine parameters} that we like in particular. They are the parameters the that give
\[ \od{u^\alpha}{\lambda} = 0. \]
This makes (as we shall see) the equation of motion the same as that for particles with mass. (TODO move section down? + wave four-vector + relativistic doppler and relativistic beaming)
\subsection{Mass-energy conversions}
In a classical collision, momentum and mass are always conserved. Kinetic energy, in general, is not. During a collision kinetic energy may be transformed into other types of energy, typically thermal (a \udef{sticky} collision); or other types of energy may be turned into more kinetic energy (an \udef{explosive} collision). If kinetic energy is conserved, we say the collision is \udef{elastic} (see also the section on classical mechanics).

In the relativistic case, momentum and total energy are always conserved, but mass and kinetic energy are not. Any change in kinetic energy means a change in rest energy, because the total energy $E$ is conserved and
\[ E_\text{rest} = E - E_\text{kin}. \]
Any change in rest energy, $E_\text{rest} = mc^2$, means a change in mass. So changes in kinetic energy cause changes in mass.

In the classical view, if a system loses kinetic energy, we typically say it gets transformed into thermal energy and lost. This is no different in the relativistic case, except we now must except that it has also gained mass. In general hotter objects are (ever so slightly) heavier than cold ones.

We again call a collision \udef{elastic} if kinetic energy is conserved. This means rest energy and mass must also be conserved.

\subsection{Practical strategies for relativistic kinematics}
TODO

\section{Relativistic dynamics}
We have so far translated some kinematic quantities into their relativistic counterparts. Now we see what happens to Newton's laws of motion.
\begin{enumerate}
\item Newton's first law is built into the principle of relativity.
\item Newton's second law can be written as follows:
\[ \vec{F} = \od{\vec{p}}{t} \]
This is still valid in relativistic mechanics, provided we use the relativistic momentum.
\item Newton's third law does not, in general hold true in relativistic mechanics.
\end{enumerate}
\subsection{The work-energy theorem}
TODO
\subsection{Minkowski force}
TODO

\section{Variational principle for free particle motion}

\chapter{Relativistic electrodynamics}
Writing in \udef{covariant form}.

Gauge invariance because of commuting derivatives

Wave equation in terms of $A^\mu$

D'Alembertian!!

\chapter{The Lorentz and Poincaré groups}

TODO: Action: The transformations $T(\Lambda, \epsilon)$ induce a unitary linear transformation on vectors in the physical Hilbert space
\[ \Psi \to U(\Lambda,a)\Psi  \]

\section{The Lorentz group}
The Lorentz group is the group of Lorentz transformations.
\subsection{Definition}
\begin{definition}
The Lorentz group is the isometry group of the Minkowski metric $\eta$
\[ \Ogroup(1,3) = \left\{\Lambda \in \R^{4\times 4}\;|\;\Lambda^\intercal\eta\Lambda = \eta\right\} \]
\end{definition}

\subsection{Connectedness}
Even though $\SO(4)$ is a connected group, $\SO(1,3)$ is \ueig{not a connected group}. There are two disconnections, meaning that $\SO(1,3)$ has 4 connected subsections. 

\begin{enumerate}
\item Taking the determinant of $\Lambda^\intercal\eta\Lambda = \eta$, we get
\[ \det(\Lambda^\intercal\eta\Lambda) = \det(\Lambda^\intercal)\det(\eta)\det(\Lambda) = \det(\eta) \qquad \Rightarrow \qquad \left[\det(\Lambda)\right]^2 = 1. \]
As a consequence either $\det(\Lambda) = 1$ or $\det(\Lambda) = -1$. There is no path from one to the other.
\item Taking the $00$-component of the equation $\eta_{\rho\sigma} = \tensor{\Lambda}{^\mu_\rho}\tensor{\Lambda}{^\nu_\sigma}\eta_{\mu\nu}$, we get
\begin{align*}
1 = \eta_{00} &= \tensor{\Lambda}{^\mu_{0}}\tensor{\Lambda}{^{\nu}_{0}}\eta_{\mu\nu} \\
&= \left(\tensor{\Lambda}{^{0}_{0}}\right)^2\eta_{00} + \tensor{\Lambda}{^{i}_{0}}\tensor{\Lambda}{^{i}_{0}}\eta_{ii} = \left(\tensor{\Lambda}{^{0}_{0}}\right)^2 - \tensor{\Lambda}{^{i}_{0}}\tensor{\Lambda}{^{i}_{0}}
\end{align*}
Rearranging yields
\[ \left(\tensor{\Lambda}{^0_{0}}\right)^2 = 1 + \sum_i \left(\tensor{\Lambda}{^i_{0}}\right)^2 \geq 1 \qquad\Rightarrow\qquad \begin{cases}
\tensor{\Lambda}{^{0}_{0}} \geq 1 \\
\tensor{\Lambda}{^{0}_{0}} \leq 1
\end{cases} \]
\end{enumerate}

So we can partition the elements of $\Ogroup(1,3)$ into 4 subsets of transformations, only one of which is a subgroup (because it contains the identity): the \udef{proper orthochronous Lorentz group $\Lambda^{++}$} with $\det(\Lambda^{++}) = 1$ and $\tensor{(\Lambda^{++})}{^0_0} \geq +1$.

Any Lorentz transformation is either part of $\Lambda^{++}$, or can be made part of it by composing it with one or both of the discrete transformations $P$ or $T$.

The \udef{space inversion $P$} and \udef{time-reversal} matrices are given by
\[ P = \begin{pmatrix}
1 & 0 & 0 & 0 \\
0 & -1 & 0 & 0 \\
0 & 0 & -1 & 0 \\
0 & 0 & 0 & -1 \\
\end{pmatrix} \qquad T = \begin{pmatrix}
-1 & 0 & 0 & 0 \\
0 & 1 & 0 & 0 \\
0 & 0 & 1 & 0 \\
0 & 0 & 0 & 1 \\
\end{pmatrix} \]

The four parts of $\Ogroup(1,3)$ are
\begin{enumerate}
\item The set of proper, orthochronous Lorentz transformations, $\Lambda^{++}$ or $\SO(1,3)_+$, with 
\[\det\Lambda = +1, \quad \tensor{\Lambda}{^0_0}\geq 1.\]
This is a subgroup.
\item The set of proper, non-orthochronous Lorentz transformations, $\Lambda^{+-}$, with 
\[\det\Lambda = +1, \ \tensor{\Lambda}{^0_0}\leq -1. \]
This can also be written as
\[ \Lambda^{+-} = PT\Lambda^{++} \]
\item The set of improper, orthochronous Lorentz transformations, $\Lambda^{-+}$, with 
\[\det\Lambda = -1, \ \tensor{\Lambda}{^0_0}\geq 1. \]
This can also be written as
\[ \Lambda^{-+} = P\Lambda^{++} \]
\item The set of improper, non-orthochronous Lorentz transformations, $\Lambda^{--}$, with 
\[\det\Lambda = -1, \ \tensor{\Lambda}{^0_0}\leq -1. \]
This can also be written as
\[ \Lambda^{--} = T\Lambda^{++} \]
\end{enumerate}

\subsection{Compactness} The Lorentz group is also non compact (not every open cover has a finite subcover) (TODO)
\begin{align*}
\SO(2) &= \begin{pmatrix}
\cos\theta & - \sin\theta \\ \sin\theta & \cos\theta
\end{pmatrix} \qquad \to \qquad \theta \in [0,2\pi) \qquad \to \qquad \text{manifold is compact} \\
\SO(1,3) &\ni \begin{pmatrix}
\cosh\epsilon & -\sinh\epsilon & 0 & 0 \\
- \sinh\epsilon & \cosh\epsilon &0&0\\
0&0&1&0\\0&0&0&1
\end{pmatrix} \qquad \epsilon \in (-\infty, +\infty) \qquad \text{(boost)} \\
&= \begin{pmatrix}
\gamma & -\beta\gamma & 0 & 0 \\
- \beta\gamma & \gamma &0&0\\
0&0&1&0\\0&0&0&1
\end{pmatrix} \qquad \gamma \in [1, +\infty]
\end{align*}

\section[The Lorentz algebra so(1,3)]{The Lorentz algebra $\mathfrak{so}(1,3)$}
Consider an infinitesimal Lorentz transformation (i.e. close to identity)
\[ \tensor{\Lambda}{^\mu_{\nu}} = \delta^\mu_\nu + \tensor{\omega}{^\mu_{\nu}} \qquad (\omega \ll 1) \]
The Lorentz condition ($\Lambda^\intercal\eta\Lambda = \eta$) gives in this case
\begin{align*}
\eta_{\mu\nu}  &= \tensor{\Lambda}{^\rho_{\mu}}\tensor{\Lambda}{^\sigma_{\nu}}\eta_{\rho\sigma} = (\delta^\rho_\mu + \tensor{\omega}{^\rho_{\mu}})(\delta^\sigma_\nu + \tensor{\omega}{^\sigma_{\nu}})\eta_{\rho\sigma} \\
&= \eta_{\mu\nu} + \eta_{\mu\sigma}\tensor{\omega}{^\sigma_{\nu}} + \eta_{\rho\nu}\tensor{\omega}{^\rho_{\mu}} + \mathcal{O}(\omega^2) = \eta_{\mu\nu} + \tensor{\omega}{_\mu_{\nu}} + \tensor{\omega}{_\nu_{\mu}} + \mathcal{O}(\omega^2)
\end{align*}
From this we get the condition
\[ \omega_{\mu\nu} = - \omega_{\nu\mu} \]
which leaves 6 parameters.

Now we write
\begin{align*}
\omega_{\mu\nu} = \frac{1}{2} \left(\omega_{\mu\nu} - \omega_{\nu\mu}\right) &= \frac{1}{2}\omega_{\alpha\beta}\left(\delta^\alpha_\mu\delta^\beta_\nu - \delta^\alpha_\nu\delta^\beta_\mu\right) \\
&= - \frac{i}{2}\omega_{\alpha\beta} \left(\mathcal{J}^{\alpha\beta}\right)_{\mu\nu}
\end{align*}
Where we define $\left(\delta^\alpha_\mu\delta^\beta_\nu - \delta^\alpha_\nu\delta^\beta_\mu\right)$ as $-i \left(\mathcal{J}^{\alpha\beta}\right)_{\mu\nu}$. So $\mathcal{J}^{\alpha\beta}$ are the generators of the Lorentz group in the defining representation and $\frac{-\omega_{\alpha\beta}}{2}$ are the coordinates of the algebra element with respect to this basis of generators. Therefore
\begin{align*}
\Lambda^\mu_{\;\nu} &= \delta^\mu_\nu - \frac{i}{2}\omega_{\alpha\beta}\left(\mathcal{J}^{\alpha\beta}\right)_{\mu\nu} \\
\Lambda &= e^{- \frac{i}{2}\omega_{\alpha\beta}\mathcal{J}^{\alpha\beta}}
\end{align*}

\begin{example}
Exercise: Prove the following:
\[ [\mathcal{J}^{\mu\nu},\mathcal{J}^{\rho\sigma}] = i \left(\eta^{\mu\rho}\mathcal{J}^{\nu\sigma}-\eta^{\nu\rho}\mathcal{J}^{\mu\sigma}+\eta^{\nu\sigma}\mathcal{J}^{\mu\rho}-\eta^{\mu\sigma}\mathcal{J}^{\nu\rho}\right) \]
Answer:
\begin{align*}
\left[\mathcal{J}^{\mu\nu}, \mathcal{J}^{\rho\sigma} \right]_{ij} &= \sum_k \left(\delta^\mu_i\delta^\nu_k - \delta^\mu_k\delta^\nu_i\right)\left(\delta^\rho_k\delta^\sigma_j-\delta_j^\rho\delta^\sigma_k\right) - \sum_k \left(\delta^\rho_i\delta^\sigma_k-\delta^\rho_k\delta^\sigma_i\right)\left(\delta^\mu_k\delta^\nu_j-\delta^\mu_j\delta^\nu_k\right) \\
&= \delta^\mu_i\delta^\rho_\nu\delta^\sigma_j - \delta^\mu_i\delta^\nu_\sigma\delta^\rho_j - \delta^\mu_\rho\delta^\nu_i\delta^\sigma_j + \delta^\mu_\sigma\delta^\nu_i\delta^\rho_j - \delta^\rho_i\delta^\sigma_\mu\delta^\nu_j + \delta^\rho_\mu\delta^\sigma_i\delta^\nu_j +\delta^\rho_i\delta^\sigma_\nu\delta^\mu_j - \delta^\rho_\nu\delta^\sigma_i\delta^\mu_j
\end{align*}
Using
\[ \left(\eta^{\mu\rho}\mathcal{J}^{\nu\sigma}\right)_{ij} = -i \left(\delta^\rho_\mu\delta^\nu_i\delta^\sigma_j - \delta^\rho_\mu\delta^\nu_j\delta^\sigma_i\right) \]
We recover the equality. But we need to take $\eta^{\mu\rho} = \delta^\rho_\mu$???
\end{example}

We now are going to derive the algebra $[\cdot,\cdot]$ for a generic representation $D$.

\begin{enumerate}
\item $D(\Lambda) = D(\mathbb{1}_\omega) \equiv \mathbb{1} - \frac{i}{2}\omega_{\alpha\beta}J^{\alpha\beta}$

Where $J^{\alpha\beta}$ are the generators in the representation $D$.
\item Group properties:
\[ D(\Lambda)^{-1}D(\tilde{\Lambda})D(\Lambda) = D(\Lambda^{-1}\tilde{\Lambda}\Lambda) \]
For an infinitesimal transformation $\tilde{\Lambda}$ we have
\[ D(\Lambda)^{-1}(\mathbb{1}- \frac{i}{2}\tilde{\omega}_{\alpha\beta}J^{\alpha\beta})D(\Lambda) = D(\mathbb{1}- \frac{i}{2}\tilde{\omega}_{\alpha\beta}\Lambda^\alpha_{\;\mu}\Lambda^\beta_{\;\nu}J^{\mu\nu}) \]
So \[ \boxed{D(\Lambda)^{-1}J^{\alpha\beta}D(\Lambda) = \Lambda^\alpha_{\;\mu}\Lambda^\beta_{\;\nu}J^{\mu\nu}} \]
\item Linearize $\Lambda \approx \mathbb{1} + \omega$ (with $\omega \ll 1$)
\[ \left(\mathbb{1}+\frac{i}{2}\omega_{\rho\sigma}J^{\rho\sigma}\right)J^{\alpha\beta}\left(\mathbb{1}- \frac{i}{2}\omega_{\rho\sigma}J^{\rho\sigma}\right)+\ldots = J^{\alpha\beta}+\frac{i}{2}\omega_{\rho\sigma}\left[J^{\rho\sigma},J^{\alpha\beta}\right]+\ldots \]
\[ = \left(\delta^\alpha_{\;\mu}+ \omega^\alpha_{\;\mu}\right)\left(\delta^\beta_{\;\nu}+\omega^\beta_{\;\nu}\right)J^{\mu\nu} = J^{\alpha\beta}+\omega^\alpha_{\;\mu}J^{\mu\beta}+\omega^\beta_{\;\nu}J^{\alpha\nu} = J^{\alpha\beta}-\omega_{\rho\sigma}\left(\eta^{\alpha\rho}J^{\beta\sigma}-\eta^{\beta\rho}J^{\alpha\sigma}\right) + \ldots \]
So we get
\[ \left[J^{\mu\nu}, J^{\rho\sigma}\right]= i \left(\eta^{\mu\sigma}J^{\nu\sigma}-\eta^{\nu\sigma}J^{\mu\sigma}+\eta^{\nu\sigma}J^{\mu\rho}-\eta^{\mu\sigma}J^{\nu\rho}\right) \]
This is the same as for the defining representation!
\end{enumerate}

\begin{example}
Recap of some important points:
\[ \Lambda^\mu_{\;\alpha}\Lambda^\nu_{\;\beta}\eta_{\mu\nu} = \eta_{\alpha\beta} \qquad \to \qquad \Lambda^\intercal\eta\Lambda = \eta \Rightarrow \Lambda^{-1}=\eta^{-1}\Lambda^\intercal\eta \]
\[\begin{cases}
\eta = \eta_{\alpha\beta}\\
\eta^{-1} = \eta^{\alpha\beta}
\end{cases}\]
\[ \left[J^{\mu\nu}, J^{\rho\sigma}\right]= i \left(\eta^{\mu\sigma}J^{\nu\sigma}-\eta^{\nu\sigma}J^{\mu\sigma}+\eta^{\nu\sigma}J^{\mu\rho}-\eta^{\mu\sigma}J^{\nu\rho}\right) \]
Defines algebra of Lorentz group.
\[ D(\Lambda)^{-1}D(\tilde{\Lambda})D(\Lambda) = D(\Lambda^{-1}\tilde{\Lambda}\Lambda) \]
\end{example}

Let's make the following redefinitions
\[ \begin{cases}
K_i \equiv J^{i0} \qquad \to \qquad \text{Boost generator} \\
J_i \equiv \frac{1}{2}\epsilon_{ijk}J^{jk} \qquad \to \qquad \text{3D rotations}
\end{cases} \]
The algebra commutators in $K_i$ and $J_i$ are given by
\[ \begin{cases}
[J_i,J_j] = i\epsilon_{ijk}J_k\\
[J_i,K_j] = i\epsilon_{ijk}K_k\\
[K_i,K_j] = -i\epsilon_{ijk}J_k
\end{cases} \]
We introduce a new basis to decouple the commutator relations
\[ N_i^{(+)} \equiv \frac{J_i+iK_i}{2}, \qquad N_i^{(-)} \equiv \frac{J_i-iK_i}{2} \]
Then the commutator relations become
\[ \begin{cases}
\left[N_i^{(+)},N_j^{(+)}\right] = i\epsilon_{ijk}N_k^{(+)} \qquad \to \qquad\SU(2) \; \text{Which is classified by an index $j_+$} \\
\left[N_i^{(-)},N_j^{(-)}\right] = i\epsilon_{ijk}N_k^{(-)} \qquad \to \qquad\SU(2) \to j_- \\
\left[N_i^{(+)},N_j^{(-)}\right] = 0
\end{cases} \]
So Lorentz group representations are classified $j=(j_+,j_-)$
\remark{dim $j = (2j_++1)(2j_-+1)$}
Locally
\begin{align*}
\SO(1,3) \approx ``\SU(2)\times\SU(2)''
\end{align*}

\subsection{Simplest (irreducible) Lorentz group representations}
\begin{tabular}{c c c}
$(j_+,j_-)$ & Dim & \\ \hline
\underline{Irreducible} \\ 
$(0,0)$ & $1$ & Scalar particles (spin 0) \\
$(1/2,0)$ & $2$ & Weyl L fermion (spin 1/2) \\
$(0,1/2)$ & $2$ & Weyl R fermion (spin 1/2) \\
$(1/2,1/2)$ & $4$ & Vector \\
$(1,1)$ & $9$ & Tensor particle (spin 2) \\ \hline
\underline{Reducible} \\
$(1/2,0)\oplus(0,1/2)$ & 4 & Dirac fermion (spin 1/2)
\end{tabular}

\subsection{Casimir of Lorentz algebra}
\[ C_2 = \frac{1}{2}J^{\mu\nu}J_{\mu\nu} \qquad C_2 = \frac{1}{4}\epsilon_{\mu\nu\rho\sigma}J^{\mu\nu}J^{\rho\sigma} \]

\section{The Poincaré group}
The Poincaré group is the group of Poincaré transformations
\[ x^\mu \to x^{\prime\mu} = \Lambda^{\mu}_{\;\nu}x^\nu + \epsilon^\mu \qquad \begin{cases}
\Lambda = \Ogroup(1,3) \\
\epsilon^\mu \in \R^4
\end{cases} \]

TODO: Poincaré is actually a group.

\begin{definition}
The Poincaré group is defined as
\[P = \left\{g(\Lambda,a)\quad | \quad \Lambda\in\R^{4\times4},\quad a \in \R^4, \quad \Lambda^\intercal\eta\Lambda = \eta \right\}\]
where $\eta = \diag(1,-1,-1,-1)$. With the following composition law:
\begin{align*}
g_2\cdot g_1 &= (\Lambda_2, a_2)(\Lambda_1, a_1) = (\Lambda_2\Lambda_1, \Lambda_2a_1+a_2) \\
g^{-1} &= (\Lambda,a)^{-1} = (\Lambda^{-1}, -\Lambda^{-1}a)
\end{align*}
\end{definition}

The Poincaré group looks like a direct product, but isn't quite one. We call it a \udef{semidirect product} 
\[ P = \Ogroup(1,3)\rtimes \R^4 \]

\section{Poincaré algebra}
TODO: Hamiltonian $P^\mu$ (or $-P^\mu$) 

First we look at translations. They form a 4 dim invariant subgroup of $\R^4$
\[ \begin{cases}
x^\mu \to x^{\prime\mu} = x^\mu+\epsilon^\mu = \left(1+i\epsilon^\nu\rho_\nu\right)x^\mu \\
\rho_\nu = -\partial_\nu, \qquad g_\epsilon = e^{i\epsilon^\mu\rho_\mu}
\end{cases} \]
\[ \left[\rho_\nu,\rho_\mu\right] = 0 \qquad \to \qquad\text{Translations commute} \]
Translations are also an abelian subgroup of $P$:
\[ (\mathbb{1},\epsilon_1)\cdot(\mathbb{1},\epsilon_2) = (\mathbb{1},\epsilon_1+\epsilon_2) = (\mathbb{1},\epsilon_2)\cdot (\mathbb{1},\epsilon_1) \]
\[ D^{-1}(g_\epsilon)D(g_{\tilde{\epsilon}})D(g_\epsilon) = D(g^{-1}_\epsilon g_{\tilde{\epsilon}} g_\epsilon ) = D(g_{\tilde{\epsilon}}) \]
We introduce the generic representation $D(g_\epsilon) = e^{i\epsilon^\mu P_\mu}$ where $P_\mu$ are generators of $D$ and we try to find properties of $P_\mu$

\[ e^{-i\epsilon P}\left(1+i\tilde{\epsilon}P\right)e^{i\epsilon P} = 1+ i\tilde{\epsilon}P \]
\[ e^{-i\epsilon P}i\tilde{\epsilon}Pe^{i\epsilon P} = i\tilde{\epsilon}P \]
\[ \left(1-i\epsilon^\mu P_\mu\right)i\tilde{\epsilon}^\nu P_\nu \left(1+i\epsilon^\mu P_\mu\right) = i\tilde{\epsilon}^\nu P_\nu \]
\[ \cancel{i\tilde{\epsilon}^\nu P_\nu} - \epsilon^\mu\tilde{\epsilon}^\nu \left[P_\mu, P_\nu\right] + \mathcal{O}(\epsilon^2) = \cancel{i\tilde{\epsilon}^\nu P_\nu} \]
So
\[ \left[P_\mu, P_\nu\right] = 0\]

Translations + rotations do, however, not commute. We write generic elements of rotations, translations and the Poincaré group in the following way
\[ g_\Lambda(\Lambda,0), \qquad g_\epsilon(\mathbb{1},\epsilon), \qquad g(\Lambda,\epsilon) \]
And we see that
\begin{align*}
g_\Lambda g_\epsilon &= (\Lambda,0)(\mathbb{1},\epsilon) = (\Lambda, \Lambda \epsilon) \\
g_\epsilon g_\Lambda &= (\mathbb{1},\epsilon)(\Lambda,0) = (\Lambda,\epsilon)
\end{align*}
These objects are not the same, so translations and rotation do not comute.

We now calculate the commutators
\[ D(g_\Lambda)^{-1}D(g_\epsilon)D(g_\Lambda) = D(g_\Lambda^{-1}g_\epsilon g_\Lambda) = D(g_{\Lambda^{-1}\epsilon^{-1}}g_{\tilde{\epsilon}}) \]
Which we expand for small $\epsilon \ll 1$:
\[ D(g_\Lambda)^{-1}\left(\mathbb{1}+i\epsilon^\mu P_\mu\right)D(g_\Lambda) = 1+i(\Lambda^{-1})^\mu_{\;\nu}\epsilon^\nu P_\mu \]
\[ \begin{cases}
D(g_\Lambda)^{-1}P_\mu D(g_\Lambda) = \Lambda_\mu^{\;\nu}P_\nu \\
\boxed{D(g_\Lambda)^{-1}P^\mu D(g_\Lambda) = \Lambda^{\mu}_{\;\nu}P^\nu}
\end{cases} \]
Expanding for small $\omega_{\alpha\beta} \ll 1$
\[ (\mathbb{1}+\frac{1}{2}\omega_{\alpha\beta}J^{\alpha\beta})P^\mu(\mathbb{1}- \frac{1}{2}\omega_{\alpha\beta}J^{\alpha\beta}) = (\delta^\mu_\nu + \omega^\mu_{\;\nu})P^\nu \]
So
\[ \frac{i}{2}\omega_{\alpha\beta}\left[J^{\alpha\beta},P^\mu\right] = \omega^\mu_{\;\nu}P^\nu \]
\[ \boxed{\left[J^{\alpha\beta},P^\mu\right] = i \left(\eta^{\mu\alpha}P^\beta-\eta^{\mu\beta}-\eta^{\mu\beta}P^\alpha\right)} \]
The full set of commutators for the Poincaré algebra, where we define
\[ K_i \equiv J^{i0}, \qquad J_i \equiv \frac{1}{2}\epsilon_{ijk}J^{jk}, \qquad H \equiv P_0, \qquad P_i \]
is as follows

\[\begin{array}{l l}
\left[J_i,J_k\right] = i\epsilon_{ijk}J_k  \qquad& \left[J_i,K_j\right] = i\epsilon_{ijk}K_k \\
\left[J_i,P_j\right] = i\epsilon_{ijk}P_k & \left[J_i,H\right] = 0 \to \text{good quantum number} \\
\left[K_i,K_j\right] = -i\epsilon_{ijk}J_k & \left[K_i,P_j\right] = i\delta_{ij}H \\
\left[K_i,H\right] = iP_i & \left[P_j,P_j\right] = 0 \\
\left[P_i, H\right] = 0 & \left[H,H\right] = 0
\end{array}\]

\subsection{Casimirs of the Poincaré algebra}
$P^\mu$ form a subalgebra. Because $\left[P^\mu, P^\nu\right] = 0$ they can be diagonalised simultaneously so that
\[ P^\mu\ket{P} = p^\mu\ket{P} \]
Where $p^\mu$ is an eigenvalue of the operator $P^\mu$.

$P^2$ is a scalar and
\[ \left[P^2, P^\mu\right] = 0, \qquad \left[P^2, J^{\mu\nu}\right] = 0 \]
\[ P^2\ket{P} = M^2\ket{P} \]
With $M$ the mass.
\begin{itemize}
\item $P^2$ is a Casimir for the Poincaré group
\item Any relativistic particle is identified by $M$
\item $M$ identifies the irreducible representation of Poincaré
\item Spin is not a good relativistic quantum number (Spin > < Boost). So we use \udef{helicity} = projection of spin along the direction of motion.
\remark{We introduce the Pauli-Lubanski (pseudo-)vector}
\[ \boxed{W^\mu = \frac{1}{2}\epsilon^{\mu\nu\rho\sigma}J_{\nu\rho}P_{\sigma}} \]
\end{itemize}
$W^2 \equiv W^\mu W_\mu$ is a Casimir of the Poincaré group
\[ \left[W^2, P^\mu\right] = 0, \qquad \left[W^2,J^{\alpha\beta}\right] = 0 \]
\[ W^2\ket{P,\sigma} \propto \frac{\sigma(\sigma+1)}{2}\ket{P,\sigma} \]
To classify any irreducible representation of the Poincaré group: $P^2$ and $W^2$ give mass and spin.

See exercises!!