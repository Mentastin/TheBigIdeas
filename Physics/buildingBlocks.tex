TODO: Different scales. How we know. Impact of measuring at small scales.
TODO:
Subatomic particles (mesons, baryons, photons, leptons, quarks) and cosmic rays; property of materials and chemistry; nuclear isotopes; phase transitions; astrophysics (planetary system, stars, galaxies, red shifts, supernovae); cosmology (cosmological models, inflationary universe theories, microwave background radiation); detection techniques.


Now that we have arrived at this point, it is useful to explore the object of our study: the universe. So far our development of physics has been based on everyday observations, but we can only get so far studying the motion of apples, cars and pendula.

We will now give an extremely brief overview of some of the things that we cannot see as easily or do not come into contact with in everyday life. No attempt will be made to justify any claims in this section as to do so requires the physics we will elaborate upon in subsequent sections. However in order to clearly present those sections, it is useful to have an idea of the building blocks discussed in this section. It is a bit of a catch-22.

Research is an iterative process, so when figuring these things out for the first time, a description of the world is used to build a better explanation of phenomena, based on which we can update our description which allows us to better test our explanations etc. So gradually we are (we hope) honing in on the truth. In a text such as this we want to just present the facts and consequently the explanation will have to be postponed.

\chapter{Structures of matter}
In our brief description of the makeup of the world, we assume familiarity with the kinds of things we come across in everyday life, i.e. things on a macroscopic scale. We now explore how matter is organized on smaller length scales. This depends very much on exactly which type of matter we are considering. 

\section{An example: water}
As a first example we observe a glass of water (in liquid state).
If we zoom in on our glass of water, it would look fairly similar until we got down to the nanometre scale. At that point we would start to see that water is not a homogeneous liquid, instead it is made up of particles bouncing against each other. We call each of those particles \udef{molecules} of water.

A molecule of water is the smallest unit we can still call water. Pure water is sometimes called TODO chem latex. H$_2$O, this refers to the makeup of a single molecule: two hydrogen atoms and an oxygen atom. Thus if we split apart a single molecule, we no longer have H$_2$O, but rather single hydrogen and oxygen atoms. A molecule of water has very different chemical and physical properties than the hydrogen and oxygen it is made up of: a collection of hydrogen atoms will naturally form the explosive hydrogen gas; oxygen gas is the stuff we breath to stay alive.

Pure water is made up of lots of such H$_2$O molecules. And I mean lots. A couple of grams has approximately TODO molecules of water in it. With such a mindbogglingly large number of molecules, it is impossible to make sure all of them are H$_2$O. There will always be \udef{impurities}. These may come from the container or be air molecules that get stuck in the liquid or come from any number of other sources. TODO purest we can get. Distilled water

In our everyday lives this purity is not really relevant. All drinking water is (intentionally) not at all pure. It contains minerals and metals and other stuff because a) it would be expensive to only drink chemically pure water and b) it would kill you (TODO why). Water, like most things outside the laboratory, is a mixture. 

\section{Molecules}

One of the most striking

\subsection{Anorganic molecules}

\subsection{Organic molecules}

\subsection{Molecular weight}

\section{Crystals}
\section{Glasses}

\chapter{Elements of chemistry}

\section{Mixtures and impurities}
\section{Phases and phase transitions}
\section{Chemical bonds}
\subsection{Charges}
\subsection{Intramolecular}
\subsection{Intermolecular}

\chapter{Atoms}

\chapter{Standard model}

\section{Fundamental forces}

\chapter{Organic matter}

\chapter{Stars}

\chapter{Galaxies}

\chapter{Black holes}

\chapter{Other things}
