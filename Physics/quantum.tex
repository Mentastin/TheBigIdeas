\chapter{Origins and justification}
Now we move onto the second of Lord Kelvin's clouds, which has to do with the equipartition theorem of classical statistical mechanics. One obvious way in which the classical theory fails is when we look at thermal radiation. Object at any temperature emit electromagnetic radiation. The hotter the object, the more it emits (it becomes ``glowing hot''). Unfortunately classical theory predicts that all objects emit an infinite amount of radiation of very short wavelength (i.e. ultraviolet radiation). This is clearly not true and is known as the ultraviolet catastrophe.

It was this problem that Planck solved in December of 1900. The way he solved it and the assumptions he made in order to solve it had profound implications and let to the development of quantum mechanics.

As with relativity we will first go through some of the seminal experiments for quantum mechanics in order to gain a sense of the phenomenology, then based on the phenomena we will have discussed, we will develop the theory of quantum mechanics.

Quantum mechanics is quite a bewildering field that makes very little intuitive sense. The \textit{mathematics} of quantum mechanics is well understood and has yielded many thoroughly tested predictions. At this point it is a mature theory, so it's quite surprising we still have not quite figured out what exactly it \textit{means}. There are several competing interpretations of quantum mechanics, which we shall very briefly mention below. These interpretations do not disagree on what they predict we observe, only on what is actually happening. Thus one may surmise that they are more philosophical in nature and not necessarily relevant to the student of physics. This instrumentalist view is prevalent in many quantum mechanics courses where it is much easier, rather than wade into the murky philosophical depths of our lack of understanding, to just quote David Mermin: "Shut up and calculate".

Unfortunately for us, this multitude of post hoc interpretations poses a serious problem for our project of Carnappian rational reconstruction: which conception of reality do we base our reconstruction on and why? Recently a possible answer to this quandary has started to emerge: quantum informational axiomatisations have emerged that do not rely on our accepting any one interpretation of quantum mechanics. This is a rather new field of research and not all problems have been ironed out yet, nevertheless it is an interesting if somewhat unsatisfactory avenue and we have a quick glance down it.

Bearing this in mind, this part on quantum mechanics is structured as follows: first we have a look at some of the strange phenomena that could not be explained classically and motivated the development of quantum mechanics; at the same time we will briefly mention some of the early ad hoc theories to explain these phenomena. These theories are now sometimes referred to as the old quantum theories and still find use as semi-classical approximations. We will then have a look at the two theories that were invented in order to explain these phenomena: matrix mechanics and wave mechanics, with an emphasis on wave mechanics because it is conceptually simpler. The introduction of these theories is still somewhat ad hoc, but they do manage to explain all of the quantum phenomena. The next step is a generalization to the elegant modern formulation of quantum mechanics using separable Hilbert spaces and a statement of the Dirac-von Neumann axioms (or postulates as they are often called). We will do our best to justify why these axioms are natural but, for the reasons mentioned above, the results of this effort will remain somewhat unsatisfactory. At this point some of the interpretations of this theory we have just constructed will be mentioned.
\section{Wave-particle duality}
\subsection{A history of particle and wave theories of light}
The nature of light has long been the subject of debate, going all the way back to the ancient Greeks. In the $11^\text{th}$ century the Arabic scientist Ibn al-Haytham (also known as Alhazen) wrote the comprehensive Book of Optics describing such phenomena as reflection and diffraction. He asserted that light rays are composed of particles of light.

In 1630 championed the wave theory of light. All other known waves had to propagate through a medium, so he posited the existence of a universal medium, the luminiferous aether.

In Opticks, published in 1704, Isaac Newton defended his corpuscular (i.e. particle) theory. He argued that only particles could move in the perfectly straight lines of light rays.

Around the same time others (including Robert Hooke, Christiaan Huygens and Augustin-Jean Fresnel) were refining the wave viewpoint.

In 1801 Thomas Young proposed his infamous double-slit experiment demonstrating wave interference of light. This, in conjunction with the inability of the particle theory to explain polerisation, began to convince the scientific community and by the middle of the $19^\text{th}$ century the wave theory was generally accepted.

In the second half of the $19^\text{th}$ century James Clerk Maxwell discovered that his equations could be applied to describe self-propagating waves of the electric and magnetic fields. It quickly became apparent that visible light, ultraviolet light and infrared light were all just electromagnetic waves of different frequencies.

\subsection{The particle nature of photons}
\subsubsection{Black body radiation}
In the section on optics (TODO) we have seen the form the spectral distribution of black body radiation takes (figure TODO). We have also already given some results based on general thermodynamic results. In order to derive an actual expression for the spectral distribution function $\rho(\lambda, T)$ we need a more detailed model.

TODO

$h$ gives quantum of action
seen as a mathematical trick by Planck trick

Wall of cavity never in equilibrium with radiation: finite degrees of freedom v infinite degrees of freedom.

\subsubsection{The photoelectric effect}
While Heinrich Hertz was performing his celebrated experiments in 1887 producing and detecting electromagnetic waves, experimentally verifying Maxwell's theory, he discovered a peculiar phenomenon: ultraviolet light falling on electrodes facilitates the passage of a spark. It was shown that this was due to electrons being ejected from the metallic surfaces when irradiated by high-frequency electromagnetic waves. This effect is called the \udef{photoelectric effect}.

Clearly the incident electromagnetic radiation is exciting the electrons in the metal enough to break free of their bonds. The photoelectric effect has a number of features that are impossible to explain with the classical theory of electromagnetic waves. Remember that in classical theory the energy of electromagnetic radiation depends on the intensity of the beam, but is independent of the frequency. (TODO:?)
\begin{enumerate}
\item There is a cutoff frequency below which no emission of electrons takes place, no matter the intensity of the beam.
\item The maximum kinetic energy of the emitted electrons depends linearly on frequency and is independent of the intensity.
\item Emission takes place immediately when light shines on the surface, there is no detectable time delay between exposure to light and emission of electrons. Based on classical theory we would expect to have a delay when using low intensity beams because it would take time to pump enough energy into a region of space of atomic proportions.
\end{enumerate}
A very the solution to these problems was proposed in 1905 by Einstein. It was for this paper that he got the Nobel prize. Einstein proposed that we take the quantization that Max Planck had introduced seriously, and not just view it as a mathematical trick. His assertion was that light consists of quanta, now called \udef{photons}, each with an energy
\[ E = h\nu = hc / \lambda. \]
The strange phenomena described above can then be explained as follows:
\begin{enumerate}
\item A single photon needs to carry enough energy to eject an electron from the surface. (The energy required to do that is called the \udef{work function} $W$ and depends on the metal being irradiated). Thus if the frequency is too low, no electrons are emitted, no matter how many photons are pumped into the system (i.e. how high the intensity is).
\item When a photon collides with an electron, it is absorbed and passes (almost) all of its energy on to the electron, thus the maximum kinetic energy of emitted electrons is the energy of the photons ($h\nu$) minus the work function.
\item If energy is not constantly being pumped into an electron, it quickly loses the energy it gained (I assume?? TODO). Thus, because the electrons receive energy from photons in small - quantized - bursts, there is no mechanism by which energy can accumulate to a point where the electron is able to break free from the bulk metal. This means emission is either instantaneous, or does not happen.
\end{enumerate}
So in conclusion, the photoelectric effect provides compelling evidence for the corpuscular nature of light. Unfortunately there is also a large amount of evidence confirming that light travels in waves. Quantum mechanics gives us a model for photons that exhibits both wave-like and particle-like behaviour. This is know as ``wave-particle duality''.


\subsubsection{The Compton effect}
Another phenomenon that helps confirm the particular nature of light is the Compton effect, named after Arthur Holly Compton. This effect occurs when irradiating matter (Compton used graphite) with X-rays. Some of the X-rays are scattered omni-directionally with no change in wavelength. This is known as \udef{Thomson scattering} and can be explained classically. Sometimes the X-rays are also scattered such that the scattered wave has a different wavelength, especially if the incident X-rays have a short wavelength. Even more strangely the wavelength of the scattered wave depends on the angle at which it is scattered. This is known as \udef{Compton scattering}. The difference between the wavelength of the incident wave and the scattered wave is called the \udef{Compton shift}.

Thomson scattering can be explained (classically) as follows: The incident oscillating electric field causes the atomic electrons to oscillate at the same frequency. These electron then radiate electromagnetic radiation in all directions at the same frequency.

Compton scattering can be explained if we assume the incident X-rays are made up of corpuscular X-rays that scatter the atomic electrons like particles. The process is very similar to the photoelectric effect, but Compton did not build on Einstein's work. The difference is that energy of the incident photon is higher and thus the photon is not absorbed, but merely relinquishes some of its energy to the electron during the scattering process and can be detected when it comes out of the sample of matter being irradiated. In fact the Compton and photoelectric effects are two of the three competing effects when photons interact with matter; the third is pair production and is dominant at even higher energies. (These effects will be discussed in more detail later).

Because scattered can only be detected if their energy is significantly higher than the binding energy of the electrons with the atoms, we can simplify our calculations by assuming the electrons are \textit{free}, i.e. not bound to any atoms. This explains why the expression for the Compton shift is independent of the nature of the material used for the target. The scattering probability does of course depend on the target and is in particular directly proportional to the density of electrons.

Assuming the incident photon is a particle (TODO image), we can use conservation of relativistic energy and momentum to derive the following equation:
\[ \lambda_s - \lambda_0 = \Delta \lambda = \frac{2h}{m_e c}\sin^2(\theta/2) \]
where $\lambda_0$ and $\lambda_s$ are the wavelength of the incident and scattered X-rays respectively; $m_e$ is the mass of an electron.

Thomson scattering (i.e. scattering with $\lambda_s = \lambda_0$) can be explained in this model by assuming it results from scattering by electrons bound so tightly that the whole atom recoils. In this case we should use the mass $M$ of the whole atom and the Compton shift $\Delta \lambda$ becomes negligible.

\subsection{The wave nature of matter: the de Broglie hypothesis}
In 1923 it was generally accepted that electromagnetic radiation can exhibit both wave-like and particle-like behaviour. In 1923 / 1924 Louis de Broglie proposed that matter may also have this dual nature. He postulated that the Planck-Einstein formula for the energy of a photon
\[ E  = h\nu \]
is not a peculiar feature of photons but holds for all particles. This means there must be some way to associate a wavelength to a particle. This wavelength is called the \udef{de Broglie wavelength} in his honor and is given by
\[ \lambda = \frac{h}{p}. \]
What is unique to photons is the relation $\lambda\nu = c$ making this second relation redundant and setting $p = E / c$ for photons.

\subsubsection{Electron diffraction}
If we compare with the results of classical optics, we realise that we will only be able to verify de Broglie's hypothesis by observing interference and diffraction effects if the length of some relevant part of the measurement setup is comparable to the wavelength. So in order to hope to measure electron diffraction, it is useful to have an idea of the de Broglie wavelength of an electron.

Let us assume we are producing a beam of electrons by accelerating them across a potential difference $V_0$, so that the kinetic energy is $mv^2/2 = eV_0$. In the non-relativistic limit, the momentum is given by $p=mv$ and thus the de Broglie wave length is given by
\[ \lambda = \frac{h}{mv} = \frac{h}{\sqrt{2meV_0}} \approx \frac{12.3}{\sqrt{V_0[\text{Volts}]}}\si{\angstrom} \]

This is much smaller than any length we would usually come across, which is why we had not realised matter had a wave-like nature, but is comparable to the spacing between atoms in a crystal. Using the theory developed by Bragg, Laue and others for the diffraction of X-rays by crystals (which will be developed later in the section on solid state physics), George Paget Thomson (son of J.J. Thomson, who gave his name to Thomson scattering among many other things), and Clinton Davidson and Lester Germer independently gave experimental evidence that electrons do indeed exhibit a wave-like nature.

Subsequent experiments have shown that other particle - protons, neutrons and even molecules - can exhibit wave-like behaviour.

\subsection{Wave-packets}
\subsubsection{Form of the wave}
So we seem to be able to associate something that behaves like a wave to matter.

Meaning? Spread out in space.

TODO after fluid dynamics. Decomposition into packets.
\subsubsection{Correspondence principle}
TODO image of broad wave to Dirac delta.

For almost all macroscopic physics, classical physics is obviously still a very good approximation.

Classically not spread out in space, but localised.  thus with vanishing wavelength. Here we may draw the parallel with geometrical optics which is the short-wavelength limit of wave optics. 
\subsubsection{The Heisenberg uncertainty principle}
De Broglie's original paper.

\subsection{Double-slit experiments}
Thomas Young's double-slit experiment can neatly be summed up by figure TODO (include labels of light source and screen). The image on the screen is then something like figure TODO(b). TODO: move to optics??

Similar experiments can be used to show the wave nature of electrons and other particles. In these cases it is much easier to record individual particles, and we get an image on the screen resembling figure TODO.

With light the interference pattern can be explained by the fact that
\begin{enumerate}
\item The amplitudes of the light-waves can be added linearly;
\item The intensity is given by the square of the amplitude.
\end{enumerate}

In the classical view, the interpretation of these statements is relatively clear: the amplitudes of the light-waves are the amplitudes of the oscillations of the electric and magnetic fields. The intensity is the power per unit area and can be measured by our eyes (TODO: see optics!!).

If we try applying this to matter that is classically thought of as consisting of particles, we run into some problems: How come we can (only) measure single electrons? Electrons invariably show up as blips on our detectors, we seem not to be able to detect the waves directly.

In light of this we may think that electrons are in essence particles and the wave-like features we measure are simply the result of interactions between electrons we do not understand yet. This turns out not to be the case. Experiments have been performed with very weak sources over the course of weeks such that there is only ever at most one electron traveling through the double-slit apparatus. These experiments yield exactly the same interference patterns.

So it seems very likely that the two aspects of classical theory that allow these interference patterns to occur have quantum analogues. 

\subsubsection{Statistical interpretation}
Max Born hypothesised an interpretation 

\subsubsection{Superposition principle}


\section{Quantised quantities}
TODO: Planck postulated quantisation of energy. There are several other phenomena and experiments that can be explained much more easily if we assume certain quantities to be quantised.
\subsection{Atomic spectra}
\subsubsection{Bohr model of the hydrogen atom}
Quantised energy and angular momentum

\subsection{Franck and Hertz experiment}
Quantisation of energy.

\subsection{Stern-Gerlach experiment}
Angular momentum and spin

\section{A history of quantum theories}


In 1925 / 1926 two different (but ultimately equivalent) theories were put forward to describe the properties of quantum mechanical objects, \textit{matrix mechanics} developed by Werner Heisenberg, Max Born and Pascual Jordan, and \textit{wave mechanics} developed by Erwin Sch\"odinger. Both are particular forms of a more general formulation of quantum mechanics developed by Paul Adrian Maurice Dirac in 1930.


\section{Wave mechanics}
\subsection{Schr\"odinger equation}
\[ E= \frac{\vec{p}^2}{2m}+V(\vec{x}) \]
Make the following substitutions:
\[ E \to i \pd{}{t}, \qquad \vec{p} \to -i \vec{\nabla}, \qquad \vec{x} \to \vec{x} \]
And we get Schrödinger's equation
\[ \boxed{i \pd{\psi}{t} = \left(- \frac{\nabla^2}{2m}+V(x)\right)\psi} \equiv H\psi \]
We are going to study conservative systems, so $H$ does not depend on time.
\subsection{Bohr condition}
\[\rho(\bar{x},t) = \left|\psi(x)\right|^2 \geq 0 \]
\[ P(t) = \int_{\R^3} \diff{^3x}\rho(x,t) = \int \diff{^3x}\left|\psi(\bar{x},t)\right|^2 = 1 \]
So $\psi \in \mathcal{C}_2$
\subsection{Is the probability constant in time?}
\begin{align*}
\od{P}{t} &= \int_{\R^3}\diff{^3x}\pd{}{t}\left|\psi(x,t)\right|^2 = \int_{\R^3}\diff{^3x}\left(\pd{\psi^*}{t}\psi + \psi^*\pd{\psi}{t}\right) \\
&= \frac{i}{2M}\int_{\R^3} \diff{^3x}\left(\psi^*\left(\bar{\nabla}^2\psi\right)- \cancel{V\psi^*\psi\cdot2M} - \left(\bar{\nabla}^2\psi^*\right)\psi + \cancel{V\psi^*\psi\cdot2M}\right) \\
&= \frac{i}{2M}\int_{\R^3}\diff{^3x}\bar{\nabla}\cdot \left(\psi^*\left(\bar{\nabla}\psi\right)- \left(\bar{\nabla}\psi^*\right)\psi\right) \\
\vec{j}(\bar{x},t) &\equiv \frac{-i}{2M}\left(\psi^*\left(\bar{\nabla}\psi\right)- \left(\bar{\nabla}\psi^*\right)\psi\right)
\end{align*}
So we get
\[ \od{P}{t} = -\int_{\R^3}\vec{\nabla}\cdot \vec{j}\diff{^3x} = -\oint_{\partial \R^3} \vec{j}\cdot \vec{u}_j \diff{^2\sigma} = 0 \]
Where
\[ \begin{cases}
\vec{j}(\bar{\infty},t) = 0\\
\psi(\bar{\infty},t) = 0 \qquad \text{because} \quad \psi \in \mathcal{C}_2
\end{cases} \]
(Assumption of conservative system important!)
\subsection{Continuity equation}
\[ \pd{}{t}\left|\psi(\bar{x},t)\right|^2 + \vec{\nabla}\cdot\vec{j} = 0 \]
Conservation of probability = continuity equation

\subsection{One-dimensional examples}
\subsubsection{The linear quantum harmonic oscillator.}
We have a potential of the form
\[ V(x) = \frac{1}{2}kx^2 \]
so the Schrödinger equation becomes
\[ - \frac{\hbar^2}{2m}\od[2]{\psi(x)}{x} + \frac{1}{2}kx^2\psi(x) = E\psi(x) \]
We can immediately see that all eigenfunctions correspond to bound states of positive energy.
\subsection{Three-dimensional examples}



\subsection{General solution of the free Schrödinger equation}
\[ \boxed{i \pd{}{t}\psi(x,t) = -\frac{\vec{\nabla}^2}{2M}\psi(x,t)} \]
\subsubsection{Derive a basis of eigenstates} of $H = - \frac{\nabla^2}{2M}$.\par
Let's suppose that there is a solution of the form $\psi(x,t) = \chi(t)\varphi(\vec{x})$. Then
\[ i \pd{}{t}\psi = \varphi(x)i \pd{\chi(t)}{t} = -\chi(t) \frac{\nabla^2}{2M}\varphi(x) = E\chi(t)\varphi(x) \]
\[ i \frac{1}{\chi(t)}\pd{\chi}{t} = - \frac{1}{\varphi}\frac{\vec{\nabla}^2}{2M}\varphi = E = \text{const.} \]
\[ \begin{cases}
\pd{\chi}{t} = -iE\chi \qquad \to \qquad \chi(t) = \chi(0)e^{-iEt} \\
\vec{\nabla}^2\varphi = -2ME\varphi \qquad \to \qquad \varphi(x) = \varphi(0)e^{i \vec{k}\cdot\vec{x}}
\end{cases} \]
Dispersion relation: $\left|\vec{k}\right|^2 = 2ME, \qquad E_k = \frac{\left|\vec{k}\right|^2}{2M}$
\[ \psi(x,t) \sim e^{-i(E_kt - \vec{k}\cdot\vec{x})} \]
\subsubsection{General solution of S.E.}
\[ \psi(\bar{x},t) = \left.\frac{1}{(2\pi)^{3/2}}\int \diff{^3k}\tilde{\psi}(k)e^{-i(E_kt- \vec{k}\cdot \vec{x})}\right|_{E_k = \frac{|\vec{k}|^2}{2M}} \]
\subsection{The Schrödinger equation is nonrelativistic}
The SE contains a first order derivative in time, but a second order derivative in space.
\[ \partial_\mu = (\partial_0, \vec{\nabla}) \qquad \to \qquad \partial_\mu\partial^\mu = \Box \]

\section{Matrix mechanics}

\section{Synthesis by Dirac and von Neumann}
TODO + losing physical intuition

\chapter{The formalism of quantum mechanics}
\section{Postulates}
Based on the exposition so far, a set of axioms may be postulated that provide a purely mathematical foundation of quantum mechanics. These are known is the Dirac-von Neumann axioms. They may be formulated in many equivalent ways.

\subsection{The space of states}
At any given time, a quantum system is in a state. We postulate that the space of states, $\mathcal{H}$, is a complex Hilbert space. This means it is a complex inner product space that is also a complete metric space with respect to the norm induced by the inner product.

Additionally any two elements in the Hilbert space that are multiples of each other refer to the same state. Thus a quantum state can be seen as the equivalence class of all vectors in the Hilbert space that are (non-zero) multiples of each other. These equivalence classes are elements of the associated projective Hilbert space.

It is however enough to postulate the existence of a Hilbert space. The fact that we are really dealing with a projective Hilbert space follows from the other postulates.

So the first postulate reads:
\remark{The space of states is a (projective) Hilbert space.}

\subsubsection{The Dirac bra-ket notation}
The inner product defined on the Hilbert space is conventionally written using Dirac's bra-ket notation:
\[ \braket{\cdot}{\cdot}: \mathcal{H} \times \mathcal{H} \to \C \]

This can be seen as the multiplication of two objects, the bra $\bra{\phi}$ and the ket $\ket{\psi}$, where $\psi$ and $\phi$ are arbitrary vectors in the Hilbert space.

TODO: using dual spaces.
TODO: $\braket[\hat{A}]{\psi}{\psi}$

\subsection{Measuring observables}
One of the main features of quantum mechanics is its acknowledgment of the importance of the measurement process. In general the act of measuring can change the state of the system. Sometimes we can infer things about the system purely from the things it sends out, but often we need to probe the system by sending something in and seeing what comes out. The probe we send in can obviously alter the system.

Quantum mechanics also makes quite a surprising assertion:
\remark{If the same observable is measured a second time, the system does not change state a second time.}

Many of the weird aspects of quantum mechanics come from this postulate, often together with the following one:

\remark{Every observable $A$ has a linear transformation $\hat{A}$ on the Hilbert space associated to it such that the expectation value of $A$ measured in a system in state $\psi$ is a real number given by
\[ \expval{A} = \frac{\braket[\hat{A}]{\psi}{\psi}}{\braket{\psi}{\psi}} \]}

The linear transformation $\hat{A}: \mathcal{H} \to \mathcal{H}$ is usually called the operator. The state $\hat{A}\psi$ is in general \textbf{NOT} the result of a measurement of $A$.

We now explore some of the consequences of these two postulates.

\begin{itemize}
\item \textbf{The operators are Hermitian}.
The postulate states that the expectation value is real. If the numerator is real, then it must equal its complex conjugate. Since the inner product is conjugate symmetric, the following equalities hold:
\[ \braket[\hat{A}]{\psi}{\psi} = \braket{\psi}{\hat{A}\psi} = \overline{\braket{\psi}{\hat{A}\psi}} = \braket{\hat{A}\psi}{\psi} \]
which assert the very definition of a Hermitian operator.
\item \textbf{Any state can be written as a sum of eigenstates}.
Thanks to the spectral theorem we know that every Hermitian operator has a complete orthonormal set of eigenvectors $\{a_i\}_{i\in I}$. TODO ?
\item \textbf{After a measurement of $A$ the system is in an eigenstate of $\hat{A}$}.
Assume the system was first in a state $\ket{\psi}$ and TODO
\item TODO probability to find it in any eigenstate.
\item TODO no need for projectivity
\end{itemize}

\subsection{Time evolution}

\section{Important theorems}
\subsection{Existence and categoricity of quantum theories}

\subsection{Equivalent representations}
Say $\Psi, \Psi'$ equivalent representations (same probabilities), then $\Psi'= U\Psi$ with $U$ unitary and linear:
\begin{align*}
(U\Psi_1, U\Psi_2) &= (\Psi_1, \Psi_2) \\
U(c_1\Psi_1 + c_2 \Psi_2) &= c_1U\Psi_1 + c_2U\Psi_2
\end{align*}
or antiunitary and antilinear:
\begin{align*}
(U\Psi_1, U\Psi_2) &= (\Psi_1, \Psi_2)^* \\
U(c_1\Psi_1 + c_2 \Psi_2) &= c_1^*U\Psi_1 + c_2^(U\Psi_2
\end{align*}

For antilinear operators we redefine adjoint $A^\dagger$.
\begin{itemize}
\item Linear:
\[ (\Psi_1, L^\dagger \Psi_2) \equiv (L\Psi_1, \Psi_2) \]
\item Antilinear:
\[ (\Psi_1, A^\dagger \Psi_2) \equiv (A\Psi_1, \Psi_2)^* = (\Psi_2, A\Psi_1) \]
\end{itemize}
Then the conditions of unitarity and antiunitarity are both expressed as
\[ U^\dagger = U^{-1} \]

In particular the identity operator $U = 1$ is unitary and linear.
\subsection{Symmetries}
TODO see Weinberg

Operator on ray vs operator on state.

Changing phase. Also superselection.

Connected Lie groups must be represented in the physical Hilbert space by unitary (rather than antiunitary) operators $U(T(\theta))$

\section{A reconstruction of quantum mechanics via quantum information}



\section{The Hamiltonian}
TODO Minimal coupling principle
\section{Angular momentum}
\section{Pictures}
Say we have a quantum mechanical system with the following initial conditions:
\[ \begin{cases}
\ket{\psi(t_0)} = \ket{\psi_0} \\
A(t_0) = A_0
\end{cases} \]
where $A$ is an operator

\subsection{Schrödinger picture}
In the Schrödinger picture the time dependence is carried by the states, according to the Schrödinger equation.  
\[ \begin{cases}
\ket{\psi(t)}_s = U(t,t_0)\ket{\psi(t_0)}_s = U(t,t_0)\ket{psi_0} \\
A_s(t) = A_s(t_0) = A_0
\end{cases} \]
(For an arbitrary operator $A$ (?))
\remark{The Schrödinger eq: $i \od{}{t}\ket{\psi(t)}_s = H\ket{\psi(t)}_s$}
So the evolution operator $U(t,t_0)$
\[ \boxed{i \od{U}{t}(t,t_0) = HU(t,t_0)} \]
\[ \begin{cases}
U(t_0,t_0) = \mathbb{1}\\
U(t,t_0) = e^{-iH(t-t_0)}
\end{cases} \qquad \to \qquad \left[U(t,t_0), H\right] = 0\]

\subsection{Heisenberg picture}
\[ \begin{cases}
\ket{\psi(t)}_H = \ket{\psi(t_0)}_H = \ket{\psi_0} \\
A_H(t) = U^\dagger(t,t_0)A_H(t_0)U(t,t_0) = U^\dagger(t,t_0)A_0U(t,t_0)
\end{cases} \]
Which is equivalent with requiring
\[ i \od{}{t}A_H(t) = i \left(\od{U^\dagger}{t}A_0U+U^\dagger A_0 \od{U}{t}\right) = \left[A_H(t), H\right] \]
Schrödinger and Heisenberg are equivalent because related by a unitary transformation.
\[ \begin{cases}
\ket{\psi(t)}_H = H^\dagger(t,t_0)\ket{\psi(t)}_s \\
A_H(t) = U^\dagger(t,t_0)A_s(t)U(t,t_0)
\end{cases} \]
\[ \boxed{\braket[A]{\psi}{\psi}_H = \braket[A]{\psi}{\psi}_S} \]
So still same physics


\subsection{The interaction picture}
Assume that the Hamiltonian is time-independent and can be expressed as the sum
\[ H = H_0 + H_I \]
where $H_0$ acting alone yields a soluble problem.

We define the state vector in the interaction picture as
\[ \ket{\Psi_I(t)} \equiv e^{iH_0t / \hbar}\ket{\Psi_S(t)} \]
We can do this because $e^{iH_0t / \hbar}$ is a unitary transformation carried out at time $t$.

An arbitrary matrix element in the Schrödinger picture can be written as 
\[ \braket[\hat{O}_S]{\Psi'_S(t)}{\Psi_S(t)} = \braket[e^{iH_0t / \hbar}\hat{O}_S e^{-iH_0t / \hbar}]{\Psi'_I(t)}{\Psi_S(t)} \]
for an arbitrary operator $\hat{O}_S$ in the Schrödinger picture.
This suggests defining the operator $\hat{O}_I$ in the interaction picture as
\[ \hat{O}_I(t) \equiv e^{i H_0 t / \hbar}\hat{O}_S e^{-iH_0t / \hbar} \]







\subsection{The interaction picture}
\[ H = H_0 + H_\text{int} \]
\[ \begin{cases}
U_0(t,t_0) = e^{-iH_0(t-t_0)} \\
U(t,t_0) = e^{-iH(t-t_0)}
\end{cases} \]
The picture is defined (in terms of Schrödinger picture, but that is arbitrary) by
\[ \begin{cases}
\ket{\psi(t)}_I = U_0^\dagger(t,0)\ket{\psi(t)}_s \\
A_I(t) = U_0^\dagger(t,0)A_s(t)U_0(t,0)
\end{cases} \qquad \to \qquad \text{interaction picture}\]
Now we write the evolution eq. of $\ket{\psi}, A$ in interaction picture.
\begin{align*}
i \od{}{t}\ket{\psi(t)}_I &= i \od{}{t}\left(U_0^\dagger(t,0)U(t,t_0)\right)\ket{\psi_0} \\
&= i \left(U_0^\dagger(t,0)(H-H_0)U(t,t_0)\right)\ket{\psi_0} \\
&= U_0^\dagger(t,0)H_\text{int}\ket{\psi(t)}_s \\
&= \underbrace{U_0^\dagger H_\text{int}U_0(t,0)}_{\equiv H^I_\text{int}(t)}\ket{\psi(t)}_I
\end{align*}
Evolution equation of $\ket{\psi(t)}_I$
\[ \boxed{i \od{}{t}\ket{\psi(t)}_I = H^I_\text{int}(t)\ket{\psi(t)}_I} \]
In general
\[ \left[H^I_\text{int}(t),H\right] \neq 0 \neq \left[H^I_\text{int}(t),U(t,t_0)\right] \]

\subsubsection{Evolution of operators in the interaction picture}
\begin{align*} i \od{A_I(t)}{t}(t) &= i \od{}{t}\left(U_0^\dagger(t,0)A_s(t_0)U_0(t,0)\right) \\
&= - U_0^\dagger(t,0)H_0A_0U_0(t,0) + U_0^\dagger(t,0)A_0H_0U_0(t,0) \\
&= \left[U^\dagger_0(t,0)A_0U_0(t,0),H_0\right] \equiv \left[A_I(t),H_0\right]
\end{align*}

Evolution of operators in interaction picture
\[ \boxed{i \od{}{t}A_I(t) = \left[A_I(t),H_0\right]} \]

Now consider $\varphi_I(t)$. Then
\[ i \od{}{t}\varphi_I(t) = \left[\varphi_I(t),H_0\right] \]
Which is the equation for the free theory in the Heisenberg picture. So the field in the interaction picture satisfies the free field equation.
\begin{align*}
\left(\Box + m^2\right)\varphi_H(t) &= -\lambda\varphi_H^3(t) \qquad \to \qquad \text{Heisenberg picture} \\
\left(\Box + m^2\right)\varphi_I(t) &= 0 \qquad \to \qquad \varphi_I \sim ae^{-ikx} + a^\dagger e^{ikx}
\end{align*}
Operators evolve like free theory. Now all of the complication is in evolution of states.

\subsubsection{States are not constant}
\begin{align*}
\ket{\psi(t)}_I &= U_0^\dagger(t,0)U(t,t_0)\ket{\psi(t_0)}_s \\
&= U_0^\dagger (t,0)U(t,t_0)U_0(t_0)\ket{\psi(t_0)}_I \\
&\equiv U_I(t,t_0)\ket{\psi(t_0)}_I
\end{align*}
So
\begin{align*}
U_I(t,t_0) &\equiv U_0^\dagger U(t,t_0)U_0(t_0,0) \\
&= e^{iH_0t}e^{-iH(t-t_0)}e^{-iH_0t_0}
\end{align*}
In general
\[ \left[H,H_0\right] \neq 0 \neq \left[H_0,H_\text{int}\right] \]

So
\[ i \od{}{t}U_I(t,t_0) = H^I_\text{int}(t)U_I(t,t_0) \]
\[ \begin{cases}
U_I(t_0,t_0) = \mathbb{1} \\
U_I(t,t_0) = \mathbb{1} - i \int^t_{t_0}\diff{\tau}H_\text{int}^I(\tau)U_I(\tau,t_0)
\end{cases} \]

\subsubsection[Perturbative solution of U\_I]{Perturbative solution of $U_I$}
Let's suppose that $H_\text{int} \propto \lambda$ with $\lambda \ll 1$.
\[U_I(t,t_0) = \sum^\infty_{u=0}C_u(t,t_0) \qquad \to \qquad C_u \sim \lambda^u \]
$U_I^{(N)}$ is the approx solution up $\lambda^N$
\[ U_I(t,t_0) = U_I^{(N)} + \mathcal(O)(\lambda^{N+1}) \]
\[ U_I^{(N)}(t,t_0) = \mathbb{1} - i\int^t_{t_0} \diff{\tau}H^I_\text{int}(\tau)U_I^{(N)}(\tau,t_0) \]
But $H^I_\text{int}$ depends on $\lambda$, so the right side is of order $N+1$, so we actually have
\[ U_I^{(N)}(t,t_0) = \mathbb{1} - i\int^t_{t_0} \diff{\tau}H^I_\text{int}(\tau)U_I^{(N-1)}(\tau,t_0) \]
This is a recursive relation. If we know the solution at $\mathcal{O}(N-1)$, we can obtain the solution at $\mathcal{O}(N)$

\begin{enumerate}
\item Solution of order $0$ ($\lambda=0, H_\text{int}=0$)
\[ U_I^{(0)}(t,t_0) = C_0 = \mathbb{1} \]
\item Solution of order $1$
\[ U_I^{(1)}(t,t_0) = \mathbb{1} - i \int^t_{t_0} \diff{\tau}H_\text{int}^I(\tau)\cdot\mathbb{1} = C_0 + C_1 \]
\item Second order term
\begin{align*}
U_I^{(2)}(t,t_0) &= \mathbb{1} - i\int^t_{t_0}\diff{\tau}H_\text{int}^I(\tau)U_I^{(1)}(\tau,t_0) \\
&= \mathbb{1} - i\int^t_{t_0}\diff{\tau}H_\text{int}^I(\tau)\left(\mathbb{1} - i \int^t_{t_0} \diff{\tau_1}H_\text{int}^I(\tau_1)\right) \\
&= \mathbb{1} - i\int^t_{t_0}\diff{\tau}H_\text{int}^I(\tau)+ (-i)^2\int^t_{t_0}\diff{\tau}\int^\tau_{t_0}\diff{\tau_1}H_\text{int}^I(\tau)H_\text{int}^I(\tau_1) \\
&= C_0 + C_1 + C_2
\end{align*}
\end{enumerate}
So in general the $u^\text{th}$ order term in the expansion is given by
\[ C_u = (-i)^u\int^t_{t_0}\diff{\tau_1}\int^{\tau_1}_{t_0}\diff{\tau_2}\ldots\int^{\tau_{u-1}}_{t_0}\diff{\tau_u}H_\text{int}^I(\tau_1)\ldots H_\text{int}^I(\tau_u) \]
And we have
\[ U_I^{(N)}(t,t_0) = \sum^N_{u=0}C_u \qquad U_I(t,t_0)=\sum^\infty_{u=0}C_u \]

\subsubsection{Put it in a better formalism}
\[ C_2(t,t_0) = (-i)^2\int^t_{t_0}\diff{\tau_1}\int^{\tau_1}_{t_0}\diff{\tau_2}H_\text{int}^I(\tau_1)H_\text{int}^I(\tau_2) \]
[Picture horizontal lines, plot t1 vs. t2]
\[ C_2(t,t_0) = (-i)^2\int^t_{t_0}\diff{\tau_2}\int^{t}_{\tau_2}\diff{\tau_1}H_\text{int}^I(\tau_1)H_\text{int}^I(\tau_2) \]
[Picture vertical lines t2 vs. t1]
\[ C_2(t,t_0) = (-i)^2\int^t_{t_0}\diff{\tau_1}\int^{t}_{\tau_1}\diff{\tau_2}H_\text{int}^I(\tau_2)H_\text{int}^I(\tau_1) \]
I can define
\[ C_2 = \frac{(-i)^2}{2}\int^t_{t_0}\diff{\tau_1}\int^{t}_{\tau_1}\diff{\tau_2}H_\text{int}^I(\tau_2)H_\text{int}^I(\tau_1) + \frac{(-i)^2}{2}\int^t_{t_0}\diff{\tau_2}\int^{t}_{\tau_2}\diff{\tau_1}H_\text{int}^I(\tau_1)H_\text{int}^I(\tau_2) \]

\begin{align*}
C_2 &= \frac{(-i)^2}{2!}\int^t_{t_0}\diff{\tau_1}\int^{t}_{t_0}\diff{\tau_2}\left\{\theta(\tau_1-\tau_2)H_\text{int}^I(\tau_1)H_\text{int}^I(\tau_2) + \theta(\tau_1-\tau_2)H_\text{int}^I(\tau_2)H_\text{int}^I(\tau_1)\right\} \\
&= \frac{(-i)^2}{2!}\int^t_{t_0}\diff{\tau_1}\diff{\tau_2}T\left(H_\text{int}^I(\tau_1)H_\text{int}^I(\tau_2)\right)
\end{align*}

Lecture 14/11
Interaction picture
\[ \begin{cases}
i \od{}{t}A_I(t) = \left[A_I(t),H_0\right] \qquad \to \qquad \text{free eq.} \\
i \od{}{t}\ket{\psi(t)}_I = H_\text{int}^I(t)\ket{\psi(t)}_I
\end{cases} \]
\[\ket{\psi(t)}_I = U_I(t,t_0)\ket{\psi(t_0)}_I  \]
\[ i \od{U_I(t)}{t} = H_\text{int}^IU_I(t,t_0) \]
Would be a simple differential equation if it commuted. Solution:
\begin{align*}
U_I(t,t_0) &= T \left(\exp{-i\int^t_{t_0}\diff{\tau}H^I_\text{int}(\tau)}\right) \\
&= \sum^\infty_{n=0} \frac{(-i)^n}{n!}\int^t_{t_0}\diff{\tau_1}\ldots \diff{\tau_n}T \left(H_\text{int}^I(\tau_1)\ldots H_\text{int}^I(\tau_n)\right)
\end{align*}


\section{Path integral formulation}
\section{Other axiomatizations}
\subsection[Dirac - von Neumann axioms in terms of a C*-algebra]{Dirac - von Neumann axioms in terms of a $C^*$-algebra}

\chapter{Measurement}
\url{https://arxiv.org/pdf/0911.2539.pdf}
\url{https://arxiv.org/pdf/1708.00769.pdf}
\url{https://chaos.if.uj.edu.pl/~karol/geometry.htm}

\section{Adiabatic theorem}
\url{https://arxiv.org/pdf/2003.03063.pdf}
\url{https://link.springer.com/content/pdf/10.1134/1.1352758.pdf}
\url{https://arxiv.org/pdf/quant-ph/0603175.pdf}

\section{Projective measurement}

\section{Generalised measurement}

\section{Continuous measurement}
\url{https://arxiv.org/pdf/quant-ph/0611067.pdf}
Book: Quantum Trajectories and Measurements in Continuous Time

\subsection{Quantum zeno}
\url{https://www.fi.muni.cz/usr/buzek/zaujimave/home.pdf}
\url{https://iopscience.iop.org/article/10.1088/1742-6596/196/1/012018/pdf}

\chapter{Quantum statistical mechanics}
We will be considering a quantum mechanical system of $N$ particles. In the Schrödinger picture and the position representation, the system is described by a wave function $\Psi(q_1, q_2, \ldots, q_N, t)$ which satisfies the Schrödinger equation
\[ i\hbar \pd{}{t}\Psi(q_1, q_2, \ldots, q_N, t) = H \Psi(q_1, q_2, \ldots, q_N, t) \]

\section{Bosons and fermions}
\subsection{Totally symmetric and antisymmetric wave functions}
\subsection{Pauli exclusion principle}

\section{Fermi gas}
TODO fermi temperature

\chapter{Approximation methods}
\section{For stationary problems}
\section{For time-dependent problems}

\chapter{Interactions of quantum systems}
\section{With radiation}
\section{With external electric and magnetic fields}
\section{Quantum collision theory}


\chapter{Relativistic wave equations}
In this section we follow the mostly minus convention
\[ \eta_{\mu\nu} = \diag(+1,-1,-1,-1). \]
We will also be setting $c$ and $\hbar$ to 1.

\section{Klein-Gordon equation}

\subsection{Derivation}
We start from the relativistic equation
\[ p^\mu p_\mu = m^2 = E^2 - \left|\vec{p}\right|^2 \]
which we rewrite as
\[ E^2 = m^2 + \left|\vec{p}\right|^2. \]

We quantise as usual by substituting the classical quantities with the usual operators:
\[ - \pd[2]{}{t}\psi = \left(-\vnabla^2+m^2\right)\psi \]
This is equivalent to writing
\[ \left[\left(\partial_0^2 - \vnabla^2\right)+m^2\right]\psi = 0 = \left[\partial_\mu\partial^\mu + m^2\right]\psi \]
and so we get the famous Klein-Gordon equation
\[ \boxed{\left(\Box + M^2\right)\psi = 0}. \]

This equation is manifestly covariant. All the components of the equation are scalars.
TODO rest of section in chosen notation

$\psi(x)$ is a scalar function
 \[ \psi'(x') \qquad \to \qquad \psi(x) \]
 And $M$ and $\Box$ are scalars
 \[ \begin{cases}
 M \qquad \to \qquad M'=M \\
 \Box \qquad \to \qquad \Box' = \Box
 \end{cases} \]
 So the Klein-Gordon equation transforms in the following way:
 \[\left(\Box + M^2\right)\psi(x) = 0 \qquad \overset{\text{Poincaré}}{\to} \qquad \left(\Box' + M^{\prime2}\right)\psi'(x) = \left(\Box + M^2\right)\psi(x) = 0\]
\subsection{The continuity equation}
Any function satisfying the Klein-Gordon equation must also satisfy a continuity equation
\[ \partial_0\rho + \vnabla\vec{j}=0 \]
In order to prove this, we start by asserting the Klein-Gordon equation holds for both $\psi$ and its complex conjugate $\psi^*$. We can then multiply both sides of the equations with either $\psi$ or $\psi^*$ to get
\[ \begin{cases}
\psi^*\left(\Box + m^2\right)\psi = 0 \\ \psi \left(\Box + m^2\right)\psi^* = 0
\end{cases} \]
Subtracting these two equations gives
\begin{align*}
 0 &= \psi^*\Box\psi + \cancel{m^2\psi^*\psi} - \psi\Box\psi^* - \cancel{m^2\psi\psi^*} \\
&= \psi^*(\Box\psi) - \psi(\Box\psi^*)
\end{align*}
Integrating by parts and expanding $\partial_\mu$ as $(\partial_0, -\vnabla)$ gives
\begin{align*}
0 &= \partial_0 \left[\psi^*(\partial_0\psi)- (\partial_0\psi^*)\psi\right] + \cancel{(\partial_0\psi)(\partial_0\psi^*)} - \cancel{(\partial_0\psi^*)(\partial_0\psi)} - \vnabla\left[\psi^*(\vnabla\psi) - (\vnabla\psi^*)\psi\right] -\cancel{(\vnabla\psi)(\vnabla\psi^*)} + \cancel{(\vnabla\psi^*)(\vnabla\psi)} \\
&= \partial_0 \left[\psi^*(\partial_0\psi)- (\partial_0\psi^*)\psi\right]  - \vnabla\left[\psi^*(\vnabla\psi) - (\vnabla\psi^*)\psi\right]
\end{align*}
Defining the quantities
\[\begin{cases}
\rho \equiv \frac{1}{2}\left[\psi^*(\partial_0\psi)-(\partial_0\psi^*)\psi\right] \equiv \frac{i}{2}\psi^*(\overset{\leftrightarrow}{\partial}_0)\psi \\
\vec{j} = - \frac{1}{2}\left[\psi^*(\vnabla\psi)-(\vnabla\psi^*)\psi\right] \equiv - \frac{2}{i}\psi^*\overset{\leftrightarrow}{\vnabla}\psi
\end{cases}\]
where we have also introduced the notation $\overset{\leftrightarrow}{\partial}_0$ and $\overset{\leftrightarrow}{\vnabla}$, we see that we have derived a continuity equation
\[ \boxed{\partial_0\rho + \vnabla\vec{j}=0}\]
Which can also be written as
\[ \partial_\mu j^\mu =0 \]
where $j^\mu = (\rho,\vec{j})$. For any continuity equation there is a conserved quantity. In casu
\[ Q = \int_{\R^3}\diff{^3x}\rho(x,t) \]
is conserved:
\[ \od{Q}{t} = \int_{\R^3} \diff{^3x}\pd{}{t}\rho(x,t) = -\int_{\R^3}\diff{^3x}\vnabla\cdot\vec{j} = - \oint_{\partial\R^3}\diff{^2\sigma} \vec{j}\cdot \vec{u}_\sigma = 0 \]
TODO redo with chosen notation
Because $\rho$ can be positive or negative, we cannot associate $Q$ to a probability.

\subsection{General solution}
The goal will be to write the solutions to the Klein-Gordon equation in a more illuminating form. A scalar solution $\varphi(x)$ can be written as the Fourier transform of $\tilde{\varphi}(k)$.
\[\varphi(x) = \frac{1}{(2\pi)^2} \int \diff{^{4}k}e^{-ik^\mu x_\mu}\tilde{\varphi}(k)\]
Conversely $\tilde{\varphi}(k)$ can be written as
\[\tilde{\varphi}(k) = \frac{1}{(2\pi)^2} \int \diff{^{4}k}e^{ik^\mu x_\mu}\varphi(x)\]
This is still a scalar function because $\diff{^{4}k}$ and $k^\mu x_\mu$ are invariant and $\varphi(x)$ was required to be scalar.

Now we require that $\varphi(x)$ is a solution of the Klein-Gordon equation
\[(\Box + m^2)\varphi(x) = \int \frac{\diff{^4k}}{(2\pi)^2} (-k^2 + m^2)e^{-ik^\mu x_\mu}\tilde{\varphi}(k) = 0 \]
This is equivalent to requiring
\[ \int_{k^2 \neq m^2} \frac{\diff{^4k}}{(2\pi)^2}e^{-ik^\mu x_\mu}\tilde{\varphi}(k) = 0. \]
Using this we can rewrite the expression for $\varphi(x)$
\begin{align*}
\varphi(x) &= \frac{1}{(2\pi)^2} \int \diff{^{4}k}e^{-ik^\mu x_\mu}\tilde{\varphi}(k) \\
&= \frac{1}{(2\pi)^2} \int_{k^2 \neq m^2} \diff{^{4}k}e^{-ik^\mu x_\mu}\tilde{\varphi}(k) + \frac{1}{(2\pi)^2} \int \diff{^{4}k}e^{-ik^\mu x_\mu}\delta(k^2-m^2)\tilde{\varphi}(k) \\
&= \frac{1}{(2\pi)^2} \int \diff{^{4}k}e^{-ik^\mu x_\mu}\delta(k^2-m^2)\tilde{\varphi}(k)
\end{align*}
Using the uniqueness of the Fourier transform, we can write $\tilde{\varphi}(k)$ as
\[\tilde{\varphi}(k) = \delta(k^2-m^2)\tilde{f}(k)\]
for a function $\tilde{f}(k)$.

The next step is to rewrite the delta function.
\begin{align*}\delta(k^2-m^2) &= \delta(k_0^2-|\vec{k}|^2-m^2) \\
&= \delta(k_0^2-\omega_k^2) = \frac{\delta(k_0-\omega_k)+\delta(k_0+\omega_k)}{2\omega_k}\end{align*}
Where we have defined $\omega_k \equiv (m^2+|\vec{k}|^2)^{1/2}$. This is the relativistic energy. The notation (writing it as a frequency) makes sense because we have set $\hbar = 1$.

The last equality follows from the result of composing the $\delta$-function with a smooth, continuously differentiable function $g$:
\[ \delta(g(x)) = \sum_i \frac{\delta(x-x_i)}{|g'(x_i)|}\]
where $x_i$ are the simple roots of $g$. In this case of course $k_0$ is the variable and $\pm\omega_k$ are the roots.

Plugging all of this back into the original equation yields
\begin{align*}
\varphi(x) &= \frac{1}{(2\pi)^2} \int \frac{\diff{^{4}k}}{2\omega_k}\left(e^{-ik^\mu x_\mu}\delta(k_0 - \omega_k)\tilde{f}(k) + e^{-ik^\mu x_\mu}\delta(k_0 + \omega_k)\tilde{f}(k)\right) \\
&= \frac{1}{(2\pi)^2} \int \frac{\diff{^{3}k}}{2\omega_k}\left(e^{-i(\omega_kx_0 - \vec{k}\cdot\vec{x})}\tilde{f}(\omega_k, \vec{k}) + e^{i(\omega_kx_0 + \vec{k}\cdot\vec{x})}\tilde{f}(-\omega_k, \vec{k})\right)
\end{align*}
Seeing as we are integrating over all $\vec{k}$s anyway, We can apply the transformation $\vec{k} \to -\vec{k}$
\[\varphi(x) = \frac{1}{(2\pi)^2} \int \frac{\diff{^{3}k}}{2\omega_k}\left(e^{-i(\omega_kx_0 + \vec{k}\cdot\vec{x})}\tilde{f}(\omega_k, -\vec{k}) + e^{i(\omega_kx_0 - \vec{k}\cdot\vec{x})}\tilde{f}(-\omega_k, -\vec{k})\right)\]
We now define the following functions:
\[ \begin{cases}
a(\vec{k}) = \frac{\tilde{f}(\omega_k, -\vec{k})}{\sqrt{(2\pi)(2\omega_k)}} \\
b^*(\vec{k}) = \frac{\tilde{f}(-\omega_k, -\vec{k})}{\sqrt{(2\pi)(2\omega_k)}}
\end{cases} \]
so we get the following general solution of the Klein-Gordon equation:
\[\varphi(x) = \frac{1}{(2\pi)^{3/2}} \int \frac{\diff{^{3}k}}{\sqrt{2\omega_k}}\left(a(k)e^{-ikx} + b^*(k)e^{ikx}\right)_{k_0 = \omega_k}\]
Some remarks about this solution:
\begin{enumerate}
\item The function $\varphi(x)$ is a scalar, even if it does not look like one. We started with a scalar function and only performed algebraic manipulations.
\item We associate $a(k)e^{-ikx}$ with positive energy and $b^*(k)e^{ikx}$ with negative energy, so we can perform the following decomposition:
\[ \varphi(x) = \varphi_+(x) + \varphi_-(x) \sim e^{-ikx} + e^{ikx} \]
\item We can further illustrate the association of $\varphi_\pm$ with energies:
\begin{align*}
i\partial_0\varphi_+ &\approx i\partial_0(e^{-ikx}a(k)) = +\omega_k(e^{-ikx}a(k)) \\
i\partial_0\varphi_- &\approx i\partial_0(e^{ikx}b^*(k)) = -\omega_k(e^{ikx}b^*(k))
\end{align*}
\item We solved the Klein-Gordon equation generally for complex $\varphi$. If we want that $\varphi$ be real ($\varphi(x) = \varphi^*(x)$), we require the following:
\begin{align*}
\varphi(x)^* &= \frac{1}{(2\pi)^{3/2}} \int \frac{\diff{^{3}k}}{\sqrt{2\omega_k}}\left(a^*(k)e^{ikx} + b(k)e^{-ikx}\right)_{k_0 = \omega_k} \\
&= \frac{1}{(2\pi)^{3/2}} \int \frac{\diff{^{3}k}}{\sqrt{2\omega_k}}\left(a(k)e^{-ikx} + b^*(k)e^{ikx}\right)_{k_0 = \omega_k} = \varphi(x)
\end{align*}
which means that $a(k) = b(k)$ and $a^*(k) = b^*(k)$, so that
\[\varphi_\R(x) = \frac{1}{(2\pi)^{3/2}} \int \frac{\diff{^{3}k}}{\sqrt{2\omega_k}}\left(a(k)e^{-ikx} + a^*(k)e^{ikx}\right)_{k_0 = \omega_k}\]
\end{enumerate}

\subsubsection{Expressions for $a(k)$ and $a^*(k)$}
\paragraph{In a real field.}
\begin{align*}
a(p) &= \left.\int \frac{\diff{^3x}}{\sqrt{2\omega_p}}\left(i\partial_0\varphi + \omega_p\varphi\right)\right|_{p_0 = \omega_p} \\
a^*(p) &= \left.\int \frac{\diff{^3x}}{\sqrt{2\omega_p}}\left(-i\partial_0\varphi + \omega_p\varphi\right)e^{-ipx}\right|_{p_0 = \omega_p}
\end{align*}
\paragraph{In a complex field.}
The complex case is totally analogous TODO

\subsection{Interpretation of the Klein-Gordon equation}
TODO
\remark{Complex scalar solution of KG eq. can be associated to a CHARGED (EM) SPIN 0 particle}

\subsubsection{Coupling to an external electromagnetic field}
External (given, fixed) EM field
\[ A^\mu = (A^0, \bar{A}) \]
\[ \begin{cases}
\bar{E} = -\bar{\nabla}A_0 - \partial_0 \bar{A} \\
\bar{B} = \bar{\nabla}\times\bar{A}
\end{cases} \]
\begin{definition}
The \udef{minimal coupling} with an (external) EM field $A^\mu$ is obtained by the following substitution:
\[ \partial_\mu \to D_\mu \equiv (\partial_mu + iqA_\mu) \]
We call $D_\mu$ the \udef{covariant derivative} 
\end{definition}
The minimal coupling
\[ P_\mu \to P_\mu-qA_\mu \qquad \begin{cases}
\omega_p \to \omega_p - q A_0 \\ \bar{p} \to \bar{p} - q \bar{A}
\end{cases} \]
S.Eq. for non relativistic particle:
\[ i \pd{\psi}{t} = \frac{P^2}{2M}\psi \to i\pd{\psi}{t} = \left(\frac{(\bar{p}-q\bar{A})^2}{2M} +qA_0\right)\psi \]
Coupled Klein-Gordon equation:
\[(\partial_\mu\partial^\mu)\varphi = 0 \to (D_\mu D^\mu + M^2)\varphi = 0 \]
\[ (D_\mu D^\mu + M^2)\varphi = \left[ (\partial_0 + iqA_0)^2 - (\bar{\nabla}-iq\bar{A})^2 + M^2\right]\varphi = 0 \]
Note: $(\partial_i + iqA_i)(\partial^i+ iqA^i) \to (\bar{\nabla}-iq\bar{A})(-\bar{\nabla}+iq\bar{A})$ (spatial part of $D^\mu D_\mu$)

\[ (D^\mu D_\mu + M^2)\varphi = [(\partial_\mu + iqA_\mu)(\partial^\mu+ iqA^\mu)+M^2]\varphi \]
\[ [\Box +2iqA^\mu\partial_\mu - q^2A^2 + iq(\partial_\mu A^\mu) + M^2]\varphi = 0 \]
Lorentz gauge: $\partial_\mu A^\mu = 0$.

$\varphi = \varphi_+ + \varphi_-$

\[ \begin{cases}
D^{\mu}D_\mu\varphi_+ \approx [(\omega_k - qA_0)^2-(\bar{k}-q\bar{A})^2]\varphi_+ = M^2\varphi_+ \\
D^{\mu}D_\mu\varphi_- \approx [(\omega_k + qA_0)^2-(\bar{k}+q\bar{A})^2]\varphi_- = M^2\varphi_-
\end{cases} \]

\[\begin{pmatrix}
\varphi_+ \\ q
\end{pmatrix} \leftrightarrow \begin{pmatrix}
\varphi_- \\ -q
\end{pmatrix}\]
Physical meaning:
\[ \begin{cases}
\varphi_+(E>0) \to \text{particle} \\
\varphi_-(E<0) \to \text{anti-particle}
\end{cases} \]

Take non-relativistic limit of coupled Klein-Gordon equation
\[ E = (M^2 + |\bar{p}|^2)^{1/2} = M(1+\frac{|p|}{M^2})^{1/2} \approx M + \frac{\bar{p}^2}{2M} + \ldots \]
Non-relativistic limit: $E \approx M$ or $M \gg |\bar{p}|$, $M>>E_k = \frac{\bar{p}^2}{2M}$.

Do following redefinition
\[\varphi(x,t)= e^{i\omega_p t} = e^{-iMt}\varphi'(x,t) = e^{-iMt}e^{-iE_k t}\]
\[ [(\partial_0 + iqA_0)^2 +M^2]e^{-iMt}\varphi'= e^{-iMt}[(\bar{\nabla}-iq\bar{A})^2]\varphi'\]
\[ e^{-iMt}[ \partial_0 - M2 - 2iM\partial_0 + 2iqA_0\partial_0 + 2iq(\partial_0A^0) + 2qMA_0 - q^2 + M^2]\varphi'= e^{-iMt}[(\bar{\nabla} - iq\bar{A})]\varphi' \]

In the non-relativistic limit:
\[ \left| \frac{\partial_0\varphi'}{\varphi'} \right| \ll M, \qquad  |qA_0| << M, \qquad \left| \frac{\partial_0A_0}{A_0} << M \right| \]

Then $2iM\partial_0$ and $2qMA_0$ dominant (single time derivative)
\[ i\partial_0\varphi' = \left[ - \frac{1}{2M} (\bar{\nabla}-iq\bar{A})^2 + q A_0 \right]\varphi' \]
$\to$ SL ofcoupled EM

\subsubsection{Problems with interpreting the equation}
Two pajor problems with interpretation of probability:
\begin{enumerate}
\item Positive and negative energies:
\[ E^2 = M^2 + |P|^2 \qquad \rightarrow \qquad \omega_p = \pm \left(M^2+|P|^2\right)^{1/2}\]
\item Interpretation of probabilities. Take a quantum barrier. 

\begin{tikzpicture}
\draw (0,0) -- (2,0) -- (2,2) -- (4,2) -- (4,0) -- (6,0);
\draw[snake=snake] (0,1) -- (2,1);
\draw[snake=snake] (4,1) -- (6,1);
\draw (1,1.5) node {R};
\draw (5,1.5) node {T};
\end{tikzpicture}
\[ \text{Continuity equation} \qquad \int \diff{^3x}\rho(x,t) > < 0\]
Klein paradox:
\[ R+T =1 \qquad \begin{cases}
R<0 \\ T> 1
\end{cases} \]
\end{enumerate}


\section{Dirac equation}
\subsection{The search for a linear equation}
The Klein-Gordon equation has some problems. It is also quadratic, not linear like the Schrödinger equation. This prompted Dirac to look for a linear relativistic wave equation.

Starting from the relativistic energy equation, we can try to factorise it like this
\[ p^\mu p_\mu - m^2 =  (a^\mu p_\mu + m)(b^\lambda p_\lambda - m) = 0 \]
with $a^\mu$ and $b^\lambda$ some constants. This is satisfied if
\[(a^\mu p_\mu + m) = a_0 E - \vec{a}\cdot \vec{p} + m = 0.\]
Rewriting in terms of $E$ we get
\[ E = \vec{\alpha}\cdot\vnabla+ \beta m \]
With $\vec{\alpha} = (\alpha_1,\alpha_2,\alpha_3)$ and $\beta$ still just some constants.
Applying the quantisation recipe, we then get
\[ i \pd{}{t}\psi = (-i \vec{\alpha}\cdot\vec{\nabla}+ \beta m)\psi = H\psi \]
We have two requirements for this equation. These impose restrictions on $\vec{\alpha}$ and $\beta$.
\begin{enumerate}
\item $H$ is a hermitian operator;
\item The energy $E$ must still satisfy $E^2 = m^2 + |\vec{p}|^2$. In other words the Dirac equation has to be consistent with the Klein-Gordon equation.
\end{enumerate}
So enforcing those, we get:
\begin{enumerate}
\item $H^\dagger = H$ implies $\alpha^\dagger_i = \alpha_i$ and $\beta^\dagger = \beta$
\item For consistency with the Klein-Gordon equation we assert that $\pd[2]{\psi}{t} = (\vnabla^2 - m^2)\psi$ :
\begin{align*} \pd[2]{\psi}{t} &= -i \pd{}{t}\left(i \pd{\psi}{t}\right) = -i \pd{}{t}\left(-i \vec{\alpha}\cdot \vnabla + \beta m\right)\psi \\
&= - \left(-i \vec{\alpha}\cdot \vnabla + \beta m\right)\left(-i \vec{\alpha}\cdot \vnabla + \beta m\right)\psi \\
&= \sum_{i,j}\left(\alpha_i\alpha_j\partial_i\partial_j + i (\alpha_i\beta + \beta \alpha_i)m\partial_i - \beta^2 m \right)\psi \\
&= (\vnabla^2 - m^2)\psi
\end{align*}
Comparing the last equality term by term, it imposes three conditions:
\begin{enumerate}
\item The term 
\begin{align*}
\sum_{i,j}\alpha_i\alpha_j\partial_i\partial_j &= \frac{1}{2}\left[\sum_{i,j}\alpha_i\alpha_j\partial_i\partial_j + \sum_{i,j}\alpha_i\alpha_j\partial_i\partial_j\right] \\
&= \frac{1}{2}\left[\sum_{i,j}\alpha_i\alpha_j\partial_i\partial_j + \sum_{i,j}\alpha_j\alpha_i\partial_j\partial_i\right] \\
&= \frac{1}{2}\left[\sum_{i,j}\alpha_i\alpha_j\partial_i\partial_j + \sum_{i,j}\alpha_j\alpha_i\partial_i\partial_j\right] \\
&= \sum_{i,j}\frac{1}{2}\left(\alpha_i\alpha_j + \alpha_j\alpha_i\right)\partial_i\partial_j
\end{align*}
 must equal
\begin{align*}
 \vnabla^2 = \sum_i \partial_i^2 = \sum_{i,j} \delta_{ij}\partial_i\partial_j
\end{align*}
So consequently
\[ \frac{1}{2}\left(\alpha_i\alpha_j + \alpha_j\alpha_i\right) = \frac{1}{2}\{\alpha_i, \alpha_j\} = \delta_{ij} \]
\item The term $(\alpha_i \beta + \beta \alpha_i)$ must vanish. In other words
 \[ \{ \alpha_i, \beta \} = 0\]
\item Lastly we have the condition
\[\beta^2 = 1\]
\end{enumerate}
\end{enumerate}

It quite quickly becomes apparent that no real numbers $\alpha_i, \beta$ can satisfy these conditions. While this is a setback, this does not quite doom our project yet. We may try making $\alpha_i$ and $\beta$ matrices. This would make $\psi$ a column-vector. 
We have not yet considered how such objects might transform, so we are hesitant to use the word vector, instead we call them \udef{(Dirac) spinors}. They are elements of an $n$ dimensional \udef{spinorial space}.

Now is a good time to explore what kind of properties the matrices $\alpha_i, \beta$ and the spinorial space have:
\begin{itemize}
\item We have been considering all possible products of two of the matrices $\alpha_i, \beta$. For this to make any sense (i.e. if we want both $\alpha_i\alpha_j$ and $\alpha_j\alpha_i$ to make sense), we need all the matrices to be square with the same dimensions. They are all $n\times n$ matrices.
\item Because we want the Hamiltonian to be Hermitian, we have already required that $\alpha_i, \beta$ be Hermitian.
\item Because $\beta$ is Hermitian, we know it must be unitarily diagonalisable:
\[ \beta^2  = \left(UDU^*\right)^2 = UD^2U^* = \mathbb{1} \]
where $D$ is a diagonal matrix with the eigenvalues of $\beta$ on the diagonal. From this equality we see that $D^2 = \mathbb{1}$ and thus the eigenvalues of $\beta$ must be $1$ an $-1$. The same holds true for the $\alpha_i$s.
\item All of the matrices anti-commute, so for $i \neq j$ 
\[ \Tr(\alpha_i) = \Tr(\alpha_i\alpha_j^2) = \Tr(\alpha_i\alpha_j\alpha_j) = -\Tr(\alpha_j\alpha_i\alpha_j) =  - \Tr(\alpha_i) \]
where the last equality holds because the trace is cyclic. Consequently 
\[ \Tr(\alpha_i) = 0 = \Tr(\beta) \]
\item When diagonalising, the trace of the diagonal matrix is the same as that of the original one, in this case zero. Because there can only be $1$ and $-1$ on the diagonal, there must be an equal number of $1$s and $-1$s, so the dimension $n$ must be even.
\end{itemize}
Now the question is, can we actually find matrices that satisfy these conditions? Searching in the lowest even dimension $n = 2$, we realise that we can always only find three linearly independent traceless Hermitian matrices (for example the Pauli matrices $\sigma_i$). We need four linearly independent matrices to satisfy the anti-commutator relations, so we will need to try a higher dimension.

\begin{note}
On a side-note, if we assume the mass $m$ is zero, we no longer need the $\beta$ matrix and then there is a two dimensional spinorial representation. This is the Weyl equation.
\end{note}

For $n = 4$ we have no problem finding matrices that fulfill all the requirements:
\[ \alpha_i = \begin{pmatrix}
0 & \sigma_i \\ \sigma_i & 0
\end{pmatrix}, \qquad \beta = \begin{pmatrix}
\mathbb{1} & 0 \\ 0 & -\mathbb{1}
\end{pmatrix} \]
So the lowest possible spinorial dimension for the Dirac spinors is $n=4$.

\subsubsection{Dirac $\gamma$-matrices in the Dirac / Pauli representation}
TODO why called representation?

Based on our choice of $\alpha_i$ and $\beta$, we define the $\gamma$-matrices.
\[ \gamma^0 \equiv \beta = \begin{pmatrix}
\mathbb{1} & 0 \\ 0 & -\mathbb{1}
\end{pmatrix}, \qquad \gamma^i \equiv \beta\alpha_i = \begin{pmatrix}
0& \sigma_i \\ -\sigma_i & 0
\end{pmatrix} \]
We can put these $\gamma$-matrices together in a four-vector-like notation
\[ \gamma^\mu \equiv (\gamma^0, \gamma^i). \]
But it is important to remember that these matrices were introduced as constants, are still constants and thus \textbf{do not transform under Poincaré transformations} like four-vectors do (TODO ??????).

The properties of the matrices $\alpha_i, \beta$ translate to the following properties of the $\gamma$-matrices:
\begin{enumerate}
\item $\gamma^0 = (\gamma^0)^\dagger, \qquad \gamma^i = - (\gamma^i)^\dagger$
\item $(\gamma^0)^2 = \mathbb{1}, \qquad (\gamma^i)^2 = -\mathbb{1}$
\item $\{ \gamma^\mu, \gamma^\nu \} = 2\eta^{\mu\nu}$
\end{enumerate}
In fact any matrices that satisfy these properties would work. Almost all results will be derived from these properties, so that they remain valid any representation.

In order to rewrite the Dirac equation in terms of $\gamma$-matrices, we multiply by $\beta$:
\[ i\beta\partial_0 \psi = \left(-i\beta\alpha_i\partial_i + \beta^2 m\right)\psi = \left(-i\gamma^i\partial_i + m\right)\psi \]
Rearranging we get
\[ \left( i\gamma^0 + i\gamma^i\partial_i -m \right)\psi \left(i\gamma^\mu\partial_\mu - m\right)\psi = 0 \]
This can be written even more succinctly with the slashed notation
\[ \slashed{a} \equiv \gamma^\mu a_\mu = \gamma^0a_0 + \gamma^i a_i \]
 for any four-vector $a^\mu$.
Using this notation, we get the most famous form of the Dirac equation:
\[ \boxed{(i\slashed{\partial} - m ) \psi = 0}\]
The Dirac equation in fact gives four equations with components
\[ \sum_\beta\left(i\gamma^\mu_{\alpha\beta}\partial_\mu - M\delta_{\alpha\beta}\right)\psi_\beta = 0 \]
where $\alpha$ and $\beta$ are spinorial indices ranging from one to four and $\mu$ is a Lorentz index.

TODO: calculus of spinor indices

\subsubsection{Other representations of $\gamma$-matrices}
The choice of the $\gamma$ representation is not unique. With a non-singular unitary matrix $C$, one can always obtain a new representation
\[ \tilde{\gamma}^\mu = C^{-1}\gamma^\mu C \]

\subsection{Covariance of Dirac equation}
Having posited the Dirac equation, we now want to verify the covariance of the equation.  

TODO theoretical grounds for transformation of spinors (coordinate based construction / basis??? Essential essence of spinors)

Considering an arbitrary Lorentz transformation $\Lambda \in \SO(1,3)$:
\[ x^{\mu} \qquad\to\qquad x'^\mu = \tensor{\Lambda}{^\mu_\nu} x^\nu \]
which, by construction, gives
\[ \partial_\mu \qquad\to\qquad \partial'_\mu = \tensor{\Lambda}{_\mu^\nu}\partial_\nu. \]
Clearly $\psi(x)$ is impacted by the coordinate transformation and becomes $\psi'(x')$. We may not just assume that each component of the spinor is a scalar. So in general $\psi(x) \neq \psi'(x')$, but we may write
\[ \psi(x) \qquad\to\qquad \psi'(x') = S(\Lambda)\psi(x). \]
Some properties of $S(\Lambda)$ are immediately apparent (TODO):
\begin{enumerate}
\item Action
\item Linear map, so representation.
\end{enumerate}
So $S(\Lambda)$ is an element of the spinorial representation of Lorentz group. The question is now which one.

We want the Dirac equation to be covariant, meaning it still holds under a Lorentz transformation
\[(i\gamma^\mu\partial_\mu - M)\psi = 0 \qquad \overset{\Lambda}{\rightarrow}\qquad (i\gamma^\mu\partial'_\mu - M)\psi'(x') = 0\]
where we have used that $\gamma^\mu$ does not transform in $\Lambda$.

Filling in the transformed quantities, we get
\begin{align*}
0 &= (i\gamma^\mu\tensor{\Lambda}{_\mu^\nu}\partial_\nu - M)S(\Lambda)\psi(x) \\
&= S(\Lambda)[i\tensor{\Lambda}{_\mu^\nu} S(\Lambda)^{-1}\gamma^\mu S(\Lambda) \partial_\mu - M]\psi \\
\end{align*}
this holds if
\[ \tensor{\Lambda}{_\mu^\nu} S(\Lambda)^{-1}\gamma^\mu S(\Lambda) = \gamma^\mu \]
or, equivalently
\[ S(\Lambda)^{-1}\gamma^\mu S(\Lambda) = \tensor{\Lambda}{^\mu_\nu}\gamma^\mu. \]

\subsubsection{Spinorial ($N=4$) representation of the Lorentz algebra}
\[ \begin{cases}
\tensor{\Lambda}{^\mu_\nu} = \delta^\mu_\nu + \tensor{\omega}{^\mu_\nu} \\
S(\Lambda) = \mathbb{1}- \frac{i}{2}\omega_{\mu\nu}\Sigma^{\mu\nu}
\end{cases} \]
($\Sigma^{\mu\nu}$ are generators of representation)

Using the ``covariance'' relation
\[ \left(\mathbb{1}+\frac{i}{2}\omega_{\mu\nu}\Sigma^{\mu\nu}\right)\gamma^\mu\left(\mathbb{1}-\frac{i}{2}\omega_{\rho\sigma}\Sigma^{\rho\sigma}\right) = \gamma^\mu + \tensor{\omega}{^\mu_\nu}\gamma^\nu \]
\[ \gamma^\mu - \frac{i}{2}\omega_{\rho\sigma}[\gamma^\mu,\Sigma^{\rho\sigma}] + \mathcal{O}(\omega^2) = \gamma^\mu + \omega_{\rho\sigma}(\eta^{\mu\rho}\gamma^\sigma) \]
\[ [\gamma^\mu, \Sigma^{\rho\sigma}] = i )(\eta^{\mu\rho}\gamma^\sigma - \eta^{\mu\sigma}\gamma^\rho) \]
So $\Sigma^{\rho\sigma}$ are the following matrices
\[ \Sigma^{\mu\nu} \equiv \frac{i}{4}[\gamma^\mu, \gamma^\nu] \equiv \frac{1}{2}\sigma^{\mu\nu} \]

\begin{example}
Exercise: verify
\[ [\gamma^\mu, \gamma^\rho\gamma^\sigma] = \{\gamma^\mu, \gamma^\rho\}\gamma^\sigma - \gamma^\rho\{\gamma^\mu, \gamma^\sigma\} \]
\end{example}

\begin{example}
Exercise: Verify $\Sigma^{\rho\sigma}$ satisfy the commutator algebra of $\SO(1,3)$.
\end{example}

\begin{eigenschap}
$\Sigma^{\mu\nu}$ is the ($D=4$) \ueig{spinorial representation} of $\SO(1,3)$
\end{eigenschap}

Lecture 16/10

ix) Definition of $\gamma^5$
\[ \gamma^5 \equiv - \frac{i}{4}\epsilon_{\mu\nu\rho\sigma}\gamma^\mu\gamma^\nu\gamma^\rho\gamma^\sigma = i\gamma^0\gamma^1\gamma^2\gamma^3 \]
Properties:
\begin{itemize}
\item $\left(\gamma^5\right)^2 = \mathbb{1}_4$
\item $\gamma^5 = \left(\gamma^5\right)^\dagger$
\item $\{\gamma^\mu,\gamma^5\} = 0$
\end{itemize}
Note: THe $\gamma^5$ matrix satisfies
\[ [\gamma^5, \Sigma^{\mu\nu}] = 0 \]
\remark{The $D=4$ representation of $\SO(1.3)$ is \ueig{reducible}}

\begin{definition}
\udef{Chirality projectors} :
\[ P_L \equiv \left(\frac{1-\gamma_T}{2}\right), \qquad P_R \equiv \left(\frac{1+\gamma_T}{2}\right)\]
\udef{Chiral components} of $\psi$:
\[ \psi_L = P_L\psi, \qquad \psi_R = P_R \psi \]
\end{definition}
(What are projectors ????)
Prove that:
\[ \gamma^5\psi_L = -\psi_L, \qquad \gamma^5\psi_R = +\psi_R\]
$\psi_L$, $\psi_R$ are (the) irreducible????
\[ \psi \sim \psi_L\oplus\psi_R \sim (\tfrac{1}{2}, 0)\oplus(0, \tfrac{1}{2}) \]
$\gamma^5$ matrix in Dirac / Pauli representation
\[ \gamma^5 = \begin{pmatrix}
0 & \mathbb{1} \\ \mathbb{1} & 0
\end{pmatrix} \]
Spin and helicity operators in $(D = 4)$:
\[ K_i \equiv \Sigma^{i0} = - \frac{i}{2}\begin{pmatrix}
0 & \sigma_i \\ \sigma_i & 0
\end{pmatrix} \]
\[ \Sigma^{i} \equiv \frac{1}{2}\epsilon_{ijk}\Sigma^{jk} = \frac{1}{2}\begin{pmatrix}
\sigma_i & 0 \\ 0 & \sigma_i
\end{pmatrix} \]
Spin is not a good quantum number, because it does not commute with Hamiltonian.
\[ [H_D, \Sigma_3] = i(\alpha_1 P_2 - \alpha_2 P_1) = i\epsilon_{ij3}\alpha_iP_j \neq 0 \]

Helicity is projection of spin operator in direction of momentum.

\begin{definition}
The \udef{helicity}  operator
\[ \sigma_P \equiv \frac{\bar{\Sigma}\cdot \bar{P}}{|\bar{P}|} \]
\end{definition}
The helicity is a good quantum number:
\[ [H_D, \sigma_P] = i \epsilon_{ijk}\alpha_iP_jP_k = 0 \]

\subsection{Pauli-Lubansky vector and helicity}
\[ \omega^\mu \equiv \frac{1}{2}\epsilon^{\mu\nu\rho\sigma}J_{\nu\rho}P_\sigma = \frac{1}{2}\epsilon^{\mu\nu\rho\sigma}\Sigma_{\nu\rho}P_\sigma \]
Where $J_{\nu\rho} = L_{\nu\rho} + \Sigma_{\nu\rho}$ and $L_{\nu\rho} = X_\nu P_\rho - X_\rho P_\nu$

Now let's calculate $\omega^\mu$ for $D=4$ spinorial representation in the rest frame $P^\mu = (M,0)$
\[ \omega^\mu \overset{R.F.}{=} \frac{M}{2}\epsilon^{\mu\nu\rho\sigma} \Sigma_{\nu\rho} \rightarrow \begin{cases}
\omega^0 = 0 \\ \omega^i = M\Sigma^i
\end{cases} \]

\begin{align*}
\frac{\omega^\mu \omega_\mu}{M^2}  \overset{R.F.}{=} &= \Sigma^i\Sigma_i = - |\bar{\Sigma}|^2 \\
&= - \left(\Sigma_1^2 + \Sigma_2^2 + \Sigma_3^2\right)^2 = - \frac{3}{4}\mathbb{1}_4 \\
&= -s(s+1)\mathbb{4} \qquad \text{with} s = \tfrac{1}{2}
\end{align*}

So $\omega^2$ is the Casimir of the Poincaré group.
Now the connection between $\omega^\mu$ and $\sigma_P$
\[ n^\mu_P \equiv \left(\frac{|\bar{P}|}{M}, \frac{\omega_P}{M}\frac{\bar{P}}{|\bar{P}|}\right) \]
\[ \frac{\omega^\mu n^P_\mu}{M} = - \frac{1}{2}\sigma_P = - \frac{1}{2} \begin{pmatrix}
\frac{\bar{\sigma}\cdot \bar{P}}{|\bar{P}|} & 0 \\ 0 & \frac{\bar{\sigma}\cdot \bar{P}}{|\bar{P}|}
\end{pmatrix}  = - \frac{1}{2}\frac{\bar{\Sigma}\cdot \bar{P}}{|\bar{P}|}\]


\subsection{Interpretation of the Dirac equation}
\subsubsection{Dirac conjugate spinor}

\[ 0 = [(i\slashed{\partial}-M)\psi]^\dagger = - \psi^\dagger(i(\gamma^\mu)^\dagger \overset{\leftarrow}{\partial}_\mu + M) \]
Using $\gamma^0(\gamma^\mu)^\dagger\gamma^0 = \gamma^\mu$
\[ = - \psi^\dagger(i\gamma^0\gamma^\mu\gamma^0 \overset{\leftarrow}{\partial}_\mu + M) = - (\psi^\dagger \gamma^0)\left(i\gamma^\mu\overset{\leftarrow}{\partial}_\mu + M\right)\gamma^0 \]
\[ \bar{\psi}\left(i\overset{\leftarrow}{\slashed{\partial}} + M\right) = 0 \]
With $\bar{\psi} \equiv \psi^\dagger\gamma^0$ the \udef{Dirac conjugate}.

So
\[ \psi'(x') = S(\Lambda)\psi(x) \]
\[ \bar{\psi}'(x') = (\psi')^\dagger(x')\gamma^0 = \psi^\dagger(x)\left(\gamma^0\right)^2S^\dagger(\Lambda)\gamma^0 = \bar{\psi}\gamma^0S^\dagger(\Lambda)\gamma^0 = \bar{\psi}(x)S(\Lambda)^{-1} \]



\subsubsection{Continuity equation}
\[ \bar{\psi}(i\overset{\rightarrow}{\slashed{\partial}} - M)\psi = 0 \]
\[ \bar{\psi}(i\overset{\rightarrow}{\slashed{\partial}} + M)\psi = 0 \]
\begin{align*}
0 &= \bar{\psi}\left(i\overset{\rightarrow}{\slashed{\partial}} - M + i\overset{\leftarrow}{\slashed{\partial}} + M\right)\psi \\
&= \bar{\psi}\left(i\gamma^\mu\overset{\rightarrow}{\partial}_\mu + i \gamma^\mu\overset{\leftarrow}{\partial}_\mu\right)\psi \\
&= i \left[\bar{\psi}\gamma^\mu(\partial_\mu\psi) + (\partial_\mu\bar{\psi})\gamma^\mu\psi\right] \\
&= i \partial_\mu [ \bar{\psi}\gamma^\mu \psi] = 0
\end{align*}

Continuity equation for Dirac:
\[ \partial_\mu J^\mu = 0 \]
With 
\begin{align*} J^\mu &\equiv \bar{\psi}\gamma^\mu \psi \\
&= (\bar{\psi\gamma^0\psi}, \bar{\psi\gamma^i\psi}) \\
&= (\psi^\dagger\psi, \psi^\dagger\partial_i\psi) \\
&= (\rho, \bar{j})
\end{align*}
So we get
\[ \partial_0\rho + \partial_i j^i = 0 \]
Now define
\[ \Theta = \int \diff{^3x}\rho(x,t) \qquad \rightarrow \qquad \od{\Theta}{t} = 0 \]
So $\Theta$ is conserved.
\[ Q = \int \diff{^3x}\psi^\dagger\psi \qquad > 0 \]

\subsubsection{Problems with the interpretation}
PROBLEM: $Q > 0, H_D > < 0$
Klein paradox (see above)
So we still have all the problems of KG.

Physical objects  $\bar{\psi}\Gamma\psi$ are bilinear.

How do bilinear transform under Lorentz?
\[ \bar{\psi}\mathbb{1}\psi \overset{\Lambda}{\rightarrow} \bar{\psi}'\psi' = \bar{\psi}S(\Lambda)^{-1}S(\Lambda)\psi \]
So scalar.
\begin{align*}
J^\mu = \bar{\psi}\gamma^\mu \psi \overset{\Lambda}{\rightarrow} \bar{\psi}'\gamma^\mu\psi' &= \bar{\psi}S(\Lambda)^{-1}\gamma^\mu S(\Lambda)\psi \\
&= \Lambda^\mu_\nu \bar{\psi}\gamma^\nu\psi \\
&= \Lambda^\mu_\nu J^\nu
\end{align*}

\subsection{General solution of the free Dirac equation}
\begin{align*}
&(i\slashed{\partial}-M)\psi(x) = 0 \\
&(\Box + M^2)\psi(x) = 0 \qquad (k^2 = M^2, \quad k_0 = \omega_k)
\end{align*}
A general solution of the Klein-Gordon equation
\[ \psi = \psi_+ + \psi_- = \left.e^{-ikx}u(k) + e^{ikx}v(k)\right|_{k_0 = \omega_k} \]
Now
\begin{align*}
& (i\slashed{\partial} - M)\psi_+ \approx e^{-ikx}(\slashed{k}-M)u(k) = 0 \\
& (i\slashed{\partial} - M)\psi_- \approx e^{ikx}(\slashed{k}+M)v(k) = 0 
\end{align*}

\[ \begin{cases}
(\slashed{k}-M)u(k) = 0 \\
(\slashed{k}+M)v(k) = 0
\end{cases} \qquad \text{Dirac eq in momentum space}\]

Find solution in the rest frame: $k^\mu \overset{R.F.}{=} (M,0)$
(Mass different from 0)
\[ \begin{cases}
(\slashed{k}-M)u(k) \overset{R.F.}{\rightarrow} M(\gamma^0 - \mathbb{1}_4)u(M) = 0 \\
(\slashed{k}+M)v(k) \overset{R.F.}{\rightarrow} M(\gamma^0 + \mathbb{1}_4)v(M) = 0
\end{cases} \]
In Dirac / Pauli representation
\[ \gamma^0 = \begin{pmatrix}
\mathbb{1} & 0 \\ 0 & -\mathbb{1}
\end{pmatrix} \]
\[ \begin{cases}
M \left[\begin{pmatrix}
\mathbb{1} & 0 \\ 0 & -\mathbb{1}
\end{pmatrix} - \begin{pmatrix}
\mathbb{1} & 0 \\ 0 & \mathbb{1}
\end{pmatrix}\right]u(M) = M \begin{pmatrix}
0 & 0 \\ 0 & -2
\end{pmatrix}u(M) = 0 \\
M \left[\begin{pmatrix}
\mathbb{1} & 0 \\ 0 & -\mathbb{1}
\end{pmatrix} + \begin{pmatrix}
\mathbb{1} & 0 \\ 0 & \mathbb{1}
\end{pmatrix}\right]v(M) = M \begin{pmatrix}
2 & 0 \\ 0 & 0
\end{pmatrix}v(M) = 0
\end{cases} \]
So we get
\begin{align*}
u_i(M) &\equiv \sqrt{2M} \begin{pmatrix}
\xi_i \\ 0
\end{pmatrix} \\
v_i(M) &\equiv \sqrt{2M} \begin{pmatrix}
0 \\ \xi_i
\end{pmatrix}
\end{align*}
With $\xi_1 = \begin{pmatrix}
1 \\ 0
\end{pmatrix}$ and $\xi_2 = \begin{pmatrix}
0 \\ 1
\end{pmatrix}$.

Usually not useful to expand fully into 4 components.

Solution in a general frame.

Notice that
\[ (\slashed{k}-M)(\slashed{k}+M) = \left(\slashed{k}^2 + M\slashed{k} - M\slashed{k} + M^2\right) = (k^2 -M^2) = 0 \]
We can therefore write
\[ \begin{cases}
u(k) = c(\slashed{k} + m)u(m) \\
u(k) = c(\slashed{k} - m)v(m)
\end{cases} \]
Using the boost operation explicitly, we can see that actually this is proportional to $k_iu_r$
Normalisation
\[ \begin{cases}
\bar{u}_r (k)  u_s(k) = 2M\delta_{rs} \\
\bar{v}_r (k)  v_s(k) = -2M\delta_{rs} 
\end{cases} \]

General solution in momentum space:

\begin{align*}
u_r(k) &= \frac{(\slashed{k}+M)}{\sqrt{2M(M_{\omega_k}+\omega_k)}}u_r(M) = \sqrt{M+\omega_k}\begin{pmatrix}
\xi_r \\ \frac{\vec{\sigma}\cdot \vec{k}}{M+\omega_k}\xi_r
\end{pmatrix} \\
v_r(k) &= \frac{(-\slashed{k}+M)}{\sqrt{2M(M_{\omega_k}+\omega_k)}}v_r(M) = \sqrt{M+\omega_k}\begin{pmatrix}
\frac{\vec{\sigma}\cdot \vec{k}}{M+\omega_k}\xi_r \\ \xi_r
\end{pmatrix}
\end{align*}
General solution in coordinate space:

\begin{align*} \psi(x,t) &= \frac{1}{(2\pi)^{3/2}}\int \frac{\diff{^3 k}}{\sqrt{2\omega_k\varphi_R}}\sum^{2}_{r=1} \left(c_r(k)u_r(k)e^{-ikx} + d_r^*(k)v_r(k)e^{ikx}\right)_{k_0=\omega_k} \\
&= \bar{\psi}(x,t) \frac{1}{(2\pi)^{3/2}}\int \frac{\diff{^3 k}}{\sqrt{2\omega_k\varphi_R}}\sum^{2}_{r=1} \left(d_r(k)\bar{v}_r(k)e^{-ikx} + c_r^*(k)\bar{u}_r(k)e^{ikx}\right)_{k_0=\omega_k}
\end{align*}

Be careful manipulating the blockmatrices:

Lecture 17/10/18 

\begin{example}
Dirac conjugate:
\[\bar{u}_r(k) = \bar{u}_r(M) \frac{\slashed{k}+M}{\sqrt{2M(M+\omega_k)}} = \left((M+\omega_k)^{1/2}\xi^\intercal_1, \ldots\right) \]
Missed rest.??

\end{example}

Watch out: $\bar{\psi}\psi\gamma^\mu$ has no meaning.


\[\Pi_\pm \equiv \frac{\mathbb{1}\pm\sigma_p}{2} = \frac{\mathbb{1}\mp\omega^\mu n_\mu^1}{2}\]
With $W_0 = 0$, $W^i=\Sigma^i$,$\bar{p}=(0,0,\bar{p})$.
\[ \begin{cases}
\pi_+ u_1(M) = +u_1(M) \\
\pi_- u_1(M) = 0 \\
\pi_+ u_2(M) = 0 \\
\pi_- u_2(M) = +u_2(M) \\
\end{cases} \]
\[ \begin{cases}
\pi_+ v_1(M) = 0 \\
\pi_- v_1(M) = -v_1(M) \\
\pi_+ v_2(M) = +v_1(M) \\
\pi_- v_2(M) = 0 \\
\end{cases} \]

Projectors over $\pm$ energy
\[ \Lambda_\pm(k) = \pm \frac{\slashed{k}+M}{2M} \qquad \begin{pmatrix}
\Lambda_+v(k) = 0 \\ \Lambda_-u(k) = 0
\end{pmatrix} \]
$\Lambda_\pm$ are projectors (verify def!)

I can rewrite the projectors as
\[ \Lambda_+(k) = \sum^2_{r=1}\frac{u_r(k)u_1(k)}{2M} \]
\[ \Lambda_-(k) = -\sum^2_{r=1}\frac{v_r(k)v_1(k)}{2M} \]
(?)

\subsection{Dirac equation coupled with an electromagnetic field}
\remark{minimal coupling prescription}
\[\partial_\mu \rightarrow D_\mu = \partial_\mu + iqA_\mu\]
So
\[ (i\slashed{\partial}-M)\psi = 0 \rightarrow (i\slashed{D}-M)\psi = 0 \]
\[\to \psi_+ \text{and} \psi_- \qquad \begin{cases}
\psi_+ \qquad \text{particle of } (+q) \\
\psi_- \qquad \text{antiparticle of } (-q) 
\end{cases}\]
Now we study the nonrelativistic limit of the Dirac equation
\[ (i\partial_0 - qA_0)\psi = \left[-i \bar{\alpha}(\bar{\nabla}-iq \bar{A})+\beta M\right]\psi \]

\[ \left[\psi(x,t) = e^{-iMt}\psi'(x,t)\right] \]
\[ e^{-iMt}\left(-\partial_0 + M - qA_0\right)\psi' = e^{-iMt}\left(-\bar{\alpha}(\bar{\nabla}-iq \bar{A}) + \beta M\right)\psi' \]
We write the 4 dim $\psi'$ as
\[ \psi' = \begin{pmatrix}
\varphi' \\ \chi'
\end{pmatrix} \]
So we get two 2 dim equations
\[ \begin{cases}
i\partial \varphi' = qA_0\varphi' - i \bar{\sigma}\cdot \left(\bar{\nabla}-iq \bar{A}\right)\chi' \\
i\partial \chi' = (qA_0-2M)\chi' - i \bar{\sigma}\cdot \left(\bar{\nabla}-iq \bar{A}\right)\varphi'
\end{cases} \]

In nonrelativistic limit one can assume
\[ \left|\frac{\partial_0\chi'}{\chi'}\right|\ll M, \qquad |qA_0| \ll M \]
The second equation becomes
\[ \chi' = - \frac{i}{2M}\bar{\sigma}\cdot \left(\bar{\nabla}-iq \bar{A}\right)\varphi' \ll \varphi' \qquad \text{CONSTRAINT}\]
Then we get from the first equation
\[ i\partial_0\varphi' = qA_0 \varphi' - \frac{1}{2M}\left(\bar{\sigma}\cdot \bar{\nabla}-iq\bar{\sigma}\cdot \bar{A}\right)^2\varphi' \]
Which is the Pauli equation. We now work towards the Schrödinger equation.
\begin{align*} \left(\bar{\sigma}\cdot \bar{\nabla}-iq\bar{\sigma}\cdot \bar{A}\right)^2\varphi' &= \sigma_i\sigma_j (\partial_i + iqA_i)(\partial_j + iqA_j)\varphi' \\
&= \left(\frac{1}{2}\{\sigma_i,\sigma_j\}+ \frac{1}{2}[\sigma_i,\sigma_j]\right)\left(\partial_i + iqA_i\right)\left(\partial_j + iqA_j\right)\varphi' \\
&=  \left(\frac{1}{2}\{\sigma_i,\sigma_j\}+ \frac{1}{2}[\sigma_i,\sigma_j]\right)\left(\partial_i\partial_j - q^2A_iA_j + iqA_j\partial_j + iqA_j\partial_i+ iq(\partial_iA_j)\right)\varphi' \\
&= \left(\nabla^2 - q^2 \bar{A}^2 - 2iq \bar{A}\cdot \bar{\nabla} - iq(\bar{\nabla}\cdot\bar{A})+ q \bar{\sigma}\cdot\bar{B}\right)\varphi' \\
&= \left[\left(\bar{\nabla}-iq \bar{A}\right)^2 + q\bar{\sigma}\cdot\bar{B}\right]\varphi'
\end{align*}

\[ iq\epsilon_{ijk}(\partial_iA_j)\sigma_k = -q(\bar{\nabla}\times \bar{A})_k \sigma_k = + q \bar{B}\cdot \bar{\sigma} \]

\[ i \pd{\varphi'}{t} = \left\{- \frac{1}{2M}\left(\bar{\nabla}-iq \bar{A}\right)^2 + q A_0 - \frac{q}{2M}\bar{\sigma}\cdot\bar{B}\right\}\varphi' \]
\[ i \pd{\varphi'}{t} = \left\{\frac{1}{2M}\left(\bar{P}-q \bar{A}\right)^2 + q A_0 - \frac{q}{2M}\bar{\sigma}\cdot\bar{B}\right\}\varphi' \]
Dipole term $\bar{\mu}_s = - \frac{q}{2M}\bar{\sigma}$
\[ H_{\text{DIP}} = -\bar{\mu}_s\cdot \bar{B} = - \frac{q}{2M}\bar{\sigma}\cdot\bar{B} \]
For an orbital momentum
\[ \bar{\mu}_L = \frac{q}{2M}\bar{L} \qquad \to \qquad \left|\frac{\bar{\mu}_L}{\bar{L}}\right|= \frac{q}{2M} \]

\[ \bar{\mu}_S = \frac{q}{2M}g_c \left(\frac{\bar{\sigma}}{2}\right) \qquad \to \qquad \left|\frac{\bar{\mu}_L}{\bar{\Sigma}_{(2)}}\right| = \frac{q}{2M}g_c \]
With $\bar{\Sigma}_{(2)} = \frac{\bar{\sigma}}{2}$.

With $g_c$ is the gyromagnetic faction for electrons (spin 1/2).

So Dirac equation predicts $g_c = 2$. (Dirac 1938)
\[ g_c^{\text{exp}} = 1.99 \pm 0.2 \]




\chapter{Measurement and interpretation}