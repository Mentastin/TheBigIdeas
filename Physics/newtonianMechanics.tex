\chapter{Setting the stage}
\section{What is mechanics?}
\begin{itemize}
\item Describe motion of objects.
\item Traditionally kinematics, dynamics and statics (Under what condition is there no motion? Important when we do not want motion such as in constructions.).
\end{itemize}
Later different formalisms:
energy, action. Derived from Newton.
\section{Some experimental facts}
\subsection{Space and time}
\begin{itemize}
\item continuous trajectories -> topology
\item straight lines -> affine structure
\item distance and time intervals -> metric structure
\end{itemize}
Absolute? Bucket experiment. (>< Mach).
\[ \mathbb{E}^3\times\mathbb{R} \]
\subsection{Galileo's principle of relativity}
Ability to switch coordinate systems.
If not absolute, then equivalence classes of coordinate systems.
 
We cannot give an example of an inertial frame, only approximants.
 
Alternatively: define inertial frames as frames where first law is true. (The law asserts their existence.)
\subsection{Determinacy}
Laplace's demon. Given position and velocity we can determine further evolution of the system.

\subsection{Divisibility}

\section{Using point particles}

\section{Newton's laws}
Posits absolute space and time.
Let $\vec{r}(t)$ be a position vector.
\[ \vec{a} = \od{\vec{v}}{t} = \od[2]{\vec{r}}{t} \]
\begin{enumerate}
\item Every body perseveres in its state either of rest or of uniform motion in a straight line, except insofar as it is compelled to change its state by impressed forces.
\item The change in motion is proportional to the impressed motive force and is made along the straight line on which the force is impressed.
\[ \vec{F} = m \vec{a}. \]
\item Action / reaction: less fundamental.
\end{enumerate}

In classical mechanics the motion of massive particles is determined by \udef{Newton's laws}. They are as follows:
\begin{eigenschap}
\begin{enumerate}
\item When all external influences on a particles are removed, the particle moves with constant velocity.
\item When a \udef{force} $\vec{F}$ acts on a particle of mass $m$, the particle moves with instantaneous acceleration $\vec{a}$ given by the formula
\[\vec{F}=m \vec{a}.\]
\item When two particles exert forces upon each other, these forces are
\begin{enumerate}
\item equal in magnitude,
\item opposite in direction, and
\item parallel to the straight line joining the two particles.
\end{enumerate}
\end{enumerate}
\end{eigenschap}

TODO: inertia, mass and force

Inertial frame: first law holds

law of multiple interactions

principle of superposition!

\chapter{Setting up the mathematics}
Remarks on intervals and integration. Over time. (What is $\diff{l(t)}$?)

\chapter{Kinematics}
TODO apples, cars and pendula


Before trying to predict motion, we first need to introduce the concepts necessary to \textit{describe} motion. This is the domain of \udef{kinematics}. To describe the motion, i.e. the trajectory, of an object (say an apple, for example) through space, it would seem quite natural to use a three dimensional curve. There is however a slight problem with this approach: a curve is infinitesimally thin, whereas an apple has a volume. This means if you track different parts of the apple, you will get different trajectories. This is especially obvious if the apple is spinning, see picture TODO ref. Intuitively we would like to call curve TODO the trajectory of the apple, but how do we know which part of the apple we need to track to get such a curve? In essence we want to track to point that the apple is spinning around, that way we have a trajectory of the apple (its \textit{translation}) without any information about its \textit{rotation}, which we can then consider separately. We will show how to find the point the apple is spinning around later, but for now in order to build up the theory, we will solve the problem in a different way: We model the motion of the apple with a particle with the same mass that is infinitely small. We define a \udef{particle} as an idealised body that occupies only a single point in space. We call a particle with a mass a \udef{point mass}. We now do not need to find the point the apple is rotating around, because the point mass is infinitely small and thus only has one point to rotate around.

TODO picture of cycloidal apple.
TODO: change of frame:
\[ \delta l = \delta u + \vec{\delta} \times (\vec{r}_A - \vec{r}_B) \]
TODO: change of frame to express constraints!!

\section{Speed, velocity and acceleration}
Many of the fundamental concepts of kinematics have been touched upon in the section on spatial curves. We have seen that we can take the derivative of a curve (at least if the curve is differentiable, but we can safely require that trajectories be modelled with differentiable curves). This derivative is, as we have seen, a vector that points in the direction tangential to the motion and its magnitude gives us an indication of how far the particle is travelling per unit of time. We call the derivative the \udef{velocity}. Sometimes we are only interested in how far the particle is travelling per unit of time, not which direction it is going in. In that case we would only be interested in the magnitude of the derivative, which we call the \udef{speed}.

Now we could also be interested in how the velocity changes over time. Velocity is a time-dependent vector; taking the derivative of it we get a new time-dependent vector, which we call the \udef{acceleration}. This is the second derivative of the position vector.

We can also take higher order derivatives of the position vector. The third derivative is known as the jolt. The fourth, fifth and sixth as jounce (or snap), crackle and pop. These higher order derivatives rarely appear in any practical applications.

In full generality there is not much more to say on the topic of particle kinematics. If we can show that a trajectory is of a certain type (e.g. linear or circular), this also imposes limits on the form of the derivatives. The mathematical expressions are also often a lot easier to deal with. We give some important examples of such systems below.

\section{Describing the motion of a particle}
In classical mechanics we assume that space is Euclidean and three dimensional. Thus we can choose an origin $O$ and three orthogonal vectors, $\hat{i}, \hat{j}, \hat{k}$. Together these form a reference frame.

Locations can be specified with a position vector $\vec{r}$, relative to the reference frame. The motion of a particle can be specified as a curve in the Euclidean space, parametrised by time $t$. So we view $\vec{r}: \R \to \E: t\mapsto \vec{r}(t)$ as giving the location of a particle in function of time.

We then define the \udef{velocity} $\vec{v}$ and \udef{acceleration} $\vec{a}$ of a particle whose motion is described by $\vec{r}(t)$ as
\[ \vec{v} = \od{\vec{r}}{t} \qquad \text{and} \qquad \vec{a} = \od{\vec{v}}{t} = \od[2]{\vec{r}}{t} \]

\subsection{Interpreting velocity and acceleration}
Assume $s(t)$ is the arc length of a trajectory $\vec{r}(t)$, so the arc length parametrisation is given by $\vec{r}(s)$. Then by the chain rule,
\begin{align*}
\vec{v} = \od{\vec{r}}{t} = \od{\vec{r}}{s}\od{s}{t} = \vec{T}v
\end{align*}
where $\vec{T}$ is the unit vector tangent to the trajectory and $v = \od{s}{t}$ is the speed.

We can write the acceleration as follows:
\begin{align*}
\vec{a} &= \od{\vec{v}}{t} = \od{v\vec{T}}{t} = \od{v}{t}\vec{T} + v \od{\vec{T}}{t} \\
&= \left(\od{v}{t}\right)\vec{T} + \left(v^2 \kappa\right)\vec{N} \\
&= \left(\od{v}{t}\right)\vec{T} + \left(\frac{v^2}{R}\right)\vec{N}
\end{align*} 
as per the Frenet-Serret formulae in three dimensions; $R$ is the radius of the osculating circle. We have split the acceleration in a component tangent to the trajectory that shows the instantaneous change in speed and a component perpendicular to the local path direction that shows how much the trajectory is curving and points in the direction the trajectory is curving towards.

\subsection{Polar coordinates}
There are other choices of basis of our reference frame we can make. If the trajectory $\vec{r}(t)$ is in a plane, then we can choose two basis vectors in the plane $\hat{r}, \hat{\theta}$ such that at anytime $\hat{r}$ is the unit vector in the direction $\vec{r}(t)$. Obviously these basisvectors are not constant, in particular they depend on the location $\vec{r}$. It is useful to express this location in polar coordinates $(r,\theta)$. Now we can write
\[ \begin{cases}
\od{\hat{r}}{r} = 0 \quad \od{\hat{r}}{\theta} = \hat{\theta} \\
\od{\hat{\theta}}{r} = 0 \quad \od{\hat{\theta}}{\theta} = - \hat{r}
\end{cases} \]

In terms of these polar coordinates we can also write expressions for the velocity and acceleration:
\[ \begin{cases}
\vec{r} = r \hat{r} \\
\vec{v} = \dot{r}\hat{r} + (r\dot{\theta})\hat{\theta} \\
\vec{a} = \left(\ddot{r}-r\dot{\theta}^2\right)\hat{r} + \left(r\ddot{\theta} + 2\dot{r}\dot{\theta}\right)\hat{\theta}
\end{cases} \]

\section{Rectilinear motion}
If the trajectory of a particle is restricted to a straight line, we do not even need vectors to describe locations, velocities or accelerations etc.

We can always choose a reference frame such that the position vector is of the form
\[ \vec{r}(t) = x(t)\hat{i}. \]
Where $x$ is a real number that depends on time.

We can define and describe some particular forms of rectilinear motion ($a_0, v_0, x_0, t_0$ are constants denoting the initial acceleration, velocity, position and time, respectively):
\begin{enumerate}
\item \textbf{Uniform rectilinear motion} is when the particle is moving at constant velocity.
\[ \begin{cases}
a(t) = 0 \\
v(t) = v_0 \\
x(t) = v_0(t-t_0) + x_0
\end{cases} \]
\item For rectilinear motion with \textbf{constant acceleration}, we get the following:
\[ \begin{cases}
a(t) = a_0 \\
v(t) = a_0(t-t_0) + v_0 \\
x(t) = \frac{a_0}{2}(t-t_0)^2 + v_0(t-t0) + x_0
\end{cases} \]
\end{enumerate}

TODO: combining these cases (projectile motion).

\section{Circular motion}
\subsection{Uniform circular motion}
TODO image.

In this case a particle is traveling along a circular trajectory with radius $R$ at a constant speed $v$. We choose a reference frame such that the origin is the centre of the circle and the trajectory is in the plane spanned by $\hat{i}, \hat{j}$. The arc length traveled in time $t$ is $vt$. In terms of the angle $\theta$ the arc length is $R\theta$, so we have
\[ \theta = \frac{vt}{R} = \omega t \]
where $\theta$ is called the \udef{angular velocity}. In general it is defined for any circular motion as $\od{\theta}{t}$.

We can then write
\[ \begin{cases}
\vec{r} = R\cos(\omega t)\hat{i} + R\sin(\omega t)\hat{j} \\
\vec{v} = - v\sin(\omega t)\hat{i} + v \cos(\omega t) \hat{j} \\
\vec{a} = - \frac{v^2}{R}\cos(\omega t)\hat{i} - \frac{v^2}{R}\sin(\omega t) \hat{j}
\end{cases} \]

Alternatively we can work in polar coordinates, in which case
\[ \begin{cases}
\vec{r} = R \hat{r}\\
\vec{v} = v \hat{\theta} \\
\vec{a} = - \left(\frac{v^2}{R}\right)\hat{r}
\end{cases} \]

\subsection{General circular motion}
In general circular motion can be described using polar coordinates as follows:
\[ \begin{cases}
\vec{r} = R \hat{r}\\
\vec{v} = v \hat{\theta} \\
\vec{a} = - \left(\frac{v^2}{R}\right)\hat{r} + \dot{v}\hat{\theta}
\end{cases} \]

TODO centripetal force and acceleration (radial component).

\section{Rigid body in planar motion}
TODO?

\chapter{Dynamics}
\section{Some forces to get us started}
Some forces traditionally associated with mechanics.
\subsection{Internal forces in a body}
\subsection{Gravity}
\subsubsection{Universal law of gravitation}
\paragraph{Gravity of spherical shells}
\paragraph{Gravity of a sphere}
\subsubsection{Approximation on the earth's surface.}
\subsection{Normal force}
\subsection{Elasticity: Hooke's law}
\subsection{Friction and drag}
\section{Net forces and free-body diagrams}
\section{General principles for solving problems in dynamics}

\chapter{Statics}
\begin{enumerate}
\item Force balance
\item Torque balance
\end{enumerate}

\chapter{Conservation principles}
\begin{itemize}
\item Homogeneity of spacetime $\rightarrow$ translational invariance $\rightarrow$ conservation of momentum (conservation of $p^\mu$)
\item Isotropicity of spacetime $\rightarrow$ rotational invariance $\rightarrow$ conservation of $\vec{L}$
\end{itemize}
\section{The energy principle}
\subsection{Work and power}
For now only definition.
\[ W = \int \vec{F}\cdot \diff{\vec{l}} \]
\[ P(t) = \vec{F}(t, \vec{r}, \dot{\vec{r}})\cdot \dot{\vec{r}} ?? \]
\[ W = \int_I P(t) \diff{t} \]
\subsection{Conservative forces}
Link work to potential energy.

A force is conservative if the work done by the force over a closed trajectory is zero. This is equivalent to saying there is some $U(\vec{r})$, called the \udef{potential}, such that
\[ \vec{F}(\vec{r}) = -\div U(\vec{r}). \]
In this case the work done by the force over a trajectory only depends on the initial and final point, not the whole trajectory:
\[ W = \int -\div U \cdot \diff{\vec{l}} = U(\vec{a}) - U(\vec{b}) = -\Delta U \]


\subsection{Kinetic energy}
We define the kinetic energy
\[ K = \frac{m \vec{v}\cdot \vec{v}}{2} = \frac{mv^2}{2}. \]
Taking the derivative with respect to time, we get
\[ \od{K}{t} = m \od{\vec{v}}{t} \cdot \vec{v} = m\vec{a}\cdot \vec{v} = \vec{F}\cdot \vec{v}. \]
Integrating over the time interval
\[ K_2-K_1 = \int_{t_1}^{t_2} \vec{F}\cdot \vec{v} \diff{t} = \int \vec{F}\cdot \diff{\vec{l}} = W\]
Work-energy principle.

\subsection{Conservation of energy}
For conservative forces.
\begin{align*}
W &= -\Delta U \\
&= \Delta K
\end{align*}
Thus $\Delta K + \Delta U = 0$ or
\[ E_1 \equiv K_1 + U_1 = K_2 + U_2 \equiv E_2 \]

TODO Gravitational potential. Newton's theorem. (See Hartle); TODO multiple forces.
\section{The linear momentum principle}
\[ \vec{F}_{\text{tot}} = \sum \vec{F} = \od{\vec{p}}{t} \]
\[ \vec{F}_{\text{ext}} = \sum \vec{F} = \od{\vec{p}}{t} \]
\section{The angular momentum principle}

\chapter{Rigid body mechanics}
\section{Rigid body kinematics}
\section{Rigid body dynamics}
TODO: torque alignment example

\chapter{Small oscillations}

\chapter{Applications}
\addtocontents{toc}{\protect\setcounter{tocdepth}{3}}
\section{Some simple systems}

\section{Harmonic (linear) oscillators}
90 percent of physics.

Potential around minimum: constant term arbitrary, first order zero (min), second order most important.

Prototype for small vibrations about stable equilibrium


\section{Non-linear oscillations and phase space}

\section{Orbits in central field}
TODO some history + elaborate whole section

Newton proves Kepler's laws
\begin{eigenschap}
Kepler's laws
\begin{enumerate}
\item[First law] Each of the planets moves on an elliptical path with the sun at one focus of the ellipse.
\item[Second law] For each of the planets, the straight line connecting the planet to the sun sweeps out equal areas in equal times.
\item[Third law] The squares of the periods of the planets are proportional to the cubes of the major axes of their orbits.
\end{enumerate}
\end{eigenschap}

one-body problem: central field
\begin{definition}
A force field $\vec{F}(\vec{r})$ is said to be a \udef{central field} with centre $\mathcal{O}$ if it has the form
\[ \vec{F}(\vec{r}) = F(r) \hat{r} \]
where $r = |\vec{r}|$ and $\hat{r} = \vec{r}/r$. A central field is thus \ueig{spherically symmetric} about its centre.
\end{definition}

\subsection{Newton's equations in a central field}
Firstly we must remark that any orbit of a particle $P$ in a central field with centre $\mathcal{O}$ must \emph{lie in a plane through $\mathcal{O}$}. This can easily be seen on grounds of symmetry. Mirroring the whole system across this plane, we see that it is unchanged. This means that were there any effect causing the particle to leave the plane one one side, a similar effect would exist to cause it to leave on the other side, clearly a contradiction.

This means all motion is two-dimensional and we can use the polar coordinates $r,\theta$ (centred on $\mathcal{O}$) to specify the location of $P$ at all times.

Using the general form of the acceleration in polar coordinates, we can write the Newtonian equations of motion as follows:
\begin{align*}
m \left(\ddot{r}-r \dot{\theta}^2\right) &= F(r) \\
m \left(r \ddot{\theta} + 2 \dot{r} \dot{\theta}\right) &= 0
\end{align*}

\subsubsection{Angular momentum conservation}
The second of these equations leads to a conservation law, as it can be rewritten:
\[ \frac{1}{r}\od{}{t}\left(mr^2 \dot{\theta}\right) = 0 \]
which can be integrated with respect to $t$ to get
\[ mr^2 \dot{\theta} = \text{constant} \equiv L \]
We call this quantity the \udef{angular momentum}. We will define it properly in a subsequent section. For now we just need to know that it is conserved.

In fact angular momentum conservation is equivalent to Kepler's second law. The area $A$ referenced in the law can be calculated as follows (TODO: why):
\[ A = \frac{1}{2}\int_0^{\theta} r^2 \diff{\theta}. \]
By the chain rule, we have
\[ \od{A}{t} = \od{A}{\theta}\od{\theta}{t} = \frac{1}{2}r^2 \dot{\theta} = \frac{L}{2m}. \]
So, assuming the mass $m$ is constant, this means the derivative of $A$ with respect to $t$ is constant and so $A$ depends only on time intervals.

\subsubsection{Newton's equations in specific form}
It will prove useful to eliminate the mass $m$ from the theory. To do that we define the outward force per unit mass $f(r)$, such that
\[ F(r) = mf(r). \]
We also define $L$ to be the angular momentum per unit mass:
\[ l \equiv \frac{L}{m} = r^2 \dot{\theta}. \]
Newton's equations in specific form are then 
\begin{align*}
\ddot{r} - r \dot{\theta}^2 &= f(r) \\
r^2 \dot{\theta} &= L
\end{align*}

\subsubsection{Energy conservation}
Every central field is \ueig{conservative}. We define $V(r)$ to be the potential per unit mass:
\[ f(r) = - \od{V}{r}. \]
We can now write down an equation expressing conservation of energy per unit mass, $E$:
\[ \frac{1}{2}\left(\dot{r}^2 + (r \dot{\theta})^2\right) + V(r) = E \]
This can be rewritten using angular momentum:
\begin{align*}
E &= \frac{1}{2}\dot{r}^2 + V(r) + \frac{l^2}{2r^2} \\
&= \frac{1}{2}\dot{r}^2 + V^*(r)
\end{align*}
Where $V^*(r) \equiv V(r) + \frac{l^2}{2r^2}$ is the \udef{effective potential}. We call this equation the \udef{radial motion equation}.

\subsection{Describing orbits}
The radial motion equation is not analytically solvable for an arbitrary potential, but we can still make some qualitative remarks.

Firstly, of course, we must have
\[ V^*(r) \leq E, \]
with equality when $\dot{r} = 0$.

Now we can make a distinction between two types of motion (TODO image):
\begin{enumerate}
\item \textbf{Bounded motion} TODO
\end{enumerate}


\subsection{The path equation}

\subsection{Nearly circular orbits}

\subsection{The attractive inverse square field}

\subsection{Space travel - Hohmann transfer orbits}


\section{Rutherford scattering}
\subsection{The experiment}
\subsection{The repulsive inverse square field}
\subsection{Cross-sections}

\addtocontents{toc}{\protect\setcounter{tocdepth}{5}}

