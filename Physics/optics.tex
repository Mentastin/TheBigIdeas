pinhole camera
specularity (diffuse / ...)
geometric optical theory
colours and theory
rainbows
X-ray crystallography (Laue diffraction)
interferometer
retroreflectors
Lambert-beer


\chapter{Geometrical optics}
\section{Properties of light and assumptions}
\section{Plane surfaces and prisms}
\section{Spherical mirrors and lenses}
\subsection{Spherical surfaces}
\subsection{Thin lenses}
\subsection{Thick lenses}
\subsection{Spherical mirrors}
\section{The effects of stops}
\section{Ray tracing}
\section{Lens aberrations}
\section{Optical instruments}


\chapter{Wave optics}
\section{Interference}
\section{Diffraction}
\subsection{Single opening}
\subsection{Double slit}
\subsection{Diffraction grating}
\subsection{Fresnel diffraction}
\section{Absorption and scattering}
\section{Dispersion}
\section{Reflection}
\section{Double refraction}
\section{Polarised light}
\section{Thermal radiation}
When looking at objects, we can see them because there is electromagnetic radiation that comes from the objects and travels to our eyes. At normal temperatures this radiation is light from a light source that was reflected by the object. We see the object as a certain colour because the object does not reflect light of all wavelengths equally.
