\chapter{The continuum model}
\section{Hypotheses and definition}
TODO + link mathematics - physics

Mesoscale + image

Description with vector field
\[ \vec{v}(\vec{x},t) \]
totally general

State variables:
\[ \vec{v}, \rho, T, p \]


\section{Observables}
Based on the flow field $\vec{v}(\vec{x},t)$ a number of quantities can be calculated that can be observed directly in an experimental setup.
\subsection{Particle paths}
For a given flow field $\vec{v}(\vec{x},t)$, the paths of fluid elements can be calculated by solving
\[ \boxed{\od{\vec{x}}{t} = \vec{v}(\vec{x},t)}. \]
Particle paths can be observed experimentally by tracking an object floating in the fluid or an ink dot. The tracking may be done by recording the placement at different times or taking a picture with a long exposure time.
\subsection{Streamlines}
A \udef{streamline} is the path a fluid element would take if the flow field was \udef{static}, meaning it did not change over time. Flows described by a static flow field are called \udef{steady flows}.

Streamlines can be observed experimentally by taking very short snapshots, for example by injecting several ink markers, taking a photo with an exposure time just long enough to get short streaks.

A path $\gamma: \R \to \E^3: u \mapsto \gamma(u)$ is a streamline, if its tangent vectors are proportional to vectors of $\vec{v}(\vec{x},t_0)$:
\[ \od{\gamma(u)}{u} = \lambda \cdot \vec{v}(\vec{x},t_0). \]
Fixing a basis, splitting $\gamma$ into its components $\gamma(u) = \left(x(u), y(u), z(u)\right)$ and writing $\vec{v} = (v_x, v_y, v_z)$, the differential equation can be written as
\[ \frac{\od{x}{u}}{v_x} = \frac{\od{y}{u}}{v_y} = \frac{\od{z}{u}}{v_z} \]

\subsection{Streaklines}
A \udef{streakline} is made up of the current location of all fluid elements that passed through a fixed position $\vec{x}_0$ during the time interval $[t_0, t]$ from a time $t_0$ until now.

Experimentally this can be realised by continuously releasing ink or smoke (depending of the type of fluid we are dealing with) at a fixed location.

Because streaklines are composed of locations of fluid elements, the tangent vectors of a streakline $\gamma(t)$ must match the vectors of the flow field, $\od{\gamma}{t} = \vec{v}(\gamma(t),t)$. Additionally the streakline must originate at a location $\vec{x}_0$ at a time $t_0$. The problem then becomes to find a curve
\[ \gamma: [t_0, t] \to \E^3: t\mapsto \gamma(t) \]
such that
\[ \begin{cases}
\od{\gamma}{t} = \vec{v}(\gamma(t),t) \\
\gamma(t_0) = \vec{x}_0.
\end{cases} \]

\section{Mathematical description}
\subsection{Eulerian and Lagrangian descriptions}
The flow of a fluid can be studied by considering the velocity, density, temperature, pressure at different points in space. In other words, for each point $\vec{x}$ and time $t$, each of the above variables assumes a certain value. This is known as the \udef{Eulerian description} of flow.

The \udef{Lagrangian description} assigns values of these variables to (infinitesimal) fluid elements, not to points in space. A particular fluid element my be referenced by supplying its location $\vec{a}$ at $t=0$. In the Lagrangian description the independent variables are $\vec{a}$ and $t$, not $\vec{x}$ and $t$.
\[ \begin{cases}
\vec{v}(\vec{x},t) & \text{Eulerian description} \\
\vec{v}(\vec{a},t) & \text{Lagrangian description} \\
\end{cases} \]

One can move from the Eulerian description to the Lagrangian description by filling in a particular path $\vec{x}(\vec{a},t)$. Such a particle path can be obtained by solving the equation cited above ($\od{\vec{x}}{t} = \vec{v}$) with the boundary condition
\[ \vec{x} = \vec{a} \qquad \text{at} \qquad t=0 \]

\subsection{Material derivative}
The two descriptions are completely equivalent, but often the Eulerian description will lead to simpler expressions. For this reason we will mostly use this description. It will however often prove useful to consider the partial derivative with respect to time in the Lagrangian description, which means keeping the initial position constant, i.e. differentiating following a fluid particle. This is as opposed to differentiating with respect to time keeping the Eulerian spatial coordinates constant.

When working in the Eulerian description, the latter is just the usual partial derivative, $\pd{}{t}$. The former derivative, following the fluid, is sometimes called the \udef{material derivative}, the \udef{convective derivative} or the \udef{Lagrangian derivative}. It will be denoted $\od{}{t}$.

Say $f(\vec{x},t) = f(x,y,z;t)$ is a scalar quantity of interest in the fluid (in the Eulerian description). The notation of the material derivative comes from the equality
\[ \od{f}{t} = \od{}{t}f(\vec{x}(t), t) = \od{}{t}f(x(t), y(t), z(t);t)\]
where $\vec{x}(t) = (x(t), y(t), z(t))$ is the path of a fluid element. The left hand derivative is material while the right hand one is an ordinary derivative seeing as $f(\vec{x}(t), t)$ only depend on $t$.

The material derivative can be calculated using the chain rule
\begin{align*}
\od{f}{t} &= \pd{f}{t} + \pd{f}{x}\od{x}{t} + \pd{f}{y}\od{y}{t} + \pd{f}{z}\od{z}{t} \\
&= \pd{f}{t} + \pd{f}{x}v_x + \pd{f}{y}v_y + \pd{f}{z}v_z \\
&= \pd{f}{t} + \vec{v}\cdot \grad f
\end{align*}
where $\vec{v} = (v_x, v_y, v_z)$ is the value of the flow field at the point $\vec{x}(t)$.

This expression for the material derivative is sometimes taken as the definition.

The material derivative is readily generalised to vectors $\vec{u}$, by viewing the derivative $\vec{v}\cdot \grad$ as the streamline directional derivative. So we may write
\[ \od{\vec{u}}{t} = \pd{\vec{u}}{t} + (\vec{v}\cdot\nabla)\vec{u}. \]
This works for arbitrary tensors as well.

Another possible generalisation is to the tensor derivative $\vec{v}\cdot(\nabla T)$. With $T$ a tensor. This gives the same result.

\subsubsection{Confusions in terminology}
Many different names are sometimes used for the material derivative, such as
\begin{itemize}
\item advective derivative
\item convective derivative
\item derivative following the motion
\item hydrodynamic derivative
\item Lagrangian derivative
\item particle derivative
\item substantial derivative
\item substantive derivative
\item Stokes derivative
\item total derivative
\end{itemize}
Some of these, such as the total derivative, are commonly used to refer to something else entirely. Often the term convective derivative is used to refer to only the spatial term $\vec{v}\cdot \grad f$.

Some authors then make the further distinction between advection for scalars and convection for tensors.

\subsubsection{Link with streamlines}
For streamlines we assume we are in a steady flow, i.e. $\pd{f}{t} = 0$. This means that
\begin{align*}
\od{f}{t} &= \vec{v}\cdot \grad f \\
&= v \;\vec{t} \cdot \grad f & \text{where $\vec{t}$ is the unit tangent vector along the streamline} \\
&= v \pd{f}{s} & \text{where $s$ is the distance along the streamline.}
\end{align*}
Thus $\vec{v}\cdot \grad f = 0$ means $f$ is constant along streamlines.

\subsubsection{From Lagrangian to Eulerian descriptions at a certain time}
The transformation $\vec{a}\to \vec{x}$ is a transformation of Euclidean space with time $t$ as a parameter. Assuming the fluid is a continuum, this transformation is continuous. On physical grounds, it must be one to one and have an inverse.

The transformation has a Jacobian
\[ J(\vec{x},t) = \pd{(x_1,x_2, x_3)}{(a_1, a_2, a_3)}. \]
Since the transformation is invertible and continuous, $J$ is bounded and non-zero and we can find the convective derivative of $J$:
\begin{align*}
\od{J}{t} &= \pd{(\od{x_1}{t},x_2, x_3)}{(a_1, a_2, a_3)}+\pd{(x_1,\od{x_2}{t}, x_3)}{(a_1, a_2, a_3)}+ \pd{(x_1,x_2, \od{x_3}{t})}{(a_1, a_2, a_3)} \\
&= \pd{(v_1,x_2, x_3)}{(a_1, a_2, a_3)} + \pd{(x_1,v_2, x_3)}{(a_1, a_2, a_3)} + \pd{(x_1,x_2, v_3)}{(a_1, a_2, a_3)}.
\end{align*}
From
\begin{align*}
\pd{(v_1,x_2, x_3)}{(a_1, a_2, a_3)} &= \det \begin{pmatrix}
\pd{v_1}{x_1}\pd{x_1}{a_1} & \pd{v_1}{x_1}\pd{x_1}{a_2} & \pd{v_1}{x_1}\pd{x_1}{a_3} \\
\pd{x_2}{a_1} & \pd{x_2}{a_2} & \pd{x_2}{a_3} \\
\pd{x_3}{a_1} & \pd{x_3}{a_2} & \pd{x_3}{a_3}
\end{pmatrix} \\
&= \pd{v_1}{x_1}J
\end{align*}
we get that
\begin{equation}
\od{J}{t} = J \div \vec{v}. \label{eq:convectiveDerivativeJacobian}
\end{equation}


\section{Reynolds' transport theorem}
Sometimes we want to follow a finite (i.e. not infinitesimal) fluid volume, called a \udef{material fluid volume}. This volume will expand, compress and deform as it moves, but always contains the same particles.

Let $L(\vec{x},t)$ be any property of the fluid. We can then associate the property 
\[\int_{V(t)} L(\vec{x},t) \diff{V(\vec{x})} \]
with the material volume $V(\vec{x})$. We are now interested in how this changes in time, i.e. the time derivative. We calculate the time derivative by transforming to Lagrangian coordinates, which makes the infinitesimal volume element $\diff{V(\vec{a})}$ no longer depend on time:
\begin{align*}
\od{}{t}\left[\int_{V(t)} L(\vec{x},t) \diff{V(\vec{x})}\right] &= \od{}{t}\left[\int_{V(0)} L(\vec{x}(\vec{a},t),t) J \diff{V(\vec{a})}\right] \\
&= \int_{V(0)} \od{}{t}\left[L(\vec{x}(\vec{a},t),t) J \right] \diff{V(\vec{a})} \\
&= \int_{V(0)} \left[J \od{L}{t} + L \od{J}{t} \right] \diff{V(\vec{a})} \\
&= \int_{V(0)} \left[\od{L}{t} + L \div \vec{v} \right] J \diff{V(\vec{a})} & \text{using equation \eqref{eq:convectiveDerivativeJacobian}} \\
&= \int_{V(t)} \left[\od{L}{t} + L \div \vec{v} \right] \diff{V(\vec{x})}
\end{align*}
The last equality gives the \udef{transport theorem}
\begin{equation}
\od{}{t}\left[\int_{V(t)} L(\vec{x},t) \diff{V(\vec{x})}\right] = \int_{V(t)} \left[\od{L}{t} + L \div \vec{v} \right] \diff{V(\vec{x})} \label{eq:transportTheorem}
\end{equation}


\section{Cauchy-Stokes decomposition theorem}
The Cauchy-Stokes decomposition shows that any flow can be fully characterised by its divergence and vorticity. This is to be expected as
\[ \begin{cases}
\div \vec{v} = f(\vec{x},t) \\
\curl \vec{v} = \vec{g}(\vec{x},t)
\end{cases} \]
gives 3 independent partial differential equations for three components $v_x, v_y, v_z$. (Only 2 of the 3 components of \vec{g} are independent as they have to satisfy $\div \vec{g} = 0$.) These should determine $\vec{v}$, if boundary conditions are given.
\subsection{Divergence} 
We would like to give a physical interpretation to the divergence $\div \vec{v}$. It turns out that the divergence is a measure for the rate of variation of the volume of a material fluid element. To see how, we make use of the transport theorem \eqref{eq:transportTheorem}. If we set $L(\vec{x},t) = 1$, then
\[ \int_{V(t)}\diff{V(t)} = \mathcal{V}(t) = \text{volume of the material fluid element.} \]
The transport theorem then gives
\[ \od{\mathcal{V}(t)}{t} = \int_{V(t)} \left(\div\vec{v}\right)\diff{V(\vec{x})}. \]
Taking the limit of $\mathcal{V} \to 0$ and using the mean value theorem for real functions of several variables we get
\[ \lim_{\mathcal{V}\to 0} \frac{1}{\mathcal{V}(t)}\od{\mathcal{V}(t)}{t} = \div \vec{v} \]
\subsubsection{Incompressible flow}
A flow is \udef{incompressible} if no material fluid volume can change its volume as it moves. Since
\[ \int \div \vec{v} \diff{V} = 0 \]
for all possible material volumes, it follows that
\[ \div \vec{v} = 0 \]
everywhere in an incompressible flow.

\subsection{Vorticity}
We define the \udef{vorticity $\vec{\omega}$} as the curl of the velocity field.
\[ \vec{\omega} \equiv \curl \vec{v} \]
The vorticity is a measure for the local rotation of the fluid.


TODO
Consider the material derivative of the velocity field.
\[\od{\vec{v}}{t} = \pd{\vec{v}}{t} + (\vec{v}\cdot\nabla)\vec{v}.\]
Using the vector identity
\[ \vec{a}\times (\curl \vec{a}) = \grad \left(\frac{a^2}{2}\right) - (\vec{a}\cdot \nabla)\vec{a} \]
we write
\begin{align*}
(\vec{v}\cdot \nabla)\vec{v} &= \nabla \left(\frac{v^2}{2}\right) - \vec{v}\times(\curl \vec{v}) \\
&= \nabla \left(\frac{v^2}{2}\right) - \vec{v}\times(\omega).
\end{align*}
This contains the kinetic energy $v^2/2$ and the vorticity explicitly.

\subsection{Decomposition}
We shall now show that in the neighbourhood of each point of the fluid $\vec{x}_0$, the velocity is the sum of a translation, a rigid rotation and a deformation.

We start by doing a Taylor expansion around a point $\vec{x}_0$:
\begin{align*}
\vec{v}(\vec{x}) &= \vec{v}(\vec{x}_0) = \begin{pmatrix}
\pd{v_x}{x} & \pd{v_x}{y} & \pd{v_x}{z} \\
\pd{v_y}{x} & \pd{v_y}{y} & \pd{v_y}{z} \\
\pd{v_z}{x} & \pd{v_z}{y} & \pd{v_z}{z}
\end{pmatrix}\begin{pmatrix}
h_x \\ h_y \\ h_z
\end{pmatrix} + O(h^2) \\
&= \vec{v}(\vec{x}_0) + M \vec{h} + O(h^2)
\end{align*}
The we write $M$ as the sum of a symmetric and an antisymmetric matrix:
\begin{align*}
M &= \frac{1}{2}(M+M^\intercal) + \frac{1}{2}(M-M^\intercal) \\
&= D+R
\end{align*}
The matrix $D$ will turn out to be associated with the deformation and $R$ with the rotation.

\subsubsection{2D flows}
In two dimensions the elements of the decomposition are simply
\begin{itemize}
\item $\vec{\omega} = \left(\pd{v_y}{x} - \pd{v_x}{y}\right)\hat{z}$
\item $D = \begin{pmatrix}
\pd{v_x}{x} & \frac{1}{2}\left(\pd{v_x}{y} + \pd{v_y}{x}\right) \\
\frac{1}{2}\left(\pd{v_x}{y} + \pd{v_y}{x}\right) & \pd{v_y}{y}
\end{pmatrix}$
\item $\frac{(\vec{\omega}_0\; \times \; \vec{x})}{2} = \frac{1}{2}(-\omega_0 y, \omega_0 x)$
\end{itemize}

\section{Examples of flow models}
\subsection{In two dimensions}
\subsubsection{Point vortex}
\subsubsection{Rankine vortex}
\subsection{In three dimensions}

\chapter{Some naive formulae}
\paragraph{Hydrostatic pressure}
\paragraph{Archimedes' principle}
\paragraph{Bernoulli's equation}

\chapter{Navier-Stokes equations}
\section{Conservation of mass}
\section{Conservation of momentum}
\section{Conservation of energy}
surface energy and tension

Young–Laplace equation
\section{Navier-Stokes equations and Euler equations}

\chapter{Properties of fluid flows}
\section{Compressibility}
\section{Viscosity}
\subsection{Reynolds number}
\section{Vorticity}
\subsection{Crocco's theorem}
\subsection{Vortex tubes}
\subsection{Kelvins's theorem for isentropic flow}

\chapter{Calculations under certain assumptions}
\section{Fluid at rest: hydrostatic pressure}
\section{Irrotational: potential flows}
\section{Bernoulli's equations}
\section{Examples of low Reynolds' number flows}
\subsection{Plane Couette flow}
\subsection{Plane Poiseuille flow}
\subsection{Poiseuille flow (blood flow)}

\chapter{Airflow around a wing}
\section{Potential flow}
\section{Flow around a cylinder}
\section{Flow around a wing}

\chapter{Waves}
TODO phase / group velocity
\section{Surface gravity waves}
\section{Internal gravity waves}
\section{Sound waves}
\subsection{Intensity of sound: Decibels}
\subsection{The Doppler effect}
\subsection{Shock Waves and the sonic boom}