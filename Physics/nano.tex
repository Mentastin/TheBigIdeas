\chapter{Introduction}

SKIPPED SLIDES: 23, 34 (Lotus effect), 39, 47-52, 59,60, 76-83, 85-91, 115-124, 131-140, 145-146

Nanoscience studies objects at the nanometer scale. Practically this means objects smaller than \SI{100}{nm}. Examples of such objects include virusses, clusters (small groups of atoms), DNA, fullerenes, nanotubes \ldots

Objects in this regime are typically too small for classical physics to make useful predictions and too large for quantum techniques to be feasible.

One major consequence of this is that material properties (such as colour, melting temperature etc.) become dependent on the size of the objects. In bulk this is obviously not the case.

There are different types of nanomaterials. Examples include
\begin{enumerate}
\item Nanoclusters. There are many types of nanoclusters, with differing levels of internal structure:
\begin{enumerate}
\item Simple nanoparticles
\item Janus particles: half one material and half another
\item Core-shell
\item Nano-planets
\item DLA fractals
\item Nano-prism
\item Nano-shell
\item Nano-hole
\end{enumerate}
\item Nanostructured materials (3D). This is material with nanoscale features such as dislocations, grain boundaries or defects.
\item Nanostructured film.
\item Multilayered materials.
\item Novel materials such as metamaterials.
\end{enumerate}

Synthesis techniques for nanomaterials can broadly be categorized as either bottom-up or top-down. Bottom-up synthesis includes techniques like nano-positioning and self-assembly. Top-down techniques include nanofabrication and nanolithography.

Applications of nanostructures include the following:
\begin{enumerate}
\item Applications based on optical properties (in which case metals such as copper, silver and gold are used).
\begin{enumerate}
\item Plasmonic sensors
\item Plasmonic waveguides
\item Nano-antennas
\item Iperthermia (cancer therapy)
\item Extroardinary transmission of light (EOT)
\item Metamaterials
\end{enumerate}
\item Applications based on magnetic properties (in which case metals such as iron, cobalt and nickel are used).
\begin{enumerate}
\item Super-paramagnetism
\item Magneto-optical properties
\item Medical diagnosis (Fe-oxides, NMR)
\end{enumerate}
\item Applications based on catalytic properties (in which case metals such as iron, cobalt, nickel, ruthenium, rhodium, palladium, osmium, iridium and platinum are used).
\begin{enumerate}
\item Catalysis and photo-catalysis
\item Water splitting and hydrogen storage.
\end{enumerate}
\end{enumerate}

\chapter{Some descriptive parameters}
\begin{itemize}
\item We denote the number of atoms in a cluster $N$.
\item We denote the (ionic) atomic radius $R_0$. In a face-centered cubic lattice, with lattice constant $a$, the distance between atoms is $a/\sqrt{2}$. So
\[ R_{0,\text{fcc}} = \frac{a}{4}\sqrt{2} \]
\item Cluster volume $V$. This definition disregards the atomic packing factor (TODO? mainly N dependence as with others).
\[ V =  \frac{4\pi}{3}R_0^3N \]
\item We can get an effective radius $R_\text{eff}$ by assuming the volume is spherical: $V = \frac{4\pi}{3}R_\text{eff}^3$. Thus we get
\[ R_\text{eff} = \left(\frac{V}{4\pi/3}\right)^{1/3} = R_0 N^{1/3} \]
\item Cluster surface $S$.
\[ S = 4\pi R^2_\text{eff} = 4\pi R_0^2N^{2/3} \]
\item Number of surface atoms $N_S$.
\[ N_S = \frac{S}{S_A} = \frac{4\pi R_0^2 N^{2/3}}{\pi R_0^2} = 4 N^{2/3} \]
With $S_A$ the fraction of the surface that a single atom occupies. This is essentially the cross-sectional surface (hence $\pi R_0^2$).
\item Finally we define the fraction of surface atoms $F$:
\[ F = \frac{N_S}{N} = \frac{4}{N^{1/3}} = \frac{4R_0}{R_\text{eff}} \]
We also give some values for $F$:
\begin{table}[h]
  \centering
  $\begin{array}{c|c}
    N & F\\
	\hline
	10^2 & 0.86\\
    10^3 & 0.40\\
    10^4 & 0.04\\
  \end{array}$
  \caption{Fraction of surface atoms}
  \label{surfaceAtomFraction}
\end{table}
\end{itemize}

\chapter{Size equations}
When we get down to nano scales, chemical and physical properties start to depend on size. We postulate that this dependency takes the following form, where $A$ represents a generic property:
\[ A(N) = A(\infty) \left(1+\frac{C_N}{N^\alpha}\right) \qquad A(R) = A(\infty) \left(1+\frac{C_R}{R^\alpha}\right) \]
The equation of the left gives the property in function of the number of atoms, $N$. The equation on the right gives it in function of the radius $R$. The bulk limit of $A$ is denoted $A(\infty)$ and $C$ and $\alpha$ are constants.

\chapter{Thermodynamic properties of nanostructured materials}
\section{Melting temperature size equation}
We want to know how melting temperature $T_M$ depends on the the size of the nanoparticles. Intuitively we can guess that the melting temperature would be lower for nanoparticles, because they have non-negligible fractions of there atoms at the surface and surface atoms  have more freedom to move and be excited by thermal energy. We would, however, like a more formal argument.


\subsection{Liquid drop}
Phase transition occurs when there is an equilibrium between the solid ($S$) and liquid ($L$) phase. (TODO: why exactly)
\[ \mu_L(T_L,P_L) = \mu_S(T_S,P_S) \]
The subscripts $L$ an $S$ are added to the temperature and pressure because they may not be the same for each phase. In fact at equilibrium there must also be thermal equilibrium, so
\[ T_L = T_P = T \]
which is the melting temperature we are looking for. The same cannot be said for the pressure.

We can expand the chemical potential to first order
\[ \mu(T,P) = \mu(T_0,P_0) + \pd{\mu}{T}(T-T_0) + \pd{\mu}{P}(P-P_0) + \ldots \]
We can get expressions for the partial derivatives from the Gibbs-Duhem relation
\[ S \diff{T} - V \diff{P} + N \diff{\mu} = 0 \]
which we can rewrite
\[ \diff{\mu} = - \frac{S}{N}\diff{T} + \frac{V}{N}\diff{P} = -s \diff{T} + \frac{1}{\rho}\diff{P} \]
where we define
\[ \begin{cases}
s \equiv \frac{S}{N} = - \left(\pd{\mu}{T}\right)_P \\
\frac{1}{\rho} \equiv \frac{V}{N} = \left(\pd{\mu}{P}\right)_T
\end{cases} \]

The equilibrium condition, to first order, is given by
\begin{align*}
0 &= \mu_L(T_0,P_0) - s_L(T-T_0) + \frac{1}{\rho_L}(P_L-P_0) - \mu_S(T_0,P_0) + s_S(T-T_0) - \frac{1}{\rho_S}(P_S-P_0) + \ldots
\end{align*}
We also choose the for expansion of the chemical potential to be around the the triple point of the bulk phase. At this point
\[ \mu_L(T_0,P_0) = \mu_S(T_0,P_0) \]
further simplifying the equilibrium condition to
\[ 0 =  (s_L-s_S)(T-T_0) - \frac{1}{\rho_L}(P_L-P_0) + \frac{1}{\rho_S}(P_S-P_0) + \ldots \]
In order to get expressions for the pressures $P_L$ and $P_S$, we use the Young-Laplace equation for the pressure difference across an interface between two static fluids. (TODO: justification, assumes Curie-Wulff crystal? Refer Ph. Buffat , J P . Borel , Phys . Rev. A 13 (1976) 2287) For spheres Laplace's law gives us the following
\[ \begin{cases}
P_L = P_\text{ext} + 2 \frac{\gamma_L}{R_L} \approx 2 \frac{\gamma_L}{R_L} \\
P_S = P_\text{ext} + 2 \frac{\gamma_S}{R_S} \approx 2 \frac{\gamma_S}{R_S}
\end{cases} \]
where $P_\text{ext}$ is the pressure not due to the interface. For small particles this in negligible compared to the pressure gradient across the interface. The constants $\gamma_L$ and $\gamma_S$ are the surface tension of the liquid ans solid phase, respectively.

Recognising that the nanoparticle has the same number of atoms, regardless of which phase it is in, we get
\[ R_S = \left(\frac{\rho_L}{\rho_S}\right)^{1/3}R_L \]

Incorporating these equations into the equilibrium condition, we get
\[ 0 = (s_L-s_S)T_0 \left(\frac{T}{T_0}-1\right) + 2 \left(\frac{\gamma_S}{R_S\rho_S}- \frac{\gamma_L}{R_L\rho_L}\right) + \ldots \]

Finally we can rewrite this using that the latent heat of fusion per atom is given by
\[ L = (s_L-s_S)T_0 \]
and assuming $\rho_L \approx \rho_S$, so we can drop the last term:
\[\frac{\Delta}{T_0} \equiv \frac{T-T_0}{T_0} = - \frac{2}{LR_S\rho_S}\left(\gamma_S -  \gamma_L \left(\frac{\rho_S}{\rho_L}\right)^{2/3}\right) = - \frac{A}{R} < 0\]
where $A$ is just a constant. In fact it is the constant in the size equation
\[ T_M(R) = T_M(\infty)\left(1- \frac{A}{R}\right) \]

In deriving this result, we made the following assumptions and approximations:
\begin{enumerate}
\item The chemical potential was only expanded to first order;
\item Spherical clusters;
\item $\rho_L \approx \rho_S$.
\end{enumerate}
Unfortunately this model does not fit reality very well. (TODO image) The obvious way to improve the model is to expand the chemical potential to second order.
To do that we need expressions for the second order partial derivatives of $\mu$:
\[ \begin{cases}
\pd[2]{\mu}{P} = - \frac{1}{\rho^2}\pd{\rho}{P} = - \frac{\chi}{\rho} \\
\pd[2]{\mu}{T} = - \pd{s}{T} = - \frac{C_P}{T} \\
\pd{\mu}{P}{T} = - \frac{1}{\rho^2}\pd{\rho}{T} \approx \frac{3\alpha}{\rho}
\end{cases} \]
where $\chi$ is the isothermal compressibility coefficient, $C_P$ is the specific heat at constant pressure and $\alpha$ is the linear-expansion coefficient.

We can the equilibrium condition is then
\[ L \frac{T_0-T}{T_0} - \frac{2}{\rho_SR_S}\left[\gamma_S-\gamma_L \left(\frac{\rho_S}{\rho_L}\right)^{2/3}\right] + \frac{C_{P,S}-C_{P,L}}{2}T_0 \left(\frac{T_0-T}{T_0}\right)^{2} - \frac{2}{\rho_SR_S}\left[\gamma_S(\eta_S - 2\alpha_S)-\gamma_L(\eta_L - 2\alpha_L)\left(\frac{\rho_S}{\rho_L}\right)^{2/3}\right](T_0-T) + \frac{2}{\rho_SR^2_S}\left[\chi_S\gamma_S^2 - \chi_L\gamma_L^2 \left(\frac{\rho_S}{\rho_L}^{1/3}\right)\right] \]
where $\eta = - \frac{1}{\gamma}\pd{\gamma}{T}$.

Before considering other improvements to the liquid drop model, we will first consider another way to obtain the same results, based on a different equilibrium condition. We now imagine the nanoparticles suspended in a (vapor, TODO: necessary?) medium. We now write the Gibbs free energy of the entire nanoparticle, in function of the radius $R$
\[ \begin{cases}
G_S = N\mu_S + 4\pi R_S^2\gamma_{SV} \\
G_L = N\mu_L + 4\pi R_L^2\gamma_{LV}
\end{cases} \]
where $\gamma_{SV}$ and $\gamma_{LV}$ are now the interfacial energies between vapor and the solid or liquid, respectively (TODO explicate difference).

Using $R_S = \left(\frac{\rho_L}{\rho_S}\right)^{1/3}R_L$ and
\begin{align*}
(\mu_L - \mu_S) &= \Delta G_\text{vol} = \Delta g_\text{vol} \\
&= \Delta h - T \Delta s \\
&= L - T \frac{L}{T_0}
\end{align*} (TODO explain)
we can write the difference in Gibbs free energy as
\begin{align*}
\Delta G &\equiv G_L - G_S \\
&= N(\mu_L-\mu_S) + 4\pi R^2 \left[\gamma_{LV} \left(\frac{\rho_S}{\rho_L}\right)^{2/3}-\gamma_{SV}\right] \\
&= \frac{4\pi R^3}{3}\rho_S L \frac{T_0-T}{T_0} + 4\pi R^2 \left[\gamma_{LV} \left(\frac{\rho_S}{\rho_L}\right)^{2/3}-\gamma_{SV}\right].
\end{align*}
We can now apply a different equilibrium condition, namely $\pd{\Delta G}{R} = 0$. (TODO: why?) This yields
\[ \frac{\Delta T}{T_0} = - \frac{2}{\rho_S LR}\left(\gamma_{SV} -  \gamma_{LV} \left(\frac{\rho_S}{\rho_L}\right)^{2/3}\right) \]
which is the same result we obtained before.


\subsection{Liquid layer model (LLM)}
Another strategy is to consider not only equilibrium, but also the transition. In a first attempt we consider the transition from solid to the situation where the core of the nanosphere is solid, but a surface layer of thickness $\delta$ is liquid. For the calculation we use the method of calculating the total Gibbs free energy and differentiating with respect to the radius as before, except now the first situation is that of a solid nanosphere and the second is the situation with the liquid layer.

In order to simplify calculations, we assume $\rho_S \approx \rho_L \approx \rho$. We can then write
\[ \begin{cases}
G_1 = N\mu_s + 4\pi R^2 \gamma_{SV} \\
G_2 = (N- N')\mu_S + N'\mu_L + 4\pi R^2 \left[\gamma_{LV}+\gamma_{SL} \left(1- \frac{\delta}{R}\right)^2\right]
\end{cases} \]
where $\gamma_{SL}$ is the interfacial energy between the solid and the liquid, and $N'$ the number of atoms in the liquid phase. So
\[ \Delta G_{2,1} = G_2 - G_1 = N'(\mu_L - \mu_S) + 4\pi R^2 \left[\gamma_{LV}+\gamma_{SL}\left(1- \frac{\delta}{R}\right)^2 - \gamma_{SV}\right] \]
We can calculate the volume contribution in a similar fashion as before
\[ N'(\mu_L - \mu_S) = V_L\rho L \frac{T_0-T}{T_0} = \frac{4\pi}{3}(R^3 - (R-\delta)^3)\rho L \frac{T_0-T}{T_0} \]
with $V_L = \frac{4\pi}{3}(R^3 - (R-\delta)^3)$ the volume of the liquid part. This gives
\[ \Delta G_{2,1} =  \frac{4\pi}{3}(R^3 - (R-\delta)^3)\rho L \frac{T_0-T}{T_0} + 4\pi R^2 \left[\gamma_{LV}+\gamma_{SL}\left(1- \frac{\delta}{R}\right)^2 - \gamma_{SV}\right] \]
Applying the condition $\pd{\Delta G_{2,1}}{R} = 0$ and assuming $\gamma_{LV} \approx \gamma_{SV}$ (TODO: is that really what we are doing??), we get
\[ \frac{\Delta T}{T_0} = - \frac{2\gamma_{SL}}{\rho_S L(R-\delta)} = - \frac{A}{R-\delta} \]
with $A$ some constant. If we no longer assume $\rho_S \approx \rho_L \approx \rho$, we get
\[ \frac{\Delta T}{T_0} = - \frac{2\gamma_{SL}}{\rho_S L(R-\delta)} - \frac{2\gamma_{LV}}{\rho_S L R}\left(1- \frac{\rho_S}{\rho_L}\right) = - \frac{A}{R-\delta} - \frac{B}{R} \] with again $B$ some constant. (TODO: why not power of $2/3$?)

\subsection{Smooth interfaces interaction (SII).}
This model is a refinement of the liquid layer model. We split the total difference in Gibbs free energy obtained above into a volume contribution,
\[ \Delta G_V = \frac{4\pi}{3}(R^3 - (R-\delta)^3)\rho L \frac{T_0-T}{T_0}, \]
and a surface contribution. The difference is that, in order to account for the surfaces not being perfectly sharp, as assumed in the liquid layer model, the surface contribution is diminished by a factor that depends on the coherence length $\xi$:
\[ \Delta G_S = 4\pi R^2 \left[\gamma_{LV}+\gamma_{SL}\left(1- \frac{\delta}{R}\right)^2 - \gamma_{SV}\right] \cdot \left(1-e^{- \frac{\delta}{\xi}}\right) \]
This gives us the following
\[ \frac{\Delta T}{T_0} = - \frac{2\gamma_{SL}}{\rho_S L(R-\delta)}\left(1-e^{- \frac{\delta}{\xi}}\right) - \frac{\Gamma R^2}{\rho_S L\xi (R-\delta)^2}e^{-\delta/\xi} = - \frac{A}{R-\delta}\left(1-e^{- \frac{\delta}{\xi}}\right) - \frac{B\Gamma R^2}{(R-\delta)^2} \]
where the $\Gamma$ has been defined for ease of notation as
\[ \Gamma \equiv \gamma_{SV} - \left[\gamma_{LV} + \gamma_{SL}\left(1- \frac{\delta}{R}\right)^2\right]. \]
We notice that the result of the SII theory approaches that of the LLM as $\xi \to 0$, which is how it should be.

\subsection{Thermal expansion variation (TEV)}
Here we simply give the result (TODO: more explanation)
\[ \frac{\Delta T}{T_0} = - \frac{2}{\rho_S RL}\left(\gamma_{SM} - \gamma_{LM}\left(\frac{\rho_S}{\rho_L}\right)^{2/3}\right) + \frac{\Delta E}{L} \]
where $\Delta E$ is the strain energy difference (thermal expansion matrix-cluster).

\chapter{Synthesis of nanostructures}
In order to synthesize nanostructured materials, we can use a variety of techniques:
\begin{itemize}
\item Physical techniques:
\begin{itemize}
\item Condensation from vapor phase
\item Free expansion molecular beams
\item Sputtering (physical vapor Deposition)
\item Ion implantation
\item Ball-milling
\item Lithography, nanofabrication
\item Laser ablation
\end{itemize}
\item Chemical techniques:
\begin{itemize}
\item Colloidal chemistry
\item Sol-gel
\item Chemical vapor deposition
\end{itemize}
\item Mixed approaches combining physical and chemical techniques
\end{itemize}

\section{Nucleation}
TODO: refer to other part??

In this section we describe to formation of droplets in a homogeneous supersaturated solution, i.e. without impurities. We assume the nucleating embryos are spherical.

The first step is to write down the Gibbs free energy of a nucleating droplet, in function of the number of atoms it has already gathered
\begin{align*}
\Delta G(N) &= - N \Delta \mu + \gamma 4\pi R^2 \\
&= - N \Delta \mu + \gamma 4\pi R_0^2 N^{2/3} 
\end{align*}
where $\Delta \mu$ is the bulk change in Gibbs free energy per atom, i.e. the chemical potential.
If we plot this in function of $N$, see figure TODO, we see that $\Delta G$ has a peak at $N^*$, which we call the \udef{critical nucleus}.

We can find an expression for the critical nucleus using the condition $\left.\pd{\Delta G}{N}\right|_{N^*} = 0$:
\[ N^* = \frac{32 \pi \gamma^3}{3\rho^2 \Delta \mu^3} \]
The difference in Gibbs free energy at this peak is
\[ \Delta G(N^*) = \frac{16\pi\gamma^3}{3\rho^2\Delta \mu^2}. \]

Particles that are smaller than the critical nucleus are unstable, because the system tries to minimise $\Delta G$. So in order for nucleation to occur, thermal fluctuations must overcome the energy barrier of $\Delta G(N^*)$. We can then use Arrhenius' equation to get the nucleation speed (TODO proper ref)
\[ J = K \exp(-\Delta G(N^*)/k_B T) \]

We can also rewrite these quantities in function of the radius $R$, using the relation
\[ N = \frac{4\pi}{3}R^3 \rho. \]
We obtain
\[ \begin{cases}
\Delta G(R) = - \frac{4\pi}{3}R^3 \rho \Delta \mu + \gamma 4\pi R^2 \\
R^* = \frac{2\gamma}{\rho \Delta \mu}
\end{cases} \]

Finally we would like to give a more explicit expression for $\Delta \mu$. In order to do so we make use of the Gibbs-Thomson equation which gives the change in vapor pressure or chemical potential across a curved surface or interface.

\subsection{Gibbs-Thomson equation.}
We consider a situation where we have a vapor in an equilibrium with the liquid phase at a pressure $P_e$ and a temperature $T$. In the previous section we saw that there is a barrier to the condensation of the vapor to form clusters via nucleation. In order to overcome that barrier, we need to raise the pressure above the equilibrium value to a new pressure $P(R)$.

To arrive at the Gibbs-Thomson equation, we need to combine several results.
\begin{enumerate}
\item \textbf{Isothermal compression of an ideal gas}. We start at the ambient external pressure $P_e$ and increase the pressure isothermically until we reach $P(R)$.

The differential of the Gibbs free energy is given by
\[ \diff{G} = V \diff{P} - S \diff{T} +\mu \diff{N}. \]
To get the total change in Gibbs free energy between the start and the end of the process, we integrate. Using the fact that there is no change in temperature or number of particles as well as making use of the ideal gas law, we can write
\[ \Delta G = \int_{P_e}^{P(R)}V \diff{P} = N k_B T\int_{P_e}^{P(R)} \frac{\diff{P}}{P} = N k_B T \ln \frac{P(R)}{P_e}\]

Using the relation $G = N\mu$, we can write
\[ \mu_V[P(R)] - \mu_V[P_e] = \frac{\Delta G}{N} = k_B T \ln \frac{P(R)}{P_e}. \]
The subscript $V$ shows we are working with a vapor.
\item \textbf{Isothermal compressions of a liquid}. We assume that the liquid is incompressible, so the compression occurs with no volume variation. We write the chemical potential inside a nanoparticle in function of the radius and the vapor pressure
\[ \mu_L(R, P). \]
We have for a liquid layer (TODO WHY???)
\[ \mu_L(\infty, P(R)) \approx \mu_L(\infty, P_e). \]
Setting $R$ to infinity, we are effectively neglecting the surface contribution to the free energy.
\end{enumerate}

Previously we calculated the critical radius in function of the volume contribution to the Gibbs free energy per atom. We can invert that relation to get 
\[ \Delta \mu = \frac{2\gamma}{\rho R}. \]

The condition for the formation of a cluster with radius $R$ at the pressure $P(R)$ is
\[ \mu_V(P(R)) = \mu_L(R,P(R)) \]
now we can split $\mu_L(R,P(R))$ into a surface and volume term:
\begin{align*}
\mu_L(R,P(R)) &= \mu_L(\infty,P(R)) + \frac{2\gamma}{\rho R} \\
&= \mu_L(\infty,P_e) + \frac{2\gamma}{\rho R} \\
&= \mu_V(P_e) + \frac{2\gamma}{\rho R}
\end{align*}
where the last step was justified because $P_e$ is the liquid-vapor equilibrium, so $\mu_L(\infty,P_e) = \mu_V(P_e)$. So we get
\[ \mu_V(P(R)) - \mu_V(P_e) = \frac{2\gamma}{\rho R}. \]
But we have already found another expression for the left side of this equition, considering the isothermal compression of an ideal gas. So we can write
\[ k_B T \ln \frac{P(R)}{P_e} = \frac{2\gamma}{\rho R} \]
or
\[ P(R) = P_e e^{\frac{2\gamma}{k_B T \rho R}}. \] 

\subsection{Making nuclei form.} There are multiple ways we can get nuclei to form:
\begin{enumerate}
\item By decreasing $P_e$ (i.e. by lowering $T$):
\begin{itemize}
\item If we expand the gas adiabatically it will cool down. If it is expanded enough nucleation will occur.
\end{itemize}
\item By getting over the critical radius:
\begin{itemize}
\item Chemical or photo chemical
decomposition in the gas phase
\item Sputtering
\item Thermal evaporation
\item Laser ablation
\item Ion implantation
\end{itemize}
\end{enumerate}

\section{Growth of nanoparticles}
In order for condensation via nucleation to start, we need a supersaturated vapor. The first step then, is for nuclei to form. This can happen spontaneously due to thermal fluctuations or there can be heterogeneous nucleation, such as with ion implantation.

So long as the vapor pressure is well above $P_e$, different clusters of atoms are free to grow independently and thus non competitively. The vapor pressure goes down as it turns to liquid. This phase is called diffusion limited aggregation (DLA). 

At some point the pressure will get low enough that the clusters start competing for resources (remember that the clusters are never stable unless they have grown as beg as they can or have shrunk out of existence). Now the bigger clusters keep on growing by shrinking the smaller clusters. This process is called coarsening, or Ostwald ripening (OR).

\subsection{Diffusion-limited aggregation}
During this phase, clusters grow as
\[ R^2(t) = R_0^2 + K_1Dt \]

To calculate the growth, we start with the total flux of atoms through the surface $\Sigma$ of the cluster (with outbound the positive direction) and assume isotropicity for ease of calculation.
\[ \Phi_\Sigma = \left.\od{N}{t}\right|_{\text{out}} - \left.\od{N}{t}\right|_{\text{in}} \]
we can rewrite this using
\[ \begin{cases}
\left.\od{N}{t}\right|_{\text{in}} = C_p \od{}{t}\left(\frac{4\pi}{3}R^3\right) = 4\pi R^2 C_p \od{R}{t} \\
\left.\od{N}{t}\right|_{\text{out}} = C_e \od{}{t}\left(\frac{4\pi}{3}R^3\right) = 4\pi R^2 C_e \od{R}{t} 
\end{cases} \]
where $C_p$ is the atomic concentration inside the nanoparticle and $C_e$ is the atomic concentration just outside. Figure TODO (+ TODO explanation of curve) gives the atomic concentration in function of the distance from the centre of the nanoparticle.

Using Fick's first law
\[ \vec{J} = - D \grad C \]
we can write the flux as
\[ \Phi_\Sigma = \int_\Sigma \vec{J}\cdot \hat{n}\diff{\Sigma} = - 4\pi R^2 D \left.\pd{C(r,t)}{r}\right|_{r=R} \]

Putting these expressions together, we get
\[ (C_p - C_e)\od{R}{t} = D \left.\pd{C(r,t)}{r}\right|_{r=R}. \]
Now we want an expression for $\left.\pd{C(r,t)}{r}\right|_{r=R}$. To get one, we linearize the gradient
\[ \left.\pd{C(r,t)}{r}\right|_{r=R} \approx \frac{\Delta C}{\Delta r}\]
We can calculate $\Delta C$ and $\Delta r$ by recognizing that the areas $A_1$ and $A_2$ in figure TODO must be the same due to conservation of the number of atoms. (TODO: Why??). The result is
\[ \frac{\Delta C}{\Delta r} = \frac{(C_s - C_e)^2}{2(C_p - C_s)R} \]

Integrating our previous result with respect to $t$, we finally get
\[ R^2(t) = R_0^2 + \frac{(C_s-C_e)^2}{(C_p-C_e)(C_p-C_s)}Dt \]
or
\[ R^2(t) = R_0^2 + K_1Dt \qquad \text{with} \qquad K_1 \equiv \frac{(C_s-C_e)^2}{(C_p-C_e)(C_p-C_s)} \approx \frac{C_s^2}{C_p^2} \]

\subsection{Ostwald ripening}
In this regime, growth goes as
\[ R^3(t) = R_0^3 + K_2 Dt \]

Ostwald ripening happens when the pressure $P$ is close to (but still above) $P_e$. If there are two particles of radii $R_1$ and $R_2$, with $R_1 < R_2$, then at some point the pressure will be below $P(R_1)$ but still higher than $P(R_2)$. At this point the first particle is no longer stable and will shrink while the second particle grows, effectively feeding off the first.

TODO 
\section{Nanostructures embedded in solid matrices}
\subsection{Ion implantation}
for synthesis and processing of metallic nanostructures
\subsection{Verification of nucleation and growth models}


\chapter{Optical properties}
We now turn to some of the optical properties of nanoparticles. They give rise to some strange phenomena. As an example, the Lycurgus cup (Roman from the fourth century AD) looks green is light is reflected of its surface and red if light is shone through the cup (TODO image). This is due to gold nanoparticles in the glass, which is otherwise similar to modern soda-lime glass.

\section{Lambert-Beer and optical quantities}
Prompted by the strange colour of the Lycurgus cup we would like to study what happens to light when it is shone through a sample containing nanoparticles.

We assume the light passing through the sample is attenuated according to the Lambert-Beer law. The intensity $I$ of the light coming out of the other side of the sample, is given by
\[ I(z) = I_0e^{-\gamma z}, \]
where $z$ is the distance traveled through the sample, $I_0$ is the initial intensity of the light entering the sample and $\gamma$ is a material constant called the \udef{extinction coefficient}.

There are two main mechanisms that cause the attenuation: scattering and absorption of photons. We also factor out the volumetric density $\rho$ in order to get the \udef{extinction cross-section} $\sigma_\text{ext}$:
\[ \gamma = \rho\sigma_\text{ext} = \rho (\sigma_\text{abs} + \sigma_\text{scat}) \]

We can then define the \udef{transmittance $T$}:
\[ T \equiv \frac{I(z)}{I_0} = e^{-\gamma z} \]
and the \udef{absorbance} (or \udef{optical density}) $A$:
\[ A \equiv \log_{10} \frac{1}{T} = \gamma z \log_{10}e =  z \rho \sigma_\text{ext} \log_{10}e \]

None of the equations so far show any frequency dependence. All of the quantities discussed so far can in principle depend on frequency. If we measure absorbance for example in function of the frequency of incoming light, we notice a peak at a particular frequency. The location and size of this peak seems to depend on the material and the size of the nanoparticles. It turns out this peak is due to surface plasmon resonance. We will discuss this phenomenon now.

\section{Localised surface plasmon resonance (LSPR)}

\section{Scattering by spheres (Mie theory)}
We want to know incoming light rays interact with small homogeneous spheres. We imagine a sphere suspended in a medium with dielectric constant $\epsilon_m$. The electromagnetic rays have their wave vector, of magnitude $k$, pointing along the $z$ axis. So, in complex notation, we can write
\[ \vbar{E} = \vbar{E}_0e^{ikz}e^{-i\omega t} \]

\subsection{Quasi-static regime.} If the electric field does not vary too fast (i.e. $R \ll \lambda$), we can use electrostatics and not take the variation of the electric field into account. As we will see, this leads us to a dipole approximation. Because we assuming the electric field to be basically static, we can write
\[ \vec{E} = E_0 \hat{z} \]
We now switch to using the electric potential $\Phi$ (we use $\Phi$ to denote teh electric field, because $V$ is used for volume in this section).

The setup (see figure TODO) is axially symmetric. This means we can use the coordinates $r$ and $\theta$.

We can write the potential outside the nanoparticle as
\[ \Phi_\text{out}(r,\theta) = \sum^\infty_{l=0} \left[B_l r^l + C_l r^{-(l+1)}\right]P_l(\cos \theta) \]
where $B_l$ and $C_l$ are constants we still need to determine and $P_l$ are the Legendre polynomials. In effect the terms with the constants $C_l$ are a multipole expansion of the field due to the polarisation of the nanoparticle. Of course there is also the potential due to incoming photons. This we write as an expansion in powers of $r$, which we can do because the Legendre polynomials form a complete set (TODO ??).

We also write a similar expansion in powers of $r$ for the potential inside the nanoparticle:
\[ \Phi_\text{in}(r,\theta) = \sum^\infty_{l=0} A_l r^l P_l(\cos\theta). \]

In order to determine the constants $A_l, B_l$ and $C_l$, we impose boundary conditions.
\begin{enumerate}
\item Very far away ($r\to \infty$) we set (TODO why??)
\[ \Phi_\text{out} = -E_0 z = -E_0 r \cos(\theta)\]
Knowing that $P_1(x) = x$, we can immediately see that this forces
\[ \begin{cases}
B_1 = -E_0 \\
B_l = 0 \qquad (l \neq 1)
\end{cases} \]
\item At the interface, the tangential component of the electric field $\vec{E}$ should be continuous (TODO better explanation):
\[ - \frac{1}{R}\left.\pd{\Phi_\text{in}}{\theta}\right|_{r=R} = - \frac{1}{R}\left.\pd{\Phi_\text{out}}{\theta}\right|_{r=R} \]
\item At the interface, the radial component of the displacement field $\vec{D}$ should be continuous:
\[ - \epsilon_0\epsilon\left.\pd{\Phi_\text{in}}{r}\right|_{r=R} = - \epsilon_0\epsilon_m\left.\pd{\Phi_\text{out}}{r}\right|_{r=R} \]
\end{enumerate}
For $l=1$ these last two conditions give
\[ \begin{cases}
A_1 R = - E_0 R + \frac{C_1}{R^2} \\
\epsilon A_1 = - \epsilon_m \left(E_0 + 2 \frac{C_1}{R^3}\right)
\end{cases} \quad \Rightarrow \quad \begin{cases}
A_1 = - \frac{3\epsilon_m}{\epsilon + 2\epsilon_m}E_0 \\
C_1 = \frac{\epsilon - \epsilon_m}{\epsilon + 2\epsilon_m}R^3 E_0
\end{cases}\] 
For $l\neq 1$ these conditions give
\[ \begin{cases}
A_lR^l = \frac{C_l}{R^{l+1}} \\
\epsilon l A_lR^{l-1} = - \epsilon_m(l+1)\frac{C_l}{R^{l+2}}
\end{cases} \quad \Rightarrow \quad \begin{cases}
A_l = 0 \\
C_l = 0
\end{cases} \]

We can write the potential outside the nanoparticle as
\[ \Phi_\text{out}(r,\theta) = - E_0r\cos\theta + \frac{1}{4\pi\epsilon_0\epsilon_m}\frac{\vec{p}\cdot\vec{r}}{r^3} \]
where $\vec{p}$ is the dipolar moment
\[ \vec{p} = 4\pi R^3 \epsilon_0\epsilon_m \frac{\epsilon - \epsilon_m}{\epsilon + 2\epsilon_m}\vec{E_0}. \]
We can write $\vec{p}$ in terms of the polerisability $\alpha$
\[ \vec{p} = \alpha \epsilon_0\epsilon_m \vec{E_0} \qquad\text{with}\qquad \alpha = 4\pi R^3 \frac{\epsilon - \epsilon_m}{\epsilon + 2\epsilon_m} \]

We can write the electric field inside the nanoparticle in terms of the \udef{local field enhancement} $f_e$
\[ \vec{E}_\text{in} = \frac{3\epsilon_m}{\epsilon + 2\epsilon_m}\vec{E_0} = f_e \vec{E_0} \]

Now allowing the incoming field $\vec{E_0}$ to vary in time again, we get
\[ \begin{cases}
\vec{E}_\text{in}(t) = \frac{3\epsilon_m}{\epsilon+2\epsilon_m}\vec{E_0}e^{-i\omega t} \\
\vec{E}_\text{out}(t) = \vec{E_0}e^{-i\omega t} + \frac{1}{4\pi \epsilon_0\epsilon_m}\frac{3(\vec{p}\cdot \hat{r})\hat{r}- \vec{p}}{r^3} \\
\vec{p}(t) = 4\pi R^3 \frac{\epsilon-\epsilon_m}{\epsilon + 2\epsilon_m}\epsilon_0\epsilon_m \vec{E_0}e^{-i\omega t}
\end{cases} \]

We can see that we have a resonance phenomenon when
\[ \epsilon + 2\epsilon_m = 0 \]
this is known as the \udef{Frölich condition}. For noble metals this condition is met within the visible range.

\subsection{Far-field properties.}

\section{Quantum confinement and photoluminescence in semiconductor quantum dots}
\chapter{Magnetic properties}
\section{Super-paramagnetism}
\chapter{Dynamics of electrons and photons}
\chapter{Confinement of electrons and photons in nanostructured or periodic materials}
\section{Photon confinement in photonic crystals}
\section{Electron confinement in metal nanoparticles}
\section{Electron confinement in semiconductor nanoparticles}
\chapter{Metamaterials and negative-refractive index materials}
\chapter{Seeing at nanoscale}
TODO: transfer to optics section

Optical microscopes use optical photons whose wavelength is between \SI{400}{nm} and \SI{800}{nm}. These photons can obviously not provide a resolution good enough to study nanomaterials.

Electron microscopes solve this problem, but are quite expensive.