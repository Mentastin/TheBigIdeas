\chapter{Introduction}



TODO
light and electromagnetic radiation

energy depends only on intensity??
notation: $\nu, \lambda$

electronics: LRC

Classical electrodynamics studies the effects of charges and currents within the context of classical Newtonian mechanics. Now if we know what force charged particles exert on each other, we can in principle solve the second order differential equation that is Newton's second law and we can find out all we want to know.

Say we have a charged particle we want to follow in particular (we call this a \udef{test charge}). We want to know what effect other charged particles (called \udef{source charges}) in the neighbourhood have on the test charge. Luckily we are aided by the \ueig{principle of superposition} baked into Newtonian mechanics: Say we have source charges $q_1, q_2, \ldots, q_n$ and we call the forces due to each source charge on its own $\vec{F}_1, \vec{F}_2, \ldots, \vec{F}_n$ respectively (i.e. if we only had source charge $q_1$, the test charge would feel a force $\vec{F}_1$). Now the principle of superposition asserts that, for the total force $\vec{F}$ on the test charge, the following holds:
\[ \vec{F} = \vec{F}_1 + \vec{F}_2 + \ldots \vec{F}_n \]
So we only need to find a formula for the force between two charged particles.

This is the good news. The bad news is that writing down such a formula in full generality is quite difficult. We will approach it in steps. This also allows us to explore some of the interesting features and useful mathematics of electromagnetic theory.

\section{Notation} Before we start, we will define some notation used in this chapter.
\begin{itemize}
\item In general we denote the test charge $Q$ and use lower case $q$ to reference source charges, indexed as necessary.
\item Because the distance between points is so important, we introduce the script $\rcurs$ to denote it. In particular if are considering two points with at locations $\vec{r}$ and $\vec{r'}$, we define
\[ \brcurs \equiv \vec{r} - \vec{r'} \]
We also introduce related notation for the unit vector in the direction of \brcurs:
\[ \hrcurs \equiv \frac{\vec{r} - \vec{r'}}{|\vec{r} - \vec{r'}|} \]
and magnitude $\rcurs$ of \brcurs.

It bears stressing that $\brcurs$ is not the same as the displacement vector $\vec{r}$. 
\end{itemize}

\chapter{Electrostatics}
Our first step will be \udef{electrostatics}, where the source charges are stationary (although the test charge may move).

TODO summary triangle

\section{Coulomb's law}

In this case the important formula is \udef{Coulomb's law}:
\[ \boxed{\vec{F} = \frac{1}{4\pi\epsilon_0}\frac{qQ}{\rcurs^2}\hrcurs} \]
We call $\epsilon_0$ the \udef{permitivity of free space}.
\[ \epsilon_0 = 8.85 \times 10^{12} \frac{\si{C^2}}{\si{N}\cdot \si{m^2}} \]
This is an experimental law. The factor $\frac{1}{4\pi\epsilon_0}$ may at this stage be viewed as just that, an experimental factor. We will later see why it is written in this form.

Notice that the force is positive (which means repulsive) if both charges are either positive or negative; the force is negative (meaning attractive) if one charge is positive and one is negative.

So we have been able to give an equation for the force in the electrostatic case. This contains all the physics. In the remainder of this section we will see some consequences of this law and elaborate some useful mathematical techniques.

\section{The electric field \textbf{E}}
\subsection{Discrete source charges}
Assume we have source charges $q_1, q_2, \ldots, q_n$ at distances $\rcurs_1, \rcurs_2, \ldots, \rcurs_n$ from the test charge $Q$. We can write
\begin{align*}
\vec{F} &= \vec{F_1} + \vec{F}_2 + \ldots  = \frac{1}{4\pi\epsilon_0} \left(\frac{q_1Q}{\rcurs_1^2}\hrcurs_1 + \frac{q_2Q}{\rcurs_2^2}\hrcurs_2 + \ldots\right) \\
& = \frac{Q}{4\pi \epsilon_0}\left(\frac{q_1\hrcurs_1}{\rcurs^2_1} + \frac{q_2\hrcurs_2}{\rcurs^2_2} + \ldots\right)
\end{align*}
Thus we can the force as
\[ \vec{F} = Q \vec{E}\]
Where we have introduced the electric field $E$. This means we can talk about electromagnetic effect without having to reference a particular test charge in a particular location, instead we define the electric field that has a (vectorial) value in every point in space. For discrete source charges, the electric field is
\[ \vec{E}(\vec{r}) \equiv \frac{1}{4\pi\epsilon_0}\sum^n_{i=1}\frac{q_i}{\rcurs_i^2}\hrcurs_i, \]
where the $\rcurs_i$'s are the distances between $\vec{r}$ and the source charge $q_i$.
\subsection{Continuous charge distributions}
The above formula can be readily extended to a continuous charge distribution:
\[ \vec{E}(\vec{r}) \equiv \frac{1}{4\pi\epsilon_0}\int\frac{1}{\rcurs^2}\hrcurs \diff{q}. \]
We can reformulate this as a slightly more tractable volume integral using the \udef{charge density} $\rho$:
\[ \vec{E}(\vec{r}) = \frac{1}{4\pi\epsilon_0}\int_\mathcal{V}\frac{\rho(\vec{r}')}{\rcurs^2}\hrcurs \diff{\tau'}, \]
where $\boxed{\diff{q} = \rho \diff{\tau'}}$.
Similarly we can calculate the electric field of a charged surface using the charge per unit area $\sigma$ (now $\diff{q} = \sigma \diff{a'}$):
\[ \vec{E}(\vec{r}) = \frac{1}{4\pi\epsilon_0}\int_\mathcal{S}\frac{\sigma(\vec{r}')}{\rcurs^2}\hrcurs \diff{a'}, \]
and of a linear charge distribution (with charge per unit length $\lambda$ and $\diff{q} = \lambda \diff{l'}$):
\[ \vec{E}(\vec{r}) = \frac{1}{4\pi\epsilon_0}\int_\mathcal{P}\frac{\lambda(\vec{r}')}{\rcurs^2}\hrcurs \diff{l'}. \]

These integrals can be very difficult to solve.

\section{Field lines and flux}
We can represent a vectorial field using \udef{field lines}. For a given vector field (that is smooth enough) a field line is a (directed) line whose tangent vectors in every point are the vectors of the vector field. [TODO image]

This way field lines convey information about the direction of the vectors in the vector field. Coincidentally for the electric field we can do even better and express the (relative) \emph{strength} of the field with field lines as well!

To see why, consider the field generated by a single source charge. We can now draw some field lines. At a certain distance $r$ from the source charge, the density of lines is the total number of lines divided by the area of the sphere: $\frac{n}{4\pi r^2} \propto 1/r^2$.

[TODO image; all images using point sources here!]

The strength of the electric field
\[ E = \frac{1}{4\pi \epsilon_0}\frac{q}{r^2} \]
is also proportional to $1/r^2$. This means we can use the density of the field lines to represent the relative strength of the electric field (i.e. the strength of the electric field up to a certain constant that depends on how many field lines we can be bothered to draw, among other things).

We can also use field line diagrams to represent more complicated fields [TODO images of examples]. Due to the fact that the electric field and the field line density behave similarly under addition, we can keep using field line density to represent the relative strength of the electric field.

Some important remarks about field lines:
\begin{itemize}
\item Field lines start in positive charges and end in negative ones. The direction is always positive to negative.
\item Field lines start and end either at charges of at infinity. They can never start or end in the middle of space.
\item Field lines never cross. If they did cross, it would mean that they had different tangent vectors at the point of crossing which is definitionally impossible.
\end{itemize}

It is importance at this stage to give a more rigourous way to conceive of the density of field lines. To do that we define the \udef{flux} of $\vec{E}$ through a surface $\mathcal{S}$ as
\[ \Phi_E \equiv \int_\mathcal{S} \vec{E}\cdot \diff{\vec{a}} \]
which gives a measure of the ``number of field lines'' going through the surface $\mathcal{S}$. TODO: why?

\section{Gauss's law}
Say we now take the flux through a closed surface. This has the following properties:
\begin{enumerate}
\item Any charges outside the closed surface do not contribute to the flux, because any field lines originating from them must both enter and exit the surface as field lines cannot stop in the middle of space. This means these field lines are both added to and subtracted from the flux.
\item  If there is a charge enclosed in the surface, its location within the surface does not impact the total flux. For any field line that starts at the charge, there are two options:
\begin{enumerate}
\item It ends at another charge within the surface, in which case it either does not exit the surface or exits an reenters. Either way it does not contribute to the total flux.
\item It ends at a charge outside the surface or at infinity, in which case it contributes to the total flux by exiting, but it can only exit once (without reentering) and is thus counted once, regardless of its location relative to the surface.
\end{enumerate}
\end{enumerate} Thus, using the superposition principle for the electric field, the flux depends only on the total enclosed charge (times a constant)! This is known as \udef{Gauss's law}.
\[  \oint_\mathcal{S} \vec{E}\cdot \diff{\vec{a}} = \frac{Q_\text{enc}}{\epsilon_0} \]
We can easily see that the constant must be $\epsilon_0$ by explicitly working out the flux for a simple setup, like the flux from a single charge through a concentric sphere. In fact this is the reason the constant in Coulomb's law was chosen the way it was.

Using the divergence theorem from vector calculus, we can obtain a differential form of Gauss's law:
\[ \div \vec{E} = \frac{\rho}{\epsilon_0} \]

\subsection{Applying Gauss's law}
Gauss's law is incredibly powerful and can make calculating the electric field very easy, if the system is symmetric enough. Otherwise it is not useful. There are three kinds of symmetry that make Gauss's law useful:
\begin{enumerate}
\item Spherical symmetry: make the Gaussian surface a concentric sphere.
\item Cylindrical symmetry: make the Gaussian surface a coaxial cylinder.
\item Planar symmetry: make the Gaussian surface a ``pillbox'' that intersects the surface.
\end{enumerate}
We can also calculate the electric field of more complex arrangements by decomposing them into elements with such symmetry and using the superposition principle.

\subsection{The curl of the electric field}
We can view the electric field as being composed of the fields of each charge in space. Each of those fields radiates straight out and is thus irrotational. This means that the total electric field is also irrotational:
\begin{equation}
\nabla \times \vec{E} = 0
\label{irrotationalE}
\end{equation}
so the path integral around any closed path is zero
\begin{equation}
\oint \vec{E}\cdot \diff{\vec{l}} = 0. \label{loopE}
\end{equation}

\section{Electric potential}
We can define the scalar field called the \udef{electric potential} $V$:
\[ V(\vec{r}) \equiv - \int_\mathcal{O}^{\vec{r}}\vec{E}\cdot \diff{\vec{l}}. \]
Using the fact that the curl of $\vec{E}$ is zero, we can see that
\[ \vec{E} = - \nabla V. \]
Some remarks:
\begin{enumerate}
\item It has been claimed the the name potential is a misnomer, because it's too much like potential energy. This is possible. It is important to remember that the potential is not potential energy (it will turn out it is more akin to potential energy per unit charge).
\item A surface in space over which the potential is constant is called and \udef{equipotential (surface)}.
\item The advantage of the potential is that it has some of the characteristics of the electric field baked in. The electric field is a \emph{vector} field which has the property of being irrotational. The potential is a \emph{scalar} field which has the irrotationality of the electric field built in.
\item The potential is defined with respect to a reference point $\mathcal{O}$. This point is arbitrary, if we chose a different one it would just add a constant to the potential:
\[ V'(\vec{r}) = - \int^{\vec{r}}_{\mathcal{O}'} \vec{E}\cdot \diff{\vec{l}} = - \int^{\mathcal{O}}_{\mathcal{O}'} \vec{E}\cdot \diff{\vec{l}} - \int^{\vec{r}}_{\mathcal{O}} \vec{E}\cdot \diff{\vec{l}} = K + V(\vec{r}),\]
where $V'(\vec{r})$ is the potential with respect to the reference point $\mathcal{O}'$. Because adding a constant does not change the gradient ($\nabla V' = \nabla V$), both are equivalent. In other words only potential \emph{differences} are physically relevant.

It is often natural to put the reference point at infinity (i.e. far away from the charges).
\item Positive test charges want to move to areas of low potential (meaning the source charges generate a force pulling them towards areas of low potential). Negative test charges want to move in the opposite direction.
\item The potential obeys the superposition principle.
\[V = V_1 + V_2 + \ldots\]
\item \textbf{Units}: Force is measured in newtons and charge in coulombs, so the electric field is measured in newtons per coulomb. The potential is then measured in newton-meters per coulomb, or joules per coulomb, also known as \udef{volts}.
\end{enumerate}

\subsection{Relationship with the charge density $\rho$}
Combining $\vec{E} = -\nabla V$ and $\nabla \cdot \vec{E} = \rho/\epsilon_0$, we get
\[ \nabla^2V = - \frac{\rho}{\epsilon_0}. \]
This is known as \udef{Poisson's equation}. In regions where there is no charge, this is known a \udef{Laplace's equation}
\[ \nabla^2 V = 0. \]
In order to work out the potential from the charge distribution, we can use the following formula which is not too difficult to derive:
\[ V(\vec{r}) = \frac{1}{4\pi \epsilon_0}\int \frac{\rho(\vec{r}')}{\rcurs}\diff{\tau'} \]
This is quite a bit easier to work with than the equivalent formula for the electric field because it is scalar.

\section{Charged surface}
We quickly make some observations about passing through a charged surface with surface charge density $\sigma$.

TODO image

Using Gauss's law, we obtain
\[ E_\text{above}^\perp - E^\perp_\text{below} = \frac{1}{\epsilon_0}\sigma. \]
Using $\oint \vec{E}\cdot \diff{\vec{l}} = 0$, we also obtain
\[ E^\parallel_\text{above} - E^\parallel_\text{below} = 0. \]
Combining them we get
\[ \vec{E}_\text{above}-\vec{E}_\text{below} = \frac{\sigma}{\epsilon_0}\hat{n} \]
The potential, however, is continuous:
\[ V_\text{above} = V_\text{below} \]

\section{Work and energy}
We can calculate the work necessary to move a charged particle from a point $\vec{a}$ 
to a point $\vec{b}$.
\[ W = \int^{\vec{b}}_{\vec{a}}\vec{F}\cdot \diff{\vec{l}} = - Q\int^{\vec{b}}_{\vec{a}}\vec{E}\cdot \diff{\vec{l}} = Q[V(\vec{b}) - V(\vec{a})].  \]
In other words
\[ V(\vec{b}) - V(\vec{a}) = \frac{W}{Q}. \]
So in this sense potential is just potential energy per unit charge.
\subsection{Energy of a point charge distribution}
TODO: superposition for potential energy: sum over all pairs of point charges.

In this section we are interested in knowing how much work it would take to assemble a collection of point charges. We can imagine bringing in the charges in one by one from infinity, which we choose as our reference point. This means that to bring a charge in from infinity we need to do the work
\[ W = q_i V_{<i}(\vec{r}_i). \]
Where $\vec{r}_i$ is the place we are taking charge $q_i$ to, and $V_{<i}$ is the potential generated by all the other charges that are already assembled (i.e. charges labeled up to $i$).

The first takes no work, because it is not fighting against any field. 
For the second charge we need to do the work
\[ W = q_2 V_{<2}(\vec{r}_2) = q_2 \frac{1}{4\pi\epsilon_0}\left(\frac{q_1}{\rcurs_{12}}\right) \]
For the third charge, we have 2 such terms. Due to the superposition principle, we know that there is one for the first charge and one for the second.

Continuing that chain of logic, we can see that the total work is
\[ W = \frac{1}{4\pi\epsilon_0}\sum^n_{i = 1}\sum^n_{\substack{j = 1 \\ j<i}}\frac{q_iq_j}{\rcurs_{ij}}. \]
If we change the $j<i$ to $j \neq i$, two things happen:
\begin{enumerate}
\item Each pair is counted twice, so we need to add a factor $\frac{1}{2}$.
\item The second summation effectively becomes an expression for the potential when all the the charges are already assembled (dropping the $j<i$ means we count all of the charges is the final configuration). We can then just call this potential $V_i(\vec{r}_i)$, meaning the potential generated by all charges other than $q_i$.
\end{enumerate}
So the expression for the work becomes
\begin{equation}
W = \frac{1}{2}\sum^n_{i=1}q_i V_i(\vec{r}_i)
\label{pointWork}
\end{equation}

\subsection{Energy of a continuous charge distribution}
By taking the individual changes to be infinitesimal, we can transition from a discrete sum, to an integral:
\[ W = \frac{1}{2}\int \rho V \diff{\tau} \]
for a volume charge density $\rho$. We can write similar expressions for line and surface charge densities. Notice we can write $V$ and not $V_i$ because the difference between the two is infinitesimal.

Now using $\rho = \epsilon_0 \nabla \cdot \vec{E}$ and integrating by parts, we get
\[ W = \frac{\epsilon_0}{2}\left(\int_\mathcal{V}E^2 \diff{\tau} + \oint_\mathcal{S} V\vec{E}\cdot \diff{\vec{a}}\right). \]
Just like the previous one, this expression is valid if we integrate over any volume that contains all the charges. In particular we can integrate over all of space, which makes the surface integral go to zero.
\begin{equation}
W = \frac{\epsilon_0}{2}\int_{\text{all space}}E^2 \diff{\tau}
\label{workE}
\end{equation}
Again there are some remarks to be made:
\begin{enumerate}
\item In deriving equation \eqref{workE} for $W$, we explicitly made use of the fact that we are working with a charge distribution when we set $V_i = V$. The final equation does not reference the charge distribution however and so we may wonder whether this equation also works for point charges. Unfortunately applying this equation to a point charge gives us
\[ W = \frac{\epsilon_0}{2(4\pi\epsilon_0)^2}\int \left(\frac{q^2}{r^4}\right)(r^2\sin\theta \diff{r}\diff{\theta}\diff{\phi}) = \frac{q^2}{8\pi\epsilon_0}\int^\infty_0 \frac{1}{r^2}\diff{r} = \infty. \]
In fact this is because the equations \eqref{pointWork} and \eqref{workE} refer to slightly different energies: for the latter we move infinitesimal charges in from infinity, for the former we have ready-made point charges we can move in from infinity. What we see here is that bringing together enough infinitesimal charge close enough together to create a finite point charge, requires an infinite amount of work. This is a singularity that is also present in quantum electrodynamics and is still an open problem.
\item In the context of radiation theory it is useful and in the context of general relativity it is essential to view the equation \eqref{workE} as the more fundamental one. In other words we view the energy as being stored in the field, not the charge. In electrostatics it does not make much difference.
\item The work depends quadratically on the electric field and thus the superposition principle does not apply.  
\end{enumerate}

\section{Multipole expansion}
Say we have a localised charge distribution that we are viewing from far away, i.e. the distance $r$ between us and the charge distribution is large. Being far away we can approximate the potential as the one generated by a point with the same charge as the net charge of the distribution: 
\[ V_\text{mon}(\vec{r}) = \frac{1}{4\pi \epsilon_0}\frac{Q}{r}. \]
The subscript mon refers to the fact that we are approximating the potential with that of a point charge, called a \udef{monopole} in this context.

This means that the potential will go to zero at least as $1/r$, and maybe quicker if the net charge is zero. This motivates us to expand the potential in powers of $1/r$.

There is a nice correspondence between the powers of $1/r$ and point charge distributions. The $1/r$ term is called the monopole term. If we take two monopoles of opposite charge and put them at a distance $d$ from each other, we have what is called a (physical) \udef{dipole}. (TODO image with $d, \theta, q, \vec{r}$ and $\vec{p}$). The potential generated by this setup can be approximated at large distances by
\[ V(\vec{r}) \approx \frac{1}{4\pi\epsilon_0}\frac{qd\cos\theta}{r^2}\]
This expression depends on the angle $\theta$, but for a fixed $\theta$ the potential drops off as $1/r^2$. We call the $1/r^2$ term of our expansion the dipole term.

If we take a second identical dipole, flip it and put it next to the first one at a distance $d$, we have what is known as a \udef{quadrupole}. Now the potential drops off as $1/r^3$ and consequently we call the $1/r^3$ term the quadrupole term. The $1/r^4$ term corresponds to an \udef{octopole} (TODO image of multipoles).

Now let us explicitly do the expansion. We start with the formula to calculate the potential from an arbitrary charge distribution
\[ V(\vec{r}) = \frac{1}{4\pi \epsilon_0} \int \frac{1}{\rcurs}\rho(\vec{r'})\diff{\tau'} \] 
Refering to figure TODO and using the law of cosines, we can write
\[\rcurs = r\sqrt{1+\epsilon} \qquad \text{with} \qquad \epsilon \equiv \left(\frac{r'}{r}\right)\left(\frac{r'}{r}- 2\cos\theta '\right).\]
We can now use a binomial expansion:
\begin{align*}
\frac{1}{\rcurs} &= \frac{1}{r}(1+\epsilon)^{-1/2} = \frac{1}{r}\left(1- \frac{1}{2}\epsilon + \frac{3}{8}\epsilon^2 - \frac{5}{16}\epsilon^3\right) \\
&= \frac{1}{r} \left[1+ \left(\frac{r'}{r}\right)\cos\theta' + \left(\frac{r'}{r}\right)^2(3\cos^2\theta'-1)/2 + \left(\frac{r'}{r}\right)^3(5\cos^3\theta'-3\cos\theta')/2 + \ldots\right] \\
&= \frac{1}{r}\sum^\infty_{n=0}\left(\frac{r'}{r}\right)^nP_n(\cos\theta')
\end{align*}
where $P_n$ are the Legendre polynomials.

Plugging this back in the expression for $V$, and noting that $r$ stays constant during integration, we get
\[ V(\vec{r}) = \frac{1}{4\pi\epsilon_0}\sum^{\infty}_{n=0}\frac{1}{r^{n+1}}\int (r')^nP_n(\cos\theta')\rho(\vec{r'})\diff{\tau'}. \]

It must be remarked that the multipole expansion depends on the chosen coordinate system.

\subsection{The monopole and dipole terms}
Usually the multipole expansion is dominated by the first term
\[ V_\text{mon}(\vec{r}) = \frac{1}{4\pi\epsilon_0}\frac{1}{r}\int \rho(\vec{r'})\diff{\tau'} = \frac{1}{4\pi\epsilon_0}\frac{Q}{r} \]
This is the potential of a point charge, which we expected given the discussion above. Thus for a point charge centered at the origin, the monopole term is exact and all higher multipole terms vanish. If the point charge is somewhere else, the monopole term stays the same, but we also have higher order corrections.

The dipole term is
\[ V_\text{dip}(\vec{r}) = \frac{1}{4\pi\epsilon_0}\frac{1}{r^2}\int r'\cos\theta'\rho(\vec{r'})\diff{\tau'}. \]
Using the fact that $r'\cos\theta'= \hat{r}\cdot \vec{r'}$, we can write
\begin{align*}
V_\text{dip}(\vec{r}) &= \frac{1}{4\pi\epsilon_0}\frac{1}{r^2}\hat{r}\cdot\int \vec{r'}\rho(\vec{r'})\diff{\tau'}\\
&= \frac{1}{4\pi\epsilon_0}\frac{\hat{r}\cdot \vec{p}}{r^2}\qquad \text{with}\qquad \vec{p} \equiv \int \vec{r'}\rho(\vec{r'})\diff{\tau'}
\end{align*}
where we have introduced the \udef{dipole moment $\vec{p}$}. We can remark on the following properties:
\begin{enumerate}
\item The dipole moment is a vector and can be manipulated accordingly. For example, if we add the dipole moments of the two dipoles that make up a quadrupole, we see that the dipole moment of the quadrupole vanishes.
\item For point charge distributions the integral becomes a sum in the usual way
\[ \vec{p} = \sum^n_{i=0} q_i \vec{r'}_i. \]
If we apply this to the physical dipole, we get
\[ \vec{p} = q \vec{d}. \]
Using this the dipole term calculated here is the same as the approximate field of the physical dipole quoted before.
\item For a physical dipole the dipole terms gives only an approximation of the potential. There are higher multipole terms. If we however shrink the distance $d$ keeping the dipole moment $p = qd$ constant, we get what is called a \udef{pure dipole} in the limit. For this distribution, all higher multipole terms vanish.
\item The dipole moment is independent of the chosen coordinate system, if and only if the monopole moment (i.e. the net charge) is zero.
\end{enumerate}

\subsubsection{A more geometric derivation of the dipole term}
Consider figure (TODO image, add $\vec{d}$ between $A$ and $B$) of a physical dipole. Let $A$ be a point far away from the dipole with displacement $\vec{r}_+$ from the positive charge and $\vec{r}_-$ from the negative charge. Let $B$ be the point such that $B = A + \vec{d}$. Let $V_+$ be the potential due the positive charge and $V_-$ due to the negative charge.

Now we note
\[ V(A) = V_+(A) + V_-(A) = \frac{q}{r_+} - \frac{q}{r_-} \qquad \text{and} \qquad V_-(A) = -\frac{q}{r_-} = -V_+(B) \]
From which we get
\begin{align*}
V(A) = V_+(A)-V_+(B) \approx - (\nabla V_+)\cdot \vec{d} = - \left(\nabla\frac{1}{r}\right)\cdot \vec{p} = \frac{\vec{p}\cdot\vec{r}}{r^3}
\end{align*}
which is exactly the dipole term we got before.

\subsection{The electric field of a dipole}
We can simply calculate
\begin{align*}
\vec{E}_\text{dip}(r,\theta) &= -\nabla V_\text{dip} \\
&= \frac{p}{4\pi\epsilon_0 r^3}(2\cos\theta \hat{r} + \sin\theta \hat{\theta}).
\end{align*}
We know that to monopole electric field falls off as $1/r^2$. The dipole electric field we now see falls off as $1/r^3$. The quadrupole field will go like $1/r^4$, the octopole as $1/r^5$ etc.

\section{Miscellaneous results in electrostatics}
In this section a collection of useful and / or interesting results from electrostatics will be presented, typically pertaining to the electric field or potential for a certain configuration.
\subsection{Average electric field inside a sphere.}
\label{subsec:Esphere}
We can split the average field into two components: the average field due to charges inside the sphere and due to charges outside the sphere.
\begin{itemize}
\item We start with the field \textbf{due to charges inside the sphere}. If there was only a single charge $q$ at point $\vec{r}$ inside the sphere, the average field would be
\[ \expval{\vec{E}} = \frac{1}{\mathcal{V}_\text{sphere}}\int \vec{E}(\vec{r'})\diff{\tau'} = \frac{1}{\frac{4}{3}\pi R^3}\frac{1}{4\pi \epsilon_0}\int \frac{q}{\rcurs^2}\hrcurs \diff{\tau'} \]
where $R$ is the radius of the sphere and $\mathcal{V}_\text{sphere}$ is the volume. The dipole moment of this point charge is
\[ \vec{p} = q \vec{r}. \]

The integral for the average field is the same as the integral we would obtain if we were calculating the field in $\vec{r}$ due to a uniform charge distribution $\rho = -q/(\frac{3}{4}\pi R^3)$ in the entire sphere. To solve this second scenario we can make use of Gauss' law because we know that the direction of $\vec{E}$ is $\hat{r}$.
\begin{alignat*}{2}
\oint \vec{E}\cdot \diff{\vec{a}} & = \int E \diff{a} && \\
& = E 4\pi r^2 && = \frac{Q_\text{enc}}{\epsilon_0} \\
& && = \rho  \frac{4}{3}\pi r^3
\end{alignat*}
So we get
\[ \vec{E} = \frac{\rho}{3}\vec{r} = - \frac{\vec{p}}{4\pi \epsilon_0 R^3} \]
We can split any arbitrary charge configuration up into point charges and calculate the average field for each one. Using the superposition principle, the total average field is just the sum of all those contributions, and the total dipole moment is the sum of all the individual dipole moments. Thus for an arbitrary charge configuration inside a sphere we have
\[ \expval{\vec{E}} = - \frac{1}{4\pi \epsilon_0}\frac{\vec{p}}{R^3}. \]
\item Now for the average field \textbf{due to external charges}. We follow the same procedure, except now the charge $q$ is at a point $\vec{r}$ outside the sphere and the charge enclosed in the gaussian surface is simply $-q$.

Therefore the average field due to a single point charge is
\[ - \frac{q}{4\pi \epsilon_0 r^2}\hat{r}. \]
This is the same as the field in the centre of the sphere. Again we can use the superposition principle to generalise. In general \textit{the average field over the sphere due to all charges outside is the same as the field they produce in the centre}.

\end{itemize}
\subsection{Earnshaw's theorem}
TODO

\chapter{Electric fields in matter}
In this section we will be studying electric fields in matter of which there are many forms. Most matter however can be approximated as either a conductor or an insulator.

\begin{itemize}
\item In a \udef{conductor} the electrons are free to move wherever they like within the conductor. In an ideal conductor there is an unlimited supply of charged particles that can move.
\item In an \udef{insulator}, also called a \udef{dielectric}, the charges are not free to move wherever they like.

A dielectric consists of a collection of electrically neutral units, which can be atoms or molecules. These units however can contain charges that can move relative to each other, \textit{within the confines of the unit}.
\end{itemize}

\section{Conductors}
We can deduce the following properties of (ideal) conductors:
\begin{enumerate}
\item \textbf{$\vec{E}=0$ inside a conductor}. If this were not the case, then the electrons would feel a force and move to counteract the field. We cannot use electrostatics to describe this process, but we do know that once it settles down into an equilibrium condition, the field inside the conductor must be zero.
\item \textbf{$\rho=0$ inside a conductor}. This follows from Gauss' law: $\vec{\nabla}\cdot \vec{E} = \rho / \epsilon_0$. This is not, however, the same in one or two dimensions.
\item \textbf{Any net charge resides on the surface.} If the field outside the conductor is not zero, the charges must gather at the edges to counteract the effect of the external field. TODO Image. 
\item \textbf{A conductor is an equipotential}:
\[ V(\vec{b}) - V(\vec{a}) = -\int^{\vec{b}}_{\vec{a}} \vec{E}\cdot \diff{\vec{l}} = 0 \]
for any two points $\vec{a}$ and $\vec{b}$ inside the conductor.
\item \textbf{Just outside a conductor, $\vec{E}$ is perpendicular to the surface}. Otherwise the tangential component would cause charges to flow along the surface, meaning we weren't in an electrostatic equilibrium.
\end{enumerate}
\subsection{Faraday cages}
Assume we have a conductor with a cavity that does not contain any charges. The field inside the cavity must be zero: every field line must start and end at the cavity wall, as there are no charges from which the field lines can originate. If we follow a loop as in figure TODO, we see that the path integral $\oint \vec{E}\cdot \diff{\vec{l}}$ is a sum of a positive part from inside the cavity and a vanishing part from inside the conductor. Thus the loop integral would be positive, but from equation \eqref{loopE}, we know that it must be zero. Thus the field inside the cavity must be zero.

We derived this fact without knowing what charges there are outside the conductor. Consequently, no matter the electrical charges outside the conductor, inside the cavity you are shielded. In fact in practice the enclosing conductor does not even have to be solid. A mesh conductor also works. This is the principle behind a \udef{Faraday cage}.

\subsection{Capacitors}
Suppose we have two (randomly shaped) conductors and we put a charge $+Q$ on one and a charge $-Q$ on the other. The potential is constant over a conductor (it's an equipotential), so we can speak unambiguously of the potential difference $V$ between them.

Because $\vec{E}$ is proportional to $\rho$ (which is proportional to $Q$) and $V$ is proportional to $\vec{E}$, $V$ is proportional to $Q$. The proportionality constant is called the \udef{capacitance} $C$.
\[ C \equiv \frac{Q}{V} \]
The capacitance is a purely geometric quantity, dependent on the sizes, shapes and separation of the two conductors. In SI units, it is measured in \udef{farads (F)}, which are coulomb-per-volts. This unit is quite large. More practical units are micro- and picofarads.

We give expressions for the capacitance of two common geometries.
\begin{enumerate}
\item A parallel-plate capacitor consists of two metal plates of area $A$, held a distance $d$ apart. The capacitance is given by
\[ C = \frac{A\epsilon_0}{d} \]
\item The capacitance of two concentric spherical metal shells with radii $a$ and $b$ is given by
\[ C = 4\pi \epsilon \frac{ab}{(b-a)}. \]
\end{enumerate}

Finally we also would like to have an idea of the energy stored in a capacitor. We charge the capacitor by moving electrons from the positive conductor to the negative conductor.
Summing over all these infinitesimal bits of charge we are moving, we get
\[ W = \int_0^QV(q) \diff{q} = \int_0^Q \left(\frac{q}{C}\right)\diff{q} = \frac{1}{2}\frac{Q^2}{C} = \frac{1}{2}CV^2. \]
In the last expression $V$ is the final potential of the capacitor.

\section{Dielectrics}
Even though charges cannot flow though dielectrics, the constituent chunks can be given a dipole moment by putting them in an electric field.
\begin{enumerate}
\item The external field can induce a dipole by moving the positive charges to one side and the negative charges to the other side of the molecule or atom.
\item The external electric field can also rotate the dipoles so they align.
\end{enumerate}

\subsection{Induced dipoles}
\subsubsection{Atoms.}
If we take a neutral atom and place it in an electric field, the charges inside it will move: the electrons will tend to be more on one side and the positive nucleus will tend to favour the other one. In effect we are \textit{polerising} the atom and giving it a dipole moment \undline{in the same direction as $\vec{E}$}. Typically the induced dipole moment depends linearly on the external field
\[ \vec{p} = \alpha \vec{E}. \]
We call $\alpha$ the \udef{atomic polarizability}. We can think of this as being a bit like Hooke's law. It is an approximation which breaks down if the electric field is strong.

\subsubsection{Molecules.}
Atoms are spherically symmetric. And so the induced dipole will be the same, whatever the direction of the electric field. The same cannot be said for molecules. So we cannot take $\alpha$ to be a scalar. Instead it is a map that maps the electric field vector onto the polarisation of the molecule. We can still assume the map to be linear though. In other words, we have a (1,1)-tensor: $\alpha$ is the \udef{polarisability tensor}. (TODO use chosen tensor notation)

\subsection{Alignment of polar molecules}
Atoms, being spherically symmetric do not have an inherent dipole moment. Any induced dipole is naturally aligned with the electric field. Molecules may have an inherent dipole moment, in which case they will feel a torque unless they align themselves with the field. Such molecules are called \udef{polar molecules}.

We can calculate the torque about the centre of the molecule, assuming $\vec{E}$ is more or less uniform over the length of the molecule (TODO image clarifying symbols):
\begin{align*}
\vec{M} &= [(\vec{d}/2) \times (q \vec{E}) + (- \vec{d}/2) \times (-q \vec{E})] \\
&= q \vec{d}\times \vec{E} \\
&= \vec{p}\times \vec{E}
\end{align*}
If the electric field is not uniform, there is a net force on the molecule, given by
\[ \vec{F} = (\vec{p}\cdot \vnabla) \vec{E}. \]

\subsection{The field inside a dielectric}
If a dielectric is placed in an electric field, the effects described above cause its atoms and / or molecules to polarise. This is a messy and complicated process that is counteracted by thermal effects. The net result is captured in a quantity called the \udef{polarisation} $P$ which is the dipole moment per unit volume.

\subsubsection{Microscopic and macroscopic field}
At first glance, the introduction of the polarisation may seem quite natural, but on closer inspection there are some subtleties.

The dipoles induced in the material are physical dipoles, not pure ones. This means they have higher multipole moments. If we are considering the field far away from the material, this is not a problem. The higher multipole terms become vanishingly small in comparison. Unfortunately we want to describe the electric field inside the dielectric as well and there these higher multipole moments most definitely do matter.

Thinking about it, we may realise that the electric field inside a dielectric must be extremely complicated. Close to an electron or a nucleus the field may be extremely strong, but just a short distance away the field may be completely different. This \textbf{microscopic} field fluctuates a lot and is incredibly difficult to calculate.
 
We want to average this field over volumes large enough that the fluctuations get smoothed out, but small enough that we do not lose all spatial variation and that the average field does not depend on the size of the volumes we average over. The field averaged in this way is called the \textbf{macroscopic field}.
 
Picture TODO qualitatively shows the average electric field over a sphere in function of the radius of the sphere. When the radius is very small the average field fluctuates a lot, because the inclusion of an extra electron or positive nucleus makes a lot of difference. At slightly larger radii the average field does not depend on the radius: if we increase the the radius a bit the added volume is close enough to all the other parts of the sphere that the external field will induce an average field that is approximately the same as the average field in the rest of the sphere. At even larger radii the average field again depends on the radius. This time it is because if we increase the radius slightly, the volume added to the sphere is so far away from the centre that the external field may be different and it may have a significantly different average field than other parts of the sphere.
 
In fact we do exactly the same averaging when we talk about the density of a material. This process will be brought up in that context when we will discuss fluid mechanics (TODO: check location).
 
Now we need to show that this macroscopic field is the field we obtain from the polarisation $\vec{P}$.

Suppose we want to calculate the macroscopic field at a point $\vec{r}$ inside the dielectric. We take a sphere of the right size, as discussed above, about $\vec{r}$. The average field inside the sphere then consists of two parts: then average field due to all the charges outside plus the average field due to all the charges inside the sphere:
\[ \vec{E} = \vec{E}_\text{out} + \vec{E}_\text{in} \]
Referring to section \ref{subsec:Esphere}, we see that
\begin{itemize}
\item The average field due to the charges outside the sphere is equal to the field strength in the centre of the sphere. These charges are far enough away that we can safely use the dipole approximation:
\begin{equation}
V_\text{out} = \frac{1}{4\pi\epsilon_0}\int_\text{outside} \frac{\hrcurs \cdot \vec{P}(\vec{r}')}{\rcurs^2}\diff{\tau'}. \label{eq:Vout}
\end{equation}

\item The average field due to the charges inside the sphere is, regardless of their distribution, the same as that of a uniformly polarised sphere:
\[ \vec{E}_\text{in} = - \frac{1}{4\pi\epsilon_0}\frac{\vec{p}}{R^3} \]
where $\vec{p}$ is the total dipole moment of the sphere: $\vec{p} = (\frac{4}{3}\pi R^3)\vec{P}$.

This is exactly the term that would be added if the integral in equation \eqref{eq:Vout} were extended over all space.
\end{itemize}
In conclusion the macroscopic field is given by
\[ V(\vec{r}) = \frac{1}{4\pi \epsilon_0}\int_\mathcal{V} \frac{\hrcurs \cdot \vec{P}(\vec{r}')}{\rcurs^2}\diff{\tau'} \]
where the integral runs over the entire volume of the dielectric. This is exactly the expression we would get if we assumed the electric field inside a dielectric was due to each volume element $\diff{\tau}$ having a dipole moment $\vec{P}\diff{\tau}$.
 
\subsubsection{Bound charges}
With a bit of manipulation, we can write the expression for the potential of the macroscopic field as the sum of a potential due to a surface charge and a volume charge.

We start by observing that
\[ \frac{\hrcurs}{\rcurs^2} = \grad'\left(\frac{1}{\rcurs}\right) \]
where the prime means that the differentiation is with respect to the source coordinates ($\vec{r}'$). Then the manipulation is as follows:
\begin{align*}
V &= \frac{1}{4\pi \epsilon_0}\int_\mathcal{V} \frac{\hrcurs \cdot \vec{P}(\vec{r}')}{\rcurs^2}\diff{\tau'} \\
&= \frac{1}{4\pi \epsilon_0}\int_\mathcal{V} \vec{P}\cdot \vnabla'\left(\frac{1}{\rcurs}\right)\diff{\tau'}  \\
&= \frac{1}{4\pi \epsilon_0} \left[\int_\mathcal{V}\vnabla'\cdot \left(\frac{\vec{P}}{\rcurs}\right)\diff{\tau'} - \int_\mathcal{V} \frac{1}{\rcurs}\left(\vnabla'\cdot \vec{P}\right)\diff{\tau'}\right] \\
&= \frac{1}{4\pi \epsilon_0}\oint_\mathcal{S} \frac{1}{\rcurs}\vec{P}\cdot \diff{\vec{a}'} - \frac{1}{4\pi \epsilon_0}\int_\mathcal{V} \frac{1}{\rcurs}\left(\vnabla'\cdot \vec{P}\right)\diff{\tau'}
\end{align*} where we have first integrated by parts and then used the divergence theorem.

The first term looks like the potential of a surface charge
\[ \sigma_b \equiv \vec{P}\cdot \hat{n} \]
(where $\hat{n}$ is the unit normal vector) and the second term looks like the potential of a volume charge
\[ \rho_b \equiv - \div \vec{P}. \]
We can treat the field caused by the polarisation of matter as being generated by the \udef{bound charges} $\rho_b$ and $\sigma_b$.

\subsubsection{Physical interpretation}
The above manipulations are quite abstract, but there is also a much more physical (if less rigorous) way to derive the expressions for the bound charges.

First we calculate the surface charge $\sigma_b$ by considering a tube through the dielectric made up of strings of back-to-back infinitesimal dipoles, as illustrated in figure TODO 4.11. The net result is the accumulation of charge at the ends of the strings. For a small section of tube (in which $\vec{P}$ constant) with faces perpendicular to the strings of dipoles, the dipole moment can be written as both $P\cdot A\cdot d$, with $d$ the length of the tube section, and $q\cdot d$ with $q$ the charge accumulation at one end. Equating those two expressions, we get $q = P A$, so
\[ \sigma_b = \frac{q}{A} = P. \]
Allowing the face to be oblique, we get
\[ \sigma_b = P\cos\theta = \vec{P}\cdot \hat{n}. \]

Considering tube element at the surface, we see that this must also be an expression for the surface charge of the dielectric, and indeed it corresponds to the expression we have already derived.

If the polarisation is nonuniform there must also be accumulations of bound charges within the dielectric, because adjacent tube elements will have slightly different surface charges that do not quite cancel.

The net bound charge in a given volume $\int \rho_b \diff{\tau}$ is equal and opposite to the amount that has been pushed out through the surface, which we have already reasoned to be $\vec{P}\cdot \hat{n}$ per unit area, so
\[ \int_\mathcal{V} \rho_b \diff{\tau} = - \oint_\mathcal{S} \vec{P}\cdot \diff{\vec{a}} = - \int_\mathcal{V} (\div \vec{P})\diff{\tau}. \]
Since this is true for any volume, we have
\[ \rho_b = - \div \vec{P}. \]
Luckily this again confirms our previous findings.

\subsection[The electric displacement field \textbf{D}]{The electric displacement field $\vec{D}$}
Now we consider both the bound charge due to the polarised dielectric, and the charge that caused the external electric field in the first place, which we call the \udef{free charge} $\rho_f$ . This is basically any charge that is not a result of the polarisation of the dielectric. The total charge density can then be written
\[ \rho = \rho_b + \rho_f \]
and Gauss's law reads
\[ \epsilon_0 \div \vec{E} = \rho = \rho_b + \rho_f = - \div \vec{P} + \rho_f. \]
Combining the divergence terms gives
\[ \rho_f  = \div(\epsilon_0 \vec{E} + \vec{P}) = \div \vec{D} \]
where
\[ \vec{D} \equiv \epsilon_0 \vec{E} + \vec{P} \]
is known as the \udef{electric displacement}. In integral form Gauss's law now reads
\[ \oint \vec{D}\cdot \diff{\vec{a}} = Q_{f,\text{enc}} \]
where $Q_{f,\text{enc}}$ is the total free charge enclosed in the volume.

At this point one may think that $\vec{D}$ is like $\vec{E}$, and try to use other equations containing $\vec{E}$ with $\vec{D}/\epsilon_0$ instead, substituting $\rho_f$ for $\rho$. This does in general not work for equations other than Gauss's law. In particular the curl of $\vec{E}$ is always zero, but the curl of $\vec{D}$ is not:
\[ \curl \vec{D} = \epsilon_0 (\curl \vec{E}) + (\curl \vec{P}) = \curl \vec{P}. \]
\subsection{Linear dielectrics}
Provided $\vec{E}$ is not too strong, the polarisation is often proportional to the electric field:
\[ \vec{P} = \epsilon_0\chi_e \vec{E} \]
where the constant of proportionality $\chi_e$ is called the \udef{electric susceptibility}. (Extracting the factor $\epsilon_0$ makes $\chi_e$ dimensionless). Dielectrics for which this is the case are called \udef{linear dielectrics}.

It is important to note that $\vec{E}$ is the \undline{total} field and thus \textit{itself depends on} $\vec{P}$. This makes it difficult to use this formula to use this formula to calculate $\vec{P}$. It is usually simpler to start with the displacement field $\vec{D}$.

In linear media we have
\[ \vec{D} = \epsilon_0 \vec{E} + \vec{P} = \epsilon_0 (1+\chi_e)\vec{E} = \epsilon \vec{D}. \]
Thus for linear dielectrics $\vec{D}$ is also proportional to $\vec{E}$ with a proportionality constant
\[ \epsilon \equiv \epsilon_0 (1+\chi_e) \]
we call the \udef{permittivity} of the material. (In a vacuum, where there is no matter to polarise, $\chi_e = 0$ and thus the permittivity is $\epsilon_0$, which is why it is called the permittivity of vacuum).

We also define the dimensionless constant
\[ \epsilon_r \equiv \frac{\epsilon}{\epsilon_0} = 1 + \chi_e \]
which is called the \udef{relative permittivity} or \udef{dielectric constant}. In later sections we will sometimes drop the subscript and refer to the relative permittivity simply as $\epsilon$.

In crystals, some directions are easier to polarise than others, but in any one direction the relation is still linear. In this case the susceptibility is a $(1,1)$-tensor. TODO tensor notation.

\subsubsection{Boundary value problems with linear dielectrics}
\subsubsection{Energy in dielectric systems}

\subsubsection{Forces on dielectrics}

\chapter{Magnetostatics}
TODO summary triangle
\section{Experimental evidence}
So far we have developed all of our equations based on Coulomb's law, which is only valid for stationary source charges. If we allow the charges to move, interesting new phenomena occur.

For example, say we have a piece of wire made out of some conductive material, like copper. In the section on electrostatics, we said that a conductor must be an equipotential, because otherwise charges would flow in order to make it one. Now say we hold the two ends of the wire at different potentials (for example, by connecting them to a battery), then obviously charges must flow and continue flowing. How they flow will be discussed in more detail in the section on electrodynamics.

Now something interesting occurs when we put two such wire next to each other. We observe that the attract or repel each other (depending on whether the charges are flowing in the same direction or not).
TODO wires neutral
TODO explain Lorentz law + historical evidence for B.
TODO: magnet-magnet interaction?? Also no work?


\section{The Lorentz force law}
Combining the electric and magnetic forces on a charged particle, we get the \udef{Lorentz force law}
\[ \vec{F} = Q[\vec{E} + (\vec{v}\times \vec{B})]. \]
Notice that this formula is not really complete. It assumes we know how to calculate the fields $\vec{E}$ and $\vec{B}$. In electrostatics we can calculate $\vec{E}$ from Coulomb's law. We need an analogue for the field $\vec{B}$. As with electrostatics our task is simpler if we consider a steady state. In magnetostatics we study such steady states.

Once we have found $\vec{E}$ and $\vec{B}$, the Lorentz force law is generally valid, even outside the electro- or magnetostatic régimes.

(TODO: $\vec{B}$ pseudovector)


\subsection{Cyclotron motion}
Assume there is a uniform magnetic field of magnitude $\vec{B}$. Now a charged particle of charge $Q$ enters the field with speed $v_\perp$ perpendicular to $\vec{B}$. The Lorentz force is then perpendicular to both the velocity and the magnetic field. It thus functions as a centripetal force satifying the \udef{cyclotron formula}:
\[ Qv_\perp B = m \frac{v_\perp^2}{R} \qquad \text{or} \qquad p_\perp = QBR \]
where $R$ is the radius of curvature, $m$ the particles mass and $p_\perp=mv_\perp$ its momentum.

\subsection{Magnetic forces do no work}
The magnetic force on a charged particle is always perpendicular to the direction of motion, due to the cross product. Consequently magnetic forces do no work:
\[ \diff{W_\text{mag}} = \vec{F}_\text{mag}\cdot \diff{\vec{l}} = Q(\vec{v}\times \vec{B})\cdot \vec{v}\diff{t} = 0 \]
because $(\vec{v}\times \vec{B})\cdot \vec{v}$ is zero.

\section{Currents}
TODO superposition of currents

The \udef{current} in a wire is the \textit{charge per unit time} passing a given point. Current is measured in coulombs-per-second, or \udef{amperes}:
\[1 \si{A} = 1\si{C\per s}\]

\paragraph{Line current.} A line charge $\lambda$ traveling along a wire at a speed $\vec{v}$ constitutes a current
\[ \vec{I} = \lambda \vec{v}. \]
Current is actually a vector, even if its vectorial character is often not important (cfr. circuit diagrams). The magnetic force on a segment of current-carrying wire is then
\[\vec{F}_\text{mag} = \int(\vec{v}\times \vec{B})\diff{q} = \int(\vec{v}\times \vec{B})\lambda\diff{l} = \int(\vec{I}\times \vec{B})\diff{l} \]
which we can rewrite as
\[\vec{F}_\text{mag} = I\int (\diff{\vec{I}}\times \vec{B}). \]
\paragraph{Surface current.}
Imagine a ribbon of current of infinitesimal width $\diff{l_\perp}$ flowing along a surface. If the current in this ribbon is $\diff{\vec{I}}$, the \udef{surface current density $\vec{K}$} (i.e. current per unit width) is
\[ \vec{K} \equiv \frac{\diff{\vec{I}}}{\diff{l_\perp}}. \]
If a surface current density $\sigma$ is moving at a velocity $\vec{v}$, then
\[ \vec{K} = \sigma \vec{v}. \]
The magnetic force on the surface current is
\[ \vec{F}_\text{mag} = \int (\vec{v}\times \vec{B})\sigma \diff{a} = \int (\vec{K}\times \vec{B})\diff{a}. \]
(TODO caveat: Just as $\vec{E}$ suffers a discontinuity at a surface charge, so $\vec{B}$ is discontinuous at a surface current. So here the average field must be used.)
\paragraph{Volume current density.}
Consider a tube of infinitesimal cross section $\diff{a_\perp}$ running parallel to the flow. If the current in the tube is $\diff{\vec{I}}$, the \udef{volume current density $\vec{J}$} (i.e. current per unit area) is
\[ \vec{J} \equiv \frac{\diff{\vec{I}}}{\diff{a_\perp}} \]
If a volume current density $\rho$ is moving at a velocity $\vec{v}$, then
\[ \vec{J} = \rho \vec{v}. \]
The magnetic force on the surface current is
\[ \vec{F}_\text{mag} = \int (\vec{v}\times \vec{B})\rho \diff{\tau} = \int (\vec{J}\times \vec{B})\diff{\tau}. \]

\subsection{Charge conservation}
Local charge conservation is expressed by the continuity equation
\[ \boxed{\div \vec{J} = - \pd{\rho}{t}} \]

\subsection{Steady currents}
This section is called magnetostatics, but the Lorentz force law is generally true. So far there has been nothing particularly static, in fact we can only consider magnetic effects if there are \textit{moving} charges.
In magnetostatics we study the effects of currents that do not change in time. We can call them \udef{steady currents}. Formally steady currents are defined by the condition
\[ \pd{\rho}{t} = 0,\qquad \pd{\vec{J}}{t} = 0 \]
at all places and all times. From charge conservation we also get
\[ \div \vec{J} \]
which means in a wire $I$ mus be the same all along the wire; otherwise, charge would be piling up somewhere and it would not be a steady current.

Point charges can never constitute as steady current.

\subsection{The Biot-Savart law}
The magnetic field of a steady line current is given by the \udef{Biot-Savart law}
\[ \boxed{\vec{B}(\vec{r}) = \frac{\mu_0}{4\pi} \int \frac{\vec{I}\times \hrcurs}{\rcurs^2}\diff{l'} = \frac{\mu_0}{4\pi}I \int \frac{\diff{\vec{l'}}\times \hrcurs}{\rcurs^2} } \]
For surface and volume currents, the Biot-Savart law becomes
\[ \boxed{ \vec{B}(\vec{r}) = \frac{\mu_0}{4\pi}\int\frac{\vec{K}(\vec{r'})\times \hrcurs}{\rcurs^2}\diff{a'} } \qquad \text{and} \qquad \boxed{ \vec{B}(\vec{r}) = \frac{\mu_0}{4\pi}\int\frac{\vec{J}(\vec{r'})\times \hrcurs}{\rcurs^2}\diff{\tau'} }. \]

The constant $\mu_0$ is called the \udef{permeability of free space}:
\[ \mu_0 = 4\pi\times 10^{-7} \si{N\per A^2}. \]

\section[The divergence and curl of \textbf{B}]{The divergence and curl of $\vec{B}$}
In the following calculations, the integration is over the \textit{primed} coordinates; the divergence and curl are with respect to the \textit{unprimed} coordinates. Let $\vec{J}(\vec{r'})$ depend only on primed coordinates, such that divergences and curls of $\vec{J}$ vanish.

We start with the Biot-Savart law 
\[ \vec{B}(\vec{r}) = \frac{\mu_0}{4\pi}\int\frac{\vec{J}(\vec{r'})\times \hrcurs}{\rcurs^2}\diff{\tau'} \]
to which we apply the divergence:
\[ \div \vec{B} = \frac{\mu_0}{4\pi}\int \div\left(\vec{J}\times \frac{\hrcurs}{\rcurs^2}\right)\diff{\tau'}. \]
Invoking the product rule,
\begin{align*}
\div\left(\vec{J}\times \frac{\hrcurs}{\rcurs^2}\right) &= \frac{\hrcurs}{\rcurs^2}\cdot \left(\curl \vec{J}\right) - \vec{J}\cdot \left(\curl \frac{\hrcurs}{\rcurs^2}\right) \\
&= 0.
\end{align*}
So the divergence of the magnetic field is zero
\[ \boxed{\div \vec{B} = 0} \]

Starting again with the Biot-Savart law 
\[ \vec{B}(\vec{r}) = \frac{\mu_0}{4\pi}\int\frac{\vec{J}(\vec{r'})\times \hrcurs}{\rcurs^2}\diff{\tau'} \]
we apply the curl:
\[ \div \vec{B} = \frac{\mu_0}{4\pi}\int \curl\left(\vec{J}\times \frac{\hrcurs}{\rcurs^2}\right)\diff{\tau'}. \]
Again using the relevant product rule:
\begin{align}
\curl\left(\vec{J}\times \frac{\hrcurs}{\rcurs^2}\right) &= \left(\frac{\hrcurs}{\rcurs^2}\cdot \vnabla\right)\vec{J} - \left(\vec{J}\cdot \vnabla\right)\frac{\hrcurs}{\rcurs^2} + \vec{J} \left(\div \frac{\hrcurs}{\rcurs^2}\right) - \frac{\hrcurs}{\rcurs^2} \left(\div \vec{J}\right) \nonumber \\
&= - \left(\vec{J}\cdot \vnabla\right)\frac{\hrcurs}{\rcurs^2} + \vec{J} \left(\div \frac{\hrcurs}{\rcurs^2}\right) \label{eq:curlBcontrib}
\end{align}
The first term integrates to zero. Because $\brcurs$ only depends on the difference between coordinates and $(\pd{}{x})f(x-x') = - (\pd{}{x'})f(x-x')$, we can write
\[ - \left(\vec{J}\cdot \vnabla\right)\frac{\hrcurs}{\rcurs^2} = \left(\vec{J}\cdot \vnabla'\right)\frac{\hrcurs}{\rcurs^2}. \]
Now taking the $x$ component $\left[\left(\vec{J}\cdot \vnabla'\right)\left(\frac{x-x'}{\rcurs^3}\right)\right]$, the product rule for $\vnabla'\cdot \left[\frac{(x-x')}{\rcurs^3}\vec{J}\right]$ can be rearranged to give
\[ \left(\vec{J}\cdot \vnabla'\right)\left(\frac{x-x'}{\rcurs^3}\right) = \vnabla'\cdot \left[\frac{(x-x')}{\rcurs^3}\vec{J}\right] - \left(\frac{x-x'}{\rcurs^3}\right)\left(\vnabla'\cdot \vec{J}\right). \]
For steady currents the divergence of $\vec{J}$ is zero, so
\[ \left[- \left(\vec{J}\cdot \vnabla\right)\frac{\hrcurs}{\rcurs^2}\right]_{x} = \vnabla'\cdot \left[\frac{(x-x')}{\rcurs^3}\vec{J}\right]. \]
Integrating the first term of equation \eqref{eq:curlBcontrib}, yields
\[ \int_\mathcal{V} \vnabla'\cdot \left[\frac{(x-x')}{\rcurs^3}\vec{J}\right]\diff{\tau'} = \oint_{\mathcal{S}} \frac{(x-x')}{\rcurs^3}\vec{J}\cdot \diff{\vec{a'}}. \]
If we integrate over a volume large enough such that all the current is inside the volume and $\vec{J}$ is zero on the boundary, the integral vanishes. (This typically still holds if $\vec{J}$ extends to infinity, as in the case of an infinite straight wire).

Thus the curl of $\vec{B}$ is obtained by integrating over the second term of equation \eqref{eq:curlBcontrib}:
\begin{align*}
\curl \vec{B} &= \frac{\mu_0}{4\pi}\int \vec{J}(\vec{r'}) \left(\div \frac{\hrcurs}{\rcurs^2}\right) \diff{\tau'} \\
&= \frac{\mu_0}{4\pi}\int \vec{J}(\vec{r'}) 4\pi \delta^3(\vec{r} - \vec{r'}) \diff{\tau'} \\
&= \mu_0 \vec{J}(\vec{r})
\end{align*}

\subsection{Ampère's law}
The equation for the curl of $\vec{B}$,
\[ \boxed{\curl \vec{B}} = \mu_0 \vec{J} \]
is called \udef{Ampère's law} (in differential form). It can be written in integral form by integrating over a surface of your choosing and using Stokes' theorem:
\[ \int (\curl \vec{B})\cdot \diff{\vec{a}} = \oint \vec{B}\cdot \diff{\vec{l}} = \mu_0 \int \vec{J}\cdot \diff{\vec{a}}. \]
Now, $\int \vec{J}\cdot \diff{\vec{a}}$ is the total current passing through the surface, which we call $I_{\text{enc}}$. It is the current enclosed by the \udef{Amperian loop}. Thus
\[\boxed{\oint \vec{B}\cdot \diff{\vec{l}} = \mu_0 I_{\text{enc}}}\]

Like Gauss's law, Ampère's law is always \textit{true} (for steady currents), but not always \textit{useful}. It is only useful if there is enough symmetry. Some situations where it is useful:
\begin{enumerate}
\item Infinite straight lines: disk perpendicular to line, centered on line.
\item Infinite planes: rectangle perpendicular to plane and to $\vec{K}$.
\item Infinite solenoids: rectangle with sides parallel to direction of solenoid; one outside and one half in, half out.
\item Toroids: the magnetic field of a toroid is circumferential at all points, both inside and outside the coil; take a circle about the axis of the toroid.
\end{enumerate}

\section{Magnetic vector potential}
Just as $\curl \vec{E}$ permitted us to introduce the scalar potential $V$ in electrostatics, so $\div \vec{B}$ allows the introduction of a \udef{vector potential $\vec{A}$} in magnetostatics:
\[ \boxed{\vec{B} = \curl \vec{A}} \]
There is some freedom in determining $\vec{A}$: you can add any function whose curl is zero (i.e. the gradient of any scalar). This freedom can be used to eliminate the divergecne of $\vec{A}$:
\[ \div\vec{A} = 0. \]

In terms of the vector potential, Ampère's law becomes
\[ \boxed{\nabla^2 \vec{A} = -\mu_0 \vec{J}. } \]
This is just three Poisson's equations. Assuming $\vec{J}$ goes to zero at infinity, the solution is
\[ \vec{A}(\vec{r}) = \frac{\mu_0}{4\pi}\int \frac{\vec{J}(\vec{r'})}{\rcurs}\diff{\tau'}. \]
For line and surface currents,
\begin{equation}
\vec{A} = \frac{\mu_0}{4\pi}\int \frac{\vec{I}}{\rcurs}\diff{l'} = \frac{\mu_0I}{4\pi}\int \frac{1}{\rcurs}\diff{\vec{l'}}; \qquad \vec{A} = \frac{\mu_0}{4\pi}\int \frac{\vec{K}}{\rcurs}\diff{a'}. \label{eq:lineSurfaceVectorPot}
\end{equation}
\subsection{Multipole expansion}
As before we wish to write the vector potential as a power series in $1/r$. To the end we recall
\[ \frac{1}{\rcurs} = \frac{1}{\sqrt{r^2 + (r')^2 - 2rr'\cos\theta'}} = \frac{1}{r}\sum^\infty_{n=0} \left(\frac{r'}{r}\right)^n P_n(\cos\theta'). \]
Using this, we can write the vector potential of a current loop as
\begin{align*}
\vec{A}(\vec{r}) &= \frac{\mu_0 I}{4\pi}\oint \frac{1}{\rcurs}\diff{\vec{l}'} = \frac{\mu_0 I}{4\pi} \sum^\infty_{n=0}\frac{1}{r^{n+1}}\oint(r')^nP_n(\cos\theta')\diff{\vec{l'}} \\
&= \frac{\mu_0 I}{4\pi}\left[\frac{1}{r}\oint \diff{\vec{l}'}+ \frac{1}{r^2}\oint r'\cos\theta'\diff{\vec{l}'}+ \frac{1}{r^3}\oint(r')^2 \left(\frac{3}{2}\cos^2\theta' - \frac{1}{2}\right)\diff{\vec{l}'}+ \ldots\right]
\end{align*}
As before, we call the first term the monopole term.  For the vector potential this term vanishes, because
\[ \oint \diff{\vec{l}'} = 0 \]
is just the vector displacement around a closed loop. The absence of magnetic monopoles is also expressed by the Maxwell's equation $\div \vec{B} = 0$, which is what allowed us to define the vector potential in the first place.

The dominant term is usually the dipole term
\begin{align*}
\vec{A}_\text{dip}(\vec{r}) &= \frac{\mu_0 I}{4\pi r^2}\oint r'\cos\theta' \diff{\vec{l}'} = \frac{\mu_0 I}{4\pi r^2}\oint (\hat{r}\cdot \vec{r'}) \diff{\vec{l}'} \\
&= \frac{\mu_0}{4\pi}\frac{\vec{m}\times \hat{r}}{r^2}
\end{align*}
where $\vec{m}$ is the \udef{magnetic dipole moment}
\[ \vec{m} \equiv I \int \diff{\vec{a}} \]

We can construct a \udef{pure magnetic dipole} by putting a current loop around the origin and shrinking it until it is infinitesimal, while keeping $\vec{m}$ constant. For a pure dipole we have
\begin{align*}
\vec{A}_\text{dip}(\vec{r}) &= \frac{\mu_0}{4\pi}\frac{m\sin\theta}{r^2}\hat{\phi} \\
\vec{B}_\text{dip}(\vec{r}) &= \curl \vec{A} = \frac{\mu_0 m }{4\pi r^3}(2\cos\theta \hat{r} + \sin \theta \hat{\theta})
\end{align*}

\section{Discontinuity of magnetic field at surface current}
TODO
\section{Miscellaneous results in magnetostatics}
\subsubsection{Helmholtz coil}
\subsubsection{Hall effect}


\chapter{Magnetic fields in matter}
TODO
\section{Magnetisation}
\section{The field inside a magnetised object}
\subsection{Bound currents}
\section[The auxiliary field \textbf{H}]{The auxiliary field $\vec{H}$}
\section{Linear and nonlinear media}


\chapter{Electrodynamics}
To make a current flow, you need to push the charges. To account for that we need to allow forces to act on charges. We define $\vec{f}$ as the force per unit charge it is acting on.

We can split $\vec{f}$ into two parts: the forces generated by a source (such as a battery, photoelectric cell or Van de Graaff generator), $\vec{f}_s$; and the electrostatic forces the charges exert on each other, which is just $\vec{E}$.
\[ \vec{f} = \vec{f}_s + \vec{E} \]

\section{Electromotive force}
We now define the \udef{electromotive force} or \udef{emf} of a circuit as
\[ \boxed{ \mathcal{E} \equiv \oint \vec{f}\cdot \diff{\vec{l}} = \oint \vec{f}_s\cdot \diff{\vec{l}}.} \]
We can use $\vec{f}$ or $\vec{f}_s$ because $\oint \vec{E}\cdot \diff{\vec{l}} = 0$. The name electromotive \textit{force} is not ideal as it is actually an integral of a force per unit charge, not a force. In fact it can be interpreted as the work done per unit charge (although remember magnetic forces never do any work).

Within an ideal voltage source, the net force on the charges is zero (there is just enough force to push the electrons through the circuit), so $\vec{E} = - \vec{f}_s$. Outside the source $\vec{f}_s = 0$. The potential difference between the terminals ($a$ and $b$) is therefore
\[ V = -\int_a^b \vec{E}\cdot \diff{\vec{l}} = \int^b_a \vec{f}_s\cdot \diff{\vec{l}} = \oint \vec{f}_s \cdot \diff{\vec{l}} = \mathcal{E}. \]

\subsection{Motional emf}
Motional emfs arise when you move a wire through a magnetic field. In order to calculate the emf we use a new quantity, the flux of $\vec{B}$ through a loop:
\[ \Phi \equiv \int \vec{B}\cdot \diff{\vec{a}} \]

Now assume that we have a loop of wire at time $t$; then advance to time $t+\diff{t}$. The loop has now moved. TODO image.
\[ \diff{\Phi} = \Phi(t+\diff{t}) - \Phi(t) = \Phi_\text{ribbon} = \int_\text{ribbon} \vec{B}\cdot \diff{\vec{a}}. \]
Let $P$ be a point on the wire. Let $\vec{v}$ be the velocity of the wire at that point, $\vec{u}$ the velocity of a charge along the wire at that point and $\vec{w} = \vec{v} + \vec{u}$ is the resultant velocity of a charge at $P$. The infinitesimal area of the ribbon can be written as
\[ \diff{\vec{a}} = (\vec{v}\times \diff{\vec{l}})\diff{t}. \]
Therefore
\[ \od{\Phi}{t} = \oint \vec{B}\cdot (\vec{v}\times \diff{\vec{l}}). \]
Since $\vec{w} = (\vec{v} + \vec{u})$ and $\vec{u}$ is parallel to $\diff{\vec{l}}$, we an just as well write this as
\[ \od{\Phi}{t} = \oint \vec{B}\cdot (\vec{w}\times \diff{\vec{l}}). \]
Using the scalar triple-product identity we get
\[ \od{\Phi}{t} = - \oint (\vec{B}\times \vec{w})\cdot \diff{\vec{l}}. \]
But $(\vec{w}\times \vec{B})$ is the magnetic force per unit charge, $\vec{f}_\text{mag}$, so
\[ \od{\Phi}{t} = - \oint \vec{f}_\text{mag}\cdot \diff{\vec{l}}, \]
and the integral of $\vec{f}_\text{mag}$ is the emf:
\[ \boxed{\mathcal{E} = - \od{\Phi}{t}} \]

\section{Electromagnetic induction}
\subsection{Faraday's law}
We have shown that moving a wire in a magnetic field can create a (motional) emf. Michael Faraday also reported on experiments where the wire was kept fixed and the \textit{magnetic field} was moved or changed. In these cases Faraday also measured a current due to an emf.

This came as a surprise, because the force generating the emf cannot be magnetic since all the charges are stationary and sationary charges do not experience magnetic forces. But one would think there was no electric field because the wire is balanced and neutral. Faraday had an ingenious inspiration:
\begin{center}
\textbf{A changing magnetic field induces an electric field.}
\end{center}
Empirically Faraday found the emf generated by the induced field to follow the same law as the motional emf:\footnote{Classically it is quite a surprising coincidence. The mechanism for Faraday's law is different from that for motional emf: one is electric, the other magnetic. That they coincide is a coincidence, but one that motivated, and is explained by, Einstein's theory of relativity.}
\[ \mathcal{E} = \oint \vec{E}\cdot \diff{\vec{l}} = - \od{\Phi}{t} = -\int \pd{\vec{B}}{t}\cdot \diff{\vec{a}} \]
This is \udef{Faraday's law}, in integral form. We can convert it to differential form by applying Stokes' theorem:
\[ \boxed{\curl \vec{E} = - \pd{\vec{B}}{t}.} \]
This reduces to the old rule of $\oint \vec{E}\cdot \diff{\vec{l}} = 0$ or $\curl \vec{E} = 0$ in the static case as it should.

\subsubsection{Lenz's law}
It is often difficult to keep track of signs in Faraday's law. \udef{Lenz's law} gives you a way to work out which way an induced current will flow:
\begin{center}
\textbf{Nature abhors a \textit{change} in flux.}
\end{center}
The induced current will flow in such a direction that the flux \textit{it} produces tends to cancel the change.
\subsubsection{The quasistatic régime}
Electromagnetic induction only occurs when there are changing magnetic fields. Often, however, we will want to use the apparatus of magnetostatics to calculate those changing fields. The is, of course, only approximately correct, but is often very close. This \textit{quasistatic approximation} only seriously breaks down when working with electromagnetic waves and radiation.

\subsection{Inductance}
Assume we have two loops of wire at rest. If you run a steady current $I_1$ around loop 1, it produces a magnetic field $\vec{B}_1$. Some of the field lines pass through loop 2; let $\Phi_2$ be the flux of $\vec{B}_1$ through the area of loop 2. From the Biot-Savart law we know that $\vec{B}_1$ is proportional to $I_1$, therefore so too is the flux though loop 2. Thus
\[ \Phi_2 = M_{21} I_1 \]
where $M_{21}$ is the constant of proportionality which we call the \udef{mutual inductance} of the two loops.

We can derive a formula for the mutual inductance as follows:
\begin{align*}
M_{21} = \frac{\Phi_2}{I_1} = \frac{1}{I_1}\int \vec{B}_1 \cdot \diff{\vec{a}_2} &= \frac{1}{I_1} \int (\curl \vec{A}_1)\cdot \diff{\vec{a}_2} \\
&= \frac{1}{I_1}\oint \vec{A}_1 \cdot \diff{\vec{l}_2} \\
&= \frac{1}{I_1}\oint \left[\frac{\mu_0 I_1}{4\pi}\oint \frac{\diff{\vec{l}_1}}{\rcurs} \right]\cdot \diff{\vec{l}_2} \\
&= \frac{\mu_0}{4\pi}\oint_{\text{loop 1}}\oint_{\text{loop 2}} \frac{\diff{\vec{l}_1}\cdot \diff{\vec{l}_2}}{\rcurs}.
\end{align*}
using equation \eqref{eq:lineSurfaceVectorPot} for the vector potential due to a line current. This is the \udef{Neumann formula}. It reveals two important facts about mutual inductance:
\begin{enumerate}
\item $M_{21}$ is a purely geometric quantity, having to do with the sizes, shapes and relative positions of the two loops.
\item We can switch the roles of the two loops without changing the integral; it follows that
\[ M_{21} = M_{12}. \]
We drop the subscripts and call them both $M$.
\end{enumerate}

If we vary the current in loop 1, the changing flux in loop 2 will induce an emf according to Faraday's law:
\[ \mathcal{E_2} = - \od{\Phi_2}{t} = - M \od{I_1}{t}. \]
(This assumes we are in the quasistatic régime.)

A changing current in a loop induces an emf not only in nearby loops, but also in itself. We can repeat the above analysis taking loop 1 and loop 2 to be the same loop. We now denote the constant of proportionality between the flux and the current as $L$ and we call it the \udef{self inductance} (or simply \udef{inductance}):
\[ \Phi = LI \qquad \mathcal{E} = - L \od{I}{t}. \]
Inductance is measured in henries (H); a henry is a volt-second per ampere.

Due to Lenz's law, inductance is an intrinsically positive quantity. The induced emf opposes the change in current. For this reason it is called a \textit{back emf}.

\subsubsection{Calculating the inductance of various configurations}
TODO

\section{Energy in magnetic fields}

\section{Maxwell's equations}

\chapter{Conservation laws}
\section{Charge}
\section{Energy}
\section{Momentum}
\subsection{Angular momentum}

\chapter{Potentials and fields in electrodynamics}
\section{The scalar and vector potentials in electrodynamics}
\subsection{Gauge transformations}
\section{Continuous distributions}
\subsection{Retarded potentials}
\subsection{Jefimenko's equations}
\section{Point charges}
\subsection{Liénard-Wiechert potentials}
\subsection{Fields of a moving charge}
\section{Lagrangian formulation}

\chapter{Electromagnetic waves}
\section{Light waves}
Phenomenology, Polerization, polerizing filter

intensity square of amplitude
\section{Electromagnetic waves in vacuum}
\section{Electromagnetic waves in matter}
\section{Absorption and dispersion}
\section{Guided waves}

\chapter{Radiation}
\section{Dipole radiation}
\section{Radiation of point charges}


\chapter{Gaussian system of units}