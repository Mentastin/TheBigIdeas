\chapter{Formal systems}
The ideas in the previous section were expressed somewhat informally. Now an attempt will be made at a more formal discussion of the ideas. This will lead us to the ideas behind formal systems.\footnote{What follows will be quite a personal exposition of formal systems. All of the results are very standard and may be found in any work on mathematical logic, but each author has their own preferred definitions and notation, depending on what problems they are trying to solve. My exact definition of a formal system is one that I have not found anywhere else, but I believe it to be a useful one.} We will start with a very broad idea and then try to narrow it down by arguing that certain constraints apply and we may disregard certain possibilities or that certain possibilities are equivalent.

\section{Towards a definition}
\subsection{Transmitting ideas and reasoning about the world}
Essentially we want a system that can be used to transmit ideas. A priori it is difficult to know what ideas are. 


At this point it must also be stressed that logic is not one monolithic entity. Different ``logical languages'' or logical systems have been developed for different uses. This may seem strange, because one imagines there is only one true truth. It is however somewhat analogous to the existence of different natural languages. It is possible to say the same thing in either English or French.

In addition there is no logical system that is capable of expressing everything. This means logical systems sometimes need to be extended. For example modal logic adds the ability to qualify statements made in classical logic. Conversely it is often both practically and theoretically useful to keep a logical system as concise as possible, dropping unnecessary extensions. This makes working with the logical system much easier.

In general a logical system is a type of \udef{formal system}. A formal system consists of a formal language as well as ways to \textit{derive} new statements from old ones. These new statements are conclusions that can be inferred from the given statements which are known as premises.

TODO: arbitrary, circular, abstraction (equivalence and different models)

\section{On the foundation of formal systems}
\subsection{What and why?}
We want a system, a method to transmit information and insights between us humans. In physics, that is specifically about the physical world, but for now we will work more generally.



In general we could use any system imaginable. 

\subsection{Pre-Socratic meditations}
\subsubsection{Anaximander and the apeiron}
\subsubsection{On change: Heraclitus and Parmenides}
\subsection{Defining formal systems}
We are allowed to do anything we can. Whether we should is another question entirely.

Relevant slices

\subsection{The art and science of modeling reality}

\subsection{Iterative development}
\subsubsection{Building physical intuition}

\subsection{A link with thoughts}
Formal argument not on paper, but in mind

\subsection{Model theory}
Finally a couple of remarks should be made about the level of generality of a formula. The conception of a formula introduced above is one of a mathematical manifestation of a consequence of a set of general ideas, which in this case are obviously ideas about how the physical world works. We may also call this ensemble of ideas, or any subset of it, a \udef{physical model}. In general a formula can be a manifestation of a model at any level of generality, from very specific cases to being fully accurate representations of the entire model.

\begin{example}
So for instance say we want to know how long it would take an apple to reach the ground if we dropped it off a building. This obviously depends on the height of the building. We may therefore write down a formula that relates the height of the building with the time in free-fall. This may well be a very nice formula, but maybe we want to allow for cases in which the apple is given some initial velocity in the longitudinal direction, i.e. the apple is either thrown upwards or downwards. We can write down a formula for that. This new formula should give the same results as the first formula if we take the initial velocity to be zero. Thus it is more general than the first formula. We can generalise the formula in many ways, maybe allowing the apple to be thrown in different directions or maybe allowing the strength of gravity to change to account for different locations on earth and in space. If we generalise enough eventually we will get Newton's equations of motion. In fact we can generalise further to obtain quantum mechanical or relativistic equations. These very general formulae can be used to describe entire models.
\end{example}


\section{A formal definition}

\subsection{The language}
\subsubsection{The apeiron and limitations}
intelligibility and resolution

\subsubsection{Formal languages}
A formal language consists of 
\begin{enumerate}
\item a collection of symbols that may be used to write in the language. This collection is called the \udef{alphabet}. All ``sentences'' (i.e. formulae) in the formal system are just strings of characters from the alphabet;
\item a way to specify which strings of characters are actually valid in the language. For example the string of characters ``ajf ekjl ljegk jlkkgkgkkgddd'' is written in the same alphabet as the English language, but is most certainly not English. This is done by giving a set of production rules known as a \udef{(formal) grammar}.\footnote{There are also other ways to specify which sentences are valid, for example using a recogniser. This topic is way beyond the scope of a text about physics. Interested readers should find a book on parsing or formal language theory, such as Ullman, Aho (TODO ref)} In logic valid sentences are called \udef{well-formed formulae} of \udef{wffs} for short.
\end{enumerate}
These aspects of a formal system are known as the \udef{syntax} of the formal system.

\subsubsection{Tokenisation}
TODO Leary p.7

\subsubsection{Grammars}
unique readability

\subsection{Texts: proofs / calculations / derivations}
We usually define formal system in order to be able to write \udef{texts} in them. Depending on the context and the system, such a text is more commonly known as a proof, a calculation or a derivation. We will (somewhat unconventionally) use the term text, in order to be as general as possible.

For a given formal system, a text typically starts with a number of wffs called \udef{premises}. Then other wffs are added to the page, but only if they follow from some so-called rule of inference.

We can write as many as we want. When we are satisfied and have finished writing the text, the last wff is called the \udef{conclusion}.

A text may be summarised by stating only the premises and conclusion, separated by $\vdash$. Such a summary looks something like
\[ \text{premise 1},\; \text{premise 2},\; \text{premise 3}\, \ldots \quad\vdash\quad \text{conclusion}. \]

\begin{example}
TODO a derivation, a calculation and a solution to a system of equations
\end{example}

The other reason we use the term text is because it is fairly neutral and we do not want to suggest an interpretation at this stage. How a text written in a formal system, and in particular what it means when a formula is written down, is not a property of the formal system itself. It is an interpretation that is given later.

In principle a formal system just tells us which orderings of symbols on a page are valid. Of course in practice we want our formal system to say something true about this reality, but that link to reality, called the \udef{semantics} (or a semantic interpretation) of the formal system, is only specified later and not really an integral part of the formal system. Although, of course, we will not talk much about formal systems that do not have a semantic interpretation that corresponds to, at least approximately, some part of reality.

\subsection{Inference rules}
\udef{Inference rules} are rules that specify how to obtain new formulae from certain old ones. They are processes that take a number of wffs as inputs and outputs a new wff. A priori there are no other restrictions on the types of processes that are allowed.

In principle we can specify inference rules however we want. One might use operations from language theory such as slicing and concatenation.
\begin{example}
An example of an inference rule is the following:
\begin{enumerate}
\item Discard the first three symbols of each premis.
\item Concatenate the resulting formulae.
\item If this formula is well-formed, it is a conclusion that follows from the premises.
\end{enumerate}
This is a perfectly good inference rule, even if it is unlikely to make any semantic sense. Unless the formal language was constructed such that the operations of slicing and concatenation were semantically meaningful.
\end{example}
\begin{example}
Another example of an inference rule that is valid for a formal system in which ``aab'', ``babb'' and ``abba'' are wffs is:
{\centering If the premises are ``aab'' and ``babb'', then conclude ``abba''}

This inference rule is very specific. Typically explicitly specifying all the inference rules we want, like this, is impractical because there are too many of them.
\end{example}

Most commonly a rule of inference is stated using a schema (plural: schemata).

\subsubsection{Schemata and metavariables}
In order to define schemata, we need to have a new set of symbols, called \udef{metavariables}.\footnote{In computer science and programming these variables are called \udef{metasyntactic variables}.} These symbols can be anything, so long as they are not part of the alphabet of the formal system. Typically we will use uppercase letters ($A, B, C \ldots$).

\udef{Schemata} are sequences of symbols that are either part of the alphabet or metavariables. Consequently all sentences in the language of the formal system are also schemata.

Now we can write whole classes of inference rules by giving several schemata as premises and a schema as conclusion. Specific inference rules are then obtained by substituting strings of symbols from the alphabet for each metavariable, such that if two metavariables are the same, their substitution is as well. Additionally we also have to require that each premise and the conclusion are all wffs.

TODO: types
This technique is especially powerful because many the grammar of many logical systems is defined such that wffs consist of sub-wffs strung together using connective symbols. In this case many schemata are guaranteed to yield wffs if the metavariables are substituted with wffs. 

\subsection{Axioms and axiom schemata}
Some authors allow the definition of a formal system to contain \udef{axioms}. These are just wffs that may be written down, regardless of which wffs came before it in the text. We will simply view them as inference rules that do not have any premises. 

Similarly, \udef{axiom schemata} are classes of axioms that are specified by a schema.

A \udef{tautology} is like an axiom in the sense that it may always be written in a text. The difference is that it is not explicitly supplied as an inference rule. Instead it is the consequence of several inference rules.

\section{Semantics and interpretation}

\section{Formalisms}
The word \udef{formalism} is a favourite of many a physics professor. It is essentially a synonym of ``formal system'', but it is typically only used for formal systems where the texts are called calculations.

TODO

\section{Equivalence of formal systems}
\subsection{Computational trinitarianism}
\subsection{On the relativity of truth}

\section{Metasystems}
\subsection{Duality}
\subsection{On the relativity of formal systems as unifying framework}
\subsubsection{Set-theoretic definition}
set is formal systems is set \ldots

\section{Incompleteness of formal systems}

